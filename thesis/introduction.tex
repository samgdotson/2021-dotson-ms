Modern energy systems provide reliable, cheap, and abundant energy for the
countries that have them. Access to this energy makes modern comforts possible. However,
most of this energy comes from fossil fuel consumption in the forms of natural gas
and coal --- whose carbon dioxide byproducts drive anthropogenic climate change
\cite{eia_frequently_2022}. Although we understand the risks posed by climate
change and that society needs to transition to cleaner forms of energy, the best
pathway to achieving a decarbonized energy system is controversial. Further, the
United States still lacks strong energy and climate policies at the federal level,
so the transition must be driven by individual states and smaller institutions.
Decision makers at all levels face several policy challenges.
\begin{enumerate}
  \item What technologies can replace fossil fuels?
  \item How much generating capacity is needed to preserve a reliable energy supply?
  \item Which technologies keep energy prices down?
\end{enumerate}

\glspl{esom} are class of modeling tools designed to answer these questions. Indeed,
the energy planning literature has many studies designed to address the transition
to a clean energy economy \cite{fattahi_systemic_2020,yue_review_2018,aryanpur_review_2021}.
However, many studies opt for greater technology options rather than temporal detail,
which risks biasing these studies towards technologies that have low capital costs without
considering operational challenges. Some existing work addresses the impact
of model time resolution with \glspl{esom} \cite{poncelet_impact_2016,
kotzur_impact_2018}. However, these modeling frameworks do not account for annual
variability, which climate change exacerbates \cite{van_der_wiel_contribution_2021}.

Even though many technology options are considered, the role of nuclear energy
remains particularly contentious \cite{lehtveer_how_2015,clemmer_nuclear_2018,
petti_future_2018}.
Nuclear power's role is disputed due to high capital costs, a lack of federally
guided waste management policy and perceived environmental, health, and weapons
proliferation risks \cite{clemmer_nuclear_2018,petti_future_2018}. Addressing
the financial risk posed by nuclear power's high capital costs is especially
important for continued operation and expansion of the United States nuclear fleet
\cite{petti_future_2018, lehtveer_how_2015}.

The gaps in current energy planning literature around temporal variability and
detail motivated this thesis' goals.
The objectives of this thesis were based on the need for greater sensitivity
analysis around temporal detail and variability. The objectives are listed below.

\subsubsection{Demonstrate a tool that reduces the learning curve for \gls{esom}
model development and facilitates external model inspection.} Since the scale
of the systems that \glspl{esom} model prohibits complete verification, transparency
and sensitivity analyses must be used as proxies for model integrity. This thesis
introduces the \gls{pygen} tool that enables rapid model development, sensitivity
analysis, and model interrogation. This tool is used in all experiments in this
thesis.

\subsubsection{Illustrate the importance of temporal detail in \gls{esom} models.}
There is a tradeoff between model complexity and computational cost. One strategy
to reduce computational cost is through dimensionality reduction by spatial
and temporal aggregation. This thesis will show how much temporal aggregation is
acceptable.

\subsubsection{Show the influence of inter-annual resource variability on model
results.} The transition to carbon-free energy systems will include highly intermittent
and variable resources such as solar and wind power. The effect of these energy
sources on system reliability and cost is influenced by temporal resolution, and
by the year-to-year changes. Current \glspl{esom} treat all modeled years
as identical. This thesis addresses that gap by analyzing model sensitivity to
annual availability.

\subsubsection{Recommend science-driven policies from case studies of two energy systems.}
This thesis presents two case-studies for the State of Illinois and the \gls{uiuc}.
Models for these energy systems are built with the \gls{pygen} and the \gls{temoa}
framework. This thesis provides policy recommendations based on these novel case
studies.
