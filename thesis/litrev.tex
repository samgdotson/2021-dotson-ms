This chapter reviews the state of energy system optimization modeling (ESOM) tools
and tools for evaluating nuclear hybrid energy systems (NHES). The role of \glspl{esom}
in energy and environmental policymaking is also discussed.

Energy systems are spatially, temporally, and topologically complex ‘machines’
that operate cohesively to supply energy where it is needed at the exact moment
it is needed. Spatial complexity arises from various interconnections that enable
the transmission of energy from one region to another. This includes transmission
lines that move electricity from one part of the United States to another or
pipelines moving natural gas. Regions can be large multistate entities like the
PJM Interconnection, smaller communities, or even individual homes. Temporal
complexity arises from the time dependence of energy demand and availability of
variable and intermittent sources of energy. Energy demand and renewable energy
production
exhibit a variety of seasonalities from annual, monthly, even daily trends.
There is also some stochasticity at the minute to hourly level caused either by
human behavior or weather conditions.
Examining energy systems at each of these temporal resolutions is vital for
generating robust energy policy and guaranteeing reliable energy delivery. Finally,
topological complexity refers to the number of interacting technologies that can
produce and move energy along with the number of unique energy demands.
A community powered by a small solar farm and a diesel
backup generator is topologically simpler than a state with many types of energy
demands, various means of production, and dozens of individual generators. Beyond
the physical and engineering aspects of energy systems is a social context that
motivates choices at every layer of complexity. These choices reflect the priorities
of a given society or community and manifest as energy policy.

\section{Energy Policy and Climate Change Mitigation}

Energy policies serve to coordinate the different aspects of energy systems.
Essentially, energy policies need to ensure reliability, affordability, and
sustainability \cite{fattahi_systemic_2020}. A reliable grid ensures an adequate
energy supply when it is needed \cite{milligan_methods_2011, ramirez-meyers_how_2021,berkeley_iii_framework_2010}.
An unreliable grid experiences an unacceptable frequency of outages or losses of
load \cite{ramirez-meyers_how_2021}. The North American Electric Reliability
Corporation sets the desired loss-of-load frequency at 0.1 days/year and recommends
a \gls{prm} of 15 percent \cite{milligan_methods_2011,reimers_impact_2019}. \gls{prm}
is defined as,
\begin{align}
  \text{PRM} &= \frac{P_{\text{firm}} - P_{\text{peak}}}{P_{\text{peak}}},
  \intertext{where}
  P_{\text{firm}} &= \text{installed firm capacity}, \nonumber\\
  P_{\text{peak}} &= \text{power demand at annual peak} \nonumber.
\end{align}
In other words, \gls{prm} is the fraction of capacity held in reserve during peak
times. ‘Affordability’ is an ill-defined term and varies significantly. An ``affordable''
energy system can be loosely understood as one with a low rate of energy poverty
\cite{brown_high_2020}. High energy costs have cascading adverse effects on every
aspect of life and significantly impact low-income households \cite{brown_high_2020}.
Measures that lead to increased costs for energy producers, such as carbon taxes,
almost always lead to higher consumer costs \cite{brown_high_2020,poelhekke_how_2019,khastar_how_2020}.
Thus, policymakers often seek a least-cost solution when designing energy policy.
Lastly, sustainability is defined as ``meeting the needs of the present without
compromising the ability of future generations to meet their own needs'' \cite{brook_why_2014,
the_united_nations_brundtland_commission_our_1987}. The twin problems of fossil
fuel consumption and climate change will significantly diminish the ability of
future generations to meet their basic needs. Droughts, forest fires, extreme
weather events, and flooding have all increased due to climate change
\cite{reidmiller_fourth_2018}. Additionally, human activities like deforestation,
strip mining, and monocropping are causing a loss of essential ecosystem services
that advanced technology cannot easily replace \cite{malhi_climate_2020,butler_climate_2018,
costanza_value_1997}. These ecosystem services have an estimated total value between
\$125 trillion/year and \$145 trillion/year and, therefore, failure to mitigate
climate change will incur severe economic damage \cite{costanza_changes_2014, malhi_climate_2020}.
Climate change will affect each of these policy goals, which means minimizing its
effects is a moral, political, and economic imperative.

\subsection{Survey of climate mitigation policies}

There are three primary policy strategies for climate change mitigation \cite{fawzy_strategies_2020}:
conventional mitigation efforts, which include all policies centered around
reducing carbon emissions, negative emissions technologies, and the final option,
geoengineering, which reduces warming by increasing the Earth’s reflectivity. \gls{esom}
studies generally focus on the first set of energy policies, conventional mitigation strategies.
These strategies include:
\begin{itemize}
  \item Investing in clean energy resources, such as renewable energy and nuclear power
  \cite{fawzy_strategies_2020}. These strategies affect the topology of the energy
  system by influencing the types of energy production technologies.
  \item Demand response policies involve reducing grid load when there is insufficient
  supply forecasted or flattening the load profile throughout the day. These policies
  are critical for energy systems with high penetration of variable and intermittent
  energy sources like wind and solar power \cite{bouckaert_expanding_2014,
  kuzemko_policies_2017}. Demand response policies influence the temporal
  complexity of the energy system.
  \item Carbon capture and storage (CCS) reduces the carbon footprint of coal and
  natural gas plants but are not considered negative emissions \cite{fawzy_strategies_2020}.
  These strategies affect grid topology by enabling the continued use of previously
  high carbon-emitting technologies.
\end{itemize}

\subsection{Climate change strategies in Illinois}
\label{section:ceja}
In September of 2021, Illinois legislators passed a historic clean energy bill,
\gls{ceja} \cite{harmon_climate_2021}. This bill does several notable things for
both equity and energy policy. Naming just a few, \gls{ceja}:
\begin{enumerate}
  \item Sets Illinois on a path toward 100\% clean energy by 2050 and pledges 40\% renewable
  energy (e.g., wind or solar) by 2030 and 50\% by 2040
  \cite{office_of_governor_jb_pritzker_gov_2021}. Ambiguously,
  the text of the bill refers to 100\% renewable energy by 2050 \cite{harmon_climate_2021}.
  The former includes nuclear power, while the latter typically excludes nuclear power.
  \item Establishes a carbon credit plan that supports “zero emissions facilities”
  (i.e., nuclear power plants). Exelon reversed the imminent closure of Byron and
  Dresden nuclear power plants based on the financial support from this bill
  \cite{brown_two_2021}.
  \item Workforce development programs focused on equity. Such as a
  “Clean Jobs Workforce Network Hubs Program” that seeks to ensure “equity-focused”
  populations have support to enter the clean energy workforce
  \cite{office_of_governor_jb_pritzker_gov_2021}.
\end{enumerate}
This bill extends support for nuclear plants in the PJM Interconnect, where its
predecessor, the 2016 Future Energy Jobs Act, gave financial support to nuclear
plants in the MISO grid. The Solar Energy Industries Association (SEIA) and Illinois
Clean Jobs Coalition estimate that Illinois will have to add 6.3 GW to its wind capacity
and add 17 GW of combined utility and residential solar energy
to achieve its renewable energy goals \cite{goeller_new_2021, the_accelerate_group_clean_2019}.
The Illinois Clean Jobs Coalition
drove much of the content of the CEJA bill. The Coalition funded a report to
evaluate the economic impact of the CEJA bill, finding that Illinois’ renewable
energy goals could drive 39 billion dollars in private investment to the state
\cite{the_accelerate_group_clean_2019}.
However, the creators of this report did not consider the impacts of these targets
on grid operations, electricity cost, and energy reliability. Although nuclear power
plants (NPP) currently generate more than half of Illinois’ electricity, almost
90 percent of its carbon-free energy, politicians and lobbyists fiercely debate the continued
role of nuclear power in Illinois’ energy mix.

\subsection{Illinois Climate Action Plan at the University of Illinois}

The University of Illinois at Urbana-Champaign (UIUC) released the first iteration
of the Illinois Climate Action Plan (iCAP) in 2010, and releases a new one every five years.
The 2015 iCAP report detailed UIUC’s goal to achieve carbon neutrality by 2050
\cite{institute_for_sustainability_energy_and_environment_illinois_2015}.
This goal includes every scope of emissions from the energy produced on campus, to
the emissions from vehicles faculty and staff use to drive to work
\cite{institute_for_sustainability_energy_and_environment_illinois_2015,
institute_for_sustainability_energy_and_environment_illinois_2020}. The
iCAP reports established several themes and objectives for each. These themes are:
\begin{itemize}
  \item Energy
  \item Transportation
  \item Land and Water
  \item Zero Waste
  \item Resilience
  \item Reporting Progress (accountability)
  \item Engagement
  \item Education
  \item Research Opportunities
  \item Funding
\end{itemize}
The greatest opportunity to reduce carbon emissions on the UIUC campus comes from
the ``Energy” theme. \gls{uiuc}’s specific energy objectives include an energy
efficiency target of 167,000 BTU/square foot by 2030, purchasing 140 GWh of clean
electricity by 2025, and reducing building-level energy use
\cite{institute_for_sustainability_energy_and_environment_illinois_2020}. The University
heats virtually all of its buildings using steam produced by Abbott Power Plant
(APP) which burns coal and natural gas. The iCAP objectives conspicuously lack a
measurable target for clean thermal energy on campus.

\section{Energy System Optimization Models}

\glspl{esom} are a class of modeling tools designed to facilitate policy decisions before
future uncertainties are resolved \cite{hunter_modeling_2013}. In particular, \glspl{esom} answer questions
related to the planning and expansion of energy generating capacity \cite{de_queiroz_repurposing_2019}.
In order to achieve this goal, \glspl{esom} have to balance spatial, temporal, and topological
complexities while accounting for profound future uncertainties \cite{martinez-gordon_review_2021}.
\glspl{esom} typically use a “time-slice” approach where temporal variability is captured in
several representative time slices \cite{fattahi_systemic_2020}. Some frameworks
handle spatial complexity by modeling different regions, the most detailed among
them incorporate data from geographical information systems (GIS) \cite{fattahi_systemic_2020}.
Lastly, most \glspl{esom} consider grid topology by grouping technology types (e.g. ‘coal power’
or ‘nuclear power’) \cite{fattahi_systemic_2020} while some frameworks look at
interactions among individual plants \cite{jenkins_enhanced_2017}. Handling
future uncertainty is one of the most important features in an \gls{esom}.

\subsection{Uncertainty Quantification}

Since perfect foresight is impossible, and policymakers often act then learn, handling
uncertainty is critical to developing robust insights
\cite{yue_review_2018, decarolis_modelling_2016}.
The two broad categories of uncertainties are parametric uncertainty and structural
uncertainty. Parametric uncertainty deals with the influence of unknown quantities
like future fuel prices, learning curves, and discount rates. Structural uncertainty
relates to factors that are not captured in a model’s equations
\cite{hunter_modeling_2013, yue_review_2018}.
The latter uncertainty includes heterogeneity of decision-makers,
social acceptance, and different priorities \cite{yue_review_2018}. Most ESOM
studies handle uncertainties using a small ensemble of scenarios where a reference
scenario is developed and policy impacts are analyzed with additional constraints
and assumptions \cite{yue_review_2018}. There are several systematic methods to
account for parametric uncertainty, including Monte Carlo, stochastic optimization,
and global sensitivity analysis \cite{yue_review_2018}. Structural uncertainty
is typically handled by either increasing model complexity to make it more
``realistic'' or studied systematically using \gls{mga}
\cite{hunter_modeling_2013,decarolis_modelling_2016, yue_review_2018}.
The framework I used in this thesis,
\gls{temoa}, developed at North Carolina State University, uses \gls{mga} to efficiently
probe decision space \cite{decarolis_temoa_2010}.


\section{Survey of \gls{esom} Literature}

\glspl{esom} are useful for providing policy insights to decision makers. \gls{temoa}
is one modeling framework among many and is structurally similar to TIMES/MARKAL
and MESSAGE \cite{yue_review_2018}. Typically, \glspl{esom} are
formulated with linear programming. A framework like OpenModelica models technologies
with much higher technical detail than traditional \glspl{esom} and are closer
to unit-commitment models. I include them in this review because OpenModelica and
the \gls{raven} tool are used to study \glspl{nhes}
\cite{baker_optimal_2018,garcia_dynamic_2016,epiney_economic_2020}.
The tradeoff between model complexity and computational cost is well documented.

The sensitivity analysis in the case study of the State of Illinois explicitly
examines the effect of annual variablility wind and solar resources. Such analysis
is absent from the literature and is important since all \gls{esom} studies select
a set of representative days which are constant throughout the model horizon.


\newgeometry{margin=1cm}
\begin{landscape}
  \begin{table}
    \centering
    \caption{Summary of ESOM Literature Survey}
    \label{tab:esom_lit}
    \begin{tabular}{l*{4}{r}p{9.5cm}}
\toprule
                           Author & Timeslices$^\text{a}$ &         Horizon$^\text{b}$ &           Place &        Modeling Tool &                                                                                                                                                                                                                                                               Notes \\
\midrule
Alzbutas, R. \& Norvaisa, E., 2012 \cite{alzbutas_uncertainty_2012} &         NS &   (2010,2025,5) &       Lithuania &              MESSAGE &                                                             Authors investigate impact of SMR reactors on energy supply. Results show that cogeneration with SMRs lead to lowest system wide costs and that discount rate has the greatest influence on total cost. \\
                Baker et al. 2018 \cite{baker_optimal_2018} &       8760 &             NaN & CAISO Microgrid &  RAVEN, OpenModelica &                                   This study considers how LCOE and water prices vary with wind penetration and availability while optimizing battery storage and desalination capacity with an SMR. The results show that higher wind penetration increases costs. \\
      Barron R. \& McJeon H., 2015 \cite{barron_differential_2015} &         NS &   (2020,2095,5) &           World &                 GCAM &                                                                      This study investigates the impact of different parameters on carbon dioxide abatement costs. They found that the capital cost of nuclear reactors had the greatest impact on abatement costs. \\
              Bennett et al. 2021 \cite{bennett_extending_2021} &         48 &   (2015,2040,5) &     Puerto Rico &                Temoa &             In this work, authors evaluate the effect different grid topologies such as distributed versus centralized generation and infrastructure resilience on cost of electricity and carbon emissions. Only renewable energy and fossil fuels are considered. \\
            Bouckaert et al. 2014 \cite{bouckaert_expanding_2014}&         16 &   (2010,2030,5) &  Reunion Island &         TIMES/MARKAL & Considers the impact of demand response policies on renewable energy penetration. Results show that increasing the penetration of intermittent renewable sources decreases system reliability. Demand response can mitigate the effects of reliability constraints. \\
           de Queiroz et al. 2019 \cite{de_queiroz_repurposing_2019}&   Variable & One month ahead &             NaN &                Temoa &                                                         This work modifies the Temoa ESOM framework to solve unit-commitment problems by remapping the time-slices in Temoa. The modified ESOM solved 96 times faster than the corresponding unit commitment model. \\
         de Sisternes et al. 2016 \cite{de_sisternes_value_2016} &        672 &             Single year &      ERCOT-like & IMRES &                                                                          The authors investigated how storage duration and capacity affects the levelized cost of electricity. They found that excluding nuclear power increases system costs by up to 8.6 percent. \\
            DeCarolis et al. 2016 \cite{decarolis_modelling_2016} &          4 &   (2015,2050,5) &   United States &                Temoa &                                              The authors demonstrate the MGA sensitivity analysis technique using the Temoa framework. They show that MGA can efficiently probe decision space and that increasing the slack variable increases technology options. \\
               Epiney et al. 2020 \cite{epiney_economic_2020} &       8760 &     Single year &             NaN &  RAVEN, OpenModelica &                                                                                                                                                                                                                      Demonstrates the RAVEN and OpenModelica tools. \\
               Garcia et al. 2016 \cite{garcia_dynamic_2016}&       8760 &     Single year &      Texas-like & OpenModelica, Dymola &                                                                                        This work is a case study of a nuclear hybrid energy system in Texas. Researchers found that integrating nuclear and renewable energy sources lead to highly flexible grids. \\
               Kotzur et al. 2017 \cite{kotzur_impact_2018} &   Variable &             NaN &             NaN &                 tsam & Method of time series aggregation plays a minor role compared with the type of system being modeled.\\
             Komiyama et al. 2015 \cite{komiyama_energy_2015} &      52560 &     Single year &           Japan &                 OPGM & Examines the integration of hydrogen storage for high penetration of renewable energy sources. Found that hydrogen is an economical for load shifting on a weekly-monthly scale.\\
                   Li et al. 2019 \cite{li_open_2020}&         96 &   (2015,2050,5) &  North Carolina &                Temoa &    This work examines a limited set of cost scenarios to determine the break even capital cost of each generating technology to show how close each technology is to deployment. They found that solar PV is the most cost-effective low-carbon technology.  \\
    Neumann, F. \& Brown, T., 2021 \cite{neumann_near-optimal_2021}&       4380 &     Single year &          Europe &                 PPSA & This study uses MGA to investigate near optimal solutions for the European electric system with 100 spatial nodes and high temporal resolution. Results indicate that wind energy and hydrogen storage are essential to keeping costs within 10\% of the optimal.\\
             Poncelet et al. 2016 \cite{poncelet_impact_2016} &   Variable &  (2020,2050,10) &             NaN &         TIMES/MARKAL & The authors look at the impact of temporal resolution on model results, as well as the method of choosing representative time slices. Temporal resolution is found to have the greatest influence on results.\\
                 Seck et al. 2020 \cite{seck_embedding_2020} &         84 &   (2014,2052,4) &          France &                TIMES & This study examines the French electric grid to find the optimal penetration of renewable energy sources while preserving grid reliability. The results show a 65\% penetration of renewables is achievable while maintaining reliability. \\
                  Yue et al. 2020 \cite{yue_least_2020}&         12 &  (2020,2050,10) &         Ireland &         TIMES/MARKAL &  This work investigates the 100\% renewable energy pathways in Ireland. The authors found that focusing on renewable energy sources was not the most cost-effective strategy to decarbonize all energy sectors. \\
\bottomrule
\multicolumn{6}{l}{$^\text{a}$ NS = Not Specified}\\
\multicolumn{6}{l}{$^\text{b}$ Horizon denoted by (start year, end year, time step)}
\end{tabular}

    % \resizebox{\textheight}{!}{}
  \end{table}
\end{landscape}
\restoregeometry
