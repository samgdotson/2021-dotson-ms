\gls{temoa} is an open-source \gls{esom} developed at North Carolina State University.
Like other \glspl{esom}, \gls{temoa} uses linear programming to solve the optimal
least-cost solution while meeting various demand and capacity constraints \cite{hunter_modeling_2013}.
The key benefits of \gls{temoa} are its open-source code, open data, and built-in
uncertainty analysis capabilities. These features address the need for greater
transparency in \gls{esom} modeling and robust assessment of future uncertainties
\cite{hunter_modeling_2013}. One of the critical limitations of \glspl{esom} is that
the nature of the systems they are intended to model make them inherently
unverifiable. By making the data and source code open,
\gls{temoa} facilitates repeatability and its transparency provides evidence for
model integrity
\cite{decarolis_case_2012}. To that end, along with open-source code and open data,
\gls{temoa} models should be solvable
with open-source linear programming solvers like \texttt{COIN-OR} \cite{noauthor_coin-or_nodate} or \texttt{GLPK} \cite{noauthor_glpk_nodate}. I used the common
\texttt{CBC} solver from \texttt{COIN-OR} \cite{noauthor_cbc_2021} in this thesis.
Another benefit of \gls{temoa} is its flexibility.
The tool is technology agnostic and is designed to model flows and transformations
of commodities created by users.
User-defined time slices allow modelers to run a model at an arbitrarily high
time resolution, restricted only by data availability and computational resources.

\subsection{Mathematics of Temoa}
\glspl{esom} like \gls{temoa} use linear programming to optimize an objective
function, in most cases the objective function represents total system cost.
A single \gls{temoa} run minimizes the following objective function \cite{noauthor_preface_nodate},

\begin{align}
  C_{total} &= C_{loans} + C_{fixed} + C_{variable}
  \intertext{where}
  C_{loans} &= \text{the sum of all investment loan costs},\nonumber\\
  C_{fixed} &= \text{the sum of all fixed operating costs},\nonumber\\
  C_{variable} &= \text{the sum of all variable operating costs}.\nonumber
\end{align}
Each term in $C_{total}$ has a mathematical expression but the most complicated
of these terms is $C_{loans}$, which is calculated by,
\begin{align}
  C_{loans} &= \sum_{t,v \in \Theta_{IC}}{\left(\left[IC_{t,v}\cdot LA_{t,v} \cdot \frac{(1+GDR)^{P_0-v+1}\cdot (1-(1+GDR)^{-LLN_{t,v}})}{GDR} \cdot \frac{1-(1+GDR)^{-LPA_{t,v}}}{1-(1+GDR)^{-LP_{t,v}}}\right] \cdot \textbf{CAP}_{t,v}\right)}
  \intertext{where}
  t &= \text{the name index of a particular technology}, \nonumber\\
  v &= \text{the vintage (i.e. construction year) of a particular technology}, \nonumber\\
  \Theta_{IC} &= \text{the set of all technologies that have associated investment costs}, \nonumber\\
  IC_{t,v} &= \text{the investment cost for a particular technology and vintage (i.e. a ``process'')},\nonumber\\
  LA_{t,v} &= \text{the amortization of the loan},\nonumber\\
  \displaybreak
  &= \frac{DR_{t,v}}{1-(1+DR_{t,v})^{-LLN_{t,v}}} \forall \{t,v\} \in \Theta_{IC}\\
  DR_{t,v} &= \text{the process specific discount rate}, \nonumber\\
  GDR &= \text{the global discount rate, applied to all processes}, \nonumber\\
  P_0 &= \text{the first year in the simulation}, \nonumber\\
  P_{last} &= \text{the last year in the simulation}, \nonumber\\
  LLN_{t,v} &= \text{the loan lifetime of a process}, \nonumber\\
  LP_{t,v} &= \text{the total lifetime of a process}, \nonumber\\
  LPA_{t,v} &= \text{the active lifetime of a process}, \nonumber\\
  &= \min{\left(LP_{t,v}, \left(P_{last}-LP_{t,v}\right)\right)}\\
  \textbf{CAP}_{t,v} &= \text{the capacity of a process}.\nonumber
\end{align}
An essential point in this formulation is that the loan period, $LLN_{t,v}$ only
influences the value of the objective function when the global discount rate
and the process-specific discount rate take on unique values. The difference between
these two values is known as the ``hurdle rate.'' When the hurdle rate is zero,
$C_{loans}$ is not a function of the loan period. I
do not specify unique discount rates for each process in the following case studies.
Some work suggests that a high discount rate may encourage capital-intensive projects \cite{alzbutas_uncertainty_2012,
decarolis_modelling_2016}, but the effect of hurdle rates is beyond the scope of
this thesis. This objective function is also subject to various constraints that
guarantee energy demands are met at all times and, if specified, ensures respect
of capacity and emissions limits. The details of these constraints and the
mathematical expressions for the other cost terms are available online
\cite{noauthor_preface_nodate}.

\subsubsection{Modeling-to-Generate-Alternatives}
\gls{mga} enables systematic evaluation of structural uncertainty by probing
near-optimal decision space. \gls{temoa} uses the Hop-Skip-Jump (HSJ) algorithm for
\gls{mga}. The steps for HSJ are \cite{decarolis_modelling_2016}:
\begin{enumerate}
  \item obtain an optimal solution by any method,
  \item add a user-specified amount of slack to the objective function value from the
  first step,
  \item use the adjusted objective function value as an upper bound constraint,
  \item generate a new objective function that minimizes the sum of all decision
  variables,
  \item iterate the procedure,
  \item stop the \gls{mga} when no significant changes are observed.
\end{enumerate}
The mathematical formulation of this algorithm is:
\begin{align}
  \intertext{Minimize:}
  p &= \sum_{k\in K} x_k,
  \intertext{Subject to:}
  f_j\left(\vec{x}\right) &\leq T_j \forall j,\\
  \vec{x}&\in X,
  \intertext{where}
  p &= \text{the new objective function}\nonumber,\\
  \displaybreak
  x_k &= \text{the $k^{th}$ decision variable with a nonzero value in previous solutions}\nonumber,\\
  f_j\left(\vec{x}\right) &= \text{the $j^{th}$ original objective function},\nonumber\\
  T_j &= \text{the slack-adjusted target value},\nonumber\\
  X &= \text{the set of all feasible solutions}.\nonumber
\end{align}
This procedure results in a small set of maximally different solutions for
modelers to interpret. In this way, \gls{mga} efficiently explores decision space
to offer alternatives that may capture unmodeled objectives, such as political
expediency or social acceptance.

\subsection{Na\"{i}ve Pathway to 100\% Clean Energy: Linear Programming Example}
\label{section:naive-example}

In this subsection, I present the results of a simple example model.
This example illustrates the influence of both structural and parametric uncertainties
and how \gls{mga} probes decision space. I include this example to give an intuition
for the results in presented in Chapter \ref{chapter:illinois} and Chapter
\ref{chapter:uiuc}. This example also motivates the need for \gls{esom} modeling
when developing science-driven policy. As discussed in Section \ref{section:ceja},
the \gls{ceja} bill promises to make Illinois' electric grid 100\% clean energy
by 2030. According to the Illinois Clean Jobs Coalition, 6.3 GW of additional wind energy
capacity and 17 GW of solar capacity should be enough to achieve this goal while
meeting Illinois' electricity demand \cite{the_accelerate_group_clean_2019}. The
Illinois Clean Jobs Coalition implicitly assumes the following:
\begin{enumerate}
  \item all of Illinois' existing nuclear fleet will be operating in 2030,
  \item all fossil fuel plants will be replaced by wind or solar energy in 2030,
  \item stagnant electricity demand,
  \item zero electricity exports. Illinois currently exports approximately 24\%
  of its total generation \cite{energy_information_administration_eia_nodate}.
\end{enumerate}
In 2019, the electricity demand of Illinois was 138 TWh, and 98.7 TWh generated from
nuclear power \cite{energy_information_administration_eia_nodate}.
Formulated as a linear programming problem, Illinois must:
\begin{align}
  \intertext{Minimize:}
  C_{total} &= C_s x_s + C_w x_w
  \intertext{Subject to:}
  D &\leq \left[CF_s x_s + CF_w x_w\right]\cdot8.76
  \intertext{where}
  x_s &= \text{capacity of solar (first decision variable), $\left[GW\right]$}\nonumber\\
  x_w &= \text{capacity of wind (first decision variable), $\left[GW\right]$}\nonumber\\
  C_s &= \text{cost of solar, $\left[\frac{B\$}{GW}\right]$} \nonumber\\
  C_w &= \text{cost of wind, $\left[\frac{B\$}{GW}\right]$}\nonumber\\
  CF_s &= \text{capacity factor of solar,  $\left[\%\right]$}  \nonumber\\
  CF_w &= \text{capacity factor of wind, $\left[\%\right]$}  \nonumber\\
  D &= \text{remaining electricity demand, $\left[TWh\right]$}\nonumber\\
  8.76 &= \text{a factor that converts power in GW to energy in TWh}\nonumber.
\end{align}
The capacity factor (CF)
is the ratio of energy produced by a generating technology and the energy it could
have produced if it operated 100\% of the time. CF is given by,
\begin{align}
  CF &= \frac{E_{tot}}{P_{np}*8.76}
  \intertext{where}
  E_{tot} &= \text{the total energy produced in a given year, $\left[TWh\right]$} \nonumber\\
  P_{np} &= \text{the nameplate capacity of a generator, $\left[GW\right]$}\nonumber.
\end{align}

Table \ref{tab:naive} summarizes the data for this toy problem.

\begin{table}[H]
  \centering
  \caption{Summary of Solar and Wind Data for the ``Na\"{i}ve Pathway"}
  \label{tab:naive}
  \resizebox{0.75\textwidth}{!}{
  \begin{tabular}{lrrrrrr}
\toprule
       & Cost in 2030 &    CF &  IL Clean Jobs Estimate & Optimal & MGA & Source \\
      &[B\$/GW] & & [GW] & [GW] & [GW] &  \\
\midrule
Demand &                    - &     - &                     4.50 &                     - &                 - &   \cite{energy_information_administration_eia_nodate}      \\
 Solar &                 0.811 &   0.20 &                    17.00 &                  12.4 &             16.65 &    \cite{nrel_2020_2020,the_accelerate_group_clean_2019}     \\
  Wind &                 1.411 &  0.35 &                     6.30 &                  5.71 &              3.28 &   \cite{nrel_2020_2020,the_accelerate_group_clean_2019}     \\
\bottomrule
\end{tabular}

  }
\end{table}

This linear problem only has two decision variables, solar and wind capacity,
which makes it graphically solvable. Figure \ref{fig:mga-fig} shows the optimal
solution and one \gls{mga} iteration to illustrate the changes to decision space.
I plot the \gls{ceja} goals for reference.
Every point above the demand constraint (the dotted blue line) represents a
feasible solution since those
combinations of solar and wind can meet annual electricity demand.
Points above the ``demand constraint'' line, represent an excess of total capacity.
The \gls{ceja} goals are well within the feasible region but highly sub-optimal;
building excess capacity.
This amount of capacity suggests the energy could be exported, curtailed, or stored,
which are unmodeled, structural uncertainties, in this formulation.
The optimal solution minimizes the total cost and lies at the intersection
of the objective function and the demand constraint. The objective function cannot
be reduced further because those solutions would not satisfy the electricity demand.
The \gls{mga} ``objective function'' added a 10\% slack to the original objective
function and became an additional upper bound constraint. The new objective function
minimizes the sum of solar and wind capacity. In this trivial case, the model will
always build solar over wind unless the cost of wind becomes cheaper than the cost of
solar. However, the example illustrates \gls{mga} generally. Finally, although \gls{mga}
scenarios may account for either structural or parametric uncertainties
\cite{decarolis_modelling_2016}, I use the former interpretation because
\gls{mga} cannot specify which uncertain parameters led to a particular scenario.
\begin{figure}[H]
  \centering
  \resizebox{0.8\columnwidth}{!}{%% Creator: Matplotlib, PGF backend
%%
%% To include the figure in your LaTeX document, write
%%   \input{<filename>.pgf}
%%
%% Make sure the required packages are loaded in your preamble
%%   \usepackage{pgf}
%%
%% Figures using additional raster images can only be included by \input if
%% they are in the same directory as the main LaTeX file. For loading figures
%% from other directories you can use the `import` package
%%   \usepackage{import}
%%
%% and then include the figures with
%%   \import{<path to file>}{<filename>.pgf}
%%
%% Matplotlib used the following preamble
%%
\begingroup%
\makeatletter%
\begin{pgfpicture}%
\pgfpathrectangle{\pgfpointorigin}{\pgfqpoint{10.079222in}{8.237269in}}%
\pgfusepath{use as bounding box, clip}%
\begin{pgfscope}%
\pgfsetbuttcap%
\pgfsetmiterjoin%
\definecolor{currentfill}{rgb}{1.000000,1.000000,1.000000}%
\pgfsetfillcolor{currentfill}%
\pgfsetlinewidth{0.000000pt}%
\definecolor{currentstroke}{rgb}{0.000000,0.000000,0.000000}%
\pgfsetstrokecolor{currentstroke}%
\pgfsetdash{}{0pt}%
\pgfpathmoveto{\pgfqpoint{0.000000in}{0.000000in}}%
\pgfpathlineto{\pgfqpoint{10.079222in}{0.000000in}}%
\pgfpathlineto{\pgfqpoint{10.079222in}{8.237269in}}%
\pgfpathlineto{\pgfqpoint{0.000000in}{8.237269in}}%
\pgfpathclose%
\pgfusepath{fill}%
\end{pgfscope}%
\begin{pgfscope}%
\pgfsetbuttcap%
\pgfsetmiterjoin%
\definecolor{currentfill}{rgb}{0.827451,0.827451,0.827451}%
\pgfsetfillcolor{currentfill}%
\pgfsetlinewidth{0.000000pt}%
\definecolor{currentstroke}{rgb}{0.000000,0.000000,0.000000}%
\pgfsetstrokecolor{currentstroke}%
\pgfsetstrokeopacity{0.000000}%
\pgfsetdash{}{0pt}%
\pgfpathmoveto{\pgfqpoint{0.679222in}{0.663790in}}%
\pgfpathlineto{\pgfqpoint{9.979222in}{0.663790in}}%
\pgfpathlineto{\pgfqpoint{9.979222in}{7.458790in}}%
\pgfpathlineto{\pgfqpoint{0.679222in}{7.458790in}}%
\pgfpathclose%
\pgfusepath{fill}%
\end{pgfscope}%
\begin{pgfscope}%
\pgfpathrectangle{\pgfqpoint{0.679222in}{0.663790in}}{\pgfqpoint{9.300000in}{6.795000in}}%
\pgfusepath{clip}%
\pgfsetbuttcap%
\pgfsetroundjoin%
\definecolor{currentfill}{rgb}{0.121569,0.466667,0.705882}%
\pgfsetfillcolor{currentfill}%
\pgfsetfillopacity{0.200000}%
\pgfsetlinewidth{0.000000pt}%
\definecolor{currentstroke}{rgb}{0.000000,0.000000,0.000000}%
\pgfsetstrokecolor{currentstroke}%
\pgfsetdash{}{0pt}%
\pgfpathmoveto{\pgfqpoint{9.979222in}{7.458790in}}%
\pgfpathlineto{\pgfqpoint{9.979222in}{0.663790in}}%
\pgfpathlineto{\pgfqpoint{0.679222in}{7.458790in}}%
\pgfpathlineto{\pgfqpoint{0.679222in}{7.458790in}}%
\pgfpathlineto{\pgfqpoint{0.679222in}{7.458790in}}%
\pgfpathlineto{\pgfqpoint{9.979222in}{7.458790in}}%
\pgfpathclose%
\pgfusepath{fill}%
\end{pgfscope}%
\begin{pgfscope}%
\pgfpathrectangle{\pgfqpoint{0.679222in}{0.663790in}}{\pgfqpoint{9.300000in}{6.795000in}}%
\pgfusepath{clip}%
\pgfsetbuttcap%
\pgfsetroundjoin%
\definecolor{currentfill}{rgb}{0.839216,0.152941,0.156863}%
\pgfsetfillcolor{currentfill}%
\pgfsetlinewidth{1.003750pt}%
\definecolor{currentstroke}{rgb}{0.839216,0.152941,0.156863}%
\pgfsetstrokecolor{currentstroke}%
\pgfsetdash{}{0pt}%
\pgfsys@defobject{currentmarker}{\pgfqpoint{-0.065881in}{-0.065881in}}{\pgfqpoint{0.065881in}{0.065881in}}{%
\pgfpathmoveto{\pgfqpoint{0.000000in}{-0.065881in}}%
\pgfpathcurveto{\pgfqpoint{0.017472in}{-0.065881in}}{\pgfqpoint{0.034230in}{-0.058939in}}{\pgfqpoint{0.046585in}{-0.046585in}}%
\pgfpathcurveto{\pgfqpoint{0.058939in}{-0.034230in}}{\pgfqpoint{0.065881in}{-0.017472in}}{\pgfqpoint{0.065881in}{0.000000in}}%
\pgfpathcurveto{\pgfqpoint{0.065881in}{0.017472in}}{\pgfqpoint{0.058939in}{0.034230in}}{\pgfqpoint{0.046585in}{0.046585in}}%
\pgfpathcurveto{\pgfqpoint{0.034230in}{0.058939in}}{\pgfqpoint{0.017472in}{0.065881in}}{\pgfqpoint{0.000000in}{0.065881in}}%
\pgfpathcurveto{\pgfqpoint{-0.017472in}{0.065881in}}{\pgfqpoint{-0.034230in}{0.058939in}}{\pgfqpoint{-0.046585in}{0.046585in}}%
\pgfpathcurveto{\pgfqpoint{-0.058939in}{0.034230in}}{\pgfqpoint{-0.065881in}{0.017472in}}{\pgfqpoint{-0.065881in}{0.000000in}}%
\pgfpathcurveto{\pgfqpoint{-0.065881in}{-0.017472in}}{\pgfqpoint{-0.058939in}{-0.034230in}}{\pgfqpoint{-0.046585in}{-0.046585in}}%
\pgfpathcurveto{\pgfqpoint{-0.034230in}{-0.058939in}}{\pgfqpoint{-0.017472in}{-0.065881in}}{\pgfqpoint{0.000000in}{-0.065881in}}%
\pgfpathclose%
\pgfusepath{stroke,fill}%
}%
\begin{pgfscope}%
\pgfsys@transformshift{7.727344in}{4.003506in}%
\pgfsys@useobject{currentmarker}{}%
\end{pgfscope}%
\end{pgfscope}%
\begin{pgfscope}%
\pgfpathrectangle{\pgfqpoint{0.679222in}{0.663790in}}{\pgfqpoint{9.300000in}{6.795000in}}%
\pgfusepath{clip}%
\pgfsetbuttcap%
\pgfsetroundjoin%
\definecolor{currentfill}{rgb}{1.000000,1.000000,1.000000}%
\pgfsetfillcolor{currentfill}%
\pgfsetlinewidth{1.003750pt}%
\definecolor{currentstroke}{rgb}{0.000000,0.000000,0.000000}%
\pgfsetstrokecolor{currentstroke}%
\pgfsetdash{}{0pt}%
\pgfsys@defobject{currentmarker}{\pgfqpoint{-0.065881in}{-0.065881in}}{\pgfqpoint{0.065881in}{0.065881in}}{%
\pgfpathmoveto{\pgfqpoint{0.000000in}{-0.065881in}}%
\pgfpathcurveto{\pgfqpoint{0.017472in}{-0.065881in}}{\pgfqpoint{0.034230in}{-0.058939in}}{\pgfqpoint{0.046585in}{-0.046585in}}%
\pgfpathcurveto{\pgfqpoint{0.058939in}{-0.034230in}}{\pgfqpoint{0.065881in}{-0.017472in}}{\pgfqpoint{0.065881in}{0.000000in}}%
\pgfpathcurveto{\pgfqpoint{0.065881in}{0.017472in}}{\pgfqpoint{0.058939in}{0.034230in}}{\pgfqpoint{0.046585in}{0.046585in}}%
\pgfpathcurveto{\pgfqpoint{0.034230in}{0.058939in}}{\pgfqpoint{0.017472in}{0.065881in}}{\pgfqpoint{0.000000in}{0.065881in}}%
\pgfpathcurveto{\pgfqpoint{-0.017472in}{0.065881in}}{\pgfqpoint{-0.034230in}{0.058939in}}{\pgfqpoint{-0.046585in}{0.046585in}}%
\pgfpathcurveto{\pgfqpoint{-0.058939in}{0.034230in}}{\pgfqpoint{-0.065881in}{0.017472in}}{\pgfqpoint{-0.065881in}{0.000000in}}%
\pgfpathcurveto{\pgfqpoint{-0.065881in}{-0.017472in}}{\pgfqpoint{-0.058939in}{-0.034230in}}{\pgfqpoint{-0.046585in}{-0.046585in}}%
\pgfpathcurveto{\pgfqpoint{-0.034230in}{-0.058939in}}{\pgfqpoint{-0.017472in}{-0.065881in}}{\pgfqpoint{0.000000in}{-0.065881in}}%
\pgfpathclose%
\pgfusepath{stroke,fill}%
}%
\begin{pgfscope}%
\pgfsys@transformshift{7.582236in}{2.402563in}%
\pgfsys@useobject{currentmarker}{}%
\end{pgfscope}%
\end{pgfscope}%
\begin{pgfscope}%
\pgfpathrectangle{\pgfqpoint{0.679222in}{0.663790in}}{\pgfqpoint{9.300000in}{6.795000in}}%
\pgfusepath{clip}%
\pgfsetbuttcap%
\pgfsetroundjoin%
\definecolor{currentfill}{rgb}{0.121569,0.466667,0.705882}%
\pgfsetfillcolor{currentfill}%
\pgfsetlinewidth{1.003750pt}%
\definecolor{currentstroke}{rgb}{0.121569,0.466667,0.705882}%
\pgfsetstrokecolor{currentstroke}%
\pgfsetdash{}{0pt}%
\pgfsys@defobject{currentmarker}{\pgfqpoint{-0.065881in}{-0.065881in}}{\pgfqpoint{0.065881in}{0.065881in}}{%
\pgfpathmoveto{\pgfqpoint{0.000000in}{-0.065881in}}%
\pgfpathcurveto{\pgfqpoint{0.017472in}{-0.065881in}}{\pgfqpoint{0.034230in}{-0.058939in}}{\pgfqpoint{0.046585in}{-0.046585in}}%
\pgfpathcurveto{\pgfqpoint{0.058939in}{-0.034230in}}{\pgfqpoint{0.065881in}{-0.017472in}}{\pgfqpoint{0.065881in}{0.000000in}}%
\pgfpathcurveto{\pgfqpoint{0.065881in}{0.017472in}}{\pgfqpoint{0.058939in}{0.034230in}}{\pgfqpoint{0.046585in}{0.046585in}}%
\pgfpathcurveto{\pgfqpoint{0.034230in}{0.058939in}}{\pgfqpoint{0.017472in}{0.065881in}}{\pgfqpoint{0.000000in}{0.065881in}}%
\pgfpathcurveto{\pgfqpoint{-0.017472in}{0.065881in}}{\pgfqpoint{-0.034230in}{0.058939in}}{\pgfqpoint{-0.046585in}{0.046585in}}%
\pgfpathcurveto{\pgfqpoint{-0.058939in}{0.034230in}}{\pgfqpoint{-0.065881in}{0.017472in}}{\pgfqpoint{-0.065881in}{0.000000in}}%
\pgfpathcurveto{\pgfqpoint{-0.065881in}{-0.017472in}}{\pgfqpoint{-0.058939in}{-0.034230in}}{\pgfqpoint{-0.046585in}{-0.046585in}}%
\pgfpathcurveto{\pgfqpoint{-0.034230in}{-0.058939in}}{\pgfqpoint{-0.017472in}{-0.065881in}}{\pgfqpoint{0.000000in}{-0.065881in}}%
\pgfpathclose%
\pgfusepath{stroke,fill}%
}%
\begin{pgfscope}%
\pgfsys@transformshift{5.820206in}{3.690739in}%
\pgfsys@useobject{currentmarker}{}%
\end{pgfscope}%
\end{pgfscope}%
\begin{pgfscope}%
\pgfpathrectangle{\pgfqpoint{0.679222in}{0.663790in}}{\pgfqpoint{9.300000in}{6.795000in}}%
\pgfusepath{clip}%
\pgfsetrectcap%
\pgfsetroundjoin%
\pgfsetlinewidth{0.803000pt}%
\definecolor{currentstroke}{rgb}{0.000000,0.000000,0.000000}%
\pgfsetstrokecolor{currentstroke}%
\pgfsetstrokeopacity{0.400000}%
\pgfsetdash{}{0pt}%
\pgfpathmoveto{\pgfqpoint{0.679222in}{0.663790in}}%
\pgfpathlineto{\pgfqpoint{0.679222in}{7.458790in}}%
\pgfusepath{stroke}%
\end{pgfscope}%
\begin{pgfscope}%
\pgfsetbuttcap%
\pgfsetroundjoin%
\definecolor{currentfill}{rgb}{0.000000,0.000000,0.000000}%
\pgfsetfillcolor{currentfill}%
\pgfsetlinewidth{0.803000pt}%
\definecolor{currentstroke}{rgb}{0.000000,0.000000,0.000000}%
\pgfsetstrokecolor{currentstroke}%
\pgfsetdash{}{0pt}%
\pgfsys@defobject{currentmarker}{\pgfqpoint{0.000000in}{-0.048611in}}{\pgfqpoint{0.000000in}{0.000000in}}{%
\pgfpathmoveto{\pgfqpoint{0.000000in}{0.000000in}}%
\pgfpathlineto{\pgfqpoint{0.000000in}{-0.048611in}}%
\pgfusepath{stroke,fill}%
}%
\begin{pgfscope}%
\pgfsys@transformshift{0.679222in}{0.663790in}%
\pgfsys@useobject{currentmarker}{}%
\end{pgfscope}%
\end{pgfscope}%
\begin{pgfscope}%
\definecolor{textcolor}{rgb}{0.000000,0.000000,0.000000}%
\pgfsetstrokecolor{textcolor}%
\pgfsetfillcolor{textcolor}%
\pgftext[x=0.679222in,y=0.566568in,,top]{\color{textcolor}\rmfamily\fontsize{10.000000}{12.000000}\selectfont \(\displaystyle {0.0}\)}%
\end{pgfscope}%
\begin{pgfscope}%
\pgfpathrectangle{\pgfqpoint{0.679222in}{0.663790in}}{\pgfqpoint{9.300000in}{6.795000in}}%
\pgfusepath{clip}%
\pgfsetrectcap%
\pgfsetroundjoin%
\pgfsetlinewidth{0.803000pt}%
\definecolor{currentstroke}{rgb}{0.000000,0.000000,0.000000}%
\pgfsetstrokecolor{currentstroke}%
\pgfsetstrokeopacity{0.400000}%
\pgfsetdash{}{0pt}%
\pgfpathmoveto{\pgfqpoint{1.715711in}{0.663790in}}%
\pgfpathlineto{\pgfqpoint{1.715711in}{7.458790in}}%
\pgfusepath{stroke}%
\end{pgfscope}%
\begin{pgfscope}%
\pgfsetbuttcap%
\pgfsetroundjoin%
\definecolor{currentfill}{rgb}{0.000000,0.000000,0.000000}%
\pgfsetfillcolor{currentfill}%
\pgfsetlinewidth{0.803000pt}%
\definecolor{currentstroke}{rgb}{0.000000,0.000000,0.000000}%
\pgfsetstrokecolor{currentstroke}%
\pgfsetdash{}{0pt}%
\pgfsys@defobject{currentmarker}{\pgfqpoint{0.000000in}{-0.048611in}}{\pgfqpoint{0.000000in}{0.000000in}}{%
\pgfpathmoveto{\pgfqpoint{0.000000in}{0.000000in}}%
\pgfpathlineto{\pgfqpoint{0.000000in}{-0.048611in}}%
\pgfusepath{stroke,fill}%
}%
\begin{pgfscope}%
\pgfsys@transformshift{1.715711in}{0.663790in}%
\pgfsys@useobject{currentmarker}{}%
\end{pgfscope}%
\end{pgfscope}%
\begin{pgfscope}%
\definecolor{textcolor}{rgb}{0.000000,0.000000,0.000000}%
\pgfsetstrokecolor{textcolor}%
\pgfsetfillcolor{textcolor}%
\pgftext[x=1.715711in,y=0.566568in,,top]{\color{textcolor}\rmfamily\fontsize{10.000000}{12.000000}\selectfont \(\displaystyle {2.5}\)}%
\end{pgfscope}%
\begin{pgfscope}%
\pgfpathrectangle{\pgfqpoint{0.679222in}{0.663790in}}{\pgfqpoint{9.300000in}{6.795000in}}%
\pgfusepath{clip}%
\pgfsetrectcap%
\pgfsetroundjoin%
\pgfsetlinewidth{0.803000pt}%
\definecolor{currentstroke}{rgb}{0.000000,0.000000,0.000000}%
\pgfsetstrokecolor{currentstroke}%
\pgfsetstrokeopacity{0.400000}%
\pgfsetdash{}{0pt}%
\pgfpathmoveto{\pgfqpoint{2.752199in}{0.663790in}}%
\pgfpathlineto{\pgfqpoint{2.752199in}{7.458790in}}%
\pgfusepath{stroke}%
\end{pgfscope}%
\begin{pgfscope}%
\pgfsetbuttcap%
\pgfsetroundjoin%
\definecolor{currentfill}{rgb}{0.000000,0.000000,0.000000}%
\pgfsetfillcolor{currentfill}%
\pgfsetlinewidth{0.803000pt}%
\definecolor{currentstroke}{rgb}{0.000000,0.000000,0.000000}%
\pgfsetstrokecolor{currentstroke}%
\pgfsetdash{}{0pt}%
\pgfsys@defobject{currentmarker}{\pgfqpoint{0.000000in}{-0.048611in}}{\pgfqpoint{0.000000in}{0.000000in}}{%
\pgfpathmoveto{\pgfqpoint{0.000000in}{0.000000in}}%
\pgfpathlineto{\pgfqpoint{0.000000in}{-0.048611in}}%
\pgfusepath{stroke,fill}%
}%
\begin{pgfscope}%
\pgfsys@transformshift{2.752199in}{0.663790in}%
\pgfsys@useobject{currentmarker}{}%
\end{pgfscope}%
\end{pgfscope}%
\begin{pgfscope}%
\definecolor{textcolor}{rgb}{0.000000,0.000000,0.000000}%
\pgfsetstrokecolor{textcolor}%
\pgfsetfillcolor{textcolor}%
\pgftext[x=2.752199in,y=0.566568in,,top]{\color{textcolor}\rmfamily\fontsize{10.000000}{12.000000}\selectfont \(\displaystyle {5.0}\)}%
\end{pgfscope}%
\begin{pgfscope}%
\pgfpathrectangle{\pgfqpoint{0.679222in}{0.663790in}}{\pgfqpoint{9.300000in}{6.795000in}}%
\pgfusepath{clip}%
\pgfsetrectcap%
\pgfsetroundjoin%
\pgfsetlinewidth{0.803000pt}%
\definecolor{currentstroke}{rgb}{0.000000,0.000000,0.000000}%
\pgfsetstrokecolor{currentstroke}%
\pgfsetstrokeopacity{0.400000}%
\pgfsetdash{}{0pt}%
\pgfpathmoveto{\pgfqpoint{3.788688in}{0.663790in}}%
\pgfpathlineto{\pgfqpoint{3.788688in}{7.458790in}}%
\pgfusepath{stroke}%
\end{pgfscope}%
\begin{pgfscope}%
\pgfsetbuttcap%
\pgfsetroundjoin%
\definecolor{currentfill}{rgb}{0.000000,0.000000,0.000000}%
\pgfsetfillcolor{currentfill}%
\pgfsetlinewidth{0.803000pt}%
\definecolor{currentstroke}{rgb}{0.000000,0.000000,0.000000}%
\pgfsetstrokecolor{currentstroke}%
\pgfsetdash{}{0pt}%
\pgfsys@defobject{currentmarker}{\pgfqpoint{0.000000in}{-0.048611in}}{\pgfqpoint{0.000000in}{0.000000in}}{%
\pgfpathmoveto{\pgfqpoint{0.000000in}{0.000000in}}%
\pgfpathlineto{\pgfqpoint{0.000000in}{-0.048611in}}%
\pgfusepath{stroke,fill}%
}%
\begin{pgfscope}%
\pgfsys@transformshift{3.788688in}{0.663790in}%
\pgfsys@useobject{currentmarker}{}%
\end{pgfscope}%
\end{pgfscope}%
\begin{pgfscope}%
\definecolor{textcolor}{rgb}{0.000000,0.000000,0.000000}%
\pgfsetstrokecolor{textcolor}%
\pgfsetfillcolor{textcolor}%
\pgftext[x=3.788688in,y=0.566568in,,top]{\color{textcolor}\rmfamily\fontsize{10.000000}{12.000000}\selectfont \(\displaystyle {7.5}\)}%
\end{pgfscope}%
\begin{pgfscope}%
\pgfpathrectangle{\pgfqpoint{0.679222in}{0.663790in}}{\pgfqpoint{9.300000in}{6.795000in}}%
\pgfusepath{clip}%
\pgfsetrectcap%
\pgfsetroundjoin%
\pgfsetlinewidth{0.803000pt}%
\definecolor{currentstroke}{rgb}{0.000000,0.000000,0.000000}%
\pgfsetstrokecolor{currentstroke}%
\pgfsetstrokeopacity{0.400000}%
\pgfsetdash{}{0pt}%
\pgfpathmoveto{\pgfqpoint{4.825177in}{0.663790in}}%
\pgfpathlineto{\pgfqpoint{4.825177in}{7.458790in}}%
\pgfusepath{stroke}%
\end{pgfscope}%
\begin{pgfscope}%
\pgfsetbuttcap%
\pgfsetroundjoin%
\definecolor{currentfill}{rgb}{0.000000,0.000000,0.000000}%
\pgfsetfillcolor{currentfill}%
\pgfsetlinewidth{0.803000pt}%
\definecolor{currentstroke}{rgb}{0.000000,0.000000,0.000000}%
\pgfsetstrokecolor{currentstroke}%
\pgfsetdash{}{0pt}%
\pgfsys@defobject{currentmarker}{\pgfqpoint{0.000000in}{-0.048611in}}{\pgfqpoint{0.000000in}{0.000000in}}{%
\pgfpathmoveto{\pgfqpoint{0.000000in}{0.000000in}}%
\pgfpathlineto{\pgfqpoint{0.000000in}{-0.048611in}}%
\pgfusepath{stroke,fill}%
}%
\begin{pgfscope}%
\pgfsys@transformshift{4.825177in}{0.663790in}%
\pgfsys@useobject{currentmarker}{}%
\end{pgfscope}%
\end{pgfscope}%
\begin{pgfscope}%
\definecolor{textcolor}{rgb}{0.000000,0.000000,0.000000}%
\pgfsetstrokecolor{textcolor}%
\pgfsetfillcolor{textcolor}%
\pgftext[x=4.825177in,y=0.566568in,,top]{\color{textcolor}\rmfamily\fontsize{10.000000}{12.000000}\selectfont \(\displaystyle {10.0}\)}%
\end{pgfscope}%
\begin{pgfscope}%
\pgfpathrectangle{\pgfqpoint{0.679222in}{0.663790in}}{\pgfqpoint{9.300000in}{6.795000in}}%
\pgfusepath{clip}%
\pgfsetrectcap%
\pgfsetroundjoin%
\pgfsetlinewidth{0.803000pt}%
\definecolor{currentstroke}{rgb}{0.000000,0.000000,0.000000}%
\pgfsetstrokecolor{currentstroke}%
\pgfsetstrokeopacity{0.400000}%
\pgfsetdash{}{0pt}%
\pgfpathmoveto{\pgfqpoint{5.861665in}{0.663790in}}%
\pgfpathlineto{\pgfqpoint{5.861665in}{7.458790in}}%
\pgfusepath{stroke}%
\end{pgfscope}%
\begin{pgfscope}%
\pgfsetbuttcap%
\pgfsetroundjoin%
\definecolor{currentfill}{rgb}{0.000000,0.000000,0.000000}%
\pgfsetfillcolor{currentfill}%
\pgfsetlinewidth{0.803000pt}%
\definecolor{currentstroke}{rgb}{0.000000,0.000000,0.000000}%
\pgfsetstrokecolor{currentstroke}%
\pgfsetdash{}{0pt}%
\pgfsys@defobject{currentmarker}{\pgfqpoint{0.000000in}{-0.048611in}}{\pgfqpoint{0.000000in}{0.000000in}}{%
\pgfpathmoveto{\pgfqpoint{0.000000in}{0.000000in}}%
\pgfpathlineto{\pgfqpoint{0.000000in}{-0.048611in}}%
\pgfusepath{stroke,fill}%
}%
\begin{pgfscope}%
\pgfsys@transformshift{5.861665in}{0.663790in}%
\pgfsys@useobject{currentmarker}{}%
\end{pgfscope}%
\end{pgfscope}%
\begin{pgfscope}%
\definecolor{textcolor}{rgb}{0.000000,0.000000,0.000000}%
\pgfsetstrokecolor{textcolor}%
\pgfsetfillcolor{textcolor}%
\pgftext[x=5.861665in,y=0.566568in,,top]{\color{textcolor}\rmfamily\fontsize{10.000000}{12.000000}\selectfont \(\displaystyle {12.5}\)}%
\end{pgfscope}%
\begin{pgfscope}%
\pgfpathrectangle{\pgfqpoint{0.679222in}{0.663790in}}{\pgfqpoint{9.300000in}{6.795000in}}%
\pgfusepath{clip}%
\pgfsetrectcap%
\pgfsetroundjoin%
\pgfsetlinewidth{0.803000pt}%
\definecolor{currentstroke}{rgb}{0.000000,0.000000,0.000000}%
\pgfsetstrokecolor{currentstroke}%
\pgfsetstrokeopacity{0.400000}%
\pgfsetdash{}{0pt}%
\pgfpathmoveto{\pgfqpoint{6.898154in}{0.663790in}}%
\pgfpathlineto{\pgfqpoint{6.898154in}{7.458790in}}%
\pgfusepath{stroke}%
\end{pgfscope}%
\begin{pgfscope}%
\pgfsetbuttcap%
\pgfsetroundjoin%
\definecolor{currentfill}{rgb}{0.000000,0.000000,0.000000}%
\pgfsetfillcolor{currentfill}%
\pgfsetlinewidth{0.803000pt}%
\definecolor{currentstroke}{rgb}{0.000000,0.000000,0.000000}%
\pgfsetstrokecolor{currentstroke}%
\pgfsetdash{}{0pt}%
\pgfsys@defobject{currentmarker}{\pgfqpoint{0.000000in}{-0.048611in}}{\pgfqpoint{0.000000in}{0.000000in}}{%
\pgfpathmoveto{\pgfqpoint{0.000000in}{0.000000in}}%
\pgfpathlineto{\pgfqpoint{0.000000in}{-0.048611in}}%
\pgfusepath{stroke,fill}%
}%
\begin{pgfscope}%
\pgfsys@transformshift{6.898154in}{0.663790in}%
\pgfsys@useobject{currentmarker}{}%
\end{pgfscope}%
\end{pgfscope}%
\begin{pgfscope}%
\definecolor{textcolor}{rgb}{0.000000,0.000000,0.000000}%
\pgfsetstrokecolor{textcolor}%
\pgfsetfillcolor{textcolor}%
\pgftext[x=6.898154in,y=0.566568in,,top]{\color{textcolor}\rmfamily\fontsize{10.000000}{12.000000}\selectfont \(\displaystyle {15.0}\)}%
\end{pgfscope}%
\begin{pgfscope}%
\pgfpathrectangle{\pgfqpoint{0.679222in}{0.663790in}}{\pgfqpoint{9.300000in}{6.795000in}}%
\pgfusepath{clip}%
\pgfsetrectcap%
\pgfsetroundjoin%
\pgfsetlinewidth{0.803000pt}%
\definecolor{currentstroke}{rgb}{0.000000,0.000000,0.000000}%
\pgfsetstrokecolor{currentstroke}%
\pgfsetstrokeopacity{0.400000}%
\pgfsetdash{}{0pt}%
\pgfpathmoveto{\pgfqpoint{7.934642in}{0.663790in}}%
\pgfpathlineto{\pgfqpoint{7.934642in}{7.458790in}}%
\pgfusepath{stroke}%
\end{pgfscope}%
\begin{pgfscope}%
\pgfsetbuttcap%
\pgfsetroundjoin%
\definecolor{currentfill}{rgb}{0.000000,0.000000,0.000000}%
\pgfsetfillcolor{currentfill}%
\pgfsetlinewidth{0.803000pt}%
\definecolor{currentstroke}{rgb}{0.000000,0.000000,0.000000}%
\pgfsetstrokecolor{currentstroke}%
\pgfsetdash{}{0pt}%
\pgfsys@defobject{currentmarker}{\pgfqpoint{0.000000in}{-0.048611in}}{\pgfqpoint{0.000000in}{0.000000in}}{%
\pgfpathmoveto{\pgfqpoint{0.000000in}{0.000000in}}%
\pgfpathlineto{\pgfqpoint{0.000000in}{-0.048611in}}%
\pgfusepath{stroke,fill}%
}%
\begin{pgfscope}%
\pgfsys@transformshift{7.934642in}{0.663790in}%
\pgfsys@useobject{currentmarker}{}%
\end{pgfscope}%
\end{pgfscope}%
\begin{pgfscope}%
\definecolor{textcolor}{rgb}{0.000000,0.000000,0.000000}%
\pgfsetstrokecolor{textcolor}%
\pgfsetfillcolor{textcolor}%
\pgftext[x=7.934642in,y=0.566568in,,top]{\color{textcolor}\rmfamily\fontsize{10.000000}{12.000000}\selectfont \(\displaystyle {17.5}\)}%
\end{pgfscope}%
\begin{pgfscope}%
\pgfpathrectangle{\pgfqpoint{0.679222in}{0.663790in}}{\pgfqpoint{9.300000in}{6.795000in}}%
\pgfusepath{clip}%
\pgfsetrectcap%
\pgfsetroundjoin%
\pgfsetlinewidth{0.803000pt}%
\definecolor{currentstroke}{rgb}{0.000000,0.000000,0.000000}%
\pgfsetstrokecolor{currentstroke}%
\pgfsetstrokeopacity{0.400000}%
\pgfsetdash{}{0pt}%
\pgfpathmoveto{\pgfqpoint{8.971131in}{0.663790in}}%
\pgfpathlineto{\pgfqpoint{8.971131in}{7.458790in}}%
\pgfusepath{stroke}%
\end{pgfscope}%
\begin{pgfscope}%
\pgfsetbuttcap%
\pgfsetroundjoin%
\definecolor{currentfill}{rgb}{0.000000,0.000000,0.000000}%
\pgfsetfillcolor{currentfill}%
\pgfsetlinewidth{0.803000pt}%
\definecolor{currentstroke}{rgb}{0.000000,0.000000,0.000000}%
\pgfsetstrokecolor{currentstroke}%
\pgfsetdash{}{0pt}%
\pgfsys@defobject{currentmarker}{\pgfqpoint{0.000000in}{-0.048611in}}{\pgfqpoint{0.000000in}{0.000000in}}{%
\pgfpathmoveto{\pgfqpoint{0.000000in}{0.000000in}}%
\pgfpathlineto{\pgfqpoint{0.000000in}{-0.048611in}}%
\pgfusepath{stroke,fill}%
}%
\begin{pgfscope}%
\pgfsys@transformshift{8.971131in}{0.663790in}%
\pgfsys@useobject{currentmarker}{}%
\end{pgfscope}%
\end{pgfscope}%
\begin{pgfscope}%
\definecolor{textcolor}{rgb}{0.000000,0.000000,0.000000}%
\pgfsetstrokecolor{textcolor}%
\pgfsetfillcolor{textcolor}%
\pgftext[x=8.971131in,y=0.566568in,,top]{\color{textcolor}\rmfamily\fontsize{10.000000}{12.000000}\selectfont \(\displaystyle {20.0}\)}%
\end{pgfscope}%
\begin{pgfscope}%
\pgfpathrectangle{\pgfqpoint{0.679222in}{0.663790in}}{\pgfqpoint{9.300000in}{6.795000in}}%
\pgfusepath{clip}%
\pgfsetbuttcap%
\pgfsetroundjoin%
\pgfsetlinewidth{0.803000pt}%
\definecolor{currentstroke}{rgb}{0.501961,0.501961,0.501961}%
\pgfsetstrokecolor{currentstroke}%
\pgfsetstrokeopacity{0.400000}%
\pgfsetdash{{2.960000pt}{1.280000pt}}{0.000000pt}%
\pgfpathmoveto{\pgfqpoint{0.886520in}{0.663790in}}%
\pgfpathlineto{\pgfqpoint{0.886520in}{7.458790in}}%
\pgfusepath{stroke}%
\end{pgfscope}%
\begin{pgfscope}%
\pgfsetbuttcap%
\pgfsetroundjoin%
\definecolor{currentfill}{rgb}{0.000000,0.000000,0.000000}%
\pgfsetfillcolor{currentfill}%
\pgfsetlinewidth{0.602250pt}%
\definecolor{currentstroke}{rgb}{0.000000,0.000000,0.000000}%
\pgfsetstrokecolor{currentstroke}%
\pgfsetdash{}{0pt}%
\pgfsys@defobject{currentmarker}{\pgfqpoint{0.000000in}{-0.027778in}}{\pgfqpoint{0.000000in}{0.000000in}}{%
\pgfpathmoveto{\pgfqpoint{0.000000in}{0.000000in}}%
\pgfpathlineto{\pgfqpoint{0.000000in}{-0.027778in}}%
\pgfusepath{stroke,fill}%
}%
\begin{pgfscope}%
\pgfsys@transformshift{0.886520in}{0.663790in}%
\pgfsys@useobject{currentmarker}{}%
\end{pgfscope}%
\end{pgfscope}%
\begin{pgfscope}%
\pgfpathrectangle{\pgfqpoint{0.679222in}{0.663790in}}{\pgfqpoint{9.300000in}{6.795000in}}%
\pgfusepath{clip}%
\pgfsetbuttcap%
\pgfsetroundjoin%
\pgfsetlinewidth{0.803000pt}%
\definecolor{currentstroke}{rgb}{0.501961,0.501961,0.501961}%
\pgfsetstrokecolor{currentstroke}%
\pgfsetstrokeopacity{0.400000}%
\pgfsetdash{{2.960000pt}{1.280000pt}}{0.000000pt}%
\pgfpathmoveto{\pgfqpoint{1.093818in}{0.663790in}}%
\pgfpathlineto{\pgfqpoint{1.093818in}{7.458790in}}%
\pgfusepath{stroke}%
\end{pgfscope}%
\begin{pgfscope}%
\pgfsetbuttcap%
\pgfsetroundjoin%
\definecolor{currentfill}{rgb}{0.000000,0.000000,0.000000}%
\pgfsetfillcolor{currentfill}%
\pgfsetlinewidth{0.602250pt}%
\definecolor{currentstroke}{rgb}{0.000000,0.000000,0.000000}%
\pgfsetstrokecolor{currentstroke}%
\pgfsetdash{}{0pt}%
\pgfsys@defobject{currentmarker}{\pgfqpoint{0.000000in}{-0.027778in}}{\pgfqpoint{0.000000in}{0.000000in}}{%
\pgfpathmoveto{\pgfqpoint{0.000000in}{0.000000in}}%
\pgfpathlineto{\pgfqpoint{0.000000in}{-0.027778in}}%
\pgfusepath{stroke,fill}%
}%
\begin{pgfscope}%
\pgfsys@transformshift{1.093818in}{0.663790in}%
\pgfsys@useobject{currentmarker}{}%
\end{pgfscope}%
\end{pgfscope}%
\begin{pgfscope}%
\pgfpathrectangle{\pgfqpoint{0.679222in}{0.663790in}}{\pgfqpoint{9.300000in}{6.795000in}}%
\pgfusepath{clip}%
\pgfsetbuttcap%
\pgfsetroundjoin%
\pgfsetlinewidth{0.803000pt}%
\definecolor{currentstroke}{rgb}{0.501961,0.501961,0.501961}%
\pgfsetstrokecolor{currentstroke}%
\pgfsetstrokeopacity{0.400000}%
\pgfsetdash{{2.960000pt}{1.280000pt}}{0.000000pt}%
\pgfpathmoveto{\pgfqpoint{1.301115in}{0.663790in}}%
\pgfpathlineto{\pgfqpoint{1.301115in}{7.458790in}}%
\pgfusepath{stroke}%
\end{pgfscope}%
\begin{pgfscope}%
\pgfsetbuttcap%
\pgfsetroundjoin%
\definecolor{currentfill}{rgb}{0.000000,0.000000,0.000000}%
\pgfsetfillcolor{currentfill}%
\pgfsetlinewidth{0.602250pt}%
\definecolor{currentstroke}{rgb}{0.000000,0.000000,0.000000}%
\pgfsetstrokecolor{currentstroke}%
\pgfsetdash{}{0pt}%
\pgfsys@defobject{currentmarker}{\pgfqpoint{0.000000in}{-0.027778in}}{\pgfqpoint{0.000000in}{0.000000in}}{%
\pgfpathmoveto{\pgfqpoint{0.000000in}{0.000000in}}%
\pgfpathlineto{\pgfqpoint{0.000000in}{-0.027778in}}%
\pgfusepath{stroke,fill}%
}%
\begin{pgfscope}%
\pgfsys@transformshift{1.301115in}{0.663790in}%
\pgfsys@useobject{currentmarker}{}%
\end{pgfscope}%
\end{pgfscope}%
\begin{pgfscope}%
\pgfpathrectangle{\pgfqpoint{0.679222in}{0.663790in}}{\pgfqpoint{9.300000in}{6.795000in}}%
\pgfusepath{clip}%
\pgfsetbuttcap%
\pgfsetroundjoin%
\pgfsetlinewidth{0.803000pt}%
\definecolor{currentstroke}{rgb}{0.501961,0.501961,0.501961}%
\pgfsetstrokecolor{currentstroke}%
\pgfsetstrokeopacity{0.400000}%
\pgfsetdash{{2.960000pt}{1.280000pt}}{0.000000pt}%
\pgfpathmoveto{\pgfqpoint{1.508413in}{0.663790in}}%
\pgfpathlineto{\pgfqpoint{1.508413in}{7.458790in}}%
\pgfusepath{stroke}%
\end{pgfscope}%
\begin{pgfscope}%
\pgfsetbuttcap%
\pgfsetroundjoin%
\definecolor{currentfill}{rgb}{0.000000,0.000000,0.000000}%
\pgfsetfillcolor{currentfill}%
\pgfsetlinewidth{0.602250pt}%
\definecolor{currentstroke}{rgb}{0.000000,0.000000,0.000000}%
\pgfsetstrokecolor{currentstroke}%
\pgfsetdash{}{0pt}%
\pgfsys@defobject{currentmarker}{\pgfqpoint{0.000000in}{-0.027778in}}{\pgfqpoint{0.000000in}{0.000000in}}{%
\pgfpathmoveto{\pgfqpoint{0.000000in}{0.000000in}}%
\pgfpathlineto{\pgfqpoint{0.000000in}{-0.027778in}}%
\pgfusepath{stroke,fill}%
}%
\begin{pgfscope}%
\pgfsys@transformshift{1.508413in}{0.663790in}%
\pgfsys@useobject{currentmarker}{}%
\end{pgfscope}%
\end{pgfscope}%
\begin{pgfscope}%
\pgfpathrectangle{\pgfqpoint{0.679222in}{0.663790in}}{\pgfqpoint{9.300000in}{6.795000in}}%
\pgfusepath{clip}%
\pgfsetbuttcap%
\pgfsetroundjoin%
\pgfsetlinewidth{0.803000pt}%
\definecolor{currentstroke}{rgb}{0.501961,0.501961,0.501961}%
\pgfsetstrokecolor{currentstroke}%
\pgfsetstrokeopacity{0.400000}%
\pgfsetdash{{2.960000pt}{1.280000pt}}{0.000000pt}%
\pgfpathmoveto{\pgfqpoint{1.923009in}{0.663790in}}%
\pgfpathlineto{\pgfqpoint{1.923009in}{7.458790in}}%
\pgfusepath{stroke}%
\end{pgfscope}%
\begin{pgfscope}%
\pgfsetbuttcap%
\pgfsetroundjoin%
\definecolor{currentfill}{rgb}{0.000000,0.000000,0.000000}%
\pgfsetfillcolor{currentfill}%
\pgfsetlinewidth{0.602250pt}%
\definecolor{currentstroke}{rgb}{0.000000,0.000000,0.000000}%
\pgfsetstrokecolor{currentstroke}%
\pgfsetdash{}{0pt}%
\pgfsys@defobject{currentmarker}{\pgfqpoint{0.000000in}{-0.027778in}}{\pgfqpoint{0.000000in}{0.000000in}}{%
\pgfpathmoveto{\pgfqpoint{0.000000in}{0.000000in}}%
\pgfpathlineto{\pgfqpoint{0.000000in}{-0.027778in}}%
\pgfusepath{stroke,fill}%
}%
\begin{pgfscope}%
\pgfsys@transformshift{1.923009in}{0.663790in}%
\pgfsys@useobject{currentmarker}{}%
\end{pgfscope}%
\end{pgfscope}%
\begin{pgfscope}%
\pgfpathrectangle{\pgfqpoint{0.679222in}{0.663790in}}{\pgfqpoint{9.300000in}{6.795000in}}%
\pgfusepath{clip}%
\pgfsetbuttcap%
\pgfsetroundjoin%
\pgfsetlinewidth{0.803000pt}%
\definecolor{currentstroke}{rgb}{0.501961,0.501961,0.501961}%
\pgfsetstrokecolor{currentstroke}%
\pgfsetstrokeopacity{0.400000}%
\pgfsetdash{{2.960000pt}{1.280000pt}}{0.000000pt}%
\pgfpathmoveto{\pgfqpoint{2.130306in}{0.663790in}}%
\pgfpathlineto{\pgfqpoint{2.130306in}{7.458790in}}%
\pgfusepath{stroke}%
\end{pgfscope}%
\begin{pgfscope}%
\pgfsetbuttcap%
\pgfsetroundjoin%
\definecolor{currentfill}{rgb}{0.000000,0.000000,0.000000}%
\pgfsetfillcolor{currentfill}%
\pgfsetlinewidth{0.602250pt}%
\definecolor{currentstroke}{rgb}{0.000000,0.000000,0.000000}%
\pgfsetstrokecolor{currentstroke}%
\pgfsetdash{}{0pt}%
\pgfsys@defobject{currentmarker}{\pgfqpoint{0.000000in}{-0.027778in}}{\pgfqpoint{0.000000in}{0.000000in}}{%
\pgfpathmoveto{\pgfqpoint{0.000000in}{0.000000in}}%
\pgfpathlineto{\pgfqpoint{0.000000in}{-0.027778in}}%
\pgfusepath{stroke,fill}%
}%
\begin{pgfscope}%
\pgfsys@transformshift{2.130306in}{0.663790in}%
\pgfsys@useobject{currentmarker}{}%
\end{pgfscope}%
\end{pgfscope}%
\begin{pgfscope}%
\pgfpathrectangle{\pgfqpoint{0.679222in}{0.663790in}}{\pgfqpoint{9.300000in}{6.795000in}}%
\pgfusepath{clip}%
\pgfsetbuttcap%
\pgfsetroundjoin%
\pgfsetlinewidth{0.803000pt}%
\definecolor{currentstroke}{rgb}{0.501961,0.501961,0.501961}%
\pgfsetstrokecolor{currentstroke}%
\pgfsetstrokeopacity{0.400000}%
\pgfsetdash{{2.960000pt}{1.280000pt}}{0.000000pt}%
\pgfpathmoveto{\pgfqpoint{2.337604in}{0.663790in}}%
\pgfpathlineto{\pgfqpoint{2.337604in}{7.458790in}}%
\pgfusepath{stroke}%
\end{pgfscope}%
\begin{pgfscope}%
\pgfsetbuttcap%
\pgfsetroundjoin%
\definecolor{currentfill}{rgb}{0.000000,0.000000,0.000000}%
\pgfsetfillcolor{currentfill}%
\pgfsetlinewidth{0.602250pt}%
\definecolor{currentstroke}{rgb}{0.000000,0.000000,0.000000}%
\pgfsetstrokecolor{currentstroke}%
\pgfsetdash{}{0pt}%
\pgfsys@defobject{currentmarker}{\pgfqpoint{0.000000in}{-0.027778in}}{\pgfqpoint{0.000000in}{0.000000in}}{%
\pgfpathmoveto{\pgfqpoint{0.000000in}{0.000000in}}%
\pgfpathlineto{\pgfqpoint{0.000000in}{-0.027778in}}%
\pgfusepath{stroke,fill}%
}%
\begin{pgfscope}%
\pgfsys@transformshift{2.337604in}{0.663790in}%
\pgfsys@useobject{currentmarker}{}%
\end{pgfscope}%
\end{pgfscope}%
\begin{pgfscope}%
\pgfpathrectangle{\pgfqpoint{0.679222in}{0.663790in}}{\pgfqpoint{9.300000in}{6.795000in}}%
\pgfusepath{clip}%
\pgfsetbuttcap%
\pgfsetroundjoin%
\pgfsetlinewidth{0.803000pt}%
\definecolor{currentstroke}{rgb}{0.501961,0.501961,0.501961}%
\pgfsetstrokecolor{currentstroke}%
\pgfsetstrokeopacity{0.400000}%
\pgfsetdash{{2.960000pt}{1.280000pt}}{0.000000pt}%
\pgfpathmoveto{\pgfqpoint{2.544902in}{0.663790in}}%
\pgfpathlineto{\pgfqpoint{2.544902in}{7.458790in}}%
\pgfusepath{stroke}%
\end{pgfscope}%
\begin{pgfscope}%
\pgfsetbuttcap%
\pgfsetroundjoin%
\definecolor{currentfill}{rgb}{0.000000,0.000000,0.000000}%
\pgfsetfillcolor{currentfill}%
\pgfsetlinewidth{0.602250pt}%
\definecolor{currentstroke}{rgb}{0.000000,0.000000,0.000000}%
\pgfsetstrokecolor{currentstroke}%
\pgfsetdash{}{0pt}%
\pgfsys@defobject{currentmarker}{\pgfqpoint{0.000000in}{-0.027778in}}{\pgfqpoint{0.000000in}{0.000000in}}{%
\pgfpathmoveto{\pgfqpoint{0.000000in}{0.000000in}}%
\pgfpathlineto{\pgfqpoint{0.000000in}{-0.027778in}}%
\pgfusepath{stroke,fill}%
}%
\begin{pgfscope}%
\pgfsys@transformshift{2.544902in}{0.663790in}%
\pgfsys@useobject{currentmarker}{}%
\end{pgfscope}%
\end{pgfscope}%
\begin{pgfscope}%
\pgfpathrectangle{\pgfqpoint{0.679222in}{0.663790in}}{\pgfqpoint{9.300000in}{6.795000in}}%
\pgfusepath{clip}%
\pgfsetbuttcap%
\pgfsetroundjoin%
\pgfsetlinewidth{0.803000pt}%
\definecolor{currentstroke}{rgb}{0.501961,0.501961,0.501961}%
\pgfsetstrokecolor{currentstroke}%
\pgfsetstrokeopacity{0.400000}%
\pgfsetdash{{2.960000pt}{1.280000pt}}{0.000000pt}%
\pgfpathmoveto{\pgfqpoint{2.959497in}{0.663790in}}%
\pgfpathlineto{\pgfqpoint{2.959497in}{7.458790in}}%
\pgfusepath{stroke}%
\end{pgfscope}%
\begin{pgfscope}%
\pgfsetbuttcap%
\pgfsetroundjoin%
\definecolor{currentfill}{rgb}{0.000000,0.000000,0.000000}%
\pgfsetfillcolor{currentfill}%
\pgfsetlinewidth{0.602250pt}%
\definecolor{currentstroke}{rgb}{0.000000,0.000000,0.000000}%
\pgfsetstrokecolor{currentstroke}%
\pgfsetdash{}{0pt}%
\pgfsys@defobject{currentmarker}{\pgfqpoint{0.000000in}{-0.027778in}}{\pgfqpoint{0.000000in}{0.000000in}}{%
\pgfpathmoveto{\pgfqpoint{0.000000in}{0.000000in}}%
\pgfpathlineto{\pgfqpoint{0.000000in}{-0.027778in}}%
\pgfusepath{stroke,fill}%
}%
\begin{pgfscope}%
\pgfsys@transformshift{2.959497in}{0.663790in}%
\pgfsys@useobject{currentmarker}{}%
\end{pgfscope}%
\end{pgfscope}%
\begin{pgfscope}%
\pgfpathrectangle{\pgfqpoint{0.679222in}{0.663790in}}{\pgfqpoint{9.300000in}{6.795000in}}%
\pgfusepath{clip}%
\pgfsetbuttcap%
\pgfsetroundjoin%
\pgfsetlinewidth{0.803000pt}%
\definecolor{currentstroke}{rgb}{0.501961,0.501961,0.501961}%
\pgfsetstrokecolor{currentstroke}%
\pgfsetstrokeopacity{0.400000}%
\pgfsetdash{{2.960000pt}{1.280000pt}}{0.000000pt}%
\pgfpathmoveto{\pgfqpoint{3.166795in}{0.663790in}}%
\pgfpathlineto{\pgfqpoint{3.166795in}{7.458790in}}%
\pgfusepath{stroke}%
\end{pgfscope}%
\begin{pgfscope}%
\pgfsetbuttcap%
\pgfsetroundjoin%
\definecolor{currentfill}{rgb}{0.000000,0.000000,0.000000}%
\pgfsetfillcolor{currentfill}%
\pgfsetlinewidth{0.602250pt}%
\definecolor{currentstroke}{rgb}{0.000000,0.000000,0.000000}%
\pgfsetstrokecolor{currentstroke}%
\pgfsetdash{}{0pt}%
\pgfsys@defobject{currentmarker}{\pgfqpoint{0.000000in}{-0.027778in}}{\pgfqpoint{0.000000in}{0.000000in}}{%
\pgfpathmoveto{\pgfqpoint{0.000000in}{0.000000in}}%
\pgfpathlineto{\pgfqpoint{0.000000in}{-0.027778in}}%
\pgfusepath{stroke,fill}%
}%
\begin{pgfscope}%
\pgfsys@transformshift{3.166795in}{0.663790in}%
\pgfsys@useobject{currentmarker}{}%
\end{pgfscope}%
\end{pgfscope}%
\begin{pgfscope}%
\pgfpathrectangle{\pgfqpoint{0.679222in}{0.663790in}}{\pgfqpoint{9.300000in}{6.795000in}}%
\pgfusepath{clip}%
\pgfsetbuttcap%
\pgfsetroundjoin%
\pgfsetlinewidth{0.803000pt}%
\definecolor{currentstroke}{rgb}{0.501961,0.501961,0.501961}%
\pgfsetstrokecolor{currentstroke}%
\pgfsetstrokeopacity{0.400000}%
\pgfsetdash{{2.960000pt}{1.280000pt}}{0.000000pt}%
\pgfpathmoveto{\pgfqpoint{3.374093in}{0.663790in}}%
\pgfpathlineto{\pgfqpoint{3.374093in}{7.458790in}}%
\pgfusepath{stroke}%
\end{pgfscope}%
\begin{pgfscope}%
\pgfsetbuttcap%
\pgfsetroundjoin%
\definecolor{currentfill}{rgb}{0.000000,0.000000,0.000000}%
\pgfsetfillcolor{currentfill}%
\pgfsetlinewidth{0.602250pt}%
\definecolor{currentstroke}{rgb}{0.000000,0.000000,0.000000}%
\pgfsetstrokecolor{currentstroke}%
\pgfsetdash{}{0pt}%
\pgfsys@defobject{currentmarker}{\pgfqpoint{0.000000in}{-0.027778in}}{\pgfqpoint{0.000000in}{0.000000in}}{%
\pgfpathmoveto{\pgfqpoint{0.000000in}{0.000000in}}%
\pgfpathlineto{\pgfqpoint{0.000000in}{-0.027778in}}%
\pgfusepath{stroke,fill}%
}%
\begin{pgfscope}%
\pgfsys@transformshift{3.374093in}{0.663790in}%
\pgfsys@useobject{currentmarker}{}%
\end{pgfscope}%
\end{pgfscope}%
\begin{pgfscope}%
\pgfpathrectangle{\pgfqpoint{0.679222in}{0.663790in}}{\pgfqpoint{9.300000in}{6.795000in}}%
\pgfusepath{clip}%
\pgfsetbuttcap%
\pgfsetroundjoin%
\pgfsetlinewidth{0.803000pt}%
\definecolor{currentstroke}{rgb}{0.501961,0.501961,0.501961}%
\pgfsetstrokecolor{currentstroke}%
\pgfsetstrokeopacity{0.400000}%
\pgfsetdash{{2.960000pt}{1.280000pt}}{0.000000pt}%
\pgfpathmoveto{\pgfqpoint{3.581390in}{0.663790in}}%
\pgfpathlineto{\pgfqpoint{3.581390in}{7.458790in}}%
\pgfusepath{stroke}%
\end{pgfscope}%
\begin{pgfscope}%
\pgfsetbuttcap%
\pgfsetroundjoin%
\definecolor{currentfill}{rgb}{0.000000,0.000000,0.000000}%
\pgfsetfillcolor{currentfill}%
\pgfsetlinewidth{0.602250pt}%
\definecolor{currentstroke}{rgb}{0.000000,0.000000,0.000000}%
\pgfsetstrokecolor{currentstroke}%
\pgfsetdash{}{0pt}%
\pgfsys@defobject{currentmarker}{\pgfqpoint{0.000000in}{-0.027778in}}{\pgfqpoint{0.000000in}{0.000000in}}{%
\pgfpathmoveto{\pgfqpoint{0.000000in}{0.000000in}}%
\pgfpathlineto{\pgfqpoint{0.000000in}{-0.027778in}}%
\pgfusepath{stroke,fill}%
}%
\begin{pgfscope}%
\pgfsys@transformshift{3.581390in}{0.663790in}%
\pgfsys@useobject{currentmarker}{}%
\end{pgfscope}%
\end{pgfscope}%
\begin{pgfscope}%
\pgfpathrectangle{\pgfqpoint{0.679222in}{0.663790in}}{\pgfqpoint{9.300000in}{6.795000in}}%
\pgfusepath{clip}%
\pgfsetbuttcap%
\pgfsetroundjoin%
\pgfsetlinewidth{0.803000pt}%
\definecolor{currentstroke}{rgb}{0.501961,0.501961,0.501961}%
\pgfsetstrokecolor{currentstroke}%
\pgfsetstrokeopacity{0.400000}%
\pgfsetdash{{2.960000pt}{1.280000pt}}{0.000000pt}%
\pgfpathmoveto{\pgfqpoint{3.995986in}{0.663790in}}%
\pgfpathlineto{\pgfqpoint{3.995986in}{7.458790in}}%
\pgfusepath{stroke}%
\end{pgfscope}%
\begin{pgfscope}%
\pgfsetbuttcap%
\pgfsetroundjoin%
\definecolor{currentfill}{rgb}{0.000000,0.000000,0.000000}%
\pgfsetfillcolor{currentfill}%
\pgfsetlinewidth{0.602250pt}%
\definecolor{currentstroke}{rgb}{0.000000,0.000000,0.000000}%
\pgfsetstrokecolor{currentstroke}%
\pgfsetdash{}{0pt}%
\pgfsys@defobject{currentmarker}{\pgfqpoint{0.000000in}{-0.027778in}}{\pgfqpoint{0.000000in}{0.000000in}}{%
\pgfpathmoveto{\pgfqpoint{0.000000in}{0.000000in}}%
\pgfpathlineto{\pgfqpoint{0.000000in}{-0.027778in}}%
\pgfusepath{stroke,fill}%
}%
\begin{pgfscope}%
\pgfsys@transformshift{3.995986in}{0.663790in}%
\pgfsys@useobject{currentmarker}{}%
\end{pgfscope}%
\end{pgfscope}%
\begin{pgfscope}%
\pgfpathrectangle{\pgfqpoint{0.679222in}{0.663790in}}{\pgfqpoint{9.300000in}{6.795000in}}%
\pgfusepath{clip}%
\pgfsetbuttcap%
\pgfsetroundjoin%
\pgfsetlinewidth{0.803000pt}%
\definecolor{currentstroke}{rgb}{0.501961,0.501961,0.501961}%
\pgfsetstrokecolor{currentstroke}%
\pgfsetstrokeopacity{0.400000}%
\pgfsetdash{{2.960000pt}{1.280000pt}}{0.000000pt}%
\pgfpathmoveto{\pgfqpoint{4.203283in}{0.663790in}}%
\pgfpathlineto{\pgfqpoint{4.203283in}{7.458790in}}%
\pgfusepath{stroke}%
\end{pgfscope}%
\begin{pgfscope}%
\pgfsetbuttcap%
\pgfsetroundjoin%
\definecolor{currentfill}{rgb}{0.000000,0.000000,0.000000}%
\pgfsetfillcolor{currentfill}%
\pgfsetlinewidth{0.602250pt}%
\definecolor{currentstroke}{rgb}{0.000000,0.000000,0.000000}%
\pgfsetstrokecolor{currentstroke}%
\pgfsetdash{}{0pt}%
\pgfsys@defobject{currentmarker}{\pgfqpoint{0.000000in}{-0.027778in}}{\pgfqpoint{0.000000in}{0.000000in}}{%
\pgfpathmoveto{\pgfqpoint{0.000000in}{0.000000in}}%
\pgfpathlineto{\pgfqpoint{0.000000in}{-0.027778in}}%
\pgfusepath{stroke,fill}%
}%
\begin{pgfscope}%
\pgfsys@transformshift{4.203283in}{0.663790in}%
\pgfsys@useobject{currentmarker}{}%
\end{pgfscope}%
\end{pgfscope}%
\begin{pgfscope}%
\pgfpathrectangle{\pgfqpoint{0.679222in}{0.663790in}}{\pgfqpoint{9.300000in}{6.795000in}}%
\pgfusepath{clip}%
\pgfsetbuttcap%
\pgfsetroundjoin%
\pgfsetlinewidth{0.803000pt}%
\definecolor{currentstroke}{rgb}{0.501961,0.501961,0.501961}%
\pgfsetstrokecolor{currentstroke}%
\pgfsetstrokeopacity{0.400000}%
\pgfsetdash{{2.960000pt}{1.280000pt}}{0.000000pt}%
\pgfpathmoveto{\pgfqpoint{4.410581in}{0.663790in}}%
\pgfpathlineto{\pgfqpoint{4.410581in}{7.458790in}}%
\pgfusepath{stroke}%
\end{pgfscope}%
\begin{pgfscope}%
\pgfsetbuttcap%
\pgfsetroundjoin%
\definecolor{currentfill}{rgb}{0.000000,0.000000,0.000000}%
\pgfsetfillcolor{currentfill}%
\pgfsetlinewidth{0.602250pt}%
\definecolor{currentstroke}{rgb}{0.000000,0.000000,0.000000}%
\pgfsetstrokecolor{currentstroke}%
\pgfsetdash{}{0pt}%
\pgfsys@defobject{currentmarker}{\pgfqpoint{0.000000in}{-0.027778in}}{\pgfqpoint{0.000000in}{0.000000in}}{%
\pgfpathmoveto{\pgfqpoint{0.000000in}{0.000000in}}%
\pgfpathlineto{\pgfqpoint{0.000000in}{-0.027778in}}%
\pgfusepath{stroke,fill}%
}%
\begin{pgfscope}%
\pgfsys@transformshift{4.410581in}{0.663790in}%
\pgfsys@useobject{currentmarker}{}%
\end{pgfscope}%
\end{pgfscope}%
\begin{pgfscope}%
\pgfpathrectangle{\pgfqpoint{0.679222in}{0.663790in}}{\pgfqpoint{9.300000in}{6.795000in}}%
\pgfusepath{clip}%
\pgfsetbuttcap%
\pgfsetroundjoin%
\pgfsetlinewidth{0.803000pt}%
\definecolor{currentstroke}{rgb}{0.501961,0.501961,0.501961}%
\pgfsetstrokecolor{currentstroke}%
\pgfsetstrokeopacity{0.400000}%
\pgfsetdash{{2.960000pt}{1.280000pt}}{0.000000pt}%
\pgfpathmoveto{\pgfqpoint{4.617879in}{0.663790in}}%
\pgfpathlineto{\pgfqpoint{4.617879in}{7.458790in}}%
\pgfusepath{stroke}%
\end{pgfscope}%
\begin{pgfscope}%
\pgfsetbuttcap%
\pgfsetroundjoin%
\definecolor{currentfill}{rgb}{0.000000,0.000000,0.000000}%
\pgfsetfillcolor{currentfill}%
\pgfsetlinewidth{0.602250pt}%
\definecolor{currentstroke}{rgb}{0.000000,0.000000,0.000000}%
\pgfsetstrokecolor{currentstroke}%
\pgfsetdash{}{0pt}%
\pgfsys@defobject{currentmarker}{\pgfqpoint{0.000000in}{-0.027778in}}{\pgfqpoint{0.000000in}{0.000000in}}{%
\pgfpathmoveto{\pgfqpoint{0.000000in}{0.000000in}}%
\pgfpathlineto{\pgfqpoint{0.000000in}{-0.027778in}}%
\pgfusepath{stroke,fill}%
}%
\begin{pgfscope}%
\pgfsys@transformshift{4.617879in}{0.663790in}%
\pgfsys@useobject{currentmarker}{}%
\end{pgfscope}%
\end{pgfscope}%
\begin{pgfscope}%
\pgfpathrectangle{\pgfqpoint{0.679222in}{0.663790in}}{\pgfqpoint{9.300000in}{6.795000in}}%
\pgfusepath{clip}%
\pgfsetbuttcap%
\pgfsetroundjoin%
\pgfsetlinewidth{0.803000pt}%
\definecolor{currentstroke}{rgb}{0.501961,0.501961,0.501961}%
\pgfsetstrokecolor{currentstroke}%
\pgfsetstrokeopacity{0.400000}%
\pgfsetdash{{2.960000pt}{1.280000pt}}{0.000000pt}%
\pgfpathmoveto{\pgfqpoint{5.032474in}{0.663790in}}%
\pgfpathlineto{\pgfqpoint{5.032474in}{7.458790in}}%
\pgfusepath{stroke}%
\end{pgfscope}%
\begin{pgfscope}%
\pgfsetbuttcap%
\pgfsetroundjoin%
\definecolor{currentfill}{rgb}{0.000000,0.000000,0.000000}%
\pgfsetfillcolor{currentfill}%
\pgfsetlinewidth{0.602250pt}%
\definecolor{currentstroke}{rgb}{0.000000,0.000000,0.000000}%
\pgfsetstrokecolor{currentstroke}%
\pgfsetdash{}{0pt}%
\pgfsys@defobject{currentmarker}{\pgfqpoint{0.000000in}{-0.027778in}}{\pgfqpoint{0.000000in}{0.000000in}}{%
\pgfpathmoveto{\pgfqpoint{0.000000in}{0.000000in}}%
\pgfpathlineto{\pgfqpoint{0.000000in}{-0.027778in}}%
\pgfusepath{stroke,fill}%
}%
\begin{pgfscope}%
\pgfsys@transformshift{5.032474in}{0.663790in}%
\pgfsys@useobject{currentmarker}{}%
\end{pgfscope}%
\end{pgfscope}%
\begin{pgfscope}%
\pgfpathrectangle{\pgfqpoint{0.679222in}{0.663790in}}{\pgfqpoint{9.300000in}{6.795000in}}%
\pgfusepath{clip}%
\pgfsetbuttcap%
\pgfsetroundjoin%
\pgfsetlinewidth{0.803000pt}%
\definecolor{currentstroke}{rgb}{0.501961,0.501961,0.501961}%
\pgfsetstrokecolor{currentstroke}%
\pgfsetstrokeopacity{0.400000}%
\pgfsetdash{{2.960000pt}{1.280000pt}}{0.000000pt}%
\pgfpathmoveto{\pgfqpoint{5.239772in}{0.663790in}}%
\pgfpathlineto{\pgfqpoint{5.239772in}{7.458790in}}%
\pgfusepath{stroke}%
\end{pgfscope}%
\begin{pgfscope}%
\pgfsetbuttcap%
\pgfsetroundjoin%
\definecolor{currentfill}{rgb}{0.000000,0.000000,0.000000}%
\pgfsetfillcolor{currentfill}%
\pgfsetlinewidth{0.602250pt}%
\definecolor{currentstroke}{rgb}{0.000000,0.000000,0.000000}%
\pgfsetstrokecolor{currentstroke}%
\pgfsetdash{}{0pt}%
\pgfsys@defobject{currentmarker}{\pgfqpoint{0.000000in}{-0.027778in}}{\pgfqpoint{0.000000in}{0.000000in}}{%
\pgfpathmoveto{\pgfqpoint{0.000000in}{0.000000in}}%
\pgfpathlineto{\pgfqpoint{0.000000in}{-0.027778in}}%
\pgfusepath{stroke,fill}%
}%
\begin{pgfscope}%
\pgfsys@transformshift{5.239772in}{0.663790in}%
\pgfsys@useobject{currentmarker}{}%
\end{pgfscope}%
\end{pgfscope}%
\begin{pgfscope}%
\pgfpathrectangle{\pgfqpoint{0.679222in}{0.663790in}}{\pgfqpoint{9.300000in}{6.795000in}}%
\pgfusepath{clip}%
\pgfsetbuttcap%
\pgfsetroundjoin%
\pgfsetlinewidth{0.803000pt}%
\definecolor{currentstroke}{rgb}{0.501961,0.501961,0.501961}%
\pgfsetstrokecolor{currentstroke}%
\pgfsetstrokeopacity{0.400000}%
\pgfsetdash{{2.960000pt}{1.280000pt}}{0.000000pt}%
\pgfpathmoveto{\pgfqpoint{5.447070in}{0.663790in}}%
\pgfpathlineto{\pgfqpoint{5.447070in}{7.458790in}}%
\pgfusepath{stroke}%
\end{pgfscope}%
\begin{pgfscope}%
\pgfsetbuttcap%
\pgfsetroundjoin%
\definecolor{currentfill}{rgb}{0.000000,0.000000,0.000000}%
\pgfsetfillcolor{currentfill}%
\pgfsetlinewidth{0.602250pt}%
\definecolor{currentstroke}{rgb}{0.000000,0.000000,0.000000}%
\pgfsetstrokecolor{currentstroke}%
\pgfsetdash{}{0pt}%
\pgfsys@defobject{currentmarker}{\pgfqpoint{0.000000in}{-0.027778in}}{\pgfqpoint{0.000000in}{0.000000in}}{%
\pgfpathmoveto{\pgfqpoint{0.000000in}{0.000000in}}%
\pgfpathlineto{\pgfqpoint{0.000000in}{-0.027778in}}%
\pgfusepath{stroke,fill}%
}%
\begin{pgfscope}%
\pgfsys@transformshift{5.447070in}{0.663790in}%
\pgfsys@useobject{currentmarker}{}%
\end{pgfscope}%
\end{pgfscope}%
\begin{pgfscope}%
\pgfpathrectangle{\pgfqpoint{0.679222in}{0.663790in}}{\pgfqpoint{9.300000in}{6.795000in}}%
\pgfusepath{clip}%
\pgfsetbuttcap%
\pgfsetroundjoin%
\pgfsetlinewidth{0.803000pt}%
\definecolor{currentstroke}{rgb}{0.501961,0.501961,0.501961}%
\pgfsetstrokecolor{currentstroke}%
\pgfsetstrokeopacity{0.400000}%
\pgfsetdash{{2.960000pt}{1.280000pt}}{0.000000pt}%
\pgfpathmoveto{\pgfqpoint{5.654367in}{0.663790in}}%
\pgfpathlineto{\pgfqpoint{5.654367in}{7.458790in}}%
\pgfusepath{stroke}%
\end{pgfscope}%
\begin{pgfscope}%
\pgfsetbuttcap%
\pgfsetroundjoin%
\definecolor{currentfill}{rgb}{0.000000,0.000000,0.000000}%
\pgfsetfillcolor{currentfill}%
\pgfsetlinewidth{0.602250pt}%
\definecolor{currentstroke}{rgb}{0.000000,0.000000,0.000000}%
\pgfsetstrokecolor{currentstroke}%
\pgfsetdash{}{0pt}%
\pgfsys@defobject{currentmarker}{\pgfqpoint{0.000000in}{-0.027778in}}{\pgfqpoint{0.000000in}{0.000000in}}{%
\pgfpathmoveto{\pgfqpoint{0.000000in}{0.000000in}}%
\pgfpathlineto{\pgfqpoint{0.000000in}{-0.027778in}}%
\pgfusepath{stroke,fill}%
}%
\begin{pgfscope}%
\pgfsys@transformshift{5.654367in}{0.663790in}%
\pgfsys@useobject{currentmarker}{}%
\end{pgfscope}%
\end{pgfscope}%
\begin{pgfscope}%
\pgfpathrectangle{\pgfqpoint{0.679222in}{0.663790in}}{\pgfqpoint{9.300000in}{6.795000in}}%
\pgfusepath{clip}%
\pgfsetbuttcap%
\pgfsetroundjoin%
\pgfsetlinewidth{0.803000pt}%
\definecolor{currentstroke}{rgb}{0.501961,0.501961,0.501961}%
\pgfsetstrokecolor{currentstroke}%
\pgfsetstrokeopacity{0.400000}%
\pgfsetdash{{2.960000pt}{1.280000pt}}{0.000000pt}%
\pgfpathmoveto{\pgfqpoint{6.068963in}{0.663790in}}%
\pgfpathlineto{\pgfqpoint{6.068963in}{7.458790in}}%
\pgfusepath{stroke}%
\end{pgfscope}%
\begin{pgfscope}%
\pgfsetbuttcap%
\pgfsetroundjoin%
\definecolor{currentfill}{rgb}{0.000000,0.000000,0.000000}%
\pgfsetfillcolor{currentfill}%
\pgfsetlinewidth{0.602250pt}%
\definecolor{currentstroke}{rgb}{0.000000,0.000000,0.000000}%
\pgfsetstrokecolor{currentstroke}%
\pgfsetdash{}{0pt}%
\pgfsys@defobject{currentmarker}{\pgfqpoint{0.000000in}{-0.027778in}}{\pgfqpoint{0.000000in}{0.000000in}}{%
\pgfpathmoveto{\pgfqpoint{0.000000in}{0.000000in}}%
\pgfpathlineto{\pgfqpoint{0.000000in}{-0.027778in}}%
\pgfusepath{stroke,fill}%
}%
\begin{pgfscope}%
\pgfsys@transformshift{6.068963in}{0.663790in}%
\pgfsys@useobject{currentmarker}{}%
\end{pgfscope}%
\end{pgfscope}%
\begin{pgfscope}%
\pgfpathrectangle{\pgfqpoint{0.679222in}{0.663790in}}{\pgfqpoint{9.300000in}{6.795000in}}%
\pgfusepath{clip}%
\pgfsetbuttcap%
\pgfsetroundjoin%
\pgfsetlinewidth{0.803000pt}%
\definecolor{currentstroke}{rgb}{0.501961,0.501961,0.501961}%
\pgfsetstrokecolor{currentstroke}%
\pgfsetstrokeopacity{0.400000}%
\pgfsetdash{{2.960000pt}{1.280000pt}}{0.000000pt}%
\pgfpathmoveto{\pgfqpoint{6.276261in}{0.663790in}}%
\pgfpathlineto{\pgfqpoint{6.276261in}{7.458790in}}%
\pgfusepath{stroke}%
\end{pgfscope}%
\begin{pgfscope}%
\pgfsetbuttcap%
\pgfsetroundjoin%
\definecolor{currentfill}{rgb}{0.000000,0.000000,0.000000}%
\pgfsetfillcolor{currentfill}%
\pgfsetlinewidth{0.602250pt}%
\definecolor{currentstroke}{rgb}{0.000000,0.000000,0.000000}%
\pgfsetstrokecolor{currentstroke}%
\pgfsetdash{}{0pt}%
\pgfsys@defobject{currentmarker}{\pgfqpoint{0.000000in}{-0.027778in}}{\pgfqpoint{0.000000in}{0.000000in}}{%
\pgfpathmoveto{\pgfqpoint{0.000000in}{0.000000in}}%
\pgfpathlineto{\pgfqpoint{0.000000in}{-0.027778in}}%
\pgfusepath{stroke,fill}%
}%
\begin{pgfscope}%
\pgfsys@transformshift{6.276261in}{0.663790in}%
\pgfsys@useobject{currentmarker}{}%
\end{pgfscope}%
\end{pgfscope}%
\begin{pgfscope}%
\pgfpathrectangle{\pgfqpoint{0.679222in}{0.663790in}}{\pgfqpoint{9.300000in}{6.795000in}}%
\pgfusepath{clip}%
\pgfsetbuttcap%
\pgfsetroundjoin%
\pgfsetlinewidth{0.803000pt}%
\definecolor{currentstroke}{rgb}{0.501961,0.501961,0.501961}%
\pgfsetstrokecolor{currentstroke}%
\pgfsetstrokeopacity{0.400000}%
\pgfsetdash{{2.960000pt}{1.280000pt}}{0.000000pt}%
\pgfpathmoveto{\pgfqpoint{6.483558in}{0.663790in}}%
\pgfpathlineto{\pgfqpoint{6.483558in}{7.458790in}}%
\pgfusepath{stroke}%
\end{pgfscope}%
\begin{pgfscope}%
\pgfsetbuttcap%
\pgfsetroundjoin%
\definecolor{currentfill}{rgb}{0.000000,0.000000,0.000000}%
\pgfsetfillcolor{currentfill}%
\pgfsetlinewidth{0.602250pt}%
\definecolor{currentstroke}{rgb}{0.000000,0.000000,0.000000}%
\pgfsetstrokecolor{currentstroke}%
\pgfsetdash{}{0pt}%
\pgfsys@defobject{currentmarker}{\pgfqpoint{0.000000in}{-0.027778in}}{\pgfqpoint{0.000000in}{0.000000in}}{%
\pgfpathmoveto{\pgfqpoint{0.000000in}{0.000000in}}%
\pgfpathlineto{\pgfqpoint{0.000000in}{-0.027778in}}%
\pgfusepath{stroke,fill}%
}%
\begin{pgfscope}%
\pgfsys@transformshift{6.483558in}{0.663790in}%
\pgfsys@useobject{currentmarker}{}%
\end{pgfscope}%
\end{pgfscope}%
\begin{pgfscope}%
\pgfpathrectangle{\pgfqpoint{0.679222in}{0.663790in}}{\pgfqpoint{9.300000in}{6.795000in}}%
\pgfusepath{clip}%
\pgfsetbuttcap%
\pgfsetroundjoin%
\pgfsetlinewidth{0.803000pt}%
\definecolor{currentstroke}{rgb}{0.501961,0.501961,0.501961}%
\pgfsetstrokecolor{currentstroke}%
\pgfsetstrokeopacity{0.400000}%
\pgfsetdash{{2.960000pt}{1.280000pt}}{0.000000pt}%
\pgfpathmoveto{\pgfqpoint{6.690856in}{0.663790in}}%
\pgfpathlineto{\pgfqpoint{6.690856in}{7.458790in}}%
\pgfusepath{stroke}%
\end{pgfscope}%
\begin{pgfscope}%
\pgfsetbuttcap%
\pgfsetroundjoin%
\definecolor{currentfill}{rgb}{0.000000,0.000000,0.000000}%
\pgfsetfillcolor{currentfill}%
\pgfsetlinewidth{0.602250pt}%
\definecolor{currentstroke}{rgb}{0.000000,0.000000,0.000000}%
\pgfsetstrokecolor{currentstroke}%
\pgfsetdash{}{0pt}%
\pgfsys@defobject{currentmarker}{\pgfqpoint{0.000000in}{-0.027778in}}{\pgfqpoint{0.000000in}{0.000000in}}{%
\pgfpathmoveto{\pgfqpoint{0.000000in}{0.000000in}}%
\pgfpathlineto{\pgfqpoint{0.000000in}{-0.027778in}}%
\pgfusepath{stroke,fill}%
}%
\begin{pgfscope}%
\pgfsys@transformshift{6.690856in}{0.663790in}%
\pgfsys@useobject{currentmarker}{}%
\end{pgfscope}%
\end{pgfscope}%
\begin{pgfscope}%
\pgfpathrectangle{\pgfqpoint{0.679222in}{0.663790in}}{\pgfqpoint{9.300000in}{6.795000in}}%
\pgfusepath{clip}%
\pgfsetbuttcap%
\pgfsetroundjoin%
\pgfsetlinewidth{0.803000pt}%
\definecolor{currentstroke}{rgb}{0.501961,0.501961,0.501961}%
\pgfsetstrokecolor{currentstroke}%
\pgfsetstrokeopacity{0.400000}%
\pgfsetdash{{2.960000pt}{1.280000pt}}{0.000000pt}%
\pgfpathmoveto{\pgfqpoint{7.105451in}{0.663790in}}%
\pgfpathlineto{\pgfqpoint{7.105451in}{7.458790in}}%
\pgfusepath{stroke}%
\end{pgfscope}%
\begin{pgfscope}%
\pgfsetbuttcap%
\pgfsetroundjoin%
\definecolor{currentfill}{rgb}{0.000000,0.000000,0.000000}%
\pgfsetfillcolor{currentfill}%
\pgfsetlinewidth{0.602250pt}%
\definecolor{currentstroke}{rgb}{0.000000,0.000000,0.000000}%
\pgfsetstrokecolor{currentstroke}%
\pgfsetdash{}{0pt}%
\pgfsys@defobject{currentmarker}{\pgfqpoint{0.000000in}{-0.027778in}}{\pgfqpoint{0.000000in}{0.000000in}}{%
\pgfpathmoveto{\pgfqpoint{0.000000in}{0.000000in}}%
\pgfpathlineto{\pgfqpoint{0.000000in}{-0.027778in}}%
\pgfusepath{stroke,fill}%
}%
\begin{pgfscope}%
\pgfsys@transformshift{7.105451in}{0.663790in}%
\pgfsys@useobject{currentmarker}{}%
\end{pgfscope}%
\end{pgfscope}%
\begin{pgfscope}%
\pgfpathrectangle{\pgfqpoint{0.679222in}{0.663790in}}{\pgfqpoint{9.300000in}{6.795000in}}%
\pgfusepath{clip}%
\pgfsetbuttcap%
\pgfsetroundjoin%
\pgfsetlinewidth{0.803000pt}%
\definecolor{currentstroke}{rgb}{0.501961,0.501961,0.501961}%
\pgfsetstrokecolor{currentstroke}%
\pgfsetstrokeopacity{0.400000}%
\pgfsetdash{{2.960000pt}{1.280000pt}}{0.000000pt}%
\pgfpathmoveto{\pgfqpoint{7.312749in}{0.663790in}}%
\pgfpathlineto{\pgfqpoint{7.312749in}{7.458790in}}%
\pgfusepath{stroke}%
\end{pgfscope}%
\begin{pgfscope}%
\pgfsetbuttcap%
\pgfsetroundjoin%
\definecolor{currentfill}{rgb}{0.000000,0.000000,0.000000}%
\pgfsetfillcolor{currentfill}%
\pgfsetlinewidth{0.602250pt}%
\definecolor{currentstroke}{rgb}{0.000000,0.000000,0.000000}%
\pgfsetstrokecolor{currentstroke}%
\pgfsetdash{}{0pt}%
\pgfsys@defobject{currentmarker}{\pgfqpoint{0.000000in}{-0.027778in}}{\pgfqpoint{0.000000in}{0.000000in}}{%
\pgfpathmoveto{\pgfqpoint{0.000000in}{0.000000in}}%
\pgfpathlineto{\pgfqpoint{0.000000in}{-0.027778in}}%
\pgfusepath{stroke,fill}%
}%
\begin{pgfscope}%
\pgfsys@transformshift{7.312749in}{0.663790in}%
\pgfsys@useobject{currentmarker}{}%
\end{pgfscope}%
\end{pgfscope}%
\begin{pgfscope}%
\pgfpathrectangle{\pgfqpoint{0.679222in}{0.663790in}}{\pgfqpoint{9.300000in}{6.795000in}}%
\pgfusepath{clip}%
\pgfsetbuttcap%
\pgfsetroundjoin%
\pgfsetlinewidth{0.803000pt}%
\definecolor{currentstroke}{rgb}{0.501961,0.501961,0.501961}%
\pgfsetstrokecolor{currentstroke}%
\pgfsetstrokeopacity{0.400000}%
\pgfsetdash{{2.960000pt}{1.280000pt}}{0.000000pt}%
\pgfpathmoveto{\pgfqpoint{7.520047in}{0.663790in}}%
\pgfpathlineto{\pgfqpoint{7.520047in}{7.458790in}}%
\pgfusepath{stroke}%
\end{pgfscope}%
\begin{pgfscope}%
\pgfsetbuttcap%
\pgfsetroundjoin%
\definecolor{currentfill}{rgb}{0.000000,0.000000,0.000000}%
\pgfsetfillcolor{currentfill}%
\pgfsetlinewidth{0.602250pt}%
\definecolor{currentstroke}{rgb}{0.000000,0.000000,0.000000}%
\pgfsetstrokecolor{currentstroke}%
\pgfsetdash{}{0pt}%
\pgfsys@defobject{currentmarker}{\pgfqpoint{0.000000in}{-0.027778in}}{\pgfqpoint{0.000000in}{0.000000in}}{%
\pgfpathmoveto{\pgfqpoint{0.000000in}{0.000000in}}%
\pgfpathlineto{\pgfqpoint{0.000000in}{-0.027778in}}%
\pgfusepath{stroke,fill}%
}%
\begin{pgfscope}%
\pgfsys@transformshift{7.520047in}{0.663790in}%
\pgfsys@useobject{currentmarker}{}%
\end{pgfscope}%
\end{pgfscope}%
\begin{pgfscope}%
\pgfpathrectangle{\pgfqpoint{0.679222in}{0.663790in}}{\pgfqpoint{9.300000in}{6.795000in}}%
\pgfusepath{clip}%
\pgfsetbuttcap%
\pgfsetroundjoin%
\pgfsetlinewidth{0.803000pt}%
\definecolor{currentstroke}{rgb}{0.501961,0.501961,0.501961}%
\pgfsetstrokecolor{currentstroke}%
\pgfsetstrokeopacity{0.400000}%
\pgfsetdash{{2.960000pt}{1.280000pt}}{0.000000pt}%
\pgfpathmoveto{\pgfqpoint{7.727344in}{0.663790in}}%
\pgfpathlineto{\pgfqpoint{7.727344in}{7.458790in}}%
\pgfusepath{stroke}%
\end{pgfscope}%
\begin{pgfscope}%
\pgfsetbuttcap%
\pgfsetroundjoin%
\definecolor{currentfill}{rgb}{0.000000,0.000000,0.000000}%
\pgfsetfillcolor{currentfill}%
\pgfsetlinewidth{0.602250pt}%
\definecolor{currentstroke}{rgb}{0.000000,0.000000,0.000000}%
\pgfsetstrokecolor{currentstroke}%
\pgfsetdash{}{0pt}%
\pgfsys@defobject{currentmarker}{\pgfqpoint{0.000000in}{-0.027778in}}{\pgfqpoint{0.000000in}{0.000000in}}{%
\pgfpathmoveto{\pgfqpoint{0.000000in}{0.000000in}}%
\pgfpathlineto{\pgfqpoint{0.000000in}{-0.027778in}}%
\pgfusepath{stroke,fill}%
}%
\begin{pgfscope}%
\pgfsys@transformshift{7.727344in}{0.663790in}%
\pgfsys@useobject{currentmarker}{}%
\end{pgfscope}%
\end{pgfscope}%
\begin{pgfscope}%
\pgfpathrectangle{\pgfqpoint{0.679222in}{0.663790in}}{\pgfqpoint{9.300000in}{6.795000in}}%
\pgfusepath{clip}%
\pgfsetbuttcap%
\pgfsetroundjoin%
\pgfsetlinewidth{0.803000pt}%
\definecolor{currentstroke}{rgb}{0.501961,0.501961,0.501961}%
\pgfsetstrokecolor{currentstroke}%
\pgfsetstrokeopacity{0.400000}%
\pgfsetdash{{2.960000pt}{1.280000pt}}{0.000000pt}%
\pgfpathmoveto{\pgfqpoint{8.141940in}{0.663790in}}%
\pgfpathlineto{\pgfqpoint{8.141940in}{7.458790in}}%
\pgfusepath{stroke}%
\end{pgfscope}%
\begin{pgfscope}%
\pgfsetbuttcap%
\pgfsetroundjoin%
\definecolor{currentfill}{rgb}{0.000000,0.000000,0.000000}%
\pgfsetfillcolor{currentfill}%
\pgfsetlinewidth{0.602250pt}%
\definecolor{currentstroke}{rgb}{0.000000,0.000000,0.000000}%
\pgfsetstrokecolor{currentstroke}%
\pgfsetdash{}{0pt}%
\pgfsys@defobject{currentmarker}{\pgfqpoint{0.000000in}{-0.027778in}}{\pgfqpoint{0.000000in}{0.000000in}}{%
\pgfpathmoveto{\pgfqpoint{0.000000in}{0.000000in}}%
\pgfpathlineto{\pgfqpoint{0.000000in}{-0.027778in}}%
\pgfusepath{stroke,fill}%
}%
\begin{pgfscope}%
\pgfsys@transformshift{8.141940in}{0.663790in}%
\pgfsys@useobject{currentmarker}{}%
\end{pgfscope}%
\end{pgfscope}%
\begin{pgfscope}%
\pgfpathrectangle{\pgfqpoint{0.679222in}{0.663790in}}{\pgfqpoint{9.300000in}{6.795000in}}%
\pgfusepath{clip}%
\pgfsetbuttcap%
\pgfsetroundjoin%
\pgfsetlinewidth{0.803000pt}%
\definecolor{currentstroke}{rgb}{0.501961,0.501961,0.501961}%
\pgfsetstrokecolor{currentstroke}%
\pgfsetstrokeopacity{0.400000}%
\pgfsetdash{{2.960000pt}{1.280000pt}}{0.000000pt}%
\pgfpathmoveto{\pgfqpoint{8.349238in}{0.663790in}}%
\pgfpathlineto{\pgfqpoint{8.349238in}{7.458790in}}%
\pgfusepath{stroke}%
\end{pgfscope}%
\begin{pgfscope}%
\pgfsetbuttcap%
\pgfsetroundjoin%
\definecolor{currentfill}{rgb}{0.000000,0.000000,0.000000}%
\pgfsetfillcolor{currentfill}%
\pgfsetlinewidth{0.602250pt}%
\definecolor{currentstroke}{rgb}{0.000000,0.000000,0.000000}%
\pgfsetstrokecolor{currentstroke}%
\pgfsetdash{}{0pt}%
\pgfsys@defobject{currentmarker}{\pgfqpoint{0.000000in}{-0.027778in}}{\pgfqpoint{0.000000in}{0.000000in}}{%
\pgfpathmoveto{\pgfqpoint{0.000000in}{0.000000in}}%
\pgfpathlineto{\pgfqpoint{0.000000in}{-0.027778in}}%
\pgfusepath{stroke,fill}%
}%
\begin{pgfscope}%
\pgfsys@transformshift{8.349238in}{0.663790in}%
\pgfsys@useobject{currentmarker}{}%
\end{pgfscope}%
\end{pgfscope}%
\begin{pgfscope}%
\pgfpathrectangle{\pgfqpoint{0.679222in}{0.663790in}}{\pgfqpoint{9.300000in}{6.795000in}}%
\pgfusepath{clip}%
\pgfsetbuttcap%
\pgfsetroundjoin%
\pgfsetlinewidth{0.803000pt}%
\definecolor{currentstroke}{rgb}{0.501961,0.501961,0.501961}%
\pgfsetstrokecolor{currentstroke}%
\pgfsetstrokeopacity{0.400000}%
\pgfsetdash{{2.960000pt}{1.280000pt}}{0.000000pt}%
\pgfpathmoveto{\pgfqpoint{8.556535in}{0.663790in}}%
\pgfpathlineto{\pgfqpoint{8.556535in}{7.458790in}}%
\pgfusepath{stroke}%
\end{pgfscope}%
\begin{pgfscope}%
\pgfsetbuttcap%
\pgfsetroundjoin%
\definecolor{currentfill}{rgb}{0.000000,0.000000,0.000000}%
\pgfsetfillcolor{currentfill}%
\pgfsetlinewidth{0.602250pt}%
\definecolor{currentstroke}{rgb}{0.000000,0.000000,0.000000}%
\pgfsetstrokecolor{currentstroke}%
\pgfsetdash{}{0pt}%
\pgfsys@defobject{currentmarker}{\pgfqpoint{0.000000in}{-0.027778in}}{\pgfqpoint{0.000000in}{0.000000in}}{%
\pgfpathmoveto{\pgfqpoint{0.000000in}{0.000000in}}%
\pgfpathlineto{\pgfqpoint{0.000000in}{-0.027778in}}%
\pgfusepath{stroke,fill}%
}%
\begin{pgfscope}%
\pgfsys@transformshift{8.556535in}{0.663790in}%
\pgfsys@useobject{currentmarker}{}%
\end{pgfscope}%
\end{pgfscope}%
\begin{pgfscope}%
\pgfpathrectangle{\pgfqpoint{0.679222in}{0.663790in}}{\pgfqpoint{9.300000in}{6.795000in}}%
\pgfusepath{clip}%
\pgfsetbuttcap%
\pgfsetroundjoin%
\pgfsetlinewidth{0.803000pt}%
\definecolor{currentstroke}{rgb}{0.501961,0.501961,0.501961}%
\pgfsetstrokecolor{currentstroke}%
\pgfsetstrokeopacity{0.400000}%
\pgfsetdash{{2.960000pt}{1.280000pt}}{0.000000pt}%
\pgfpathmoveto{\pgfqpoint{8.763833in}{0.663790in}}%
\pgfpathlineto{\pgfqpoint{8.763833in}{7.458790in}}%
\pgfusepath{stroke}%
\end{pgfscope}%
\begin{pgfscope}%
\pgfsetbuttcap%
\pgfsetroundjoin%
\definecolor{currentfill}{rgb}{0.000000,0.000000,0.000000}%
\pgfsetfillcolor{currentfill}%
\pgfsetlinewidth{0.602250pt}%
\definecolor{currentstroke}{rgb}{0.000000,0.000000,0.000000}%
\pgfsetstrokecolor{currentstroke}%
\pgfsetdash{}{0pt}%
\pgfsys@defobject{currentmarker}{\pgfqpoint{0.000000in}{-0.027778in}}{\pgfqpoint{0.000000in}{0.000000in}}{%
\pgfpathmoveto{\pgfqpoint{0.000000in}{0.000000in}}%
\pgfpathlineto{\pgfqpoint{0.000000in}{-0.027778in}}%
\pgfusepath{stroke,fill}%
}%
\begin{pgfscope}%
\pgfsys@transformshift{8.763833in}{0.663790in}%
\pgfsys@useobject{currentmarker}{}%
\end{pgfscope}%
\end{pgfscope}%
\begin{pgfscope}%
\pgfpathrectangle{\pgfqpoint{0.679222in}{0.663790in}}{\pgfqpoint{9.300000in}{6.795000in}}%
\pgfusepath{clip}%
\pgfsetbuttcap%
\pgfsetroundjoin%
\pgfsetlinewidth{0.803000pt}%
\definecolor{currentstroke}{rgb}{0.501961,0.501961,0.501961}%
\pgfsetstrokecolor{currentstroke}%
\pgfsetstrokeopacity{0.400000}%
\pgfsetdash{{2.960000pt}{1.280000pt}}{0.000000pt}%
\pgfpathmoveto{\pgfqpoint{9.178428in}{0.663790in}}%
\pgfpathlineto{\pgfqpoint{9.178428in}{7.458790in}}%
\pgfusepath{stroke}%
\end{pgfscope}%
\begin{pgfscope}%
\pgfsetbuttcap%
\pgfsetroundjoin%
\definecolor{currentfill}{rgb}{0.000000,0.000000,0.000000}%
\pgfsetfillcolor{currentfill}%
\pgfsetlinewidth{0.602250pt}%
\definecolor{currentstroke}{rgb}{0.000000,0.000000,0.000000}%
\pgfsetstrokecolor{currentstroke}%
\pgfsetdash{}{0pt}%
\pgfsys@defobject{currentmarker}{\pgfqpoint{0.000000in}{-0.027778in}}{\pgfqpoint{0.000000in}{0.000000in}}{%
\pgfpathmoveto{\pgfqpoint{0.000000in}{0.000000in}}%
\pgfpathlineto{\pgfqpoint{0.000000in}{-0.027778in}}%
\pgfusepath{stroke,fill}%
}%
\begin{pgfscope}%
\pgfsys@transformshift{9.178428in}{0.663790in}%
\pgfsys@useobject{currentmarker}{}%
\end{pgfscope}%
\end{pgfscope}%
\begin{pgfscope}%
\pgfpathrectangle{\pgfqpoint{0.679222in}{0.663790in}}{\pgfqpoint{9.300000in}{6.795000in}}%
\pgfusepath{clip}%
\pgfsetbuttcap%
\pgfsetroundjoin%
\pgfsetlinewidth{0.803000pt}%
\definecolor{currentstroke}{rgb}{0.501961,0.501961,0.501961}%
\pgfsetstrokecolor{currentstroke}%
\pgfsetstrokeopacity{0.400000}%
\pgfsetdash{{2.960000pt}{1.280000pt}}{0.000000pt}%
\pgfpathmoveto{\pgfqpoint{9.385726in}{0.663790in}}%
\pgfpathlineto{\pgfqpoint{9.385726in}{7.458790in}}%
\pgfusepath{stroke}%
\end{pgfscope}%
\begin{pgfscope}%
\pgfsetbuttcap%
\pgfsetroundjoin%
\definecolor{currentfill}{rgb}{0.000000,0.000000,0.000000}%
\pgfsetfillcolor{currentfill}%
\pgfsetlinewidth{0.602250pt}%
\definecolor{currentstroke}{rgb}{0.000000,0.000000,0.000000}%
\pgfsetstrokecolor{currentstroke}%
\pgfsetdash{}{0pt}%
\pgfsys@defobject{currentmarker}{\pgfqpoint{0.000000in}{-0.027778in}}{\pgfqpoint{0.000000in}{0.000000in}}{%
\pgfpathmoveto{\pgfqpoint{0.000000in}{0.000000in}}%
\pgfpathlineto{\pgfqpoint{0.000000in}{-0.027778in}}%
\pgfusepath{stroke,fill}%
}%
\begin{pgfscope}%
\pgfsys@transformshift{9.385726in}{0.663790in}%
\pgfsys@useobject{currentmarker}{}%
\end{pgfscope}%
\end{pgfscope}%
\begin{pgfscope}%
\pgfpathrectangle{\pgfqpoint{0.679222in}{0.663790in}}{\pgfqpoint{9.300000in}{6.795000in}}%
\pgfusepath{clip}%
\pgfsetbuttcap%
\pgfsetroundjoin%
\pgfsetlinewidth{0.803000pt}%
\definecolor{currentstroke}{rgb}{0.501961,0.501961,0.501961}%
\pgfsetstrokecolor{currentstroke}%
\pgfsetstrokeopacity{0.400000}%
\pgfsetdash{{2.960000pt}{1.280000pt}}{0.000000pt}%
\pgfpathmoveto{\pgfqpoint{9.593024in}{0.663790in}}%
\pgfpathlineto{\pgfqpoint{9.593024in}{7.458790in}}%
\pgfusepath{stroke}%
\end{pgfscope}%
\begin{pgfscope}%
\pgfsetbuttcap%
\pgfsetroundjoin%
\definecolor{currentfill}{rgb}{0.000000,0.000000,0.000000}%
\pgfsetfillcolor{currentfill}%
\pgfsetlinewidth{0.602250pt}%
\definecolor{currentstroke}{rgb}{0.000000,0.000000,0.000000}%
\pgfsetstrokecolor{currentstroke}%
\pgfsetdash{}{0pt}%
\pgfsys@defobject{currentmarker}{\pgfqpoint{0.000000in}{-0.027778in}}{\pgfqpoint{0.000000in}{0.000000in}}{%
\pgfpathmoveto{\pgfqpoint{0.000000in}{0.000000in}}%
\pgfpathlineto{\pgfqpoint{0.000000in}{-0.027778in}}%
\pgfusepath{stroke,fill}%
}%
\begin{pgfscope}%
\pgfsys@transformshift{9.593024in}{0.663790in}%
\pgfsys@useobject{currentmarker}{}%
\end{pgfscope}%
\end{pgfscope}%
\begin{pgfscope}%
\pgfpathrectangle{\pgfqpoint{0.679222in}{0.663790in}}{\pgfqpoint{9.300000in}{6.795000in}}%
\pgfusepath{clip}%
\pgfsetbuttcap%
\pgfsetroundjoin%
\pgfsetlinewidth{0.803000pt}%
\definecolor{currentstroke}{rgb}{0.501961,0.501961,0.501961}%
\pgfsetstrokecolor{currentstroke}%
\pgfsetstrokeopacity{0.400000}%
\pgfsetdash{{2.960000pt}{1.280000pt}}{0.000000pt}%
\pgfpathmoveto{\pgfqpoint{9.800322in}{0.663790in}}%
\pgfpathlineto{\pgfqpoint{9.800322in}{7.458790in}}%
\pgfusepath{stroke}%
\end{pgfscope}%
\begin{pgfscope}%
\pgfsetbuttcap%
\pgfsetroundjoin%
\definecolor{currentfill}{rgb}{0.000000,0.000000,0.000000}%
\pgfsetfillcolor{currentfill}%
\pgfsetlinewidth{0.602250pt}%
\definecolor{currentstroke}{rgb}{0.000000,0.000000,0.000000}%
\pgfsetstrokecolor{currentstroke}%
\pgfsetdash{}{0pt}%
\pgfsys@defobject{currentmarker}{\pgfqpoint{0.000000in}{-0.027778in}}{\pgfqpoint{0.000000in}{0.000000in}}{%
\pgfpathmoveto{\pgfqpoint{0.000000in}{0.000000in}}%
\pgfpathlineto{\pgfqpoint{0.000000in}{-0.027778in}}%
\pgfusepath{stroke,fill}%
}%
\begin{pgfscope}%
\pgfsys@transformshift{9.800322in}{0.663790in}%
\pgfsys@useobject{currentmarker}{}%
\end{pgfscope}%
\end{pgfscope}%
\begin{pgfscope}%
\definecolor{textcolor}{rgb}{0.000000,0.000000,0.000000}%
\pgfsetstrokecolor{textcolor}%
\pgfsetfillcolor{textcolor}%
\pgftext[x=5.329222in,y=0.387555in,,top]{\color{textcolor}\rmfamily\fontsize{20.000000}{24.000000}\selectfont Solar Capacity [GW]}%
\end{pgfscope}%
\begin{pgfscope}%
\pgfpathrectangle{\pgfqpoint{0.679222in}{0.663790in}}{\pgfqpoint{9.300000in}{6.795000in}}%
\pgfusepath{clip}%
\pgfsetrectcap%
\pgfsetroundjoin%
\pgfsetlinewidth{0.803000pt}%
\definecolor{currentstroke}{rgb}{0.000000,0.000000,0.000000}%
\pgfsetstrokecolor{currentstroke}%
\pgfsetstrokeopacity{0.400000}%
\pgfsetdash{}{0pt}%
\pgfpathmoveto{\pgfqpoint{0.679222in}{0.663790in}}%
\pgfpathlineto{\pgfqpoint{9.979222in}{0.663790in}}%
\pgfusepath{stroke}%
\end{pgfscope}%
\begin{pgfscope}%
\pgfsetbuttcap%
\pgfsetroundjoin%
\definecolor{currentfill}{rgb}{0.000000,0.000000,0.000000}%
\pgfsetfillcolor{currentfill}%
\pgfsetlinewidth{0.803000pt}%
\definecolor{currentstroke}{rgb}{0.000000,0.000000,0.000000}%
\pgfsetstrokecolor{currentstroke}%
\pgfsetdash{}{0pt}%
\pgfsys@defobject{currentmarker}{\pgfqpoint{-0.048611in}{0.000000in}}{\pgfqpoint{-0.000000in}{0.000000in}}{%
\pgfpathmoveto{\pgfqpoint{-0.000000in}{0.000000in}}%
\pgfpathlineto{\pgfqpoint{-0.048611in}{0.000000in}}%
\pgfusepath{stroke,fill}%
}%
\begin{pgfscope}%
\pgfsys@transformshift{0.679222in}{0.663790in}%
\pgfsys@useobject{currentmarker}{}%
\end{pgfscope}%
\end{pgfscope}%
\begin{pgfscope}%
\definecolor{textcolor}{rgb}{0.000000,0.000000,0.000000}%
\pgfsetstrokecolor{textcolor}%
\pgfsetfillcolor{textcolor}%
\pgftext[x=0.512555in, y=0.615564in, left, base]{\color{textcolor}\rmfamily\fontsize{10.000000}{12.000000}\selectfont \(\displaystyle {0}\)}%
\end{pgfscope}%
\begin{pgfscope}%
\pgfpathrectangle{\pgfqpoint{0.679222in}{0.663790in}}{\pgfqpoint{9.300000in}{6.795000in}}%
\pgfusepath{clip}%
\pgfsetrectcap%
\pgfsetroundjoin%
\pgfsetlinewidth{0.803000pt}%
\definecolor{currentstroke}{rgb}{0.000000,0.000000,0.000000}%
\pgfsetstrokecolor{currentstroke}%
\pgfsetstrokeopacity{0.400000}%
\pgfsetdash{}{0pt}%
\pgfpathmoveto{\pgfqpoint{0.679222in}{1.724017in}}%
\pgfpathlineto{\pgfqpoint{9.979222in}{1.724017in}}%
\pgfusepath{stroke}%
\end{pgfscope}%
\begin{pgfscope}%
\pgfsetbuttcap%
\pgfsetroundjoin%
\definecolor{currentfill}{rgb}{0.000000,0.000000,0.000000}%
\pgfsetfillcolor{currentfill}%
\pgfsetlinewidth{0.803000pt}%
\definecolor{currentstroke}{rgb}{0.000000,0.000000,0.000000}%
\pgfsetstrokecolor{currentstroke}%
\pgfsetdash{}{0pt}%
\pgfsys@defobject{currentmarker}{\pgfqpoint{-0.048611in}{0.000000in}}{\pgfqpoint{-0.000000in}{0.000000in}}{%
\pgfpathmoveto{\pgfqpoint{-0.000000in}{0.000000in}}%
\pgfpathlineto{\pgfqpoint{-0.048611in}{0.000000in}}%
\pgfusepath{stroke,fill}%
}%
\begin{pgfscope}%
\pgfsys@transformshift{0.679222in}{1.724017in}%
\pgfsys@useobject{currentmarker}{}%
\end{pgfscope}%
\end{pgfscope}%
\begin{pgfscope}%
\definecolor{textcolor}{rgb}{0.000000,0.000000,0.000000}%
\pgfsetstrokecolor{textcolor}%
\pgfsetfillcolor{textcolor}%
\pgftext[x=0.512555in, y=1.675792in, left, base]{\color{textcolor}\rmfamily\fontsize{10.000000}{12.000000}\selectfont \(\displaystyle {2}\)}%
\end{pgfscope}%
\begin{pgfscope}%
\pgfpathrectangle{\pgfqpoint{0.679222in}{0.663790in}}{\pgfqpoint{9.300000in}{6.795000in}}%
\pgfusepath{clip}%
\pgfsetrectcap%
\pgfsetroundjoin%
\pgfsetlinewidth{0.803000pt}%
\definecolor{currentstroke}{rgb}{0.000000,0.000000,0.000000}%
\pgfsetstrokecolor{currentstroke}%
\pgfsetstrokeopacity{0.400000}%
\pgfsetdash{}{0pt}%
\pgfpathmoveto{\pgfqpoint{0.679222in}{2.784245in}}%
\pgfpathlineto{\pgfqpoint{9.979222in}{2.784245in}}%
\pgfusepath{stroke}%
\end{pgfscope}%
\begin{pgfscope}%
\pgfsetbuttcap%
\pgfsetroundjoin%
\definecolor{currentfill}{rgb}{0.000000,0.000000,0.000000}%
\pgfsetfillcolor{currentfill}%
\pgfsetlinewidth{0.803000pt}%
\definecolor{currentstroke}{rgb}{0.000000,0.000000,0.000000}%
\pgfsetstrokecolor{currentstroke}%
\pgfsetdash{}{0pt}%
\pgfsys@defobject{currentmarker}{\pgfqpoint{-0.048611in}{0.000000in}}{\pgfqpoint{-0.000000in}{0.000000in}}{%
\pgfpathmoveto{\pgfqpoint{-0.000000in}{0.000000in}}%
\pgfpathlineto{\pgfqpoint{-0.048611in}{0.000000in}}%
\pgfusepath{stroke,fill}%
}%
\begin{pgfscope}%
\pgfsys@transformshift{0.679222in}{2.784245in}%
\pgfsys@useobject{currentmarker}{}%
\end{pgfscope}%
\end{pgfscope}%
\begin{pgfscope}%
\definecolor{textcolor}{rgb}{0.000000,0.000000,0.000000}%
\pgfsetstrokecolor{textcolor}%
\pgfsetfillcolor{textcolor}%
\pgftext[x=0.512555in, y=2.736019in, left, base]{\color{textcolor}\rmfamily\fontsize{10.000000}{12.000000}\selectfont \(\displaystyle {4}\)}%
\end{pgfscope}%
\begin{pgfscope}%
\pgfpathrectangle{\pgfqpoint{0.679222in}{0.663790in}}{\pgfqpoint{9.300000in}{6.795000in}}%
\pgfusepath{clip}%
\pgfsetrectcap%
\pgfsetroundjoin%
\pgfsetlinewidth{0.803000pt}%
\definecolor{currentstroke}{rgb}{0.000000,0.000000,0.000000}%
\pgfsetstrokecolor{currentstroke}%
\pgfsetstrokeopacity{0.400000}%
\pgfsetdash{}{0pt}%
\pgfpathmoveto{\pgfqpoint{0.679222in}{3.844472in}}%
\pgfpathlineto{\pgfqpoint{9.979222in}{3.844472in}}%
\pgfusepath{stroke}%
\end{pgfscope}%
\begin{pgfscope}%
\pgfsetbuttcap%
\pgfsetroundjoin%
\definecolor{currentfill}{rgb}{0.000000,0.000000,0.000000}%
\pgfsetfillcolor{currentfill}%
\pgfsetlinewidth{0.803000pt}%
\definecolor{currentstroke}{rgb}{0.000000,0.000000,0.000000}%
\pgfsetstrokecolor{currentstroke}%
\pgfsetdash{}{0pt}%
\pgfsys@defobject{currentmarker}{\pgfqpoint{-0.048611in}{0.000000in}}{\pgfqpoint{-0.000000in}{0.000000in}}{%
\pgfpathmoveto{\pgfqpoint{-0.000000in}{0.000000in}}%
\pgfpathlineto{\pgfqpoint{-0.048611in}{0.000000in}}%
\pgfusepath{stroke,fill}%
}%
\begin{pgfscope}%
\pgfsys@transformshift{0.679222in}{3.844472in}%
\pgfsys@useobject{currentmarker}{}%
\end{pgfscope}%
\end{pgfscope}%
\begin{pgfscope}%
\definecolor{textcolor}{rgb}{0.000000,0.000000,0.000000}%
\pgfsetstrokecolor{textcolor}%
\pgfsetfillcolor{textcolor}%
\pgftext[x=0.512555in, y=3.796247in, left, base]{\color{textcolor}\rmfamily\fontsize{10.000000}{12.000000}\selectfont \(\displaystyle {6}\)}%
\end{pgfscope}%
\begin{pgfscope}%
\pgfpathrectangle{\pgfqpoint{0.679222in}{0.663790in}}{\pgfqpoint{9.300000in}{6.795000in}}%
\pgfusepath{clip}%
\pgfsetrectcap%
\pgfsetroundjoin%
\pgfsetlinewidth{0.803000pt}%
\definecolor{currentstroke}{rgb}{0.000000,0.000000,0.000000}%
\pgfsetstrokecolor{currentstroke}%
\pgfsetstrokeopacity{0.400000}%
\pgfsetdash{}{0pt}%
\pgfpathmoveto{\pgfqpoint{0.679222in}{4.904700in}}%
\pgfpathlineto{\pgfqpoint{9.979222in}{4.904700in}}%
\pgfusepath{stroke}%
\end{pgfscope}%
\begin{pgfscope}%
\pgfsetbuttcap%
\pgfsetroundjoin%
\definecolor{currentfill}{rgb}{0.000000,0.000000,0.000000}%
\pgfsetfillcolor{currentfill}%
\pgfsetlinewidth{0.803000pt}%
\definecolor{currentstroke}{rgb}{0.000000,0.000000,0.000000}%
\pgfsetstrokecolor{currentstroke}%
\pgfsetdash{}{0pt}%
\pgfsys@defobject{currentmarker}{\pgfqpoint{-0.048611in}{0.000000in}}{\pgfqpoint{-0.000000in}{0.000000in}}{%
\pgfpathmoveto{\pgfqpoint{-0.000000in}{0.000000in}}%
\pgfpathlineto{\pgfqpoint{-0.048611in}{0.000000in}}%
\pgfusepath{stroke,fill}%
}%
\begin{pgfscope}%
\pgfsys@transformshift{0.679222in}{4.904700in}%
\pgfsys@useobject{currentmarker}{}%
\end{pgfscope}%
\end{pgfscope}%
\begin{pgfscope}%
\definecolor{textcolor}{rgb}{0.000000,0.000000,0.000000}%
\pgfsetstrokecolor{textcolor}%
\pgfsetfillcolor{textcolor}%
\pgftext[x=0.512555in, y=4.856474in, left, base]{\color{textcolor}\rmfamily\fontsize{10.000000}{12.000000}\selectfont \(\displaystyle {8}\)}%
\end{pgfscope}%
\begin{pgfscope}%
\pgfpathrectangle{\pgfqpoint{0.679222in}{0.663790in}}{\pgfqpoint{9.300000in}{6.795000in}}%
\pgfusepath{clip}%
\pgfsetrectcap%
\pgfsetroundjoin%
\pgfsetlinewidth{0.803000pt}%
\definecolor{currentstroke}{rgb}{0.000000,0.000000,0.000000}%
\pgfsetstrokecolor{currentstroke}%
\pgfsetstrokeopacity{0.400000}%
\pgfsetdash{}{0pt}%
\pgfpathmoveto{\pgfqpoint{0.679222in}{5.964927in}}%
\pgfpathlineto{\pgfqpoint{9.979222in}{5.964927in}}%
\pgfusepath{stroke}%
\end{pgfscope}%
\begin{pgfscope}%
\pgfsetbuttcap%
\pgfsetroundjoin%
\definecolor{currentfill}{rgb}{0.000000,0.000000,0.000000}%
\pgfsetfillcolor{currentfill}%
\pgfsetlinewidth{0.803000pt}%
\definecolor{currentstroke}{rgb}{0.000000,0.000000,0.000000}%
\pgfsetstrokecolor{currentstroke}%
\pgfsetdash{}{0pt}%
\pgfsys@defobject{currentmarker}{\pgfqpoint{-0.048611in}{0.000000in}}{\pgfqpoint{-0.000000in}{0.000000in}}{%
\pgfpathmoveto{\pgfqpoint{-0.000000in}{0.000000in}}%
\pgfpathlineto{\pgfqpoint{-0.048611in}{0.000000in}}%
\pgfusepath{stroke,fill}%
}%
\begin{pgfscope}%
\pgfsys@transformshift{0.679222in}{5.964927in}%
\pgfsys@useobject{currentmarker}{}%
\end{pgfscope}%
\end{pgfscope}%
\begin{pgfscope}%
\definecolor{textcolor}{rgb}{0.000000,0.000000,0.000000}%
\pgfsetstrokecolor{textcolor}%
\pgfsetfillcolor{textcolor}%
\pgftext[x=0.443111in, y=5.916702in, left, base]{\color{textcolor}\rmfamily\fontsize{10.000000}{12.000000}\selectfont \(\displaystyle {10}\)}%
\end{pgfscope}%
\begin{pgfscope}%
\pgfpathrectangle{\pgfqpoint{0.679222in}{0.663790in}}{\pgfqpoint{9.300000in}{6.795000in}}%
\pgfusepath{clip}%
\pgfsetrectcap%
\pgfsetroundjoin%
\pgfsetlinewidth{0.803000pt}%
\definecolor{currentstroke}{rgb}{0.000000,0.000000,0.000000}%
\pgfsetstrokecolor{currentstroke}%
\pgfsetstrokeopacity{0.400000}%
\pgfsetdash{}{0pt}%
\pgfpathmoveto{\pgfqpoint{0.679222in}{7.025155in}}%
\pgfpathlineto{\pgfqpoint{9.979222in}{7.025155in}}%
\pgfusepath{stroke}%
\end{pgfscope}%
\begin{pgfscope}%
\pgfsetbuttcap%
\pgfsetroundjoin%
\definecolor{currentfill}{rgb}{0.000000,0.000000,0.000000}%
\pgfsetfillcolor{currentfill}%
\pgfsetlinewidth{0.803000pt}%
\definecolor{currentstroke}{rgb}{0.000000,0.000000,0.000000}%
\pgfsetstrokecolor{currentstroke}%
\pgfsetdash{}{0pt}%
\pgfsys@defobject{currentmarker}{\pgfqpoint{-0.048611in}{0.000000in}}{\pgfqpoint{-0.000000in}{0.000000in}}{%
\pgfpathmoveto{\pgfqpoint{-0.000000in}{0.000000in}}%
\pgfpathlineto{\pgfqpoint{-0.048611in}{0.000000in}}%
\pgfusepath{stroke,fill}%
}%
\begin{pgfscope}%
\pgfsys@transformshift{0.679222in}{7.025155in}%
\pgfsys@useobject{currentmarker}{}%
\end{pgfscope}%
\end{pgfscope}%
\begin{pgfscope}%
\definecolor{textcolor}{rgb}{0.000000,0.000000,0.000000}%
\pgfsetstrokecolor{textcolor}%
\pgfsetfillcolor{textcolor}%
\pgftext[x=0.443111in, y=6.976929in, left, base]{\color{textcolor}\rmfamily\fontsize{10.000000}{12.000000}\selectfont \(\displaystyle {12}\)}%
\end{pgfscope}%
\begin{pgfscope}%
\pgfpathrectangle{\pgfqpoint{0.679222in}{0.663790in}}{\pgfqpoint{9.300000in}{6.795000in}}%
\pgfusepath{clip}%
\pgfsetbuttcap%
\pgfsetroundjoin%
\pgfsetlinewidth{0.803000pt}%
\definecolor{currentstroke}{rgb}{0.501961,0.501961,0.501961}%
\pgfsetstrokecolor{currentstroke}%
\pgfsetstrokeopacity{0.400000}%
\pgfsetdash{{2.960000pt}{1.280000pt}}{0.000000pt}%
\pgfpathmoveto{\pgfqpoint{0.679222in}{0.928847in}}%
\pgfpathlineto{\pgfqpoint{9.979222in}{0.928847in}}%
\pgfusepath{stroke}%
\end{pgfscope}%
\begin{pgfscope}%
\pgfsetbuttcap%
\pgfsetroundjoin%
\definecolor{currentfill}{rgb}{0.000000,0.000000,0.000000}%
\pgfsetfillcolor{currentfill}%
\pgfsetlinewidth{0.602250pt}%
\definecolor{currentstroke}{rgb}{0.000000,0.000000,0.000000}%
\pgfsetstrokecolor{currentstroke}%
\pgfsetdash{}{0pt}%
\pgfsys@defobject{currentmarker}{\pgfqpoint{-0.027778in}{0.000000in}}{\pgfqpoint{-0.000000in}{0.000000in}}{%
\pgfpathmoveto{\pgfqpoint{-0.000000in}{0.000000in}}%
\pgfpathlineto{\pgfqpoint{-0.027778in}{0.000000in}}%
\pgfusepath{stroke,fill}%
}%
\begin{pgfscope}%
\pgfsys@transformshift{0.679222in}{0.928847in}%
\pgfsys@useobject{currentmarker}{}%
\end{pgfscope}%
\end{pgfscope}%
\begin{pgfscope}%
\pgfpathrectangle{\pgfqpoint{0.679222in}{0.663790in}}{\pgfqpoint{9.300000in}{6.795000in}}%
\pgfusepath{clip}%
\pgfsetbuttcap%
\pgfsetroundjoin%
\pgfsetlinewidth{0.803000pt}%
\definecolor{currentstroke}{rgb}{0.501961,0.501961,0.501961}%
\pgfsetstrokecolor{currentstroke}%
\pgfsetstrokeopacity{0.400000}%
\pgfsetdash{{2.960000pt}{1.280000pt}}{0.000000pt}%
\pgfpathmoveto{\pgfqpoint{0.679222in}{1.193903in}}%
\pgfpathlineto{\pgfqpoint{9.979222in}{1.193903in}}%
\pgfusepath{stroke}%
\end{pgfscope}%
\begin{pgfscope}%
\pgfsetbuttcap%
\pgfsetroundjoin%
\definecolor{currentfill}{rgb}{0.000000,0.000000,0.000000}%
\pgfsetfillcolor{currentfill}%
\pgfsetlinewidth{0.602250pt}%
\definecolor{currentstroke}{rgb}{0.000000,0.000000,0.000000}%
\pgfsetstrokecolor{currentstroke}%
\pgfsetdash{}{0pt}%
\pgfsys@defobject{currentmarker}{\pgfqpoint{-0.027778in}{0.000000in}}{\pgfqpoint{-0.000000in}{0.000000in}}{%
\pgfpathmoveto{\pgfqpoint{-0.000000in}{0.000000in}}%
\pgfpathlineto{\pgfqpoint{-0.027778in}{0.000000in}}%
\pgfusepath{stroke,fill}%
}%
\begin{pgfscope}%
\pgfsys@transformshift{0.679222in}{1.193903in}%
\pgfsys@useobject{currentmarker}{}%
\end{pgfscope}%
\end{pgfscope}%
\begin{pgfscope}%
\pgfpathrectangle{\pgfqpoint{0.679222in}{0.663790in}}{\pgfqpoint{9.300000in}{6.795000in}}%
\pgfusepath{clip}%
\pgfsetbuttcap%
\pgfsetroundjoin%
\pgfsetlinewidth{0.803000pt}%
\definecolor{currentstroke}{rgb}{0.501961,0.501961,0.501961}%
\pgfsetstrokecolor{currentstroke}%
\pgfsetstrokeopacity{0.400000}%
\pgfsetdash{{2.960000pt}{1.280000pt}}{0.000000pt}%
\pgfpathmoveto{\pgfqpoint{0.679222in}{1.458960in}}%
\pgfpathlineto{\pgfqpoint{9.979222in}{1.458960in}}%
\pgfusepath{stroke}%
\end{pgfscope}%
\begin{pgfscope}%
\pgfsetbuttcap%
\pgfsetroundjoin%
\definecolor{currentfill}{rgb}{0.000000,0.000000,0.000000}%
\pgfsetfillcolor{currentfill}%
\pgfsetlinewidth{0.602250pt}%
\definecolor{currentstroke}{rgb}{0.000000,0.000000,0.000000}%
\pgfsetstrokecolor{currentstroke}%
\pgfsetdash{}{0pt}%
\pgfsys@defobject{currentmarker}{\pgfqpoint{-0.027778in}{0.000000in}}{\pgfqpoint{-0.000000in}{0.000000in}}{%
\pgfpathmoveto{\pgfqpoint{-0.000000in}{0.000000in}}%
\pgfpathlineto{\pgfqpoint{-0.027778in}{0.000000in}}%
\pgfusepath{stroke,fill}%
}%
\begin{pgfscope}%
\pgfsys@transformshift{0.679222in}{1.458960in}%
\pgfsys@useobject{currentmarker}{}%
\end{pgfscope}%
\end{pgfscope}%
\begin{pgfscope}%
\pgfpathrectangle{\pgfqpoint{0.679222in}{0.663790in}}{\pgfqpoint{9.300000in}{6.795000in}}%
\pgfusepath{clip}%
\pgfsetbuttcap%
\pgfsetroundjoin%
\pgfsetlinewidth{0.803000pt}%
\definecolor{currentstroke}{rgb}{0.501961,0.501961,0.501961}%
\pgfsetstrokecolor{currentstroke}%
\pgfsetstrokeopacity{0.400000}%
\pgfsetdash{{2.960000pt}{1.280000pt}}{0.000000pt}%
\pgfpathmoveto{\pgfqpoint{0.679222in}{1.989074in}}%
\pgfpathlineto{\pgfqpoint{9.979222in}{1.989074in}}%
\pgfusepath{stroke}%
\end{pgfscope}%
\begin{pgfscope}%
\pgfsetbuttcap%
\pgfsetroundjoin%
\definecolor{currentfill}{rgb}{0.000000,0.000000,0.000000}%
\pgfsetfillcolor{currentfill}%
\pgfsetlinewidth{0.602250pt}%
\definecolor{currentstroke}{rgb}{0.000000,0.000000,0.000000}%
\pgfsetstrokecolor{currentstroke}%
\pgfsetdash{}{0pt}%
\pgfsys@defobject{currentmarker}{\pgfqpoint{-0.027778in}{0.000000in}}{\pgfqpoint{-0.000000in}{0.000000in}}{%
\pgfpathmoveto{\pgfqpoint{-0.000000in}{0.000000in}}%
\pgfpathlineto{\pgfqpoint{-0.027778in}{0.000000in}}%
\pgfusepath{stroke,fill}%
}%
\begin{pgfscope}%
\pgfsys@transformshift{0.679222in}{1.989074in}%
\pgfsys@useobject{currentmarker}{}%
\end{pgfscope}%
\end{pgfscope}%
\begin{pgfscope}%
\pgfpathrectangle{\pgfqpoint{0.679222in}{0.663790in}}{\pgfqpoint{9.300000in}{6.795000in}}%
\pgfusepath{clip}%
\pgfsetbuttcap%
\pgfsetroundjoin%
\pgfsetlinewidth{0.803000pt}%
\definecolor{currentstroke}{rgb}{0.501961,0.501961,0.501961}%
\pgfsetstrokecolor{currentstroke}%
\pgfsetstrokeopacity{0.400000}%
\pgfsetdash{{2.960000pt}{1.280000pt}}{0.000000pt}%
\pgfpathmoveto{\pgfqpoint{0.679222in}{2.254131in}}%
\pgfpathlineto{\pgfqpoint{9.979222in}{2.254131in}}%
\pgfusepath{stroke}%
\end{pgfscope}%
\begin{pgfscope}%
\pgfsetbuttcap%
\pgfsetroundjoin%
\definecolor{currentfill}{rgb}{0.000000,0.000000,0.000000}%
\pgfsetfillcolor{currentfill}%
\pgfsetlinewidth{0.602250pt}%
\definecolor{currentstroke}{rgb}{0.000000,0.000000,0.000000}%
\pgfsetstrokecolor{currentstroke}%
\pgfsetdash{}{0pt}%
\pgfsys@defobject{currentmarker}{\pgfqpoint{-0.027778in}{0.000000in}}{\pgfqpoint{-0.000000in}{0.000000in}}{%
\pgfpathmoveto{\pgfqpoint{-0.000000in}{0.000000in}}%
\pgfpathlineto{\pgfqpoint{-0.027778in}{0.000000in}}%
\pgfusepath{stroke,fill}%
}%
\begin{pgfscope}%
\pgfsys@transformshift{0.679222in}{2.254131in}%
\pgfsys@useobject{currentmarker}{}%
\end{pgfscope}%
\end{pgfscope}%
\begin{pgfscope}%
\pgfpathrectangle{\pgfqpoint{0.679222in}{0.663790in}}{\pgfqpoint{9.300000in}{6.795000in}}%
\pgfusepath{clip}%
\pgfsetbuttcap%
\pgfsetroundjoin%
\pgfsetlinewidth{0.803000pt}%
\definecolor{currentstroke}{rgb}{0.501961,0.501961,0.501961}%
\pgfsetstrokecolor{currentstroke}%
\pgfsetstrokeopacity{0.400000}%
\pgfsetdash{{2.960000pt}{1.280000pt}}{0.000000pt}%
\pgfpathmoveto{\pgfqpoint{0.679222in}{2.519188in}}%
\pgfpathlineto{\pgfqpoint{9.979222in}{2.519188in}}%
\pgfusepath{stroke}%
\end{pgfscope}%
\begin{pgfscope}%
\pgfsetbuttcap%
\pgfsetroundjoin%
\definecolor{currentfill}{rgb}{0.000000,0.000000,0.000000}%
\pgfsetfillcolor{currentfill}%
\pgfsetlinewidth{0.602250pt}%
\definecolor{currentstroke}{rgb}{0.000000,0.000000,0.000000}%
\pgfsetstrokecolor{currentstroke}%
\pgfsetdash{}{0pt}%
\pgfsys@defobject{currentmarker}{\pgfqpoint{-0.027778in}{0.000000in}}{\pgfqpoint{-0.000000in}{0.000000in}}{%
\pgfpathmoveto{\pgfqpoint{-0.000000in}{0.000000in}}%
\pgfpathlineto{\pgfqpoint{-0.027778in}{0.000000in}}%
\pgfusepath{stroke,fill}%
}%
\begin{pgfscope}%
\pgfsys@transformshift{0.679222in}{2.519188in}%
\pgfsys@useobject{currentmarker}{}%
\end{pgfscope}%
\end{pgfscope}%
\begin{pgfscope}%
\pgfpathrectangle{\pgfqpoint{0.679222in}{0.663790in}}{\pgfqpoint{9.300000in}{6.795000in}}%
\pgfusepath{clip}%
\pgfsetbuttcap%
\pgfsetroundjoin%
\pgfsetlinewidth{0.803000pt}%
\definecolor{currentstroke}{rgb}{0.501961,0.501961,0.501961}%
\pgfsetstrokecolor{currentstroke}%
\pgfsetstrokeopacity{0.400000}%
\pgfsetdash{{2.960000pt}{1.280000pt}}{0.000000pt}%
\pgfpathmoveto{\pgfqpoint{0.679222in}{3.049302in}}%
\pgfpathlineto{\pgfqpoint{9.979222in}{3.049302in}}%
\pgfusepath{stroke}%
\end{pgfscope}%
\begin{pgfscope}%
\pgfsetbuttcap%
\pgfsetroundjoin%
\definecolor{currentfill}{rgb}{0.000000,0.000000,0.000000}%
\pgfsetfillcolor{currentfill}%
\pgfsetlinewidth{0.602250pt}%
\definecolor{currentstroke}{rgb}{0.000000,0.000000,0.000000}%
\pgfsetstrokecolor{currentstroke}%
\pgfsetdash{}{0pt}%
\pgfsys@defobject{currentmarker}{\pgfqpoint{-0.027778in}{0.000000in}}{\pgfqpoint{-0.000000in}{0.000000in}}{%
\pgfpathmoveto{\pgfqpoint{-0.000000in}{0.000000in}}%
\pgfpathlineto{\pgfqpoint{-0.027778in}{0.000000in}}%
\pgfusepath{stroke,fill}%
}%
\begin{pgfscope}%
\pgfsys@transformshift{0.679222in}{3.049302in}%
\pgfsys@useobject{currentmarker}{}%
\end{pgfscope}%
\end{pgfscope}%
\begin{pgfscope}%
\pgfpathrectangle{\pgfqpoint{0.679222in}{0.663790in}}{\pgfqpoint{9.300000in}{6.795000in}}%
\pgfusepath{clip}%
\pgfsetbuttcap%
\pgfsetroundjoin%
\pgfsetlinewidth{0.803000pt}%
\definecolor{currentstroke}{rgb}{0.501961,0.501961,0.501961}%
\pgfsetstrokecolor{currentstroke}%
\pgfsetstrokeopacity{0.400000}%
\pgfsetdash{{2.960000pt}{1.280000pt}}{0.000000pt}%
\pgfpathmoveto{\pgfqpoint{0.679222in}{3.314358in}}%
\pgfpathlineto{\pgfqpoint{9.979222in}{3.314358in}}%
\pgfusepath{stroke}%
\end{pgfscope}%
\begin{pgfscope}%
\pgfsetbuttcap%
\pgfsetroundjoin%
\definecolor{currentfill}{rgb}{0.000000,0.000000,0.000000}%
\pgfsetfillcolor{currentfill}%
\pgfsetlinewidth{0.602250pt}%
\definecolor{currentstroke}{rgb}{0.000000,0.000000,0.000000}%
\pgfsetstrokecolor{currentstroke}%
\pgfsetdash{}{0pt}%
\pgfsys@defobject{currentmarker}{\pgfqpoint{-0.027778in}{0.000000in}}{\pgfqpoint{-0.000000in}{0.000000in}}{%
\pgfpathmoveto{\pgfqpoint{-0.000000in}{0.000000in}}%
\pgfpathlineto{\pgfqpoint{-0.027778in}{0.000000in}}%
\pgfusepath{stroke,fill}%
}%
\begin{pgfscope}%
\pgfsys@transformshift{0.679222in}{3.314358in}%
\pgfsys@useobject{currentmarker}{}%
\end{pgfscope}%
\end{pgfscope}%
\begin{pgfscope}%
\pgfpathrectangle{\pgfqpoint{0.679222in}{0.663790in}}{\pgfqpoint{9.300000in}{6.795000in}}%
\pgfusepath{clip}%
\pgfsetbuttcap%
\pgfsetroundjoin%
\pgfsetlinewidth{0.803000pt}%
\definecolor{currentstroke}{rgb}{0.501961,0.501961,0.501961}%
\pgfsetstrokecolor{currentstroke}%
\pgfsetstrokeopacity{0.400000}%
\pgfsetdash{{2.960000pt}{1.280000pt}}{0.000000pt}%
\pgfpathmoveto{\pgfqpoint{0.679222in}{3.579415in}}%
\pgfpathlineto{\pgfqpoint{9.979222in}{3.579415in}}%
\pgfusepath{stroke}%
\end{pgfscope}%
\begin{pgfscope}%
\pgfsetbuttcap%
\pgfsetroundjoin%
\definecolor{currentfill}{rgb}{0.000000,0.000000,0.000000}%
\pgfsetfillcolor{currentfill}%
\pgfsetlinewidth{0.602250pt}%
\definecolor{currentstroke}{rgb}{0.000000,0.000000,0.000000}%
\pgfsetstrokecolor{currentstroke}%
\pgfsetdash{}{0pt}%
\pgfsys@defobject{currentmarker}{\pgfqpoint{-0.027778in}{0.000000in}}{\pgfqpoint{-0.000000in}{0.000000in}}{%
\pgfpathmoveto{\pgfqpoint{-0.000000in}{0.000000in}}%
\pgfpathlineto{\pgfqpoint{-0.027778in}{0.000000in}}%
\pgfusepath{stroke,fill}%
}%
\begin{pgfscope}%
\pgfsys@transformshift{0.679222in}{3.579415in}%
\pgfsys@useobject{currentmarker}{}%
\end{pgfscope}%
\end{pgfscope}%
\begin{pgfscope}%
\pgfpathrectangle{\pgfqpoint{0.679222in}{0.663790in}}{\pgfqpoint{9.300000in}{6.795000in}}%
\pgfusepath{clip}%
\pgfsetbuttcap%
\pgfsetroundjoin%
\pgfsetlinewidth{0.803000pt}%
\definecolor{currentstroke}{rgb}{0.501961,0.501961,0.501961}%
\pgfsetstrokecolor{currentstroke}%
\pgfsetstrokeopacity{0.400000}%
\pgfsetdash{{2.960000pt}{1.280000pt}}{0.000000pt}%
\pgfpathmoveto{\pgfqpoint{0.679222in}{4.109529in}}%
\pgfpathlineto{\pgfqpoint{9.979222in}{4.109529in}}%
\pgfusepath{stroke}%
\end{pgfscope}%
\begin{pgfscope}%
\pgfsetbuttcap%
\pgfsetroundjoin%
\definecolor{currentfill}{rgb}{0.000000,0.000000,0.000000}%
\pgfsetfillcolor{currentfill}%
\pgfsetlinewidth{0.602250pt}%
\definecolor{currentstroke}{rgb}{0.000000,0.000000,0.000000}%
\pgfsetstrokecolor{currentstroke}%
\pgfsetdash{}{0pt}%
\pgfsys@defobject{currentmarker}{\pgfqpoint{-0.027778in}{0.000000in}}{\pgfqpoint{-0.000000in}{0.000000in}}{%
\pgfpathmoveto{\pgfqpoint{-0.000000in}{0.000000in}}%
\pgfpathlineto{\pgfqpoint{-0.027778in}{0.000000in}}%
\pgfusepath{stroke,fill}%
}%
\begin{pgfscope}%
\pgfsys@transformshift{0.679222in}{4.109529in}%
\pgfsys@useobject{currentmarker}{}%
\end{pgfscope}%
\end{pgfscope}%
\begin{pgfscope}%
\pgfpathrectangle{\pgfqpoint{0.679222in}{0.663790in}}{\pgfqpoint{9.300000in}{6.795000in}}%
\pgfusepath{clip}%
\pgfsetbuttcap%
\pgfsetroundjoin%
\pgfsetlinewidth{0.803000pt}%
\definecolor{currentstroke}{rgb}{0.501961,0.501961,0.501961}%
\pgfsetstrokecolor{currentstroke}%
\pgfsetstrokeopacity{0.400000}%
\pgfsetdash{{2.960000pt}{1.280000pt}}{0.000000pt}%
\pgfpathmoveto{\pgfqpoint{0.679222in}{4.374586in}}%
\pgfpathlineto{\pgfqpoint{9.979222in}{4.374586in}}%
\pgfusepath{stroke}%
\end{pgfscope}%
\begin{pgfscope}%
\pgfsetbuttcap%
\pgfsetroundjoin%
\definecolor{currentfill}{rgb}{0.000000,0.000000,0.000000}%
\pgfsetfillcolor{currentfill}%
\pgfsetlinewidth{0.602250pt}%
\definecolor{currentstroke}{rgb}{0.000000,0.000000,0.000000}%
\pgfsetstrokecolor{currentstroke}%
\pgfsetdash{}{0pt}%
\pgfsys@defobject{currentmarker}{\pgfqpoint{-0.027778in}{0.000000in}}{\pgfqpoint{-0.000000in}{0.000000in}}{%
\pgfpathmoveto{\pgfqpoint{-0.000000in}{0.000000in}}%
\pgfpathlineto{\pgfqpoint{-0.027778in}{0.000000in}}%
\pgfusepath{stroke,fill}%
}%
\begin{pgfscope}%
\pgfsys@transformshift{0.679222in}{4.374586in}%
\pgfsys@useobject{currentmarker}{}%
\end{pgfscope}%
\end{pgfscope}%
\begin{pgfscope}%
\pgfpathrectangle{\pgfqpoint{0.679222in}{0.663790in}}{\pgfqpoint{9.300000in}{6.795000in}}%
\pgfusepath{clip}%
\pgfsetbuttcap%
\pgfsetroundjoin%
\pgfsetlinewidth{0.803000pt}%
\definecolor{currentstroke}{rgb}{0.501961,0.501961,0.501961}%
\pgfsetstrokecolor{currentstroke}%
\pgfsetstrokeopacity{0.400000}%
\pgfsetdash{{2.960000pt}{1.280000pt}}{0.000000pt}%
\pgfpathmoveto{\pgfqpoint{0.679222in}{4.639643in}}%
\pgfpathlineto{\pgfqpoint{9.979222in}{4.639643in}}%
\pgfusepath{stroke}%
\end{pgfscope}%
\begin{pgfscope}%
\pgfsetbuttcap%
\pgfsetroundjoin%
\definecolor{currentfill}{rgb}{0.000000,0.000000,0.000000}%
\pgfsetfillcolor{currentfill}%
\pgfsetlinewidth{0.602250pt}%
\definecolor{currentstroke}{rgb}{0.000000,0.000000,0.000000}%
\pgfsetstrokecolor{currentstroke}%
\pgfsetdash{}{0pt}%
\pgfsys@defobject{currentmarker}{\pgfqpoint{-0.027778in}{0.000000in}}{\pgfqpoint{-0.000000in}{0.000000in}}{%
\pgfpathmoveto{\pgfqpoint{-0.000000in}{0.000000in}}%
\pgfpathlineto{\pgfqpoint{-0.027778in}{0.000000in}}%
\pgfusepath{stroke,fill}%
}%
\begin{pgfscope}%
\pgfsys@transformshift{0.679222in}{4.639643in}%
\pgfsys@useobject{currentmarker}{}%
\end{pgfscope}%
\end{pgfscope}%
\begin{pgfscope}%
\pgfpathrectangle{\pgfqpoint{0.679222in}{0.663790in}}{\pgfqpoint{9.300000in}{6.795000in}}%
\pgfusepath{clip}%
\pgfsetbuttcap%
\pgfsetroundjoin%
\pgfsetlinewidth{0.803000pt}%
\definecolor{currentstroke}{rgb}{0.501961,0.501961,0.501961}%
\pgfsetstrokecolor{currentstroke}%
\pgfsetstrokeopacity{0.400000}%
\pgfsetdash{{2.960000pt}{1.280000pt}}{0.000000pt}%
\pgfpathmoveto{\pgfqpoint{0.679222in}{5.169757in}}%
\pgfpathlineto{\pgfqpoint{9.979222in}{5.169757in}}%
\pgfusepath{stroke}%
\end{pgfscope}%
\begin{pgfscope}%
\pgfsetbuttcap%
\pgfsetroundjoin%
\definecolor{currentfill}{rgb}{0.000000,0.000000,0.000000}%
\pgfsetfillcolor{currentfill}%
\pgfsetlinewidth{0.602250pt}%
\definecolor{currentstroke}{rgb}{0.000000,0.000000,0.000000}%
\pgfsetstrokecolor{currentstroke}%
\pgfsetdash{}{0pt}%
\pgfsys@defobject{currentmarker}{\pgfqpoint{-0.027778in}{0.000000in}}{\pgfqpoint{-0.000000in}{0.000000in}}{%
\pgfpathmoveto{\pgfqpoint{-0.000000in}{0.000000in}}%
\pgfpathlineto{\pgfqpoint{-0.027778in}{0.000000in}}%
\pgfusepath{stroke,fill}%
}%
\begin{pgfscope}%
\pgfsys@transformshift{0.679222in}{5.169757in}%
\pgfsys@useobject{currentmarker}{}%
\end{pgfscope}%
\end{pgfscope}%
\begin{pgfscope}%
\pgfpathrectangle{\pgfqpoint{0.679222in}{0.663790in}}{\pgfqpoint{9.300000in}{6.795000in}}%
\pgfusepath{clip}%
\pgfsetbuttcap%
\pgfsetroundjoin%
\pgfsetlinewidth{0.803000pt}%
\definecolor{currentstroke}{rgb}{0.501961,0.501961,0.501961}%
\pgfsetstrokecolor{currentstroke}%
\pgfsetstrokeopacity{0.400000}%
\pgfsetdash{{2.960000pt}{1.280000pt}}{0.000000pt}%
\pgfpathmoveto{\pgfqpoint{0.679222in}{5.434813in}}%
\pgfpathlineto{\pgfqpoint{9.979222in}{5.434813in}}%
\pgfusepath{stroke}%
\end{pgfscope}%
\begin{pgfscope}%
\pgfsetbuttcap%
\pgfsetroundjoin%
\definecolor{currentfill}{rgb}{0.000000,0.000000,0.000000}%
\pgfsetfillcolor{currentfill}%
\pgfsetlinewidth{0.602250pt}%
\definecolor{currentstroke}{rgb}{0.000000,0.000000,0.000000}%
\pgfsetstrokecolor{currentstroke}%
\pgfsetdash{}{0pt}%
\pgfsys@defobject{currentmarker}{\pgfqpoint{-0.027778in}{0.000000in}}{\pgfqpoint{-0.000000in}{0.000000in}}{%
\pgfpathmoveto{\pgfqpoint{-0.000000in}{0.000000in}}%
\pgfpathlineto{\pgfqpoint{-0.027778in}{0.000000in}}%
\pgfusepath{stroke,fill}%
}%
\begin{pgfscope}%
\pgfsys@transformshift{0.679222in}{5.434813in}%
\pgfsys@useobject{currentmarker}{}%
\end{pgfscope}%
\end{pgfscope}%
\begin{pgfscope}%
\pgfpathrectangle{\pgfqpoint{0.679222in}{0.663790in}}{\pgfqpoint{9.300000in}{6.795000in}}%
\pgfusepath{clip}%
\pgfsetbuttcap%
\pgfsetroundjoin%
\pgfsetlinewidth{0.803000pt}%
\definecolor{currentstroke}{rgb}{0.501961,0.501961,0.501961}%
\pgfsetstrokecolor{currentstroke}%
\pgfsetstrokeopacity{0.400000}%
\pgfsetdash{{2.960000pt}{1.280000pt}}{0.000000pt}%
\pgfpathmoveto{\pgfqpoint{0.679222in}{5.699870in}}%
\pgfpathlineto{\pgfqpoint{9.979222in}{5.699870in}}%
\pgfusepath{stroke}%
\end{pgfscope}%
\begin{pgfscope}%
\pgfsetbuttcap%
\pgfsetroundjoin%
\definecolor{currentfill}{rgb}{0.000000,0.000000,0.000000}%
\pgfsetfillcolor{currentfill}%
\pgfsetlinewidth{0.602250pt}%
\definecolor{currentstroke}{rgb}{0.000000,0.000000,0.000000}%
\pgfsetstrokecolor{currentstroke}%
\pgfsetdash{}{0pt}%
\pgfsys@defobject{currentmarker}{\pgfqpoint{-0.027778in}{0.000000in}}{\pgfqpoint{-0.000000in}{0.000000in}}{%
\pgfpathmoveto{\pgfqpoint{-0.000000in}{0.000000in}}%
\pgfpathlineto{\pgfqpoint{-0.027778in}{0.000000in}}%
\pgfusepath{stroke,fill}%
}%
\begin{pgfscope}%
\pgfsys@transformshift{0.679222in}{5.699870in}%
\pgfsys@useobject{currentmarker}{}%
\end{pgfscope}%
\end{pgfscope}%
\begin{pgfscope}%
\pgfpathrectangle{\pgfqpoint{0.679222in}{0.663790in}}{\pgfqpoint{9.300000in}{6.795000in}}%
\pgfusepath{clip}%
\pgfsetbuttcap%
\pgfsetroundjoin%
\pgfsetlinewidth{0.803000pt}%
\definecolor{currentstroke}{rgb}{0.501961,0.501961,0.501961}%
\pgfsetstrokecolor{currentstroke}%
\pgfsetstrokeopacity{0.400000}%
\pgfsetdash{{2.960000pt}{1.280000pt}}{0.000000pt}%
\pgfpathmoveto{\pgfqpoint{0.679222in}{6.229984in}}%
\pgfpathlineto{\pgfqpoint{9.979222in}{6.229984in}}%
\pgfusepath{stroke}%
\end{pgfscope}%
\begin{pgfscope}%
\pgfsetbuttcap%
\pgfsetroundjoin%
\definecolor{currentfill}{rgb}{0.000000,0.000000,0.000000}%
\pgfsetfillcolor{currentfill}%
\pgfsetlinewidth{0.602250pt}%
\definecolor{currentstroke}{rgb}{0.000000,0.000000,0.000000}%
\pgfsetstrokecolor{currentstroke}%
\pgfsetdash{}{0pt}%
\pgfsys@defobject{currentmarker}{\pgfqpoint{-0.027778in}{0.000000in}}{\pgfqpoint{-0.000000in}{0.000000in}}{%
\pgfpathmoveto{\pgfqpoint{-0.000000in}{0.000000in}}%
\pgfpathlineto{\pgfqpoint{-0.027778in}{0.000000in}}%
\pgfusepath{stroke,fill}%
}%
\begin{pgfscope}%
\pgfsys@transformshift{0.679222in}{6.229984in}%
\pgfsys@useobject{currentmarker}{}%
\end{pgfscope}%
\end{pgfscope}%
\begin{pgfscope}%
\pgfpathrectangle{\pgfqpoint{0.679222in}{0.663790in}}{\pgfqpoint{9.300000in}{6.795000in}}%
\pgfusepath{clip}%
\pgfsetbuttcap%
\pgfsetroundjoin%
\pgfsetlinewidth{0.803000pt}%
\definecolor{currentstroke}{rgb}{0.501961,0.501961,0.501961}%
\pgfsetstrokecolor{currentstroke}%
\pgfsetstrokeopacity{0.400000}%
\pgfsetdash{{2.960000pt}{1.280000pt}}{0.000000pt}%
\pgfpathmoveto{\pgfqpoint{0.679222in}{6.495041in}}%
\pgfpathlineto{\pgfqpoint{9.979222in}{6.495041in}}%
\pgfusepath{stroke}%
\end{pgfscope}%
\begin{pgfscope}%
\pgfsetbuttcap%
\pgfsetroundjoin%
\definecolor{currentfill}{rgb}{0.000000,0.000000,0.000000}%
\pgfsetfillcolor{currentfill}%
\pgfsetlinewidth{0.602250pt}%
\definecolor{currentstroke}{rgb}{0.000000,0.000000,0.000000}%
\pgfsetstrokecolor{currentstroke}%
\pgfsetdash{}{0pt}%
\pgfsys@defobject{currentmarker}{\pgfqpoint{-0.027778in}{0.000000in}}{\pgfqpoint{-0.000000in}{0.000000in}}{%
\pgfpathmoveto{\pgfqpoint{-0.000000in}{0.000000in}}%
\pgfpathlineto{\pgfqpoint{-0.027778in}{0.000000in}}%
\pgfusepath{stroke,fill}%
}%
\begin{pgfscope}%
\pgfsys@transformshift{0.679222in}{6.495041in}%
\pgfsys@useobject{currentmarker}{}%
\end{pgfscope}%
\end{pgfscope}%
\begin{pgfscope}%
\pgfpathrectangle{\pgfqpoint{0.679222in}{0.663790in}}{\pgfqpoint{9.300000in}{6.795000in}}%
\pgfusepath{clip}%
\pgfsetbuttcap%
\pgfsetroundjoin%
\pgfsetlinewidth{0.803000pt}%
\definecolor{currentstroke}{rgb}{0.501961,0.501961,0.501961}%
\pgfsetstrokecolor{currentstroke}%
\pgfsetstrokeopacity{0.400000}%
\pgfsetdash{{2.960000pt}{1.280000pt}}{0.000000pt}%
\pgfpathmoveto{\pgfqpoint{0.679222in}{6.760098in}}%
\pgfpathlineto{\pgfqpoint{9.979222in}{6.760098in}}%
\pgfusepath{stroke}%
\end{pgfscope}%
\begin{pgfscope}%
\pgfsetbuttcap%
\pgfsetroundjoin%
\definecolor{currentfill}{rgb}{0.000000,0.000000,0.000000}%
\pgfsetfillcolor{currentfill}%
\pgfsetlinewidth{0.602250pt}%
\definecolor{currentstroke}{rgb}{0.000000,0.000000,0.000000}%
\pgfsetstrokecolor{currentstroke}%
\pgfsetdash{}{0pt}%
\pgfsys@defobject{currentmarker}{\pgfqpoint{-0.027778in}{0.000000in}}{\pgfqpoint{-0.000000in}{0.000000in}}{%
\pgfpathmoveto{\pgfqpoint{-0.000000in}{0.000000in}}%
\pgfpathlineto{\pgfqpoint{-0.027778in}{0.000000in}}%
\pgfusepath{stroke,fill}%
}%
\begin{pgfscope}%
\pgfsys@transformshift{0.679222in}{6.760098in}%
\pgfsys@useobject{currentmarker}{}%
\end{pgfscope}%
\end{pgfscope}%
\begin{pgfscope}%
\pgfpathrectangle{\pgfqpoint{0.679222in}{0.663790in}}{\pgfqpoint{9.300000in}{6.795000in}}%
\pgfusepath{clip}%
\pgfsetbuttcap%
\pgfsetroundjoin%
\pgfsetlinewidth{0.803000pt}%
\definecolor{currentstroke}{rgb}{0.501961,0.501961,0.501961}%
\pgfsetstrokecolor{currentstroke}%
\pgfsetstrokeopacity{0.400000}%
\pgfsetdash{{2.960000pt}{1.280000pt}}{0.000000pt}%
\pgfpathmoveto{\pgfqpoint{0.679222in}{7.290211in}}%
\pgfpathlineto{\pgfqpoint{9.979222in}{7.290211in}}%
\pgfusepath{stroke}%
\end{pgfscope}%
\begin{pgfscope}%
\pgfsetbuttcap%
\pgfsetroundjoin%
\definecolor{currentfill}{rgb}{0.000000,0.000000,0.000000}%
\pgfsetfillcolor{currentfill}%
\pgfsetlinewidth{0.602250pt}%
\definecolor{currentstroke}{rgb}{0.000000,0.000000,0.000000}%
\pgfsetstrokecolor{currentstroke}%
\pgfsetdash{}{0pt}%
\pgfsys@defobject{currentmarker}{\pgfqpoint{-0.027778in}{0.000000in}}{\pgfqpoint{-0.000000in}{0.000000in}}{%
\pgfpathmoveto{\pgfqpoint{-0.000000in}{0.000000in}}%
\pgfpathlineto{\pgfqpoint{-0.027778in}{0.000000in}}%
\pgfusepath{stroke,fill}%
}%
\begin{pgfscope}%
\pgfsys@transformshift{0.679222in}{7.290211in}%
\pgfsys@useobject{currentmarker}{}%
\end{pgfscope}%
\end{pgfscope}%
\begin{pgfscope}%
\definecolor{textcolor}{rgb}{0.000000,0.000000,0.000000}%
\pgfsetstrokecolor{textcolor}%
\pgfsetfillcolor{textcolor}%
\pgftext[x=0.387555in,y=4.061290in,,bottom,rotate=90.000000]{\color{textcolor}\rmfamily\fontsize{20.000000}{24.000000}\selectfont Wind Capacity [GW]}%
\end{pgfscope}%
\begin{pgfscope}%
\pgfpathrectangle{\pgfqpoint{0.679222in}{0.663790in}}{\pgfqpoint{9.300000in}{6.795000in}}%
\pgfusepath{clip}%
\pgfsetbuttcap%
\pgfsetroundjoin%
\pgfsetlinewidth{1.505625pt}%
\definecolor{currentstroke}{rgb}{0.121569,0.466667,0.705882}%
\pgfsetstrokecolor{currentstroke}%
\pgfsetdash{{5.550000pt}{2.400000pt}}{0.000000pt}%
\pgfpathmoveto{\pgfqpoint{9.979222in}{0.663790in}}%
\pgfpathlineto{\pgfqpoint{0.679222in}{7.458790in}}%
\pgfusepath{stroke}%
\end{pgfscope}%
\begin{pgfscope}%
\pgfpathrectangle{\pgfqpoint{0.679222in}{0.663790in}}{\pgfqpoint{9.300000in}{6.795000in}}%
\pgfusepath{clip}%
\pgfsetrectcap%
\pgfsetroundjoin%
\pgfsetlinewidth{1.505625pt}%
\definecolor{currentstroke}{rgb}{0.839216,0.152941,0.156863}%
\pgfsetstrokecolor{currentstroke}%
\pgfsetdash{}{0pt}%
\pgfpathmoveto{\pgfqpoint{8.221522in}{0.663790in}}%
\pgfpathlineto{\pgfqpoint{2.874198in}{7.461290in}}%
\pgfusepath{stroke}%
\end{pgfscope}%
\begin{pgfscope}%
\pgfpathrectangle{\pgfqpoint{0.679222in}{0.663790in}}{\pgfqpoint{9.300000in}{6.795000in}}%
\pgfusepath{clip}%
\pgfsetrectcap%
\pgfsetroundjoin%
\pgfsetlinewidth{1.505625pt}%
\definecolor{currentstroke}{rgb}{0.000000,0.000000,0.000000}%
\pgfsetstrokecolor{currentstroke}%
\pgfsetdash{}{0pt}%
\pgfpathmoveto{\pgfqpoint{8.975752in}{0.663790in}}%
\pgfpathlineto{\pgfqpoint{3.628428in}{7.461290in}}%
\pgfusepath{stroke}%
\end{pgfscope}%
\begin{pgfscope}%
\pgfsetrectcap%
\pgfsetmiterjoin%
\pgfsetlinewidth{0.803000pt}%
\definecolor{currentstroke}{rgb}{0.000000,0.000000,0.000000}%
\pgfsetstrokecolor{currentstroke}%
\pgfsetdash{}{0pt}%
\pgfpathmoveto{\pgfqpoint{0.679222in}{0.663790in}}%
\pgfpathlineto{\pgfqpoint{0.679222in}{7.458790in}}%
\pgfusepath{stroke}%
\end{pgfscope}%
\begin{pgfscope}%
\pgfsetrectcap%
\pgfsetmiterjoin%
\pgfsetlinewidth{0.803000pt}%
\definecolor{currentstroke}{rgb}{0.000000,0.000000,0.000000}%
\pgfsetstrokecolor{currentstroke}%
\pgfsetdash{}{0pt}%
\pgfpathmoveto{\pgfqpoint{9.979222in}{0.663790in}}%
\pgfpathlineto{\pgfqpoint{9.979222in}{7.458790in}}%
\pgfusepath{stroke}%
\end{pgfscope}%
\begin{pgfscope}%
\pgfsetrectcap%
\pgfsetmiterjoin%
\pgfsetlinewidth{0.803000pt}%
\definecolor{currentstroke}{rgb}{0.000000,0.000000,0.000000}%
\pgfsetstrokecolor{currentstroke}%
\pgfsetdash{}{0pt}%
\pgfpathmoveto{\pgfqpoint{0.679222in}{0.663790in}}%
\pgfpathlineto{\pgfqpoint{9.979222in}{0.663790in}}%
\pgfusepath{stroke}%
\end{pgfscope}%
\begin{pgfscope}%
\pgfsetrectcap%
\pgfsetmiterjoin%
\pgfsetlinewidth{0.803000pt}%
\definecolor{currentstroke}{rgb}{0.000000,0.000000,0.000000}%
\pgfsetstrokecolor{currentstroke}%
\pgfsetdash{}{0pt}%
\pgfpathmoveto{\pgfqpoint{0.679222in}{7.458790in}}%
\pgfpathlineto{\pgfqpoint{9.979222in}{7.458790in}}%
\pgfusepath{stroke}%
\end{pgfscope}%
\begin{pgfscope}%
\pgfsetroundcap%
\pgfsetroundjoin%
\pgfsetlinewidth{1.003750pt}%
\definecolor{currentstroke}{rgb}{0.000000,0.000000,0.000000}%
\pgfsetstrokecolor{currentstroke}%
\pgfsetdash{}{0pt}%
\pgfpathmoveto{\pgfqpoint{5.032998in}{3.349452in}}%
\pgfpathquadraticcurveto{\pgfqpoint{5.413860in}{3.514572in}}{\pgfqpoint{5.785047in}{3.675496in}}%
\pgfusepath{stroke}%
\end{pgfscope}%
\begin{pgfscope}%
\pgfsetroundcap%
\pgfsetroundjoin%
\pgfsetlinewidth{1.003750pt}%
\definecolor{currentstroke}{rgb}{0.000000,0.000000,0.000000}%
\pgfsetstrokecolor{currentstroke}%
\pgfsetdash{}{0pt}%
\pgfpathmoveto{\pgfqpoint{5.686617in}{3.706999in}}%
\pgfpathlineto{\pgfqpoint{5.785047in}{3.675496in}}%
\pgfpathlineto{\pgfqpoint{5.740757in}{3.582119in}}%
\pgfusepath{stroke}%
\end{pgfscope}%
\begin{pgfscope}%
\definecolor{textcolor}{rgb}{0.000000,0.000000,0.000000}%
\pgfsetstrokecolor{textcolor}%
\pgfsetfillcolor{textcolor}%
\pgftext[x=3.747228in,y=3.160625in,left,base]{\color{textcolor}\rmfamily\fontsize{14.000000}{16.800000}\selectfont Optimum: (12.4, 5.71)}%
\end{pgfscope}%
\begin{pgfscope}%
\pgfsetroundcap%
\pgfsetroundjoin%
\pgfsetlinewidth{1.003750pt}%
\definecolor{currentstroke}{rgb}{0.000000,0.000000,0.000000}%
\pgfsetstrokecolor{currentstroke}%
\pgfsetdash{}{0pt}%
\pgfpathmoveto{\pgfqpoint{6.077904in}{2.056455in}}%
\pgfpathquadraticcurveto{\pgfqpoint{6.816533in}{2.226394in}}{\pgfqpoint{7.544884in}{2.393969in}}%
\pgfusepath{stroke}%
\end{pgfscope}%
\begin{pgfscope}%
\pgfsetroundcap%
\pgfsetroundjoin%
\pgfsetlinewidth{1.003750pt}%
\definecolor{currentstroke}{rgb}{0.000000,0.000000,0.000000}%
\pgfsetstrokecolor{currentstroke}%
\pgfsetdash{}{0pt}%
\pgfpathmoveto{\pgfqpoint{7.453828in}{2.442853in}}%
\pgfpathlineto{\pgfqpoint{7.544884in}{2.393969in}}%
\pgfpathlineto{\pgfqpoint{7.484346in}{2.310207in}}%
\pgfusepath{stroke}%
\end{pgfscope}%
\begin{pgfscope}%
\definecolor{textcolor}{rgb}{0.000000,0.000000,0.000000}%
\pgfsetstrokecolor{textcolor}%
\pgfsetfillcolor{textcolor}%
\pgftext[x=4.265473in,y=1.872449in,left,base]{\color{textcolor}\rmfamily\fontsize{14.000000}{16.800000}\selectfont MGA Solution: (16.65, 3.28)}%
\end{pgfscope}%
\begin{pgfscope}%
\definecolor{textcolor}{rgb}{0.000000,0.000000,0.000000}%
\pgfsetstrokecolor{textcolor}%
\pgfsetfillcolor{textcolor}%
\pgftext[x=1.950329in, y=7.897300in, left, base]{\color{textcolor}\rmfamily\fontsize{24.000000}{28.800000}\selectfont What's the Optimal Mix of Renewable Energy? }%
\end{pgfscope}%
\begin{pgfscope}%
\definecolor{textcolor}{rgb}{0.000000,0.000000,0.000000}%
\pgfsetstrokecolor{textcolor}%
\pgfsetfillcolor{textcolor}%
\pgftext[x=1.621529in, y=7.542123in, left, base]{\color{textcolor}\rmfamily\fontsize{24.000000}{28.800000}\selectfont Demonstration of Modelling to Generate Alternatives}%
\end{pgfscope}%
\begin{pgfscope}%
\pgfsetbuttcap%
\pgfsetmiterjoin%
\definecolor{currentfill}{rgb}{0.248235,0.248235,0.248235}%
\pgfsetfillcolor{currentfill}%
\pgfsetfillopacity{0.500000}%
\pgfsetlinewidth{1.003750pt}%
\definecolor{currentstroke}{rgb}{0.248235,0.248235,0.248235}%
\pgfsetstrokecolor{currentstroke}%
\pgfsetstrokeopacity{0.500000}%
\pgfsetdash{}{0pt}%
\pgfpathmoveto{\pgfqpoint{0.843111in}{0.733234in}}%
\pgfpathlineto{\pgfqpoint{3.412762in}{0.733234in}}%
\pgfpathquadraticcurveto{\pgfqpoint{3.451651in}{0.733234in}}{\pgfqpoint{3.451651in}{0.772123in}}%
\pgfpathlineto{\pgfqpoint{3.451651in}{2.688787in}}%
\pgfpathquadraticcurveto{\pgfqpoint{3.451651in}{2.727676in}}{\pgfqpoint{3.412762in}{2.727676in}}%
\pgfpathlineto{\pgfqpoint{0.843111in}{2.727676in}}%
\pgfpathquadraticcurveto{\pgfqpoint{0.804222in}{2.727676in}}{\pgfqpoint{0.804222in}{2.688787in}}%
\pgfpathlineto{\pgfqpoint{0.804222in}{0.772123in}}%
\pgfpathquadraticcurveto{\pgfqpoint{0.804222in}{0.733234in}}{\pgfqpoint{0.843111in}{0.733234in}}%
\pgfpathclose%
\pgfusepath{stroke,fill}%
\end{pgfscope}%
\begin{pgfscope}%
\pgfsetbuttcap%
\pgfsetmiterjoin%
\definecolor{currentfill}{rgb}{0.827451,0.827451,0.827451}%
\pgfsetfillcolor{currentfill}%
\pgfsetlinewidth{1.003750pt}%
\definecolor{currentstroke}{rgb}{0.800000,0.800000,0.800000}%
\pgfsetstrokecolor{currentstroke}%
\pgfsetdash{}{0pt}%
\pgfpathmoveto{\pgfqpoint{0.815333in}{0.761012in}}%
\pgfpathlineto{\pgfqpoint{3.384984in}{0.761012in}}%
\pgfpathquadraticcurveto{\pgfqpoint{3.423873in}{0.761012in}}{\pgfqpoint{3.423873in}{0.799901in}}%
\pgfpathlineto{\pgfqpoint{3.423873in}{2.716565in}}%
\pgfpathquadraticcurveto{\pgfqpoint{3.423873in}{2.755454in}}{\pgfqpoint{3.384984in}{2.755454in}}%
\pgfpathlineto{\pgfqpoint{0.815333in}{2.755454in}}%
\pgfpathquadraticcurveto{\pgfqpoint{0.776445in}{2.755454in}}{\pgfqpoint{0.776445in}{2.716565in}}%
\pgfpathlineto{\pgfqpoint{0.776445in}{0.799901in}}%
\pgfpathquadraticcurveto{\pgfqpoint{0.776445in}{0.761012in}}{\pgfqpoint{0.815333in}{0.761012in}}%
\pgfpathclose%
\pgfusepath{stroke,fill}%
\end{pgfscope}%
\begin{pgfscope}%
\pgfsetbuttcap%
\pgfsetroundjoin%
\pgfsetlinewidth{1.505625pt}%
\definecolor{currentstroke}{rgb}{0.121569,0.466667,0.705882}%
\pgfsetstrokecolor{currentstroke}%
\pgfsetdash{{5.550000pt}{2.400000pt}}{0.000000pt}%
\pgfpathmoveto{\pgfqpoint{0.854222in}{2.606843in}}%
\pgfpathlineto{\pgfqpoint{1.243111in}{2.606843in}}%
\pgfusepath{stroke}%
\end{pgfscope}%
\begin{pgfscope}%
\definecolor{textcolor}{rgb}{0.000000,0.000000,0.000000}%
\pgfsetstrokecolor{textcolor}%
\pgfsetfillcolor{textcolor}%
\pgftext[x=1.398667in,y=2.538787in,left,base]{\color{textcolor}\rmfamily\fontsize{14.000000}{16.800000}\selectfont Demand Constraint}%
\end{pgfscope}%
\begin{pgfscope}%
\pgfsetrectcap%
\pgfsetroundjoin%
\pgfsetlinewidth{1.505625pt}%
\definecolor{currentstroke}{rgb}{0.839216,0.152941,0.156863}%
\pgfsetstrokecolor{currentstroke}%
\pgfsetdash{}{0pt}%
\pgfpathmoveto{\pgfqpoint{0.854222in}{2.331843in}}%
\pgfpathlineto{\pgfqpoint{1.243111in}{2.331843in}}%
\pgfusepath{stroke}%
\end{pgfscope}%
\begin{pgfscope}%
\definecolor{textcolor}{rgb}{0.000000,0.000000,0.000000}%
\pgfsetstrokecolor{textcolor}%
\pgfsetfillcolor{textcolor}%
\pgftext[x=1.398667in,y=2.263788in,left,base]{\color{textcolor}\rmfamily\fontsize{14.000000}{16.800000}\selectfont Objective Function}%
\end{pgfscope}%
\begin{pgfscope}%
\pgfsetrectcap%
\pgfsetroundjoin%
\pgfsetlinewidth{1.505625pt}%
\definecolor{currentstroke}{rgb}{0.000000,0.000000,0.000000}%
\pgfsetstrokecolor{currentstroke}%
\pgfsetdash{}{0pt}%
\pgfpathmoveto{\pgfqpoint{0.854222in}{2.045732in}}%
\pgfpathlineto{\pgfqpoint{1.243111in}{2.045732in}}%
\pgfusepath{stroke}%
\end{pgfscope}%
\begin{pgfscope}%
\definecolor{textcolor}{rgb}{0.000000,0.000000,0.000000}%
\pgfsetstrokecolor{textcolor}%
\pgfsetfillcolor{textcolor}%
\pgftext[x=1.398667in,y=1.977677in,left,base]{\color{textcolor}\rmfamily\fontsize{14.000000}{16.800000}\selectfont 10\% Slacked Objective}%
\end{pgfscope}%
\begin{pgfscope}%
\pgfsetbuttcap%
\pgfsetmiterjoin%
\definecolor{currentfill}{rgb}{0.121569,0.466667,0.705882}%
\pgfsetfillcolor{currentfill}%
\pgfsetfillopacity{0.200000}%
\pgfsetlinewidth{0.000000pt}%
\definecolor{currentstroke}{rgb}{0.000000,0.000000,0.000000}%
\pgfsetstrokecolor{currentstroke}%
\pgfsetstrokeopacity{0.000000}%
\pgfsetdash{}{0pt}%
\pgfpathmoveto{\pgfqpoint{0.854222in}{1.702677in}}%
\pgfpathlineto{\pgfqpoint{1.243111in}{1.702677in}}%
\pgfpathlineto{\pgfqpoint{1.243111in}{1.838788in}}%
\pgfpathlineto{\pgfqpoint{0.854222in}{1.838788in}}%
\pgfpathclose%
\pgfusepath{fill}%
\end{pgfscope}%
\begin{pgfscope}%
\definecolor{textcolor}{rgb}{0.000000,0.000000,0.000000}%
\pgfsetstrokecolor{textcolor}%
\pgfsetfillcolor{textcolor}%
\pgftext[x=1.398667in,y=1.702677in,left,base]{\color{textcolor}\rmfamily\fontsize{14.000000}{16.800000}\selectfont Feasible Space}%
\end{pgfscope}%
\begin{pgfscope}%
\pgfsetbuttcap%
\pgfsetroundjoin%
\definecolor{currentfill}{rgb}{0.839216,0.152941,0.156863}%
\pgfsetfillcolor{currentfill}%
\pgfsetlinewidth{1.003750pt}%
\definecolor{currentstroke}{rgb}{0.839216,0.152941,0.156863}%
\pgfsetstrokecolor{currentstroke}%
\pgfsetdash{}{0pt}%
\pgfsys@defobject{currentmarker}{\pgfqpoint{-0.065881in}{-0.065881in}}{\pgfqpoint{0.065881in}{0.065881in}}{%
\pgfpathmoveto{\pgfqpoint{0.000000in}{-0.065881in}}%
\pgfpathcurveto{\pgfqpoint{0.017472in}{-0.065881in}}{\pgfqpoint{0.034230in}{-0.058939in}}{\pgfqpoint{0.046585in}{-0.046585in}}%
\pgfpathcurveto{\pgfqpoint{0.058939in}{-0.034230in}}{\pgfqpoint{0.065881in}{-0.017472in}}{\pgfqpoint{0.065881in}{0.000000in}}%
\pgfpathcurveto{\pgfqpoint{0.065881in}{0.017472in}}{\pgfqpoint{0.058939in}{0.034230in}}{\pgfqpoint{0.046585in}{0.046585in}}%
\pgfpathcurveto{\pgfqpoint{0.034230in}{0.058939in}}{\pgfqpoint{0.017472in}{0.065881in}}{\pgfqpoint{0.000000in}{0.065881in}}%
\pgfpathcurveto{\pgfqpoint{-0.017472in}{0.065881in}}{\pgfqpoint{-0.034230in}{0.058939in}}{\pgfqpoint{-0.046585in}{0.046585in}}%
\pgfpathcurveto{\pgfqpoint{-0.058939in}{0.034230in}}{\pgfqpoint{-0.065881in}{0.017472in}}{\pgfqpoint{-0.065881in}{0.000000in}}%
\pgfpathcurveto{\pgfqpoint{-0.065881in}{-0.017472in}}{\pgfqpoint{-0.058939in}{-0.034230in}}{\pgfqpoint{-0.046585in}{-0.046585in}}%
\pgfpathcurveto{\pgfqpoint{-0.034230in}{-0.058939in}}{\pgfqpoint{-0.017472in}{-0.065881in}}{\pgfqpoint{0.000000in}{-0.065881in}}%
\pgfpathclose%
\pgfusepath{stroke,fill}%
}%
\begin{pgfscope}%
\pgfsys@transformshift{1.048667in}{1.478719in}%
\pgfsys@useobject{currentmarker}{}%
\end{pgfscope}%
\end{pgfscope}%
\begin{pgfscope}%
\definecolor{textcolor}{rgb}{0.000000,0.000000,0.000000}%
\pgfsetstrokecolor{textcolor}%
\pgfsetfillcolor{textcolor}%
\pgftext[x=1.398667in,y=1.427678in,left,base]{\color{textcolor}\rmfamily\fontsize{14.000000}{16.800000}\selectfont CEJA Goal}%
\end{pgfscope}%
\begin{pgfscope}%
\pgfsetbuttcap%
\pgfsetroundjoin%
\definecolor{currentfill}{rgb}{1.000000,1.000000,1.000000}%
\pgfsetfillcolor{currentfill}%
\pgfsetlinewidth{1.003750pt}%
\definecolor{currentstroke}{rgb}{0.000000,0.000000,0.000000}%
\pgfsetstrokecolor{currentstroke}%
\pgfsetdash{}{0pt}%
\pgfsys@defobject{currentmarker}{\pgfqpoint{-0.065881in}{-0.065881in}}{\pgfqpoint{0.065881in}{0.065881in}}{%
\pgfpathmoveto{\pgfqpoint{0.000000in}{-0.065881in}}%
\pgfpathcurveto{\pgfqpoint{0.017472in}{-0.065881in}}{\pgfqpoint{0.034230in}{-0.058939in}}{\pgfqpoint{0.046585in}{-0.046585in}}%
\pgfpathcurveto{\pgfqpoint{0.058939in}{-0.034230in}}{\pgfqpoint{0.065881in}{-0.017472in}}{\pgfqpoint{0.065881in}{0.000000in}}%
\pgfpathcurveto{\pgfqpoint{0.065881in}{0.017472in}}{\pgfqpoint{0.058939in}{0.034230in}}{\pgfqpoint{0.046585in}{0.046585in}}%
\pgfpathcurveto{\pgfqpoint{0.034230in}{0.058939in}}{\pgfqpoint{0.017472in}{0.065881in}}{\pgfqpoint{0.000000in}{0.065881in}}%
\pgfpathcurveto{\pgfqpoint{-0.017472in}{0.065881in}}{\pgfqpoint{-0.034230in}{0.058939in}}{\pgfqpoint{-0.046585in}{0.046585in}}%
\pgfpathcurveto{\pgfqpoint{-0.058939in}{0.034230in}}{\pgfqpoint{-0.065881in}{0.017472in}}{\pgfqpoint{-0.065881in}{0.000000in}}%
\pgfpathcurveto{\pgfqpoint{-0.065881in}{-0.017472in}}{\pgfqpoint{-0.058939in}{-0.034230in}}{\pgfqpoint{-0.046585in}{-0.046585in}}%
\pgfpathcurveto{\pgfqpoint{-0.034230in}{-0.058939in}}{\pgfqpoint{-0.017472in}{-0.065881in}}{\pgfqpoint{0.000000in}{-0.065881in}}%
\pgfpathclose%
\pgfusepath{stroke,fill}%
}%
\begin{pgfscope}%
\pgfsys@transformshift{1.048667in}{1.203720in}%
\pgfsys@useobject{currentmarker}{}%
\end{pgfscope}%
\end{pgfscope}%
\begin{pgfscope}%
\definecolor{textcolor}{rgb}{0.000000,0.000000,0.000000}%
\pgfsetstrokecolor{textcolor}%
\pgfsetfillcolor{textcolor}%
\pgftext[x=1.398667in,y=1.152678in,left,base]{\color{textcolor}\rmfamily\fontsize{14.000000}{16.800000}\selectfont MGA Solution}%
\end{pgfscope}%
\begin{pgfscope}%
\pgfsetbuttcap%
\pgfsetroundjoin%
\definecolor{currentfill}{rgb}{0.121569,0.466667,0.705882}%
\pgfsetfillcolor{currentfill}%
\pgfsetlinewidth{1.003750pt}%
\definecolor{currentstroke}{rgb}{0.121569,0.466667,0.705882}%
\pgfsetstrokecolor{currentstroke}%
\pgfsetdash{}{0pt}%
\pgfsys@defobject{currentmarker}{\pgfqpoint{-0.065881in}{-0.065881in}}{\pgfqpoint{0.065881in}{0.065881in}}{%
\pgfpathmoveto{\pgfqpoint{0.000000in}{-0.065881in}}%
\pgfpathcurveto{\pgfqpoint{0.017472in}{-0.065881in}}{\pgfqpoint{0.034230in}{-0.058939in}}{\pgfqpoint{0.046585in}{-0.046585in}}%
\pgfpathcurveto{\pgfqpoint{0.058939in}{-0.034230in}}{\pgfqpoint{0.065881in}{-0.017472in}}{\pgfqpoint{0.065881in}{0.000000in}}%
\pgfpathcurveto{\pgfqpoint{0.065881in}{0.017472in}}{\pgfqpoint{0.058939in}{0.034230in}}{\pgfqpoint{0.046585in}{0.046585in}}%
\pgfpathcurveto{\pgfqpoint{0.034230in}{0.058939in}}{\pgfqpoint{0.017472in}{0.065881in}}{\pgfqpoint{0.000000in}{0.065881in}}%
\pgfpathcurveto{\pgfqpoint{-0.017472in}{0.065881in}}{\pgfqpoint{-0.034230in}{0.058939in}}{\pgfqpoint{-0.046585in}{0.046585in}}%
\pgfpathcurveto{\pgfqpoint{-0.058939in}{0.034230in}}{\pgfqpoint{-0.065881in}{0.017472in}}{\pgfqpoint{-0.065881in}{0.000000in}}%
\pgfpathcurveto{\pgfqpoint{-0.065881in}{-0.017472in}}{\pgfqpoint{-0.058939in}{-0.034230in}}{\pgfqpoint{-0.046585in}{-0.046585in}}%
\pgfpathcurveto{\pgfqpoint{-0.034230in}{-0.058939in}}{\pgfqpoint{-0.017472in}{-0.065881in}}{\pgfqpoint{0.000000in}{-0.065881in}}%
\pgfpathclose%
\pgfusepath{stroke,fill}%
}%
\begin{pgfscope}%
\pgfsys@transformshift{1.048667in}{0.928720in}%
\pgfsys@useobject{currentmarker}{}%
\end{pgfscope}%
\end{pgfscope}%
\begin{pgfscope}%
\definecolor{textcolor}{rgb}{0.000000,0.000000,0.000000}%
\pgfsetstrokecolor{textcolor}%
\pgfsetfillcolor{textcolor}%
\pgftext[x=1.398667in,y=0.877678in,left,base]{\color{textcolor}\rmfamily\fontsize{14.000000}{16.800000}\selectfont Optimum Solution}%
\end{pgfscope}%
\end{pgfpicture}%
\makeatother%
\endgroup%
}
  \caption{The optimal solution lies at the intersection between the objective
  function (orange line) and the demand constraint (dashed blue line). An MGA
  iteration is shown by adding a 10\% slack to the objective function (black line).
  The red dot shows the \gls{ceja} goals for renewable energy, for reference.}
  \label{fig:mga-fig}
\end{figure}

While \gls{mga} helps modelers explore alternative energy futures and structural
uncertainty, quantifying parametric uncertainty is essential for creating robust insights
for policy decisions. Figure \ref{fig:param-fig} shows the same decision space for
the na\"{i}ve capacity expansion problem as before with upper and lower bounds for
the demand constraint and objective function, with 10\% uncertainty applied to all
parameters. The upper bound for the demand constraint, illustrated in blue,
represents a scenario where
the capacity factors for wind and solar are both 10\% less than average, and electricity
demand is 10\% higher. The upper and lower bounds on the objective function, in red,
represent futures where solar and wind energy costs are 10\% higher or lower,
respectively. The red dot in Figure \ref{fig:param-fig} represents the
capacity goal set by CEJA. There are several methods to explore this optimal space and
\gls{temoa} has some functionality for two: \gls{mca} \cite{yue_review_2018} and
stochastic optimization \cite{decarolis_multi-stage_2012, bennett_extending_2021}.
\gls{mca} is useful for global sensitivity analysis since its computational
cost increases with the number of iterations but not the number of uncertain
parameters. However, \gls{mca} depends strongly on the underlying data distribution
and cannot give a probability for any given scenario \cite{yue_review_2018}.
Stochastic optimization offers a likelihood and a possible ``hedging'' strategy,
but computational cost limits the number of uncertain parameters that can be
tested \cite{yue_review_2018}.

\begin{figure}[H]
  \centering
  \resizebox{0.8\columnwidth}{!}{%% Creator: Matplotlib, PGF backend
%%
%% To include the figure in your LaTeX document, write
%%   \input{<filename>.pgf}
%%
%% Make sure the required packages are loaded in your preamble
%%   \usepackage{pgf}
%%
%% Figures using additional raster images can only be included by \input if
%% they are in the same directory as the main LaTeX file. For loading figures
%% from other directories you can use the `import` package
%%   \usepackage{import}
%%
%% and then include the figures with
%%   \import{<path to file>}{<filename>.pgf}
%%
%% Matplotlib used the following preamble
%%
\begingroup%
\makeatletter%
\begin{pgfpicture}%
\pgfpathrectangle{\pgfpointorigin}{\pgfqpoint{10.103526in}{8.261960in}}%
\pgfusepath{use as bounding box, clip}%
\begin{pgfscope}%
\pgfsetbuttcap%
\pgfsetmiterjoin%
\definecolor{currentfill}{rgb}{1.000000,1.000000,1.000000}%
\pgfsetfillcolor{currentfill}%
\pgfsetlinewidth{0.000000pt}%
\definecolor{currentstroke}{rgb}{0.000000,0.000000,0.000000}%
\pgfsetstrokecolor{currentstroke}%
\pgfsetdash{}{0pt}%
\pgfpathmoveto{\pgfqpoint{0.000000in}{0.000000in}}%
\pgfpathlineto{\pgfqpoint{10.103526in}{0.000000in}}%
\pgfpathlineto{\pgfqpoint{10.103526in}{8.261960in}}%
\pgfpathlineto{\pgfqpoint{0.000000in}{8.261960in}}%
\pgfpathclose%
\pgfusepath{fill}%
\end{pgfscope}%
\begin{pgfscope}%
\pgfsetbuttcap%
\pgfsetmiterjoin%
\definecolor{currentfill}{rgb}{0.827451,0.827451,0.827451}%
\pgfsetfillcolor{currentfill}%
\pgfsetlinewidth{0.000000pt}%
\definecolor{currentstroke}{rgb}{0.000000,0.000000,0.000000}%
\pgfsetstrokecolor{currentstroke}%
\pgfsetstrokeopacity{0.000000}%
\pgfsetdash{}{0pt}%
\pgfpathmoveto{\pgfqpoint{0.703526in}{0.688481in}}%
\pgfpathlineto{\pgfqpoint{10.003526in}{0.688481in}}%
\pgfpathlineto{\pgfqpoint{10.003526in}{7.483481in}}%
\pgfpathlineto{\pgfqpoint{0.703526in}{7.483481in}}%
\pgfpathclose%
\pgfusepath{fill}%
\end{pgfscope}%
\begin{pgfscope}%
\pgfpathrectangle{\pgfqpoint{0.703526in}{0.688481in}}{\pgfqpoint{9.300000in}{6.795000in}}%
\pgfusepath{clip}%
\pgfsetbuttcap%
\pgfsetroundjoin%
\definecolor{currentfill}{rgb}{0.839216,0.152941,0.156863}%
\pgfsetfillcolor{currentfill}%
\pgfsetlinewidth{1.003750pt}%
\definecolor{currentstroke}{rgb}{0.839216,0.152941,0.156863}%
\pgfsetstrokecolor{currentstroke}%
\pgfsetdash{}{0pt}%
\pgfsys@defobject{currentmarker}{\pgfqpoint{-0.065881in}{-0.065881in}}{\pgfqpoint{0.065881in}{0.065881in}}{%
\pgfpathmoveto{\pgfqpoint{0.000000in}{-0.065881in}}%
\pgfpathcurveto{\pgfqpoint{0.017472in}{-0.065881in}}{\pgfqpoint{0.034230in}{-0.058939in}}{\pgfqpoint{0.046585in}{-0.046585in}}%
\pgfpathcurveto{\pgfqpoint{0.058939in}{-0.034230in}}{\pgfqpoint{0.065881in}{-0.017472in}}{\pgfqpoint{0.065881in}{0.000000in}}%
\pgfpathcurveto{\pgfqpoint{0.065881in}{0.017472in}}{\pgfqpoint{0.058939in}{0.034230in}}{\pgfqpoint{0.046585in}{0.046585in}}%
\pgfpathcurveto{\pgfqpoint{0.034230in}{0.058939in}}{\pgfqpoint{0.017472in}{0.065881in}}{\pgfqpoint{0.000000in}{0.065881in}}%
\pgfpathcurveto{\pgfqpoint{-0.017472in}{0.065881in}}{\pgfqpoint{-0.034230in}{0.058939in}}{\pgfqpoint{-0.046585in}{0.046585in}}%
\pgfpathcurveto{\pgfqpoint{-0.058939in}{0.034230in}}{\pgfqpoint{-0.065881in}{0.017472in}}{\pgfqpoint{-0.065881in}{0.000000in}}%
\pgfpathcurveto{\pgfqpoint{-0.065881in}{-0.017472in}}{\pgfqpoint{-0.058939in}{-0.034230in}}{\pgfqpoint{-0.046585in}{-0.046585in}}%
\pgfpathcurveto{\pgfqpoint{-0.034230in}{-0.058939in}}{\pgfqpoint{-0.017472in}{-0.065881in}}{\pgfqpoint{0.000000in}{-0.065881in}}%
\pgfpathclose%
\pgfusepath{stroke,fill}%
}%
\begin{pgfscope}%
\pgfsys@transformshift{6.470171in}{3.420976in}%
\pgfsys@useobject{currentmarker}{}%
\end{pgfscope}%
\end{pgfscope}%
\begin{pgfscope}%
\pgfpathrectangle{\pgfqpoint{0.703526in}{0.688481in}}{\pgfqpoint{9.300000in}{6.795000in}}%
\pgfusepath{clip}%
\pgfsetbuttcap%
\pgfsetroundjoin%
\definecolor{currentfill}{rgb}{0.121569,0.466667,0.705882}%
\pgfsetfillcolor{currentfill}%
\pgfsetfillopacity{0.100000}%
\pgfsetlinewidth{0.000000pt}%
\definecolor{currentstroke}{rgb}{0.000000,0.000000,0.000000}%
\pgfsetstrokecolor{currentstroke}%
\pgfsetdash{}{0pt}%
\pgfpathmoveto{\pgfqpoint{0.703526in}{7.483481in}}%
\pgfpathlineto{\pgfqpoint{0.703526in}{5.237200in}}%
\pgfpathlineto{\pgfqpoint{0.703526in}{5.237200in}}%
\pgfpathlineto{\pgfqpoint{0.709758in}{5.232646in}}%
\pgfpathlineto{\pgfqpoint{0.712835in}{5.230398in}}%
\pgfpathlineto{\pgfqpoint{0.715990in}{5.228093in}}%
\pgfpathlineto{\pgfqpoint{0.722144in}{5.223596in}}%
\pgfpathlineto{\pgfqpoint{0.722221in}{5.223540in}}%
\pgfpathlineto{\pgfqpoint{0.728453in}{5.218986in}}%
\pgfpathlineto{\pgfqpoint{0.731454in}{5.216794in}}%
\pgfpathlineto{\pgfqpoint{0.734685in}{5.214433in}}%
\pgfpathlineto{\pgfqpoint{0.740763in}{5.209992in}}%
\pgfpathlineto{\pgfqpoint{0.740917in}{5.209880in}}%
\pgfpathlineto{\pgfqpoint{0.747149in}{5.205327in}}%
\pgfpathlineto{\pgfqpoint{0.750072in}{5.203191in}}%
\pgfpathlineto{\pgfqpoint{0.753381in}{5.200773in}}%
\pgfpathlineto{\pgfqpoint{0.759382in}{5.196389in}}%
\pgfpathlineto{\pgfqpoint{0.759612in}{5.196220in}}%
\pgfpathlineto{\pgfqpoint{0.765844in}{5.191667in}}%
\pgfpathlineto{\pgfqpoint{0.768691in}{5.189587in}}%
\pgfpathlineto{\pgfqpoint{0.772076in}{5.187114in}}%
\pgfpathlineto{\pgfqpoint{0.778000in}{5.182785in}}%
\pgfpathlineto{\pgfqpoint{0.778308in}{5.182560in}}%
\pgfpathlineto{\pgfqpoint{0.784540in}{5.178007in}}%
\pgfpathlineto{\pgfqpoint{0.787310in}{5.175983in}}%
\pgfpathlineto{\pgfqpoint{0.790772in}{5.173454in}}%
\pgfpathlineto{\pgfqpoint{0.796619in}{5.169182in}}%
\pgfpathlineto{\pgfqpoint{0.797004in}{5.168900in}}%
\pgfpathlineto{\pgfqpoint{0.803235in}{5.164347in}}%
\pgfpathlineto{\pgfqpoint{0.805928in}{5.162380in}}%
\pgfpathlineto{\pgfqpoint{0.809467in}{5.159794in}}%
\pgfpathlineto{\pgfqpoint{0.815238in}{5.155578in}}%
\pgfpathlineto{\pgfqpoint{0.815699in}{5.155241in}}%
\pgfpathlineto{\pgfqpoint{0.821931in}{5.150687in}}%
\pgfpathlineto{\pgfqpoint{0.824547in}{5.148776in}}%
\pgfpathlineto{\pgfqpoint{0.828163in}{5.146134in}}%
\pgfpathlineto{\pgfqpoint{0.833856in}{5.141974in}}%
\pgfpathlineto{\pgfqpoint{0.834395in}{5.141581in}}%
\pgfpathlineto{\pgfqpoint{0.840627in}{5.137028in}}%
\pgfpathlineto{\pgfqpoint{0.843165in}{5.135173in}}%
\pgfpathlineto{\pgfqpoint{0.846858in}{5.132474in}}%
\pgfpathlineto{\pgfqpoint{0.852475in}{5.128371in}}%
\pgfpathlineto{\pgfqpoint{0.853090in}{5.127921in}}%
\pgfpathlineto{\pgfqpoint{0.859322in}{5.123368in}}%
\pgfpathlineto{\pgfqpoint{0.861784in}{5.121569in}}%
\pgfpathlineto{\pgfqpoint{0.865554in}{5.118814in}}%
\pgfpathlineto{\pgfqpoint{0.871093in}{5.114767in}}%
\pgfpathlineto{\pgfqpoint{0.871786in}{5.114261in}}%
\pgfpathlineto{\pgfqpoint{0.878018in}{5.109708in}}%
\pgfpathlineto{\pgfqpoint{0.880403in}{5.107965in}}%
\pgfpathlineto{\pgfqpoint{0.884250in}{5.105155in}}%
\pgfpathlineto{\pgfqpoint{0.889712in}{5.101164in}}%
\pgfpathlineto{\pgfqpoint{0.890481in}{5.100601in}}%
\pgfpathlineto{\pgfqpoint{0.896713in}{5.096048in}}%
\pgfpathlineto{\pgfqpoint{0.899021in}{5.094362in}}%
\pgfpathlineto{\pgfqpoint{0.902945in}{5.091495in}}%
\pgfpathlineto{\pgfqpoint{0.908331in}{5.087560in}}%
\pgfpathlineto{\pgfqpoint{0.909177in}{5.086942in}}%
\pgfpathlineto{\pgfqpoint{0.915409in}{5.082388in}}%
\pgfpathlineto{\pgfqpoint{0.917640in}{5.080758in}}%
\pgfpathlineto{\pgfqpoint{0.921641in}{5.077835in}}%
\pgfpathlineto{\pgfqpoint{0.926949in}{5.073956in}}%
\pgfpathlineto{\pgfqpoint{0.927872in}{5.073282in}}%
\pgfpathlineto{\pgfqpoint{0.934104in}{5.068728in}}%
\pgfpathlineto{\pgfqpoint{0.936259in}{5.067155in}}%
\pgfpathlineto{\pgfqpoint{0.940336in}{5.064175in}}%
\pgfpathlineto{\pgfqpoint{0.945568in}{5.060353in}}%
\pgfpathlineto{\pgfqpoint{0.946568in}{5.059622in}}%
\pgfpathlineto{\pgfqpoint{0.952800in}{5.055069in}}%
\pgfpathlineto{\pgfqpoint{0.954877in}{5.053551in}}%
\pgfpathlineto{\pgfqpoint{0.959032in}{5.050515in}}%
\pgfpathlineto{\pgfqpoint{0.964186in}{5.046749in}}%
\pgfpathlineto{\pgfqpoint{0.965264in}{5.045962in}}%
\pgfpathlineto{\pgfqpoint{0.971495in}{5.041409in}}%
\pgfpathlineto{\pgfqpoint{0.973496in}{5.039947in}}%
\pgfpathlineto{\pgfqpoint{0.977727in}{5.036856in}}%
\pgfpathlineto{\pgfqpoint{0.982805in}{5.033146in}}%
\pgfpathlineto{\pgfqpoint{0.983959in}{5.032302in}}%
\pgfpathlineto{\pgfqpoint{0.990191in}{5.027749in}}%
\pgfpathlineto{\pgfqpoint{0.992114in}{5.026344in}}%
\pgfpathlineto{\pgfqpoint{0.996423in}{5.023196in}}%
\pgfpathlineto{\pgfqpoint{1.001424in}{5.019542in}}%
\pgfpathlineto{\pgfqpoint{1.002655in}{5.018642in}}%
\pgfpathlineto{\pgfqpoint{1.008887in}{5.014089in}}%
\pgfpathlineto{\pgfqpoint{1.010733in}{5.012740in}}%
\pgfpathlineto{\pgfqpoint{1.015118in}{5.009536in}}%
\pgfpathlineto{\pgfqpoint{1.020042in}{5.005938in}}%
\pgfpathlineto{\pgfqpoint{1.021350in}{5.004983in}}%
\pgfpathlineto{\pgfqpoint{1.027582in}{5.000429in}}%
\pgfpathlineto{\pgfqpoint{1.029352in}{4.999137in}}%
\pgfpathlineto{\pgfqpoint{1.033814in}{4.995876in}}%
\pgfpathlineto{\pgfqpoint{1.038661in}{4.992335in}}%
\pgfpathlineto{\pgfqpoint{1.040046in}{4.991323in}}%
\pgfpathlineto{\pgfqpoint{1.046278in}{4.986770in}}%
\pgfpathlineto{\pgfqpoint{1.047970in}{4.985533in}}%
\pgfpathlineto{\pgfqpoint{1.052510in}{4.982216in}}%
\pgfpathlineto{\pgfqpoint{1.057280in}{4.978731in}}%
\pgfpathlineto{\pgfqpoint{1.058741in}{4.977663in}}%
\pgfpathlineto{\pgfqpoint{1.064973in}{4.973110in}}%
\pgfpathlineto{\pgfqpoint{1.066589in}{4.971929in}}%
\pgfpathlineto{\pgfqpoint{1.071205in}{4.968556in}}%
\pgfpathlineto{\pgfqpoint{1.075898in}{4.965127in}}%
\pgfpathlineto{\pgfqpoint{1.077437in}{4.964003in}}%
\pgfpathlineto{\pgfqpoint{1.083669in}{4.959450in}}%
\pgfpathlineto{\pgfqpoint{1.085208in}{4.958326in}}%
\pgfpathlineto{\pgfqpoint{1.089901in}{4.954897in}}%
\pgfpathlineto{\pgfqpoint{1.094517in}{4.951524in}}%
\pgfpathlineto{\pgfqpoint{1.096132in}{4.950343in}}%
\pgfpathlineto{\pgfqpoint{1.102364in}{4.945790in}}%
\pgfpathlineto{\pgfqpoint{1.103826in}{4.944722in}}%
\pgfpathlineto{\pgfqpoint{1.108596in}{4.941237in}}%
\pgfpathlineto{\pgfqpoint{1.113135in}{4.937920in}}%
\pgfpathlineto{\pgfqpoint{1.114828in}{4.936684in}}%
\pgfpathlineto{\pgfqpoint{1.121060in}{4.932130in}}%
\pgfpathlineto{\pgfqpoint{1.122445in}{4.931118in}}%
\pgfpathlineto{\pgfqpoint{1.127292in}{4.927577in}}%
\pgfpathlineto{\pgfqpoint{1.131754in}{4.924317in}}%
\pgfpathlineto{\pgfqpoint{1.133524in}{4.923024in}}%
\pgfpathlineto{\pgfqpoint{1.139755in}{4.918471in}}%
\pgfpathlineto{\pgfqpoint{1.141063in}{4.917515in}}%
\pgfpathlineto{\pgfqpoint{1.145987in}{4.913917in}}%
\pgfpathlineto{\pgfqpoint{1.150373in}{4.910713in}}%
\pgfpathlineto{\pgfqpoint{1.152219in}{4.909364in}}%
\pgfpathlineto{\pgfqpoint{1.158451in}{4.904811in}}%
\pgfpathlineto{\pgfqpoint{1.159682in}{4.903911in}}%
\pgfpathlineto{\pgfqpoint{1.164683in}{4.900257in}}%
\pgfpathlineto{\pgfqpoint{1.168991in}{4.897109in}}%
\pgfpathlineto{\pgfqpoint{1.170915in}{4.895704in}}%
\pgfpathlineto{\pgfqpoint{1.177147in}{4.891151in}}%
\pgfpathlineto{\pgfqpoint{1.178301in}{4.890308in}}%
\pgfpathlineto{\pgfqpoint{1.183378in}{4.886598in}}%
\pgfpathlineto{\pgfqpoint{1.187610in}{4.883506in}}%
\pgfpathlineto{\pgfqpoint{1.189610in}{4.882044in}}%
\pgfpathlineto{\pgfqpoint{1.195842in}{4.877491in}}%
\pgfpathlineto{\pgfqpoint{1.196919in}{4.876704in}}%
\pgfpathlineto{\pgfqpoint{1.202074in}{4.872938in}}%
\pgfpathlineto{\pgfqpoint{1.206229in}{4.869902in}}%
\pgfpathlineto{\pgfqpoint{1.208306in}{4.868385in}}%
\pgfpathlineto{\pgfqpoint{1.214538in}{4.863831in}}%
\pgfpathlineto{\pgfqpoint{1.215538in}{4.863100in}}%
\pgfpathlineto{\pgfqpoint{1.220770in}{4.859278in}}%
\pgfpathlineto{\pgfqpoint{1.224847in}{4.856299in}}%
\pgfpathlineto{\pgfqpoint{1.227001in}{4.854725in}}%
\pgfpathlineto{\pgfqpoint{1.233233in}{4.850171in}}%
\pgfpathlineto{\pgfqpoint{1.234156in}{4.849497in}}%
\pgfpathlineto{\pgfqpoint{1.239465in}{4.845618in}}%
\pgfpathlineto{\pgfqpoint{1.243466in}{4.842695in}}%
\pgfpathlineto{\pgfqpoint{1.245697in}{4.841065in}}%
\pgfpathlineto{\pgfqpoint{1.251929in}{4.836512in}}%
\pgfpathlineto{\pgfqpoint{1.252775in}{4.835893in}}%
\pgfpathlineto{\pgfqpoint{1.258161in}{4.831958in}}%
\pgfpathlineto{\pgfqpoint{1.262084in}{4.829091in}}%
\pgfpathlineto{\pgfqpoint{1.264392in}{4.827405in}}%
\pgfpathlineto{\pgfqpoint{1.270624in}{4.822852in}}%
\pgfpathlineto{\pgfqpoint{1.271394in}{4.822290in}}%
\pgfpathlineto{\pgfqpoint{1.276856in}{4.818299in}}%
\pgfpathlineto{\pgfqpoint{1.280703in}{4.815488in}}%
\pgfpathlineto{\pgfqpoint{1.283088in}{4.813745in}}%
\pgfpathlineto{\pgfqpoint{1.289320in}{4.809192in}}%
\pgfpathlineto{\pgfqpoint{1.290012in}{4.808686in}}%
\pgfpathlineto{\pgfqpoint{1.295552in}{4.804639in}}%
\pgfpathlineto{\pgfqpoint{1.299322in}{4.801884in}}%
\pgfpathlineto{\pgfqpoint{1.301784in}{4.800085in}}%
\pgfpathlineto{\pgfqpoint{1.308015in}{4.795532in}}%
\pgfpathlineto{\pgfqpoint{1.308631in}{4.795082in}}%
\pgfpathlineto{\pgfqpoint{1.314247in}{4.790979in}}%
\pgfpathlineto{\pgfqpoint{1.317940in}{4.788281in}}%
\pgfpathlineto{\pgfqpoint{1.320479in}{4.786426in}}%
\pgfpathlineto{\pgfqpoint{1.326711in}{4.781872in}}%
\pgfpathlineto{\pgfqpoint{1.327250in}{4.781479in}}%
\pgfpathlineto{\pgfqpoint{1.332943in}{4.777319in}}%
\pgfpathlineto{\pgfqpoint{1.336559in}{4.774677in}}%
\pgfpathlineto{\pgfqpoint{1.339175in}{4.772766in}}%
\pgfpathlineto{\pgfqpoint{1.345407in}{4.768213in}}%
\pgfpathlineto{\pgfqpoint{1.345868in}{4.767875in}}%
\pgfpathlineto{\pgfqpoint{1.351638in}{4.763659in}}%
\pgfpathlineto{\pgfqpoint{1.355177in}{4.761073in}}%
\pgfpathlineto{\pgfqpoint{1.357870in}{4.759106in}}%
\pgfpathlineto{\pgfqpoint{1.364102in}{4.754553in}}%
\pgfpathlineto{\pgfqpoint{1.364487in}{4.754272in}}%
\pgfpathlineto{\pgfqpoint{1.370334in}{4.749999in}}%
\pgfpathlineto{\pgfqpoint{1.373796in}{4.747470in}}%
\pgfpathlineto{\pgfqpoint{1.376566in}{4.745446in}}%
\pgfpathlineto{\pgfqpoint{1.382798in}{4.740893in}}%
\pgfpathlineto{\pgfqpoint{1.383105in}{4.740668in}}%
\pgfpathlineto{\pgfqpoint{1.389030in}{4.736340in}}%
\pgfpathlineto{\pgfqpoint{1.392415in}{4.733866in}}%
\pgfpathlineto{\pgfqpoint{1.395261in}{4.731786in}}%
\pgfpathlineto{\pgfqpoint{1.401493in}{4.727233in}}%
\pgfpathlineto{\pgfqpoint{1.401724in}{4.727064in}}%
\pgfpathlineto{\pgfqpoint{1.407725in}{4.722680in}}%
\pgfpathlineto{\pgfqpoint{1.411033in}{4.720263in}}%
\pgfpathlineto{\pgfqpoint{1.413957in}{4.718127in}}%
\pgfpathlineto{\pgfqpoint{1.420189in}{4.713573in}}%
\pgfpathlineto{\pgfqpoint{1.420343in}{4.713461in}}%
\pgfpathlineto{\pgfqpoint{1.426421in}{4.709020in}}%
\pgfpathlineto{\pgfqpoint{1.429652in}{4.706659in}}%
\pgfpathlineto{\pgfqpoint{1.432652in}{4.704467in}}%
\pgfpathlineto{\pgfqpoint{1.438884in}{4.699913in}}%
\pgfpathlineto{\pgfqpoint{1.438961in}{4.699857in}}%
\pgfpathlineto{\pgfqpoint{1.445116in}{4.695360in}}%
\pgfpathlineto{\pgfqpoint{1.448271in}{4.693055in}}%
\pgfpathlineto{\pgfqpoint{1.451348in}{4.690807in}}%
\pgfpathlineto{\pgfqpoint{1.457580in}{4.686254in}}%
\pgfpathlineto{\pgfqpoint{1.457580in}{4.686254in}}%
\pgfpathlineto{\pgfqpoint{1.463812in}{4.681700in}}%
\pgfpathlineto{\pgfqpoint{1.466889in}{4.679452in}}%
\pgfpathlineto{\pgfqpoint{1.470044in}{4.677147in}}%
\pgfpathlineto{\pgfqpoint{1.476199in}{4.672650in}}%
\pgfpathlineto{\pgfqpoint{1.476275in}{4.672594in}}%
\pgfpathlineto{\pgfqpoint{1.482507in}{4.668041in}}%
\pgfpathlineto{\pgfqpoint{1.485508in}{4.665848in}}%
\pgfpathlineto{\pgfqpoint{1.488739in}{4.663487in}}%
\pgfpathlineto{\pgfqpoint{1.494817in}{4.659046in}}%
\pgfpathlineto{\pgfqpoint{1.494971in}{4.658934in}}%
\pgfpathlineto{\pgfqpoint{1.501203in}{4.654381in}}%
\pgfpathlineto{\pgfqpoint{1.504126in}{4.652245in}}%
\pgfpathlineto{\pgfqpoint{1.507435in}{4.649827in}}%
\pgfpathlineto{\pgfqpoint{1.513436in}{4.645443in}}%
\pgfpathlineto{\pgfqpoint{1.513667in}{4.645274in}}%
\pgfpathlineto{\pgfqpoint{1.519898in}{4.640721in}}%
\pgfpathlineto{\pgfqpoint{1.522745in}{4.638641in}}%
\pgfpathlineto{\pgfqpoint{1.526130in}{4.636168in}}%
\pgfpathlineto{\pgfqpoint{1.532054in}{4.631839in}}%
\pgfpathlineto{\pgfqpoint{1.532362in}{4.631614in}}%
\pgfpathlineto{\pgfqpoint{1.538594in}{4.627061in}}%
\pgfpathlineto{\pgfqpoint{1.541364in}{4.625037in}}%
\pgfpathlineto{\pgfqpoint{1.544826in}{4.622508in}}%
\pgfpathlineto{\pgfqpoint{1.550673in}{4.618236in}}%
\pgfpathlineto{\pgfqpoint{1.551058in}{4.617955in}}%
\pgfpathlineto{\pgfqpoint{1.557290in}{4.613401in}}%
\pgfpathlineto{\pgfqpoint{1.559982in}{4.611434in}}%
\pgfpathlineto{\pgfqpoint{1.563521in}{4.608848in}}%
\pgfpathlineto{\pgfqpoint{1.569292in}{4.604632in}}%
\pgfpathlineto{\pgfqpoint{1.569753in}{4.604295in}}%
\pgfpathlineto{\pgfqpoint{1.575985in}{4.599741in}}%
\pgfpathlineto{\pgfqpoint{1.578601in}{4.597830in}}%
\pgfpathlineto{\pgfqpoint{1.582217in}{4.595188in}}%
\pgfpathlineto{\pgfqpoint{1.587910in}{4.591028in}}%
\pgfpathlineto{\pgfqpoint{1.588449in}{4.590635in}}%
\pgfpathlineto{\pgfqpoint{1.594681in}{4.586082in}}%
\pgfpathlineto{\pgfqpoint{1.597220in}{4.584227in}}%
\pgfpathlineto{\pgfqpoint{1.600912in}{4.581528in}}%
\pgfpathlineto{\pgfqpoint{1.606529in}{4.577425in}}%
\pgfpathlineto{\pgfqpoint{1.607144in}{4.576975in}}%
\pgfpathlineto{\pgfqpoint{1.613376in}{4.572422in}}%
\pgfpathlineto{\pgfqpoint{1.615838in}{4.570623in}}%
\pgfpathlineto{\pgfqpoint{1.619608in}{4.567869in}}%
\pgfpathlineto{\pgfqpoint{1.625147in}{4.563821in}}%
\pgfpathlineto{\pgfqpoint{1.625840in}{4.563315in}}%
\pgfpathlineto{\pgfqpoint{1.632072in}{4.558762in}}%
\pgfpathlineto{\pgfqpoint{1.634457in}{4.557019in}}%
\pgfpathlineto{\pgfqpoint{1.638304in}{4.554209in}}%
\pgfpathlineto{\pgfqpoint{1.643766in}{4.550218in}}%
\pgfpathlineto{\pgfqpoint{1.644535in}{4.549655in}}%
\pgfpathlineto{\pgfqpoint{1.650767in}{4.545102in}}%
\pgfpathlineto{\pgfqpoint{1.653075in}{4.543416in}}%
\pgfpathlineto{\pgfqpoint{1.656999in}{4.540549in}}%
\pgfpathlineto{\pgfqpoint{1.662385in}{4.536614in}}%
\pgfpathlineto{\pgfqpoint{1.663231in}{4.535996in}}%
\pgfpathlineto{\pgfqpoint{1.669463in}{4.531442in}}%
\pgfpathlineto{\pgfqpoint{1.671694in}{4.529812in}}%
\pgfpathlineto{\pgfqpoint{1.675695in}{4.526889in}}%
\pgfpathlineto{\pgfqpoint{1.681003in}{4.523010in}}%
\pgfpathlineto{\pgfqpoint{1.681927in}{4.522336in}}%
\pgfpathlineto{\pgfqpoint{1.688158in}{4.517783in}}%
\pgfpathlineto{\pgfqpoint{1.690313in}{4.516209in}}%
\pgfpathlineto{\pgfqpoint{1.694390in}{4.513229in}}%
\pgfpathlineto{\pgfqpoint{1.699622in}{4.509407in}}%
\pgfpathlineto{\pgfqpoint{1.700622in}{4.508676in}}%
\pgfpathlineto{\pgfqpoint{1.706854in}{4.504123in}}%
\pgfpathlineto{\pgfqpoint{1.708931in}{4.502605in}}%
\pgfpathlineto{\pgfqpoint{1.713086in}{4.499569in}}%
\pgfpathlineto{\pgfqpoint{1.718241in}{4.495803in}}%
\pgfpathlineto{\pgfqpoint{1.719318in}{4.495016in}}%
\pgfpathlineto{\pgfqpoint{1.725550in}{4.490463in}}%
\pgfpathlineto{\pgfqpoint{1.727550in}{4.489001in}}%
\pgfpathlineto{\pgfqpoint{1.731781in}{4.485910in}}%
\pgfpathlineto{\pgfqpoint{1.736859in}{4.482200in}}%
\pgfpathlineto{\pgfqpoint{1.738013in}{4.481356in}}%
\pgfpathlineto{\pgfqpoint{1.744245in}{4.476803in}}%
\pgfpathlineto{\pgfqpoint{1.746168in}{4.475398in}}%
\pgfpathlineto{\pgfqpoint{1.750477in}{4.472250in}}%
\pgfpathlineto{\pgfqpoint{1.755478in}{4.468596in}}%
\pgfpathlineto{\pgfqpoint{1.756709in}{4.467697in}}%
\pgfpathlineto{\pgfqpoint{1.762941in}{4.463143in}}%
\pgfpathlineto{\pgfqpoint{1.764787in}{4.461794in}}%
\pgfpathlineto{\pgfqpoint{1.769172in}{4.458590in}}%
\pgfpathlineto{\pgfqpoint{1.774096in}{4.454992in}}%
\pgfpathlineto{\pgfqpoint{1.775404in}{4.454037in}}%
\pgfpathlineto{\pgfqpoint{1.781636in}{4.449483in}}%
\pgfpathlineto{\pgfqpoint{1.783406in}{4.448191in}}%
\pgfpathlineto{\pgfqpoint{1.787868in}{4.444930in}}%
\pgfpathlineto{\pgfqpoint{1.792715in}{4.441389in}}%
\pgfpathlineto{\pgfqpoint{1.794100in}{4.440377in}}%
\pgfpathlineto{\pgfqpoint{1.800332in}{4.435824in}}%
\pgfpathlineto{\pgfqpoint{1.802024in}{4.434587in}}%
\pgfpathlineto{\pgfqpoint{1.806564in}{4.431270in}}%
\pgfpathlineto{\pgfqpoint{1.811334in}{4.427785in}}%
\pgfpathlineto{\pgfqpoint{1.812795in}{4.426717in}}%
\pgfpathlineto{\pgfqpoint{1.819027in}{4.422164in}}%
\pgfpathlineto{\pgfqpoint{1.820643in}{4.420983in}}%
\pgfpathlineto{\pgfqpoint{1.825259in}{4.417611in}}%
\pgfpathlineto{\pgfqpoint{1.829952in}{4.414182in}}%
\pgfpathlineto{\pgfqpoint{1.831491in}{4.413057in}}%
\pgfpathlineto{\pgfqpoint{1.837723in}{4.408504in}}%
\pgfpathlineto{\pgfqpoint{1.839262in}{4.407380in}}%
\pgfpathlineto{\pgfqpoint{1.843955in}{4.403951in}}%
\pgfpathlineto{\pgfqpoint{1.848571in}{4.400578in}}%
\pgfpathlineto{\pgfqpoint{1.850187in}{4.399397in}}%
\pgfpathlineto{\pgfqpoint{1.856418in}{4.394844in}}%
\pgfpathlineto{\pgfqpoint{1.857880in}{4.393776in}}%
\pgfpathlineto{\pgfqpoint{1.862650in}{4.390291in}}%
\pgfpathlineto{\pgfqpoint{1.867189in}{4.386974in}}%
\pgfpathlineto{\pgfqpoint{1.868882in}{4.385738in}}%
\pgfpathlineto{\pgfqpoint{1.875114in}{4.381184in}}%
\pgfpathlineto{\pgfqpoint{1.876499in}{4.380173in}}%
\pgfpathlineto{\pgfqpoint{1.881346in}{4.376631in}}%
\pgfpathlineto{\pgfqpoint{1.885808in}{4.373371in}}%
\pgfpathlineto{\pgfqpoint{1.887578in}{4.372078in}}%
\pgfpathlineto{\pgfqpoint{1.893810in}{4.367525in}}%
\pgfpathlineto{\pgfqpoint{1.895117in}{4.366569in}}%
\pgfpathlineto{\pgfqpoint{1.900041in}{4.362971in}}%
\pgfpathlineto{\pgfqpoint{1.904427in}{4.359767in}}%
\pgfpathlineto{\pgfqpoint{1.906273in}{4.358418in}}%
\pgfpathlineto{\pgfqpoint{1.912505in}{4.353865in}}%
\pgfpathlineto{\pgfqpoint{1.913736in}{4.352965in}}%
\pgfpathlineto{\pgfqpoint{1.918737in}{4.349311in}}%
\pgfpathlineto{\pgfqpoint{1.923045in}{4.346164in}}%
\pgfpathlineto{\pgfqpoint{1.924969in}{4.344758in}}%
\pgfpathlineto{\pgfqpoint{1.931201in}{4.340205in}}%
\pgfpathlineto{\pgfqpoint{1.932355in}{4.339362in}}%
\pgfpathlineto{\pgfqpoint{1.937432in}{4.335652in}}%
\pgfpathlineto{\pgfqpoint{1.941664in}{4.332560in}}%
\pgfpathlineto{\pgfqpoint{1.943664in}{4.331098in}}%
\pgfpathlineto{\pgfqpoint{1.949896in}{4.326545in}}%
\pgfpathlineto{\pgfqpoint{1.950973in}{4.325758in}}%
\pgfpathlineto{\pgfqpoint{1.956128in}{4.321992in}}%
\pgfpathlineto{\pgfqpoint{1.960283in}{4.318956in}}%
\pgfpathlineto{\pgfqpoint{1.962360in}{4.317439in}}%
\pgfpathlineto{\pgfqpoint{1.968592in}{4.312885in}}%
\pgfpathlineto{\pgfqpoint{1.969592in}{4.312155in}}%
\pgfpathlineto{\pgfqpoint{1.974824in}{4.308332in}}%
\pgfpathlineto{\pgfqpoint{1.978901in}{4.305353in}}%
\pgfpathlineto{\pgfqpoint{1.981055in}{4.303779in}}%
\pgfpathlineto{\pgfqpoint{1.987287in}{4.299225in}}%
\pgfpathlineto{\pgfqpoint{1.988211in}{4.298551in}}%
\pgfpathlineto{\pgfqpoint{1.993519in}{4.294672in}}%
\pgfpathlineto{\pgfqpoint{1.997520in}{4.291749in}}%
\pgfpathlineto{\pgfqpoint{1.999751in}{4.290119in}}%
\pgfpathlineto{\pgfqpoint{2.005983in}{4.285566in}}%
\pgfpathlineto{\pgfqpoint{2.006829in}{4.284947in}}%
\pgfpathlineto{\pgfqpoint{2.012215in}{4.281012in}}%
\pgfpathlineto{\pgfqpoint{2.016138in}{4.278146in}}%
\pgfpathlineto{\pgfqpoint{2.018447in}{4.276459in}}%
\pgfpathlineto{\pgfqpoint{2.024678in}{4.271906in}}%
\pgfpathlineto{\pgfqpoint{2.025448in}{4.271344in}}%
\pgfpathlineto{\pgfqpoint{2.030910in}{4.267353in}}%
\pgfpathlineto{\pgfqpoint{2.034757in}{4.264542in}}%
\pgfpathlineto{\pgfqpoint{2.037142in}{4.262799in}}%
\pgfpathlineto{\pgfqpoint{2.043374in}{4.258246in}}%
\pgfpathlineto{\pgfqpoint{2.044066in}{4.257740in}}%
\pgfpathlineto{\pgfqpoint{2.049606in}{4.253693in}}%
\pgfpathlineto{\pgfqpoint{2.053376in}{4.250938in}}%
\pgfpathlineto{\pgfqpoint{2.055838in}{4.249139in}}%
\pgfpathlineto{\pgfqpoint{2.062070in}{4.244586in}}%
\pgfpathlineto{\pgfqpoint{2.062685in}{4.244137in}}%
\pgfpathlineto{\pgfqpoint{2.068301in}{4.240033in}}%
\pgfpathlineto{\pgfqpoint{2.071994in}{4.237335in}}%
\pgfpathlineto{\pgfqpoint{2.074533in}{4.235480in}}%
\pgfpathlineto{\pgfqpoint{2.080765in}{4.230926in}}%
\pgfpathlineto{\pgfqpoint{2.081304in}{4.230533in}}%
\pgfpathlineto{\pgfqpoint{2.086997in}{4.226373in}}%
\pgfpathlineto{\pgfqpoint{2.090613in}{4.223731in}}%
\pgfpathlineto{\pgfqpoint{2.093229in}{4.221820in}}%
\pgfpathlineto{\pgfqpoint{2.099461in}{4.217267in}}%
\pgfpathlineto{\pgfqpoint{2.099922in}{4.216929in}}%
\pgfpathlineto{\pgfqpoint{2.105692in}{4.212713in}}%
\pgfpathlineto{\pgfqpoint{2.109232in}{4.210127in}}%
\pgfpathlineto{\pgfqpoint{2.111924in}{4.208160in}}%
\pgfpathlineto{\pgfqpoint{2.118156in}{4.203607in}}%
\pgfpathlineto{\pgfqpoint{2.118541in}{4.203326in}}%
\pgfpathlineto{\pgfqpoint{2.124388in}{4.199053in}}%
\pgfpathlineto{\pgfqpoint{2.127850in}{4.196524in}}%
\pgfpathlineto{\pgfqpoint{2.130620in}{4.194500in}}%
\pgfpathlineto{\pgfqpoint{2.136852in}{4.189947in}}%
\pgfpathlineto{\pgfqpoint{2.137159in}{4.189722in}}%
\pgfpathlineto{\pgfqpoint{2.143084in}{4.185394in}}%
\pgfpathlineto{\pgfqpoint{2.146469in}{4.182920in}}%
\pgfpathlineto{\pgfqpoint{2.149315in}{4.180840in}}%
\pgfpathlineto{\pgfqpoint{2.155547in}{4.176287in}}%
\pgfpathlineto{\pgfqpoint{2.155778in}{4.176118in}}%
\pgfpathlineto{\pgfqpoint{2.161779in}{4.171734in}}%
\pgfpathlineto{\pgfqpoint{2.165087in}{4.169317in}}%
\pgfpathlineto{\pgfqpoint{2.168011in}{4.167181in}}%
\pgfpathlineto{\pgfqpoint{2.174243in}{4.162627in}}%
\pgfpathlineto{\pgfqpoint{2.174397in}{4.162515in}}%
\pgfpathlineto{\pgfqpoint{2.180475in}{4.158074in}}%
\pgfpathlineto{\pgfqpoint{2.183706in}{4.155713in}}%
\pgfpathlineto{\pgfqpoint{2.186707in}{4.153521in}}%
\pgfpathlineto{\pgfqpoint{2.192938in}{4.148967in}}%
\pgfpathlineto{\pgfqpoint{2.193015in}{4.148911in}}%
\pgfpathlineto{\pgfqpoint{2.199170in}{4.144414in}}%
\pgfpathlineto{\pgfqpoint{2.202325in}{4.142109in}}%
\pgfpathlineto{\pgfqpoint{2.205402in}{4.139861in}}%
\pgfpathlineto{\pgfqpoint{2.211634in}{4.135308in}}%
\pgfpathlineto{\pgfqpoint{2.211634in}{4.135308in}}%
\pgfpathlineto{\pgfqpoint{2.217866in}{4.130754in}}%
\pgfpathlineto{\pgfqpoint{2.220943in}{4.128506in}}%
\pgfpathlineto{\pgfqpoint{2.224098in}{4.126201in}}%
\pgfpathlineto{\pgfqpoint{2.230253in}{4.121704in}}%
\pgfpathlineto{\pgfqpoint{2.230329in}{4.121648in}}%
\pgfpathlineto{\pgfqpoint{2.236561in}{4.117095in}}%
\pgfpathlineto{\pgfqpoint{2.239562in}{4.114902in}}%
\pgfpathlineto{\pgfqpoint{2.242793in}{4.112541in}}%
\pgfpathlineto{\pgfqpoint{2.248871in}{4.108100in}}%
\pgfpathlineto{\pgfqpoint{2.249025in}{4.107988in}}%
\pgfpathlineto{\pgfqpoint{2.255257in}{4.103435in}}%
\pgfpathlineto{\pgfqpoint{2.258180in}{4.101299in}}%
\pgfpathlineto{\pgfqpoint{2.261489in}{4.098881in}}%
\pgfpathlineto{\pgfqpoint{2.267490in}{4.094497in}}%
\pgfpathlineto{\pgfqpoint{2.267721in}{4.094328in}}%
\pgfpathlineto{\pgfqpoint{2.273952in}{4.089775in}}%
\pgfpathlineto{\pgfqpoint{2.276799in}{4.087695in}}%
\pgfpathlineto{\pgfqpoint{2.280184in}{4.085222in}}%
\pgfpathlineto{\pgfqpoint{2.286108in}{4.080893in}}%
\pgfpathlineto{\pgfqpoint{2.286416in}{4.080668in}}%
\pgfpathlineto{\pgfqpoint{2.292648in}{4.076115in}}%
\pgfpathlineto{\pgfqpoint{2.295418in}{4.074091in}}%
\pgfpathlineto{\pgfqpoint{2.298880in}{4.071562in}}%
\pgfpathlineto{\pgfqpoint{2.304727in}{4.067290in}}%
\pgfpathlineto{\pgfqpoint{2.305112in}{4.067009in}}%
\pgfpathlineto{\pgfqpoint{2.311344in}{4.062455in}}%
\pgfpathlineto{\pgfqpoint{2.314036in}{4.060488in}}%
\pgfpathlineto{\pgfqpoint{2.317575in}{4.057902in}}%
\pgfpathlineto{\pgfqpoint{2.323346in}{4.053686in}}%
\pgfpathlineto{\pgfqpoint{2.323807in}{4.053349in}}%
\pgfpathlineto{\pgfqpoint{2.330039in}{4.048795in}}%
\pgfpathlineto{\pgfqpoint{2.332655in}{4.046884in}}%
\pgfpathlineto{\pgfqpoint{2.336271in}{4.044242in}}%
\pgfpathlineto{\pgfqpoint{2.341964in}{4.040082in}}%
\pgfpathlineto{\pgfqpoint{2.342503in}{4.039689in}}%
\pgfpathlineto{\pgfqpoint{2.348735in}{4.035136in}}%
\pgfpathlineto{\pgfqpoint{2.351274in}{4.033281in}}%
\pgfpathlineto{\pgfqpoint{2.354967in}{4.030582in}}%
\pgfpathlineto{\pgfqpoint{2.360583in}{4.026479in}}%
\pgfpathlineto{\pgfqpoint{2.361198in}{4.026029in}}%
\pgfpathlineto{\pgfqpoint{2.367430in}{4.021476in}}%
\pgfpathlineto{\pgfqpoint{2.369892in}{4.019677in}}%
\pgfpathlineto{\pgfqpoint{2.373662in}{4.016923in}}%
\pgfpathlineto{\pgfqpoint{2.379202in}{4.012875in}}%
\pgfpathlineto{\pgfqpoint{2.379894in}{4.012369in}}%
\pgfpathlineto{\pgfqpoint{2.386126in}{4.007816in}}%
\pgfpathlineto{\pgfqpoint{2.388511in}{4.006073in}}%
\pgfpathlineto{\pgfqpoint{2.392358in}{4.003263in}}%
\pgfpathlineto{\pgfqpoint{2.397820in}{3.999272in}}%
\pgfpathlineto{\pgfqpoint{2.398589in}{3.998710in}}%
\pgfpathlineto{\pgfqpoint{2.404821in}{3.994156in}}%
\pgfpathlineto{\pgfqpoint{2.407129in}{3.992470in}}%
\pgfpathlineto{\pgfqpoint{2.411053in}{3.989603in}}%
\pgfpathlineto{\pgfqpoint{2.416439in}{3.985668in}}%
\pgfpathlineto{\pgfqpoint{2.417285in}{3.985050in}}%
\pgfpathlineto{\pgfqpoint{2.423517in}{3.980496in}}%
\pgfpathlineto{\pgfqpoint{2.425748in}{3.978866in}}%
\pgfpathlineto{\pgfqpoint{2.429749in}{3.975943in}}%
\pgfpathlineto{\pgfqpoint{2.435057in}{3.972064in}}%
\pgfpathlineto{\pgfqpoint{2.435981in}{3.971390in}}%
\pgfpathlineto{\pgfqpoint{2.442212in}{3.966837in}}%
\pgfpathlineto{\pgfqpoint{2.444367in}{3.965263in}}%
\pgfpathlineto{\pgfqpoint{2.448444in}{3.962283in}}%
\pgfpathlineto{\pgfqpoint{2.453676in}{3.958461in}}%
\pgfpathlineto{\pgfqpoint{2.454676in}{3.957730in}}%
\pgfpathlineto{\pgfqpoint{2.460908in}{3.953177in}}%
\pgfpathlineto{\pgfqpoint{2.462985in}{3.951659in}}%
\pgfpathlineto{\pgfqpoint{2.467140in}{3.948624in}}%
\pgfpathlineto{\pgfqpoint{2.472295in}{3.944857in}}%
\pgfpathlineto{\pgfqpoint{2.473372in}{3.944070in}}%
\pgfpathlineto{\pgfqpoint{2.479604in}{3.939517in}}%
\pgfpathlineto{\pgfqpoint{2.481604in}{3.938055in}}%
\pgfpathlineto{\pgfqpoint{2.485835in}{3.934964in}}%
\pgfpathlineto{\pgfqpoint{2.490913in}{3.931254in}}%
\pgfpathlineto{\pgfqpoint{2.492067in}{3.930410in}}%
\pgfpathlineto{\pgfqpoint{2.498299in}{3.925857in}}%
\pgfpathlineto{\pgfqpoint{2.500223in}{3.924452in}}%
\pgfpathlineto{\pgfqpoint{2.504531in}{3.921304in}}%
\pgfpathlineto{\pgfqpoint{2.509532in}{3.917650in}}%
\pgfpathlineto{\pgfqpoint{2.510763in}{3.916751in}}%
\pgfpathlineto{\pgfqpoint{2.516995in}{3.912197in}}%
\pgfpathlineto{\pgfqpoint{2.518841in}{3.910848in}}%
\pgfpathlineto{\pgfqpoint{2.523227in}{3.907644in}}%
\pgfpathlineto{\pgfqpoint{2.528150in}{3.904046in}}%
\pgfpathlineto{\pgfqpoint{2.529458in}{3.903091in}}%
\pgfpathlineto{\pgfqpoint{2.535690in}{3.898538in}}%
\pgfpathlineto{\pgfqpoint{2.537460in}{3.897245in}}%
\pgfpathlineto{\pgfqpoint{2.541922in}{3.893984in}}%
\pgfpathlineto{\pgfqpoint{2.546769in}{3.890443in}}%
\pgfpathlineto{\pgfqpoint{2.548154in}{3.889431in}}%
\pgfpathlineto{\pgfqpoint{2.554386in}{3.884878in}}%
\pgfpathlineto{\pgfqpoint{2.556078in}{3.883641in}}%
\pgfpathlineto{\pgfqpoint{2.560618in}{3.880324in}}%
\pgfpathlineto{\pgfqpoint{2.565388in}{3.876839in}}%
\pgfpathlineto{\pgfqpoint{2.566849in}{3.875771in}}%
\pgfpathlineto{\pgfqpoint{2.573081in}{3.871218in}}%
\pgfpathlineto{\pgfqpoint{2.574697in}{3.870037in}}%
\pgfpathlineto{\pgfqpoint{2.579313in}{3.866665in}}%
\pgfpathlineto{\pgfqpoint{2.584006in}{3.863236in}}%
\pgfpathlineto{\pgfqpoint{2.585545in}{3.862111in}}%
\pgfpathlineto{\pgfqpoint{2.591777in}{3.857558in}}%
\pgfpathlineto{\pgfqpoint{2.593316in}{3.856434in}}%
\pgfpathlineto{\pgfqpoint{2.598009in}{3.853005in}}%
\pgfpathlineto{\pgfqpoint{2.602625in}{3.849632in}}%
\pgfpathlineto{\pgfqpoint{2.604241in}{3.848452in}}%
\pgfpathlineto{\pgfqpoint{2.610472in}{3.843898in}}%
\pgfpathlineto{\pgfqpoint{2.611934in}{3.842830in}}%
\pgfpathlineto{\pgfqpoint{2.616704in}{3.839345in}}%
\pgfpathlineto{\pgfqpoint{2.621244in}{3.836028in}}%
\pgfpathlineto{\pgfqpoint{2.622936in}{3.834792in}}%
\pgfpathlineto{\pgfqpoint{2.629168in}{3.830238in}}%
\pgfpathlineto{\pgfqpoint{2.630553in}{3.829227in}}%
\pgfpathlineto{\pgfqpoint{2.635400in}{3.825685in}}%
\pgfpathlineto{\pgfqpoint{2.639862in}{3.822425in}}%
\pgfpathlineto{\pgfqpoint{2.641632in}{3.821132in}}%
\pgfpathlineto{\pgfqpoint{2.647864in}{3.816579in}}%
\pgfpathlineto{\pgfqpoint{2.649171in}{3.815623in}}%
\pgfpathlineto{\pgfqpoint{2.654095in}{3.812025in}}%
\pgfpathlineto{\pgfqpoint{2.658481in}{3.808821in}}%
\pgfpathlineto{\pgfqpoint{2.660327in}{3.807472in}}%
\pgfpathlineto{\pgfqpoint{2.666559in}{3.802919in}}%
\pgfpathlineto{\pgfqpoint{2.667790in}{3.802019in}}%
\pgfpathlineto{\pgfqpoint{2.672791in}{3.798366in}}%
\pgfpathlineto{\pgfqpoint{2.677099in}{3.795218in}}%
\pgfpathlineto{\pgfqpoint{2.679023in}{3.793812in}}%
\pgfpathlineto{\pgfqpoint{2.685255in}{3.789259in}}%
\pgfpathlineto{\pgfqpoint{2.686409in}{3.788416in}}%
\pgfpathlineto{\pgfqpoint{2.691487in}{3.784706in}}%
\pgfpathlineto{\pgfqpoint{2.695718in}{3.781614in}}%
\pgfpathlineto{\pgfqpoint{2.697718in}{3.780152in}}%
\pgfpathlineto{\pgfqpoint{2.703950in}{3.775599in}}%
\pgfpathlineto{\pgfqpoint{2.705027in}{3.774812in}}%
\pgfpathlineto{\pgfqpoint{2.710182in}{3.771046in}}%
\pgfpathlineto{\pgfqpoint{2.714337in}{3.768010in}}%
\pgfpathlineto{\pgfqpoint{2.716414in}{3.766493in}}%
\pgfpathlineto{\pgfqpoint{2.722646in}{3.761939in}}%
\pgfpathlineto{\pgfqpoint{2.723646in}{3.761209in}}%
\pgfpathlineto{\pgfqpoint{2.728878in}{3.757386in}}%
\pgfpathlineto{\pgfqpoint{2.732955in}{3.754407in}}%
\pgfpathlineto{\pgfqpoint{2.735109in}{3.752833in}}%
\pgfpathlineto{\pgfqpoint{2.741341in}{3.748280in}}%
\pgfpathlineto{\pgfqpoint{2.742265in}{3.747605in}}%
\pgfpathlineto{\pgfqpoint{2.747573in}{3.743726in}}%
\pgfpathlineto{\pgfqpoint{2.751574in}{3.740803in}}%
\pgfpathlineto{\pgfqpoint{2.753805in}{3.739173in}}%
\pgfpathlineto{\pgfqpoint{2.760037in}{3.734620in}}%
\pgfpathlineto{\pgfqpoint{2.760883in}{3.734001in}}%
\pgfpathlineto{\pgfqpoint{2.766269in}{3.730066in}}%
\pgfpathlineto{\pgfqpoint{2.770192in}{3.727200in}}%
\pgfpathlineto{\pgfqpoint{2.772501in}{3.725513in}}%
\pgfpathlineto{\pgfqpoint{2.778732in}{3.720960in}}%
\pgfpathlineto{\pgfqpoint{2.779502in}{3.720398in}}%
\pgfpathlineto{\pgfqpoint{2.784964in}{3.716407in}}%
\pgfpathlineto{\pgfqpoint{2.788811in}{3.713596in}}%
\pgfpathlineto{\pgfqpoint{2.791196in}{3.711853in}}%
\pgfpathlineto{\pgfqpoint{2.797428in}{3.707300in}}%
\pgfpathlineto{\pgfqpoint{2.798120in}{3.706794in}}%
\pgfpathlineto{\pgfqpoint{2.803660in}{3.702747in}}%
\pgfpathlineto{\pgfqpoint{2.807430in}{3.699992in}}%
\pgfpathlineto{\pgfqpoint{2.809892in}{3.698194in}}%
\pgfpathlineto{\pgfqpoint{2.816124in}{3.693640in}}%
\pgfpathlineto{\pgfqpoint{2.816739in}{3.693191in}}%
\pgfpathlineto{\pgfqpoint{2.822355in}{3.689087in}}%
\pgfpathlineto{\pgfqpoint{2.826048in}{3.686389in}}%
\pgfpathlineto{\pgfqpoint{2.828587in}{3.684534in}}%
\pgfpathlineto{\pgfqpoint{2.834819in}{3.679980in}}%
\pgfpathlineto{\pgfqpoint{2.835358in}{3.679587in}}%
\pgfpathlineto{\pgfqpoint{2.841051in}{3.675427in}}%
\pgfpathlineto{\pgfqpoint{2.844667in}{3.672785in}}%
\pgfpathlineto{\pgfqpoint{2.847283in}{3.670874in}}%
\pgfpathlineto{\pgfqpoint{2.853515in}{3.666321in}}%
\pgfpathlineto{\pgfqpoint{2.853976in}{3.665983in}}%
\pgfpathlineto{\pgfqpoint{2.859747in}{3.661767in}}%
\pgfpathlineto{\pgfqpoint{2.863286in}{3.659182in}}%
\pgfpathlineto{\pgfqpoint{2.865978in}{3.657214in}}%
\pgfpathlineto{\pgfqpoint{2.872210in}{3.652661in}}%
\pgfpathlineto{\pgfqpoint{2.872595in}{3.652380in}}%
\pgfpathlineto{\pgfqpoint{2.878442in}{3.648108in}}%
\pgfpathlineto{\pgfqpoint{2.881904in}{3.645578in}}%
\pgfpathlineto{\pgfqpoint{2.884674in}{3.643554in}}%
\pgfpathlineto{\pgfqpoint{2.890906in}{3.639001in}}%
\pgfpathlineto{\pgfqpoint{2.891214in}{3.638776in}}%
\pgfpathlineto{\pgfqpoint{2.897138in}{3.634448in}}%
\pgfpathlineto{\pgfqpoint{2.900523in}{3.631974in}}%
\pgfpathlineto{\pgfqpoint{2.903369in}{3.629894in}}%
\pgfpathlineto{\pgfqpoint{2.909601in}{3.625341in}}%
\pgfpathlineto{\pgfqpoint{2.909832in}{3.625173in}}%
\pgfpathlineto{\pgfqpoint{2.915833in}{3.620788in}}%
\pgfpathlineto{\pgfqpoint{2.919141in}{3.618371in}}%
\pgfpathlineto{\pgfqpoint{2.922065in}{3.616235in}}%
\pgfpathlineto{\pgfqpoint{2.928297in}{3.611681in}}%
\pgfpathlineto{\pgfqpoint{2.928451in}{3.611569in}}%
\pgfpathlineto{\pgfqpoint{2.934529in}{3.607128in}}%
\pgfpathlineto{\pgfqpoint{2.937760in}{3.604767in}}%
\pgfpathlineto{\pgfqpoint{2.940761in}{3.602575in}}%
\pgfpathlineto{\pgfqpoint{2.946992in}{3.598022in}}%
\pgfpathlineto{\pgfqpoint{2.947069in}{3.597965in}}%
\pgfpathlineto{\pgfqpoint{2.953224in}{3.593468in}}%
\pgfpathlineto{\pgfqpoint{2.956379in}{3.591164in}}%
\pgfpathlineto{\pgfqpoint{2.959456in}{3.588915in}}%
\pgfpathlineto{\pgfqpoint{2.965688in}{3.584362in}}%
\pgfpathlineto{\pgfqpoint{2.965688in}{3.584362in}}%
\pgfpathlineto{\pgfqpoint{2.971920in}{3.579808in}}%
\pgfpathlineto{\pgfqpoint{2.974997in}{3.577560in}}%
\pgfpathlineto{\pgfqpoint{2.978152in}{3.575255in}}%
\pgfpathlineto{\pgfqpoint{2.984307in}{3.570758in}}%
\pgfpathlineto{\pgfqpoint{2.984384in}{3.570702in}}%
\pgfpathlineto{\pgfqpoint{2.990615in}{3.566149in}}%
\pgfpathlineto{\pgfqpoint{2.993616in}{3.563956in}}%
\pgfpathlineto{\pgfqpoint{2.996847in}{3.561595in}}%
\pgfpathlineto{\pgfqpoint{3.002925in}{3.557155in}}%
\pgfpathlineto{\pgfqpoint{3.003079in}{3.557042in}}%
\pgfpathlineto{\pgfqpoint{3.009311in}{3.552489in}}%
\pgfpathlineto{\pgfqpoint{3.012235in}{3.550353in}}%
\pgfpathlineto{\pgfqpoint{3.015543in}{3.547936in}}%
\pgfpathlineto{\pgfqpoint{3.021544in}{3.543551in}}%
\pgfpathlineto{\pgfqpoint{3.021775in}{3.543382in}}%
\pgfpathlineto{\pgfqpoint{3.028007in}{3.538829in}}%
\pgfpathlineto{\pgfqpoint{3.030853in}{3.536749in}}%
\pgfpathlineto{\pgfqpoint{3.034238in}{3.534276in}}%
\pgfpathlineto{\pgfqpoint{3.040162in}{3.529947in}}%
\pgfpathlineto{\pgfqpoint{3.040470in}{3.529722in}}%
\pgfpathlineto{\pgfqpoint{3.046702in}{3.525169in}}%
\pgfpathlineto{\pgfqpoint{3.049472in}{3.523146in}}%
\pgfpathlineto{\pgfqpoint{3.052934in}{3.520616in}}%
\pgfpathlineto{\pgfqpoint{3.058781in}{3.516344in}}%
\pgfpathlineto{\pgfqpoint{3.059166in}{3.516063in}}%
\pgfpathlineto{\pgfqpoint{3.065398in}{3.511509in}}%
\pgfpathlineto{\pgfqpoint{3.068090in}{3.509542in}}%
\pgfpathlineto{\pgfqpoint{3.071629in}{3.506956in}}%
\pgfpathlineto{\pgfqpoint{3.077400in}{3.502740in}}%
\pgfpathlineto{\pgfqpoint{3.077861in}{3.502403in}}%
\pgfpathlineto{\pgfqpoint{3.084093in}{3.497850in}}%
\pgfpathlineto{\pgfqpoint{3.086709in}{3.495938in}}%
\pgfpathlineto{\pgfqpoint{3.090325in}{3.493296in}}%
\pgfpathlineto{\pgfqpoint{3.096018in}{3.489137in}}%
\pgfpathlineto{\pgfqpoint{3.096557in}{3.488743in}}%
\pgfpathlineto{\pgfqpoint{3.102789in}{3.484190in}}%
\pgfpathlineto{\pgfqpoint{3.105328in}{3.482335in}}%
\pgfpathlineto{\pgfqpoint{3.109021in}{3.479636in}}%
\pgfpathlineto{\pgfqpoint{3.114637in}{3.475533in}}%
\pgfpathlineto{\pgfqpoint{3.115252in}{3.475083in}}%
\pgfpathlineto{\pgfqpoint{3.121484in}{3.470530in}}%
\pgfpathlineto{\pgfqpoint{3.123946in}{3.468731in}}%
\pgfpathlineto{\pgfqpoint{3.127716in}{3.465977in}}%
\pgfpathlineto{\pgfqpoint{3.133256in}{3.461929in}}%
\pgfpathlineto{\pgfqpoint{3.133948in}{3.461423in}}%
\pgfpathlineto{\pgfqpoint{3.140180in}{3.456870in}}%
\pgfpathlineto{\pgfqpoint{3.142565in}{3.455127in}}%
\pgfpathlineto{\pgfqpoint{3.146412in}{3.452317in}}%
\pgfpathlineto{\pgfqpoint{3.151874in}{3.448326in}}%
\pgfpathlineto{\pgfqpoint{3.152644in}{3.447764in}}%
\pgfpathlineto{\pgfqpoint{3.158875in}{3.443210in}}%
\pgfpathlineto{\pgfqpoint{3.161183in}{3.441524in}}%
\pgfpathlineto{\pgfqpoint{3.165107in}{3.438657in}}%
\pgfpathlineto{\pgfqpoint{3.170493in}{3.434722in}}%
\pgfpathlineto{\pgfqpoint{3.171339in}{3.434104in}}%
\pgfpathlineto{\pgfqpoint{3.177571in}{3.429550in}}%
\pgfpathlineto{\pgfqpoint{3.179802in}{3.427920in}}%
\pgfpathlineto{\pgfqpoint{3.183803in}{3.424997in}}%
\pgfpathlineto{\pgfqpoint{3.189111in}{3.421118in}}%
\pgfpathlineto{\pgfqpoint{3.190035in}{3.420444in}}%
\pgfpathlineto{\pgfqpoint{3.196267in}{3.415891in}}%
\pgfpathlineto{\pgfqpoint{3.198421in}{3.414317in}}%
\pgfpathlineto{\pgfqpoint{3.202498in}{3.411337in}}%
\pgfpathlineto{\pgfqpoint{3.207730in}{3.407515in}}%
\pgfpathlineto{\pgfqpoint{3.208730in}{3.406784in}}%
\pgfpathlineto{\pgfqpoint{3.214962in}{3.402231in}}%
\pgfpathlineto{\pgfqpoint{3.217039in}{3.400713in}}%
\pgfpathlineto{\pgfqpoint{3.221194in}{3.397678in}}%
\pgfpathlineto{\pgfqpoint{3.226349in}{3.393911in}}%
\pgfpathlineto{\pgfqpoint{3.227426in}{3.393124in}}%
\pgfpathlineto{\pgfqpoint{3.233658in}{3.388571in}}%
\pgfpathlineto{\pgfqpoint{3.235658in}{3.387109in}}%
\pgfpathlineto{\pgfqpoint{3.239889in}{3.384018in}}%
\pgfpathlineto{\pgfqpoint{3.244967in}{3.380308in}}%
\pgfpathlineto{\pgfqpoint{3.246121in}{3.379464in}}%
\pgfpathlineto{\pgfqpoint{3.252353in}{3.374911in}}%
\pgfpathlineto{\pgfqpoint{3.254277in}{3.373506in}}%
\pgfpathlineto{\pgfqpoint{3.258585in}{3.370358in}}%
\pgfpathlineto{\pgfqpoint{3.263586in}{3.366704in}}%
\pgfpathlineto{\pgfqpoint{3.264817in}{3.365805in}}%
\pgfpathlineto{\pgfqpoint{3.271049in}{3.361251in}}%
\pgfpathlineto{\pgfqpoint{3.272895in}{3.359902in}}%
\pgfpathlineto{\pgfqpoint{3.277281in}{3.356698in}}%
\pgfpathlineto{\pgfqpoint{3.282205in}{3.353100in}}%
\pgfpathlineto{\pgfqpoint{3.283512in}{3.352145in}}%
\pgfpathlineto{\pgfqpoint{3.289744in}{3.347592in}}%
\pgfpathlineto{\pgfqpoint{3.291514in}{3.346299in}}%
\pgfpathlineto{\pgfqpoint{3.295976in}{3.343038in}}%
\pgfpathlineto{\pgfqpoint{3.300823in}{3.339497in}}%
\pgfpathlineto{\pgfqpoint{3.302208in}{3.338485in}}%
\pgfpathlineto{\pgfqpoint{3.308440in}{3.333932in}}%
\pgfpathlineto{\pgfqpoint{3.310132in}{3.332695in}}%
\pgfpathlineto{\pgfqpoint{3.314672in}{3.329378in}}%
\pgfpathlineto{\pgfqpoint{3.319442in}{3.325893in}}%
\pgfpathlineto{\pgfqpoint{3.320904in}{3.324825in}}%
\pgfpathlineto{\pgfqpoint{3.327135in}{3.320272in}}%
\pgfpathlineto{\pgfqpoint{3.328751in}{3.319091in}}%
\pgfpathlineto{\pgfqpoint{3.333367in}{3.315719in}}%
\pgfpathlineto{\pgfqpoint{3.338060in}{3.312290in}}%
\pgfpathlineto{\pgfqpoint{3.339599in}{3.311165in}}%
\pgfpathlineto{\pgfqpoint{3.345831in}{3.306612in}}%
\pgfpathlineto{\pgfqpoint{3.347370in}{3.305488in}}%
\pgfpathlineto{\pgfqpoint{3.352063in}{3.302059in}}%
\pgfpathlineto{\pgfqpoint{3.356679in}{3.298686in}}%
\pgfpathlineto{\pgfqpoint{3.358295in}{3.297506in}}%
\pgfpathlineto{\pgfqpoint{3.364527in}{3.292952in}}%
\pgfpathlineto{\pgfqpoint{3.365988in}{3.291884in}}%
\pgfpathlineto{\pgfqpoint{3.370758in}{3.288399in}}%
\pgfpathlineto{\pgfqpoint{3.375298in}{3.285082in}}%
\pgfpathlineto{\pgfqpoint{3.376990in}{3.283846in}}%
\pgfpathlineto{\pgfqpoint{3.383222in}{3.279292in}}%
\pgfpathlineto{\pgfqpoint{3.384607in}{3.278281in}}%
\pgfpathlineto{\pgfqpoint{3.389454in}{3.274739in}}%
\pgfpathlineto{\pgfqpoint{3.393916in}{3.271479in}}%
\pgfpathlineto{\pgfqpoint{3.395686in}{3.270186in}}%
\pgfpathlineto{\pgfqpoint{3.401918in}{3.265633in}}%
\pgfpathlineto{\pgfqpoint{3.403226in}{3.264677in}}%
\pgfpathlineto{\pgfqpoint{3.408149in}{3.261079in}}%
\pgfpathlineto{\pgfqpoint{3.412535in}{3.257875in}}%
\pgfpathlineto{\pgfqpoint{3.414381in}{3.256526in}}%
\pgfpathlineto{\pgfqpoint{3.420613in}{3.251973in}}%
\pgfpathlineto{\pgfqpoint{3.421844in}{3.251073in}}%
\pgfpathlineto{\pgfqpoint{3.426845in}{3.247420in}}%
\pgfpathlineto{\pgfqpoint{3.431153in}{3.244272in}}%
\pgfpathlineto{\pgfqpoint{3.433077in}{3.242866in}}%
\pgfpathlineto{\pgfqpoint{3.439309in}{3.238313in}}%
\pgfpathlineto{\pgfqpoint{3.440463in}{3.237470in}}%
\pgfpathlineto{\pgfqpoint{3.445541in}{3.233760in}}%
\pgfpathlineto{\pgfqpoint{3.449772in}{3.230668in}}%
\pgfpathlineto{\pgfqpoint{3.451772in}{3.229206in}}%
\pgfpathlineto{\pgfqpoint{3.458004in}{3.224653in}}%
\pgfpathlineto{\pgfqpoint{3.459081in}{3.223866in}}%
\pgfpathlineto{\pgfqpoint{3.464236in}{3.220100in}}%
\pgfpathlineto{\pgfqpoint{3.468391in}{3.217064in}}%
\pgfpathlineto{\pgfqpoint{3.470468in}{3.215547in}}%
\pgfpathlineto{\pgfqpoint{3.476700in}{3.210993in}}%
\pgfpathlineto{\pgfqpoint{3.477700in}{3.210263in}}%
\pgfpathlineto{\pgfqpoint{3.482932in}{3.206440in}}%
\pgfpathlineto{\pgfqpoint{3.487009in}{3.203461in}}%
\pgfpathlineto{\pgfqpoint{3.489164in}{3.201887in}}%
\pgfpathlineto{\pgfqpoint{3.495395in}{3.197334in}}%
\pgfpathlineto{\pgfqpoint{3.496319in}{3.196659in}}%
\pgfpathlineto{\pgfqpoint{3.501627in}{3.192780in}}%
\pgfpathlineto{\pgfqpoint{3.505628in}{3.189857in}}%
\pgfpathlineto{\pgfqpoint{3.507859in}{3.188227in}}%
\pgfpathlineto{\pgfqpoint{3.514091in}{3.183674in}}%
\pgfpathlineto{\pgfqpoint{3.514937in}{3.183055in}}%
\pgfpathlineto{\pgfqpoint{3.520323in}{3.179120in}}%
\pgfpathlineto{\pgfqpoint{3.524247in}{3.176254in}}%
\pgfpathlineto{\pgfqpoint{3.526555in}{3.174567in}}%
\pgfpathlineto{\pgfqpoint{3.532786in}{3.170014in}}%
\pgfpathlineto{\pgfqpoint{3.533556in}{3.169452in}}%
\pgfpathlineto{\pgfqpoint{3.539018in}{3.165461in}}%
\pgfpathlineto{\pgfqpoint{3.542865in}{3.162650in}}%
\pgfpathlineto{\pgfqpoint{3.545250in}{3.160907in}}%
\pgfpathlineto{\pgfqpoint{3.551482in}{3.156354in}}%
\pgfpathlineto{\pgfqpoint{3.552174in}{3.155848in}}%
\pgfpathlineto{\pgfqpoint{3.557714in}{3.151801in}}%
\pgfpathlineto{\pgfqpoint{3.561484in}{3.149046in}}%
\pgfpathlineto{\pgfqpoint{3.563946in}{3.147248in}}%
\pgfpathlineto{\pgfqpoint{3.570178in}{3.142694in}}%
\pgfpathlineto{\pgfqpoint{3.570793in}{3.142245in}}%
\pgfpathlineto{\pgfqpoint{3.576409in}{3.138141in}}%
\pgfpathlineto{\pgfqpoint{3.580102in}{3.135443in}}%
\pgfpathlineto{\pgfqpoint{3.582641in}{3.133588in}}%
\pgfpathlineto{\pgfqpoint{3.588873in}{3.129034in}}%
\pgfpathlineto{\pgfqpoint{3.589412in}{3.128641in}}%
\pgfpathlineto{\pgfqpoint{3.595105in}{3.124481in}}%
\pgfpathlineto{\pgfqpoint{3.598721in}{3.121839in}}%
\pgfpathlineto{\pgfqpoint{3.601337in}{3.119928in}}%
\pgfpathlineto{\pgfqpoint{3.607569in}{3.115375in}}%
\pgfpathlineto{\pgfqpoint{3.608030in}{3.115037in}}%
\pgfpathlineto{\pgfqpoint{3.613801in}{3.110821in}}%
\pgfpathlineto{\pgfqpoint{3.617340in}{3.108236in}}%
\pgfpathlineto{\pgfqpoint{3.620032in}{3.106268in}}%
\pgfpathlineto{\pgfqpoint{3.626264in}{3.101715in}}%
\pgfpathlineto{\pgfqpoint{3.626649in}{3.101434in}}%
\pgfpathlineto{\pgfqpoint{3.632496in}{3.097162in}}%
\pgfpathlineto{\pgfqpoint{3.635958in}{3.094632in}}%
\pgfpathlineto{\pgfqpoint{3.638728in}{3.092608in}}%
\pgfpathlineto{\pgfqpoint{3.644960in}{3.088055in}}%
\pgfpathlineto{\pgfqpoint{3.645268in}{3.087830in}}%
\pgfpathlineto{\pgfqpoint{3.651192in}{3.083502in}}%
\pgfpathlineto{\pgfqpoint{3.654577in}{3.081028in}}%
\pgfpathlineto{\pgfqpoint{3.657424in}{3.078949in}}%
\pgfpathlineto{\pgfqpoint{3.663655in}{3.074395in}}%
\pgfpathlineto{\pgfqpoint{3.663886in}{3.074227in}}%
\pgfpathlineto{\pgfqpoint{3.669887in}{3.069842in}}%
\pgfpathlineto{\pgfqpoint{3.673196in}{3.067425in}}%
\pgfpathlineto{\pgfqpoint{3.676119in}{3.065289in}}%
\pgfpathlineto{\pgfqpoint{3.682351in}{3.060735in}}%
\pgfpathlineto{\pgfqpoint{3.682505in}{3.060623in}}%
\pgfpathlineto{\pgfqpoint{3.688583in}{3.056182in}}%
\pgfpathlineto{\pgfqpoint{3.691814in}{3.053821in}}%
\pgfpathlineto{\pgfqpoint{3.694815in}{3.051629in}}%
\pgfpathlineto{\pgfqpoint{3.701046in}{3.047076in}}%
\pgfpathlineto{\pgfqpoint{3.701123in}{3.047019in}}%
\pgfpathlineto{\pgfqpoint{3.707278in}{3.042522in}}%
\pgfpathlineto{\pgfqpoint{3.710433in}{3.040218in}}%
\pgfpathlineto{\pgfqpoint{3.713510in}{3.037969in}}%
\pgfpathlineto{\pgfqpoint{3.719742in}{3.033416in}}%
\pgfpathlineto{\pgfqpoint{3.719742in}{3.033416in}}%
\pgfpathlineto{\pgfqpoint{3.725974in}{3.028863in}}%
\pgfpathlineto{\pgfqpoint{3.729051in}{3.026614in}}%
\pgfpathlineto{\pgfqpoint{3.732206in}{3.024309in}}%
\pgfpathlineto{\pgfqpoint{3.738361in}{3.019812in}}%
\pgfpathlineto{\pgfqpoint{3.738438in}{3.019756in}}%
\pgfpathlineto{\pgfqpoint{3.744669in}{3.015203in}}%
\pgfpathlineto{\pgfqpoint{3.747670in}{3.013010in}}%
\pgfpathlineto{\pgfqpoint{3.750901in}{3.010649in}}%
\pgfpathlineto{\pgfqpoint{3.756979in}{3.006209in}}%
\pgfpathlineto{\pgfqpoint{3.757133in}{3.006096in}}%
\pgfpathlineto{\pgfqpoint{3.763365in}{3.001543in}}%
\pgfpathlineto{\pgfqpoint{3.766289in}{2.999407in}}%
\pgfpathlineto{\pgfqpoint{3.769597in}{2.996990in}}%
\pgfpathlineto{\pgfqpoint{3.775598in}{2.992605in}}%
\pgfpathlineto{\pgfqpoint{3.775829in}{2.992436in}}%
\pgfpathlineto{\pgfqpoint{3.782061in}{2.987883in}}%
\pgfpathlineto{\pgfqpoint{3.784907in}{2.985803in}}%
\pgfpathlineto{\pgfqpoint{3.788292in}{2.983330in}}%
\pgfpathlineto{\pgfqpoint{3.794217in}{2.979001in}}%
\pgfpathlineto{\pgfqpoint{3.794524in}{2.978777in}}%
\pgfpathlineto{\pgfqpoint{3.800756in}{2.974223in}}%
\pgfpathlineto{\pgfqpoint{3.803526in}{2.972200in}}%
\pgfpathlineto{\pgfqpoint{3.806988in}{2.969670in}}%
\pgfpathlineto{\pgfqpoint{3.812835in}{2.965398in}}%
\pgfpathlineto{\pgfqpoint{3.813220in}{2.965117in}}%
\pgfpathlineto{\pgfqpoint{3.819452in}{2.960563in}}%
\pgfpathlineto{\pgfqpoint{3.822144in}{2.958596in}}%
\pgfpathlineto{\pgfqpoint{3.825684in}{2.956010in}}%
\pgfpathlineto{\pgfqpoint{3.831454in}{2.951794in}}%
\pgfpathlineto{\pgfqpoint{3.831915in}{2.951457in}}%
\pgfpathlineto{\pgfqpoint{3.838147in}{2.946904in}}%
\pgfpathlineto{\pgfqpoint{3.840763in}{2.944992in}}%
\pgfpathlineto{\pgfqpoint{3.844379in}{2.942350in}}%
\pgfpathlineto{\pgfqpoint{3.850072in}{2.938191in}}%
\pgfpathlineto{\pgfqpoint{3.850611in}{2.937797in}}%
\pgfpathlineto{\pgfqpoint{3.856843in}{2.933244in}}%
\pgfpathlineto{\pgfqpoint{3.859382in}{2.931389in}}%
\pgfpathlineto{\pgfqpoint{3.863075in}{2.928691in}}%
\pgfpathlineto{\pgfqpoint{3.868691in}{2.924587in}}%
\pgfpathlineto{\pgfqpoint{3.869306in}{2.924137in}}%
\pgfpathlineto{\pgfqpoint{3.875538in}{2.919584in}}%
\pgfpathlineto{\pgfqpoint{3.878000in}{2.917785in}}%
\pgfpathlineto{\pgfqpoint{3.881770in}{2.915031in}}%
\pgfpathlineto{\pgfqpoint{3.887310in}{2.910983in}}%
\pgfpathlineto{\pgfqpoint{3.888002in}{2.910477in}}%
\pgfpathlineto{\pgfqpoint{3.894234in}{2.905924in}}%
\pgfpathlineto{\pgfqpoint{3.896619in}{2.904182in}}%
\pgfpathlineto{\pgfqpoint{3.900466in}{2.901371in}}%
\pgfpathlineto{\pgfqpoint{3.905928in}{2.897380in}}%
\pgfpathlineto{\pgfqpoint{3.906698in}{2.896818in}}%
\pgfpathlineto{\pgfqpoint{3.912929in}{2.892264in}}%
\pgfpathlineto{\pgfqpoint{3.915238in}{2.890578in}}%
\pgfpathlineto{\pgfqpoint{3.919161in}{2.887711in}}%
\pgfpathlineto{\pgfqpoint{3.924547in}{2.883776in}}%
\pgfpathlineto{\pgfqpoint{3.925393in}{2.883158in}}%
\pgfpathlineto{\pgfqpoint{3.931625in}{2.878605in}}%
\pgfpathlineto{\pgfqpoint{3.933856in}{2.876974in}}%
\pgfpathlineto{\pgfqpoint{3.937857in}{2.874051in}}%
\pgfpathlineto{\pgfqpoint{3.943165in}{2.870173in}}%
\pgfpathlineto{\pgfqpoint{3.944089in}{2.869498in}}%
\pgfpathlineto{\pgfqpoint{3.950321in}{2.864945in}}%
\pgfpathlineto{\pgfqpoint{3.952475in}{2.863371in}}%
\pgfpathlineto{\pgfqpoint{3.956552in}{2.860391in}}%
\pgfpathlineto{\pgfqpoint{3.961784in}{2.856569in}}%
\pgfpathlineto{\pgfqpoint{3.962784in}{2.855838in}}%
\pgfpathlineto{\pgfqpoint{3.969016in}{2.851285in}}%
\pgfpathlineto{\pgfqpoint{3.971093in}{2.849767in}}%
\pgfpathlineto{\pgfqpoint{3.975248in}{2.846732in}}%
\pgfpathlineto{\pgfqpoint{3.980403in}{2.842965in}}%
\pgfpathlineto{\pgfqpoint{3.981480in}{2.842178in}}%
\pgfpathlineto{\pgfqpoint{3.987712in}{2.837625in}}%
\pgfpathlineto{\pgfqpoint{3.989712in}{2.836164in}}%
\pgfpathlineto{\pgfqpoint{3.993944in}{2.833072in}}%
\pgfpathlineto{\pgfqpoint{3.999021in}{2.829362in}}%
\pgfpathlineto{\pgfqpoint{4.000175in}{2.828519in}}%
\pgfpathlineto{\pgfqpoint{4.006407in}{2.823965in}}%
\pgfpathlineto{\pgfqpoint{4.008331in}{2.822560in}}%
\pgfpathlineto{\pgfqpoint{4.012639in}{2.819412in}}%
\pgfpathlineto{\pgfqpoint{4.017640in}{2.815758in}}%
\pgfpathlineto{\pgfqpoint{4.018871in}{2.814859in}}%
\pgfpathlineto{\pgfqpoint{4.025103in}{2.810305in}}%
\pgfpathlineto{\pgfqpoint{4.026949in}{2.808956in}}%
\pgfpathlineto{\pgfqpoint{4.031335in}{2.805752in}}%
\pgfpathlineto{\pgfqpoint{4.036259in}{2.802155in}}%
\pgfpathlineto{\pgfqpoint{4.037566in}{2.801199in}}%
\pgfpathlineto{\pgfqpoint{4.043798in}{2.796646in}}%
\pgfpathlineto{\pgfqpoint{4.045568in}{2.795353in}}%
\pgfpathlineto{\pgfqpoint{4.050030in}{2.792092in}}%
\pgfpathlineto{\pgfqpoint{4.054877in}{2.788551in}}%
\pgfpathlineto{\pgfqpoint{4.056262in}{2.787539in}}%
\pgfpathlineto{\pgfqpoint{4.062494in}{2.782986in}}%
\pgfpathlineto{\pgfqpoint{4.064186in}{2.781749in}}%
\pgfpathlineto{\pgfqpoint{4.068726in}{2.778433in}}%
\pgfpathlineto{\pgfqpoint{4.073496in}{2.774947in}}%
\pgfpathlineto{\pgfqpoint{4.074958in}{2.773879in}}%
\pgfpathlineto{\pgfqpoint{4.081189in}{2.769326in}}%
\pgfpathlineto{\pgfqpoint{4.082805in}{2.768146in}}%
\pgfpathlineto{\pgfqpoint{4.087421in}{2.764773in}}%
\pgfpathlineto{\pgfqpoint{4.092114in}{2.761344in}}%
\pgfpathlineto{\pgfqpoint{4.093653in}{2.760219in}}%
\pgfpathlineto{\pgfqpoint{4.099885in}{2.755666in}}%
\pgfpathlineto{\pgfqpoint{4.101424in}{2.754542in}}%
\pgfpathlineto{\pgfqpoint{4.106117in}{2.751113in}}%
\pgfpathlineto{\pgfqpoint{4.110733in}{2.747740in}}%
\pgfpathlineto{\pgfqpoint{4.112349in}{2.746560in}}%
\pgfpathlineto{\pgfqpoint{4.118581in}{2.742006in}}%
\pgfpathlineto{\pgfqpoint{4.120042in}{2.740938in}}%
\pgfpathlineto{\pgfqpoint{4.124812in}{2.737453in}}%
\pgfpathlineto{\pgfqpoint{4.129352in}{2.734137in}}%
\pgfpathlineto{\pgfqpoint{4.131044in}{2.732900in}}%
\pgfpathlineto{\pgfqpoint{4.137276in}{2.728347in}}%
\pgfpathlineto{\pgfqpoint{4.138661in}{2.727335in}}%
\pgfpathlineto{\pgfqpoint{4.143508in}{2.723793in}}%
\pgfpathlineto{\pgfqpoint{4.147970in}{2.720533in}}%
\pgfpathlineto{\pgfqpoint{4.149740in}{2.719240in}}%
\pgfpathlineto{\pgfqpoint{4.155972in}{2.714687in}}%
\pgfpathlineto{\pgfqpoint{4.157280in}{2.713731in}}%
\pgfpathlineto{\pgfqpoint{4.162204in}{2.710133in}}%
\pgfpathlineto{\pgfqpoint{4.166589in}{2.706929in}}%
\pgfpathlineto{\pgfqpoint{4.168435in}{2.705580in}}%
\pgfpathlineto{\pgfqpoint{4.174667in}{2.701027in}}%
\pgfpathlineto{\pgfqpoint{4.175898in}{2.700127in}}%
\pgfpathlineto{\pgfqpoint{4.180899in}{2.696474in}}%
\pgfpathlineto{\pgfqpoint{4.185208in}{2.693326in}}%
\pgfpathlineto{\pgfqpoint{4.187131in}{2.691920in}}%
\pgfpathlineto{\pgfqpoint{4.193363in}{2.687367in}}%
\pgfpathlineto{\pgfqpoint{4.194517in}{2.686524in}}%
\pgfpathlineto{\pgfqpoint{4.199595in}{2.682814in}}%
\pgfpathlineto{\pgfqpoint{4.203826in}{2.679722in}}%
\pgfpathlineto{\pgfqpoint{4.205826in}{2.678261in}}%
\pgfpathlineto{\pgfqpoint{4.212058in}{2.673707in}}%
\pgfpathlineto{\pgfqpoint{4.213135in}{2.672920in}}%
\pgfpathlineto{\pgfqpoint{4.218290in}{2.669154in}}%
\pgfpathlineto{\pgfqpoint{4.222445in}{2.666118in}}%
\pgfpathlineto{\pgfqpoint{4.224522in}{2.664601in}}%
\pgfpathlineto{\pgfqpoint{4.230754in}{2.660047in}}%
\pgfpathlineto{\pgfqpoint{4.231754in}{2.659317in}}%
\pgfpathlineto{\pgfqpoint{4.236986in}{2.655494in}}%
\pgfpathlineto{\pgfqpoint{4.241063in}{2.652515in}}%
\pgfpathlineto{\pgfqpoint{4.243218in}{2.650941in}}%
\pgfpathlineto{\pgfqpoint{4.249449in}{2.646388in}}%
\pgfpathlineto{\pgfqpoint{4.250373in}{2.645713in}}%
\pgfpathlineto{\pgfqpoint{4.255681in}{2.641834in}}%
\pgfpathlineto{\pgfqpoint{4.259682in}{2.638911in}}%
\pgfpathlineto{\pgfqpoint{4.261913in}{2.637281in}}%
\pgfpathlineto{\pgfqpoint{4.268145in}{2.632728in}}%
\pgfpathlineto{\pgfqpoint{4.268991in}{2.632109in}}%
\pgfpathlineto{\pgfqpoint{4.274377in}{2.628175in}}%
\pgfpathlineto{\pgfqpoint{4.278301in}{2.625308in}}%
\pgfpathlineto{\pgfqpoint{4.280609in}{2.623621in}}%
\pgfpathlineto{\pgfqpoint{4.286841in}{2.619068in}}%
\pgfpathlineto{\pgfqpoint{4.287610in}{2.618506in}}%
\pgfpathlineto{\pgfqpoint{4.293072in}{2.614515in}}%
\pgfpathlineto{\pgfqpoint{4.296919in}{2.611704in}}%
\pgfpathlineto{\pgfqpoint{4.299304in}{2.609961in}}%
\pgfpathlineto{\pgfqpoint{4.305536in}{2.605408in}}%
\pgfpathlineto{\pgfqpoint{4.306229in}{2.604902in}}%
\pgfpathlineto{\pgfqpoint{4.311768in}{2.600855in}}%
\pgfpathlineto{\pgfqpoint{4.315538in}{2.598100in}}%
\pgfpathlineto{\pgfqpoint{4.318000in}{2.596302in}}%
\pgfpathlineto{\pgfqpoint{4.324232in}{2.591748in}}%
\pgfpathlineto{\pgfqpoint{4.324847in}{2.591299in}}%
\pgfpathlineto{\pgfqpoint{4.330464in}{2.587195in}}%
\pgfpathlineto{\pgfqpoint{4.334156in}{2.584497in}}%
\pgfpathlineto{\pgfqpoint{4.336695in}{2.582642in}}%
\pgfpathlineto{\pgfqpoint{4.342927in}{2.578089in}}%
\pgfpathlineto{\pgfqpoint{4.343466in}{2.577695in}}%
\pgfpathlineto{\pgfqpoint{4.349159in}{2.573535in}}%
\pgfpathlineto{\pgfqpoint{4.352775in}{2.570893in}}%
\pgfpathlineto{\pgfqpoint{4.355391in}{2.568982in}}%
\pgfpathlineto{\pgfqpoint{4.361623in}{2.564429in}}%
\pgfpathlineto{\pgfqpoint{4.362084in}{2.564091in}}%
\pgfpathlineto{\pgfqpoint{4.367855in}{2.559875in}}%
\pgfpathlineto{\pgfqpoint{4.371394in}{2.557290in}}%
\pgfpathlineto{\pgfqpoint{4.374086in}{2.555322in}}%
\pgfpathlineto{\pgfqpoint{4.380318in}{2.550769in}}%
\pgfpathlineto{\pgfqpoint{4.380703in}{2.550488in}}%
\pgfpathlineto{\pgfqpoint{4.386550in}{2.546216in}}%
\pgfpathlineto{\pgfqpoint{4.390012in}{2.543686in}}%
\pgfpathlineto{\pgfqpoint{4.392782in}{2.541662in}}%
\pgfpathlineto{\pgfqpoint{4.399014in}{2.537109in}}%
\pgfpathlineto{\pgfqpoint{4.399322in}{2.536884in}}%
\pgfpathlineto{\pgfqpoint{4.405246in}{2.532556in}}%
\pgfpathlineto{\pgfqpoint{4.408631in}{2.530082in}}%
\pgfpathlineto{\pgfqpoint{4.411478in}{2.528003in}}%
\pgfpathlineto{\pgfqpoint{4.417709in}{2.523449in}}%
\pgfpathlineto{\pgfqpoint{4.417940in}{2.523281in}}%
\pgfpathlineto{\pgfqpoint{4.423941in}{2.518896in}}%
\pgfpathlineto{\pgfqpoint{4.427250in}{2.516479in}}%
\pgfpathlineto{\pgfqpoint{4.430173in}{2.514343in}}%
\pgfpathlineto{\pgfqpoint{4.436405in}{2.509789in}}%
\pgfpathlineto{\pgfqpoint{4.436559in}{2.509677in}}%
\pgfpathlineto{\pgfqpoint{4.442637in}{2.505236in}}%
\pgfpathlineto{\pgfqpoint{4.445868in}{2.502875in}}%
\pgfpathlineto{\pgfqpoint{4.448869in}{2.500683in}}%
\pgfpathlineto{\pgfqpoint{4.455101in}{2.496130in}}%
\pgfpathlineto{\pgfqpoint{4.455177in}{2.496073in}}%
\pgfpathlineto{\pgfqpoint{4.461332in}{2.491576in}}%
\pgfpathlineto{\pgfqpoint{4.464487in}{2.489272in}}%
\pgfpathlineto{\pgfqpoint{4.467564in}{2.487023in}}%
\pgfpathlineto{\pgfqpoint{4.473796in}{2.482470in}}%
\pgfpathlineto{\pgfqpoint{4.473796in}{2.482470in}}%
\pgfpathlineto{\pgfqpoint{4.480028in}{2.477917in}}%
\pgfpathlineto{\pgfqpoint{4.483105in}{2.475668in}}%
\pgfpathlineto{\pgfqpoint{4.486260in}{2.473363in}}%
\pgfpathlineto{\pgfqpoint{4.492415in}{2.468866in}}%
\pgfpathlineto{\pgfqpoint{4.492492in}{2.468810in}}%
\pgfpathlineto{\pgfqpoint{4.498724in}{2.464257in}}%
\pgfpathlineto{\pgfqpoint{4.501724in}{2.462064in}}%
\pgfpathlineto{\pgfqpoint{4.504955in}{2.459703in}}%
\pgfpathlineto{\pgfqpoint{4.511033in}{2.455263in}}%
\pgfpathlineto{\pgfqpoint{4.511187in}{2.455150in}}%
\pgfpathlineto{\pgfqpoint{4.517419in}{2.450597in}}%
\pgfpathlineto{\pgfqpoint{4.520343in}{2.448461in}}%
\pgfpathlineto{\pgfqpoint{4.523651in}{2.446044in}}%
\pgfpathlineto{\pgfqpoint{4.529652in}{2.441659in}}%
\pgfpathlineto{\pgfqpoint{4.529883in}{2.441490in}}%
\pgfpathlineto{\pgfqpoint{4.536115in}{2.436937in}}%
\pgfpathlineto{\pgfqpoint{4.538961in}{2.434857in}}%
\pgfpathlineto{\pgfqpoint{4.542346in}{2.432384in}}%
\pgfpathlineto{\pgfqpoint{4.548271in}{2.428055in}}%
\pgfpathlineto{\pgfqpoint{4.548578in}{2.427831in}}%
\pgfpathlineto{\pgfqpoint{4.554810in}{2.423277in}}%
\pgfpathlineto{\pgfqpoint{4.557580in}{2.421254in}}%
\pgfpathlineto{\pgfqpoint{4.561042in}{2.418724in}}%
\pgfpathlineto{\pgfqpoint{4.566889in}{2.414452in}}%
\pgfpathlineto{\pgfqpoint{4.567274in}{2.414171in}}%
\pgfpathlineto{\pgfqpoint{4.573506in}{2.409617in}}%
\pgfpathlineto{\pgfqpoint{4.576199in}{2.407650in}}%
\pgfpathlineto{\pgfqpoint{4.579738in}{2.405064in}}%
\pgfpathlineto{\pgfqpoint{4.585508in}{2.400848in}}%
\pgfpathlineto{\pgfqpoint{4.585969in}{2.400511in}}%
\pgfpathlineto{\pgfqpoint{4.592201in}{2.395958in}}%
\pgfpathlineto{\pgfqpoint{4.594817in}{2.394046in}}%
\pgfpathlineto{\pgfqpoint{4.598433in}{2.391404in}}%
\pgfpathlineto{\pgfqpoint{4.604126in}{2.387245in}}%
\pgfpathlineto{\pgfqpoint{4.604665in}{2.386851in}}%
\pgfpathlineto{\pgfqpoint{4.610897in}{2.382298in}}%
\pgfpathlineto{\pgfqpoint{4.613436in}{2.380443in}}%
\pgfpathlineto{\pgfqpoint{4.617129in}{2.377745in}}%
\pgfpathlineto{\pgfqpoint{4.622745in}{2.373641in}}%
\pgfpathlineto{\pgfqpoint{4.623361in}{2.373191in}}%
\pgfpathlineto{\pgfqpoint{4.629592in}{2.368638in}}%
\pgfpathlineto{\pgfqpoint{4.632054in}{2.366839in}}%
\pgfpathlineto{\pgfqpoint{4.635824in}{2.364085in}}%
\pgfpathlineto{\pgfqpoint{4.641364in}{2.360037in}}%
\pgfpathlineto{\pgfqpoint{4.642056in}{2.359531in}}%
\pgfpathlineto{\pgfqpoint{4.648288in}{2.354978in}}%
\pgfpathlineto{\pgfqpoint{4.650673in}{2.353236in}}%
\pgfpathlineto{\pgfqpoint{4.654520in}{2.350425in}}%
\pgfpathlineto{\pgfqpoint{4.659982in}{2.346434in}}%
\pgfpathlineto{\pgfqpoint{4.660752in}{2.345872in}}%
\pgfpathlineto{\pgfqpoint{4.666984in}{2.341318in}}%
\pgfpathlineto{\pgfqpoint{4.669292in}{2.339632in}}%
\pgfpathlineto{\pgfqpoint{4.673215in}{2.336765in}}%
\pgfpathlineto{\pgfqpoint{4.678601in}{2.332830in}}%
\pgfpathlineto{\pgfqpoint{4.679447in}{2.332212in}}%
\pgfpathlineto{\pgfqpoint{4.685679in}{2.327659in}}%
\pgfpathlineto{\pgfqpoint{4.687910in}{2.326028in}}%
\pgfpathlineto{\pgfqpoint{4.691911in}{2.323105in}}%
\pgfpathlineto{\pgfqpoint{4.697220in}{2.319227in}}%
\pgfpathlineto{\pgfqpoint{4.698143in}{2.318552in}}%
\pgfpathlineto{\pgfqpoint{4.704375in}{2.313999in}}%
\pgfpathlineto{\pgfqpoint{4.706529in}{2.312425in}}%
\pgfpathlineto{\pgfqpoint{4.710606in}{2.309445in}}%
\pgfpathlineto{\pgfqpoint{4.715838in}{2.305623in}}%
\pgfpathlineto{\pgfqpoint{4.716838in}{2.304892in}}%
\pgfpathlineto{\pgfqpoint{4.723070in}{2.300339in}}%
\pgfpathlineto{\pgfqpoint{4.725147in}{2.298821in}}%
\pgfpathlineto{\pgfqpoint{4.729302in}{2.295786in}}%
\pgfpathlineto{\pgfqpoint{4.734457in}{2.292019in}}%
\pgfpathlineto{\pgfqpoint{4.735534in}{2.291232in}}%
\pgfpathlineto{\pgfqpoint{4.741766in}{2.286679in}}%
\pgfpathlineto{\pgfqpoint{4.743766in}{2.285218in}}%
\pgfpathlineto{\pgfqpoint{4.747998in}{2.282126in}}%
\pgfpathlineto{\pgfqpoint{4.753075in}{2.278416in}}%
\pgfpathlineto{\pgfqpoint{4.754229in}{2.277573in}}%
\pgfpathlineto{\pgfqpoint{4.760461in}{2.273019in}}%
\pgfpathlineto{\pgfqpoint{4.762385in}{2.271614in}}%
\pgfpathlineto{\pgfqpoint{4.766693in}{2.268466in}}%
\pgfpathlineto{\pgfqpoint{4.771694in}{2.264812in}}%
\pgfpathlineto{\pgfqpoint{4.772925in}{2.263913in}}%
\pgfpathlineto{\pgfqpoint{4.779157in}{2.259359in}}%
\pgfpathlineto{\pgfqpoint{4.781003in}{2.258010in}}%
\pgfpathlineto{\pgfqpoint{4.785389in}{2.254806in}}%
\pgfpathlineto{\pgfqpoint{4.790313in}{2.251209in}}%
\pgfpathlineto{\pgfqpoint{4.791621in}{2.250253in}}%
\pgfpathlineto{\pgfqpoint{4.797852in}{2.245700in}}%
\pgfpathlineto{\pgfqpoint{4.799622in}{2.244407in}}%
\pgfpathlineto{\pgfqpoint{4.804084in}{2.241146in}}%
\pgfpathlineto{\pgfqpoint{4.808931in}{2.237605in}}%
\pgfpathlineto{\pgfqpoint{4.810316in}{2.236593in}}%
\pgfpathlineto{\pgfqpoint{4.816548in}{2.232040in}}%
\pgfpathlineto{\pgfqpoint{4.818241in}{2.230803in}}%
\pgfpathlineto{\pgfqpoint{4.822780in}{2.227487in}}%
\pgfpathlineto{\pgfqpoint{4.827550in}{2.224001in}}%
\pgfpathlineto{\pgfqpoint{4.829012in}{2.222933in}}%
\pgfpathlineto{\pgfqpoint{4.835244in}{2.218380in}}%
\pgfpathlineto{\pgfqpoint{4.836859in}{2.217200in}}%
\pgfpathlineto{\pgfqpoint{4.841475in}{2.213827in}}%
\pgfpathlineto{\pgfqpoint{4.846168in}{2.210398in}}%
\pgfpathlineto{\pgfqpoint{4.847707in}{2.209273in}}%
\pgfpathlineto{\pgfqpoint{4.853939in}{2.204720in}}%
\pgfpathlineto{\pgfqpoint{4.855478in}{2.203596in}}%
\pgfpathlineto{\pgfqpoint{4.860171in}{2.200167in}}%
\pgfpathlineto{\pgfqpoint{4.864787in}{2.196794in}}%
\pgfpathlineto{\pgfqpoint{4.866403in}{2.195614in}}%
\pgfpathlineto{\pgfqpoint{4.872635in}{2.191060in}}%
\pgfpathlineto{\pgfqpoint{4.874096in}{2.189992in}}%
\pgfpathlineto{\pgfqpoint{4.878866in}{2.186507in}}%
\pgfpathlineto{\pgfqpoint{4.883406in}{2.183191in}}%
\pgfpathlineto{\pgfqpoint{4.885098in}{2.181954in}}%
\pgfpathlineto{\pgfqpoint{4.891330in}{2.177401in}}%
\pgfpathlineto{\pgfqpoint{4.892715in}{2.176389in}}%
\pgfpathlineto{\pgfqpoint{4.897562in}{2.172847in}}%
\pgfpathlineto{\pgfqpoint{4.902024in}{2.169587in}}%
\pgfpathlineto{\pgfqpoint{4.903794in}{2.168294in}}%
\pgfpathlineto{\pgfqpoint{4.910026in}{2.163741in}}%
\pgfpathlineto{\pgfqpoint{4.911334in}{2.162785in}}%
\pgfpathlineto{\pgfqpoint{4.916258in}{2.159188in}}%
\pgfpathlineto{\pgfqpoint{4.920643in}{2.155983in}}%
\pgfpathlineto{\pgfqpoint{4.922489in}{2.154634in}}%
\pgfpathlineto{\pgfqpoint{4.928721in}{2.150081in}}%
\pgfpathlineto{\pgfqpoint{4.929952in}{2.149182in}}%
\pgfpathlineto{\pgfqpoint{4.934953in}{2.145528in}}%
\pgfpathlineto{\pgfqpoint{4.939262in}{2.142380in}}%
\pgfpathlineto{\pgfqpoint{4.941185in}{2.140974in}}%
\pgfpathlineto{\pgfqpoint{4.947417in}{2.136421in}}%
\pgfpathlineto{\pgfqpoint{4.948571in}{2.135578in}}%
\pgfpathlineto{\pgfqpoint{4.953649in}{2.131868in}}%
\pgfpathlineto{\pgfqpoint{4.957880in}{2.128776in}}%
\pgfpathlineto{\pgfqpoint{4.959881in}{2.127315in}}%
\pgfpathlineto{\pgfqpoint{4.966112in}{2.122761in}}%
\pgfpathlineto{\pgfqpoint{4.967189in}{2.121974in}}%
\pgfpathlineto{\pgfqpoint{4.972344in}{2.118208in}}%
\pgfpathlineto{\pgfqpoint{4.976499in}{2.115173in}}%
\pgfpathlineto{\pgfqpoint{4.978576in}{2.113655in}}%
\pgfpathlineto{\pgfqpoint{4.984808in}{2.109102in}}%
\pgfpathlineto{\pgfqpoint{4.985808in}{2.108371in}}%
\pgfpathlineto{\pgfqpoint{4.991040in}{2.104548in}}%
\pgfpathlineto{\pgfqpoint{4.995117in}{2.101569in}}%
\pgfpathlineto{\pgfqpoint{4.997272in}{2.099995in}}%
\pgfpathlineto{\pgfqpoint{5.003503in}{2.095442in}}%
\pgfpathlineto{\pgfqpoint{5.004427in}{2.094767in}}%
\pgfpathlineto{\pgfqpoint{5.009735in}{2.090888in}}%
\pgfpathlineto{\pgfqpoint{5.013736in}{2.087965in}}%
\pgfpathlineto{\pgfqpoint{5.015967in}{2.086335in}}%
\pgfpathlineto{\pgfqpoint{5.022199in}{2.081782in}}%
\pgfpathlineto{\pgfqpoint{5.023045in}{2.081164in}}%
\pgfpathlineto{\pgfqpoint{5.028431in}{2.077229in}}%
\pgfpathlineto{\pgfqpoint{5.032355in}{2.074362in}}%
\pgfpathlineto{\pgfqpoint{5.034663in}{2.072675in}}%
\pgfpathlineto{\pgfqpoint{5.040895in}{2.068122in}}%
\pgfpathlineto{\pgfqpoint{5.041664in}{2.067560in}}%
\pgfpathlineto{\pgfqpoint{5.047126in}{2.063569in}}%
\pgfpathlineto{\pgfqpoint{5.050973in}{2.060758in}}%
\pgfpathlineto{\pgfqpoint{5.053358in}{2.059016in}}%
\pgfpathlineto{\pgfqpoint{5.059590in}{2.054462in}}%
\pgfpathlineto{\pgfqpoint{5.060283in}{2.053956in}}%
\pgfpathlineto{\pgfqpoint{5.065822in}{2.049909in}}%
\pgfpathlineto{\pgfqpoint{5.069592in}{2.047155in}}%
\pgfpathlineto{\pgfqpoint{5.072054in}{2.045356in}}%
\pgfpathlineto{\pgfqpoint{5.078286in}{2.040802in}}%
\pgfpathlineto{\pgfqpoint{5.078901in}{2.040353in}}%
\pgfpathlineto{\pgfqpoint{5.084518in}{2.036249in}}%
\pgfpathlineto{\pgfqpoint{5.088211in}{2.033551in}}%
\pgfpathlineto{\pgfqpoint{5.090749in}{2.031696in}}%
\pgfpathlineto{\pgfqpoint{5.096981in}{2.027143in}}%
\pgfpathlineto{\pgfqpoint{5.097520in}{2.026749in}}%
\pgfpathlineto{\pgfqpoint{5.103213in}{2.022589in}}%
\pgfpathlineto{\pgfqpoint{5.106829in}{2.019947in}}%
\pgfpathlineto{\pgfqpoint{5.109445in}{2.018036in}}%
\pgfpathlineto{\pgfqpoint{5.115677in}{2.013483in}}%
\pgfpathlineto{\pgfqpoint{5.116138in}{2.013146in}}%
\pgfpathlineto{\pgfqpoint{5.121909in}{2.008930in}}%
\pgfpathlineto{\pgfqpoint{5.125448in}{2.006344in}}%
\pgfpathlineto{\pgfqpoint{5.128141in}{2.004376in}}%
\pgfpathlineto{\pgfqpoint{5.134372in}{1.999823in}}%
\pgfpathlineto{\pgfqpoint{5.134757in}{1.999542in}}%
\pgfpathlineto{\pgfqpoint{5.140604in}{1.995270in}}%
\pgfpathlineto{\pgfqpoint{5.144066in}{1.992740in}}%
\pgfpathlineto{\pgfqpoint{5.146836in}{1.990716in}}%
\pgfpathlineto{\pgfqpoint{5.153068in}{1.986163in}}%
\pgfpathlineto{\pgfqpoint{5.153376in}{1.985938in}}%
\pgfpathlineto{\pgfqpoint{5.159300in}{1.981610in}}%
\pgfpathlineto{\pgfqpoint{5.162685in}{1.979137in}}%
\pgfpathlineto{\pgfqpoint{5.165532in}{1.977057in}}%
\pgfpathlineto{\pgfqpoint{5.171763in}{1.972503in}}%
\pgfpathlineto{\pgfqpoint{5.171994in}{1.972335in}}%
\pgfpathlineto{\pgfqpoint{5.177995in}{1.967950in}}%
\pgfpathlineto{\pgfqpoint{5.181304in}{1.965533in}}%
\pgfpathlineto{\pgfqpoint{5.184227in}{1.963397in}}%
\pgfpathlineto{\pgfqpoint{5.190459in}{1.958844in}}%
\pgfpathlineto{\pgfqpoint{5.190613in}{1.958731in}}%
\pgfpathlineto{\pgfqpoint{5.196691in}{1.954290in}}%
\pgfpathlineto{\pgfqpoint{5.199922in}{1.951929in}}%
\pgfpathlineto{\pgfqpoint{5.202923in}{1.949737in}}%
\pgfpathlineto{\pgfqpoint{5.209155in}{1.945184in}}%
\pgfpathlineto{\pgfqpoint{5.209232in}{1.945127in}}%
\pgfpathlineto{\pgfqpoint{5.215386in}{1.940630in}}%
\pgfpathlineto{\pgfqpoint{5.218541in}{1.938326in}}%
\pgfpathlineto{\pgfqpoint{5.221618in}{1.936077in}}%
\pgfpathlineto{\pgfqpoint{5.227850in}{1.931524in}}%
\pgfpathlineto{\pgfqpoint{5.227850in}{1.931524in}}%
\pgfpathlineto{\pgfqpoint{5.234082in}{1.926971in}}%
\pgfpathlineto{\pgfqpoint{5.237159in}{1.924722in}}%
\pgfpathlineto{\pgfqpoint{5.240314in}{1.922417in}}%
\pgfpathlineto{\pgfqpoint{5.246469in}{1.917920in}}%
\pgfpathlineto{\pgfqpoint{5.246546in}{1.917864in}}%
\pgfpathlineto{\pgfqpoint{5.252778in}{1.913311in}}%
\pgfpathlineto{\pgfqpoint{5.255778in}{1.911118in}}%
\pgfpathlineto{\pgfqpoint{5.259009in}{1.908758in}}%
\pgfpathlineto{\pgfqpoint{5.265087in}{1.904317in}}%
\pgfpathlineto{\pgfqpoint{5.265241in}{1.904204in}}%
\pgfpathlineto{\pgfqpoint{5.271473in}{1.899651in}}%
\pgfpathlineto{\pgfqpoint{5.274397in}{1.897515in}}%
\pgfpathlineto{\pgfqpoint{5.277705in}{1.895098in}}%
\pgfpathlineto{\pgfqpoint{5.283706in}{1.890713in}}%
\pgfpathlineto{\pgfqpoint{5.283937in}{1.890544in}}%
\pgfpathlineto{\pgfqpoint{5.290169in}{1.885991in}}%
\pgfpathlineto{\pgfqpoint{5.293015in}{1.883911in}}%
\pgfpathlineto{\pgfqpoint{5.296401in}{1.881438in}}%
\pgfpathlineto{\pgfqpoint{5.302325in}{1.877109in}}%
\pgfpathlineto{\pgfqpoint{5.302632in}{1.876885in}}%
\pgfpathlineto{\pgfqpoint{5.308864in}{1.872331in}}%
\pgfpathlineto{\pgfqpoint{5.311634in}{1.870308in}}%
\pgfpathlineto{\pgfqpoint{5.315096in}{1.867778in}}%
\pgfpathlineto{\pgfqpoint{5.320943in}{1.863506in}}%
\pgfpathlineto{\pgfqpoint{5.321328in}{1.863225in}}%
\pgfpathlineto{\pgfqpoint{5.327560in}{1.858672in}}%
\pgfpathlineto{\pgfqpoint{5.330253in}{1.856704in}}%
\pgfpathlineto{\pgfqpoint{5.333792in}{1.854118in}}%
\pgfpathlineto{\pgfqpoint{5.339562in}{1.849902in}}%
\pgfpathlineto{\pgfqpoint{5.340023in}{1.849565in}}%
\pgfpathlineto{\pgfqpoint{5.346255in}{1.845012in}}%
\pgfpathlineto{\pgfqpoint{5.348871in}{1.843100in}}%
\pgfpathlineto{\pgfqpoint{5.352487in}{1.840458in}}%
\pgfpathlineto{\pgfqpoint{5.358180in}{1.836299in}}%
\pgfpathlineto{\pgfqpoint{5.358719in}{1.835905in}}%
\pgfpathlineto{\pgfqpoint{5.364951in}{1.831352in}}%
\pgfpathlineto{\pgfqpoint{5.367490in}{1.829497in}}%
\pgfpathlineto{\pgfqpoint{5.371183in}{1.826799in}}%
\pgfpathlineto{\pgfqpoint{5.376799in}{1.822695in}}%
\pgfpathlineto{\pgfqpoint{5.377415in}{1.822245in}}%
\pgfpathlineto{\pgfqpoint{5.383646in}{1.817692in}}%
\pgfpathlineto{\pgfqpoint{5.386108in}{1.815893in}}%
\pgfpathlineto{\pgfqpoint{5.389878in}{1.813139in}}%
\pgfpathlineto{\pgfqpoint{5.395418in}{1.809091in}}%
\pgfpathlineto{\pgfqpoint{5.396110in}{1.808586in}}%
\pgfpathlineto{\pgfqpoint{5.402342in}{1.804032in}}%
\pgfpathlineto{\pgfqpoint{5.404727in}{1.802290in}}%
\pgfpathlineto{\pgfqpoint{5.408574in}{1.799479in}}%
\pgfpathlineto{\pgfqpoint{5.414036in}{1.795488in}}%
\pgfpathlineto{\pgfqpoint{5.414806in}{1.794926in}}%
\pgfpathlineto{\pgfqpoint{5.421038in}{1.790372in}}%
\pgfpathlineto{\pgfqpoint{5.423346in}{1.788686in}}%
\pgfpathlineto{\pgfqpoint{5.427269in}{1.785819in}}%
\pgfpathlineto{\pgfqpoint{5.432655in}{1.781884in}}%
\pgfpathlineto{\pgfqpoint{5.433501in}{1.781266in}}%
\pgfpathlineto{\pgfqpoint{5.439733in}{1.776713in}}%
\pgfpathlineto{\pgfqpoint{5.441964in}{1.775082in}}%
\pgfpathlineto{\pgfqpoint{5.445965in}{1.772159in}}%
\pgfpathlineto{\pgfqpoint{5.451274in}{1.768281in}}%
\pgfpathlineto{\pgfqpoint{5.452197in}{1.767606in}}%
\pgfpathlineto{\pgfqpoint{5.458429in}{1.763053in}}%
\pgfpathlineto{\pgfqpoint{5.460583in}{1.761479in}}%
\pgfpathlineto{\pgfqpoint{5.464661in}{1.758500in}}%
\pgfpathlineto{\pgfqpoint{5.469892in}{1.754677in}}%
\pgfpathlineto{\pgfqpoint{5.470892in}{1.753946in}}%
\pgfpathlineto{\pgfqpoint{5.477124in}{1.749393in}}%
\pgfpathlineto{\pgfqpoint{5.479202in}{1.747875in}}%
\pgfpathlineto{\pgfqpoint{5.483356in}{1.744840in}}%
\pgfpathlineto{\pgfqpoint{5.488511in}{1.741073in}}%
\pgfpathlineto{\pgfqpoint{5.489588in}{1.740286in}}%
\pgfpathlineto{\pgfqpoint{5.495820in}{1.735733in}}%
\pgfpathlineto{\pgfqpoint{5.497820in}{1.734272in}}%
\pgfpathlineto{\pgfqpoint{5.502052in}{1.731180in}}%
\pgfpathlineto{\pgfqpoint{5.507129in}{1.727470in}}%
\pgfpathlineto{\pgfqpoint{5.508283in}{1.726627in}}%
\pgfpathlineto{\pgfqpoint{5.514515in}{1.722073in}}%
\pgfpathlineto{\pgfqpoint{5.516439in}{1.720668in}}%
\pgfpathlineto{\pgfqpoint{5.520747in}{1.717520in}}%
\pgfpathlineto{\pgfqpoint{5.525748in}{1.713866in}}%
\pgfpathlineto{\pgfqpoint{5.526979in}{1.712967in}}%
\pgfpathlineto{\pgfqpoint{5.533211in}{1.708414in}}%
\pgfpathlineto{\pgfqpoint{5.535057in}{1.707064in}}%
\pgfpathlineto{\pgfqpoint{5.539443in}{1.703860in}}%
\pgfpathlineto{\pgfqpoint{5.544367in}{1.700263in}}%
\pgfpathlineto{\pgfqpoint{5.545675in}{1.699307in}}%
\pgfpathlineto{\pgfqpoint{5.551906in}{1.694754in}}%
\pgfpathlineto{\pgfqpoint{5.553676in}{1.693461in}}%
\pgfpathlineto{\pgfqpoint{5.558138in}{1.690200in}}%
\pgfpathlineto{\pgfqpoint{5.562985in}{1.686659in}}%
\pgfpathlineto{\pgfqpoint{5.564370in}{1.685647in}}%
\pgfpathlineto{\pgfqpoint{5.570602in}{1.681094in}}%
\pgfpathlineto{\pgfqpoint{5.572295in}{1.679857in}}%
\pgfpathlineto{\pgfqpoint{5.576834in}{1.676541in}}%
\pgfpathlineto{\pgfqpoint{5.581604in}{1.673055in}}%
\pgfpathlineto{\pgfqpoint{5.583066in}{1.671987in}}%
\pgfpathlineto{\pgfqpoint{5.589298in}{1.667434in}}%
\pgfpathlineto{\pgfqpoint{5.590913in}{1.666254in}}%
\pgfpathlineto{\pgfqpoint{5.595529in}{1.662881in}}%
\pgfpathlineto{\pgfqpoint{5.600223in}{1.659452in}}%
\pgfpathlineto{\pgfqpoint{5.601761in}{1.658328in}}%
\pgfpathlineto{\pgfqpoint{5.607993in}{1.653774in}}%
\pgfpathlineto{\pgfqpoint{5.609532in}{1.652650in}}%
\pgfpathlineto{\pgfqpoint{5.614225in}{1.649221in}}%
\pgfpathlineto{\pgfqpoint{5.618841in}{1.645848in}}%
\pgfpathlineto{\pgfqpoint{5.620457in}{1.644668in}}%
\pgfpathlineto{\pgfqpoint{5.626689in}{1.640114in}}%
\pgfpathlineto{\pgfqpoint{5.628150in}{1.639046in}}%
\pgfpathlineto{\pgfqpoint{5.632921in}{1.635561in}}%
\pgfpathlineto{\pgfqpoint{5.637460in}{1.632245in}}%
\pgfpathlineto{\pgfqpoint{5.639152in}{1.631008in}}%
\pgfpathlineto{\pgfqpoint{5.645384in}{1.626455in}}%
\pgfpathlineto{\pgfqpoint{5.646769in}{1.625443in}}%
\pgfpathlineto{\pgfqpoint{5.651616in}{1.621901in}}%
\pgfpathlineto{\pgfqpoint{5.656078in}{1.618641in}}%
\pgfpathlineto{\pgfqpoint{5.657848in}{1.617348in}}%
\pgfpathlineto{\pgfqpoint{5.664080in}{1.612795in}}%
\pgfpathlineto{\pgfqpoint{5.665388in}{1.611839in}}%
\pgfpathlineto{\pgfqpoint{5.670312in}{1.608242in}}%
\pgfpathlineto{\pgfqpoint{5.674697in}{1.605037in}}%
\pgfpathlineto{\pgfqpoint{5.676543in}{1.603688in}}%
\pgfpathlineto{\pgfqpoint{5.682775in}{1.599135in}}%
\pgfpathlineto{\pgfqpoint{5.684006in}{1.598236in}}%
\pgfpathlineto{\pgfqpoint{5.689007in}{1.594582in}}%
\pgfpathlineto{\pgfqpoint{5.693316in}{1.591434in}}%
\pgfpathlineto{\pgfqpoint{5.695239in}{1.590028in}}%
\pgfpathlineto{\pgfqpoint{5.701471in}{1.585475in}}%
\pgfpathlineto{\pgfqpoint{5.702625in}{1.584632in}}%
\pgfpathlineto{\pgfqpoint{5.707703in}{1.580922in}}%
\pgfpathlineto{\pgfqpoint{5.711934in}{1.577830in}}%
\pgfpathlineto{\pgfqpoint{5.713935in}{1.576369in}}%
\pgfpathlineto{\pgfqpoint{5.720166in}{1.571815in}}%
\pgfpathlineto{\pgfqpoint{5.721244in}{1.571028in}}%
\pgfpathlineto{\pgfqpoint{5.726398in}{1.567262in}}%
\pgfpathlineto{\pgfqpoint{5.730553in}{1.564227in}}%
\pgfpathlineto{\pgfqpoint{5.732630in}{1.562709in}}%
\pgfpathlineto{\pgfqpoint{5.738862in}{1.558156in}}%
\pgfpathlineto{\pgfqpoint{5.739862in}{1.557425in}}%
\pgfpathlineto{\pgfqpoint{5.745094in}{1.553602in}}%
\pgfpathlineto{\pgfqpoint{5.749171in}{1.550623in}}%
\pgfpathlineto{\pgfqpoint{5.751326in}{1.549049in}}%
\pgfpathlineto{\pgfqpoint{5.757558in}{1.544496in}}%
\pgfpathlineto{\pgfqpoint{5.758481in}{1.543821in}}%
\pgfpathlineto{\pgfqpoint{5.763789in}{1.539942in}}%
\pgfpathlineto{\pgfqpoint{5.767790in}{1.537019in}}%
\pgfpathlineto{\pgfqpoint{5.770021in}{1.535389in}}%
\pgfpathlineto{\pgfqpoint{5.776253in}{1.530836in}}%
\pgfpathlineto{\pgfqpoint{5.777099in}{1.530218in}}%
\pgfpathlineto{\pgfqpoint{5.782485in}{1.526283in}}%
\pgfpathlineto{\pgfqpoint{5.786409in}{1.523416in}}%
\pgfpathlineto{\pgfqpoint{5.788717in}{1.521729in}}%
\pgfpathlineto{\pgfqpoint{5.794949in}{1.517176in}}%
\pgfpathlineto{\pgfqpoint{5.795718in}{1.516614in}}%
\pgfpathlineto{\pgfqpoint{5.801181in}{1.512623in}}%
\pgfpathlineto{\pgfqpoint{5.805027in}{1.509812in}}%
\pgfpathlineto{\pgfqpoint{5.807412in}{1.508070in}}%
\pgfpathlineto{\pgfqpoint{5.813644in}{1.503516in}}%
\pgfpathlineto{\pgfqpoint{5.814337in}{1.503010in}}%
\pgfpathlineto{\pgfqpoint{5.819876in}{1.498963in}}%
\pgfpathlineto{\pgfqpoint{5.823646in}{1.496209in}}%
\pgfpathlineto{\pgfqpoint{5.826108in}{1.494410in}}%
\pgfpathlineto{\pgfqpoint{5.832340in}{1.489856in}}%
\pgfpathlineto{\pgfqpoint{5.832955in}{1.489407in}}%
\pgfpathlineto{\pgfqpoint{5.838572in}{1.485303in}}%
\pgfpathlineto{\pgfqpoint{5.842265in}{1.482605in}}%
\pgfpathlineto{\pgfqpoint{5.844803in}{1.480750in}}%
\pgfpathlineto{\pgfqpoint{5.851035in}{1.476197in}}%
\pgfpathlineto{\pgfqpoint{5.851574in}{1.475803in}}%
\pgfpathlineto{\pgfqpoint{5.857267in}{1.471643in}}%
\pgfpathlineto{\pgfqpoint{5.860883in}{1.469001in}}%
\pgfpathlineto{\pgfqpoint{5.863499in}{1.467090in}}%
\pgfpathlineto{\pgfqpoint{5.869731in}{1.462537in}}%
\pgfpathlineto{\pgfqpoint{5.870192in}{1.462200in}}%
\pgfpathlineto{\pgfqpoint{5.875963in}{1.457984in}}%
\pgfpathlineto{\pgfqpoint{5.879502in}{1.455398in}}%
\pgfpathlineto{\pgfqpoint{5.882195in}{1.453430in}}%
\pgfpathlineto{\pgfqpoint{5.888426in}{1.448877in}}%
\pgfpathlineto{\pgfqpoint{5.888811in}{1.448596in}}%
\pgfpathlineto{\pgfqpoint{5.894658in}{1.444324in}}%
\pgfpathlineto{\pgfqpoint{5.898120in}{1.441794in}}%
\pgfpathlineto{\pgfqpoint{5.900890in}{1.439770in}}%
\pgfpathlineto{\pgfqpoint{5.907122in}{1.435217in}}%
\pgfpathlineto{\pgfqpoint{5.907430in}{1.434992in}}%
\pgfpathlineto{\pgfqpoint{5.913354in}{1.430664in}}%
\pgfpathlineto{\pgfqpoint{5.916739in}{1.428191in}}%
\pgfpathlineto{\pgfqpoint{5.919586in}{1.426111in}}%
\pgfpathlineto{\pgfqpoint{5.925818in}{1.421557in}}%
\pgfpathlineto{\pgfqpoint{5.926048in}{1.421389in}}%
\pgfpathlineto{\pgfqpoint{5.932049in}{1.417004in}}%
\pgfpathlineto{\pgfqpoint{5.935358in}{1.414587in}}%
\pgfpathlineto{\pgfqpoint{5.938281in}{1.412451in}}%
\pgfpathlineto{\pgfqpoint{5.944513in}{1.407898in}}%
\pgfpathlineto{\pgfqpoint{5.944667in}{1.407785in}}%
\pgfpathlineto{\pgfqpoint{5.950745in}{1.403344in}}%
\pgfpathlineto{\pgfqpoint{5.953976in}{1.400983in}}%
\pgfpathlineto{\pgfqpoint{5.956977in}{1.398791in}}%
\pgfpathlineto{\pgfqpoint{5.963209in}{1.394238in}}%
\pgfpathlineto{\pgfqpoint{5.963286in}{1.394182in}}%
\pgfpathlineto{\pgfqpoint{5.969441in}{1.389684in}}%
\pgfpathlineto{\pgfqpoint{5.972595in}{1.387380in}}%
\pgfpathlineto{\pgfqpoint{5.975672in}{1.385131in}}%
\pgfpathlineto{\pgfqpoint{5.981904in}{1.380578in}}%
\pgfpathlineto{\pgfqpoint{5.981904in}{1.380578in}}%
\pgfpathlineto{\pgfqpoint{5.988136in}{1.376025in}}%
\pgfpathlineto{\pgfqpoint{5.991214in}{1.373776in}}%
\pgfpathlineto{\pgfqpoint{5.994368in}{1.371471in}}%
\pgfpathlineto{\pgfqpoint{6.000523in}{1.366974in}}%
\pgfpathlineto{\pgfqpoint{6.000600in}{1.366918in}}%
\pgfpathlineto{\pgfqpoint{6.006832in}{1.362365in}}%
\pgfpathlineto{\pgfqpoint{6.009832in}{1.360173in}}%
\pgfpathlineto{\pgfqpoint{6.013063in}{1.357812in}}%
\pgfpathlineto{\pgfqpoint{6.019141in}{1.353371in}}%
\pgfpathlineto{\pgfqpoint{6.019295in}{1.353258in}}%
\pgfpathlineto{\pgfqpoint{6.025527in}{1.348705in}}%
\pgfpathlineto{\pgfqpoint{6.028451in}{1.346569in}}%
\pgfpathlineto{\pgfqpoint{6.031759in}{1.344152in}}%
\pgfpathlineto{\pgfqpoint{6.037760in}{1.339767in}}%
\pgfpathlineto{\pgfqpoint{6.037991in}{1.339598in}}%
\pgfpathlineto{\pgfqpoint{6.044223in}{1.335045in}}%
\pgfpathlineto{\pgfqpoint{6.047069in}{1.332965in}}%
\pgfpathlineto{\pgfqpoint{6.050455in}{1.330492in}}%
\pgfpathlineto{\pgfqpoint{6.056379in}{1.326164in}}%
\pgfpathlineto{\pgfqpoint{6.056686in}{1.325939in}}%
\pgfpathlineto{\pgfqpoint{6.062918in}{1.321385in}}%
\pgfpathlineto{\pgfqpoint{6.065688in}{1.319362in}}%
\pgfpathlineto{\pgfqpoint{6.069150in}{1.316832in}}%
\pgfpathlineto{\pgfqpoint{6.074997in}{1.312560in}}%
\pgfpathlineto{\pgfqpoint{6.075382in}{1.312279in}}%
\pgfpathlineto{\pgfqpoint{6.081614in}{1.307726in}}%
\pgfpathlineto{\pgfqpoint{6.084307in}{1.305758in}}%
\pgfpathlineto{\pgfqpoint{6.087846in}{1.303172in}}%
\pgfpathlineto{\pgfqpoint{6.093616in}{1.298956in}}%
\pgfpathlineto{\pgfqpoint{6.094078in}{1.298619in}}%
\pgfpathlineto{\pgfqpoint{6.100309in}{1.294066in}}%
\pgfpathlineto{\pgfqpoint{6.102925in}{1.292155in}}%
\pgfpathlineto{\pgfqpoint{6.106541in}{1.289512in}}%
\pgfpathlineto{\pgfqpoint{6.112235in}{1.285353in}}%
\pgfpathlineto{\pgfqpoint{6.112773in}{1.284959in}}%
\pgfpathlineto{\pgfqpoint{6.119005in}{1.280406in}}%
\pgfpathlineto{\pgfqpoint{6.121544in}{1.278551in}}%
\pgfpathlineto{\pgfqpoint{6.125237in}{1.275853in}}%
\pgfpathlineto{\pgfqpoint{6.130853in}{1.271749in}}%
\pgfpathlineto{\pgfqpoint{6.131469in}{1.271299in}}%
\pgfpathlineto{\pgfqpoint{6.137701in}{1.266746in}}%
\pgfpathlineto{\pgfqpoint{6.140162in}{1.264947in}}%
\pgfpathlineto{\pgfqpoint{6.143932in}{1.262193in}}%
\pgfpathlineto{\pgfqpoint{6.149472in}{1.258146in}}%
\pgfpathlineto{\pgfqpoint{6.150164in}{1.257640in}}%
\pgfpathlineto{\pgfqpoint{6.156396in}{1.253086in}}%
\pgfpathlineto{\pgfqpoint{6.158781in}{1.251344in}}%
\pgfpathlineto{\pgfqpoint{6.162628in}{1.248533in}}%
\pgfpathlineto{\pgfqpoint{6.168090in}{1.244542in}}%
\pgfpathlineto{\pgfqpoint{6.168860in}{1.243980in}}%
\pgfpathlineto{\pgfqpoint{6.175092in}{1.239427in}}%
\pgfpathlineto{\pgfqpoint{6.177400in}{1.237740in}}%
\pgfpathlineto{\pgfqpoint{6.181323in}{1.234873in}}%
\pgfpathlineto{\pgfqpoint{6.186709in}{1.230938in}}%
\pgfpathlineto{\pgfqpoint{6.187555in}{1.230320in}}%
\pgfpathlineto{\pgfqpoint{6.193787in}{1.225767in}}%
\pgfpathlineto{\pgfqpoint{6.196018in}{1.224137in}}%
\pgfpathlineto{\pgfqpoint{6.200019in}{1.221213in}}%
\pgfpathlineto{\pgfqpoint{6.205328in}{1.217335in}}%
\pgfpathlineto{\pgfqpoint{6.206251in}{1.216660in}}%
\pgfpathlineto{\pgfqpoint{6.212483in}{1.212107in}}%
\pgfpathlineto{\pgfqpoint{6.214637in}{1.210533in}}%
\pgfpathlineto{\pgfqpoint{6.218715in}{1.207554in}}%
\pgfpathlineto{\pgfqpoint{6.223946in}{1.203731in}}%
\pgfpathlineto{\pgfqpoint{6.224946in}{1.203000in}}%
\pgfpathlineto{\pgfqpoint{6.231178in}{1.198447in}}%
\pgfpathlineto{\pgfqpoint{6.233256in}{1.196929in}}%
\pgfpathlineto{\pgfqpoint{6.237410in}{1.193894in}}%
\pgfpathlineto{\pgfqpoint{6.242565in}{1.190127in}}%
\pgfpathlineto{\pgfqpoint{6.243642in}{1.189341in}}%
\pgfpathlineto{\pgfqpoint{6.249874in}{1.184787in}}%
\pgfpathlineto{\pgfqpoint{6.251874in}{1.183326in}}%
\pgfpathlineto{\pgfqpoint{6.256106in}{1.180234in}}%
\pgfpathlineto{\pgfqpoint{6.261183in}{1.176524in}}%
\pgfpathlineto{\pgfqpoint{6.262338in}{1.175681in}}%
\pgfpathlineto{\pgfqpoint{6.268569in}{1.171127in}}%
\pgfpathlineto{\pgfqpoint{6.270493in}{1.169722in}}%
\pgfpathlineto{\pgfqpoint{6.274801in}{1.166574in}}%
\pgfpathlineto{\pgfqpoint{6.279802in}{1.162920in}}%
\pgfpathlineto{\pgfqpoint{6.281033in}{1.162021in}}%
\pgfpathlineto{\pgfqpoint{6.287265in}{1.157468in}}%
\pgfpathlineto{\pgfqpoint{6.289111in}{1.156118in}}%
\pgfpathlineto{\pgfqpoint{6.293497in}{1.152914in}}%
\pgfpathlineto{\pgfqpoint{6.298421in}{1.149317in}}%
\pgfpathlineto{\pgfqpoint{6.299729in}{1.148361in}}%
\pgfpathlineto{\pgfqpoint{6.305960in}{1.143808in}}%
\pgfpathlineto{\pgfqpoint{6.307730in}{1.142515in}}%
\pgfpathlineto{\pgfqpoint{6.312192in}{1.139255in}}%
\pgfpathlineto{\pgfqpoint{6.317039in}{1.135713in}}%
\pgfpathlineto{\pgfqpoint{6.318424in}{1.134701in}}%
\pgfpathlineto{\pgfqpoint{6.324656in}{1.130148in}}%
\pgfpathlineto{\pgfqpoint{6.326349in}{1.128911in}}%
\pgfpathlineto{\pgfqpoint{6.330888in}{1.125595in}}%
\pgfpathlineto{\pgfqpoint{6.335658in}{1.122109in}}%
\pgfpathlineto{\pgfqpoint{6.337120in}{1.121041in}}%
\pgfpathlineto{\pgfqpoint{6.343352in}{1.116488in}}%
\pgfpathlineto{\pgfqpoint{6.344967in}{1.115308in}}%
\pgfpathlineto{\pgfqpoint{6.349583in}{1.111935in}}%
\pgfpathlineto{\pgfqpoint{6.354277in}{1.108506in}}%
\pgfpathlineto{\pgfqpoint{6.355815in}{1.107382in}}%
\pgfpathlineto{\pgfqpoint{6.362047in}{1.102828in}}%
\pgfpathlineto{\pgfqpoint{6.363586in}{1.101704in}}%
\pgfpathlineto{\pgfqpoint{6.368279in}{1.098275in}}%
\pgfpathlineto{\pgfqpoint{6.372895in}{1.094902in}}%
\pgfpathlineto{\pgfqpoint{6.374511in}{1.093722in}}%
\pgfpathlineto{\pgfqpoint{6.380743in}{1.089169in}}%
\pgfpathlineto{\pgfqpoint{6.382205in}{1.088100in}}%
\pgfpathlineto{\pgfqpoint{6.386975in}{1.084615in}}%
\pgfpathlineto{\pgfqpoint{6.391514in}{1.081299in}}%
\pgfpathlineto{\pgfqpoint{6.393206in}{1.080062in}}%
\pgfpathlineto{\pgfqpoint{6.399438in}{1.075509in}}%
\pgfpathlineto{\pgfqpoint{6.400823in}{1.074497in}}%
\pgfpathlineto{\pgfqpoint{6.405670in}{1.070955in}}%
\pgfpathlineto{\pgfqpoint{6.410132in}{1.067695in}}%
\pgfpathlineto{\pgfqpoint{6.411902in}{1.066402in}}%
\pgfpathlineto{\pgfqpoint{6.418134in}{1.061849in}}%
\pgfpathlineto{\pgfqpoint{6.419442in}{1.060893in}}%
\pgfpathlineto{\pgfqpoint{6.424366in}{1.057296in}}%
\pgfpathlineto{\pgfqpoint{6.428751in}{1.054091in}}%
\pgfpathlineto{\pgfqpoint{6.430598in}{1.052742in}}%
\pgfpathlineto{\pgfqpoint{6.436829in}{1.048189in}}%
\pgfpathlineto{\pgfqpoint{6.438060in}{1.047290in}}%
\pgfpathlineto{\pgfqpoint{6.443061in}{1.043636in}}%
\pgfpathlineto{\pgfqpoint{6.447370in}{1.040488in}}%
\pgfpathlineto{\pgfqpoint{6.449293in}{1.039083in}}%
\pgfpathlineto{\pgfqpoint{6.455525in}{1.034529in}}%
\pgfpathlineto{\pgfqpoint{6.456679in}{1.033686in}}%
\pgfpathlineto{\pgfqpoint{6.461757in}{1.029976in}}%
\pgfpathlineto{\pgfqpoint{6.465988in}{1.026884in}}%
\pgfpathlineto{\pgfqpoint{6.467989in}{1.025423in}}%
\pgfpathlineto{\pgfqpoint{6.474220in}{1.020869in}}%
\pgfpathlineto{\pgfqpoint{6.475298in}{1.020082in}}%
\pgfpathlineto{\pgfqpoint{6.480452in}{1.016316in}}%
\pgfpathlineto{\pgfqpoint{6.484607in}{1.013281in}}%
\pgfpathlineto{\pgfqpoint{6.486684in}{1.011763in}}%
\pgfpathlineto{\pgfqpoint{6.492916in}{1.007210in}}%
\pgfpathlineto{\pgfqpoint{6.493916in}{1.006479in}}%
\pgfpathlineto{\pgfqpoint{6.499148in}{1.002656in}}%
\pgfpathlineto{\pgfqpoint{6.503226in}{0.999677in}}%
\pgfpathlineto{\pgfqpoint{6.505380in}{0.998103in}}%
\pgfpathlineto{\pgfqpoint{6.511612in}{0.993550in}}%
\pgfpathlineto{\pgfqpoint{6.512535in}{0.992875in}}%
\pgfpathlineto{\pgfqpoint{6.517843in}{0.988997in}}%
\pgfpathlineto{\pgfqpoint{6.521844in}{0.986073in}}%
\pgfpathlineto{\pgfqpoint{6.524075in}{0.984443in}}%
\pgfpathlineto{\pgfqpoint{6.530307in}{0.979890in}}%
\pgfpathlineto{\pgfqpoint{6.531153in}{0.979272in}}%
\pgfpathlineto{\pgfqpoint{6.536539in}{0.975337in}}%
\pgfpathlineto{\pgfqpoint{6.540463in}{0.972470in}}%
\pgfpathlineto{\pgfqpoint{6.542771in}{0.970783in}}%
\pgfpathlineto{\pgfqpoint{6.549003in}{0.966230in}}%
\pgfpathlineto{\pgfqpoint{6.549772in}{0.965668in}}%
\pgfpathlineto{\pgfqpoint{6.555235in}{0.961677in}}%
\pgfpathlineto{\pgfqpoint{6.559081in}{0.958866in}}%
\pgfpathlineto{\pgfqpoint{6.561466in}{0.957124in}}%
\pgfpathlineto{\pgfqpoint{6.567698in}{0.952570in}}%
\pgfpathlineto{\pgfqpoint{6.568391in}{0.952064in}}%
\pgfpathlineto{\pgfqpoint{6.573930in}{0.948017in}}%
\pgfpathlineto{\pgfqpoint{6.577700in}{0.945263in}}%
\pgfpathlineto{\pgfqpoint{6.580162in}{0.943464in}}%
\pgfpathlineto{\pgfqpoint{6.586394in}{0.938911in}}%
\pgfpathlineto{\pgfqpoint{6.587009in}{0.938461in}}%
\pgfpathlineto{\pgfqpoint{6.592626in}{0.934357in}}%
\pgfpathlineto{\pgfqpoint{6.596319in}{0.931659in}}%
\pgfpathlineto{\pgfqpoint{6.598858in}{0.929804in}}%
\pgfpathlineto{\pgfqpoint{6.605089in}{0.925251in}}%
\pgfpathlineto{\pgfqpoint{6.605628in}{0.924857in}}%
\pgfpathlineto{\pgfqpoint{6.611321in}{0.920697in}}%
\pgfpathlineto{\pgfqpoint{6.614937in}{0.918055in}}%
\pgfpathlineto{\pgfqpoint{6.617553in}{0.916144in}}%
\pgfpathlineto{\pgfqpoint{6.623785in}{0.911591in}}%
\pgfpathlineto{\pgfqpoint{6.624247in}{0.911254in}}%
\pgfpathlineto{\pgfqpoint{6.630017in}{0.907038in}}%
\pgfpathlineto{\pgfqpoint{6.633556in}{0.904452in}}%
\pgfpathlineto{\pgfqpoint{6.636249in}{0.902484in}}%
\pgfpathlineto{\pgfqpoint{6.642480in}{0.897931in}}%
\pgfpathlineto{\pgfqpoint{6.642865in}{0.897650in}}%
\pgfpathlineto{\pgfqpoint{6.648712in}{0.893378in}}%
\pgfpathlineto{\pgfqpoint{6.652174in}{0.890848in}}%
\pgfpathlineto{\pgfqpoint{6.654944in}{0.888825in}}%
\pgfpathlineto{\pgfqpoint{6.661176in}{0.884271in}}%
\pgfpathlineto{\pgfqpoint{6.661484in}{0.884046in}}%
\pgfpathlineto{\pgfqpoint{6.667408in}{0.879718in}}%
\pgfpathlineto{\pgfqpoint{6.670793in}{0.877245in}}%
\pgfpathlineto{\pgfqpoint{6.673640in}{0.875165in}}%
\pgfpathlineto{\pgfqpoint{6.679872in}{0.870611in}}%
\pgfpathlineto{\pgfqpoint{6.680102in}{0.870443in}}%
\pgfpathlineto{\pgfqpoint{6.686103in}{0.866058in}}%
\pgfpathlineto{\pgfqpoint{6.689412in}{0.863641in}}%
\pgfpathlineto{\pgfqpoint{6.692335in}{0.861505in}}%
\pgfpathlineto{\pgfqpoint{6.698567in}{0.856952in}}%
\pgfpathlineto{\pgfqpoint{6.698721in}{0.856839in}}%
\pgfpathlineto{\pgfqpoint{6.704799in}{0.852398in}}%
\pgfpathlineto{\pgfqpoint{6.708030in}{0.850037in}}%
\pgfpathlineto{\pgfqpoint{6.711031in}{0.847845in}}%
\pgfpathlineto{\pgfqpoint{6.717263in}{0.843292in}}%
\pgfpathlineto{\pgfqpoint{6.717340in}{0.843236in}}%
\pgfpathlineto{\pgfqpoint{6.723495in}{0.838739in}}%
\pgfpathlineto{\pgfqpoint{6.726649in}{0.836434in}}%
\pgfpathlineto{\pgfqpoint{6.729726in}{0.834185in}}%
\pgfpathlineto{\pgfqpoint{6.735958in}{0.829632in}}%
\pgfpathlineto{\pgfqpoint{6.735958in}{0.829632in}}%
\pgfpathlineto{\pgfqpoint{6.742190in}{0.825079in}}%
\pgfpathlineto{\pgfqpoint{6.745268in}{0.822830in}}%
\pgfpathlineto{\pgfqpoint{6.748422in}{0.820525in}}%
\pgfpathlineto{\pgfqpoint{6.754577in}{0.816028in}}%
\pgfpathlineto{\pgfqpoint{6.754654in}{0.815972in}}%
\pgfpathlineto{\pgfqpoint{6.760886in}{0.811419in}}%
\pgfpathlineto{\pgfqpoint{6.763886in}{0.809227in}}%
\pgfpathlineto{\pgfqpoint{6.767118in}{0.806866in}}%
\pgfpathlineto{\pgfqpoint{6.773196in}{0.802425in}}%
\pgfpathlineto{\pgfqpoint{6.773349in}{0.802312in}}%
\pgfpathlineto{\pgfqpoint{6.779581in}{0.797759in}}%
\pgfpathlineto{\pgfqpoint{6.782505in}{0.795623in}}%
\pgfpathlineto{\pgfqpoint{6.785813in}{0.793206in}}%
\pgfpathlineto{\pgfqpoint{6.791814in}{0.788821in}}%
\pgfpathlineto{\pgfqpoint{6.792045in}{0.788653in}}%
\pgfpathlineto{\pgfqpoint{6.798277in}{0.784099in}}%
\pgfpathlineto{\pgfqpoint{6.801123in}{0.782019in}}%
\pgfpathlineto{\pgfqpoint{6.804509in}{0.779546in}}%
\pgfpathlineto{\pgfqpoint{6.810433in}{0.775218in}}%
\pgfpathlineto{\pgfqpoint{6.810740in}{0.774993in}}%
\pgfpathlineto{\pgfqpoint{6.816972in}{0.770439in}}%
\pgfpathlineto{\pgfqpoint{6.819742in}{0.768416in}}%
\pgfpathlineto{\pgfqpoint{6.823204in}{0.765886in}}%
\pgfpathlineto{\pgfqpoint{6.829051in}{0.761614in}}%
\pgfpathlineto{\pgfqpoint{6.829436in}{0.761333in}}%
\pgfpathlineto{\pgfqpoint{6.835668in}{0.756780in}}%
\pgfpathlineto{\pgfqpoint{6.838361in}{0.754812in}}%
\pgfpathlineto{\pgfqpoint{6.841900in}{0.752226in}}%
\pgfpathlineto{\pgfqpoint{6.847670in}{0.748010in}}%
\pgfpathlineto{\pgfqpoint{6.848132in}{0.747673in}}%
\pgfpathlineto{\pgfqpoint{6.854363in}{0.743120in}}%
\pgfpathlineto{\pgfqpoint{6.856979in}{0.741209in}}%
\pgfpathlineto{\pgfqpoint{6.860595in}{0.738567in}}%
\pgfpathlineto{\pgfqpoint{6.866289in}{0.734407in}}%
\pgfpathlineto{\pgfqpoint{6.866827in}{0.734013in}}%
\pgfpathlineto{\pgfqpoint{6.873059in}{0.729460in}}%
\pgfpathlineto{\pgfqpoint{6.875598in}{0.727605in}}%
\pgfpathlineto{\pgfqpoint{6.879291in}{0.724907in}}%
\pgfpathlineto{\pgfqpoint{6.884907in}{0.720803in}}%
\pgfpathlineto{\pgfqpoint{6.885523in}{0.720353in}}%
\pgfpathlineto{\pgfqpoint{6.891755in}{0.715800in}}%
\pgfpathlineto{\pgfqpoint{6.894217in}{0.714001in}}%
\pgfpathlineto{\pgfqpoint{6.897986in}{0.711247in}}%
\pgfpathlineto{\pgfqpoint{6.903526in}{0.707200in}}%
\pgfpathlineto{\pgfqpoint{6.904218in}{0.706694in}}%
\pgfpathlineto{\pgfqpoint{6.910450in}{0.702140in}}%
\pgfpathlineto{\pgfqpoint{6.912835in}{0.700398in}}%
\pgfpathlineto{\pgfqpoint{6.916682in}{0.697587in}}%
\pgfpathlineto{\pgfqpoint{6.922144in}{0.693596in}}%
\pgfpathlineto{\pgfqpoint{6.922914in}{0.693034in}}%
\pgfpathlineto{\pgfqpoint{6.929146in}{0.688481in}}%
\pgfpathlineto{\pgfqpoint{6.931454in}{0.688481in}}%
\pgfpathlineto{\pgfqpoint{6.940763in}{0.688481in}}%
\pgfpathlineto{\pgfqpoint{6.950072in}{0.688481in}}%
\pgfpathlineto{\pgfqpoint{6.959382in}{0.688481in}}%
\pgfpathlineto{\pgfqpoint{6.968691in}{0.688481in}}%
\pgfpathlineto{\pgfqpoint{6.978000in}{0.688481in}}%
\pgfpathlineto{\pgfqpoint{6.987310in}{0.688481in}}%
\pgfpathlineto{\pgfqpoint{6.996619in}{0.688481in}}%
\pgfpathlineto{\pgfqpoint{7.005928in}{0.688481in}}%
\pgfpathlineto{\pgfqpoint{7.015238in}{0.688481in}}%
\pgfpathlineto{\pgfqpoint{7.024547in}{0.688481in}}%
\pgfpathlineto{\pgfqpoint{7.033856in}{0.688481in}}%
\pgfpathlineto{\pgfqpoint{7.043165in}{0.688481in}}%
\pgfpathlineto{\pgfqpoint{7.052475in}{0.688481in}}%
\pgfpathlineto{\pgfqpoint{7.061784in}{0.688481in}}%
\pgfpathlineto{\pgfqpoint{7.071093in}{0.688481in}}%
\pgfpathlineto{\pgfqpoint{7.080403in}{0.688481in}}%
\pgfpathlineto{\pgfqpoint{7.089712in}{0.688481in}}%
\pgfpathlineto{\pgfqpoint{7.099021in}{0.688481in}}%
\pgfpathlineto{\pgfqpoint{7.108331in}{0.688481in}}%
\pgfpathlineto{\pgfqpoint{7.117640in}{0.688481in}}%
\pgfpathlineto{\pgfqpoint{7.126949in}{0.688481in}}%
\pgfpathlineto{\pgfqpoint{7.136259in}{0.688481in}}%
\pgfpathlineto{\pgfqpoint{7.145568in}{0.688481in}}%
\pgfpathlineto{\pgfqpoint{7.154877in}{0.688481in}}%
\pgfpathlineto{\pgfqpoint{7.164186in}{0.688481in}}%
\pgfpathlineto{\pgfqpoint{7.173496in}{0.688481in}}%
\pgfpathlineto{\pgfqpoint{7.182805in}{0.688481in}}%
\pgfpathlineto{\pgfqpoint{7.192114in}{0.688481in}}%
\pgfpathlineto{\pgfqpoint{7.201424in}{0.688481in}}%
\pgfpathlineto{\pgfqpoint{7.210733in}{0.688481in}}%
\pgfpathlineto{\pgfqpoint{7.220042in}{0.688481in}}%
\pgfpathlineto{\pgfqpoint{7.229352in}{0.688481in}}%
\pgfpathlineto{\pgfqpoint{7.238661in}{0.688481in}}%
\pgfpathlineto{\pgfqpoint{7.247970in}{0.688481in}}%
\pgfpathlineto{\pgfqpoint{7.257280in}{0.688481in}}%
\pgfpathlineto{\pgfqpoint{7.266589in}{0.688481in}}%
\pgfpathlineto{\pgfqpoint{7.275898in}{0.688481in}}%
\pgfpathlineto{\pgfqpoint{7.285208in}{0.688481in}}%
\pgfpathlineto{\pgfqpoint{7.294517in}{0.688481in}}%
\pgfpathlineto{\pgfqpoint{7.303826in}{0.688481in}}%
\pgfpathlineto{\pgfqpoint{7.313135in}{0.688481in}}%
\pgfpathlineto{\pgfqpoint{7.322445in}{0.688481in}}%
\pgfpathlineto{\pgfqpoint{7.331754in}{0.688481in}}%
\pgfpathlineto{\pgfqpoint{7.341063in}{0.688481in}}%
\pgfpathlineto{\pgfqpoint{7.350373in}{0.688481in}}%
\pgfpathlineto{\pgfqpoint{7.359682in}{0.688481in}}%
\pgfpathlineto{\pgfqpoint{7.368991in}{0.688481in}}%
\pgfpathlineto{\pgfqpoint{7.378301in}{0.688481in}}%
\pgfpathlineto{\pgfqpoint{7.387610in}{0.688481in}}%
\pgfpathlineto{\pgfqpoint{7.396919in}{0.688481in}}%
\pgfpathlineto{\pgfqpoint{7.406229in}{0.688481in}}%
\pgfpathlineto{\pgfqpoint{7.415538in}{0.688481in}}%
\pgfpathlineto{\pgfqpoint{7.424847in}{0.688481in}}%
\pgfpathlineto{\pgfqpoint{7.434156in}{0.688481in}}%
\pgfpathlineto{\pgfqpoint{7.443466in}{0.688481in}}%
\pgfpathlineto{\pgfqpoint{7.452775in}{0.688481in}}%
\pgfpathlineto{\pgfqpoint{7.462084in}{0.688481in}}%
\pgfpathlineto{\pgfqpoint{7.471394in}{0.688481in}}%
\pgfpathlineto{\pgfqpoint{7.480703in}{0.688481in}}%
\pgfpathlineto{\pgfqpoint{7.490012in}{0.688481in}}%
\pgfpathlineto{\pgfqpoint{7.499322in}{0.688481in}}%
\pgfpathlineto{\pgfqpoint{7.508631in}{0.688481in}}%
\pgfpathlineto{\pgfqpoint{7.517940in}{0.688481in}}%
\pgfpathlineto{\pgfqpoint{7.527250in}{0.688481in}}%
\pgfpathlineto{\pgfqpoint{7.536559in}{0.688481in}}%
\pgfpathlineto{\pgfqpoint{7.545868in}{0.688481in}}%
\pgfpathlineto{\pgfqpoint{7.555177in}{0.688481in}}%
\pgfpathlineto{\pgfqpoint{7.564487in}{0.688481in}}%
\pgfpathlineto{\pgfqpoint{7.573796in}{0.688481in}}%
\pgfpathlineto{\pgfqpoint{7.583105in}{0.688481in}}%
\pgfpathlineto{\pgfqpoint{7.592415in}{0.688481in}}%
\pgfpathlineto{\pgfqpoint{7.601724in}{0.688481in}}%
\pgfpathlineto{\pgfqpoint{7.611033in}{0.688481in}}%
\pgfpathlineto{\pgfqpoint{7.620343in}{0.688481in}}%
\pgfpathlineto{\pgfqpoint{7.629652in}{0.688481in}}%
\pgfpathlineto{\pgfqpoint{7.638961in}{0.688481in}}%
\pgfpathlineto{\pgfqpoint{7.648271in}{0.688481in}}%
\pgfpathlineto{\pgfqpoint{7.657580in}{0.688481in}}%
\pgfpathlineto{\pgfqpoint{7.666889in}{0.688481in}}%
\pgfpathlineto{\pgfqpoint{7.676199in}{0.688481in}}%
\pgfpathlineto{\pgfqpoint{7.685508in}{0.688481in}}%
\pgfpathlineto{\pgfqpoint{7.694817in}{0.688481in}}%
\pgfpathlineto{\pgfqpoint{7.704126in}{0.688481in}}%
\pgfpathlineto{\pgfqpoint{7.713436in}{0.688481in}}%
\pgfpathlineto{\pgfqpoint{7.722745in}{0.688481in}}%
\pgfpathlineto{\pgfqpoint{7.732054in}{0.688481in}}%
\pgfpathlineto{\pgfqpoint{7.741364in}{0.688481in}}%
\pgfpathlineto{\pgfqpoint{7.750673in}{0.688481in}}%
\pgfpathlineto{\pgfqpoint{7.759982in}{0.688481in}}%
\pgfpathlineto{\pgfqpoint{7.769292in}{0.688481in}}%
\pgfpathlineto{\pgfqpoint{7.778601in}{0.688481in}}%
\pgfpathlineto{\pgfqpoint{7.787910in}{0.688481in}}%
\pgfpathlineto{\pgfqpoint{7.797220in}{0.688481in}}%
\pgfpathlineto{\pgfqpoint{7.806529in}{0.688481in}}%
\pgfpathlineto{\pgfqpoint{7.815838in}{0.688481in}}%
\pgfpathlineto{\pgfqpoint{7.825147in}{0.688481in}}%
\pgfpathlineto{\pgfqpoint{7.834457in}{0.688481in}}%
\pgfpathlineto{\pgfqpoint{7.843766in}{0.688481in}}%
\pgfpathlineto{\pgfqpoint{7.853075in}{0.688481in}}%
\pgfpathlineto{\pgfqpoint{7.862385in}{0.688481in}}%
\pgfpathlineto{\pgfqpoint{7.871694in}{0.688481in}}%
\pgfpathlineto{\pgfqpoint{7.881003in}{0.688481in}}%
\pgfpathlineto{\pgfqpoint{7.890313in}{0.688481in}}%
\pgfpathlineto{\pgfqpoint{7.899622in}{0.688481in}}%
\pgfpathlineto{\pgfqpoint{7.908931in}{0.688481in}}%
\pgfpathlineto{\pgfqpoint{7.918241in}{0.688481in}}%
\pgfpathlineto{\pgfqpoint{7.927550in}{0.688481in}}%
\pgfpathlineto{\pgfqpoint{7.936859in}{0.688481in}}%
\pgfpathlineto{\pgfqpoint{7.946168in}{0.688481in}}%
\pgfpathlineto{\pgfqpoint{7.955478in}{0.688481in}}%
\pgfpathlineto{\pgfqpoint{7.964787in}{0.688481in}}%
\pgfpathlineto{\pgfqpoint{7.974096in}{0.688481in}}%
\pgfpathlineto{\pgfqpoint{7.983406in}{0.688481in}}%
\pgfpathlineto{\pgfqpoint{7.992715in}{0.688481in}}%
\pgfpathlineto{\pgfqpoint{8.002024in}{0.688481in}}%
\pgfpathlineto{\pgfqpoint{8.011334in}{0.688481in}}%
\pgfpathlineto{\pgfqpoint{8.020643in}{0.688481in}}%
\pgfpathlineto{\pgfqpoint{8.029952in}{0.688481in}}%
\pgfpathlineto{\pgfqpoint{8.039262in}{0.688481in}}%
\pgfpathlineto{\pgfqpoint{8.048571in}{0.688481in}}%
\pgfpathlineto{\pgfqpoint{8.057880in}{0.688481in}}%
\pgfpathlineto{\pgfqpoint{8.067189in}{0.688481in}}%
\pgfpathlineto{\pgfqpoint{8.076499in}{0.688481in}}%
\pgfpathlineto{\pgfqpoint{8.085808in}{0.688481in}}%
\pgfpathlineto{\pgfqpoint{8.095117in}{0.688481in}}%
\pgfpathlineto{\pgfqpoint{8.104427in}{0.688481in}}%
\pgfpathlineto{\pgfqpoint{8.113736in}{0.688481in}}%
\pgfpathlineto{\pgfqpoint{8.123045in}{0.688481in}}%
\pgfpathlineto{\pgfqpoint{8.132355in}{0.688481in}}%
\pgfpathlineto{\pgfqpoint{8.141664in}{0.688481in}}%
\pgfpathlineto{\pgfqpoint{8.150973in}{0.688481in}}%
\pgfpathlineto{\pgfqpoint{8.160283in}{0.688481in}}%
\pgfpathlineto{\pgfqpoint{8.169592in}{0.688481in}}%
\pgfpathlineto{\pgfqpoint{8.178901in}{0.688481in}}%
\pgfpathlineto{\pgfqpoint{8.188211in}{0.688481in}}%
\pgfpathlineto{\pgfqpoint{8.197520in}{0.688481in}}%
\pgfpathlineto{\pgfqpoint{8.206829in}{0.688481in}}%
\pgfpathlineto{\pgfqpoint{8.216138in}{0.688481in}}%
\pgfpathlineto{\pgfqpoint{8.225448in}{0.688481in}}%
\pgfpathlineto{\pgfqpoint{8.234757in}{0.688481in}}%
\pgfpathlineto{\pgfqpoint{8.244066in}{0.688481in}}%
\pgfpathlineto{\pgfqpoint{8.253376in}{0.688481in}}%
\pgfpathlineto{\pgfqpoint{8.262685in}{0.688481in}}%
\pgfpathlineto{\pgfqpoint{8.271994in}{0.688481in}}%
\pgfpathlineto{\pgfqpoint{8.281304in}{0.688481in}}%
\pgfpathlineto{\pgfqpoint{8.290613in}{0.688481in}}%
\pgfpathlineto{\pgfqpoint{8.299922in}{0.688481in}}%
\pgfpathlineto{\pgfqpoint{8.309232in}{0.688481in}}%
\pgfpathlineto{\pgfqpoint{8.318541in}{0.688481in}}%
\pgfpathlineto{\pgfqpoint{8.327850in}{0.688481in}}%
\pgfpathlineto{\pgfqpoint{8.337159in}{0.688481in}}%
\pgfpathlineto{\pgfqpoint{8.346469in}{0.688481in}}%
\pgfpathlineto{\pgfqpoint{8.355778in}{0.688481in}}%
\pgfpathlineto{\pgfqpoint{8.365087in}{0.688481in}}%
\pgfpathlineto{\pgfqpoint{8.374397in}{0.688481in}}%
\pgfpathlineto{\pgfqpoint{8.383706in}{0.688481in}}%
\pgfpathlineto{\pgfqpoint{8.393015in}{0.688481in}}%
\pgfpathlineto{\pgfqpoint{8.402325in}{0.688481in}}%
\pgfpathlineto{\pgfqpoint{8.411634in}{0.688481in}}%
\pgfpathlineto{\pgfqpoint{8.420943in}{0.688481in}}%
\pgfpathlineto{\pgfqpoint{8.430253in}{0.688481in}}%
\pgfpathlineto{\pgfqpoint{8.439562in}{0.688481in}}%
\pgfpathlineto{\pgfqpoint{8.448871in}{0.688481in}}%
\pgfpathlineto{\pgfqpoint{8.458180in}{0.688481in}}%
\pgfpathlineto{\pgfqpoint{8.467490in}{0.688481in}}%
\pgfpathlineto{\pgfqpoint{8.476799in}{0.688481in}}%
\pgfpathlineto{\pgfqpoint{8.486108in}{0.688481in}}%
\pgfpathlineto{\pgfqpoint{8.495418in}{0.688481in}}%
\pgfpathlineto{\pgfqpoint{8.504727in}{0.688481in}}%
\pgfpathlineto{\pgfqpoint{8.514036in}{0.688481in}}%
\pgfpathlineto{\pgfqpoint{8.523346in}{0.688481in}}%
\pgfpathlineto{\pgfqpoint{8.532655in}{0.688481in}}%
\pgfpathlineto{\pgfqpoint{8.541964in}{0.688481in}}%
\pgfpathlineto{\pgfqpoint{8.551274in}{0.688481in}}%
\pgfpathlineto{\pgfqpoint{8.560583in}{0.688481in}}%
\pgfpathlineto{\pgfqpoint{8.569892in}{0.688481in}}%
\pgfpathlineto{\pgfqpoint{8.579202in}{0.688481in}}%
\pgfpathlineto{\pgfqpoint{8.588511in}{0.688481in}}%
\pgfpathlineto{\pgfqpoint{8.597820in}{0.688481in}}%
\pgfpathlineto{\pgfqpoint{8.607129in}{0.688481in}}%
\pgfpathlineto{\pgfqpoint{8.616439in}{0.688481in}}%
\pgfpathlineto{\pgfqpoint{8.625748in}{0.688481in}}%
\pgfpathlineto{\pgfqpoint{8.635057in}{0.688481in}}%
\pgfpathlineto{\pgfqpoint{8.644367in}{0.688481in}}%
\pgfpathlineto{\pgfqpoint{8.653676in}{0.688481in}}%
\pgfpathlineto{\pgfqpoint{8.662985in}{0.688481in}}%
\pgfpathlineto{\pgfqpoint{8.672295in}{0.688481in}}%
\pgfpathlineto{\pgfqpoint{8.681604in}{0.688481in}}%
\pgfpathlineto{\pgfqpoint{8.690913in}{0.688481in}}%
\pgfpathlineto{\pgfqpoint{8.700223in}{0.688481in}}%
\pgfpathlineto{\pgfqpoint{8.709532in}{0.688481in}}%
\pgfpathlineto{\pgfqpoint{8.718841in}{0.688481in}}%
\pgfpathlineto{\pgfqpoint{8.728150in}{0.688481in}}%
\pgfpathlineto{\pgfqpoint{8.737460in}{0.688481in}}%
\pgfpathlineto{\pgfqpoint{8.746769in}{0.688481in}}%
\pgfpathlineto{\pgfqpoint{8.756078in}{0.688481in}}%
\pgfpathlineto{\pgfqpoint{8.765388in}{0.688481in}}%
\pgfpathlineto{\pgfqpoint{8.774697in}{0.688481in}}%
\pgfpathlineto{\pgfqpoint{8.784006in}{0.688481in}}%
\pgfpathlineto{\pgfqpoint{8.793316in}{0.688481in}}%
\pgfpathlineto{\pgfqpoint{8.802625in}{0.688481in}}%
\pgfpathlineto{\pgfqpoint{8.811934in}{0.688481in}}%
\pgfpathlineto{\pgfqpoint{8.821244in}{0.688481in}}%
\pgfpathlineto{\pgfqpoint{8.830553in}{0.688481in}}%
\pgfpathlineto{\pgfqpoint{8.839862in}{0.688481in}}%
\pgfpathlineto{\pgfqpoint{8.849171in}{0.688481in}}%
\pgfpathlineto{\pgfqpoint{8.858481in}{0.688481in}}%
\pgfpathlineto{\pgfqpoint{8.867790in}{0.688481in}}%
\pgfpathlineto{\pgfqpoint{8.877099in}{0.688481in}}%
\pgfpathlineto{\pgfqpoint{8.886409in}{0.688481in}}%
\pgfpathlineto{\pgfqpoint{8.895718in}{0.688481in}}%
\pgfpathlineto{\pgfqpoint{8.905027in}{0.688481in}}%
\pgfpathlineto{\pgfqpoint{8.914337in}{0.688481in}}%
\pgfpathlineto{\pgfqpoint{8.923646in}{0.688481in}}%
\pgfpathlineto{\pgfqpoint{8.932955in}{0.688481in}}%
\pgfpathlineto{\pgfqpoint{8.942265in}{0.688481in}}%
\pgfpathlineto{\pgfqpoint{8.951574in}{0.688481in}}%
\pgfpathlineto{\pgfqpoint{8.960883in}{0.688481in}}%
\pgfpathlineto{\pgfqpoint{8.970192in}{0.688481in}}%
\pgfpathlineto{\pgfqpoint{8.979502in}{0.688481in}}%
\pgfpathlineto{\pgfqpoint{8.988811in}{0.688481in}}%
\pgfpathlineto{\pgfqpoint{8.998120in}{0.688481in}}%
\pgfpathlineto{\pgfqpoint{9.007430in}{0.688481in}}%
\pgfpathlineto{\pgfqpoint{9.016739in}{0.688481in}}%
\pgfpathlineto{\pgfqpoint{9.026048in}{0.688481in}}%
\pgfpathlineto{\pgfqpoint{9.035358in}{0.688481in}}%
\pgfpathlineto{\pgfqpoint{9.044667in}{0.688481in}}%
\pgfpathlineto{\pgfqpoint{9.053976in}{0.688481in}}%
\pgfpathlineto{\pgfqpoint{9.063286in}{0.688481in}}%
\pgfpathlineto{\pgfqpoint{9.072595in}{0.688481in}}%
\pgfpathlineto{\pgfqpoint{9.081904in}{0.688481in}}%
\pgfpathlineto{\pgfqpoint{9.091214in}{0.688481in}}%
\pgfpathlineto{\pgfqpoint{9.100523in}{0.688481in}}%
\pgfpathlineto{\pgfqpoint{9.109832in}{0.688481in}}%
\pgfpathlineto{\pgfqpoint{9.119141in}{0.688481in}}%
\pgfpathlineto{\pgfqpoint{9.128451in}{0.688481in}}%
\pgfpathlineto{\pgfqpoint{9.137760in}{0.688481in}}%
\pgfpathlineto{\pgfqpoint{9.147069in}{0.688481in}}%
\pgfpathlineto{\pgfqpoint{9.156379in}{0.688481in}}%
\pgfpathlineto{\pgfqpoint{9.165688in}{0.688481in}}%
\pgfpathlineto{\pgfqpoint{9.174997in}{0.688481in}}%
\pgfpathlineto{\pgfqpoint{9.184307in}{0.688481in}}%
\pgfpathlineto{\pgfqpoint{9.193616in}{0.688481in}}%
\pgfpathlineto{\pgfqpoint{9.202925in}{0.688481in}}%
\pgfpathlineto{\pgfqpoint{9.212235in}{0.688481in}}%
\pgfpathlineto{\pgfqpoint{9.221544in}{0.688481in}}%
\pgfpathlineto{\pgfqpoint{9.230853in}{0.688481in}}%
\pgfpathlineto{\pgfqpoint{9.240162in}{0.688481in}}%
\pgfpathlineto{\pgfqpoint{9.249472in}{0.688481in}}%
\pgfpathlineto{\pgfqpoint{9.258781in}{0.688481in}}%
\pgfpathlineto{\pgfqpoint{9.268090in}{0.688481in}}%
\pgfpathlineto{\pgfqpoint{9.277400in}{0.688481in}}%
\pgfpathlineto{\pgfqpoint{9.286709in}{0.688481in}}%
\pgfpathlineto{\pgfqpoint{9.296018in}{0.688481in}}%
\pgfpathlineto{\pgfqpoint{9.305328in}{0.688481in}}%
\pgfpathlineto{\pgfqpoint{9.314637in}{0.688481in}}%
\pgfpathlineto{\pgfqpoint{9.323946in}{0.688481in}}%
\pgfpathlineto{\pgfqpoint{9.333256in}{0.688481in}}%
\pgfpathlineto{\pgfqpoint{9.342565in}{0.688481in}}%
\pgfpathlineto{\pgfqpoint{9.351874in}{0.688481in}}%
\pgfpathlineto{\pgfqpoint{9.361183in}{0.688481in}}%
\pgfpathlineto{\pgfqpoint{9.370493in}{0.688481in}}%
\pgfpathlineto{\pgfqpoint{9.379802in}{0.688481in}}%
\pgfpathlineto{\pgfqpoint{9.389111in}{0.688481in}}%
\pgfpathlineto{\pgfqpoint{9.398421in}{0.688481in}}%
\pgfpathlineto{\pgfqpoint{9.407730in}{0.688481in}}%
\pgfpathlineto{\pgfqpoint{9.417039in}{0.688481in}}%
\pgfpathlineto{\pgfqpoint{9.426349in}{0.688481in}}%
\pgfpathlineto{\pgfqpoint{9.435658in}{0.688481in}}%
\pgfpathlineto{\pgfqpoint{9.444967in}{0.688481in}}%
\pgfpathlineto{\pgfqpoint{9.454277in}{0.688481in}}%
\pgfpathlineto{\pgfqpoint{9.463586in}{0.688481in}}%
\pgfpathlineto{\pgfqpoint{9.472895in}{0.688481in}}%
\pgfpathlineto{\pgfqpoint{9.482205in}{0.688481in}}%
\pgfpathlineto{\pgfqpoint{9.491514in}{0.688481in}}%
\pgfpathlineto{\pgfqpoint{9.500823in}{0.688481in}}%
\pgfpathlineto{\pgfqpoint{9.510132in}{0.688481in}}%
\pgfpathlineto{\pgfqpoint{9.519442in}{0.688481in}}%
\pgfpathlineto{\pgfqpoint{9.528751in}{0.688481in}}%
\pgfpathlineto{\pgfqpoint{9.538060in}{0.688481in}}%
\pgfpathlineto{\pgfqpoint{9.547370in}{0.688481in}}%
\pgfpathlineto{\pgfqpoint{9.556679in}{0.688481in}}%
\pgfpathlineto{\pgfqpoint{9.565988in}{0.688481in}}%
\pgfpathlineto{\pgfqpoint{9.575298in}{0.688481in}}%
\pgfpathlineto{\pgfqpoint{9.584607in}{0.688481in}}%
\pgfpathlineto{\pgfqpoint{9.593916in}{0.688481in}}%
\pgfpathlineto{\pgfqpoint{9.603226in}{0.688481in}}%
\pgfpathlineto{\pgfqpoint{9.612535in}{0.688481in}}%
\pgfpathlineto{\pgfqpoint{9.621844in}{0.688481in}}%
\pgfpathlineto{\pgfqpoint{9.631153in}{0.688481in}}%
\pgfpathlineto{\pgfqpoint{9.640463in}{0.688481in}}%
\pgfpathlineto{\pgfqpoint{9.649772in}{0.688481in}}%
\pgfpathlineto{\pgfqpoint{9.659081in}{0.688481in}}%
\pgfpathlineto{\pgfqpoint{9.668391in}{0.688481in}}%
\pgfpathlineto{\pgfqpoint{9.677700in}{0.688481in}}%
\pgfpathlineto{\pgfqpoint{9.687009in}{0.688481in}}%
\pgfpathlineto{\pgfqpoint{9.696319in}{0.688481in}}%
\pgfpathlineto{\pgfqpoint{9.705628in}{0.688481in}}%
\pgfpathlineto{\pgfqpoint{9.714937in}{0.688481in}}%
\pgfpathlineto{\pgfqpoint{9.724247in}{0.688481in}}%
\pgfpathlineto{\pgfqpoint{9.733556in}{0.688481in}}%
\pgfpathlineto{\pgfqpoint{9.742865in}{0.688481in}}%
\pgfpathlineto{\pgfqpoint{9.752174in}{0.688481in}}%
\pgfpathlineto{\pgfqpoint{9.761484in}{0.688481in}}%
\pgfpathlineto{\pgfqpoint{9.770793in}{0.688481in}}%
\pgfpathlineto{\pgfqpoint{9.780102in}{0.688481in}}%
\pgfpathlineto{\pgfqpoint{9.789412in}{0.688481in}}%
\pgfpathlineto{\pgfqpoint{9.798721in}{0.688481in}}%
\pgfpathlineto{\pgfqpoint{9.808030in}{0.688481in}}%
\pgfpathlineto{\pgfqpoint{9.817340in}{0.688481in}}%
\pgfpathlineto{\pgfqpoint{9.826649in}{0.688481in}}%
\pgfpathlineto{\pgfqpoint{9.835958in}{0.688481in}}%
\pgfpathlineto{\pgfqpoint{9.845268in}{0.688481in}}%
\pgfpathlineto{\pgfqpoint{9.854577in}{0.688481in}}%
\pgfpathlineto{\pgfqpoint{9.863886in}{0.688481in}}%
\pgfpathlineto{\pgfqpoint{9.873196in}{0.688481in}}%
\pgfpathlineto{\pgfqpoint{9.882505in}{0.688481in}}%
\pgfpathlineto{\pgfqpoint{9.891814in}{0.688481in}}%
\pgfpathlineto{\pgfqpoint{9.901123in}{0.688481in}}%
\pgfpathlineto{\pgfqpoint{9.910433in}{0.688481in}}%
\pgfpathlineto{\pgfqpoint{9.919742in}{0.688481in}}%
\pgfpathlineto{\pgfqpoint{9.929051in}{0.688481in}}%
\pgfpathlineto{\pgfqpoint{9.938361in}{0.688481in}}%
\pgfpathlineto{\pgfqpoint{9.947670in}{0.688481in}}%
\pgfpathlineto{\pgfqpoint{9.956979in}{0.688481in}}%
\pgfpathlineto{\pgfqpoint{9.966289in}{0.688481in}}%
\pgfpathlineto{\pgfqpoint{9.975598in}{0.688481in}}%
\pgfpathlineto{\pgfqpoint{9.984907in}{0.688481in}}%
\pgfpathlineto{\pgfqpoint{9.994217in}{0.688481in}}%
\pgfpathlineto{\pgfqpoint{10.003526in}{0.688481in}}%
\pgfpathlineto{\pgfqpoint{10.003526in}{0.688481in}}%
\pgfpathlineto{\pgfqpoint{10.003526in}{0.688481in}}%
\pgfpathlineto{\pgfqpoint{9.994217in}{0.695282in}}%
\pgfpathlineto{\pgfqpoint{9.984907in}{0.702084in}}%
\pgfpathlineto{\pgfqpoint{9.975598in}{0.708886in}}%
\pgfpathlineto{\pgfqpoint{9.966289in}{0.715688in}}%
\pgfpathlineto{\pgfqpoint{9.956979in}{0.722490in}}%
\pgfpathlineto{\pgfqpoint{9.947670in}{0.729291in}}%
\pgfpathlineto{\pgfqpoint{9.938361in}{0.736093in}}%
\pgfpathlineto{\pgfqpoint{9.929051in}{0.742895in}}%
\pgfpathlineto{\pgfqpoint{9.919742in}{0.749697in}}%
\pgfpathlineto{\pgfqpoint{9.910433in}{0.756499in}}%
\pgfpathlineto{\pgfqpoint{9.901123in}{0.763300in}}%
\pgfpathlineto{\pgfqpoint{9.891814in}{0.770102in}}%
\pgfpathlineto{\pgfqpoint{9.882505in}{0.776904in}}%
\pgfpathlineto{\pgfqpoint{9.873196in}{0.783706in}}%
\pgfpathlineto{\pgfqpoint{9.863886in}{0.790508in}}%
\pgfpathlineto{\pgfqpoint{9.854577in}{0.797309in}}%
\pgfpathlineto{\pgfqpoint{9.845268in}{0.804111in}}%
\pgfpathlineto{\pgfqpoint{9.835958in}{0.810913in}}%
\pgfpathlineto{\pgfqpoint{9.826649in}{0.817715in}}%
\pgfpathlineto{\pgfqpoint{9.817340in}{0.824517in}}%
\pgfpathlineto{\pgfqpoint{9.808030in}{0.831318in}}%
\pgfpathlineto{\pgfqpoint{9.798721in}{0.838120in}}%
\pgfpathlineto{\pgfqpoint{9.789412in}{0.844922in}}%
\pgfpathlineto{\pgfqpoint{9.780102in}{0.851724in}}%
\pgfpathlineto{\pgfqpoint{9.770793in}{0.858526in}}%
\pgfpathlineto{\pgfqpoint{9.761484in}{0.865327in}}%
\pgfpathlineto{\pgfqpoint{9.752174in}{0.872129in}}%
\pgfpathlineto{\pgfqpoint{9.742865in}{0.878931in}}%
\pgfpathlineto{\pgfqpoint{9.733556in}{0.885733in}}%
\pgfpathlineto{\pgfqpoint{9.724247in}{0.892535in}}%
\pgfpathlineto{\pgfqpoint{9.714937in}{0.899336in}}%
\pgfpathlineto{\pgfqpoint{9.705628in}{0.906138in}}%
\pgfpathlineto{\pgfqpoint{9.696319in}{0.912940in}}%
\pgfpathlineto{\pgfqpoint{9.687009in}{0.919742in}}%
\pgfpathlineto{\pgfqpoint{9.677700in}{0.926544in}}%
\pgfpathlineto{\pgfqpoint{9.668391in}{0.933345in}}%
\pgfpathlineto{\pgfqpoint{9.659081in}{0.940147in}}%
\pgfpathlineto{\pgfqpoint{9.649772in}{0.946949in}}%
\pgfpathlineto{\pgfqpoint{9.640463in}{0.953751in}}%
\pgfpathlineto{\pgfqpoint{9.631153in}{0.960553in}}%
\pgfpathlineto{\pgfqpoint{9.621844in}{0.967354in}}%
\pgfpathlineto{\pgfqpoint{9.612535in}{0.974156in}}%
\pgfpathlineto{\pgfqpoint{9.603226in}{0.980958in}}%
\pgfpathlineto{\pgfqpoint{9.593916in}{0.987760in}}%
\pgfpathlineto{\pgfqpoint{9.584607in}{0.994562in}}%
\pgfpathlineto{\pgfqpoint{9.575298in}{1.001363in}}%
\pgfpathlineto{\pgfqpoint{9.565988in}{1.008165in}}%
\pgfpathlineto{\pgfqpoint{9.556679in}{1.014967in}}%
\pgfpathlineto{\pgfqpoint{9.547370in}{1.021769in}}%
\pgfpathlineto{\pgfqpoint{9.538060in}{1.028571in}}%
\pgfpathlineto{\pgfqpoint{9.528751in}{1.035372in}}%
\pgfpathlineto{\pgfqpoint{9.519442in}{1.042174in}}%
\pgfpathlineto{\pgfqpoint{9.510132in}{1.048976in}}%
\pgfpathlineto{\pgfqpoint{9.500823in}{1.055778in}}%
\pgfpathlineto{\pgfqpoint{9.491514in}{1.062580in}}%
\pgfpathlineto{\pgfqpoint{9.482205in}{1.069381in}}%
\pgfpathlineto{\pgfqpoint{9.472895in}{1.076183in}}%
\pgfpathlineto{\pgfqpoint{9.463586in}{1.082985in}}%
\pgfpathlineto{\pgfqpoint{9.454277in}{1.089787in}}%
\pgfpathlineto{\pgfqpoint{9.444967in}{1.096589in}}%
\pgfpathlineto{\pgfqpoint{9.435658in}{1.103390in}}%
\pgfpathlineto{\pgfqpoint{9.426349in}{1.110192in}}%
\pgfpathlineto{\pgfqpoint{9.417039in}{1.116994in}}%
\pgfpathlineto{\pgfqpoint{9.407730in}{1.123796in}}%
\pgfpathlineto{\pgfqpoint{9.398421in}{1.130598in}}%
\pgfpathlineto{\pgfqpoint{9.389111in}{1.137399in}}%
\pgfpathlineto{\pgfqpoint{9.379802in}{1.144201in}}%
\pgfpathlineto{\pgfqpoint{9.370493in}{1.151003in}}%
\pgfpathlineto{\pgfqpoint{9.361183in}{1.157805in}}%
\pgfpathlineto{\pgfqpoint{9.351874in}{1.164607in}}%
\pgfpathlineto{\pgfqpoint{9.342565in}{1.171408in}}%
\pgfpathlineto{\pgfqpoint{9.333256in}{1.178210in}}%
\pgfpathlineto{\pgfqpoint{9.323946in}{1.185012in}}%
\pgfpathlineto{\pgfqpoint{9.314637in}{1.191814in}}%
\pgfpathlineto{\pgfqpoint{9.305328in}{1.198616in}}%
\pgfpathlineto{\pgfqpoint{9.296018in}{1.205417in}}%
\pgfpathlineto{\pgfqpoint{9.286709in}{1.212219in}}%
\pgfpathlineto{\pgfqpoint{9.277400in}{1.219021in}}%
\pgfpathlineto{\pgfqpoint{9.268090in}{1.225823in}}%
\pgfpathlineto{\pgfqpoint{9.258781in}{1.232625in}}%
\pgfpathlineto{\pgfqpoint{9.249472in}{1.239427in}}%
\pgfpathlineto{\pgfqpoint{9.240162in}{1.246228in}}%
\pgfpathlineto{\pgfqpoint{9.230853in}{1.253030in}}%
\pgfpathlineto{\pgfqpoint{9.221544in}{1.259832in}}%
\pgfpathlineto{\pgfqpoint{9.212235in}{1.266634in}}%
\pgfpathlineto{\pgfqpoint{9.202925in}{1.273436in}}%
\pgfpathlineto{\pgfqpoint{9.193616in}{1.280237in}}%
\pgfpathlineto{\pgfqpoint{9.184307in}{1.287039in}}%
\pgfpathlineto{\pgfqpoint{9.174997in}{1.293841in}}%
\pgfpathlineto{\pgfqpoint{9.165688in}{1.300643in}}%
\pgfpathlineto{\pgfqpoint{9.156379in}{1.307445in}}%
\pgfpathlineto{\pgfqpoint{9.147069in}{1.314246in}}%
\pgfpathlineto{\pgfqpoint{9.137760in}{1.321048in}}%
\pgfpathlineto{\pgfqpoint{9.128451in}{1.327850in}}%
\pgfpathlineto{\pgfqpoint{9.119141in}{1.334652in}}%
\pgfpathlineto{\pgfqpoint{9.109832in}{1.341454in}}%
\pgfpathlineto{\pgfqpoint{9.100523in}{1.348255in}}%
\pgfpathlineto{\pgfqpoint{9.091214in}{1.355057in}}%
\pgfpathlineto{\pgfqpoint{9.081904in}{1.361859in}}%
\pgfpathlineto{\pgfqpoint{9.072595in}{1.368661in}}%
\pgfpathlineto{\pgfqpoint{9.063286in}{1.375463in}}%
\pgfpathlineto{\pgfqpoint{9.053976in}{1.382264in}}%
\pgfpathlineto{\pgfqpoint{9.044667in}{1.389066in}}%
\pgfpathlineto{\pgfqpoint{9.035358in}{1.395868in}}%
\pgfpathlineto{\pgfqpoint{9.026048in}{1.402670in}}%
\pgfpathlineto{\pgfqpoint{9.016739in}{1.409472in}}%
\pgfpathlineto{\pgfqpoint{9.007430in}{1.416273in}}%
\pgfpathlineto{\pgfqpoint{8.998120in}{1.423075in}}%
\pgfpathlineto{\pgfqpoint{8.988811in}{1.429877in}}%
\pgfpathlineto{\pgfqpoint{8.979502in}{1.436679in}}%
\pgfpathlineto{\pgfqpoint{8.970192in}{1.443481in}}%
\pgfpathlineto{\pgfqpoint{8.960883in}{1.450282in}}%
\pgfpathlineto{\pgfqpoint{8.951574in}{1.457084in}}%
\pgfpathlineto{\pgfqpoint{8.942265in}{1.463886in}}%
\pgfpathlineto{\pgfqpoint{8.932955in}{1.470688in}}%
\pgfpathlineto{\pgfqpoint{8.923646in}{1.477490in}}%
\pgfpathlineto{\pgfqpoint{8.914337in}{1.484291in}}%
\pgfpathlineto{\pgfqpoint{8.905027in}{1.491093in}}%
\pgfpathlineto{\pgfqpoint{8.895718in}{1.497895in}}%
\pgfpathlineto{\pgfqpoint{8.886409in}{1.504697in}}%
\pgfpathlineto{\pgfqpoint{8.877099in}{1.511499in}}%
\pgfpathlineto{\pgfqpoint{8.867790in}{1.518300in}}%
\pgfpathlineto{\pgfqpoint{8.858481in}{1.525102in}}%
\pgfpathlineto{\pgfqpoint{8.849171in}{1.531904in}}%
\pgfpathlineto{\pgfqpoint{8.839862in}{1.538706in}}%
\pgfpathlineto{\pgfqpoint{8.830553in}{1.545508in}}%
\pgfpathlineto{\pgfqpoint{8.821244in}{1.552309in}}%
\pgfpathlineto{\pgfqpoint{8.811934in}{1.559111in}}%
\pgfpathlineto{\pgfqpoint{8.802625in}{1.565913in}}%
\pgfpathlineto{\pgfqpoint{8.793316in}{1.572715in}}%
\pgfpathlineto{\pgfqpoint{8.784006in}{1.579517in}}%
\pgfpathlineto{\pgfqpoint{8.774697in}{1.586318in}}%
\pgfpathlineto{\pgfqpoint{8.765388in}{1.593120in}}%
\pgfpathlineto{\pgfqpoint{8.756078in}{1.599922in}}%
\pgfpathlineto{\pgfqpoint{8.746769in}{1.606724in}}%
\pgfpathlineto{\pgfqpoint{8.737460in}{1.613526in}}%
\pgfpathlineto{\pgfqpoint{8.728150in}{1.620327in}}%
\pgfpathlineto{\pgfqpoint{8.718841in}{1.627129in}}%
\pgfpathlineto{\pgfqpoint{8.709532in}{1.633931in}}%
\pgfpathlineto{\pgfqpoint{8.700223in}{1.640733in}}%
\pgfpathlineto{\pgfqpoint{8.690913in}{1.647535in}}%
\pgfpathlineto{\pgfqpoint{8.681604in}{1.654336in}}%
\pgfpathlineto{\pgfqpoint{8.672295in}{1.661138in}}%
\pgfpathlineto{\pgfqpoint{8.662985in}{1.667940in}}%
\pgfpathlineto{\pgfqpoint{8.653676in}{1.674742in}}%
\pgfpathlineto{\pgfqpoint{8.644367in}{1.681544in}}%
\pgfpathlineto{\pgfqpoint{8.635057in}{1.688345in}}%
\pgfpathlineto{\pgfqpoint{8.625748in}{1.695147in}}%
\pgfpathlineto{\pgfqpoint{8.616439in}{1.701949in}}%
\pgfpathlineto{\pgfqpoint{8.607129in}{1.708751in}}%
\pgfpathlineto{\pgfqpoint{8.597820in}{1.715553in}}%
\pgfpathlineto{\pgfqpoint{8.588511in}{1.722354in}}%
\pgfpathlineto{\pgfqpoint{8.579202in}{1.729156in}}%
\pgfpathlineto{\pgfqpoint{8.569892in}{1.735958in}}%
\pgfpathlineto{\pgfqpoint{8.560583in}{1.742760in}}%
\pgfpathlineto{\pgfqpoint{8.551274in}{1.749562in}}%
\pgfpathlineto{\pgfqpoint{8.541964in}{1.756363in}}%
\pgfpathlineto{\pgfqpoint{8.532655in}{1.763165in}}%
\pgfpathlineto{\pgfqpoint{8.523346in}{1.769967in}}%
\pgfpathlineto{\pgfqpoint{8.514036in}{1.776769in}}%
\pgfpathlineto{\pgfqpoint{8.504727in}{1.783571in}}%
\pgfpathlineto{\pgfqpoint{8.495418in}{1.790372in}}%
\pgfpathlineto{\pgfqpoint{8.486108in}{1.797174in}}%
\pgfpathlineto{\pgfqpoint{8.476799in}{1.803976in}}%
\pgfpathlineto{\pgfqpoint{8.467490in}{1.810778in}}%
\pgfpathlineto{\pgfqpoint{8.458180in}{1.817580in}}%
\pgfpathlineto{\pgfqpoint{8.448871in}{1.824381in}}%
\pgfpathlineto{\pgfqpoint{8.439562in}{1.831183in}}%
\pgfpathlineto{\pgfqpoint{8.430253in}{1.837985in}}%
\pgfpathlineto{\pgfqpoint{8.420943in}{1.844787in}}%
\pgfpathlineto{\pgfqpoint{8.411634in}{1.851589in}}%
\pgfpathlineto{\pgfqpoint{8.402325in}{1.858390in}}%
\pgfpathlineto{\pgfqpoint{8.393015in}{1.865192in}}%
\pgfpathlineto{\pgfqpoint{8.383706in}{1.871994in}}%
\pgfpathlineto{\pgfqpoint{8.374397in}{1.878796in}}%
\pgfpathlineto{\pgfqpoint{8.365087in}{1.885598in}}%
\pgfpathlineto{\pgfqpoint{8.355778in}{1.892399in}}%
\pgfpathlineto{\pgfqpoint{8.346469in}{1.899201in}}%
\pgfpathlineto{\pgfqpoint{8.337159in}{1.906003in}}%
\pgfpathlineto{\pgfqpoint{8.327850in}{1.912805in}}%
\pgfpathlineto{\pgfqpoint{8.318541in}{1.919607in}}%
\pgfpathlineto{\pgfqpoint{8.309232in}{1.926408in}}%
\pgfpathlineto{\pgfqpoint{8.299922in}{1.933210in}}%
\pgfpathlineto{\pgfqpoint{8.290613in}{1.940012in}}%
\pgfpathlineto{\pgfqpoint{8.281304in}{1.946814in}}%
\pgfpathlineto{\pgfqpoint{8.271994in}{1.953616in}}%
\pgfpathlineto{\pgfqpoint{8.262685in}{1.960417in}}%
\pgfpathlineto{\pgfqpoint{8.253376in}{1.967219in}}%
\pgfpathlineto{\pgfqpoint{8.244066in}{1.974021in}}%
\pgfpathlineto{\pgfqpoint{8.234757in}{1.980823in}}%
\pgfpathlineto{\pgfqpoint{8.225448in}{1.987625in}}%
\pgfpathlineto{\pgfqpoint{8.216138in}{1.994427in}}%
\pgfpathlineto{\pgfqpoint{8.206829in}{2.001228in}}%
\pgfpathlineto{\pgfqpoint{8.197520in}{2.008030in}}%
\pgfpathlineto{\pgfqpoint{8.188211in}{2.014832in}}%
\pgfpathlineto{\pgfqpoint{8.178901in}{2.021634in}}%
\pgfpathlineto{\pgfqpoint{8.169592in}{2.028436in}}%
\pgfpathlineto{\pgfqpoint{8.160283in}{2.035237in}}%
\pgfpathlineto{\pgfqpoint{8.150973in}{2.042039in}}%
\pgfpathlineto{\pgfqpoint{8.141664in}{2.048841in}}%
\pgfpathlineto{\pgfqpoint{8.132355in}{2.055643in}}%
\pgfpathlineto{\pgfqpoint{8.123045in}{2.062445in}}%
\pgfpathlineto{\pgfqpoint{8.113736in}{2.069246in}}%
\pgfpathlineto{\pgfqpoint{8.104427in}{2.076048in}}%
\pgfpathlineto{\pgfqpoint{8.095117in}{2.082850in}}%
\pgfpathlineto{\pgfqpoint{8.085808in}{2.089652in}}%
\pgfpathlineto{\pgfqpoint{8.076499in}{2.096454in}}%
\pgfpathlineto{\pgfqpoint{8.067189in}{2.103255in}}%
\pgfpathlineto{\pgfqpoint{8.057880in}{2.110057in}}%
\pgfpathlineto{\pgfqpoint{8.048571in}{2.116859in}}%
\pgfpathlineto{\pgfqpoint{8.039262in}{2.123661in}}%
\pgfpathlineto{\pgfqpoint{8.029952in}{2.130463in}}%
\pgfpathlineto{\pgfqpoint{8.020643in}{2.137264in}}%
\pgfpathlineto{\pgfqpoint{8.011334in}{2.144066in}}%
\pgfpathlineto{\pgfqpoint{8.002024in}{2.150868in}}%
\pgfpathlineto{\pgfqpoint{7.992715in}{2.157670in}}%
\pgfpathlineto{\pgfqpoint{7.983406in}{2.164472in}}%
\pgfpathlineto{\pgfqpoint{7.974096in}{2.171273in}}%
\pgfpathlineto{\pgfqpoint{7.964787in}{2.178075in}}%
\pgfpathlineto{\pgfqpoint{7.955478in}{2.184877in}}%
\pgfpathlineto{\pgfqpoint{7.946168in}{2.191679in}}%
\pgfpathlineto{\pgfqpoint{7.936859in}{2.198481in}}%
\pgfpathlineto{\pgfqpoint{7.927550in}{2.205282in}}%
\pgfpathlineto{\pgfqpoint{7.918241in}{2.212084in}}%
\pgfpathlineto{\pgfqpoint{7.908931in}{2.218886in}}%
\pgfpathlineto{\pgfqpoint{7.899622in}{2.225688in}}%
\pgfpathlineto{\pgfqpoint{7.890313in}{2.232490in}}%
\pgfpathlineto{\pgfqpoint{7.881003in}{2.239291in}}%
\pgfpathlineto{\pgfqpoint{7.871694in}{2.246093in}}%
\pgfpathlineto{\pgfqpoint{7.862385in}{2.252895in}}%
\pgfpathlineto{\pgfqpoint{7.853075in}{2.259697in}}%
\pgfpathlineto{\pgfqpoint{7.843766in}{2.266499in}}%
\pgfpathlineto{\pgfqpoint{7.834457in}{2.273300in}}%
\pgfpathlineto{\pgfqpoint{7.825147in}{2.280102in}}%
\pgfpathlineto{\pgfqpoint{7.815838in}{2.286904in}}%
\pgfpathlineto{\pgfqpoint{7.806529in}{2.293706in}}%
\pgfpathlineto{\pgfqpoint{7.797220in}{2.300508in}}%
\pgfpathlineto{\pgfqpoint{7.787910in}{2.307309in}}%
\pgfpathlineto{\pgfqpoint{7.778601in}{2.314111in}}%
\pgfpathlineto{\pgfqpoint{7.769292in}{2.320913in}}%
\pgfpathlineto{\pgfqpoint{7.759982in}{2.327715in}}%
\pgfpathlineto{\pgfqpoint{7.750673in}{2.334517in}}%
\pgfpathlineto{\pgfqpoint{7.741364in}{2.341318in}}%
\pgfpathlineto{\pgfqpoint{7.732054in}{2.348120in}}%
\pgfpathlineto{\pgfqpoint{7.722745in}{2.354922in}}%
\pgfpathlineto{\pgfqpoint{7.713436in}{2.361724in}}%
\pgfpathlineto{\pgfqpoint{7.704126in}{2.368526in}}%
\pgfpathlineto{\pgfqpoint{7.694817in}{2.375327in}}%
\pgfpathlineto{\pgfqpoint{7.685508in}{2.382129in}}%
\pgfpathlineto{\pgfqpoint{7.676199in}{2.388931in}}%
\pgfpathlineto{\pgfqpoint{7.666889in}{2.395733in}}%
\pgfpathlineto{\pgfqpoint{7.657580in}{2.402535in}}%
\pgfpathlineto{\pgfqpoint{7.648271in}{2.409336in}}%
\pgfpathlineto{\pgfqpoint{7.638961in}{2.416138in}}%
\pgfpathlineto{\pgfqpoint{7.629652in}{2.422940in}}%
\pgfpathlineto{\pgfqpoint{7.620343in}{2.429742in}}%
\pgfpathlineto{\pgfqpoint{7.611033in}{2.436544in}}%
\pgfpathlineto{\pgfqpoint{7.601724in}{2.443345in}}%
\pgfpathlineto{\pgfqpoint{7.592415in}{2.450147in}}%
\pgfpathlineto{\pgfqpoint{7.583105in}{2.456949in}}%
\pgfpathlineto{\pgfqpoint{7.573796in}{2.463751in}}%
\pgfpathlineto{\pgfqpoint{7.564487in}{2.470553in}}%
\pgfpathlineto{\pgfqpoint{7.555177in}{2.477354in}}%
\pgfpathlineto{\pgfqpoint{7.545868in}{2.484156in}}%
\pgfpathlineto{\pgfqpoint{7.536559in}{2.490958in}}%
\pgfpathlineto{\pgfqpoint{7.527250in}{2.497760in}}%
\pgfpathlineto{\pgfqpoint{7.517940in}{2.504562in}}%
\pgfpathlineto{\pgfqpoint{7.508631in}{2.511363in}}%
\pgfpathlineto{\pgfqpoint{7.499322in}{2.518165in}}%
\pgfpathlineto{\pgfqpoint{7.490012in}{2.524967in}}%
\pgfpathlineto{\pgfqpoint{7.480703in}{2.531769in}}%
\pgfpathlineto{\pgfqpoint{7.471394in}{2.538571in}}%
\pgfpathlineto{\pgfqpoint{7.462084in}{2.545372in}}%
\pgfpathlineto{\pgfqpoint{7.452775in}{2.552174in}}%
\pgfpathlineto{\pgfqpoint{7.443466in}{2.558976in}}%
\pgfpathlineto{\pgfqpoint{7.434156in}{2.565778in}}%
\pgfpathlineto{\pgfqpoint{7.424847in}{2.572580in}}%
\pgfpathlineto{\pgfqpoint{7.415538in}{2.579381in}}%
\pgfpathlineto{\pgfqpoint{7.406229in}{2.586183in}}%
\pgfpathlineto{\pgfqpoint{7.396919in}{2.592985in}}%
\pgfpathlineto{\pgfqpoint{7.387610in}{2.599787in}}%
\pgfpathlineto{\pgfqpoint{7.378301in}{2.606589in}}%
\pgfpathlineto{\pgfqpoint{7.368991in}{2.613390in}}%
\pgfpathlineto{\pgfqpoint{7.359682in}{2.620192in}}%
\pgfpathlineto{\pgfqpoint{7.350373in}{2.626994in}}%
\pgfpathlineto{\pgfqpoint{7.341063in}{2.633796in}}%
\pgfpathlineto{\pgfqpoint{7.331754in}{2.640598in}}%
\pgfpathlineto{\pgfqpoint{7.322445in}{2.647399in}}%
\pgfpathlineto{\pgfqpoint{7.313135in}{2.654201in}}%
\pgfpathlineto{\pgfqpoint{7.303826in}{2.661003in}}%
\pgfpathlineto{\pgfqpoint{7.294517in}{2.667805in}}%
\pgfpathlineto{\pgfqpoint{7.285208in}{2.674607in}}%
\pgfpathlineto{\pgfqpoint{7.275898in}{2.681408in}}%
\pgfpathlineto{\pgfqpoint{7.266589in}{2.688210in}}%
\pgfpathlineto{\pgfqpoint{7.257280in}{2.695012in}}%
\pgfpathlineto{\pgfqpoint{7.247970in}{2.701814in}}%
\pgfpathlineto{\pgfqpoint{7.238661in}{2.708616in}}%
\pgfpathlineto{\pgfqpoint{7.229352in}{2.715417in}}%
\pgfpathlineto{\pgfqpoint{7.220042in}{2.722219in}}%
\pgfpathlineto{\pgfqpoint{7.210733in}{2.729021in}}%
\pgfpathlineto{\pgfqpoint{7.201424in}{2.735823in}}%
\pgfpathlineto{\pgfqpoint{7.192114in}{2.742625in}}%
\pgfpathlineto{\pgfqpoint{7.182805in}{2.749427in}}%
\pgfpathlineto{\pgfqpoint{7.173496in}{2.756228in}}%
\pgfpathlineto{\pgfqpoint{7.164186in}{2.763030in}}%
\pgfpathlineto{\pgfqpoint{7.154877in}{2.769832in}}%
\pgfpathlineto{\pgfqpoint{7.145568in}{2.776634in}}%
\pgfpathlineto{\pgfqpoint{7.136259in}{2.783436in}}%
\pgfpathlineto{\pgfqpoint{7.126949in}{2.790237in}}%
\pgfpathlineto{\pgfqpoint{7.117640in}{2.797039in}}%
\pgfpathlineto{\pgfqpoint{7.108331in}{2.803841in}}%
\pgfpathlineto{\pgfqpoint{7.099021in}{2.810643in}}%
\pgfpathlineto{\pgfqpoint{7.089712in}{2.817445in}}%
\pgfpathlineto{\pgfqpoint{7.080403in}{2.824246in}}%
\pgfpathlineto{\pgfqpoint{7.071093in}{2.831048in}}%
\pgfpathlineto{\pgfqpoint{7.061784in}{2.837850in}}%
\pgfpathlineto{\pgfqpoint{7.052475in}{2.844652in}}%
\pgfpathlineto{\pgfqpoint{7.043165in}{2.851454in}}%
\pgfpathlineto{\pgfqpoint{7.033856in}{2.858255in}}%
\pgfpathlineto{\pgfqpoint{7.024547in}{2.865057in}}%
\pgfpathlineto{\pgfqpoint{7.015238in}{2.871859in}}%
\pgfpathlineto{\pgfqpoint{7.005928in}{2.878661in}}%
\pgfpathlineto{\pgfqpoint{6.996619in}{2.885463in}}%
\pgfpathlineto{\pgfqpoint{6.987310in}{2.892264in}}%
\pgfpathlineto{\pgfqpoint{6.978000in}{2.899066in}}%
\pgfpathlineto{\pgfqpoint{6.968691in}{2.905868in}}%
\pgfpathlineto{\pgfqpoint{6.959382in}{2.912670in}}%
\pgfpathlineto{\pgfqpoint{6.950072in}{2.919472in}}%
\pgfpathlineto{\pgfqpoint{6.940763in}{2.926273in}}%
\pgfpathlineto{\pgfqpoint{6.931454in}{2.933075in}}%
\pgfpathlineto{\pgfqpoint{6.929146in}{2.934762in}}%
\pgfpathlineto{\pgfqpoint{6.922914in}{2.939315in}}%
\pgfpathlineto{\pgfqpoint{6.922144in}{2.939877in}}%
\pgfpathlineto{\pgfqpoint{6.916682in}{2.943868in}}%
\pgfpathlineto{\pgfqpoint{6.912835in}{2.946679in}}%
\pgfpathlineto{\pgfqpoint{6.910450in}{2.948421in}}%
\pgfpathlineto{\pgfqpoint{6.904218in}{2.952975in}}%
\pgfpathlineto{\pgfqpoint{6.903526in}{2.953481in}}%
\pgfpathlineto{\pgfqpoint{6.897986in}{2.957528in}}%
\pgfpathlineto{\pgfqpoint{6.894217in}{2.960282in}}%
\pgfpathlineto{\pgfqpoint{6.891755in}{2.962081in}}%
\pgfpathlineto{\pgfqpoint{6.885523in}{2.966634in}}%
\pgfpathlineto{\pgfqpoint{6.884907in}{2.967084in}}%
\pgfpathlineto{\pgfqpoint{6.879291in}{2.971188in}}%
\pgfpathlineto{\pgfqpoint{6.875598in}{2.973886in}}%
\pgfpathlineto{\pgfqpoint{6.873059in}{2.975741in}}%
\pgfpathlineto{\pgfqpoint{6.866827in}{2.980294in}}%
\pgfpathlineto{\pgfqpoint{6.866289in}{2.980688in}}%
\pgfpathlineto{\pgfqpoint{6.860595in}{2.984848in}}%
\pgfpathlineto{\pgfqpoint{6.856979in}{2.987490in}}%
\pgfpathlineto{\pgfqpoint{6.854363in}{2.989401in}}%
\pgfpathlineto{\pgfqpoint{6.848132in}{2.993954in}}%
\pgfpathlineto{\pgfqpoint{6.847670in}{2.994291in}}%
\pgfpathlineto{\pgfqpoint{6.841900in}{2.998507in}}%
\pgfpathlineto{\pgfqpoint{6.838361in}{3.001093in}}%
\pgfpathlineto{\pgfqpoint{6.835668in}{3.003061in}}%
\pgfpathlineto{\pgfqpoint{6.829436in}{3.007614in}}%
\pgfpathlineto{\pgfqpoint{6.829051in}{3.007895in}}%
\pgfpathlineto{\pgfqpoint{6.823204in}{3.012167in}}%
\pgfpathlineto{\pgfqpoint{6.819742in}{3.014697in}}%
\pgfpathlineto{\pgfqpoint{6.816972in}{3.016720in}}%
\pgfpathlineto{\pgfqpoint{6.810740in}{3.021274in}}%
\pgfpathlineto{\pgfqpoint{6.810433in}{3.021499in}}%
\pgfpathlineto{\pgfqpoint{6.804509in}{3.025827in}}%
\pgfpathlineto{\pgfqpoint{6.801123in}{3.028300in}}%
\pgfpathlineto{\pgfqpoint{6.798277in}{3.030380in}}%
\pgfpathlineto{\pgfqpoint{6.792045in}{3.034934in}}%
\pgfpathlineto{\pgfqpoint{6.791814in}{3.035102in}}%
\pgfpathlineto{\pgfqpoint{6.785813in}{3.039487in}}%
\pgfpathlineto{\pgfqpoint{6.782505in}{3.041904in}}%
\pgfpathlineto{\pgfqpoint{6.779581in}{3.044040in}}%
\pgfpathlineto{\pgfqpoint{6.773349in}{3.048593in}}%
\pgfpathlineto{\pgfqpoint{6.773196in}{3.048706in}}%
\pgfpathlineto{\pgfqpoint{6.767118in}{3.053147in}}%
\pgfpathlineto{\pgfqpoint{6.763886in}{3.055508in}}%
\pgfpathlineto{\pgfqpoint{6.760886in}{3.057700in}}%
\pgfpathlineto{\pgfqpoint{6.754654in}{3.062253in}}%
\pgfpathlineto{\pgfqpoint{6.754577in}{3.062309in}}%
\pgfpathlineto{\pgfqpoint{6.748422in}{3.066806in}}%
\pgfpathlineto{\pgfqpoint{6.745268in}{3.069111in}}%
\pgfpathlineto{\pgfqpoint{6.742190in}{3.071360in}}%
\pgfpathlineto{\pgfqpoint{6.735958in}{3.075913in}}%
\pgfpathlineto{\pgfqpoint{6.735958in}{3.075913in}}%
\pgfpathlineto{\pgfqpoint{6.729726in}{3.080466in}}%
\pgfpathlineto{\pgfqpoint{6.726649in}{3.082715in}}%
\pgfpathlineto{\pgfqpoint{6.723495in}{3.085020in}}%
\pgfpathlineto{\pgfqpoint{6.717340in}{3.089517in}}%
\pgfpathlineto{\pgfqpoint{6.717263in}{3.089573in}}%
\pgfpathlineto{\pgfqpoint{6.711031in}{3.094126in}}%
\pgfpathlineto{\pgfqpoint{6.708030in}{3.096318in}}%
\pgfpathlineto{\pgfqpoint{6.704799in}{3.098679in}}%
\pgfpathlineto{\pgfqpoint{6.698721in}{3.103120in}}%
\pgfpathlineto{\pgfqpoint{6.698567in}{3.103233in}}%
\pgfpathlineto{\pgfqpoint{6.692335in}{3.107786in}}%
\pgfpathlineto{\pgfqpoint{6.689412in}{3.109922in}}%
\pgfpathlineto{\pgfqpoint{6.686103in}{3.112339in}}%
\pgfpathlineto{\pgfqpoint{6.680102in}{3.116724in}}%
\pgfpathlineto{\pgfqpoint{6.679872in}{3.116892in}}%
\pgfpathlineto{\pgfqpoint{6.673640in}{3.121446in}}%
\pgfpathlineto{\pgfqpoint{6.670793in}{3.123526in}}%
\pgfpathlineto{\pgfqpoint{6.667408in}{3.125999in}}%
\pgfpathlineto{\pgfqpoint{6.661484in}{3.130327in}}%
\pgfpathlineto{\pgfqpoint{6.661176in}{3.130552in}}%
\pgfpathlineto{\pgfqpoint{6.654944in}{3.135106in}}%
\pgfpathlineto{\pgfqpoint{6.652174in}{3.137129in}}%
\pgfpathlineto{\pgfqpoint{6.648712in}{3.139659in}}%
\pgfpathlineto{\pgfqpoint{6.642865in}{3.143931in}}%
\pgfpathlineto{\pgfqpoint{6.642480in}{3.144212in}}%
\pgfpathlineto{\pgfqpoint{6.636249in}{3.148765in}}%
\pgfpathlineto{\pgfqpoint{6.633556in}{3.150733in}}%
\pgfpathlineto{\pgfqpoint{6.630017in}{3.153319in}}%
\pgfpathlineto{\pgfqpoint{6.624247in}{3.157535in}}%
\pgfpathlineto{\pgfqpoint{6.623785in}{3.157872in}}%
\pgfpathlineto{\pgfqpoint{6.617553in}{3.162425in}}%
\pgfpathlineto{\pgfqpoint{6.614937in}{3.164336in}}%
\pgfpathlineto{\pgfqpoint{6.611321in}{3.166978in}}%
\pgfpathlineto{\pgfqpoint{6.605628in}{3.171138in}}%
\pgfpathlineto{\pgfqpoint{6.605089in}{3.171532in}}%
\pgfpathlineto{\pgfqpoint{6.598858in}{3.176085in}}%
\pgfpathlineto{\pgfqpoint{6.596319in}{3.177940in}}%
\pgfpathlineto{\pgfqpoint{6.592626in}{3.180638in}}%
\pgfpathlineto{\pgfqpoint{6.587009in}{3.184742in}}%
\pgfpathlineto{\pgfqpoint{6.586394in}{3.185192in}}%
\pgfpathlineto{\pgfqpoint{6.580162in}{3.189745in}}%
\pgfpathlineto{\pgfqpoint{6.577700in}{3.191544in}}%
\pgfpathlineto{\pgfqpoint{6.573930in}{3.194298in}}%
\pgfpathlineto{\pgfqpoint{6.568391in}{3.198345in}}%
\pgfpathlineto{\pgfqpoint{6.567698in}{3.198851in}}%
\pgfpathlineto{\pgfqpoint{6.561466in}{3.203405in}}%
\pgfpathlineto{\pgfqpoint{6.559081in}{3.205147in}}%
\pgfpathlineto{\pgfqpoint{6.555235in}{3.207958in}}%
\pgfpathlineto{\pgfqpoint{6.549772in}{3.211949in}}%
\pgfpathlineto{\pgfqpoint{6.549003in}{3.212511in}}%
\pgfpathlineto{\pgfqpoint{6.542771in}{3.217064in}}%
\pgfpathlineto{\pgfqpoint{6.540463in}{3.218751in}}%
\pgfpathlineto{\pgfqpoint{6.536539in}{3.221618in}}%
\pgfpathlineto{\pgfqpoint{6.531153in}{3.225553in}}%
\pgfpathlineto{\pgfqpoint{6.530307in}{3.226171in}}%
\pgfpathlineto{\pgfqpoint{6.524075in}{3.230724in}}%
\pgfpathlineto{\pgfqpoint{6.521844in}{3.232354in}}%
\pgfpathlineto{\pgfqpoint{6.517843in}{3.235278in}}%
\pgfpathlineto{\pgfqpoint{6.512535in}{3.239156in}}%
\pgfpathlineto{\pgfqpoint{6.511612in}{3.239831in}}%
\pgfpathlineto{\pgfqpoint{6.505380in}{3.244384in}}%
\pgfpathlineto{\pgfqpoint{6.503226in}{3.245958in}}%
\pgfpathlineto{\pgfqpoint{6.499148in}{3.248937in}}%
\pgfpathlineto{\pgfqpoint{6.493916in}{3.252760in}}%
\pgfpathlineto{\pgfqpoint{6.492916in}{3.253491in}}%
\pgfpathlineto{\pgfqpoint{6.486684in}{3.258044in}}%
\pgfpathlineto{\pgfqpoint{6.484607in}{3.259562in}}%
\pgfpathlineto{\pgfqpoint{6.480452in}{3.262597in}}%
\pgfpathlineto{\pgfqpoint{6.475298in}{3.266363in}}%
\pgfpathlineto{\pgfqpoint{6.474220in}{3.267150in}}%
\pgfpathlineto{\pgfqpoint{6.467989in}{3.271704in}}%
\pgfpathlineto{\pgfqpoint{6.465988in}{3.273165in}}%
\pgfpathlineto{\pgfqpoint{6.461757in}{3.276257in}}%
\pgfpathlineto{\pgfqpoint{6.456679in}{3.279967in}}%
\pgfpathlineto{\pgfqpoint{6.455525in}{3.280810in}}%
\pgfpathlineto{\pgfqpoint{6.449293in}{3.285364in}}%
\pgfpathlineto{\pgfqpoint{6.447370in}{3.286769in}}%
\pgfpathlineto{\pgfqpoint{6.443061in}{3.289917in}}%
\pgfpathlineto{\pgfqpoint{6.438060in}{3.293571in}}%
\pgfpathlineto{\pgfqpoint{6.436829in}{3.294470in}}%
\pgfpathlineto{\pgfqpoint{6.430598in}{3.299023in}}%
\pgfpathlineto{\pgfqpoint{6.428751in}{3.300372in}}%
\pgfpathlineto{\pgfqpoint{6.424366in}{3.303577in}}%
\pgfpathlineto{\pgfqpoint{6.419442in}{3.307174in}}%
\pgfpathlineto{\pgfqpoint{6.418134in}{3.308130in}}%
\pgfpathlineto{\pgfqpoint{6.411902in}{3.312683in}}%
\pgfpathlineto{\pgfqpoint{6.410132in}{3.313976in}}%
\pgfpathlineto{\pgfqpoint{6.405670in}{3.317236in}}%
\pgfpathlineto{\pgfqpoint{6.400823in}{3.320778in}}%
\pgfpathlineto{\pgfqpoint{6.399438in}{3.321790in}}%
\pgfpathlineto{\pgfqpoint{6.393206in}{3.326343in}}%
\pgfpathlineto{\pgfqpoint{6.391514in}{3.327580in}}%
\pgfpathlineto{\pgfqpoint{6.386975in}{3.330896in}}%
\pgfpathlineto{\pgfqpoint{6.382205in}{3.334381in}}%
\pgfpathlineto{\pgfqpoint{6.380743in}{3.335450in}}%
\pgfpathlineto{\pgfqpoint{6.374511in}{3.340003in}}%
\pgfpathlineto{\pgfqpoint{6.372895in}{3.341183in}}%
\pgfpathlineto{\pgfqpoint{6.368279in}{3.344556in}}%
\pgfpathlineto{\pgfqpoint{6.363586in}{3.347985in}}%
\pgfpathlineto{\pgfqpoint{6.362047in}{3.349109in}}%
\pgfpathlineto{\pgfqpoint{6.355815in}{3.353663in}}%
\pgfpathlineto{\pgfqpoint{6.354277in}{3.354787in}}%
\pgfpathlineto{\pgfqpoint{6.349583in}{3.358216in}}%
\pgfpathlineto{\pgfqpoint{6.344967in}{3.361589in}}%
\pgfpathlineto{\pgfqpoint{6.343352in}{3.362769in}}%
\pgfpathlineto{\pgfqpoint{6.337120in}{3.367322in}}%
\pgfpathlineto{\pgfqpoint{6.335658in}{3.368390in}}%
\pgfpathlineto{\pgfqpoint{6.330888in}{3.371876in}}%
\pgfpathlineto{\pgfqpoint{6.326349in}{3.375192in}}%
\pgfpathlineto{\pgfqpoint{6.324656in}{3.376429in}}%
\pgfpathlineto{\pgfqpoint{6.318424in}{3.380982in}}%
\pgfpathlineto{\pgfqpoint{6.317039in}{3.381994in}}%
\pgfpathlineto{\pgfqpoint{6.312192in}{3.385536in}}%
\pgfpathlineto{\pgfqpoint{6.307730in}{3.388796in}}%
\pgfpathlineto{\pgfqpoint{6.305960in}{3.390089in}}%
\pgfpathlineto{\pgfqpoint{6.299729in}{3.394642in}}%
\pgfpathlineto{\pgfqpoint{6.298421in}{3.395598in}}%
\pgfpathlineto{\pgfqpoint{6.293497in}{3.399195in}}%
\pgfpathlineto{\pgfqpoint{6.289111in}{3.402399in}}%
\pgfpathlineto{\pgfqpoint{6.287265in}{3.403749in}}%
\pgfpathlineto{\pgfqpoint{6.281033in}{3.408302in}}%
\pgfpathlineto{\pgfqpoint{6.279802in}{3.409201in}}%
\pgfpathlineto{\pgfqpoint{6.274801in}{3.412855in}}%
\pgfpathlineto{\pgfqpoint{6.270493in}{3.416003in}}%
\pgfpathlineto{\pgfqpoint{6.268569in}{3.417408in}}%
\pgfpathlineto{\pgfqpoint{6.262338in}{3.421962in}}%
\pgfpathlineto{\pgfqpoint{6.261183in}{3.422805in}}%
\pgfpathlineto{\pgfqpoint{6.256106in}{3.426515in}}%
\pgfpathlineto{\pgfqpoint{6.251874in}{3.429607in}}%
\pgfpathlineto{\pgfqpoint{6.249874in}{3.431068in}}%
\pgfpathlineto{\pgfqpoint{6.243642in}{3.435621in}}%
\pgfpathlineto{\pgfqpoint{6.242565in}{3.436408in}}%
\pgfpathlineto{\pgfqpoint{6.237410in}{3.440175in}}%
\pgfpathlineto{\pgfqpoint{6.233256in}{3.443210in}}%
\pgfpathlineto{\pgfqpoint{6.231178in}{3.444728in}}%
\pgfpathlineto{\pgfqpoint{6.224946in}{3.449281in}}%
\pgfpathlineto{\pgfqpoint{6.223946in}{3.450012in}}%
\pgfpathlineto{\pgfqpoint{6.218715in}{3.453835in}}%
\pgfpathlineto{\pgfqpoint{6.214637in}{3.456814in}}%
\pgfpathlineto{\pgfqpoint{6.212483in}{3.458388in}}%
\pgfpathlineto{\pgfqpoint{6.206251in}{3.462941in}}%
\pgfpathlineto{\pgfqpoint{6.205328in}{3.463616in}}%
\pgfpathlineto{\pgfqpoint{6.200019in}{3.467494in}}%
\pgfpathlineto{\pgfqpoint{6.196018in}{3.470417in}}%
\pgfpathlineto{\pgfqpoint{6.193787in}{3.472048in}}%
\pgfpathlineto{\pgfqpoint{6.187555in}{3.476601in}}%
\pgfpathlineto{\pgfqpoint{6.186709in}{3.477219in}}%
\pgfpathlineto{\pgfqpoint{6.181323in}{3.481154in}}%
\pgfpathlineto{\pgfqpoint{6.177400in}{3.484021in}}%
\pgfpathlineto{\pgfqpoint{6.175092in}{3.485707in}}%
\pgfpathlineto{\pgfqpoint{6.168860in}{3.490261in}}%
\pgfpathlineto{\pgfqpoint{6.168090in}{3.490823in}}%
\pgfpathlineto{\pgfqpoint{6.162628in}{3.494814in}}%
\pgfpathlineto{\pgfqpoint{6.158781in}{3.497625in}}%
\pgfpathlineto{\pgfqpoint{6.156396in}{3.499367in}}%
\pgfpathlineto{\pgfqpoint{6.150164in}{3.503921in}}%
\pgfpathlineto{\pgfqpoint{6.149472in}{3.504427in}}%
\pgfpathlineto{\pgfqpoint{6.143932in}{3.508474in}}%
\pgfpathlineto{\pgfqpoint{6.140162in}{3.511228in}}%
\pgfpathlineto{\pgfqpoint{6.137701in}{3.513027in}}%
\pgfpathlineto{\pgfqpoint{6.131469in}{3.517580in}}%
\pgfpathlineto{\pgfqpoint{6.130853in}{3.518030in}}%
\pgfpathlineto{\pgfqpoint{6.125237in}{3.522134in}}%
\pgfpathlineto{\pgfqpoint{6.121544in}{3.524832in}}%
\pgfpathlineto{\pgfqpoint{6.119005in}{3.526687in}}%
\pgfpathlineto{\pgfqpoint{6.112773in}{3.531240in}}%
\pgfpathlineto{\pgfqpoint{6.112235in}{3.531634in}}%
\pgfpathlineto{\pgfqpoint{6.106541in}{3.535793in}}%
\pgfpathlineto{\pgfqpoint{6.102925in}{3.538436in}}%
\pgfpathlineto{\pgfqpoint{6.100309in}{3.540347in}}%
\pgfpathlineto{\pgfqpoint{6.094078in}{3.544900in}}%
\pgfpathlineto{\pgfqpoint{6.093616in}{3.545237in}}%
\pgfpathlineto{\pgfqpoint{6.087846in}{3.549453in}}%
\pgfpathlineto{\pgfqpoint{6.084307in}{3.552039in}}%
\pgfpathlineto{\pgfqpoint{6.081614in}{3.554007in}}%
\pgfpathlineto{\pgfqpoint{6.075382in}{3.558560in}}%
\pgfpathlineto{\pgfqpoint{6.074997in}{3.558841in}}%
\pgfpathlineto{\pgfqpoint{6.069150in}{3.563113in}}%
\pgfpathlineto{\pgfqpoint{6.065688in}{3.565643in}}%
\pgfpathlineto{\pgfqpoint{6.062918in}{3.567666in}}%
\pgfpathlineto{\pgfqpoint{6.056686in}{3.572220in}}%
\pgfpathlineto{\pgfqpoint{6.056379in}{3.572445in}}%
\pgfpathlineto{\pgfqpoint{6.050455in}{3.576773in}}%
\pgfpathlineto{\pgfqpoint{6.047069in}{3.579246in}}%
\pgfpathlineto{\pgfqpoint{6.044223in}{3.581326in}}%
\pgfpathlineto{\pgfqpoint{6.037991in}{3.585879in}}%
\pgfpathlineto{\pgfqpoint{6.037760in}{3.586048in}}%
\pgfpathlineto{\pgfqpoint{6.031759in}{3.590433in}}%
\pgfpathlineto{\pgfqpoint{6.028451in}{3.592850in}}%
\pgfpathlineto{\pgfqpoint{6.025527in}{3.594986in}}%
\pgfpathlineto{\pgfqpoint{6.019295in}{3.599539in}}%
\pgfpathlineto{\pgfqpoint{6.019141in}{3.599652in}}%
\pgfpathlineto{\pgfqpoint{6.013063in}{3.604093in}}%
\pgfpathlineto{\pgfqpoint{6.009832in}{3.606454in}}%
\pgfpathlineto{\pgfqpoint{6.006832in}{3.608646in}}%
\pgfpathlineto{\pgfqpoint{6.000600in}{3.613199in}}%
\pgfpathlineto{\pgfqpoint{6.000523in}{3.613255in}}%
\pgfpathlineto{\pgfqpoint{5.994368in}{3.617752in}}%
\pgfpathlineto{\pgfqpoint{5.991214in}{3.620057in}}%
\pgfpathlineto{\pgfqpoint{5.988136in}{3.622306in}}%
\pgfpathlineto{\pgfqpoint{5.981904in}{3.626859in}}%
\pgfpathlineto{\pgfqpoint{5.981904in}{3.626859in}}%
\pgfpathlineto{\pgfqpoint{5.975672in}{3.631412in}}%
\pgfpathlineto{\pgfqpoint{5.972595in}{3.633661in}}%
\pgfpathlineto{\pgfqpoint{5.969441in}{3.635965in}}%
\pgfpathlineto{\pgfqpoint{5.963286in}{3.640463in}}%
\pgfpathlineto{\pgfqpoint{5.963209in}{3.640519in}}%
\pgfpathlineto{\pgfqpoint{5.956977in}{3.645072in}}%
\pgfpathlineto{\pgfqpoint{5.953976in}{3.647264in}}%
\pgfpathlineto{\pgfqpoint{5.950745in}{3.649625in}}%
\pgfpathlineto{\pgfqpoint{5.944667in}{3.654066in}}%
\pgfpathlineto{\pgfqpoint{5.944513in}{3.654179in}}%
\pgfpathlineto{\pgfqpoint{5.938281in}{3.658732in}}%
\pgfpathlineto{\pgfqpoint{5.935358in}{3.660868in}}%
\pgfpathlineto{\pgfqpoint{5.932049in}{3.663285in}}%
\pgfpathlineto{\pgfqpoint{5.926048in}{3.667670in}}%
\pgfpathlineto{\pgfqpoint{5.925818in}{3.667838in}}%
\pgfpathlineto{\pgfqpoint{5.919586in}{3.672392in}}%
\pgfpathlineto{\pgfqpoint{5.916739in}{3.674472in}}%
\pgfpathlineto{\pgfqpoint{5.913354in}{3.676945in}}%
\pgfpathlineto{\pgfqpoint{5.907430in}{3.681273in}}%
\pgfpathlineto{\pgfqpoint{5.907122in}{3.681498in}}%
\pgfpathlineto{\pgfqpoint{5.900890in}{3.686051in}}%
\pgfpathlineto{\pgfqpoint{5.898120in}{3.688075in}}%
\pgfpathlineto{\pgfqpoint{5.894658in}{3.690605in}}%
\pgfpathlineto{\pgfqpoint{5.888811in}{3.694877in}}%
\pgfpathlineto{\pgfqpoint{5.888426in}{3.695158in}}%
\pgfpathlineto{\pgfqpoint{5.882195in}{3.699711in}}%
\pgfpathlineto{\pgfqpoint{5.879502in}{3.701679in}}%
\pgfpathlineto{\pgfqpoint{5.875963in}{3.704265in}}%
\pgfpathlineto{\pgfqpoint{5.870192in}{3.708481in}}%
\pgfpathlineto{\pgfqpoint{5.869731in}{3.708818in}}%
\pgfpathlineto{\pgfqpoint{5.863499in}{3.713371in}}%
\pgfpathlineto{\pgfqpoint{5.860883in}{3.715282in}}%
\pgfpathlineto{\pgfqpoint{5.857267in}{3.717924in}}%
\pgfpathlineto{\pgfqpoint{5.851574in}{3.722084in}}%
\pgfpathlineto{\pgfqpoint{5.851035in}{3.722478in}}%
\pgfpathlineto{\pgfqpoint{5.844803in}{3.727031in}}%
\pgfpathlineto{\pgfqpoint{5.842265in}{3.728886in}}%
\pgfpathlineto{\pgfqpoint{5.838572in}{3.731584in}}%
\pgfpathlineto{\pgfqpoint{5.832955in}{3.735688in}}%
\pgfpathlineto{\pgfqpoint{5.832340in}{3.736137in}}%
\pgfpathlineto{\pgfqpoint{5.826108in}{3.740691in}}%
\pgfpathlineto{\pgfqpoint{5.823646in}{3.742490in}}%
\pgfpathlineto{\pgfqpoint{5.819876in}{3.745244in}}%
\pgfpathlineto{\pgfqpoint{5.814337in}{3.749291in}}%
\pgfpathlineto{\pgfqpoint{5.813644in}{3.749797in}}%
\pgfpathlineto{\pgfqpoint{5.807412in}{3.754351in}}%
\pgfpathlineto{\pgfqpoint{5.805027in}{3.756093in}}%
\pgfpathlineto{\pgfqpoint{5.801181in}{3.758904in}}%
\pgfpathlineto{\pgfqpoint{5.795718in}{3.762895in}}%
\pgfpathlineto{\pgfqpoint{5.794949in}{3.763457in}}%
\pgfpathlineto{\pgfqpoint{5.788717in}{3.768010in}}%
\pgfpathlineto{\pgfqpoint{5.786409in}{3.769697in}}%
\pgfpathlineto{\pgfqpoint{5.782485in}{3.772564in}}%
\pgfpathlineto{\pgfqpoint{5.777099in}{3.776499in}}%
\pgfpathlineto{\pgfqpoint{5.776253in}{3.777117in}}%
\pgfpathlineto{\pgfqpoint{5.770021in}{3.781670in}}%
\pgfpathlineto{\pgfqpoint{5.767790in}{3.783300in}}%
\pgfpathlineto{\pgfqpoint{5.763789in}{3.786223in}}%
\pgfpathlineto{\pgfqpoint{5.758481in}{3.790102in}}%
\pgfpathlineto{\pgfqpoint{5.757558in}{3.790777in}}%
\pgfpathlineto{\pgfqpoint{5.751326in}{3.795330in}}%
\pgfpathlineto{\pgfqpoint{5.749171in}{3.796904in}}%
\pgfpathlineto{\pgfqpoint{5.745094in}{3.799883in}}%
\pgfpathlineto{\pgfqpoint{5.739862in}{3.803706in}}%
\pgfpathlineto{\pgfqpoint{5.738862in}{3.804437in}}%
\pgfpathlineto{\pgfqpoint{5.732630in}{3.808990in}}%
\pgfpathlineto{\pgfqpoint{5.730553in}{3.810508in}}%
\pgfpathlineto{\pgfqpoint{5.726398in}{3.813543in}}%
\pgfpathlineto{\pgfqpoint{5.721244in}{3.817309in}}%
\pgfpathlineto{\pgfqpoint{5.720166in}{3.818096in}}%
\pgfpathlineto{\pgfqpoint{5.713935in}{3.822650in}}%
\pgfpathlineto{\pgfqpoint{5.711934in}{3.824111in}}%
\pgfpathlineto{\pgfqpoint{5.707703in}{3.827203in}}%
\pgfpathlineto{\pgfqpoint{5.702625in}{3.830913in}}%
\pgfpathlineto{\pgfqpoint{5.701471in}{3.831756in}}%
\pgfpathlineto{\pgfqpoint{5.695239in}{3.836309in}}%
\pgfpathlineto{\pgfqpoint{5.693316in}{3.837715in}}%
\pgfpathlineto{\pgfqpoint{5.689007in}{3.840863in}}%
\pgfpathlineto{\pgfqpoint{5.684006in}{3.844517in}}%
\pgfpathlineto{\pgfqpoint{5.682775in}{3.845416in}}%
\pgfpathlineto{\pgfqpoint{5.676543in}{3.849969in}}%
\pgfpathlineto{\pgfqpoint{5.674697in}{3.851318in}}%
\pgfpathlineto{\pgfqpoint{5.670312in}{3.854523in}}%
\pgfpathlineto{\pgfqpoint{5.665388in}{3.858120in}}%
\pgfpathlineto{\pgfqpoint{5.664080in}{3.859076in}}%
\pgfpathlineto{\pgfqpoint{5.657848in}{3.863629in}}%
\pgfpathlineto{\pgfqpoint{5.656078in}{3.864922in}}%
\pgfpathlineto{\pgfqpoint{5.651616in}{3.868182in}}%
\pgfpathlineto{\pgfqpoint{5.646769in}{3.871724in}}%
\pgfpathlineto{\pgfqpoint{5.645384in}{3.872736in}}%
\pgfpathlineto{\pgfqpoint{5.639152in}{3.877289in}}%
\pgfpathlineto{\pgfqpoint{5.637460in}{3.878526in}}%
\pgfpathlineto{\pgfqpoint{5.632921in}{3.881842in}}%
\pgfpathlineto{\pgfqpoint{5.628150in}{3.885327in}}%
\pgfpathlineto{\pgfqpoint{5.626689in}{3.886395in}}%
\pgfpathlineto{\pgfqpoint{5.620457in}{3.890949in}}%
\pgfpathlineto{\pgfqpoint{5.618841in}{3.892129in}}%
\pgfpathlineto{\pgfqpoint{5.614225in}{3.895502in}}%
\pgfpathlineto{\pgfqpoint{5.609532in}{3.898931in}}%
\pgfpathlineto{\pgfqpoint{5.607993in}{3.900055in}}%
\pgfpathlineto{\pgfqpoint{5.601761in}{3.904609in}}%
\pgfpathlineto{\pgfqpoint{5.600223in}{3.905733in}}%
\pgfpathlineto{\pgfqpoint{5.595529in}{3.909162in}}%
\pgfpathlineto{\pgfqpoint{5.590913in}{3.912535in}}%
\pgfpathlineto{\pgfqpoint{5.589298in}{3.913715in}}%
\pgfpathlineto{\pgfqpoint{5.583066in}{3.918268in}}%
\pgfpathlineto{\pgfqpoint{5.581604in}{3.919336in}}%
\pgfpathlineto{\pgfqpoint{5.576834in}{3.922822in}}%
\pgfpathlineto{\pgfqpoint{5.572295in}{3.926138in}}%
\pgfpathlineto{\pgfqpoint{5.570602in}{3.927375in}}%
\pgfpathlineto{\pgfqpoint{5.564370in}{3.931928in}}%
\pgfpathlineto{\pgfqpoint{5.562985in}{3.932940in}}%
\pgfpathlineto{\pgfqpoint{5.558138in}{3.936481in}}%
\pgfpathlineto{\pgfqpoint{5.553676in}{3.939742in}}%
\pgfpathlineto{\pgfqpoint{5.551906in}{3.941035in}}%
\pgfpathlineto{\pgfqpoint{5.545675in}{3.945588in}}%
\pgfpathlineto{\pgfqpoint{5.544367in}{3.946544in}}%
\pgfpathlineto{\pgfqpoint{5.539443in}{3.950141in}}%
\pgfpathlineto{\pgfqpoint{5.535057in}{3.953345in}}%
\pgfpathlineto{\pgfqpoint{5.533211in}{3.954695in}}%
\pgfpathlineto{\pgfqpoint{5.526979in}{3.959248in}}%
\pgfpathlineto{\pgfqpoint{5.525748in}{3.960147in}}%
\pgfpathlineto{\pgfqpoint{5.520747in}{3.963801in}}%
\pgfpathlineto{\pgfqpoint{5.516439in}{3.966949in}}%
\pgfpathlineto{\pgfqpoint{5.514515in}{3.968354in}}%
\pgfpathlineto{\pgfqpoint{5.508283in}{3.972908in}}%
\pgfpathlineto{\pgfqpoint{5.507129in}{3.973751in}}%
\pgfpathlineto{\pgfqpoint{5.502052in}{3.977461in}}%
\pgfpathlineto{\pgfqpoint{5.497820in}{3.980553in}}%
\pgfpathlineto{\pgfqpoint{5.495820in}{3.982014in}}%
\pgfpathlineto{\pgfqpoint{5.489588in}{3.986567in}}%
\pgfpathlineto{\pgfqpoint{5.488511in}{3.987354in}}%
\pgfpathlineto{\pgfqpoint{5.483356in}{3.991121in}}%
\pgfpathlineto{\pgfqpoint{5.479202in}{3.994156in}}%
\pgfpathlineto{\pgfqpoint{5.477124in}{3.995674in}}%
\pgfpathlineto{\pgfqpoint{5.470892in}{4.000227in}}%
\pgfpathlineto{\pgfqpoint{5.469892in}{4.000958in}}%
\pgfpathlineto{\pgfqpoint{5.464661in}{4.004781in}}%
\pgfpathlineto{\pgfqpoint{5.460583in}{4.007760in}}%
\pgfpathlineto{\pgfqpoint{5.458429in}{4.009334in}}%
\pgfpathlineto{\pgfqpoint{5.452197in}{4.013887in}}%
\pgfpathlineto{\pgfqpoint{5.451274in}{4.014562in}}%
\pgfpathlineto{\pgfqpoint{5.445965in}{4.018440in}}%
\pgfpathlineto{\pgfqpoint{5.441964in}{4.021363in}}%
\pgfpathlineto{\pgfqpoint{5.439733in}{4.022994in}}%
\pgfpathlineto{\pgfqpoint{5.433501in}{4.027547in}}%
\pgfpathlineto{\pgfqpoint{5.432655in}{4.028165in}}%
\pgfpathlineto{\pgfqpoint{5.427269in}{4.032100in}}%
\pgfpathlineto{\pgfqpoint{5.423346in}{4.034967in}}%
\pgfpathlineto{\pgfqpoint{5.421038in}{4.036653in}}%
\pgfpathlineto{\pgfqpoint{5.414806in}{4.041207in}}%
\pgfpathlineto{\pgfqpoint{5.414036in}{4.041769in}}%
\pgfpathlineto{\pgfqpoint{5.408574in}{4.045760in}}%
\pgfpathlineto{\pgfqpoint{5.404727in}{4.048571in}}%
\pgfpathlineto{\pgfqpoint{5.402342in}{4.050313in}}%
\pgfpathlineto{\pgfqpoint{5.396110in}{4.054867in}}%
\pgfpathlineto{\pgfqpoint{5.395418in}{4.055372in}}%
\pgfpathlineto{\pgfqpoint{5.389878in}{4.059420in}}%
\pgfpathlineto{\pgfqpoint{5.386108in}{4.062174in}}%
\pgfpathlineto{\pgfqpoint{5.383646in}{4.063973in}}%
\pgfpathlineto{\pgfqpoint{5.377415in}{4.068526in}}%
\pgfpathlineto{\pgfqpoint{5.376799in}{4.068976in}}%
\pgfpathlineto{\pgfqpoint{5.371183in}{4.073080in}}%
\pgfpathlineto{\pgfqpoint{5.367490in}{4.075778in}}%
\pgfpathlineto{\pgfqpoint{5.364951in}{4.077633in}}%
\pgfpathlineto{\pgfqpoint{5.358719in}{4.082186in}}%
\pgfpathlineto{\pgfqpoint{5.358180in}{4.082580in}}%
\pgfpathlineto{\pgfqpoint{5.352487in}{4.086739in}}%
\pgfpathlineto{\pgfqpoint{5.348871in}{4.089381in}}%
\pgfpathlineto{\pgfqpoint{5.346255in}{4.091293in}}%
\pgfpathlineto{\pgfqpoint{5.340023in}{4.095846in}}%
\pgfpathlineto{\pgfqpoint{5.339562in}{4.096183in}}%
\pgfpathlineto{\pgfqpoint{5.333792in}{4.100399in}}%
\pgfpathlineto{\pgfqpoint{5.330253in}{4.102985in}}%
\pgfpathlineto{\pgfqpoint{5.327560in}{4.104953in}}%
\pgfpathlineto{\pgfqpoint{5.321328in}{4.109506in}}%
\pgfpathlineto{\pgfqpoint{5.320943in}{4.109787in}}%
\pgfpathlineto{\pgfqpoint{5.315096in}{4.114059in}}%
\pgfpathlineto{\pgfqpoint{5.311634in}{4.116589in}}%
\pgfpathlineto{\pgfqpoint{5.308864in}{4.118612in}}%
\pgfpathlineto{\pgfqpoint{5.302632in}{4.123166in}}%
\pgfpathlineto{\pgfqpoint{5.302325in}{4.123390in}}%
\pgfpathlineto{\pgfqpoint{5.296401in}{4.127719in}}%
\pgfpathlineto{\pgfqpoint{5.293015in}{4.130192in}}%
\pgfpathlineto{\pgfqpoint{5.290169in}{4.132272in}}%
\pgfpathlineto{\pgfqpoint{5.283937in}{4.136825in}}%
\pgfpathlineto{\pgfqpoint{5.283706in}{4.136994in}}%
\pgfpathlineto{\pgfqpoint{5.277705in}{4.141379in}}%
\pgfpathlineto{\pgfqpoint{5.274397in}{4.143796in}}%
\pgfpathlineto{\pgfqpoint{5.271473in}{4.145932in}}%
\pgfpathlineto{\pgfqpoint{5.265241in}{4.150485in}}%
\pgfpathlineto{\pgfqpoint{5.265087in}{4.150598in}}%
\pgfpathlineto{\pgfqpoint{5.259009in}{4.155039in}}%
\pgfpathlineto{\pgfqpoint{5.255778in}{4.157399in}}%
\pgfpathlineto{\pgfqpoint{5.252778in}{4.159592in}}%
\pgfpathlineto{\pgfqpoint{5.246546in}{4.164145in}}%
\pgfpathlineto{\pgfqpoint{5.246469in}{4.164201in}}%
\pgfpathlineto{\pgfqpoint{5.240314in}{4.168698in}}%
\pgfpathlineto{\pgfqpoint{5.237159in}{4.171003in}}%
\pgfpathlineto{\pgfqpoint{5.234082in}{4.173252in}}%
\pgfpathlineto{\pgfqpoint{5.227850in}{4.177805in}}%
\pgfpathlineto{\pgfqpoint{5.227850in}{4.177805in}}%
\pgfpathlineto{\pgfqpoint{5.221618in}{4.182358in}}%
\pgfpathlineto{\pgfqpoint{5.218541in}{4.184607in}}%
\pgfpathlineto{\pgfqpoint{5.215386in}{4.186911in}}%
\pgfpathlineto{\pgfqpoint{5.209232in}{4.191408in}}%
\pgfpathlineto{\pgfqpoint{5.209155in}{4.191465in}}%
\pgfpathlineto{\pgfqpoint{5.202923in}{4.196018in}}%
\pgfpathlineto{\pgfqpoint{5.199922in}{4.198210in}}%
\pgfpathlineto{\pgfqpoint{5.196691in}{4.200571in}}%
\pgfpathlineto{\pgfqpoint{5.190613in}{4.205012in}}%
\pgfpathlineto{\pgfqpoint{5.190459in}{4.205125in}}%
\pgfpathlineto{\pgfqpoint{5.184227in}{4.209678in}}%
\pgfpathlineto{\pgfqpoint{5.181304in}{4.211814in}}%
\pgfpathlineto{\pgfqpoint{5.177995in}{4.214231in}}%
\pgfpathlineto{\pgfqpoint{5.171994in}{4.218616in}}%
\pgfpathlineto{\pgfqpoint{5.171763in}{4.218784in}}%
\pgfpathlineto{\pgfqpoint{5.165532in}{4.223338in}}%
\pgfpathlineto{\pgfqpoint{5.162685in}{4.225417in}}%
\pgfpathlineto{\pgfqpoint{5.159300in}{4.227891in}}%
\pgfpathlineto{\pgfqpoint{5.153376in}{4.232219in}}%
\pgfpathlineto{\pgfqpoint{5.153068in}{4.232444in}}%
\pgfpathlineto{\pgfqpoint{5.146836in}{4.236997in}}%
\pgfpathlineto{\pgfqpoint{5.144066in}{4.239021in}}%
\pgfpathlineto{\pgfqpoint{5.140604in}{4.241551in}}%
\pgfpathlineto{\pgfqpoint{5.134757in}{4.245823in}}%
\pgfpathlineto{\pgfqpoint{5.134372in}{4.246104in}}%
\pgfpathlineto{\pgfqpoint{5.128141in}{4.250657in}}%
\pgfpathlineto{\pgfqpoint{5.125448in}{4.252625in}}%
\pgfpathlineto{\pgfqpoint{5.121909in}{4.255211in}}%
\pgfpathlineto{\pgfqpoint{5.116138in}{4.259427in}}%
\pgfpathlineto{\pgfqpoint{5.115677in}{4.259764in}}%
\pgfpathlineto{\pgfqpoint{5.109445in}{4.264317in}}%
\pgfpathlineto{\pgfqpoint{5.106829in}{4.266228in}}%
\pgfpathlineto{\pgfqpoint{5.103213in}{4.268870in}}%
\pgfpathlineto{\pgfqpoint{5.097520in}{4.273030in}}%
\pgfpathlineto{\pgfqpoint{5.096981in}{4.273424in}}%
\pgfpathlineto{\pgfqpoint{5.090749in}{4.277977in}}%
\pgfpathlineto{\pgfqpoint{5.088211in}{4.279832in}}%
\pgfpathlineto{\pgfqpoint{5.084518in}{4.282530in}}%
\pgfpathlineto{\pgfqpoint{5.078901in}{4.286634in}}%
\pgfpathlineto{\pgfqpoint{5.078286in}{4.287083in}}%
\pgfpathlineto{\pgfqpoint{5.072054in}{4.291637in}}%
\pgfpathlineto{\pgfqpoint{5.069592in}{4.293436in}}%
\pgfpathlineto{\pgfqpoint{5.065822in}{4.296190in}}%
\pgfpathlineto{\pgfqpoint{5.060283in}{4.300237in}}%
\pgfpathlineto{\pgfqpoint{5.059590in}{4.300743in}}%
\pgfpathlineto{\pgfqpoint{5.053358in}{4.305297in}}%
\pgfpathlineto{\pgfqpoint{5.050973in}{4.307039in}}%
\pgfpathlineto{\pgfqpoint{5.047126in}{4.309850in}}%
\pgfpathlineto{\pgfqpoint{5.041664in}{4.313841in}}%
\pgfpathlineto{\pgfqpoint{5.040895in}{4.314403in}}%
\pgfpathlineto{\pgfqpoint{5.034663in}{4.318956in}}%
\pgfpathlineto{\pgfqpoint{5.032355in}{4.320643in}}%
\pgfpathlineto{\pgfqpoint{5.028431in}{4.323510in}}%
\pgfpathlineto{\pgfqpoint{5.023045in}{4.327445in}}%
\pgfpathlineto{\pgfqpoint{5.022199in}{4.328063in}}%
\pgfpathlineto{\pgfqpoint{5.015967in}{4.332616in}}%
\pgfpathlineto{\pgfqpoint{5.013736in}{4.334246in}}%
\pgfpathlineto{\pgfqpoint{5.009735in}{4.337169in}}%
\pgfpathlineto{\pgfqpoint{5.004427in}{4.341048in}}%
\pgfpathlineto{\pgfqpoint{5.003503in}{4.341723in}}%
\pgfpathlineto{\pgfqpoint{4.997272in}{4.346276in}}%
\pgfpathlineto{\pgfqpoint{4.995117in}{4.347850in}}%
\pgfpathlineto{\pgfqpoint{4.991040in}{4.350829in}}%
\pgfpathlineto{\pgfqpoint{4.985808in}{4.354652in}}%
\pgfpathlineto{\pgfqpoint{4.984808in}{4.355382in}}%
\pgfpathlineto{\pgfqpoint{4.978576in}{4.359936in}}%
\pgfpathlineto{\pgfqpoint{4.976499in}{4.361454in}}%
\pgfpathlineto{\pgfqpoint{4.972344in}{4.364489in}}%
\pgfpathlineto{\pgfqpoint{4.967189in}{4.368255in}}%
\pgfpathlineto{\pgfqpoint{4.966112in}{4.369042in}}%
\pgfpathlineto{\pgfqpoint{4.959881in}{4.373596in}}%
\pgfpathlineto{\pgfqpoint{4.957880in}{4.375057in}}%
\pgfpathlineto{\pgfqpoint{4.953649in}{4.378149in}}%
\pgfpathlineto{\pgfqpoint{4.948571in}{4.381859in}}%
\pgfpathlineto{\pgfqpoint{4.947417in}{4.382702in}}%
\pgfpathlineto{\pgfqpoint{4.941185in}{4.387255in}}%
\pgfpathlineto{\pgfqpoint{4.939262in}{4.388661in}}%
\pgfpathlineto{\pgfqpoint{4.934953in}{4.391809in}}%
\pgfpathlineto{\pgfqpoint{4.929952in}{4.395463in}}%
\pgfpathlineto{\pgfqpoint{4.928721in}{4.396362in}}%
\pgfpathlineto{\pgfqpoint{4.922489in}{4.400915in}}%
\pgfpathlineto{\pgfqpoint{4.920643in}{4.402264in}}%
\pgfpathlineto{\pgfqpoint{4.916258in}{4.405468in}}%
\pgfpathlineto{\pgfqpoint{4.911334in}{4.409066in}}%
\pgfpathlineto{\pgfqpoint{4.910026in}{4.410022in}}%
\pgfpathlineto{\pgfqpoint{4.903794in}{4.414575in}}%
\pgfpathlineto{\pgfqpoint{4.902024in}{4.415868in}}%
\pgfpathlineto{\pgfqpoint{4.897562in}{4.419128in}}%
\pgfpathlineto{\pgfqpoint{4.892715in}{4.422670in}}%
\pgfpathlineto{\pgfqpoint{4.891330in}{4.423682in}}%
\pgfpathlineto{\pgfqpoint{4.885098in}{4.428235in}}%
\pgfpathlineto{\pgfqpoint{4.883406in}{4.429472in}}%
\pgfpathlineto{\pgfqpoint{4.878866in}{4.432788in}}%
\pgfpathlineto{\pgfqpoint{4.874096in}{4.436273in}}%
\pgfpathlineto{\pgfqpoint{4.872635in}{4.437341in}}%
\pgfpathlineto{\pgfqpoint{4.866403in}{4.441895in}}%
\pgfpathlineto{\pgfqpoint{4.864787in}{4.443075in}}%
\pgfpathlineto{\pgfqpoint{4.860171in}{4.446448in}}%
\pgfpathlineto{\pgfqpoint{4.855478in}{4.449877in}}%
\pgfpathlineto{\pgfqpoint{4.853939in}{4.451001in}}%
\pgfpathlineto{\pgfqpoint{4.847707in}{4.455554in}}%
\pgfpathlineto{\pgfqpoint{4.846168in}{4.456679in}}%
\pgfpathlineto{\pgfqpoint{4.841475in}{4.460108in}}%
\pgfpathlineto{\pgfqpoint{4.836859in}{4.463481in}}%
\pgfpathlineto{\pgfqpoint{4.835244in}{4.464661in}}%
\pgfpathlineto{\pgfqpoint{4.829012in}{4.469214in}}%
\pgfpathlineto{\pgfqpoint{4.827550in}{4.470282in}}%
\pgfpathlineto{\pgfqpoint{4.822780in}{4.473768in}}%
\pgfpathlineto{\pgfqpoint{4.818241in}{4.477084in}}%
\pgfpathlineto{\pgfqpoint{4.816548in}{4.478321in}}%
\pgfpathlineto{\pgfqpoint{4.810316in}{4.482874in}}%
\pgfpathlineto{\pgfqpoint{4.808931in}{4.483886in}}%
\pgfpathlineto{\pgfqpoint{4.804084in}{4.487427in}}%
\pgfpathlineto{\pgfqpoint{4.799622in}{4.490688in}}%
\pgfpathlineto{\pgfqpoint{4.797852in}{4.491981in}}%
\pgfpathlineto{\pgfqpoint{4.791621in}{4.496534in}}%
\pgfpathlineto{\pgfqpoint{4.790313in}{4.497490in}}%
\pgfpathlineto{\pgfqpoint{4.785389in}{4.501087in}}%
\pgfpathlineto{\pgfqpoint{4.781003in}{4.504291in}}%
\pgfpathlineto{\pgfqpoint{4.779157in}{4.505640in}}%
\pgfpathlineto{\pgfqpoint{4.772925in}{4.510194in}}%
\pgfpathlineto{\pgfqpoint{4.771694in}{4.511093in}}%
\pgfpathlineto{\pgfqpoint{4.766693in}{4.514747in}}%
\pgfpathlineto{\pgfqpoint{4.762385in}{4.517895in}}%
\pgfpathlineto{\pgfqpoint{4.760461in}{4.519300in}}%
\pgfpathlineto{\pgfqpoint{4.754229in}{4.523854in}}%
\pgfpathlineto{\pgfqpoint{4.753075in}{4.524697in}}%
\pgfpathlineto{\pgfqpoint{4.747998in}{4.528407in}}%
\pgfpathlineto{\pgfqpoint{4.743766in}{4.531499in}}%
\pgfpathlineto{\pgfqpoint{4.741766in}{4.532960in}}%
\pgfpathlineto{\pgfqpoint{4.735534in}{4.537513in}}%
\pgfpathlineto{\pgfqpoint{4.734457in}{4.538300in}}%
\pgfpathlineto{\pgfqpoint{4.729302in}{4.542067in}}%
\pgfpathlineto{\pgfqpoint{4.725147in}{4.545102in}}%
\pgfpathlineto{\pgfqpoint{4.723070in}{4.546620in}}%
\pgfpathlineto{\pgfqpoint{4.716838in}{4.551173in}}%
\pgfpathlineto{\pgfqpoint{4.715838in}{4.551904in}}%
\pgfpathlineto{\pgfqpoint{4.710606in}{4.555726in}}%
\pgfpathlineto{\pgfqpoint{4.706529in}{4.558706in}}%
\pgfpathlineto{\pgfqpoint{4.704375in}{4.560280in}}%
\pgfpathlineto{\pgfqpoint{4.698143in}{4.564833in}}%
\pgfpathlineto{\pgfqpoint{4.697220in}{4.565508in}}%
\pgfpathlineto{\pgfqpoint{4.691911in}{4.569386in}}%
\pgfpathlineto{\pgfqpoint{4.687910in}{4.572309in}}%
\pgfpathlineto{\pgfqpoint{4.685679in}{4.573940in}}%
\pgfpathlineto{\pgfqpoint{4.679447in}{4.578493in}}%
\pgfpathlineto{\pgfqpoint{4.678601in}{4.579111in}}%
\pgfpathlineto{\pgfqpoint{4.673215in}{4.583046in}}%
\pgfpathlineto{\pgfqpoint{4.669292in}{4.585913in}}%
\pgfpathlineto{\pgfqpoint{4.666984in}{4.587599in}}%
\pgfpathlineto{\pgfqpoint{4.660752in}{4.592153in}}%
\pgfpathlineto{\pgfqpoint{4.659982in}{4.592715in}}%
\pgfpathlineto{\pgfqpoint{4.654520in}{4.596706in}}%
\pgfpathlineto{\pgfqpoint{4.650673in}{4.599517in}}%
\pgfpathlineto{\pgfqpoint{4.648288in}{4.601259in}}%
\pgfpathlineto{\pgfqpoint{4.642056in}{4.605812in}}%
\pgfpathlineto{\pgfqpoint{4.641364in}{4.606318in}}%
\pgfpathlineto{\pgfqpoint{4.635824in}{4.610366in}}%
\pgfpathlineto{\pgfqpoint{4.632054in}{4.613120in}}%
\pgfpathlineto{\pgfqpoint{4.629592in}{4.614919in}}%
\pgfpathlineto{\pgfqpoint{4.623361in}{4.619472in}}%
\pgfpathlineto{\pgfqpoint{4.622745in}{4.619922in}}%
\pgfpathlineto{\pgfqpoint{4.617129in}{4.624026in}}%
\pgfpathlineto{\pgfqpoint{4.613436in}{4.626724in}}%
\pgfpathlineto{\pgfqpoint{4.610897in}{4.628579in}}%
\pgfpathlineto{\pgfqpoint{4.604665in}{4.633132in}}%
\pgfpathlineto{\pgfqpoint{4.604126in}{4.633526in}}%
\pgfpathlineto{\pgfqpoint{4.598433in}{4.637685in}}%
\pgfpathlineto{\pgfqpoint{4.594817in}{4.640327in}}%
\pgfpathlineto{\pgfqpoint{4.592201in}{4.642239in}}%
\pgfpathlineto{\pgfqpoint{4.585969in}{4.646792in}}%
\pgfpathlineto{\pgfqpoint{4.585508in}{4.647129in}}%
\pgfpathlineto{\pgfqpoint{4.579738in}{4.651345in}}%
\pgfpathlineto{\pgfqpoint{4.576199in}{4.653931in}}%
\pgfpathlineto{\pgfqpoint{4.573506in}{4.655898in}}%
\pgfpathlineto{\pgfqpoint{4.567274in}{4.660452in}}%
\pgfpathlineto{\pgfqpoint{4.566889in}{4.660733in}}%
\pgfpathlineto{\pgfqpoint{4.561042in}{4.665005in}}%
\pgfpathlineto{\pgfqpoint{4.557580in}{4.667535in}}%
\pgfpathlineto{\pgfqpoint{4.554810in}{4.669558in}}%
\pgfpathlineto{\pgfqpoint{4.548578in}{4.674112in}}%
\pgfpathlineto{\pgfqpoint{4.548271in}{4.674336in}}%
\pgfpathlineto{\pgfqpoint{4.542346in}{4.678665in}}%
\pgfpathlineto{\pgfqpoint{4.538961in}{4.681138in}}%
\pgfpathlineto{\pgfqpoint{4.536115in}{4.683218in}}%
\pgfpathlineto{\pgfqpoint{4.529883in}{4.687771in}}%
\pgfpathlineto{\pgfqpoint{4.529652in}{4.687940in}}%
\pgfpathlineto{\pgfqpoint{4.523651in}{4.692325in}}%
\pgfpathlineto{\pgfqpoint{4.520343in}{4.694742in}}%
\pgfpathlineto{\pgfqpoint{4.517419in}{4.696878in}}%
\pgfpathlineto{\pgfqpoint{4.511187in}{4.701431in}}%
\pgfpathlineto{\pgfqpoint{4.511033in}{4.701544in}}%
\pgfpathlineto{\pgfqpoint{4.504955in}{4.705984in}}%
\pgfpathlineto{\pgfqpoint{4.501724in}{4.708345in}}%
\pgfpathlineto{\pgfqpoint{4.498724in}{4.710538in}}%
\pgfpathlineto{\pgfqpoint{4.492492in}{4.715091in}}%
\pgfpathlineto{\pgfqpoint{4.492415in}{4.715147in}}%
\pgfpathlineto{\pgfqpoint{4.486260in}{4.719644in}}%
\pgfpathlineto{\pgfqpoint{4.483105in}{4.721949in}}%
\pgfpathlineto{\pgfqpoint{4.480028in}{4.724198in}}%
\pgfpathlineto{\pgfqpoint{4.473796in}{4.728751in}}%
\pgfpathlineto{\pgfqpoint{4.473796in}{4.728751in}}%
\pgfpathlineto{\pgfqpoint{4.467564in}{4.733304in}}%
\pgfpathlineto{\pgfqpoint{4.464487in}{4.735553in}}%
\pgfpathlineto{\pgfqpoint{4.461332in}{4.737857in}}%
\pgfpathlineto{\pgfqpoint{4.455177in}{4.742354in}}%
\pgfpathlineto{\pgfqpoint{4.455101in}{4.742411in}}%
\pgfpathlineto{\pgfqpoint{4.448869in}{4.746964in}}%
\pgfpathlineto{\pgfqpoint{4.445868in}{4.749156in}}%
\pgfpathlineto{\pgfqpoint{4.442637in}{4.751517in}}%
\pgfpathlineto{\pgfqpoint{4.436559in}{4.755958in}}%
\pgfpathlineto{\pgfqpoint{4.436405in}{4.756070in}}%
\pgfpathlineto{\pgfqpoint{4.430173in}{4.760624in}}%
\pgfpathlineto{\pgfqpoint{4.427250in}{4.762760in}}%
\pgfpathlineto{\pgfqpoint{4.423941in}{4.765177in}}%
\pgfpathlineto{\pgfqpoint{4.417940in}{4.769562in}}%
\pgfpathlineto{\pgfqpoint{4.417709in}{4.769730in}}%
\pgfpathlineto{\pgfqpoint{4.411478in}{4.774284in}}%
\pgfpathlineto{\pgfqpoint{4.408631in}{4.776363in}}%
\pgfpathlineto{\pgfqpoint{4.405246in}{4.778837in}}%
\pgfpathlineto{\pgfqpoint{4.399322in}{4.783165in}}%
\pgfpathlineto{\pgfqpoint{4.399014in}{4.783390in}}%
\pgfpathlineto{\pgfqpoint{4.392782in}{4.787943in}}%
\pgfpathlineto{\pgfqpoint{4.390012in}{4.789967in}}%
\pgfpathlineto{\pgfqpoint{4.386550in}{4.792497in}}%
\pgfpathlineto{\pgfqpoint{4.380703in}{4.796769in}}%
\pgfpathlineto{\pgfqpoint{4.380318in}{4.797050in}}%
\pgfpathlineto{\pgfqpoint{4.374086in}{4.801603in}}%
\pgfpathlineto{\pgfqpoint{4.371394in}{4.803571in}}%
\pgfpathlineto{\pgfqpoint{4.367855in}{4.806156in}}%
\pgfpathlineto{\pgfqpoint{4.362084in}{4.810372in}}%
\pgfpathlineto{\pgfqpoint{4.361623in}{4.810710in}}%
\pgfpathlineto{\pgfqpoint{4.355391in}{4.815263in}}%
\pgfpathlineto{\pgfqpoint{4.352775in}{4.817174in}}%
\pgfpathlineto{\pgfqpoint{4.349159in}{4.819816in}}%
\pgfpathlineto{\pgfqpoint{4.343466in}{4.823976in}}%
\pgfpathlineto{\pgfqpoint{4.342927in}{4.824370in}}%
\pgfpathlineto{\pgfqpoint{4.336695in}{4.828923in}}%
\pgfpathlineto{\pgfqpoint{4.334156in}{4.830778in}}%
\pgfpathlineto{\pgfqpoint{4.330464in}{4.833476in}}%
\pgfpathlineto{\pgfqpoint{4.324847in}{4.837580in}}%
\pgfpathlineto{\pgfqpoint{4.324232in}{4.838029in}}%
\pgfpathlineto{\pgfqpoint{4.318000in}{4.842583in}}%
\pgfpathlineto{\pgfqpoint{4.315538in}{4.844381in}}%
\pgfpathlineto{\pgfqpoint{4.311768in}{4.847136in}}%
\pgfpathlineto{\pgfqpoint{4.306229in}{4.851183in}}%
\pgfpathlineto{\pgfqpoint{4.305536in}{4.851689in}}%
\pgfpathlineto{\pgfqpoint{4.299304in}{4.856242in}}%
\pgfpathlineto{\pgfqpoint{4.296919in}{4.857985in}}%
\pgfpathlineto{\pgfqpoint{4.293072in}{4.860796in}}%
\pgfpathlineto{\pgfqpoint{4.287610in}{4.864787in}}%
\pgfpathlineto{\pgfqpoint{4.286841in}{4.865349in}}%
\pgfpathlineto{\pgfqpoint{4.280609in}{4.869902in}}%
\pgfpathlineto{\pgfqpoint{4.278301in}{4.871589in}}%
\pgfpathlineto{\pgfqpoint{4.274377in}{4.874456in}}%
\pgfpathlineto{\pgfqpoint{4.268991in}{4.878390in}}%
\pgfpathlineto{\pgfqpoint{4.268145in}{4.879009in}}%
\pgfpathlineto{\pgfqpoint{4.261913in}{4.883562in}}%
\pgfpathlineto{\pgfqpoint{4.259682in}{4.885192in}}%
\pgfpathlineto{\pgfqpoint{4.255681in}{4.888115in}}%
\pgfpathlineto{\pgfqpoint{4.250373in}{4.891994in}}%
\pgfpathlineto{\pgfqpoint{4.249449in}{4.892669in}}%
\pgfpathlineto{\pgfqpoint{4.243218in}{4.897222in}}%
\pgfpathlineto{\pgfqpoint{4.241063in}{4.898796in}}%
\pgfpathlineto{\pgfqpoint{4.236986in}{4.901775in}}%
\pgfpathlineto{\pgfqpoint{4.231754in}{4.905598in}}%
\pgfpathlineto{\pgfqpoint{4.230754in}{4.906328in}}%
\pgfpathlineto{\pgfqpoint{4.224522in}{4.910882in}}%
\pgfpathlineto{\pgfqpoint{4.222445in}{4.912399in}}%
\pgfpathlineto{\pgfqpoint{4.218290in}{4.915435in}}%
\pgfpathlineto{\pgfqpoint{4.213135in}{4.919201in}}%
\pgfpathlineto{\pgfqpoint{4.212058in}{4.919988in}}%
\pgfpathlineto{\pgfqpoint{4.205826in}{4.924542in}}%
\pgfpathlineto{\pgfqpoint{4.203826in}{4.926003in}}%
\pgfpathlineto{\pgfqpoint{4.199595in}{4.929095in}}%
\pgfpathlineto{\pgfqpoint{4.194517in}{4.932805in}}%
\pgfpathlineto{\pgfqpoint{4.193363in}{4.933648in}}%
\pgfpathlineto{\pgfqpoint{4.187131in}{4.938201in}}%
\pgfpathlineto{\pgfqpoint{4.185208in}{4.939607in}}%
\pgfpathlineto{\pgfqpoint{4.180899in}{4.942755in}}%
\pgfpathlineto{\pgfqpoint{4.175898in}{4.946408in}}%
\pgfpathlineto{\pgfqpoint{4.174667in}{4.947308in}}%
\pgfpathlineto{\pgfqpoint{4.168435in}{4.951861in}}%
\pgfpathlineto{\pgfqpoint{4.166589in}{4.953210in}}%
\pgfpathlineto{\pgfqpoint{4.162204in}{4.956414in}}%
\pgfpathlineto{\pgfqpoint{4.157280in}{4.960012in}}%
\pgfpathlineto{\pgfqpoint{4.155972in}{4.960968in}}%
\pgfpathlineto{\pgfqpoint{4.149740in}{4.965521in}}%
\pgfpathlineto{\pgfqpoint{4.147970in}{4.966814in}}%
\pgfpathlineto{\pgfqpoint{4.143508in}{4.970074in}}%
\pgfpathlineto{\pgfqpoint{4.138661in}{4.973616in}}%
\pgfpathlineto{\pgfqpoint{4.137276in}{4.974628in}}%
\pgfpathlineto{\pgfqpoint{4.131044in}{4.979181in}}%
\pgfpathlineto{\pgfqpoint{4.129352in}{4.980417in}}%
\pgfpathlineto{\pgfqpoint{4.124812in}{4.983734in}}%
\pgfpathlineto{\pgfqpoint{4.120042in}{4.987219in}}%
\pgfpathlineto{\pgfqpoint{4.118581in}{4.988287in}}%
\pgfpathlineto{\pgfqpoint{4.112349in}{4.992841in}}%
\pgfpathlineto{\pgfqpoint{4.110733in}{4.994021in}}%
\pgfpathlineto{\pgfqpoint{4.106117in}{4.997394in}}%
\pgfpathlineto{\pgfqpoint{4.101424in}{5.000823in}}%
\pgfpathlineto{\pgfqpoint{4.099885in}{5.001947in}}%
\pgfpathlineto{\pgfqpoint{4.093653in}{5.006500in}}%
\pgfpathlineto{\pgfqpoint{4.092114in}{5.007625in}}%
\pgfpathlineto{\pgfqpoint{4.087421in}{5.011054in}}%
\pgfpathlineto{\pgfqpoint{4.082805in}{5.014427in}}%
\pgfpathlineto{\pgfqpoint{4.081189in}{5.015607in}}%
\pgfpathlineto{\pgfqpoint{4.074958in}{5.020160in}}%
\pgfpathlineto{\pgfqpoint{4.073496in}{5.021228in}}%
\pgfpathlineto{\pgfqpoint{4.068726in}{5.024714in}}%
\pgfpathlineto{\pgfqpoint{4.064186in}{5.028030in}}%
\pgfpathlineto{\pgfqpoint{4.062494in}{5.029267in}}%
\pgfpathlineto{\pgfqpoint{4.056262in}{5.033820in}}%
\pgfpathlineto{\pgfqpoint{4.054877in}{5.034832in}}%
\pgfpathlineto{\pgfqpoint{4.050030in}{5.038373in}}%
\pgfpathlineto{\pgfqpoint{4.045568in}{5.041634in}}%
\pgfpathlineto{\pgfqpoint{4.043798in}{5.042927in}}%
\pgfpathlineto{\pgfqpoint{4.037566in}{5.047480in}}%
\pgfpathlineto{\pgfqpoint{4.036259in}{5.048436in}}%
\pgfpathlineto{\pgfqpoint{4.031335in}{5.052033in}}%
\pgfpathlineto{\pgfqpoint{4.026949in}{5.055237in}}%
\pgfpathlineto{\pgfqpoint{4.025103in}{5.056586in}}%
\pgfpathlineto{\pgfqpoint{4.018871in}{5.061140in}}%
\pgfpathlineto{\pgfqpoint{4.017640in}{5.062039in}}%
\pgfpathlineto{\pgfqpoint{4.012639in}{5.065693in}}%
\pgfpathlineto{\pgfqpoint{4.008331in}{5.068841in}}%
\pgfpathlineto{\pgfqpoint{4.006407in}{5.070246in}}%
\pgfpathlineto{\pgfqpoint{4.000175in}{5.074800in}}%
\pgfpathlineto{\pgfqpoint{3.999021in}{5.075643in}}%
\pgfpathlineto{\pgfqpoint{3.993944in}{5.079353in}}%
\pgfpathlineto{\pgfqpoint{3.989712in}{5.082445in}}%
\pgfpathlineto{\pgfqpoint{3.987712in}{5.083906in}}%
\pgfpathlineto{\pgfqpoint{3.981480in}{5.088459in}}%
\pgfpathlineto{\pgfqpoint{3.980403in}{5.089246in}}%
\pgfpathlineto{\pgfqpoint{3.975248in}{5.093013in}}%
\pgfpathlineto{\pgfqpoint{3.971093in}{5.096048in}}%
\pgfpathlineto{\pgfqpoint{3.969016in}{5.097566in}}%
\pgfpathlineto{\pgfqpoint{3.962784in}{5.102119in}}%
\pgfpathlineto{\pgfqpoint{3.961784in}{5.102850in}}%
\pgfpathlineto{\pgfqpoint{3.956552in}{5.106672in}}%
\pgfpathlineto{\pgfqpoint{3.952475in}{5.109652in}}%
\pgfpathlineto{\pgfqpoint{3.950321in}{5.111226in}}%
\pgfpathlineto{\pgfqpoint{3.944089in}{5.115779in}}%
\pgfpathlineto{\pgfqpoint{3.943165in}{5.116454in}}%
\pgfpathlineto{\pgfqpoint{3.937857in}{5.120332in}}%
\pgfpathlineto{\pgfqpoint{3.933856in}{5.123255in}}%
\pgfpathlineto{\pgfqpoint{3.931625in}{5.124886in}}%
\pgfpathlineto{\pgfqpoint{3.925393in}{5.129439in}}%
\pgfpathlineto{\pgfqpoint{3.924547in}{5.130057in}}%
\pgfpathlineto{\pgfqpoint{3.919161in}{5.133992in}}%
\pgfpathlineto{\pgfqpoint{3.915238in}{5.136859in}}%
\pgfpathlineto{\pgfqpoint{3.912929in}{5.138545in}}%
\pgfpathlineto{\pgfqpoint{3.906698in}{5.143099in}}%
\pgfpathlineto{\pgfqpoint{3.905928in}{5.143661in}}%
\pgfpathlineto{\pgfqpoint{3.900466in}{5.147652in}}%
\pgfpathlineto{\pgfqpoint{3.896619in}{5.150463in}}%
\pgfpathlineto{\pgfqpoint{3.894234in}{5.152205in}}%
\pgfpathlineto{\pgfqpoint{3.888002in}{5.156758in}}%
\pgfpathlineto{\pgfqpoint{3.887310in}{5.157264in}}%
\pgfpathlineto{\pgfqpoint{3.881770in}{5.161312in}}%
\pgfpathlineto{\pgfqpoint{3.878000in}{5.164066in}}%
\pgfpathlineto{\pgfqpoint{3.875538in}{5.165865in}}%
\pgfpathlineto{\pgfqpoint{3.869306in}{5.170418in}}%
\pgfpathlineto{\pgfqpoint{3.868691in}{5.170868in}}%
\pgfpathlineto{\pgfqpoint{3.863075in}{5.174972in}}%
\pgfpathlineto{\pgfqpoint{3.859382in}{5.177670in}}%
\pgfpathlineto{\pgfqpoint{3.856843in}{5.179525in}}%
\pgfpathlineto{\pgfqpoint{3.850611in}{5.184078in}}%
\pgfpathlineto{\pgfqpoint{3.850072in}{5.184472in}}%
\pgfpathlineto{\pgfqpoint{3.844379in}{5.188631in}}%
\pgfpathlineto{\pgfqpoint{3.840763in}{5.191273in}}%
\pgfpathlineto{\pgfqpoint{3.838147in}{5.193185in}}%
\pgfpathlineto{\pgfqpoint{3.831915in}{5.197738in}}%
\pgfpathlineto{\pgfqpoint{3.831454in}{5.198075in}}%
\pgfpathlineto{\pgfqpoint{3.825684in}{5.202291in}}%
\pgfpathlineto{\pgfqpoint{3.822144in}{5.204877in}}%
\pgfpathlineto{\pgfqpoint{3.819452in}{5.206844in}}%
\pgfpathlineto{\pgfqpoint{3.813220in}{5.211398in}}%
\pgfpathlineto{\pgfqpoint{3.812835in}{5.211679in}}%
\pgfpathlineto{\pgfqpoint{3.806988in}{5.215951in}}%
\pgfpathlineto{\pgfqpoint{3.803526in}{5.218481in}}%
\pgfpathlineto{\pgfqpoint{3.800756in}{5.220504in}}%
\pgfpathlineto{\pgfqpoint{3.794524in}{5.225058in}}%
\pgfpathlineto{\pgfqpoint{3.794217in}{5.225282in}}%
\pgfpathlineto{\pgfqpoint{3.788292in}{5.229611in}}%
\pgfpathlineto{\pgfqpoint{3.784907in}{5.232084in}}%
\pgfpathlineto{\pgfqpoint{3.782061in}{5.234164in}}%
\pgfpathlineto{\pgfqpoint{3.775829in}{5.238717in}}%
\pgfpathlineto{\pgfqpoint{3.775598in}{5.238886in}}%
\pgfpathlineto{\pgfqpoint{3.769597in}{5.243271in}}%
\pgfpathlineto{\pgfqpoint{3.766289in}{5.245688in}}%
\pgfpathlineto{\pgfqpoint{3.763365in}{5.247824in}}%
\pgfpathlineto{\pgfqpoint{3.757133in}{5.252377in}}%
\pgfpathlineto{\pgfqpoint{3.756979in}{5.252490in}}%
\pgfpathlineto{\pgfqpoint{3.750901in}{5.256930in}}%
\pgfpathlineto{\pgfqpoint{3.747670in}{5.259291in}}%
\pgfpathlineto{\pgfqpoint{3.744669in}{5.261484in}}%
\pgfpathlineto{\pgfqpoint{3.738438in}{5.266037in}}%
\pgfpathlineto{\pgfqpoint{3.738361in}{5.266093in}}%
\pgfpathlineto{\pgfqpoint{3.732206in}{5.270590in}}%
\pgfpathlineto{\pgfqpoint{3.729051in}{5.272895in}}%
\pgfpathlineto{\pgfqpoint{3.725974in}{5.275143in}}%
\pgfpathlineto{\pgfqpoint{3.719742in}{5.279697in}}%
\pgfpathlineto{\pgfqpoint{3.719742in}{5.279697in}}%
\pgfpathlineto{\pgfqpoint{3.713510in}{5.284250in}}%
\pgfpathlineto{\pgfqpoint{3.710433in}{5.286499in}}%
\pgfpathlineto{\pgfqpoint{3.707278in}{5.288803in}}%
\pgfpathlineto{\pgfqpoint{3.701123in}{5.293300in}}%
\pgfpathlineto{\pgfqpoint{3.701046in}{5.293357in}}%
\pgfpathlineto{\pgfqpoint{3.694815in}{5.297910in}}%
\pgfpathlineto{\pgfqpoint{3.691814in}{5.300102in}}%
\pgfpathlineto{\pgfqpoint{3.688583in}{5.302463in}}%
\pgfpathlineto{\pgfqpoint{3.682505in}{5.306904in}}%
\pgfpathlineto{\pgfqpoint{3.682351in}{5.307016in}}%
\pgfpathlineto{\pgfqpoint{3.676119in}{5.311570in}}%
\pgfpathlineto{\pgfqpoint{3.673196in}{5.313706in}}%
\pgfpathlineto{\pgfqpoint{3.669887in}{5.316123in}}%
\pgfpathlineto{\pgfqpoint{3.663886in}{5.320508in}}%
\pgfpathlineto{\pgfqpoint{3.663655in}{5.320676in}}%
\pgfpathlineto{\pgfqpoint{3.657424in}{5.325229in}}%
\pgfpathlineto{\pgfqpoint{3.654577in}{5.327309in}}%
\pgfpathlineto{\pgfqpoint{3.651192in}{5.329783in}}%
\pgfpathlineto{\pgfqpoint{3.645268in}{5.334111in}}%
\pgfpathlineto{\pgfqpoint{3.644960in}{5.334336in}}%
\pgfpathlineto{\pgfqpoint{3.638728in}{5.338889in}}%
\pgfpathlineto{\pgfqpoint{3.635958in}{5.340913in}}%
\pgfpathlineto{\pgfqpoint{3.632496in}{5.343443in}}%
\pgfpathlineto{\pgfqpoint{3.626649in}{5.347715in}}%
\pgfpathlineto{\pgfqpoint{3.626264in}{5.347996in}}%
\pgfpathlineto{\pgfqpoint{3.620032in}{5.352549in}}%
\pgfpathlineto{\pgfqpoint{3.617340in}{5.354517in}}%
\pgfpathlineto{\pgfqpoint{3.613801in}{5.357102in}}%
\pgfpathlineto{\pgfqpoint{3.608030in}{5.361318in}}%
\pgfpathlineto{\pgfqpoint{3.607569in}{5.361656in}}%
\pgfpathlineto{\pgfqpoint{3.601337in}{5.366209in}}%
\pgfpathlineto{\pgfqpoint{3.598721in}{5.368120in}}%
\pgfpathlineto{\pgfqpoint{3.595105in}{5.370762in}}%
\pgfpathlineto{\pgfqpoint{3.589412in}{5.374922in}}%
\pgfpathlineto{\pgfqpoint{3.588873in}{5.375315in}}%
\pgfpathlineto{\pgfqpoint{3.582641in}{5.379869in}}%
\pgfpathlineto{\pgfqpoint{3.580102in}{5.381724in}}%
\pgfpathlineto{\pgfqpoint{3.576409in}{5.384422in}}%
\pgfpathlineto{\pgfqpoint{3.570793in}{5.388526in}}%
\pgfpathlineto{\pgfqpoint{3.570178in}{5.388975in}}%
\pgfpathlineto{\pgfqpoint{3.563946in}{5.393529in}}%
\pgfpathlineto{\pgfqpoint{3.561484in}{5.395327in}}%
\pgfpathlineto{\pgfqpoint{3.557714in}{5.398082in}}%
\pgfpathlineto{\pgfqpoint{3.552174in}{5.402129in}}%
\pgfpathlineto{\pgfqpoint{3.551482in}{5.402635in}}%
\pgfpathlineto{\pgfqpoint{3.545250in}{5.407188in}}%
\pgfpathlineto{\pgfqpoint{3.542865in}{5.408931in}}%
\pgfpathlineto{\pgfqpoint{3.539018in}{5.411742in}}%
\pgfpathlineto{\pgfqpoint{3.533556in}{5.415733in}}%
\pgfpathlineto{\pgfqpoint{3.532786in}{5.416295in}}%
\pgfpathlineto{\pgfqpoint{3.526555in}{5.420848in}}%
\pgfpathlineto{\pgfqpoint{3.524247in}{5.422535in}}%
\pgfpathlineto{\pgfqpoint{3.520323in}{5.425401in}}%
\pgfpathlineto{\pgfqpoint{3.514937in}{5.429336in}}%
\pgfpathlineto{\pgfqpoint{3.514091in}{5.429955in}}%
\pgfpathlineto{\pgfqpoint{3.507859in}{5.434508in}}%
\pgfpathlineto{\pgfqpoint{3.505628in}{5.436138in}}%
\pgfpathlineto{\pgfqpoint{3.501627in}{5.439061in}}%
\pgfpathlineto{\pgfqpoint{3.496319in}{5.442940in}}%
\pgfpathlineto{\pgfqpoint{3.495395in}{5.443615in}}%
\pgfpathlineto{\pgfqpoint{3.489164in}{5.448168in}}%
\pgfpathlineto{\pgfqpoint{3.487009in}{5.449742in}}%
\pgfpathlineto{\pgfqpoint{3.482932in}{5.452721in}}%
\pgfpathlineto{\pgfqpoint{3.477700in}{5.456544in}}%
\pgfpathlineto{\pgfqpoint{3.476700in}{5.457274in}}%
\pgfpathlineto{\pgfqpoint{3.470468in}{5.461828in}}%
\pgfpathlineto{\pgfqpoint{3.468391in}{5.463345in}}%
\pgfpathlineto{\pgfqpoint{3.464236in}{5.466381in}}%
\pgfpathlineto{\pgfqpoint{3.459081in}{5.470147in}}%
\pgfpathlineto{\pgfqpoint{3.458004in}{5.470934in}}%
\pgfpathlineto{\pgfqpoint{3.451772in}{5.475487in}}%
\pgfpathlineto{\pgfqpoint{3.449772in}{5.476949in}}%
\pgfpathlineto{\pgfqpoint{3.445541in}{5.480041in}}%
\pgfpathlineto{\pgfqpoint{3.440463in}{5.483751in}}%
\pgfpathlineto{\pgfqpoint{3.439309in}{5.484594in}}%
\pgfpathlineto{\pgfqpoint{3.433077in}{5.489147in}}%
\pgfpathlineto{\pgfqpoint{3.431153in}{5.490553in}}%
\pgfpathlineto{\pgfqpoint{3.426845in}{5.493701in}}%
\pgfpathlineto{\pgfqpoint{3.421844in}{5.497354in}}%
\pgfpathlineto{\pgfqpoint{3.420613in}{5.498254in}}%
\pgfpathlineto{\pgfqpoint{3.414381in}{5.502807in}}%
\pgfpathlineto{\pgfqpoint{3.412535in}{5.504156in}}%
\pgfpathlineto{\pgfqpoint{3.408149in}{5.507360in}}%
\pgfpathlineto{\pgfqpoint{3.403226in}{5.510958in}}%
\pgfpathlineto{\pgfqpoint{3.401918in}{5.511914in}}%
\pgfpathlineto{\pgfqpoint{3.395686in}{5.516467in}}%
\pgfpathlineto{\pgfqpoint{3.393916in}{5.517760in}}%
\pgfpathlineto{\pgfqpoint{3.389454in}{5.521020in}}%
\pgfpathlineto{\pgfqpoint{3.384607in}{5.524562in}}%
\pgfpathlineto{\pgfqpoint{3.383222in}{5.525573in}}%
\pgfpathlineto{\pgfqpoint{3.376990in}{5.530127in}}%
\pgfpathlineto{\pgfqpoint{3.375298in}{5.531363in}}%
\pgfpathlineto{\pgfqpoint{3.370758in}{5.534680in}}%
\pgfpathlineto{\pgfqpoint{3.365988in}{5.538165in}}%
\pgfpathlineto{\pgfqpoint{3.364527in}{5.539233in}}%
\pgfpathlineto{\pgfqpoint{3.358295in}{5.543787in}}%
\pgfpathlineto{\pgfqpoint{3.356679in}{5.544967in}}%
\pgfpathlineto{\pgfqpoint{3.352063in}{5.548340in}}%
\pgfpathlineto{\pgfqpoint{3.347370in}{5.551769in}}%
\pgfpathlineto{\pgfqpoint{3.345831in}{5.552893in}}%
\pgfpathlineto{\pgfqpoint{3.339599in}{5.557446in}}%
\pgfpathlineto{\pgfqpoint{3.338060in}{5.558571in}}%
\pgfpathlineto{\pgfqpoint{3.333367in}{5.562000in}}%
\pgfpathlineto{\pgfqpoint{3.328751in}{5.565372in}}%
\pgfpathlineto{\pgfqpoint{3.327135in}{5.566553in}}%
\pgfpathlineto{\pgfqpoint{3.320904in}{5.571106in}}%
\pgfpathlineto{\pgfqpoint{3.319442in}{5.572174in}}%
\pgfpathlineto{\pgfqpoint{3.314672in}{5.575659in}}%
\pgfpathlineto{\pgfqpoint{3.310132in}{5.578976in}}%
\pgfpathlineto{\pgfqpoint{3.308440in}{5.580213in}}%
\pgfpathlineto{\pgfqpoint{3.302208in}{5.584766in}}%
\pgfpathlineto{\pgfqpoint{3.300823in}{5.585778in}}%
\pgfpathlineto{\pgfqpoint{3.295976in}{5.589319in}}%
\pgfpathlineto{\pgfqpoint{3.291514in}{5.592580in}}%
\pgfpathlineto{\pgfqpoint{3.289744in}{5.593873in}}%
\pgfpathlineto{\pgfqpoint{3.283512in}{5.598426in}}%
\pgfpathlineto{\pgfqpoint{3.282205in}{5.599381in}}%
\pgfpathlineto{\pgfqpoint{3.277281in}{5.602979in}}%
\pgfpathlineto{\pgfqpoint{3.272895in}{5.606183in}}%
\pgfpathlineto{\pgfqpoint{3.271049in}{5.607532in}}%
\pgfpathlineto{\pgfqpoint{3.264817in}{5.612086in}}%
\pgfpathlineto{\pgfqpoint{3.263586in}{5.612985in}}%
\pgfpathlineto{\pgfqpoint{3.258585in}{5.616639in}}%
\pgfpathlineto{\pgfqpoint{3.254277in}{5.619787in}}%
\pgfpathlineto{\pgfqpoint{3.252353in}{5.621192in}}%
\pgfpathlineto{\pgfqpoint{3.246121in}{5.625745in}}%
\pgfpathlineto{\pgfqpoint{3.244967in}{5.626589in}}%
\pgfpathlineto{\pgfqpoint{3.239889in}{5.630299in}}%
\pgfpathlineto{\pgfqpoint{3.235658in}{5.633390in}}%
\pgfpathlineto{\pgfqpoint{3.233658in}{5.634852in}}%
\pgfpathlineto{\pgfqpoint{3.227426in}{5.639405in}}%
\pgfpathlineto{\pgfqpoint{3.226349in}{5.640192in}}%
\pgfpathlineto{\pgfqpoint{3.221194in}{5.643959in}}%
\pgfpathlineto{\pgfqpoint{3.217039in}{5.646994in}}%
\pgfpathlineto{\pgfqpoint{3.214962in}{5.648512in}}%
\pgfpathlineto{\pgfqpoint{3.208730in}{5.653065in}}%
\pgfpathlineto{\pgfqpoint{3.207730in}{5.653796in}}%
\pgfpathlineto{\pgfqpoint{3.202498in}{5.657618in}}%
\pgfpathlineto{\pgfqpoint{3.198421in}{5.660598in}}%
\pgfpathlineto{\pgfqpoint{3.196267in}{5.662172in}}%
\pgfpathlineto{\pgfqpoint{3.190035in}{5.666725in}}%
\pgfpathlineto{\pgfqpoint{3.189111in}{5.667399in}}%
\pgfpathlineto{\pgfqpoint{3.183803in}{5.671278in}}%
\pgfpathlineto{\pgfqpoint{3.179802in}{5.674201in}}%
\pgfpathlineto{\pgfqpoint{3.177571in}{5.675831in}}%
\pgfpathlineto{\pgfqpoint{3.171339in}{5.680385in}}%
\pgfpathlineto{\pgfqpoint{3.170493in}{5.681003in}}%
\pgfpathlineto{\pgfqpoint{3.165107in}{5.684938in}}%
\pgfpathlineto{\pgfqpoint{3.161183in}{5.687805in}}%
\pgfpathlineto{\pgfqpoint{3.158875in}{5.689491in}}%
\pgfpathlineto{\pgfqpoint{3.152644in}{5.694045in}}%
\pgfpathlineto{\pgfqpoint{3.151874in}{5.694607in}}%
\pgfpathlineto{\pgfqpoint{3.146412in}{5.698598in}}%
\pgfpathlineto{\pgfqpoint{3.142565in}{5.701408in}}%
\pgfpathlineto{\pgfqpoint{3.140180in}{5.703151in}}%
\pgfpathlineto{\pgfqpoint{3.133948in}{5.707704in}}%
\pgfpathlineto{\pgfqpoint{3.133256in}{5.708210in}}%
\pgfpathlineto{\pgfqpoint{3.127716in}{5.712258in}}%
\pgfpathlineto{\pgfqpoint{3.123946in}{5.715012in}}%
\pgfpathlineto{\pgfqpoint{3.121484in}{5.716811in}}%
\pgfpathlineto{\pgfqpoint{3.115252in}{5.721364in}}%
\pgfpathlineto{\pgfqpoint{3.114637in}{5.721814in}}%
\pgfpathlineto{\pgfqpoint{3.109021in}{5.725917in}}%
\pgfpathlineto{\pgfqpoint{3.105328in}{5.728616in}}%
\pgfpathlineto{\pgfqpoint{3.102789in}{5.730471in}}%
\pgfpathlineto{\pgfqpoint{3.096557in}{5.735024in}}%
\pgfpathlineto{\pgfqpoint{3.096018in}{5.735417in}}%
\pgfpathlineto{\pgfqpoint{3.090325in}{5.739577in}}%
\pgfpathlineto{\pgfqpoint{3.086709in}{5.742219in}}%
\pgfpathlineto{\pgfqpoint{3.084093in}{5.744131in}}%
\pgfpathlineto{\pgfqpoint{3.077861in}{5.748684in}}%
\pgfpathlineto{\pgfqpoint{3.077400in}{5.749021in}}%
\pgfpathlineto{\pgfqpoint{3.071629in}{5.753237in}}%
\pgfpathlineto{\pgfqpoint{3.068090in}{5.755823in}}%
\pgfpathlineto{\pgfqpoint{3.065398in}{5.757790in}}%
\pgfpathlineto{\pgfqpoint{3.059166in}{5.762344in}}%
\pgfpathlineto{\pgfqpoint{3.058781in}{5.762625in}}%
\pgfpathlineto{\pgfqpoint{3.052934in}{5.766897in}}%
\pgfpathlineto{\pgfqpoint{3.049472in}{5.769427in}}%
\pgfpathlineto{\pgfqpoint{3.046702in}{5.771450in}}%
\pgfpathlineto{\pgfqpoint{3.040470in}{5.776003in}}%
\pgfpathlineto{\pgfqpoint{3.040162in}{5.776228in}}%
\pgfpathlineto{\pgfqpoint{3.034238in}{5.780557in}}%
\pgfpathlineto{\pgfqpoint{3.030853in}{5.783030in}}%
\pgfpathlineto{\pgfqpoint{3.028007in}{5.785110in}}%
\pgfpathlineto{\pgfqpoint{3.021775in}{5.789663in}}%
\pgfpathlineto{\pgfqpoint{3.021544in}{5.789832in}}%
\pgfpathlineto{\pgfqpoint{3.015543in}{5.794217in}}%
\pgfpathlineto{\pgfqpoint{3.012235in}{5.796634in}}%
\pgfpathlineto{\pgfqpoint{3.009311in}{5.798770in}}%
\pgfpathlineto{\pgfqpoint{3.003079in}{5.803323in}}%
\pgfpathlineto{\pgfqpoint{3.002925in}{5.803436in}}%
\pgfpathlineto{\pgfqpoint{2.996847in}{5.807876in}}%
\pgfpathlineto{\pgfqpoint{2.993616in}{5.810237in}}%
\pgfpathlineto{\pgfqpoint{2.990615in}{5.812430in}}%
\pgfpathlineto{\pgfqpoint{2.984384in}{5.816983in}}%
\pgfpathlineto{\pgfqpoint{2.984307in}{5.817039in}}%
\pgfpathlineto{\pgfqpoint{2.978152in}{5.821536in}}%
\pgfpathlineto{\pgfqpoint{2.974997in}{5.823841in}}%
\pgfpathlineto{\pgfqpoint{2.971920in}{5.826089in}}%
\pgfpathlineto{\pgfqpoint{2.965688in}{5.830643in}}%
\pgfpathlineto{\pgfqpoint{2.965688in}{5.830643in}}%
\pgfpathlineto{\pgfqpoint{2.959456in}{5.835196in}}%
\pgfpathlineto{\pgfqpoint{2.956379in}{5.837445in}}%
\pgfpathlineto{\pgfqpoint{2.953224in}{5.839749in}}%
\pgfpathlineto{\pgfqpoint{2.947069in}{5.844246in}}%
\pgfpathlineto{\pgfqpoint{2.946992in}{5.844303in}}%
\pgfpathlineto{\pgfqpoint{2.940761in}{5.848856in}}%
\pgfpathlineto{\pgfqpoint{2.937760in}{5.851048in}}%
\pgfpathlineto{\pgfqpoint{2.934529in}{5.853409in}}%
\pgfpathlineto{\pgfqpoint{2.928451in}{5.857850in}}%
\pgfpathlineto{\pgfqpoint{2.928297in}{5.857962in}}%
\pgfpathlineto{\pgfqpoint{2.922065in}{5.862516in}}%
\pgfpathlineto{\pgfqpoint{2.919141in}{5.864652in}}%
\pgfpathlineto{\pgfqpoint{2.915833in}{5.867069in}}%
\pgfpathlineto{\pgfqpoint{2.909832in}{5.871454in}}%
\pgfpathlineto{\pgfqpoint{2.909601in}{5.871622in}}%
\pgfpathlineto{\pgfqpoint{2.903369in}{5.876175in}}%
\pgfpathlineto{\pgfqpoint{2.900523in}{5.878255in}}%
\pgfpathlineto{\pgfqpoint{2.897138in}{5.880729in}}%
\pgfpathlineto{\pgfqpoint{2.891214in}{5.885057in}}%
\pgfpathlineto{\pgfqpoint{2.890906in}{5.885282in}}%
\pgfpathlineto{\pgfqpoint{2.884674in}{5.889835in}}%
\pgfpathlineto{\pgfqpoint{2.881904in}{5.891859in}}%
\pgfpathlineto{\pgfqpoint{2.878442in}{5.894389in}}%
\pgfpathlineto{\pgfqpoint{2.872595in}{5.898661in}}%
\pgfpathlineto{\pgfqpoint{2.872210in}{5.898942in}}%
\pgfpathlineto{\pgfqpoint{2.865978in}{5.903495in}}%
\pgfpathlineto{\pgfqpoint{2.863286in}{5.905463in}}%
\pgfpathlineto{\pgfqpoint{2.859747in}{5.908048in}}%
\pgfpathlineto{\pgfqpoint{2.853976in}{5.912264in}}%
\pgfpathlineto{\pgfqpoint{2.853515in}{5.912602in}}%
\pgfpathlineto{\pgfqpoint{2.847283in}{5.917155in}}%
\pgfpathlineto{\pgfqpoint{2.844667in}{5.919066in}}%
\pgfpathlineto{\pgfqpoint{2.841051in}{5.921708in}}%
\pgfpathlineto{\pgfqpoint{2.835358in}{5.925868in}}%
\pgfpathlineto{\pgfqpoint{2.834819in}{5.926261in}}%
\pgfpathlineto{\pgfqpoint{2.828587in}{5.930815in}}%
\pgfpathlineto{\pgfqpoint{2.826048in}{5.932670in}}%
\pgfpathlineto{\pgfqpoint{2.822355in}{5.935368in}}%
\pgfpathlineto{\pgfqpoint{2.816739in}{5.939472in}}%
\pgfpathlineto{\pgfqpoint{2.816124in}{5.939921in}}%
\pgfpathlineto{\pgfqpoint{2.809892in}{5.944475in}}%
\pgfpathlineto{\pgfqpoint{2.807430in}{5.946273in}}%
\pgfpathlineto{\pgfqpoint{2.803660in}{5.949028in}}%
\pgfpathlineto{\pgfqpoint{2.798120in}{5.953075in}}%
\pgfpathlineto{\pgfqpoint{2.797428in}{5.953581in}}%
\pgfpathlineto{\pgfqpoint{2.791196in}{5.958134in}}%
\pgfpathlineto{\pgfqpoint{2.788811in}{5.959877in}}%
\pgfpathlineto{\pgfqpoint{2.784964in}{5.962688in}}%
\pgfpathlineto{\pgfqpoint{2.779502in}{5.966679in}}%
\pgfpathlineto{\pgfqpoint{2.778732in}{5.967241in}}%
\pgfpathlineto{\pgfqpoint{2.772501in}{5.971794in}}%
\pgfpathlineto{\pgfqpoint{2.770192in}{5.973481in}}%
\pgfpathlineto{\pgfqpoint{2.766269in}{5.976347in}}%
\pgfpathlineto{\pgfqpoint{2.760883in}{5.980282in}}%
\pgfpathlineto{\pgfqpoint{2.760037in}{5.980901in}}%
\pgfpathlineto{\pgfqpoint{2.753805in}{5.985454in}}%
\pgfpathlineto{\pgfqpoint{2.751574in}{5.987084in}}%
\pgfpathlineto{\pgfqpoint{2.747573in}{5.990007in}}%
\pgfpathlineto{\pgfqpoint{2.742265in}{5.993886in}}%
\pgfpathlineto{\pgfqpoint{2.741341in}{5.994561in}}%
\pgfpathlineto{\pgfqpoint{2.735109in}{5.999114in}}%
\pgfpathlineto{\pgfqpoint{2.732955in}{6.000688in}}%
\pgfpathlineto{\pgfqpoint{2.728878in}{6.003667in}}%
\pgfpathlineto{\pgfqpoint{2.723646in}{6.007490in}}%
\pgfpathlineto{\pgfqpoint{2.722646in}{6.008220in}}%
\pgfpathlineto{\pgfqpoint{2.716414in}{6.012774in}}%
\pgfpathlineto{\pgfqpoint{2.714337in}{6.014291in}}%
\pgfpathlineto{\pgfqpoint{2.710182in}{6.017327in}}%
\pgfpathlineto{\pgfqpoint{2.705027in}{6.021093in}}%
\pgfpathlineto{\pgfqpoint{2.703950in}{6.021880in}}%
\pgfpathlineto{\pgfqpoint{2.697718in}{6.026433in}}%
\pgfpathlineto{\pgfqpoint{2.695718in}{6.027895in}}%
\pgfpathlineto{\pgfqpoint{2.691487in}{6.030987in}}%
\pgfpathlineto{\pgfqpoint{2.686409in}{6.034697in}}%
\pgfpathlineto{\pgfqpoint{2.685255in}{6.035540in}}%
\pgfpathlineto{\pgfqpoint{2.679023in}{6.040093in}}%
\pgfpathlineto{\pgfqpoint{2.677099in}{6.041499in}}%
\pgfpathlineto{\pgfqpoint{2.672791in}{6.044647in}}%
\pgfpathlineto{\pgfqpoint{2.667790in}{6.048300in}}%
\pgfpathlineto{\pgfqpoint{2.666559in}{6.049200in}}%
\pgfpathlineto{\pgfqpoint{2.660327in}{6.053753in}}%
\pgfpathlineto{\pgfqpoint{2.658481in}{6.055102in}}%
\pgfpathlineto{\pgfqpoint{2.654095in}{6.058306in}}%
\pgfpathlineto{\pgfqpoint{2.649171in}{6.061904in}}%
\pgfpathlineto{\pgfqpoint{2.647864in}{6.062860in}}%
\pgfpathlineto{\pgfqpoint{2.641632in}{6.067413in}}%
\pgfpathlineto{\pgfqpoint{2.639862in}{6.068706in}}%
\pgfpathlineto{\pgfqpoint{2.635400in}{6.071966in}}%
\pgfpathlineto{\pgfqpoint{2.630553in}{6.075508in}}%
\pgfpathlineto{\pgfqpoint{2.629168in}{6.076519in}}%
\pgfpathlineto{\pgfqpoint{2.622936in}{6.081073in}}%
\pgfpathlineto{\pgfqpoint{2.621244in}{6.082309in}}%
\pgfpathlineto{\pgfqpoint{2.616704in}{6.085626in}}%
\pgfpathlineto{\pgfqpoint{2.611934in}{6.089111in}}%
\pgfpathlineto{\pgfqpoint{2.610472in}{6.090179in}}%
\pgfpathlineto{\pgfqpoint{2.604241in}{6.094733in}}%
\pgfpathlineto{\pgfqpoint{2.602625in}{6.095913in}}%
\pgfpathlineto{\pgfqpoint{2.598009in}{6.099286in}}%
\pgfpathlineto{\pgfqpoint{2.593316in}{6.102715in}}%
\pgfpathlineto{\pgfqpoint{2.591777in}{6.103839in}}%
\pgfpathlineto{\pgfqpoint{2.585545in}{6.108392in}}%
\pgfpathlineto{\pgfqpoint{2.584006in}{6.109517in}}%
\pgfpathlineto{\pgfqpoint{2.579313in}{6.112946in}}%
\pgfpathlineto{\pgfqpoint{2.574697in}{6.116318in}}%
\pgfpathlineto{\pgfqpoint{2.573081in}{6.117499in}}%
\pgfpathlineto{\pgfqpoint{2.566849in}{6.122052in}}%
\pgfpathlineto{\pgfqpoint{2.565388in}{6.123120in}}%
\pgfpathlineto{\pgfqpoint{2.560618in}{6.126605in}}%
\pgfpathlineto{\pgfqpoint{2.556078in}{6.129922in}}%
\pgfpathlineto{\pgfqpoint{2.554386in}{6.131159in}}%
\pgfpathlineto{\pgfqpoint{2.548154in}{6.135712in}}%
\pgfpathlineto{\pgfqpoint{2.546769in}{6.136724in}}%
\pgfpathlineto{\pgfqpoint{2.541922in}{6.140265in}}%
\pgfpathlineto{\pgfqpoint{2.537460in}{6.143526in}}%
\pgfpathlineto{\pgfqpoint{2.535690in}{6.144819in}}%
\pgfpathlineto{\pgfqpoint{2.529458in}{6.149372in}}%
\pgfpathlineto{\pgfqpoint{2.528150in}{6.150327in}}%
\pgfpathlineto{\pgfqpoint{2.523227in}{6.153925in}}%
\pgfpathlineto{\pgfqpoint{2.518841in}{6.157129in}}%
\pgfpathlineto{\pgfqpoint{2.516995in}{6.158478in}}%
\pgfpathlineto{\pgfqpoint{2.510763in}{6.163032in}}%
\pgfpathlineto{\pgfqpoint{2.509532in}{6.163931in}}%
\pgfpathlineto{\pgfqpoint{2.504531in}{6.167585in}}%
\pgfpathlineto{\pgfqpoint{2.500223in}{6.170733in}}%
\pgfpathlineto{\pgfqpoint{2.498299in}{6.172138in}}%
\pgfpathlineto{\pgfqpoint{2.492067in}{6.176691in}}%
\pgfpathlineto{\pgfqpoint{2.490913in}{6.177535in}}%
\pgfpathlineto{\pgfqpoint{2.485835in}{6.181245in}}%
\pgfpathlineto{\pgfqpoint{2.481604in}{6.184336in}}%
\pgfpathlineto{\pgfqpoint{2.479604in}{6.185798in}}%
\pgfpathlineto{\pgfqpoint{2.473372in}{6.190351in}}%
\pgfpathlineto{\pgfqpoint{2.472295in}{6.191138in}}%
\pgfpathlineto{\pgfqpoint{2.467140in}{6.194905in}}%
\pgfpathlineto{\pgfqpoint{2.462985in}{6.197940in}}%
\pgfpathlineto{\pgfqpoint{2.460908in}{6.199458in}}%
\pgfpathlineto{\pgfqpoint{2.454676in}{6.204011in}}%
\pgfpathlineto{\pgfqpoint{2.453676in}{6.204742in}}%
\pgfpathlineto{\pgfqpoint{2.448444in}{6.208564in}}%
\pgfpathlineto{\pgfqpoint{2.444367in}{6.211544in}}%
\pgfpathlineto{\pgfqpoint{2.442212in}{6.213118in}}%
\pgfpathlineto{\pgfqpoint{2.435981in}{6.217671in}}%
\pgfpathlineto{\pgfqpoint{2.435057in}{6.218345in}}%
\pgfpathlineto{\pgfqpoint{2.429749in}{6.222224in}}%
\pgfpathlineto{\pgfqpoint{2.425748in}{6.225147in}}%
\pgfpathlineto{\pgfqpoint{2.423517in}{6.226777in}}%
\pgfpathlineto{\pgfqpoint{2.417285in}{6.231331in}}%
\pgfpathlineto{\pgfqpoint{2.416439in}{6.231949in}}%
\pgfpathlineto{\pgfqpoint{2.411053in}{6.235884in}}%
\pgfpathlineto{\pgfqpoint{2.407129in}{6.238751in}}%
\pgfpathlineto{\pgfqpoint{2.404821in}{6.240437in}}%
\pgfpathlineto{\pgfqpoint{2.398589in}{6.244990in}}%
\pgfpathlineto{\pgfqpoint{2.397820in}{6.245553in}}%
\pgfpathlineto{\pgfqpoint{2.392358in}{6.249544in}}%
\pgfpathlineto{\pgfqpoint{2.388511in}{6.252354in}}%
\pgfpathlineto{\pgfqpoint{2.386126in}{6.254097in}}%
\pgfpathlineto{\pgfqpoint{2.379894in}{6.258650in}}%
\pgfpathlineto{\pgfqpoint{2.379202in}{6.259156in}}%
\pgfpathlineto{\pgfqpoint{2.373662in}{6.263204in}}%
\pgfpathlineto{\pgfqpoint{2.369892in}{6.265958in}}%
\pgfpathlineto{\pgfqpoint{2.367430in}{6.267757in}}%
\pgfpathlineto{\pgfqpoint{2.361198in}{6.272310in}}%
\pgfpathlineto{\pgfqpoint{2.360583in}{6.272760in}}%
\pgfpathlineto{\pgfqpoint{2.354967in}{6.276863in}}%
\pgfpathlineto{\pgfqpoint{2.351274in}{6.279562in}}%
\pgfpathlineto{\pgfqpoint{2.348735in}{6.281417in}}%
\pgfpathlineto{\pgfqpoint{2.342503in}{6.285970in}}%
\pgfpathlineto{\pgfqpoint{2.341964in}{6.286363in}}%
\pgfpathlineto{\pgfqpoint{2.336271in}{6.290523in}}%
\pgfpathlineto{\pgfqpoint{2.332655in}{6.293165in}}%
\pgfpathlineto{\pgfqpoint{2.330039in}{6.295076in}}%
\pgfpathlineto{\pgfqpoint{2.323807in}{6.299630in}}%
\pgfpathlineto{\pgfqpoint{2.323346in}{6.299967in}}%
\pgfpathlineto{\pgfqpoint{2.317575in}{6.304183in}}%
\pgfpathlineto{\pgfqpoint{2.314036in}{6.306769in}}%
\pgfpathlineto{\pgfqpoint{2.311344in}{6.308736in}}%
\pgfpathlineto{\pgfqpoint{2.305112in}{6.313290in}}%
\pgfpathlineto{\pgfqpoint{2.304727in}{6.313571in}}%
\pgfpathlineto{\pgfqpoint{2.298880in}{6.317843in}}%
\pgfpathlineto{\pgfqpoint{2.295418in}{6.320372in}}%
\pgfpathlineto{\pgfqpoint{2.292648in}{6.322396in}}%
\pgfpathlineto{\pgfqpoint{2.286416in}{6.326949in}}%
\pgfpathlineto{\pgfqpoint{2.286108in}{6.327174in}}%
\pgfpathlineto{\pgfqpoint{2.280184in}{6.331503in}}%
\pgfpathlineto{\pgfqpoint{2.276799in}{6.333976in}}%
\pgfpathlineto{\pgfqpoint{2.273952in}{6.336056in}}%
\pgfpathlineto{\pgfqpoint{2.267721in}{6.340609in}}%
\pgfpathlineto{\pgfqpoint{2.267490in}{6.340778in}}%
\pgfpathlineto{\pgfqpoint{2.261489in}{6.345162in}}%
\pgfpathlineto{\pgfqpoint{2.258180in}{6.347580in}}%
\pgfpathlineto{\pgfqpoint{2.255257in}{6.349716in}}%
\pgfpathlineto{\pgfqpoint{2.249025in}{6.354269in}}%
\pgfpathlineto{\pgfqpoint{2.248871in}{6.354381in}}%
\pgfpathlineto{\pgfqpoint{2.242793in}{6.358822in}}%
\pgfpathlineto{\pgfqpoint{2.239562in}{6.361183in}}%
\pgfpathlineto{\pgfqpoint{2.236561in}{6.363376in}}%
\pgfpathlineto{\pgfqpoint{2.230329in}{6.367929in}}%
\pgfpathlineto{\pgfqpoint{2.230253in}{6.367985in}}%
\pgfpathlineto{\pgfqpoint{2.224098in}{6.372482in}}%
\pgfpathlineto{\pgfqpoint{2.220943in}{6.374787in}}%
\pgfpathlineto{\pgfqpoint{2.217866in}{6.377035in}}%
\pgfpathlineto{\pgfqpoint{2.211634in}{6.381589in}}%
\pgfpathlineto{\pgfqpoint{2.211634in}{6.381589in}}%
\pgfpathlineto{\pgfqpoint{2.205402in}{6.386142in}}%
\pgfpathlineto{\pgfqpoint{2.202325in}{6.388390in}}%
\pgfpathlineto{\pgfqpoint{2.199170in}{6.390695in}}%
\pgfpathlineto{\pgfqpoint{2.193015in}{6.395192in}}%
\pgfpathlineto{\pgfqpoint{2.192938in}{6.395248in}}%
\pgfpathlineto{\pgfqpoint{2.186707in}{6.399802in}}%
\pgfpathlineto{\pgfqpoint{2.183706in}{6.401994in}}%
\pgfpathlineto{\pgfqpoint{2.180475in}{6.404355in}}%
\pgfpathlineto{\pgfqpoint{2.174397in}{6.408796in}}%
\pgfpathlineto{\pgfqpoint{2.174243in}{6.408908in}}%
\pgfpathlineto{\pgfqpoint{2.168011in}{6.413462in}}%
\pgfpathlineto{\pgfqpoint{2.165087in}{6.415598in}}%
\pgfpathlineto{\pgfqpoint{2.161779in}{6.418015in}}%
\pgfpathlineto{\pgfqpoint{2.155778in}{6.422399in}}%
\pgfpathlineto{\pgfqpoint{2.155547in}{6.422568in}}%
\pgfpathlineto{\pgfqpoint{2.149315in}{6.427121in}}%
\pgfpathlineto{\pgfqpoint{2.146469in}{6.429201in}}%
\pgfpathlineto{\pgfqpoint{2.143084in}{6.431675in}}%
\pgfpathlineto{\pgfqpoint{2.137159in}{6.436003in}}%
\pgfpathlineto{\pgfqpoint{2.136852in}{6.436228in}}%
\pgfpathlineto{\pgfqpoint{2.130620in}{6.440781in}}%
\pgfpathlineto{\pgfqpoint{2.127850in}{6.442805in}}%
\pgfpathlineto{\pgfqpoint{2.124388in}{6.445334in}}%
\pgfpathlineto{\pgfqpoint{2.118541in}{6.449607in}}%
\pgfpathlineto{\pgfqpoint{2.118156in}{6.449888in}}%
\pgfpathlineto{\pgfqpoint{2.111924in}{6.454441in}}%
\pgfpathlineto{\pgfqpoint{2.109232in}{6.456408in}}%
\pgfpathlineto{\pgfqpoint{2.105692in}{6.458994in}}%
\pgfpathlineto{\pgfqpoint{2.099922in}{6.463210in}}%
\pgfpathlineto{\pgfqpoint{2.099461in}{6.463548in}}%
\pgfpathlineto{\pgfqpoint{2.093229in}{6.468101in}}%
\pgfpathlineto{\pgfqpoint{2.090613in}{6.470012in}}%
\pgfpathlineto{\pgfqpoint{2.086997in}{6.472654in}}%
\pgfpathlineto{\pgfqpoint{2.081304in}{6.476814in}}%
\pgfpathlineto{\pgfqpoint{2.080765in}{6.477207in}}%
\pgfpathlineto{\pgfqpoint{2.074533in}{6.481761in}}%
\pgfpathlineto{\pgfqpoint{2.071994in}{6.483616in}}%
\pgfpathlineto{\pgfqpoint{2.068301in}{6.486314in}}%
\pgfpathlineto{\pgfqpoint{2.062685in}{6.490417in}}%
\pgfpathlineto{\pgfqpoint{2.062070in}{6.490867in}}%
\pgfpathlineto{\pgfqpoint{2.055838in}{6.495420in}}%
\pgfpathlineto{\pgfqpoint{2.053376in}{6.497219in}}%
\pgfpathlineto{\pgfqpoint{2.049606in}{6.499974in}}%
\pgfpathlineto{\pgfqpoint{2.044066in}{6.504021in}}%
\pgfpathlineto{\pgfqpoint{2.043374in}{6.504527in}}%
\pgfpathlineto{\pgfqpoint{2.037142in}{6.509080in}}%
\pgfpathlineto{\pgfqpoint{2.034757in}{6.510823in}}%
\pgfpathlineto{\pgfqpoint{2.030910in}{6.513634in}}%
\pgfpathlineto{\pgfqpoint{2.025448in}{6.517625in}}%
\pgfpathlineto{\pgfqpoint{2.024678in}{6.518187in}}%
\pgfpathlineto{\pgfqpoint{2.018447in}{6.522740in}}%
\pgfpathlineto{\pgfqpoint{2.016138in}{6.524427in}}%
\pgfpathlineto{\pgfqpoint{2.012215in}{6.527293in}}%
\pgfpathlineto{\pgfqpoint{2.006829in}{6.531228in}}%
\pgfpathlineto{\pgfqpoint{2.005983in}{6.531847in}}%
\pgfpathlineto{\pgfqpoint{1.999751in}{6.536400in}}%
\pgfpathlineto{\pgfqpoint{1.997520in}{6.538030in}}%
\pgfpathlineto{\pgfqpoint{1.993519in}{6.540953in}}%
\pgfpathlineto{\pgfqpoint{1.988211in}{6.544832in}}%
\pgfpathlineto{\pgfqpoint{1.987287in}{6.545506in}}%
\pgfpathlineto{\pgfqpoint{1.981055in}{6.550060in}}%
\pgfpathlineto{\pgfqpoint{1.978901in}{6.551634in}}%
\pgfpathlineto{\pgfqpoint{1.974824in}{6.554613in}}%
\pgfpathlineto{\pgfqpoint{1.969592in}{6.558436in}}%
\pgfpathlineto{\pgfqpoint{1.968592in}{6.559166in}}%
\pgfpathlineto{\pgfqpoint{1.962360in}{6.563720in}}%
\pgfpathlineto{\pgfqpoint{1.960283in}{6.565237in}}%
\pgfpathlineto{\pgfqpoint{1.956128in}{6.568273in}}%
\pgfpathlineto{\pgfqpoint{1.950973in}{6.572039in}}%
\pgfpathlineto{\pgfqpoint{1.949896in}{6.572826in}}%
\pgfpathlineto{\pgfqpoint{1.943664in}{6.577379in}}%
\pgfpathlineto{\pgfqpoint{1.941664in}{6.578841in}}%
\pgfpathlineto{\pgfqpoint{1.937432in}{6.581933in}}%
\pgfpathlineto{\pgfqpoint{1.932355in}{6.585643in}}%
\pgfpathlineto{\pgfqpoint{1.931201in}{6.586486in}}%
\pgfpathlineto{\pgfqpoint{1.924969in}{6.591039in}}%
\pgfpathlineto{\pgfqpoint{1.923045in}{6.592445in}}%
\pgfpathlineto{\pgfqpoint{1.918737in}{6.595592in}}%
\pgfpathlineto{\pgfqpoint{1.913736in}{6.599246in}}%
\pgfpathlineto{\pgfqpoint{1.912505in}{6.600146in}}%
\pgfpathlineto{\pgfqpoint{1.906273in}{6.604699in}}%
\pgfpathlineto{\pgfqpoint{1.904427in}{6.606048in}}%
\pgfpathlineto{\pgfqpoint{1.900041in}{6.609252in}}%
\pgfpathlineto{\pgfqpoint{1.895117in}{6.612850in}}%
\pgfpathlineto{\pgfqpoint{1.893810in}{6.613806in}}%
\pgfpathlineto{\pgfqpoint{1.887578in}{6.618359in}}%
\pgfpathlineto{\pgfqpoint{1.885808in}{6.619652in}}%
\pgfpathlineto{\pgfqpoint{1.881346in}{6.622912in}}%
\pgfpathlineto{\pgfqpoint{1.876499in}{6.626454in}}%
\pgfpathlineto{\pgfqpoint{1.875114in}{6.627465in}}%
\pgfpathlineto{\pgfqpoint{1.868882in}{6.632019in}}%
\pgfpathlineto{\pgfqpoint{1.867189in}{6.633255in}}%
\pgfpathlineto{\pgfqpoint{1.862650in}{6.636572in}}%
\pgfpathlineto{\pgfqpoint{1.857880in}{6.640057in}}%
\pgfpathlineto{\pgfqpoint{1.856418in}{6.641125in}}%
\pgfpathlineto{\pgfqpoint{1.850187in}{6.645678in}}%
\pgfpathlineto{\pgfqpoint{1.848571in}{6.646859in}}%
\pgfpathlineto{\pgfqpoint{1.843955in}{6.650232in}}%
\pgfpathlineto{\pgfqpoint{1.839262in}{6.653661in}}%
\pgfpathlineto{\pgfqpoint{1.837723in}{6.654785in}}%
\pgfpathlineto{\pgfqpoint{1.831491in}{6.659338in}}%
\pgfpathlineto{\pgfqpoint{1.829952in}{6.660463in}}%
\pgfpathlineto{\pgfqpoint{1.825259in}{6.663892in}}%
\pgfpathlineto{\pgfqpoint{1.820643in}{6.667264in}}%
\pgfpathlineto{\pgfqpoint{1.819027in}{6.668445in}}%
\pgfpathlineto{\pgfqpoint{1.812795in}{6.672998in}}%
\pgfpathlineto{\pgfqpoint{1.811334in}{6.674066in}}%
\pgfpathlineto{\pgfqpoint{1.806564in}{6.677551in}}%
\pgfpathlineto{\pgfqpoint{1.802024in}{6.680868in}}%
\pgfpathlineto{\pgfqpoint{1.800332in}{6.682105in}}%
\pgfpathlineto{\pgfqpoint{1.794100in}{6.686658in}}%
\pgfpathlineto{\pgfqpoint{1.792715in}{6.687670in}}%
\pgfpathlineto{\pgfqpoint{1.787868in}{6.691211in}}%
\pgfpathlineto{\pgfqpoint{1.783406in}{6.694472in}}%
\pgfpathlineto{\pgfqpoint{1.781636in}{6.695764in}}%
\pgfpathlineto{\pgfqpoint{1.775404in}{6.700318in}}%
\pgfpathlineto{\pgfqpoint{1.774096in}{6.701273in}}%
\pgfpathlineto{\pgfqpoint{1.769172in}{6.704871in}}%
\pgfpathlineto{\pgfqpoint{1.764787in}{6.708075in}}%
\pgfpathlineto{\pgfqpoint{1.762941in}{6.709424in}}%
\pgfpathlineto{\pgfqpoint{1.756709in}{6.713978in}}%
\pgfpathlineto{\pgfqpoint{1.755478in}{6.714877in}}%
\pgfpathlineto{\pgfqpoint{1.750477in}{6.718531in}}%
\pgfpathlineto{\pgfqpoint{1.746168in}{6.721679in}}%
\pgfpathlineto{\pgfqpoint{1.744245in}{6.723084in}}%
\pgfpathlineto{\pgfqpoint{1.738013in}{6.727637in}}%
\pgfpathlineto{\pgfqpoint{1.736859in}{6.728481in}}%
\pgfpathlineto{\pgfqpoint{1.731781in}{6.732191in}}%
\pgfpathlineto{\pgfqpoint{1.727550in}{6.735282in}}%
\pgfpathlineto{\pgfqpoint{1.725550in}{6.736744in}}%
\pgfpathlineto{\pgfqpoint{1.719318in}{6.741297in}}%
\pgfpathlineto{\pgfqpoint{1.718241in}{6.742084in}}%
\pgfpathlineto{\pgfqpoint{1.713086in}{6.745850in}}%
\pgfpathlineto{\pgfqpoint{1.708931in}{6.748886in}}%
\pgfpathlineto{\pgfqpoint{1.706854in}{6.750404in}}%
\pgfpathlineto{\pgfqpoint{1.700622in}{6.754957in}}%
\pgfpathlineto{\pgfqpoint{1.699622in}{6.755688in}}%
\pgfpathlineto{\pgfqpoint{1.694390in}{6.759510in}}%
\pgfpathlineto{\pgfqpoint{1.690313in}{6.762490in}}%
\pgfpathlineto{\pgfqpoint{1.688158in}{6.764064in}}%
\pgfpathlineto{\pgfqpoint{1.681927in}{6.768617in}}%
\pgfpathlineto{\pgfqpoint{1.681003in}{6.769291in}}%
\pgfpathlineto{\pgfqpoint{1.675695in}{6.773170in}}%
\pgfpathlineto{\pgfqpoint{1.671694in}{6.776093in}}%
\pgfpathlineto{\pgfqpoint{1.669463in}{6.777723in}}%
\pgfpathlineto{\pgfqpoint{1.663231in}{6.782277in}}%
\pgfpathlineto{\pgfqpoint{1.662385in}{6.782895in}}%
\pgfpathlineto{\pgfqpoint{1.656999in}{6.786830in}}%
\pgfpathlineto{\pgfqpoint{1.653075in}{6.789697in}}%
\pgfpathlineto{\pgfqpoint{1.650767in}{6.791383in}}%
\pgfpathlineto{\pgfqpoint{1.644535in}{6.795936in}}%
\pgfpathlineto{\pgfqpoint{1.643766in}{6.796499in}}%
\pgfpathlineto{\pgfqpoint{1.638304in}{6.800490in}}%
\pgfpathlineto{\pgfqpoint{1.634457in}{6.803300in}}%
\pgfpathlineto{\pgfqpoint{1.632072in}{6.805043in}}%
\pgfpathlineto{\pgfqpoint{1.625840in}{6.809596in}}%
\pgfpathlineto{\pgfqpoint{1.625147in}{6.810102in}}%
\pgfpathlineto{\pgfqpoint{1.619608in}{6.814150in}}%
\pgfpathlineto{\pgfqpoint{1.615838in}{6.816904in}}%
\pgfpathlineto{\pgfqpoint{1.613376in}{6.818703in}}%
\pgfpathlineto{\pgfqpoint{1.607144in}{6.823256in}}%
\pgfpathlineto{\pgfqpoint{1.606529in}{6.823706in}}%
\pgfpathlineto{\pgfqpoint{1.600912in}{6.827809in}}%
\pgfpathlineto{\pgfqpoint{1.597220in}{6.830508in}}%
\pgfpathlineto{\pgfqpoint{1.594681in}{6.832363in}}%
\pgfpathlineto{\pgfqpoint{1.588449in}{6.836916in}}%
\pgfpathlineto{\pgfqpoint{1.587910in}{6.837309in}}%
\pgfpathlineto{\pgfqpoint{1.582217in}{6.841469in}}%
\pgfpathlineto{\pgfqpoint{1.578601in}{6.844111in}}%
\pgfpathlineto{\pgfqpoint{1.575985in}{6.846022in}}%
\pgfpathlineto{\pgfqpoint{1.569753in}{6.850576in}}%
\pgfpathlineto{\pgfqpoint{1.569292in}{6.850913in}}%
\pgfpathlineto{\pgfqpoint{1.563521in}{6.855129in}}%
\pgfpathlineto{\pgfqpoint{1.559982in}{6.857715in}}%
\pgfpathlineto{\pgfqpoint{1.557290in}{6.859682in}}%
\pgfpathlineto{\pgfqpoint{1.551058in}{6.864236in}}%
\pgfpathlineto{\pgfqpoint{1.550673in}{6.864517in}}%
\pgfpathlineto{\pgfqpoint{1.544826in}{6.868789in}}%
\pgfpathlineto{\pgfqpoint{1.541364in}{6.871318in}}%
\pgfpathlineto{\pgfqpoint{1.538594in}{6.873342in}}%
\pgfpathlineto{\pgfqpoint{1.532362in}{6.877895in}}%
\pgfpathlineto{\pgfqpoint{1.532054in}{6.878120in}}%
\pgfpathlineto{\pgfqpoint{1.526130in}{6.882449in}}%
\pgfpathlineto{\pgfqpoint{1.522745in}{6.884922in}}%
\pgfpathlineto{\pgfqpoint{1.519898in}{6.887002in}}%
\pgfpathlineto{\pgfqpoint{1.513667in}{6.891555in}}%
\pgfpathlineto{\pgfqpoint{1.513436in}{6.891724in}}%
\pgfpathlineto{\pgfqpoint{1.507435in}{6.896108in}}%
\pgfpathlineto{\pgfqpoint{1.504126in}{6.898526in}}%
\pgfpathlineto{\pgfqpoint{1.501203in}{6.900662in}}%
\pgfpathlineto{\pgfqpoint{1.494971in}{6.905215in}}%
\pgfpathlineto{\pgfqpoint{1.494817in}{6.905327in}}%
\pgfpathlineto{\pgfqpoint{1.488739in}{6.909768in}}%
\pgfpathlineto{\pgfqpoint{1.485508in}{6.912129in}}%
\pgfpathlineto{\pgfqpoint{1.482507in}{6.914322in}}%
\pgfpathlineto{\pgfqpoint{1.476275in}{6.918875in}}%
\pgfpathlineto{\pgfqpoint{1.476199in}{6.918931in}}%
\pgfpathlineto{\pgfqpoint{1.470044in}{6.923428in}}%
\pgfpathlineto{\pgfqpoint{1.466889in}{6.925733in}}%
\pgfpathlineto{\pgfqpoint{1.463812in}{6.927981in}}%
\pgfpathlineto{\pgfqpoint{1.457580in}{6.932535in}}%
\pgfpathlineto{\pgfqpoint{1.457580in}{6.932535in}}%
\pgfpathlineto{\pgfqpoint{1.451348in}{6.937088in}}%
\pgfpathlineto{\pgfqpoint{1.448271in}{6.939336in}}%
\pgfpathlineto{\pgfqpoint{1.445116in}{6.941641in}}%
\pgfpathlineto{\pgfqpoint{1.438961in}{6.946138in}}%
\pgfpathlineto{\pgfqpoint{1.438884in}{6.946194in}}%
\pgfpathlineto{\pgfqpoint{1.432652in}{6.950748in}}%
\pgfpathlineto{\pgfqpoint{1.429652in}{6.952940in}}%
\pgfpathlineto{\pgfqpoint{1.426421in}{6.955301in}}%
\pgfpathlineto{\pgfqpoint{1.420343in}{6.959742in}}%
\pgfpathlineto{\pgfqpoint{1.420189in}{6.959854in}}%
\pgfpathlineto{\pgfqpoint{1.413957in}{6.964408in}}%
\pgfpathlineto{\pgfqpoint{1.411033in}{6.966544in}}%
\pgfpathlineto{\pgfqpoint{1.407725in}{6.968961in}}%
\pgfpathlineto{\pgfqpoint{1.401724in}{6.973345in}}%
\pgfpathlineto{\pgfqpoint{1.401493in}{6.973514in}}%
\pgfpathlineto{\pgfqpoint{1.395261in}{6.978067in}}%
\pgfpathlineto{\pgfqpoint{1.392415in}{6.980147in}}%
\pgfpathlineto{\pgfqpoint{1.389030in}{6.982621in}}%
\pgfpathlineto{\pgfqpoint{1.383105in}{6.986949in}}%
\pgfpathlineto{\pgfqpoint{1.382798in}{6.987174in}}%
\pgfpathlineto{\pgfqpoint{1.376566in}{6.991727in}}%
\pgfpathlineto{\pgfqpoint{1.373796in}{6.993751in}}%
\pgfpathlineto{\pgfqpoint{1.370334in}{6.996280in}}%
\pgfpathlineto{\pgfqpoint{1.364487in}{7.000553in}}%
\pgfpathlineto{\pgfqpoint{1.364102in}{7.000834in}}%
\pgfpathlineto{\pgfqpoint{1.357870in}{7.005387in}}%
\pgfpathlineto{\pgfqpoint{1.355177in}{7.007354in}}%
\pgfpathlineto{\pgfqpoint{1.351638in}{7.009940in}}%
\pgfpathlineto{\pgfqpoint{1.345868in}{7.014156in}}%
\pgfpathlineto{\pgfqpoint{1.345407in}{7.014494in}}%
\pgfpathlineto{\pgfqpoint{1.339175in}{7.019047in}}%
\pgfpathlineto{\pgfqpoint{1.336559in}{7.020958in}}%
\pgfpathlineto{\pgfqpoint{1.332943in}{7.023600in}}%
\pgfpathlineto{\pgfqpoint{1.327250in}{7.027760in}}%
\pgfpathlineto{\pgfqpoint{1.326711in}{7.028153in}}%
\pgfpathlineto{\pgfqpoint{1.320479in}{7.032707in}}%
\pgfpathlineto{\pgfqpoint{1.317940in}{7.034562in}}%
\pgfpathlineto{\pgfqpoint{1.314247in}{7.037260in}}%
\pgfpathlineto{\pgfqpoint{1.308631in}{7.041363in}}%
\pgfpathlineto{\pgfqpoint{1.308015in}{7.041813in}}%
\pgfpathlineto{\pgfqpoint{1.301784in}{7.046366in}}%
\pgfpathlineto{\pgfqpoint{1.299322in}{7.048165in}}%
\pgfpathlineto{\pgfqpoint{1.295552in}{7.050920in}}%
\pgfpathlineto{\pgfqpoint{1.290012in}{7.054967in}}%
\pgfpathlineto{\pgfqpoint{1.289320in}{7.055473in}}%
\pgfpathlineto{\pgfqpoint{1.283088in}{7.060026in}}%
\pgfpathlineto{\pgfqpoint{1.280703in}{7.061769in}}%
\pgfpathlineto{\pgfqpoint{1.276856in}{7.064580in}}%
\pgfpathlineto{\pgfqpoint{1.271394in}{7.068571in}}%
\pgfpathlineto{\pgfqpoint{1.270624in}{7.069133in}}%
\pgfpathlineto{\pgfqpoint{1.264392in}{7.073686in}}%
\pgfpathlineto{\pgfqpoint{1.262084in}{7.075372in}}%
\pgfpathlineto{\pgfqpoint{1.258161in}{7.078239in}}%
\pgfpathlineto{\pgfqpoint{1.252775in}{7.082174in}}%
\pgfpathlineto{\pgfqpoint{1.251929in}{7.082793in}}%
\pgfpathlineto{\pgfqpoint{1.245697in}{7.087346in}}%
\pgfpathlineto{\pgfqpoint{1.243466in}{7.088976in}}%
\pgfpathlineto{\pgfqpoint{1.239465in}{7.091899in}}%
\pgfpathlineto{\pgfqpoint{1.234156in}{7.095778in}}%
\pgfpathlineto{\pgfqpoint{1.233233in}{7.096452in}}%
\pgfpathlineto{\pgfqpoint{1.227001in}{7.101006in}}%
\pgfpathlineto{\pgfqpoint{1.224847in}{7.102580in}}%
\pgfpathlineto{\pgfqpoint{1.220770in}{7.105559in}}%
\pgfpathlineto{\pgfqpoint{1.215538in}{7.109381in}}%
\pgfpathlineto{\pgfqpoint{1.214538in}{7.110112in}}%
\pgfpathlineto{\pgfqpoint{1.208306in}{7.114666in}}%
\pgfpathlineto{\pgfqpoint{1.206229in}{7.116183in}}%
\pgfpathlineto{\pgfqpoint{1.202074in}{7.119219in}}%
\pgfpathlineto{\pgfqpoint{1.196919in}{7.122985in}}%
\pgfpathlineto{\pgfqpoint{1.195842in}{7.123772in}}%
\pgfpathlineto{\pgfqpoint{1.189610in}{7.128325in}}%
\pgfpathlineto{\pgfqpoint{1.187610in}{7.129787in}}%
\pgfpathlineto{\pgfqpoint{1.183378in}{7.132879in}}%
\pgfpathlineto{\pgfqpoint{1.178301in}{7.136589in}}%
\pgfpathlineto{\pgfqpoint{1.177147in}{7.137432in}}%
\pgfpathlineto{\pgfqpoint{1.170915in}{7.141985in}}%
\pgfpathlineto{\pgfqpoint{1.168991in}{7.143390in}}%
\pgfpathlineto{\pgfqpoint{1.164683in}{7.146538in}}%
\pgfpathlineto{\pgfqpoint{1.159682in}{7.150192in}}%
\pgfpathlineto{\pgfqpoint{1.158451in}{7.151092in}}%
\pgfpathlineto{\pgfqpoint{1.152219in}{7.155645in}}%
\pgfpathlineto{\pgfqpoint{1.150373in}{7.156994in}}%
\pgfpathlineto{\pgfqpoint{1.145987in}{7.160198in}}%
\pgfpathlineto{\pgfqpoint{1.141063in}{7.163796in}}%
\pgfpathlineto{\pgfqpoint{1.139755in}{7.164751in}}%
\pgfpathlineto{\pgfqpoint{1.133524in}{7.169305in}}%
\pgfpathlineto{\pgfqpoint{1.131754in}{7.170598in}}%
\pgfpathlineto{\pgfqpoint{1.127292in}{7.173858in}}%
\pgfpathlineto{\pgfqpoint{1.122445in}{7.177399in}}%
\pgfpathlineto{\pgfqpoint{1.121060in}{7.178411in}}%
\pgfpathlineto{\pgfqpoint{1.114828in}{7.182965in}}%
\pgfpathlineto{\pgfqpoint{1.113135in}{7.184201in}}%
\pgfpathlineto{\pgfqpoint{1.108596in}{7.187518in}}%
\pgfpathlineto{\pgfqpoint{1.103826in}{7.191003in}}%
\pgfpathlineto{\pgfqpoint{1.102364in}{7.192071in}}%
\pgfpathlineto{\pgfqpoint{1.096132in}{7.196624in}}%
\pgfpathlineto{\pgfqpoint{1.094517in}{7.197805in}}%
\pgfpathlineto{\pgfqpoint{1.089901in}{7.201178in}}%
\pgfpathlineto{\pgfqpoint{1.085208in}{7.204607in}}%
\pgfpathlineto{\pgfqpoint{1.083669in}{7.205731in}}%
\pgfpathlineto{\pgfqpoint{1.077437in}{7.210284in}}%
\pgfpathlineto{\pgfqpoint{1.075898in}{7.211408in}}%
\pgfpathlineto{\pgfqpoint{1.071205in}{7.214837in}}%
\pgfpathlineto{\pgfqpoint{1.066589in}{7.218210in}}%
\pgfpathlineto{\pgfqpoint{1.064973in}{7.219391in}}%
\pgfpathlineto{\pgfqpoint{1.058741in}{7.223944in}}%
\pgfpathlineto{\pgfqpoint{1.057280in}{7.225012in}}%
\pgfpathlineto{\pgfqpoint{1.052510in}{7.228497in}}%
\pgfpathlineto{\pgfqpoint{1.047970in}{7.231814in}}%
\pgfpathlineto{\pgfqpoint{1.046278in}{7.233051in}}%
\pgfpathlineto{\pgfqpoint{1.040046in}{7.237604in}}%
\pgfpathlineto{\pgfqpoint{1.038661in}{7.238616in}}%
\pgfpathlineto{\pgfqpoint{1.033814in}{7.242157in}}%
\pgfpathlineto{\pgfqpoint{1.029352in}{7.245417in}}%
\pgfpathlineto{\pgfqpoint{1.027582in}{7.246710in}}%
\pgfpathlineto{\pgfqpoint{1.021350in}{7.251264in}}%
\pgfpathlineto{\pgfqpoint{1.020042in}{7.252219in}}%
\pgfpathlineto{\pgfqpoint{1.015118in}{7.255817in}}%
\pgfpathlineto{\pgfqpoint{1.010733in}{7.259021in}}%
\pgfpathlineto{\pgfqpoint{1.008887in}{7.260370in}}%
\pgfpathlineto{\pgfqpoint{1.002655in}{7.264923in}}%
\pgfpathlineto{\pgfqpoint{1.001424in}{7.265823in}}%
\pgfpathlineto{\pgfqpoint{0.996423in}{7.269477in}}%
\pgfpathlineto{\pgfqpoint{0.992114in}{7.272625in}}%
\pgfpathlineto{\pgfqpoint{0.990191in}{7.274030in}}%
\pgfpathlineto{\pgfqpoint{0.983959in}{7.278583in}}%
\pgfpathlineto{\pgfqpoint{0.982805in}{7.279427in}}%
\pgfpathlineto{\pgfqpoint{0.977727in}{7.283137in}}%
\pgfpathlineto{\pgfqpoint{0.973496in}{7.286228in}}%
\pgfpathlineto{\pgfqpoint{0.971495in}{7.287690in}}%
\pgfpathlineto{\pgfqpoint{0.965264in}{7.292243in}}%
\pgfpathlineto{\pgfqpoint{0.964186in}{7.293030in}}%
\pgfpathlineto{\pgfqpoint{0.959032in}{7.296796in}}%
\pgfpathlineto{\pgfqpoint{0.954877in}{7.299832in}}%
\pgfpathlineto{\pgfqpoint{0.952800in}{7.301350in}}%
\pgfpathlineto{\pgfqpoint{0.946568in}{7.305903in}}%
\pgfpathlineto{\pgfqpoint{0.945568in}{7.306634in}}%
\pgfpathlineto{\pgfqpoint{0.940336in}{7.310456in}}%
\pgfpathlineto{\pgfqpoint{0.936259in}{7.313436in}}%
\pgfpathlineto{\pgfqpoint{0.934104in}{7.315009in}}%
\pgfpathlineto{\pgfqpoint{0.927872in}{7.319563in}}%
\pgfpathlineto{\pgfqpoint{0.926949in}{7.320237in}}%
\pgfpathlineto{\pgfqpoint{0.921641in}{7.324116in}}%
\pgfpathlineto{\pgfqpoint{0.917640in}{7.327039in}}%
\pgfpathlineto{\pgfqpoint{0.915409in}{7.328669in}}%
\pgfpathlineto{\pgfqpoint{0.909177in}{7.333223in}}%
\pgfpathlineto{\pgfqpoint{0.908331in}{7.333841in}}%
\pgfpathlineto{\pgfqpoint{0.902945in}{7.337776in}}%
\pgfpathlineto{\pgfqpoint{0.899021in}{7.340643in}}%
\pgfpathlineto{\pgfqpoint{0.896713in}{7.342329in}}%
\pgfpathlineto{\pgfqpoint{0.890481in}{7.346882in}}%
\pgfpathlineto{\pgfqpoint{0.889712in}{7.347445in}}%
\pgfpathlineto{\pgfqpoint{0.884250in}{7.351436in}}%
\pgfpathlineto{\pgfqpoint{0.880403in}{7.354246in}}%
\pgfpathlineto{\pgfqpoint{0.878018in}{7.355989in}}%
\pgfpathlineto{\pgfqpoint{0.871786in}{7.360542in}}%
\pgfpathlineto{\pgfqpoint{0.871093in}{7.361048in}}%
\pgfpathlineto{\pgfqpoint{0.865554in}{7.365095in}}%
\pgfpathlineto{\pgfqpoint{0.861784in}{7.367850in}}%
\pgfpathlineto{\pgfqpoint{0.859322in}{7.369649in}}%
\pgfpathlineto{\pgfqpoint{0.853090in}{7.374202in}}%
\pgfpathlineto{\pgfqpoint{0.852475in}{7.374652in}}%
\pgfpathlineto{\pgfqpoint{0.846858in}{7.378755in}}%
\pgfpathlineto{\pgfqpoint{0.843165in}{7.381454in}}%
\pgfpathlineto{\pgfqpoint{0.840627in}{7.383309in}}%
\pgfpathlineto{\pgfqpoint{0.834395in}{7.387862in}}%
\pgfpathlineto{\pgfqpoint{0.833856in}{7.388255in}}%
\pgfpathlineto{\pgfqpoint{0.828163in}{7.392415in}}%
\pgfpathlineto{\pgfqpoint{0.824547in}{7.395057in}}%
\pgfpathlineto{\pgfqpoint{0.821931in}{7.396968in}}%
\pgfpathlineto{\pgfqpoint{0.815699in}{7.401522in}}%
\pgfpathlineto{\pgfqpoint{0.815238in}{7.401859in}}%
\pgfpathlineto{\pgfqpoint{0.809467in}{7.406075in}}%
\pgfpathlineto{\pgfqpoint{0.805928in}{7.408661in}}%
\pgfpathlineto{\pgfqpoint{0.803235in}{7.410628in}}%
\pgfpathlineto{\pgfqpoint{0.797004in}{7.415181in}}%
\pgfpathlineto{\pgfqpoint{0.796619in}{7.415463in}}%
\pgfpathlineto{\pgfqpoint{0.790772in}{7.419735in}}%
\pgfpathlineto{\pgfqpoint{0.787310in}{7.422264in}}%
\pgfpathlineto{\pgfqpoint{0.784540in}{7.424288in}}%
\pgfpathlineto{\pgfqpoint{0.778308in}{7.428841in}}%
\pgfpathlineto{\pgfqpoint{0.778000in}{7.429066in}}%
\pgfpathlineto{\pgfqpoint{0.772076in}{7.433395in}}%
\pgfpathlineto{\pgfqpoint{0.768691in}{7.435868in}}%
\pgfpathlineto{\pgfqpoint{0.765844in}{7.437948in}}%
\pgfpathlineto{\pgfqpoint{0.759612in}{7.442501in}}%
\pgfpathlineto{\pgfqpoint{0.759382in}{7.442670in}}%
\pgfpathlineto{\pgfqpoint{0.753381in}{7.447054in}}%
\pgfpathlineto{\pgfqpoint{0.750072in}{7.449472in}}%
\pgfpathlineto{\pgfqpoint{0.747149in}{7.451608in}}%
\pgfpathlineto{\pgfqpoint{0.740917in}{7.456161in}}%
\pgfpathlineto{\pgfqpoint{0.740763in}{7.456273in}}%
\pgfpathlineto{\pgfqpoint{0.734685in}{7.460714in}}%
\pgfpathlineto{\pgfqpoint{0.731454in}{7.463075in}}%
\pgfpathlineto{\pgfqpoint{0.728453in}{7.465267in}}%
\pgfpathlineto{\pgfqpoint{0.722221in}{7.469821in}}%
\pgfpathlineto{\pgfqpoint{0.722144in}{7.469877in}}%
\pgfpathlineto{\pgfqpoint{0.715990in}{7.474374in}}%
\pgfpathlineto{\pgfqpoint{0.712835in}{7.476679in}}%
\pgfpathlineto{\pgfqpoint{0.709758in}{7.478927in}}%
\pgfpathlineto{\pgfqpoint{0.703526in}{7.483481in}}%
\pgfpathlineto{\pgfqpoint{0.703526in}{7.483481in}}%
\pgfpathclose%
\pgfusepath{fill}%
\end{pgfscope}%
\begin{pgfscope}%
\pgfpathrectangle{\pgfqpoint{0.703526in}{0.688481in}}{\pgfqpoint{9.300000in}{6.795000in}}%
\pgfusepath{clip}%
\pgfsetbuttcap%
\pgfsetroundjoin%
\definecolor{currentfill}{rgb}{0.839216,0.152941,0.156863}%
\pgfsetfillcolor{currentfill}%
\pgfsetfillopacity{0.100000}%
\pgfsetlinewidth{1.003750pt}%
\definecolor{currentstroke}{rgb}{0.839216,0.152941,0.156863}%
\pgfsetstrokecolor{currentstroke}%
\pgfsetstrokeopacity{0.100000}%
\pgfsetdash{}{0pt}%
\pgfpathmoveto{\pgfqpoint{0.703526in}{11.235000in}}%
\pgfpathlineto{\pgfqpoint{0.703526in}{6.464899in}}%
\pgfpathlineto{\pgfqpoint{0.703526in}{6.464899in}}%
\pgfpathlineto{\pgfqpoint{0.708074in}{6.459117in}}%
\pgfpathlineto{\pgfqpoint{0.711831in}{6.454342in}}%
\pgfpathlineto{\pgfqpoint{0.712623in}{6.453334in}}%
\pgfpathlineto{\pgfqpoint{0.717172in}{6.447552in}}%
\pgfpathlineto{\pgfqpoint{0.720136in}{6.443785in}}%
\pgfpathlineto{\pgfqpoint{0.721720in}{6.441770in}}%
\pgfpathlineto{\pgfqpoint{0.726269in}{6.435988in}}%
\pgfpathlineto{\pgfqpoint{0.728440in}{6.433228in}}%
\pgfpathlineto{\pgfqpoint{0.730818in}{6.430206in}}%
\pgfpathlineto{\pgfqpoint{0.735366in}{6.424423in}}%
\pgfpathlineto{\pgfqpoint{0.736745in}{6.422671in}}%
\pgfpathlineto{\pgfqpoint{0.739915in}{6.418641in}}%
\pgfpathlineto{\pgfqpoint{0.744463in}{6.412859in}}%
\pgfpathlineto{\pgfqpoint{0.745050in}{6.412113in}}%
\pgfpathlineto{\pgfqpoint{0.749012in}{6.407077in}}%
\pgfpathlineto{\pgfqpoint{0.753355in}{6.401556in}}%
\pgfpathlineto{\pgfqpoint{0.753561in}{6.401295in}}%
\pgfpathlineto{\pgfqpoint{0.758109in}{6.395512in}}%
\pgfpathlineto{\pgfqpoint{0.761660in}{6.390999in}}%
\pgfpathlineto{\pgfqpoint{0.762658in}{6.389730in}}%
\pgfpathlineto{\pgfqpoint{0.767207in}{6.383948in}}%
\pgfpathlineto{\pgfqpoint{0.769965in}{6.380442in}}%
\pgfpathlineto{\pgfqpoint{0.771755in}{6.378166in}}%
\pgfpathlineto{\pgfqpoint{0.776304in}{6.372384in}}%
\pgfpathlineto{\pgfqpoint{0.778269in}{6.369885in}}%
\pgfpathlineto{\pgfqpoint{0.780853in}{6.366601in}}%
\pgfpathlineto{\pgfqpoint{0.785401in}{6.360819in}}%
\pgfpathlineto{\pgfqpoint{0.786574in}{6.359328in}}%
\pgfpathlineto{\pgfqpoint{0.789950in}{6.355037in}}%
\pgfpathlineto{\pgfqpoint{0.794498in}{6.349255in}}%
\pgfpathlineto{\pgfqpoint{0.794879in}{6.348771in}}%
\pgfpathlineto{\pgfqpoint{0.799047in}{6.343473in}}%
\pgfpathlineto{\pgfqpoint{0.803184in}{6.338214in}}%
\pgfpathlineto{\pgfqpoint{0.803596in}{6.337690in}}%
\pgfpathlineto{\pgfqpoint{0.808144in}{6.331908in}}%
\pgfpathlineto{\pgfqpoint{0.811489in}{6.327657in}}%
\pgfpathlineto{\pgfqpoint{0.812693in}{6.326126in}}%
\pgfpathlineto{\pgfqpoint{0.817242in}{6.320344in}}%
\pgfpathlineto{\pgfqpoint{0.819794in}{6.317100in}}%
\pgfpathlineto{\pgfqpoint{0.821790in}{6.314562in}}%
\pgfpathlineto{\pgfqpoint{0.826339in}{6.308779in}}%
\pgfpathlineto{\pgfqpoint{0.828098in}{6.306543in}}%
\pgfpathlineto{\pgfqpoint{0.830887in}{6.302997in}}%
\pgfpathlineto{\pgfqpoint{0.835436in}{6.297215in}}%
\pgfpathlineto{\pgfqpoint{0.836403in}{6.295986in}}%
\pgfpathlineto{\pgfqpoint{0.839985in}{6.291433in}}%
\pgfpathlineto{\pgfqpoint{0.844533in}{6.285651in}}%
\pgfpathlineto{\pgfqpoint{0.844708in}{6.285429in}}%
\pgfpathlineto{\pgfqpoint{0.849082in}{6.279868in}}%
\pgfpathlineto{\pgfqpoint{0.853013in}{6.274871in}}%
\pgfpathlineto{\pgfqpoint{0.853631in}{6.274086in}}%
\pgfpathlineto{\pgfqpoint{0.858179in}{6.268304in}}%
\pgfpathlineto{\pgfqpoint{0.861318in}{6.264314in}}%
\pgfpathlineto{\pgfqpoint{0.862728in}{6.262522in}}%
\pgfpathlineto{\pgfqpoint{0.867276in}{6.256740in}}%
\pgfpathlineto{\pgfqpoint{0.869623in}{6.253757in}}%
\pgfpathlineto{\pgfqpoint{0.871825in}{6.250957in}}%
\pgfpathlineto{\pgfqpoint{0.876374in}{6.245175in}}%
\pgfpathlineto{\pgfqpoint{0.877927in}{6.243200in}}%
\pgfpathlineto{\pgfqpoint{0.880922in}{6.239393in}}%
\pgfpathlineto{\pgfqpoint{0.885471in}{6.233611in}}%
\pgfpathlineto{\pgfqpoint{0.886232in}{6.232643in}}%
\pgfpathlineto{\pgfqpoint{0.890020in}{6.227829in}}%
\pgfpathlineto{\pgfqpoint{0.894537in}{6.222086in}}%
\pgfpathlineto{\pgfqpoint{0.894568in}{6.222046in}}%
\pgfpathlineto{\pgfqpoint{0.899117in}{6.216264in}}%
\pgfpathlineto{\pgfqpoint{0.902842in}{6.211529in}}%
\pgfpathlineto{\pgfqpoint{0.903665in}{6.210482in}}%
\pgfpathlineto{\pgfqpoint{0.908214in}{6.204700in}}%
\pgfpathlineto{\pgfqpoint{0.911147in}{6.200972in}}%
\pgfpathlineto{\pgfqpoint{0.912763in}{6.198918in}}%
\pgfpathlineto{\pgfqpoint{0.917311in}{6.193135in}}%
\pgfpathlineto{\pgfqpoint{0.919452in}{6.190415in}}%
\pgfpathlineto{\pgfqpoint{0.921860in}{6.187353in}}%
\pgfpathlineto{\pgfqpoint{0.926409in}{6.181571in}}%
\pgfpathlineto{\pgfqpoint{0.927756in}{6.179858in}}%
\pgfpathlineto{\pgfqpoint{0.930957in}{6.175789in}}%
\pgfpathlineto{\pgfqpoint{0.935506in}{6.170007in}}%
\pgfpathlineto{\pgfqpoint{0.936061in}{6.169301in}}%
\pgfpathlineto{\pgfqpoint{0.940055in}{6.164224in}}%
\pgfpathlineto{\pgfqpoint{0.944366in}{6.158744in}}%
\pgfpathlineto{\pgfqpoint{0.944603in}{6.158442in}}%
\pgfpathlineto{\pgfqpoint{0.949152in}{6.152660in}}%
\pgfpathlineto{\pgfqpoint{0.952671in}{6.148187in}}%
\pgfpathlineto{\pgfqpoint{0.953700in}{6.146878in}}%
\pgfpathlineto{\pgfqpoint{0.958249in}{6.141096in}}%
\pgfpathlineto{\pgfqpoint{0.960976in}{6.137629in}}%
\pgfpathlineto{\pgfqpoint{0.962798in}{6.135313in}}%
\pgfpathlineto{\pgfqpoint{0.967346in}{6.129531in}}%
\pgfpathlineto{\pgfqpoint{0.969281in}{6.127072in}}%
\pgfpathlineto{\pgfqpoint{0.971895in}{6.123749in}}%
\pgfpathlineto{\pgfqpoint{0.976444in}{6.117967in}}%
\pgfpathlineto{\pgfqpoint{0.977585in}{6.116515in}}%
\pgfpathlineto{\pgfqpoint{0.980992in}{6.112185in}}%
\pgfpathlineto{\pgfqpoint{0.985541in}{6.106402in}}%
\pgfpathlineto{\pgfqpoint{0.985890in}{6.105958in}}%
\pgfpathlineto{\pgfqpoint{0.990089in}{6.100620in}}%
\pgfpathlineto{\pgfqpoint{0.994195in}{6.095401in}}%
\pgfpathlineto{\pgfqpoint{0.994638in}{6.094838in}}%
\pgfpathlineto{\pgfqpoint{0.999187in}{6.089056in}}%
\pgfpathlineto{\pgfqpoint{1.002500in}{6.084844in}}%
\pgfpathlineto{\pgfqpoint{1.003735in}{6.083274in}}%
\pgfpathlineto{\pgfqpoint{1.008284in}{6.077491in}}%
\pgfpathlineto{\pgfqpoint{1.010805in}{6.074287in}}%
\pgfpathlineto{\pgfqpoint{1.012833in}{6.071709in}}%
\pgfpathlineto{\pgfqpoint{1.017381in}{6.065927in}}%
\pgfpathlineto{\pgfqpoint{1.019110in}{6.063730in}}%
\pgfpathlineto{\pgfqpoint{1.021930in}{6.060145in}}%
\pgfpathlineto{\pgfqpoint{1.026478in}{6.054363in}}%
\pgfpathlineto{\pgfqpoint{1.027414in}{6.053173in}}%
\pgfpathlineto{\pgfqpoint{1.031027in}{6.048580in}}%
\pgfpathlineto{\pgfqpoint{1.035576in}{6.042798in}}%
\pgfpathlineto{\pgfqpoint{1.035719in}{6.042616in}}%
\pgfpathlineto{\pgfqpoint{1.040124in}{6.037016in}}%
\pgfpathlineto{\pgfqpoint{1.044024in}{6.032059in}}%
\pgfpathlineto{\pgfqpoint{1.044673in}{6.031234in}}%
\pgfpathlineto{\pgfqpoint{1.049222in}{6.025452in}}%
\pgfpathlineto{\pgfqpoint{1.052329in}{6.021502in}}%
\pgfpathlineto{\pgfqpoint{1.053770in}{6.019669in}}%
\pgfpathlineto{\pgfqpoint{1.058319in}{6.013887in}}%
\pgfpathlineto{\pgfqpoint{1.060634in}{6.010945in}}%
\pgfpathlineto{\pgfqpoint{1.062867in}{6.008105in}}%
\pgfpathlineto{\pgfqpoint{1.067416in}{6.002323in}}%
\pgfpathlineto{\pgfqpoint{1.068939in}{6.000387in}}%
\pgfpathlineto{\pgfqpoint{1.071965in}{5.996541in}}%
\pgfpathlineto{\pgfqpoint{1.076513in}{5.990758in}}%
\pgfpathlineto{\pgfqpoint{1.077243in}{5.989830in}}%
\pgfpathlineto{\pgfqpoint{1.081062in}{5.984976in}}%
\pgfpathlineto{\pgfqpoint{1.085548in}{5.979273in}}%
\pgfpathlineto{\pgfqpoint{1.085611in}{5.979194in}}%
\pgfpathlineto{\pgfqpoint{1.090159in}{5.973412in}}%
\pgfpathlineto{\pgfqpoint{1.093853in}{5.968716in}}%
\pgfpathlineto{\pgfqpoint{1.094708in}{5.967630in}}%
\pgfpathlineto{\pgfqpoint{1.099257in}{5.961847in}}%
\pgfpathlineto{\pgfqpoint{1.102158in}{5.958159in}}%
\pgfpathlineto{\pgfqpoint{1.103805in}{5.956065in}}%
\pgfpathlineto{\pgfqpoint{1.108354in}{5.950283in}}%
\pgfpathlineto{\pgfqpoint{1.110463in}{5.947602in}}%
\pgfpathlineto{\pgfqpoint{1.112902in}{5.944501in}}%
\pgfpathlineto{\pgfqpoint{1.117451in}{5.938719in}}%
\pgfpathlineto{\pgfqpoint{1.118768in}{5.937045in}}%
\pgfpathlineto{\pgfqpoint{1.122000in}{5.932936in}}%
\pgfpathlineto{\pgfqpoint{1.126548in}{5.927154in}}%
\pgfpathlineto{\pgfqpoint{1.127072in}{5.926488in}}%
\pgfpathlineto{\pgfqpoint{1.131097in}{5.921372in}}%
\pgfpathlineto{\pgfqpoint{1.135377in}{5.915931in}}%
\pgfpathlineto{\pgfqpoint{1.135646in}{5.915590in}}%
\pgfpathlineto{\pgfqpoint{1.140194in}{5.909808in}}%
\pgfpathlineto{\pgfqpoint{1.143682in}{5.905374in}}%
\pgfpathlineto{\pgfqpoint{1.144743in}{5.904025in}}%
\pgfpathlineto{\pgfqpoint{1.149291in}{5.898243in}}%
\pgfpathlineto{\pgfqpoint{1.151987in}{5.894817in}}%
\pgfpathlineto{\pgfqpoint{1.153840in}{5.892461in}}%
\pgfpathlineto{\pgfqpoint{1.158389in}{5.886679in}}%
\pgfpathlineto{\pgfqpoint{1.160292in}{5.884260in}}%
\pgfpathlineto{\pgfqpoint{1.162937in}{5.880897in}}%
\pgfpathlineto{\pgfqpoint{1.167486in}{5.875114in}}%
\pgfpathlineto{\pgfqpoint{1.168597in}{5.873703in}}%
\pgfpathlineto{\pgfqpoint{1.172035in}{5.869332in}}%
\pgfpathlineto{\pgfqpoint{1.176583in}{5.863550in}}%
\pgfpathlineto{\pgfqpoint{1.176901in}{5.863145in}}%
\pgfpathlineto{\pgfqpoint{1.181132in}{5.857768in}}%
\pgfpathlineto{\pgfqpoint{1.185206in}{5.852588in}}%
\pgfpathlineto{\pgfqpoint{1.185680in}{5.851986in}}%
\pgfpathlineto{\pgfqpoint{1.190229in}{5.846203in}}%
\pgfpathlineto{\pgfqpoint{1.193511in}{5.842031in}}%
\pgfpathlineto{\pgfqpoint{1.194778in}{5.840421in}}%
\pgfpathlineto{\pgfqpoint{1.199326in}{5.834639in}}%
\pgfpathlineto{\pgfqpoint{1.201816in}{5.831474in}}%
\pgfpathlineto{\pgfqpoint{1.203875in}{5.828857in}}%
\pgfpathlineto{\pgfqpoint{1.208424in}{5.823075in}}%
\pgfpathlineto{\pgfqpoint{1.210121in}{5.820917in}}%
\pgfpathlineto{\pgfqpoint{1.212972in}{5.817292in}}%
\pgfpathlineto{\pgfqpoint{1.217521in}{5.811510in}}%
\pgfpathlineto{\pgfqpoint{1.218426in}{5.810360in}}%
\pgfpathlineto{\pgfqpoint{1.222069in}{5.805728in}}%
\pgfpathlineto{\pgfqpoint{1.226618in}{5.799946in}}%
\pgfpathlineto{\pgfqpoint{1.226730in}{5.799803in}}%
\pgfpathlineto{\pgfqpoint{1.231167in}{5.794164in}}%
\pgfpathlineto{\pgfqpoint{1.235035in}{5.789246in}}%
\pgfpathlineto{\pgfqpoint{1.235715in}{5.788381in}}%
\pgfpathlineto{\pgfqpoint{1.240264in}{5.782599in}}%
\pgfpathlineto{\pgfqpoint{1.243340in}{5.778689in}}%
\pgfpathlineto{\pgfqpoint{1.244813in}{5.776817in}}%
\pgfpathlineto{\pgfqpoint{1.249361in}{5.771035in}}%
\pgfpathlineto{\pgfqpoint{1.251645in}{5.768132in}}%
\pgfpathlineto{\pgfqpoint{1.253910in}{5.765253in}}%
\pgfpathlineto{\pgfqpoint{1.258459in}{5.759470in}}%
\pgfpathlineto{\pgfqpoint{1.259950in}{5.757575in}}%
\pgfpathlineto{\pgfqpoint{1.263007in}{5.753688in}}%
\pgfpathlineto{\pgfqpoint{1.267556in}{5.747906in}}%
\pgfpathlineto{\pgfqpoint{1.268255in}{5.747018in}}%
\pgfpathlineto{\pgfqpoint{1.272104in}{5.742124in}}%
\pgfpathlineto{\pgfqpoint{1.276559in}{5.736461in}}%
\pgfpathlineto{\pgfqpoint{1.276653in}{5.736342in}}%
\pgfpathlineto{\pgfqpoint{1.281202in}{5.730559in}}%
\pgfpathlineto{\pgfqpoint{1.284864in}{5.725903in}}%
\pgfpathlineto{\pgfqpoint{1.285750in}{5.724777in}}%
\pgfpathlineto{\pgfqpoint{1.290299in}{5.718995in}}%
\pgfpathlineto{\pgfqpoint{1.293169in}{5.715346in}}%
\pgfpathlineto{\pgfqpoint{1.294848in}{5.713213in}}%
\pgfpathlineto{\pgfqpoint{1.299396in}{5.707431in}}%
\pgfpathlineto{\pgfqpoint{1.301474in}{5.704789in}}%
\pgfpathlineto{\pgfqpoint{1.303945in}{5.701648in}}%
\pgfpathlineto{\pgfqpoint{1.308493in}{5.695866in}}%
\pgfpathlineto{\pgfqpoint{1.309779in}{5.694232in}}%
\pgfpathlineto{\pgfqpoint{1.313042in}{5.690084in}}%
\pgfpathlineto{\pgfqpoint{1.317591in}{5.684302in}}%
\pgfpathlineto{\pgfqpoint{1.318084in}{5.683675in}}%
\pgfpathlineto{\pgfqpoint{1.322139in}{5.678520in}}%
\pgfpathlineto{\pgfqpoint{1.326388in}{5.673118in}}%
\pgfpathlineto{\pgfqpoint{1.326688in}{5.672737in}}%
\pgfpathlineto{\pgfqpoint{1.331237in}{5.666955in}}%
\pgfpathlineto{\pgfqpoint{1.334693in}{5.662561in}}%
\pgfpathlineto{\pgfqpoint{1.335785in}{5.661173in}}%
\pgfpathlineto{\pgfqpoint{1.340334in}{5.655391in}}%
\pgfpathlineto{\pgfqpoint{1.342998in}{5.652004in}}%
\pgfpathlineto{\pgfqpoint{1.344882in}{5.649609in}}%
\pgfpathlineto{\pgfqpoint{1.349431in}{5.643826in}}%
\pgfpathlineto{\pgfqpoint{1.351303in}{5.641447in}}%
\pgfpathlineto{\pgfqpoint{1.353980in}{5.638044in}}%
\pgfpathlineto{\pgfqpoint{1.358528in}{5.632262in}}%
\pgfpathlineto{\pgfqpoint{1.359608in}{5.630890in}}%
\pgfpathlineto{\pgfqpoint{1.363077in}{5.626480in}}%
\pgfpathlineto{\pgfqpoint{1.367626in}{5.620698in}}%
\pgfpathlineto{\pgfqpoint{1.367913in}{5.620333in}}%
\pgfpathlineto{\pgfqpoint{1.372174in}{5.614915in}}%
\pgfpathlineto{\pgfqpoint{1.376217in}{5.609776in}}%
\pgfpathlineto{\pgfqpoint{1.376723in}{5.609133in}}%
\pgfpathlineto{\pgfqpoint{1.381271in}{5.603351in}}%
\pgfpathlineto{\pgfqpoint{1.384522in}{5.599219in}}%
\pgfpathlineto{\pgfqpoint{1.385820in}{5.597569in}}%
\pgfpathlineto{\pgfqpoint{1.390369in}{5.591787in}}%
\pgfpathlineto{\pgfqpoint{1.392827in}{5.588661in}}%
\pgfpathlineto{\pgfqpoint{1.394917in}{5.586004in}}%
\pgfpathlineto{\pgfqpoint{1.399466in}{5.580222in}}%
\pgfpathlineto{\pgfqpoint{1.401132in}{5.578104in}}%
\pgfpathlineto{\pgfqpoint{1.404015in}{5.574440in}}%
\pgfpathlineto{\pgfqpoint{1.408563in}{5.568658in}}%
\pgfpathlineto{\pgfqpoint{1.409437in}{5.567547in}}%
\pgfpathlineto{\pgfqpoint{1.413112in}{5.562876in}}%
\pgfpathlineto{\pgfqpoint{1.417661in}{5.557093in}}%
\pgfpathlineto{\pgfqpoint{1.417742in}{5.556990in}}%
\pgfpathlineto{\pgfqpoint{1.422209in}{5.551311in}}%
\pgfpathlineto{\pgfqpoint{1.426046in}{5.546433in}}%
\pgfpathlineto{\pgfqpoint{1.426758in}{5.545529in}}%
\pgfpathlineto{\pgfqpoint{1.431306in}{5.539747in}}%
\pgfpathlineto{\pgfqpoint{1.434351in}{5.535876in}}%
\pgfpathlineto{\pgfqpoint{1.435855in}{5.533965in}}%
\pgfpathlineto{\pgfqpoint{1.440404in}{5.528182in}}%
\pgfpathlineto{\pgfqpoint{1.442656in}{5.525319in}}%
\pgfpathlineto{\pgfqpoint{1.444952in}{5.522400in}}%
\pgfpathlineto{\pgfqpoint{1.449501in}{5.516618in}}%
\pgfpathlineto{\pgfqpoint{1.450961in}{5.514762in}}%
\pgfpathlineto{\pgfqpoint{1.454050in}{5.510836in}}%
\pgfpathlineto{\pgfqpoint{1.458598in}{5.505054in}}%
\pgfpathlineto{\pgfqpoint{1.459266in}{5.504205in}}%
\pgfpathlineto{\pgfqpoint{1.463147in}{5.499271in}}%
\pgfpathlineto{\pgfqpoint{1.467571in}{5.493648in}}%
\pgfpathlineto{\pgfqpoint{1.467695in}{5.493489in}}%
\pgfpathlineto{\pgfqpoint{1.472244in}{5.487707in}}%
\pgfpathlineto{\pgfqpoint{1.475875in}{5.483091in}}%
\pgfpathlineto{\pgfqpoint{1.476793in}{5.481925in}}%
\pgfpathlineto{\pgfqpoint{1.481341in}{5.476143in}}%
\pgfpathlineto{\pgfqpoint{1.484180in}{5.472534in}}%
\pgfpathlineto{\pgfqpoint{1.485890in}{5.470360in}}%
\pgfpathlineto{\pgfqpoint{1.490439in}{5.464578in}}%
\pgfpathlineto{\pgfqpoint{1.492485in}{5.461977in}}%
\pgfpathlineto{\pgfqpoint{1.494987in}{5.458796in}}%
\pgfpathlineto{\pgfqpoint{1.499536in}{5.453014in}}%
\pgfpathlineto{\pgfqpoint{1.500790in}{5.451419in}}%
\pgfpathlineto{\pgfqpoint{1.504084in}{5.447232in}}%
\pgfpathlineto{\pgfqpoint{1.508633in}{5.441449in}}%
\pgfpathlineto{\pgfqpoint{1.509095in}{5.440862in}}%
\pgfpathlineto{\pgfqpoint{1.513182in}{5.435667in}}%
\pgfpathlineto{\pgfqpoint{1.517400in}{5.430305in}}%
\pgfpathlineto{\pgfqpoint{1.517730in}{5.429885in}}%
\pgfpathlineto{\pgfqpoint{1.522279in}{5.424103in}}%
\pgfpathlineto{\pgfqpoint{1.525704in}{5.419748in}}%
\pgfpathlineto{\pgfqpoint{1.526828in}{5.418321in}}%
\pgfpathlineto{\pgfqpoint{1.531376in}{5.412538in}}%
\pgfpathlineto{\pgfqpoint{1.534009in}{5.409191in}}%
\pgfpathlineto{\pgfqpoint{1.535925in}{5.406756in}}%
\pgfpathlineto{\pgfqpoint{1.540473in}{5.400974in}}%
\pgfpathlineto{\pgfqpoint{1.542314in}{5.398634in}}%
\pgfpathlineto{\pgfqpoint{1.545022in}{5.395192in}}%
\pgfpathlineto{\pgfqpoint{1.549571in}{5.389410in}}%
\pgfpathlineto{\pgfqpoint{1.550619in}{5.388077in}}%
\pgfpathlineto{\pgfqpoint{1.554119in}{5.383627in}}%
\pgfpathlineto{\pgfqpoint{1.558668in}{5.377845in}}%
\pgfpathlineto{\pgfqpoint{1.558924in}{5.377520in}}%
\pgfpathlineto{\pgfqpoint{1.563217in}{5.372063in}}%
\pgfpathlineto{\pgfqpoint{1.567229in}{5.366963in}}%
\pgfpathlineto{\pgfqpoint{1.567765in}{5.366281in}}%
\pgfpathlineto{\pgfqpoint{1.572314in}{5.360499in}}%
\pgfpathlineto{\pgfqpoint{1.575533in}{5.356406in}}%
\pgfpathlineto{\pgfqpoint{1.576863in}{5.354716in}}%
\pgfpathlineto{\pgfqpoint{1.581411in}{5.348934in}}%
\pgfpathlineto{\pgfqpoint{1.583838in}{5.345849in}}%
\pgfpathlineto{\pgfqpoint{1.585960in}{5.343152in}}%
\pgfpathlineto{\pgfqpoint{1.590508in}{5.337370in}}%
\pgfpathlineto{\pgfqpoint{1.592143in}{5.335292in}}%
\pgfpathlineto{\pgfqpoint{1.595057in}{5.331588in}}%
\pgfpathlineto{\pgfqpoint{1.599606in}{5.325805in}}%
\pgfpathlineto{\pgfqpoint{1.600448in}{5.324735in}}%
\pgfpathlineto{\pgfqpoint{1.604154in}{5.320023in}}%
\pgfpathlineto{\pgfqpoint{1.608703in}{5.314241in}}%
\pgfpathlineto{\pgfqpoint{1.608753in}{5.314177in}}%
\pgfpathlineto{\pgfqpoint{1.613252in}{5.308459in}}%
\pgfpathlineto{\pgfqpoint{1.617058in}{5.303620in}}%
\pgfpathlineto{\pgfqpoint{1.617800in}{5.302677in}}%
\pgfpathlineto{\pgfqpoint{1.622349in}{5.296894in}}%
\pgfpathlineto{\pgfqpoint{1.625363in}{5.293063in}}%
\pgfpathlineto{\pgfqpoint{1.626897in}{5.291112in}}%
\pgfpathlineto{\pgfqpoint{1.631446in}{5.285330in}}%
\pgfpathlineto{\pgfqpoint{1.633667in}{5.282506in}}%
\pgfpathlineto{\pgfqpoint{1.635995in}{5.279548in}}%
\pgfpathlineto{\pgfqpoint{1.640543in}{5.273766in}}%
\pgfpathlineto{\pgfqpoint{1.641972in}{5.271949in}}%
\pgfpathlineto{\pgfqpoint{1.645092in}{5.267983in}}%
\pgfpathlineto{\pgfqpoint{1.649641in}{5.262201in}}%
\pgfpathlineto{\pgfqpoint{1.650277in}{5.261392in}}%
\pgfpathlineto{\pgfqpoint{1.654189in}{5.256419in}}%
\pgfpathlineto{\pgfqpoint{1.658582in}{5.250835in}}%
\pgfpathlineto{\pgfqpoint{1.658738in}{5.250637in}}%
\pgfpathlineto{\pgfqpoint{1.663286in}{5.244855in}}%
\pgfpathlineto{\pgfqpoint{1.666887in}{5.240278in}}%
\pgfpathlineto{\pgfqpoint{1.667835in}{5.239072in}}%
\pgfpathlineto{\pgfqpoint{1.672384in}{5.233290in}}%
\pgfpathlineto{\pgfqpoint{1.675192in}{5.229721in}}%
\pgfpathlineto{\pgfqpoint{1.676932in}{5.227508in}}%
\pgfpathlineto{\pgfqpoint{1.681481in}{5.221726in}}%
\pgfpathlineto{\pgfqpoint{1.683496in}{5.219164in}}%
\pgfpathlineto{\pgfqpoint{1.686030in}{5.215944in}}%
\pgfpathlineto{\pgfqpoint{1.690578in}{5.210161in}}%
\pgfpathlineto{\pgfqpoint{1.691801in}{5.208607in}}%
\pgfpathlineto{\pgfqpoint{1.695127in}{5.204379in}}%
\pgfpathlineto{\pgfqpoint{1.699675in}{5.198597in}}%
\pgfpathlineto{\pgfqpoint{1.700106in}{5.198050in}}%
\pgfpathlineto{\pgfqpoint{1.704224in}{5.192815in}}%
\pgfpathlineto{\pgfqpoint{1.708411in}{5.187493in}}%
\pgfpathlineto{\pgfqpoint{1.708773in}{5.187033in}}%
\pgfpathlineto{\pgfqpoint{1.713321in}{5.181250in}}%
\pgfpathlineto{\pgfqpoint{1.716716in}{5.176935in}}%
\pgfpathlineto{\pgfqpoint{1.717870in}{5.175468in}}%
\pgfpathlineto{\pgfqpoint{1.722419in}{5.169686in}}%
\pgfpathlineto{\pgfqpoint{1.725021in}{5.166378in}}%
\pgfpathlineto{\pgfqpoint{1.726967in}{5.163904in}}%
\pgfpathlineto{\pgfqpoint{1.731516in}{5.158122in}}%
\pgfpathlineto{\pgfqpoint{1.733325in}{5.155821in}}%
\pgfpathlineto{\pgfqpoint{1.736065in}{5.152339in}}%
\pgfpathlineto{\pgfqpoint{1.740613in}{5.146557in}}%
\pgfpathlineto{\pgfqpoint{1.741630in}{5.145264in}}%
\pgfpathlineto{\pgfqpoint{1.745162in}{5.140775in}}%
\pgfpathlineto{\pgfqpoint{1.749710in}{5.134993in}}%
\pgfpathlineto{\pgfqpoint{1.749935in}{5.134707in}}%
\pgfpathlineto{\pgfqpoint{1.754259in}{5.129211in}}%
\pgfpathlineto{\pgfqpoint{1.758240in}{5.124150in}}%
\pgfpathlineto{\pgfqpoint{1.758808in}{5.123428in}}%
\pgfpathlineto{\pgfqpoint{1.763356in}{5.117646in}}%
\pgfpathlineto{\pgfqpoint{1.766545in}{5.113593in}}%
\pgfpathlineto{\pgfqpoint{1.767905in}{5.111864in}}%
\pgfpathlineto{\pgfqpoint{1.772454in}{5.106082in}}%
\pgfpathlineto{\pgfqpoint{1.774850in}{5.103036in}}%
\pgfpathlineto{\pgfqpoint{1.777002in}{5.100300in}}%
\pgfpathlineto{\pgfqpoint{1.781551in}{5.094517in}}%
\pgfpathlineto{\pgfqpoint{1.783154in}{5.092479in}}%
\pgfpathlineto{\pgfqpoint{1.786099in}{5.088735in}}%
\pgfpathlineto{\pgfqpoint{1.790648in}{5.082953in}}%
\pgfpathlineto{\pgfqpoint{1.791459in}{5.081922in}}%
\pgfpathlineto{\pgfqpoint{1.795197in}{5.077171in}}%
\pgfpathlineto{\pgfqpoint{1.799745in}{5.071389in}}%
\pgfpathlineto{\pgfqpoint{1.799764in}{5.071365in}}%
\pgfpathlineto{\pgfqpoint{1.804294in}{5.065606in}}%
\pgfpathlineto{\pgfqpoint{1.808069in}{5.060808in}}%
\pgfpathlineto{\pgfqpoint{1.808843in}{5.059824in}}%
\pgfpathlineto{\pgfqpoint{1.813391in}{5.054042in}}%
\pgfpathlineto{\pgfqpoint{1.816374in}{5.050251in}}%
\pgfpathlineto{\pgfqpoint{1.817940in}{5.048260in}}%
\pgfpathlineto{\pgfqpoint{1.822488in}{5.042478in}}%
\pgfpathlineto{\pgfqpoint{1.824679in}{5.039693in}}%
\pgfpathlineto{\pgfqpoint{1.827037in}{5.036695in}}%
\pgfpathlineto{\pgfqpoint{1.831586in}{5.030913in}}%
\pgfpathlineto{\pgfqpoint{1.832983in}{5.029136in}}%
\pgfpathlineto{\pgfqpoint{1.836134in}{5.025131in}}%
\pgfpathlineto{\pgfqpoint{1.840683in}{5.019349in}}%
\pgfpathlineto{\pgfqpoint{1.841288in}{5.018579in}}%
\pgfpathlineto{\pgfqpoint{1.845232in}{5.013567in}}%
\pgfpathlineto{\pgfqpoint{1.849593in}{5.008022in}}%
\pgfpathlineto{\pgfqpoint{1.849780in}{5.007784in}}%
\pgfpathlineto{\pgfqpoint{1.854329in}{5.002002in}}%
\pgfpathlineto{\pgfqpoint{1.857898in}{4.997465in}}%
\pgfpathlineto{\pgfqpoint{1.858877in}{4.996220in}}%
\pgfpathlineto{\pgfqpoint{1.863426in}{4.990438in}}%
\pgfpathlineto{\pgfqpoint{1.866203in}{4.986908in}}%
\pgfpathlineto{\pgfqpoint{1.867975in}{4.984656in}}%
\pgfpathlineto{\pgfqpoint{1.872523in}{4.978873in}}%
\pgfpathlineto{\pgfqpoint{1.874508in}{4.976351in}}%
\pgfpathlineto{\pgfqpoint{1.877072in}{4.973091in}}%
\pgfpathlineto{\pgfqpoint{1.881621in}{4.967309in}}%
\pgfpathlineto{\pgfqpoint{1.882812in}{4.965794in}}%
\pgfpathlineto{\pgfqpoint{1.886169in}{4.961527in}}%
\pgfpathlineto{\pgfqpoint{1.890718in}{4.955745in}}%
\pgfpathlineto{\pgfqpoint{1.891117in}{4.955237in}}%
\pgfpathlineto{\pgfqpoint{1.895267in}{4.949962in}}%
\pgfpathlineto{\pgfqpoint{1.899422in}{4.944680in}}%
\pgfpathlineto{\pgfqpoint{1.899815in}{4.944180in}}%
\pgfpathlineto{\pgfqpoint{1.904364in}{4.938398in}}%
\pgfpathlineto{\pgfqpoint{1.907727in}{4.934123in}}%
\pgfpathlineto{\pgfqpoint{1.908912in}{4.932616in}}%
\pgfpathlineto{\pgfqpoint{1.913461in}{4.926833in}}%
\pgfpathlineto{\pgfqpoint{1.916032in}{4.923566in}}%
\pgfpathlineto{\pgfqpoint{1.918010in}{4.921051in}}%
\pgfpathlineto{\pgfqpoint{1.922558in}{4.915269in}}%
\pgfpathlineto{\pgfqpoint{1.924337in}{4.913009in}}%
\pgfpathlineto{\pgfqpoint{1.927107in}{4.909487in}}%
\pgfpathlineto{\pgfqpoint{1.931656in}{4.903705in}}%
\pgfpathlineto{\pgfqpoint{1.932641in}{4.902451in}}%
\pgfpathlineto{\pgfqpoint{1.936204in}{4.897922in}}%
\pgfpathlineto{\pgfqpoint{1.940753in}{4.892140in}}%
\pgfpathlineto{\pgfqpoint{1.940946in}{4.891894in}}%
\pgfpathlineto{\pgfqpoint{1.945301in}{4.886358in}}%
\pgfpathlineto{\pgfqpoint{1.949251in}{4.881337in}}%
\pgfpathlineto{\pgfqpoint{1.949850in}{4.880576in}}%
\pgfpathlineto{\pgfqpoint{1.954399in}{4.874794in}}%
\pgfpathlineto{\pgfqpoint{1.957556in}{4.870780in}}%
\pgfpathlineto{\pgfqpoint{1.958947in}{4.869011in}}%
\pgfpathlineto{\pgfqpoint{1.963496in}{4.863229in}}%
\pgfpathlineto{\pgfqpoint{1.965861in}{4.860223in}}%
\pgfpathlineto{\pgfqpoint{1.968045in}{4.857447in}}%
\pgfpathlineto{\pgfqpoint{1.972593in}{4.851665in}}%
\pgfpathlineto{\pgfqpoint{1.974166in}{4.849666in}}%
\pgfpathlineto{\pgfqpoint{1.977142in}{4.845883in}}%
\pgfpathlineto{\pgfqpoint{1.981690in}{4.840100in}}%
\pgfpathlineto{\pgfqpoint{1.982470in}{4.839109in}}%
\pgfpathlineto{\pgfqpoint{1.986239in}{4.834318in}}%
\pgfpathlineto{\pgfqpoint{1.990775in}{4.828552in}}%
\pgfpathlineto{\pgfqpoint{1.990788in}{4.828536in}}%
\pgfpathlineto{\pgfqpoint{1.995336in}{4.822754in}}%
\pgfpathlineto{\pgfqpoint{1.999080in}{4.817995in}}%
\pgfpathlineto{\pgfqpoint{1.999885in}{4.816972in}}%
\pgfpathlineto{\pgfqpoint{2.004434in}{4.811189in}}%
\pgfpathlineto{\pgfqpoint{2.007385in}{4.807438in}}%
\pgfpathlineto{\pgfqpoint{2.008982in}{4.805407in}}%
\pgfpathlineto{\pgfqpoint{2.013531in}{4.799625in}}%
\pgfpathlineto{\pgfqpoint{2.015690in}{4.796881in}}%
\pgfpathlineto{\pgfqpoint{2.018079in}{4.793843in}}%
\pgfpathlineto{\pgfqpoint{2.022628in}{4.788061in}}%
\pgfpathlineto{\pgfqpoint{2.023995in}{4.786324in}}%
\pgfpathlineto{\pgfqpoint{2.027177in}{4.782278in}}%
\pgfpathlineto{\pgfqpoint{2.031725in}{4.776496in}}%
\pgfpathlineto{\pgfqpoint{2.032299in}{4.775767in}}%
\pgfpathlineto{\pgfqpoint{2.036274in}{4.770714in}}%
\pgfpathlineto{\pgfqpoint{2.040604in}{4.765209in}}%
\pgfpathlineto{\pgfqpoint{2.040823in}{4.764932in}}%
\pgfpathlineto{\pgfqpoint{2.045371in}{4.759150in}}%
\pgfpathlineto{\pgfqpoint{2.048909in}{4.754652in}}%
\pgfpathlineto{\pgfqpoint{2.049920in}{4.753367in}}%
\pgfpathlineto{\pgfqpoint{2.054469in}{4.747585in}}%
\pgfpathlineto{\pgfqpoint{2.057214in}{4.744095in}}%
\pgfpathlineto{\pgfqpoint{2.059017in}{4.741803in}}%
\pgfpathlineto{\pgfqpoint{2.063566in}{4.736021in}}%
\pgfpathlineto{\pgfqpoint{2.065519in}{4.733538in}}%
\pgfpathlineto{\pgfqpoint{2.068114in}{4.730239in}}%
\pgfpathlineto{\pgfqpoint{2.072663in}{4.724456in}}%
\pgfpathlineto{\pgfqpoint{2.073824in}{4.722981in}}%
\pgfpathlineto{\pgfqpoint{2.077212in}{4.718674in}}%
\pgfpathlineto{\pgfqpoint{2.081760in}{4.712892in}}%
\pgfpathlineto{\pgfqpoint{2.082128in}{4.712424in}}%
\pgfpathlineto{\pgfqpoint{2.086309in}{4.707110in}}%
\pgfpathlineto{\pgfqpoint{2.090433in}{4.701867in}}%
\pgfpathlineto{\pgfqpoint{2.090858in}{4.701328in}}%
\pgfpathlineto{\pgfqpoint{2.095406in}{4.695545in}}%
\pgfpathlineto{\pgfqpoint{2.098738in}{4.691310in}}%
\pgfpathlineto{\pgfqpoint{2.099955in}{4.689763in}}%
\pgfpathlineto{\pgfqpoint{2.104503in}{4.683981in}}%
\pgfpathlineto{\pgfqpoint{2.107043in}{4.680753in}}%
\pgfpathlineto{\pgfqpoint{2.109052in}{4.678199in}}%
\pgfpathlineto{\pgfqpoint{2.113601in}{4.672417in}}%
\pgfpathlineto{\pgfqpoint{2.115348in}{4.670196in}}%
\pgfpathlineto{\pgfqpoint{2.118149in}{4.666634in}}%
\pgfpathlineto{\pgfqpoint{2.122698in}{4.660852in}}%
\pgfpathlineto{\pgfqpoint{2.123653in}{4.659639in}}%
\pgfpathlineto{\pgfqpoint{2.127247in}{4.655070in}}%
\pgfpathlineto{\pgfqpoint{2.131795in}{4.649288in}}%
\pgfpathlineto{\pgfqpoint{2.131957in}{4.649082in}}%
\pgfpathlineto{\pgfqpoint{2.136344in}{4.643506in}}%
\pgfpathlineto{\pgfqpoint{2.140262in}{4.638525in}}%
\pgfpathlineto{\pgfqpoint{2.140892in}{4.637723in}}%
\pgfpathlineto{\pgfqpoint{2.145441in}{4.631941in}}%
\pgfpathlineto{\pgfqpoint{2.148567in}{4.627967in}}%
\pgfpathlineto{\pgfqpoint{2.149990in}{4.626159in}}%
\pgfpathlineto{\pgfqpoint{2.154538in}{4.620377in}}%
\pgfpathlineto{\pgfqpoint{2.156872in}{4.617410in}}%
\pgfpathlineto{\pgfqpoint{2.159087in}{4.614595in}}%
\pgfpathlineto{\pgfqpoint{2.163636in}{4.608812in}}%
\pgfpathlineto{\pgfqpoint{2.165177in}{4.606853in}}%
\pgfpathlineto{\pgfqpoint{2.168184in}{4.603030in}}%
\pgfpathlineto{\pgfqpoint{2.172733in}{4.597248in}}%
\pgfpathlineto{\pgfqpoint{2.173482in}{4.596296in}}%
\pgfpathlineto{\pgfqpoint{2.177281in}{4.591466in}}%
\pgfpathlineto{\pgfqpoint{2.181786in}{4.585739in}}%
\pgfpathlineto{\pgfqpoint{2.181830in}{4.585684in}}%
\pgfpathlineto{\pgfqpoint{2.186379in}{4.579901in}}%
\pgfpathlineto{\pgfqpoint{2.190091in}{4.575182in}}%
\pgfpathlineto{\pgfqpoint{2.190927in}{4.574119in}}%
\pgfpathlineto{\pgfqpoint{2.195476in}{4.568337in}}%
\pgfpathlineto{\pgfqpoint{2.198396in}{4.564625in}}%
\pgfpathlineto{\pgfqpoint{2.200025in}{4.562555in}}%
\pgfpathlineto{\pgfqpoint{2.204573in}{4.556773in}}%
\pgfpathlineto{\pgfqpoint{2.206701in}{4.554068in}}%
\pgfpathlineto{\pgfqpoint{2.209122in}{4.550990in}}%
\pgfpathlineto{\pgfqpoint{2.213671in}{4.545208in}}%
\pgfpathlineto{\pgfqpoint{2.215006in}{4.543511in}}%
\pgfpathlineto{\pgfqpoint{2.218219in}{4.539426in}}%
\pgfpathlineto{\pgfqpoint{2.222768in}{4.533644in}}%
\pgfpathlineto{\pgfqpoint{2.223311in}{4.532954in}}%
\pgfpathlineto{\pgfqpoint{2.227316in}{4.527862in}}%
\pgfpathlineto{\pgfqpoint{2.231615in}{4.522397in}}%
\pgfpathlineto{\pgfqpoint{2.231865in}{4.522079in}}%
\pgfpathlineto{\pgfqpoint{2.236414in}{4.516297in}}%
\pgfpathlineto{\pgfqpoint{2.239920in}{4.511840in}}%
\pgfpathlineto{\pgfqpoint{2.240962in}{4.510515in}}%
\pgfpathlineto{\pgfqpoint{2.245511in}{4.504733in}}%
\pgfpathlineto{\pgfqpoint{2.248225in}{4.501283in}}%
\pgfpathlineto{\pgfqpoint{2.250060in}{4.498951in}}%
\pgfpathlineto{\pgfqpoint{2.254608in}{4.493168in}}%
\pgfpathlineto{\pgfqpoint{2.256530in}{4.490726in}}%
\pgfpathlineto{\pgfqpoint{2.259157in}{4.487386in}}%
\pgfpathlineto{\pgfqpoint{2.263705in}{4.481604in}}%
\pgfpathlineto{\pgfqpoint{2.264835in}{4.480168in}}%
\pgfpathlineto{\pgfqpoint{2.268254in}{4.475822in}}%
\pgfpathlineto{\pgfqpoint{2.272803in}{4.470040in}}%
\pgfpathlineto{\pgfqpoint{2.273140in}{4.469611in}}%
\pgfpathlineto{\pgfqpoint{2.277351in}{4.464257in}}%
\pgfpathlineto{\pgfqpoint{2.281444in}{4.459054in}}%
\pgfpathlineto{\pgfqpoint{2.281900in}{4.458475in}}%
\pgfpathlineto{\pgfqpoint{2.286449in}{4.452693in}}%
\pgfpathlineto{\pgfqpoint{2.289749in}{4.448497in}}%
\pgfpathlineto{\pgfqpoint{2.290997in}{4.446911in}}%
\pgfpathlineto{\pgfqpoint{2.295546in}{4.441129in}}%
\pgfpathlineto{\pgfqpoint{2.298054in}{4.437940in}}%
\pgfpathlineto{\pgfqpoint{2.300094in}{4.435346in}}%
\pgfpathlineto{\pgfqpoint{2.304643in}{4.429564in}}%
\pgfpathlineto{\pgfqpoint{2.306359in}{4.427383in}}%
\pgfpathlineto{\pgfqpoint{2.309192in}{4.423782in}}%
\pgfpathlineto{\pgfqpoint{2.313740in}{4.418000in}}%
\pgfpathlineto{\pgfqpoint{2.314664in}{4.416826in}}%
\pgfpathlineto{\pgfqpoint{2.318289in}{4.412218in}}%
\pgfpathlineto{\pgfqpoint{2.322838in}{4.406435in}}%
\pgfpathlineto{\pgfqpoint{2.322969in}{4.406269in}}%
\pgfpathlineto{\pgfqpoint{2.327386in}{4.400653in}}%
\pgfpathlineto{\pgfqpoint{2.331273in}{4.395712in}}%
\pgfpathlineto{\pgfqpoint{2.331935in}{4.394871in}}%
\pgfpathlineto{\pgfqpoint{2.336483in}{4.389089in}}%
\pgfpathlineto{\pgfqpoint{2.339578in}{4.385155in}}%
\pgfpathlineto{\pgfqpoint{2.341032in}{4.383307in}}%
\pgfpathlineto{\pgfqpoint{2.345581in}{4.377524in}}%
\pgfpathlineto{\pgfqpoint{2.347883in}{4.374598in}}%
\pgfpathlineto{\pgfqpoint{2.350129in}{4.371742in}}%
\pgfpathlineto{\pgfqpoint{2.354678in}{4.365960in}}%
\pgfpathlineto{\pgfqpoint{2.356188in}{4.364041in}}%
\pgfpathlineto{\pgfqpoint{2.359227in}{4.360178in}}%
\pgfpathlineto{\pgfqpoint{2.363775in}{4.354396in}}%
\pgfpathlineto{\pgfqpoint{2.364493in}{4.353484in}}%
\pgfpathlineto{\pgfqpoint{2.368324in}{4.348613in}}%
\pgfpathlineto{\pgfqpoint{2.372798in}{4.342926in}}%
\pgfpathlineto{\pgfqpoint{2.372873in}{4.342831in}}%
\pgfpathlineto{\pgfqpoint{2.377421in}{4.337049in}}%
\pgfpathlineto{\pgfqpoint{2.381102in}{4.332369in}}%
\pgfpathlineto{\pgfqpoint{2.381970in}{4.331267in}}%
\pgfpathlineto{\pgfqpoint{2.386518in}{4.325485in}}%
\pgfpathlineto{\pgfqpoint{2.389407in}{4.321812in}}%
\pgfpathlineto{\pgfqpoint{2.391067in}{4.319702in}}%
\pgfpathlineto{\pgfqpoint{2.395616in}{4.313920in}}%
\pgfpathlineto{\pgfqpoint{2.397712in}{4.311255in}}%
\pgfpathlineto{\pgfqpoint{2.400164in}{4.308138in}}%
\pgfpathlineto{\pgfqpoint{2.404713in}{4.302356in}}%
\pgfpathlineto{\pgfqpoint{2.406017in}{4.300698in}}%
\pgfpathlineto{\pgfqpoint{2.409262in}{4.296574in}}%
\pgfpathlineto{\pgfqpoint{2.413810in}{4.290791in}}%
\pgfpathlineto{\pgfqpoint{2.414322in}{4.290141in}}%
\pgfpathlineto{\pgfqpoint{2.418359in}{4.285009in}}%
\pgfpathlineto{\pgfqpoint{2.422627in}{4.279584in}}%
\pgfpathlineto{\pgfqpoint{2.422907in}{4.279227in}}%
\pgfpathlineto{\pgfqpoint{2.427456in}{4.273445in}}%
\pgfpathlineto{\pgfqpoint{2.430931in}{4.269027in}}%
\pgfpathlineto{\pgfqpoint{2.432005in}{4.267663in}}%
\pgfpathlineto{\pgfqpoint{2.436553in}{4.261880in}}%
\pgfpathlineto{\pgfqpoint{2.439236in}{4.258470in}}%
\pgfpathlineto{\pgfqpoint{2.441102in}{4.256098in}}%
\pgfpathlineto{\pgfqpoint{2.445651in}{4.250316in}}%
\pgfpathlineto{\pgfqpoint{2.447541in}{4.247913in}}%
\pgfpathlineto{\pgfqpoint{2.450199in}{4.244534in}}%
\pgfpathlineto{\pgfqpoint{2.454748in}{4.238752in}}%
\pgfpathlineto{\pgfqpoint{2.455846in}{4.237356in}}%
\pgfpathlineto{\pgfqpoint{2.459296in}{4.232969in}}%
\pgfpathlineto{\pgfqpoint{2.463845in}{4.227187in}}%
\pgfpathlineto{\pgfqpoint{2.464151in}{4.226799in}}%
\pgfpathlineto{\pgfqpoint{2.468394in}{4.221405in}}%
\pgfpathlineto{\pgfqpoint{2.472456in}{4.216242in}}%
\pgfpathlineto{\pgfqpoint{2.472942in}{4.215623in}}%
\pgfpathlineto{\pgfqpoint{2.477491in}{4.209841in}}%
\pgfpathlineto{\pgfqpoint{2.480760in}{4.205684in}}%
\pgfpathlineto{\pgfqpoint{2.482040in}{4.204058in}}%
\pgfpathlineto{\pgfqpoint{2.486588in}{4.198276in}}%
\pgfpathlineto{\pgfqpoint{2.489065in}{4.195127in}}%
\pgfpathlineto{\pgfqpoint{2.491137in}{4.192494in}}%
\pgfpathlineto{\pgfqpoint{2.495685in}{4.186712in}}%
\pgfpathlineto{\pgfqpoint{2.497370in}{4.184570in}}%
\pgfpathlineto{\pgfqpoint{2.500234in}{4.180930in}}%
\pgfpathlineto{\pgfqpoint{2.504783in}{4.175147in}}%
\pgfpathlineto{\pgfqpoint{2.505675in}{4.174013in}}%
\pgfpathlineto{\pgfqpoint{2.509331in}{4.169365in}}%
\pgfpathlineto{\pgfqpoint{2.513880in}{4.163583in}}%
\pgfpathlineto{\pgfqpoint{2.513980in}{4.163456in}}%
\pgfpathlineto{\pgfqpoint{2.518429in}{4.157801in}}%
\pgfpathlineto{\pgfqpoint{2.522285in}{4.152899in}}%
\pgfpathlineto{\pgfqpoint{2.522977in}{4.152019in}}%
\pgfpathlineto{\pgfqpoint{2.527526in}{4.146236in}}%
\pgfpathlineto{\pgfqpoint{2.530589in}{4.142342in}}%
\pgfpathlineto{\pgfqpoint{2.532075in}{4.140454in}}%
\pgfpathlineto{\pgfqpoint{2.536623in}{4.134672in}}%
\pgfpathlineto{\pgfqpoint{2.538894in}{4.131785in}}%
\pgfpathlineto{\pgfqpoint{2.541172in}{4.128890in}}%
\pgfpathlineto{\pgfqpoint{2.545720in}{4.123108in}}%
\pgfpathlineto{\pgfqpoint{2.547199in}{4.121228in}}%
\pgfpathlineto{\pgfqpoint{2.550269in}{4.117325in}}%
\pgfpathlineto{\pgfqpoint{2.554818in}{4.111543in}}%
\pgfpathlineto{\pgfqpoint{2.555504in}{4.110671in}}%
\pgfpathlineto{\pgfqpoint{2.559366in}{4.105761in}}%
\pgfpathlineto{\pgfqpoint{2.563809in}{4.100114in}}%
\pgfpathlineto{\pgfqpoint{2.563915in}{4.099979in}}%
\pgfpathlineto{\pgfqpoint{2.568464in}{4.094197in}}%
\pgfpathlineto{\pgfqpoint{2.572114in}{4.089557in}}%
\pgfpathlineto{\pgfqpoint{2.573012in}{4.088414in}}%
\pgfpathlineto{\pgfqpoint{2.577561in}{4.082632in}}%
\pgfpathlineto{\pgfqpoint{2.580419in}{4.079000in}}%
\pgfpathlineto{\pgfqpoint{2.582109in}{4.076850in}}%
\pgfpathlineto{\pgfqpoint{2.586658in}{4.071068in}}%
\pgfpathlineto{\pgfqpoint{2.588723in}{4.068442in}}%
\pgfpathlineto{\pgfqpoint{2.591207in}{4.065286in}}%
\pgfpathlineto{\pgfqpoint{2.595755in}{4.059503in}}%
\pgfpathlineto{\pgfqpoint{2.597028in}{4.057885in}}%
\pgfpathlineto{\pgfqpoint{2.600304in}{4.053721in}}%
\pgfpathlineto{\pgfqpoint{2.604853in}{4.047939in}}%
\pgfpathlineto{\pgfqpoint{2.605333in}{4.047328in}}%
\pgfpathlineto{\pgfqpoint{2.609401in}{4.042157in}}%
\pgfpathlineto{\pgfqpoint{2.613638in}{4.036771in}}%
\pgfpathlineto{\pgfqpoint{2.613950in}{4.036375in}}%
\pgfpathlineto{\pgfqpoint{2.618498in}{4.030592in}}%
\pgfpathlineto{\pgfqpoint{2.621943in}{4.026214in}}%
\pgfpathlineto{\pgfqpoint{2.623047in}{4.024810in}}%
\pgfpathlineto{\pgfqpoint{2.627596in}{4.019028in}}%
\pgfpathlineto{\pgfqpoint{2.630248in}{4.015657in}}%
\pgfpathlineto{\pgfqpoint{2.632144in}{4.013246in}}%
\pgfpathlineto{\pgfqpoint{2.636693in}{4.007464in}}%
\pgfpathlineto{\pgfqpoint{2.638552in}{4.005100in}}%
\pgfpathlineto{\pgfqpoint{2.641242in}{4.001681in}}%
\pgfpathlineto{\pgfqpoint{2.645790in}{3.995899in}}%
\pgfpathlineto{\pgfqpoint{2.646857in}{3.994543in}}%
\pgfpathlineto{\pgfqpoint{2.650339in}{3.990117in}}%
\pgfpathlineto{\pgfqpoint{2.654887in}{3.984335in}}%
\pgfpathlineto{\pgfqpoint{2.655162in}{3.983986in}}%
\pgfpathlineto{\pgfqpoint{2.659436in}{3.978553in}}%
\pgfpathlineto{\pgfqpoint{2.663467in}{3.973429in}}%
\pgfpathlineto{\pgfqpoint{2.663985in}{3.972770in}}%
\pgfpathlineto{\pgfqpoint{2.668533in}{3.966988in}}%
\pgfpathlineto{\pgfqpoint{2.671772in}{3.962872in}}%
\pgfpathlineto{\pgfqpoint{2.673082in}{3.961206in}}%
\pgfpathlineto{\pgfqpoint{2.677631in}{3.955424in}}%
\pgfpathlineto{\pgfqpoint{2.680077in}{3.952315in}}%
\pgfpathlineto{\pgfqpoint{2.682179in}{3.949642in}}%
\pgfpathlineto{\pgfqpoint{2.686728in}{3.943859in}}%
\pgfpathlineto{\pgfqpoint{2.688381in}{3.941758in}}%
\pgfpathlineto{\pgfqpoint{2.691277in}{3.938077in}}%
\pgfpathlineto{\pgfqpoint{2.695825in}{3.932295in}}%
\pgfpathlineto{\pgfqpoint{2.696686in}{3.931200in}}%
\pgfpathlineto{\pgfqpoint{2.700374in}{3.926513in}}%
\pgfpathlineto{\pgfqpoint{2.704922in}{3.920731in}}%
\pgfpathlineto{\pgfqpoint{2.704991in}{3.920643in}}%
\pgfpathlineto{\pgfqpoint{2.709471in}{3.914948in}}%
\pgfpathlineto{\pgfqpoint{2.713296in}{3.910086in}}%
\pgfpathlineto{\pgfqpoint{2.714020in}{3.909166in}}%
\pgfpathlineto{\pgfqpoint{2.718568in}{3.903384in}}%
\pgfpathlineto{\pgfqpoint{2.721601in}{3.899529in}}%
\pgfpathlineto{\pgfqpoint{2.723117in}{3.897602in}}%
\pgfpathlineto{\pgfqpoint{2.727666in}{3.891820in}}%
\pgfpathlineto{\pgfqpoint{2.729906in}{3.888972in}}%
\pgfpathlineto{\pgfqpoint{2.732214in}{3.886037in}}%
\pgfpathlineto{\pgfqpoint{2.736763in}{3.880255in}}%
\pgfpathlineto{\pgfqpoint{2.738210in}{3.878415in}}%
\pgfpathlineto{\pgfqpoint{2.741311in}{3.874473in}}%
\pgfpathlineto{\pgfqpoint{2.745860in}{3.868691in}}%
\pgfpathlineto{\pgfqpoint{2.746515in}{3.867858in}}%
\pgfpathlineto{\pgfqpoint{2.750409in}{3.862909in}}%
\pgfpathlineto{\pgfqpoint{2.754820in}{3.857301in}}%
\pgfpathlineto{\pgfqpoint{2.754957in}{3.857126in}}%
\pgfpathlineto{\pgfqpoint{2.759506in}{3.851344in}}%
\pgfpathlineto{\pgfqpoint{2.763125in}{3.846744in}}%
\pgfpathlineto{\pgfqpoint{2.764055in}{3.845562in}}%
\pgfpathlineto{\pgfqpoint{2.768603in}{3.839780in}}%
\pgfpathlineto{\pgfqpoint{2.771430in}{3.836187in}}%
\pgfpathlineto{\pgfqpoint{2.773152in}{3.833998in}}%
\pgfpathlineto{\pgfqpoint{2.777700in}{3.828215in}}%
\pgfpathlineto{\pgfqpoint{2.779735in}{3.825630in}}%
\pgfpathlineto{\pgfqpoint{2.782249in}{3.822433in}}%
\pgfpathlineto{\pgfqpoint{2.786798in}{3.816651in}}%
\pgfpathlineto{\pgfqpoint{2.788039in}{3.815073in}}%
\pgfpathlineto{\pgfqpoint{2.791346in}{3.810869in}}%
\pgfpathlineto{\pgfqpoint{2.795895in}{3.805087in}}%
\pgfpathlineto{\pgfqpoint{2.796344in}{3.804516in}}%
\pgfpathlineto{\pgfqpoint{2.800444in}{3.799304in}}%
\pgfpathlineto{\pgfqpoint{2.804649in}{3.793958in}}%
\pgfpathlineto{\pgfqpoint{2.804992in}{3.793522in}}%
\pgfpathlineto{\pgfqpoint{2.809541in}{3.787740in}}%
\pgfpathlineto{\pgfqpoint{2.812954in}{3.783401in}}%
\pgfpathlineto{\pgfqpoint{2.814089in}{3.781958in}}%
\pgfpathlineto{\pgfqpoint{2.818638in}{3.776176in}}%
\pgfpathlineto{\pgfqpoint{2.821259in}{3.772844in}}%
\pgfpathlineto{\pgfqpoint{2.823187in}{3.770393in}}%
\pgfpathlineto{\pgfqpoint{2.827735in}{3.764611in}}%
\pgfpathlineto{\pgfqpoint{2.829564in}{3.762287in}}%
\pgfpathlineto{\pgfqpoint{2.832284in}{3.758829in}}%
\pgfpathlineto{\pgfqpoint{2.836833in}{3.753047in}}%
\pgfpathlineto{\pgfqpoint{2.837868in}{3.751730in}}%
\pgfpathlineto{\pgfqpoint{2.841381in}{3.747265in}}%
\pgfpathlineto{\pgfqpoint{2.845930in}{3.741482in}}%
\pgfpathlineto{\pgfqpoint{2.846173in}{3.741173in}}%
\pgfpathlineto{\pgfqpoint{2.850479in}{3.735700in}}%
\pgfpathlineto{\pgfqpoint{2.854478in}{3.730616in}}%
\pgfpathlineto{\pgfqpoint{2.855027in}{3.729918in}}%
\pgfpathlineto{\pgfqpoint{2.859576in}{3.724136in}}%
\pgfpathlineto{\pgfqpoint{2.862783in}{3.720059in}}%
\pgfpathlineto{\pgfqpoint{2.864124in}{3.718354in}}%
\pgfpathlineto{\pgfqpoint{2.868673in}{3.712571in}}%
\pgfpathlineto{\pgfqpoint{2.871088in}{3.709502in}}%
\pgfpathlineto{\pgfqpoint{2.873222in}{3.706789in}}%
\pgfpathlineto{\pgfqpoint{2.877770in}{3.701007in}}%
\pgfpathlineto{\pgfqpoint{2.879393in}{3.698945in}}%
\pgfpathlineto{\pgfqpoint{2.882319in}{3.695225in}}%
\pgfpathlineto{\pgfqpoint{2.886868in}{3.689443in}}%
\pgfpathlineto{\pgfqpoint{2.887697in}{3.688388in}}%
\pgfpathlineto{\pgfqpoint{2.891416in}{3.683660in}}%
\pgfpathlineto{\pgfqpoint{2.895965in}{3.677878in}}%
\pgfpathlineto{\pgfqpoint{2.896002in}{3.677831in}}%
\pgfpathlineto{\pgfqpoint{2.900513in}{3.672096in}}%
\pgfpathlineto{\pgfqpoint{2.904307in}{3.667274in}}%
\pgfpathlineto{\pgfqpoint{2.905062in}{3.666314in}}%
\pgfpathlineto{\pgfqpoint{2.909611in}{3.660532in}}%
\pgfpathlineto{\pgfqpoint{2.912612in}{3.656716in}}%
\pgfpathlineto{\pgfqpoint{2.914159in}{3.654749in}}%
\pgfpathlineto{\pgfqpoint{2.918708in}{3.648967in}}%
\pgfpathlineto{\pgfqpoint{2.920917in}{3.646159in}}%
\pgfpathlineto{\pgfqpoint{2.923257in}{3.643185in}}%
\pgfpathlineto{\pgfqpoint{2.927805in}{3.637403in}}%
\pgfpathlineto{\pgfqpoint{2.929222in}{3.635602in}}%
\pgfpathlineto{\pgfqpoint{2.932354in}{3.631621in}}%
\pgfpathlineto{\pgfqpoint{2.936902in}{3.625838in}}%
\pgfpathlineto{\pgfqpoint{2.937526in}{3.625045in}}%
\pgfpathlineto{\pgfqpoint{2.941451in}{3.620056in}}%
\pgfpathlineto{\pgfqpoint{2.945831in}{3.614488in}}%
\pgfpathlineto{\pgfqpoint{2.946000in}{3.614274in}}%
\pgfpathlineto{\pgfqpoint{2.950548in}{3.608492in}}%
\pgfpathlineto{\pgfqpoint{2.954136in}{3.603931in}}%
\pgfpathlineto{\pgfqpoint{2.955097in}{3.602710in}}%
\pgfpathlineto{\pgfqpoint{2.959646in}{3.596927in}}%
\pgfpathlineto{\pgfqpoint{2.962441in}{3.593374in}}%
\pgfpathlineto{\pgfqpoint{2.964194in}{3.591145in}}%
\pgfpathlineto{\pgfqpoint{2.968743in}{3.585363in}}%
\pgfpathlineto{\pgfqpoint{2.970746in}{3.582817in}}%
\pgfpathlineto{\pgfqpoint{2.973291in}{3.579581in}}%
\pgfpathlineto{\pgfqpoint{2.977840in}{3.573799in}}%
\pgfpathlineto{\pgfqpoint{2.979051in}{3.572260in}}%
\pgfpathlineto{\pgfqpoint{2.982389in}{3.568016in}}%
\pgfpathlineto{\pgfqpoint{2.986937in}{3.562234in}}%
\pgfpathlineto{\pgfqpoint{2.987355in}{3.561703in}}%
\pgfpathlineto{\pgfqpoint{2.991486in}{3.556452in}}%
\pgfpathlineto{\pgfqpoint{2.995660in}{3.551146in}}%
\pgfpathlineto{\pgfqpoint{2.996035in}{3.550670in}}%
\pgfpathlineto{\pgfqpoint{3.000583in}{3.544888in}}%
\pgfpathlineto{\pgfqpoint{3.003965in}{3.540589in}}%
\pgfpathlineto{\pgfqpoint{3.005132in}{3.539105in}}%
\pgfpathlineto{\pgfqpoint{3.009681in}{3.533323in}}%
\pgfpathlineto{\pgfqpoint{3.012270in}{3.530032in}}%
\pgfpathlineto{\pgfqpoint{3.014229in}{3.527541in}}%
\pgfpathlineto{\pgfqpoint{3.018778in}{3.521759in}}%
\pgfpathlineto{\pgfqpoint{3.020575in}{3.519474in}}%
\pgfpathlineto{\pgfqpoint{3.023326in}{3.515977in}}%
\pgfpathlineto{\pgfqpoint{3.027875in}{3.510194in}}%
\pgfpathlineto{\pgfqpoint{3.028880in}{3.508917in}}%
\pgfpathlineto{\pgfqpoint{3.032424in}{3.504412in}}%
\pgfpathlineto{\pgfqpoint{3.036972in}{3.498630in}}%
\pgfpathlineto{\pgfqpoint{3.037184in}{3.498360in}}%
\pgfpathlineto{\pgfqpoint{3.041521in}{3.492848in}}%
\pgfpathlineto{\pgfqpoint{3.045489in}{3.487803in}}%
\pgfpathlineto{\pgfqpoint{3.046070in}{3.487066in}}%
\pgfpathlineto{\pgfqpoint{3.050618in}{3.481283in}}%
\pgfpathlineto{\pgfqpoint{3.053794in}{3.477246in}}%
\pgfpathlineto{\pgfqpoint{3.055167in}{3.475501in}}%
\pgfpathlineto{\pgfqpoint{3.059715in}{3.469719in}}%
\pgfpathlineto{\pgfqpoint{3.062099in}{3.466689in}}%
\pgfpathlineto{\pgfqpoint{3.064264in}{3.463937in}}%
\pgfpathlineto{\pgfqpoint{3.068813in}{3.458155in}}%
\pgfpathlineto{\pgfqpoint{3.070404in}{3.456132in}}%
\pgfpathlineto{\pgfqpoint{3.073361in}{3.452372in}}%
\pgfpathlineto{\pgfqpoint{3.077910in}{3.446590in}}%
\pgfpathlineto{\pgfqpoint{3.078709in}{3.445575in}}%
\pgfpathlineto{\pgfqpoint{3.082459in}{3.440808in}}%
\pgfpathlineto{\pgfqpoint{3.087007in}{3.435026in}}%
\pgfpathlineto{\pgfqpoint{3.087013in}{3.435018in}}%
\pgfpathlineto{\pgfqpoint{3.091556in}{3.429244in}}%
\pgfpathlineto{\pgfqpoint{3.095318in}{3.424461in}}%
\pgfpathlineto{\pgfqpoint{3.096104in}{3.423461in}}%
\pgfpathlineto{\pgfqpoint{3.100653in}{3.417679in}}%
\pgfpathlineto{\pgfqpoint{3.103623in}{3.413904in}}%
\pgfpathlineto{\pgfqpoint{3.105202in}{3.411897in}}%
\pgfpathlineto{\pgfqpoint{3.109750in}{3.406115in}}%
\pgfpathlineto{\pgfqpoint{3.111928in}{3.403347in}}%
\pgfpathlineto{\pgfqpoint{3.114299in}{3.400333in}}%
\pgfpathlineto{\pgfqpoint{3.118848in}{3.394550in}}%
\pgfpathlineto{\pgfqpoint{3.120233in}{3.392790in}}%
\pgfpathlineto{\pgfqpoint{3.123396in}{3.388768in}}%
\pgfpathlineto{\pgfqpoint{3.127945in}{3.382986in}}%
\pgfpathlineto{\pgfqpoint{3.128538in}{3.382232in}}%
\pgfpathlineto{\pgfqpoint{3.132493in}{3.377204in}}%
\pgfpathlineto{\pgfqpoint{3.136842in}{3.371675in}}%
\pgfpathlineto{\pgfqpoint{3.137042in}{3.371422in}}%
\pgfpathlineto{\pgfqpoint{3.141591in}{3.365639in}}%
\pgfpathlineto{\pgfqpoint{3.145147in}{3.361118in}}%
\pgfpathlineto{\pgfqpoint{3.146139in}{3.359857in}}%
\pgfpathlineto{\pgfqpoint{3.150688in}{3.354075in}}%
\pgfpathlineto{\pgfqpoint{3.153452in}{3.350561in}}%
\pgfpathlineto{\pgfqpoint{3.155237in}{3.348293in}}%
\pgfpathlineto{\pgfqpoint{3.159785in}{3.342511in}}%
\pgfpathlineto{\pgfqpoint{3.161757in}{3.340004in}}%
\pgfpathlineto{\pgfqpoint{3.164334in}{3.336728in}}%
\pgfpathlineto{\pgfqpoint{3.168883in}{3.330946in}}%
\pgfpathlineto{\pgfqpoint{3.170062in}{3.329447in}}%
\pgfpathlineto{\pgfqpoint{3.173431in}{3.325164in}}%
\pgfpathlineto{\pgfqpoint{3.177980in}{3.319382in}}%
\pgfpathlineto{\pgfqpoint{3.178367in}{3.318890in}}%
\pgfpathlineto{\pgfqpoint{3.182528in}{3.313600in}}%
\pgfpathlineto{\pgfqpoint{3.186671in}{3.308333in}}%
\pgfpathlineto{\pgfqpoint{3.187077in}{3.307817in}}%
\pgfpathlineto{\pgfqpoint{3.191626in}{3.302035in}}%
\pgfpathlineto{\pgfqpoint{3.194976in}{3.297776in}}%
\pgfpathlineto{\pgfqpoint{3.196174in}{3.296253in}}%
\pgfpathlineto{\pgfqpoint{3.200723in}{3.290471in}}%
\pgfpathlineto{\pgfqpoint{3.203281in}{3.287219in}}%
\pgfpathlineto{\pgfqpoint{3.205272in}{3.284689in}}%
\pgfpathlineto{\pgfqpoint{3.209820in}{3.278906in}}%
\pgfpathlineto{\pgfqpoint{3.211586in}{3.276662in}}%
\pgfpathlineto{\pgfqpoint{3.214369in}{3.273124in}}%
\pgfpathlineto{\pgfqpoint{3.218917in}{3.267342in}}%
\pgfpathlineto{\pgfqpoint{3.219891in}{3.266105in}}%
\pgfpathlineto{\pgfqpoint{3.223466in}{3.261560in}}%
\pgfpathlineto{\pgfqpoint{3.228015in}{3.255778in}}%
\pgfpathlineto{\pgfqpoint{3.228196in}{3.255548in}}%
\pgfpathlineto{\pgfqpoint{3.232563in}{3.249995in}}%
\pgfpathlineto{\pgfqpoint{3.236500in}{3.244990in}}%
\pgfpathlineto{\pgfqpoint{3.237112in}{3.244213in}}%
\pgfpathlineto{\pgfqpoint{3.241661in}{3.238431in}}%
\pgfpathlineto{\pgfqpoint{3.244805in}{3.234433in}}%
\pgfpathlineto{\pgfqpoint{3.246209in}{3.232649in}}%
\pgfpathlineto{\pgfqpoint{3.250758in}{3.226867in}}%
\pgfpathlineto{\pgfqpoint{3.253110in}{3.223876in}}%
\pgfpathlineto{\pgfqpoint{3.255306in}{3.221084in}}%
\pgfpathlineto{\pgfqpoint{3.259855in}{3.215302in}}%
\pgfpathlineto{\pgfqpoint{3.261415in}{3.213319in}}%
\pgfpathlineto{\pgfqpoint{3.264404in}{3.209520in}}%
\pgfpathlineto{\pgfqpoint{3.268952in}{3.203738in}}%
\pgfpathlineto{\pgfqpoint{3.269720in}{3.202762in}}%
\pgfpathlineto{\pgfqpoint{3.273501in}{3.197956in}}%
\pgfpathlineto{\pgfqpoint{3.278025in}{3.192205in}}%
\pgfpathlineto{\pgfqpoint{3.278050in}{3.192173in}}%
\pgfpathlineto{\pgfqpoint{3.282598in}{3.186391in}}%
\pgfpathlineto{\pgfqpoint{3.286329in}{3.181648in}}%
\pgfpathlineto{\pgfqpoint{3.287147in}{3.180609in}}%
\pgfpathlineto{\pgfqpoint{3.291695in}{3.174827in}}%
\pgfpathlineto{\pgfqpoint{3.294634in}{3.171091in}}%
\pgfpathlineto{\pgfqpoint{3.296244in}{3.169045in}}%
\pgfpathlineto{\pgfqpoint{3.300793in}{3.163262in}}%
\pgfpathlineto{\pgfqpoint{3.302939in}{3.160534in}}%
\pgfpathlineto{\pgfqpoint{3.305341in}{3.157480in}}%
\pgfpathlineto{\pgfqpoint{3.309890in}{3.151698in}}%
\pgfpathlineto{\pgfqpoint{3.311244in}{3.149977in}}%
\pgfpathlineto{\pgfqpoint{3.314439in}{3.145916in}}%
\pgfpathlineto{\pgfqpoint{3.318987in}{3.140134in}}%
\pgfpathlineto{\pgfqpoint{3.319549in}{3.139420in}}%
\pgfpathlineto{\pgfqpoint{3.323536in}{3.134351in}}%
\pgfpathlineto{\pgfqpoint{3.327854in}{3.128863in}}%
\pgfpathlineto{\pgfqpoint{3.328085in}{3.128569in}}%
\pgfpathlineto{\pgfqpoint{3.332633in}{3.122787in}}%
\pgfpathlineto{\pgfqpoint{3.336158in}{3.118306in}}%
\pgfpathlineto{\pgfqpoint{3.337182in}{3.117005in}}%
\pgfpathlineto{\pgfqpoint{3.341730in}{3.111223in}}%
\pgfpathlineto{\pgfqpoint{3.344463in}{3.107748in}}%
\pgfpathlineto{\pgfqpoint{3.346279in}{3.105440in}}%
\pgfpathlineto{\pgfqpoint{3.350828in}{3.099658in}}%
\pgfpathlineto{\pgfqpoint{3.352768in}{3.097191in}}%
\pgfpathlineto{\pgfqpoint{3.355376in}{3.093876in}}%
\pgfpathlineto{\pgfqpoint{3.359925in}{3.088094in}}%
\pgfpathlineto{\pgfqpoint{3.361073in}{3.086634in}}%
\pgfpathlineto{\pgfqpoint{3.364474in}{3.082312in}}%
\pgfpathlineto{\pgfqpoint{3.369022in}{3.076529in}}%
\pgfpathlineto{\pgfqpoint{3.369378in}{3.076077in}}%
\pgfpathlineto{\pgfqpoint{3.373571in}{3.070747in}}%
\pgfpathlineto{\pgfqpoint{3.377683in}{3.065520in}}%
\pgfpathlineto{\pgfqpoint{3.378119in}{3.064965in}}%
\pgfpathlineto{\pgfqpoint{3.382668in}{3.059183in}}%
\pgfpathlineto{\pgfqpoint{3.385987in}{3.054963in}}%
\pgfpathlineto{\pgfqpoint{3.387217in}{3.053401in}}%
\pgfpathlineto{\pgfqpoint{3.391765in}{3.047618in}}%
\pgfpathlineto{\pgfqpoint{3.394292in}{3.044406in}}%
\pgfpathlineto{\pgfqpoint{3.396314in}{3.041836in}}%
\pgfpathlineto{\pgfqpoint{3.400863in}{3.036054in}}%
\pgfpathlineto{\pgfqpoint{3.402597in}{3.033849in}}%
\pgfpathlineto{\pgfqpoint{3.405411in}{3.030272in}}%
\pgfpathlineto{\pgfqpoint{3.409960in}{3.024490in}}%
\pgfpathlineto{\pgfqpoint{3.410902in}{3.023292in}}%
\pgfpathlineto{\pgfqpoint{3.414508in}{3.018707in}}%
\pgfpathlineto{\pgfqpoint{3.419057in}{3.012925in}}%
\pgfpathlineto{\pgfqpoint{3.419207in}{3.012735in}}%
\pgfpathlineto{\pgfqpoint{3.423606in}{3.007143in}}%
\pgfpathlineto{\pgfqpoint{3.427512in}{3.002178in}}%
\pgfpathlineto{\pgfqpoint{3.428154in}{3.001361in}}%
\pgfpathlineto{\pgfqpoint{3.432703in}{2.995579in}}%
\pgfpathlineto{\pgfqpoint{3.435816in}{2.991621in}}%
\pgfpathlineto{\pgfqpoint{3.437252in}{2.989796in}}%
\pgfpathlineto{\pgfqpoint{3.441800in}{2.984014in}}%
\pgfpathlineto{\pgfqpoint{3.444121in}{2.981064in}}%
\pgfpathlineto{\pgfqpoint{3.446349in}{2.978232in}}%
\pgfpathlineto{\pgfqpoint{3.450897in}{2.972450in}}%
\pgfpathlineto{\pgfqpoint{3.452426in}{2.970506in}}%
\pgfpathlineto{\pgfqpoint{3.455446in}{2.966668in}}%
\pgfpathlineto{\pgfqpoint{3.459995in}{2.960885in}}%
\pgfpathlineto{\pgfqpoint{3.460731in}{2.959949in}}%
\pgfpathlineto{\pgfqpoint{3.464543in}{2.955103in}}%
\pgfpathlineto{\pgfqpoint{3.469036in}{2.949392in}}%
\pgfpathlineto{\pgfqpoint{3.469092in}{2.949321in}}%
\pgfpathlineto{\pgfqpoint{3.473641in}{2.943539in}}%
\pgfpathlineto{\pgfqpoint{3.477341in}{2.938835in}}%
\pgfpathlineto{\pgfqpoint{3.478189in}{2.937757in}}%
\pgfpathlineto{\pgfqpoint{3.482738in}{2.931974in}}%
\pgfpathlineto{\pgfqpoint{3.485646in}{2.928278in}}%
\pgfpathlineto{\pgfqpoint{3.487287in}{2.926192in}}%
\pgfpathlineto{\pgfqpoint{3.491835in}{2.920410in}}%
\pgfpathlineto{\pgfqpoint{3.493950in}{2.917721in}}%
\pgfpathlineto{\pgfqpoint{3.496384in}{2.914628in}}%
\pgfpathlineto{\pgfqpoint{3.500932in}{2.908846in}}%
\pgfpathlineto{\pgfqpoint{3.502255in}{2.907164in}}%
\pgfpathlineto{\pgfqpoint{3.505481in}{2.903063in}}%
\pgfpathlineto{\pgfqpoint{3.510030in}{2.897281in}}%
\pgfpathlineto{\pgfqpoint{3.510560in}{2.896607in}}%
\pgfpathlineto{\pgfqpoint{3.514578in}{2.891499in}}%
\pgfpathlineto{\pgfqpoint{3.518865in}{2.886050in}}%
\pgfpathlineto{\pgfqpoint{3.519127in}{2.885717in}}%
\pgfpathlineto{\pgfqpoint{3.523676in}{2.879935in}}%
\pgfpathlineto{\pgfqpoint{3.527170in}{2.875493in}}%
\pgfpathlineto{\pgfqpoint{3.528224in}{2.874152in}}%
\pgfpathlineto{\pgfqpoint{3.532773in}{2.868370in}}%
\pgfpathlineto{\pgfqpoint{3.535475in}{2.864936in}}%
\pgfpathlineto{\pgfqpoint{3.537321in}{2.862588in}}%
\pgfpathlineto{\pgfqpoint{3.541870in}{2.856806in}}%
\pgfpathlineto{\pgfqpoint{3.543779in}{2.854379in}}%
\pgfpathlineto{\pgfqpoint{3.546419in}{2.851024in}}%
\pgfpathlineto{\pgfqpoint{3.550967in}{2.845241in}}%
\pgfpathlineto{\pgfqpoint{3.552084in}{2.843822in}}%
\pgfpathlineto{\pgfqpoint{3.555516in}{2.839459in}}%
\pgfpathlineto{\pgfqpoint{3.560065in}{2.833677in}}%
\pgfpathlineto{\pgfqpoint{3.560389in}{2.833264in}}%
\pgfpathlineto{\pgfqpoint{3.564613in}{2.827895in}}%
\pgfpathlineto{\pgfqpoint{3.568694in}{2.822707in}}%
\pgfpathlineto{\pgfqpoint{3.569162in}{2.822113in}}%
\pgfpathlineto{\pgfqpoint{3.573710in}{2.816330in}}%
\pgfpathlineto{\pgfqpoint{3.576999in}{2.812150in}}%
\pgfpathlineto{\pgfqpoint{3.578259in}{2.810548in}}%
\pgfpathlineto{\pgfqpoint{3.582808in}{2.804766in}}%
\pgfpathlineto{\pgfqpoint{3.585304in}{2.801593in}}%
\pgfpathlineto{\pgfqpoint{3.587356in}{2.798984in}}%
\pgfpathlineto{\pgfqpoint{3.591905in}{2.793202in}}%
\pgfpathlineto{\pgfqpoint{3.593608in}{2.791036in}}%
\pgfpathlineto{\pgfqpoint{3.596454in}{2.787419in}}%
\pgfpathlineto{\pgfqpoint{3.601002in}{2.781637in}}%
\pgfpathlineto{\pgfqpoint{3.601913in}{2.780479in}}%
\pgfpathlineto{\pgfqpoint{3.605551in}{2.775855in}}%
\pgfpathlineto{\pgfqpoint{3.610099in}{2.770073in}}%
\pgfpathlineto{\pgfqpoint{3.610218in}{2.769922in}}%
\pgfpathlineto{\pgfqpoint{3.614648in}{2.764291in}}%
\pgfpathlineto{\pgfqpoint{3.618523in}{2.759365in}}%
\pgfpathlineto{\pgfqpoint{3.619197in}{2.758508in}}%
\pgfpathlineto{\pgfqpoint{3.623745in}{2.752726in}}%
\pgfpathlineto{\pgfqpoint{3.626828in}{2.748808in}}%
\pgfpathlineto{\pgfqpoint{3.628294in}{2.746944in}}%
\pgfpathlineto{\pgfqpoint{3.632843in}{2.741162in}}%
\pgfpathlineto{\pgfqpoint{3.635133in}{2.738251in}}%
\pgfpathlineto{\pgfqpoint{3.637391in}{2.735380in}}%
\pgfpathlineto{\pgfqpoint{3.641940in}{2.729597in}}%
\pgfpathlineto{\pgfqpoint{3.643437in}{2.727694in}}%
\pgfpathlineto{\pgfqpoint{3.646489in}{2.723815in}}%
\pgfpathlineto{\pgfqpoint{3.651037in}{2.718033in}}%
\pgfpathlineto{\pgfqpoint{3.651742in}{2.717137in}}%
\pgfpathlineto{\pgfqpoint{3.655586in}{2.712251in}}%
\pgfpathlineto{\pgfqpoint{3.660047in}{2.706580in}}%
\pgfpathlineto{\pgfqpoint{3.660134in}{2.706469in}}%
\pgfpathlineto{\pgfqpoint{3.664683in}{2.700686in}}%
\pgfpathlineto{\pgfqpoint{3.668352in}{2.696022in}}%
\pgfpathlineto{\pgfqpoint{3.669232in}{2.694904in}}%
\pgfpathlineto{\pgfqpoint{3.673780in}{2.689122in}}%
\pgfpathlineto{\pgfqpoint{3.676657in}{2.685465in}}%
\pgfpathlineto{\pgfqpoint{3.678329in}{2.683340in}}%
\pgfpathlineto{\pgfqpoint{3.682878in}{2.677558in}}%
\pgfpathlineto{\pgfqpoint{3.684962in}{2.674908in}}%
\pgfpathlineto{\pgfqpoint{3.687426in}{2.671775in}}%
\pgfpathlineto{\pgfqpoint{3.691975in}{2.665993in}}%
\pgfpathlineto{\pgfqpoint{3.693266in}{2.664351in}}%
\pgfpathlineto{\pgfqpoint{3.696523in}{2.660211in}}%
\pgfpathlineto{\pgfqpoint{3.701072in}{2.654429in}}%
\pgfpathlineto{\pgfqpoint{3.701571in}{2.653794in}}%
\pgfpathlineto{\pgfqpoint{3.705621in}{2.648647in}}%
\pgfpathlineto{\pgfqpoint{3.709876in}{2.643237in}}%
\pgfpathlineto{\pgfqpoint{3.710169in}{2.642864in}}%
\pgfpathlineto{\pgfqpoint{3.714718in}{2.637082in}}%
\pgfpathlineto{\pgfqpoint{3.718181in}{2.632680in}}%
\pgfpathlineto{\pgfqpoint{3.719267in}{2.631300in}}%
\pgfpathlineto{\pgfqpoint{3.723815in}{2.625518in}}%
\pgfpathlineto{\pgfqpoint{3.726486in}{2.622123in}}%
\pgfpathlineto{\pgfqpoint{3.728364in}{2.619736in}}%
\pgfpathlineto{\pgfqpoint{3.732912in}{2.613953in}}%
\pgfpathlineto{\pgfqpoint{3.734791in}{2.611566in}}%
\pgfpathlineto{\pgfqpoint{3.737461in}{2.608171in}}%
\pgfpathlineto{\pgfqpoint{3.742010in}{2.602389in}}%
\pgfpathlineto{\pgfqpoint{3.743095in}{2.601009in}}%
\pgfpathlineto{\pgfqpoint{3.746558in}{2.596607in}}%
\pgfpathlineto{\pgfqpoint{3.751107in}{2.590825in}}%
\pgfpathlineto{\pgfqpoint{3.751400in}{2.590452in}}%
\pgfpathlineto{\pgfqpoint{3.755656in}{2.585042in}}%
\pgfpathlineto{\pgfqpoint{3.759705in}{2.579895in}}%
\pgfpathlineto{\pgfqpoint{3.760204in}{2.579260in}}%
\pgfpathlineto{\pgfqpoint{3.764753in}{2.573478in}}%
\pgfpathlineto{\pgfqpoint{3.768010in}{2.569338in}}%
\pgfpathlineto{\pgfqpoint{3.769301in}{2.567696in}}%
\pgfpathlineto{\pgfqpoint{3.773850in}{2.561914in}}%
\pgfpathlineto{\pgfqpoint{3.776315in}{2.558780in}}%
\pgfpathlineto{\pgfqpoint{3.778399in}{2.556131in}}%
\pgfpathlineto{\pgfqpoint{3.782947in}{2.550349in}}%
\pgfpathlineto{\pgfqpoint{3.784620in}{2.548223in}}%
\pgfpathlineto{\pgfqpoint{3.787496in}{2.544567in}}%
\pgfpathlineto{\pgfqpoint{3.792045in}{2.538785in}}%
\pgfpathlineto{\pgfqpoint{3.792924in}{2.537666in}}%
\pgfpathlineto{\pgfqpoint{3.796593in}{2.533003in}}%
\pgfpathlineto{\pgfqpoint{3.801142in}{2.527220in}}%
\pgfpathlineto{\pgfqpoint{3.801229in}{2.527109in}}%
\pgfpathlineto{\pgfqpoint{3.805691in}{2.521438in}}%
\pgfpathlineto{\pgfqpoint{3.809534in}{2.516552in}}%
\pgfpathlineto{\pgfqpoint{3.810239in}{2.515656in}}%
\pgfpathlineto{\pgfqpoint{3.814788in}{2.509874in}}%
\pgfpathlineto{\pgfqpoint{3.817839in}{2.505995in}}%
\pgfpathlineto{\pgfqpoint{3.819336in}{2.504092in}}%
\pgfpathlineto{\pgfqpoint{3.823885in}{2.498309in}}%
\pgfpathlineto{\pgfqpoint{3.826144in}{2.495438in}}%
\pgfpathlineto{\pgfqpoint{3.828434in}{2.492527in}}%
\pgfpathlineto{\pgfqpoint{3.832982in}{2.486745in}}%
\pgfpathlineto{\pgfqpoint{3.834449in}{2.484881in}}%
\pgfpathlineto{\pgfqpoint{3.837531in}{2.480963in}}%
\pgfpathlineto{\pgfqpoint{3.842080in}{2.475181in}}%
\pgfpathlineto{\pgfqpoint{3.842753in}{2.474324in}}%
\pgfpathlineto{\pgfqpoint{3.846628in}{2.469398in}}%
\pgfpathlineto{\pgfqpoint{3.851058in}{2.463767in}}%
\pgfpathlineto{\pgfqpoint{3.851177in}{2.463616in}}%
\pgfpathlineto{\pgfqpoint{3.855725in}{2.457834in}}%
\pgfpathlineto{\pgfqpoint{3.859363in}{2.453210in}}%
\pgfpathlineto{\pgfqpoint{3.860274in}{2.452052in}}%
\pgfpathlineto{\pgfqpoint{3.864823in}{2.446269in}}%
\pgfpathlineto{\pgfqpoint{3.867668in}{2.442653in}}%
\pgfpathlineto{\pgfqpoint{3.869371in}{2.440487in}}%
\pgfpathlineto{\pgfqpoint{3.873920in}{2.434705in}}%
\pgfpathlineto{\pgfqpoint{3.875973in}{2.432096in}}%
\pgfpathlineto{\pgfqpoint{3.878469in}{2.428923in}}%
\pgfpathlineto{\pgfqpoint{3.883017in}{2.423141in}}%
\pgfpathlineto{\pgfqpoint{3.884278in}{2.421538in}}%
\pgfpathlineto{\pgfqpoint{3.887566in}{2.417358in}}%
\pgfpathlineto{\pgfqpoint{3.892114in}{2.411576in}}%
\pgfpathlineto{\pgfqpoint{3.892582in}{2.410981in}}%
\pgfpathlineto{\pgfqpoint{3.896663in}{2.405794in}}%
\pgfpathlineto{\pgfqpoint{3.900887in}{2.400424in}}%
\pgfpathlineto{\pgfqpoint{3.901212in}{2.400012in}}%
\pgfpathlineto{\pgfqpoint{3.905760in}{2.394230in}}%
\pgfpathlineto{\pgfqpoint{3.909192in}{2.389867in}}%
\pgfpathlineto{\pgfqpoint{3.910309in}{2.388447in}}%
\pgfpathlineto{\pgfqpoint{3.914858in}{2.382665in}}%
\pgfpathlineto{\pgfqpoint{3.917497in}{2.379310in}}%
\pgfpathlineto{\pgfqpoint{3.919406in}{2.376883in}}%
\pgfpathlineto{\pgfqpoint{3.923955in}{2.371101in}}%
\pgfpathlineto{\pgfqpoint{3.925802in}{2.368753in}}%
\pgfpathlineto{\pgfqpoint{3.928503in}{2.365319in}}%
\pgfpathlineto{\pgfqpoint{3.933052in}{2.359536in}}%
\pgfpathlineto{\pgfqpoint{3.934107in}{2.358196in}}%
\pgfpathlineto{\pgfqpoint{3.937601in}{2.353754in}}%
\pgfpathlineto{\pgfqpoint{3.942149in}{2.347972in}}%
\pgfpathlineto{\pgfqpoint{3.942411in}{2.347639in}}%
\pgfpathlineto{\pgfqpoint{3.946698in}{2.342190in}}%
\pgfpathlineto{\pgfqpoint{3.950716in}{2.337082in}}%
\pgfpathlineto{\pgfqpoint{3.951247in}{2.336408in}}%
\pgfpathlineto{\pgfqpoint{3.955795in}{2.330625in}}%
\pgfpathlineto{\pgfqpoint{3.959021in}{2.326525in}}%
\pgfpathlineto{\pgfqpoint{3.960344in}{2.324843in}}%
\pgfpathlineto{\pgfqpoint{3.964893in}{2.319061in}}%
\pgfpathlineto{\pgfqpoint{3.967326in}{2.315968in}}%
\pgfpathlineto{\pgfqpoint{3.969441in}{2.313279in}}%
\pgfpathlineto{\pgfqpoint{3.973990in}{2.307497in}}%
\pgfpathlineto{\pgfqpoint{3.975631in}{2.305411in}}%
\pgfpathlineto{\pgfqpoint{3.978538in}{2.301714in}}%
\pgfpathlineto{\pgfqpoint{3.983087in}{2.295932in}}%
\pgfpathlineto{\pgfqpoint{3.983936in}{2.294854in}}%
\pgfpathlineto{\pgfqpoint{3.987636in}{2.290150in}}%
\pgfpathlineto{\pgfqpoint{3.992184in}{2.284368in}}%
\pgfpathlineto{\pgfqpoint{3.992240in}{2.284296in}}%
\pgfpathlineto{\pgfqpoint{3.996733in}{2.278586in}}%
\pgfpathlineto{\pgfqpoint{4.000545in}{2.273739in}}%
\pgfpathlineto{\pgfqpoint{4.001282in}{2.272803in}}%
\pgfpathlineto{\pgfqpoint{4.005830in}{2.267021in}}%
\pgfpathlineto{\pgfqpoint{4.008850in}{2.263182in}}%
\pgfpathlineto{\pgfqpoint{4.010379in}{2.261239in}}%
\pgfpathlineto{\pgfqpoint{4.014927in}{2.255457in}}%
\pgfpathlineto{\pgfqpoint{4.017155in}{2.252625in}}%
\pgfpathlineto{\pgfqpoint{4.019476in}{2.249675in}}%
\pgfpathlineto{\pgfqpoint{4.024025in}{2.243892in}}%
\pgfpathlineto{\pgfqpoint{4.025460in}{2.242068in}}%
\pgfpathlineto{\pgfqpoint{4.028573in}{2.238110in}}%
\pgfpathlineto{\pgfqpoint{4.033122in}{2.232328in}}%
\pgfpathlineto{\pgfqpoint{4.033765in}{2.231511in}}%
\pgfpathlineto{\pgfqpoint{4.037671in}{2.226546in}}%
\pgfpathlineto{\pgfqpoint{4.042069in}{2.220954in}}%
\pgfpathlineto{\pgfqpoint{4.042219in}{2.220764in}}%
\pgfpathlineto{\pgfqpoint{4.046768in}{2.214981in}}%
\pgfpathlineto{\pgfqpoint{4.050374in}{2.210397in}}%
\pgfpathlineto{\pgfqpoint{4.051316in}{2.209199in}}%
\pgfpathlineto{\pgfqpoint{4.055865in}{2.203417in}}%
\pgfpathlineto{\pgfqpoint{4.058679in}{2.199840in}}%
\pgfpathlineto{\pgfqpoint{4.060414in}{2.197635in}}%
\pgfpathlineto{\pgfqpoint{4.064962in}{2.191853in}}%
\pgfpathlineto{\pgfqpoint{4.066984in}{2.189283in}}%
\pgfpathlineto{\pgfqpoint{4.069511in}{2.186070in}}%
\pgfpathlineto{\pgfqpoint{4.074060in}{2.180288in}}%
\pgfpathlineto{\pgfqpoint{4.075289in}{2.178726in}}%
\pgfpathlineto{\pgfqpoint{4.078608in}{2.174506in}}%
\pgfpathlineto{\pgfqpoint{4.083157in}{2.168724in}}%
\pgfpathlineto{\pgfqpoint{4.083594in}{2.168169in}}%
\pgfpathlineto{\pgfqpoint{4.087705in}{2.162942in}}%
\pgfpathlineto{\pgfqpoint{4.091898in}{2.157612in}}%
\pgfpathlineto{\pgfqpoint{4.092254in}{2.157159in}}%
\pgfpathlineto{\pgfqpoint{4.096803in}{2.151377in}}%
\pgfpathlineto{\pgfqpoint{4.100203in}{2.147055in}}%
\pgfpathlineto{\pgfqpoint{4.101351in}{2.145595in}}%
\pgfpathlineto{\pgfqpoint{4.105900in}{2.139813in}}%
\pgfpathlineto{\pgfqpoint{4.108508in}{2.136497in}}%
\pgfpathlineto{\pgfqpoint{4.110449in}{2.134031in}}%
\pgfpathlineto{\pgfqpoint{4.114997in}{2.128248in}}%
\pgfpathlineto{\pgfqpoint{4.116813in}{2.125940in}}%
\pgfpathlineto{\pgfqpoint{4.119546in}{2.122466in}}%
\pgfpathlineto{\pgfqpoint{4.124094in}{2.116684in}}%
\pgfpathlineto{\pgfqpoint{4.125118in}{2.115383in}}%
\pgfpathlineto{\pgfqpoint{4.128643in}{2.110902in}}%
\pgfpathlineto{\pgfqpoint{4.133192in}{2.105120in}}%
\pgfpathlineto{\pgfqpoint{4.133423in}{2.104826in}}%
\pgfpathlineto{\pgfqpoint{4.137740in}{2.099337in}}%
\pgfpathlineto{\pgfqpoint{4.141727in}{2.094269in}}%
\pgfpathlineto{\pgfqpoint{4.142289in}{2.093555in}}%
\pgfpathlineto{\pgfqpoint{4.146838in}{2.087773in}}%
\pgfpathlineto{\pgfqpoint{4.150032in}{2.083712in}}%
\pgfpathlineto{\pgfqpoint{4.151386in}{2.081991in}}%
\pgfpathlineto{\pgfqpoint{4.155935in}{2.076209in}}%
\pgfpathlineto{\pgfqpoint{4.158337in}{2.073155in}}%
\pgfpathlineto{\pgfqpoint{4.160484in}{2.070426in}}%
\pgfpathlineto{\pgfqpoint{4.165032in}{2.064644in}}%
\pgfpathlineto{\pgfqpoint{4.166642in}{2.062598in}}%
\pgfpathlineto{\pgfqpoint{4.169581in}{2.058862in}}%
\pgfpathlineto{\pgfqpoint{4.174129in}{2.053080in}}%
\pgfpathlineto{\pgfqpoint{4.174947in}{2.052041in}}%
\pgfpathlineto{\pgfqpoint{4.178678in}{2.047298in}}%
\pgfpathlineto{\pgfqpoint{4.183227in}{2.041515in}}%
\pgfpathlineto{\pgfqpoint{4.183252in}{2.041484in}}%
\pgfpathlineto{\pgfqpoint{4.187775in}{2.035733in}}%
\pgfpathlineto{\pgfqpoint{4.191556in}{2.030927in}}%
\pgfpathlineto{\pgfqpoint{4.192324in}{2.029951in}}%
\pgfpathlineto{\pgfqpoint{4.196873in}{2.024169in}}%
\pgfpathlineto{\pgfqpoint{4.199861in}{2.020370in}}%
\pgfpathlineto{\pgfqpoint{4.201421in}{2.018387in}}%
\pgfpathlineto{\pgfqpoint{4.205970in}{2.012604in}}%
\pgfpathlineto{\pgfqpoint{4.208166in}{2.009813in}}%
\pgfpathlineto{\pgfqpoint{4.210518in}{2.006822in}}%
\pgfpathlineto{\pgfqpoint{4.215067in}{2.001040in}}%
\pgfpathlineto{\pgfqpoint{4.216471in}{1.999255in}}%
\pgfpathlineto{\pgfqpoint{4.219616in}{1.995258in}}%
\pgfpathlineto{\pgfqpoint{4.224164in}{1.989476in}}%
\pgfpathlineto{\pgfqpoint{4.224776in}{1.988698in}}%
\pgfpathlineto{\pgfqpoint{4.228713in}{1.983693in}}%
\pgfpathlineto{\pgfqpoint{4.233081in}{1.978141in}}%
\pgfpathlineto{\pgfqpoint{4.233262in}{1.977911in}}%
\pgfpathlineto{\pgfqpoint{4.237810in}{1.972129in}}%
\pgfpathlineto{\pgfqpoint{4.241385in}{1.967584in}}%
\pgfpathlineto{\pgfqpoint{4.242359in}{1.966347in}}%
\pgfpathlineto{\pgfqpoint{4.246907in}{1.960565in}}%
\pgfpathlineto{\pgfqpoint{4.249690in}{1.957027in}}%
\pgfpathlineto{\pgfqpoint{4.251456in}{1.954782in}}%
\pgfpathlineto{\pgfqpoint{4.256005in}{1.949000in}}%
\pgfpathlineto{\pgfqpoint{4.257995in}{1.946470in}}%
\pgfpathlineto{\pgfqpoint{4.260553in}{1.943218in}}%
\pgfpathlineto{\pgfqpoint{4.265102in}{1.937436in}}%
\pgfpathlineto{\pgfqpoint{4.266300in}{1.935913in}}%
\pgfpathlineto{\pgfqpoint{4.269651in}{1.931654in}}%
\pgfpathlineto{\pgfqpoint{4.274199in}{1.925871in}}%
\pgfpathlineto{\pgfqpoint{4.274605in}{1.925356in}}%
\pgfpathlineto{\pgfqpoint{4.278748in}{1.920089in}}%
\pgfpathlineto{\pgfqpoint{4.282910in}{1.914799in}}%
\pgfpathlineto{\pgfqpoint{4.283296in}{1.914307in}}%
\pgfpathlineto{\pgfqpoint{4.287845in}{1.908525in}}%
\pgfpathlineto{\pgfqpoint{4.291214in}{1.904242in}}%
\pgfpathlineto{\pgfqpoint{4.292394in}{1.902743in}}%
\pgfpathlineto{\pgfqpoint{4.296942in}{1.896960in}}%
\pgfpathlineto{\pgfqpoint{4.299519in}{1.893685in}}%
\pgfpathlineto{\pgfqpoint{4.301491in}{1.891178in}}%
\pgfpathlineto{\pgfqpoint{4.306040in}{1.885396in}}%
\pgfpathlineto{\pgfqpoint{4.307824in}{1.883128in}}%
\pgfpathlineto{\pgfqpoint{4.310588in}{1.879614in}}%
\pgfpathlineto{\pgfqpoint{4.315137in}{1.873832in}}%
\pgfpathlineto{\pgfqpoint{4.316129in}{1.872571in}}%
\pgfpathlineto{\pgfqpoint{4.319686in}{1.868049in}}%
\pgfpathlineto{\pgfqpoint{4.324234in}{1.862267in}}%
\pgfpathlineto{\pgfqpoint{4.324434in}{1.862013in}}%
\pgfpathlineto{\pgfqpoint{4.328783in}{1.856485in}}%
\pgfpathlineto{\pgfqpoint{4.332739in}{1.851456in}}%
\pgfpathlineto{\pgfqpoint{4.333331in}{1.850703in}}%
\pgfpathlineto{\pgfqpoint{4.337880in}{1.844921in}}%
\pgfpathlineto{\pgfqpoint{4.341043in}{1.840899in}}%
\pgfpathlineto{\pgfqpoint{4.342429in}{1.839138in}}%
\pgfpathlineto{\pgfqpoint{4.346977in}{1.833356in}}%
\pgfpathlineto{\pgfqpoint{4.349348in}{1.830342in}}%
\pgfpathlineto{\pgfqpoint{4.351526in}{1.827574in}}%
\pgfpathlineto{\pgfqpoint{4.356075in}{1.821792in}}%
\pgfpathlineto{\pgfqpoint{4.357653in}{1.819785in}}%
\pgfpathlineto{\pgfqpoint{4.360623in}{1.816010in}}%
\pgfpathlineto{\pgfqpoint{4.365172in}{1.810227in}}%
\pgfpathlineto{\pgfqpoint{4.365958in}{1.809228in}}%
\pgfpathlineto{\pgfqpoint{4.369720in}{1.804445in}}%
\pgfpathlineto{\pgfqpoint{4.374263in}{1.798671in}}%
\pgfpathlineto{\pgfqpoint{4.374269in}{1.798663in}}%
\pgfpathlineto{\pgfqpoint{4.378818in}{1.792881in}}%
\pgfpathlineto{\pgfqpoint{4.382568in}{1.788114in}}%
\pgfpathlineto{\pgfqpoint{4.383366in}{1.787099in}}%
\pgfpathlineto{\pgfqpoint{4.387915in}{1.781316in}}%
\pgfpathlineto{\pgfqpoint{4.390873in}{1.777557in}}%
\pgfpathlineto{\pgfqpoint{4.392464in}{1.775534in}}%
\pgfpathlineto{\pgfqpoint{4.397012in}{1.769752in}}%
\pgfpathlineto{\pgfqpoint{4.399177in}{1.767000in}}%
\pgfpathlineto{\pgfqpoint{4.401561in}{1.763970in}}%
\pgfpathlineto{\pgfqpoint{4.406109in}{1.758188in}}%
\pgfpathlineto{\pgfqpoint{4.407482in}{1.756443in}}%
\pgfpathlineto{\pgfqpoint{4.410658in}{1.752405in}}%
\pgfpathlineto{\pgfqpoint{4.415207in}{1.746623in}}%
\pgfpathlineto{\pgfqpoint{4.415787in}{1.745886in}}%
\pgfpathlineto{\pgfqpoint{4.419755in}{1.740841in}}%
\pgfpathlineto{\pgfqpoint{4.424092in}{1.735329in}}%
\pgfpathlineto{\pgfqpoint{4.424304in}{1.735059in}}%
\pgfpathlineto{\pgfqpoint{4.428853in}{1.729277in}}%
\pgfpathlineto{\pgfqpoint{4.432397in}{1.724771in}}%
\pgfpathlineto{\pgfqpoint{4.433401in}{1.723494in}}%
\pgfpathlineto{\pgfqpoint{4.437950in}{1.717712in}}%
\pgfpathlineto{\pgfqpoint{4.440702in}{1.714214in}}%
\pgfpathlineto{\pgfqpoint{4.442498in}{1.711930in}}%
\pgfpathlineto{\pgfqpoint{4.447047in}{1.706148in}}%
\pgfpathlineto{\pgfqpoint{4.449006in}{1.703657in}}%
\pgfpathlineto{\pgfqpoint{4.451596in}{1.700366in}}%
\pgfpathlineto{\pgfqpoint{4.456144in}{1.694583in}}%
\pgfpathlineto{\pgfqpoint{4.457311in}{1.693100in}}%
\pgfpathlineto{\pgfqpoint{4.460693in}{1.688801in}}%
\pgfpathlineto{\pgfqpoint{4.465242in}{1.683019in}}%
\pgfpathlineto{\pgfqpoint{4.465616in}{1.682543in}}%
\pgfpathlineto{\pgfqpoint{4.469790in}{1.677237in}}%
\pgfpathlineto{\pgfqpoint{4.473921in}{1.671986in}}%
\pgfpathlineto{\pgfqpoint{4.474339in}{1.671455in}}%
\pgfpathlineto{\pgfqpoint{4.478888in}{1.665672in}}%
\pgfpathlineto{\pgfqpoint{4.482226in}{1.661429in}}%
\pgfpathlineto{\pgfqpoint{4.483436in}{1.659890in}}%
\pgfpathlineto{\pgfqpoint{4.487985in}{1.654108in}}%
\pgfpathlineto{\pgfqpoint{4.490531in}{1.650872in}}%
\pgfpathlineto{\pgfqpoint{4.492533in}{1.648326in}}%
\pgfpathlineto{\pgfqpoint{4.497082in}{1.642544in}}%
\pgfpathlineto{\pgfqpoint{4.498835in}{1.640315in}}%
\pgfpathlineto{\pgfqpoint{4.501631in}{1.636761in}}%
\pgfpathlineto{\pgfqpoint{4.506179in}{1.630979in}}%
\pgfpathlineto{\pgfqpoint{4.507140in}{1.629758in}}%
\pgfpathlineto{\pgfqpoint{4.510728in}{1.625197in}}%
\pgfpathlineto{\pgfqpoint{4.515277in}{1.619415in}}%
\pgfpathlineto{\pgfqpoint{4.515445in}{1.619201in}}%
\pgfpathlineto{\pgfqpoint{4.519825in}{1.613633in}}%
\pgfpathlineto{\pgfqpoint{4.523750in}{1.608644in}}%
\pgfpathlineto{\pgfqpoint{4.524374in}{1.607850in}}%
\pgfpathlineto{\pgfqpoint{4.528922in}{1.602068in}}%
\pgfpathlineto{\pgfqpoint{4.532055in}{1.598087in}}%
\pgfpathlineto{\pgfqpoint{4.533471in}{1.596286in}}%
\pgfpathlineto{\pgfqpoint{4.538020in}{1.590504in}}%
\pgfpathlineto{\pgfqpoint{4.540360in}{1.587529in}}%
\pgfpathlineto{\pgfqpoint{4.542568in}{1.584722in}}%
\pgfpathlineto{\pgfqpoint{4.547117in}{1.578939in}}%
\pgfpathlineto{\pgfqpoint{4.548664in}{1.576972in}}%
\pgfpathlineto{\pgfqpoint{4.551666in}{1.573157in}}%
\pgfpathlineto{\pgfqpoint{4.556214in}{1.567375in}}%
\pgfpathlineto{\pgfqpoint{4.556969in}{1.566415in}}%
\pgfpathlineto{\pgfqpoint{4.560763in}{1.561593in}}%
\pgfpathlineto{\pgfqpoint{4.565274in}{1.555858in}}%
\pgfpathlineto{\pgfqpoint{4.565311in}{1.555811in}}%
\pgfpathlineto{\pgfqpoint{4.569860in}{1.550028in}}%
\pgfpathlineto{\pgfqpoint{4.573579in}{1.545301in}}%
\pgfpathlineto{\pgfqpoint{4.574409in}{1.544246in}}%
\pgfpathlineto{\pgfqpoint{4.578957in}{1.538464in}}%
\pgfpathlineto{\pgfqpoint{4.581884in}{1.534744in}}%
\pgfpathlineto{\pgfqpoint{4.583506in}{1.532682in}}%
\pgfpathlineto{\pgfqpoint{4.588055in}{1.526900in}}%
\pgfpathlineto{\pgfqpoint{4.590189in}{1.524187in}}%
\pgfpathlineto{\pgfqpoint{4.592603in}{1.521117in}}%
\pgfpathlineto{\pgfqpoint{4.597152in}{1.515335in}}%
\pgfpathlineto{\pgfqpoint{4.598493in}{1.513630in}}%
\pgfpathlineto{\pgfqpoint{4.601700in}{1.509553in}}%
\pgfpathlineto{\pgfqpoint{4.606249in}{1.503771in}}%
\pgfpathlineto{\pgfqpoint{4.606798in}{1.503073in}}%
\pgfpathlineto{\pgfqpoint{4.610798in}{1.497989in}}%
\pgfpathlineto{\pgfqpoint{4.615103in}{1.492516in}}%
\pgfpathlineto{\pgfqpoint{4.615346in}{1.492206in}}%
\pgfpathlineto{\pgfqpoint{4.619895in}{1.486424in}}%
\pgfpathlineto{\pgfqpoint{4.623408in}{1.481959in}}%
\pgfpathlineto{\pgfqpoint{4.624444in}{1.480642in}}%
\pgfpathlineto{\pgfqpoint{4.628992in}{1.474860in}}%
\pgfpathlineto{\pgfqpoint{4.631713in}{1.471402in}}%
\pgfpathlineto{\pgfqpoint{4.633541in}{1.469078in}}%
\pgfpathlineto{\pgfqpoint{4.638090in}{1.463295in}}%
\pgfpathlineto{\pgfqpoint{4.640018in}{1.460845in}}%
\pgfpathlineto{\pgfqpoint{4.642638in}{1.457513in}}%
\pgfpathlineto{\pgfqpoint{4.647187in}{1.451731in}}%
\pgfpathlineto{\pgfqpoint{4.648322in}{1.450287in}}%
\pgfpathlineto{\pgfqpoint{4.651735in}{1.445949in}}%
\pgfpathlineto{\pgfqpoint{4.656284in}{1.440167in}}%
\pgfpathlineto{\pgfqpoint{4.656627in}{1.439730in}}%
\pgfpathlineto{\pgfqpoint{4.660833in}{1.434384in}}%
\pgfpathlineto{\pgfqpoint{4.664932in}{1.429173in}}%
\pgfpathlineto{\pgfqpoint{4.665381in}{1.428602in}}%
\pgfpathlineto{\pgfqpoint{4.669930in}{1.422820in}}%
\pgfpathlineto{\pgfqpoint{4.673237in}{1.418616in}}%
\pgfpathlineto{\pgfqpoint{4.674479in}{1.417038in}}%
\pgfpathlineto{\pgfqpoint{4.679027in}{1.411256in}}%
\pgfpathlineto{\pgfqpoint{4.681542in}{1.408059in}}%
\pgfpathlineto{\pgfqpoint{4.683576in}{1.405473in}}%
\pgfpathlineto{\pgfqpoint{4.688124in}{1.399691in}}%
\pgfpathlineto{\pgfqpoint{4.689847in}{1.397502in}}%
\pgfpathlineto{\pgfqpoint{4.692673in}{1.393909in}}%
\pgfpathlineto{\pgfqpoint{4.697222in}{1.388127in}}%
\pgfpathlineto{\pgfqpoint{4.698151in}{1.386945in}}%
\pgfpathlineto{\pgfqpoint{4.701770in}{1.382345in}}%
\pgfpathlineto{\pgfqpoint{4.706319in}{1.376562in}}%
\pgfpathlineto{\pgfqpoint{4.706456in}{1.376388in}}%
\pgfpathlineto{\pgfqpoint{4.710868in}{1.370780in}}%
\pgfpathlineto{\pgfqpoint{4.714761in}{1.365831in}}%
\pgfpathlineto{\pgfqpoint{4.715416in}{1.364998in}}%
\pgfpathlineto{\pgfqpoint{4.719965in}{1.359216in}}%
\pgfpathlineto{\pgfqpoint{4.723066in}{1.355274in}}%
\pgfpathlineto{\pgfqpoint{4.724513in}{1.353434in}}%
\pgfpathlineto{\pgfqpoint{4.729062in}{1.347651in}}%
\pgfpathlineto{\pgfqpoint{4.731371in}{1.344717in}}%
\pgfpathlineto{\pgfqpoint{4.733611in}{1.341869in}}%
\pgfpathlineto{\pgfqpoint{4.738159in}{1.336087in}}%
\pgfpathlineto{\pgfqpoint{4.739676in}{1.334160in}}%
\pgfpathlineto{\pgfqpoint{4.742708in}{1.330305in}}%
\pgfpathlineto{\pgfqpoint{4.747257in}{1.324523in}}%
\pgfpathlineto{\pgfqpoint{4.747980in}{1.323603in}}%
\pgfpathlineto{\pgfqpoint{4.751805in}{1.318740in}}%
\pgfpathlineto{\pgfqpoint{4.756285in}{1.313045in}}%
\pgfpathlineto{\pgfqpoint{4.756354in}{1.312958in}}%
\pgfpathlineto{\pgfqpoint{4.760902in}{1.307176in}}%
\pgfpathlineto{\pgfqpoint{4.764590in}{1.302488in}}%
\pgfpathlineto{\pgfqpoint{4.765451in}{1.301394in}}%
\pgfpathlineto{\pgfqpoint{4.770000in}{1.295612in}}%
\pgfpathlineto{\pgfqpoint{4.772895in}{1.291931in}}%
\pgfpathlineto{\pgfqpoint{4.774548in}{1.289829in}}%
\pgfpathlineto{\pgfqpoint{4.779097in}{1.284047in}}%
\pgfpathlineto{\pgfqpoint{4.781200in}{1.281374in}}%
\pgfpathlineto{\pgfqpoint{4.783646in}{1.278265in}}%
\pgfpathlineto{\pgfqpoint{4.788194in}{1.272483in}}%
\pgfpathlineto{\pgfqpoint{4.789505in}{1.270817in}}%
\pgfpathlineto{\pgfqpoint{4.792743in}{1.266701in}}%
\pgfpathlineto{\pgfqpoint{4.797292in}{1.260918in}}%
\pgfpathlineto{\pgfqpoint{4.797809in}{1.260260in}}%
\pgfpathlineto{\pgfqpoint{4.801840in}{1.255136in}}%
\pgfpathlineto{\pgfqpoint{4.806114in}{1.249703in}}%
\pgfpathlineto{\pgfqpoint{4.806389in}{1.249354in}}%
\pgfpathlineto{\pgfqpoint{4.810937in}{1.243572in}}%
\pgfpathlineto{\pgfqpoint{4.814419in}{1.239146in}}%
\pgfpathlineto{\pgfqpoint{4.815486in}{1.237790in}}%
\pgfpathlineto{\pgfqpoint{4.820035in}{1.232007in}}%
\pgfpathlineto{\pgfqpoint{4.822724in}{1.228589in}}%
\pgfpathlineto{\pgfqpoint{4.824583in}{1.226225in}}%
\pgfpathlineto{\pgfqpoint{4.829132in}{1.220443in}}%
\pgfpathlineto{\pgfqpoint{4.831029in}{1.218032in}}%
\pgfpathlineto{\pgfqpoint{4.833681in}{1.214661in}}%
\pgfpathlineto{\pgfqpoint{4.838229in}{1.208879in}}%
\pgfpathlineto{\pgfqpoint{4.839334in}{1.207475in}}%
\pgfpathlineto{\pgfqpoint{4.842778in}{1.203096in}}%
\pgfpathlineto{\pgfqpoint{4.847326in}{1.197314in}}%
\pgfpathlineto{\pgfqpoint{4.847638in}{1.196918in}}%
\pgfpathlineto{\pgfqpoint{4.851875in}{1.191532in}}%
\pgfpathlineto{\pgfqpoint{4.855943in}{1.186361in}}%
\pgfpathlineto{\pgfqpoint{4.856424in}{1.185750in}}%
\pgfpathlineto{\pgfqpoint{4.860972in}{1.179968in}}%
\pgfpathlineto{\pgfqpoint{4.864248in}{1.175803in}}%
\pgfpathlineto{\pgfqpoint{4.865521in}{1.174185in}}%
\pgfpathlineto{\pgfqpoint{4.870070in}{1.168403in}}%
\pgfpathlineto{\pgfqpoint{4.872553in}{1.165246in}}%
\pgfpathlineto{\pgfqpoint{4.874618in}{1.162621in}}%
\pgfpathlineto{\pgfqpoint{4.879167in}{1.156839in}}%
\pgfpathlineto{\pgfqpoint{4.880858in}{1.154689in}}%
\pgfpathlineto{\pgfqpoint{4.883715in}{1.151057in}}%
\pgfpathlineto{\pgfqpoint{4.888264in}{1.145274in}}%
\pgfpathlineto{\pgfqpoint{4.889163in}{1.144132in}}%
\pgfpathlineto{\pgfqpoint{4.892813in}{1.139492in}}%
\pgfpathlineto{\pgfqpoint{4.897361in}{1.133710in}}%
\pgfpathlineto{\pgfqpoint{4.897467in}{1.133575in}}%
\pgfpathlineto{\pgfqpoint{4.901910in}{1.127928in}}%
\pgfpathlineto{\pgfqpoint{4.905772in}{1.123018in}}%
\pgfpathlineto{\pgfqpoint{4.906459in}{1.122146in}}%
\pgfpathlineto{\pgfqpoint{4.911007in}{1.116363in}}%
\pgfpathlineto{\pgfqpoint{4.914077in}{1.112461in}}%
\pgfpathlineto{\pgfqpoint{4.915556in}{1.110581in}}%
\pgfpathlineto{\pgfqpoint{4.920104in}{1.104799in}}%
\pgfpathlineto{\pgfqpoint{4.922382in}{1.101904in}}%
\pgfpathlineto{\pgfqpoint{4.924653in}{1.099017in}}%
\pgfpathlineto{\pgfqpoint{4.929202in}{1.093235in}}%
\pgfpathlineto{\pgfqpoint{4.930687in}{1.091347in}}%
\pgfpathlineto{\pgfqpoint{4.933750in}{1.087452in}}%
\pgfpathlineto{\pgfqpoint{4.938299in}{1.081670in}}%
\pgfpathlineto{\pgfqpoint{4.938992in}{1.080790in}}%
\pgfpathlineto{\pgfqpoint{4.942848in}{1.075888in}}%
\pgfpathlineto{\pgfqpoint{4.947296in}{1.070233in}}%
\pgfpathlineto{\pgfqpoint{4.947396in}{1.070106in}}%
\pgfpathlineto{\pgfqpoint{4.951945in}{1.064324in}}%
\pgfpathlineto{\pgfqpoint{4.955601in}{1.059676in}}%
\pgfpathlineto{\pgfqpoint{4.956494in}{1.058541in}}%
\pgfpathlineto{\pgfqpoint{4.961042in}{1.052759in}}%
\pgfpathlineto{\pgfqpoint{4.963906in}{1.049119in}}%
\pgfpathlineto{\pgfqpoint{4.965591in}{1.046977in}}%
\pgfpathlineto{\pgfqpoint{4.970139in}{1.041195in}}%
\pgfpathlineto{\pgfqpoint{4.972211in}{1.038561in}}%
\pgfpathlineto{\pgfqpoint{4.974688in}{1.035413in}}%
\pgfpathlineto{\pgfqpoint{4.979237in}{1.029630in}}%
\pgfpathlineto{\pgfqpoint{4.980516in}{1.028004in}}%
\pgfpathlineto{\pgfqpoint{4.983785in}{1.023848in}}%
\pgfpathlineto{\pgfqpoint{4.988334in}{1.018066in}}%
\pgfpathlineto{\pgfqpoint{4.988821in}{1.017447in}}%
\pgfpathlineto{\pgfqpoint{4.992883in}{1.012284in}}%
\pgfpathlineto{\pgfqpoint{4.997125in}{1.006890in}}%
\pgfpathlineto{\pgfqpoint{4.997431in}{1.006502in}}%
\pgfpathlineto{\pgfqpoint{5.001980in}{1.000719in}}%
\pgfpathlineto{\pgfqpoint{5.005430in}{0.996333in}}%
\pgfpathlineto{\pgfqpoint{5.006528in}{0.994937in}}%
\pgfpathlineto{\pgfqpoint{5.011077in}{0.989155in}}%
\pgfpathlineto{\pgfqpoint{5.013735in}{0.985776in}}%
\pgfpathlineto{\pgfqpoint{5.015626in}{0.983373in}}%
\pgfpathlineto{\pgfqpoint{5.020174in}{0.977591in}}%
\pgfpathlineto{\pgfqpoint{5.022040in}{0.975219in}}%
\pgfpathlineto{\pgfqpoint{5.024723in}{0.971808in}}%
\pgfpathlineto{\pgfqpoint{5.029272in}{0.966026in}}%
\pgfpathlineto{\pgfqpoint{5.030345in}{0.964662in}}%
\pgfpathlineto{\pgfqpoint{5.033820in}{0.960244in}}%
\pgfpathlineto{\pgfqpoint{5.038369in}{0.954462in}}%
\pgfpathlineto{\pgfqpoint{5.038650in}{0.954105in}}%
\pgfpathlineto{\pgfqpoint{5.042917in}{0.948680in}}%
\pgfpathlineto{\pgfqpoint{5.046954in}{0.943548in}}%
\pgfpathlineto{\pgfqpoint{5.047466in}{0.942897in}}%
\pgfpathlineto{\pgfqpoint{5.052015in}{0.937115in}}%
\pgfpathlineto{\pgfqpoint{5.055259in}{0.932991in}}%
\pgfpathlineto{\pgfqpoint{5.056563in}{0.931333in}}%
\pgfpathlineto{\pgfqpoint{5.061112in}{0.925551in}}%
\pgfpathlineto{\pgfqpoint{5.063564in}{0.922434in}}%
\pgfpathlineto{\pgfqpoint{5.065661in}{0.919769in}}%
\pgfpathlineto{\pgfqpoint{5.070209in}{0.913986in}}%
\pgfpathlineto{\pgfqpoint{5.071869in}{0.911877in}}%
\pgfpathlineto{\pgfqpoint{5.074758in}{0.908204in}}%
\pgfpathlineto{\pgfqpoint{5.079306in}{0.902422in}}%
\pgfpathlineto{\pgfqpoint{5.080174in}{0.901319in}}%
\pgfpathlineto{\pgfqpoint{5.083855in}{0.896640in}}%
\pgfpathlineto{\pgfqpoint{5.088404in}{0.890858in}}%
\pgfpathlineto{\pgfqpoint{5.088479in}{0.890762in}}%
\pgfpathlineto{\pgfqpoint{5.092952in}{0.885075in}}%
\pgfpathlineto{\pgfqpoint{5.096783in}{0.880205in}}%
\pgfpathlineto{\pgfqpoint{5.097501in}{0.879293in}}%
\pgfpathlineto{\pgfqpoint{5.102050in}{0.873511in}}%
\pgfpathlineto{\pgfqpoint{5.105088in}{0.869648in}}%
\pgfpathlineto{\pgfqpoint{5.106598in}{0.867729in}}%
\pgfpathlineto{\pgfqpoint{5.111147in}{0.861947in}}%
\pgfpathlineto{\pgfqpoint{5.113393in}{0.859091in}}%
\pgfpathlineto{\pgfqpoint{5.115696in}{0.856164in}}%
\pgfpathlineto{\pgfqpoint{5.120244in}{0.850382in}}%
\pgfpathlineto{\pgfqpoint{5.121698in}{0.848534in}}%
\pgfpathlineto{\pgfqpoint{5.124793in}{0.844600in}}%
\pgfpathlineto{\pgfqpoint{5.129341in}{0.838818in}}%
\pgfpathlineto{\pgfqpoint{5.130003in}{0.837977in}}%
\pgfpathlineto{\pgfqpoint{5.133890in}{0.833036in}}%
\pgfpathlineto{\pgfqpoint{5.138308in}{0.827420in}}%
\pgfpathlineto{\pgfqpoint{5.138439in}{0.827253in}}%
\pgfpathlineto{\pgfqpoint{5.142987in}{0.821471in}}%
\pgfpathlineto{\pgfqpoint{5.146612in}{0.816863in}}%
\pgfpathlineto{\pgfqpoint{5.147536in}{0.815689in}}%
\pgfpathlineto{\pgfqpoint{5.152085in}{0.809907in}}%
\pgfpathlineto{\pgfqpoint{5.154917in}{0.806306in}}%
\pgfpathlineto{\pgfqpoint{5.156633in}{0.804125in}}%
\pgfpathlineto{\pgfqpoint{5.161182in}{0.798342in}}%
\pgfpathlineto{\pgfqpoint{5.163222in}{0.795749in}}%
\pgfpathlineto{\pgfqpoint{5.165730in}{0.792560in}}%
\pgfpathlineto{\pgfqpoint{5.170279in}{0.786778in}}%
\pgfpathlineto{\pgfqpoint{5.171527in}{0.785192in}}%
\pgfpathlineto{\pgfqpoint{5.174828in}{0.780996in}}%
\pgfpathlineto{\pgfqpoint{5.179376in}{0.775214in}}%
\pgfpathlineto{\pgfqpoint{5.179832in}{0.774635in}}%
\pgfpathlineto{\pgfqpoint{5.183925in}{0.769431in}}%
\pgfpathlineto{\pgfqpoint{5.188137in}{0.764077in}}%
\pgfpathlineto{\pgfqpoint{5.188474in}{0.763649in}}%
\pgfpathlineto{\pgfqpoint{5.193022in}{0.757867in}}%
\pgfpathlineto{\pgfqpoint{5.196441in}{0.753520in}}%
\pgfpathlineto{\pgfqpoint{5.197571in}{0.752085in}}%
\pgfpathlineto{\pgfqpoint{5.202119in}{0.746303in}}%
\pgfpathlineto{\pgfqpoint{5.204746in}{0.742963in}}%
\pgfpathlineto{\pgfqpoint{5.206668in}{0.740520in}}%
\pgfpathlineto{\pgfqpoint{5.211217in}{0.734738in}}%
\pgfpathlineto{\pgfqpoint{5.213051in}{0.732406in}}%
\pgfpathlineto{\pgfqpoint{5.215765in}{0.728956in}}%
\pgfpathlineto{\pgfqpoint{5.220314in}{0.723174in}}%
\pgfpathlineto{\pgfqpoint{5.221356in}{0.721849in}}%
\pgfpathlineto{\pgfqpoint{5.224863in}{0.717392in}}%
\pgfpathlineto{\pgfqpoint{5.229411in}{0.711609in}}%
\pgfpathlineto{\pgfqpoint{5.229661in}{0.711292in}}%
\pgfpathlineto{\pgfqpoint{5.233960in}{0.705827in}}%
\pgfpathlineto{\pgfqpoint{5.237966in}{0.700735in}}%
\pgfpathlineto{\pgfqpoint{5.238508in}{0.700045in}}%
\pgfpathlineto{\pgfqpoint{5.243057in}{0.694263in}}%
\pgfpathlineto{\pgfqpoint{5.246270in}{0.690178in}}%
\pgfpathlineto{\pgfqpoint{5.247606in}{0.688481in}}%
\pgfpathlineto{\pgfqpoint{5.254575in}{0.688481in}}%
\pgfpathlineto{\pgfqpoint{5.262880in}{0.688481in}}%
\pgfpathlineto{\pgfqpoint{5.271185in}{0.688481in}}%
\pgfpathlineto{\pgfqpoint{5.279490in}{0.688481in}}%
\pgfpathlineto{\pgfqpoint{5.287795in}{0.688481in}}%
\pgfpathlineto{\pgfqpoint{5.296099in}{0.688481in}}%
\pgfpathlineto{\pgfqpoint{5.304404in}{0.688481in}}%
\pgfpathlineto{\pgfqpoint{5.312709in}{0.688481in}}%
\pgfpathlineto{\pgfqpoint{5.321014in}{0.688481in}}%
\pgfpathlineto{\pgfqpoint{5.329319in}{0.688481in}}%
\pgfpathlineto{\pgfqpoint{5.337624in}{0.688481in}}%
\pgfpathlineto{\pgfqpoint{5.345929in}{0.688481in}}%
\pgfpathlineto{\pgfqpoint{5.354233in}{0.688481in}}%
\pgfpathlineto{\pgfqpoint{5.362538in}{0.688481in}}%
\pgfpathlineto{\pgfqpoint{5.370843in}{0.688481in}}%
\pgfpathlineto{\pgfqpoint{5.379148in}{0.688481in}}%
\pgfpathlineto{\pgfqpoint{5.387453in}{0.688481in}}%
\pgfpathlineto{\pgfqpoint{5.395758in}{0.688481in}}%
\pgfpathlineto{\pgfqpoint{5.404062in}{0.688481in}}%
\pgfpathlineto{\pgfqpoint{5.412367in}{0.688481in}}%
\pgfpathlineto{\pgfqpoint{5.420672in}{0.688481in}}%
\pgfpathlineto{\pgfqpoint{5.428977in}{0.688481in}}%
\pgfpathlineto{\pgfqpoint{5.437282in}{0.688481in}}%
\pgfpathlineto{\pgfqpoint{5.445587in}{0.688481in}}%
\pgfpathlineto{\pgfqpoint{5.453891in}{0.688481in}}%
\pgfpathlineto{\pgfqpoint{5.462196in}{0.688481in}}%
\pgfpathlineto{\pgfqpoint{5.470501in}{0.688481in}}%
\pgfpathlineto{\pgfqpoint{5.478806in}{0.688481in}}%
\pgfpathlineto{\pgfqpoint{5.487111in}{0.688481in}}%
\pgfpathlineto{\pgfqpoint{5.495416in}{0.688481in}}%
\pgfpathlineto{\pgfqpoint{5.503720in}{0.688481in}}%
\pgfpathlineto{\pgfqpoint{5.512025in}{0.688481in}}%
\pgfpathlineto{\pgfqpoint{5.520330in}{0.688481in}}%
\pgfpathlineto{\pgfqpoint{5.528635in}{0.688481in}}%
\pgfpathlineto{\pgfqpoint{5.536940in}{0.688481in}}%
\pgfpathlineto{\pgfqpoint{5.545245in}{0.688481in}}%
\pgfpathlineto{\pgfqpoint{5.553549in}{0.688481in}}%
\pgfpathlineto{\pgfqpoint{5.561854in}{0.688481in}}%
\pgfpathlineto{\pgfqpoint{5.570159in}{0.688481in}}%
\pgfpathlineto{\pgfqpoint{5.578464in}{0.688481in}}%
\pgfpathlineto{\pgfqpoint{5.586769in}{0.688481in}}%
\pgfpathlineto{\pgfqpoint{5.595074in}{0.688481in}}%
\pgfpathlineto{\pgfqpoint{5.603378in}{0.688481in}}%
\pgfpathlineto{\pgfqpoint{5.611683in}{0.688481in}}%
\pgfpathlineto{\pgfqpoint{5.619988in}{0.688481in}}%
\pgfpathlineto{\pgfqpoint{5.628293in}{0.688481in}}%
\pgfpathlineto{\pgfqpoint{5.636598in}{0.688481in}}%
\pgfpathlineto{\pgfqpoint{5.644903in}{0.688481in}}%
\pgfpathlineto{\pgfqpoint{5.653207in}{0.688481in}}%
\pgfpathlineto{\pgfqpoint{5.661512in}{0.688481in}}%
\pgfpathlineto{\pgfqpoint{5.669817in}{0.688481in}}%
\pgfpathlineto{\pgfqpoint{5.678122in}{0.688481in}}%
\pgfpathlineto{\pgfqpoint{5.686427in}{0.688481in}}%
\pgfpathlineto{\pgfqpoint{5.694732in}{0.688481in}}%
\pgfpathlineto{\pgfqpoint{5.703036in}{0.688481in}}%
\pgfpathlineto{\pgfqpoint{5.711341in}{0.688481in}}%
\pgfpathlineto{\pgfqpoint{5.719646in}{0.688481in}}%
\pgfpathlineto{\pgfqpoint{5.727951in}{0.688481in}}%
\pgfpathlineto{\pgfqpoint{5.736256in}{0.688481in}}%
\pgfpathlineto{\pgfqpoint{5.744561in}{0.688481in}}%
\pgfpathlineto{\pgfqpoint{5.752865in}{0.688481in}}%
\pgfpathlineto{\pgfqpoint{5.761170in}{0.688481in}}%
\pgfpathlineto{\pgfqpoint{5.769475in}{0.688481in}}%
\pgfpathlineto{\pgfqpoint{5.777780in}{0.688481in}}%
\pgfpathlineto{\pgfqpoint{5.786085in}{0.688481in}}%
\pgfpathlineto{\pgfqpoint{5.794390in}{0.688481in}}%
\pgfpathlineto{\pgfqpoint{5.802694in}{0.688481in}}%
\pgfpathlineto{\pgfqpoint{5.810999in}{0.688481in}}%
\pgfpathlineto{\pgfqpoint{5.819304in}{0.688481in}}%
\pgfpathlineto{\pgfqpoint{5.827609in}{0.688481in}}%
\pgfpathlineto{\pgfqpoint{5.835914in}{0.688481in}}%
\pgfpathlineto{\pgfqpoint{5.844219in}{0.688481in}}%
\pgfpathlineto{\pgfqpoint{5.852523in}{0.688481in}}%
\pgfpathlineto{\pgfqpoint{5.860828in}{0.688481in}}%
\pgfpathlineto{\pgfqpoint{5.869133in}{0.688481in}}%
\pgfpathlineto{\pgfqpoint{5.877438in}{0.688481in}}%
\pgfpathlineto{\pgfqpoint{5.885743in}{0.688481in}}%
\pgfpathlineto{\pgfqpoint{5.894048in}{0.688481in}}%
\pgfpathlineto{\pgfqpoint{5.902352in}{0.688481in}}%
\pgfpathlineto{\pgfqpoint{5.910657in}{0.688481in}}%
\pgfpathlineto{\pgfqpoint{5.918962in}{0.688481in}}%
\pgfpathlineto{\pgfqpoint{5.927267in}{0.688481in}}%
\pgfpathlineto{\pgfqpoint{5.935572in}{0.688481in}}%
\pgfpathlineto{\pgfqpoint{5.943877in}{0.688481in}}%
\pgfpathlineto{\pgfqpoint{5.952181in}{0.688481in}}%
\pgfpathlineto{\pgfqpoint{5.960486in}{0.688481in}}%
\pgfpathlineto{\pgfqpoint{5.968791in}{0.688481in}}%
\pgfpathlineto{\pgfqpoint{5.977096in}{0.688481in}}%
\pgfpathlineto{\pgfqpoint{5.985401in}{0.688481in}}%
\pgfpathlineto{\pgfqpoint{5.993706in}{0.688481in}}%
\pgfpathlineto{\pgfqpoint{6.002010in}{0.688481in}}%
\pgfpathlineto{\pgfqpoint{6.010315in}{0.688481in}}%
\pgfpathlineto{\pgfqpoint{6.018620in}{0.688481in}}%
\pgfpathlineto{\pgfqpoint{6.026925in}{0.688481in}}%
\pgfpathlineto{\pgfqpoint{6.035230in}{0.688481in}}%
\pgfpathlineto{\pgfqpoint{6.043535in}{0.688481in}}%
\pgfpathlineto{\pgfqpoint{6.051839in}{0.688481in}}%
\pgfpathlineto{\pgfqpoint{6.060144in}{0.688481in}}%
\pgfpathlineto{\pgfqpoint{6.068449in}{0.688481in}}%
\pgfpathlineto{\pgfqpoint{6.076754in}{0.688481in}}%
\pgfpathlineto{\pgfqpoint{6.085059in}{0.688481in}}%
\pgfpathlineto{\pgfqpoint{6.093364in}{0.688481in}}%
\pgfpathlineto{\pgfqpoint{6.101668in}{0.688481in}}%
\pgfpathlineto{\pgfqpoint{6.109973in}{0.688481in}}%
\pgfpathlineto{\pgfqpoint{6.118278in}{0.688481in}}%
\pgfpathlineto{\pgfqpoint{6.126583in}{0.688481in}}%
\pgfpathlineto{\pgfqpoint{6.134888in}{0.688481in}}%
\pgfpathlineto{\pgfqpoint{6.143193in}{0.688481in}}%
\pgfpathlineto{\pgfqpoint{6.151497in}{0.688481in}}%
\pgfpathlineto{\pgfqpoint{6.159802in}{0.688481in}}%
\pgfpathlineto{\pgfqpoint{6.168107in}{0.688481in}}%
\pgfpathlineto{\pgfqpoint{6.176412in}{0.688481in}}%
\pgfpathlineto{\pgfqpoint{6.184717in}{0.688481in}}%
\pgfpathlineto{\pgfqpoint{6.193022in}{0.688481in}}%
\pgfpathlineto{\pgfqpoint{6.201326in}{0.688481in}}%
\pgfpathlineto{\pgfqpoint{6.209631in}{0.688481in}}%
\pgfpathlineto{\pgfqpoint{6.217936in}{0.688481in}}%
\pgfpathlineto{\pgfqpoint{6.226241in}{0.688481in}}%
\pgfpathlineto{\pgfqpoint{6.234546in}{0.688481in}}%
\pgfpathlineto{\pgfqpoint{6.242851in}{0.688481in}}%
\pgfpathlineto{\pgfqpoint{6.251156in}{0.688481in}}%
\pgfpathlineto{\pgfqpoint{6.259460in}{0.688481in}}%
\pgfpathlineto{\pgfqpoint{6.267765in}{0.688481in}}%
\pgfpathlineto{\pgfqpoint{6.276070in}{0.688481in}}%
\pgfpathlineto{\pgfqpoint{6.284375in}{0.688481in}}%
\pgfpathlineto{\pgfqpoint{6.292680in}{0.688481in}}%
\pgfpathlineto{\pgfqpoint{6.300985in}{0.688481in}}%
\pgfpathlineto{\pgfqpoint{6.309289in}{0.688481in}}%
\pgfpathlineto{\pgfqpoint{6.317594in}{0.688481in}}%
\pgfpathlineto{\pgfqpoint{6.325899in}{0.688481in}}%
\pgfpathlineto{\pgfqpoint{6.334204in}{0.688481in}}%
\pgfpathlineto{\pgfqpoint{6.342509in}{0.688481in}}%
\pgfpathlineto{\pgfqpoint{6.350814in}{0.688481in}}%
\pgfpathlineto{\pgfqpoint{6.359118in}{0.688481in}}%
\pgfpathlineto{\pgfqpoint{6.367423in}{0.688481in}}%
\pgfpathlineto{\pgfqpoint{6.375728in}{0.688481in}}%
\pgfpathlineto{\pgfqpoint{6.384033in}{0.688481in}}%
\pgfpathlineto{\pgfqpoint{6.392338in}{0.688481in}}%
\pgfpathlineto{\pgfqpoint{6.400643in}{0.688481in}}%
\pgfpathlineto{\pgfqpoint{6.408947in}{0.688481in}}%
\pgfpathlineto{\pgfqpoint{6.417252in}{0.688481in}}%
\pgfpathlineto{\pgfqpoint{6.425557in}{0.688481in}}%
\pgfpathlineto{\pgfqpoint{6.433862in}{0.688481in}}%
\pgfpathlineto{\pgfqpoint{6.442167in}{0.688481in}}%
\pgfpathlineto{\pgfqpoint{6.450472in}{0.688481in}}%
\pgfpathlineto{\pgfqpoint{6.458776in}{0.688481in}}%
\pgfpathlineto{\pgfqpoint{6.467081in}{0.688481in}}%
\pgfpathlineto{\pgfqpoint{6.475386in}{0.688481in}}%
\pgfpathlineto{\pgfqpoint{6.483691in}{0.688481in}}%
\pgfpathlineto{\pgfqpoint{6.491996in}{0.688481in}}%
\pgfpathlineto{\pgfqpoint{6.500301in}{0.688481in}}%
\pgfpathlineto{\pgfqpoint{6.508605in}{0.688481in}}%
\pgfpathlineto{\pgfqpoint{6.516910in}{0.688481in}}%
\pgfpathlineto{\pgfqpoint{6.525215in}{0.688481in}}%
\pgfpathlineto{\pgfqpoint{6.533520in}{0.688481in}}%
\pgfpathlineto{\pgfqpoint{6.541825in}{0.688481in}}%
\pgfpathlineto{\pgfqpoint{6.550130in}{0.688481in}}%
\pgfpathlineto{\pgfqpoint{6.558434in}{0.688481in}}%
\pgfpathlineto{\pgfqpoint{6.566739in}{0.688481in}}%
\pgfpathlineto{\pgfqpoint{6.575044in}{0.688481in}}%
\pgfpathlineto{\pgfqpoint{6.583349in}{0.688481in}}%
\pgfpathlineto{\pgfqpoint{6.591654in}{0.688481in}}%
\pgfpathlineto{\pgfqpoint{6.599959in}{0.688481in}}%
\pgfpathlineto{\pgfqpoint{6.608263in}{0.688481in}}%
\pgfpathlineto{\pgfqpoint{6.616568in}{0.688481in}}%
\pgfpathlineto{\pgfqpoint{6.624873in}{0.688481in}}%
\pgfpathlineto{\pgfqpoint{6.633178in}{0.688481in}}%
\pgfpathlineto{\pgfqpoint{6.641483in}{0.688481in}}%
\pgfpathlineto{\pgfqpoint{6.649788in}{0.688481in}}%
\pgfpathlineto{\pgfqpoint{6.658092in}{0.688481in}}%
\pgfpathlineto{\pgfqpoint{6.666397in}{0.688481in}}%
\pgfpathlineto{\pgfqpoint{6.674702in}{0.688481in}}%
\pgfpathlineto{\pgfqpoint{6.683007in}{0.688481in}}%
\pgfpathlineto{\pgfqpoint{6.691312in}{0.688481in}}%
\pgfpathlineto{\pgfqpoint{6.699617in}{0.688481in}}%
\pgfpathlineto{\pgfqpoint{6.707921in}{0.688481in}}%
\pgfpathlineto{\pgfqpoint{6.716226in}{0.688481in}}%
\pgfpathlineto{\pgfqpoint{6.724531in}{0.688481in}}%
\pgfpathlineto{\pgfqpoint{6.732836in}{0.688481in}}%
\pgfpathlineto{\pgfqpoint{6.741141in}{0.688481in}}%
\pgfpathlineto{\pgfqpoint{6.749446in}{0.688481in}}%
\pgfpathlineto{\pgfqpoint{6.757750in}{0.688481in}}%
\pgfpathlineto{\pgfqpoint{6.766055in}{0.688481in}}%
\pgfpathlineto{\pgfqpoint{6.774360in}{0.688481in}}%
\pgfpathlineto{\pgfqpoint{6.782665in}{0.688481in}}%
\pgfpathlineto{\pgfqpoint{6.790970in}{0.688481in}}%
\pgfpathlineto{\pgfqpoint{6.799275in}{0.688481in}}%
\pgfpathlineto{\pgfqpoint{6.807579in}{0.688481in}}%
\pgfpathlineto{\pgfqpoint{6.815884in}{0.688481in}}%
\pgfpathlineto{\pgfqpoint{6.824189in}{0.688481in}}%
\pgfpathlineto{\pgfqpoint{6.832494in}{0.688481in}}%
\pgfpathlineto{\pgfqpoint{6.840799in}{0.688481in}}%
\pgfpathlineto{\pgfqpoint{6.849104in}{0.688481in}}%
\pgfpathlineto{\pgfqpoint{6.857408in}{0.688481in}}%
\pgfpathlineto{\pgfqpoint{6.865713in}{0.688481in}}%
\pgfpathlineto{\pgfqpoint{6.874018in}{0.688481in}}%
\pgfpathlineto{\pgfqpoint{6.882323in}{0.688481in}}%
\pgfpathlineto{\pgfqpoint{6.890628in}{0.688481in}}%
\pgfpathlineto{\pgfqpoint{6.898933in}{0.688481in}}%
\pgfpathlineto{\pgfqpoint{6.907237in}{0.688481in}}%
\pgfpathlineto{\pgfqpoint{6.915542in}{0.688481in}}%
\pgfpathlineto{\pgfqpoint{6.923847in}{0.688481in}}%
\pgfpathlineto{\pgfqpoint{6.932152in}{0.688481in}}%
\pgfpathlineto{\pgfqpoint{6.940457in}{0.688481in}}%
\pgfpathlineto{\pgfqpoint{6.948762in}{0.688481in}}%
\pgfpathlineto{\pgfqpoint{6.957066in}{0.688481in}}%
\pgfpathlineto{\pgfqpoint{6.965371in}{0.688481in}}%
\pgfpathlineto{\pgfqpoint{6.973676in}{0.688481in}}%
\pgfpathlineto{\pgfqpoint{6.981981in}{0.688481in}}%
\pgfpathlineto{\pgfqpoint{6.990286in}{0.688481in}}%
\pgfpathlineto{\pgfqpoint{6.998591in}{0.688481in}}%
\pgfpathlineto{\pgfqpoint{7.006895in}{0.688481in}}%
\pgfpathlineto{\pgfqpoint{7.015200in}{0.688481in}}%
\pgfpathlineto{\pgfqpoint{7.023505in}{0.688481in}}%
\pgfpathlineto{\pgfqpoint{7.031810in}{0.688481in}}%
\pgfpathlineto{\pgfqpoint{7.040115in}{0.688481in}}%
\pgfpathlineto{\pgfqpoint{7.048420in}{0.688481in}}%
\pgfpathlineto{\pgfqpoint{7.056724in}{0.688481in}}%
\pgfpathlineto{\pgfqpoint{7.065029in}{0.688481in}}%
\pgfpathlineto{\pgfqpoint{7.073334in}{0.688481in}}%
\pgfpathlineto{\pgfqpoint{7.081639in}{0.688481in}}%
\pgfpathlineto{\pgfqpoint{7.089944in}{0.688481in}}%
\pgfpathlineto{\pgfqpoint{7.098249in}{0.688481in}}%
\pgfpathlineto{\pgfqpoint{7.106553in}{0.688481in}}%
\pgfpathlineto{\pgfqpoint{7.114858in}{0.688481in}}%
\pgfpathlineto{\pgfqpoint{7.123163in}{0.688481in}}%
\pgfpathlineto{\pgfqpoint{7.131468in}{0.688481in}}%
\pgfpathlineto{\pgfqpoint{7.139773in}{0.688481in}}%
\pgfpathlineto{\pgfqpoint{7.148078in}{0.688481in}}%
\pgfpathlineto{\pgfqpoint{7.156382in}{0.688481in}}%
\pgfpathlineto{\pgfqpoint{7.164687in}{0.688481in}}%
\pgfpathlineto{\pgfqpoint{7.172992in}{0.688481in}}%
\pgfpathlineto{\pgfqpoint{7.181297in}{0.688481in}}%
\pgfpathlineto{\pgfqpoint{7.189602in}{0.688481in}}%
\pgfpathlineto{\pgfqpoint{7.197907in}{0.688481in}}%
\pgfpathlineto{\pgfqpoint{7.206212in}{0.688481in}}%
\pgfpathlineto{\pgfqpoint{7.214516in}{0.688481in}}%
\pgfpathlineto{\pgfqpoint{7.222821in}{0.688481in}}%
\pgfpathlineto{\pgfqpoint{7.231126in}{0.688481in}}%
\pgfpathlineto{\pgfqpoint{7.239431in}{0.688481in}}%
\pgfpathlineto{\pgfqpoint{7.247736in}{0.688481in}}%
\pgfpathlineto{\pgfqpoint{7.256041in}{0.688481in}}%
\pgfpathlineto{\pgfqpoint{7.264345in}{0.688481in}}%
\pgfpathlineto{\pgfqpoint{7.272650in}{0.688481in}}%
\pgfpathlineto{\pgfqpoint{7.280955in}{0.688481in}}%
\pgfpathlineto{\pgfqpoint{7.289260in}{0.688481in}}%
\pgfpathlineto{\pgfqpoint{7.297565in}{0.688481in}}%
\pgfpathlineto{\pgfqpoint{7.305870in}{0.688481in}}%
\pgfpathlineto{\pgfqpoint{7.314174in}{0.688481in}}%
\pgfpathlineto{\pgfqpoint{7.322479in}{0.688481in}}%
\pgfpathlineto{\pgfqpoint{7.330784in}{0.688481in}}%
\pgfpathlineto{\pgfqpoint{7.339089in}{0.688481in}}%
\pgfpathlineto{\pgfqpoint{7.347394in}{0.688481in}}%
\pgfpathlineto{\pgfqpoint{7.355699in}{0.688481in}}%
\pgfpathlineto{\pgfqpoint{7.364003in}{0.688481in}}%
\pgfpathlineto{\pgfqpoint{7.372308in}{0.688481in}}%
\pgfpathlineto{\pgfqpoint{7.380613in}{0.688481in}}%
\pgfpathlineto{\pgfqpoint{7.388918in}{0.688481in}}%
\pgfpathlineto{\pgfqpoint{7.397223in}{0.688481in}}%
\pgfpathlineto{\pgfqpoint{7.405528in}{0.688481in}}%
\pgfpathlineto{\pgfqpoint{7.413832in}{0.688481in}}%
\pgfpathlineto{\pgfqpoint{7.422137in}{0.688481in}}%
\pgfpathlineto{\pgfqpoint{7.430442in}{0.688481in}}%
\pgfpathlineto{\pgfqpoint{7.438747in}{0.688481in}}%
\pgfpathlineto{\pgfqpoint{7.447052in}{0.688481in}}%
\pgfpathlineto{\pgfqpoint{7.455357in}{0.688481in}}%
\pgfpathlineto{\pgfqpoint{7.463661in}{0.688481in}}%
\pgfpathlineto{\pgfqpoint{7.471966in}{0.688481in}}%
\pgfpathlineto{\pgfqpoint{7.480271in}{0.688481in}}%
\pgfpathlineto{\pgfqpoint{7.488576in}{0.688481in}}%
\pgfpathlineto{\pgfqpoint{7.496881in}{0.688481in}}%
\pgfpathlineto{\pgfqpoint{7.505186in}{0.688481in}}%
\pgfpathlineto{\pgfqpoint{7.513490in}{0.688481in}}%
\pgfpathlineto{\pgfqpoint{7.521795in}{0.688481in}}%
\pgfpathlineto{\pgfqpoint{7.530100in}{0.688481in}}%
\pgfpathlineto{\pgfqpoint{7.538405in}{0.688481in}}%
\pgfpathlineto{\pgfqpoint{7.546710in}{0.688481in}}%
\pgfpathlineto{\pgfqpoint{7.555015in}{0.688481in}}%
\pgfpathlineto{\pgfqpoint{7.563319in}{0.688481in}}%
\pgfpathlineto{\pgfqpoint{7.571624in}{0.688481in}}%
\pgfpathlineto{\pgfqpoint{7.579929in}{0.688481in}}%
\pgfpathlineto{\pgfqpoint{7.588234in}{0.688481in}}%
\pgfpathlineto{\pgfqpoint{7.596539in}{0.688481in}}%
\pgfpathlineto{\pgfqpoint{7.604844in}{0.688481in}}%
\pgfpathlineto{\pgfqpoint{7.613148in}{0.688481in}}%
\pgfpathlineto{\pgfqpoint{7.621453in}{0.688481in}}%
\pgfpathlineto{\pgfqpoint{7.629758in}{0.688481in}}%
\pgfpathlineto{\pgfqpoint{7.638063in}{0.688481in}}%
\pgfpathlineto{\pgfqpoint{7.646368in}{0.688481in}}%
\pgfpathlineto{\pgfqpoint{7.654673in}{0.688481in}}%
\pgfpathlineto{\pgfqpoint{7.662977in}{0.688481in}}%
\pgfpathlineto{\pgfqpoint{7.671282in}{0.688481in}}%
\pgfpathlineto{\pgfqpoint{7.679587in}{0.688481in}}%
\pgfpathlineto{\pgfqpoint{7.687892in}{0.688481in}}%
\pgfpathlineto{\pgfqpoint{7.696197in}{0.688481in}}%
\pgfpathlineto{\pgfqpoint{7.704502in}{0.688481in}}%
\pgfpathlineto{\pgfqpoint{7.712806in}{0.688481in}}%
\pgfpathlineto{\pgfqpoint{7.721111in}{0.688481in}}%
\pgfpathlineto{\pgfqpoint{7.729416in}{0.688481in}}%
\pgfpathlineto{\pgfqpoint{7.737721in}{0.688481in}}%
\pgfpathlineto{\pgfqpoint{7.746026in}{0.688481in}}%
\pgfpathlineto{\pgfqpoint{7.754331in}{0.688481in}}%
\pgfpathlineto{\pgfqpoint{7.762635in}{0.688481in}}%
\pgfpathlineto{\pgfqpoint{7.770940in}{0.688481in}}%
\pgfpathlineto{\pgfqpoint{7.779245in}{0.688481in}}%
\pgfpathlineto{\pgfqpoint{7.787550in}{0.688481in}}%
\pgfpathlineto{\pgfqpoint{7.795855in}{0.688481in}}%
\pgfpathlineto{\pgfqpoint{7.804160in}{0.688481in}}%
\pgfpathlineto{\pgfqpoint{7.812464in}{0.688481in}}%
\pgfpathlineto{\pgfqpoint{7.820769in}{0.688481in}}%
\pgfpathlineto{\pgfqpoint{7.829074in}{0.688481in}}%
\pgfpathlineto{\pgfqpoint{7.837379in}{0.688481in}}%
\pgfpathlineto{\pgfqpoint{7.845684in}{0.688481in}}%
\pgfpathlineto{\pgfqpoint{7.853989in}{0.688481in}}%
\pgfpathlineto{\pgfqpoint{7.862293in}{0.688481in}}%
\pgfpathlineto{\pgfqpoint{7.870598in}{0.688481in}}%
\pgfpathlineto{\pgfqpoint{7.878903in}{0.688481in}}%
\pgfpathlineto{\pgfqpoint{7.887208in}{0.688481in}}%
\pgfpathlineto{\pgfqpoint{7.895513in}{0.688481in}}%
\pgfpathlineto{\pgfqpoint{7.903818in}{0.688481in}}%
\pgfpathlineto{\pgfqpoint{7.912122in}{0.688481in}}%
\pgfpathlineto{\pgfqpoint{7.920427in}{0.688481in}}%
\pgfpathlineto{\pgfqpoint{7.928732in}{0.688481in}}%
\pgfpathlineto{\pgfqpoint{7.937037in}{0.688481in}}%
\pgfpathlineto{\pgfqpoint{7.945342in}{0.688481in}}%
\pgfpathlineto{\pgfqpoint{7.953647in}{0.688481in}}%
\pgfpathlineto{\pgfqpoint{7.961951in}{0.688481in}}%
\pgfpathlineto{\pgfqpoint{7.970256in}{0.688481in}}%
\pgfpathlineto{\pgfqpoint{7.978561in}{0.688481in}}%
\pgfpathlineto{\pgfqpoint{7.986866in}{0.688481in}}%
\pgfpathlineto{\pgfqpoint{7.995171in}{0.688481in}}%
\pgfpathlineto{\pgfqpoint{8.003476in}{0.688481in}}%
\pgfpathlineto{\pgfqpoint{8.011780in}{0.688481in}}%
\pgfpathlineto{\pgfqpoint{8.020085in}{0.688481in}}%
\pgfpathlineto{\pgfqpoint{8.028390in}{0.688481in}}%
\pgfpathlineto{\pgfqpoint{8.036695in}{0.688481in}}%
\pgfpathlineto{\pgfqpoint{8.045000in}{0.688481in}}%
\pgfpathlineto{\pgfqpoint{8.053305in}{0.688481in}}%
\pgfpathlineto{\pgfqpoint{8.061609in}{0.688481in}}%
\pgfpathlineto{\pgfqpoint{8.069914in}{0.688481in}}%
\pgfpathlineto{\pgfqpoint{8.078219in}{0.688481in}}%
\pgfpathlineto{\pgfqpoint{8.086524in}{0.688481in}}%
\pgfpathlineto{\pgfqpoint{8.094829in}{0.688481in}}%
\pgfpathlineto{\pgfqpoint{8.103134in}{0.688481in}}%
\pgfpathlineto{\pgfqpoint{8.111439in}{0.688481in}}%
\pgfpathlineto{\pgfqpoint{8.119743in}{0.688481in}}%
\pgfpathlineto{\pgfqpoint{8.128048in}{0.688481in}}%
\pgfpathlineto{\pgfqpoint{8.136353in}{0.688481in}}%
\pgfpathlineto{\pgfqpoint{8.144658in}{0.688481in}}%
\pgfpathlineto{\pgfqpoint{8.152963in}{0.688481in}}%
\pgfpathlineto{\pgfqpoint{8.161268in}{0.688481in}}%
\pgfpathlineto{\pgfqpoint{8.169572in}{0.688481in}}%
\pgfpathlineto{\pgfqpoint{8.177877in}{0.688481in}}%
\pgfpathlineto{\pgfqpoint{8.186182in}{0.688481in}}%
\pgfpathlineto{\pgfqpoint{8.194487in}{0.688481in}}%
\pgfpathlineto{\pgfqpoint{8.202792in}{0.688481in}}%
\pgfpathlineto{\pgfqpoint{8.211097in}{0.688481in}}%
\pgfpathlineto{\pgfqpoint{8.219401in}{0.688481in}}%
\pgfpathlineto{\pgfqpoint{8.227706in}{0.688481in}}%
\pgfpathlineto{\pgfqpoint{8.236011in}{0.688481in}}%
\pgfpathlineto{\pgfqpoint{8.244316in}{0.688481in}}%
\pgfpathlineto{\pgfqpoint{8.252621in}{0.688481in}}%
\pgfpathlineto{\pgfqpoint{8.260926in}{0.688481in}}%
\pgfpathlineto{\pgfqpoint{8.269230in}{0.688481in}}%
\pgfpathlineto{\pgfqpoint{8.277535in}{0.688481in}}%
\pgfpathlineto{\pgfqpoint{8.285840in}{0.688481in}}%
\pgfpathlineto{\pgfqpoint{8.294145in}{0.688481in}}%
\pgfpathlineto{\pgfqpoint{8.302450in}{0.688481in}}%
\pgfpathlineto{\pgfqpoint{8.310755in}{0.688481in}}%
\pgfpathlineto{\pgfqpoint{8.319059in}{0.688481in}}%
\pgfpathlineto{\pgfqpoint{8.327364in}{0.688481in}}%
\pgfpathlineto{\pgfqpoint{8.335669in}{0.688481in}}%
\pgfpathlineto{\pgfqpoint{8.343974in}{0.688481in}}%
\pgfpathlineto{\pgfqpoint{8.352279in}{0.688481in}}%
\pgfpathlineto{\pgfqpoint{8.360584in}{0.688481in}}%
\pgfpathlineto{\pgfqpoint{8.368888in}{0.688481in}}%
\pgfpathlineto{\pgfqpoint{8.377193in}{0.688481in}}%
\pgfpathlineto{\pgfqpoint{8.385498in}{0.688481in}}%
\pgfpathlineto{\pgfqpoint{8.393803in}{0.688481in}}%
\pgfpathlineto{\pgfqpoint{8.402108in}{0.688481in}}%
\pgfpathlineto{\pgfqpoint{8.410413in}{0.688481in}}%
\pgfpathlineto{\pgfqpoint{8.418717in}{0.688481in}}%
\pgfpathlineto{\pgfqpoint{8.427022in}{0.688481in}}%
\pgfpathlineto{\pgfqpoint{8.435327in}{0.688481in}}%
\pgfpathlineto{\pgfqpoint{8.443632in}{0.688481in}}%
\pgfpathlineto{\pgfqpoint{8.451937in}{0.688481in}}%
\pgfpathlineto{\pgfqpoint{8.460242in}{0.688481in}}%
\pgfpathlineto{\pgfqpoint{8.468546in}{0.688481in}}%
\pgfpathlineto{\pgfqpoint{8.476851in}{0.688481in}}%
\pgfpathlineto{\pgfqpoint{8.485156in}{0.688481in}}%
\pgfpathlineto{\pgfqpoint{8.493461in}{0.688481in}}%
\pgfpathlineto{\pgfqpoint{8.501766in}{0.688481in}}%
\pgfpathlineto{\pgfqpoint{8.510071in}{0.688481in}}%
\pgfpathlineto{\pgfqpoint{8.518375in}{0.688481in}}%
\pgfpathlineto{\pgfqpoint{8.526680in}{0.688481in}}%
\pgfpathlineto{\pgfqpoint{8.534985in}{0.688481in}}%
\pgfpathlineto{\pgfqpoint{8.543290in}{0.688481in}}%
\pgfpathlineto{\pgfqpoint{8.551595in}{0.688481in}}%
\pgfpathlineto{\pgfqpoint{8.559900in}{0.688481in}}%
\pgfpathlineto{\pgfqpoint{8.568204in}{0.688481in}}%
\pgfpathlineto{\pgfqpoint{8.576509in}{0.688481in}}%
\pgfpathlineto{\pgfqpoint{8.584814in}{0.688481in}}%
\pgfpathlineto{\pgfqpoint{8.593119in}{0.688481in}}%
\pgfpathlineto{\pgfqpoint{8.601424in}{0.688481in}}%
\pgfpathlineto{\pgfqpoint{8.609729in}{0.688481in}}%
\pgfpathlineto{\pgfqpoint{8.618033in}{0.688481in}}%
\pgfpathlineto{\pgfqpoint{8.626338in}{0.688481in}}%
\pgfpathlineto{\pgfqpoint{8.634643in}{0.688481in}}%
\pgfpathlineto{\pgfqpoint{8.642948in}{0.688481in}}%
\pgfpathlineto{\pgfqpoint{8.651253in}{0.688481in}}%
\pgfpathlineto{\pgfqpoint{8.659558in}{0.688481in}}%
\pgfpathlineto{\pgfqpoint{8.667862in}{0.688481in}}%
\pgfpathlineto{\pgfqpoint{8.676167in}{0.688481in}}%
\pgfpathlineto{\pgfqpoint{8.684472in}{0.688481in}}%
\pgfpathlineto{\pgfqpoint{8.692777in}{0.688481in}}%
\pgfpathlineto{\pgfqpoint{8.701082in}{0.688481in}}%
\pgfpathlineto{\pgfqpoint{8.709387in}{0.688481in}}%
\pgfpathlineto{\pgfqpoint{8.717691in}{0.688481in}}%
\pgfpathlineto{\pgfqpoint{8.725996in}{0.688481in}}%
\pgfpathlineto{\pgfqpoint{8.734301in}{0.688481in}}%
\pgfpathlineto{\pgfqpoint{8.742606in}{0.688481in}}%
\pgfpathlineto{\pgfqpoint{8.750911in}{0.688481in}}%
\pgfpathlineto{\pgfqpoint{8.759216in}{0.688481in}}%
\pgfpathlineto{\pgfqpoint{8.767520in}{0.688481in}}%
\pgfpathlineto{\pgfqpoint{8.775825in}{0.688481in}}%
\pgfpathlineto{\pgfqpoint{8.784130in}{0.688481in}}%
\pgfpathlineto{\pgfqpoint{8.792435in}{0.688481in}}%
\pgfpathlineto{\pgfqpoint{8.800740in}{0.688481in}}%
\pgfpathlineto{\pgfqpoint{8.809045in}{0.688481in}}%
\pgfpathlineto{\pgfqpoint{8.817349in}{0.688481in}}%
\pgfpathlineto{\pgfqpoint{8.825654in}{0.688481in}}%
\pgfpathlineto{\pgfqpoint{8.833959in}{0.688481in}}%
\pgfpathlineto{\pgfqpoint{8.842264in}{0.688481in}}%
\pgfpathlineto{\pgfqpoint{8.850569in}{0.688481in}}%
\pgfpathlineto{\pgfqpoint{8.858874in}{0.688481in}}%
\pgfpathlineto{\pgfqpoint{8.867178in}{0.688481in}}%
\pgfpathlineto{\pgfqpoint{8.875483in}{0.688481in}}%
\pgfpathlineto{\pgfqpoint{8.883788in}{0.688481in}}%
\pgfpathlineto{\pgfqpoint{8.892093in}{0.688481in}}%
\pgfpathlineto{\pgfqpoint{8.900398in}{0.688481in}}%
\pgfpathlineto{\pgfqpoint{8.908703in}{0.688481in}}%
\pgfpathlineto{\pgfqpoint{8.917007in}{0.688481in}}%
\pgfpathlineto{\pgfqpoint{8.925312in}{0.688481in}}%
\pgfpathlineto{\pgfqpoint{8.933617in}{0.688481in}}%
\pgfpathlineto{\pgfqpoint{8.941922in}{0.688481in}}%
\pgfpathlineto{\pgfqpoint{8.950227in}{0.688481in}}%
\pgfpathlineto{\pgfqpoint{8.958532in}{0.688481in}}%
\pgfpathlineto{\pgfqpoint{8.966836in}{0.688481in}}%
\pgfpathlineto{\pgfqpoint{8.975141in}{0.688481in}}%
\pgfpathlineto{\pgfqpoint{8.983446in}{0.688481in}}%
\pgfpathlineto{\pgfqpoint{8.991751in}{0.688481in}}%
\pgfpathlineto{\pgfqpoint{9.000056in}{0.688481in}}%
\pgfpathlineto{\pgfqpoint{9.000056in}{0.688481in}}%
\pgfpathlineto{\pgfqpoint{9.000056in}{0.688481in}}%
\pgfpathlineto{\pgfqpoint{8.991751in}{0.699038in}}%
\pgfpathlineto{\pgfqpoint{8.983446in}{0.709595in}}%
\pgfpathlineto{\pgfqpoint{8.975141in}{0.720152in}}%
\pgfpathlineto{\pgfqpoint{8.966836in}{0.730709in}}%
\pgfpathlineto{\pgfqpoint{8.958532in}{0.741266in}}%
\pgfpathlineto{\pgfqpoint{8.950227in}{0.751823in}}%
\pgfpathlineto{\pgfqpoint{8.941922in}{0.762380in}}%
\pgfpathlineto{\pgfqpoint{8.933617in}{0.772937in}}%
\pgfpathlineto{\pgfqpoint{8.925312in}{0.783494in}}%
\pgfpathlineto{\pgfqpoint{8.917007in}{0.794051in}}%
\pgfpathlineto{\pgfqpoint{8.908703in}{0.804608in}}%
\pgfpathlineto{\pgfqpoint{8.900398in}{0.815165in}}%
\pgfpathlineto{\pgfqpoint{8.892093in}{0.825723in}}%
\pgfpathlineto{\pgfqpoint{8.883788in}{0.836280in}}%
\pgfpathlineto{\pgfqpoint{8.875483in}{0.846837in}}%
\pgfpathlineto{\pgfqpoint{8.867178in}{0.857394in}}%
\pgfpathlineto{\pgfqpoint{8.858874in}{0.867951in}}%
\pgfpathlineto{\pgfqpoint{8.850569in}{0.878508in}}%
\pgfpathlineto{\pgfqpoint{8.842264in}{0.889065in}}%
\pgfpathlineto{\pgfqpoint{8.833959in}{0.899622in}}%
\pgfpathlineto{\pgfqpoint{8.825654in}{0.910179in}}%
\pgfpathlineto{\pgfqpoint{8.817349in}{0.920736in}}%
\pgfpathlineto{\pgfqpoint{8.809045in}{0.931293in}}%
\pgfpathlineto{\pgfqpoint{8.800740in}{0.941850in}}%
\pgfpathlineto{\pgfqpoint{8.792435in}{0.952407in}}%
\pgfpathlineto{\pgfqpoint{8.784130in}{0.962965in}}%
\pgfpathlineto{\pgfqpoint{8.775825in}{0.973522in}}%
\pgfpathlineto{\pgfqpoint{8.767520in}{0.984079in}}%
\pgfpathlineto{\pgfqpoint{8.759216in}{0.994636in}}%
\pgfpathlineto{\pgfqpoint{8.750911in}{1.005193in}}%
\pgfpathlineto{\pgfqpoint{8.742606in}{1.015750in}}%
\pgfpathlineto{\pgfqpoint{8.734301in}{1.026307in}}%
\pgfpathlineto{\pgfqpoint{8.725996in}{1.036864in}}%
\pgfpathlineto{\pgfqpoint{8.717691in}{1.047421in}}%
\pgfpathlineto{\pgfqpoint{8.709387in}{1.057978in}}%
\pgfpathlineto{\pgfqpoint{8.701082in}{1.068535in}}%
\pgfpathlineto{\pgfqpoint{8.692777in}{1.079092in}}%
\pgfpathlineto{\pgfqpoint{8.684472in}{1.089649in}}%
\pgfpathlineto{\pgfqpoint{8.676167in}{1.100207in}}%
\pgfpathlineto{\pgfqpoint{8.667862in}{1.110764in}}%
\pgfpathlineto{\pgfqpoint{8.659558in}{1.121321in}}%
\pgfpathlineto{\pgfqpoint{8.651253in}{1.131878in}}%
\pgfpathlineto{\pgfqpoint{8.642948in}{1.142435in}}%
\pgfpathlineto{\pgfqpoint{8.634643in}{1.152992in}}%
\pgfpathlineto{\pgfqpoint{8.626338in}{1.163549in}}%
\pgfpathlineto{\pgfqpoint{8.618033in}{1.174106in}}%
\pgfpathlineto{\pgfqpoint{8.609729in}{1.184663in}}%
\pgfpathlineto{\pgfqpoint{8.601424in}{1.195220in}}%
\pgfpathlineto{\pgfqpoint{8.593119in}{1.205777in}}%
\pgfpathlineto{\pgfqpoint{8.584814in}{1.216334in}}%
\pgfpathlineto{\pgfqpoint{8.576509in}{1.226891in}}%
\pgfpathlineto{\pgfqpoint{8.568204in}{1.237449in}}%
\pgfpathlineto{\pgfqpoint{8.559900in}{1.248006in}}%
\pgfpathlineto{\pgfqpoint{8.551595in}{1.258563in}}%
\pgfpathlineto{\pgfqpoint{8.543290in}{1.269120in}}%
\pgfpathlineto{\pgfqpoint{8.534985in}{1.279677in}}%
\pgfpathlineto{\pgfqpoint{8.526680in}{1.290234in}}%
\pgfpathlineto{\pgfqpoint{8.518375in}{1.300791in}}%
\pgfpathlineto{\pgfqpoint{8.510071in}{1.311348in}}%
\pgfpathlineto{\pgfqpoint{8.501766in}{1.321905in}}%
\pgfpathlineto{\pgfqpoint{8.493461in}{1.332462in}}%
\pgfpathlineto{\pgfqpoint{8.485156in}{1.343019in}}%
\pgfpathlineto{\pgfqpoint{8.476851in}{1.353576in}}%
\pgfpathlineto{\pgfqpoint{8.468546in}{1.364133in}}%
\pgfpathlineto{\pgfqpoint{8.460242in}{1.374691in}}%
\pgfpathlineto{\pgfqpoint{8.451937in}{1.385248in}}%
\pgfpathlineto{\pgfqpoint{8.443632in}{1.395805in}}%
\pgfpathlineto{\pgfqpoint{8.435327in}{1.406362in}}%
\pgfpathlineto{\pgfqpoint{8.427022in}{1.416919in}}%
\pgfpathlineto{\pgfqpoint{8.418717in}{1.427476in}}%
\pgfpathlineto{\pgfqpoint{8.410413in}{1.438033in}}%
\pgfpathlineto{\pgfqpoint{8.402108in}{1.448590in}}%
\pgfpathlineto{\pgfqpoint{8.393803in}{1.459147in}}%
\pgfpathlineto{\pgfqpoint{8.385498in}{1.469704in}}%
\pgfpathlineto{\pgfqpoint{8.377193in}{1.480261in}}%
\pgfpathlineto{\pgfqpoint{8.368888in}{1.490818in}}%
\pgfpathlineto{\pgfqpoint{8.360584in}{1.501375in}}%
\pgfpathlineto{\pgfqpoint{8.352279in}{1.511933in}}%
\pgfpathlineto{\pgfqpoint{8.343974in}{1.522490in}}%
\pgfpathlineto{\pgfqpoint{8.335669in}{1.533047in}}%
\pgfpathlineto{\pgfqpoint{8.327364in}{1.543604in}}%
\pgfpathlineto{\pgfqpoint{8.319059in}{1.554161in}}%
\pgfpathlineto{\pgfqpoint{8.310755in}{1.564718in}}%
\pgfpathlineto{\pgfqpoint{8.302450in}{1.575275in}}%
\pgfpathlineto{\pgfqpoint{8.294145in}{1.585832in}}%
\pgfpathlineto{\pgfqpoint{8.285840in}{1.596389in}}%
\pgfpathlineto{\pgfqpoint{8.277535in}{1.606946in}}%
\pgfpathlineto{\pgfqpoint{8.269230in}{1.617503in}}%
\pgfpathlineto{\pgfqpoint{8.260926in}{1.628060in}}%
\pgfpathlineto{\pgfqpoint{8.252621in}{1.638617in}}%
\pgfpathlineto{\pgfqpoint{8.244316in}{1.649175in}}%
\pgfpathlineto{\pgfqpoint{8.236011in}{1.659732in}}%
\pgfpathlineto{\pgfqpoint{8.227706in}{1.670289in}}%
\pgfpathlineto{\pgfqpoint{8.219401in}{1.680846in}}%
\pgfpathlineto{\pgfqpoint{8.211097in}{1.691403in}}%
\pgfpathlineto{\pgfqpoint{8.202792in}{1.701960in}}%
\pgfpathlineto{\pgfqpoint{8.194487in}{1.712517in}}%
\pgfpathlineto{\pgfqpoint{8.186182in}{1.723074in}}%
\pgfpathlineto{\pgfqpoint{8.177877in}{1.733631in}}%
\pgfpathlineto{\pgfqpoint{8.169572in}{1.744188in}}%
\pgfpathlineto{\pgfqpoint{8.161268in}{1.754745in}}%
\pgfpathlineto{\pgfqpoint{8.152963in}{1.765302in}}%
\pgfpathlineto{\pgfqpoint{8.144658in}{1.775859in}}%
\pgfpathlineto{\pgfqpoint{8.136353in}{1.786417in}}%
\pgfpathlineto{\pgfqpoint{8.128048in}{1.796974in}}%
\pgfpathlineto{\pgfqpoint{8.119743in}{1.807531in}}%
\pgfpathlineto{\pgfqpoint{8.111439in}{1.818088in}}%
\pgfpathlineto{\pgfqpoint{8.103134in}{1.828645in}}%
\pgfpathlineto{\pgfqpoint{8.094829in}{1.839202in}}%
\pgfpathlineto{\pgfqpoint{8.086524in}{1.849759in}}%
\pgfpathlineto{\pgfqpoint{8.078219in}{1.860316in}}%
\pgfpathlineto{\pgfqpoint{8.069914in}{1.870873in}}%
\pgfpathlineto{\pgfqpoint{8.061609in}{1.881430in}}%
\pgfpathlineto{\pgfqpoint{8.053305in}{1.891987in}}%
\pgfpathlineto{\pgfqpoint{8.045000in}{1.902544in}}%
\pgfpathlineto{\pgfqpoint{8.036695in}{1.913101in}}%
\pgfpathlineto{\pgfqpoint{8.028390in}{1.923659in}}%
\pgfpathlineto{\pgfqpoint{8.020085in}{1.934216in}}%
\pgfpathlineto{\pgfqpoint{8.011780in}{1.944773in}}%
\pgfpathlineto{\pgfqpoint{8.003476in}{1.955330in}}%
\pgfpathlineto{\pgfqpoint{7.995171in}{1.965887in}}%
\pgfpathlineto{\pgfqpoint{7.986866in}{1.976444in}}%
\pgfpathlineto{\pgfqpoint{7.978561in}{1.987001in}}%
\pgfpathlineto{\pgfqpoint{7.970256in}{1.997558in}}%
\pgfpathlineto{\pgfqpoint{7.961951in}{2.008115in}}%
\pgfpathlineto{\pgfqpoint{7.953647in}{2.018672in}}%
\pgfpathlineto{\pgfqpoint{7.945342in}{2.029229in}}%
\pgfpathlineto{\pgfqpoint{7.937037in}{2.039786in}}%
\pgfpathlineto{\pgfqpoint{7.928732in}{2.050343in}}%
\pgfpathlineto{\pgfqpoint{7.920427in}{2.060901in}}%
\pgfpathlineto{\pgfqpoint{7.912122in}{2.071458in}}%
\pgfpathlineto{\pgfqpoint{7.903818in}{2.082015in}}%
\pgfpathlineto{\pgfqpoint{7.895513in}{2.092572in}}%
\pgfpathlineto{\pgfqpoint{7.887208in}{2.103129in}}%
\pgfpathlineto{\pgfqpoint{7.878903in}{2.113686in}}%
\pgfpathlineto{\pgfqpoint{7.870598in}{2.124243in}}%
\pgfpathlineto{\pgfqpoint{7.862293in}{2.134800in}}%
\pgfpathlineto{\pgfqpoint{7.853989in}{2.145357in}}%
\pgfpathlineto{\pgfqpoint{7.845684in}{2.155914in}}%
\pgfpathlineto{\pgfqpoint{7.837379in}{2.166471in}}%
\pgfpathlineto{\pgfqpoint{7.829074in}{2.177028in}}%
\pgfpathlineto{\pgfqpoint{7.820769in}{2.187585in}}%
\pgfpathlineto{\pgfqpoint{7.812464in}{2.198143in}}%
\pgfpathlineto{\pgfqpoint{7.804160in}{2.208700in}}%
\pgfpathlineto{\pgfqpoint{7.795855in}{2.219257in}}%
\pgfpathlineto{\pgfqpoint{7.787550in}{2.229814in}}%
\pgfpathlineto{\pgfqpoint{7.779245in}{2.240371in}}%
\pgfpathlineto{\pgfqpoint{7.770940in}{2.250928in}}%
\pgfpathlineto{\pgfqpoint{7.762635in}{2.261485in}}%
\pgfpathlineto{\pgfqpoint{7.754331in}{2.272042in}}%
\pgfpathlineto{\pgfqpoint{7.746026in}{2.282599in}}%
\pgfpathlineto{\pgfqpoint{7.737721in}{2.293156in}}%
\pgfpathlineto{\pgfqpoint{7.729416in}{2.303713in}}%
\pgfpathlineto{\pgfqpoint{7.721111in}{2.314270in}}%
\pgfpathlineto{\pgfqpoint{7.712806in}{2.324827in}}%
\pgfpathlineto{\pgfqpoint{7.704502in}{2.335385in}}%
\pgfpathlineto{\pgfqpoint{7.696197in}{2.345942in}}%
\pgfpathlineto{\pgfqpoint{7.687892in}{2.356499in}}%
\pgfpathlineto{\pgfqpoint{7.679587in}{2.367056in}}%
\pgfpathlineto{\pgfqpoint{7.671282in}{2.377613in}}%
\pgfpathlineto{\pgfqpoint{7.662977in}{2.388170in}}%
\pgfpathlineto{\pgfqpoint{7.654673in}{2.398727in}}%
\pgfpathlineto{\pgfqpoint{7.646368in}{2.409284in}}%
\pgfpathlineto{\pgfqpoint{7.638063in}{2.419841in}}%
\pgfpathlineto{\pgfqpoint{7.629758in}{2.430398in}}%
\pgfpathlineto{\pgfqpoint{7.621453in}{2.440955in}}%
\pgfpathlineto{\pgfqpoint{7.613148in}{2.451512in}}%
\pgfpathlineto{\pgfqpoint{7.604844in}{2.462069in}}%
\pgfpathlineto{\pgfqpoint{7.596539in}{2.472626in}}%
\pgfpathlineto{\pgfqpoint{7.588234in}{2.483184in}}%
\pgfpathlineto{\pgfqpoint{7.579929in}{2.493741in}}%
\pgfpathlineto{\pgfqpoint{7.571624in}{2.504298in}}%
\pgfpathlineto{\pgfqpoint{7.563319in}{2.514855in}}%
\pgfpathlineto{\pgfqpoint{7.555015in}{2.525412in}}%
\pgfpathlineto{\pgfqpoint{7.546710in}{2.535969in}}%
\pgfpathlineto{\pgfqpoint{7.538405in}{2.546526in}}%
\pgfpathlineto{\pgfqpoint{7.530100in}{2.557083in}}%
\pgfpathlineto{\pgfqpoint{7.521795in}{2.567640in}}%
\pgfpathlineto{\pgfqpoint{7.513490in}{2.578197in}}%
\pgfpathlineto{\pgfqpoint{7.505186in}{2.588754in}}%
\pgfpathlineto{\pgfqpoint{7.496881in}{2.599311in}}%
\pgfpathlineto{\pgfqpoint{7.488576in}{2.609868in}}%
\pgfpathlineto{\pgfqpoint{7.480271in}{2.620426in}}%
\pgfpathlineto{\pgfqpoint{7.471966in}{2.630983in}}%
\pgfpathlineto{\pgfqpoint{7.463661in}{2.641540in}}%
\pgfpathlineto{\pgfqpoint{7.455357in}{2.652097in}}%
\pgfpathlineto{\pgfqpoint{7.447052in}{2.662654in}}%
\pgfpathlineto{\pgfqpoint{7.438747in}{2.673211in}}%
\pgfpathlineto{\pgfqpoint{7.430442in}{2.683768in}}%
\pgfpathlineto{\pgfqpoint{7.422137in}{2.694325in}}%
\pgfpathlineto{\pgfqpoint{7.413832in}{2.704882in}}%
\pgfpathlineto{\pgfqpoint{7.405528in}{2.715439in}}%
\pgfpathlineto{\pgfqpoint{7.397223in}{2.725996in}}%
\pgfpathlineto{\pgfqpoint{7.388918in}{2.736553in}}%
\pgfpathlineto{\pgfqpoint{7.380613in}{2.747110in}}%
\pgfpathlineto{\pgfqpoint{7.372308in}{2.757668in}}%
\pgfpathlineto{\pgfqpoint{7.364003in}{2.768225in}}%
\pgfpathlineto{\pgfqpoint{7.355699in}{2.778782in}}%
\pgfpathlineto{\pgfqpoint{7.347394in}{2.789339in}}%
\pgfpathlineto{\pgfqpoint{7.339089in}{2.799896in}}%
\pgfpathlineto{\pgfqpoint{7.330784in}{2.810453in}}%
\pgfpathlineto{\pgfqpoint{7.322479in}{2.821010in}}%
\pgfpathlineto{\pgfqpoint{7.314174in}{2.831567in}}%
\pgfpathlineto{\pgfqpoint{7.305870in}{2.842124in}}%
\pgfpathlineto{\pgfqpoint{7.297565in}{2.852681in}}%
\pgfpathlineto{\pgfqpoint{7.289260in}{2.863238in}}%
\pgfpathlineto{\pgfqpoint{7.280955in}{2.873795in}}%
\pgfpathlineto{\pgfqpoint{7.272650in}{2.884352in}}%
\pgfpathlineto{\pgfqpoint{7.264345in}{2.894910in}}%
\pgfpathlineto{\pgfqpoint{7.256041in}{2.905467in}}%
\pgfpathlineto{\pgfqpoint{7.247736in}{2.916024in}}%
\pgfpathlineto{\pgfqpoint{7.239431in}{2.926581in}}%
\pgfpathlineto{\pgfqpoint{7.231126in}{2.937138in}}%
\pgfpathlineto{\pgfqpoint{7.222821in}{2.947695in}}%
\pgfpathlineto{\pgfqpoint{7.214516in}{2.958252in}}%
\pgfpathlineto{\pgfqpoint{7.206212in}{2.968809in}}%
\pgfpathlineto{\pgfqpoint{7.197907in}{2.979366in}}%
\pgfpathlineto{\pgfqpoint{7.189602in}{2.989923in}}%
\pgfpathlineto{\pgfqpoint{7.181297in}{3.000480in}}%
\pgfpathlineto{\pgfqpoint{7.172992in}{3.011037in}}%
\pgfpathlineto{\pgfqpoint{7.164687in}{3.021594in}}%
\pgfpathlineto{\pgfqpoint{7.156382in}{3.032152in}}%
\pgfpathlineto{\pgfqpoint{7.148078in}{3.042709in}}%
\pgfpathlineto{\pgfqpoint{7.139773in}{3.053266in}}%
\pgfpathlineto{\pgfqpoint{7.131468in}{3.063823in}}%
\pgfpathlineto{\pgfqpoint{7.123163in}{3.074380in}}%
\pgfpathlineto{\pgfqpoint{7.114858in}{3.084937in}}%
\pgfpathlineto{\pgfqpoint{7.106553in}{3.095494in}}%
\pgfpathlineto{\pgfqpoint{7.098249in}{3.106051in}}%
\pgfpathlineto{\pgfqpoint{7.089944in}{3.116608in}}%
\pgfpathlineto{\pgfqpoint{7.081639in}{3.127165in}}%
\pgfpathlineto{\pgfqpoint{7.073334in}{3.137722in}}%
\pgfpathlineto{\pgfqpoint{7.065029in}{3.148279in}}%
\pgfpathlineto{\pgfqpoint{7.056724in}{3.158836in}}%
\pgfpathlineto{\pgfqpoint{7.048420in}{3.169394in}}%
\pgfpathlineto{\pgfqpoint{7.040115in}{3.179951in}}%
\pgfpathlineto{\pgfqpoint{7.031810in}{3.190508in}}%
\pgfpathlineto{\pgfqpoint{7.023505in}{3.201065in}}%
\pgfpathlineto{\pgfqpoint{7.015200in}{3.211622in}}%
\pgfpathlineto{\pgfqpoint{7.006895in}{3.222179in}}%
\pgfpathlineto{\pgfqpoint{6.998591in}{3.232736in}}%
\pgfpathlineto{\pgfqpoint{6.990286in}{3.243293in}}%
\pgfpathlineto{\pgfqpoint{6.981981in}{3.253850in}}%
\pgfpathlineto{\pgfqpoint{6.973676in}{3.264407in}}%
\pgfpathlineto{\pgfqpoint{6.965371in}{3.274964in}}%
\pgfpathlineto{\pgfqpoint{6.957066in}{3.285521in}}%
\pgfpathlineto{\pgfqpoint{6.948762in}{3.296078in}}%
\pgfpathlineto{\pgfqpoint{6.940457in}{3.306636in}}%
\pgfpathlineto{\pgfqpoint{6.932152in}{3.317193in}}%
\pgfpathlineto{\pgfqpoint{6.923847in}{3.327750in}}%
\pgfpathlineto{\pgfqpoint{6.915542in}{3.338307in}}%
\pgfpathlineto{\pgfqpoint{6.907237in}{3.348864in}}%
\pgfpathlineto{\pgfqpoint{6.898933in}{3.359421in}}%
\pgfpathlineto{\pgfqpoint{6.890628in}{3.369978in}}%
\pgfpathlineto{\pgfqpoint{6.882323in}{3.380535in}}%
\pgfpathlineto{\pgfqpoint{6.874018in}{3.391092in}}%
\pgfpathlineto{\pgfqpoint{6.865713in}{3.401649in}}%
\pgfpathlineto{\pgfqpoint{6.857408in}{3.412206in}}%
\pgfpathlineto{\pgfqpoint{6.849104in}{3.422763in}}%
\pgfpathlineto{\pgfqpoint{6.840799in}{3.433320in}}%
\pgfpathlineto{\pgfqpoint{6.832494in}{3.443878in}}%
\pgfpathlineto{\pgfqpoint{6.824189in}{3.454435in}}%
\pgfpathlineto{\pgfqpoint{6.815884in}{3.464992in}}%
\pgfpathlineto{\pgfqpoint{6.807579in}{3.475549in}}%
\pgfpathlineto{\pgfqpoint{6.799275in}{3.486106in}}%
\pgfpathlineto{\pgfqpoint{6.790970in}{3.496663in}}%
\pgfpathlineto{\pgfqpoint{6.782665in}{3.507220in}}%
\pgfpathlineto{\pgfqpoint{6.774360in}{3.517777in}}%
\pgfpathlineto{\pgfqpoint{6.766055in}{3.528334in}}%
\pgfpathlineto{\pgfqpoint{6.757750in}{3.538891in}}%
\pgfpathlineto{\pgfqpoint{6.749446in}{3.549448in}}%
\pgfpathlineto{\pgfqpoint{6.741141in}{3.560005in}}%
\pgfpathlineto{\pgfqpoint{6.732836in}{3.570562in}}%
\pgfpathlineto{\pgfqpoint{6.724531in}{3.581120in}}%
\pgfpathlineto{\pgfqpoint{6.716226in}{3.591677in}}%
\pgfpathlineto{\pgfqpoint{6.707921in}{3.602234in}}%
\pgfpathlineto{\pgfqpoint{6.699617in}{3.612791in}}%
\pgfpathlineto{\pgfqpoint{6.691312in}{3.623348in}}%
\pgfpathlineto{\pgfqpoint{6.683007in}{3.633905in}}%
\pgfpathlineto{\pgfqpoint{6.674702in}{3.644462in}}%
\pgfpathlineto{\pgfqpoint{6.666397in}{3.655019in}}%
\pgfpathlineto{\pgfqpoint{6.658092in}{3.665576in}}%
\pgfpathlineto{\pgfqpoint{6.649788in}{3.676133in}}%
\pgfpathlineto{\pgfqpoint{6.641483in}{3.686690in}}%
\pgfpathlineto{\pgfqpoint{6.633178in}{3.697247in}}%
\pgfpathlineto{\pgfqpoint{6.624873in}{3.707804in}}%
\pgfpathlineto{\pgfqpoint{6.616568in}{3.718362in}}%
\pgfpathlineto{\pgfqpoint{6.608263in}{3.728919in}}%
\pgfpathlineto{\pgfqpoint{6.599959in}{3.739476in}}%
\pgfpathlineto{\pgfqpoint{6.591654in}{3.750033in}}%
\pgfpathlineto{\pgfqpoint{6.583349in}{3.760590in}}%
\pgfpathlineto{\pgfqpoint{6.575044in}{3.771147in}}%
\pgfpathlineto{\pgfqpoint{6.566739in}{3.781704in}}%
\pgfpathlineto{\pgfqpoint{6.558434in}{3.792261in}}%
\pgfpathlineto{\pgfqpoint{6.550130in}{3.802818in}}%
\pgfpathlineto{\pgfqpoint{6.541825in}{3.813375in}}%
\pgfpathlineto{\pgfqpoint{6.533520in}{3.823932in}}%
\pgfpathlineto{\pgfqpoint{6.525215in}{3.834489in}}%
\pgfpathlineto{\pgfqpoint{6.516910in}{3.845046in}}%
\pgfpathlineto{\pgfqpoint{6.508605in}{3.855604in}}%
\pgfpathlineto{\pgfqpoint{6.500301in}{3.866161in}}%
\pgfpathlineto{\pgfqpoint{6.491996in}{3.876718in}}%
\pgfpathlineto{\pgfqpoint{6.483691in}{3.887275in}}%
\pgfpathlineto{\pgfqpoint{6.475386in}{3.897832in}}%
\pgfpathlineto{\pgfqpoint{6.467081in}{3.908389in}}%
\pgfpathlineto{\pgfqpoint{6.458776in}{3.918946in}}%
\pgfpathlineto{\pgfqpoint{6.450472in}{3.929503in}}%
\pgfpathlineto{\pgfqpoint{6.442167in}{3.940060in}}%
\pgfpathlineto{\pgfqpoint{6.433862in}{3.950617in}}%
\pgfpathlineto{\pgfqpoint{6.425557in}{3.961174in}}%
\pgfpathlineto{\pgfqpoint{6.417252in}{3.971731in}}%
\pgfpathlineto{\pgfqpoint{6.408947in}{3.982288in}}%
\pgfpathlineto{\pgfqpoint{6.400643in}{3.992846in}}%
\pgfpathlineto{\pgfqpoint{6.392338in}{4.003403in}}%
\pgfpathlineto{\pgfqpoint{6.384033in}{4.013960in}}%
\pgfpathlineto{\pgfqpoint{6.375728in}{4.024517in}}%
\pgfpathlineto{\pgfqpoint{6.367423in}{4.035074in}}%
\pgfpathlineto{\pgfqpoint{6.359118in}{4.045631in}}%
\pgfpathlineto{\pgfqpoint{6.350814in}{4.056188in}}%
\pgfpathlineto{\pgfqpoint{6.342509in}{4.066745in}}%
\pgfpathlineto{\pgfqpoint{6.334204in}{4.077302in}}%
\pgfpathlineto{\pgfqpoint{6.325899in}{4.087859in}}%
\pgfpathlineto{\pgfqpoint{6.317594in}{4.098416in}}%
\pgfpathlineto{\pgfqpoint{6.309289in}{4.108973in}}%
\pgfpathlineto{\pgfqpoint{6.300985in}{4.119530in}}%
\pgfpathlineto{\pgfqpoint{6.292680in}{4.130088in}}%
\pgfpathlineto{\pgfqpoint{6.284375in}{4.140645in}}%
\pgfpathlineto{\pgfqpoint{6.276070in}{4.151202in}}%
\pgfpathlineto{\pgfqpoint{6.267765in}{4.161759in}}%
\pgfpathlineto{\pgfqpoint{6.259460in}{4.172316in}}%
\pgfpathlineto{\pgfqpoint{6.251156in}{4.182873in}}%
\pgfpathlineto{\pgfqpoint{6.242851in}{4.193430in}}%
\pgfpathlineto{\pgfqpoint{6.234546in}{4.203987in}}%
\pgfpathlineto{\pgfqpoint{6.226241in}{4.214544in}}%
\pgfpathlineto{\pgfqpoint{6.217936in}{4.225101in}}%
\pgfpathlineto{\pgfqpoint{6.209631in}{4.235658in}}%
\pgfpathlineto{\pgfqpoint{6.201326in}{4.246215in}}%
\pgfpathlineto{\pgfqpoint{6.193022in}{4.256772in}}%
\pgfpathlineto{\pgfqpoint{6.184717in}{4.267330in}}%
\pgfpathlineto{\pgfqpoint{6.176412in}{4.277887in}}%
\pgfpathlineto{\pgfqpoint{6.168107in}{4.288444in}}%
\pgfpathlineto{\pgfqpoint{6.159802in}{4.299001in}}%
\pgfpathlineto{\pgfqpoint{6.151497in}{4.309558in}}%
\pgfpathlineto{\pgfqpoint{6.143193in}{4.320115in}}%
\pgfpathlineto{\pgfqpoint{6.134888in}{4.330672in}}%
\pgfpathlineto{\pgfqpoint{6.126583in}{4.341229in}}%
\pgfpathlineto{\pgfqpoint{6.118278in}{4.351786in}}%
\pgfpathlineto{\pgfqpoint{6.109973in}{4.362343in}}%
\pgfpathlineto{\pgfqpoint{6.101668in}{4.372900in}}%
\pgfpathlineto{\pgfqpoint{6.093364in}{4.383457in}}%
\pgfpathlineto{\pgfqpoint{6.085059in}{4.394014in}}%
\pgfpathlineto{\pgfqpoint{6.076754in}{4.404572in}}%
\pgfpathlineto{\pgfqpoint{6.068449in}{4.415129in}}%
\pgfpathlineto{\pgfqpoint{6.060144in}{4.425686in}}%
\pgfpathlineto{\pgfqpoint{6.051839in}{4.436243in}}%
\pgfpathlineto{\pgfqpoint{6.043535in}{4.446800in}}%
\pgfpathlineto{\pgfqpoint{6.035230in}{4.457357in}}%
\pgfpathlineto{\pgfqpoint{6.026925in}{4.467914in}}%
\pgfpathlineto{\pgfqpoint{6.018620in}{4.478471in}}%
\pgfpathlineto{\pgfqpoint{6.010315in}{4.489028in}}%
\pgfpathlineto{\pgfqpoint{6.002010in}{4.499585in}}%
\pgfpathlineto{\pgfqpoint{5.993706in}{4.510142in}}%
\pgfpathlineto{\pgfqpoint{5.985401in}{4.520699in}}%
\pgfpathlineto{\pgfqpoint{5.977096in}{4.531256in}}%
\pgfpathlineto{\pgfqpoint{5.968791in}{4.541814in}}%
\pgfpathlineto{\pgfqpoint{5.960486in}{4.552371in}}%
\pgfpathlineto{\pgfqpoint{5.952181in}{4.562928in}}%
\pgfpathlineto{\pgfqpoint{5.943877in}{4.573485in}}%
\pgfpathlineto{\pgfqpoint{5.935572in}{4.584042in}}%
\pgfpathlineto{\pgfqpoint{5.927267in}{4.594599in}}%
\pgfpathlineto{\pgfqpoint{5.918962in}{4.605156in}}%
\pgfpathlineto{\pgfqpoint{5.910657in}{4.615713in}}%
\pgfpathlineto{\pgfqpoint{5.902352in}{4.626270in}}%
\pgfpathlineto{\pgfqpoint{5.894048in}{4.636827in}}%
\pgfpathlineto{\pgfqpoint{5.885743in}{4.647384in}}%
\pgfpathlineto{\pgfqpoint{5.877438in}{4.657941in}}%
\pgfpathlineto{\pgfqpoint{5.869133in}{4.668498in}}%
\pgfpathlineto{\pgfqpoint{5.860828in}{4.679056in}}%
\pgfpathlineto{\pgfqpoint{5.852523in}{4.689613in}}%
\pgfpathlineto{\pgfqpoint{5.844219in}{4.700170in}}%
\pgfpathlineto{\pgfqpoint{5.835914in}{4.710727in}}%
\pgfpathlineto{\pgfqpoint{5.827609in}{4.721284in}}%
\pgfpathlineto{\pgfqpoint{5.819304in}{4.731841in}}%
\pgfpathlineto{\pgfqpoint{5.810999in}{4.742398in}}%
\pgfpathlineto{\pgfqpoint{5.802694in}{4.752955in}}%
\pgfpathlineto{\pgfqpoint{5.794390in}{4.763512in}}%
\pgfpathlineto{\pgfqpoint{5.786085in}{4.774069in}}%
\pgfpathlineto{\pgfqpoint{5.777780in}{4.784626in}}%
\pgfpathlineto{\pgfqpoint{5.769475in}{4.795183in}}%
\pgfpathlineto{\pgfqpoint{5.761170in}{4.805740in}}%
\pgfpathlineto{\pgfqpoint{5.752865in}{4.816297in}}%
\pgfpathlineto{\pgfqpoint{5.744561in}{4.826855in}}%
\pgfpathlineto{\pgfqpoint{5.736256in}{4.837412in}}%
\pgfpathlineto{\pgfqpoint{5.727951in}{4.847969in}}%
\pgfpathlineto{\pgfqpoint{5.719646in}{4.858526in}}%
\pgfpathlineto{\pgfqpoint{5.711341in}{4.869083in}}%
\pgfpathlineto{\pgfqpoint{5.703036in}{4.879640in}}%
\pgfpathlineto{\pgfqpoint{5.694732in}{4.890197in}}%
\pgfpathlineto{\pgfqpoint{5.686427in}{4.900754in}}%
\pgfpathlineto{\pgfqpoint{5.678122in}{4.911311in}}%
\pgfpathlineto{\pgfqpoint{5.669817in}{4.921868in}}%
\pgfpathlineto{\pgfqpoint{5.661512in}{4.932425in}}%
\pgfpathlineto{\pgfqpoint{5.653207in}{4.942982in}}%
\pgfpathlineto{\pgfqpoint{5.644903in}{4.953539in}}%
\pgfpathlineto{\pgfqpoint{5.636598in}{4.964097in}}%
\pgfpathlineto{\pgfqpoint{5.628293in}{4.974654in}}%
\pgfpathlineto{\pgfqpoint{5.619988in}{4.985211in}}%
\pgfpathlineto{\pgfqpoint{5.611683in}{4.995768in}}%
\pgfpathlineto{\pgfqpoint{5.603378in}{5.006325in}}%
\pgfpathlineto{\pgfqpoint{5.595074in}{5.016882in}}%
\pgfpathlineto{\pgfqpoint{5.586769in}{5.027439in}}%
\pgfpathlineto{\pgfqpoint{5.578464in}{5.037996in}}%
\pgfpathlineto{\pgfqpoint{5.570159in}{5.048553in}}%
\pgfpathlineto{\pgfqpoint{5.561854in}{5.059110in}}%
\pgfpathlineto{\pgfqpoint{5.553549in}{5.069667in}}%
\pgfpathlineto{\pgfqpoint{5.545245in}{5.080224in}}%
\pgfpathlineto{\pgfqpoint{5.536940in}{5.090781in}}%
\pgfpathlineto{\pgfqpoint{5.528635in}{5.101339in}}%
\pgfpathlineto{\pgfqpoint{5.520330in}{5.111896in}}%
\pgfpathlineto{\pgfqpoint{5.512025in}{5.122453in}}%
\pgfpathlineto{\pgfqpoint{5.503720in}{5.133010in}}%
\pgfpathlineto{\pgfqpoint{5.495416in}{5.143567in}}%
\pgfpathlineto{\pgfqpoint{5.487111in}{5.154124in}}%
\pgfpathlineto{\pgfqpoint{5.478806in}{5.164681in}}%
\pgfpathlineto{\pgfqpoint{5.470501in}{5.175238in}}%
\pgfpathlineto{\pgfqpoint{5.462196in}{5.185795in}}%
\pgfpathlineto{\pgfqpoint{5.453891in}{5.196352in}}%
\pgfpathlineto{\pgfqpoint{5.445587in}{5.206909in}}%
\pgfpathlineto{\pgfqpoint{5.437282in}{5.217466in}}%
\pgfpathlineto{\pgfqpoint{5.428977in}{5.228023in}}%
\pgfpathlineto{\pgfqpoint{5.420672in}{5.238581in}}%
\pgfpathlineto{\pgfqpoint{5.412367in}{5.249138in}}%
\pgfpathlineto{\pgfqpoint{5.404062in}{5.259695in}}%
\pgfpathlineto{\pgfqpoint{5.395758in}{5.270252in}}%
\pgfpathlineto{\pgfqpoint{5.387453in}{5.280809in}}%
\pgfpathlineto{\pgfqpoint{5.379148in}{5.291366in}}%
\pgfpathlineto{\pgfqpoint{5.370843in}{5.301923in}}%
\pgfpathlineto{\pgfqpoint{5.362538in}{5.312480in}}%
\pgfpathlineto{\pgfqpoint{5.354233in}{5.323037in}}%
\pgfpathlineto{\pgfqpoint{5.345929in}{5.333594in}}%
\pgfpathlineto{\pgfqpoint{5.337624in}{5.344151in}}%
\pgfpathlineto{\pgfqpoint{5.329319in}{5.354708in}}%
\pgfpathlineto{\pgfqpoint{5.321014in}{5.365265in}}%
\pgfpathlineto{\pgfqpoint{5.312709in}{5.375823in}}%
\pgfpathlineto{\pgfqpoint{5.304404in}{5.386380in}}%
\pgfpathlineto{\pgfqpoint{5.296099in}{5.396937in}}%
\pgfpathlineto{\pgfqpoint{5.287795in}{5.407494in}}%
\pgfpathlineto{\pgfqpoint{5.279490in}{5.418051in}}%
\pgfpathlineto{\pgfqpoint{5.271185in}{5.428608in}}%
\pgfpathlineto{\pgfqpoint{5.262880in}{5.439165in}}%
\pgfpathlineto{\pgfqpoint{5.254575in}{5.449722in}}%
\pgfpathlineto{\pgfqpoint{5.247606in}{5.458582in}}%
\pgfpathlineto{\pgfqpoint{5.246270in}{5.460279in}}%
\pgfpathlineto{\pgfqpoint{5.243057in}{5.464364in}}%
\pgfpathlineto{\pgfqpoint{5.238508in}{5.470146in}}%
\pgfpathlineto{\pgfqpoint{5.237966in}{5.470836in}}%
\pgfpathlineto{\pgfqpoint{5.233960in}{5.475928in}}%
\pgfpathlineto{\pgfqpoint{5.229661in}{5.481393in}}%
\pgfpathlineto{\pgfqpoint{5.229411in}{5.481711in}}%
\pgfpathlineto{\pgfqpoint{5.224863in}{5.487493in}}%
\pgfpathlineto{\pgfqpoint{5.221356in}{5.491950in}}%
\pgfpathlineto{\pgfqpoint{5.220314in}{5.493275in}}%
\pgfpathlineto{\pgfqpoint{5.215765in}{5.499057in}}%
\pgfpathlineto{\pgfqpoint{5.213051in}{5.502507in}}%
\pgfpathlineto{\pgfqpoint{5.211217in}{5.504839in}}%
\pgfpathlineto{\pgfqpoint{5.206668in}{5.510622in}}%
\pgfpathlineto{\pgfqpoint{5.204746in}{5.513065in}}%
\pgfpathlineto{\pgfqpoint{5.202119in}{5.516404in}}%
\pgfpathlineto{\pgfqpoint{5.197571in}{5.522186in}}%
\pgfpathlineto{\pgfqpoint{5.196441in}{5.523622in}}%
\pgfpathlineto{\pgfqpoint{5.193022in}{5.527968in}}%
\pgfpathlineto{\pgfqpoint{5.188474in}{5.533750in}}%
\pgfpathlineto{\pgfqpoint{5.188137in}{5.534179in}}%
\pgfpathlineto{\pgfqpoint{5.183925in}{5.539533in}}%
\pgfpathlineto{\pgfqpoint{5.179832in}{5.544736in}}%
\pgfpathlineto{\pgfqpoint{5.179376in}{5.545315in}}%
\pgfpathlineto{\pgfqpoint{5.174828in}{5.551097in}}%
\pgfpathlineto{\pgfqpoint{5.171527in}{5.555293in}}%
\pgfpathlineto{\pgfqpoint{5.170279in}{5.556879in}}%
\pgfpathlineto{\pgfqpoint{5.165730in}{5.562661in}}%
\pgfpathlineto{\pgfqpoint{5.163222in}{5.565850in}}%
\pgfpathlineto{\pgfqpoint{5.161182in}{5.568444in}}%
\pgfpathlineto{\pgfqpoint{5.156633in}{5.574226in}}%
\pgfpathlineto{\pgfqpoint{5.154917in}{5.576407in}}%
\pgfpathlineto{\pgfqpoint{5.152085in}{5.580008in}}%
\pgfpathlineto{\pgfqpoint{5.147536in}{5.585790in}}%
\pgfpathlineto{\pgfqpoint{5.146612in}{5.586964in}}%
\pgfpathlineto{\pgfqpoint{5.142987in}{5.591572in}}%
\pgfpathlineto{\pgfqpoint{5.138439in}{5.597355in}}%
\pgfpathlineto{\pgfqpoint{5.138308in}{5.597521in}}%
\pgfpathlineto{\pgfqpoint{5.133890in}{5.603137in}}%
\pgfpathlineto{\pgfqpoint{5.130003in}{5.608078in}}%
\pgfpathlineto{\pgfqpoint{5.129341in}{5.608919in}}%
\pgfpathlineto{\pgfqpoint{5.124793in}{5.614701in}}%
\pgfpathlineto{\pgfqpoint{5.121698in}{5.618635in}}%
\pgfpathlineto{\pgfqpoint{5.120244in}{5.620483in}}%
\pgfpathlineto{\pgfqpoint{5.115696in}{5.626266in}}%
\pgfpathlineto{\pgfqpoint{5.113393in}{5.629192in}}%
\pgfpathlineto{\pgfqpoint{5.111147in}{5.632048in}}%
\pgfpathlineto{\pgfqpoint{5.106598in}{5.637830in}}%
\pgfpathlineto{\pgfqpoint{5.105088in}{5.639749in}}%
\pgfpathlineto{\pgfqpoint{5.102050in}{5.643612in}}%
\pgfpathlineto{\pgfqpoint{5.097501in}{5.649394in}}%
\pgfpathlineto{\pgfqpoint{5.096783in}{5.650307in}}%
\pgfpathlineto{\pgfqpoint{5.092952in}{5.655177in}}%
\pgfpathlineto{\pgfqpoint{5.088479in}{5.660864in}}%
\pgfpathlineto{\pgfqpoint{5.088404in}{5.660959in}}%
\pgfpathlineto{\pgfqpoint{5.083855in}{5.666741in}}%
\pgfpathlineto{\pgfqpoint{5.080174in}{5.671421in}}%
\pgfpathlineto{\pgfqpoint{5.079306in}{5.672523in}}%
\pgfpathlineto{\pgfqpoint{5.074758in}{5.678305in}}%
\pgfpathlineto{\pgfqpoint{5.071869in}{5.681978in}}%
\pgfpathlineto{\pgfqpoint{5.070209in}{5.684088in}}%
\pgfpathlineto{\pgfqpoint{5.065661in}{5.689870in}}%
\pgfpathlineto{\pgfqpoint{5.063564in}{5.692535in}}%
\pgfpathlineto{\pgfqpoint{5.061112in}{5.695652in}}%
\pgfpathlineto{\pgfqpoint{5.056563in}{5.701434in}}%
\pgfpathlineto{\pgfqpoint{5.055259in}{5.703092in}}%
\pgfpathlineto{\pgfqpoint{5.052015in}{5.707216in}}%
\pgfpathlineto{\pgfqpoint{5.047466in}{5.712999in}}%
\pgfpathlineto{\pgfqpoint{5.046954in}{5.713649in}}%
\pgfpathlineto{\pgfqpoint{5.042917in}{5.718781in}}%
\pgfpathlineto{\pgfqpoint{5.038650in}{5.724206in}}%
\pgfpathlineto{\pgfqpoint{5.038369in}{5.724563in}}%
\pgfpathlineto{\pgfqpoint{5.033820in}{5.730345in}}%
\pgfpathlineto{\pgfqpoint{5.030345in}{5.734763in}}%
\pgfpathlineto{\pgfqpoint{5.029272in}{5.736127in}}%
\pgfpathlineto{\pgfqpoint{5.024723in}{5.741910in}}%
\pgfpathlineto{\pgfqpoint{5.022040in}{5.745320in}}%
\pgfpathlineto{\pgfqpoint{5.020174in}{5.747692in}}%
\pgfpathlineto{\pgfqpoint{5.015626in}{5.753474in}}%
\pgfpathlineto{\pgfqpoint{5.013735in}{5.755877in}}%
\pgfpathlineto{\pgfqpoint{5.011077in}{5.759256in}}%
\pgfpathlineto{\pgfqpoint{5.006528in}{5.765038in}}%
\pgfpathlineto{\pgfqpoint{5.005430in}{5.766434in}}%
\pgfpathlineto{\pgfqpoint{5.001980in}{5.770821in}}%
\pgfpathlineto{\pgfqpoint{4.997431in}{5.776603in}}%
\pgfpathlineto{\pgfqpoint{4.997125in}{5.776991in}}%
\pgfpathlineto{\pgfqpoint{4.992883in}{5.782385in}}%
\pgfpathlineto{\pgfqpoint{4.988821in}{5.787549in}}%
\pgfpathlineto{\pgfqpoint{4.988334in}{5.788167in}}%
\pgfpathlineto{\pgfqpoint{4.983785in}{5.793949in}}%
\pgfpathlineto{\pgfqpoint{4.980516in}{5.798106in}}%
\pgfpathlineto{\pgfqpoint{4.979237in}{5.799732in}}%
\pgfpathlineto{\pgfqpoint{4.974688in}{5.805514in}}%
\pgfpathlineto{\pgfqpoint{4.972211in}{5.808663in}}%
\pgfpathlineto{\pgfqpoint{4.970139in}{5.811296in}}%
\pgfpathlineto{\pgfqpoint{4.965591in}{5.817078in}}%
\pgfpathlineto{\pgfqpoint{4.963906in}{5.819220in}}%
\pgfpathlineto{\pgfqpoint{4.961042in}{5.822860in}}%
\pgfpathlineto{\pgfqpoint{4.956494in}{5.828643in}}%
\pgfpathlineto{\pgfqpoint{4.955601in}{5.829777in}}%
\pgfpathlineto{\pgfqpoint{4.951945in}{5.834425in}}%
\pgfpathlineto{\pgfqpoint{4.947396in}{5.840207in}}%
\pgfpathlineto{\pgfqpoint{4.947296in}{5.840334in}}%
\pgfpathlineto{\pgfqpoint{4.942848in}{5.845989in}}%
\pgfpathlineto{\pgfqpoint{4.938992in}{5.850891in}}%
\pgfpathlineto{\pgfqpoint{4.938299in}{5.851771in}}%
\pgfpathlineto{\pgfqpoint{4.933750in}{5.857554in}}%
\pgfpathlineto{\pgfqpoint{4.930687in}{5.861448in}}%
\pgfpathlineto{\pgfqpoint{4.929202in}{5.863336in}}%
\pgfpathlineto{\pgfqpoint{4.924653in}{5.869118in}}%
\pgfpathlineto{\pgfqpoint{4.922382in}{5.872005in}}%
\pgfpathlineto{\pgfqpoint{4.920104in}{5.874900in}}%
\pgfpathlineto{\pgfqpoint{4.915556in}{5.880682in}}%
\pgfpathlineto{\pgfqpoint{4.914077in}{5.882562in}}%
\pgfpathlineto{\pgfqpoint{4.911007in}{5.886465in}}%
\pgfpathlineto{\pgfqpoint{4.906459in}{5.892247in}}%
\pgfpathlineto{\pgfqpoint{4.905772in}{5.893119in}}%
\pgfpathlineto{\pgfqpoint{4.901910in}{5.898029in}}%
\pgfpathlineto{\pgfqpoint{4.897467in}{5.903676in}}%
\pgfpathlineto{\pgfqpoint{4.897361in}{5.903811in}}%
\pgfpathlineto{\pgfqpoint{4.892813in}{5.909593in}}%
\pgfpathlineto{\pgfqpoint{4.889163in}{5.914233in}}%
\pgfpathlineto{\pgfqpoint{4.888264in}{5.915376in}}%
\pgfpathlineto{\pgfqpoint{4.883715in}{5.921158in}}%
\pgfpathlineto{\pgfqpoint{4.880858in}{5.924791in}}%
\pgfpathlineto{\pgfqpoint{4.879167in}{5.926940in}}%
\pgfpathlineto{\pgfqpoint{4.874618in}{5.932722in}}%
\pgfpathlineto{\pgfqpoint{4.872553in}{5.935348in}}%
\pgfpathlineto{\pgfqpoint{4.870070in}{5.938504in}}%
\pgfpathlineto{\pgfqpoint{4.865521in}{5.944287in}}%
\pgfpathlineto{\pgfqpoint{4.864248in}{5.945905in}}%
\pgfpathlineto{\pgfqpoint{4.860972in}{5.950069in}}%
\pgfpathlineto{\pgfqpoint{4.856424in}{5.955851in}}%
\pgfpathlineto{\pgfqpoint{4.855943in}{5.956462in}}%
\pgfpathlineto{\pgfqpoint{4.851875in}{5.961633in}}%
\pgfpathlineto{\pgfqpoint{4.847638in}{5.967019in}}%
\pgfpathlineto{\pgfqpoint{4.847326in}{5.967415in}}%
\pgfpathlineto{\pgfqpoint{4.842778in}{5.973198in}}%
\pgfpathlineto{\pgfqpoint{4.839334in}{5.977576in}}%
\pgfpathlineto{\pgfqpoint{4.838229in}{5.978980in}}%
\pgfpathlineto{\pgfqpoint{4.833681in}{5.984762in}}%
\pgfpathlineto{\pgfqpoint{4.831029in}{5.988133in}}%
\pgfpathlineto{\pgfqpoint{4.829132in}{5.990544in}}%
\pgfpathlineto{\pgfqpoint{4.824583in}{5.996326in}}%
\pgfpathlineto{\pgfqpoint{4.822724in}{5.998690in}}%
\pgfpathlineto{\pgfqpoint{4.820035in}{6.002109in}}%
\pgfpathlineto{\pgfqpoint{4.815486in}{6.007891in}}%
\pgfpathlineto{\pgfqpoint{4.814419in}{6.009247in}}%
\pgfpathlineto{\pgfqpoint{4.810937in}{6.013673in}}%
\pgfpathlineto{\pgfqpoint{4.806389in}{6.019455in}}%
\pgfpathlineto{\pgfqpoint{4.806114in}{6.019804in}}%
\pgfpathlineto{\pgfqpoint{4.801840in}{6.025237in}}%
\pgfpathlineto{\pgfqpoint{4.797809in}{6.030361in}}%
\pgfpathlineto{\pgfqpoint{4.797292in}{6.031020in}}%
\pgfpathlineto{\pgfqpoint{4.792743in}{6.036802in}}%
\pgfpathlineto{\pgfqpoint{4.789505in}{6.040918in}}%
\pgfpathlineto{\pgfqpoint{4.788194in}{6.042584in}}%
\pgfpathlineto{\pgfqpoint{4.783646in}{6.048366in}}%
\pgfpathlineto{\pgfqpoint{4.781200in}{6.051475in}}%
\pgfpathlineto{\pgfqpoint{4.779097in}{6.054148in}}%
\pgfpathlineto{\pgfqpoint{4.774548in}{6.059931in}}%
\pgfpathlineto{\pgfqpoint{4.772895in}{6.062033in}}%
\pgfpathlineto{\pgfqpoint{4.770000in}{6.065713in}}%
\pgfpathlineto{\pgfqpoint{4.765451in}{6.071495in}}%
\pgfpathlineto{\pgfqpoint{4.764590in}{6.072590in}}%
\pgfpathlineto{\pgfqpoint{4.760902in}{6.077277in}}%
\pgfpathlineto{\pgfqpoint{4.756354in}{6.083059in}}%
\pgfpathlineto{\pgfqpoint{4.756285in}{6.083147in}}%
\pgfpathlineto{\pgfqpoint{4.751805in}{6.088842in}}%
\pgfpathlineto{\pgfqpoint{4.747980in}{6.093704in}}%
\pgfpathlineto{\pgfqpoint{4.747257in}{6.094624in}}%
\pgfpathlineto{\pgfqpoint{4.742708in}{6.100406in}}%
\pgfpathlineto{\pgfqpoint{4.739676in}{6.104261in}}%
\pgfpathlineto{\pgfqpoint{4.738159in}{6.106188in}}%
\pgfpathlineto{\pgfqpoint{4.733611in}{6.111970in}}%
\pgfpathlineto{\pgfqpoint{4.731371in}{6.114818in}}%
\pgfpathlineto{\pgfqpoint{4.729062in}{6.117753in}}%
\pgfpathlineto{\pgfqpoint{4.724513in}{6.123535in}}%
\pgfpathlineto{\pgfqpoint{4.723066in}{6.125375in}}%
\pgfpathlineto{\pgfqpoint{4.719965in}{6.129317in}}%
\pgfpathlineto{\pgfqpoint{4.715416in}{6.135099in}}%
\pgfpathlineto{\pgfqpoint{4.714761in}{6.135932in}}%
\pgfpathlineto{\pgfqpoint{4.710868in}{6.140881in}}%
\pgfpathlineto{\pgfqpoint{4.706456in}{6.146489in}}%
\pgfpathlineto{\pgfqpoint{4.706319in}{6.146664in}}%
\pgfpathlineto{\pgfqpoint{4.701770in}{6.152446in}}%
\pgfpathlineto{\pgfqpoint{4.698151in}{6.157046in}}%
\pgfpathlineto{\pgfqpoint{4.697222in}{6.158228in}}%
\pgfpathlineto{\pgfqpoint{4.692673in}{6.164010in}}%
\pgfpathlineto{\pgfqpoint{4.689847in}{6.167603in}}%
\pgfpathlineto{\pgfqpoint{4.688124in}{6.169792in}}%
\pgfpathlineto{\pgfqpoint{4.683576in}{6.175575in}}%
\pgfpathlineto{\pgfqpoint{4.681542in}{6.178160in}}%
\pgfpathlineto{\pgfqpoint{4.679027in}{6.181357in}}%
\pgfpathlineto{\pgfqpoint{4.674479in}{6.187139in}}%
\pgfpathlineto{\pgfqpoint{4.673237in}{6.188717in}}%
\pgfpathlineto{\pgfqpoint{4.669930in}{6.192921in}}%
\pgfpathlineto{\pgfqpoint{4.665381in}{6.198703in}}%
\pgfpathlineto{\pgfqpoint{4.664932in}{6.199275in}}%
\pgfpathlineto{\pgfqpoint{4.660833in}{6.204486in}}%
\pgfpathlineto{\pgfqpoint{4.656627in}{6.209832in}}%
\pgfpathlineto{\pgfqpoint{4.656284in}{6.210268in}}%
\pgfpathlineto{\pgfqpoint{4.651735in}{6.216050in}}%
\pgfpathlineto{\pgfqpoint{4.648322in}{6.220389in}}%
\pgfpathlineto{\pgfqpoint{4.647187in}{6.221832in}}%
\pgfpathlineto{\pgfqpoint{4.642638in}{6.227614in}}%
\pgfpathlineto{\pgfqpoint{4.640018in}{6.230946in}}%
\pgfpathlineto{\pgfqpoint{4.638090in}{6.233397in}}%
\pgfpathlineto{\pgfqpoint{4.633541in}{6.239179in}}%
\pgfpathlineto{\pgfqpoint{4.631713in}{6.241503in}}%
\pgfpathlineto{\pgfqpoint{4.628992in}{6.244961in}}%
\pgfpathlineto{\pgfqpoint{4.624444in}{6.250743in}}%
\pgfpathlineto{\pgfqpoint{4.623408in}{6.252060in}}%
\pgfpathlineto{\pgfqpoint{4.619895in}{6.256525in}}%
\pgfpathlineto{\pgfqpoint{4.615346in}{6.262308in}}%
\pgfpathlineto{\pgfqpoint{4.615103in}{6.262617in}}%
\pgfpathlineto{\pgfqpoint{4.610798in}{6.268090in}}%
\pgfpathlineto{\pgfqpoint{4.606798in}{6.273174in}}%
\pgfpathlineto{\pgfqpoint{4.606249in}{6.273872in}}%
\pgfpathlineto{\pgfqpoint{4.601700in}{6.279654in}}%
\pgfpathlineto{\pgfqpoint{4.598493in}{6.283731in}}%
\pgfpathlineto{\pgfqpoint{4.597152in}{6.285436in}}%
\pgfpathlineto{\pgfqpoint{4.592603in}{6.291219in}}%
\pgfpathlineto{\pgfqpoint{4.590189in}{6.294288in}}%
\pgfpathlineto{\pgfqpoint{4.588055in}{6.297001in}}%
\pgfpathlineto{\pgfqpoint{4.583506in}{6.302783in}}%
\pgfpathlineto{\pgfqpoint{4.581884in}{6.304845in}}%
\pgfpathlineto{\pgfqpoint{4.578957in}{6.308565in}}%
\pgfpathlineto{\pgfqpoint{4.574409in}{6.314347in}}%
\pgfpathlineto{\pgfqpoint{4.573579in}{6.315402in}}%
\pgfpathlineto{\pgfqpoint{4.569860in}{6.320130in}}%
\pgfpathlineto{\pgfqpoint{4.565311in}{6.325912in}}%
\pgfpathlineto{\pgfqpoint{4.565274in}{6.325959in}}%
\pgfpathlineto{\pgfqpoint{4.560763in}{6.331694in}}%
\pgfpathlineto{\pgfqpoint{4.556969in}{6.336517in}}%
\pgfpathlineto{\pgfqpoint{4.556214in}{6.337476in}}%
\pgfpathlineto{\pgfqpoint{4.551666in}{6.343258in}}%
\pgfpathlineto{\pgfqpoint{4.548664in}{6.347074in}}%
\pgfpathlineto{\pgfqpoint{4.547117in}{6.349041in}}%
\pgfpathlineto{\pgfqpoint{4.542568in}{6.354823in}}%
\pgfpathlineto{\pgfqpoint{4.540360in}{6.357631in}}%
\pgfpathlineto{\pgfqpoint{4.538020in}{6.360605in}}%
\pgfpathlineto{\pgfqpoint{4.533471in}{6.366387in}}%
\pgfpathlineto{\pgfqpoint{4.532055in}{6.368188in}}%
\pgfpathlineto{\pgfqpoint{4.528922in}{6.372169in}}%
\pgfpathlineto{\pgfqpoint{4.524374in}{6.377952in}}%
\pgfpathlineto{\pgfqpoint{4.523750in}{6.378745in}}%
\pgfpathlineto{\pgfqpoint{4.519825in}{6.383734in}}%
\pgfpathlineto{\pgfqpoint{4.515445in}{6.389302in}}%
\pgfpathlineto{\pgfqpoint{4.515277in}{6.389516in}}%
\pgfpathlineto{\pgfqpoint{4.510728in}{6.395298in}}%
\pgfpathlineto{\pgfqpoint{4.507140in}{6.399859in}}%
\pgfpathlineto{\pgfqpoint{4.506179in}{6.401080in}}%
\pgfpathlineto{\pgfqpoint{4.501631in}{6.406863in}}%
\pgfpathlineto{\pgfqpoint{4.498835in}{6.410416in}}%
\pgfpathlineto{\pgfqpoint{4.497082in}{6.412645in}}%
\pgfpathlineto{\pgfqpoint{4.492533in}{6.418427in}}%
\pgfpathlineto{\pgfqpoint{4.490531in}{6.420973in}}%
\pgfpathlineto{\pgfqpoint{4.487985in}{6.424209in}}%
\pgfpathlineto{\pgfqpoint{4.483436in}{6.429991in}}%
\pgfpathlineto{\pgfqpoint{4.482226in}{6.431530in}}%
\pgfpathlineto{\pgfqpoint{4.478888in}{6.435774in}}%
\pgfpathlineto{\pgfqpoint{4.474339in}{6.441556in}}%
\pgfpathlineto{\pgfqpoint{4.473921in}{6.442087in}}%
\pgfpathlineto{\pgfqpoint{4.469790in}{6.447338in}}%
\pgfpathlineto{\pgfqpoint{4.465616in}{6.452644in}}%
\pgfpathlineto{\pgfqpoint{4.465242in}{6.453120in}}%
\pgfpathlineto{\pgfqpoint{4.460693in}{6.458902in}}%
\pgfpathlineto{\pgfqpoint{4.457311in}{6.463201in}}%
\pgfpathlineto{\pgfqpoint{4.456144in}{6.464685in}}%
\pgfpathlineto{\pgfqpoint{4.451596in}{6.470467in}}%
\pgfpathlineto{\pgfqpoint{4.449006in}{6.473759in}}%
\pgfpathlineto{\pgfqpoint{4.447047in}{6.476249in}}%
\pgfpathlineto{\pgfqpoint{4.442498in}{6.482031in}}%
\pgfpathlineto{\pgfqpoint{4.440702in}{6.484316in}}%
\pgfpathlineto{\pgfqpoint{4.437950in}{6.487813in}}%
\pgfpathlineto{\pgfqpoint{4.433401in}{6.493596in}}%
\pgfpathlineto{\pgfqpoint{4.432397in}{6.494873in}}%
\pgfpathlineto{\pgfqpoint{4.428853in}{6.499378in}}%
\pgfpathlineto{\pgfqpoint{4.424304in}{6.505160in}}%
\pgfpathlineto{\pgfqpoint{4.424092in}{6.505430in}}%
\pgfpathlineto{\pgfqpoint{4.419755in}{6.510942in}}%
\pgfpathlineto{\pgfqpoint{4.415787in}{6.515987in}}%
\pgfpathlineto{\pgfqpoint{4.415207in}{6.516724in}}%
\pgfpathlineto{\pgfqpoint{4.410658in}{6.522507in}}%
\pgfpathlineto{\pgfqpoint{4.407482in}{6.526544in}}%
\pgfpathlineto{\pgfqpoint{4.406109in}{6.528289in}}%
\pgfpathlineto{\pgfqpoint{4.401561in}{6.534071in}}%
\pgfpathlineto{\pgfqpoint{4.399177in}{6.537101in}}%
\pgfpathlineto{\pgfqpoint{4.397012in}{6.539853in}}%
\pgfpathlineto{\pgfqpoint{4.392464in}{6.545635in}}%
\pgfpathlineto{\pgfqpoint{4.390873in}{6.547658in}}%
\pgfpathlineto{\pgfqpoint{4.387915in}{6.551418in}}%
\pgfpathlineto{\pgfqpoint{4.383366in}{6.557200in}}%
\pgfpathlineto{\pgfqpoint{4.382568in}{6.558215in}}%
\pgfpathlineto{\pgfqpoint{4.378818in}{6.562982in}}%
\pgfpathlineto{\pgfqpoint{4.374269in}{6.568764in}}%
\pgfpathlineto{\pgfqpoint{4.374263in}{6.568772in}}%
\pgfpathlineto{\pgfqpoint{4.369720in}{6.574546in}}%
\pgfpathlineto{\pgfqpoint{4.365958in}{6.579329in}}%
\pgfpathlineto{\pgfqpoint{4.365172in}{6.580329in}}%
\pgfpathlineto{\pgfqpoint{4.360623in}{6.586111in}}%
\pgfpathlineto{\pgfqpoint{4.357653in}{6.589886in}}%
\pgfpathlineto{\pgfqpoint{4.356075in}{6.591893in}}%
\pgfpathlineto{\pgfqpoint{4.351526in}{6.597675in}}%
\pgfpathlineto{\pgfqpoint{4.349348in}{6.600443in}}%
\pgfpathlineto{\pgfqpoint{4.346977in}{6.603457in}}%
\pgfpathlineto{\pgfqpoint{4.342429in}{6.609240in}}%
\pgfpathlineto{\pgfqpoint{4.341043in}{6.611001in}}%
\pgfpathlineto{\pgfqpoint{4.337880in}{6.615022in}}%
\pgfpathlineto{\pgfqpoint{4.333331in}{6.620804in}}%
\pgfpathlineto{\pgfqpoint{4.332739in}{6.621558in}}%
\pgfpathlineto{\pgfqpoint{4.328783in}{6.626586in}}%
\pgfpathlineto{\pgfqpoint{4.324434in}{6.632115in}}%
\pgfpathlineto{\pgfqpoint{4.324234in}{6.632368in}}%
\pgfpathlineto{\pgfqpoint{4.319686in}{6.638151in}}%
\pgfpathlineto{\pgfqpoint{4.316129in}{6.642672in}}%
\pgfpathlineto{\pgfqpoint{4.315137in}{6.643933in}}%
\pgfpathlineto{\pgfqpoint{4.310588in}{6.649715in}}%
\pgfpathlineto{\pgfqpoint{4.307824in}{6.653229in}}%
\pgfpathlineto{\pgfqpoint{4.306040in}{6.655497in}}%
\pgfpathlineto{\pgfqpoint{4.301491in}{6.661279in}}%
\pgfpathlineto{\pgfqpoint{4.299519in}{6.663786in}}%
\pgfpathlineto{\pgfqpoint{4.296942in}{6.667062in}}%
\pgfpathlineto{\pgfqpoint{4.292394in}{6.672844in}}%
\pgfpathlineto{\pgfqpoint{4.291214in}{6.674343in}}%
\pgfpathlineto{\pgfqpoint{4.287845in}{6.678626in}}%
\pgfpathlineto{\pgfqpoint{4.283296in}{6.684408in}}%
\pgfpathlineto{\pgfqpoint{4.282910in}{6.684900in}}%
\pgfpathlineto{\pgfqpoint{4.278748in}{6.690190in}}%
\pgfpathlineto{\pgfqpoint{4.274605in}{6.695457in}}%
\pgfpathlineto{\pgfqpoint{4.274199in}{6.695973in}}%
\pgfpathlineto{\pgfqpoint{4.269651in}{6.701755in}}%
\pgfpathlineto{\pgfqpoint{4.266300in}{6.706014in}}%
\pgfpathlineto{\pgfqpoint{4.265102in}{6.707537in}}%
\pgfpathlineto{\pgfqpoint{4.260553in}{6.713319in}}%
\pgfpathlineto{\pgfqpoint{4.257995in}{6.716571in}}%
\pgfpathlineto{\pgfqpoint{4.256005in}{6.719101in}}%
\pgfpathlineto{\pgfqpoint{4.251456in}{6.724884in}}%
\pgfpathlineto{\pgfqpoint{4.249690in}{6.727128in}}%
\pgfpathlineto{\pgfqpoint{4.246907in}{6.730666in}}%
\pgfpathlineto{\pgfqpoint{4.242359in}{6.736448in}}%
\pgfpathlineto{\pgfqpoint{4.241385in}{6.737685in}}%
\pgfpathlineto{\pgfqpoint{4.237810in}{6.742230in}}%
\pgfpathlineto{\pgfqpoint{4.233262in}{6.748012in}}%
\pgfpathlineto{\pgfqpoint{4.233081in}{6.748243in}}%
\pgfpathlineto{\pgfqpoint{4.228713in}{6.753795in}}%
\pgfpathlineto{\pgfqpoint{4.224776in}{6.758800in}}%
\pgfpathlineto{\pgfqpoint{4.224164in}{6.759577in}}%
\pgfpathlineto{\pgfqpoint{4.219616in}{6.765359in}}%
\pgfpathlineto{\pgfqpoint{4.216471in}{6.769357in}}%
\pgfpathlineto{\pgfqpoint{4.215067in}{6.771141in}}%
\pgfpathlineto{\pgfqpoint{4.210518in}{6.776923in}}%
\pgfpathlineto{\pgfqpoint{4.208166in}{6.779914in}}%
\pgfpathlineto{\pgfqpoint{4.205970in}{6.782706in}}%
\pgfpathlineto{\pgfqpoint{4.201421in}{6.788488in}}%
\pgfpathlineto{\pgfqpoint{4.199861in}{6.790471in}}%
\pgfpathlineto{\pgfqpoint{4.196873in}{6.794270in}}%
\pgfpathlineto{\pgfqpoint{4.192324in}{6.800052in}}%
\pgfpathlineto{\pgfqpoint{4.191556in}{6.801028in}}%
\pgfpathlineto{\pgfqpoint{4.187775in}{6.805834in}}%
\pgfpathlineto{\pgfqpoint{4.183252in}{6.811585in}}%
\pgfpathlineto{\pgfqpoint{4.183227in}{6.811617in}}%
\pgfpathlineto{\pgfqpoint{4.178678in}{6.817399in}}%
\pgfpathlineto{\pgfqpoint{4.174947in}{6.822142in}}%
\pgfpathlineto{\pgfqpoint{4.174129in}{6.823181in}}%
\pgfpathlineto{\pgfqpoint{4.169581in}{6.828963in}}%
\pgfpathlineto{\pgfqpoint{4.166642in}{6.832699in}}%
\pgfpathlineto{\pgfqpoint{4.165032in}{6.834745in}}%
\pgfpathlineto{\pgfqpoint{4.160484in}{6.840528in}}%
\pgfpathlineto{\pgfqpoint{4.158337in}{6.843256in}}%
\pgfpathlineto{\pgfqpoint{4.155935in}{6.846310in}}%
\pgfpathlineto{\pgfqpoint{4.151386in}{6.852092in}}%
\pgfpathlineto{\pgfqpoint{4.150032in}{6.853813in}}%
\pgfpathlineto{\pgfqpoint{4.146838in}{6.857874in}}%
\pgfpathlineto{\pgfqpoint{4.142289in}{6.863657in}}%
\pgfpathlineto{\pgfqpoint{4.141727in}{6.864370in}}%
\pgfpathlineto{\pgfqpoint{4.137740in}{6.869439in}}%
\pgfpathlineto{\pgfqpoint{4.133423in}{6.874927in}}%
\pgfpathlineto{\pgfqpoint{4.133192in}{6.875221in}}%
\pgfpathlineto{\pgfqpoint{4.128643in}{6.881003in}}%
\pgfpathlineto{\pgfqpoint{4.125118in}{6.885485in}}%
\pgfpathlineto{\pgfqpoint{4.124094in}{6.886785in}}%
\pgfpathlineto{\pgfqpoint{4.119546in}{6.892568in}}%
\pgfpathlineto{\pgfqpoint{4.116813in}{6.896042in}}%
\pgfpathlineto{\pgfqpoint{4.114997in}{6.898350in}}%
\pgfpathlineto{\pgfqpoint{4.110449in}{6.904132in}}%
\pgfpathlineto{\pgfqpoint{4.108508in}{6.906599in}}%
\pgfpathlineto{\pgfqpoint{4.105900in}{6.909914in}}%
\pgfpathlineto{\pgfqpoint{4.101351in}{6.915696in}}%
\pgfpathlineto{\pgfqpoint{4.100203in}{6.917156in}}%
\pgfpathlineto{\pgfqpoint{4.096803in}{6.921479in}}%
\pgfpathlineto{\pgfqpoint{4.092254in}{6.927261in}}%
\pgfpathlineto{\pgfqpoint{4.091898in}{6.927713in}}%
\pgfpathlineto{\pgfqpoint{4.087705in}{6.933043in}}%
\pgfpathlineto{\pgfqpoint{4.083594in}{6.938270in}}%
\pgfpathlineto{\pgfqpoint{4.083157in}{6.938825in}}%
\pgfpathlineto{\pgfqpoint{4.078608in}{6.944607in}}%
\pgfpathlineto{\pgfqpoint{4.075289in}{6.948827in}}%
\pgfpathlineto{\pgfqpoint{4.074060in}{6.950390in}}%
\pgfpathlineto{\pgfqpoint{4.069511in}{6.956172in}}%
\pgfpathlineto{\pgfqpoint{4.066984in}{6.959384in}}%
\pgfpathlineto{\pgfqpoint{4.064962in}{6.961954in}}%
\pgfpathlineto{\pgfqpoint{4.060414in}{6.967736in}}%
\pgfpathlineto{\pgfqpoint{4.058679in}{6.969941in}}%
\pgfpathlineto{\pgfqpoint{4.055865in}{6.973518in}}%
\pgfpathlineto{\pgfqpoint{4.051316in}{6.979301in}}%
\pgfpathlineto{\pgfqpoint{4.050374in}{6.980498in}}%
\pgfpathlineto{\pgfqpoint{4.046768in}{6.985083in}}%
\pgfpathlineto{\pgfqpoint{4.042219in}{6.990865in}}%
\pgfpathlineto{\pgfqpoint{4.042069in}{6.991055in}}%
\pgfpathlineto{\pgfqpoint{4.037671in}{6.996647in}}%
\pgfpathlineto{\pgfqpoint{4.033765in}{7.001612in}}%
\pgfpathlineto{\pgfqpoint{4.033122in}{7.002429in}}%
\pgfpathlineto{\pgfqpoint{4.028573in}{7.008212in}}%
\pgfpathlineto{\pgfqpoint{4.025460in}{7.012169in}}%
\pgfpathlineto{\pgfqpoint{4.024025in}{7.013994in}}%
\pgfpathlineto{\pgfqpoint{4.019476in}{7.019776in}}%
\pgfpathlineto{\pgfqpoint{4.017155in}{7.022727in}}%
\pgfpathlineto{\pgfqpoint{4.014927in}{7.025558in}}%
\pgfpathlineto{\pgfqpoint{4.010379in}{7.031340in}}%
\pgfpathlineto{\pgfqpoint{4.008850in}{7.033284in}}%
\pgfpathlineto{\pgfqpoint{4.005830in}{7.037123in}}%
\pgfpathlineto{\pgfqpoint{4.001282in}{7.042905in}}%
\pgfpathlineto{\pgfqpoint{4.000545in}{7.043841in}}%
\pgfpathlineto{\pgfqpoint{3.996733in}{7.048687in}}%
\pgfpathlineto{\pgfqpoint{3.992240in}{7.054398in}}%
\pgfpathlineto{\pgfqpoint{3.992184in}{7.054469in}}%
\pgfpathlineto{\pgfqpoint{3.987636in}{7.060251in}}%
\pgfpathlineto{\pgfqpoint{3.983936in}{7.064955in}}%
\pgfpathlineto{\pgfqpoint{3.983087in}{7.066034in}}%
\pgfpathlineto{\pgfqpoint{3.978538in}{7.071816in}}%
\pgfpathlineto{\pgfqpoint{3.975631in}{7.075512in}}%
\pgfpathlineto{\pgfqpoint{3.973990in}{7.077598in}}%
\pgfpathlineto{\pgfqpoint{3.969441in}{7.083380in}}%
\pgfpathlineto{\pgfqpoint{3.967326in}{7.086069in}}%
\pgfpathlineto{\pgfqpoint{3.964893in}{7.089162in}}%
\pgfpathlineto{\pgfqpoint{3.960344in}{7.094945in}}%
\pgfpathlineto{\pgfqpoint{3.959021in}{7.096626in}}%
\pgfpathlineto{\pgfqpoint{3.955795in}{7.100727in}}%
\pgfpathlineto{\pgfqpoint{3.951247in}{7.106509in}}%
\pgfpathlineto{\pgfqpoint{3.950716in}{7.107183in}}%
\pgfpathlineto{\pgfqpoint{3.946698in}{7.112291in}}%
\pgfpathlineto{\pgfqpoint{3.942411in}{7.117740in}}%
\pgfpathlineto{\pgfqpoint{3.942149in}{7.118073in}}%
\pgfpathlineto{\pgfqpoint{3.937601in}{7.123856in}}%
\pgfpathlineto{\pgfqpoint{3.934107in}{7.128297in}}%
\pgfpathlineto{\pgfqpoint{3.933052in}{7.129638in}}%
\pgfpathlineto{\pgfqpoint{3.928503in}{7.135420in}}%
\pgfpathlineto{\pgfqpoint{3.925802in}{7.138854in}}%
\pgfpathlineto{\pgfqpoint{3.923955in}{7.141202in}}%
\pgfpathlineto{\pgfqpoint{3.919406in}{7.146984in}}%
\pgfpathlineto{\pgfqpoint{3.917497in}{7.149411in}}%
\pgfpathlineto{\pgfqpoint{3.914858in}{7.152767in}}%
\pgfpathlineto{\pgfqpoint{3.910309in}{7.158549in}}%
\pgfpathlineto{\pgfqpoint{3.909192in}{7.159968in}}%
\pgfpathlineto{\pgfqpoint{3.905760in}{7.164331in}}%
\pgfpathlineto{\pgfqpoint{3.901212in}{7.170113in}}%
\pgfpathlineto{\pgfqpoint{3.900887in}{7.170526in}}%
\pgfpathlineto{\pgfqpoint{3.896663in}{7.175895in}}%
\pgfpathlineto{\pgfqpoint{3.892582in}{7.181083in}}%
\pgfpathlineto{\pgfqpoint{3.892114in}{7.181678in}}%
\pgfpathlineto{\pgfqpoint{3.887566in}{7.187460in}}%
\pgfpathlineto{\pgfqpoint{3.884278in}{7.191640in}}%
\pgfpathlineto{\pgfqpoint{3.883017in}{7.193242in}}%
\pgfpathlineto{\pgfqpoint{3.878469in}{7.199024in}}%
\pgfpathlineto{\pgfqpoint{3.875973in}{7.202197in}}%
\pgfpathlineto{\pgfqpoint{3.873920in}{7.204806in}}%
\pgfpathlineto{\pgfqpoint{3.869371in}{7.210589in}}%
\pgfpathlineto{\pgfqpoint{3.867668in}{7.212754in}}%
\pgfpathlineto{\pgfqpoint{3.864823in}{7.216371in}}%
\pgfpathlineto{\pgfqpoint{3.860274in}{7.222153in}}%
\pgfpathlineto{\pgfqpoint{3.859363in}{7.223311in}}%
\pgfpathlineto{\pgfqpoint{3.855725in}{7.227935in}}%
\pgfpathlineto{\pgfqpoint{3.851177in}{7.233717in}}%
\pgfpathlineto{\pgfqpoint{3.851058in}{7.233868in}}%
\pgfpathlineto{\pgfqpoint{3.846628in}{7.239500in}}%
\pgfpathlineto{\pgfqpoint{3.842753in}{7.244425in}}%
\pgfpathlineto{\pgfqpoint{3.842080in}{7.245282in}}%
\pgfpathlineto{\pgfqpoint{3.837531in}{7.251064in}}%
\pgfpathlineto{\pgfqpoint{3.834449in}{7.254982in}}%
\pgfpathlineto{\pgfqpoint{3.832982in}{7.256846in}}%
\pgfpathlineto{\pgfqpoint{3.828434in}{7.262628in}}%
\pgfpathlineto{\pgfqpoint{3.826144in}{7.265539in}}%
\pgfpathlineto{\pgfqpoint{3.823885in}{7.268411in}}%
\pgfpathlineto{\pgfqpoint{3.819336in}{7.274193in}}%
\pgfpathlineto{\pgfqpoint{3.817839in}{7.276096in}}%
\pgfpathlineto{\pgfqpoint{3.814788in}{7.279975in}}%
\pgfpathlineto{\pgfqpoint{3.810239in}{7.285757in}}%
\pgfpathlineto{\pgfqpoint{3.809534in}{7.286653in}}%
\pgfpathlineto{\pgfqpoint{3.805691in}{7.291539in}}%
\pgfpathlineto{\pgfqpoint{3.801229in}{7.297210in}}%
\pgfpathlineto{\pgfqpoint{3.801142in}{7.297322in}}%
\pgfpathlineto{\pgfqpoint{3.796593in}{7.303104in}}%
\pgfpathlineto{\pgfqpoint{3.792924in}{7.307768in}}%
\pgfpathlineto{\pgfqpoint{3.792045in}{7.308886in}}%
\pgfpathlineto{\pgfqpoint{3.787496in}{7.314668in}}%
\pgfpathlineto{\pgfqpoint{3.784620in}{7.318325in}}%
\pgfpathlineto{\pgfqpoint{3.782947in}{7.320450in}}%
\pgfpathlineto{\pgfqpoint{3.778399in}{7.326233in}}%
\pgfpathlineto{\pgfqpoint{3.776315in}{7.328882in}}%
\pgfpathlineto{\pgfqpoint{3.773850in}{7.332015in}}%
\pgfpathlineto{\pgfqpoint{3.769301in}{7.337797in}}%
\pgfpathlineto{\pgfqpoint{3.768010in}{7.339439in}}%
\pgfpathlineto{\pgfqpoint{3.764753in}{7.343579in}}%
\pgfpathlineto{\pgfqpoint{3.760204in}{7.349361in}}%
\pgfpathlineto{\pgfqpoint{3.759705in}{7.349996in}}%
\pgfpathlineto{\pgfqpoint{3.755656in}{7.355144in}}%
\pgfpathlineto{\pgfqpoint{3.751400in}{7.360553in}}%
\pgfpathlineto{\pgfqpoint{3.751107in}{7.360926in}}%
\pgfpathlineto{\pgfqpoint{3.746558in}{7.366708in}}%
\pgfpathlineto{\pgfqpoint{3.743095in}{7.371110in}}%
\pgfpathlineto{\pgfqpoint{3.742010in}{7.372490in}}%
\pgfpathlineto{\pgfqpoint{3.737461in}{7.378272in}}%
\pgfpathlineto{\pgfqpoint{3.734791in}{7.381667in}}%
\pgfpathlineto{\pgfqpoint{3.732912in}{7.384055in}}%
\pgfpathlineto{\pgfqpoint{3.728364in}{7.389837in}}%
\pgfpathlineto{\pgfqpoint{3.726486in}{7.392224in}}%
\pgfpathlineto{\pgfqpoint{3.723815in}{7.395619in}}%
\pgfpathlineto{\pgfqpoint{3.719267in}{7.401401in}}%
\pgfpathlineto{\pgfqpoint{3.718181in}{7.402781in}}%
\pgfpathlineto{\pgfqpoint{3.714718in}{7.407183in}}%
\pgfpathlineto{\pgfqpoint{3.710169in}{7.412966in}}%
\pgfpathlineto{\pgfqpoint{3.709876in}{7.413338in}}%
\pgfpathlineto{\pgfqpoint{3.705621in}{7.418748in}}%
\pgfpathlineto{\pgfqpoint{3.701571in}{7.423895in}}%
\pgfpathlineto{\pgfqpoint{3.701072in}{7.424530in}}%
\pgfpathlineto{\pgfqpoint{3.696523in}{7.430312in}}%
\pgfpathlineto{\pgfqpoint{3.693266in}{7.434452in}}%
\pgfpathlineto{\pgfqpoint{3.691975in}{7.436094in}}%
\pgfpathlineto{\pgfqpoint{3.687426in}{7.441877in}}%
\pgfpathlineto{\pgfqpoint{3.684962in}{7.445010in}}%
\pgfpathlineto{\pgfqpoint{3.682878in}{7.447659in}}%
\pgfpathlineto{\pgfqpoint{3.678329in}{7.453441in}}%
\pgfpathlineto{\pgfqpoint{3.676657in}{7.455567in}}%
\pgfpathlineto{\pgfqpoint{3.673780in}{7.459223in}}%
\pgfpathlineto{\pgfqpoint{3.669232in}{7.465005in}}%
\pgfpathlineto{\pgfqpoint{3.668352in}{7.466124in}}%
\pgfpathlineto{\pgfqpoint{3.664683in}{7.470788in}}%
\pgfpathlineto{\pgfqpoint{3.660134in}{7.476570in}}%
\pgfpathlineto{\pgfqpoint{3.660047in}{7.476681in}}%
\pgfpathlineto{\pgfqpoint{3.655586in}{7.482352in}}%
\pgfpathlineto{\pgfqpoint{3.651742in}{7.487238in}}%
\pgfpathlineto{\pgfqpoint{3.651037in}{7.488134in}}%
\pgfpathlineto{\pgfqpoint{3.646489in}{7.493916in}}%
\pgfpathlineto{\pgfqpoint{3.643437in}{7.497795in}}%
\pgfpathlineto{\pgfqpoint{3.641940in}{7.499699in}}%
\pgfpathlineto{\pgfqpoint{3.637391in}{7.505481in}}%
\pgfpathlineto{\pgfqpoint{3.635133in}{7.508352in}}%
\pgfpathlineto{\pgfqpoint{3.632843in}{7.511263in}}%
\pgfpathlineto{\pgfqpoint{3.628294in}{7.517045in}}%
\pgfpathlineto{\pgfqpoint{3.626828in}{7.518909in}}%
\pgfpathlineto{\pgfqpoint{3.623745in}{7.522827in}}%
\pgfpathlineto{\pgfqpoint{3.619197in}{7.528610in}}%
\pgfpathlineto{\pgfqpoint{3.618523in}{7.529466in}}%
\pgfpathlineto{\pgfqpoint{3.614648in}{7.534392in}}%
\pgfpathlineto{\pgfqpoint{3.610218in}{7.540023in}}%
\pgfpathlineto{\pgfqpoint{3.610099in}{7.540174in}}%
\pgfpathlineto{\pgfqpoint{3.605551in}{7.545956in}}%
\pgfpathlineto{\pgfqpoint{3.601913in}{7.550580in}}%
\pgfpathlineto{\pgfqpoint{3.601002in}{7.551738in}}%
\pgfpathlineto{\pgfqpoint{3.596454in}{7.557521in}}%
\pgfpathlineto{\pgfqpoint{3.593608in}{7.561137in}}%
\pgfpathlineto{\pgfqpoint{3.591905in}{7.563303in}}%
\pgfpathlineto{\pgfqpoint{3.587356in}{7.569085in}}%
\pgfpathlineto{\pgfqpoint{3.585304in}{7.571694in}}%
\pgfpathlineto{\pgfqpoint{3.582808in}{7.574867in}}%
\pgfpathlineto{\pgfqpoint{3.578259in}{7.580649in}}%
\pgfpathlineto{\pgfqpoint{3.576999in}{7.582252in}}%
\pgfpathlineto{\pgfqpoint{3.573710in}{7.586432in}}%
\pgfpathlineto{\pgfqpoint{3.569162in}{7.592214in}}%
\pgfpathlineto{\pgfqpoint{3.568694in}{7.592809in}}%
\pgfpathlineto{\pgfqpoint{3.564613in}{7.597996in}}%
\pgfpathlineto{\pgfqpoint{3.560389in}{7.603366in}}%
\pgfpathlineto{\pgfqpoint{3.560065in}{7.603778in}}%
\pgfpathlineto{\pgfqpoint{3.555516in}{7.609560in}}%
\pgfpathlineto{\pgfqpoint{3.552084in}{7.613923in}}%
\pgfpathlineto{\pgfqpoint{3.550967in}{7.615343in}}%
\pgfpathlineto{\pgfqpoint{3.546419in}{7.621125in}}%
\pgfpathlineto{\pgfqpoint{3.543779in}{7.624480in}}%
\pgfpathlineto{\pgfqpoint{3.541870in}{7.626907in}}%
\pgfpathlineto{\pgfqpoint{3.537321in}{7.632689in}}%
\pgfpathlineto{\pgfqpoint{3.535475in}{7.635037in}}%
\pgfpathlineto{\pgfqpoint{3.532773in}{7.638471in}}%
\pgfpathlineto{\pgfqpoint{3.528224in}{7.644254in}}%
\pgfpathlineto{\pgfqpoint{3.527170in}{7.645594in}}%
\pgfpathlineto{\pgfqpoint{3.523676in}{7.650036in}}%
\pgfpathlineto{\pgfqpoint{3.519127in}{7.655818in}}%
\pgfpathlineto{\pgfqpoint{3.518865in}{7.656151in}}%
\pgfpathlineto{\pgfqpoint{3.514578in}{7.661600in}}%
\pgfpathlineto{\pgfqpoint{3.510560in}{7.666708in}}%
\pgfpathlineto{\pgfqpoint{3.510030in}{7.667382in}}%
\pgfpathlineto{\pgfqpoint{3.505481in}{7.673165in}}%
\pgfpathlineto{\pgfqpoint{3.502255in}{7.677265in}}%
\pgfpathlineto{\pgfqpoint{3.500932in}{7.678947in}}%
\pgfpathlineto{\pgfqpoint{3.496384in}{7.684729in}}%
\pgfpathlineto{\pgfqpoint{3.493950in}{7.687822in}}%
\pgfpathlineto{\pgfqpoint{3.491835in}{7.690511in}}%
\pgfpathlineto{\pgfqpoint{3.487287in}{7.696293in}}%
\pgfpathlineto{\pgfqpoint{3.485646in}{7.698379in}}%
\pgfpathlineto{\pgfqpoint{3.482738in}{7.702076in}}%
\pgfpathlineto{\pgfqpoint{3.478189in}{7.707858in}}%
\pgfpathlineto{\pgfqpoint{3.477341in}{7.708936in}}%
\pgfpathlineto{\pgfqpoint{3.473641in}{7.713640in}}%
\pgfpathlineto{\pgfqpoint{3.469092in}{7.719422in}}%
\pgfpathlineto{\pgfqpoint{3.469036in}{7.719494in}}%
\pgfpathlineto{\pgfqpoint{3.464543in}{7.725204in}}%
\pgfpathlineto{\pgfqpoint{3.460731in}{7.730051in}}%
\pgfpathlineto{\pgfqpoint{3.459995in}{7.730987in}}%
\pgfpathlineto{\pgfqpoint{3.455446in}{7.736769in}}%
\pgfpathlineto{\pgfqpoint{3.452426in}{7.740608in}}%
\pgfpathlineto{\pgfqpoint{3.450897in}{7.742551in}}%
\pgfpathlineto{\pgfqpoint{3.446349in}{7.748333in}}%
\pgfpathlineto{\pgfqpoint{3.444121in}{7.751165in}}%
\pgfpathlineto{\pgfqpoint{3.441800in}{7.754115in}}%
\pgfpathlineto{\pgfqpoint{3.437252in}{7.759898in}}%
\pgfpathlineto{\pgfqpoint{3.435816in}{7.761722in}}%
\pgfpathlineto{\pgfqpoint{3.432703in}{7.765680in}}%
\pgfpathlineto{\pgfqpoint{3.428154in}{7.771462in}}%
\pgfpathlineto{\pgfqpoint{3.427512in}{7.772279in}}%
\pgfpathlineto{\pgfqpoint{3.423606in}{7.777244in}}%
\pgfpathlineto{\pgfqpoint{3.419207in}{7.782836in}}%
\pgfpathlineto{\pgfqpoint{3.419057in}{7.783026in}}%
\pgfpathlineto{\pgfqpoint{3.414508in}{7.788809in}}%
\pgfpathlineto{\pgfqpoint{3.410902in}{7.793393in}}%
\pgfpathlineto{\pgfqpoint{3.409960in}{7.794591in}}%
\pgfpathlineto{\pgfqpoint{3.405411in}{7.800373in}}%
\pgfpathlineto{\pgfqpoint{3.402597in}{7.803950in}}%
\pgfpathlineto{\pgfqpoint{3.400863in}{7.806155in}}%
\pgfpathlineto{\pgfqpoint{3.396314in}{7.811937in}}%
\pgfpathlineto{\pgfqpoint{3.394292in}{7.814507in}}%
\pgfpathlineto{\pgfqpoint{3.391765in}{7.817720in}}%
\pgfpathlineto{\pgfqpoint{3.387217in}{7.823502in}}%
\pgfpathlineto{\pgfqpoint{3.385987in}{7.825064in}}%
\pgfpathlineto{\pgfqpoint{3.382668in}{7.829284in}}%
\pgfpathlineto{\pgfqpoint{3.378119in}{7.835066in}}%
\pgfpathlineto{\pgfqpoint{3.377683in}{7.835621in}}%
\pgfpathlineto{\pgfqpoint{3.373571in}{7.840848in}}%
\pgfpathlineto{\pgfqpoint{3.369378in}{7.846178in}}%
\pgfpathlineto{\pgfqpoint{3.369022in}{7.846631in}}%
\pgfpathlineto{\pgfqpoint{3.364474in}{7.852413in}}%
\pgfpathlineto{\pgfqpoint{3.361073in}{7.856736in}}%
\pgfpathlineto{\pgfqpoint{3.359925in}{7.858195in}}%
\pgfpathlineto{\pgfqpoint{3.355376in}{7.863977in}}%
\pgfpathlineto{\pgfqpoint{3.352768in}{7.867293in}}%
\pgfpathlineto{\pgfqpoint{3.350828in}{7.869759in}}%
\pgfpathlineto{\pgfqpoint{3.346279in}{7.875542in}}%
\pgfpathlineto{\pgfqpoint{3.344463in}{7.877850in}}%
\pgfpathlineto{\pgfqpoint{3.341730in}{7.881324in}}%
\pgfpathlineto{\pgfqpoint{3.337182in}{7.887106in}}%
\pgfpathlineto{\pgfqpoint{3.336158in}{7.888407in}}%
\pgfpathlineto{\pgfqpoint{3.332633in}{7.892888in}}%
\pgfpathlineto{\pgfqpoint{3.328085in}{7.898670in}}%
\pgfpathlineto{\pgfqpoint{3.327854in}{7.898964in}}%
\pgfpathlineto{\pgfqpoint{3.323536in}{7.904453in}}%
\pgfpathlineto{\pgfqpoint{3.319549in}{7.909521in}}%
\pgfpathlineto{\pgfqpoint{3.318987in}{7.910235in}}%
\pgfpathlineto{\pgfqpoint{3.314439in}{7.916017in}}%
\pgfpathlineto{\pgfqpoint{3.311244in}{7.920078in}}%
\pgfpathlineto{\pgfqpoint{3.309890in}{7.921799in}}%
\pgfpathlineto{\pgfqpoint{3.305341in}{7.927581in}}%
\pgfpathlineto{\pgfqpoint{3.302939in}{7.930635in}}%
\pgfpathlineto{\pgfqpoint{3.300793in}{7.933364in}}%
\pgfpathlineto{\pgfqpoint{3.296244in}{7.939146in}}%
\pgfpathlineto{\pgfqpoint{3.294634in}{7.941192in}}%
\pgfpathlineto{\pgfqpoint{3.291695in}{7.944928in}}%
\pgfpathlineto{\pgfqpoint{3.287147in}{7.950710in}}%
\pgfpathlineto{\pgfqpoint{3.286329in}{7.951749in}}%
\pgfpathlineto{\pgfqpoint{3.282598in}{7.956492in}}%
\pgfpathlineto{\pgfqpoint{3.278050in}{7.962275in}}%
\pgfpathlineto{\pgfqpoint{3.278025in}{7.962306in}}%
\pgfpathlineto{\pgfqpoint{3.273501in}{7.968057in}}%
\pgfpathlineto{\pgfqpoint{3.269720in}{7.972863in}}%
\pgfpathlineto{\pgfqpoint{3.268952in}{7.973839in}}%
\pgfpathlineto{\pgfqpoint{3.264404in}{7.979621in}}%
\pgfpathlineto{\pgfqpoint{3.261415in}{7.983420in}}%
\pgfpathlineto{\pgfqpoint{3.259855in}{7.985403in}}%
\pgfpathlineto{\pgfqpoint{3.255306in}{7.991186in}}%
\pgfpathlineto{\pgfqpoint{3.253110in}{7.993978in}}%
\pgfpathlineto{\pgfqpoint{3.250758in}{7.996968in}}%
\pgfpathlineto{\pgfqpoint{3.246209in}{8.002750in}}%
\pgfpathlineto{\pgfqpoint{3.244805in}{8.004535in}}%
\pgfpathlineto{\pgfqpoint{3.241661in}{8.008532in}}%
\pgfpathlineto{\pgfqpoint{3.237112in}{8.014314in}}%
\pgfpathlineto{\pgfqpoint{3.236500in}{8.015092in}}%
\pgfpathlineto{\pgfqpoint{3.232563in}{8.020097in}}%
\pgfpathlineto{\pgfqpoint{3.228196in}{8.025649in}}%
\pgfpathlineto{\pgfqpoint{3.228015in}{8.025879in}}%
\pgfpathlineto{\pgfqpoint{3.223466in}{8.031661in}}%
\pgfpathlineto{\pgfqpoint{3.219891in}{8.036206in}}%
\pgfpathlineto{\pgfqpoint{3.218917in}{8.037443in}}%
\pgfpathlineto{\pgfqpoint{3.214369in}{8.043225in}}%
\pgfpathlineto{\pgfqpoint{3.211586in}{8.046763in}}%
\pgfpathlineto{\pgfqpoint{3.209820in}{8.049008in}}%
\pgfpathlineto{\pgfqpoint{3.205272in}{8.054790in}}%
\pgfpathlineto{\pgfqpoint{3.203281in}{8.057320in}}%
\pgfpathlineto{\pgfqpoint{3.200723in}{8.060572in}}%
\pgfpathlineto{\pgfqpoint{3.196174in}{8.066354in}}%
\pgfpathlineto{\pgfqpoint{3.194976in}{8.067877in}}%
\pgfpathlineto{\pgfqpoint{3.191626in}{8.072136in}}%
\pgfpathlineto{\pgfqpoint{3.187077in}{8.077919in}}%
\pgfpathlineto{\pgfqpoint{3.186671in}{8.078434in}}%
\pgfpathlineto{\pgfqpoint{3.182528in}{8.083701in}}%
\pgfpathlineto{\pgfqpoint{3.178367in}{8.088991in}}%
\pgfpathlineto{\pgfqpoint{3.177980in}{8.089483in}}%
\pgfpathlineto{\pgfqpoint{3.173431in}{8.095265in}}%
\pgfpathlineto{\pgfqpoint{3.170062in}{8.099548in}}%
\pgfpathlineto{\pgfqpoint{3.168883in}{8.101047in}}%
\pgfpathlineto{\pgfqpoint{3.164334in}{8.106830in}}%
\pgfpathlineto{\pgfqpoint{3.161757in}{8.110105in}}%
\pgfpathlineto{\pgfqpoint{3.159785in}{8.112612in}}%
\pgfpathlineto{\pgfqpoint{3.155237in}{8.118394in}}%
\pgfpathlineto{\pgfqpoint{3.153452in}{8.120662in}}%
\pgfpathlineto{\pgfqpoint{3.150688in}{8.124176in}}%
\pgfpathlineto{\pgfqpoint{3.146139in}{8.129958in}}%
\pgfpathlineto{\pgfqpoint{3.145147in}{8.131220in}}%
\pgfpathlineto{\pgfqpoint{3.141591in}{8.135741in}}%
\pgfpathlineto{\pgfqpoint{3.137042in}{8.141523in}}%
\pgfpathlineto{\pgfqpoint{3.136842in}{8.141777in}}%
\pgfpathlineto{\pgfqpoint{3.132493in}{8.147305in}}%
\pgfpathlineto{\pgfqpoint{3.128538in}{8.152334in}}%
\pgfpathlineto{\pgfqpoint{3.127945in}{8.153087in}}%
\pgfpathlineto{\pgfqpoint{3.123396in}{8.158869in}}%
\pgfpathlineto{\pgfqpoint{3.120233in}{8.162891in}}%
\pgfpathlineto{\pgfqpoint{3.118848in}{8.164652in}}%
\pgfpathlineto{\pgfqpoint{3.114299in}{8.170434in}}%
\pgfpathlineto{\pgfqpoint{3.111928in}{8.173448in}}%
\pgfpathlineto{\pgfqpoint{3.109750in}{8.176216in}}%
\pgfpathlineto{\pgfqpoint{3.105202in}{8.181998in}}%
\pgfpathlineto{\pgfqpoint{3.103623in}{8.184005in}}%
\pgfpathlineto{\pgfqpoint{3.100653in}{8.187780in}}%
\pgfpathlineto{\pgfqpoint{3.096104in}{8.193563in}}%
\pgfpathlineto{\pgfqpoint{3.095318in}{8.194562in}}%
\pgfpathlineto{\pgfqpoint{3.091556in}{8.199345in}}%
\pgfpathlineto{\pgfqpoint{3.087013in}{8.205119in}}%
\pgfpathlineto{\pgfqpoint{3.087007in}{8.205127in}}%
\pgfpathlineto{\pgfqpoint{3.082459in}{8.210909in}}%
\pgfpathlineto{\pgfqpoint{3.078709in}{8.215676in}}%
\pgfpathlineto{\pgfqpoint{3.077910in}{8.216691in}}%
\pgfpathlineto{\pgfqpoint{3.073361in}{8.222474in}}%
\pgfpathlineto{\pgfqpoint{3.070404in}{8.226233in}}%
\pgfpathlineto{\pgfqpoint{3.068813in}{8.228256in}}%
\pgfpathlineto{\pgfqpoint{3.064264in}{8.234038in}}%
\pgfpathlineto{\pgfqpoint{3.062099in}{8.236790in}}%
\pgfpathlineto{\pgfqpoint{3.059715in}{8.239820in}}%
\pgfpathlineto{\pgfqpoint{3.055167in}{8.245602in}}%
\pgfpathlineto{\pgfqpoint{3.053794in}{8.247347in}}%
\pgfpathlineto{\pgfqpoint{3.050618in}{8.251385in}}%
\pgfpathlineto{\pgfqpoint{3.046070in}{8.257167in}}%
\pgfpathlineto{\pgfqpoint{3.045489in}{8.257904in}}%
\pgfpathlineto{\pgfqpoint{3.041521in}{8.262949in}}%
\pgfpathlineto{\pgfqpoint{3.037184in}{8.268462in}}%
\pgfpathlineto{\pgfqpoint{3.036972in}{8.268731in}}%
\pgfpathlineto{\pgfqpoint{3.032424in}{8.274513in}}%
\pgfpathlineto{\pgfqpoint{3.028880in}{8.279019in}}%
\pgfpathlineto{\pgfqpoint{3.027875in}{8.280296in}}%
\pgfpathlineto{\pgfqpoint{3.023326in}{8.286078in}}%
\pgfpathlineto{\pgfqpoint{3.020575in}{8.289576in}}%
\pgfpathlineto{\pgfqpoint{3.018778in}{8.291860in}}%
\pgfpathlineto{\pgfqpoint{3.014229in}{8.297642in}}%
\pgfpathlineto{\pgfqpoint{3.012270in}{8.300133in}}%
\pgfpathlineto{\pgfqpoint{3.009681in}{8.303424in}}%
\pgfpathlineto{\pgfqpoint{3.005132in}{8.309207in}}%
\pgfpathlineto{\pgfqpoint{3.003965in}{8.310690in}}%
\pgfpathlineto{\pgfqpoint{3.000583in}{8.314989in}}%
\pgfpathlineto{\pgfqpoint{2.996035in}{8.320771in}}%
\pgfpathlineto{\pgfqpoint{2.995660in}{8.321247in}}%
\pgfpathlineto{\pgfqpoint{2.991486in}{8.326553in}}%
\pgfpathlineto{\pgfqpoint{2.987355in}{8.331804in}}%
\pgfpathlineto{\pgfqpoint{2.986937in}{8.332335in}}%
\pgfpathlineto{\pgfqpoint{2.982389in}{8.338118in}}%
\pgfpathlineto{\pgfqpoint{2.979051in}{8.342361in}}%
\pgfpathlineto{\pgfqpoint{2.977840in}{8.343900in}}%
\pgfpathlineto{\pgfqpoint{2.973291in}{8.349682in}}%
\pgfpathlineto{\pgfqpoint{2.970746in}{8.352918in}}%
\pgfpathlineto{\pgfqpoint{2.968743in}{8.355464in}}%
\pgfpathlineto{\pgfqpoint{2.964194in}{8.361246in}}%
\pgfpathlineto{\pgfqpoint{2.962441in}{8.363475in}}%
\pgfpathlineto{\pgfqpoint{2.959646in}{8.367029in}}%
\pgfpathlineto{\pgfqpoint{2.955097in}{8.372811in}}%
\pgfpathlineto{\pgfqpoint{2.954136in}{8.374032in}}%
\pgfpathlineto{\pgfqpoint{2.950548in}{8.378593in}}%
\pgfpathlineto{\pgfqpoint{2.946000in}{8.384375in}}%
\pgfpathlineto{\pgfqpoint{2.945831in}{8.384589in}}%
\pgfpathlineto{\pgfqpoint{2.941451in}{8.390157in}}%
\pgfpathlineto{\pgfqpoint{2.937526in}{8.395146in}}%
\pgfpathlineto{\pgfqpoint{2.936902in}{8.395940in}}%
\pgfpathlineto{\pgfqpoint{2.932354in}{8.401722in}}%
\pgfpathlineto{\pgfqpoint{2.929222in}{8.405704in}}%
\pgfpathlineto{\pgfqpoint{2.927805in}{8.407504in}}%
\pgfpathlineto{\pgfqpoint{2.923257in}{8.413286in}}%
\pgfpathlineto{\pgfqpoint{2.920917in}{8.416261in}}%
\pgfpathlineto{\pgfqpoint{2.918708in}{8.419068in}}%
\pgfpathlineto{\pgfqpoint{2.914159in}{8.424851in}}%
\pgfpathlineto{\pgfqpoint{2.912612in}{8.426818in}}%
\pgfpathlineto{\pgfqpoint{2.909611in}{8.430633in}}%
\pgfpathlineto{\pgfqpoint{2.905062in}{8.436415in}}%
\pgfpathlineto{\pgfqpoint{2.904307in}{8.437375in}}%
\pgfpathlineto{\pgfqpoint{2.900513in}{8.442197in}}%
\pgfpathlineto{\pgfqpoint{2.896002in}{8.447932in}}%
\pgfpathlineto{\pgfqpoint{2.895965in}{8.447979in}}%
\pgfpathlineto{\pgfqpoint{2.891416in}{8.453762in}}%
\pgfpathlineto{\pgfqpoint{2.887697in}{8.458489in}}%
\pgfpathlineto{\pgfqpoint{2.886868in}{8.459544in}}%
\pgfpathlineto{\pgfqpoint{2.882319in}{8.465326in}}%
\pgfpathlineto{\pgfqpoint{2.879393in}{8.469046in}}%
\pgfpathlineto{\pgfqpoint{2.877770in}{8.471108in}}%
\pgfpathlineto{\pgfqpoint{2.873222in}{8.476890in}}%
\pgfpathlineto{\pgfqpoint{2.871088in}{8.479603in}}%
\pgfpathlineto{\pgfqpoint{2.868673in}{8.482673in}}%
\pgfpathlineto{\pgfqpoint{2.864124in}{8.488455in}}%
\pgfpathlineto{\pgfqpoint{2.862783in}{8.490160in}}%
\pgfpathlineto{\pgfqpoint{2.859576in}{8.494237in}}%
\pgfpathlineto{\pgfqpoint{2.855027in}{8.500019in}}%
\pgfpathlineto{\pgfqpoint{2.854478in}{8.500717in}}%
\pgfpathlineto{\pgfqpoint{2.850479in}{8.505801in}}%
\pgfpathlineto{\pgfqpoint{2.846173in}{8.511274in}}%
\pgfpathlineto{\pgfqpoint{2.845930in}{8.511584in}}%
\pgfpathlineto{\pgfqpoint{2.841381in}{8.517366in}}%
\pgfpathlineto{\pgfqpoint{2.837868in}{8.521831in}}%
\pgfpathlineto{\pgfqpoint{2.836833in}{8.523148in}}%
\pgfpathlineto{\pgfqpoint{2.832284in}{8.528930in}}%
\pgfpathlineto{\pgfqpoint{2.829564in}{8.532388in}}%
\pgfpathlineto{\pgfqpoint{2.827735in}{8.534712in}}%
\pgfpathlineto{\pgfqpoint{2.823187in}{8.540495in}}%
\pgfpathlineto{\pgfqpoint{2.821259in}{8.542946in}}%
\pgfpathlineto{\pgfqpoint{2.818638in}{8.546277in}}%
\pgfpathlineto{\pgfqpoint{2.814089in}{8.552059in}}%
\pgfpathlineto{\pgfqpoint{2.812954in}{8.553503in}}%
\pgfpathlineto{\pgfqpoint{2.809541in}{8.557841in}}%
\pgfpathlineto{\pgfqpoint{2.804992in}{8.563623in}}%
\pgfpathlineto{\pgfqpoint{2.804649in}{8.564060in}}%
\pgfpathlineto{\pgfqpoint{2.800444in}{8.569406in}}%
\pgfpathlineto{\pgfqpoint{2.796344in}{8.574617in}}%
\pgfpathlineto{\pgfqpoint{2.795895in}{8.575188in}}%
\pgfpathlineto{\pgfqpoint{2.791346in}{8.580970in}}%
\pgfpathlineto{\pgfqpoint{2.788039in}{8.585174in}}%
\pgfpathlineto{\pgfqpoint{2.786798in}{8.586752in}}%
\pgfpathlineto{\pgfqpoint{2.782249in}{8.592534in}}%
\pgfpathlineto{\pgfqpoint{2.779735in}{8.595731in}}%
\pgfpathlineto{\pgfqpoint{2.777700in}{8.598317in}}%
\pgfpathlineto{\pgfqpoint{2.773152in}{8.604099in}}%
\pgfpathlineto{\pgfqpoint{2.771430in}{8.606288in}}%
\pgfpathlineto{\pgfqpoint{2.768603in}{8.609881in}}%
\pgfpathlineto{\pgfqpoint{2.764055in}{8.615663in}}%
\pgfpathlineto{\pgfqpoint{2.763125in}{8.616845in}}%
\pgfpathlineto{\pgfqpoint{2.759506in}{8.621445in}}%
\pgfpathlineto{\pgfqpoint{2.754957in}{8.627228in}}%
\pgfpathlineto{\pgfqpoint{2.754820in}{8.627402in}}%
\pgfpathlineto{\pgfqpoint{2.750409in}{8.633010in}}%
\pgfpathlineto{\pgfqpoint{2.746515in}{8.637959in}}%
\pgfpathlineto{\pgfqpoint{2.745860in}{8.638792in}}%
\pgfpathlineto{\pgfqpoint{2.741311in}{8.644574in}}%
\pgfpathlineto{\pgfqpoint{2.738210in}{8.648516in}}%
\pgfpathlineto{\pgfqpoint{2.736763in}{8.650356in}}%
\pgfpathlineto{\pgfqpoint{2.732214in}{8.656139in}}%
\pgfpathlineto{\pgfqpoint{2.729906in}{8.659073in}}%
\pgfpathlineto{\pgfqpoint{2.727666in}{8.661921in}}%
\pgfpathlineto{\pgfqpoint{2.723117in}{8.667703in}}%
\pgfpathlineto{\pgfqpoint{2.721601in}{8.669630in}}%
\pgfpathlineto{\pgfqpoint{2.718568in}{8.673485in}}%
\pgfpathlineto{\pgfqpoint{2.714020in}{8.679267in}}%
\pgfpathlineto{\pgfqpoint{2.713296in}{8.680188in}}%
\pgfpathlineto{\pgfqpoint{2.709471in}{8.685050in}}%
\pgfpathlineto{\pgfqpoint{2.704991in}{8.690745in}}%
\pgfpathlineto{\pgfqpoint{2.704922in}{8.690832in}}%
\pgfpathlineto{\pgfqpoint{2.700374in}{8.696614in}}%
\pgfpathlineto{\pgfqpoint{2.696686in}{8.701302in}}%
\pgfpathlineto{\pgfqpoint{2.695825in}{8.702396in}}%
\pgfpathlineto{\pgfqpoint{2.691277in}{8.708178in}}%
\pgfpathlineto{\pgfqpoint{2.688381in}{8.711859in}}%
\pgfpathlineto{\pgfqpoint{2.686728in}{8.713961in}}%
\pgfpathlineto{\pgfqpoint{2.682179in}{8.719743in}}%
\pgfpathlineto{\pgfqpoint{2.680077in}{8.722416in}}%
\pgfpathlineto{\pgfqpoint{2.677631in}{8.725525in}}%
\pgfpathlineto{\pgfqpoint{2.673082in}{8.731307in}}%
\pgfpathlineto{\pgfqpoint{2.671772in}{8.732973in}}%
\pgfpathlineto{\pgfqpoint{2.668533in}{8.737089in}}%
\pgfpathlineto{\pgfqpoint{2.663985in}{8.742872in}}%
\pgfpathlineto{\pgfqpoint{2.663467in}{8.743530in}}%
\pgfpathlineto{\pgfqpoint{2.659436in}{8.748654in}}%
\pgfpathlineto{\pgfqpoint{2.655162in}{8.754087in}}%
\pgfpathlineto{\pgfqpoint{2.654887in}{8.754436in}}%
\pgfpathlineto{\pgfqpoint{2.650339in}{8.760218in}}%
\pgfpathlineto{\pgfqpoint{2.646857in}{8.764644in}}%
\pgfpathlineto{\pgfqpoint{2.645790in}{8.766000in}}%
\pgfpathlineto{\pgfqpoint{2.641242in}{8.771783in}}%
\pgfpathlineto{\pgfqpoint{2.638552in}{8.775201in}}%
\pgfpathlineto{\pgfqpoint{2.636693in}{8.777565in}}%
\pgfpathlineto{\pgfqpoint{2.632144in}{8.783347in}}%
\pgfpathlineto{\pgfqpoint{2.630248in}{8.785758in}}%
\pgfpathlineto{\pgfqpoint{2.627596in}{8.789129in}}%
\pgfpathlineto{\pgfqpoint{2.623047in}{8.794911in}}%
\pgfpathlineto{\pgfqpoint{2.621943in}{8.796315in}}%
\pgfpathlineto{\pgfqpoint{2.618498in}{8.800694in}}%
\pgfpathlineto{\pgfqpoint{2.613950in}{8.806476in}}%
\pgfpathlineto{\pgfqpoint{2.613638in}{8.806872in}}%
\pgfpathlineto{\pgfqpoint{2.609401in}{8.812258in}}%
\pgfpathlineto{\pgfqpoint{2.605333in}{8.817430in}}%
\pgfpathlineto{\pgfqpoint{2.604853in}{8.818040in}}%
\pgfpathlineto{\pgfqpoint{2.600304in}{8.823822in}}%
\pgfpathlineto{\pgfqpoint{2.597028in}{8.827987in}}%
\pgfpathlineto{\pgfqpoint{2.595755in}{8.829605in}}%
\pgfpathlineto{\pgfqpoint{2.591207in}{8.835387in}}%
\pgfpathlineto{\pgfqpoint{2.588723in}{8.838544in}}%
\pgfpathlineto{\pgfqpoint{2.586658in}{8.841169in}}%
\pgfpathlineto{\pgfqpoint{2.582109in}{8.846951in}}%
\pgfpathlineto{\pgfqpoint{2.580419in}{8.849101in}}%
\pgfpathlineto{\pgfqpoint{2.577561in}{8.852733in}}%
\pgfpathlineto{\pgfqpoint{2.573012in}{8.858516in}}%
\pgfpathlineto{\pgfqpoint{2.572114in}{8.859658in}}%
\pgfpathlineto{\pgfqpoint{2.568464in}{8.864298in}}%
\pgfpathlineto{\pgfqpoint{2.563915in}{8.870080in}}%
\pgfpathlineto{\pgfqpoint{2.563809in}{8.870215in}}%
\pgfpathlineto{\pgfqpoint{2.559366in}{8.875862in}}%
\pgfpathlineto{\pgfqpoint{2.555504in}{8.880772in}}%
\pgfpathlineto{\pgfqpoint{2.554818in}{8.881644in}}%
\pgfpathlineto{\pgfqpoint{2.550269in}{8.887427in}}%
\pgfpathlineto{\pgfqpoint{2.547199in}{8.891329in}}%
\pgfpathlineto{\pgfqpoint{2.545720in}{8.893209in}}%
\pgfpathlineto{\pgfqpoint{2.541172in}{8.898991in}}%
\pgfpathlineto{\pgfqpoint{2.538894in}{8.901886in}}%
\pgfpathlineto{\pgfqpoint{2.536623in}{8.904773in}}%
\pgfpathlineto{\pgfqpoint{2.532075in}{8.910555in}}%
\pgfpathlineto{\pgfqpoint{2.530589in}{8.912443in}}%
\pgfpathlineto{\pgfqpoint{2.527526in}{8.916338in}}%
\pgfpathlineto{\pgfqpoint{2.522977in}{8.922120in}}%
\pgfpathlineto{\pgfqpoint{2.522285in}{8.923000in}}%
\pgfpathlineto{\pgfqpoint{2.518429in}{8.927902in}}%
\pgfpathlineto{\pgfqpoint{2.513980in}{8.933557in}}%
\pgfpathlineto{\pgfqpoint{2.513880in}{8.933684in}}%
\pgfpathlineto{\pgfqpoint{2.509331in}{8.939466in}}%
\pgfpathlineto{\pgfqpoint{2.505675in}{8.944114in}}%
\pgfpathlineto{\pgfqpoint{2.504783in}{8.945249in}}%
\pgfpathlineto{\pgfqpoint{2.500234in}{8.951031in}}%
\pgfpathlineto{\pgfqpoint{2.497370in}{8.954672in}}%
\pgfpathlineto{\pgfqpoint{2.495685in}{8.956813in}}%
\pgfpathlineto{\pgfqpoint{2.491137in}{8.962595in}}%
\pgfpathlineto{\pgfqpoint{2.489065in}{8.965229in}}%
\pgfpathlineto{\pgfqpoint{2.486588in}{8.968377in}}%
\pgfpathlineto{\pgfqpoint{2.482040in}{8.974160in}}%
\pgfpathlineto{\pgfqpoint{2.480760in}{8.975786in}}%
\pgfpathlineto{\pgfqpoint{2.477491in}{8.979942in}}%
\pgfpathlineto{\pgfqpoint{2.472942in}{8.985724in}}%
\pgfpathlineto{\pgfqpoint{2.472456in}{8.986343in}}%
\pgfpathlineto{\pgfqpoint{2.468394in}{8.991506in}}%
\pgfpathlineto{\pgfqpoint{2.464151in}{8.996900in}}%
\pgfpathlineto{\pgfqpoint{2.463845in}{8.997288in}}%
\pgfpathlineto{\pgfqpoint{2.459296in}{9.003071in}}%
\pgfpathlineto{\pgfqpoint{2.455846in}{9.007457in}}%
\pgfpathlineto{\pgfqpoint{2.454748in}{9.008853in}}%
\pgfpathlineto{\pgfqpoint{2.450199in}{9.014635in}}%
\pgfpathlineto{\pgfqpoint{2.447541in}{9.018014in}}%
\pgfpathlineto{\pgfqpoint{2.445651in}{9.020417in}}%
\pgfpathlineto{\pgfqpoint{2.441102in}{9.026199in}}%
\pgfpathlineto{\pgfqpoint{2.439236in}{9.028571in}}%
\pgfpathlineto{\pgfqpoint{2.436553in}{9.031982in}}%
\pgfpathlineto{\pgfqpoint{2.432005in}{9.037764in}}%
\pgfpathlineto{\pgfqpoint{2.430931in}{9.039128in}}%
\pgfpathlineto{\pgfqpoint{2.427456in}{9.043546in}}%
\pgfpathlineto{\pgfqpoint{2.422907in}{9.049328in}}%
\pgfpathlineto{\pgfqpoint{2.422627in}{9.049685in}}%
\pgfpathlineto{\pgfqpoint{2.418359in}{9.055110in}}%
\pgfpathlineto{\pgfqpoint{2.414322in}{9.060242in}}%
\pgfpathlineto{\pgfqpoint{2.413810in}{9.060893in}}%
\pgfpathlineto{\pgfqpoint{2.409262in}{9.066675in}}%
\pgfpathlineto{\pgfqpoint{2.406017in}{9.070799in}}%
\pgfpathlineto{\pgfqpoint{2.404713in}{9.072457in}}%
\pgfpathlineto{\pgfqpoint{2.400164in}{9.078239in}}%
\pgfpathlineto{\pgfqpoint{2.397712in}{9.081356in}}%
\pgfpathlineto{\pgfqpoint{2.395616in}{9.084021in}}%
\pgfpathlineto{\pgfqpoint{2.391067in}{9.089804in}}%
\pgfpathlineto{\pgfqpoint{2.389407in}{9.091914in}}%
\pgfpathlineto{\pgfqpoint{2.386518in}{9.095586in}}%
\pgfpathlineto{\pgfqpoint{2.381970in}{9.101368in}}%
\pgfpathlineto{\pgfqpoint{2.381102in}{9.102471in}}%
\pgfpathlineto{\pgfqpoint{2.377421in}{9.107150in}}%
\pgfpathlineto{\pgfqpoint{2.372873in}{9.112932in}}%
\pgfpathlineto{\pgfqpoint{2.372798in}{9.113028in}}%
\pgfpathlineto{\pgfqpoint{2.368324in}{9.118715in}}%
\pgfpathlineto{\pgfqpoint{2.364493in}{9.123585in}}%
\pgfpathlineto{\pgfqpoint{2.363775in}{9.124497in}}%
\pgfpathlineto{\pgfqpoint{2.359227in}{9.130279in}}%
\pgfpathlineto{\pgfqpoint{2.356188in}{9.134142in}}%
\pgfpathlineto{\pgfqpoint{2.354678in}{9.136061in}}%
\pgfpathlineto{\pgfqpoint{2.350129in}{9.141843in}}%
\pgfpathlineto{\pgfqpoint{2.347883in}{9.144699in}}%
\pgfpathlineto{\pgfqpoint{2.345581in}{9.147626in}}%
\pgfpathlineto{\pgfqpoint{2.341032in}{9.153408in}}%
\pgfpathlineto{\pgfqpoint{2.339578in}{9.155256in}}%
\pgfpathlineto{\pgfqpoint{2.336483in}{9.159190in}}%
\pgfpathlineto{\pgfqpoint{2.331935in}{9.164972in}}%
\pgfpathlineto{\pgfqpoint{2.331273in}{9.165813in}}%
\pgfpathlineto{\pgfqpoint{2.327386in}{9.170754in}}%
\pgfpathlineto{\pgfqpoint{2.322969in}{9.176370in}}%
\pgfpathlineto{\pgfqpoint{2.322838in}{9.176537in}}%
\pgfpathlineto{\pgfqpoint{2.318289in}{9.182319in}}%
\pgfpathlineto{\pgfqpoint{2.314664in}{9.186927in}}%
\pgfpathlineto{\pgfqpoint{2.313740in}{9.188101in}}%
\pgfpathlineto{\pgfqpoint{2.309192in}{9.193883in}}%
\pgfpathlineto{\pgfqpoint{2.306359in}{9.197484in}}%
\pgfpathlineto{\pgfqpoint{2.304643in}{9.199665in}}%
\pgfpathlineto{\pgfqpoint{2.300094in}{9.205448in}}%
\pgfpathlineto{\pgfqpoint{2.298054in}{9.208041in}}%
\pgfpathlineto{\pgfqpoint{2.295546in}{9.211230in}}%
\pgfpathlineto{\pgfqpoint{2.290997in}{9.217012in}}%
\pgfpathlineto{\pgfqpoint{2.289749in}{9.218598in}}%
\pgfpathlineto{\pgfqpoint{2.286449in}{9.222794in}}%
\pgfpathlineto{\pgfqpoint{2.281900in}{9.228576in}}%
\pgfpathlineto{\pgfqpoint{2.281444in}{9.229156in}}%
\pgfpathlineto{\pgfqpoint{2.277351in}{9.234359in}}%
\pgfpathlineto{\pgfqpoint{2.273140in}{9.239713in}}%
\pgfpathlineto{\pgfqpoint{2.272803in}{9.240141in}}%
\pgfpathlineto{\pgfqpoint{2.268254in}{9.245923in}}%
\pgfpathlineto{\pgfqpoint{2.264835in}{9.250270in}}%
\pgfpathlineto{\pgfqpoint{2.263705in}{9.251705in}}%
\pgfpathlineto{\pgfqpoint{2.259157in}{9.257487in}}%
\pgfpathlineto{\pgfqpoint{2.256530in}{9.260827in}}%
\pgfpathlineto{\pgfqpoint{2.254608in}{9.263270in}}%
\pgfpathlineto{\pgfqpoint{2.250060in}{9.269052in}}%
\pgfpathlineto{\pgfqpoint{2.248225in}{9.271384in}}%
\pgfpathlineto{\pgfqpoint{2.245511in}{9.274834in}}%
\pgfpathlineto{\pgfqpoint{2.240962in}{9.280616in}}%
\pgfpathlineto{\pgfqpoint{2.239920in}{9.281941in}}%
\pgfpathlineto{\pgfqpoint{2.236414in}{9.286398in}}%
\pgfpathlineto{\pgfqpoint{2.231865in}{9.292181in}}%
\pgfpathlineto{\pgfqpoint{2.231615in}{9.292498in}}%
\pgfpathlineto{\pgfqpoint{2.227316in}{9.297963in}}%
\pgfpathlineto{\pgfqpoint{2.223311in}{9.303055in}}%
\pgfpathlineto{\pgfqpoint{2.222768in}{9.303745in}}%
\pgfpathlineto{\pgfqpoint{2.218219in}{9.309527in}}%
\pgfpathlineto{\pgfqpoint{2.215006in}{9.313612in}}%
\pgfpathlineto{\pgfqpoint{2.213671in}{9.315309in}}%
\pgfpathlineto{\pgfqpoint{2.209122in}{9.321092in}}%
\pgfpathlineto{\pgfqpoint{2.206701in}{9.324169in}}%
\pgfpathlineto{\pgfqpoint{2.204573in}{9.326874in}}%
\pgfpathlineto{\pgfqpoint{2.200025in}{9.332656in}}%
\pgfpathlineto{\pgfqpoint{2.198396in}{9.334726in}}%
\pgfpathlineto{\pgfqpoint{2.195476in}{9.338438in}}%
\pgfpathlineto{\pgfqpoint{2.190927in}{9.344221in}}%
\pgfpathlineto{\pgfqpoint{2.190091in}{9.345283in}}%
\pgfpathlineto{\pgfqpoint{2.186379in}{9.350003in}}%
\pgfpathlineto{\pgfqpoint{2.181830in}{9.355785in}}%
\pgfpathlineto{\pgfqpoint{2.181786in}{9.355840in}}%
\pgfpathlineto{\pgfqpoint{2.177281in}{9.361567in}}%
\pgfpathlineto{\pgfqpoint{2.173482in}{9.366398in}}%
\pgfpathlineto{\pgfqpoint{2.172733in}{9.367349in}}%
\pgfpathlineto{\pgfqpoint{2.168184in}{9.373132in}}%
\pgfpathlineto{\pgfqpoint{2.165177in}{9.376955in}}%
\pgfpathlineto{\pgfqpoint{2.163636in}{9.378914in}}%
\pgfpathlineto{\pgfqpoint{2.159087in}{9.384696in}}%
\pgfpathlineto{\pgfqpoint{2.156872in}{9.387512in}}%
\pgfpathlineto{\pgfqpoint{2.154538in}{9.390478in}}%
\pgfpathlineto{\pgfqpoint{2.149990in}{9.396260in}}%
\pgfpathlineto{\pgfqpoint{2.148567in}{9.398069in}}%
\pgfpathlineto{\pgfqpoint{2.145441in}{9.402043in}}%
\pgfpathlineto{\pgfqpoint{2.140892in}{9.407825in}}%
\pgfpathlineto{\pgfqpoint{2.140262in}{9.408626in}}%
\pgfpathlineto{\pgfqpoint{2.136344in}{9.413607in}}%
\pgfpathlineto{\pgfqpoint{2.131957in}{9.419183in}}%
\pgfpathlineto{\pgfqpoint{2.131795in}{9.419389in}}%
\pgfpathlineto{\pgfqpoint{2.127247in}{9.425171in}}%
\pgfpathlineto{\pgfqpoint{2.123653in}{9.429740in}}%
\pgfpathlineto{\pgfqpoint{2.122698in}{9.430954in}}%
\pgfpathlineto{\pgfqpoint{2.118149in}{9.436736in}}%
\pgfpathlineto{\pgfqpoint{2.115348in}{9.440297in}}%
\pgfpathlineto{\pgfqpoint{2.113601in}{9.442518in}}%
\pgfpathlineto{\pgfqpoint{2.109052in}{9.448300in}}%
\pgfpathlineto{\pgfqpoint{2.107043in}{9.450854in}}%
\pgfpathlineto{\pgfqpoint{2.104503in}{9.454082in}}%
\pgfpathlineto{\pgfqpoint{2.099955in}{9.459865in}}%
\pgfpathlineto{\pgfqpoint{2.098738in}{9.461411in}}%
\pgfpathlineto{\pgfqpoint{2.095406in}{9.465647in}}%
\pgfpathlineto{\pgfqpoint{2.090858in}{9.471429in}}%
\pgfpathlineto{\pgfqpoint{2.090433in}{9.471968in}}%
\pgfpathlineto{\pgfqpoint{2.086309in}{9.477211in}}%
\pgfpathlineto{\pgfqpoint{2.082128in}{9.482525in}}%
\pgfpathlineto{\pgfqpoint{2.081760in}{9.482993in}}%
\pgfpathlineto{\pgfqpoint{2.077212in}{9.488776in}}%
\pgfpathlineto{\pgfqpoint{2.073824in}{9.493082in}}%
\pgfpathlineto{\pgfqpoint{2.072663in}{9.494558in}}%
\pgfpathlineto{\pgfqpoint{2.068114in}{9.500340in}}%
\pgfpathlineto{\pgfqpoint{2.065519in}{9.503639in}}%
\pgfpathlineto{\pgfqpoint{2.063566in}{9.506122in}}%
\pgfpathlineto{\pgfqpoint{2.059017in}{9.511904in}}%
\pgfpathlineto{\pgfqpoint{2.057214in}{9.514197in}}%
\pgfpathlineto{\pgfqpoint{2.054469in}{9.517687in}}%
\pgfpathlineto{\pgfqpoint{2.049920in}{9.523469in}}%
\pgfpathlineto{\pgfqpoint{2.048909in}{9.524754in}}%
\pgfpathlineto{\pgfqpoint{2.045371in}{9.529251in}}%
\pgfpathlineto{\pgfqpoint{2.040823in}{9.535033in}}%
\pgfpathlineto{\pgfqpoint{2.040604in}{9.535311in}}%
\pgfpathlineto{\pgfqpoint{2.036274in}{9.540815in}}%
\pgfpathlineto{\pgfqpoint{2.032299in}{9.545868in}}%
\pgfpathlineto{\pgfqpoint{2.031725in}{9.546598in}}%
\pgfpathlineto{\pgfqpoint{2.027177in}{9.552380in}}%
\pgfpathlineto{\pgfqpoint{2.023995in}{9.556425in}}%
\pgfpathlineto{\pgfqpoint{2.022628in}{9.558162in}}%
\pgfpathlineto{\pgfqpoint{2.018079in}{9.563944in}}%
\pgfpathlineto{\pgfqpoint{2.015690in}{9.566982in}}%
\pgfpathlineto{\pgfqpoint{2.013531in}{9.569726in}}%
\pgfpathlineto{\pgfqpoint{2.008982in}{9.575509in}}%
\pgfpathlineto{\pgfqpoint{2.007385in}{9.577539in}}%
\pgfpathlineto{\pgfqpoint{2.004434in}{9.581291in}}%
\pgfpathlineto{\pgfqpoint{1.999885in}{9.587073in}}%
\pgfpathlineto{\pgfqpoint{1.999080in}{9.588096in}}%
\pgfpathlineto{\pgfqpoint{1.995336in}{9.592855in}}%
\pgfpathlineto{\pgfqpoint{1.990788in}{9.598637in}}%
\pgfpathlineto{\pgfqpoint{1.990775in}{9.598653in}}%
\pgfpathlineto{\pgfqpoint{1.986239in}{9.604420in}}%
\pgfpathlineto{\pgfqpoint{1.982470in}{9.609210in}}%
\pgfpathlineto{\pgfqpoint{1.981690in}{9.610202in}}%
\pgfpathlineto{\pgfqpoint{1.977142in}{9.615984in}}%
\pgfpathlineto{\pgfqpoint{1.974166in}{9.619767in}}%
\pgfpathlineto{\pgfqpoint{1.972593in}{9.621766in}}%
\pgfpathlineto{\pgfqpoint{1.968045in}{9.627548in}}%
\pgfpathlineto{\pgfqpoint{1.965861in}{9.630324in}}%
\pgfpathlineto{\pgfqpoint{1.963496in}{9.633331in}}%
\pgfpathlineto{\pgfqpoint{1.958947in}{9.639113in}}%
\pgfpathlineto{\pgfqpoint{1.957556in}{9.640881in}}%
\pgfpathlineto{\pgfqpoint{1.954399in}{9.644895in}}%
\pgfpathlineto{\pgfqpoint{1.949850in}{9.650677in}}%
\pgfpathlineto{\pgfqpoint{1.949251in}{9.651439in}}%
\pgfpathlineto{\pgfqpoint{1.945301in}{9.656459in}}%
\pgfpathlineto{\pgfqpoint{1.940946in}{9.661996in}}%
\pgfpathlineto{\pgfqpoint{1.940753in}{9.662242in}}%
\pgfpathlineto{\pgfqpoint{1.936204in}{9.668024in}}%
\pgfpathlineto{\pgfqpoint{1.932641in}{9.672553in}}%
\pgfpathlineto{\pgfqpoint{1.931656in}{9.673806in}}%
\pgfpathlineto{\pgfqpoint{1.927107in}{9.679588in}}%
\pgfpathlineto{\pgfqpoint{1.924337in}{9.683110in}}%
\pgfpathlineto{\pgfqpoint{1.922558in}{9.685370in}}%
\pgfpathlineto{\pgfqpoint{1.918010in}{9.691153in}}%
\pgfpathlineto{\pgfqpoint{1.916032in}{9.693667in}}%
\pgfpathlineto{\pgfqpoint{1.913461in}{9.696935in}}%
\pgfpathlineto{\pgfqpoint{1.908912in}{9.702717in}}%
\pgfpathlineto{\pgfqpoint{1.907727in}{9.704224in}}%
\pgfpathlineto{\pgfqpoint{1.904364in}{9.708499in}}%
\pgfpathlineto{\pgfqpoint{1.899815in}{9.714281in}}%
\pgfpathlineto{\pgfqpoint{1.899422in}{9.714781in}}%
\pgfpathlineto{\pgfqpoint{1.895267in}{9.720064in}}%
\pgfpathlineto{\pgfqpoint{1.891117in}{9.725338in}}%
\pgfpathlineto{\pgfqpoint{1.890718in}{9.725846in}}%
\pgfpathlineto{\pgfqpoint{1.886169in}{9.731628in}}%
\pgfpathlineto{\pgfqpoint{1.882812in}{9.735895in}}%
\pgfpathlineto{\pgfqpoint{1.881621in}{9.737410in}}%
\pgfpathlineto{\pgfqpoint{1.877072in}{9.743192in}}%
\pgfpathlineto{\pgfqpoint{1.874508in}{9.746452in}}%
\pgfpathlineto{\pgfqpoint{1.872523in}{9.748975in}}%
\pgfpathlineto{\pgfqpoint{1.867975in}{9.754757in}}%
\pgfpathlineto{\pgfqpoint{1.866203in}{9.757009in}}%
\pgfpathlineto{\pgfqpoint{1.863426in}{9.760539in}}%
\pgfpathlineto{\pgfqpoint{1.858877in}{9.766321in}}%
\pgfpathlineto{\pgfqpoint{1.857898in}{9.767566in}}%
\pgfpathlineto{\pgfqpoint{1.854329in}{9.772103in}}%
\pgfpathlineto{\pgfqpoint{1.849780in}{9.777886in}}%
\pgfpathlineto{\pgfqpoint{1.849593in}{9.778123in}}%
\pgfpathlineto{\pgfqpoint{1.845232in}{9.783668in}}%
\pgfpathlineto{\pgfqpoint{1.841288in}{9.788681in}}%
\pgfpathlineto{\pgfqpoint{1.840683in}{9.789450in}}%
\pgfpathlineto{\pgfqpoint{1.836134in}{9.795232in}}%
\pgfpathlineto{\pgfqpoint{1.832983in}{9.799238in}}%
\pgfpathlineto{\pgfqpoint{1.831586in}{9.801014in}}%
\pgfpathlineto{\pgfqpoint{1.827037in}{9.806797in}}%
\pgfpathlineto{\pgfqpoint{1.824679in}{9.809795in}}%
\pgfpathlineto{\pgfqpoint{1.822488in}{9.812579in}}%
\pgfpathlineto{\pgfqpoint{1.817940in}{9.818361in}}%
\pgfpathlineto{\pgfqpoint{1.816374in}{9.820352in}}%
\pgfpathlineto{\pgfqpoint{1.813391in}{9.824143in}}%
\pgfpathlineto{\pgfqpoint{1.808843in}{9.829925in}}%
\pgfpathlineto{\pgfqpoint{1.808069in}{9.830909in}}%
\pgfpathlineto{\pgfqpoint{1.804294in}{9.835708in}}%
\pgfpathlineto{\pgfqpoint{1.799764in}{9.841466in}}%
\pgfpathlineto{\pgfqpoint{1.799745in}{9.841490in}}%
\pgfpathlineto{\pgfqpoint{1.795197in}{9.847272in}}%
\pgfpathlineto{\pgfqpoint{1.791459in}{9.852023in}}%
\pgfpathlineto{\pgfqpoint{1.790648in}{9.853054in}}%
\pgfpathlineto{\pgfqpoint{1.786099in}{9.858836in}}%
\pgfpathlineto{\pgfqpoint{1.783154in}{9.862580in}}%
\pgfpathlineto{\pgfqpoint{1.781551in}{9.864619in}}%
\pgfpathlineto{\pgfqpoint{1.777002in}{9.870401in}}%
\pgfpathlineto{\pgfqpoint{1.774850in}{9.873137in}}%
\pgfpathlineto{\pgfqpoint{1.772454in}{9.876183in}}%
\pgfpathlineto{\pgfqpoint{1.767905in}{9.881965in}}%
\pgfpathlineto{\pgfqpoint{1.766545in}{9.883694in}}%
\pgfpathlineto{\pgfqpoint{1.763356in}{9.887747in}}%
\pgfpathlineto{\pgfqpoint{1.758808in}{9.893530in}}%
\pgfpathlineto{\pgfqpoint{1.758240in}{9.894251in}}%
\pgfpathlineto{\pgfqpoint{1.754259in}{9.899312in}}%
\pgfpathlineto{\pgfqpoint{1.749935in}{9.904808in}}%
\pgfpathlineto{\pgfqpoint{1.749710in}{9.905094in}}%
\pgfpathlineto{\pgfqpoint{1.745162in}{9.910876in}}%
\pgfpathlineto{\pgfqpoint{1.741630in}{9.915365in}}%
\pgfpathlineto{\pgfqpoint{1.740613in}{9.916658in}}%
\pgfpathlineto{\pgfqpoint{1.736065in}{9.922441in}}%
\pgfpathlineto{\pgfqpoint{1.733325in}{9.925923in}}%
\pgfpathlineto{\pgfqpoint{1.731516in}{9.928223in}}%
\pgfpathlineto{\pgfqpoint{1.726967in}{9.934005in}}%
\pgfpathlineto{\pgfqpoint{1.725021in}{9.936480in}}%
\pgfpathlineto{\pgfqpoint{1.722419in}{9.939787in}}%
\pgfpathlineto{\pgfqpoint{1.717870in}{9.945569in}}%
\pgfpathlineto{\pgfqpoint{1.716716in}{9.947037in}}%
\pgfpathlineto{\pgfqpoint{1.713321in}{9.951352in}}%
\pgfpathlineto{\pgfqpoint{1.708773in}{9.957134in}}%
\pgfpathlineto{\pgfqpoint{1.708411in}{9.957594in}}%
\pgfpathlineto{\pgfqpoint{1.704224in}{9.962916in}}%
\pgfpathlineto{\pgfqpoint{1.700106in}{9.968151in}}%
\pgfpathlineto{\pgfqpoint{1.699675in}{9.968698in}}%
\pgfpathlineto{\pgfqpoint{1.695127in}{9.974480in}}%
\pgfpathlineto{\pgfqpoint{1.691801in}{9.978708in}}%
\pgfpathlineto{\pgfqpoint{1.690578in}{9.980263in}}%
\pgfpathlineto{\pgfqpoint{1.686030in}{9.986045in}}%
\pgfpathlineto{\pgfqpoint{1.683496in}{9.989265in}}%
\pgfpathlineto{\pgfqpoint{1.681481in}{9.991827in}}%
\pgfpathlineto{\pgfqpoint{1.676932in}{9.997609in}}%
\pgfpathlineto{\pgfqpoint{1.675192in}{9.999822in}}%
\pgfpathlineto{\pgfqpoint{1.672384in}{10.003391in}}%
\pgfpathlineto{\pgfqpoint{1.667835in}{10.009174in}}%
\pgfpathlineto{\pgfqpoint{1.666887in}{10.010379in}}%
\pgfpathlineto{\pgfqpoint{1.663286in}{10.014956in}}%
\pgfpathlineto{\pgfqpoint{1.658738in}{10.020738in}}%
\pgfpathlineto{\pgfqpoint{1.658582in}{10.020936in}}%
\pgfpathlineto{\pgfqpoint{1.654189in}{10.026520in}}%
\pgfpathlineto{\pgfqpoint{1.650277in}{10.031493in}}%
\pgfpathlineto{\pgfqpoint{1.649641in}{10.032302in}}%
\pgfpathlineto{\pgfqpoint{1.645092in}{10.038085in}}%
\pgfpathlineto{\pgfqpoint{1.641972in}{10.042050in}}%
\pgfpathlineto{\pgfqpoint{1.640543in}{10.043867in}}%
\pgfpathlineto{\pgfqpoint{1.635995in}{10.049649in}}%
\pgfpathlineto{\pgfqpoint{1.633667in}{10.052607in}}%
\pgfpathlineto{\pgfqpoint{1.631446in}{10.055431in}}%
\pgfpathlineto{\pgfqpoint{1.626897in}{10.061213in}}%
\pgfpathlineto{\pgfqpoint{1.625363in}{10.063165in}}%
\pgfpathlineto{\pgfqpoint{1.622349in}{10.066996in}}%
\pgfpathlineto{\pgfqpoint{1.617800in}{10.072778in}}%
\pgfpathlineto{\pgfqpoint{1.617058in}{10.073722in}}%
\pgfpathlineto{\pgfqpoint{1.613252in}{10.078560in}}%
\pgfpathlineto{\pgfqpoint{1.608753in}{10.084279in}}%
\pgfpathlineto{\pgfqpoint{1.608703in}{10.084342in}}%
\pgfpathlineto{\pgfqpoint{1.604154in}{10.090124in}}%
\pgfpathlineto{\pgfqpoint{1.600448in}{10.094836in}}%
\pgfpathlineto{\pgfqpoint{1.599606in}{10.095907in}}%
\pgfpathlineto{\pgfqpoint{1.595057in}{10.101689in}}%
\pgfpathlineto{\pgfqpoint{1.592143in}{10.105393in}}%
\pgfpathlineto{\pgfqpoint{1.590508in}{10.107471in}}%
\pgfpathlineto{\pgfqpoint{1.585960in}{10.113253in}}%
\pgfpathlineto{\pgfqpoint{1.583838in}{10.115950in}}%
\pgfpathlineto{\pgfqpoint{1.581411in}{10.119035in}}%
\pgfpathlineto{\pgfqpoint{1.576863in}{10.124818in}}%
\pgfpathlineto{\pgfqpoint{1.575533in}{10.126507in}}%
\pgfpathlineto{\pgfqpoint{1.572314in}{10.130600in}}%
\pgfpathlineto{\pgfqpoint{1.567765in}{10.136382in}}%
\pgfpathlineto{\pgfqpoint{1.567229in}{10.137064in}}%
\pgfpathlineto{\pgfqpoint{1.563217in}{10.142164in}}%
\pgfpathlineto{\pgfqpoint{1.558924in}{10.147621in}}%
\pgfpathlineto{\pgfqpoint{1.558668in}{10.147946in}}%
\pgfpathlineto{\pgfqpoint{1.554119in}{10.153729in}}%
\pgfpathlineto{\pgfqpoint{1.550619in}{10.158178in}}%
\pgfpathlineto{\pgfqpoint{1.549571in}{10.159511in}}%
\pgfpathlineto{\pgfqpoint{1.545022in}{10.165293in}}%
\pgfpathlineto{\pgfqpoint{1.542314in}{10.168735in}}%
\pgfpathlineto{\pgfqpoint{1.540473in}{10.171075in}}%
\pgfpathlineto{\pgfqpoint{1.535925in}{10.176857in}}%
\pgfpathlineto{\pgfqpoint{1.534009in}{10.179292in}}%
\pgfpathlineto{\pgfqpoint{1.531376in}{10.182640in}}%
\pgfpathlineto{\pgfqpoint{1.526828in}{10.188422in}}%
\pgfpathlineto{\pgfqpoint{1.525704in}{10.189849in}}%
\pgfpathlineto{\pgfqpoint{1.522279in}{10.194204in}}%
\pgfpathlineto{\pgfqpoint{1.517730in}{10.199986in}}%
\pgfpathlineto{\pgfqpoint{1.517400in}{10.200407in}}%
\pgfpathlineto{\pgfqpoint{1.513182in}{10.205768in}}%
\pgfpathlineto{\pgfqpoint{1.509095in}{10.210964in}}%
\pgfpathlineto{\pgfqpoint{1.508633in}{10.211551in}}%
\pgfpathlineto{\pgfqpoint{1.504084in}{10.217333in}}%
\pgfpathlineto{\pgfqpoint{1.500790in}{10.221521in}}%
\pgfpathlineto{\pgfqpoint{1.499536in}{10.223115in}}%
\pgfpathlineto{\pgfqpoint{1.494987in}{10.228897in}}%
\pgfpathlineto{\pgfqpoint{1.492485in}{10.232078in}}%
\pgfpathlineto{\pgfqpoint{1.490439in}{10.234679in}}%
\pgfpathlineto{\pgfqpoint{1.485890in}{10.240462in}}%
\pgfpathlineto{\pgfqpoint{1.484180in}{10.242635in}}%
\pgfpathlineto{\pgfqpoint{1.481341in}{10.246244in}}%
\pgfpathlineto{\pgfqpoint{1.476793in}{10.252026in}}%
\pgfpathlineto{\pgfqpoint{1.475875in}{10.253192in}}%
\pgfpathlineto{\pgfqpoint{1.472244in}{10.257808in}}%
\pgfpathlineto{\pgfqpoint{1.467695in}{10.263590in}}%
\pgfpathlineto{\pgfqpoint{1.467571in}{10.263749in}}%
\pgfpathlineto{\pgfqpoint{1.463147in}{10.269373in}}%
\pgfpathlineto{\pgfqpoint{1.459266in}{10.274306in}}%
\pgfpathlineto{\pgfqpoint{1.458598in}{10.275155in}}%
\pgfpathlineto{\pgfqpoint{1.454050in}{10.280937in}}%
\pgfpathlineto{\pgfqpoint{1.450961in}{10.284863in}}%
\pgfpathlineto{\pgfqpoint{1.449501in}{10.286719in}}%
\pgfpathlineto{\pgfqpoint{1.444952in}{10.292501in}}%
\pgfpathlineto{\pgfqpoint{1.442656in}{10.295420in}}%
\pgfpathlineto{\pgfqpoint{1.440404in}{10.298284in}}%
\pgfpathlineto{\pgfqpoint{1.435855in}{10.304066in}}%
\pgfpathlineto{\pgfqpoint{1.434351in}{10.305977in}}%
\pgfpathlineto{\pgfqpoint{1.431306in}{10.309848in}}%
\pgfpathlineto{\pgfqpoint{1.426758in}{10.315630in}}%
\pgfpathlineto{\pgfqpoint{1.426046in}{10.316534in}}%
\pgfpathlineto{\pgfqpoint{1.422209in}{10.321412in}}%
\pgfpathlineto{\pgfqpoint{1.417742in}{10.327091in}}%
\pgfpathlineto{\pgfqpoint{1.417661in}{10.327195in}}%
\pgfpathlineto{\pgfqpoint{1.413112in}{10.332977in}}%
\pgfpathlineto{\pgfqpoint{1.409437in}{10.337649in}}%
\pgfpathlineto{\pgfqpoint{1.408563in}{10.338759in}}%
\pgfpathlineto{\pgfqpoint{1.404015in}{10.344541in}}%
\pgfpathlineto{\pgfqpoint{1.401132in}{10.348206in}}%
\pgfpathlineto{\pgfqpoint{1.399466in}{10.350323in}}%
\pgfpathlineto{\pgfqpoint{1.394917in}{10.356106in}}%
\pgfpathlineto{\pgfqpoint{1.392827in}{10.358763in}}%
\pgfpathlineto{\pgfqpoint{1.390369in}{10.361888in}}%
\pgfpathlineto{\pgfqpoint{1.385820in}{10.367670in}}%
\pgfpathlineto{\pgfqpoint{1.384522in}{10.369320in}}%
\pgfpathlineto{\pgfqpoint{1.381271in}{10.373452in}}%
\pgfpathlineto{\pgfqpoint{1.376723in}{10.379234in}}%
\pgfpathlineto{\pgfqpoint{1.376217in}{10.379877in}}%
\pgfpathlineto{\pgfqpoint{1.372174in}{10.385017in}}%
\pgfpathlineto{\pgfqpoint{1.367913in}{10.390434in}}%
\pgfpathlineto{\pgfqpoint{1.367626in}{10.390799in}}%
\pgfpathlineto{\pgfqpoint{1.363077in}{10.396581in}}%
\pgfpathlineto{\pgfqpoint{1.359608in}{10.400991in}}%
\pgfpathlineto{\pgfqpoint{1.358528in}{10.402363in}}%
\pgfpathlineto{\pgfqpoint{1.353980in}{10.408145in}}%
\pgfpathlineto{\pgfqpoint{1.351303in}{10.411548in}}%
\pgfpathlineto{\pgfqpoint{1.349431in}{10.413928in}}%
\pgfpathlineto{\pgfqpoint{1.344882in}{10.419710in}}%
\pgfpathlineto{\pgfqpoint{1.342998in}{10.422105in}}%
\pgfpathlineto{\pgfqpoint{1.340334in}{10.425492in}}%
\pgfpathlineto{\pgfqpoint{1.335785in}{10.431274in}}%
\pgfpathlineto{\pgfqpoint{1.334693in}{10.432662in}}%
\pgfpathlineto{\pgfqpoint{1.331237in}{10.437056in}}%
\pgfpathlineto{\pgfqpoint{1.326688in}{10.442839in}}%
\pgfpathlineto{\pgfqpoint{1.326388in}{10.443219in}}%
\pgfpathlineto{\pgfqpoint{1.322139in}{10.448621in}}%
\pgfpathlineto{\pgfqpoint{1.318084in}{10.453776in}}%
\pgfpathlineto{\pgfqpoint{1.317591in}{10.454403in}}%
\pgfpathlineto{\pgfqpoint{1.313042in}{10.460185in}}%
\pgfpathlineto{\pgfqpoint{1.309779in}{10.464333in}}%
\pgfpathlineto{\pgfqpoint{1.308493in}{10.465967in}}%
\pgfpathlineto{\pgfqpoint{1.303945in}{10.471750in}}%
\pgfpathlineto{\pgfqpoint{1.301474in}{10.474891in}}%
\pgfpathlineto{\pgfqpoint{1.299396in}{10.477532in}}%
\pgfpathlineto{\pgfqpoint{1.294848in}{10.483314in}}%
\pgfpathlineto{\pgfqpoint{1.293169in}{10.485448in}}%
\pgfpathlineto{\pgfqpoint{1.290299in}{10.489096in}}%
\pgfpathlineto{\pgfqpoint{1.285750in}{10.494878in}}%
\pgfpathlineto{\pgfqpoint{1.284864in}{10.496005in}}%
\pgfpathlineto{\pgfqpoint{1.281202in}{10.500661in}}%
\pgfpathlineto{\pgfqpoint{1.276653in}{10.506443in}}%
\pgfpathlineto{\pgfqpoint{1.276559in}{10.506562in}}%
\pgfpathlineto{\pgfqpoint{1.272104in}{10.512225in}}%
\pgfpathlineto{\pgfqpoint{1.268255in}{10.517119in}}%
\pgfpathlineto{\pgfqpoint{1.267556in}{10.518007in}}%
\pgfpathlineto{\pgfqpoint{1.263007in}{10.523789in}}%
\pgfpathlineto{\pgfqpoint{1.259950in}{10.527676in}}%
\pgfpathlineto{\pgfqpoint{1.258459in}{10.529572in}}%
\pgfpathlineto{\pgfqpoint{1.253910in}{10.535354in}}%
\pgfpathlineto{\pgfqpoint{1.251645in}{10.538233in}}%
\pgfpathlineto{\pgfqpoint{1.249361in}{10.541136in}}%
\pgfpathlineto{\pgfqpoint{1.244813in}{10.546918in}}%
\pgfpathlineto{\pgfqpoint{1.243340in}{10.548790in}}%
\pgfpathlineto{\pgfqpoint{1.240264in}{10.552700in}}%
\pgfpathlineto{\pgfqpoint{1.235715in}{10.558483in}}%
\pgfpathlineto{\pgfqpoint{1.235035in}{10.559347in}}%
\pgfpathlineto{\pgfqpoint{1.231167in}{10.564265in}}%
\pgfpathlineto{\pgfqpoint{1.226730in}{10.569904in}}%
\pgfpathlineto{\pgfqpoint{1.226618in}{10.570047in}}%
\pgfpathlineto{\pgfqpoint{1.222069in}{10.575829in}}%
\pgfpathlineto{\pgfqpoint{1.218426in}{10.580461in}}%
\pgfpathlineto{\pgfqpoint{1.217521in}{10.581611in}}%
\pgfpathlineto{\pgfqpoint{1.212972in}{10.587394in}}%
\pgfpathlineto{\pgfqpoint{1.210121in}{10.591018in}}%
\pgfpathlineto{\pgfqpoint{1.208424in}{10.593176in}}%
\pgfpathlineto{\pgfqpoint{1.203875in}{10.598958in}}%
\pgfpathlineto{\pgfqpoint{1.201816in}{10.601575in}}%
\pgfpathlineto{\pgfqpoint{1.199326in}{10.604740in}}%
\pgfpathlineto{\pgfqpoint{1.194778in}{10.610522in}}%
\pgfpathlineto{\pgfqpoint{1.193511in}{10.612133in}}%
\pgfpathlineto{\pgfqpoint{1.190229in}{10.616305in}}%
\pgfpathlineto{\pgfqpoint{1.185680in}{10.622087in}}%
\pgfpathlineto{\pgfqpoint{1.185206in}{10.622690in}}%
\pgfpathlineto{\pgfqpoint{1.181132in}{10.627869in}}%
\pgfpathlineto{\pgfqpoint{1.176901in}{10.633247in}}%
\pgfpathlineto{\pgfqpoint{1.176583in}{10.633651in}}%
\pgfpathlineto{\pgfqpoint{1.172035in}{10.639433in}}%
\pgfpathlineto{\pgfqpoint{1.168597in}{10.643804in}}%
\pgfpathlineto{\pgfqpoint{1.167486in}{10.645216in}}%
\pgfpathlineto{\pgfqpoint{1.162937in}{10.650998in}}%
\pgfpathlineto{\pgfqpoint{1.160292in}{10.654361in}}%
\pgfpathlineto{\pgfqpoint{1.158389in}{10.656780in}}%
\pgfpathlineto{\pgfqpoint{1.153840in}{10.662562in}}%
\pgfpathlineto{\pgfqpoint{1.151987in}{10.664918in}}%
\pgfpathlineto{\pgfqpoint{1.149291in}{10.668344in}}%
\pgfpathlineto{\pgfqpoint{1.144743in}{10.674127in}}%
\pgfpathlineto{\pgfqpoint{1.143682in}{10.675475in}}%
\pgfpathlineto{\pgfqpoint{1.140194in}{10.679909in}}%
\pgfpathlineto{\pgfqpoint{1.135646in}{10.685691in}}%
\pgfpathlineto{\pgfqpoint{1.135377in}{10.686032in}}%
\pgfpathlineto{\pgfqpoint{1.131097in}{10.691473in}}%
\pgfpathlineto{\pgfqpoint{1.127072in}{10.696589in}}%
\pgfpathlineto{\pgfqpoint{1.126548in}{10.697255in}}%
\pgfpathlineto{\pgfqpoint{1.122000in}{10.703038in}}%
\pgfpathlineto{\pgfqpoint{1.118768in}{10.707146in}}%
\pgfpathlineto{\pgfqpoint{1.117451in}{10.708820in}}%
\pgfpathlineto{\pgfqpoint{1.112902in}{10.714602in}}%
\pgfpathlineto{\pgfqpoint{1.110463in}{10.717703in}}%
\pgfpathlineto{\pgfqpoint{1.108354in}{10.720384in}}%
\pgfpathlineto{\pgfqpoint{1.103805in}{10.726166in}}%
\pgfpathlineto{\pgfqpoint{1.102158in}{10.728260in}}%
\pgfpathlineto{\pgfqpoint{1.099257in}{10.731949in}}%
\pgfpathlineto{\pgfqpoint{1.094708in}{10.737731in}}%
\pgfpathlineto{\pgfqpoint{1.093853in}{10.738817in}}%
\pgfpathlineto{\pgfqpoint{1.090159in}{10.743513in}}%
\pgfpathlineto{\pgfqpoint{1.085611in}{10.749295in}}%
\pgfpathlineto{\pgfqpoint{1.085548in}{10.749375in}}%
\pgfpathlineto{\pgfqpoint{1.081062in}{10.755077in}}%
\pgfpathlineto{\pgfqpoint{1.077243in}{10.759932in}}%
\pgfpathlineto{\pgfqpoint{1.076513in}{10.760860in}}%
\pgfpathlineto{\pgfqpoint{1.071965in}{10.766642in}}%
\pgfpathlineto{\pgfqpoint{1.068939in}{10.770489in}}%
\pgfpathlineto{\pgfqpoint{1.067416in}{10.772424in}}%
\pgfpathlineto{\pgfqpoint{1.062867in}{10.778206in}}%
\pgfpathlineto{\pgfqpoint{1.060634in}{10.781046in}}%
\pgfpathlineto{\pgfqpoint{1.058319in}{10.783988in}}%
\pgfpathlineto{\pgfqpoint{1.053770in}{10.789771in}}%
\pgfpathlineto{\pgfqpoint{1.052329in}{10.791603in}}%
\pgfpathlineto{\pgfqpoint{1.049222in}{10.795553in}}%
\pgfpathlineto{\pgfqpoint{1.044673in}{10.801335in}}%
\pgfpathlineto{\pgfqpoint{1.044024in}{10.802160in}}%
\pgfpathlineto{\pgfqpoint{1.040124in}{10.807117in}}%
\pgfpathlineto{\pgfqpoint{1.035719in}{10.812717in}}%
\pgfpathlineto{\pgfqpoint{1.035576in}{10.812899in}}%
\pgfpathlineto{\pgfqpoint{1.031027in}{10.818682in}}%
\pgfpathlineto{\pgfqpoint{1.027414in}{10.823274in}}%
\pgfpathlineto{\pgfqpoint{1.026478in}{10.824464in}}%
\pgfpathlineto{\pgfqpoint{1.021930in}{10.830246in}}%
\pgfpathlineto{\pgfqpoint{1.019110in}{10.833831in}}%
\pgfpathlineto{\pgfqpoint{1.017381in}{10.836028in}}%
\pgfpathlineto{\pgfqpoint{1.012833in}{10.841810in}}%
\pgfpathlineto{\pgfqpoint{1.010805in}{10.844388in}}%
\pgfpathlineto{\pgfqpoint{1.008284in}{10.847593in}}%
\pgfpathlineto{\pgfqpoint{1.003735in}{10.853375in}}%
\pgfpathlineto{\pgfqpoint{1.002500in}{10.854945in}}%
\pgfpathlineto{\pgfqpoint{0.999187in}{10.859157in}}%
\pgfpathlineto{\pgfqpoint{0.994638in}{10.864939in}}%
\pgfpathlineto{\pgfqpoint{0.994195in}{10.865502in}}%
\pgfpathlineto{\pgfqpoint{0.990089in}{10.870721in}}%
\pgfpathlineto{\pgfqpoint{0.985890in}{10.876059in}}%
\pgfpathlineto{\pgfqpoint{0.985541in}{10.876504in}}%
\pgfpathlineto{\pgfqpoint{0.980992in}{10.882286in}}%
\pgfpathlineto{\pgfqpoint{0.977585in}{10.886617in}}%
\pgfpathlineto{\pgfqpoint{0.976444in}{10.888068in}}%
\pgfpathlineto{\pgfqpoint{0.971895in}{10.893850in}}%
\pgfpathlineto{\pgfqpoint{0.969281in}{10.897174in}}%
\pgfpathlineto{\pgfqpoint{0.967346in}{10.899632in}}%
\pgfpathlineto{\pgfqpoint{0.962798in}{10.905415in}}%
\pgfpathlineto{\pgfqpoint{0.960976in}{10.907731in}}%
\pgfpathlineto{\pgfqpoint{0.958249in}{10.911197in}}%
\pgfpathlineto{\pgfqpoint{0.953700in}{10.916979in}}%
\pgfpathlineto{\pgfqpoint{0.952671in}{10.918288in}}%
\pgfpathlineto{\pgfqpoint{0.949152in}{10.922761in}}%
\pgfpathlineto{\pgfqpoint{0.944603in}{10.928543in}}%
\pgfpathlineto{\pgfqpoint{0.944366in}{10.928845in}}%
\pgfpathlineto{\pgfqpoint{0.940055in}{10.934326in}}%
\pgfpathlineto{\pgfqpoint{0.936061in}{10.939402in}}%
\pgfpathlineto{\pgfqpoint{0.935506in}{10.940108in}}%
\pgfpathlineto{\pgfqpoint{0.930957in}{10.945890in}}%
\pgfpathlineto{\pgfqpoint{0.927756in}{10.949959in}}%
\pgfpathlineto{\pgfqpoint{0.926409in}{10.951672in}}%
\pgfpathlineto{\pgfqpoint{0.921860in}{10.957454in}}%
\pgfpathlineto{\pgfqpoint{0.919452in}{10.960516in}}%
\pgfpathlineto{\pgfqpoint{0.917311in}{10.963237in}}%
\pgfpathlineto{\pgfqpoint{0.912763in}{10.969019in}}%
\pgfpathlineto{\pgfqpoint{0.911147in}{10.971073in}}%
\pgfpathlineto{\pgfqpoint{0.908214in}{10.974801in}}%
\pgfpathlineto{\pgfqpoint{0.903665in}{10.980583in}}%
\pgfpathlineto{\pgfqpoint{0.902842in}{10.981630in}}%
\pgfpathlineto{\pgfqpoint{0.899117in}{10.986365in}}%
\pgfpathlineto{\pgfqpoint{0.894568in}{10.992148in}}%
\pgfpathlineto{\pgfqpoint{0.894537in}{10.992187in}}%
\pgfpathlineto{\pgfqpoint{0.890020in}{10.997930in}}%
\pgfpathlineto{\pgfqpoint{0.886232in}{11.002744in}}%
\pgfpathlineto{\pgfqpoint{0.885471in}{11.003712in}}%
\pgfpathlineto{\pgfqpoint{0.880922in}{11.009494in}}%
\pgfpathlineto{\pgfqpoint{0.877927in}{11.013301in}}%
\pgfpathlineto{\pgfqpoint{0.876374in}{11.015276in}}%
\pgfpathlineto{\pgfqpoint{0.871825in}{11.021059in}}%
\pgfpathlineto{\pgfqpoint{0.869623in}{11.023859in}}%
\pgfpathlineto{\pgfqpoint{0.867276in}{11.026841in}}%
\pgfpathlineto{\pgfqpoint{0.862728in}{11.032623in}}%
\pgfpathlineto{\pgfqpoint{0.861318in}{11.034416in}}%
\pgfpathlineto{\pgfqpoint{0.858179in}{11.038405in}}%
\pgfpathlineto{\pgfqpoint{0.853631in}{11.044187in}}%
\pgfpathlineto{\pgfqpoint{0.853013in}{11.044973in}}%
\pgfpathlineto{\pgfqpoint{0.849082in}{11.049970in}}%
\pgfpathlineto{\pgfqpoint{0.844708in}{11.055530in}}%
\pgfpathlineto{\pgfqpoint{0.844533in}{11.055752in}}%
\pgfpathlineto{\pgfqpoint{0.839985in}{11.061534in}}%
\pgfpathlineto{\pgfqpoint{0.836403in}{11.066087in}}%
\pgfpathlineto{\pgfqpoint{0.835436in}{11.067316in}}%
\pgfpathlineto{\pgfqpoint{0.830887in}{11.073098in}}%
\pgfpathlineto{\pgfqpoint{0.828098in}{11.076644in}}%
\pgfpathlineto{\pgfqpoint{0.826339in}{11.078881in}}%
\pgfpathlineto{\pgfqpoint{0.821790in}{11.084663in}}%
\pgfpathlineto{\pgfqpoint{0.819794in}{11.087201in}}%
\pgfpathlineto{\pgfqpoint{0.817242in}{11.090445in}}%
\pgfpathlineto{\pgfqpoint{0.812693in}{11.096227in}}%
\pgfpathlineto{\pgfqpoint{0.811489in}{11.097758in}}%
\pgfpathlineto{\pgfqpoint{0.808144in}{11.102009in}}%
\pgfpathlineto{\pgfqpoint{0.803596in}{11.107792in}}%
\pgfpathlineto{\pgfqpoint{0.803184in}{11.108315in}}%
\pgfpathlineto{\pgfqpoint{0.799047in}{11.113574in}}%
\pgfpathlineto{\pgfqpoint{0.794879in}{11.118872in}}%
\pgfpathlineto{\pgfqpoint{0.794498in}{11.119356in}}%
\pgfpathlineto{\pgfqpoint{0.789950in}{11.125138in}}%
\pgfpathlineto{\pgfqpoint{0.786574in}{11.129429in}}%
\pgfpathlineto{\pgfqpoint{0.785401in}{11.130920in}}%
\pgfpathlineto{\pgfqpoint{0.780853in}{11.136703in}}%
\pgfpathlineto{\pgfqpoint{0.778269in}{11.139986in}}%
\pgfpathlineto{\pgfqpoint{0.776304in}{11.142485in}}%
\pgfpathlineto{\pgfqpoint{0.771755in}{11.148267in}}%
\pgfpathlineto{\pgfqpoint{0.769965in}{11.150543in}}%
\pgfpathlineto{\pgfqpoint{0.767207in}{11.154049in}}%
\pgfpathlineto{\pgfqpoint{0.762658in}{11.159831in}}%
\pgfpathlineto{\pgfqpoint{0.761660in}{11.161101in}}%
\pgfpathlineto{\pgfqpoint{0.758109in}{11.165614in}}%
\pgfpathlineto{\pgfqpoint{0.753561in}{11.171396in}}%
\pgfpathlineto{\pgfqpoint{0.753355in}{11.171658in}}%
\pgfpathlineto{\pgfqpoint{0.749012in}{11.177178in}}%
\pgfpathlineto{\pgfqpoint{0.745050in}{11.182215in}}%
\pgfpathlineto{\pgfqpoint{0.744463in}{11.182960in}}%
\pgfpathlineto{\pgfqpoint{0.739915in}{11.188742in}}%
\pgfpathlineto{\pgfqpoint{0.736745in}{11.192772in}}%
\pgfpathlineto{\pgfqpoint{0.735366in}{11.194525in}}%
\pgfpathlineto{\pgfqpoint{0.730818in}{11.200307in}}%
\pgfpathlineto{\pgfqpoint{0.728440in}{11.203329in}}%
\pgfpathlineto{\pgfqpoint{0.726269in}{11.206089in}}%
\pgfpathlineto{\pgfqpoint{0.721720in}{11.211871in}}%
\pgfpathlineto{\pgfqpoint{0.720136in}{11.213886in}}%
\pgfpathlineto{\pgfqpoint{0.717172in}{11.217653in}}%
\pgfpathlineto{\pgfqpoint{0.712623in}{11.223436in}}%
\pgfpathlineto{\pgfqpoint{0.711831in}{11.224443in}}%
\pgfpathlineto{\pgfqpoint{0.708074in}{11.229218in}}%
\pgfpathlineto{\pgfqpoint{0.703526in}{11.235000in}}%
\pgfpathlineto{\pgfqpoint{0.703526in}{11.235000in}}%
\pgfpathclose%
\pgfusepath{stroke,fill}%
\end{pgfscope}%
\begin{pgfscope}%
\pgfpathrectangle{\pgfqpoint{0.703526in}{0.688481in}}{\pgfqpoint{9.300000in}{6.795000in}}%
\pgfusepath{clip}%
\pgfsetrectcap%
\pgfsetroundjoin%
\pgfsetlinewidth{0.803000pt}%
\definecolor{currentstroke}{rgb}{1.000000,1.000000,1.000000}%
\pgfsetstrokecolor{currentstroke}%
\pgfsetstrokeopacity{0.400000}%
\pgfsetdash{}{0pt}%
\pgfpathmoveto{\pgfqpoint{0.703526in}{0.688481in}}%
\pgfpathlineto{\pgfqpoint{0.703526in}{7.483481in}}%
\pgfusepath{stroke}%
\end{pgfscope}%
\begin{pgfscope}%
\pgfsetbuttcap%
\pgfsetroundjoin%
\definecolor{currentfill}{rgb}{0.000000,0.000000,0.000000}%
\pgfsetfillcolor{currentfill}%
\pgfsetlinewidth{0.803000pt}%
\definecolor{currentstroke}{rgb}{0.000000,0.000000,0.000000}%
\pgfsetstrokecolor{currentstroke}%
\pgfsetdash{}{0pt}%
\pgfsys@defobject{currentmarker}{\pgfqpoint{0.000000in}{-0.048611in}}{\pgfqpoint{0.000000in}{0.000000in}}{%
\pgfpathmoveto{\pgfqpoint{0.000000in}{0.000000in}}%
\pgfpathlineto{\pgfqpoint{0.000000in}{-0.048611in}}%
\pgfusepath{stroke,fill}%
}%
\begin{pgfscope}%
\pgfsys@transformshift{0.703526in}{0.688481in}%
\pgfsys@useobject{currentmarker}{}%
\end{pgfscope}%
\end{pgfscope}%
\begin{pgfscope}%
\definecolor{textcolor}{rgb}{0.000000,0.000000,0.000000}%
\pgfsetstrokecolor{textcolor}%
\pgfsetfillcolor{textcolor}%
\pgftext[x=0.703526in,y=0.591258in,,top]{\color{textcolor}\rmfamily\fontsize{12.000000}{14.400000}\selectfont \(\displaystyle {0}\)}%
\end{pgfscope}%
\begin{pgfscope}%
\pgfpathrectangle{\pgfqpoint{0.703526in}{0.688481in}}{\pgfqpoint{9.300000in}{6.795000in}}%
\pgfusepath{clip}%
\pgfsetrectcap%
\pgfsetroundjoin%
\pgfsetlinewidth{0.803000pt}%
\definecolor{currentstroke}{rgb}{1.000000,1.000000,1.000000}%
\pgfsetstrokecolor{currentstroke}%
\pgfsetstrokeopacity{0.400000}%
\pgfsetdash{}{0pt}%
\pgfpathmoveto{\pgfqpoint{2.399598in}{0.688481in}}%
\pgfpathlineto{\pgfqpoint{2.399598in}{7.483481in}}%
\pgfusepath{stroke}%
\end{pgfscope}%
\begin{pgfscope}%
\pgfsetbuttcap%
\pgfsetroundjoin%
\definecolor{currentfill}{rgb}{0.000000,0.000000,0.000000}%
\pgfsetfillcolor{currentfill}%
\pgfsetlinewidth{0.803000pt}%
\definecolor{currentstroke}{rgb}{0.000000,0.000000,0.000000}%
\pgfsetstrokecolor{currentstroke}%
\pgfsetdash{}{0pt}%
\pgfsys@defobject{currentmarker}{\pgfqpoint{0.000000in}{-0.048611in}}{\pgfqpoint{0.000000in}{0.000000in}}{%
\pgfpathmoveto{\pgfqpoint{0.000000in}{0.000000in}}%
\pgfpathlineto{\pgfqpoint{0.000000in}{-0.048611in}}%
\pgfusepath{stroke,fill}%
}%
\begin{pgfscope}%
\pgfsys@transformshift{2.399598in}{0.688481in}%
\pgfsys@useobject{currentmarker}{}%
\end{pgfscope}%
\end{pgfscope}%
\begin{pgfscope}%
\definecolor{textcolor}{rgb}{0.000000,0.000000,0.000000}%
\pgfsetstrokecolor{textcolor}%
\pgfsetfillcolor{textcolor}%
\pgftext[x=2.399598in,y=0.591258in,,top]{\color{textcolor}\rmfamily\fontsize{12.000000}{14.400000}\selectfont \(\displaystyle {5}\)}%
\end{pgfscope}%
\begin{pgfscope}%
\pgfpathrectangle{\pgfqpoint{0.703526in}{0.688481in}}{\pgfqpoint{9.300000in}{6.795000in}}%
\pgfusepath{clip}%
\pgfsetrectcap%
\pgfsetroundjoin%
\pgfsetlinewidth{0.803000pt}%
\definecolor{currentstroke}{rgb}{1.000000,1.000000,1.000000}%
\pgfsetstrokecolor{currentstroke}%
\pgfsetstrokeopacity{0.400000}%
\pgfsetdash{}{0pt}%
\pgfpathmoveto{\pgfqpoint{4.095670in}{0.688481in}}%
\pgfpathlineto{\pgfqpoint{4.095670in}{7.483481in}}%
\pgfusepath{stroke}%
\end{pgfscope}%
\begin{pgfscope}%
\pgfsetbuttcap%
\pgfsetroundjoin%
\definecolor{currentfill}{rgb}{0.000000,0.000000,0.000000}%
\pgfsetfillcolor{currentfill}%
\pgfsetlinewidth{0.803000pt}%
\definecolor{currentstroke}{rgb}{0.000000,0.000000,0.000000}%
\pgfsetstrokecolor{currentstroke}%
\pgfsetdash{}{0pt}%
\pgfsys@defobject{currentmarker}{\pgfqpoint{0.000000in}{-0.048611in}}{\pgfqpoint{0.000000in}{0.000000in}}{%
\pgfpathmoveto{\pgfqpoint{0.000000in}{0.000000in}}%
\pgfpathlineto{\pgfqpoint{0.000000in}{-0.048611in}}%
\pgfusepath{stroke,fill}%
}%
\begin{pgfscope}%
\pgfsys@transformshift{4.095670in}{0.688481in}%
\pgfsys@useobject{currentmarker}{}%
\end{pgfscope}%
\end{pgfscope}%
\begin{pgfscope}%
\definecolor{textcolor}{rgb}{0.000000,0.000000,0.000000}%
\pgfsetstrokecolor{textcolor}%
\pgfsetfillcolor{textcolor}%
\pgftext[x=4.095670in,y=0.591258in,,top]{\color{textcolor}\rmfamily\fontsize{12.000000}{14.400000}\selectfont \(\displaystyle {10}\)}%
\end{pgfscope}%
\begin{pgfscope}%
\pgfpathrectangle{\pgfqpoint{0.703526in}{0.688481in}}{\pgfqpoint{9.300000in}{6.795000in}}%
\pgfusepath{clip}%
\pgfsetrectcap%
\pgfsetroundjoin%
\pgfsetlinewidth{0.803000pt}%
\definecolor{currentstroke}{rgb}{1.000000,1.000000,1.000000}%
\pgfsetstrokecolor{currentstroke}%
\pgfsetstrokeopacity{0.400000}%
\pgfsetdash{}{0pt}%
\pgfpathmoveto{\pgfqpoint{5.791742in}{0.688481in}}%
\pgfpathlineto{\pgfqpoint{5.791742in}{7.483481in}}%
\pgfusepath{stroke}%
\end{pgfscope}%
\begin{pgfscope}%
\pgfsetbuttcap%
\pgfsetroundjoin%
\definecolor{currentfill}{rgb}{0.000000,0.000000,0.000000}%
\pgfsetfillcolor{currentfill}%
\pgfsetlinewidth{0.803000pt}%
\definecolor{currentstroke}{rgb}{0.000000,0.000000,0.000000}%
\pgfsetstrokecolor{currentstroke}%
\pgfsetdash{}{0pt}%
\pgfsys@defobject{currentmarker}{\pgfqpoint{0.000000in}{-0.048611in}}{\pgfqpoint{0.000000in}{0.000000in}}{%
\pgfpathmoveto{\pgfqpoint{0.000000in}{0.000000in}}%
\pgfpathlineto{\pgfqpoint{0.000000in}{-0.048611in}}%
\pgfusepath{stroke,fill}%
}%
\begin{pgfscope}%
\pgfsys@transformshift{5.791742in}{0.688481in}%
\pgfsys@useobject{currentmarker}{}%
\end{pgfscope}%
\end{pgfscope}%
\begin{pgfscope}%
\definecolor{textcolor}{rgb}{0.000000,0.000000,0.000000}%
\pgfsetstrokecolor{textcolor}%
\pgfsetfillcolor{textcolor}%
\pgftext[x=5.791742in,y=0.591258in,,top]{\color{textcolor}\rmfamily\fontsize{12.000000}{14.400000}\selectfont \(\displaystyle {15}\)}%
\end{pgfscope}%
\begin{pgfscope}%
\pgfpathrectangle{\pgfqpoint{0.703526in}{0.688481in}}{\pgfqpoint{9.300000in}{6.795000in}}%
\pgfusepath{clip}%
\pgfsetrectcap%
\pgfsetroundjoin%
\pgfsetlinewidth{0.803000pt}%
\definecolor{currentstroke}{rgb}{1.000000,1.000000,1.000000}%
\pgfsetstrokecolor{currentstroke}%
\pgfsetstrokeopacity{0.400000}%
\pgfsetdash{}{0pt}%
\pgfpathmoveto{\pgfqpoint{7.487815in}{0.688481in}}%
\pgfpathlineto{\pgfqpoint{7.487815in}{7.483481in}}%
\pgfusepath{stroke}%
\end{pgfscope}%
\begin{pgfscope}%
\pgfsetbuttcap%
\pgfsetroundjoin%
\definecolor{currentfill}{rgb}{0.000000,0.000000,0.000000}%
\pgfsetfillcolor{currentfill}%
\pgfsetlinewidth{0.803000pt}%
\definecolor{currentstroke}{rgb}{0.000000,0.000000,0.000000}%
\pgfsetstrokecolor{currentstroke}%
\pgfsetdash{}{0pt}%
\pgfsys@defobject{currentmarker}{\pgfqpoint{0.000000in}{-0.048611in}}{\pgfqpoint{0.000000in}{0.000000in}}{%
\pgfpathmoveto{\pgfqpoint{0.000000in}{0.000000in}}%
\pgfpathlineto{\pgfqpoint{0.000000in}{-0.048611in}}%
\pgfusepath{stroke,fill}%
}%
\begin{pgfscope}%
\pgfsys@transformshift{7.487815in}{0.688481in}%
\pgfsys@useobject{currentmarker}{}%
\end{pgfscope}%
\end{pgfscope}%
\begin{pgfscope}%
\definecolor{textcolor}{rgb}{0.000000,0.000000,0.000000}%
\pgfsetstrokecolor{textcolor}%
\pgfsetfillcolor{textcolor}%
\pgftext[x=7.487815in,y=0.591258in,,top]{\color{textcolor}\rmfamily\fontsize{12.000000}{14.400000}\selectfont \(\displaystyle {20}\)}%
\end{pgfscope}%
\begin{pgfscope}%
\pgfpathrectangle{\pgfqpoint{0.703526in}{0.688481in}}{\pgfqpoint{9.300000in}{6.795000in}}%
\pgfusepath{clip}%
\pgfsetrectcap%
\pgfsetroundjoin%
\pgfsetlinewidth{0.803000pt}%
\definecolor{currentstroke}{rgb}{1.000000,1.000000,1.000000}%
\pgfsetstrokecolor{currentstroke}%
\pgfsetstrokeopacity{0.400000}%
\pgfsetdash{}{0pt}%
\pgfpathmoveto{\pgfqpoint{9.183887in}{0.688481in}}%
\pgfpathlineto{\pgfqpoint{9.183887in}{7.483481in}}%
\pgfusepath{stroke}%
\end{pgfscope}%
\begin{pgfscope}%
\pgfsetbuttcap%
\pgfsetroundjoin%
\definecolor{currentfill}{rgb}{0.000000,0.000000,0.000000}%
\pgfsetfillcolor{currentfill}%
\pgfsetlinewidth{0.803000pt}%
\definecolor{currentstroke}{rgb}{0.000000,0.000000,0.000000}%
\pgfsetstrokecolor{currentstroke}%
\pgfsetdash{}{0pt}%
\pgfsys@defobject{currentmarker}{\pgfqpoint{0.000000in}{-0.048611in}}{\pgfqpoint{0.000000in}{0.000000in}}{%
\pgfpathmoveto{\pgfqpoint{0.000000in}{0.000000in}}%
\pgfpathlineto{\pgfqpoint{0.000000in}{-0.048611in}}%
\pgfusepath{stroke,fill}%
}%
\begin{pgfscope}%
\pgfsys@transformshift{9.183887in}{0.688481in}%
\pgfsys@useobject{currentmarker}{}%
\end{pgfscope}%
\end{pgfscope}%
\begin{pgfscope}%
\definecolor{textcolor}{rgb}{0.000000,0.000000,0.000000}%
\pgfsetstrokecolor{textcolor}%
\pgfsetfillcolor{textcolor}%
\pgftext[x=9.183887in,y=0.591258in,,top]{\color{textcolor}\rmfamily\fontsize{12.000000}{14.400000}\selectfont \(\displaystyle {25}\)}%
\end{pgfscope}%
\begin{pgfscope}%
\pgfsetbuttcap%
\pgfsetroundjoin%
\definecolor{currentfill}{rgb}{0.000000,0.000000,0.000000}%
\pgfsetfillcolor{currentfill}%
\pgfsetlinewidth{0.602250pt}%
\definecolor{currentstroke}{rgb}{0.000000,0.000000,0.000000}%
\pgfsetstrokecolor{currentstroke}%
\pgfsetdash{}{0pt}%
\pgfsys@defobject{currentmarker}{\pgfqpoint{0.000000in}{-0.027778in}}{\pgfqpoint{0.000000in}{0.000000in}}{%
\pgfpathmoveto{\pgfqpoint{0.000000in}{0.000000in}}%
\pgfpathlineto{\pgfqpoint{0.000000in}{-0.027778in}}%
\pgfusepath{stroke,fill}%
}%
\begin{pgfscope}%
\pgfsys@transformshift{1.042740in}{0.688481in}%
\pgfsys@useobject{currentmarker}{}%
\end{pgfscope}%
\end{pgfscope}%
\begin{pgfscope}%
\pgfsetbuttcap%
\pgfsetroundjoin%
\definecolor{currentfill}{rgb}{0.000000,0.000000,0.000000}%
\pgfsetfillcolor{currentfill}%
\pgfsetlinewidth{0.602250pt}%
\definecolor{currentstroke}{rgb}{0.000000,0.000000,0.000000}%
\pgfsetstrokecolor{currentstroke}%
\pgfsetdash{}{0pt}%
\pgfsys@defobject{currentmarker}{\pgfqpoint{0.000000in}{-0.027778in}}{\pgfqpoint{0.000000in}{0.000000in}}{%
\pgfpathmoveto{\pgfqpoint{0.000000in}{0.000000in}}%
\pgfpathlineto{\pgfqpoint{0.000000in}{-0.027778in}}%
\pgfusepath{stroke,fill}%
}%
\begin{pgfscope}%
\pgfsys@transformshift{1.381955in}{0.688481in}%
\pgfsys@useobject{currentmarker}{}%
\end{pgfscope}%
\end{pgfscope}%
\begin{pgfscope}%
\pgfsetbuttcap%
\pgfsetroundjoin%
\definecolor{currentfill}{rgb}{0.000000,0.000000,0.000000}%
\pgfsetfillcolor{currentfill}%
\pgfsetlinewidth{0.602250pt}%
\definecolor{currentstroke}{rgb}{0.000000,0.000000,0.000000}%
\pgfsetstrokecolor{currentstroke}%
\pgfsetdash{}{0pt}%
\pgfsys@defobject{currentmarker}{\pgfqpoint{0.000000in}{-0.027778in}}{\pgfqpoint{0.000000in}{0.000000in}}{%
\pgfpathmoveto{\pgfqpoint{0.000000in}{0.000000in}}%
\pgfpathlineto{\pgfqpoint{0.000000in}{-0.027778in}}%
\pgfusepath{stroke,fill}%
}%
\begin{pgfscope}%
\pgfsys@transformshift{1.721169in}{0.688481in}%
\pgfsys@useobject{currentmarker}{}%
\end{pgfscope}%
\end{pgfscope}%
\begin{pgfscope}%
\pgfsetbuttcap%
\pgfsetroundjoin%
\definecolor{currentfill}{rgb}{0.000000,0.000000,0.000000}%
\pgfsetfillcolor{currentfill}%
\pgfsetlinewidth{0.602250pt}%
\definecolor{currentstroke}{rgb}{0.000000,0.000000,0.000000}%
\pgfsetstrokecolor{currentstroke}%
\pgfsetdash{}{0pt}%
\pgfsys@defobject{currentmarker}{\pgfqpoint{0.000000in}{-0.027778in}}{\pgfqpoint{0.000000in}{0.000000in}}{%
\pgfpathmoveto{\pgfqpoint{0.000000in}{0.000000in}}%
\pgfpathlineto{\pgfqpoint{0.000000in}{-0.027778in}}%
\pgfusepath{stroke,fill}%
}%
\begin{pgfscope}%
\pgfsys@transformshift{2.060384in}{0.688481in}%
\pgfsys@useobject{currentmarker}{}%
\end{pgfscope}%
\end{pgfscope}%
\begin{pgfscope}%
\pgfsetbuttcap%
\pgfsetroundjoin%
\definecolor{currentfill}{rgb}{0.000000,0.000000,0.000000}%
\pgfsetfillcolor{currentfill}%
\pgfsetlinewidth{0.602250pt}%
\definecolor{currentstroke}{rgb}{0.000000,0.000000,0.000000}%
\pgfsetstrokecolor{currentstroke}%
\pgfsetdash{}{0pt}%
\pgfsys@defobject{currentmarker}{\pgfqpoint{0.000000in}{-0.027778in}}{\pgfqpoint{0.000000in}{0.000000in}}{%
\pgfpathmoveto{\pgfqpoint{0.000000in}{0.000000in}}%
\pgfpathlineto{\pgfqpoint{0.000000in}{-0.027778in}}%
\pgfusepath{stroke,fill}%
}%
\begin{pgfscope}%
\pgfsys@transformshift{2.738812in}{0.688481in}%
\pgfsys@useobject{currentmarker}{}%
\end{pgfscope}%
\end{pgfscope}%
\begin{pgfscope}%
\pgfsetbuttcap%
\pgfsetroundjoin%
\definecolor{currentfill}{rgb}{0.000000,0.000000,0.000000}%
\pgfsetfillcolor{currentfill}%
\pgfsetlinewidth{0.602250pt}%
\definecolor{currentstroke}{rgb}{0.000000,0.000000,0.000000}%
\pgfsetstrokecolor{currentstroke}%
\pgfsetdash{}{0pt}%
\pgfsys@defobject{currentmarker}{\pgfqpoint{0.000000in}{-0.027778in}}{\pgfqpoint{0.000000in}{0.000000in}}{%
\pgfpathmoveto{\pgfqpoint{0.000000in}{0.000000in}}%
\pgfpathlineto{\pgfqpoint{0.000000in}{-0.027778in}}%
\pgfusepath{stroke,fill}%
}%
\begin{pgfscope}%
\pgfsys@transformshift{3.078027in}{0.688481in}%
\pgfsys@useobject{currentmarker}{}%
\end{pgfscope}%
\end{pgfscope}%
\begin{pgfscope}%
\pgfsetbuttcap%
\pgfsetroundjoin%
\definecolor{currentfill}{rgb}{0.000000,0.000000,0.000000}%
\pgfsetfillcolor{currentfill}%
\pgfsetlinewidth{0.602250pt}%
\definecolor{currentstroke}{rgb}{0.000000,0.000000,0.000000}%
\pgfsetstrokecolor{currentstroke}%
\pgfsetdash{}{0pt}%
\pgfsys@defobject{currentmarker}{\pgfqpoint{0.000000in}{-0.027778in}}{\pgfqpoint{0.000000in}{0.000000in}}{%
\pgfpathmoveto{\pgfqpoint{0.000000in}{0.000000in}}%
\pgfpathlineto{\pgfqpoint{0.000000in}{-0.027778in}}%
\pgfusepath{stroke,fill}%
}%
\begin{pgfscope}%
\pgfsys@transformshift{3.417241in}{0.688481in}%
\pgfsys@useobject{currentmarker}{}%
\end{pgfscope}%
\end{pgfscope}%
\begin{pgfscope}%
\pgfsetbuttcap%
\pgfsetroundjoin%
\definecolor{currentfill}{rgb}{0.000000,0.000000,0.000000}%
\pgfsetfillcolor{currentfill}%
\pgfsetlinewidth{0.602250pt}%
\definecolor{currentstroke}{rgb}{0.000000,0.000000,0.000000}%
\pgfsetstrokecolor{currentstroke}%
\pgfsetdash{}{0pt}%
\pgfsys@defobject{currentmarker}{\pgfqpoint{0.000000in}{-0.027778in}}{\pgfqpoint{0.000000in}{0.000000in}}{%
\pgfpathmoveto{\pgfqpoint{0.000000in}{0.000000in}}%
\pgfpathlineto{\pgfqpoint{0.000000in}{-0.027778in}}%
\pgfusepath{stroke,fill}%
}%
\begin{pgfscope}%
\pgfsys@transformshift{3.756456in}{0.688481in}%
\pgfsys@useobject{currentmarker}{}%
\end{pgfscope}%
\end{pgfscope}%
\begin{pgfscope}%
\pgfsetbuttcap%
\pgfsetroundjoin%
\definecolor{currentfill}{rgb}{0.000000,0.000000,0.000000}%
\pgfsetfillcolor{currentfill}%
\pgfsetlinewidth{0.602250pt}%
\definecolor{currentstroke}{rgb}{0.000000,0.000000,0.000000}%
\pgfsetstrokecolor{currentstroke}%
\pgfsetdash{}{0pt}%
\pgfsys@defobject{currentmarker}{\pgfqpoint{0.000000in}{-0.027778in}}{\pgfqpoint{0.000000in}{0.000000in}}{%
\pgfpathmoveto{\pgfqpoint{0.000000in}{0.000000in}}%
\pgfpathlineto{\pgfqpoint{0.000000in}{-0.027778in}}%
\pgfusepath{stroke,fill}%
}%
\begin{pgfscope}%
\pgfsys@transformshift{4.434885in}{0.688481in}%
\pgfsys@useobject{currentmarker}{}%
\end{pgfscope}%
\end{pgfscope}%
\begin{pgfscope}%
\pgfsetbuttcap%
\pgfsetroundjoin%
\definecolor{currentfill}{rgb}{0.000000,0.000000,0.000000}%
\pgfsetfillcolor{currentfill}%
\pgfsetlinewidth{0.602250pt}%
\definecolor{currentstroke}{rgb}{0.000000,0.000000,0.000000}%
\pgfsetstrokecolor{currentstroke}%
\pgfsetdash{}{0pt}%
\pgfsys@defobject{currentmarker}{\pgfqpoint{0.000000in}{-0.027778in}}{\pgfqpoint{0.000000in}{0.000000in}}{%
\pgfpathmoveto{\pgfqpoint{0.000000in}{0.000000in}}%
\pgfpathlineto{\pgfqpoint{0.000000in}{-0.027778in}}%
\pgfusepath{stroke,fill}%
}%
\begin{pgfscope}%
\pgfsys@transformshift{4.774099in}{0.688481in}%
\pgfsys@useobject{currentmarker}{}%
\end{pgfscope}%
\end{pgfscope}%
\begin{pgfscope}%
\pgfsetbuttcap%
\pgfsetroundjoin%
\definecolor{currentfill}{rgb}{0.000000,0.000000,0.000000}%
\pgfsetfillcolor{currentfill}%
\pgfsetlinewidth{0.602250pt}%
\definecolor{currentstroke}{rgb}{0.000000,0.000000,0.000000}%
\pgfsetstrokecolor{currentstroke}%
\pgfsetdash{}{0pt}%
\pgfsys@defobject{currentmarker}{\pgfqpoint{0.000000in}{-0.027778in}}{\pgfqpoint{0.000000in}{0.000000in}}{%
\pgfpathmoveto{\pgfqpoint{0.000000in}{0.000000in}}%
\pgfpathlineto{\pgfqpoint{0.000000in}{-0.027778in}}%
\pgfusepath{stroke,fill}%
}%
\begin{pgfscope}%
\pgfsys@transformshift{5.113313in}{0.688481in}%
\pgfsys@useobject{currentmarker}{}%
\end{pgfscope}%
\end{pgfscope}%
\begin{pgfscope}%
\pgfsetbuttcap%
\pgfsetroundjoin%
\definecolor{currentfill}{rgb}{0.000000,0.000000,0.000000}%
\pgfsetfillcolor{currentfill}%
\pgfsetlinewidth{0.602250pt}%
\definecolor{currentstroke}{rgb}{0.000000,0.000000,0.000000}%
\pgfsetstrokecolor{currentstroke}%
\pgfsetdash{}{0pt}%
\pgfsys@defobject{currentmarker}{\pgfqpoint{0.000000in}{-0.027778in}}{\pgfqpoint{0.000000in}{0.000000in}}{%
\pgfpathmoveto{\pgfqpoint{0.000000in}{0.000000in}}%
\pgfpathlineto{\pgfqpoint{0.000000in}{-0.027778in}}%
\pgfusepath{stroke,fill}%
}%
\begin{pgfscope}%
\pgfsys@transformshift{5.452528in}{0.688481in}%
\pgfsys@useobject{currentmarker}{}%
\end{pgfscope}%
\end{pgfscope}%
\begin{pgfscope}%
\pgfsetbuttcap%
\pgfsetroundjoin%
\definecolor{currentfill}{rgb}{0.000000,0.000000,0.000000}%
\pgfsetfillcolor{currentfill}%
\pgfsetlinewidth{0.602250pt}%
\definecolor{currentstroke}{rgb}{0.000000,0.000000,0.000000}%
\pgfsetstrokecolor{currentstroke}%
\pgfsetdash{}{0pt}%
\pgfsys@defobject{currentmarker}{\pgfqpoint{0.000000in}{-0.027778in}}{\pgfqpoint{0.000000in}{0.000000in}}{%
\pgfpathmoveto{\pgfqpoint{0.000000in}{0.000000in}}%
\pgfpathlineto{\pgfqpoint{0.000000in}{-0.027778in}}%
\pgfusepath{stroke,fill}%
}%
\begin{pgfscope}%
\pgfsys@transformshift{6.130957in}{0.688481in}%
\pgfsys@useobject{currentmarker}{}%
\end{pgfscope}%
\end{pgfscope}%
\begin{pgfscope}%
\pgfsetbuttcap%
\pgfsetroundjoin%
\definecolor{currentfill}{rgb}{0.000000,0.000000,0.000000}%
\pgfsetfillcolor{currentfill}%
\pgfsetlinewidth{0.602250pt}%
\definecolor{currentstroke}{rgb}{0.000000,0.000000,0.000000}%
\pgfsetstrokecolor{currentstroke}%
\pgfsetdash{}{0pt}%
\pgfsys@defobject{currentmarker}{\pgfqpoint{0.000000in}{-0.027778in}}{\pgfqpoint{0.000000in}{0.000000in}}{%
\pgfpathmoveto{\pgfqpoint{0.000000in}{0.000000in}}%
\pgfpathlineto{\pgfqpoint{0.000000in}{-0.027778in}}%
\pgfusepath{stroke,fill}%
}%
\begin{pgfscope}%
\pgfsys@transformshift{6.470171in}{0.688481in}%
\pgfsys@useobject{currentmarker}{}%
\end{pgfscope}%
\end{pgfscope}%
\begin{pgfscope}%
\pgfsetbuttcap%
\pgfsetroundjoin%
\definecolor{currentfill}{rgb}{0.000000,0.000000,0.000000}%
\pgfsetfillcolor{currentfill}%
\pgfsetlinewidth{0.602250pt}%
\definecolor{currentstroke}{rgb}{0.000000,0.000000,0.000000}%
\pgfsetstrokecolor{currentstroke}%
\pgfsetdash{}{0pt}%
\pgfsys@defobject{currentmarker}{\pgfqpoint{0.000000in}{-0.027778in}}{\pgfqpoint{0.000000in}{0.000000in}}{%
\pgfpathmoveto{\pgfqpoint{0.000000in}{0.000000in}}%
\pgfpathlineto{\pgfqpoint{0.000000in}{-0.027778in}}%
\pgfusepath{stroke,fill}%
}%
\begin{pgfscope}%
\pgfsys@transformshift{6.809386in}{0.688481in}%
\pgfsys@useobject{currentmarker}{}%
\end{pgfscope}%
\end{pgfscope}%
\begin{pgfscope}%
\pgfsetbuttcap%
\pgfsetroundjoin%
\definecolor{currentfill}{rgb}{0.000000,0.000000,0.000000}%
\pgfsetfillcolor{currentfill}%
\pgfsetlinewidth{0.602250pt}%
\definecolor{currentstroke}{rgb}{0.000000,0.000000,0.000000}%
\pgfsetstrokecolor{currentstroke}%
\pgfsetdash{}{0pt}%
\pgfsys@defobject{currentmarker}{\pgfqpoint{0.000000in}{-0.027778in}}{\pgfqpoint{0.000000in}{0.000000in}}{%
\pgfpathmoveto{\pgfqpoint{0.000000in}{0.000000in}}%
\pgfpathlineto{\pgfqpoint{0.000000in}{-0.027778in}}%
\pgfusepath{stroke,fill}%
}%
\begin{pgfscope}%
\pgfsys@transformshift{7.148600in}{0.688481in}%
\pgfsys@useobject{currentmarker}{}%
\end{pgfscope}%
\end{pgfscope}%
\begin{pgfscope}%
\pgfsetbuttcap%
\pgfsetroundjoin%
\definecolor{currentfill}{rgb}{0.000000,0.000000,0.000000}%
\pgfsetfillcolor{currentfill}%
\pgfsetlinewidth{0.602250pt}%
\definecolor{currentstroke}{rgb}{0.000000,0.000000,0.000000}%
\pgfsetstrokecolor{currentstroke}%
\pgfsetdash{}{0pt}%
\pgfsys@defobject{currentmarker}{\pgfqpoint{0.000000in}{-0.027778in}}{\pgfqpoint{0.000000in}{0.000000in}}{%
\pgfpathmoveto{\pgfqpoint{0.000000in}{0.000000in}}%
\pgfpathlineto{\pgfqpoint{0.000000in}{-0.027778in}}%
\pgfusepath{stroke,fill}%
}%
\begin{pgfscope}%
\pgfsys@transformshift{7.827029in}{0.688481in}%
\pgfsys@useobject{currentmarker}{}%
\end{pgfscope}%
\end{pgfscope}%
\begin{pgfscope}%
\pgfsetbuttcap%
\pgfsetroundjoin%
\definecolor{currentfill}{rgb}{0.000000,0.000000,0.000000}%
\pgfsetfillcolor{currentfill}%
\pgfsetlinewidth{0.602250pt}%
\definecolor{currentstroke}{rgb}{0.000000,0.000000,0.000000}%
\pgfsetstrokecolor{currentstroke}%
\pgfsetdash{}{0pt}%
\pgfsys@defobject{currentmarker}{\pgfqpoint{0.000000in}{-0.027778in}}{\pgfqpoint{0.000000in}{0.000000in}}{%
\pgfpathmoveto{\pgfqpoint{0.000000in}{0.000000in}}%
\pgfpathlineto{\pgfqpoint{0.000000in}{-0.027778in}}%
\pgfusepath{stroke,fill}%
}%
\begin{pgfscope}%
\pgfsys@transformshift{8.166243in}{0.688481in}%
\pgfsys@useobject{currentmarker}{}%
\end{pgfscope}%
\end{pgfscope}%
\begin{pgfscope}%
\pgfsetbuttcap%
\pgfsetroundjoin%
\definecolor{currentfill}{rgb}{0.000000,0.000000,0.000000}%
\pgfsetfillcolor{currentfill}%
\pgfsetlinewidth{0.602250pt}%
\definecolor{currentstroke}{rgb}{0.000000,0.000000,0.000000}%
\pgfsetstrokecolor{currentstroke}%
\pgfsetdash{}{0pt}%
\pgfsys@defobject{currentmarker}{\pgfqpoint{0.000000in}{-0.027778in}}{\pgfqpoint{0.000000in}{0.000000in}}{%
\pgfpathmoveto{\pgfqpoint{0.000000in}{0.000000in}}%
\pgfpathlineto{\pgfqpoint{0.000000in}{-0.027778in}}%
\pgfusepath{stroke,fill}%
}%
\begin{pgfscope}%
\pgfsys@transformshift{8.505458in}{0.688481in}%
\pgfsys@useobject{currentmarker}{}%
\end{pgfscope}%
\end{pgfscope}%
\begin{pgfscope}%
\pgfsetbuttcap%
\pgfsetroundjoin%
\definecolor{currentfill}{rgb}{0.000000,0.000000,0.000000}%
\pgfsetfillcolor{currentfill}%
\pgfsetlinewidth{0.602250pt}%
\definecolor{currentstroke}{rgb}{0.000000,0.000000,0.000000}%
\pgfsetstrokecolor{currentstroke}%
\pgfsetdash{}{0pt}%
\pgfsys@defobject{currentmarker}{\pgfqpoint{0.000000in}{-0.027778in}}{\pgfqpoint{0.000000in}{0.000000in}}{%
\pgfpathmoveto{\pgfqpoint{0.000000in}{0.000000in}}%
\pgfpathlineto{\pgfqpoint{0.000000in}{-0.027778in}}%
\pgfusepath{stroke,fill}%
}%
\begin{pgfscope}%
\pgfsys@transformshift{8.844672in}{0.688481in}%
\pgfsys@useobject{currentmarker}{}%
\end{pgfscope}%
\end{pgfscope}%
\begin{pgfscope}%
\pgfsetbuttcap%
\pgfsetroundjoin%
\definecolor{currentfill}{rgb}{0.000000,0.000000,0.000000}%
\pgfsetfillcolor{currentfill}%
\pgfsetlinewidth{0.602250pt}%
\definecolor{currentstroke}{rgb}{0.000000,0.000000,0.000000}%
\pgfsetstrokecolor{currentstroke}%
\pgfsetdash{}{0pt}%
\pgfsys@defobject{currentmarker}{\pgfqpoint{0.000000in}{-0.027778in}}{\pgfqpoint{0.000000in}{0.000000in}}{%
\pgfpathmoveto{\pgfqpoint{0.000000in}{0.000000in}}%
\pgfpathlineto{\pgfqpoint{0.000000in}{-0.027778in}}%
\pgfusepath{stroke,fill}%
}%
\begin{pgfscope}%
\pgfsys@transformshift{9.523101in}{0.688481in}%
\pgfsys@useobject{currentmarker}{}%
\end{pgfscope}%
\end{pgfscope}%
\begin{pgfscope}%
\pgfsetbuttcap%
\pgfsetroundjoin%
\definecolor{currentfill}{rgb}{0.000000,0.000000,0.000000}%
\pgfsetfillcolor{currentfill}%
\pgfsetlinewidth{0.602250pt}%
\definecolor{currentstroke}{rgb}{0.000000,0.000000,0.000000}%
\pgfsetstrokecolor{currentstroke}%
\pgfsetdash{}{0pt}%
\pgfsys@defobject{currentmarker}{\pgfqpoint{0.000000in}{-0.027778in}}{\pgfqpoint{0.000000in}{0.000000in}}{%
\pgfpathmoveto{\pgfqpoint{0.000000in}{0.000000in}}%
\pgfpathlineto{\pgfqpoint{0.000000in}{-0.027778in}}%
\pgfusepath{stroke,fill}%
}%
\begin{pgfscope}%
\pgfsys@transformshift{9.862316in}{0.688481in}%
\pgfsys@useobject{currentmarker}{}%
\end{pgfscope}%
\end{pgfscope}%
\begin{pgfscope}%
\definecolor{textcolor}{rgb}{0.000000,0.000000,0.000000}%
\pgfsetstrokecolor{textcolor}%
\pgfsetfillcolor{textcolor}%
\pgftext[x=5.353526in,y=0.387555in,,top]{\color{textcolor}\rmfamily\fontsize{20.000000}{24.000000}\selectfont Solar Capacity [GW]}%
\end{pgfscope}%
\begin{pgfscope}%
\pgfpathrectangle{\pgfqpoint{0.703526in}{0.688481in}}{\pgfqpoint{9.300000in}{6.795000in}}%
\pgfusepath{clip}%
\pgfsetrectcap%
\pgfsetroundjoin%
\pgfsetlinewidth{0.803000pt}%
\definecolor{currentstroke}{rgb}{1.000000,1.000000,1.000000}%
\pgfsetstrokecolor{currentstroke}%
\pgfsetstrokeopacity{0.400000}%
\pgfsetdash{}{0pt}%
\pgfpathmoveto{\pgfqpoint{0.703526in}{0.688481in}}%
\pgfpathlineto{\pgfqpoint{10.003526in}{0.688481in}}%
\pgfusepath{stroke}%
\end{pgfscope}%
\begin{pgfscope}%
\pgfsetbuttcap%
\pgfsetroundjoin%
\definecolor{currentfill}{rgb}{0.000000,0.000000,0.000000}%
\pgfsetfillcolor{currentfill}%
\pgfsetlinewidth{0.803000pt}%
\definecolor{currentstroke}{rgb}{0.000000,0.000000,0.000000}%
\pgfsetstrokecolor{currentstroke}%
\pgfsetdash{}{0pt}%
\pgfsys@defobject{currentmarker}{\pgfqpoint{-0.048611in}{0.000000in}}{\pgfqpoint{-0.000000in}{0.000000in}}{%
\pgfpathmoveto{\pgfqpoint{-0.000000in}{0.000000in}}%
\pgfpathlineto{\pgfqpoint{-0.048611in}{0.000000in}}%
\pgfusepath{stroke,fill}%
}%
\begin{pgfscope}%
\pgfsys@transformshift{0.703526in}{0.688481in}%
\pgfsys@useobject{currentmarker}{}%
\end{pgfscope}%
\end{pgfscope}%
\begin{pgfscope}%
\definecolor{textcolor}{rgb}{0.000000,0.000000,0.000000}%
\pgfsetstrokecolor{textcolor}%
\pgfsetfillcolor{textcolor}%
\pgftext[x=0.524707in, y=0.630610in, left, base]{\color{textcolor}\rmfamily\fontsize{12.000000}{14.400000}\selectfont \(\displaystyle {0}\)}%
\end{pgfscope}%
\begin{pgfscope}%
\pgfpathrectangle{\pgfqpoint{0.703526in}{0.688481in}}{\pgfqpoint{9.300000in}{6.795000in}}%
\pgfusepath{clip}%
\pgfsetrectcap%
\pgfsetroundjoin%
\pgfsetlinewidth{0.803000pt}%
\definecolor{currentstroke}{rgb}{1.000000,1.000000,1.000000}%
\pgfsetstrokecolor{currentstroke}%
\pgfsetstrokeopacity{0.400000}%
\pgfsetdash{}{0pt}%
\pgfpathmoveto{\pgfqpoint{0.703526in}{1.555939in}}%
\pgfpathlineto{\pgfqpoint{10.003526in}{1.555939in}}%
\pgfusepath{stroke}%
\end{pgfscope}%
\begin{pgfscope}%
\pgfsetbuttcap%
\pgfsetroundjoin%
\definecolor{currentfill}{rgb}{0.000000,0.000000,0.000000}%
\pgfsetfillcolor{currentfill}%
\pgfsetlinewidth{0.803000pt}%
\definecolor{currentstroke}{rgb}{0.000000,0.000000,0.000000}%
\pgfsetstrokecolor{currentstroke}%
\pgfsetdash{}{0pt}%
\pgfsys@defobject{currentmarker}{\pgfqpoint{-0.048611in}{0.000000in}}{\pgfqpoint{-0.000000in}{0.000000in}}{%
\pgfpathmoveto{\pgfqpoint{-0.000000in}{0.000000in}}%
\pgfpathlineto{\pgfqpoint{-0.048611in}{0.000000in}}%
\pgfusepath{stroke,fill}%
}%
\begin{pgfscope}%
\pgfsys@transformshift{0.703526in}{1.555939in}%
\pgfsys@useobject{currentmarker}{}%
\end{pgfscope}%
\end{pgfscope}%
\begin{pgfscope}%
\definecolor{textcolor}{rgb}{0.000000,0.000000,0.000000}%
\pgfsetstrokecolor{textcolor}%
\pgfsetfillcolor{textcolor}%
\pgftext[x=0.524707in, y=1.498069in, left, base]{\color{textcolor}\rmfamily\fontsize{12.000000}{14.400000}\selectfont \(\displaystyle {2}\)}%
\end{pgfscope}%
\begin{pgfscope}%
\pgfpathrectangle{\pgfqpoint{0.703526in}{0.688481in}}{\pgfqpoint{9.300000in}{6.795000in}}%
\pgfusepath{clip}%
\pgfsetrectcap%
\pgfsetroundjoin%
\pgfsetlinewidth{0.803000pt}%
\definecolor{currentstroke}{rgb}{1.000000,1.000000,1.000000}%
\pgfsetstrokecolor{currentstroke}%
\pgfsetstrokeopacity{0.400000}%
\pgfsetdash{}{0pt}%
\pgfpathmoveto{\pgfqpoint{0.703526in}{2.423398in}}%
\pgfpathlineto{\pgfqpoint{10.003526in}{2.423398in}}%
\pgfusepath{stroke}%
\end{pgfscope}%
\begin{pgfscope}%
\pgfsetbuttcap%
\pgfsetroundjoin%
\definecolor{currentfill}{rgb}{0.000000,0.000000,0.000000}%
\pgfsetfillcolor{currentfill}%
\pgfsetlinewidth{0.803000pt}%
\definecolor{currentstroke}{rgb}{0.000000,0.000000,0.000000}%
\pgfsetstrokecolor{currentstroke}%
\pgfsetdash{}{0pt}%
\pgfsys@defobject{currentmarker}{\pgfqpoint{-0.048611in}{0.000000in}}{\pgfqpoint{-0.000000in}{0.000000in}}{%
\pgfpathmoveto{\pgfqpoint{-0.000000in}{0.000000in}}%
\pgfpathlineto{\pgfqpoint{-0.048611in}{0.000000in}}%
\pgfusepath{stroke,fill}%
}%
\begin{pgfscope}%
\pgfsys@transformshift{0.703526in}{2.423398in}%
\pgfsys@useobject{currentmarker}{}%
\end{pgfscope}%
\end{pgfscope}%
\begin{pgfscope}%
\definecolor{textcolor}{rgb}{0.000000,0.000000,0.000000}%
\pgfsetstrokecolor{textcolor}%
\pgfsetfillcolor{textcolor}%
\pgftext[x=0.524707in, y=2.365528in, left, base]{\color{textcolor}\rmfamily\fontsize{12.000000}{14.400000}\selectfont \(\displaystyle {4}\)}%
\end{pgfscope}%
\begin{pgfscope}%
\pgfpathrectangle{\pgfqpoint{0.703526in}{0.688481in}}{\pgfqpoint{9.300000in}{6.795000in}}%
\pgfusepath{clip}%
\pgfsetrectcap%
\pgfsetroundjoin%
\pgfsetlinewidth{0.803000pt}%
\definecolor{currentstroke}{rgb}{1.000000,1.000000,1.000000}%
\pgfsetstrokecolor{currentstroke}%
\pgfsetstrokeopacity{0.400000}%
\pgfsetdash{}{0pt}%
\pgfpathmoveto{\pgfqpoint{0.703526in}{3.290857in}}%
\pgfpathlineto{\pgfqpoint{10.003526in}{3.290857in}}%
\pgfusepath{stroke}%
\end{pgfscope}%
\begin{pgfscope}%
\pgfsetbuttcap%
\pgfsetroundjoin%
\definecolor{currentfill}{rgb}{0.000000,0.000000,0.000000}%
\pgfsetfillcolor{currentfill}%
\pgfsetlinewidth{0.803000pt}%
\definecolor{currentstroke}{rgb}{0.000000,0.000000,0.000000}%
\pgfsetstrokecolor{currentstroke}%
\pgfsetdash{}{0pt}%
\pgfsys@defobject{currentmarker}{\pgfqpoint{-0.048611in}{0.000000in}}{\pgfqpoint{-0.000000in}{0.000000in}}{%
\pgfpathmoveto{\pgfqpoint{-0.000000in}{0.000000in}}%
\pgfpathlineto{\pgfqpoint{-0.048611in}{0.000000in}}%
\pgfusepath{stroke,fill}%
}%
\begin{pgfscope}%
\pgfsys@transformshift{0.703526in}{3.290857in}%
\pgfsys@useobject{currentmarker}{}%
\end{pgfscope}%
\end{pgfscope}%
\begin{pgfscope}%
\definecolor{textcolor}{rgb}{0.000000,0.000000,0.000000}%
\pgfsetstrokecolor{textcolor}%
\pgfsetfillcolor{textcolor}%
\pgftext[x=0.524707in, y=3.232987in, left, base]{\color{textcolor}\rmfamily\fontsize{12.000000}{14.400000}\selectfont \(\displaystyle {6}\)}%
\end{pgfscope}%
\begin{pgfscope}%
\pgfpathrectangle{\pgfqpoint{0.703526in}{0.688481in}}{\pgfqpoint{9.300000in}{6.795000in}}%
\pgfusepath{clip}%
\pgfsetrectcap%
\pgfsetroundjoin%
\pgfsetlinewidth{0.803000pt}%
\definecolor{currentstroke}{rgb}{1.000000,1.000000,1.000000}%
\pgfsetstrokecolor{currentstroke}%
\pgfsetstrokeopacity{0.400000}%
\pgfsetdash{}{0pt}%
\pgfpathmoveto{\pgfqpoint{0.703526in}{4.158316in}}%
\pgfpathlineto{\pgfqpoint{10.003526in}{4.158316in}}%
\pgfusepath{stroke}%
\end{pgfscope}%
\begin{pgfscope}%
\pgfsetbuttcap%
\pgfsetroundjoin%
\definecolor{currentfill}{rgb}{0.000000,0.000000,0.000000}%
\pgfsetfillcolor{currentfill}%
\pgfsetlinewidth{0.803000pt}%
\definecolor{currentstroke}{rgb}{0.000000,0.000000,0.000000}%
\pgfsetstrokecolor{currentstroke}%
\pgfsetdash{}{0pt}%
\pgfsys@defobject{currentmarker}{\pgfqpoint{-0.048611in}{0.000000in}}{\pgfqpoint{-0.000000in}{0.000000in}}{%
\pgfpathmoveto{\pgfqpoint{-0.000000in}{0.000000in}}%
\pgfpathlineto{\pgfqpoint{-0.048611in}{0.000000in}}%
\pgfusepath{stroke,fill}%
}%
\begin{pgfscope}%
\pgfsys@transformshift{0.703526in}{4.158316in}%
\pgfsys@useobject{currentmarker}{}%
\end{pgfscope}%
\end{pgfscope}%
\begin{pgfscope}%
\definecolor{textcolor}{rgb}{0.000000,0.000000,0.000000}%
\pgfsetstrokecolor{textcolor}%
\pgfsetfillcolor{textcolor}%
\pgftext[x=0.524707in, y=4.100446in, left, base]{\color{textcolor}\rmfamily\fontsize{12.000000}{14.400000}\selectfont \(\displaystyle {8}\)}%
\end{pgfscope}%
\begin{pgfscope}%
\pgfpathrectangle{\pgfqpoint{0.703526in}{0.688481in}}{\pgfqpoint{9.300000in}{6.795000in}}%
\pgfusepath{clip}%
\pgfsetrectcap%
\pgfsetroundjoin%
\pgfsetlinewidth{0.803000pt}%
\definecolor{currentstroke}{rgb}{1.000000,1.000000,1.000000}%
\pgfsetstrokecolor{currentstroke}%
\pgfsetstrokeopacity{0.400000}%
\pgfsetdash{}{0pt}%
\pgfpathmoveto{\pgfqpoint{0.703526in}{5.025775in}}%
\pgfpathlineto{\pgfqpoint{10.003526in}{5.025775in}}%
\pgfusepath{stroke}%
\end{pgfscope}%
\begin{pgfscope}%
\pgfsetbuttcap%
\pgfsetroundjoin%
\definecolor{currentfill}{rgb}{0.000000,0.000000,0.000000}%
\pgfsetfillcolor{currentfill}%
\pgfsetlinewidth{0.803000pt}%
\definecolor{currentstroke}{rgb}{0.000000,0.000000,0.000000}%
\pgfsetstrokecolor{currentstroke}%
\pgfsetdash{}{0pt}%
\pgfsys@defobject{currentmarker}{\pgfqpoint{-0.048611in}{0.000000in}}{\pgfqpoint{-0.000000in}{0.000000in}}{%
\pgfpathmoveto{\pgfqpoint{-0.000000in}{0.000000in}}%
\pgfpathlineto{\pgfqpoint{-0.048611in}{0.000000in}}%
\pgfusepath{stroke,fill}%
}%
\begin{pgfscope}%
\pgfsys@transformshift{0.703526in}{5.025775in}%
\pgfsys@useobject{currentmarker}{}%
\end{pgfscope}%
\end{pgfscope}%
\begin{pgfscope}%
\definecolor{textcolor}{rgb}{0.000000,0.000000,0.000000}%
\pgfsetstrokecolor{textcolor}%
\pgfsetfillcolor{textcolor}%
\pgftext[x=0.443111in, y=4.967905in, left, base]{\color{textcolor}\rmfamily\fontsize{12.000000}{14.400000}\selectfont \(\displaystyle {10}\)}%
\end{pgfscope}%
\begin{pgfscope}%
\pgfpathrectangle{\pgfqpoint{0.703526in}{0.688481in}}{\pgfqpoint{9.300000in}{6.795000in}}%
\pgfusepath{clip}%
\pgfsetrectcap%
\pgfsetroundjoin%
\pgfsetlinewidth{0.803000pt}%
\definecolor{currentstroke}{rgb}{1.000000,1.000000,1.000000}%
\pgfsetstrokecolor{currentstroke}%
\pgfsetstrokeopacity{0.400000}%
\pgfsetdash{}{0pt}%
\pgfpathmoveto{\pgfqpoint{0.703526in}{5.893234in}}%
\pgfpathlineto{\pgfqpoint{10.003526in}{5.893234in}}%
\pgfusepath{stroke}%
\end{pgfscope}%
\begin{pgfscope}%
\pgfsetbuttcap%
\pgfsetroundjoin%
\definecolor{currentfill}{rgb}{0.000000,0.000000,0.000000}%
\pgfsetfillcolor{currentfill}%
\pgfsetlinewidth{0.803000pt}%
\definecolor{currentstroke}{rgb}{0.000000,0.000000,0.000000}%
\pgfsetstrokecolor{currentstroke}%
\pgfsetdash{}{0pt}%
\pgfsys@defobject{currentmarker}{\pgfqpoint{-0.048611in}{0.000000in}}{\pgfqpoint{-0.000000in}{0.000000in}}{%
\pgfpathmoveto{\pgfqpoint{-0.000000in}{0.000000in}}%
\pgfpathlineto{\pgfqpoint{-0.048611in}{0.000000in}}%
\pgfusepath{stroke,fill}%
}%
\begin{pgfscope}%
\pgfsys@transformshift{0.703526in}{5.893234in}%
\pgfsys@useobject{currentmarker}{}%
\end{pgfscope}%
\end{pgfscope}%
\begin{pgfscope}%
\definecolor{textcolor}{rgb}{0.000000,0.000000,0.000000}%
\pgfsetstrokecolor{textcolor}%
\pgfsetfillcolor{textcolor}%
\pgftext[x=0.443111in, y=5.835363in, left, base]{\color{textcolor}\rmfamily\fontsize{12.000000}{14.400000}\selectfont \(\displaystyle {12}\)}%
\end{pgfscope}%
\begin{pgfscope}%
\pgfpathrectangle{\pgfqpoint{0.703526in}{0.688481in}}{\pgfqpoint{9.300000in}{6.795000in}}%
\pgfusepath{clip}%
\pgfsetrectcap%
\pgfsetroundjoin%
\pgfsetlinewidth{0.803000pt}%
\definecolor{currentstroke}{rgb}{1.000000,1.000000,1.000000}%
\pgfsetstrokecolor{currentstroke}%
\pgfsetstrokeopacity{0.400000}%
\pgfsetdash{}{0pt}%
\pgfpathmoveto{\pgfqpoint{0.703526in}{6.760692in}}%
\pgfpathlineto{\pgfqpoint{10.003526in}{6.760692in}}%
\pgfusepath{stroke}%
\end{pgfscope}%
\begin{pgfscope}%
\pgfsetbuttcap%
\pgfsetroundjoin%
\definecolor{currentfill}{rgb}{0.000000,0.000000,0.000000}%
\pgfsetfillcolor{currentfill}%
\pgfsetlinewidth{0.803000pt}%
\definecolor{currentstroke}{rgb}{0.000000,0.000000,0.000000}%
\pgfsetstrokecolor{currentstroke}%
\pgfsetdash{}{0pt}%
\pgfsys@defobject{currentmarker}{\pgfqpoint{-0.048611in}{0.000000in}}{\pgfqpoint{-0.000000in}{0.000000in}}{%
\pgfpathmoveto{\pgfqpoint{-0.000000in}{0.000000in}}%
\pgfpathlineto{\pgfqpoint{-0.048611in}{0.000000in}}%
\pgfusepath{stroke,fill}%
}%
\begin{pgfscope}%
\pgfsys@transformshift{0.703526in}{6.760692in}%
\pgfsys@useobject{currentmarker}{}%
\end{pgfscope}%
\end{pgfscope}%
\begin{pgfscope}%
\definecolor{textcolor}{rgb}{0.000000,0.000000,0.000000}%
\pgfsetstrokecolor{textcolor}%
\pgfsetfillcolor{textcolor}%
\pgftext[x=0.443111in, y=6.702822in, left, base]{\color{textcolor}\rmfamily\fontsize{12.000000}{14.400000}\selectfont \(\displaystyle {14}\)}%
\end{pgfscope}%
\begin{pgfscope}%
\pgfsetbuttcap%
\pgfsetroundjoin%
\definecolor{currentfill}{rgb}{0.000000,0.000000,0.000000}%
\pgfsetfillcolor{currentfill}%
\pgfsetlinewidth{0.602250pt}%
\definecolor{currentstroke}{rgb}{0.000000,0.000000,0.000000}%
\pgfsetstrokecolor{currentstroke}%
\pgfsetdash{}{0pt}%
\pgfsys@defobject{currentmarker}{\pgfqpoint{-0.027778in}{0.000000in}}{\pgfqpoint{-0.000000in}{0.000000in}}{%
\pgfpathmoveto{\pgfqpoint{-0.000000in}{0.000000in}}%
\pgfpathlineto{\pgfqpoint{-0.027778in}{0.000000in}}%
\pgfusepath{stroke,fill}%
}%
\begin{pgfscope}%
\pgfsys@transformshift{0.703526in}{0.905345in}%
\pgfsys@useobject{currentmarker}{}%
\end{pgfscope}%
\end{pgfscope}%
\begin{pgfscope}%
\pgfsetbuttcap%
\pgfsetroundjoin%
\definecolor{currentfill}{rgb}{0.000000,0.000000,0.000000}%
\pgfsetfillcolor{currentfill}%
\pgfsetlinewidth{0.602250pt}%
\definecolor{currentstroke}{rgb}{0.000000,0.000000,0.000000}%
\pgfsetstrokecolor{currentstroke}%
\pgfsetdash{}{0pt}%
\pgfsys@defobject{currentmarker}{\pgfqpoint{-0.027778in}{0.000000in}}{\pgfqpoint{-0.000000in}{0.000000in}}{%
\pgfpathmoveto{\pgfqpoint{-0.000000in}{0.000000in}}%
\pgfpathlineto{\pgfqpoint{-0.027778in}{0.000000in}}%
\pgfusepath{stroke,fill}%
}%
\begin{pgfscope}%
\pgfsys@transformshift{0.703526in}{1.122210in}%
\pgfsys@useobject{currentmarker}{}%
\end{pgfscope}%
\end{pgfscope}%
\begin{pgfscope}%
\pgfsetbuttcap%
\pgfsetroundjoin%
\definecolor{currentfill}{rgb}{0.000000,0.000000,0.000000}%
\pgfsetfillcolor{currentfill}%
\pgfsetlinewidth{0.602250pt}%
\definecolor{currentstroke}{rgb}{0.000000,0.000000,0.000000}%
\pgfsetstrokecolor{currentstroke}%
\pgfsetdash{}{0pt}%
\pgfsys@defobject{currentmarker}{\pgfqpoint{-0.027778in}{0.000000in}}{\pgfqpoint{-0.000000in}{0.000000in}}{%
\pgfpathmoveto{\pgfqpoint{-0.000000in}{0.000000in}}%
\pgfpathlineto{\pgfqpoint{-0.027778in}{0.000000in}}%
\pgfusepath{stroke,fill}%
}%
\begin{pgfscope}%
\pgfsys@transformshift{0.703526in}{1.339075in}%
\pgfsys@useobject{currentmarker}{}%
\end{pgfscope}%
\end{pgfscope}%
\begin{pgfscope}%
\pgfsetbuttcap%
\pgfsetroundjoin%
\definecolor{currentfill}{rgb}{0.000000,0.000000,0.000000}%
\pgfsetfillcolor{currentfill}%
\pgfsetlinewidth{0.602250pt}%
\definecolor{currentstroke}{rgb}{0.000000,0.000000,0.000000}%
\pgfsetstrokecolor{currentstroke}%
\pgfsetdash{}{0pt}%
\pgfsys@defobject{currentmarker}{\pgfqpoint{-0.027778in}{0.000000in}}{\pgfqpoint{-0.000000in}{0.000000in}}{%
\pgfpathmoveto{\pgfqpoint{-0.000000in}{0.000000in}}%
\pgfpathlineto{\pgfqpoint{-0.027778in}{0.000000in}}%
\pgfusepath{stroke,fill}%
}%
\begin{pgfscope}%
\pgfsys@transformshift{0.703526in}{1.772804in}%
\pgfsys@useobject{currentmarker}{}%
\end{pgfscope}%
\end{pgfscope}%
\begin{pgfscope}%
\pgfsetbuttcap%
\pgfsetroundjoin%
\definecolor{currentfill}{rgb}{0.000000,0.000000,0.000000}%
\pgfsetfillcolor{currentfill}%
\pgfsetlinewidth{0.602250pt}%
\definecolor{currentstroke}{rgb}{0.000000,0.000000,0.000000}%
\pgfsetstrokecolor{currentstroke}%
\pgfsetdash{}{0pt}%
\pgfsys@defobject{currentmarker}{\pgfqpoint{-0.027778in}{0.000000in}}{\pgfqpoint{-0.000000in}{0.000000in}}{%
\pgfpathmoveto{\pgfqpoint{-0.000000in}{0.000000in}}%
\pgfpathlineto{\pgfqpoint{-0.027778in}{0.000000in}}%
\pgfusepath{stroke,fill}%
}%
\begin{pgfscope}%
\pgfsys@transformshift{0.703526in}{1.989669in}%
\pgfsys@useobject{currentmarker}{}%
\end{pgfscope}%
\end{pgfscope}%
\begin{pgfscope}%
\pgfsetbuttcap%
\pgfsetroundjoin%
\definecolor{currentfill}{rgb}{0.000000,0.000000,0.000000}%
\pgfsetfillcolor{currentfill}%
\pgfsetlinewidth{0.602250pt}%
\definecolor{currentstroke}{rgb}{0.000000,0.000000,0.000000}%
\pgfsetstrokecolor{currentstroke}%
\pgfsetdash{}{0pt}%
\pgfsys@defobject{currentmarker}{\pgfqpoint{-0.027778in}{0.000000in}}{\pgfqpoint{-0.000000in}{0.000000in}}{%
\pgfpathmoveto{\pgfqpoint{-0.000000in}{0.000000in}}%
\pgfpathlineto{\pgfqpoint{-0.027778in}{0.000000in}}%
\pgfusepath{stroke,fill}%
}%
\begin{pgfscope}%
\pgfsys@transformshift{0.703526in}{2.206534in}%
\pgfsys@useobject{currentmarker}{}%
\end{pgfscope}%
\end{pgfscope}%
\begin{pgfscope}%
\pgfsetbuttcap%
\pgfsetroundjoin%
\definecolor{currentfill}{rgb}{0.000000,0.000000,0.000000}%
\pgfsetfillcolor{currentfill}%
\pgfsetlinewidth{0.602250pt}%
\definecolor{currentstroke}{rgb}{0.000000,0.000000,0.000000}%
\pgfsetstrokecolor{currentstroke}%
\pgfsetdash{}{0pt}%
\pgfsys@defobject{currentmarker}{\pgfqpoint{-0.027778in}{0.000000in}}{\pgfqpoint{-0.000000in}{0.000000in}}{%
\pgfpathmoveto{\pgfqpoint{-0.000000in}{0.000000in}}%
\pgfpathlineto{\pgfqpoint{-0.027778in}{0.000000in}}%
\pgfusepath{stroke,fill}%
}%
\begin{pgfscope}%
\pgfsys@transformshift{0.703526in}{2.640263in}%
\pgfsys@useobject{currentmarker}{}%
\end{pgfscope}%
\end{pgfscope}%
\begin{pgfscope}%
\pgfsetbuttcap%
\pgfsetroundjoin%
\definecolor{currentfill}{rgb}{0.000000,0.000000,0.000000}%
\pgfsetfillcolor{currentfill}%
\pgfsetlinewidth{0.602250pt}%
\definecolor{currentstroke}{rgb}{0.000000,0.000000,0.000000}%
\pgfsetstrokecolor{currentstroke}%
\pgfsetdash{}{0pt}%
\pgfsys@defobject{currentmarker}{\pgfqpoint{-0.027778in}{0.000000in}}{\pgfqpoint{-0.000000in}{0.000000in}}{%
\pgfpathmoveto{\pgfqpoint{-0.000000in}{0.000000in}}%
\pgfpathlineto{\pgfqpoint{-0.027778in}{0.000000in}}%
\pgfusepath{stroke,fill}%
}%
\begin{pgfscope}%
\pgfsys@transformshift{0.703526in}{2.857128in}%
\pgfsys@useobject{currentmarker}{}%
\end{pgfscope}%
\end{pgfscope}%
\begin{pgfscope}%
\pgfsetbuttcap%
\pgfsetroundjoin%
\definecolor{currentfill}{rgb}{0.000000,0.000000,0.000000}%
\pgfsetfillcolor{currentfill}%
\pgfsetlinewidth{0.602250pt}%
\definecolor{currentstroke}{rgb}{0.000000,0.000000,0.000000}%
\pgfsetstrokecolor{currentstroke}%
\pgfsetdash{}{0pt}%
\pgfsys@defobject{currentmarker}{\pgfqpoint{-0.027778in}{0.000000in}}{\pgfqpoint{-0.000000in}{0.000000in}}{%
\pgfpathmoveto{\pgfqpoint{-0.000000in}{0.000000in}}%
\pgfpathlineto{\pgfqpoint{-0.027778in}{0.000000in}}%
\pgfusepath{stroke,fill}%
}%
\begin{pgfscope}%
\pgfsys@transformshift{0.703526in}{3.073992in}%
\pgfsys@useobject{currentmarker}{}%
\end{pgfscope}%
\end{pgfscope}%
\begin{pgfscope}%
\pgfsetbuttcap%
\pgfsetroundjoin%
\definecolor{currentfill}{rgb}{0.000000,0.000000,0.000000}%
\pgfsetfillcolor{currentfill}%
\pgfsetlinewidth{0.602250pt}%
\definecolor{currentstroke}{rgb}{0.000000,0.000000,0.000000}%
\pgfsetstrokecolor{currentstroke}%
\pgfsetdash{}{0pt}%
\pgfsys@defobject{currentmarker}{\pgfqpoint{-0.027778in}{0.000000in}}{\pgfqpoint{-0.000000in}{0.000000in}}{%
\pgfpathmoveto{\pgfqpoint{-0.000000in}{0.000000in}}%
\pgfpathlineto{\pgfqpoint{-0.027778in}{0.000000in}}%
\pgfusepath{stroke,fill}%
}%
\begin{pgfscope}%
\pgfsys@transformshift{0.703526in}{3.507722in}%
\pgfsys@useobject{currentmarker}{}%
\end{pgfscope}%
\end{pgfscope}%
\begin{pgfscope}%
\pgfsetbuttcap%
\pgfsetroundjoin%
\definecolor{currentfill}{rgb}{0.000000,0.000000,0.000000}%
\pgfsetfillcolor{currentfill}%
\pgfsetlinewidth{0.602250pt}%
\definecolor{currentstroke}{rgb}{0.000000,0.000000,0.000000}%
\pgfsetstrokecolor{currentstroke}%
\pgfsetdash{}{0pt}%
\pgfsys@defobject{currentmarker}{\pgfqpoint{-0.027778in}{0.000000in}}{\pgfqpoint{-0.000000in}{0.000000in}}{%
\pgfpathmoveto{\pgfqpoint{-0.000000in}{0.000000in}}%
\pgfpathlineto{\pgfqpoint{-0.027778in}{0.000000in}}%
\pgfusepath{stroke,fill}%
}%
\begin{pgfscope}%
\pgfsys@transformshift{0.703526in}{3.724587in}%
\pgfsys@useobject{currentmarker}{}%
\end{pgfscope}%
\end{pgfscope}%
\begin{pgfscope}%
\pgfsetbuttcap%
\pgfsetroundjoin%
\definecolor{currentfill}{rgb}{0.000000,0.000000,0.000000}%
\pgfsetfillcolor{currentfill}%
\pgfsetlinewidth{0.602250pt}%
\definecolor{currentstroke}{rgb}{0.000000,0.000000,0.000000}%
\pgfsetstrokecolor{currentstroke}%
\pgfsetdash{}{0pt}%
\pgfsys@defobject{currentmarker}{\pgfqpoint{-0.027778in}{0.000000in}}{\pgfqpoint{-0.000000in}{0.000000in}}{%
\pgfpathmoveto{\pgfqpoint{-0.000000in}{0.000000in}}%
\pgfpathlineto{\pgfqpoint{-0.027778in}{0.000000in}}%
\pgfusepath{stroke,fill}%
}%
\begin{pgfscope}%
\pgfsys@transformshift{0.703526in}{3.941451in}%
\pgfsys@useobject{currentmarker}{}%
\end{pgfscope}%
\end{pgfscope}%
\begin{pgfscope}%
\pgfsetbuttcap%
\pgfsetroundjoin%
\definecolor{currentfill}{rgb}{0.000000,0.000000,0.000000}%
\pgfsetfillcolor{currentfill}%
\pgfsetlinewidth{0.602250pt}%
\definecolor{currentstroke}{rgb}{0.000000,0.000000,0.000000}%
\pgfsetstrokecolor{currentstroke}%
\pgfsetdash{}{0pt}%
\pgfsys@defobject{currentmarker}{\pgfqpoint{-0.027778in}{0.000000in}}{\pgfqpoint{-0.000000in}{0.000000in}}{%
\pgfpathmoveto{\pgfqpoint{-0.000000in}{0.000000in}}%
\pgfpathlineto{\pgfqpoint{-0.027778in}{0.000000in}}%
\pgfusepath{stroke,fill}%
}%
\begin{pgfscope}%
\pgfsys@transformshift{0.703526in}{4.375181in}%
\pgfsys@useobject{currentmarker}{}%
\end{pgfscope}%
\end{pgfscope}%
\begin{pgfscope}%
\pgfsetbuttcap%
\pgfsetroundjoin%
\definecolor{currentfill}{rgb}{0.000000,0.000000,0.000000}%
\pgfsetfillcolor{currentfill}%
\pgfsetlinewidth{0.602250pt}%
\definecolor{currentstroke}{rgb}{0.000000,0.000000,0.000000}%
\pgfsetstrokecolor{currentstroke}%
\pgfsetdash{}{0pt}%
\pgfsys@defobject{currentmarker}{\pgfqpoint{-0.027778in}{0.000000in}}{\pgfqpoint{-0.000000in}{0.000000in}}{%
\pgfpathmoveto{\pgfqpoint{-0.000000in}{0.000000in}}%
\pgfpathlineto{\pgfqpoint{-0.027778in}{0.000000in}}%
\pgfusepath{stroke,fill}%
}%
\begin{pgfscope}%
\pgfsys@transformshift{0.703526in}{4.592045in}%
\pgfsys@useobject{currentmarker}{}%
\end{pgfscope}%
\end{pgfscope}%
\begin{pgfscope}%
\pgfsetbuttcap%
\pgfsetroundjoin%
\definecolor{currentfill}{rgb}{0.000000,0.000000,0.000000}%
\pgfsetfillcolor{currentfill}%
\pgfsetlinewidth{0.602250pt}%
\definecolor{currentstroke}{rgb}{0.000000,0.000000,0.000000}%
\pgfsetstrokecolor{currentstroke}%
\pgfsetdash{}{0pt}%
\pgfsys@defobject{currentmarker}{\pgfqpoint{-0.027778in}{0.000000in}}{\pgfqpoint{-0.000000in}{0.000000in}}{%
\pgfpathmoveto{\pgfqpoint{-0.000000in}{0.000000in}}%
\pgfpathlineto{\pgfqpoint{-0.027778in}{0.000000in}}%
\pgfusepath{stroke,fill}%
}%
\begin{pgfscope}%
\pgfsys@transformshift{0.703526in}{4.808910in}%
\pgfsys@useobject{currentmarker}{}%
\end{pgfscope}%
\end{pgfscope}%
\begin{pgfscope}%
\pgfsetbuttcap%
\pgfsetroundjoin%
\definecolor{currentfill}{rgb}{0.000000,0.000000,0.000000}%
\pgfsetfillcolor{currentfill}%
\pgfsetlinewidth{0.602250pt}%
\definecolor{currentstroke}{rgb}{0.000000,0.000000,0.000000}%
\pgfsetstrokecolor{currentstroke}%
\pgfsetdash{}{0pt}%
\pgfsys@defobject{currentmarker}{\pgfqpoint{-0.027778in}{0.000000in}}{\pgfqpoint{-0.000000in}{0.000000in}}{%
\pgfpathmoveto{\pgfqpoint{-0.000000in}{0.000000in}}%
\pgfpathlineto{\pgfqpoint{-0.027778in}{0.000000in}}%
\pgfusepath{stroke,fill}%
}%
\begin{pgfscope}%
\pgfsys@transformshift{0.703526in}{5.242640in}%
\pgfsys@useobject{currentmarker}{}%
\end{pgfscope}%
\end{pgfscope}%
\begin{pgfscope}%
\pgfsetbuttcap%
\pgfsetroundjoin%
\definecolor{currentfill}{rgb}{0.000000,0.000000,0.000000}%
\pgfsetfillcolor{currentfill}%
\pgfsetlinewidth{0.602250pt}%
\definecolor{currentstroke}{rgb}{0.000000,0.000000,0.000000}%
\pgfsetstrokecolor{currentstroke}%
\pgfsetdash{}{0pt}%
\pgfsys@defobject{currentmarker}{\pgfqpoint{-0.027778in}{0.000000in}}{\pgfqpoint{-0.000000in}{0.000000in}}{%
\pgfpathmoveto{\pgfqpoint{-0.000000in}{0.000000in}}%
\pgfpathlineto{\pgfqpoint{-0.027778in}{0.000000in}}%
\pgfusepath{stroke,fill}%
}%
\begin{pgfscope}%
\pgfsys@transformshift{0.703526in}{5.459504in}%
\pgfsys@useobject{currentmarker}{}%
\end{pgfscope}%
\end{pgfscope}%
\begin{pgfscope}%
\pgfsetbuttcap%
\pgfsetroundjoin%
\definecolor{currentfill}{rgb}{0.000000,0.000000,0.000000}%
\pgfsetfillcolor{currentfill}%
\pgfsetlinewidth{0.602250pt}%
\definecolor{currentstroke}{rgb}{0.000000,0.000000,0.000000}%
\pgfsetstrokecolor{currentstroke}%
\pgfsetdash{}{0pt}%
\pgfsys@defobject{currentmarker}{\pgfqpoint{-0.027778in}{0.000000in}}{\pgfqpoint{-0.000000in}{0.000000in}}{%
\pgfpathmoveto{\pgfqpoint{-0.000000in}{0.000000in}}%
\pgfpathlineto{\pgfqpoint{-0.027778in}{0.000000in}}%
\pgfusepath{stroke,fill}%
}%
\begin{pgfscope}%
\pgfsys@transformshift{0.703526in}{5.676369in}%
\pgfsys@useobject{currentmarker}{}%
\end{pgfscope}%
\end{pgfscope}%
\begin{pgfscope}%
\pgfsetbuttcap%
\pgfsetroundjoin%
\definecolor{currentfill}{rgb}{0.000000,0.000000,0.000000}%
\pgfsetfillcolor{currentfill}%
\pgfsetlinewidth{0.602250pt}%
\definecolor{currentstroke}{rgb}{0.000000,0.000000,0.000000}%
\pgfsetstrokecolor{currentstroke}%
\pgfsetdash{}{0pt}%
\pgfsys@defobject{currentmarker}{\pgfqpoint{-0.027778in}{0.000000in}}{\pgfqpoint{-0.000000in}{0.000000in}}{%
\pgfpathmoveto{\pgfqpoint{-0.000000in}{0.000000in}}%
\pgfpathlineto{\pgfqpoint{-0.027778in}{0.000000in}}%
\pgfusepath{stroke,fill}%
}%
\begin{pgfscope}%
\pgfsys@transformshift{0.703526in}{6.110098in}%
\pgfsys@useobject{currentmarker}{}%
\end{pgfscope}%
\end{pgfscope}%
\begin{pgfscope}%
\pgfsetbuttcap%
\pgfsetroundjoin%
\definecolor{currentfill}{rgb}{0.000000,0.000000,0.000000}%
\pgfsetfillcolor{currentfill}%
\pgfsetlinewidth{0.602250pt}%
\definecolor{currentstroke}{rgb}{0.000000,0.000000,0.000000}%
\pgfsetstrokecolor{currentstroke}%
\pgfsetdash{}{0pt}%
\pgfsys@defobject{currentmarker}{\pgfqpoint{-0.027778in}{0.000000in}}{\pgfqpoint{-0.000000in}{0.000000in}}{%
\pgfpathmoveto{\pgfqpoint{-0.000000in}{0.000000in}}%
\pgfpathlineto{\pgfqpoint{-0.027778in}{0.000000in}}%
\pgfusepath{stroke,fill}%
}%
\begin{pgfscope}%
\pgfsys@transformshift{0.703526in}{6.326963in}%
\pgfsys@useobject{currentmarker}{}%
\end{pgfscope}%
\end{pgfscope}%
\begin{pgfscope}%
\pgfsetbuttcap%
\pgfsetroundjoin%
\definecolor{currentfill}{rgb}{0.000000,0.000000,0.000000}%
\pgfsetfillcolor{currentfill}%
\pgfsetlinewidth{0.602250pt}%
\definecolor{currentstroke}{rgb}{0.000000,0.000000,0.000000}%
\pgfsetstrokecolor{currentstroke}%
\pgfsetdash{}{0pt}%
\pgfsys@defobject{currentmarker}{\pgfqpoint{-0.027778in}{0.000000in}}{\pgfqpoint{-0.000000in}{0.000000in}}{%
\pgfpathmoveto{\pgfqpoint{-0.000000in}{0.000000in}}%
\pgfpathlineto{\pgfqpoint{-0.027778in}{0.000000in}}%
\pgfusepath{stroke,fill}%
}%
\begin{pgfscope}%
\pgfsys@transformshift{0.703526in}{6.543828in}%
\pgfsys@useobject{currentmarker}{}%
\end{pgfscope}%
\end{pgfscope}%
\begin{pgfscope}%
\pgfsetbuttcap%
\pgfsetroundjoin%
\definecolor{currentfill}{rgb}{0.000000,0.000000,0.000000}%
\pgfsetfillcolor{currentfill}%
\pgfsetlinewidth{0.602250pt}%
\definecolor{currentstroke}{rgb}{0.000000,0.000000,0.000000}%
\pgfsetstrokecolor{currentstroke}%
\pgfsetdash{}{0pt}%
\pgfsys@defobject{currentmarker}{\pgfqpoint{-0.027778in}{0.000000in}}{\pgfqpoint{-0.000000in}{0.000000in}}{%
\pgfpathmoveto{\pgfqpoint{-0.000000in}{0.000000in}}%
\pgfpathlineto{\pgfqpoint{-0.027778in}{0.000000in}}%
\pgfusepath{stroke,fill}%
}%
\begin{pgfscope}%
\pgfsys@transformshift{0.703526in}{6.977557in}%
\pgfsys@useobject{currentmarker}{}%
\end{pgfscope}%
\end{pgfscope}%
\begin{pgfscope}%
\pgfsetbuttcap%
\pgfsetroundjoin%
\definecolor{currentfill}{rgb}{0.000000,0.000000,0.000000}%
\pgfsetfillcolor{currentfill}%
\pgfsetlinewidth{0.602250pt}%
\definecolor{currentstroke}{rgb}{0.000000,0.000000,0.000000}%
\pgfsetstrokecolor{currentstroke}%
\pgfsetdash{}{0pt}%
\pgfsys@defobject{currentmarker}{\pgfqpoint{-0.027778in}{0.000000in}}{\pgfqpoint{-0.000000in}{0.000000in}}{%
\pgfpathmoveto{\pgfqpoint{-0.000000in}{0.000000in}}%
\pgfpathlineto{\pgfqpoint{-0.027778in}{0.000000in}}%
\pgfusepath{stroke,fill}%
}%
\begin{pgfscope}%
\pgfsys@transformshift{0.703526in}{7.194422in}%
\pgfsys@useobject{currentmarker}{}%
\end{pgfscope}%
\end{pgfscope}%
\begin{pgfscope}%
\pgfsetbuttcap%
\pgfsetroundjoin%
\definecolor{currentfill}{rgb}{0.000000,0.000000,0.000000}%
\pgfsetfillcolor{currentfill}%
\pgfsetlinewidth{0.602250pt}%
\definecolor{currentstroke}{rgb}{0.000000,0.000000,0.000000}%
\pgfsetstrokecolor{currentstroke}%
\pgfsetdash{}{0pt}%
\pgfsys@defobject{currentmarker}{\pgfqpoint{-0.027778in}{0.000000in}}{\pgfqpoint{-0.000000in}{0.000000in}}{%
\pgfpathmoveto{\pgfqpoint{-0.000000in}{0.000000in}}%
\pgfpathlineto{\pgfqpoint{-0.027778in}{0.000000in}}%
\pgfusepath{stroke,fill}%
}%
\begin{pgfscope}%
\pgfsys@transformshift{0.703526in}{7.411287in}%
\pgfsys@useobject{currentmarker}{}%
\end{pgfscope}%
\end{pgfscope}%
\begin{pgfscope}%
\definecolor{textcolor}{rgb}{0.000000,0.000000,0.000000}%
\pgfsetstrokecolor{textcolor}%
\pgfsetfillcolor{textcolor}%
\pgftext[x=0.387555in,y=4.085981in,,bottom,rotate=90.000000]{\color{textcolor}\rmfamily\fontsize{20.000000}{24.000000}\selectfont Wind Capacity [GW]}%
\end{pgfscope}%
\begin{pgfscope}%
\pgfpathrectangle{\pgfqpoint{0.703526in}{0.688481in}}{\pgfqpoint{9.300000in}{6.795000in}}%
\pgfusepath{clip}%
\pgfsetrectcap%
\pgfsetroundjoin%
\pgfsetlinewidth{1.505625pt}%
\definecolor{currentstroke}{rgb}{0.121569,0.466667,0.705882}%
\pgfsetstrokecolor{currentstroke}%
\pgfsetdash{}{0pt}%
\pgfpathmoveto{\pgfqpoint{0.703526in}{6.248026in}}%
\pgfpathlineto{\pgfqpoint{8.312617in}{0.688481in}}%
\pgfpathlineto{\pgfqpoint{8.312617in}{0.688481in}}%
\pgfusepath{stroke}%
\end{pgfscope}%
\begin{pgfscope}%
\pgfpathrectangle{\pgfqpoint{0.703526in}{0.688481in}}{\pgfqpoint{9.300000in}{6.795000in}}%
\pgfusepath{clip}%
\pgfsetbuttcap%
\pgfsetroundjoin%
\pgfsetlinewidth{1.505625pt}%
\definecolor{currentstroke}{rgb}{0.121569,0.466667,0.705882}%
\pgfsetstrokecolor{currentstroke}%
\pgfsetdash{{5.550000pt}{2.400000pt}}{0.000000pt}%
\pgfpathmoveto{\pgfqpoint{0.703526in}{5.237200in}}%
\pgfpathlineto{\pgfqpoint{6.929146in}{0.688481in}}%
\pgfpathlineto{\pgfqpoint{6.929146in}{0.688481in}}%
\pgfusepath{stroke}%
\end{pgfscope}%
\begin{pgfscope}%
\pgfpathrectangle{\pgfqpoint{0.703526in}{0.688481in}}{\pgfqpoint{9.300000in}{6.795000in}}%
\pgfusepath{clip}%
\pgfsetbuttcap%
\pgfsetroundjoin%
\pgfsetlinewidth{1.505625pt}%
\definecolor{currentstroke}{rgb}{0.121569,0.466667,0.705882}%
\pgfsetstrokecolor{currentstroke}%
\pgfsetdash{{5.550000pt}{2.400000pt}}{0.000000pt}%
\pgfpathmoveto{\pgfqpoint{0.703526in}{7.483481in}}%
\pgfpathlineto{\pgfqpoint{10.003526in}{0.688481in}}%
\pgfpathlineto{\pgfqpoint{10.003526in}{0.688481in}}%
\pgfusepath{stroke}%
\end{pgfscope}%
\begin{pgfscope}%
\pgfpathrectangle{\pgfqpoint{0.703526in}{0.688481in}}{\pgfqpoint{9.300000in}{6.795000in}}%
\pgfusepath{clip}%
\pgfsetrectcap%
\pgfsetroundjoin%
\pgfsetlinewidth{1.505625pt}%
\definecolor{currentstroke}{rgb}{0.839216,0.152941,0.156863}%
\pgfsetstrokecolor{currentstroke}%
\pgfsetdash{}{0pt}%
\pgfpathmoveto{\pgfqpoint{1.527174in}{7.485981in}}%
\pgfpathlineto{\pgfqpoint{6.874499in}{0.688481in}}%
\pgfpathlineto{\pgfqpoint{6.874499in}{0.688481in}}%
\pgfusepath{stroke}%
\end{pgfscope}%
\begin{pgfscope}%
\pgfpathrectangle{\pgfqpoint{0.703526in}{0.688481in}}{\pgfqpoint{9.300000in}{6.795000in}}%
\pgfusepath{clip}%
\pgfsetbuttcap%
\pgfsetroundjoin%
\pgfsetlinewidth{1.505625pt}%
\definecolor{currentstroke}{rgb}{0.839216,0.152941,0.156863}%
\pgfsetstrokecolor{currentstroke}%
\pgfsetdash{{5.550000pt}{2.400000pt}}{0.000000pt}%
\pgfpathmoveto{\pgfqpoint{0.703526in}{6.464899in}}%
\pgfpathlineto{\pgfqpoint{5.247606in}{0.688481in}}%
\pgfpathlineto{\pgfqpoint{5.247606in}{0.688481in}}%
\pgfusepath{stroke}%
\end{pgfscope}%
\begin{pgfscope}%
\pgfpathrectangle{\pgfqpoint{0.703526in}{0.688481in}}{\pgfqpoint{9.300000in}{6.795000in}}%
\pgfusepath{clip}%
\pgfsetbuttcap%
\pgfsetroundjoin%
\pgfsetlinewidth{1.505625pt}%
\definecolor{currentstroke}{rgb}{0.839216,0.152941,0.156863}%
\pgfsetstrokecolor{currentstroke}%
\pgfsetdash{{5.550000pt}{2.400000pt}}{0.000000pt}%
\pgfpathmoveto{\pgfqpoint{3.652731in}{7.485981in}}%
\pgfpathlineto{\pgfqpoint{9.000056in}{0.688481in}}%
\pgfpathlineto{\pgfqpoint{9.000056in}{0.688481in}}%
\pgfusepath{stroke}%
\end{pgfscope}%
\begin{pgfscope}%
\pgfsetrectcap%
\pgfsetmiterjoin%
\pgfsetlinewidth{0.803000pt}%
\definecolor{currentstroke}{rgb}{0.000000,0.000000,0.000000}%
\pgfsetstrokecolor{currentstroke}%
\pgfsetdash{}{0pt}%
\pgfpathmoveto{\pgfqpoint{0.703526in}{0.688481in}}%
\pgfpathlineto{\pgfqpoint{0.703526in}{7.483481in}}%
\pgfusepath{stroke}%
\end{pgfscope}%
\begin{pgfscope}%
\pgfsetrectcap%
\pgfsetmiterjoin%
\pgfsetlinewidth{0.803000pt}%
\definecolor{currentstroke}{rgb}{0.000000,0.000000,0.000000}%
\pgfsetstrokecolor{currentstroke}%
\pgfsetdash{}{0pt}%
\pgfpathmoveto{\pgfqpoint{10.003526in}{0.688481in}}%
\pgfpathlineto{\pgfqpoint{10.003526in}{7.483481in}}%
\pgfusepath{stroke}%
\end{pgfscope}%
\begin{pgfscope}%
\pgfsetrectcap%
\pgfsetmiterjoin%
\pgfsetlinewidth{0.803000pt}%
\definecolor{currentstroke}{rgb}{0.000000,0.000000,0.000000}%
\pgfsetstrokecolor{currentstroke}%
\pgfsetdash{}{0pt}%
\pgfpathmoveto{\pgfqpoint{0.703526in}{0.688481in}}%
\pgfpathlineto{\pgfqpoint{10.003526in}{0.688481in}}%
\pgfusepath{stroke}%
\end{pgfscope}%
\begin{pgfscope}%
\pgfsetrectcap%
\pgfsetmiterjoin%
\pgfsetlinewidth{0.803000pt}%
\definecolor{currentstroke}{rgb}{0.000000,0.000000,0.000000}%
\pgfsetstrokecolor{currentstroke}%
\pgfsetdash{}{0pt}%
\pgfpathmoveto{\pgfqpoint{0.703526in}{7.483481in}}%
\pgfpathlineto{\pgfqpoint{10.003526in}{7.483481in}}%
\pgfusepath{stroke}%
\end{pgfscope}%
\begin{pgfscope}%
\definecolor{textcolor}{rgb}{0.000000,0.000000,0.000000}%
\pgfsetstrokecolor{textcolor}%
\pgfsetfillcolor{textcolor}%
\pgftext[x=2.044845in, y=7.921991in, left, base]{\color{textcolor}\rmfamily\fontsize{24.000000}{28.800000}\selectfont What's the Optimal Mix of Renewable Energy?}%
\end{pgfscope}%
\begin{pgfscope}%
\definecolor{textcolor}{rgb}{0.000000,0.000000,0.000000}%
\pgfsetstrokecolor{textcolor}%
\pgfsetfillcolor{textcolor}%
\pgftext[x=2.865121in, y=7.566814in, left, base]{\color{textcolor}\rmfamily\fontsize{24.000000}{28.800000}\selectfont Parametric Uncertainties Visualized}%
\end{pgfscope}%
\begin{pgfscope}%
\pgfsetbuttcap%
\pgfsetmiterjoin%
\definecolor{currentfill}{rgb}{0.248235,0.248235,0.248235}%
\pgfsetfillcolor{currentfill}%
\pgfsetfillopacity{0.500000}%
\pgfsetlinewidth{1.003750pt}%
\definecolor{currentstroke}{rgb}{0.248235,0.248235,0.248235}%
\pgfsetstrokecolor{currentstroke}%
\pgfsetstrokeopacity{0.500000}%
\pgfsetdash{}{0pt}%
\pgfpathmoveto{\pgfqpoint{7.018300in}{5.375150in}}%
\pgfpathlineto{\pgfqpoint{9.895193in}{5.375150in}}%
\pgfpathquadraticcurveto{\pgfqpoint{9.934081in}{5.375150in}}{\pgfqpoint{9.934081in}{5.414039in}}%
\pgfpathlineto{\pgfqpoint{9.934081in}{7.319592in}}%
\pgfpathquadraticcurveto{\pgfqpoint{9.934081in}{7.358481in}}{\pgfqpoint{9.895193in}{7.358481in}}%
\pgfpathlineto{\pgfqpoint{7.018300in}{7.358481in}}%
\pgfpathquadraticcurveto{\pgfqpoint{6.979411in}{7.358481in}}{\pgfqpoint{6.979411in}{7.319592in}}%
\pgfpathlineto{\pgfqpoint{6.979411in}{5.414039in}}%
\pgfpathquadraticcurveto{\pgfqpoint{6.979411in}{5.375150in}}{\pgfqpoint{7.018300in}{5.375150in}}%
\pgfpathclose%
\pgfusepath{stroke,fill}%
\end{pgfscope}%
\begin{pgfscope}%
\pgfsetbuttcap%
\pgfsetmiterjoin%
\definecolor{currentfill}{rgb}{0.827451,0.827451,0.827451}%
\pgfsetfillcolor{currentfill}%
\pgfsetlinewidth{1.003750pt}%
\definecolor{currentstroke}{rgb}{0.800000,0.800000,0.800000}%
\pgfsetstrokecolor{currentstroke}%
\pgfsetdash{}{0pt}%
\pgfpathmoveto{\pgfqpoint{6.990522in}{5.402928in}}%
\pgfpathlineto{\pgfqpoint{9.867415in}{5.402928in}}%
\pgfpathquadraticcurveto{\pgfqpoint{9.906304in}{5.402928in}}{\pgfqpoint{9.906304in}{5.441817in}}%
\pgfpathlineto{\pgfqpoint{9.906304in}{7.347369in}}%
\pgfpathquadraticcurveto{\pgfqpoint{9.906304in}{7.386258in}}{\pgfqpoint{9.867415in}{7.386258in}}%
\pgfpathlineto{\pgfqpoint{6.990522in}{7.386258in}}%
\pgfpathquadraticcurveto{\pgfqpoint{6.951633in}{7.386258in}}{\pgfqpoint{6.951633in}{7.347369in}}%
\pgfpathlineto{\pgfqpoint{6.951633in}{5.441817in}}%
\pgfpathquadraticcurveto{\pgfqpoint{6.951633in}{5.402928in}}{\pgfqpoint{6.990522in}{5.402928in}}%
\pgfpathclose%
\pgfusepath{stroke,fill}%
\end{pgfscope}%
\begin{pgfscope}%
\pgfsetrectcap%
\pgfsetroundjoin%
\pgfsetlinewidth{1.505625pt}%
\definecolor{currentstroke}{rgb}{0.121569,0.466667,0.705882}%
\pgfsetstrokecolor{currentstroke}%
\pgfsetdash{}{0pt}%
\pgfpathmoveto{\pgfqpoint{7.029411in}{7.237648in}}%
\pgfpathlineto{\pgfqpoint{7.418300in}{7.237648in}}%
\pgfusepath{stroke}%
\end{pgfscope}%
\begin{pgfscope}%
\definecolor{textcolor}{rgb}{0.000000,0.000000,0.000000}%
\pgfsetstrokecolor{textcolor}%
\pgfsetfillcolor{textcolor}%
\pgftext[x=7.573855in,y=7.169592in,left,base]{\color{textcolor}\rmfamily\fontsize{14.000000}{16.800000}\selectfont Demand Constraint:Mean}%
\end{pgfscope}%
\begin{pgfscope}%
\pgfsetbuttcap%
\pgfsetroundjoin%
\pgfsetlinewidth{1.505625pt}%
\definecolor{currentstroke}{rgb}{0.121569,0.466667,0.705882}%
\pgfsetstrokecolor{currentstroke}%
\pgfsetdash{{5.550000pt}{2.400000pt}}{0.000000pt}%
\pgfpathmoveto{\pgfqpoint{7.029411in}{6.962648in}}%
\pgfpathlineto{\pgfqpoint{7.418300in}{6.962648in}}%
\pgfusepath{stroke}%
\end{pgfscope}%
\begin{pgfscope}%
\definecolor{textcolor}{rgb}{0.000000,0.000000,0.000000}%
\pgfsetstrokecolor{textcolor}%
\pgfsetfillcolor{textcolor}%
\pgftext[x=7.573855in,y=6.894592in,left,base]{\color{textcolor}\rmfamily\fontsize{14.000000}{16.800000}\selectfont Demand Constraint: LB}%
\end{pgfscope}%
\begin{pgfscope}%
\pgfsetbuttcap%
\pgfsetroundjoin%
\pgfsetlinewidth{1.505625pt}%
\definecolor{currentstroke}{rgb}{0.121569,0.466667,0.705882}%
\pgfsetstrokecolor{currentstroke}%
\pgfsetdash{{5.550000pt}{2.400000pt}}{0.000000pt}%
\pgfpathmoveto{\pgfqpoint{7.029411in}{6.687648in}}%
\pgfpathlineto{\pgfqpoint{7.418300in}{6.687648in}}%
\pgfusepath{stroke}%
\end{pgfscope}%
\begin{pgfscope}%
\definecolor{textcolor}{rgb}{0.000000,0.000000,0.000000}%
\pgfsetstrokecolor{textcolor}%
\pgfsetfillcolor{textcolor}%
\pgftext[x=7.573855in,y=6.619593in,left,base]{\color{textcolor}\rmfamily\fontsize{14.000000}{16.800000}\selectfont Demand Constraint: UB}%
\end{pgfscope}%
\begin{pgfscope}%
\pgfsetrectcap%
\pgfsetroundjoin%
\pgfsetlinewidth{1.505625pt}%
\definecolor{currentstroke}{rgb}{0.839216,0.152941,0.156863}%
\pgfsetstrokecolor{currentstroke}%
\pgfsetdash{}{0pt}%
\pgfpathmoveto{\pgfqpoint{7.029411in}{6.412649in}}%
\pgfpathlineto{\pgfqpoint{7.418300in}{6.412649in}}%
\pgfusepath{stroke}%
\end{pgfscope}%
\begin{pgfscope}%
\definecolor{textcolor}{rgb}{0.000000,0.000000,0.000000}%
\pgfsetstrokecolor{textcolor}%
\pgfsetfillcolor{textcolor}%
\pgftext[x=7.573855in,y=6.344593in,left,base]{\color{textcolor}\rmfamily\fontsize{14.000000}{16.800000}\selectfont Objective Function: Mean}%
\end{pgfscope}%
\begin{pgfscope}%
\pgfsetbuttcap%
\pgfsetroundjoin%
\pgfsetlinewidth{1.505625pt}%
\definecolor{currentstroke}{rgb}{0.839216,0.152941,0.156863}%
\pgfsetstrokecolor{currentstroke}%
\pgfsetdash{{5.550000pt}{2.400000pt}}{0.000000pt}%
\pgfpathmoveto{\pgfqpoint{7.029411in}{6.137649in}}%
\pgfpathlineto{\pgfqpoint{7.418300in}{6.137649in}}%
\pgfusepath{stroke}%
\end{pgfscope}%
\begin{pgfscope}%
\definecolor{textcolor}{rgb}{0.000000,0.000000,0.000000}%
\pgfsetstrokecolor{textcolor}%
\pgfsetfillcolor{textcolor}%
\pgftext[x=7.573855in,y=6.069594in,left,base]{\color{textcolor}\rmfamily\fontsize{14.000000}{16.800000}\selectfont Objective Function: LB}%
\end{pgfscope}%
\begin{pgfscope}%
\pgfsetbuttcap%
\pgfsetroundjoin%
\pgfsetlinewidth{1.505625pt}%
\definecolor{currentstroke}{rgb}{0.839216,0.152941,0.156863}%
\pgfsetstrokecolor{currentstroke}%
\pgfsetdash{{5.550000pt}{2.400000pt}}{0.000000pt}%
\pgfpathmoveto{\pgfqpoint{7.029411in}{5.862650in}}%
\pgfpathlineto{\pgfqpoint{7.418300in}{5.862650in}}%
\pgfusepath{stroke}%
\end{pgfscope}%
\begin{pgfscope}%
\definecolor{textcolor}{rgb}{0.000000,0.000000,0.000000}%
\pgfsetstrokecolor{textcolor}%
\pgfsetfillcolor{textcolor}%
\pgftext[x=7.573855in,y=5.794594in,left,base]{\color{textcolor}\rmfamily\fontsize{14.000000}{16.800000}\selectfont Objective Function: UB}%
\end{pgfscope}%
\begin{pgfscope}%
\pgfsetbuttcap%
\pgfsetroundjoin%
\definecolor{currentfill}{rgb}{0.839216,0.152941,0.156863}%
\pgfsetfillcolor{currentfill}%
\pgfsetlinewidth{1.003750pt}%
\definecolor{currentstroke}{rgb}{0.839216,0.152941,0.156863}%
\pgfsetstrokecolor{currentstroke}%
\pgfsetdash{}{0pt}%
\pgfsys@defobject{currentmarker}{\pgfqpoint{-0.065881in}{-0.065881in}}{\pgfqpoint{0.065881in}{0.065881in}}{%
\pgfpathmoveto{\pgfqpoint{0.000000in}{-0.065881in}}%
\pgfpathcurveto{\pgfqpoint{0.017472in}{-0.065881in}}{\pgfqpoint{0.034230in}{-0.058939in}}{\pgfqpoint{0.046585in}{-0.046585in}}%
\pgfpathcurveto{\pgfqpoint{0.058939in}{-0.034230in}}{\pgfqpoint{0.065881in}{-0.017472in}}{\pgfqpoint{0.065881in}{0.000000in}}%
\pgfpathcurveto{\pgfqpoint{0.065881in}{0.017472in}}{\pgfqpoint{0.058939in}{0.034230in}}{\pgfqpoint{0.046585in}{0.046585in}}%
\pgfpathcurveto{\pgfqpoint{0.034230in}{0.058939in}}{\pgfqpoint{0.017472in}{0.065881in}}{\pgfqpoint{0.000000in}{0.065881in}}%
\pgfpathcurveto{\pgfqpoint{-0.017472in}{0.065881in}}{\pgfqpoint{-0.034230in}{0.058939in}}{\pgfqpoint{-0.046585in}{0.046585in}}%
\pgfpathcurveto{\pgfqpoint{-0.058939in}{0.034230in}}{\pgfqpoint{-0.065881in}{0.017472in}}{\pgfqpoint{-0.065881in}{0.000000in}}%
\pgfpathcurveto{\pgfqpoint{-0.065881in}{-0.017472in}}{\pgfqpoint{-0.058939in}{-0.034230in}}{\pgfqpoint{-0.046585in}{-0.046585in}}%
\pgfpathcurveto{\pgfqpoint{-0.034230in}{-0.058939in}}{\pgfqpoint{-0.017472in}{-0.065881in}}{\pgfqpoint{0.000000in}{-0.065881in}}%
\pgfpathclose%
\pgfusepath{stroke,fill}%
}%
\begin{pgfscope}%
\pgfsys@transformshift{7.223855in}{5.570636in}%
\pgfsys@useobject{currentmarker}{}%
\end{pgfscope}%
\end{pgfscope}%
\begin{pgfscope}%
\definecolor{textcolor}{rgb}{0.000000,0.000000,0.000000}%
\pgfsetstrokecolor{textcolor}%
\pgfsetfillcolor{textcolor}%
\pgftext[x=7.573855in,y=5.519594in,left,base]{\color{textcolor}\rmfamily\fontsize{14.000000}{16.800000}\selectfont CEJA Goal}%
\end{pgfscope}%
\end{pgfpicture}%
\makeatother%
\endgroup%
}
  \caption{Decision space for the na\"{i}ve capacity expansion problem with
  10\% uncertainty applied to all parameters. The optimal space lies within
  overlapping blue and red regions. The red dot shows the \gls{ceja} goals for
  renewable energy, for reference.}
  \label{fig:param-fig}
\end{figure}

This example illustrates the \gls{mga} method and shows how the optimal space
expands under parametric uncertainties.
Although a useful demonstration, considering the optimal mixture of wind and solar
to achieve Illinois' clean energy goals in this manner and comparing it to
current policy oversimplifies the problem.
Many features of energy systems are not captured in this simple formulation,
such as the availability of solar power, need for backup or energy storage when
the weather fails, and other clean technology options. Thus \glspl{esom} are necessary.

In the following case studies, I apply \gls{mga} to explore near-optimal solutions
for the \gls{uiuc} energy system (Chapter \ref{chapter:uiuc}) and study the
sensitivity of Illinois' grid to wind and solar capacity factor profiles and time
resolution (Chapter \ref{chapter:illinois}).
