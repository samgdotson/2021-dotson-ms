\begin{tabular}{l*{4}{r}p{9.5cm}}
\toprule
                           Author & Timeslices$^\text{a}$ &         Horizon$^\text{b}$ &           Place &        Modeling Tool &                                                                                                                                                                                                                                                               Notes \\
\midrule
Alzbutas, R. \& Norvaisa, E., 2012 \cite{alzbutas_uncertainty_2012} &         NS &   (2010,2025,5) &       Lithuania &              MESSAGE &                                                             Authors investigate impact of SMR reactors on energy supply. Results show that cogeneration with SMRs lead to lowest system wide costs and that discount rate has the greatest influence on total cost. \\
                Baker et al. 2018 \cite{baker_optimal_2018} &       8760 &             NaN & CAISO Microgrid &  RAVEN, OpenModelica &                                   This study considers how LCOE and water prices vary with wind penetration and availability while optimizing battery storage and desalination capacity with an SMR. The results show that higher wind penetration increases costs. \\
      Barron R. \& McJeon H., 2015 \cite{barron_differential_2015} &         NS &   (2020,2095,5) &           World &                 GCAM &                                                                      This study investigates the impact of different parameters on carbon dioxide abatement costs. They found that the capital cost of nuclear reactors had the greatest impact on abatement costs. \\
              Bennett et al. 2021 \cite{bennett_extending_2021} &         48 &   (2015,2040,5) &     Puerto Rico &                Temoa &             In this work, authors evaluate the effect different grid topologies such as distributed versus centralized generation and infrastructure resilience on cost of electricity and carbon emissions. Only renewable energy and fossil fuels are considered. \\
            Bouckaert et al. 2014 \cite{bouckaert_expanding_2014}&         16 &   (2010,2030,5) &  Reunion Island &         TIMES/MARKAL & Considers the impact of demand response policies on renewable energy penetration. Results show that increasing the penetration of intermittent renewable sources decreases system reliability. Demand response can mitigate the effects of reliability constraints. \\
           de Queiroz et al. 2019 \cite{de_queiroz_repurposing_2019}&   Variable & One month ahead &             NaN &                Temoa &                                                         This work modifies the Temoa ESOM framework to solve unit-commitment problems by remapping the time-slices in Temoa. The modified ESOM solved 96 times faster than the corresponding unit commitment model. \\
         de Sisternes et al. 2016 \cite{de_sisternes_value_2016} &        672 &             NaN &      ERCOT-like &                 GenX &                                                                          The authors investigated how storage duration and capacity affects the levelized cost of electricity. They found that excluding nuclear power increases system costs by up to 8.6 percent. \\
            DeCarolis et al. 2016 \cite{decarolis_modelling_2016} &          4 &   (2015,2050,5) &   United States &                Temoa &                                              The authors demonstrate the MGA sensitivity analysis technique using the Temoa framework. They show that MGA can efficiently probe decision space and that increasing the slack variable increases technology options. \\
               Epiney et al. 2020 \cite{epiney_economic_2020} &       8760 &     Single year &             NaN &  RAVEN, OpenModelica &                                                                                                                                                                                                                      Demonstrates the RAVEN and OpenModelica tools. \\
               Garcia et al. 2016 \cite{garcia_dynamic_2016}&       8760 &     Single year &      Texas-like & OpenModelica, Dymola &                                                                                        This work is a case study of a nuclear hybrid energy system in Texas. Researchers found that integrating nuclear and renewable energy sources lead to highly flexible grids. \\
               Kotzur et al. 2017 \cite{kotzur_impact_2018} &   Variable &             NaN &             NaN &                 tsam & Method of time series aggregation plays a minor role compared with the type of system being modeled.\\
             Komiyama et al. 2015 \cite{komiyama_energy_2015} &      52560 &     Single year &           Japan &                 OPGM & Examines the integration of hydrogen storage for high penetration of renewable energy sources. Found that hydrogen is an economical for load shifting on a weekly-monthly scale.\\
                   Li et al. 2019 \cite{li_open_2020}&         96 &   (2015,2050,5) &  North Carolina &                Temoa &    This work examines a limited set of cost scenarios to determine the break even capital cost of each generating technology to show how close each technology is to deployment. They found that solar PV is the most cost-effective low-carbon technology.  \\
    Neumann, F. \& Brown, T., 2021 \cite{neumann_near-optimal_2021}&       4380 &     Single year &          Europe &                 PPSA & This study uses MGA to investigate near optimal solutions for the European electric system with 100 spatial nodes and high temporal resolution. Results indicate that wind energy and hydrogen storage are essential to keeping costs within 10\% of the optimal.\\
             Poncelet et al. 2016 \cite{poncelet_impact_2016} &   Variable &  (2020,2050,10) &             NaN &         TIMES/MARKAL & The authors look at the impact of temporal resolution on model results, as well as the method of choosing representative time slices. Temporal resolution is found to have the greatest influence on results.\\
                 Seck et al. 2020 \cite{seck_embedding_2020} &         84 &   (2014,2052,4) &          France &                TIMES & This study examines the French electric grid to find the optimal penetration of renewable energy sources while preserving grid reliability. The results show a 65\% penetration of renewables is achievable while maintaining reliability. \\
                  Yue et al. 2020 \cite{yue_least_2020}&         12 &  (2020,2050,10) &         Ireland &         TIMES/MARKAL &  This work investigates the 100\% renewable energy pathways in Ireland. The authors found that focusing on renewable energy sources was not the most cost-effective strategy to decarbonize all energy sectors. \\
\bottomrule
\multicolumn{6}{l}{$^\text{a}$ NS = Not Specified}\\
\multicolumn{6}{l}{$^\text{b}$ Horizon denoted by (start year, end year, time step)}
\end{tabular}
