The University of Illinois at Urbana-Champaign pledged to achieve net-zero carbon
emissions from all campus sectors by 2050
\cite{institute_for_sustainability_energy_and_environment_illinois_2015}. This
chapter presents the results of the \gls{uiuc} case study. Since section
\ref{section:time_res} established the importance of time resolution in model results,
accordingly all simulations in this chapter use a 52-week time resolution to ensure
computational tractability. The sensitivity analysis explores other feasible options
for the university with \gls{mga} and varies the \gls{mga} slack variable.

\section{Least Cost Pathway to Net-Zero}
This section presents the lowest cost pathway to achieving \gls{uiuc}'s carbon
neutrality goal.
Figure \ref{fig:uiuc_elc_cap} shows the annual capacity expansion in the cost
minimized solution. Notably, the capacity for all electric generating technologies
grows throughout the model horizon. In particular, the \texttt{ABBOTT\_TB} capacity
increases to just over 100 MW$_e$, even though Figure \ref{fig:uiuc_elc_gen}
shows its electric generation dropping to zero by 2050. This behavior is likely due to
the model using \texttt{ABBOTT\_TB} to meet the planning reserve margin, even
though it does not generate electricity. Additionally, electric import capacity,
\texttt{IMP\_ELC}, rises
to almost 120 MW and accounts for roughly 35\% of total electricity by 2050.
This result is consistent with the recommendation from the Campus Master Plan
\cite{affiliated_engineers_inc_utilities_2015} which recommends increased
electricity imports.

\begin{figure}[H]
  \begin{minipage}{0.48\textwidth}
    \captionsetup{type=figure}
    \centering
    \resizebox{\columnwidth}{!}{%% Creator: Matplotlib, PGF backend
%%
%% To include the figure in your LaTeX document, write
%%   \input{<filename>.pgf}
%%
%% Make sure the required packages are loaded in your preamble
%%   \usepackage{pgf}
%%
%% Figures using additional raster images can only be included by \input if
%% they are in the same directory as the main LaTeX file. For loading figures
%% from other directories you can use the `import` package
%%   \usepackage{import}
%%
%% and then include the figures with
%%   \import{<path to file>}{<filename>.pgf}
%%
%% Matplotlib used the following preamble
%%
\begingroup%
\makeatletter%
\begin{pgfpicture}%
\pgfpathrectangle{\pgfpointorigin}{\pgfqpoint{10.335815in}{10.120798in}}%
\pgfusepath{use as bounding box, clip}%
\begin{pgfscope}%
\pgfsetbuttcap%
\pgfsetmiterjoin%
\definecolor{currentfill}{rgb}{1.000000,1.000000,1.000000}%
\pgfsetfillcolor{currentfill}%
\pgfsetlinewidth{0.000000pt}%
\definecolor{currentstroke}{rgb}{0.000000,0.000000,0.000000}%
\pgfsetstrokecolor{currentstroke}%
\pgfsetdash{}{0pt}%
\pgfpathmoveto{\pgfqpoint{0.000000in}{0.000000in}}%
\pgfpathlineto{\pgfqpoint{10.335815in}{0.000000in}}%
\pgfpathlineto{\pgfqpoint{10.335815in}{10.120798in}}%
\pgfpathlineto{\pgfqpoint{0.000000in}{10.120798in}}%
\pgfpathclose%
\pgfusepath{fill}%
\end{pgfscope}%
\begin{pgfscope}%
\pgfsetbuttcap%
\pgfsetmiterjoin%
\definecolor{currentfill}{rgb}{0.898039,0.898039,0.898039}%
\pgfsetfillcolor{currentfill}%
\pgfsetlinewidth{0.000000pt}%
\definecolor{currentstroke}{rgb}{0.000000,0.000000,0.000000}%
\pgfsetstrokecolor{currentstroke}%
\pgfsetstrokeopacity{0.000000}%
\pgfsetdash{}{0pt}%
\pgfpathmoveto{\pgfqpoint{0.935815in}{0.637495in}}%
\pgfpathlineto{\pgfqpoint{10.235815in}{0.637495in}}%
\pgfpathlineto{\pgfqpoint{10.235815in}{9.697495in}}%
\pgfpathlineto{\pgfqpoint{0.935815in}{9.697495in}}%
\pgfpathclose%
\pgfusepath{fill}%
\end{pgfscope}%
\begin{pgfscope}%
\pgfpathrectangle{\pgfqpoint{0.935815in}{0.637495in}}{\pgfqpoint{9.300000in}{9.060000in}}%
\pgfusepath{clip}%
\pgfsetrectcap%
\pgfsetroundjoin%
\pgfsetlinewidth{0.803000pt}%
\definecolor{currentstroke}{rgb}{1.000000,1.000000,1.000000}%
\pgfsetstrokecolor{currentstroke}%
\pgfsetdash{}{0pt}%
\pgfpathmoveto{\pgfqpoint{1.710815in}{0.637495in}}%
\pgfpathlineto{\pgfqpoint{1.710815in}{9.697495in}}%
\pgfusepath{stroke}%
\end{pgfscope}%
\begin{pgfscope}%
\pgfsetbuttcap%
\pgfsetroundjoin%
\definecolor{currentfill}{rgb}{0.333333,0.333333,0.333333}%
\pgfsetfillcolor{currentfill}%
\pgfsetlinewidth{0.803000pt}%
\definecolor{currentstroke}{rgb}{0.333333,0.333333,0.333333}%
\pgfsetstrokecolor{currentstroke}%
\pgfsetdash{}{0pt}%
\pgfsys@defobject{currentmarker}{\pgfqpoint{0.000000in}{-0.048611in}}{\pgfqpoint{0.000000in}{0.000000in}}{%
\pgfpathmoveto{\pgfqpoint{0.000000in}{0.000000in}}%
\pgfpathlineto{\pgfqpoint{0.000000in}{-0.048611in}}%
\pgfusepath{stroke,fill}%
}%
\begin{pgfscope}%
\pgfsys@transformshift{1.710815in}{0.637495in}%
\pgfsys@useobject{currentmarker}{}%
\end{pgfscope}%
\end{pgfscope}%
\begin{pgfscope}%
\definecolor{textcolor}{rgb}{0.333333,0.333333,0.333333}%
\pgfsetstrokecolor{textcolor}%
\pgfsetfillcolor{textcolor}%
\pgftext[x=1.770807in, y=0.100000in, left, base,rotate=90.000000]{\color{textcolor}\rmfamily\fontsize{16.000000}{19.200000}\selectfont 2025}%
\end{pgfscope}%
\begin{pgfscope}%
\pgfpathrectangle{\pgfqpoint{0.935815in}{0.637495in}}{\pgfqpoint{9.300000in}{9.060000in}}%
\pgfusepath{clip}%
\pgfsetrectcap%
\pgfsetroundjoin%
\pgfsetlinewidth{0.803000pt}%
\definecolor{currentstroke}{rgb}{1.000000,1.000000,1.000000}%
\pgfsetstrokecolor{currentstroke}%
\pgfsetdash{}{0pt}%
\pgfpathmoveto{\pgfqpoint{3.260815in}{0.637495in}}%
\pgfpathlineto{\pgfqpoint{3.260815in}{9.697495in}}%
\pgfusepath{stroke}%
\end{pgfscope}%
\begin{pgfscope}%
\pgfsetbuttcap%
\pgfsetroundjoin%
\definecolor{currentfill}{rgb}{0.333333,0.333333,0.333333}%
\pgfsetfillcolor{currentfill}%
\pgfsetlinewidth{0.803000pt}%
\definecolor{currentstroke}{rgb}{0.333333,0.333333,0.333333}%
\pgfsetstrokecolor{currentstroke}%
\pgfsetdash{}{0pt}%
\pgfsys@defobject{currentmarker}{\pgfqpoint{0.000000in}{-0.048611in}}{\pgfqpoint{0.000000in}{0.000000in}}{%
\pgfpathmoveto{\pgfqpoint{0.000000in}{0.000000in}}%
\pgfpathlineto{\pgfqpoint{0.000000in}{-0.048611in}}%
\pgfusepath{stroke,fill}%
}%
\begin{pgfscope}%
\pgfsys@transformshift{3.260815in}{0.637495in}%
\pgfsys@useobject{currentmarker}{}%
\end{pgfscope}%
\end{pgfscope}%
\begin{pgfscope}%
\definecolor{textcolor}{rgb}{0.333333,0.333333,0.333333}%
\pgfsetstrokecolor{textcolor}%
\pgfsetfillcolor{textcolor}%
\pgftext[x=3.320807in, y=0.100000in, left, base,rotate=90.000000]{\color{textcolor}\rmfamily\fontsize{16.000000}{19.200000}\selectfont 2030}%
\end{pgfscope}%
\begin{pgfscope}%
\pgfpathrectangle{\pgfqpoint{0.935815in}{0.637495in}}{\pgfqpoint{9.300000in}{9.060000in}}%
\pgfusepath{clip}%
\pgfsetrectcap%
\pgfsetroundjoin%
\pgfsetlinewidth{0.803000pt}%
\definecolor{currentstroke}{rgb}{1.000000,1.000000,1.000000}%
\pgfsetstrokecolor{currentstroke}%
\pgfsetdash{}{0pt}%
\pgfpathmoveto{\pgfqpoint{4.810815in}{0.637495in}}%
\pgfpathlineto{\pgfqpoint{4.810815in}{9.697495in}}%
\pgfusepath{stroke}%
\end{pgfscope}%
\begin{pgfscope}%
\pgfsetbuttcap%
\pgfsetroundjoin%
\definecolor{currentfill}{rgb}{0.333333,0.333333,0.333333}%
\pgfsetfillcolor{currentfill}%
\pgfsetlinewidth{0.803000pt}%
\definecolor{currentstroke}{rgb}{0.333333,0.333333,0.333333}%
\pgfsetstrokecolor{currentstroke}%
\pgfsetdash{}{0pt}%
\pgfsys@defobject{currentmarker}{\pgfqpoint{0.000000in}{-0.048611in}}{\pgfqpoint{0.000000in}{0.000000in}}{%
\pgfpathmoveto{\pgfqpoint{0.000000in}{0.000000in}}%
\pgfpathlineto{\pgfqpoint{0.000000in}{-0.048611in}}%
\pgfusepath{stroke,fill}%
}%
\begin{pgfscope}%
\pgfsys@transformshift{4.810815in}{0.637495in}%
\pgfsys@useobject{currentmarker}{}%
\end{pgfscope}%
\end{pgfscope}%
\begin{pgfscope}%
\definecolor{textcolor}{rgb}{0.333333,0.333333,0.333333}%
\pgfsetstrokecolor{textcolor}%
\pgfsetfillcolor{textcolor}%
\pgftext[x=4.870807in, y=0.100000in, left, base,rotate=90.000000]{\color{textcolor}\rmfamily\fontsize{16.000000}{19.200000}\selectfont 2035}%
\end{pgfscope}%
\begin{pgfscope}%
\pgfpathrectangle{\pgfqpoint{0.935815in}{0.637495in}}{\pgfqpoint{9.300000in}{9.060000in}}%
\pgfusepath{clip}%
\pgfsetrectcap%
\pgfsetroundjoin%
\pgfsetlinewidth{0.803000pt}%
\definecolor{currentstroke}{rgb}{1.000000,1.000000,1.000000}%
\pgfsetstrokecolor{currentstroke}%
\pgfsetdash{}{0pt}%
\pgfpathmoveto{\pgfqpoint{6.360815in}{0.637495in}}%
\pgfpathlineto{\pgfqpoint{6.360815in}{9.697495in}}%
\pgfusepath{stroke}%
\end{pgfscope}%
\begin{pgfscope}%
\pgfsetbuttcap%
\pgfsetroundjoin%
\definecolor{currentfill}{rgb}{0.333333,0.333333,0.333333}%
\pgfsetfillcolor{currentfill}%
\pgfsetlinewidth{0.803000pt}%
\definecolor{currentstroke}{rgb}{0.333333,0.333333,0.333333}%
\pgfsetstrokecolor{currentstroke}%
\pgfsetdash{}{0pt}%
\pgfsys@defobject{currentmarker}{\pgfqpoint{0.000000in}{-0.048611in}}{\pgfqpoint{0.000000in}{0.000000in}}{%
\pgfpathmoveto{\pgfqpoint{0.000000in}{0.000000in}}%
\pgfpathlineto{\pgfqpoint{0.000000in}{-0.048611in}}%
\pgfusepath{stroke,fill}%
}%
\begin{pgfscope}%
\pgfsys@transformshift{6.360815in}{0.637495in}%
\pgfsys@useobject{currentmarker}{}%
\end{pgfscope}%
\end{pgfscope}%
\begin{pgfscope}%
\definecolor{textcolor}{rgb}{0.333333,0.333333,0.333333}%
\pgfsetstrokecolor{textcolor}%
\pgfsetfillcolor{textcolor}%
\pgftext[x=6.420807in, y=0.100000in, left, base,rotate=90.000000]{\color{textcolor}\rmfamily\fontsize{16.000000}{19.200000}\selectfont 2040}%
\end{pgfscope}%
\begin{pgfscope}%
\pgfpathrectangle{\pgfqpoint{0.935815in}{0.637495in}}{\pgfqpoint{9.300000in}{9.060000in}}%
\pgfusepath{clip}%
\pgfsetrectcap%
\pgfsetroundjoin%
\pgfsetlinewidth{0.803000pt}%
\definecolor{currentstroke}{rgb}{1.000000,1.000000,1.000000}%
\pgfsetstrokecolor{currentstroke}%
\pgfsetdash{}{0pt}%
\pgfpathmoveto{\pgfqpoint{7.910815in}{0.637495in}}%
\pgfpathlineto{\pgfqpoint{7.910815in}{9.697495in}}%
\pgfusepath{stroke}%
\end{pgfscope}%
\begin{pgfscope}%
\pgfsetbuttcap%
\pgfsetroundjoin%
\definecolor{currentfill}{rgb}{0.333333,0.333333,0.333333}%
\pgfsetfillcolor{currentfill}%
\pgfsetlinewidth{0.803000pt}%
\definecolor{currentstroke}{rgb}{0.333333,0.333333,0.333333}%
\pgfsetstrokecolor{currentstroke}%
\pgfsetdash{}{0pt}%
\pgfsys@defobject{currentmarker}{\pgfqpoint{0.000000in}{-0.048611in}}{\pgfqpoint{0.000000in}{0.000000in}}{%
\pgfpathmoveto{\pgfqpoint{0.000000in}{0.000000in}}%
\pgfpathlineto{\pgfqpoint{0.000000in}{-0.048611in}}%
\pgfusepath{stroke,fill}%
}%
\begin{pgfscope}%
\pgfsys@transformshift{7.910815in}{0.637495in}%
\pgfsys@useobject{currentmarker}{}%
\end{pgfscope}%
\end{pgfscope}%
\begin{pgfscope}%
\definecolor{textcolor}{rgb}{0.333333,0.333333,0.333333}%
\pgfsetstrokecolor{textcolor}%
\pgfsetfillcolor{textcolor}%
\pgftext[x=7.970807in, y=0.100000in, left, base,rotate=90.000000]{\color{textcolor}\rmfamily\fontsize{16.000000}{19.200000}\selectfont 2045}%
\end{pgfscope}%
\begin{pgfscope}%
\pgfpathrectangle{\pgfqpoint{0.935815in}{0.637495in}}{\pgfqpoint{9.300000in}{9.060000in}}%
\pgfusepath{clip}%
\pgfsetrectcap%
\pgfsetroundjoin%
\pgfsetlinewidth{0.803000pt}%
\definecolor{currentstroke}{rgb}{1.000000,1.000000,1.000000}%
\pgfsetstrokecolor{currentstroke}%
\pgfsetdash{}{0pt}%
\pgfpathmoveto{\pgfqpoint{9.460815in}{0.637495in}}%
\pgfpathlineto{\pgfqpoint{9.460815in}{9.697495in}}%
\pgfusepath{stroke}%
\end{pgfscope}%
\begin{pgfscope}%
\pgfsetbuttcap%
\pgfsetroundjoin%
\definecolor{currentfill}{rgb}{0.333333,0.333333,0.333333}%
\pgfsetfillcolor{currentfill}%
\pgfsetlinewidth{0.803000pt}%
\definecolor{currentstroke}{rgb}{0.333333,0.333333,0.333333}%
\pgfsetstrokecolor{currentstroke}%
\pgfsetdash{}{0pt}%
\pgfsys@defobject{currentmarker}{\pgfqpoint{0.000000in}{-0.048611in}}{\pgfqpoint{0.000000in}{0.000000in}}{%
\pgfpathmoveto{\pgfqpoint{0.000000in}{0.000000in}}%
\pgfpathlineto{\pgfqpoint{0.000000in}{-0.048611in}}%
\pgfusepath{stroke,fill}%
}%
\begin{pgfscope}%
\pgfsys@transformshift{9.460815in}{0.637495in}%
\pgfsys@useobject{currentmarker}{}%
\end{pgfscope}%
\end{pgfscope}%
\begin{pgfscope}%
\definecolor{textcolor}{rgb}{0.333333,0.333333,0.333333}%
\pgfsetstrokecolor{textcolor}%
\pgfsetfillcolor{textcolor}%
\pgftext[x=9.520807in, y=0.100000in, left, base,rotate=90.000000]{\color{textcolor}\rmfamily\fontsize{16.000000}{19.200000}\selectfont 2050}%
\end{pgfscope}%
\begin{pgfscope}%
\pgfpathrectangle{\pgfqpoint{0.935815in}{0.637495in}}{\pgfqpoint{9.300000in}{9.060000in}}%
\pgfusepath{clip}%
\pgfsetrectcap%
\pgfsetroundjoin%
\pgfsetlinewidth{0.803000pt}%
\definecolor{currentstroke}{rgb}{1.000000,1.000000,1.000000}%
\pgfsetstrokecolor{currentstroke}%
\pgfsetdash{}{0pt}%
\pgfpathmoveto{\pgfqpoint{0.935815in}{0.637495in}}%
\pgfpathlineto{\pgfqpoint{10.235815in}{0.637495in}}%
\pgfusepath{stroke}%
\end{pgfscope}%
\begin{pgfscope}%
\pgfsetbuttcap%
\pgfsetroundjoin%
\definecolor{currentfill}{rgb}{0.333333,0.333333,0.333333}%
\pgfsetfillcolor{currentfill}%
\pgfsetlinewidth{0.803000pt}%
\definecolor{currentstroke}{rgb}{0.333333,0.333333,0.333333}%
\pgfsetstrokecolor{currentstroke}%
\pgfsetdash{}{0pt}%
\pgfsys@defobject{currentmarker}{\pgfqpoint{-0.048611in}{0.000000in}}{\pgfqpoint{-0.000000in}{0.000000in}}{%
\pgfpathmoveto{\pgfqpoint{-0.000000in}{0.000000in}}%
\pgfpathlineto{\pgfqpoint{-0.048611in}{0.000000in}}%
\pgfusepath{stroke,fill}%
}%
\begin{pgfscope}%
\pgfsys@transformshift{0.935815in}{0.637495in}%
\pgfsys@useobject{currentmarker}{}%
\end{pgfscope}%
\end{pgfscope}%
\begin{pgfscope}%
\definecolor{textcolor}{rgb}{0.333333,0.333333,0.333333}%
\pgfsetstrokecolor{textcolor}%
\pgfsetfillcolor{textcolor}%
\pgftext[x=0.443111in, y=0.554162in, left, base]{\color{textcolor}\rmfamily\fontsize{16.000000}{19.200000}\selectfont \(\displaystyle {0.00}\)}%
\end{pgfscope}%
\begin{pgfscope}%
\pgfpathrectangle{\pgfqpoint{0.935815in}{0.637495in}}{\pgfqpoint{9.300000in}{9.060000in}}%
\pgfusepath{clip}%
\pgfsetrectcap%
\pgfsetroundjoin%
\pgfsetlinewidth{0.803000pt}%
\definecolor{currentstroke}{rgb}{1.000000,1.000000,1.000000}%
\pgfsetstrokecolor{currentstroke}%
\pgfsetdash{}{0pt}%
\pgfpathmoveto{\pgfqpoint{0.935815in}{1.671945in}}%
\pgfpathlineto{\pgfqpoint{10.235815in}{1.671945in}}%
\pgfusepath{stroke}%
\end{pgfscope}%
\begin{pgfscope}%
\pgfsetbuttcap%
\pgfsetroundjoin%
\definecolor{currentfill}{rgb}{0.333333,0.333333,0.333333}%
\pgfsetfillcolor{currentfill}%
\pgfsetlinewidth{0.803000pt}%
\definecolor{currentstroke}{rgb}{0.333333,0.333333,0.333333}%
\pgfsetstrokecolor{currentstroke}%
\pgfsetdash{}{0pt}%
\pgfsys@defobject{currentmarker}{\pgfqpoint{-0.048611in}{0.000000in}}{\pgfqpoint{-0.000000in}{0.000000in}}{%
\pgfpathmoveto{\pgfqpoint{-0.000000in}{0.000000in}}%
\pgfpathlineto{\pgfqpoint{-0.048611in}{0.000000in}}%
\pgfusepath{stroke,fill}%
}%
\begin{pgfscope}%
\pgfsys@transformshift{0.935815in}{1.671945in}%
\pgfsys@useobject{currentmarker}{}%
\end{pgfscope}%
\end{pgfscope}%
\begin{pgfscope}%
\definecolor{textcolor}{rgb}{0.333333,0.333333,0.333333}%
\pgfsetstrokecolor{textcolor}%
\pgfsetfillcolor{textcolor}%
\pgftext[x=0.443111in, y=1.588611in, left, base]{\color{textcolor}\rmfamily\fontsize{16.000000}{19.200000}\selectfont \(\displaystyle {0.05}\)}%
\end{pgfscope}%
\begin{pgfscope}%
\pgfpathrectangle{\pgfqpoint{0.935815in}{0.637495in}}{\pgfqpoint{9.300000in}{9.060000in}}%
\pgfusepath{clip}%
\pgfsetrectcap%
\pgfsetroundjoin%
\pgfsetlinewidth{0.803000pt}%
\definecolor{currentstroke}{rgb}{1.000000,1.000000,1.000000}%
\pgfsetstrokecolor{currentstroke}%
\pgfsetdash{}{0pt}%
\pgfpathmoveto{\pgfqpoint{0.935815in}{2.706394in}}%
\pgfpathlineto{\pgfqpoint{10.235815in}{2.706394in}}%
\pgfusepath{stroke}%
\end{pgfscope}%
\begin{pgfscope}%
\pgfsetbuttcap%
\pgfsetroundjoin%
\definecolor{currentfill}{rgb}{0.333333,0.333333,0.333333}%
\pgfsetfillcolor{currentfill}%
\pgfsetlinewidth{0.803000pt}%
\definecolor{currentstroke}{rgb}{0.333333,0.333333,0.333333}%
\pgfsetstrokecolor{currentstroke}%
\pgfsetdash{}{0pt}%
\pgfsys@defobject{currentmarker}{\pgfqpoint{-0.048611in}{0.000000in}}{\pgfqpoint{-0.000000in}{0.000000in}}{%
\pgfpathmoveto{\pgfqpoint{-0.000000in}{0.000000in}}%
\pgfpathlineto{\pgfqpoint{-0.048611in}{0.000000in}}%
\pgfusepath{stroke,fill}%
}%
\begin{pgfscope}%
\pgfsys@transformshift{0.935815in}{2.706394in}%
\pgfsys@useobject{currentmarker}{}%
\end{pgfscope}%
\end{pgfscope}%
\begin{pgfscope}%
\definecolor{textcolor}{rgb}{0.333333,0.333333,0.333333}%
\pgfsetstrokecolor{textcolor}%
\pgfsetfillcolor{textcolor}%
\pgftext[x=0.443111in, y=2.623061in, left, base]{\color{textcolor}\rmfamily\fontsize{16.000000}{19.200000}\selectfont \(\displaystyle {0.10}\)}%
\end{pgfscope}%
\begin{pgfscope}%
\pgfpathrectangle{\pgfqpoint{0.935815in}{0.637495in}}{\pgfqpoint{9.300000in}{9.060000in}}%
\pgfusepath{clip}%
\pgfsetrectcap%
\pgfsetroundjoin%
\pgfsetlinewidth{0.803000pt}%
\definecolor{currentstroke}{rgb}{1.000000,1.000000,1.000000}%
\pgfsetstrokecolor{currentstroke}%
\pgfsetdash{}{0pt}%
\pgfpathmoveto{\pgfqpoint{0.935815in}{3.740843in}}%
\pgfpathlineto{\pgfqpoint{10.235815in}{3.740843in}}%
\pgfusepath{stroke}%
\end{pgfscope}%
\begin{pgfscope}%
\pgfsetbuttcap%
\pgfsetroundjoin%
\definecolor{currentfill}{rgb}{0.333333,0.333333,0.333333}%
\pgfsetfillcolor{currentfill}%
\pgfsetlinewidth{0.803000pt}%
\definecolor{currentstroke}{rgb}{0.333333,0.333333,0.333333}%
\pgfsetstrokecolor{currentstroke}%
\pgfsetdash{}{0pt}%
\pgfsys@defobject{currentmarker}{\pgfqpoint{-0.048611in}{0.000000in}}{\pgfqpoint{-0.000000in}{0.000000in}}{%
\pgfpathmoveto{\pgfqpoint{-0.000000in}{0.000000in}}%
\pgfpathlineto{\pgfqpoint{-0.048611in}{0.000000in}}%
\pgfusepath{stroke,fill}%
}%
\begin{pgfscope}%
\pgfsys@transformshift{0.935815in}{3.740843in}%
\pgfsys@useobject{currentmarker}{}%
\end{pgfscope}%
\end{pgfscope}%
\begin{pgfscope}%
\definecolor{textcolor}{rgb}{0.333333,0.333333,0.333333}%
\pgfsetstrokecolor{textcolor}%
\pgfsetfillcolor{textcolor}%
\pgftext[x=0.443111in, y=3.657510in, left, base]{\color{textcolor}\rmfamily\fontsize{16.000000}{19.200000}\selectfont \(\displaystyle {0.15}\)}%
\end{pgfscope}%
\begin{pgfscope}%
\pgfpathrectangle{\pgfqpoint{0.935815in}{0.637495in}}{\pgfqpoint{9.300000in}{9.060000in}}%
\pgfusepath{clip}%
\pgfsetrectcap%
\pgfsetroundjoin%
\pgfsetlinewidth{0.803000pt}%
\definecolor{currentstroke}{rgb}{1.000000,1.000000,1.000000}%
\pgfsetstrokecolor{currentstroke}%
\pgfsetdash{}{0pt}%
\pgfpathmoveto{\pgfqpoint{0.935815in}{4.775293in}}%
\pgfpathlineto{\pgfqpoint{10.235815in}{4.775293in}}%
\pgfusepath{stroke}%
\end{pgfscope}%
\begin{pgfscope}%
\pgfsetbuttcap%
\pgfsetroundjoin%
\definecolor{currentfill}{rgb}{0.333333,0.333333,0.333333}%
\pgfsetfillcolor{currentfill}%
\pgfsetlinewidth{0.803000pt}%
\definecolor{currentstroke}{rgb}{0.333333,0.333333,0.333333}%
\pgfsetstrokecolor{currentstroke}%
\pgfsetdash{}{0pt}%
\pgfsys@defobject{currentmarker}{\pgfqpoint{-0.048611in}{0.000000in}}{\pgfqpoint{-0.000000in}{0.000000in}}{%
\pgfpathmoveto{\pgfqpoint{-0.000000in}{0.000000in}}%
\pgfpathlineto{\pgfqpoint{-0.048611in}{0.000000in}}%
\pgfusepath{stroke,fill}%
}%
\begin{pgfscope}%
\pgfsys@transformshift{0.935815in}{4.775293in}%
\pgfsys@useobject{currentmarker}{}%
\end{pgfscope}%
\end{pgfscope}%
\begin{pgfscope}%
\definecolor{textcolor}{rgb}{0.333333,0.333333,0.333333}%
\pgfsetstrokecolor{textcolor}%
\pgfsetfillcolor{textcolor}%
\pgftext[x=0.443111in, y=4.691960in, left, base]{\color{textcolor}\rmfamily\fontsize{16.000000}{19.200000}\selectfont \(\displaystyle {0.20}\)}%
\end{pgfscope}%
\begin{pgfscope}%
\pgfpathrectangle{\pgfqpoint{0.935815in}{0.637495in}}{\pgfqpoint{9.300000in}{9.060000in}}%
\pgfusepath{clip}%
\pgfsetrectcap%
\pgfsetroundjoin%
\pgfsetlinewidth{0.803000pt}%
\definecolor{currentstroke}{rgb}{1.000000,1.000000,1.000000}%
\pgfsetstrokecolor{currentstroke}%
\pgfsetdash{}{0pt}%
\pgfpathmoveto{\pgfqpoint{0.935815in}{5.809742in}}%
\pgfpathlineto{\pgfqpoint{10.235815in}{5.809742in}}%
\pgfusepath{stroke}%
\end{pgfscope}%
\begin{pgfscope}%
\pgfsetbuttcap%
\pgfsetroundjoin%
\definecolor{currentfill}{rgb}{0.333333,0.333333,0.333333}%
\pgfsetfillcolor{currentfill}%
\pgfsetlinewidth{0.803000pt}%
\definecolor{currentstroke}{rgb}{0.333333,0.333333,0.333333}%
\pgfsetstrokecolor{currentstroke}%
\pgfsetdash{}{0pt}%
\pgfsys@defobject{currentmarker}{\pgfqpoint{-0.048611in}{0.000000in}}{\pgfqpoint{-0.000000in}{0.000000in}}{%
\pgfpathmoveto{\pgfqpoint{-0.000000in}{0.000000in}}%
\pgfpathlineto{\pgfqpoint{-0.048611in}{0.000000in}}%
\pgfusepath{stroke,fill}%
}%
\begin{pgfscope}%
\pgfsys@transformshift{0.935815in}{5.809742in}%
\pgfsys@useobject{currentmarker}{}%
\end{pgfscope}%
\end{pgfscope}%
\begin{pgfscope}%
\definecolor{textcolor}{rgb}{0.333333,0.333333,0.333333}%
\pgfsetstrokecolor{textcolor}%
\pgfsetfillcolor{textcolor}%
\pgftext[x=0.443111in, y=5.726409in, left, base]{\color{textcolor}\rmfamily\fontsize{16.000000}{19.200000}\selectfont \(\displaystyle {0.25}\)}%
\end{pgfscope}%
\begin{pgfscope}%
\pgfpathrectangle{\pgfqpoint{0.935815in}{0.637495in}}{\pgfqpoint{9.300000in}{9.060000in}}%
\pgfusepath{clip}%
\pgfsetrectcap%
\pgfsetroundjoin%
\pgfsetlinewidth{0.803000pt}%
\definecolor{currentstroke}{rgb}{1.000000,1.000000,1.000000}%
\pgfsetstrokecolor{currentstroke}%
\pgfsetdash{}{0pt}%
\pgfpathmoveto{\pgfqpoint{0.935815in}{6.844192in}}%
\pgfpathlineto{\pgfqpoint{10.235815in}{6.844192in}}%
\pgfusepath{stroke}%
\end{pgfscope}%
\begin{pgfscope}%
\pgfsetbuttcap%
\pgfsetroundjoin%
\definecolor{currentfill}{rgb}{0.333333,0.333333,0.333333}%
\pgfsetfillcolor{currentfill}%
\pgfsetlinewidth{0.803000pt}%
\definecolor{currentstroke}{rgb}{0.333333,0.333333,0.333333}%
\pgfsetstrokecolor{currentstroke}%
\pgfsetdash{}{0pt}%
\pgfsys@defobject{currentmarker}{\pgfqpoint{-0.048611in}{0.000000in}}{\pgfqpoint{-0.000000in}{0.000000in}}{%
\pgfpathmoveto{\pgfqpoint{-0.000000in}{0.000000in}}%
\pgfpathlineto{\pgfqpoint{-0.048611in}{0.000000in}}%
\pgfusepath{stroke,fill}%
}%
\begin{pgfscope}%
\pgfsys@transformshift{0.935815in}{6.844192in}%
\pgfsys@useobject{currentmarker}{}%
\end{pgfscope}%
\end{pgfscope}%
\begin{pgfscope}%
\definecolor{textcolor}{rgb}{0.333333,0.333333,0.333333}%
\pgfsetstrokecolor{textcolor}%
\pgfsetfillcolor{textcolor}%
\pgftext[x=0.443111in, y=6.760858in, left, base]{\color{textcolor}\rmfamily\fontsize{16.000000}{19.200000}\selectfont \(\displaystyle {0.30}\)}%
\end{pgfscope}%
\begin{pgfscope}%
\pgfpathrectangle{\pgfqpoint{0.935815in}{0.637495in}}{\pgfqpoint{9.300000in}{9.060000in}}%
\pgfusepath{clip}%
\pgfsetrectcap%
\pgfsetroundjoin%
\pgfsetlinewidth{0.803000pt}%
\definecolor{currentstroke}{rgb}{1.000000,1.000000,1.000000}%
\pgfsetstrokecolor{currentstroke}%
\pgfsetdash{}{0pt}%
\pgfpathmoveto{\pgfqpoint{0.935815in}{7.878641in}}%
\pgfpathlineto{\pgfqpoint{10.235815in}{7.878641in}}%
\pgfusepath{stroke}%
\end{pgfscope}%
\begin{pgfscope}%
\pgfsetbuttcap%
\pgfsetroundjoin%
\definecolor{currentfill}{rgb}{0.333333,0.333333,0.333333}%
\pgfsetfillcolor{currentfill}%
\pgfsetlinewidth{0.803000pt}%
\definecolor{currentstroke}{rgb}{0.333333,0.333333,0.333333}%
\pgfsetstrokecolor{currentstroke}%
\pgfsetdash{}{0pt}%
\pgfsys@defobject{currentmarker}{\pgfqpoint{-0.048611in}{0.000000in}}{\pgfqpoint{-0.000000in}{0.000000in}}{%
\pgfpathmoveto{\pgfqpoint{-0.000000in}{0.000000in}}%
\pgfpathlineto{\pgfqpoint{-0.048611in}{0.000000in}}%
\pgfusepath{stroke,fill}%
}%
\begin{pgfscope}%
\pgfsys@transformshift{0.935815in}{7.878641in}%
\pgfsys@useobject{currentmarker}{}%
\end{pgfscope}%
\end{pgfscope}%
\begin{pgfscope}%
\definecolor{textcolor}{rgb}{0.333333,0.333333,0.333333}%
\pgfsetstrokecolor{textcolor}%
\pgfsetfillcolor{textcolor}%
\pgftext[x=0.443111in, y=7.795308in, left, base]{\color{textcolor}\rmfamily\fontsize{16.000000}{19.200000}\selectfont \(\displaystyle {0.35}\)}%
\end{pgfscope}%
\begin{pgfscope}%
\pgfpathrectangle{\pgfqpoint{0.935815in}{0.637495in}}{\pgfqpoint{9.300000in}{9.060000in}}%
\pgfusepath{clip}%
\pgfsetrectcap%
\pgfsetroundjoin%
\pgfsetlinewidth{0.803000pt}%
\definecolor{currentstroke}{rgb}{1.000000,1.000000,1.000000}%
\pgfsetstrokecolor{currentstroke}%
\pgfsetdash{}{0pt}%
\pgfpathmoveto{\pgfqpoint{0.935815in}{8.913091in}}%
\pgfpathlineto{\pgfqpoint{10.235815in}{8.913091in}}%
\pgfusepath{stroke}%
\end{pgfscope}%
\begin{pgfscope}%
\pgfsetbuttcap%
\pgfsetroundjoin%
\definecolor{currentfill}{rgb}{0.333333,0.333333,0.333333}%
\pgfsetfillcolor{currentfill}%
\pgfsetlinewidth{0.803000pt}%
\definecolor{currentstroke}{rgb}{0.333333,0.333333,0.333333}%
\pgfsetstrokecolor{currentstroke}%
\pgfsetdash{}{0pt}%
\pgfsys@defobject{currentmarker}{\pgfqpoint{-0.048611in}{0.000000in}}{\pgfqpoint{-0.000000in}{0.000000in}}{%
\pgfpathmoveto{\pgfqpoint{-0.000000in}{0.000000in}}%
\pgfpathlineto{\pgfqpoint{-0.048611in}{0.000000in}}%
\pgfusepath{stroke,fill}%
}%
\begin{pgfscope}%
\pgfsys@transformshift{0.935815in}{8.913091in}%
\pgfsys@useobject{currentmarker}{}%
\end{pgfscope}%
\end{pgfscope}%
\begin{pgfscope}%
\definecolor{textcolor}{rgb}{0.333333,0.333333,0.333333}%
\pgfsetstrokecolor{textcolor}%
\pgfsetfillcolor{textcolor}%
\pgftext[x=0.443111in, y=8.829757in, left, base]{\color{textcolor}\rmfamily\fontsize{16.000000}{19.200000}\selectfont \(\displaystyle {0.40}\)}%
\end{pgfscope}%
\begin{pgfscope}%
\definecolor{textcolor}{rgb}{0.333333,0.333333,0.333333}%
\pgfsetstrokecolor{textcolor}%
\pgfsetfillcolor{textcolor}%
\pgftext[x=0.387555in,y=5.167495in,,bottom,rotate=90.000000]{\color{textcolor}\rmfamily\fontsize{20.000000}{24.000000}\selectfont Capacity [GW]}%
\end{pgfscope}%
\begin{pgfscope}%
\pgfpathrectangle{\pgfqpoint{0.935815in}{0.637495in}}{\pgfqpoint{9.300000in}{9.060000in}}%
\pgfusepath{clip}%
\pgfsetbuttcap%
\pgfsetmiterjoin%
\definecolor{currentfill}{rgb}{0.839216,0.152941,0.156863}%
\pgfsetfillcolor{currentfill}%
\pgfsetfillopacity{0.990000}%
\pgfsetlinewidth{0.000000pt}%
\definecolor{currentstroke}{rgb}{0.000000,0.000000,0.000000}%
\pgfsetstrokecolor{currentstroke}%
\pgfsetstrokeopacity{0.990000}%
\pgfsetdash{}{0pt}%
\pgfpathmoveto{\pgfqpoint{1.323315in}{0.637495in}}%
\pgfpathlineto{\pgfqpoint{2.098315in}{0.637495in}}%
\pgfpathlineto{\pgfqpoint{2.098315in}{1.426245in}}%
\pgfpathlineto{\pgfqpoint{1.323315in}{1.426245in}}%
\pgfpathclose%
\pgfusepath{fill}%
\end{pgfscope}%
\begin{pgfscope}%
\pgfsetbuttcap%
\pgfsetmiterjoin%
\definecolor{currentfill}{rgb}{0.839216,0.152941,0.156863}%
\pgfsetfillcolor{currentfill}%
\pgfsetfillopacity{0.990000}%
\pgfsetlinewidth{0.000000pt}%
\definecolor{currentstroke}{rgb}{0.000000,0.000000,0.000000}%
\pgfsetstrokecolor{currentstroke}%
\pgfsetstrokeopacity{0.990000}%
\pgfsetdash{}{0pt}%
\pgfpathrectangle{\pgfqpoint{0.935815in}{0.637495in}}{\pgfqpoint{9.300000in}{9.060000in}}%
\pgfusepath{clip}%
\pgfpathmoveto{\pgfqpoint{1.323315in}{0.637495in}}%
\pgfpathlineto{\pgfqpoint{2.098315in}{0.637495in}}%
\pgfpathlineto{\pgfqpoint{2.098315in}{1.426245in}}%
\pgfpathlineto{\pgfqpoint{1.323315in}{1.426245in}}%
\pgfpathclose%
\pgfusepath{clip}%
\pgfsys@defobject{currentpattern}{\pgfqpoint{0in}{0in}}{\pgfqpoint{1in}{1in}}{%
\begin{pgfscope}%
\pgfpathrectangle{\pgfqpoint{0in}{0in}}{\pgfqpoint{1in}{1in}}%
\pgfusepath{clip}%
\pgfpathmoveto{\pgfqpoint{-0.500000in}{0.500000in}}%
\pgfpathlineto{\pgfqpoint{0.500000in}{1.500000in}}%
\pgfpathmoveto{\pgfqpoint{-0.333333in}{0.333333in}}%
\pgfpathlineto{\pgfqpoint{0.666667in}{1.333333in}}%
\pgfpathmoveto{\pgfqpoint{-0.166667in}{0.166667in}}%
\pgfpathlineto{\pgfqpoint{0.833333in}{1.166667in}}%
\pgfpathmoveto{\pgfqpoint{0.000000in}{0.000000in}}%
\pgfpathlineto{\pgfqpoint{1.000000in}{1.000000in}}%
\pgfpathmoveto{\pgfqpoint{0.166667in}{-0.166667in}}%
\pgfpathlineto{\pgfqpoint{1.166667in}{0.833333in}}%
\pgfpathmoveto{\pgfqpoint{0.333333in}{-0.333333in}}%
\pgfpathlineto{\pgfqpoint{1.333333in}{0.666667in}}%
\pgfpathmoveto{\pgfqpoint{0.500000in}{-0.500000in}}%
\pgfpathlineto{\pgfqpoint{1.500000in}{0.500000in}}%
\pgfpathmoveto{\pgfqpoint{-0.500000in}{0.500000in}}%
\pgfpathlineto{\pgfqpoint{0.500000in}{-0.500000in}}%
\pgfpathmoveto{\pgfqpoint{-0.333333in}{0.666667in}}%
\pgfpathlineto{\pgfqpoint{0.666667in}{-0.333333in}}%
\pgfpathmoveto{\pgfqpoint{-0.166667in}{0.833333in}}%
\pgfpathlineto{\pgfqpoint{0.833333in}{-0.166667in}}%
\pgfpathmoveto{\pgfqpoint{0.000000in}{1.000000in}}%
\pgfpathlineto{\pgfqpoint{1.000000in}{0.000000in}}%
\pgfpathmoveto{\pgfqpoint{0.166667in}{1.166667in}}%
\pgfpathlineto{\pgfqpoint{1.166667in}{0.166667in}}%
\pgfpathmoveto{\pgfqpoint{0.333333in}{1.333333in}}%
\pgfpathlineto{\pgfqpoint{1.333333in}{0.333333in}}%
\pgfpathmoveto{\pgfqpoint{0.500000in}{1.500000in}}%
\pgfpathlineto{\pgfqpoint{1.500000in}{0.500000in}}%
\pgfusepath{stroke}%
\end{pgfscope}%
}%
\pgfsys@transformshift{1.323315in}{0.637495in}%
\pgfsys@useobject{currentpattern}{}%
\pgfsys@transformshift{1in}{0in}%
\pgfsys@transformshift{-1in}{0in}%
\pgfsys@transformshift{0in}{1in}%
\end{pgfscope}%
\begin{pgfscope}%
\pgfpathrectangle{\pgfqpoint{0.935815in}{0.637495in}}{\pgfqpoint{9.300000in}{9.060000in}}%
\pgfusepath{clip}%
\pgfsetbuttcap%
\pgfsetmiterjoin%
\definecolor{currentfill}{rgb}{0.839216,0.152941,0.156863}%
\pgfsetfillcolor{currentfill}%
\pgfsetfillopacity{0.990000}%
\pgfsetlinewidth{0.000000pt}%
\definecolor{currentstroke}{rgb}{0.000000,0.000000,0.000000}%
\pgfsetstrokecolor{currentstroke}%
\pgfsetstrokeopacity{0.990000}%
\pgfsetdash{}{0pt}%
\pgfpathmoveto{\pgfqpoint{2.873315in}{0.637495in}}%
\pgfpathlineto{\pgfqpoint{3.648315in}{0.637495in}}%
\pgfpathlineto{\pgfqpoint{3.648315in}{2.180049in}}%
\pgfpathlineto{\pgfqpoint{2.873315in}{2.180049in}}%
\pgfpathclose%
\pgfusepath{fill}%
\end{pgfscope}%
\begin{pgfscope}%
\pgfsetbuttcap%
\pgfsetmiterjoin%
\definecolor{currentfill}{rgb}{0.839216,0.152941,0.156863}%
\pgfsetfillcolor{currentfill}%
\pgfsetfillopacity{0.990000}%
\pgfsetlinewidth{0.000000pt}%
\definecolor{currentstroke}{rgb}{0.000000,0.000000,0.000000}%
\pgfsetstrokecolor{currentstroke}%
\pgfsetstrokeopacity{0.990000}%
\pgfsetdash{}{0pt}%
\pgfpathrectangle{\pgfqpoint{0.935815in}{0.637495in}}{\pgfqpoint{9.300000in}{9.060000in}}%
\pgfusepath{clip}%
\pgfpathmoveto{\pgfqpoint{2.873315in}{0.637495in}}%
\pgfpathlineto{\pgfqpoint{3.648315in}{0.637495in}}%
\pgfpathlineto{\pgfqpoint{3.648315in}{2.180049in}}%
\pgfpathlineto{\pgfqpoint{2.873315in}{2.180049in}}%
\pgfpathclose%
\pgfusepath{clip}%
\pgfsys@defobject{currentpattern}{\pgfqpoint{0in}{0in}}{\pgfqpoint{1in}{1in}}{%
\begin{pgfscope}%
\pgfpathrectangle{\pgfqpoint{0in}{0in}}{\pgfqpoint{1in}{1in}}%
\pgfusepath{clip}%
\pgfpathmoveto{\pgfqpoint{-0.500000in}{0.500000in}}%
\pgfpathlineto{\pgfqpoint{0.500000in}{1.500000in}}%
\pgfpathmoveto{\pgfqpoint{-0.333333in}{0.333333in}}%
\pgfpathlineto{\pgfqpoint{0.666667in}{1.333333in}}%
\pgfpathmoveto{\pgfqpoint{-0.166667in}{0.166667in}}%
\pgfpathlineto{\pgfqpoint{0.833333in}{1.166667in}}%
\pgfpathmoveto{\pgfqpoint{0.000000in}{0.000000in}}%
\pgfpathlineto{\pgfqpoint{1.000000in}{1.000000in}}%
\pgfpathmoveto{\pgfqpoint{0.166667in}{-0.166667in}}%
\pgfpathlineto{\pgfqpoint{1.166667in}{0.833333in}}%
\pgfpathmoveto{\pgfqpoint{0.333333in}{-0.333333in}}%
\pgfpathlineto{\pgfqpoint{1.333333in}{0.666667in}}%
\pgfpathmoveto{\pgfqpoint{0.500000in}{-0.500000in}}%
\pgfpathlineto{\pgfqpoint{1.500000in}{0.500000in}}%
\pgfpathmoveto{\pgfqpoint{-0.500000in}{0.500000in}}%
\pgfpathlineto{\pgfqpoint{0.500000in}{-0.500000in}}%
\pgfpathmoveto{\pgfqpoint{-0.333333in}{0.666667in}}%
\pgfpathlineto{\pgfqpoint{0.666667in}{-0.333333in}}%
\pgfpathmoveto{\pgfqpoint{-0.166667in}{0.833333in}}%
\pgfpathlineto{\pgfqpoint{0.833333in}{-0.166667in}}%
\pgfpathmoveto{\pgfqpoint{0.000000in}{1.000000in}}%
\pgfpathlineto{\pgfqpoint{1.000000in}{0.000000in}}%
\pgfpathmoveto{\pgfqpoint{0.166667in}{1.166667in}}%
\pgfpathlineto{\pgfqpoint{1.166667in}{0.166667in}}%
\pgfpathmoveto{\pgfqpoint{0.333333in}{1.333333in}}%
\pgfpathlineto{\pgfqpoint{1.333333in}{0.333333in}}%
\pgfpathmoveto{\pgfqpoint{0.500000in}{1.500000in}}%
\pgfpathlineto{\pgfqpoint{1.500000in}{0.500000in}}%
\pgfusepath{stroke}%
\end{pgfscope}%
}%
\pgfsys@transformshift{2.873315in}{0.637495in}%
\pgfsys@useobject{currentpattern}{}%
\pgfsys@transformshift{1in}{0in}%
\pgfsys@transformshift{-1in}{0in}%
\pgfsys@transformshift{0in}{1in}%
\pgfsys@useobject{currentpattern}{}%
\pgfsys@transformshift{1in}{0in}%
\pgfsys@transformshift{-1in}{0in}%
\pgfsys@transformshift{0in}{1in}%
\end{pgfscope}%
\begin{pgfscope}%
\pgfpathrectangle{\pgfqpoint{0.935815in}{0.637495in}}{\pgfqpoint{9.300000in}{9.060000in}}%
\pgfusepath{clip}%
\pgfsetbuttcap%
\pgfsetmiterjoin%
\definecolor{currentfill}{rgb}{0.839216,0.152941,0.156863}%
\pgfsetfillcolor{currentfill}%
\pgfsetfillopacity{0.990000}%
\pgfsetlinewidth{0.000000pt}%
\definecolor{currentstroke}{rgb}{0.000000,0.000000,0.000000}%
\pgfsetstrokecolor{currentstroke}%
\pgfsetstrokeopacity{0.990000}%
\pgfsetdash{}{0pt}%
\pgfpathmoveto{\pgfqpoint{4.423315in}{0.637495in}}%
\pgfpathlineto{\pgfqpoint{5.198315in}{0.637495in}}%
\pgfpathlineto{\pgfqpoint{5.198315in}{2.758071in}}%
\pgfpathlineto{\pgfqpoint{4.423315in}{2.758071in}}%
\pgfpathclose%
\pgfusepath{fill}%
\end{pgfscope}%
\begin{pgfscope}%
\pgfsetbuttcap%
\pgfsetmiterjoin%
\definecolor{currentfill}{rgb}{0.839216,0.152941,0.156863}%
\pgfsetfillcolor{currentfill}%
\pgfsetfillopacity{0.990000}%
\pgfsetlinewidth{0.000000pt}%
\definecolor{currentstroke}{rgb}{0.000000,0.000000,0.000000}%
\pgfsetstrokecolor{currentstroke}%
\pgfsetstrokeopacity{0.990000}%
\pgfsetdash{}{0pt}%
\pgfpathrectangle{\pgfqpoint{0.935815in}{0.637495in}}{\pgfqpoint{9.300000in}{9.060000in}}%
\pgfusepath{clip}%
\pgfpathmoveto{\pgfqpoint{4.423315in}{0.637495in}}%
\pgfpathlineto{\pgfqpoint{5.198315in}{0.637495in}}%
\pgfpathlineto{\pgfqpoint{5.198315in}{2.758071in}}%
\pgfpathlineto{\pgfqpoint{4.423315in}{2.758071in}}%
\pgfpathclose%
\pgfusepath{clip}%
\pgfsys@defobject{currentpattern}{\pgfqpoint{0in}{0in}}{\pgfqpoint{1in}{1in}}{%
\begin{pgfscope}%
\pgfpathrectangle{\pgfqpoint{0in}{0in}}{\pgfqpoint{1in}{1in}}%
\pgfusepath{clip}%
\pgfpathmoveto{\pgfqpoint{-0.500000in}{0.500000in}}%
\pgfpathlineto{\pgfqpoint{0.500000in}{1.500000in}}%
\pgfpathmoveto{\pgfqpoint{-0.333333in}{0.333333in}}%
\pgfpathlineto{\pgfqpoint{0.666667in}{1.333333in}}%
\pgfpathmoveto{\pgfqpoint{-0.166667in}{0.166667in}}%
\pgfpathlineto{\pgfqpoint{0.833333in}{1.166667in}}%
\pgfpathmoveto{\pgfqpoint{0.000000in}{0.000000in}}%
\pgfpathlineto{\pgfqpoint{1.000000in}{1.000000in}}%
\pgfpathmoveto{\pgfqpoint{0.166667in}{-0.166667in}}%
\pgfpathlineto{\pgfqpoint{1.166667in}{0.833333in}}%
\pgfpathmoveto{\pgfqpoint{0.333333in}{-0.333333in}}%
\pgfpathlineto{\pgfqpoint{1.333333in}{0.666667in}}%
\pgfpathmoveto{\pgfqpoint{0.500000in}{-0.500000in}}%
\pgfpathlineto{\pgfqpoint{1.500000in}{0.500000in}}%
\pgfpathmoveto{\pgfqpoint{-0.500000in}{0.500000in}}%
\pgfpathlineto{\pgfqpoint{0.500000in}{-0.500000in}}%
\pgfpathmoveto{\pgfqpoint{-0.333333in}{0.666667in}}%
\pgfpathlineto{\pgfqpoint{0.666667in}{-0.333333in}}%
\pgfpathmoveto{\pgfqpoint{-0.166667in}{0.833333in}}%
\pgfpathlineto{\pgfqpoint{0.833333in}{-0.166667in}}%
\pgfpathmoveto{\pgfqpoint{0.000000in}{1.000000in}}%
\pgfpathlineto{\pgfqpoint{1.000000in}{0.000000in}}%
\pgfpathmoveto{\pgfqpoint{0.166667in}{1.166667in}}%
\pgfpathlineto{\pgfqpoint{1.166667in}{0.166667in}}%
\pgfpathmoveto{\pgfqpoint{0.333333in}{1.333333in}}%
\pgfpathlineto{\pgfqpoint{1.333333in}{0.333333in}}%
\pgfpathmoveto{\pgfqpoint{0.500000in}{1.500000in}}%
\pgfpathlineto{\pgfqpoint{1.500000in}{0.500000in}}%
\pgfusepath{stroke}%
\end{pgfscope}%
}%
\pgfsys@transformshift{4.423315in}{0.637495in}%
\pgfsys@useobject{currentpattern}{}%
\pgfsys@transformshift{1in}{0in}%
\pgfsys@transformshift{-1in}{0in}%
\pgfsys@transformshift{0in}{1in}%
\pgfsys@useobject{currentpattern}{}%
\pgfsys@transformshift{1in}{0in}%
\pgfsys@transformshift{-1in}{0in}%
\pgfsys@transformshift{0in}{1in}%
\pgfsys@useobject{currentpattern}{}%
\pgfsys@transformshift{1in}{0in}%
\pgfsys@transformshift{-1in}{0in}%
\pgfsys@transformshift{0in}{1in}%
\end{pgfscope}%
\begin{pgfscope}%
\pgfpathrectangle{\pgfqpoint{0.935815in}{0.637495in}}{\pgfqpoint{9.300000in}{9.060000in}}%
\pgfusepath{clip}%
\pgfsetbuttcap%
\pgfsetmiterjoin%
\definecolor{currentfill}{rgb}{0.839216,0.152941,0.156863}%
\pgfsetfillcolor{currentfill}%
\pgfsetfillopacity{0.990000}%
\pgfsetlinewidth{0.000000pt}%
\definecolor{currentstroke}{rgb}{0.000000,0.000000,0.000000}%
\pgfsetstrokecolor{currentstroke}%
\pgfsetstrokeopacity{0.990000}%
\pgfsetdash{}{0pt}%
\pgfpathmoveto{\pgfqpoint{5.973315in}{0.637495in}}%
\pgfpathlineto{\pgfqpoint{6.748315in}{0.637495in}}%
\pgfpathlineto{\pgfqpoint{6.748315in}{2.878536in}}%
\pgfpathlineto{\pgfqpoint{5.973315in}{2.878536in}}%
\pgfpathclose%
\pgfusepath{fill}%
\end{pgfscope}%
\begin{pgfscope}%
\pgfsetbuttcap%
\pgfsetmiterjoin%
\definecolor{currentfill}{rgb}{0.839216,0.152941,0.156863}%
\pgfsetfillcolor{currentfill}%
\pgfsetfillopacity{0.990000}%
\pgfsetlinewidth{0.000000pt}%
\definecolor{currentstroke}{rgb}{0.000000,0.000000,0.000000}%
\pgfsetstrokecolor{currentstroke}%
\pgfsetstrokeopacity{0.990000}%
\pgfsetdash{}{0pt}%
\pgfpathrectangle{\pgfqpoint{0.935815in}{0.637495in}}{\pgfqpoint{9.300000in}{9.060000in}}%
\pgfusepath{clip}%
\pgfpathmoveto{\pgfqpoint{5.973315in}{0.637495in}}%
\pgfpathlineto{\pgfqpoint{6.748315in}{0.637495in}}%
\pgfpathlineto{\pgfqpoint{6.748315in}{2.878536in}}%
\pgfpathlineto{\pgfqpoint{5.973315in}{2.878536in}}%
\pgfpathclose%
\pgfusepath{clip}%
\pgfsys@defobject{currentpattern}{\pgfqpoint{0in}{0in}}{\pgfqpoint{1in}{1in}}{%
\begin{pgfscope}%
\pgfpathrectangle{\pgfqpoint{0in}{0in}}{\pgfqpoint{1in}{1in}}%
\pgfusepath{clip}%
\pgfpathmoveto{\pgfqpoint{-0.500000in}{0.500000in}}%
\pgfpathlineto{\pgfqpoint{0.500000in}{1.500000in}}%
\pgfpathmoveto{\pgfqpoint{-0.333333in}{0.333333in}}%
\pgfpathlineto{\pgfqpoint{0.666667in}{1.333333in}}%
\pgfpathmoveto{\pgfqpoint{-0.166667in}{0.166667in}}%
\pgfpathlineto{\pgfqpoint{0.833333in}{1.166667in}}%
\pgfpathmoveto{\pgfqpoint{0.000000in}{0.000000in}}%
\pgfpathlineto{\pgfqpoint{1.000000in}{1.000000in}}%
\pgfpathmoveto{\pgfqpoint{0.166667in}{-0.166667in}}%
\pgfpathlineto{\pgfqpoint{1.166667in}{0.833333in}}%
\pgfpathmoveto{\pgfqpoint{0.333333in}{-0.333333in}}%
\pgfpathlineto{\pgfqpoint{1.333333in}{0.666667in}}%
\pgfpathmoveto{\pgfqpoint{0.500000in}{-0.500000in}}%
\pgfpathlineto{\pgfqpoint{1.500000in}{0.500000in}}%
\pgfpathmoveto{\pgfqpoint{-0.500000in}{0.500000in}}%
\pgfpathlineto{\pgfqpoint{0.500000in}{-0.500000in}}%
\pgfpathmoveto{\pgfqpoint{-0.333333in}{0.666667in}}%
\pgfpathlineto{\pgfqpoint{0.666667in}{-0.333333in}}%
\pgfpathmoveto{\pgfqpoint{-0.166667in}{0.833333in}}%
\pgfpathlineto{\pgfqpoint{0.833333in}{-0.166667in}}%
\pgfpathmoveto{\pgfqpoint{0.000000in}{1.000000in}}%
\pgfpathlineto{\pgfqpoint{1.000000in}{0.000000in}}%
\pgfpathmoveto{\pgfqpoint{0.166667in}{1.166667in}}%
\pgfpathlineto{\pgfqpoint{1.166667in}{0.166667in}}%
\pgfpathmoveto{\pgfqpoint{0.333333in}{1.333333in}}%
\pgfpathlineto{\pgfqpoint{1.333333in}{0.333333in}}%
\pgfpathmoveto{\pgfqpoint{0.500000in}{1.500000in}}%
\pgfpathlineto{\pgfqpoint{1.500000in}{0.500000in}}%
\pgfusepath{stroke}%
\end{pgfscope}%
}%
\pgfsys@transformshift{5.973315in}{0.637495in}%
\pgfsys@useobject{currentpattern}{}%
\pgfsys@transformshift{1in}{0in}%
\pgfsys@transformshift{-1in}{0in}%
\pgfsys@transformshift{0in}{1in}%
\pgfsys@useobject{currentpattern}{}%
\pgfsys@transformshift{1in}{0in}%
\pgfsys@transformshift{-1in}{0in}%
\pgfsys@transformshift{0in}{1in}%
\pgfsys@useobject{currentpattern}{}%
\pgfsys@transformshift{1in}{0in}%
\pgfsys@transformshift{-1in}{0in}%
\pgfsys@transformshift{0in}{1in}%
\end{pgfscope}%
\begin{pgfscope}%
\pgfpathrectangle{\pgfqpoint{0.935815in}{0.637495in}}{\pgfqpoint{9.300000in}{9.060000in}}%
\pgfusepath{clip}%
\pgfsetbuttcap%
\pgfsetmiterjoin%
\definecolor{currentfill}{rgb}{0.839216,0.152941,0.156863}%
\pgfsetfillcolor{currentfill}%
\pgfsetfillopacity{0.990000}%
\pgfsetlinewidth{0.000000pt}%
\definecolor{currentstroke}{rgb}{0.000000,0.000000,0.000000}%
\pgfsetstrokecolor{currentstroke}%
\pgfsetstrokeopacity{0.990000}%
\pgfsetdash{}{0pt}%
\pgfpathmoveto{\pgfqpoint{7.523315in}{0.637495in}}%
\pgfpathlineto{\pgfqpoint{8.298315in}{0.637495in}}%
\pgfpathlineto{\pgfqpoint{8.298315in}{3.003473in}}%
\pgfpathlineto{\pgfqpoint{7.523315in}{3.003473in}}%
\pgfpathclose%
\pgfusepath{fill}%
\end{pgfscope}%
\begin{pgfscope}%
\pgfsetbuttcap%
\pgfsetmiterjoin%
\definecolor{currentfill}{rgb}{0.839216,0.152941,0.156863}%
\pgfsetfillcolor{currentfill}%
\pgfsetfillopacity{0.990000}%
\pgfsetlinewidth{0.000000pt}%
\definecolor{currentstroke}{rgb}{0.000000,0.000000,0.000000}%
\pgfsetstrokecolor{currentstroke}%
\pgfsetstrokeopacity{0.990000}%
\pgfsetdash{}{0pt}%
\pgfpathrectangle{\pgfqpoint{0.935815in}{0.637495in}}{\pgfqpoint{9.300000in}{9.060000in}}%
\pgfusepath{clip}%
\pgfpathmoveto{\pgfqpoint{7.523315in}{0.637495in}}%
\pgfpathlineto{\pgfqpoint{8.298315in}{0.637495in}}%
\pgfpathlineto{\pgfqpoint{8.298315in}{3.003473in}}%
\pgfpathlineto{\pgfqpoint{7.523315in}{3.003473in}}%
\pgfpathclose%
\pgfusepath{clip}%
\pgfsys@defobject{currentpattern}{\pgfqpoint{0in}{0in}}{\pgfqpoint{1in}{1in}}{%
\begin{pgfscope}%
\pgfpathrectangle{\pgfqpoint{0in}{0in}}{\pgfqpoint{1in}{1in}}%
\pgfusepath{clip}%
\pgfpathmoveto{\pgfqpoint{-0.500000in}{0.500000in}}%
\pgfpathlineto{\pgfqpoint{0.500000in}{1.500000in}}%
\pgfpathmoveto{\pgfqpoint{-0.333333in}{0.333333in}}%
\pgfpathlineto{\pgfqpoint{0.666667in}{1.333333in}}%
\pgfpathmoveto{\pgfqpoint{-0.166667in}{0.166667in}}%
\pgfpathlineto{\pgfqpoint{0.833333in}{1.166667in}}%
\pgfpathmoveto{\pgfqpoint{0.000000in}{0.000000in}}%
\pgfpathlineto{\pgfqpoint{1.000000in}{1.000000in}}%
\pgfpathmoveto{\pgfqpoint{0.166667in}{-0.166667in}}%
\pgfpathlineto{\pgfqpoint{1.166667in}{0.833333in}}%
\pgfpathmoveto{\pgfqpoint{0.333333in}{-0.333333in}}%
\pgfpathlineto{\pgfqpoint{1.333333in}{0.666667in}}%
\pgfpathmoveto{\pgfqpoint{0.500000in}{-0.500000in}}%
\pgfpathlineto{\pgfqpoint{1.500000in}{0.500000in}}%
\pgfpathmoveto{\pgfqpoint{-0.500000in}{0.500000in}}%
\pgfpathlineto{\pgfqpoint{0.500000in}{-0.500000in}}%
\pgfpathmoveto{\pgfqpoint{-0.333333in}{0.666667in}}%
\pgfpathlineto{\pgfqpoint{0.666667in}{-0.333333in}}%
\pgfpathmoveto{\pgfqpoint{-0.166667in}{0.833333in}}%
\pgfpathlineto{\pgfqpoint{0.833333in}{-0.166667in}}%
\pgfpathmoveto{\pgfqpoint{0.000000in}{1.000000in}}%
\pgfpathlineto{\pgfqpoint{1.000000in}{0.000000in}}%
\pgfpathmoveto{\pgfqpoint{0.166667in}{1.166667in}}%
\pgfpathlineto{\pgfqpoint{1.166667in}{0.166667in}}%
\pgfpathmoveto{\pgfqpoint{0.333333in}{1.333333in}}%
\pgfpathlineto{\pgfqpoint{1.333333in}{0.333333in}}%
\pgfpathmoveto{\pgfqpoint{0.500000in}{1.500000in}}%
\pgfpathlineto{\pgfqpoint{1.500000in}{0.500000in}}%
\pgfusepath{stroke}%
\end{pgfscope}%
}%
\pgfsys@transformshift{7.523315in}{0.637495in}%
\pgfsys@useobject{currentpattern}{}%
\pgfsys@transformshift{1in}{0in}%
\pgfsys@transformshift{-1in}{0in}%
\pgfsys@transformshift{0in}{1in}%
\pgfsys@useobject{currentpattern}{}%
\pgfsys@transformshift{1in}{0in}%
\pgfsys@transformshift{-1in}{0in}%
\pgfsys@transformshift{0in}{1in}%
\pgfsys@useobject{currentpattern}{}%
\pgfsys@transformshift{1in}{0in}%
\pgfsys@transformshift{-1in}{0in}%
\pgfsys@transformshift{0in}{1in}%
\end{pgfscope}%
\begin{pgfscope}%
\pgfpathrectangle{\pgfqpoint{0.935815in}{0.637495in}}{\pgfqpoint{9.300000in}{9.060000in}}%
\pgfusepath{clip}%
\pgfsetbuttcap%
\pgfsetmiterjoin%
\definecolor{currentfill}{rgb}{0.839216,0.152941,0.156863}%
\pgfsetfillcolor{currentfill}%
\pgfsetfillopacity{0.990000}%
\pgfsetlinewidth{0.000000pt}%
\definecolor{currentstroke}{rgb}{0.000000,0.000000,0.000000}%
\pgfsetstrokecolor{currentstroke}%
\pgfsetstrokeopacity{0.990000}%
\pgfsetdash{}{0pt}%
\pgfpathmoveto{\pgfqpoint{9.073315in}{0.637495in}}%
\pgfpathlineto{\pgfqpoint{9.848315in}{0.637495in}}%
\pgfpathlineto{\pgfqpoint{9.848315in}{3.003473in}}%
\pgfpathlineto{\pgfqpoint{9.073315in}{3.003473in}}%
\pgfpathclose%
\pgfusepath{fill}%
\end{pgfscope}%
\begin{pgfscope}%
\pgfsetbuttcap%
\pgfsetmiterjoin%
\definecolor{currentfill}{rgb}{0.839216,0.152941,0.156863}%
\pgfsetfillcolor{currentfill}%
\pgfsetfillopacity{0.990000}%
\pgfsetlinewidth{0.000000pt}%
\definecolor{currentstroke}{rgb}{0.000000,0.000000,0.000000}%
\pgfsetstrokecolor{currentstroke}%
\pgfsetstrokeopacity{0.990000}%
\pgfsetdash{}{0pt}%
\pgfpathrectangle{\pgfqpoint{0.935815in}{0.637495in}}{\pgfqpoint{9.300000in}{9.060000in}}%
\pgfusepath{clip}%
\pgfpathmoveto{\pgfqpoint{9.073315in}{0.637495in}}%
\pgfpathlineto{\pgfqpoint{9.848315in}{0.637495in}}%
\pgfpathlineto{\pgfqpoint{9.848315in}{3.003473in}}%
\pgfpathlineto{\pgfqpoint{9.073315in}{3.003473in}}%
\pgfpathclose%
\pgfusepath{clip}%
\pgfsys@defobject{currentpattern}{\pgfqpoint{0in}{0in}}{\pgfqpoint{1in}{1in}}{%
\begin{pgfscope}%
\pgfpathrectangle{\pgfqpoint{0in}{0in}}{\pgfqpoint{1in}{1in}}%
\pgfusepath{clip}%
\pgfpathmoveto{\pgfqpoint{-0.500000in}{0.500000in}}%
\pgfpathlineto{\pgfqpoint{0.500000in}{1.500000in}}%
\pgfpathmoveto{\pgfqpoint{-0.333333in}{0.333333in}}%
\pgfpathlineto{\pgfqpoint{0.666667in}{1.333333in}}%
\pgfpathmoveto{\pgfqpoint{-0.166667in}{0.166667in}}%
\pgfpathlineto{\pgfqpoint{0.833333in}{1.166667in}}%
\pgfpathmoveto{\pgfqpoint{0.000000in}{0.000000in}}%
\pgfpathlineto{\pgfqpoint{1.000000in}{1.000000in}}%
\pgfpathmoveto{\pgfqpoint{0.166667in}{-0.166667in}}%
\pgfpathlineto{\pgfqpoint{1.166667in}{0.833333in}}%
\pgfpathmoveto{\pgfqpoint{0.333333in}{-0.333333in}}%
\pgfpathlineto{\pgfqpoint{1.333333in}{0.666667in}}%
\pgfpathmoveto{\pgfqpoint{0.500000in}{-0.500000in}}%
\pgfpathlineto{\pgfqpoint{1.500000in}{0.500000in}}%
\pgfpathmoveto{\pgfqpoint{-0.500000in}{0.500000in}}%
\pgfpathlineto{\pgfqpoint{0.500000in}{-0.500000in}}%
\pgfpathmoveto{\pgfqpoint{-0.333333in}{0.666667in}}%
\pgfpathlineto{\pgfqpoint{0.666667in}{-0.333333in}}%
\pgfpathmoveto{\pgfqpoint{-0.166667in}{0.833333in}}%
\pgfpathlineto{\pgfqpoint{0.833333in}{-0.166667in}}%
\pgfpathmoveto{\pgfqpoint{0.000000in}{1.000000in}}%
\pgfpathlineto{\pgfqpoint{1.000000in}{0.000000in}}%
\pgfpathmoveto{\pgfqpoint{0.166667in}{1.166667in}}%
\pgfpathlineto{\pgfqpoint{1.166667in}{0.166667in}}%
\pgfpathmoveto{\pgfqpoint{0.333333in}{1.333333in}}%
\pgfpathlineto{\pgfqpoint{1.333333in}{0.333333in}}%
\pgfpathmoveto{\pgfqpoint{0.500000in}{1.500000in}}%
\pgfpathlineto{\pgfqpoint{1.500000in}{0.500000in}}%
\pgfusepath{stroke}%
\end{pgfscope}%
}%
\pgfsys@transformshift{9.073315in}{0.637495in}%
\pgfsys@useobject{currentpattern}{}%
\pgfsys@transformshift{1in}{0in}%
\pgfsys@transformshift{-1in}{0in}%
\pgfsys@transformshift{0in}{1in}%
\pgfsys@useobject{currentpattern}{}%
\pgfsys@transformshift{1in}{0in}%
\pgfsys@transformshift{-1in}{0in}%
\pgfsys@transformshift{0in}{1in}%
\pgfsys@useobject{currentpattern}{}%
\pgfsys@transformshift{1in}{0in}%
\pgfsys@transformshift{-1in}{0in}%
\pgfsys@transformshift{0in}{1in}%
\end{pgfscope}%
\begin{pgfscope}%
\pgfpathrectangle{\pgfqpoint{0.935815in}{0.637495in}}{\pgfqpoint{9.300000in}{9.060000in}}%
\pgfusepath{clip}%
\pgfsetbuttcap%
\pgfsetmiterjoin%
\definecolor{currentfill}{rgb}{0.549020,0.337255,0.294118}%
\pgfsetfillcolor{currentfill}%
\pgfsetfillopacity{0.990000}%
\pgfsetlinewidth{0.000000pt}%
\definecolor{currentstroke}{rgb}{0.000000,0.000000,0.000000}%
\pgfsetstrokecolor{currentstroke}%
\pgfsetstrokeopacity{0.990000}%
\pgfsetdash{}{0pt}%
\pgfpathmoveto{\pgfqpoint{1.323315in}{1.426245in}}%
\pgfpathlineto{\pgfqpoint{2.098315in}{1.426245in}}%
\pgfpathlineto{\pgfqpoint{2.098315in}{2.460694in}}%
\pgfpathlineto{\pgfqpoint{1.323315in}{2.460694in}}%
\pgfpathclose%
\pgfusepath{fill}%
\end{pgfscope}%
\begin{pgfscope}%
\pgfsetbuttcap%
\pgfsetmiterjoin%
\definecolor{currentfill}{rgb}{0.549020,0.337255,0.294118}%
\pgfsetfillcolor{currentfill}%
\pgfsetfillopacity{0.990000}%
\pgfsetlinewidth{0.000000pt}%
\definecolor{currentstroke}{rgb}{0.000000,0.000000,0.000000}%
\pgfsetstrokecolor{currentstroke}%
\pgfsetstrokeopacity{0.990000}%
\pgfsetdash{}{0pt}%
\pgfpathrectangle{\pgfqpoint{0.935815in}{0.637495in}}{\pgfqpoint{9.300000in}{9.060000in}}%
\pgfusepath{clip}%
\pgfpathmoveto{\pgfqpoint{1.323315in}{1.426245in}}%
\pgfpathlineto{\pgfqpoint{2.098315in}{1.426245in}}%
\pgfpathlineto{\pgfqpoint{2.098315in}{2.460694in}}%
\pgfpathlineto{\pgfqpoint{1.323315in}{2.460694in}}%
\pgfpathclose%
\pgfusepath{clip}%
\pgfsys@defobject{currentpattern}{\pgfqpoint{0in}{0in}}{\pgfqpoint{1in}{1in}}{%
\begin{pgfscope}%
\pgfpathrectangle{\pgfqpoint{0in}{0in}}{\pgfqpoint{1in}{1in}}%
\pgfusepath{clip}%
\pgfpathmoveto{\pgfqpoint{0.000000in}{-0.058333in}}%
\pgfpathcurveto{\pgfqpoint{0.015470in}{-0.058333in}}{\pgfqpoint{0.030309in}{-0.052187in}}{\pgfqpoint{0.041248in}{-0.041248in}}%
\pgfpathcurveto{\pgfqpoint{0.052187in}{-0.030309in}}{\pgfqpoint{0.058333in}{-0.015470in}}{\pgfqpoint{0.058333in}{0.000000in}}%
\pgfpathcurveto{\pgfqpoint{0.058333in}{0.015470in}}{\pgfqpoint{0.052187in}{0.030309in}}{\pgfqpoint{0.041248in}{0.041248in}}%
\pgfpathcurveto{\pgfqpoint{0.030309in}{0.052187in}}{\pgfqpoint{0.015470in}{0.058333in}}{\pgfqpoint{0.000000in}{0.058333in}}%
\pgfpathcurveto{\pgfqpoint{-0.015470in}{0.058333in}}{\pgfqpoint{-0.030309in}{0.052187in}}{\pgfqpoint{-0.041248in}{0.041248in}}%
\pgfpathcurveto{\pgfqpoint{-0.052187in}{0.030309in}}{\pgfqpoint{-0.058333in}{0.015470in}}{\pgfqpoint{-0.058333in}{0.000000in}}%
\pgfpathcurveto{\pgfqpoint{-0.058333in}{-0.015470in}}{\pgfqpoint{-0.052187in}{-0.030309in}}{\pgfqpoint{-0.041248in}{-0.041248in}}%
\pgfpathcurveto{\pgfqpoint{-0.030309in}{-0.052187in}}{\pgfqpoint{-0.015470in}{-0.058333in}}{\pgfqpoint{0.000000in}{-0.058333in}}%
\pgfpathclose%
\pgfpathmoveto{\pgfqpoint{0.000000in}{-0.052500in}}%
\pgfpathcurveto{\pgfqpoint{0.000000in}{-0.052500in}}{\pgfqpoint{-0.013923in}{-0.052500in}}{\pgfqpoint{-0.027278in}{-0.046968in}}%
\pgfpathcurveto{\pgfqpoint{-0.037123in}{-0.037123in}}{\pgfqpoint{-0.046968in}{-0.027278in}}{\pgfqpoint{-0.052500in}{-0.013923in}}%
\pgfpathcurveto{\pgfqpoint{-0.052500in}{0.000000in}}{\pgfqpoint{-0.052500in}{0.013923in}}{\pgfqpoint{-0.046968in}{0.027278in}}%
\pgfpathcurveto{\pgfqpoint{-0.037123in}{0.037123in}}{\pgfqpoint{-0.027278in}{0.046968in}}{\pgfqpoint{-0.013923in}{0.052500in}}%
\pgfpathcurveto{\pgfqpoint{0.000000in}{0.052500in}}{\pgfqpoint{0.013923in}{0.052500in}}{\pgfqpoint{0.027278in}{0.046968in}}%
\pgfpathcurveto{\pgfqpoint{0.037123in}{0.037123in}}{\pgfqpoint{0.046968in}{0.027278in}}{\pgfqpoint{0.052500in}{0.013923in}}%
\pgfpathcurveto{\pgfqpoint{0.052500in}{0.000000in}}{\pgfqpoint{0.052500in}{-0.013923in}}{\pgfqpoint{0.046968in}{-0.027278in}}%
\pgfpathcurveto{\pgfqpoint{0.037123in}{-0.037123in}}{\pgfqpoint{0.027278in}{-0.046968in}}{\pgfqpoint{0.013923in}{-0.052500in}}%
\pgfpathclose%
\pgfpathmoveto{\pgfqpoint{0.166667in}{-0.058333in}}%
\pgfpathcurveto{\pgfqpoint{0.182137in}{-0.058333in}}{\pgfqpoint{0.196975in}{-0.052187in}}{\pgfqpoint{0.207915in}{-0.041248in}}%
\pgfpathcurveto{\pgfqpoint{0.218854in}{-0.030309in}}{\pgfqpoint{0.225000in}{-0.015470in}}{\pgfqpoint{0.225000in}{0.000000in}}%
\pgfpathcurveto{\pgfqpoint{0.225000in}{0.015470in}}{\pgfqpoint{0.218854in}{0.030309in}}{\pgfqpoint{0.207915in}{0.041248in}}%
\pgfpathcurveto{\pgfqpoint{0.196975in}{0.052187in}}{\pgfqpoint{0.182137in}{0.058333in}}{\pgfqpoint{0.166667in}{0.058333in}}%
\pgfpathcurveto{\pgfqpoint{0.151196in}{0.058333in}}{\pgfqpoint{0.136358in}{0.052187in}}{\pgfqpoint{0.125419in}{0.041248in}}%
\pgfpathcurveto{\pgfqpoint{0.114480in}{0.030309in}}{\pgfqpoint{0.108333in}{0.015470in}}{\pgfqpoint{0.108333in}{0.000000in}}%
\pgfpathcurveto{\pgfqpoint{0.108333in}{-0.015470in}}{\pgfqpoint{0.114480in}{-0.030309in}}{\pgfqpoint{0.125419in}{-0.041248in}}%
\pgfpathcurveto{\pgfqpoint{0.136358in}{-0.052187in}}{\pgfqpoint{0.151196in}{-0.058333in}}{\pgfqpoint{0.166667in}{-0.058333in}}%
\pgfpathclose%
\pgfpathmoveto{\pgfqpoint{0.166667in}{-0.052500in}}%
\pgfpathcurveto{\pgfqpoint{0.166667in}{-0.052500in}}{\pgfqpoint{0.152744in}{-0.052500in}}{\pgfqpoint{0.139389in}{-0.046968in}}%
\pgfpathcurveto{\pgfqpoint{0.129544in}{-0.037123in}}{\pgfqpoint{0.119698in}{-0.027278in}}{\pgfqpoint{0.114167in}{-0.013923in}}%
\pgfpathcurveto{\pgfqpoint{0.114167in}{0.000000in}}{\pgfqpoint{0.114167in}{0.013923in}}{\pgfqpoint{0.119698in}{0.027278in}}%
\pgfpathcurveto{\pgfqpoint{0.129544in}{0.037123in}}{\pgfqpoint{0.139389in}{0.046968in}}{\pgfqpoint{0.152744in}{0.052500in}}%
\pgfpathcurveto{\pgfqpoint{0.166667in}{0.052500in}}{\pgfqpoint{0.180590in}{0.052500in}}{\pgfqpoint{0.193945in}{0.046968in}}%
\pgfpathcurveto{\pgfqpoint{0.203790in}{0.037123in}}{\pgfqpoint{0.213635in}{0.027278in}}{\pgfqpoint{0.219167in}{0.013923in}}%
\pgfpathcurveto{\pgfqpoint{0.219167in}{0.000000in}}{\pgfqpoint{0.219167in}{-0.013923in}}{\pgfqpoint{0.213635in}{-0.027278in}}%
\pgfpathcurveto{\pgfqpoint{0.203790in}{-0.037123in}}{\pgfqpoint{0.193945in}{-0.046968in}}{\pgfqpoint{0.180590in}{-0.052500in}}%
\pgfpathclose%
\pgfpathmoveto{\pgfqpoint{0.333333in}{-0.058333in}}%
\pgfpathcurveto{\pgfqpoint{0.348804in}{-0.058333in}}{\pgfqpoint{0.363642in}{-0.052187in}}{\pgfqpoint{0.374581in}{-0.041248in}}%
\pgfpathcurveto{\pgfqpoint{0.385520in}{-0.030309in}}{\pgfqpoint{0.391667in}{-0.015470in}}{\pgfqpoint{0.391667in}{0.000000in}}%
\pgfpathcurveto{\pgfqpoint{0.391667in}{0.015470in}}{\pgfqpoint{0.385520in}{0.030309in}}{\pgfqpoint{0.374581in}{0.041248in}}%
\pgfpathcurveto{\pgfqpoint{0.363642in}{0.052187in}}{\pgfqpoint{0.348804in}{0.058333in}}{\pgfqpoint{0.333333in}{0.058333in}}%
\pgfpathcurveto{\pgfqpoint{0.317863in}{0.058333in}}{\pgfqpoint{0.303025in}{0.052187in}}{\pgfqpoint{0.292085in}{0.041248in}}%
\pgfpathcurveto{\pgfqpoint{0.281146in}{0.030309in}}{\pgfqpoint{0.275000in}{0.015470in}}{\pgfqpoint{0.275000in}{0.000000in}}%
\pgfpathcurveto{\pgfqpoint{0.275000in}{-0.015470in}}{\pgfqpoint{0.281146in}{-0.030309in}}{\pgfqpoint{0.292085in}{-0.041248in}}%
\pgfpathcurveto{\pgfqpoint{0.303025in}{-0.052187in}}{\pgfqpoint{0.317863in}{-0.058333in}}{\pgfqpoint{0.333333in}{-0.058333in}}%
\pgfpathclose%
\pgfpathmoveto{\pgfqpoint{0.333333in}{-0.052500in}}%
\pgfpathcurveto{\pgfqpoint{0.333333in}{-0.052500in}}{\pgfqpoint{0.319410in}{-0.052500in}}{\pgfqpoint{0.306055in}{-0.046968in}}%
\pgfpathcurveto{\pgfqpoint{0.296210in}{-0.037123in}}{\pgfqpoint{0.286365in}{-0.027278in}}{\pgfqpoint{0.280833in}{-0.013923in}}%
\pgfpathcurveto{\pgfqpoint{0.280833in}{0.000000in}}{\pgfqpoint{0.280833in}{0.013923in}}{\pgfqpoint{0.286365in}{0.027278in}}%
\pgfpathcurveto{\pgfqpoint{0.296210in}{0.037123in}}{\pgfqpoint{0.306055in}{0.046968in}}{\pgfqpoint{0.319410in}{0.052500in}}%
\pgfpathcurveto{\pgfqpoint{0.333333in}{0.052500in}}{\pgfqpoint{0.347256in}{0.052500in}}{\pgfqpoint{0.360611in}{0.046968in}}%
\pgfpathcurveto{\pgfqpoint{0.370456in}{0.037123in}}{\pgfqpoint{0.380302in}{0.027278in}}{\pgfqpoint{0.385833in}{0.013923in}}%
\pgfpathcurveto{\pgfqpoint{0.385833in}{0.000000in}}{\pgfqpoint{0.385833in}{-0.013923in}}{\pgfqpoint{0.380302in}{-0.027278in}}%
\pgfpathcurveto{\pgfqpoint{0.370456in}{-0.037123in}}{\pgfqpoint{0.360611in}{-0.046968in}}{\pgfqpoint{0.347256in}{-0.052500in}}%
\pgfpathclose%
\pgfpathmoveto{\pgfqpoint{0.500000in}{-0.058333in}}%
\pgfpathcurveto{\pgfqpoint{0.515470in}{-0.058333in}}{\pgfqpoint{0.530309in}{-0.052187in}}{\pgfqpoint{0.541248in}{-0.041248in}}%
\pgfpathcurveto{\pgfqpoint{0.552187in}{-0.030309in}}{\pgfqpoint{0.558333in}{-0.015470in}}{\pgfqpoint{0.558333in}{0.000000in}}%
\pgfpathcurveto{\pgfqpoint{0.558333in}{0.015470in}}{\pgfqpoint{0.552187in}{0.030309in}}{\pgfqpoint{0.541248in}{0.041248in}}%
\pgfpathcurveto{\pgfqpoint{0.530309in}{0.052187in}}{\pgfqpoint{0.515470in}{0.058333in}}{\pgfqpoint{0.500000in}{0.058333in}}%
\pgfpathcurveto{\pgfqpoint{0.484530in}{0.058333in}}{\pgfqpoint{0.469691in}{0.052187in}}{\pgfqpoint{0.458752in}{0.041248in}}%
\pgfpathcurveto{\pgfqpoint{0.447813in}{0.030309in}}{\pgfqpoint{0.441667in}{0.015470in}}{\pgfqpoint{0.441667in}{0.000000in}}%
\pgfpathcurveto{\pgfqpoint{0.441667in}{-0.015470in}}{\pgfqpoint{0.447813in}{-0.030309in}}{\pgfqpoint{0.458752in}{-0.041248in}}%
\pgfpathcurveto{\pgfqpoint{0.469691in}{-0.052187in}}{\pgfqpoint{0.484530in}{-0.058333in}}{\pgfqpoint{0.500000in}{-0.058333in}}%
\pgfpathclose%
\pgfpathmoveto{\pgfqpoint{0.500000in}{-0.052500in}}%
\pgfpathcurveto{\pgfqpoint{0.500000in}{-0.052500in}}{\pgfqpoint{0.486077in}{-0.052500in}}{\pgfqpoint{0.472722in}{-0.046968in}}%
\pgfpathcurveto{\pgfqpoint{0.462877in}{-0.037123in}}{\pgfqpoint{0.453032in}{-0.027278in}}{\pgfqpoint{0.447500in}{-0.013923in}}%
\pgfpathcurveto{\pgfqpoint{0.447500in}{0.000000in}}{\pgfqpoint{0.447500in}{0.013923in}}{\pgfqpoint{0.453032in}{0.027278in}}%
\pgfpathcurveto{\pgfqpoint{0.462877in}{0.037123in}}{\pgfqpoint{0.472722in}{0.046968in}}{\pgfqpoint{0.486077in}{0.052500in}}%
\pgfpathcurveto{\pgfqpoint{0.500000in}{0.052500in}}{\pgfqpoint{0.513923in}{0.052500in}}{\pgfqpoint{0.527278in}{0.046968in}}%
\pgfpathcurveto{\pgfqpoint{0.537123in}{0.037123in}}{\pgfqpoint{0.546968in}{0.027278in}}{\pgfqpoint{0.552500in}{0.013923in}}%
\pgfpathcurveto{\pgfqpoint{0.552500in}{0.000000in}}{\pgfqpoint{0.552500in}{-0.013923in}}{\pgfqpoint{0.546968in}{-0.027278in}}%
\pgfpathcurveto{\pgfqpoint{0.537123in}{-0.037123in}}{\pgfqpoint{0.527278in}{-0.046968in}}{\pgfqpoint{0.513923in}{-0.052500in}}%
\pgfpathclose%
\pgfpathmoveto{\pgfqpoint{0.666667in}{-0.058333in}}%
\pgfpathcurveto{\pgfqpoint{0.682137in}{-0.058333in}}{\pgfqpoint{0.696975in}{-0.052187in}}{\pgfqpoint{0.707915in}{-0.041248in}}%
\pgfpathcurveto{\pgfqpoint{0.718854in}{-0.030309in}}{\pgfqpoint{0.725000in}{-0.015470in}}{\pgfqpoint{0.725000in}{0.000000in}}%
\pgfpathcurveto{\pgfqpoint{0.725000in}{0.015470in}}{\pgfqpoint{0.718854in}{0.030309in}}{\pgfqpoint{0.707915in}{0.041248in}}%
\pgfpathcurveto{\pgfqpoint{0.696975in}{0.052187in}}{\pgfqpoint{0.682137in}{0.058333in}}{\pgfqpoint{0.666667in}{0.058333in}}%
\pgfpathcurveto{\pgfqpoint{0.651196in}{0.058333in}}{\pgfqpoint{0.636358in}{0.052187in}}{\pgfqpoint{0.625419in}{0.041248in}}%
\pgfpathcurveto{\pgfqpoint{0.614480in}{0.030309in}}{\pgfqpoint{0.608333in}{0.015470in}}{\pgfqpoint{0.608333in}{0.000000in}}%
\pgfpathcurveto{\pgfqpoint{0.608333in}{-0.015470in}}{\pgfqpoint{0.614480in}{-0.030309in}}{\pgfqpoint{0.625419in}{-0.041248in}}%
\pgfpathcurveto{\pgfqpoint{0.636358in}{-0.052187in}}{\pgfqpoint{0.651196in}{-0.058333in}}{\pgfqpoint{0.666667in}{-0.058333in}}%
\pgfpathclose%
\pgfpathmoveto{\pgfqpoint{0.666667in}{-0.052500in}}%
\pgfpathcurveto{\pgfqpoint{0.666667in}{-0.052500in}}{\pgfqpoint{0.652744in}{-0.052500in}}{\pgfqpoint{0.639389in}{-0.046968in}}%
\pgfpathcurveto{\pgfqpoint{0.629544in}{-0.037123in}}{\pgfqpoint{0.619698in}{-0.027278in}}{\pgfqpoint{0.614167in}{-0.013923in}}%
\pgfpathcurveto{\pgfqpoint{0.614167in}{0.000000in}}{\pgfqpoint{0.614167in}{0.013923in}}{\pgfqpoint{0.619698in}{0.027278in}}%
\pgfpathcurveto{\pgfqpoint{0.629544in}{0.037123in}}{\pgfqpoint{0.639389in}{0.046968in}}{\pgfqpoint{0.652744in}{0.052500in}}%
\pgfpathcurveto{\pgfqpoint{0.666667in}{0.052500in}}{\pgfqpoint{0.680590in}{0.052500in}}{\pgfqpoint{0.693945in}{0.046968in}}%
\pgfpathcurveto{\pgfqpoint{0.703790in}{0.037123in}}{\pgfqpoint{0.713635in}{0.027278in}}{\pgfqpoint{0.719167in}{0.013923in}}%
\pgfpathcurveto{\pgfqpoint{0.719167in}{0.000000in}}{\pgfqpoint{0.719167in}{-0.013923in}}{\pgfqpoint{0.713635in}{-0.027278in}}%
\pgfpathcurveto{\pgfqpoint{0.703790in}{-0.037123in}}{\pgfqpoint{0.693945in}{-0.046968in}}{\pgfqpoint{0.680590in}{-0.052500in}}%
\pgfpathclose%
\pgfpathmoveto{\pgfqpoint{0.833333in}{-0.058333in}}%
\pgfpathcurveto{\pgfqpoint{0.848804in}{-0.058333in}}{\pgfqpoint{0.863642in}{-0.052187in}}{\pgfqpoint{0.874581in}{-0.041248in}}%
\pgfpathcurveto{\pgfqpoint{0.885520in}{-0.030309in}}{\pgfqpoint{0.891667in}{-0.015470in}}{\pgfqpoint{0.891667in}{0.000000in}}%
\pgfpathcurveto{\pgfqpoint{0.891667in}{0.015470in}}{\pgfqpoint{0.885520in}{0.030309in}}{\pgfqpoint{0.874581in}{0.041248in}}%
\pgfpathcurveto{\pgfqpoint{0.863642in}{0.052187in}}{\pgfqpoint{0.848804in}{0.058333in}}{\pgfqpoint{0.833333in}{0.058333in}}%
\pgfpathcurveto{\pgfqpoint{0.817863in}{0.058333in}}{\pgfqpoint{0.803025in}{0.052187in}}{\pgfqpoint{0.792085in}{0.041248in}}%
\pgfpathcurveto{\pgfqpoint{0.781146in}{0.030309in}}{\pgfqpoint{0.775000in}{0.015470in}}{\pgfqpoint{0.775000in}{0.000000in}}%
\pgfpathcurveto{\pgfqpoint{0.775000in}{-0.015470in}}{\pgfqpoint{0.781146in}{-0.030309in}}{\pgfqpoint{0.792085in}{-0.041248in}}%
\pgfpathcurveto{\pgfqpoint{0.803025in}{-0.052187in}}{\pgfqpoint{0.817863in}{-0.058333in}}{\pgfqpoint{0.833333in}{-0.058333in}}%
\pgfpathclose%
\pgfpathmoveto{\pgfqpoint{0.833333in}{-0.052500in}}%
\pgfpathcurveto{\pgfqpoint{0.833333in}{-0.052500in}}{\pgfqpoint{0.819410in}{-0.052500in}}{\pgfqpoint{0.806055in}{-0.046968in}}%
\pgfpathcurveto{\pgfqpoint{0.796210in}{-0.037123in}}{\pgfqpoint{0.786365in}{-0.027278in}}{\pgfqpoint{0.780833in}{-0.013923in}}%
\pgfpathcurveto{\pgfqpoint{0.780833in}{0.000000in}}{\pgfqpoint{0.780833in}{0.013923in}}{\pgfqpoint{0.786365in}{0.027278in}}%
\pgfpathcurveto{\pgfqpoint{0.796210in}{0.037123in}}{\pgfqpoint{0.806055in}{0.046968in}}{\pgfqpoint{0.819410in}{0.052500in}}%
\pgfpathcurveto{\pgfqpoint{0.833333in}{0.052500in}}{\pgfqpoint{0.847256in}{0.052500in}}{\pgfqpoint{0.860611in}{0.046968in}}%
\pgfpathcurveto{\pgfqpoint{0.870456in}{0.037123in}}{\pgfqpoint{0.880302in}{0.027278in}}{\pgfqpoint{0.885833in}{0.013923in}}%
\pgfpathcurveto{\pgfqpoint{0.885833in}{0.000000in}}{\pgfqpoint{0.885833in}{-0.013923in}}{\pgfqpoint{0.880302in}{-0.027278in}}%
\pgfpathcurveto{\pgfqpoint{0.870456in}{-0.037123in}}{\pgfqpoint{0.860611in}{-0.046968in}}{\pgfqpoint{0.847256in}{-0.052500in}}%
\pgfpathclose%
\pgfpathmoveto{\pgfqpoint{1.000000in}{-0.058333in}}%
\pgfpathcurveto{\pgfqpoint{1.015470in}{-0.058333in}}{\pgfqpoint{1.030309in}{-0.052187in}}{\pgfqpoint{1.041248in}{-0.041248in}}%
\pgfpathcurveto{\pgfqpoint{1.052187in}{-0.030309in}}{\pgfqpoint{1.058333in}{-0.015470in}}{\pgfqpoint{1.058333in}{0.000000in}}%
\pgfpathcurveto{\pgfqpoint{1.058333in}{0.015470in}}{\pgfqpoint{1.052187in}{0.030309in}}{\pgfqpoint{1.041248in}{0.041248in}}%
\pgfpathcurveto{\pgfqpoint{1.030309in}{0.052187in}}{\pgfqpoint{1.015470in}{0.058333in}}{\pgfqpoint{1.000000in}{0.058333in}}%
\pgfpathcurveto{\pgfqpoint{0.984530in}{0.058333in}}{\pgfqpoint{0.969691in}{0.052187in}}{\pgfqpoint{0.958752in}{0.041248in}}%
\pgfpathcurveto{\pgfqpoint{0.947813in}{0.030309in}}{\pgfqpoint{0.941667in}{0.015470in}}{\pgfqpoint{0.941667in}{0.000000in}}%
\pgfpathcurveto{\pgfqpoint{0.941667in}{-0.015470in}}{\pgfqpoint{0.947813in}{-0.030309in}}{\pgfqpoint{0.958752in}{-0.041248in}}%
\pgfpathcurveto{\pgfqpoint{0.969691in}{-0.052187in}}{\pgfqpoint{0.984530in}{-0.058333in}}{\pgfqpoint{1.000000in}{-0.058333in}}%
\pgfpathclose%
\pgfpathmoveto{\pgfqpoint{1.000000in}{-0.052500in}}%
\pgfpathcurveto{\pgfqpoint{1.000000in}{-0.052500in}}{\pgfqpoint{0.986077in}{-0.052500in}}{\pgfqpoint{0.972722in}{-0.046968in}}%
\pgfpathcurveto{\pgfqpoint{0.962877in}{-0.037123in}}{\pgfqpoint{0.953032in}{-0.027278in}}{\pgfqpoint{0.947500in}{-0.013923in}}%
\pgfpathcurveto{\pgfqpoint{0.947500in}{0.000000in}}{\pgfqpoint{0.947500in}{0.013923in}}{\pgfqpoint{0.953032in}{0.027278in}}%
\pgfpathcurveto{\pgfqpoint{0.962877in}{0.037123in}}{\pgfqpoint{0.972722in}{0.046968in}}{\pgfqpoint{0.986077in}{0.052500in}}%
\pgfpathcurveto{\pgfqpoint{1.000000in}{0.052500in}}{\pgfqpoint{1.013923in}{0.052500in}}{\pgfqpoint{1.027278in}{0.046968in}}%
\pgfpathcurveto{\pgfqpoint{1.037123in}{0.037123in}}{\pgfqpoint{1.046968in}{0.027278in}}{\pgfqpoint{1.052500in}{0.013923in}}%
\pgfpathcurveto{\pgfqpoint{1.052500in}{0.000000in}}{\pgfqpoint{1.052500in}{-0.013923in}}{\pgfqpoint{1.046968in}{-0.027278in}}%
\pgfpathcurveto{\pgfqpoint{1.037123in}{-0.037123in}}{\pgfqpoint{1.027278in}{-0.046968in}}{\pgfqpoint{1.013923in}{-0.052500in}}%
\pgfpathclose%
\pgfpathmoveto{\pgfqpoint{0.083333in}{0.108333in}}%
\pgfpathcurveto{\pgfqpoint{0.098804in}{0.108333in}}{\pgfqpoint{0.113642in}{0.114480in}}{\pgfqpoint{0.124581in}{0.125419in}}%
\pgfpathcurveto{\pgfqpoint{0.135520in}{0.136358in}}{\pgfqpoint{0.141667in}{0.151196in}}{\pgfqpoint{0.141667in}{0.166667in}}%
\pgfpathcurveto{\pgfqpoint{0.141667in}{0.182137in}}{\pgfqpoint{0.135520in}{0.196975in}}{\pgfqpoint{0.124581in}{0.207915in}}%
\pgfpathcurveto{\pgfqpoint{0.113642in}{0.218854in}}{\pgfqpoint{0.098804in}{0.225000in}}{\pgfqpoint{0.083333in}{0.225000in}}%
\pgfpathcurveto{\pgfqpoint{0.067863in}{0.225000in}}{\pgfqpoint{0.053025in}{0.218854in}}{\pgfqpoint{0.042085in}{0.207915in}}%
\pgfpathcurveto{\pgfqpoint{0.031146in}{0.196975in}}{\pgfqpoint{0.025000in}{0.182137in}}{\pgfqpoint{0.025000in}{0.166667in}}%
\pgfpathcurveto{\pgfqpoint{0.025000in}{0.151196in}}{\pgfqpoint{0.031146in}{0.136358in}}{\pgfqpoint{0.042085in}{0.125419in}}%
\pgfpathcurveto{\pgfqpoint{0.053025in}{0.114480in}}{\pgfqpoint{0.067863in}{0.108333in}}{\pgfqpoint{0.083333in}{0.108333in}}%
\pgfpathclose%
\pgfpathmoveto{\pgfqpoint{0.083333in}{0.114167in}}%
\pgfpathcurveto{\pgfqpoint{0.083333in}{0.114167in}}{\pgfqpoint{0.069410in}{0.114167in}}{\pgfqpoint{0.056055in}{0.119698in}}%
\pgfpathcurveto{\pgfqpoint{0.046210in}{0.129544in}}{\pgfqpoint{0.036365in}{0.139389in}}{\pgfqpoint{0.030833in}{0.152744in}}%
\pgfpathcurveto{\pgfqpoint{0.030833in}{0.166667in}}{\pgfqpoint{0.030833in}{0.180590in}}{\pgfqpoint{0.036365in}{0.193945in}}%
\pgfpathcurveto{\pgfqpoint{0.046210in}{0.203790in}}{\pgfqpoint{0.056055in}{0.213635in}}{\pgfqpoint{0.069410in}{0.219167in}}%
\pgfpathcurveto{\pgfqpoint{0.083333in}{0.219167in}}{\pgfqpoint{0.097256in}{0.219167in}}{\pgfqpoint{0.110611in}{0.213635in}}%
\pgfpathcurveto{\pgfqpoint{0.120456in}{0.203790in}}{\pgfqpoint{0.130302in}{0.193945in}}{\pgfqpoint{0.135833in}{0.180590in}}%
\pgfpathcurveto{\pgfqpoint{0.135833in}{0.166667in}}{\pgfqpoint{0.135833in}{0.152744in}}{\pgfqpoint{0.130302in}{0.139389in}}%
\pgfpathcurveto{\pgfqpoint{0.120456in}{0.129544in}}{\pgfqpoint{0.110611in}{0.119698in}}{\pgfqpoint{0.097256in}{0.114167in}}%
\pgfpathclose%
\pgfpathmoveto{\pgfqpoint{0.250000in}{0.108333in}}%
\pgfpathcurveto{\pgfqpoint{0.265470in}{0.108333in}}{\pgfqpoint{0.280309in}{0.114480in}}{\pgfqpoint{0.291248in}{0.125419in}}%
\pgfpathcurveto{\pgfqpoint{0.302187in}{0.136358in}}{\pgfqpoint{0.308333in}{0.151196in}}{\pgfqpoint{0.308333in}{0.166667in}}%
\pgfpathcurveto{\pgfqpoint{0.308333in}{0.182137in}}{\pgfqpoint{0.302187in}{0.196975in}}{\pgfqpoint{0.291248in}{0.207915in}}%
\pgfpathcurveto{\pgfqpoint{0.280309in}{0.218854in}}{\pgfqpoint{0.265470in}{0.225000in}}{\pgfqpoint{0.250000in}{0.225000in}}%
\pgfpathcurveto{\pgfqpoint{0.234530in}{0.225000in}}{\pgfqpoint{0.219691in}{0.218854in}}{\pgfqpoint{0.208752in}{0.207915in}}%
\pgfpathcurveto{\pgfqpoint{0.197813in}{0.196975in}}{\pgfqpoint{0.191667in}{0.182137in}}{\pgfqpoint{0.191667in}{0.166667in}}%
\pgfpathcurveto{\pgfqpoint{0.191667in}{0.151196in}}{\pgfqpoint{0.197813in}{0.136358in}}{\pgfqpoint{0.208752in}{0.125419in}}%
\pgfpathcurveto{\pgfqpoint{0.219691in}{0.114480in}}{\pgfqpoint{0.234530in}{0.108333in}}{\pgfqpoint{0.250000in}{0.108333in}}%
\pgfpathclose%
\pgfpathmoveto{\pgfqpoint{0.250000in}{0.114167in}}%
\pgfpathcurveto{\pgfqpoint{0.250000in}{0.114167in}}{\pgfqpoint{0.236077in}{0.114167in}}{\pgfqpoint{0.222722in}{0.119698in}}%
\pgfpathcurveto{\pgfqpoint{0.212877in}{0.129544in}}{\pgfqpoint{0.203032in}{0.139389in}}{\pgfqpoint{0.197500in}{0.152744in}}%
\pgfpathcurveto{\pgfqpoint{0.197500in}{0.166667in}}{\pgfqpoint{0.197500in}{0.180590in}}{\pgfqpoint{0.203032in}{0.193945in}}%
\pgfpathcurveto{\pgfqpoint{0.212877in}{0.203790in}}{\pgfqpoint{0.222722in}{0.213635in}}{\pgfqpoint{0.236077in}{0.219167in}}%
\pgfpathcurveto{\pgfqpoint{0.250000in}{0.219167in}}{\pgfqpoint{0.263923in}{0.219167in}}{\pgfqpoint{0.277278in}{0.213635in}}%
\pgfpathcurveto{\pgfqpoint{0.287123in}{0.203790in}}{\pgfqpoint{0.296968in}{0.193945in}}{\pgfqpoint{0.302500in}{0.180590in}}%
\pgfpathcurveto{\pgfqpoint{0.302500in}{0.166667in}}{\pgfqpoint{0.302500in}{0.152744in}}{\pgfqpoint{0.296968in}{0.139389in}}%
\pgfpathcurveto{\pgfqpoint{0.287123in}{0.129544in}}{\pgfqpoint{0.277278in}{0.119698in}}{\pgfqpoint{0.263923in}{0.114167in}}%
\pgfpathclose%
\pgfpathmoveto{\pgfqpoint{0.416667in}{0.108333in}}%
\pgfpathcurveto{\pgfqpoint{0.432137in}{0.108333in}}{\pgfqpoint{0.446975in}{0.114480in}}{\pgfqpoint{0.457915in}{0.125419in}}%
\pgfpathcurveto{\pgfqpoint{0.468854in}{0.136358in}}{\pgfqpoint{0.475000in}{0.151196in}}{\pgfqpoint{0.475000in}{0.166667in}}%
\pgfpathcurveto{\pgfqpoint{0.475000in}{0.182137in}}{\pgfqpoint{0.468854in}{0.196975in}}{\pgfqpoint{0.457915in}{0.207915in}}%
\pgfpathcurveto{\pgfqpoint{0.446975in}{0.218854in}}{\pgfqpoint{0.432137in}{0.225000in}}{\pgfqpoint{0.416667in}{0.225000in}}%
\pgfpathcurveto{\pgfqpoint{0.401196in}{0.225000in}}{\pgfqpoint{0.386358in}{0.218854in}}{\pgfqpoint{0.375419in}{0.207915in}}%
\pgfpathcurveto{\pgfqpoint{0.364480in}{0.196975in}}{\pgfqpoint{0.358333in}{0.182137in}}{\pgfqpoint{0.358333in}{0.166667in}}%
\pgfpathcurveto{\pgfqpoint{0.358333in}{0.151196in}}{\pgfqpoint{0.364480in}{0.136358in}}{\pgfqpoint{0.375419in}{0.125419in}}%
\pgfpathcurveto{\pgfqpoint{0.386358in}{0.114480in}}{\pgfqpoint{0.401196in}{0.108333in}}{\pgfqpoint{0.416667in}{0.108333in}}%
\pgfpathclose%
\pgfpathmoveto{\pgfqpoint{0.416667in}{0.114167in}}%
\pgfpathcurveto{\pgfqpoint{0.416667in}{0.114167in}}{\pgfqpoint{0.402744in}{0.114167in}}{\pgfqpoint{0.389389in}{0.119698in}}%
\pgfpathcurveto{\pgfqpoint{0.379544in}{0.129544in}}{\pgfqpoint{0.369698in}{0.139389in}}{\pgfqpoint{0.364167in}{0.152744in}}%
\pgfpathcurveto{\pgfqpoint{0.364167in}{0.166667in}}{\pgfqpoint{0.364167in}{0.180590in}}{\pgfqpoint{0.369698in}{0.193945in}}%
\pgfpathcurveto{\pgfqpoint{0.379544in}{0.203790in}}{\pgfqpoint{0.389389in}{0.213635in}}{\pgfqpoint{0.402744in}{0.219167in}}%
\pgfpathcurveto{\pgfqpoint{0.416667in}{0.219167in}}{\pgfqpoint{0.430590in}{0.219167in}}{\pgfqpoint{0.443945in}{0.213635in}}%
\pgfpathcurveto{\pgfqpoint{0.453790in}{0.203790in}}{\pgfqpoint{0.463635in}{0.193945in}}{\pgfqpoint{0.469167in}{0.180590in}}%
\pgfpathcurveto{\pgfqpoint{0.469167in}{0.166667in}}{\pgfqpoint{0.469167in}{0.152744in}}{\pgfqpoint{0.463635in}{0.139389in}}%
\pgfpathcurveto{\pgfqpoint{0.453790in}{0.129544in}}{\pgfqpoint{0.443945in}{0.119698in}}{\pgfqpoint{0.430590in}{0.114167in}}%
\pgfpathclose%
\pgfpathmoveto{\pgfqpoint{0.583333in}{0.108333in}}%
\pgfpathcurveto{\pgfqpoint{0.598804in}{0.108333in}}{\pgfqpoint{0.613642in}{0.114480in}}{\pgfqpoint{0.624581in}{0.125419in}}%
\pgfpathcurveto{\pgfqpoint{0.635520in}{0.136358in}}{\pgfqpoint{0.641667in}{0.151196in}}{\pgfqpoint{0.641667in}{0.166667in}}%
\pgfpathcurveto{\pgfqpoint{0.641667in}{0.182137in}}{\pgfqpoint{0.635520in}{0.196975in}}{\pgfqpoint{0.624581in}{0.207915in}}%
\pgfpathcurveto{\pgfqpoint{0.613642in}{0.218854in}}{\pgfqpoint{0.598804in}{0.225000in}}{\pgfqpoint{0.583333in}{0.225000in}}%
\pgfpathcurveto{\pgfqpoint{0.567863in}{0.225000in}}{\pgfqpoint{0.553025in}{0.218854in}}{\pgfqpoint{0.542085in}{0.207915in}}%
\pgfpathcurveto{\pgfqpoint{0.531146in}{0.196975in}}{\pgfqpoint{0.525000in}{0.182137in}}{\pgfqpoint{0.525000in}{0.166667in}}%
\pgfpathcurveto{\pgfqpoint{0.525000in}{0.151196in}}{\pgfqpoint{0.531146in}{0.136358in}}{\pgfqpoint{0.542085in}{0.125419in}}%
\pgfpathcurveto{\pgfqpoint{0.553025in}{0.114480in}}{\pgfqpoint{0.567863in}{0.108333in}}{\pgfqpoint{0.583333in}{0.108333in}}%
\pgfpathclose%
\pgfpathmoveto{\pgfqpoint{0.583333in}{0.114167in}}%
\pgfpathcurveto{\pgfqpoint{0.583333in}{0.114167in}}{\pgfqpoint{0.569410in}{0.114167in}}{\pgfqpoint{0.556055in}{0.119698in}}%
\pgfpathcurveto{\pgfqpoint{0.546210in}{0.129544in}}{\pgfqpoint{0.536365in}{0.139389in}}{\pgfqpoint{0.530833in}{0.152744in}}%
\pgfpathcurveto{\pgfqpoint{0.530833in}{0.166667in}}{\pgfqpoint{0.530833in}{0.180590in}}{\pgfqpoint{0.536365in}{0.193945in}}%
\pgfpathcurveto{\pgfqpoint{0.546210in}{0.203790in}}{\pgfqpoint{0.556055in}{0.213635in}}{\pgfqpoint{0.569410in}{0.219167in}}%
\pgfpathcurveto{\pgfqpoint{0.583333in}{0.219167in}}{\pgfqpoint{0.597256in}{0.219167in}}{\pgfqpoint{0.610611in}{0.213635in}}%
\pgfpathcurveto{\pgfqpoint{0.620456in}{0.203790in}}{\pgfqpoint{0.630302in}{0.193945in}}{\pgfqpoint{0.635833in}{0.180590in}}%
\pgfpathcurveto{\pgfqpoint{0.635833in}{0.166667in}}{\pgfqpoint{0.635833in}{0.152744in}}{\pgfqpoint{0.630302in}{0.139389in}}%
\pgfpathcurveto{\pgfqpoint{0.620456in}{0.129544in}}{\pgfqpoint{0.610611in}{0.119698in}}{\pgfqpoint{0.597256in}{0.114167in}}%
\pgfpathclose%
\pgfpathmoveto{\pgfqpoint{0.750000in}{0.108333in}}%
\pgfpathcurveto{\pgfqpoint{0.765470in}{0.108333in}}{\pgfqpoint{0.780309in}{0.114480in}}{\pgfqpoint{0.791248in}{0.125419in}}%
\pgfpathcurveto{\pgfqpoint{0.802187in}{0.136358in}}{\pgfqpoint{0.808333in}{0.151196in}}{\pgfqpoint{0.808333in}{0.166667in}}%
\pgfpathcurveto{\pgfqpoint{0.808333in}{0.182137in}}{\pgfqpoint{0.802187in}{0.196975in}}{\pgfqpoint{0.791248in}{0.207915in}}%
\pgfpathcurveto{\pgfqpoint{0.780309in}{0.218854in}}{\pgfqpoint{0.765470in}{0.225000in}}{\pgfqpoint{0.750000in}{0.225000in}}%
\pgfpathcurveto{\pgfqpoint{0.734530in}{0.225000in}}{\pgfqpoint{0.719691in}{0.218854in}}{\pgfqpoint{0.708752in}{0.207915in}}%
\pgfpathcurveto{\pgfqpoint{0.697813in}{0.196975in}}{\pgfqpoint{0.691667in}{0.182137in}}{\pgfqpoint{0.691667in}{0.166667in}}%
\pgfpathcurveto{\pgfqpoint{0.691667in}{0.151196in}}{\pgfqpoint{0.697813in}{0.136358in}}{\pgfqpoint{0.708752in}{0.125419in}}%
\pgfpathcurveto{\pgfqpoint{0.719691in}{0.114480in}}{\pgfqpoint{0.734530in}{0.108333in}}{\pgfqpoint{0.750000in}{0.108333in}}%
\pgfpathclose%
\pgfpathmoveto{\pgfqpoint{0.750000in}{0.114167in}}%
\pgfpathcurveto{\pgfqpoint{0.750000in}{0.114167in}}{\pgfqpoint{0.736077in}{0.114167in}}{\pgfqpoint{0.722722in}{0.119698in}}%
\pgfpathcurveto{\pgfqpoint{0.712877in}{0.129544in}}{\pgfqpoint{0.703032in}{0.139389in}}{\pgfqpoint{0.697500in}{0.152744in}}%
\pgfpathcurveto{\pgfqpoint{0.697500in}{0.166667in}}{\pgfqpoint{0.697500in}{0.180590in}}{\pgfqpoint{0.703032in}{0.193945in}}%
\pgfpathcurveto{\pgfqpoint{0.712877in}{0.203790in}}{\pgfqpoint{0.722722in}{0.213635in}}{\pgfqpoint{0.736077in}{0.219167in}}%
\pgfpathcurveto{\pgfqpoint{0.750000in}{0.219167in}}{\pgfqpoint{0.763923in}{0.219167in}}{\pgfqpoint{0.777278in}{0.213635in}}%
\pgfpathcurveto{\pgfqpoint{0.787123in}{0.203790in}}{\pgfqpoint{0.796968in}{0.193945in}}{\pgfqpoint{0.802500in}{0.180590in}}%
\pgfpathcurveto{\pgfqpoint{0.802500in}{0.166667in}}{\pgfqpoint{0.802500in}{0.152744in}}{\pgfqpoint{0.796968in}{0.139389in}}%
\pgfpathcurveto{\pgfqpoint{0.787123in}{0.129544in}}{\pgfqpoint{0.777278in}{0.119698in}}{\pgfqpoint{0.763923in}{0.114167in}}%
\pgfpathclose%
\pgfpathmoveto{\pgfqpoint{0.916667in}{0.108333in}}%
\pgfpathcurveto{\pgfqpoint{0.932137in}{0.108333in}}{\pgfqpoint{0.946975in}{0.114480in}}{\pgfqpoint{0.957915in}{0.125419in}}%
\pgfpathcurveto{\pgfqpoint{0.968854in}{0.136358in}}{\pgfqpoint{0.975000in}{0.151196in}}{\pgfqpoint{0.975000in}{0.166667in}}%
\pgfpathcurveto{\pgfqpoint{0.975000in}{0.182137in}}{\pgfqpoint{0.968854in}{0.196975in}}{\pgfqpoint{0.957915in}{0.207915in}}%
\pgfpathcurveto{\pgfqpoint{0.946975in}{0.218854in}}{\pgfqpoint{0.932137in}{0.225000in}}{\pgfqpoint{0.916667in}{0.225000in}}%
\pgfpathcurveto{\pgfqpoint{0.901196in}{0.225000in}}{\pgfqpoint{0.886358in}{0.218854in}}{\pgfqpoint{0.875419in}{0.207915in}}%
\pgfpathcurveto{\pgfqpoint{0.864480in}{0.196975in}}{\pgfqpoint{0.858333in}{0.182137in}}{\pgfqpoint{0.858333in}{0.166667in}}%
\pgfpathcurveto{\pgfqpoint{0.858333in}{0.151196in}}{\pgfqpoint{0.864480in}{0.136358in}}{\pgfqpoint{0.875419in}{0.125419in}}%
\pgfpathcurveto{\pgfqpoint{0.886358in}{0.114480in}}{\pgfqpoint{0.901196in}{0.108333in}}{\pgfqpoint{0.916667in}{0.108333in}}%
\pgfpathclose%
\pgfpathmoveto{\pgfqpoint{0.916667in}{0.114167in}}%
\pgfpathcurveto{\pgfqpoint{0.916667in}{0.114167in}}{\pgfqpoint{0.902744in}{0.114167in}}{\pgfqpoint{0.889389in}{0.119698in}}%
\pgfpathcurveto{\pgfqpoint{0.879544in}{0.129544in}}{\pgfqpoint{0.869698in}{0.139389in}}{\pgfqpoint{0.864167in}{0.152744in}}%
\pgfpathcurveto{\pgfqpoint{0.864167in}{0.166667in}}{\pgfqpoint{0.864167in}{0.180590in}}{\pgfqpoint{0.869698in}{0.193945in}}%
\pgfpathcurveto{\pgfqpoint{0.879544in}{0.203790in}}{\pgfqpoint{0.889389in}{0.213635in}}{\pgfqpoint{0.902744in}{0.219167in}}%
\pgfpathcurveto{\pgfqpoint{0.916667in}{0.219167in}}{\pgfqpoint{0.930590in}{0.219167in}}{\pgfqpoint{0.943945in}{0.213635in}}%
\pgfpathcurveto{\pgfqpoint{0.953790in}{0.203790in}}{\pgfqpoint{0.963635in}{0.193945in}}{\pgfqpoint{0.969167in}{0.180590in}}%
\pgfpathcurveto{\pgfqpoint{0.969167in}{0.166667in}}{\pgfqpoint{0.969167in}{0.152744in}}{\pgfqpoint{0.963635in}{0.139389in}}%
\pgfpathcurveto{\pgfqpoint{0.953790in}{0.129544in}}{\pgfqpoint{0.943945in}{0.119698in}}{\pgfqpoint{0.930590in}{0.114167in}}%
\pgfpathclose%
\pgfpathmoveto{\pgfqpoint{0.000000in}{0.275000in}}%
\pgfpathcurveto{\pgfqpoint{0.015470in}{0.275000in}}{\pgfqpoint{0.030309in}{0.281146in}}{\pgfqpoint{0.041248in}{0.292085in}}%
\pgfpathcurveto{\pgfqpoint{0.052187in}{0.303025in}}{\pgfqpoint{0.058333in}{0.317863in}}{\pgfqpoint{0.058333in}{0.333333in}}%
\pgfpathcurveto{\pgfqpoint{0.058333in}{0.348804in}}{\pgfqpoint{0.052187in}{0.363642in}}{\pgfqpoint{0.041248in}{0.374581in}}%
\pgfpathcurveto{\pgfqpoint{0.030309in}{0.385520in}}{\pgfqpoint{0.015470in}{0.391667in}}{\pgfqpoint{0.000000in}{0.391667in}}%
\pgfpathcurveto{\pgfqpoint{-0.015470in}{0.391667in}}{\pgfqpoint{-0.030309in}{0.385520in}}{\pgfqpoint{-0.041248in}{0.374581in}}%
\pgfpathcurveto{\pgfqpoint{-0.052187in}{0.363642in}}{\pgfqpoint{-0.058333in}{0.348804in}}{\pgfqpoint{-0.058333in}{0.333333in}}%
\pgfpathcurveto{\pgfqpoint{-0.058333in}{0.317863in}}{\pgfqpoint{-0.052187in}{0.303025in}}{\pgfqpoint{-0.041248in}{0.292085in}}%
\pgfpathcurveto{\pgfqpoint{-0.030309in}{0.281146in}}{\pgfqpoint{-0.015470in}{0.275000in}}{\pgfqpoint{0.000000in}{0.275000in}}%
\pgfpathclose%
\pgfpathmoveto{\pgfqpoint{0.000000in}{0.280833in}}%
\pgfpathcurveto{\pgfqpoint{0.000000in}{0.280833in}}{\pgfqpoint{-0.013923in}{0.280833in}}{\pgfqpoint{-0.027278in}{0.286365in}}%
\pgfpathcurveto{\pgfqpoint{-0.037123in}{0.296210in}}{\pgfqpoint{-0.046968in}{0.306055in}}{\pgfqpoint{-0.052500in}{0.319410in}}%
\pgfpathcurveto{\pgfqpoint{-0.052500in}{0.333333in}}{\pgfqpoint{-0.052500in}{0.347256in}}{\pgfqpoint{-0.046968in}{0.360611in}}%
\pgfpathcurveto{\pgfqpoint{-0.037123in}{0.370456in}}{\pgfqpoint{-0.027278in}{0.380302in}}{\pgfqpoint{-0.013923in}{0.385833in}}%
\pgfpathcurveto{\pgfqpoint{0.000000in}{0.385833in}}{\pgfqpoint{0.013923in}{0.385833in}}{\pgfqpoint{0.027278in}{0.380302in}}%
\pgfpathcurveto{\pgfqpoint{0.037123in}{0.370456in}}{\pgfqpoint{0.046968in}{0.360611in}}{\pgfqpoint{0.052500in}{0.347256in}}%
\pgfpathcurveto{\pgfqpoint{0.052500in}{0.333333in}}{\pgfqpoint{0.052500in}{0.319410in}}{\pgfqpoint{0.046968in}{0.306055in}}%
\pgfpathcurveto{\pgfqpoint{0.037123in}{0.296210in}}{\pgfqpoint{0.027278in}{0.286365in}}{\pgfqpoint{0.013923in}{0.280833in}}%
\pgfpathclose%
\pgfpathmoveto{\pgfqpoint{0.166667in}{0.275000in}}%
\pgfpathcurveto{\pgfqpoint{0.182137in}{0.275000in}}{\pgfqpoint{0.196975in}{0.281146in}}{\pgfqpoint{0.207915in}{0.292085in}}%
\pgfpathcurveto{\pgfqpoint{0.218854in}{0.303025in}}{\pgfqpoint{0.225000in}{0.317863in}}{\pgfqpoint{0.225000in}{0.333333in}}%
\pgfpathcurveto{\pgfqpoint{0.225000in}{0.348804in}}{\pgfqpoint{0.218854in}{0.363642in}}{\pgfqpoint{0.207915in}{0.374581in}}%
\pgfpathcurveto{\pgfqpoint{0.196975in}{0.385520in}}{\pgfqpoint{0.182137in}{0.391667in}}{\pgfqpoint{0.166667in}{0.391667in}}%
\pgfpathcurveto{\pgfqpoint{0.151196in}{0.391667in}}{\pgfqpoint{0.136358in}{0.385520in}}{\pgfqpoint{0.125419in}{0.374581in}}%
\pgfpathcurveto{\pgfqpoint{0.114480in}{0.363642in}}{\pgfqpoint{0.108333in}{0.348804in}}{\pgfqpoint{0.108333in}{0.333333in}}%
\pgfpathcurveto{\pgfqpoint{0.108333in}{0.317863in}}{\pgfqpoint{0.114480in}{0.303025in}}{\pgfqpoint{0.125419in}{0.292085in}}%
\pgfpathcurveto{\pgfqpoint{0.136358in}{0.281146in}}{\pgfqpoint{0.151196in}{0.275000in}}{\pgfqpoint{0.166667in}{0.275000in}}%
\pgfpathclose%
\pgfpathmoveto{\pgfqpoint{0.166667in}{0.280833in}}%
\pgfpathcurveto{\pgfqpoint{0.166667in}{0.280833in}}{\pgfqpoint{0.152744in}{0.280833in}}{\pgfqpoint{0.139389in}{0.286365in}}%
\pgfpathcurveto{\pgfqpoint{0.129544in}{0.296210in}}{\pgfqpoint{0.119698in}{0.306055in}}{\pgfqpoint{0.114167in}{0.319410in}}%
\pgfpathcurveto{\pgfqpoint{0.114167in}{0.333333in}}{\pgfqpoint{0.114167in}{0.347256in}}{\pgfqpoint{0.119698in}{0.360611in}}%
\pgfpathcurveto{\pgfqpoint{0.129544in}{0.370456in}}{\pgfqpoint{0.139389in}{0.380302in}}{\pgfqpoint{0.152744in}{0.385833in}}%
\pgfpathcurveto{\pgfqpoint{0.166667in}{0.385833in}}{\pgfqpoint{0.180590in}{0.385833in}}{\pgfqpoint{0.193945in}{0.380302in}}%
\pgfpathcurveto{\pgfqpoint{0.203790in}{0.370456in}}{\pgfqpoint{0.213635in}{0.360611in}}{\pgfqpoint{0.219167in}{0.347256in}}%
\pgfpathcurveto{\pgfqpoint{0.219167in}{0.333333in}}{\pgfqpoint{0.219167in}{0.319410in}}{\pgfqpoint{0.213635in}{0.306055in}}%
\pgfpathcurveto{\pgfqpoint{0.203790in}{0.296210in}}{\pgfqpoint{0.193945in}{0.286365in}}{\pgfqpoint{0.180590in}{0.280833in}}%
\pgfpathclose%
\pgfpathmoveto{\pgfqpoint{0.333333in}{0.275000in}}%
\pgfpathcurveto{\pgfqpoint{0.348804in}{0.275000in}}{\pgfqpoint{0.363642in}{0.281146in}}{\pgfqpoint{0.374581in}{0.292085in}}%
\pgfpathcurveto{\pgfqpoint{0.385520in}{0.303025in}}{\pgfqpoint{0.391667in}{0.317863in}}{\pgfqpoint{0.391667in}{0.333333in}}%
\pgfpathcurveto{\pgfqpoint{0.391667in}{0.348804in}}{\pgfqpoint{0.385520in}{0.363642in}}{\pgfqpoint{0.374581in}{0.374581in}}%
\pgfpathcurveto{\pgfqpoint{0.363642in}{0.385520in}}{\pgfqpoint{0.348804in}{0.391667in}}{\pgfqpoint{0.333333in}{0.391667in}}%
\pgfpathcurveto{\pgfqpoint{0.317863in}{0.391667in}}{\pgfqpoint{0.303025in}{0.385520in}}{\pgfqpoint{0.292085in}{0.374581in}}%
\pgfpathcurveto{\pgfqpoint{0.281146in}{0.363642in}}{\pgfqpoint{0.275000in}{0.348804in}}{\pgfqpoint{0.275000in}{0.333333in}}%
\pgfpathcurveto{\pgfqpoint{0.275000in}{0.317863in}}{\pgfqpoint{0.281146in}{0.303025in}}{\pgfqpoint{0.292085in}{0.292085in}}%
\pgfpathcurveto{\pgfqpoint{0.303025in}{0.281146in}}{\pgfqpoint{0.317863in}{0.275000in}}{\pgfqpoint{0.333333in}{0.275000in}}%
\pgfpathclose%
\pgfpathmoveto{\pgfqpoint{0.333333in}{0.280833in}}%
\pgfpathcurveto{\pgfqpoint{0.333333in}{0.280833in}}{\pgfqpoint{0.319410in}{0.280833in}}{\pgfqpoint{0.306055in}{0.286365in}}%
\pgfpathcurveto{\pgfqpoint{0.296210in}{0.296210in}}{\pgfqpoint{0.286365in}{0.306055in}}{\pgfqpoint{0.280833in}{0.319410in}}%
\pgfpathcurveto{\pgfqpoint{0.280833in}{0.333333in}}{\pgfqpoint{0.280833in}{0.347256in}}{\pgfqpoint{0.286365in}{0.360611in}}%
\pgfpathcurveto{\pgfqpoint{0.296210in}{0.370456in}}{\pgfqpoint{0.306055in}{0.380302in}}{\pgfqpoint{0.319410in}{0.385833in}}%
\pgfpathcurveto{\pgfqpoint{0.333333in}{0.385833in}}{\pgfqpoint{0.347256in}{0.385833in}}{\pgfqpoint{0.360611in}{0.380302in}}%
\pgfpathcurveto{\pgfqpoint{0.370456in}{0.370456in}}{\pgfqpoint{0.380302in}{0.360611in}}{\pgfqpoint{0.385833in}{0.347256in}}%
\pgfpathcurveto{\pgfqpoint{0.385833in}{0.333333in}}{\pgfqpoint{0.385833in}{0.319410in}}{\pgfqpoint{0.380302in}{0.306055in}}%
\pgfpathcurveto{\pgfqpoint{0.370456in}{0.296210in}}{\pgfqpoint{0.360611in}{0.286365in}}{\pgfqpoint{0.347256in}{0.280833in}}%
\pgfpathclose%
\pgfpathmoveto{\pgfqpoint{0.500000in}{0.275000in}}%
\pgfpathcurveto{\pgfqpoint{0.515470in}{0.275000in}}{\pgfqpoint{0.530309in}{0.281146in}}{\pgfqpoint{0.541248in}{0.292085in}}%
\pgfpathcurveto{\pgfqpoint{0.552187in}{0.303025in}}{\pgfqpoint{0.558333in}{0.317863in}}{\pgfqpoint{0.558333in}{0.333333in}}%
\pgfpathcurveto{\pgfqpoint{0.558333in}{0.348804in}}{\pgfqpoint{0.552187in}{0.363642in}}{\pgfqpoint{0.541248in}{0.374581in}}%
\pgfpathcurveto{\pgfqpoint{0.530309in}{0.385520in}}{\pgfqpoint{0.515470in}{0.391667in}}{\pgfqpoint{0.500000in}{0.391667in}}%
\pgfpathcurveto{\pgfqpoint{0.484530in}{0.391667in}}{\pgfqpoint{0.469691in}{0.385520in}}{\pgfqpoint{0.458752in}{0.374581in}}%
\pgfpathcurveto{\pgfqpoint{0.447813in}{0.363642in}}{\pgfqpoint{0.441667in}{0.348804in}}{\pgfqpoint{0.441667in}{0.333333in}}%
\pgfpathcurveto{\pgfqpoint{0.441667in}{0.317863in}}{\pgfqpoint{0.447813in}{0.303025in}}{\pgfqpoint{0.458752in}{0.292085in}}%
\pgfpathcurveto{\pgfqpoint{0.469691in}{0.281146in}}{\pgfqpoint{0.484530in}{0.275000in}}{\pgfqpoint{0.500000in}{0.275000in}}%
\pgfpathclose%
\pgfpathmoveto{\pgfqpoint{0.500000in}{0.280833in}}%
\pgfpathcurveto{\pgfqpoint{0.500000in}{0.280833in}}{\pgfqpoint{0.486077in}{0.280833in}}{\pgfqpoint{0.472722in}{0.286365in}}%
\pgfpathcurveto{\pgfqpoint{0.462877in}{0.296210in}}{\pgfqpoint{0.453032in}{0.306055in}}{\pgfqpoint{0.447500in}{0.319410in}}%
\pgfpathcurveto{\pgfqpoint{0.447500in}{0.333333in}}{\pgfqpoint{0.447500in}{0.347256in}}{\pgfqpoint{0.453032in}{0.360611in}}%
\pgfpathcurveto{\pgfqpoint{0.462877in}{0.370456in}}{\pgfqpoint{0.472722in}{0.380302in}}{\pgfqpoint{0.486077in}{0.385833in}}%
\pgfpathcurveto{\pgfqpoint{0.500000in}{0.385833in}}{\pgfqpoint{0.513923in}{0.385833in}}{\pgfqpoint{0.527278in}{0.380302in}}%
\pgfpathcurveto{\pgfqpoint{0.537123in}{0.370456in}}{\pgfqpoint{0.546968in}{0.360611in}}{\pgfqpoint{0.552500in}{0.347256in}}%
\pgfpathcurveto{\pgfqpoint{0.552500in}{0.333333in}}{\pgfqpoint{0.552500in}{0.319410in}}{\pgfqpoint{0.546968in}{0.306055in}}%
\pgfpathcurveto{\pgfqpoint{0.537123in}{0.296210in}}{\pgfqpoint{0.527278in}{0.286365in}}{\pgfqpoint{0.513923in}{0.280833in}}%
\pgfpathclose%
\pgfpathmoveto{\pgfqpoint{0.666667in}{0.275000in}}%
\pgfpathcurveto{\pgfqpoint{0.682137in}{0.275000in}}{\pgfqpoint{0.696975in}{0.281146in}}{\pgfqpoint{0.707915in}{0.292085in}}%
\pgfpathcurveto{\pgfqpoint{0.718854in}{0.303025in}}{\pgfqpoint{0.725000in}{0.317863in}}{\pgfqpoint{0.725000in}{0.333333in}}%
\pgfpathcurveto{\pgfqpoint{0.725000in}{0.348804in}}{\pgfqpoint{0.718854in}{0.363642in}}{\pgfqpoint{0.707915in}{0.374581in}}%
\pgfpathcurveto{\pgfqpoint{0.696975in}{0.385520in}}{\pgfqpoint{0.682137in}{0.391667in}}{\pgfqpoint{0.666667in}{0.391667in}}%
\pgfpathcurveto{\pgfqpoint{0.651196in}{0.391667in}}{\pgfqpoint{0.636358in}{0.385520in}}{\pgfqpoint{0.625419in}{0.374581in}}%
\pgfpathcurveto{\pgfqpoint{0.614480in}{0.363642in}}{\pgfqpoint{0.608333in}{0.348804in}}{\pgfqpoint{0.608333in}{0.333333in}}%
\pgfpathcurveto{\pgfqpoint{0.608333in}{0.317863in}}{\pgfqpoint{0.614480in}{0.303025in}}{\pgfqpoint{0.625419in}{0.292085in}}%
\pgfpathcurveto{\pgfqpoint{0.636358in}{0.281146in}}{\pgfqpoint{0.651196in}{0.275000in}}{\pgfqpoint{0.666667in}{0.275000in}}%
\pgfpathclose%
\pgfpathmoveto{\pgfqpoint{0.666667in}{0.280833in}}%
\pgfpathcurveto{\pgfqpoint{0.666667in}{0.280833in}}{\pgfqpoint{0.652744in}{0.280833in}}{\pgfqpoint{0.639389in}{0.286365in}}%
\pgfpathcurveto{\pgfqpoint{0.629544in}{0.296210in}}{\pgfqpoint{0.619698in}{0.306055in}}{\pgfqpoint{0.614167in}{0.319410in}}%
\pgfpathcurveto{\pgfqpoint{0.614167in}{0.333333in}}{\pgfqpoint{0.614167in}{0.347256in}}{\pgfqpoint{0.619698in}{0.360611in}}%
\pgfpathcurveto{\pgfqpoint{0.629544in}{0.370456in}}{\pgfqpoint{0.639389in}{0.380302in}}{\pgfqpoint{0.652744in}{0.385833in}}%
\pgfpathcurveto{\pgfqpoint{0.666667in}{0.385833in}}{\pgfqpoint{0.680590in}{0.385833in}}{\pgfqpoint{0.693945in}{0.380302in}}%
\pgfpathcurveto{\pgfqpoint{0.703790in}{0.370456in}}{\pgfqpoint{0.713635in}{0.360611in}}{\pgfqpoint{0.719167in}{0.347256in}}%
\pgfpathcurveto{\pgfqpoint{0.719167in}{0.333333in}}{\pgfqpoint{0.719167in}{0.319410in}}{\pgfqpoint{0.713635in}{0.306055in}}%
\pgfpathcurveto{\pgfqpoint{0.703790in}{0.296210in}}{\pgfqpoint{0.693945in}{0.286365in}}{\pgfqpoint{0.680590in}{0.280833in}}%
\pgfpathclose%
\pgfpathmoveto{\pgfqpoint{0.833333in}{0.275000in}}%
\pgfpathcurveto{\pgfqpoint{0.848804in}{0.275000in}}{\pgfqpoint{0.863642in}{0.281146in}}{\pgfqpoint{0.874581in}{0.292085in}}%
\pgfpathcurveto{\pgfqpoint{0.885520in}{0.303025in}}{\pgfqpoint{0.891667in}{0.317863in}}{\pgfqpoint{0.891667in}{0.333333in}}%
\pgfpathcurveto{\pgfqpoint{0.891667in}{0.348804in}}{\pgfqpoint{0.885520in}{0.363642in}}{\pgfqpoint{0.874581in}{0.374581in}}%
\pgfpathcurveto{\pgfqpoint{0.863642in}{0.385520in}}{\pgfqpoint{0.848804in}{0.391667in}}{\pgfqpoint{0.833333in}{0.391667in}}%
\pgfpathcurveto{\pgfqpoint{0.817863in}{0.391667in}}{\pgfqpoint{0.803025in}{0.385520in}}{\pgfqpoint{0.792085in}{0.374581in}}%
\pgfpathcurveto{\pgfqpoint{0.781146in}{0.363642in}}{\pgfqpoint{0.775000in}{0.348804in}}{\pgfqpoint{0.775000in}{0.333333in}}%
\pgfpathcurveto{\pgfqpoint{0.775000in}{0.317863in}}{\pgfqpoint{0.781146in}{0.303025in}}{\pgfqpoint{0.792085in}{0.292085in}}%
\pgfpathcurveto{\pgfqpoint{0.803025in}{0.281146in}}{\pgfqpoint{0.817863in}{0.275000in}}{\pgfqpoint{0.833333in}{0.275000in}}%
\pgfpathclose%
\pgfpathmoveto{\pgfqpoint{0.833333in}{0.280833in}}%
\pgfpathcurveto{\pgfqpoint{0.833333in}{0.280833in}}{\pgfqpoint{0.819410in}{0.280833in}}{\pgfqpoint{0.806055in}{0.286365in}}%
\pgfpathcurveto{\pgfqpoint{0.796210in}{0.296210in}}{\pgfqpoint{0.786365in}{0.306055in}}{\pgfqpoint{0.780833in}{0.319410in}}%
\pgfpathcurveto{\pgfqpoint{0.780833in}{0.333333in}}{\pgfqpoint{0.780833in}{0.347256in}}{\pgfqpoint{0.786365in}{0.360611in}}%
\pgfpathcurveto{\pgfqpoint{0.796210in}{0.370456in}}{\pgfqpoint{0.806055in}{0.380302in}}{\pgfqpoint{0.819410in}{0.385833in}}%
\pgfpathcurveto{\pgfqpoint{0.833333in}{0.385833in}}{\pgfqpoint{0.847256in}{0.385833in}}{\pgfqpoint{0.860611in}{0.380302in}}%
\pgfpathcurveto{\pgfqpoint{0.870456in}{0.370456in}}{\pgfqpoint{0.880302in}{0.360611in}}{\pgfqpoint{0.885833in}{0.347256in}}%
\pgfpathcurveto{\pgfqpoint{0.885833in}{0.333333in}}{\pgfqpoint{0.885833in}{0.319410in}}{\pgfqpoint{0.880302in}{0.306055in}}%
\pgfpathcurveto{\pgfqpoint{0.870456in}{0.296210in}}{\pgfqpoint{0.860611in}{0.286365in}}{\pgfqpoint{0.847256in}{0.280833in}}%
\pgfpathclose%
\pgfpathmoveto{\pgfqpoint{1.000000in}{0.275000in}}%
\pgfpathcurveto{\pgfqpoint{1.015470in}{0.275000in}}{\pgfqpoint{1.030309in}{0.281146in}}{\pgfqpoint{1.041248in}{0.292085in}}%
\pgfpathcurveto{\pgfqpoint{1.052187in}{0.303025in}}{\pgfqpoint{1.058333in}{0.317863in}}{\pgfqpoint{1.058333in}{0.333333in}}%
\pgfpathcurveto{\pgfqpoint{1.058333in}{0.348804in}}{\pgfqpoint{1.052187in}{0.363642in}}{\pgfqpoint{1.041248in}{0.374581in}}%
\pgfpathcurveto{\pgfqpoint{1.030309in}{0.385520in}}{\pgfqpoint{1.015470in}{0.391667in}}{\pgfqpoint{1.000000in}{0.391667in}}%
\pgfpathcurveto{\pgfqpoint{0.984530in}{0.391667in}}{\pgfqpoint{0.969691in}{0.385520in}}{\pgfqpoint{0.958752in}{0.374581in}}%
\pgfpathcurveto{\pgfqpoint{0.947813in}{0.363642in}}{\pgfqpoint{0.941667in}{0.348804in}}{\pgfqpoint{0.941667in}{0.333333in}}%
\pgfpathcurveto{\pgfqpoint{0.941667in}{0.317863in}}{\pgfqpoint{0.947813in}{0.303025in}}{\pgfqpoint{0.958752in}{0.292085in}}%
\pgfpathcurveto{\pgfqpoint{0.969691in}{0.281146in}}{\pgfqpoint{0.984530in}{0.275000in}}{\pgfqpoint{1.000000in}{0.275000in}}%
\pgfpathclose%
\pgfpathmoveto{\pgfqpoint{1.000000in}{0.280833in}}%
\pgfpathcurveto{\pgfqpoint{1.000000in}{0.280833in}}{\pgfqpoint{0.986077in}{0.280833in}}{\pgfqpoint{0.972722in}{0.286365in}}%
\pgfpathcurveto{\pgfqpoint{0.962877in}{0.296210in}}{\pgfqpoint{0.953032in}{0.306055in}}{\pgfqpoint{0.947500in}{0.319410in}}%
\pgfpathcurveto{\pgfqpoint{0.947500in}{0.333333in}}{\pgfqpoint{0.947500in}{0.347256in}}{\pgfqpoint{0.953032in}{0.360611in}}%
\pgfpathcurveto{\pgfqpoint{0.962877in}{0.370456in}}{\pgfqpoint{0.972722in}{0.380302in}}{\pgfqpoint{0.986077in}{0.385833in}}%
\pgfpathcurveto{\pgfqpoint{1.000000in}{0.385833in}}{\pgfqpoint{1.013923in}{0.385833in}}{\pgfqpoint{1.027278in}{0.380302in}}%
\pgfpathcurveto{\pgfqpoint{1.037123in}{0.370456in}}{\pgfqpoint{1.046968in}{0.360611in}}{\pgfqpoint{1.052500in}{0.347256in}}%
\pgfpathcurveto{\pgfqpoint{1.052500in}{0.333333in}}{\pgfqpoint{1.052500in}{0.319410in}}{\pgfqpoint{1.046968in}{0.306055in}}%
\pgfpathcurveto{\pgfqpoint{1.037123in}{0.296210in}}{\pgfqpoint{1.027278in}{0.286365in}}{\pgfqpoint{1.013923in}{0.280833in}}%
\pgfpathclose%
\pgfpathmoveto{\pgfqpoint{0.083333in}{0.441667in}}%
\pgfpathcurveto{\pgfqpoint{0.098804in}{0.441667in}}{\pgfqpoint{0.113642in}{0.447813in}}{\pgfqpoint{0.124581in}{0.458752in}}%
\pgfpathcurveto{\pgfqpoint{0.135520in}{0.469691in}}{\pgfqpoint{0.141667in}{0.484530in}}{\pgfqpoint{0.141667in}{0.500000in}}%
\pgfpathcurveto{\pgfqpoint{0.141667in}{0.515470in}}{\pgfqpoint{0.135520in}{0.530309in}}{\pgfqpoint{0.124581in}{0.541248in}}%
\pgfpathcurveto{\pgfqpoint{0.113642in}{0.552187in}}{\pgfqpoint{0.098804in}{0.558333in}}{\pgfqpoint{0.083333in}{0.558333in}}%
\pgfpathcurveto{\pgfqpoint{0.067863in}{0.558333in}}{\pgfqpoint{0.053025in}{0.552187in}}{\pgfqpoint{0.042085in}{0.541248in}}%
\pgfpathcurveto{\pgfqpoint{0.031146in}{0.530309in}}{\pgfqpoint{0.025000in}{0.515470in}}{\pgfqpoint{0.025000in}{0.500000in}}%
\pgfpathcurveto{\pgfqpoint{0.025000in}{0.484530in}}{\pgfqpoint{0.031146in}{0.469691in}}{\pgfqpoint{0.042085in}{0.458752in}}%
\pgfpathcurveto{\pgfqpoint{0.053025in}{0.447813in}}{\pgfqpoint{0.067863in}{0.441667in}}{\pgfqpoint{0.083333in}{0.441667in}}%
\pgfpathclose%
\pgfpathmoveto{\pgfqpoint{0.083333in}{0.447500in}}%
\pgfpathcurveto{\pgfqpoint{0.083333in}{0.447500in}}{\pgfqpoint{0.069410in}{0.447500in}}{\pgfqpoint{0.056055in}{0.453032in}}%
\pgfpathcurveto{\pgfqpoint{0.046210in}{0.462877in}}{\pgfqpoint{0.036365in}{0.472722in}}{\pgfqpoint{0.030833in}{0.486077in}}%
\pgfpathcurveto{\pgfqpoint{0.030833in}{0.500000in}}{\pgfqpoint{0.030833in}{0.513923in}}{\pgfqpoint{0.036365in}{0.527278in}}%
\pgfpathcurveto{\pgfqpoint{0.046210in}{0.537123in}}{\pgfqpoint{0.056055in}{0.546968in}}{\pgfqpoint{0.069410in}{0.552500in}}%
\pgfpathcurveto{\pgfqpoint{0.083333in}{0.552500in}}{\pgfqpoint{0.097256in}{0.552500in}}{\pgfqpoint{0.110611in}{0.546968in}}%
\pgfpathcurveto{\pgfqpoint{0.120456in}{0.537123in}}{\pgfqpoint{0.130302in}{0.527278in}}{\pgfqpoint{0.135833in}{0.513923in}}%
\pgfpathcurveto{\pgfqpoint{0.135833in}{0.500000in}}{\pgfqpoint{0.135833in}{0.486077in}}{\pgfqpoint{0.130302in}{0.472722in}}%
\pgfpathcurveto{\pgfqpoint{0.120456in}{0.462877in}}{\pgfqpoint{0.110611in}{0.453032in}}{\pgfqpoint{0.097256in}{0.447500in}}%
\pgfpathclose%
\pgfpathmoveto{\pgfqpoint{0.250000in}{0.441667in}}%
\pgfpathcurveto{\pgfqpoint{0.265470in}{0.441667in}}{\pgfqpoint{0.280309in}{0.447813in}}{\pgfqpoint{0.291248in}{0.458752in}}%
\pgfpathcurveto{\pgfqpoint{0.302187in}{0.469691in}}{\pgfqpoint{0.308333in}{0.484530in}}{\pgfqpoint{0.308333in}{0.500000in}}%
\pgfpathcurveto{\pgfqpoint{0.308333in}{0.515470in}}{\pgfqpoint{0.302187in}{0.530309in}}{\pgfqpoint{0.291248in}{0.541248in}}%
\pgfpathcurveto{\pgfqpoint{0.280309in}{0.552187in}}{\pgfqpoint{0.265470in}{0.558333in}}{\pgfqpoint{0.250000in}{0.558333in}}%
\pgfpathcurveto{\pgfqpoint{0.234530in}{0.558333in}}{\pgfqpoint{0.219691in}{0.552187in}}{\pgfqpoint{0.208752in}{0.541248in}}%
\pgfpathcurveto{\pgfqpoint{0.197813in}{0.530309in}}{\pgfqpoint{0.191667in}{0.515470in}}{\pgfqpoint{0.191667in}{0.500000in}}%
\pgfpathcurveto{\pgfqpoint{0.191667in}{0.484530in}}{\pgfqpoint{0.197813in}{0.469691in}}{\pgfqpoint{0.208752in}{0.458752in}}%
\pgfpathcurveto{\pgfqpoint{0.219691in}{0.447813in}}{\pgfqpoint{0.234530in}{0.441667in}}{\pgfqpoint{0.250000in}{0.441667in}}%
\pgfpathclose%
\pgfpathmoveto{\pgfqpoint{0.250000in}{0.447500in}}%
\pgfpathcurveto{\pgfqpoint{0.250000in}{0.447500in}}{\pgfqpoint{0.236077in}{0.447500in}}{\pgfqpoint{0.222722in}{0.453032in}}%
\pgfpathcurveto{\pgfqpoint{0.212877in}{0.462877in}}{\pgfqpoint{0.203032in}{0.472722in}}{\pgfqpoint{0.197500in}{0.486077in}}%
\pgfpathcurveto{\pgfqpoint{0.197500in}{0.500000in}}{\pgfqpoint{0.197500in}{0.513923in}}{\pgfqpoint{0.203032in}{0.527278in}}%
\pgfpathcurveto{\pgfqpoint{0.212877in}{0.537123in}}{\pgfqpoint{0.222722in}{0.546968in}}{\pgfqpoint{0.236077in}{0.552500in}}%
\pgfpathcurveto{\pgfqpoint{0.250000in}{0.552500in}}{\pgfqpoint{0.263923in}{0.552500in}}{\pgfqpoint{0.277278in}{0.546968in}}%
\pgfpathcurveto{\pgfqpoint{0.287123in}{0.537123in}}{\pgfqpoint{0.296968in}{0.527278in}}{\pgfqpoint{0.302500in}{0.513923in}}%
\pgfpathcurveto{\pgfqpoint{0.302500in}{0.500000in}}{\pgfqpoint{0.302500in}{0.486077in}}{\pgfqpoint{0.296968in}{0.472722in}}%
\pgfpathcurveto{\pgfqpoint{0.287123in}{0.462877in}}{\pgfqpoint{0.277278in}{0.453032in}}{\pgfqpoint{0.263923in}{0.447500in}}%
\pgfpathclose%
\pgfpathmoveto{\pgfqpoint{0.416667in}{0.441667in}}%
\pgfpathcurveto{\pgfqpoint{0.432137in}{0.441667in}}{\pgfqpoint{0.446975in}{0.447813in}}{\pgfqpoint{0.457915in}{0.458752in}}%
\pgfpathcurveto{\pgfqpoint{0.468854in}{0.469691in}}{\pgfqpoint{0.475000in}{0.484530in}}{\pgfqpoint{0.475000in}{0.500000in}}%
\pgfpathcurveto{\pgfqpoint{0.475000in}{0.515470in}}{\pgfqpoint{0.468854in}{0.530309in}}{\pgfqpoint{0.457915in}{0.541248in}}%
\pgfpathcurveto{\pgfqpoint{0.446975in}{0.552187in}}{\pgfqpoint{0.432137in}{0.558333in}}{\pgfqpoint{0.416667in}{0.558333in}}%
\pgfpathcurveto{\pgfqpoint{0.401196in}{0.558333in}}{\pgfqpoint{0.386358in}{0.552187in}}{\pgfqpoint{0.375419in}{0.541248in}}%
\pgfpathcurveto{\pgfqpoint{0.364480in}{0.530309in}}{\pgfqpoint{0.358333in}{0.515470in}}{\pgfqpoint{0.358333in}{0.500000in}}%
\pgfpathcurveto{\pgfqpoint{0.358333in}{0.484530in}}{\pgfqpoint{0.364480in}{0.469691in}}{\pgfqpoint{0.375419in}{0.458752in}}%
\pgfpathcurveto{\pgfqpoint{0.386358in}{0.447813in}}{\pgfqpoint{0.401196in}{0.441667in}}{\pgfqpoint{0.416667in}{0.441667in}}%
\pgfpathclose%
\pgfpathmoveto{\pgfqpoint{0.416667in}{0.447500in}}%
\pgfpathcurveto{\pgfqpoint{0.416667in}{0.447500in}}{\pgfqpoint{0.402744in}{0.447500in}}{\pgfqpoint{0.389389in}{0.453032in}}%
\pgfpathcurveto{\pgfqpoint{0.379544in}{0.462877in}}{\pgfqpoint{0.369698in}{0.472722in}}{\pgfqpoint{0.364167in}{0.486077in}}%
\pgfpathcurveto{\pgfqpoint{0.364167in}{0.500000in}}{\pgfqpoint{0.364167in}{0.513923in}}{\pgfqpoint{0.369698in}{0.527278in}}%
\pgfpathcurveto{\pgfqpoint{0.379544in}{0.537123in}}{\pgfqpoint{0.389389in}{0.546968in}}{\pgfqpoint{0.402744in}{0.552500in}}%
\pgfpathcurveto{\pgfqpoint{0.416667in}{0.552500in}}{\pgfqpoint{0.430590in}{0.552500in}}{\pgfqpoint{0.443945in}{0.546968in}}%
\pgfpathcurveto{\pgfqpoint{0.453790in}{0.537123in}}{\pgfqpoint{0.463635in}{0.527278in}}{\pgfqpoint{0.469167in}{0.513923in}}%
\pgfpathcurveto{\pgfqpoint{0.469167in}{0.500000in}}{\pgfqpoint{0.469167in}{0.486077in}}{\pgfqpoint{0.463635in}{0.472722in}}%
\pgfpathcurveto{\pgfqpoint{0.453790in}{0.462877in}}{\pgfqpoint{0.443945in}{0.453032in}}{\pgfqpoint{0.430590in}{0.447500in}}%
\pgfpathclose%
\pgfpathmoveto{\pgfqpoint{0.583333in}{0.441667in}}%
\pgfpathcurveto{\pgfqpoint{0.598804in}{0.441667in}}{\pgfqpoint{0.613642in}{0.447813in}}{\pgfqpoint{0.624581in}{0.458752in}}%
\pgfpathcurveto{\pgfqpoint{0.635520in}{0.469691in}}{\pgfqpoint{0.641667in}{0.484530in}}{\pgfqpoint{0.641667in}{0.500000in}}%
\pgfpathcurveto{\pgfqpoint{0.641667in}{0.515470in}}{\pgfqpoint{0.635520in}{0.530309in}}{\pgfqpoint{0.624581in}{0.541248in}}%
\pgfpathcurveto{\pgfqpoint{0.613642in}{0.552187in}}{\pgfqpoint{0.598804in}{0.558333in}}{\pgfqpoint{0.583333in}{0.558333in}}%
\pgfpathcurveto{\pgfqpoint{0.567863in}{0.558333in}}{\pgfqpoint{0.553025in}{0.552187in}}{\pgfqpoint{0.542085in}{0.541248in}}%
\pgfpathcurveto{\pgfqpoint{0.531146in}{0.530309in}}{\pgfqpoint{0.525000in}{0.515470in}}{\pgfqpoint{0.525000in}{0.500000in}}%
\pgfpathcurveto{\pgfqpoint{0.525000in}{0.484530in}}{\pgfqpoint{0.531146in}{0.469691in}}{\pgfqpoint{0.542085in}{0.458752in}}%
\pgfpathcurveto{\pgfqpoint{0.553025in}{0.447813in}}{\pgfqpoint{0.567863in}{0.441667in}}{\pgfqpoint{0.583333in}{0.441667in}}%
\pgfpathclose%
\pgfpathmoveto{\pgfqpoint{0.583333in}{0.447500in}}%
\pgfpathcurveto{\pgfqpoint{0.583333in}{0.447500in}}{\pgfqpoint{0.569410in}{0.447500in}}{\pgfqpoint{0.556055in}{0.453032in}}%
\pgfpathcurveto{\pgfqpoint{0.546210in}{0.462877in}}{\pgfqpoint{0.536365in}{0.472722in}}{\pgfqpoint{0.530833in}{0.486077in}}%
\pgfpathcurveto{\pgfqpoint{0.530833in}{0.500000in}}{\pgfqpoint{0.530833in}{0.513923in}}{\pgfqpoint{0.536365in}{0.527278in}}%
\pgfpathcurveto{\pgfqpoint{0.546210in}{0.537123in}}{\pgfqpoint{0.556055in}{0.546968in}}{\pgfqpoint{0.569410in}{0.552500in}}%
\pgfpathcurveto{\pgfqpoint{0.583333in}{0.552500in}}{\pgfqpoint{0.597256in}{0.552500in}}{\pgfqpoint{0.610611in}{0.546968in}}%
\pgfpathcurveto{\pgfqpoint{0.620456in}{0.537123in}}{\pgfqpoint{0.630302in}{0.527278in}}{\pgfqpoint{0.635833in}{0.513923in}}%
\pgfpathcurveto{\pgfqpoint{0.635833in}{0.500000in}}{\pgfqpoint{0.635833in}{0.486077in}}{\pgfqpoint{0.630302in}{0.472722in}}%
\pgfpathcurveto{\pgfqpoint{0.620456in}{0.462877in}}{\pgfqpoint{0.610611in}{0.453032in}}{\pgfqpoint{0.597256in}{0.447500in}}%
\pgfpathclose%
\pgfpathmoveto{\pgfqpoint{0.750000in}{0.441667in}}%
\pgfpathcurveto{\pgfqpoint{0.765470in}{0.441667in}}{\pgfqpoint{0.780309in}{0.447813in}}{\pgfqpoint{0.791248in}{0.458752in}}%
\pgfpathcurveto{\pgfqpoint{0.802187in}{0.469691in}}{\pgfqpoint{0.808333in}{0.484530in}}{\pgfqpoint{0.808333in}{0.500000in}}%
\pgfpathcurveto{\pgfqpoint{0.808333in}{0.515470in}}{\pgfqpoint{0.802187in}{0.530309in}}{\pgfqpoint{0.791248in}{0.541248in}}%
\pgfpathcurveto{\pgfqpoint{0.780309in}{0.552187in}}{\pgfqpoint{0.765470in}{0.558333in}}{\pgfqpoint{0.750000in}{0.558333in}}%
\pgfpathcurveto{\pgfqpoint{0.734530in}{0.558333in}}{\pgfqpoint{0.719691in}{0.552187in}}{\pgfqpoint{0.708752in}{0.541248in}}%
\pgfpathcurveto{\pgfqpoint{0.697813in}{0.530309in}}{\pgfqpoint{0.691667in}{0.515470in}}{\pgfqpoint{0.691667in}{0.500000in}}%
\pgfpathcurveto{\pgfqpoint{0.691667in}{0.484530in}}{\pgfqpoint{0.697813in}{0.469691in}}{\pgfqpoint{0.708752in}{0.458752in}}%
\pgfpathcurveto{\pgfqpoint{0.719691in}{0.447813in}}{\pgfqpoint{0.734530in}{0.441667in}}{\pgfqpoint{0.750000in}{0.441667in}}%
\pgfpathclose%
\pgfpathmoveto{\pgfqpoint{0.750000in}{0.447500in}}%
\pgfpathcurveto{\pgfqpoint{0.750000in}{0.447500in}}{\pgfqpoint{0.736077in}{0.447500in}}{\pgfqpoint{0.722722in}{0.453032in}}%
\pgfpathcurveto{\pgfqpoint{0.712877in}{0.462877in}}{\pgfqpoint{0.703032in}{0.472722in}}{\pgfqpoint{0.697500in}{0.486077in}}%
\pgfpathcurveto{\pgfqpoint{0.697500in}{0.500000in}}{\pgfqpoint{0.697500in}{0.513923in}}{\pgfqpoint{0.703032in}{0.527278in}}%
\pgfpathcurveto{\pgfqpoint{0.712877in}{0.537123in}}{\pgfqpoint{0.722722in}{0.546968in}}{\pgfqpoint{0.736077in}{0.552500in}}%
\pgfpathcurveto{\pgfqpoint{0.750000in}{0.552500in}}{\pgfqpoint{0.763923in}{0.552500in}}{\pgfqpoint{0.777278in}{0.546968in}}%
\pgfpathcurveto{\pgfqpoint{0.787123in}{0.537123in}}{\pgfqpoint{0.796968in}{0.527278in}}{\pgfqpoint{0.802500in}{0.513923in}}%
\pgfpathcurveto{\pgfqpoint{0.802500in}{0.500000in}}{\pgfqpoint{0.802500in}{0.486077in}}{\pgfqpoint{0.796968in}{0.472722in}}%
\pgfpathcurveto{\pgfqpoint{0.787123in}{0.462877in}}{\pgfqpoint{0.777278in}{0.453032in}}{\pgfqpoint{0.763923in}{0.447500in}}%
\pgfpathclose%
\pgfpathmoveto{\pgfqpoint{0.916667in}{0.441667in}}%
\pgfpathcurveto{\pgfqpoint{0.932137in}{0.441667in}}{\pgfqpoint{0.946975in}{0.447813in}}{\pgfqpoint{0.957915in}{0.458752in}}%
\pgfpathcurveto{\pgfqpoint{0.968854in}{0.469691in}}{\pgfqpoint{0.975000in}{0.484530in}}{\pgfqpoint{0.975000in}{0.500000in}}%
\pgfpathcurveto{\pgfqpoint{0.975000in}{0.515470in}}{\pgfqpoint{0.968854in}{0.530309in}}{\pgfqpoint{0.957915in}{0.541248in}}%
\pgfpathcurveto{\pgfqpoint{0.946975in}{0.552187in}}{\pgfqpoint{0.932137in}{0.558333in}}{\pgfqpoint{0.916667in}{0.558333in}}%
\pgfpathcurveto{\pgfqpoint{0.901196in}{0.558333in}}{\pgfqpoint{0.886358in}{0.552187in}}{\pgfqpoint{0.875419in}{0.541248in}}%
\pgfpathcurveto{\pgfqpoint{0.864480in}{0.530309in}}{\pgfqpoint{0.858333in}{0.515470in}}{\pgfqpoint{0.858333in}{0.500000in}}%
\pgfpathcurveto{\pgfqpoint{0.858333in}{0.484530in}}{\pgfqpoint{0.864480in}{0.469691in}}{\pgfqpoint{0.875419in}{0.458752in}}%
\pgfpathcurveto{\pgfqpoint{0.886358in}{0.447813in}}{\pgfqpoint{0.901196in}{0.441667in}}{\pgfqpoint{0.916667in}{0.441667in}}%
\pgfpathclose%
\pgfpathmoveto{\pgfqpoint{0.916667in}{0.447500in}}%
\pgfpathcurveto{\pgfqpoint{0.916667in}{0.447500in}}{\pgfqpoint{0.902744in}{0.447500in}}{\pgfqpoint{0.889389in}{0.453032in}}%
\pgfpathcurveto{\pgfqpoint{0.879544in}{0.462877in}}{\pgfqpoint{0.869698in}{0.472722in}}{\pgfqpoint{0.864167in}{0.486077in}}%
\pgfpathcurveto{\pgfqpoint{0.864167in}{0.500000in}}{\pgfqpoint{0.864167in}{0.513923in}}{\pgfqpoint{0.869698in}{0.527278in}}%
\pgfpathcurveto{\pgfqpoint{0.879544in}{0.537123in}}{\pgfqpoint{0.889389in}{0.546968in}}{\pgfqpoint{0.902744in}{0.552500in}}%
\pgfpathcurveto{\pgfqpoint{0.916667in}{0.552500in}}{\pgfqpoint{0.930590in}{0.552500in}}{\pgfqpoint{0.943945in}{0.546968in}}%
\pgfpathcurveto{\pgfqpoint{0.953790in}{0.537123in}}{\pgfqpoint{0.963635in}{0.527278in}}{\pgfqpoint{0.969167in}{0.513923in}}%
\pgfpathcurveto{\pgfqpoint{0.969167in}{0.500000in}}{\pgfqpoint{0.969167in}{0.486077in}}{\pgfqpoint{0.963635in}{0.472722in}}%
\pgfpathcurveto{\pgfqpoint{0.953790in}{0.462877in}}{\pgfqpoint{0.943945in}{0.453032in}}{\pgfqpoint{0.930590in}{0.447500in}}%
\pgfpathclose%
\pgfpathmoveto{\pgfqpoint{0.000000in}{0.608333in}}%
\pgfpathcurveto{\pgfqpoint{0.015470in}{0.608333in}}{\pgfqpoint{0.030309in}{0.614480in}}{\pgfqpoint{0.041248in}{0.625419in}}%
\pgfpathcurveto{\pgfqpoint{0.052187in}{0.636358in}}{\pgfqpoint{0.058333in}{0.651196in}}{\pgfqpoint{0.058333in}{0.666667in}}%
\pgfpathcurveto{\pgfqpoint{0.058333in}{0.682137in}}{\pgfqpoint{0.052187in}{0.696975in}}{\pgfqpoint{0.041248in}{0.707915in}}%
\pgfpathcurveto{\pgfqpoint{0.030309in}{0.718854in}}{\pgfqpoint{0.015470in}{0.725000in}}{\pgfqpoint{0.000000in}{0.725000in}}%
\pgfpathcurveto{\pgfqpoint{-0.015470in}{0.725000in}}{\pgfqpoint{-0.030309in}{0.718854in}}{\pgfqpoint{-0.041248in}{0.707915in}}%
\pgfpathcurveto{\pgfqpoint{-0.052187in}{0.696975in}}{\pgfqpoint{-0.058333in}{0.682137in}}{\pgfqpoint{-0.058333in}{0.666667in}}%
\pgfpathcurveto{\pgfqpoint{-0.058333in}{0.651196in}}{\pgfqpoint{-0.052187in}{0.636358in}}{\pgfqpoint{-0.041248in}{0.625419in}}%
\pgfpathcurveto{\pgfqpoint{-0.030309in}{0.614480in}}{\pgfqpoint{-0.015470in}{0.608333in}}{\pgfqpoint{0.000000in}{0.608333in}}%
\pgfpathclose%
\pgfpathmoveto{\pgfqpoint{0.000000in}{0.614167in}}%
\pgfpathcurveto{\pgfqpoint{0.000000in}{0.614167in}}{\pgfqpoint{-0.013923in}{0.614167in}}{\pgfqpoint{-0.027278in}{0.619698in}}%
\pgfpathcurveto{\pgfqpoint{-0.037123in}{0.629544in}}{\pgfqpoint{-0.046968in}{0.639389in}}{\pgfqpoint{-0.052500in}{0.652744in}}%
\pgfpathcurveto{\pgfqpoint{-0.052500in}{0.666667in}}{\pgfqpoint{-0.052500in}{0.680590in}}{\pgfqpoint{-0.046968in}{0.693945in}}%
\pgfpathcurveto{\pgfqpoint{-0.037123in}{0.703790in}}{\pgfqpoint{-0.027278in}{0.713635in}}{\pgfqpoint{-0.013923in}{0.719167in}}%
\pgfpathcurveto{\pgfqpoint{0.000000in}{0.719167in}}{\pgfqpoint{0.013923in}{0.719167in}}{\pgfqpoint{0.027278in}{0.713635in}}%
\pgfpathcurveto{\pgfqpoint{0.037123in}{0.703790in}}{\pgfqpoint{0.046968in}{0.693945in}}{\pgfqpoint{0.052500in}{0.680590in}}%
\pgfpathcurveto{\pgfqpoint{0.052500in}{0.666667in}}{\pgfqpoint{0.052500in}{0.652744in}}{\pgfqpoint{0.046968in}{0.639389in}}%
\pgfpathcurveto{\pgfqpoint{0.037123in}{0.629544in}}{\pgfqpoint{0.027278in}{0.619698in}}{\pgfqpoint{0.013923in}{0.614167in}}%
\pgfpathclose%
\pgfpathmoveto{\pgfqpoint{0.166667in}{0.608333in}}%
\pgfpathcurveto{\pgfqpoint{0.182137in}{0.608333in}}{\pgfqpoint{0.196975in}{0.614480in}}{\pgfqpoint{0.207915in}{0.625419in}}%
\pgfpathcurveto{\pgfqpoint{0.218854in}{0.636358in}}{\pgfqpoint{0.225000in}{0.651196in}}{\pgfqpoint{0.225000in}{0.666667in}}%
\pgfpathcurveto{\pgfqpoint{0.225000in}{0.682137in}}{\pgfqpoint{0.218854in}{0.696975in}}{\pgfqpoint{0.207915in}{0.707915in}}%
\pgfpathcurveto{\pgfqpoint{0.196975in}{0.718854in}}{\pgfqpoint{0.182137in}{0.725000in}}{\pgfqpoint{0.166667in}{0.725000in}}%
\pgfpathcurveto{\pgfqpoint{0.151196in}{0.725000in}}{\pgfqpoint{0.136358in}{0.718854in}}{\pgfqpoint{0.125419in}{0.707915in}}%
\pgfpathcurveto{\pgfqpoint{0.114480in}{0.696975in}}{\pgfqpoint{0.108333in}{0.682137in}}{\pgfqpoint{0.108333in}{0.666667in}}%
\pgfpathcurveto{\pgfqpoint{0.108333in}{0.651196in}}{\pgfqpoint{0.114480in}{0.636358in}}{\pgfqpoint{0.125419in}{0.625419in}}%
\pgfpathcurveto{\pgfqpoint{0.136358in}{0.614480in}}{\pgfqpoint{0.151196in}{0.608333in}}{\pgfqpoint{0.166667in}{0.608333in}}%
\pgfpathclose%
\pgfpathmoveto{\pgfqpoint{0.166667in}{0.614167in}}%
\pgfpathcurveto{\pgfqpoint{0.166667in}{0.614167in}}{\pgfqpoint{0.152744in}{0.614167in}}{\pgfqpoint{0.139389in}{0.619698in}}%
\pgfpathcurveto{\pgfqpoint{0.129544in}{0.629544in}}{\pgfqpoint{0.119698in}{0.639389in}}{\pgfqpoint{0.114167in}{0.652744in}}%
\pgfpathcurveto{\pgfqpoint{0.114167in}{0.666667in}}{\pgfqpoint{0.114167in}{0.680590in}}{\pgfqpoint{0.119698in}{0.693945in}}%
\pgfpathcurveto{\pgfqpoint{0.129544in}{0.703790in}}{\pgfqpoint{0.139389in}{0.713635in}}{\pgfqpoint{0.152744in}{0.719167in}}%
\pgfpathcurveto{\pgfqpoint{0.166667in}{0.719167in}}{\pgfqpoint{0.180590in}{0.719167in}}{\pgfqpoint{0.193945in}{0.713635in}}%
\pgfpathcurveto{\pgfqpoint{0.203790in}{0.703790in}}{\pgfqpoint{0.213635in}{0.693945in}}{\pgfqpoint{0.219167in}{0.680590in}}%
\pgfpathcurveto{\pgfqpoint{0.219167in}{0.666667in}}{\pgfqpoint{0.219167in}{0.652744in}}{\pgfqpoint{0.213635in}{0.639389in}}%
\pgfpathcurveto{\pgfqpoint{0.203790in}{0.629544in}}{\pgfqpoint{0.193945in}{0.619698in}}{\pgfqpoint{0.180590in}{0.614167in}}%
\pgfpathclose%
\pgfpathmoveto{\pgfqpoint{0.333333in}{0.608333in}}%
\pgfpathcurveto{\pgfqpoint{0.348804in}{0.608333in}}{\pgfqpoint{0.363642in}{0.614480in}}{\pgfqpoint{0.374581in}{0.625419in}}%
\pgfpathcurveto{\pgfqpoint{0.385520in}{0.636358in}}{\pgfqpoint{0.391667in}{0.651196in}}{\pgfqpoint{0.391667in}{0.666667in}}%
\pgfpathcurveto{\pgfqpoint{0.391667in}{0.682137in}}{\pgfqpoint{0.385520in}{0.696975in}}{\pgfqpoint{0.374581in}{0.707915in}}%
\pgfpathcurveto{\pgfqpoint{0.363642in}{0.718854in}}{\pgfqpoint{0.348804in}{0.725000in}}{\pgfqpoint{0.333333in}{0.725000in}}%
\pgfpathcurveto{\pgfqpoint{0.317863in}{0.725000in}}{\pgfqpoint{0.303025in}{0.718854in}}{\pgfqpoint{0.292085in}{0.707915in}}%
\pgfpathcurveto{\pgfqpoint{0.281146in}{0.696975in}}{\pgfqpoint{0.275000in}{0.682137in}}{\pgfqpoint{0.275000in}{0.666667in}}%
\pgfpathcurveto{\pgfqpoint{0.275000in}{0.651196in}}{\pgfqpoint{0.281146in}{0.636358in}}{\pgfqpoint{0.292085in}{0.625419in}}%
\pgfpathcurveto{\pgfqpoint{0.303025in}{0.614480in}}{\pgfqpoint{0.317863in}{0.608333in}}{\pgfqpoint{0.333333in}{0.608333in}}%
\pgfpathclose%
\pgfpathmoveto{\pgfqpoint{0.333333in}{0.614167in}}%
\pgfpathcurveto{\pgfqpoint{0.333333in}{0.614167in}}{\pgfqpoint{0.319410in}{0.614167in}}{\pgfqpoint{0.306055in}{0.619698in}}%
\pgfpathcurveto{\pgfqpoint{0.296210in}{0.629544in}}{\pgfqpoint{0.286365in}{0.639389in}}{\pgfqpoint{0.280833in}{0.652744in}}%
\pgfpathcurveto{\pgfqpoint{0.280833in}{0.666667in}}{\pgfqpoint{0.280833in}{0.680590in}}{\pgfqpoint{0.286365in}{0.693945in}}%
\pgfpathcurveto{\pgfqpoint{0.296210in}{0.703790in}}{\pgfqpoint{0.306055in}{0.713635in}}{\pgfqpoint{0.319410in}{0.719167in}}%
\pgfpathcurveto{\pgfqpoint{0.333333in}{0.719167in}}{\pgfqpoint{0.347256in}{0.719167in}}{\pgfqpoint{0.360611in}{0.713635in}}%
\pgfpathcurveto{\pgfqpoint{0.370456in}{0.703790in}}{\pgfqpoint{0.380302in}{0.693945in}}{\pgfqpoint{0.385833in}{0.680590in}}%
\pgfpathcurveto{\pgfqpoint{0.385833in}{0.666667in}}{\pgfqpoint{0.385833in}{0.652744in}}{\pgfqpoint{0.380302in}{0.639389in}}%
\pgfpathcurveto{\pgfqpoint{0.370456in}{0.629544in}}{\pgfqpoint{0.360611in}{0.619698in}}{\pgfqpoint{0.347256in}{0.614167in}}%
\pgfpathclose%
\pgfpathmoveto{\pgfqpoint{0.500000in}{0.608333in}}%
\pgfpathcurveto{\pgfqpoint{0.515470in}{0.608333in}}{\pgfqpoint{0.530309in}{0.614480in}}{\pgfqpoint{0.541248in}{0.625419in}}%
\pgfpathcurveto{\pgfqpoint{0.552187in}{0.636358in}}{\pgfqpoint{0.558333in}{0.651196in}}{\pgfqpoint{0.558333in}{0.666667in}}%
\pgfpathcurveto{\pgfqpoint{0.558333in}{0.682137in}}{\pgfqpoint{0.552187in}{0.696975in}}{\pgfqpoint{0.541248in}{0.707915in}}%
\pgfpathcurveto{\pgfqpoint{0.530309in}{0.718854in}}{\pgfqpoint{0.515470in}{0.725000in}}{\pgfqpoint{0.500000in}{0.725000in}}%
\pgfpathcurveto{\pgfqpoint{0.484530in}{0.725000in}}{\pgfqpoint{0.469691in}{0.718854in}}{\pgfqpoint{0.458752in}{0.707915in}}%
\pgfpathcurveto{\pgfqpoint{0.447813in}{0.696975in}}{\pgfqpoint{0.441667in}{0.682137in}}{\pgfqpoint{0.441667in}{0.666667in}}%
\pgfpathcurveto{\pgfqpoint{0.441667in}{0.651196in}}{\pgfqpoint{0.447813in}{0.636358in}}{\pgfqpoint{0.458752in}{0.625419in}}%
\pgfpathcurveto{\pgfqpoint{0.469691in}{0.614480in}}{\pgfqpoint{0.484530in}{0.608333in}}{\pgfqpoint{0.500000in}{0.608333in}}%
\pgfpathclose%
\pgfpathmoveto{\pgfqpoint{0.500000in}{0.614167in}}%
\pgfpathcurveto{\pgfqpoint{0.500000in}{0.614167in}}{\pgfqpoint{0.486077in}{0.614167in}}{\pgfqpoint{0.472722in}{0.619698in}}%
\pgfpathcurveto{\pgfqpoint{0.462877in}{0.629544in}}{\pgfqpoint{0.453032in}{0.639389in}}{\pgfqpoint{0.447500in}{0.652744in}}%
\pgfpathcurveto{\pgfqpoint{0.447500in}{0.666667in}}{\pgfqpoint{0.447500in}{0.680590in}}{\pgfqpoint{0.453032in}{0.693945in}}%
\pgfpathcurveto{\pgfqpoint{0.462877in}{0.703790in}}{\pgfqpoint{0.472722in}{0.713635in}}{\pgfqpoint{0.486077in}{0.719167in}}%
\pgfpathcurveto{\pgfqpoint{0.500000in}{0.719167in}}{\pgfqpoint{0.513923in}{0.719167in}}{\pgfqpoint{0.527278in}{0.713635in}}%
\pgfpathcurveto{\pgfqpoint{0.537123in}{0.703790in}}{\pgfqpoint{0.546968in}{0.693945in}}{\pgfqpoint{0.552500in}{0.680590in}}%
\pgfpathcurveto{\pgfqpoint{0.552500in}{0.666667in}}{\pgfqpoint{0.552500in}{0.652744in}}{\pgfqpoint{0.546968in}{0.639389in}}%
\pgfpathcurveto{\pgfqpoint{0.537123in}{0.629544in}}{\pgfqpoint{0.527278in}{0.619698in}}{\pgfqpoint{0.513923in}{0.614167in}}%
\pgfpathclose%
\pgfpathmoveto{\pgfqpoint{0.666667in}{0.608333in}}%
\pgfpathcurveto{\pgfqpoint{0.682137in}{0.608333in}}{\pgfqpoint{0.696975in}{0.614480in}}{\pgfqpoint{0.707915in}{0.625419in}}%
\pgfpathcurveto{\pgfqpoint{0.718854in}{0.636358in}}{\pgfqpoint{0.725000in}{0.651196in}}{\pgfqpoint{0.725000in}{0.666667in}}%
\pgfpathcurveto{\pgfqpoint{0.725000in}{0.682137in}}{\pgfqpoint{0.718854in}{0.696975in}}{\pgfqpoint{0.707915in}{0.707915in}}%
\pgfpathcurveto{\pgfqpoint{0.696975in}{0.718854in}}{\pgfqpoint{0.682137in}{0.725000in}}{\pgfqpoint{0.666667in}{0.725000in}}%
\pgfpathcurveto{\pgfqpoint{0.651196in}{0.725000in}}{\pgfqpoint{0.636358in}{0.718854in}}{\pgfqpoint{0.625419in}{0.707915in}}%
\pgfpathcurveto{\pgfqpoint{0.614480in}{0.696975in}}{\pgfqpoint{0.608333in}{0.682137in}}{\pgfqpoint{0.608333in}{0.666667in}}%
\pgfpathcurveto{\pgfqpoint{0.608333in}{0.651196in}}{\pgfqpoint{0.614480in}{0.636358in}}{\pgfqpoint{0.625419in}{0.625419in}}%
\pgfpathcurveto{\pgfqpoint{0.636358in}{0.614480in}}{\pgfqpoint{0.651196in}{0.608333in}}{\pgfqpoint{0.666667in}{0.608333in}}%
\pgfpathclose%
\pgfpathmoveto{\pgfqpoint{0.666667in}{0.614167in}}%
\pgfpathcurveto{\pgfqpoint{0.666667in}{0.614167in}}{\pgfqpoint{0.652744in}{0.614167in}}{\pgfqpoint{0.639389in}{0.619698in}}%
\pgfpathcurveto{\pgfqpoint{0.629544in}{0.629544in}}{\pgfqpoint{0.619698in}{0.639389in}}{\pgfqpoint{0.614167in}{0.652744in}}%
\pgfpathcurveto{\pgfqpoint{0.614167in}{0.666667in}}{\pgfqpoint{0.614167in}{0.680590in}}{\pgfqpoint{0.619698in}{0.693945in}}%
\pgfpathcurveto{\pgfqpoint{0.629544in}{0.703790in}}{\pgfqpoint{0.639389in}{0.713635in}}{\pgfqpoint{0.652744in}{0.719167in}}%
\pgfpathcurveto{\pgfqpoint{0.666667in}{0.719167in}}{\pgfqpoint{0.680590in}{0.719167in}}{\pgfqpoint{0.693945in}{0.713635in}}%
\pgfpathcurveto{\pgfqpoint{0.703790in}{0.703790in}}{\pgfqpoint{0.713635in}{0.693945in}}{\pgfqpoint{0.719167in}{0.680590in}}%
\pgfpathcurveto{\pgfqpoint{0.719167in}{0.666667in}}{\pgfqpoint{0.719167in}{0.652744in}}{\pgfqpoint{0.713635in}{0.639389in}}%
\pgfpathcurveto{\pgfqpoint{0.703790in}{0.629544in}}{\pgfqpoint{0.693945in}{0.619698in}}{\pgfqpoint{0.680590in}{0.614167in}}%
\pgfpathclose%
\pgfpathmoveto{\pgfqpoint{0.833333in}{0.608333in}}%
\pgfpathcurveto{\pgfqpoint{0.848804in}{0.608333in}}{\pgfqpoint{0.863642in}{0.614480in}}{\pgfqpoint{0.874581in}{0.625419in}}%
\pgfpathcurveto{\pgfqpoint{0.885520in}{0.636358in}}{\pgfqpoint{0.891667in}{0.651196in}}{\pgfqpoint{0.891667in}{0.666667in}}%
\pgfpathcurveto{\pgfqpoint{0.891667in}{0.682137in}}{\pgfqpoint{0.885520in}{0.696975in}}{\pgfqpoint{0.874581in}{0.707915in}}%
\pgfpathcurveto{\pgfqpoint{0.863642in}{0.718854in}}{\pgfqpoint{0.848804in}{0.725000in}}{\pgfqpoint{0.833333in}{0.725000in}}%
\pgfpathcurveto{\pgfqpoint{0.817863in}{0.725000in}}{\pgfqpoint{0.803025in}{0.718854in}}{\pgfqpoint{0.792085in}{0.707915in}}%
\pgfpathcurveto{\pgfqpoint{0.781146in}{0.696975in}}{\pgfqpoint{0.775000in}{0.682137in}}{\pgfqpoint{0.775000in}{0.666667in}}%
\pgfpathcurveto{\pgfqpoint{0.775000in}{0.651196in}}{\pgfqpoint{0.781146in}{0.636358in}}{\pgfqpoint{0.792085in}{0.625419in}}%
\pgfpathcurveto{\pgfqpoint{0.803025in}{0.614480in}}{\pgfqpoint{0.817863in}{0.608333in}}{\pgfqpoint{0.833333in}{0.608333in}}%
\pgfpathclose%
\pgfpathmoveto{\pgfqpoint{0.833333in}{0.614167in}}%
\pgfpathcurveto{\pgfqpoint{0.833333in}{0.614167in}}{\pgfqpoint{0.819410in}{0.614167in}}{\pgfqpoint{0.806055in}{0.619698in}}%
\pgfpathcurveto{\pgfqpoint{0.796210in}{0.629544in}}{\pgfqpoint{0.786365in}{0.639389in}}{\pgfqpoint{0.780833in}{0.652744in}}%
\pgfpathcurveto{\pgfqpoint{0.780833in}{0.666667in}}{\pgfqpoint{0.780833in}{0.680590in}}{\pgfqpoint{0.786365in}{0.693945in}}%
\pgfpathcurveto{\pgfqpoint{0.796210in}{0.703790in}}{\pgfqpoint{0.806055in}{0.713635in}}{\pgfqpoint{0.819410in}{0.719167in}}%
\pgfpathcurveto{\pgfqpoint{0.833333in}{0.719167in}}{\pgfqpoint{0.847256in}{0.719167in}}{\pgfqpoint{0.860611in}{0.713635in}}%
\pgfpathcurveto{\pgfqpoint{0.870456in}{0.703790in}}{\pgfqpoint{0.880302in}{0.693945in}}{\pgfqpoint{0.885833in}{0.680590in}}%
\pgfpathcurveto{\pgfqpoint{0.885833in}{0.666667in}}{\pgfqpoint{0.885833in}{0.652744in}}{\pgfqpoint{0.880302in}{0.639389in}}%
\pgfpathcurveto{\pgfqpoint{0.870456in}{0.629544in}}{\pgfqpoint{0.860611in}{0.619698in}}{\pgfqpoint{0.847256in}{0.614167in}}%
\pgfpathclose%
\pgfpathmoveto{\pgfqpoint{1.000000in}{0.608333in}}%
\pgfpathcurveto{\pgfqpoint{1.015470in}{0.608333in}}{\pgfqpoint{1.030309in}{0.614480in}}{\pgfqpoint{1.041248in}{0.625419in}}%
\pgfpathcurveto{\pgfqpoint{1.052187in}{0.636358in}}{\pgfqpoint{1.058333in}{0.651196in}}{\pgfqpoint{1.058333in}{0.666667in}}%
\pgfpathcurveto{\pgfqpoint{1.058333in}{0.682137in}}{\pgfqpoint{1.052187in}{0.696975in}}{\pgfqpoint{1.041248in}{0.707915in}}%
\pgfpathcurveto{\pgfqpoint{1.030309in}{0.718854in}}{\pgfqpoint{1.015470in}{0.725000in}}{\pgfqpoint{1.000000in}{0.725000in}}%
\pgfpathcurveto{\pgfqpoint{0.984530in}{0.725000in}}{\pgfqpoint{0.969691in}{0.718854in}}{\pgfqpoint{0.958752in}{0.707915in}}%
\pgfpathcurveto{\pgfqpoint{0.947813in}{0.696975in}}{\pgfqpoint{0.941667in}{0.682137in}}{\pgfqpoint{0.941667in}{0.666667in}}%
\pgfpathcurveto{\pgfqpoint{0.941667in}{0.651196in}}{\pgfqpoint{0.947813in}{0.636358in}}{\pgfqpoint{0.958752in}{0.625419in}}%
\pgfpathcurveto{\pgfqpoint{0.969691in}{0.614480in}}{\pgfqpoint{0.984530in}{0.608333in}}{\pgfqpoint{1.000000in}{0.608333in}}%
\pgfpathclose%
\pgfpathmoveto{\pgfqpoint{1.000000in}{0.614167in}}%
\pgfpathcurveto{\pgfqpoint{1.000000in}{0.614167in}}{\pgfqpoint{0.986077in}{0.614167in}}{\pgfqpoint{0.972722in}{0.619698in}}%
\pgfpathcurveto{\pgfqpoint{0.962877in}{0.629544in}}{\pgfqpoint{0.953032in}{0.639389in}}{\pgfqpoint{0.947500in}{0.652744in}}%
\pgfpathcurveto{\pgfqpoint{0.947500in}{0.666667in}}{\pgfqpoint{0.947500in}{0.680590in}}{\pgfqpoint{0.953032in}{0.693945in}}%
\pgfpathcurveto{\pgfqpoint{0.962877in}{0.703790in}}{\pgfqpoint{0.972722in}{0.713635in}}{\pgfqpoint{0.986077in}{0.719167in}}%
\pgfpathcurveto{\pgfqpoint{1.000000in}{0.719167in}}{\pgfqpoint{1.013923in}{0.719167in}}{\pgfqpoint{1.027278in}{0.713635in}}%
\pgfpathcurveto{\pgfqpoint{1.037123in}{0.703790in}}{\pgfqpoint{1.046968in}{0.693945in}}{\pgfqpoint{1.052500in}{0.680590in}}%
\pgfpathcurveto{\pgfqpoint{1.052500in}{0.666667in}}{\pgfqpoint{1.052500in}{0.652744in}}{\pgfqpoint{1.046968in}{0.639389in}}%
\pgfpathcurveto{\pgfqpoint{1.037123in}{0.629544in}}{\pgfqpoint{1.027278in}{0.619698in}}{\pgfqpoint{1.013923in}{0.614167in}}%
\pgfpathclose%
\pgfpathmoveto{\pgfqpoint{0.083333in}{0.775000in}}%
\pgfpathcurveto{\pgfqpoint{0.098804in}{0.775000in}}{\pgfqpoint{0.113642in}{0.781146in}}{\pgfqpoint{0.124581in}{0.792085in}}%
\pgfpathcurveto{\pgfqpoint{0.135520in}{0.803025in}}{\pgfqpoint{0.141667in}{0.817863in}}{\pgfqpoint{0.141667in}{0.833333in}}%
\pgfpathcurveto{\pgfqpoint{0.141667in}{0.848804in}}{\pgfqpoint{0.135520in}{0.863642in}}{\pgfqpoint{0.124581in}{0.874581in}}%
\pgfpathcurveto{\pgfqpoint{0.113642in}{0.885520in}}{\pgfqpoint{0.098804in}{0.891667in}}{\pgfqpoint{0.083333in}{0.891667in}}%
\pgfpathcurveto{\pgfqpoint{0.067863in}{0.891667in}}{\pgfqpoint{0.053025in}{0.885520in}}{\pgfqpoint{0.042085in}{0.874581in}}%
\pgfpathcurveto{\pgfqpoint{0.031146in}{0.863642in}}{\pgfqpoint{0.025000in}{0.848804in}}{\pgfqpoint{0.025000in}{0.833333in}}%
\pgfpathcurveto{\pgfqpoint{0.025000in}{0.817863in}}{\pgfqpoint{0.031146in}{0.803025in}}{\pgfqpoint{0.042085in}{0.792085in}}%
\pgfpathcurveto{\pgfqpoint{0.053025in}{0.781146in}}{\pgfqpoint{0.067863in}{0.775000in}}{\pgfqpoint{0.083333in}{0.775000in}}%
\pgfpathclose%
\pgfpathmoveto{\pgfqpoint{0.083333in}{0.780833in}}%
\pgfpathcurveto{\pgfqpoint{0.083333in}{0.780833in}}{\pgfqpoint{0.069410in}{0.780833in}}{\pgfqpoint{0.056055in}{0.786365in}}%
\pgfpathcurveto{\pgfqpoint{0.046210in}{0.796210in}}{\pgfqpoint{0.036365in}{0.806055in}}{\pgfqpoint{0.030833in}{0.819410in}}%
\pgfpathcurveto{\pgfqpoint{0.030833in}{0.833333in}}{\pgfqpoint{0.030833in}{0.847256in}}{\pgfqpoint{0.036365in}{0.860611in}}%
\pgfpathcurveto{\pgfqpoint{0.046210in}{0.870456in}}{\pgfqpoint{0.056055in}{0.880302in}}{\pgfqpoint{0.069410in}{0.885833in}}%
\pgfpathcurveto{\pgfqpoint{0.083333in}{0.885833in}}{\pgfqpoint{0.097256in}{0.885833in}}{\pgfqpoint{0.110611in}{0.880302in}}%
\pgfpathcurveto{\pgfqpoint{0.120456in}{0.870456in}}{\pgfqpoint{0.130302in}{0.860611in}}{\pgfqpoint{0.135833in}{0.847256in}}%
\pgfpathcurveto{\pgfqpoint{0.135833in}{0.833333in}}{\pgfqpoint{0.135833in}{0.819410in}}{\pgfqpoint{0.130302in}{0.806055in}}%
\pgfpathcurveto{\pgfqpoint{0.120456in}{0.796210in}}{\pgfqpoint{0.110611in}{0.786365in}}{\pgfqpoint{0.097256in}{0.780833in}}%
\pgfpathclose%
\pgfpathmoveto{\pgfqpoint{0.250000in}{0.775000in}}%
\pgfpathcurveto{\pgfqpoint{0.265470in}{0.775000in}}{\pgfqpoint{0.280309in}{0.781146in}}{\pgfqpoint{0.291248in}{0.792085in}}%
\pgfpathcurveto{\pgfqpoint{0.302187in}{0.803025in}}{\pgfqpoint{0.308333in}{0.817863in}}{\pgfqpoint{0.308333in}{0.833333in}}%
\pgfpathcurveto{\pgfqpoint{0.308333in}{0.848804in}}{\pgfqpoint{0.302187in}{0.863642in}}{\pgfqpoint{0.291248in}{0.874581in}}%
\pgfpathcurveto{\pgfqpoint{0.280309in}{0.885520in}}{\pgfqpoint{0.265470in}{0.891667in}}{\pgfqpoint{0.250000in}{0.891667in}}%
\pgfpathcurveto{\pgfqpoint{0.234530in}{0.891667in}}{\pgfqpoint{0.219691in}{0.885520in}}{\pgfqpoint{0.208752in}{0.874581in}}%
\pgfpathcurveto{\pgfqpoint{0.197813in}{0.863642in}}{\pgfqpoint{0.191667in}{0.848804in}}{\pgfqpoint{0.191667in}{0.833333in}}%
\pgfpathcurveto{\pgfqpoint{0.191667in}{0.817863in}}{\pgfqpoint{0.197813in}{0.803025in}}{\pgfqpoint{0.208752in}{0.792085in}}%
\pgfpathcurveto{\pgfqpoint{0.219691in}{0.781146in}}{\pgfqpoint{0.234530in}{0.775000in}}{\pgfqpoint{0.250000in}{0.775000in}}%
\pgfpathclose%
\pgfpathmoveto{\pgfqpoint{0.250000in}{0.780833in}}%
\pgfpathcurveto{\pgfqpoint{0.250000in}{0.780833in}}{\pgfqpoint{0.236077in}{0.780833in}}{\pgfqpoint{0.222722in}{0.786365in}}%
\pgfpathcurveto{\pgfqpoint{0.212877in}{0.796210in}}{\pgfqpoint{0.203032in}{0.806055in}}{\pgfqpoint{0.197500in}{0.819410in}}%
\pgfpathcurveto{\pgfqpoint{0.197500in}{0.833333in}}{\pgfqpoint{0.197500in}{0.847256in}}{\pgfqpoint{0.203032in}{0.860611in}}%
\pgfpathcurveto{\pgfqpoint{0.212877in}{0.870456in}}{\pgfqpoint{0.222722in}{0.880302in}}{\pgfqpoint{0.236077in}{0.885833in}}%
\pgfpathcurveto{\pgfqpoint{0.250000in}{0.885833in}}{\pgfqpoint{0.263923in}{0.885833in}}{\pgfqpoint{0.277278in}{0.880302in}}%
\pgfpathcurveto{\pgfqpoint{0.287123in}{0.870456in}}{\pgfqpoint{0.296968in}{0.860611in}}{\pgfqpoint{0.302500in}{0.847256in}}%
\pgfpathcurveto{\pgfqpoint{0.302500in}{0.833333in}}{\pgfqpoint{0.302500in}{0.819410in}}{\pgfqpoint{0.296968in}{0.806055in}}%
\pgfpathcurveto{\pgfqpoint{0.287123in}{0.796210in}}{\pgfqpoint{0.277278in}{0.786365in}}{\pgfqpoint{0.263923in}{0.780833in}}%
\pgfpathclose%
\pgfpathmoveto{\pgfqpoint{0.416667in}{0.775000in}}%
\pgfpathcurveto{\pgfqpoint{0.432137in}{0.775000in}}{\pgfqpoint{0.446975in}{0.781146in}}{\pgfqpoint{0.457915in}{0.792085in}}%
\pgfpathcurveto{\pgfqpoint{0.468854in}{0.803025in}}{\pgfqpoint{0.475000in}{0.817863in}}{\pgfqpoint{0.475000in}{0.833333in}}%
\pgfpathcurveto{\pgfqpoint{0.475000in}{0.848804in}}{\pgfqpoint{0.468854in}{0.863642in}}{\pgfqpoint{0.457915in}{0.874581in}}%
\pgfpathcurveto{\pgfqpoint{0.446975in}{0.885520in}}{\pgfqpoint{0.432137in}{0.891667in}}{\pgfqpoint{0.416667in}{0.891667in}}%
\pgfpathcurveto{\pgfqpoint{0.401196in}{0.891667in}}{\pgfqpoint{0.386358in}{0.885520in}}{\pgfqpoint{0.375419in}{0.874581in}}%
\pgfpathcurveto{\pgfqpoint{0.364480in}{0.863642in}}{\pgfqpoint{0.358333in}{0.848804in}}{\pgfqpoint{0.358333in}{0.833333in}}%
\pgfpathcurveto{\pgfqpoint{0.358333in}{0.817863in}}{\pgfqpoint{0.364480in}{0.803025in}}{\pgfqpoint{0.375419in}{0.792085in}}%
\pgfpathcurveto{\pgfqpoint{0.386358in}{0.781146in}}{\pgfqpoint{0.401196in}{0.775000in}}{\pgfqpoint{0.416667in}{0.775000in}}%
\pgfpathclose%
\pgfpathmoveto{\pgfqpoint{0.416667in}{0.780833in}}%
\pgfpathcurveto{\pgfqpoint{0.416667in}{0.780833in}}{\pgfqpoint{0.402744in}{0.780833in}}{\pgfqpoint{0.389389in}{0.786365in}}%
\pgfpathcurveto{\pgfqpoint{0.379544in}{0.796210in}}{\pgfqpoint{0.369698in}{0.806055in}}{\pgfqpoint{0.364167in}{0.819410in}}%
\pgfpathcurveto{\pgfqpoint{0.364167in}{0.833333in}}{\pgfqpoint{0.364167in}{0.847256in}}{\pgfqpoint{0.369698in}{0.860611in}}%
\pgfpathcurveto{\pgfqpoint{0.379544in}{0.870456in}}{\pgfqpoint{0.389389in}{0.880302in}}{\pgfqpoint{0.402744in}{0.885833in}}%
\pgfpathcurveto{\pgfqpoint{0.416667in}{0.885833in}}{\pgfqpoint{0.430590in}{0.885833in}}{\pgfqpoint{0.443945in}{0.880302in}}%
\pgfpathcurveto{\pgfqpoint{0.453790in}{0.870456in}}{\pgfqpoint{0.463635in}{0.860611in}}{\pgfqpoint{0.469167in}{0.847256in}}%
\pgfpathcurveto{\pgfqpoint{0.469167in}{0.833333in}}{\pgfqpoint{0.469167in}{0.819410in}}{\pgfqpoint{0.463635in}{0.806055in}}%
\pgfpathcurveto{\pgfqpoint{0.453790in}{0.796210in}}{\pgfqpoint{0.443945in}{0.786365in}}{\pgfqpoint{0.430590in}{0.780833in}}%
\pgfpathclose%
\pgfpathmoveto{\pgfqpoint{0.583333in}{0.775000in}}%
\pgfpathcurveto{\pgfqpoint{0.598804in}{0.775000in}}{\pgfqpoint{0.613642in}{0.781146in}}{\pgfqpoint{0.624581in}{0.792085in}}%
\pgfpathcurveto{\pgfqpoint{0.635520in}{0.803025in}}{\pgfqpoint{0.641667in}{0.817863in}}{\pgfqpoint{0.641667in}{0.833333in}}%
\pgfpathcurveto{\pgfqpoint{0.641667in}{0.848804in}}{\pgfqpoint{0.635520in}{0.863642in}}{\pgfqpoint{0.624581in}{0.874581in}}%
\pgfpathcurveto{\pgfqpoint{0.613642in}{0.885520in}}{\pgfqpoint{0.598804in}{0.891667in}}{\pgfqpoint{0.583333in}{0.891667in}}%
\pgfpathcurveto{\pgfqpoint{0.567863in}{0.891667in}}{\pgfqpoint{0.553025in}{0.885520in}}{\pgfqpoint{0.542085in}{0.874581in}}%
\pgfpathcurveto{\pgfqpoint{0.531146in}{0.863642in}}{\pgfqpoint{0.525000in}{0.848804in}}{\pgfqpoint{0.525000in}{0.833333in}}%
\pgfpathcurveto{\pgfqpoint{0.525000in}{0.817863in}}{\pgfqpoint{0.531146in}{0.803025in}}{\pgfqpoint{0.542085in}{0.792085in}}%
\pgfpathcurveto{\pgfqpoint{0.553025in}{0.781146in}}{\pgfqpoint{0.567863in}{0.775000in}}{\pgfqpoint{0.583333in}{0.775000in}}%
\pgfpathclose%
\pgfpathmoveto{\pgfqpoint{0.583333in}{0.780833in}}%
\pgfpathcurveto{\pgfqpoint{0.583333in}{0.780833in}}{\pgfqpoint{0.569410in}{0.780833in}}{\pgfqpoint{0.556055in}{0.786365in}}%
\pgfpathcurveto{\pgfqpoint{0.546210in}{0.796210in}}{\pgfqpoint{0.536365in}{0.806055in}}{\pgfqpoint{0.530833in}{0.819410in}}%
\pgfpathcurveto{\pgfqpoint{0.530833in}{0.833333in}}{\pgfqpoint{0.530833in}{0.847256in}}{\pgfqpoint{0.536365in}{0.860611in}}%
\pgfpathcurveto{\pgfqpoint{0.546210in}{0.870456in}}{\pgfqpoint{0.556055in}{0.880302in}}{\pgfqpoint{0.569410in}{0.885833in}}%
\pgfpathcurveto{\pgfqpoint{0.583333in}{0.885833in}}{\pgfqpoint{0.597256in}{0.885833in}}{\pgfqpoint{0.610611in}{0.880302in}}%
\pgfpathcurveto{\pgfqpoint{0.620456in}{0.870456in}}{\pgfqpoint{0.630302in}{0.860611in}}{\pgfqpoint{0.635833in}{0.847256in}}%
\pgfpathcurveto{\pgfqpoint{0.635833in}{0.833333in}}{\pgfqpoint{0.635833in}{0.819410in}}{\pgfqpoint{0.630302in}{0.806055in}}%
\pgfpathcurveto{\pgfqpoint{0.620456in}{0.796210in}}{\pgfqpoint{0.610611in}{0.786365in}}{\pgfqpoint{0.597256in}{0.780833in}}%
\pgfpathclose%
\pgfpathmoveto{\pgfqpoint{0.750000in}{0.775000in}}%
\pgfpathcurveto{\pgfqpoint{0.765470in}{0.775000in}}{\pgfqpoint{0.780309in}{0.781146in}}{\pgfqpoint{0.791248in}{0.792085in}}%
\pgfpathcurveto{\pgfqpoint{0.802187in}{0.803025in}}{\pgfqpoint{0.808333in}{0.817863in}}{\pgfqpoint{0.808333in}{0.833333in}}%
\pgfpathcurveto{\pgfqpoint{0.808333in}{0.848804in}}{\pgfqpoint{0.802187in}{0.863642in}}{\pgfqpoint{0.791248in}{0.874581in}}%
\pgfpathcurveto{\pgfqpoint{0.780309in}{0.885520in}}{\pgfqpoint{0.765470in}{0.891667in}}{\pgfqpoint{0.750000in}{0.891667in}}%
\pgfpathcurveto{\pgfqpoint{0.734530in}{0.891667in}}{\pgfqpoint{0.719691in}{0.885520in}}{\pgfqpoint{0.708752in}{0.874581in}}%
\pgfpathcurveto{\pgfqpoint{0.697813in}{0.863642in}}{\pgfqpoint{0.691667in}{0.848804in}}{\pgfqpoint{0.691667in}{0.833333in}}%
\pgfpathcurveto{\pgfqpoint{0.691667in}{0.817863in}}{\pgfqpoint{0.697813in}{0.803025in}}{\pgfqpoint{0.708752in}{0.792085in}}%
\pgfpathcurveto{\pgfqpoint{0.719691in}{0.781146in}}{\pgfqpoint{0.734530in}{0.775000in}}{\pgfqpoint{0.750000in}{0.775000in}}%
\pgfpathclose%
\pgfpathmoveto{\pgfqpoint{0.750000in}{0.780833in}}%
\pgfpathcurveto{\pgfqpoint{0.750000in}{0.780833in}}{\pgfqpoint{0.736077in}{0.780833in}}{\pgfqpoint{0.722722in}{0.786365in}}%
\pgfpathcurveto{\pgfqpoint{0.712877in}{0.796210in}}{\pgfqpoint{0.703032in}{0.806055in}}{\pgfqpoint{0.697500in}{0.819410in}}%
\pgfpathcurveto{\pgfqpoint{0.697500in}{0.833333in}}{\pgfqpoint{0.697500in}{0.847256in}}{\pgfqpoint{0.703032in}{0.860611in}}%
\pgfpathcurveto{\pgfqpoint{0.712877in}{0.870456in}}{\pgfqpoint{0.722722in}{0.880302in}}{\pgfqpoint{0.736077in}{0.885833in}}%
\pgfpathcurveto{\pgfqpoint{0.750000in}{0.885833in}}{\pgfqpoint{0.763923in}{0.885833in}}{\pgfqpoint{0.777278in}{0.880302in}}%
\pgfpathcurveto{\pgfqpoint{0.787123in}{0.870456in}}{\pgfqpoint{0.796968in}{0.860611in}}{\pgfqpoint{0.802500in}{0.847256in}}%
\pgfpathcurveto{\pgfqpoint{0.802500in}{0.833333in}}{\pgfqpoint{0.802500in}{0.819410in}}{\pgfqpoint{0.796968in}{0.806055in}}%
\pgfpathcurveto{\pgfqpoint{0.787123in}{0.796210in}}{\pgfqpoint{0.777278in}{0.786365in}}{\pgfqpoint{0.763923in}{0.780833in}}%
\pgfpathclose%
\pgfpathmoveto{\pgfqpoint{0.916667in}{0.775000in}}%
\pgfpathcurveto{\pgfqpoint{0.932137in}{0.775000in}}{\pgfqpoint{0.946975in}{0.781146in}}{\pgfqpoint{0.957915in}{0.792085in}}%
\pgfpathcurveto{\pgfqpoint{0.968854in}{0.803025in}}{\pgfqpoint{0.975000in}{0.817863in}}{\pgfqpoint{0.975000in}{0.833333in}}%
\pgfpathcurveto{\pgfqpoint{0.975000in}{0.848804in}}{\pgfqpoint{0.968854in}{0.863642in}}{\pgfqpoint{0.957915in}{0.874581in}}%
\pgfpathcurveto{\pgfqpoint{0.946975in}{0.885520in}}{\pgfqpoint{0.932137in}{0.891667in}}{\pgfqpoint{0.916667in}{0.891667in}}%
\pgfpathcurveto{\pgfqpoint{0.901196in}{0.891667in}}{\pgfqpoint{0.886358in}{0.885520in}}{\pgfqpoint{0.875419in}{0.874581in}}%
\pgfpathcurveto{\pgfqpoint{0.864480in}{0.863642in}}{\pgfqpoint{0.858333in}{0.848804in}}{\pgfqpoint{0.858333in}{0.833333in}}%
\pgfpathcurveto{\pgfqpoint{0.858333in}{0.817863in}}{\pgfqpoint{0.864480in}{0.803025in}}{\pgfqpoint{0.875419in}{0.792085in}}%
\pgfpathcurveto{\pgfqpoint{0.886358in}{0.781146in}}{\pgfqpoint{0.901196in}{0.775000in}}{\pgfqpoint{0.916667in}{0.775000in}}%
\pgfpathclose%
\pgfpathmoveto{\pgfqpoint{0.916667in}{0.780833in}}%
\pgfpathcurveto{\pgfqpoint{0.916667in}{0.780833in}}{\pgfqpoint{0.902744in}{0.780833in}}{\pgfqpoint{0.889389in}{0.786365in}}%
\pgfpathcurveto{\pgfqpoint{0.879544in}{0.796210in}}{\pgfqpoint{0.869698in}{0.806055in}}{\pgfqpoint{0.864167in}{0.819410in}}%
\pgfpathcurveto{\pgfqpoint{0.864167in}{0.833333in}}{\pgfqpoint{0.864167in}{0.847256in}}{\pgfqpoint{0.869698in}{0.860611in}}%
\pgfpathcurveto{\pgfqpoint{0.879544in}{0.870456in}}{\pgfqpoint{0.889389in}{0.880302in}}{\pgfqpoint{0.902744in}{0.885833in}}%
\pgfpathcurveto{\pgfqpoint{0.916667in}{0.885833in}}{\pgfqpoint{0.930590in}{0.885833in}}{\pgfqpoint{0.943945in}{0.880302in}}%
\pgfpathcurveto{\pgfqpoint{0.953790in}{0.870456in}}{\pgfqpoint{0.963635in}{0.860611in}}{\pgfqpoint{0.969167in}{0.847256in}}%
\pgfpathcurveto{\pgfqpoint{0.969167in}{0.833333in}}{\pgfqpoint{0.969167in}{0.819410in}}{\pgfqpoint{0.963635in}{0.806055in}}%
\pgfpathcurveto{\pgfqpoint{0.953790in}{0.796210in}}{\pgfqpoint{0.943945in}{0.786365in}}{\pgfqpoint{0.930590in}{0.780833in}}%
\pgfpathclose%
\pgfpathmoveto{\pgfqpoint{0.000000in}{0.941667in}}%
\pgfpathcurveto{\pgfqpoint{0.015470in}{0.941667in}}{\pgfqpoint{0.030309in}{0.947813in}}{\pgfqpoint{0.041248in}{0.958752in}}%
\pgfpathcurveto{\pgfqpoint{0.052187in}{0.969691in}}{\pgfqpoint{0.058333in}{0.984530in}}{\pgfqpoint{0.058333in}{1.000000in}}%
\pgfpathcurveto{\pgfqpoint{0.058333in}{1.015470in}}{\pgfqpoint{0.052187in}{1.030309in}}{\pgfqpoint{0.041248in}{1.041248in}}%
\pgfpathcurveto{\pgfqpoint{0.030309in}{1.052187in}}{\pgfqpoint{0.015470in}{1.058333in}}{\pgfqpoint{0.000000in}{1.058333in}}%
\pgfpathcurveto{\pgfqpoint{-0.015470in}{1.058333in}}{\pgfqpoint{-0.030309in}{1.052187in}}{\pgfqpoint{-0.041248in}{1.041248in}}%
\pgfpathcurveto{\pgfqpoint{-0.052187in}{1.030309in}}{\pgfqpoint{-0.058333in}{1.015470in}}{\pgfqpoint{-0.058333in}{1.000000in}}%
\pgfpathcurveto{\pgfqpoint{-0.058333in}{0.984530in}}{\pgfqpoint{-0.052187in}{0.969691in}}{\pgfqpoint{-0.041248in}{0.958752in}}%
\pgfpathcurveto{\pgfqpoint{-0.030309in}{0.947813in}}{\pgfqpoint{-0.015470in}{0.941667in}}{\pgfqpoint{0.000000in}{0.941667in}}%
\pgfpathclose%
\pgfpathmoveto{\pgfqpoint{0.000000in}{0.947500in}}%
\pgfpathcurveto{\pgfqpoint{0.000000in}{0.947500in}}{\pgfqpoint{-0.013923in}{0.947500in}}{\pgfqpoint{-0.027278in}{0.953032in}}%
\pgfpathcurveto{\pgfqpoint{-0.037123in}{0.962877in}}{\pgfqpoint{-0.046968in}{0.972722in}}{\pgfqpoint{-0.052500in}{0.986077in}}%
\pgfpathcurveto{\pgfqpoint{-0.052500in}{1.000000in}}{\pgfqpoint{-0.052500in}{1.013923in}}{\pgfqpoint{-0.046968in}{1.027278in}}%
\pgfpathcurveto{\pgfqpoint{-0.037123in}{1.037123in}}{\pgfqpoint{-0.027278in}{1.046968in}}{\pgfqpoint{-0.013923in}{1.052500in}}%
\pgfpathcurveto{\pgfqpoint{0.000000in}{1.052500in}}{\pgfqpoint{0.013923in}{1.052500in}}{\pgfqpoint{0.027278in}{1.046968in}}%
\pgfpathcurveto{\pgfqpoint{0.037123in}{1.037123in}}{\pgfqpoint{0.046968in}{1.027278in}}{\pgfqpoint{0.052500in}{1.013923in}}%
\pgfpathcurveto{\pgfqpoint{0.052500in}{1.000000in}}{\pgfqpoint{0.052500in}{0.986077in}}{\pgfqpoint{0.046968in}{0.972722in}}%
\pgfpathcurveto{\pgfqpoint{0.037123in}{0.962877in}}{\pgfqpoint{0.027278in}{0.953032in}}{\pgfqpoint{0.013923in}{0.947500in}}%
\pgfpathclose%
\pgfpathmoveto{\pgfqpoint{0.166667in}{0.941667in}}%
\pgfpathcurveto{\pgfqpoint{0.182137in}{0.941667in}}{\pgfqpoint{0.196975in}{0.947813in}}{\pgfqpoint{0.207915in}{0.958752in}}%
\pgfpathcurveto{\pgfqpoint{0.218854in}{0.969691in}}{\pgfqpoint{0.225000in}{0.984530in}}{\pgfqpoint{0.225000in}{1.000000in}}%
\pgfpathcurveto{\pgfqpoint{0.225000in}{1.015470in}}{\pgfqpoint{0.218854in}{1.030309in}}{\pgfqpoint{0.207915in}{1.041248in}}%
\pgfpathcurveto{\pgfqpoint{0.196975in}{1.052187in}}{\pgfqpoint{0.182137in}{1.058333in}}{\pgfqpoint{0.166667in}{1.058333in}}%
\pgfpathcurveto{\pgfqpoint{0.151196in}{1.058333in}}{\pgfqpoint{0.136358in}{1.052187in}}{\pgfqpoint{0.125419in}{1.041248in}}%
\pgfpathcurveto{\pgfqpoint{0.114480in}{1.030309in}}{\pgfqpoint{0.108333in}{1.015470in}}{\pgfqpoint{0.108333in}{1.000000in}}%
\pgfpathcurveto{\pgfqpoint{0.108333in}{0.984530in}}{\pgfqpoint{0.114480in}{0.969691in}}{\pgfqpoint{0.125419in}{0.958752in}}%
\pgfpathcurveto{\pgfqpoint{0.136358in}{0.947813in}}{\pgfqpoint{0.151196in}{0.941667in}}{\pgfqpoint{0.166667in}{0.941667in}}%
\pgfpathclose%
\pgfpathmoveto{\pgfqpoint{0.166667in}{0.947500in}}%
\pgfpathcurveto{\pgfqpoint{0.166667in}{0.947500in}}{\pgfqpoint{0.152744in}{0.947500in}}{\pgfqpoint{0.139389in}{0.953032in}}%
\pgfpathcurveto{\pgfqpoint{0.129544in}{0.962877in}}{\pgfqpoint{0.119698in}{0.972722in}}{\pgfqpoint{0.114167in}{0.986077in}}%
\pgfpathcurveto{\pgfqpoint{0.114167in}{1.000000in}}{\pgfqpoint{0.114167in}{1.013923in}}{\pgfqpoint{0.119698in}{1.027278in}}%
\pgfpathcurveto{\pgfqpoint{0.129544in}{1.037123in}}{\pgfqpoint{0.139389in}{1.046968in}}{\pgfqpoint{0.152744in}{1.052500in}}%
\pgfpathcurveto{\pgfqpoint{0.166667in}{1.052500in}}{\pgfqpoint{0.180590in}{1.052500in}}{\pgfqpoint{0.193945in}{1.046968in}}%
\pgfpathcurveto{\pgfqpoint{0.203790in}{1.037123in}}{\pgfqpoint{0.213635in}{1.027278in}}{\pgfqpoint{0.219167in}{1.013923in}}%
\pgfpathcurveto{\pgfqpoint{0.219167in}{1.000000in}}{\pgfqpoint{0.219167in}{0.986077in}}{\pgfqpoint{0.213635in}{0.972722in}}%
\pgfpathcurveto{\pgfqpoint{0.203790in}{0.962877in}}{\pgfqpoint{0.193945in}{0.953032in}}{\pgfqpoint{0.180590in}{0.947500in}}%
\pgfpathclose%
\pgfpathmoveto{\pgfqpoint{0.333333in}{0.941667in}}%
\pgfpathcurveto{\pgfqpoint{0.348804in}{0.941667in}}{\pgfqpoint{0.363642in}{0.947813in}}{\pgfqpoint{0.374581in}{0.958752in}}%
\pgfpathcurveto{\pgfqpoint{0.385520in}{0.969691in}}{\pgfqpoint{0.391667in}{0.984530in}}{\pgfqpoint{0.391667in}{1.000000in}}%
\pgfpathcurveto{\pgfqpoint{0.391667in}{1.015470in}}{\pgfqpoint{0.385520in}{1.030309in}}{\pgfqpoint{0.374581in}{1.041248in}}%
\pgfpathcurveto{\pgfqpoint{0.363642in}{1.052187in}}{\pgfqpoint{0.348804in}{1.058333in}}{\pgfqpoint{0.333333in}{1.058333in}}%
\pgfpathcurveto{\pgfqpoint{0.317863in}{1.058333in}}{\pgfqpoint{0.303025in}{1.052187in}}{\pgfqpoint{0.292085in}{1.041248in}}%
\pgfpathcurveto{\pgfqpoint{0.281146in}{1.030309in}}{\pgfqpoint{0.275000in}{1.015470in}}{\pgfqpoint{0.275000in}{1.000000in}}%
\pgfpathcurveto{\pgfqpoint{0.275000in}{0.984530in}}{\pgfqpoint{0.281146in}{0.969691in}}{\pgfqpoint{0.292085in}{0.958752in}}%
\pgfpathcurveto{\pgfqpoint{0.303025in}{0.947813in}}{\pgfqpoint{0.317863in}{0.941667in}}{\pgfqpoint{0.333333in}{0.941667in}}%
\pgfpathclose%
\pgfpathmoveto{\pgfqpoint{0.333333in}{0.947500in}}%
\pgfpathcurveto{\pgfqpoint{0.333333in}{0.947500in}}{\pgfqpoint{0.319410in}{0.947500in}}{\pgfqpoint{0.306055in}{0.953032in}}%
\pgfpathcurveto{\pgfqpoint{0.296210in}{0.962877in}}{\pgfqpoint{0.286365in}{0.972722in}}{\pgfqpoint{0.280833in}{0.986077in}}%
\pgfpathcurveto{\pgfqpoint{0.280833in}{1.000000in}}{\pgfqpoint{0.280833in}{1.013923in}}{\pgfqpoint{0.286365in}{1.027278in}}%
\pgfpathcurveto{\pgfqpoint{0.296210in}{1.037123in}}{\pgfqpoint{0.306055in}{1.046968in}}{\pgfqpoint{0.319410in}{1.052500in}}%
\pgfpathcurveto{\pgfqpoint{0.333333in}{1.052500in}}{\pgfqpoint{0.347256in}{1.052500in}}{\pgfqpoint{0.360611in}{1.046968in}}%
\pgfpathcurveto{\pgfqpoint{0.370456in}{1.037123in}}{\pgfqpoint{0.380302in}{1.027278in}}{\pgfqpoint{0.385833in}{1.013923in}}%
\pgfpathcurveto{\pgfqpoint{0.385833in}{1.000000in}}{\pgfqpoint{0.385833in}{0.986077in}}{\pgfqpoint{0.380302in}{0.972722in}}%
\pgfpathcurveto{\pgfqpoint{0.370456in}{0.962877in}}{\pgfqpoint{0.360611in}{0.953032in}}{\pgfqpoint{0.347256in}{0.947500in}}%
\pgfpathclose%
\pgfpathmoveto{\pgfqpoint{0.500000in}{0.941667in}}%
\pgfpathcurveto{\pgfqpoint{0.515470in}{0.941667in}}{\pgfqpoint{0.530309in}{0.947813in}}{\pgfqpoint{0.541248in}{0.958752in}}%
\pgfpathcurveto{\pgfqpoint{0.552187in}{0.969691in}}{\pgfqpoint{0.558333in}{0.984530in}}{\pgfqpoint{0.558333in}{1.000000in}}%
\pgfpathcurveto{\pgfqpoint{0.558333in}{1.015470in}}{\pgfqpoint{0.552187in}{1.030309in}}{\pgfqpoint{0.541248in}{1.041248in}}%
\pgfpathcurveto{\pgfqpoint{0.530309in}{1.052187in}}{\pgfqpoint{0.515470in}{1.058333in}}{\pgfqpoint{0.500000in}{1.058333in}}%
\pgfpathcurveto{\pgfqpoint{0.484530in}{1.058333in}}{\pgfqpoint{0.469691in}{1.052187in}}{\pgfqpoint{0.458752in}{1.041248in}}%
\pgfpathcurveto{\pgfqpoint{0.447813in}{1.030309in}}{\pgfqpoint{0.441667in}{1.015470in}}{\pgfqpoint{0.441667in}{1.000000in}}%
\pgfpathcurveto{\pgfqpoint{0.441667in}{0.984530in}}{\pgfqpoint{0.447813in}{0.969691in}}{\pgfqpoint{0.458752in}{0.958752in}}%
\pgfpathcurveto{\pgfqpoint{0.469691in}{0.947813in}}{\pgfqpoint{0.484530in}{0.941667in}}{\pgfqpoint{0.500000in}{0.941667in}}%
\pgfpathclose%
\pgfpathmoveto{\pgfqpoint{0.500000in}{0.947500in}}%
\pgfpathcurveto{\pgfqpoint{0.500000in}{0.947500in}}{\pgfqpoint{0.486077in}{0.947500in}}{\pgfqpoint{0.472722in}{0.953032in}}%
\pgfpathcurveto{\pgfqpoint{0.462877in}{0.962877in}}{\pgfqpoint{0.453032in}{0.972722in}}{\pgfqpoint{0.447500in}{0.986077in}}%
\pgfpathcurveto{\pgfqpoint{0.447500in}{1.000000in}}{\pgfqpoint{0.447500in}{1.013923in}}{\pgfqpoint{0.453032in}{1.027278in}}%
\pgfpathcurveto{\pgfqpoint{0.462877in}{1.037123in}}{\pgfqpoint{0.472722in}{1.046968in}}{\pgfqpoint{0.486077in}{1.052500in}}%
\pgfpathcurveto{\pgfqpoint{0.500000in}{1.052500in}}{\pgfqpoint{0.513923in}{1.052500in}}{\pgfqpoint{0.527278in}{1.046968in}}%
\pgfpathcurveto{\pgfqpoint{0.537123in}{1.037123in}}{\pgfqpoint{0.546968in}{1.027278in}}{\pgfqpoint{0.552500in}{1.013923in}}%
\pgfpathcurveto{\pgfqpoint{0.552500in}{1.000000in}}{\pgfqpoint{0.552500in}{0.986077in}}{\pgfqpoint{0.546968in}{0.972722in}}%
\pgfpathcurveto{\pgfqpoint{0.537123in}{0.962877in}}{\pgfqpoint{0.527278in}{0.953032in}}{\pgfqpoint{0.513923in}{0.947500in}}%
\pgfpathclose%
\pgfpathmoveto{\pgfqpoint{0.666667in}{0.941667in}}%
\pgfpathcurveto{\pgfqpoint{0.682137in}{0.941667in}}{\pgfqpoint{0.696975in}{0.947813in}}{\pgfqpoint{0.707915in}{0.958752in}}%
\pgfpathcurveto{\pgfqpoint{0.718854in}{0.969691in}}{\pgfqpoint{0.725000in}{0.984530in}}{\pgfqpoint{0.725000in}{1.000000in}}%
\pgfpathcurveto{\pgfqpoint{0.725000in}{1.015470in}}{\pgfqpoint{0.718854in}{1.030309in}}{\pgfqpoint{0.707915in}{1.041248in}}%
\pgfpathcurveto{\pgfqpoint{0.696975in}{1.052187in}}{\pgfqpoint{0.682137in}{1.058333in}}{\pgfqpoint{0.666667in}{1.058333in}}%
\pgfpathcurveto{\pgfqpoint{0.651196in}{1.058333in}}{\pgfqpoint{0.636358in}{1.052187in}}{\pgfqpoint{0.625419in}{1.041248in}}%
\pgfpathcurveto{\pgfqpoint{0.614480in}{1.030309in}}{\pgfqpoint{0.608333in}{1.015470in}}{\pgfqpoint{0.608333in}{1.000000in}}%
\pgfpathcurveto{\pgfqpoint{0.608333in}{0.984530in}}{\pgfqpoint{0.614480in}{0.969691in}}{\pgfqpoint{0.625419in}{0.958752in}}%
\pgfpathcurveto{\pgfqpoint{0.636358in}{0.947813in}}{\pgfqpoint{0.651196in}{0.941667in}}{\pgfqpoint{0.666667in}{0.941667in}}%
\pgfpathclose%
\pgfpathmoveto{\pgfqpoint{0.666667in}{0.947500in}}%
\pgfpathcurveto{\pgfqpoint{0.666667in}{0.947500in}}{\pgfqpoint{0.652744in}{0.947500in}}{\pgfqpoint{0.639389in}{0.953032in}}%
\pgfpathcurveto{\pgfqpoint{0.629544in}{0.962877in}}{\pgfqpoint{0.619698in}{0.972722in}}{\pgfqpoint{0.614167in}{0.986077in}}%
\pgfpathcurveto{\pgfqpoint{0.614167in}{1.000000in}}{\pgfqpoint{0.614167in}{1.013923in}}{\pgfqpoint{0.619698in}{1.027278in}}%
\pgfpathcurveto{\pgfqpoint{0.629544in}{1.037123in}}{\pgfqpoint{0.639389in}{1.046968in}}{\pgfqpoint{0.652744in}{1.052500in}}%
\pgfpathcurveto{\pgfqpoint{0.666667in}{1.052500in}}{\pgfqpoint{0.680590in}{1.052500in}}{\pgfqpoint{0.693945in}{1.046968in}}%
\pgfpathcurveto{\pgfqpoint{0.703790in}{1.037123in}}{\pgfqpoint{0.713635in}{1.027278in}}{\pgfqpoint{0.719167in}{1.013923in}}%
\pgfpathcurveto{\pgfqpoint{0.719167in}{1.000000in}}{\pgfqpoint{0.719167in}{0.986077in}}{\pgfqpoint{0.713635in}{0.972722in}}%
\pgfpathcurveto{\pgfqpoint{0.703790in}{0.962877in}}{\pgfqpoint{0.693945in}{0.953032in}}{\pgfqpoint{0.680590in}{0.947500in}}%
\pgfpathclose%
\pgfpathmoveto{\pgfqpoint{0.833333in}{0.941667in}}%
\pgfpathcurveto{\pgfqpoint{0.848804in}{0.941667in}}{\pgfqpoint{0.863642in}{0.947813in}}{\pgfqpoint{0.874581in}{0.958752in}}%
\pgfpathcurveto{\pgfqpoint{0.885520in}{0.969691in}}{\pgfqpoint{0.891667in}{0.984530in}}{\pgfqpoint{0.891667in}{1.000000in}}%
\pgfpathcurveto{\pgfqpoint{0.891667in}{1.015470in}}{\pgfqpoint{0.885520in}{1.030309in}}{\pgfqpoint{0.874581in}{1.041248in}}%
\pgfpathcurveto{\pgfqpoint{0.863642in}{1.052187in}}{\pgfqpoint{0.848804in}{1.058333in}}{\pgfqpoint{0.833333in}{1.058333in}}%
\pgfpathcurveto{\pgfqpoint{0.817863in}{1.058333in}}{\pgfqpoint{0.803025in}{1.052187in}}{\pgfqpoint{0.792085in}{1.041248in}}%
\pgfpathcurveto{\pgfqpoint{0.781146in}{1.030309in}}{\pgfqpoint{0.775000in}{1.015470in}}{\pgfqpoint{0.775000in}{1.000000in}}%
\pgfpathcurveto{\pgfqpoint{0.775000in}{0.984530in}}{\pgfqpoint{0.781146in}{0.969691in}}{\pgfqpoint{0.792085in}{0.958752in}}%
\pgfpathcurveto{\pgfqpoint{0.803025in}{0.947813in}}{\pgfqpoint{0.817863in}{0.941667in}}{\pgfqpoint{0.833333in}{0.941667in}}%
\pgfpathclose%
\pgfpathmoveto{\pgfqpoint{0.833333in}{0.947500in}}%
\pgfpathcurveto{\pgfqpoint{0.833333in}{0.947500in}}{\pgfqpoint{0.819410in}{0.947500in}}{\pgfqpoint{0.806055in}{0.953032in}}%
\pgfpathcurveto{\pgfqpoint{0.796210in}{0.962877in}}{\pgfqpoint{0.786365in}{0.972722in}}{\pgfqpoint{0.780833in}{0.986077in}}%
\pgfpathcurveto{\pgfqpoint{0.780833in}{1.000000in}}{\pgfqpoint{0.780833in}{1.013923in}}{\pgfqpoint{0.786365in}{1.027278in}}%
\pgfpathcurveto{\pgfqpoint{0.796210in}{1.037123in}}{\pgfqpoint{0.806055in}{1.046968in}}{\pgfqpoint{0.819410in}{1.052500in}}%
\pgfpathcurveto{\pgfqpoint{0.833333in}{1.052500in}}{\pgfqpoint{0.847256in}{1.052500in}}{\pgfqpoint{0.860611in}{1.046968in}}%
\pgfpathcurveto{\pgfqpoint{0.870456in}{1.037123in}}{\pgfqpoint{0.880302in}{1.027278in}}{\pgfqpoint{0.885833in}{1.013923in}}%
\pgfpathcurveto{\pgfqpoint{0.885833in}{1.000000in}}{\pgfqpoint{0.885833in}{0.986077in}}{\pgfqpoint{0.880302in}{0.972722in}}%
\pgfpathcurveto{\pgfqpoint{0.870456in}{0.962877in}}{\pgfqpoint{0.860611in}{0.953032in}}{\pgfqpoint{0.847256in}{0.947500in}}%
\pgfpathclose%
\pgfpathmoveto{\pgfqpoint{1.000000in}{0.941667in}}%
\pgfpathcurveto{\pgfqpoint{1.015470in}{0.941667in}}{\pgfqpoint{1.030309in}{0.947813in}}{\pgfqpoint{1.041248in}{0.958752in}}%
\pgfpathcurveto{\pgfqpoint{1.052187in}{0.969691in}}{\pgfqpoint{1.058333in}{0.984530in}}{\pgfqpoint{1.058333in}{1.000000in}}%
\pgfpathcurveto{\pgfqpoint{1.058333in}{1.015470in}}{\pgfqpoint{1.052187in}{1.030309in}}{\pgfqpoint{1.041248in}{1.041248in}}%
\pgfpathcurveto{\pgfqpoint{1.030309in}{1.052187in}}{\pgfqpoint{1.015470in}{1.058333in}}{\pgfqpoint{1.000000in}{1.058333in}}%
\pgfpathcurveto{\pgfqpoint{0.984530in}{1.058333in}}{\pgfqpoint{0.969691in}{1.052187in}}{\pgfqpoint{0.958752in}{1.041248in}}%
\pgfpathcurveto{\pgfqpoint{0.947813in}{1.030309in}}{\pgfqpoint{0.941667in}{1.015470in}}{\pgfqpoint{0.941667in}{1.000000in}}%
\pgfpathcurveto{\pgfqpoint{0.941667in}{0.984530in}}{\pgfqpoint{0.947813in}{0.969691in}}{\pgfqpoint{0.958752in}{0.958752in}}%
\pgfpathcurveto{\pgfqpoint{0.969691in}{0.947813in}}{\pgfqpoint{0.984530in}{0.941667in}}{\pgfqpoint{1.000000in}{0.941667in}}%
\pgfpathclose%
\pgfpathmoveto{\pgfqpoint{1.000000in}{0.947500in}}%
\pgfpathcurveto{\pgfqpoint{1.000000in}{0.947500in}}{\pgfqpoint{0.986077in}{0.947500in}}{\pgfqpoint{0.972722in}{0.953032in}}%
\pgfpathcurveto{\pgfqpoint{0.962877in}{0.962877in}}{\pgfqpoint{0.953032in}{0.972722in}}{\pgfqpoint{0.947500in}{0.986077in}}%
\pgfpathcurveto{\pgfqpoint{0.947500in}{1.000000in}}{\pgfqpoint{0.947500in}{1.013923in}}{\pgfqpoint{0.953032in}{1.027278in}}%
\pgfpathcurveto{\pgfqpoint{0.962877in}{1.037123in}}{\pgfqpoint{0.972722in}{1.046968in}}{\pgfqpoint{0.986077in}{1.052500in}}%
\pgfpathcurveto{\pgfqpoint{1.000000in}{1.052500in}}{\pgfqpoint{1.013923in}{1.052500in}}{\pgfqpoint{1.027278in}{1.046968in}}%
\pgfpathcurveto{\pgfqpoint{1.037123in}{1.037123in}}{\pgfqpoint{1.046968in}{1.027278in}}{\pgfqpoint{1.052500in}{1.013923in}}%
\pgfpathcurveto{\pgfqpoint{1.052500in}{1.000000in}}{\pgfqpoint{1.052500in}{0.986077in}}{\pgfqpoint{1.046968in}{0.972722in}}%
\pgfpathcurveto{\pgfqpoint{1.037123in}{0.962877in}}{\pgfqpoint{1.027278in}{0.953032in}}{\pgfqpoint{1.013923in}{0.947500in}}%
\pgfpathclose%
\pgfusepath{stroke}%
\end{pgfscope}%
}%
\pgfsys@transformshift{1.323315in}{1.426245in}%
\pgfsys@useobject{currentpattern}{}%
\pgfsys@transformshift{1in}{0in}%
\pgfsys@transformshift{-1in}{0in}%
\pgfsys@transformshift{0in}{1in}%
\pgfsys@useobject{currentpattern}{}%
\pgfsys@transformshift{1in}{0in}%
\pgfsys@transformshift{-1in}{0in}%
\pgfsys@transformshift{0in}{1in}%
\end{pgfscope}%
\begin{pgfscope}%
\pgfpathrectangle{\pgfqpoint{0.935815in}{0.637495in}}{\pgfqpoint{9.300000in}{9.060000in}}%
\pgfusepath{clip}%
\pgfsetbuttcap%
\pgfsetmiterjoin%
\definecolor{currentfill}{rgb}{0.549020,0.337255,0.294118}%
\pgfsetfillcolor{currentfill}%
\pgfsetfillopacity{0.990000}%
\pgfsetlinewidth{0.000000pt}%
\definecolor{currentstroke}{rgb}{0.000000,0.000000,0.000000}%
\pgfsetstrokecolor{currentstroke}%
\pgfsetstrokeopacity{0.990000}%
\pgfsetdash{}{0pt}%
\pgfpathmoveto{\pgfqpoint{2.873315in}{2.180049in}}%
\pgfpathlineto{\pgfqpoint{3.648315in}{2.180049in}}%
\pgfpathlineto{\pgfqpoint{3.648315in}{3.551179in}}%
\pgfpathlineto{\pgfqpoint{2.873315in}{3.551179in}}%
\pgfpathclose%
\pgfusepath{fill}%
\end{pgfscope}%
\begin{pgfscope}%
\pgfsetbuttcap%
\pgfsetmiterjoin%
\definecolor{currentfill}{rgb}{0.549020,0.337255,0.294118}%
\pgfsetfillcolor{currentfill}%
\pgfsetfillopacity{0.990000}%
\pgfsetlinewidth{0.000000pt}%
\definecolor{currentstroke}{rgb}{0.000000,0.000000,0.000000}%
\pgfsetstrokecolor{currentstroke}%
\pgfsetstrokeopacity{0.990000}%
\pgfsetdash{}{0pt}%
\pgfpathrectangle{\pgfqpoint{0.935815in}{0.637495in}}{\pgfqpoint{9.300000in}{9.060000in}}%
\pgfusepath{clip}%
\pgfpathmoveto{\pgfqpoint{2.873315in}{2.180049in}}%
\pgfpathlineto{\pgfqpoint{3.648315in}{2.180049in}}%
\pgfpathlineto{\pgfqpoint{3.648315in}{3.551179in}}%
\pgfpathlineto{\pgfqpoint{2.873315in}{3.551179in}}%
\pgfpathclose%
\pgfusepath{clip}%
\pgfsys@defobject{currentpattern}{\pgfqpoint{0in}{0in}}{\pgfqpoint{1in}{1in}}{%
\begin{pgfscope}%
\pgfpathrectangle{\pgfqpoint{0in}{0in}}{\pgfqpoint{1in}{1in}}%
\pgfusepath{clip}%
\pgfpathmoveto{\pgfqpoint{0.000000in}{-0.058333in}}%
\pgfpathcurveto{\pgfqpoint{0.015470in}{-0.058333in}}{\pgfqpoint{0.030309in}{-0.052187in}}{\pgfqpoint{0.041248in}{-0.041248in}}%
\pgfpathcurveto{\pgfqpoint{0.052187in}{-0.030309in}}{\pgfqpoint{0.058333in}{-0.015470in}}{\pgfqpoint{0.058333in}{0.000000in}}%
\pgfpathcurveto{\pgfqpoint{0.058333in}{0.015470in}}{\pgfqpoint{0.052187in}{0.030309in}}{\pgfqpoint{0.041248in}{0.041248in}}%
\pgfpathcurveto{\pgfqpoint{0.030309in}{0.052187in}}{\pgfqpoint{0.015470in}{0.058333in}}{\pgfqpoint{0.000000in}{0.058333in}}%
\pgfpathcurveto{\pgfqpoint{-0.015470in}{0.058333in}}{\pgfqpoint{-0.030309in}{0.052187in}}{\pgfqpoint{-0.041248in}{0.041248in}}%
\pgfpathcurveto{\pgfqpoint{-0.052187in}{0.030309in}}{\pgfqpoint{-0.058333in}{0.015470in}}{\pgfqpoint{-0.058333in}{0.000000in}}%
\pgfpathcurveto{\pgfqpoint{-0.058333in}{-0.015470in}}{\pgfqpoint{-0.052187in}{-0.030309in}}{\pgfqpoint{-0.041248in}{-0.041248in}}%
\pgfpathcurveto{\pgfqpoint{-0.030309in}{-0.052187in}}{\pgfqpoint{-0.015470in}{-0.058333in}}{\pgfqpoint{0.000000in}{-0.058333in}}%
\pgfpathclose%
\pgfpathmoveto{\pgfqpoint{0.000000in}{-0.052500in}}%
\pgfpathcurveto{\pgfqpoint{0.000000in}{-0.052500in}}{\pgfqpoint{-0.013923in}{-0.052500in}}{\pgfqpoint{-0.027278in}{-0.046968in}}%
\pgfpathcurveto{\pgfqpoint{-0.037123in}{-0.037123in}}{\pgfqpoint{-0.046968in}{-0.027278in}}{\pgfqpoint{-0.052500in}{-0.013923in}}%
\pgfpathcurveto{\pgfqpoint{-0.052500in}{0.000000in}}{\pgfqpoint{-0.052500in}{0.013923in}}{\pgfqpoint{-0.046968in}{0.027278in}}%
\pgfpathcurveto{\pgfqpoint{-0.037123in}{0.037123in}}{\pgfqpoint{-0.027278in}{0.046968in}}{\pgfqpoint{-0.013923in}{0.052500in}}%
\pgfpathcurveto{\pgfqpoint{0.000000in}{0.052500in}}{\pgfqpoint{0.013923in}{0.052500in}}{\pgfqpoint{0.027278in}{0.046968in}}%
\pgfpathcurveto{\pgfqpoint{0.037123in}{0.037123in}}{\pgfqpoint{0.046968in}{0.027278in}}{\pgfqpoint{0.052500in}{0.013923in}}%
\pgfpathcurveto{\pgfqpoint{0.052500in}{0.000000in}}{\pgfqpoint{0.052500in}{-0.013923in}}{\pgfqpoint{0.046968in}{-0.027278in}}%
\pgfpathcurveto{\pgfqpoint{0.037123in}{-0.037123in}}{\pgfqpoint{0.027278in}{-0.046968in}}{\pgfqpoint{0.013923in}{-0.052500in}}%
\pgfpathclose%
\pgfpathmoveto{\pgfqpoint{0.166667in}{-0.058333in}}%
\pgfpathcurveto{\pgfqpoint{0.182137in}{-0.058333in}}{\pgfqpoint{0.196975in}{-0.052187in}}{\pgfqpoint{0.207915in}{-0.041248in}}%
\pgfpathcurveto{\pgfqpoint{0.218854in}{-0.030309in}}{\pgfqpoint{0.225000in}{-0.015470in}}{\pgfqpoint{0.225000in}{0.000000in}}%
\pgfpathcurveto{\pgfqpoint{0.225000in}{0.015470in}}{\pgfqpoint{0.218854in}{0.030309in}}{\pgfqpoint{0.207915in}{0.041248in}}%
\pgfpathcurveto{\pgfqpoint{0.196975in}{0.052187in}}{\pgfqpoint{0.182137in}{0.058333in}}{\pgfqpoint{0.166667in}{0.058333in}}%
\pgfpathcurveto{\pgfqpoint{0.151196in}{0.058333in}}{\pgfqpoint{0.136358in}{0.052187in}}{\pgfqpoint{0.125419in}{0.041248in}}%
\pgfpathcurveto{\pgfqpoint{0.114480in}{0.030309in}}{\pgfqpoint{0.108333in}{0.015470in}}{\pgfqpoint{0.108333in}{0.000000in}}%
\pgfpathcurveto{\pgfqpoint{0.108333in}{-0.015470in}}{\pgfqpoint{0.114480in}{-0.030309in}}{\pgfqpoint{0.125419in}{-0.041248in}}%
\pgfpathcurveto{\pgfqpoint{0.136358in}{-0.052187in}}{\pgfqpoint{0.151196in}{-0.058333in}}{\pgfqpoint{0.166667in}{-0.058333in}}%
\pgfpathclose%
\pgfpathmoveto{\pgfqpoint{0.166667in}{-0.052500in}}%
\pgfpathcurveto{\pgfqpoint{0.166667in}{-0.052500in}}{\pgfqpoint{0.152744in}{-0.052500in}}{\pgfqpoint{0.139389in}{-0.046968in}}%
\pgfpathcurveto{\pgfqpoint{0.129544in}{-0.037123in}}{\pgfqpoint{0.119698in}{-0.027278in}}{\pgfqpoint{0.114167in}{-0.013923in}}%
\pgfpathcurveto{\pgfqpoint{0.114167in}{0.000000in}}{\pgfqpoint{0.114167in}{0.013923in}}{\pgfqpoint{0.119698in}{0.027278in}}%
\pgfpathcurveto{\pgfqpoint{0.129544in}{0.037123in}}{\pgfqpoint{0.139389in}{0.046968in}}{\pgfqpoint{0.152744in}{0.052500in}}%
\pgfpathcurveto{\pgfqpoint{0.166667in}{0.052500in}}{\pgfqpoint{0.180590in}{0.052500in}}{\pgfqpoint{0.193945in}{0.046968in}}%
\pgfpathcurveto{\pgfqpoint{0.203790in}{0.037123in}}{\pgfqpoint{0.213635in}{0.027278in}}{\pgfqpoint{0.219167in}{0.013923in}}%
\pgfpathcurveto{\pgfqpoint{0.219167in}{0.000000in}}{\pgfqpoint{0.219167in}{-0.013923in}}{\pgfqpoint{0.213635in}{-0.027278in}}%
\pgfpathcurveto{\pgfqpoint{0.203790in}{-0.037123in}}{\pgfqpoint{0.193945in}{-0.046968in}}{\pgfqpoint{0.180590in}{-0.052500in}}%
\pgfpathclose%
\pgfpathmoveto{\pgfqpoint{0.333333in}{-0.058333in}}%
\pgfpathcurveto{\pgfqpoint{0.348804in}{-0.058333in}}{\pgfqpoint{0.363642in}{-0.052187in}}{\pgfqpoint{0.374581in}{-0.041248in}}%
\pgfpathcurveto{\pgfqpoint{0.385520in}{-0.030309in}}{\pgfqpoint{0.391667in}{-0.015470in}}{\pgfqpoint{0.391667in}{0.000000in}}%
\pgfpathcurveto{\pgfqpoint{0.391667in}{0.015470in}}{\pgfqpoint{0.385520in}{0.030309in}}{\pgfqpoint{0.374581in}{0.041248in}}%
\pgfpathcurveto{\pgfqpoint{0.363642in}{0.052187in}}{\pgfqpoint{0.348804in}{0.058333in}}{\pgfqpoint{0.333333in}{0.058333in}}%
\pgfpathcurveto{\pgfqpoint{0.317863in}{0.058333in}}{\pgfqpoint{0.303025in}{0.052187in}}{\pgfqpoint{0.292085in}{0.041248in}}%
\pgfpathcurveto{\pgfqpoint{0.281146in}{0.030309in}}{\pgfqpoint{0.275000in}{0.015470in}}{\pgfqpoint{0.275000in}{0.000000in}}%
\pgfpathcurveto{\pgfqpoint{0.275000in}{-0.015470in}}{\pgfqpoint{0.281146in}{-0.030309in}}{\pgfqpoint{0.292085in}{-0.041248in}}%
\pgfpathcurveto{\pgfqpoint{0.303025in}{-0.052187in}}{\pgfqpoint{0.317863in}{-0.058333in}}{\pgfqpoint{0.333333in}{-0.058333in}}%
\pgfpathclose%
\pgfpathmoveto{\pgfqpoint{0.333333in}{-0.052500in}}%
\pgfpathcurveto{\pgfqpoint{0.333333in}{-0.052500in}}{\pgfqpoint{0.319410in}{-0.052500in}}{\pgfqpoint{0.306055in}{-0.046968in}}%
\pgfpathcurveto{\pgfqpoint{0.296210in}{-0.037123in}}{\pgfqpoint{0.286365in}{-0.027278in}}{\pgfqpoint{0.280833in}{-0.013923in}}%
\pgfpathcurveto{\pgfqpoint{0.280833in}{0.000000in}}{\pgfqpoint{0.280833in}{0.013923in}}{\pgfqpoint{0.286365in}{0.027278in}}%
\pgfpathcurveto{\pgfqpoint{0.296210in}{0.037123in}}{\pgfqpoint{0.306055in}{0.046968in}}{\pgfqpoint{0.319410in}{0.052500in}}%
\pgfpathcurveto{\pgfqpoint{0.333333in}{0.052500in}}{\pgfqpoint{0.347256in}{0.052500in}}{\pgfqpoint{0.360611in}{0.046968in}}%
\pgfpathcurveto{\pgfqpoint{0.370456in}{0.037123in}}{\pgfqpoint{0.380302in}{0.027278in}}{\pgfqpoint{0.385833in}{0.013923in}}%
\pgfpathcurveto{\pgfqpoint{0.385833in}{0.000000in}}{\pgfqpoint{0.385833in}{-0.013923in}}{\pgfqpoint{0.380302in}{-0.027278in}}%
\pgfpathcurveto{\pgfqpoint{0.370456in}{-0.037123in}}{\pgfqpoint{0.360611in}{-0.046968in}}{\pgfqpoint{0.347256in}{-0.052500in}}%
\pgfpathclose%
\pgfpathmoveto{\pgfqpoint{0.500000in}{-0.058333in}}%
\pgfpathcurveto{\pgfqpoint{0.515470in}{-0.058333in}}{\pgfqpoint{0.530309in}{-0.052187in}}{\pgfqpoint{0.541248in}{-0.041248in}}%
\pgfpathcurveto{\pgfqpoint{0.552187in}{-0.030309in}}{\pgfqpoint{0.558333in}{-0.015470in}}{\pgfqpoint{0.558333in}{0.000000in}}%
\pgfpathcurveto{\pgfqpoint{0.558333in}{0.015470in}}{\pgfqpoint{0.552187in}{0.030309in}}{\pgfqpoint{0.541248in}{0.041248in}}%
\pgfpathcurveto{\pgfqpoint{0.530309in}{0.052187in}}{\pgfqpoint{0.515470in}{0.058333in}}{\pgfqpoint{0.500000in}{0.058333in}}%
\pgfpathcurveto{\pgfqpoint{0.484530in}{0.058333in}}{\pgfqpoint{0.469691in}{0.052187in}}{\pgfqpoint{0.458752in}{0.041248in}}%
\pgfpathcurveto{\pgfqpoint{0.447813in}{0.030309in}}{\pgfqpoint{0.441667in}{0.015470in}}{\pgfqpoint{0.441667in}{0.000000in}}%
\pgfpathcurveto{\pgfqpoint{0.441667in}{-0.015470in}}{\pgfqpoint{0.447813in}{-0.030309in}}{\pgfqpoint{0.458752in}{-0.041248in}}%
\pgfpathcurveto{\pgfqpoint{0.469691in}{-0.052187in}}{\pgfqpoint{0.484530in}{-0.058333in}}{\pgfqpoint{0.500000in}{-0.058333in}}%
\pgfpathclose%
\pgfpathmoveto{\pgfqpoint{0.500000in}{-0.052500in}}%
\pgfpathcurveto{\pgfqpoint{0.500000in}{-0.052500in}}{\pgfqpoint{0.486077in}{-0.052500in}}{\pgfqpoint{0.472722in}{-0.046968in}}%
\pgfpathcurveto{\pgfqpoint{0.462877in}{-0.037123in}}{\pgfqpoint{0.453032in}{-0.027278in}}{\pgfqpoint{0.447500in}{-0.013923in}}%
\pgfpathcurveto{\pgfqpoint{0.447500in}{0.000000in}}{\pgfqpoint{0.447500in}{0.013923in}}{\pgfqpoint{0.453032in}{0.027278in}}%
\pgfpathcurveto{\pgfqpoint{0.462877in}{0.037123in}}{\pgfqpoint{0.472722in}{0.046968in}}{\pgfqpoint{0.486077in}{0.052500in}}%
\pgfpathcurveto{\pgfqpoint{0.500000in}{0.052500in}}{\pgfqpoint{0.513923in}{0.052500in}}{\pgfqpoint{0.527278in}{0.046968in}}%
\pgfpathcurveto{\pgfqpoint{0.537123in}{0.037123in}}{\pgfqpoint{0.546968in}{0.027278in}}{\pgfqpoint{0.552500in}{0.013923in}}%
\pgfpathcurveto{\pgfqpoint{0.552500in}{0.000000in}}{\pgfqpoint{0.552500in}{-0.013923in}}{\pgfqpoint{0.546968in}{-0.027278in}}%
\pgfpathcurveto{\pgfqpoint{0.537123in}{-0.037123in}}{\pgfqpoint{0.527278in}{-0.046968in}}{\pgfqpoint{0.513923in}{-0.052500in}}%
\pgfpathclose%
\pgfpathmoveto{\pgfqpoint{0.666667in}{-0.058333in}}%
\pgfpathcurveto{\pgfqpoint{0.682137in}{-0.058333in}}{\pgfqpoint{0.696975in}{-0.052187in}}{\pgfqpoint{0.707915in}{-0.041248in}}%
\pgfpathcurveto{\pgfqpoint{0.718854in}{-0.030309in}}{\pgfqpoint{0.725000in}{-0.015470in}}{\pgfqpoint{0.725000in}{0.000000in}}%
\pgfpathcurveto{\pgfqpoint{0.725000in}{0.015470in}}{\pgfqpoint{0.718854in}{0.030309in}}{\pgfqpoint{0.707915in}{0.041248in}}%
\pgfpathcurveto{\pgfqpoint{0.696975in}{0.052187in}}{\pgfqpoint{0.682137in}{0.058333in}}{\pgfqpoint{0.666667in}{0.058333in}}%
\pgfpathcurveto{\pgfqpoint{0.651196in}{0.058333in}}{\pgfqpoint{0.636358in}{0.052187in}}{\pgfqpoint{0.625419in}{0.041248in}}%
\pgfpathcurveto{\pgfqpoint{0.614480in}{0.030309in}}{\pgfqpoint{0.608333in}{0.015470in}}{\pgfqpoint{0.608333in}{0.000000in}}%
\pgfpathcurveto{\pgfqpoint{0.608333in}{-0.015470in}}{\pgfqpoint{0.614480in}{-0.030309in}}{\pgfqpoint{0.625419in}{-0.041248in}}%
\pgfpathcurveto{\pgfqpoint{0.636358in}{-0.052187in}}{\pgfqpoint{0.651196in}{-0.058333in}}{\pgfqpoint{0.666667in}{-0.058333in}}%
\pgfpathclose%
\pgfpathmoveto{\pgfqpoint{0.666667in}{-0.052500in}}%
\pgfpathcurveto{\pgfqpoint{0.666667in}{-0.052500in}}{\pgfqpoint{0.652744in}{-0.052500in}}{\pgfqpoint{0.639389in}{-0.046968in}}%
\pgfpathcurveto{\pgfqpoint{0.629544in}{-0.037123in}}{\pgfqpoint{0.619698in}{-0.027278in}}{\pgfqpoint{0.614167in}{-0.013923in}}%
\pgfpathcurveto{\pgfqpoint{0.614167in}{0.000000in}}{\pgfqpoint{0.614167in}{0.013923in}}{\pgfqpoint{0.619698in}{0.027278in}}%
\pgfpathcurveto{\pgfqpoint{0.629544in}{0.037123in}}{\pgfqpoint{0.639389in}{0.046968in}}{\pgfqpoint{0.652744in}{0.052500in}}%
\pgfpathcurveto{\pgfqpoint{0.666667in}{0.052500in}}{\pgfqpoint{0.680590in}{0.052500in}}{\pgfqpoint{0.693945in}{0.046968in}}%
\pgfpathcurveto{\pgfqpoint{0.703790in}{0.037123in}}{\pgfqpoint{0.713635in}{0.027278in}}{\pgfqpoint{0.719167in}{0.013923in}}%
\pgfpathcurveto{\pgfqpoint{0.719167in}{0.000000in}}{\pgfqpoint{0.719167in}{-0.013923in}}{\pgfqpoint{0.713635in}{-0.027278in}}%
\pgfpathcurveto{\pgfqpoint{0.703790in}{-0.037123in}}{\pgfqpoint{0.693945in}{-0.046968in}}{\pgfqpoint{0.680590in}{-0.052500in}}%
\pgfpathclose%
\pgfpathmoveto{\pgfqpoint{0.833333in}{-0.058333in}}%
\pgfpathcurveto{\pgfqpoint{0.848804in}{-0.058333in}}{\pgfqpoint{0.863642in}{-0.052187in}}{\pgfqpoint{0.874581in}{-0.041248in}}%
\pgfpathcurveto{\pgfqpoint{0.885520in}{-0.030309in}}{\pgfqpoint{0.891667in}{-0.015470in}}{\pgfqpoint{0.891667in}{0.000000in}}%
\pgfpathcurveto{\pgfqpoint{0.891667in}{0.015470in}}{\pgfqpoint{0.885520in}{0.030309in}}{\pgfqpoint{0.874581in}{0.041248in}}%
\pgfpathcurveto{\pgfqpoint{0.863642in}{0.052187in}}{\pgfqpoint{0.848804in}{0.058333in}}{\pgfqpoint{0.833333in}{0.058333in}}%
\pgfpathcurveto{\pgfqpoint{0.817863in}{0.058333in}}{\pgfqpoint{0.803025in}{0.052187in}}{\pgfqpoint{0.792085in}{0.041248in}}%
\pgfpathcurveto{\pgfqpoint{0.781146in}{0.030309in}}{\pgfqpoint{0.775000in}{0.015470in}}{\pgfqpoint{0.775000in}{0.000000in}}%
\pgfpathcurveto{\pgfqpoint{0.775000in}{-0.015470in}}{\pgfqpoint{0.781146in}{-0.030309in}}{\pgfqpoint{0.792085in}{-0.041248in}}%
\pgfpathcurveto{\pgfqpoint{0.803025in}{-0.052187in}}{\pgfqpoint{0.817863in}{-0.058333in}}{\pgfqpoint{0.833333in}{-0.058333in}}%
\pgfpathclose%
\pgfpathmoveto{\pgfqpoint{0.833333in}{-0.052500in}}%
\pgfpathcurveto{\pgfqpoint{0.833333in}{-0.052500in}}{\pgfqpoint{0.819410in}{-0.052500in}}{\pgfqpoint{0.806055in}{-0.046968in}}%
\pgfpathcurveto{\pgfqpoint{0.796210in}{-0.037123in}}{\pgfqpoint{0.786365in}{-0.027278in}}{\pgfqpoint{0.780833in}{-0.013923in}}%
\pgfpathcurveto{\pgfqpoint{0.780833in}{0.000000in}}{\pgfqpoint{0.780833in}{0.013923in}}{\pgfqpoint{0.786365in}{0.027278in}}%
\pgfpathcurveto{\pgfqpoint{0.796210in}{0.037123in}}{\pgfqpoint{0.806055in}{0.046968in}}{\pgfqpoint{0.819410in}{0.052500in}}%
\pgfpathcurveto{\pgfqpoint{0.833333in}{0.052500in}}{\pgfqpoint{0.847256in}{0.052500in}}{\pgfqpoint{0.860611in}{0.046968in}}%
\pgfpathcurveto{\pgfqpoint{0.870456in}{0.037123in}}{\pgfqpoint{0.880302in}{0.027278in}}{\pgfqpoint{0.885833in}{0.013923in}}%
\pgfpathcurveto{\pgfqpoint{0.885833in}{0.000000in}}{\pgfqpoint{0.885833in}{-0.013923in}}{\pgfqpoint{0.880302in}{-0.027278in}}%
\pgfpathcurveto{\pgfqpoint{0.870456in}{-0.037123in}}{\pgfqpoint{0.860611in}{-0.046968in}}{\pgfqpoint{0.847256in}{-0.052500in}}%
\pgfpathclose%
\pgfpathmoveto{\pgfqpoint{1.000000in}{-0.058333in}}%
\pgfpathcurveto{\pgfqpoint{1.015470in}{-0.058333in}}{\pgfqpoint{1.030309in}{-0.052187in}}{\pgfqpoint{1.041248in}{-0.041248in}}%
\pgfpathcurveto{\pgfqpoint{1.052187in}{-0.030309in}}{\pgfqpoint{1.058333in}{-0.015470in}}{\pgfqpoint{1.058333in}{0.000000in}}%
\pgfpathcurveto{\pgfqpoint{1.058333in}{0.015470in}}{\pgfqpoint{1.052187in}{0.030309in}}{\pgfqpoint{1.041248in}{0.041248in}}%
\pgfpathcurveto{\pgfqpoint{1.030309in}{0.052187in}}{\pgfqpoint{1.015470in}{0.058333in}}{\pgfqpoint{1.000000in}{0.058333in}}%
\pgfpathcurveto{\pgfqpoint{0.984530in}{0.058333in}}{\pgfqpoint{0.969691in}{0.052187in}}{\pgfqpoint{0.958752in}{0.041248in}}%
\pgfpathcurveto{\pgfqpoint{0.947813in}{0.030309in}}{\pgfqpoint{0.941667in}{0.015470in}}{\pgfqpoint{0.941667in}{0.000000in}}%
\pgfpathcurveto{\pgfqpoint{0.941667in}{-0.015470in}}{\pgfqpoint{0.947813in}{-0.030309in}}{\pgfqpoint{0.958752in}{-0.041248in}}%
\pgfpathcurveto{\pgfqpoint{0.969691in}{-0.052187in}}{\pgfqpoint{0.984530in}{-0.058333in}}{\pgfqpoint{1.000000in}{-0.058333in}}%
\pgfpathclose%
\pgfpathmoveto{\pgfqpoint{1.000000in}{-0.052500in}}%
\pgfpathcurveto{\pgfqpoint{1.000000in}{-0.052500in}}{\pgfqpoint{0.986077in}{-0.052500in}}{\pgfqpoint{0.972722in}{-0.046968in}}%
\pgfpathcurveto{\pgfqpoint{0.962877in}{-0.037123in}}{\pgfqpoint{0.953032in}{-0.027278in}}{\pgfqpoint{0.947500in}{-0.013923in}}%
\pgfpathcurveto{\pgfqpoint{0.947500in}{0.000000in}}{\pgfqpoint{0.947500in}{0.013923in}}{\pgfqpoint{0.953032in}{0.027278in}}%
\pgfpathcurveto{\pgfqpoint{0.962877in}{0.037123in}}{\pgfqpoint{0.972722in}{0.046968in}}{\pgfqpoint{0.986077in}{0.052500in}}%
\pgfpathcurveto{\pgfqpoint{1.000000in}{0.052500in}}{\pgfqpoint{1.013923in}{0.052500in}}{\pgfqpoint{1.027278in}{0.046968in}}%
\pgfpathcurveto{\pgfqpoint{1.037123in}{0.037123in}}{\pgfqpoint{1.046968in}{0.027278in}}{\pgfqpoint{1.052500in}{0.013923in}}%
\pgfpathcurveto{\pgfqpoint{1.052500in}{0.000000in}}{\pgfqpoint{1.052500in}{-0.013923in}}{\pgfqpoint{1.046968in}{-0.027278in}}%
\pgfpathcurveto{\pgfqpoint{1.037123in}{-0.037123in}}{\pgfqpoint{1.027278in}{-0.046968in}}{\pgfqpoint{1.013923in}{-0.052500in}}%
\pgfpathclose%
\pgfpathmoveto{\pgfqpoint{0.083333in}{0.108333in}}%
\pgfpathcurveto{\pgfqpoint{0.098804in}{0.108333in}}{\pgfqpoint{0.113642in}{0.114480in}}{\pgfqpoint{0.124581in}{0.125419in}}%
\pgfpathcurveto{\pgfqpoint{0.135520in}{0.136358in}}{\pgfqpoint{0.141667in}{0.151196in}}{\pgfqpoint{0.141667in}{0.166667in}}%
\pgfpathcurveto{\pgfqpoint{0.141667in}{0.182137in}}{\pgfqpoint{0.135520in}{0.196975in}}{\pgfqpoint{0.124581in}{0.207915in}}%
\pgfpathcurveto{\pgfqpoint{0.113642in}{0.218854in}}{\pgfqpoint{0.098804in}{0.225000in}}{\pgfqpoint{0.083333in}{0.225000in}}%
\pgfpathcurveto{\pgfqpoint{0.067863in}{0.225000in}}{\pgfqpoint{0.053025in}{0.218854in}}{\pgfqpoint{0.042085in}{0.207915in}}%
\pgfpathcurveto{\pgfqpoint{0.031146in}{0.196975in}}{\pgfqpoint{0.025000in}{0.182137in}}{\pgfqpoint{0.025000in}{0.166667in}}%
\pgfpathcurveto{\pgfqpoint{0.025000in}{0.151196in}}{\pgfqpoint{0.031146in}{0.136358in}}{\pgfqpoint{0.042085in}{0.125419in}}%
\pgfpathcurveto{\pgfqpoint{0.053025in}{0.114480in}}{\pgfqpoint{0.067863in}{0.108333in}}{\pgfqpoint{0.083333in}{0.108333in}}%
\pgfpathclose%
\pgfpathmoveto{\pgfqpoint{0.083333in}{0.114167in}}%
\pgfpathcurveto{\pgfqpoint{0.083333in}{0.114167in}}{\pgfqpoint{0.069410in}{0.114167in}}{\pgfqpoint{0.056055in}{0.119698in}}%
\pgfpathcurveto{\pgfqpoint{0.046210in}{0.129544in}}{\pgfqpoint{0.036365in}{0.139389in}}{\pgfqpoint{0.030833in}{0.152744in}}%
\pgfpathcurveto{\pgfqpoint{0.030833in}{0.166667in}}{\pgfqpoint{0.030833in}{0.180590in}}{\pgfqpoint{0.036365in}{0.193945in}}%
\pgfpathcurveto{\pgfqpoint{0.046210in}{0.203790in}}{\pgfqpoint{0.056055in}{0.213635in}}{\pgfqpoint{0.069410in}{0.219167in}}%
\pgfpathcurveto{\pgfqpoint{0.083333in}{0.219167in}}{\pgfqpoint{0.097256in}{0.219167in}}{\pgfqpoint{0.110611in}{0.213635in}}%
\pgfpathcurveto{\pgfqpoint{0.120456in}{0.203790in}}{\pgfqpoint{0.130302in}{0.193945in}}{\pgfqpoint{0.135833in}{0.180590in}}%
\pgfpathcurveto{\pgfqpoint{0.135833in}{0.166667in}}{\pgfqpoint{0.135833in}{0.152744in}}{\pgfqpoint{0.130302in}{0.139389in}}%
\pgfpathcurveto{\pgfqpoint{0.120456in}{0.129544in}}{\pgfqpoint{0.110611in}{0.119698in}}{\pgfqpoint{0.097256in}{0.114167in}}%
\pgfpathclose%
\pgfpathmoveto{\pgfqpoint{0.250000in}{0.108333in}}%
\pgfpathcurveto{\pgfqpoint{0.265470in}{0.108333in}}{\pgfqpoint{0.280309in}{0.114480in}}{\pgfqpoint{0.291248in}{0.125419in}}%
\pgfpathcurveto{\pgfqpoint{0.302187in}{0.136358in}}{\pgfqpoint{0.308333in}{0.151196in}}{\pgfqpoint{0.308333in}{0.166667in}}%
\pgfpathcurveto{\pgfqpoint{0.308333in}{0.182137in}}{\pgfqpoint{0.302187in}{0.196975in}}{\pgfqpoint{0.291248in}{0.207915in}}%
\pgfpathcurveto{\pgfqpoint{0.280309in}{0.218854in}}{\pgfqpoint{0.265470in}{0.225000in}}{\pgfqpoint{0.250000in}{0.225000in}}%
\pgfpathcurveto{\pgfqpoint{0.234530in}{0.225000in}}{\pgfqpoint{0.219691in}{0.218854in}}{\pgfqpoint{0.208752in}{0.207915in}}%
\pgfpathcurveto{\pgfqpoint{0.197813in}{0.196975in}}{\pgfqpoint{0.191667in}{0.182137in}}{\pgfqpoint{0.191667in}{0.166667in}}%
\pgfpathcurveto{\pgfqpoint{0.191667in}{0.151196in}}{\pgfqpoint{0.197813in}{0.136358in}}{\pgfqpoint{0.208752in}{0.125419in}}%
\pgfpathcurveto{\pgfqpoint{0.219691in}{0.114480in}}{\pgfqpoint{0.234530in}{0.108333in}}{\pgfqpoint{0.250000in}{0.108333in}}%
\pgfpathclose%
\pgfpathmoveto{\pgfqpoint{0.250000in}{0.114167in}}%
\pgfpathcurveto{\pgfqpoint{0.250000in}{0.114167in}}{\pgfqpoint{0.236077in}{0.114167in}}{\pgfqpoint{0.222722in}{0.119698in}}%
\pgfpathcurveto{\pgfqpoint{0.212877in}{0.129544in}}{\pgfqpoint{0.203032in}{0.139389in}}{\pgfqpoint{0.197500in}{0.152744in}}%
\pgfpathcurveto{\pgfqpoint{0.197500in}{0.166667in}}{\pgfqpoint{0.197500in}{0.180590in}}{\pgfqpoint{0.203032in}{0.193945in}}%
\pgfpathcurveto{\pgfqpoint{0.212877in}{0.203790in}}{\pgfqpoint{0.222722in}{0.213635in}}{\pgfqpoint{0.236077in}{0.219167in}}%
\pgfpathcurveto{\pgfqpoint{0.250000in}{0.219167in}}{\pgfqpoint{0.263923in}{0.219167in}}{\pgfqpoint{0.277278in}{0.213635in}}%
\pgfpathcurveto{\pgfqpoint{0.287123in}{0.203790in}}{\pgfqpoint{0.296968in}{0.193945in}}{\pgfqpoint{0.302500in}{0.180590in}}%
\pgfpathcurveto{\pgfqpoint{0.302500in}{0.166667in}}{\pgfqpoint{0.302500in}{0.152744in}}{\pgfqpoint{0.296968in}{0.139389in}}%
\pgfpathcurveto{\pgfqpoint{0.287123in}{0.129544in}}{\pgfqpoint{0.277278in}{0.119698in}}{\pgfqpoint{0.263923in}{0.114167in}}%
\pgfpathclose%
\pgfpathmoveto{\pgfqpoint{0.416667in}{0.108333in}}%
\pgfpathcurveto{\pgfqpoint{0.432137in}{0.108333in}}{\pgfqpoint{0.446975in}{0.114480in}}{\pgfqpoint{0.457915in}{0.125419in}}%
\pgfpathcurveto{\pgfqpoint{0.468854in}{0.136358in}}{\pgfqpoint{0.475000in}{0.151196in}}{\pgfqpoint{0.475000in}{0.166667in}}%
\pgfpathcurveto{\pgfqpoint{0.475000in}{0.182137in}}{\pgfqpoint{0.468854in}{0.196975in}}{\pgfqpoint{0.457915in}{0.207915in}}%
\pgfpathcurveto{\pgfqpoint{0.446975in}{0.218854in}}{\pgfqpoint{0.432137in}{0.225000in}}{\pgfqpoint{0.416667in}{0.225000in}}%
\pgfpathcurveto{\pgfqpoint{0.401196in}{0.225000in}}{\pgfqpoint{0.386358in}{0.218854in}}{\pgfqpoint{0.375419in}{0.207915in}}%
\pgfpathcurveto{\pgfqpoint{0.364480in}{0.196975in}}{\pgfqpoint{0.358333in}{0.182137in}}{\pgfqpoint{0.358333in}{0.166667in}}%
\pgfpathcurveto{\pgfqpoint{0.358333in}{0.151196in}}{\pgfqpoint{0.364480in}{0.136358in}}{\pgfqpoint{0.375419in}{0.125419in}}%
\pgfpathcurveto{\pgfqpoint{0.386358in}{0.114480in}}{\pgfqpoint{0.401196in}{0.108333in}}{\pgfqpoint{0.416667in}{0.108333in}}%
\pgfpathclose%
\pgfpathmoveto{\pgfqpoint{0.416667in}{0.114167in}}%
\pgfpathcurveto{\pgfqpoint{0.416667in}{0.114167in}}{\pgfqpoint{0.402744in}{0.114167in}}{\pgfqpoint{0.389389in}{0.119698in}}%
\pgfpathcurveto{\pgfqpoint{0.379544in}{0.129544in}}{\pgfqpoint{0.369698in}{0.139389in}}{\pgfqpoint{0.364167in}{0.152744in}}%
\pgfpathcurveto{\pgfqpoint{0.364167in}{0.166667in}}{\pgfqpoint{0.364167in}{0.180590in}}{\pgfqpoint{0.369698in}{0.193945in}}%
\pgfpathcurveto{\pgfqpoint{0.379544in}{0.203790in}}{\pgfqpoint{0.389389in}{0.213635in}}{\pgfqpoint{0.402744in}{0.219167in}}%
\pgfpathcurveto{\pgfqpoint{0.416667in}{0.219167in}}{\pgfqpoint{0.430590in}{0.219167in}}{\pgfqpoint{0.443945in}{0.213635in}}%
\pgfpathcurveto{\pgfqpoint{0.453790in}{0.203790in}}{\pgfqpoint{0.463635in}{0.193945in}}{\pgfqpoint{0.469167in}{0.180590in}}%
\pgfpathcurveto{\pgfqpoint{0.469167in}{0.166667in}}{\pgfqpoint{0.469167in}{0.152744in}}{\pgfqpoint{0.463635in}{0.139389in}}%
\pgfpathcurveto{\pgfqpoint{0.453790in}{0.129544in}}{\pgfqpoint{0.443945in}{0.119698in}}{\pgfqpoint{0.430590in}{0.114167in}}%
\pgfpathclose%
\pgfpathmoveto{\pgfqpoint{0.583333in}{0.108333in}}%
\pgfpathcurveto{\pgfqpoint{0.598804in}{0.108333in}}{\pgfqpoint{0.613642in}{0.114480in}}{\pgfqpoint{0.624581in}{0.125419in}}%
\pgfpathcurveto{\pgfqpoint{0.635520in}{0.136358in}}{\pgfqpoint{0.641667in}{0.151196in}}{\pgfqpoint{0.641667in}{0.166667in}}%
\pgfpathcurveto{\pgfqpoint{0.641667in}{0.182137in}}{\pgfqpoint{0.635520in}{0.196975in}}{\pgfqpoint{0.624581in}{0.207915in}}%
\pgfpathcurveto{\pgfqpoint{0.613642in}{0.218854in}}{\pgfqpoint{0.598804in}{0.225000in}}{\pgfqpoint{0.583333in}{0.225000in}}%
\pgfpathcurveto{\pgfqpoint{0.567863in}{0.225000in}}{\pgfqpoint{0.553025in}{0.218854in}}{\pgfqpoint{0.542085in}{0.207915in}}%
\pgfpathcurveto{\pgfqpoint{0.531146in}{0.196975in}}{\pgfqpoint{0.525000in}{0.182137in}}{\pgfqpoint{0.525000in}{0.166667in}}%
\pgfpathcurveto{\pgfqpoint{0.525000in}{0.151196in}}{\pgfqpoint{0.531146in}{0.136358in}}{\pgfqpoint{0.542085in}{0.125419in}}%
\pgfpathcurveto{\pgfqpoint{0.553025in}{0.114480in}}{\pgfqpoint{0.567863in}{0.108333in}}{\pgfqpoint{0.583333in}{0.108333in}}%
\pgfpathclose%
\pgfpathmoveto{\pgfqpoint{0.583333in}{0.114167in}}%
\pgfpathcurveto{\pgfqpoint{0.583333in}{0.114167in}}{\pgfqpoint{0.569410in}{0.114167in}}{\pgfqpoint{0.556055in}{0.119698in}}%
\pgfpathcurveto{\pgfqpoint{0.546210in}{0.129544in}}{\pgfqpoint{0.536365in}{0.139389in}}{\pgfqpoint{0.530833in}{0.152744in}}%
\pgfpathcurveto{\pgfqpoint{0.530833in}{0.166667in}}{\pgfqpoint{0.530833in}{0.180590in}}{\pgfqpoint{0.536365in}{0.193945in}}%
\pgfpathcurveto{\pgfqpoint{0.546210in}{0.203790in}}{\pgfqpoint{0.556055in}{0.213635in}}{\pgfqpoint{0.569410in}{0.219167in}}%
\pgfpathcurveto{\pgfqpoint{0.583333in}{0.219167in}}{\pgfqpoint{0.597256in}{0.219167in}}{\pgfqpoint{0.610611in}{0.213635in}}%
\pgfpathcurveto{\pgfqpoint{0.620456in}{0.203790in}}{\pgfqpoint{0.630302in}{0.193945in}}{\pgfqpoint{0.635833in}{0.180590in}}%
\pgfpathcurveto{\pgfqpoint{0.635833in}{0.166667in}}{\pgfqpoint{0.635833in}{0.152744in}}{\pgfqpoint{0.630302in}{0.139389in}}%
\pgfpathcurveto{\pgfqpoint{0.620456in}{0.129544in}}{\pgfqpoint{0.610611in}{0.119698in}}{\pgfqpoint{0.597256in}{0.114167in}}%
\pgfpathclose%
\pgfpathmoveto{\pgfqpoint{0.750000in}{0.108333in}}%
\pgfpathcurveto{\pgfqpoint{0.765470in}{0.108333in}}{\pgfqpoint{0.780309in}{0.114480in}}{\pgfqpoint{0.791248in}{0.125419in}}%
\pgfpathcurveto{\pgfqpoint{0.802187in}{0.136358in}}{\pgfqpoint{0.808333in}{0.151196in}}{\pgfqpoint{0.808333in}{0.166667in}}%
\pgfpathcurveto{\pgfqpoint{0.808333in}{0.182137in}}{\pgfqpoint{0.802187in}{0.196975in}}{\pgfqpoint{0.791248in}{0.207915in}}%
\pgfpathcurveto{\pgfqpoint{0.780309in}{0.218854in}}{\pgfqpoint{0.765470in}{0.225000in}}{\pgfqpoint{0.750000in}{0.225000in}}%
\pgfpathcurveto{\pgfqpoint{0.734530in}{0.225000in}}{\pgfqpoint{0.719691in}{0.218854in}}{\pgfqpoint{0.708752in}{0.207915in}}%
\pgfpathcurveto{\pgfqpoint{0.697813in}{0.196975in}}{\pgfqpoint{0.691667in}{0.182137in}}{\pgfqpoint{0.691667in}{0.166667in}}%
\pgfpathcurveto{\pgfqpoint{0.691667in}{0.151196in}}{\pgfqpoint{0.697813in}{0.136358in}}{\pgfqpoint{0.708752in}{0.125419in}}%
\pgfpathcurveto{\pgfqpoint{0.719691in}{0.114480in}}{\pgfqpoint{0.734530in}{0.108333in}}{\pgfqpoint{0.750000in}{0.108333in}}%
\pgfpathclose%
\pgfpathmoveto{\pgfqpoint{0.750000in}{0.114167in}}%
\pgfpathcurveto{\pgfqpoint{0.750000in}{0.114167in}}{\pgfqpoint{0.736077in}{0.114167in}}{\pgfqpoint{0.722722in}{0.119698in}}%
\pgfpathcurveto{\pgfqpoint{0.712877in}{0.129544in}}{\pgfqpoint{0.703032in}{0.139389in}}{\pgfqpoint{0.697500in}{0.152744in}}%
\pgfpathcurveto{\pgfqpoint{0.697500in}{0.166667in}}{\pgfqpoint{0.697500in}{0.180590in}}{\pgfqpoint{0.703032in}{0.193945in}}%
\pgfpathcurveto{\pgfqpoint{0.712877in}{0.203790in}}{\pgfqpoint{0.722722in}{0.213635in}}{\pgfqpoint{0.736077in}{0.219167in}}%
\pgfpathcurveto{\pgfqpoint{0.750000in}{0.219167in}}{\pgfqpoint{0.763923in}{0.219167in}}{\pgfqpoint{0.777278in}{0.213635in}}%
\pgfpathcurveto{\pgfqpoint{0.787123in}{0.203790in}}{\pgfqpoint{0.796968in}{0.193945in}}{\pgfqpoint{0.802500in}{0.180590in}}%
\pgfpathcurveto{\pgfqpoint{0.802500in}{0.166667in}}{\pgfqpoint{0.802500in}{0.152744in}}{\pgfqpoint{0.796968in}{0.139389in}}%
\pgfpathcurveto{\pgfqpoint{0.787123in}{0.129544in}}{\pgfqpoint{0.777278in}{0.119698in}}{\pgfqpoint{0.763923in}{0.114167in}}%
\pgfpathclose%
\pgfpathmoveto{\pgfqpoint{0.916667in}{0.108333in}}%
\pgfpathcurveto{\pgfqpoint{0.932137in}{0.108333in}}{\pgfqpoint{0.946975in}{0.114480in}}{\pgfqpoint{0.957915in}{0.125419in}}%
\pgfpathcurveto{\pgfqpoint{0.968854in}{0.136358in}}{\pgfqpoint{0.975000in}{0.151196in}}{\pgfqpoint{0.975000in}{0.166667in}}%
\pgfpathcurveto{\pgfqpoint{0.975000in}{0.182137in}}{\pgfqpoint{0.968854in}{0.196975in}}{\pgfqpoint{0.957915in}{0.207915in}}%
\pgfpathcurveto{\pgfqpoint{0.946975in}{0.218854in}}{\pgfqpoint{0.932137in}{0.225000in}}{\pgfqpoint{0.916667in}{0.225000in}}%
\pgfpathcurveto{\pgfqpoint{0.901196in}{0.225000in}}{\pgfqpoint{0.886358in}{0.218854in}}{\pgfqpoint{0.875419in}{0.207915in}}%
\pgfpathcurveto{\pgfqpoint{0.864480in}{0.196975in}}{\pgfqpoint{0.858333in}{0.182137in}}{\pgfqpoint{0.858333in}{0.166667in}}%
\pgfpathcurveto{\pgfqpoint{0.858333in}{0.151196in}}{\pgfqpoint{0.864480in}{0.136358in}}{\pgfqpoint{0.875419in}{0.125419in}}%
\pgfpathcurveto{\pgfqpoint{0.886358in}{0.114480in}}{\pgfqpoint{0.901196in}{0.108333in}}{\pgfqpoint{0.916667in}{0.108333in}}%
\pgfpathclose%
\pgfpathmoveto{\pgfqpoint{0.916667in}{0.114167in}}%
\pgfpathcurveto{\pgfqpoint{0.916667in}{0.114167in}}{\pgfqpoint{0.902744in}{0.114167in}}{\pgfqpoint{0.889389in}{0.119698in}}%
\pgfpathcurveto{\pgfqpoint{0.879544in}{0.129544in}}{\pgfqpoint{0.869698in}{0.139389in}}{\pgfqpoint{0.864167in}{0.152744in}}%
\pgfpathcurveto{\pgfqpoint{0.864167in}{0.166667in}}{\pgfqpoint{0.864167in}{0.180590in}}{\pgfqpoint{0.869698in}{0.193945in}}%
\pgfpathcurveto{\pgfqpoint{0.879544in}{0.203790in}}{\pgfqpoint{0.889389in}{0.213635in}}{\pgfqpoint{0.902744in}{0.219167in}}%
\pgfpathcurveto{\pgfqpoint{0.916667in}{0.219167in}}{\pgfqpoint{0.930590in}{0.219167in}}{\pgfqpoint{0.943945in}{0.213635in}}%
\pgfpathcurveto{\pgfqpoint{0.953790in}{0.203790in}}{\pgfqpoint{0.963635in}{0.193945in}}{\pgfqpoint{0.969167in}{0.180590in}}%
\pgfpathcurveto{\pgfqpoint{0.969167in}{0.166667in}}{\pgfqpoint{0.969167in}{0.152744in}}{\pgfqpoint{0.963635in}{0.139389in}}%
\pgfpathcurveto{\pgfqpoint{0.953790in}{0.129544in}}{\pgfqpoint{0.943945in}{0.119698in}}{\pgfqpoint{0.930590in}{0.114167in}}%
\pgfpathclose%
\pgfpathmoveto{\pgfqpoint{0.000000in}{0.275000in}}%
\pgfpathcurveto{\pgfqpoint{0.015470in}{0.275000in}}{\pgfqpoint{0.030309in}{0.281146in}}{\pgfqpoint{0.041248in}{0.292085in}}%
\pgfpathcurveto{\pgfqpoint{0.052187in}{0.303025in}}{\pgfqpoint{0.058333in}{0.317863in}}{\pgfqpoint{0.058333in}{0.333333in}}%
\pgfpathcurveto{\pgfqpoint{0.058333in}{0.348804in}}{\pgfqpoint{0.052187in}{0.363642in}}{\pgfqpoint{0.041248in}{0.374581in}}%
\pgfpathcurveto{\pgfqpoint{0.030309in}{0.385520in}}{\pgfqpoint{0.015470in}{0.391667in}}{\pgfqpoint{0.000000in}{0.391667in}}%
\pgfpathcurveto{\pgfqpoint{-0.015470in}{0.391667in}}{\pgfqpoint{-0.030309in}{0.385520in}}{\pgfqpoint{-0.041248in}{0.374581in}}%
\pgfpathcurveto{\pgfqpoint{-0.052187in}{0.363642in}}{\pgfqpoint{-0.058333in}{0.348804in}}{\pgfqpoint{-0.058333in}{0.333333in}}%
\pgfpathcurveto{\pgfqpoint{-0.058333in}{0.317863in}}{\pgfqpoint{-0.052187in}{0.303025in}}{\pgfqpoint{-0.041248in}{0.292085in}}%
\pgfpathcurveto{\pgfqpoint{-0.030309in}{0.281146in}}{\pgfqpoint{-0.015470in}{0.275000in}}{\pgfqpoint{0.000000in}{0.275000in}}%
\pgfpathclose%
\pgfpathmoveto{\pgfqpoint{0.000000in}{0.280833in}}%
\pgfpathcurveto{\pgfqpoint{0.000000in}{0.280833in}}{\pgfqpoint{-0.013923in}{0.280833in}}{\pgfqpoint{-0.027278in}{0.286365in}}%
\pgfpathcurveto{\pgfqpoint{-0.037123in}{0.296210in}}{\pgfqpoint{-0.046968in}{0.306055in}}{\pgfqpoint{-0.052500in}{0.319410in}}%
\pgfpathcurveto{\pgfqpoint{-0.052500in}{0.333333in}}{\pgfqpoint{-0.052500in}{0.347256in}}{\pgfqpoint{-0.046968in}{0.360611in}}%
\pgfpathcurveto{\pgfqpoint{-0.037123in}{0.370456in}}{\pgfqpoint{-0.027278in}{0.380302in}}{\pgfqpoint{-0.013923in}{0.385833in}}%
\pgfpathcurveto{\pgfqpoint{0.000000in}{0.385833in}}{\pgfqpoint{0.013923in}{0.385833in}}{\pgfqpoint{0.027278in}{0.380302in}}%
\pgfpathcurveto{\pgfqpoint{0.037123in}{0.370456in}}{\pgfqpoint{0.046968in}{0.360611in}}{\pgfqpoint{0.052500in}{0.347256in}}%
\pgfpathcurveto{\pgfqpoint{0.052500in}{0.333333in}}{\pgfqpoint{0.052500in}{0.319410in}}{\pgfqpoint{0.046968in}{0.306055in}}%
\pgfpathcurveto{\pgfqpoint{0.037123in}{0.296210in}}{\pgfqpoint{0.027278in}{0.286365in}}{\pgfqpoint{0.013923in}{0.280833in}}%
\pgfpathclose%
\pgfpathmoveto{\pgfqpoint{0.166667in}{0.275000in}}%
\pgfpathcurveto{\pgfqpoint{0.182137in}{0.275000in}}{\pgfqpoint{0.196975in}{0.281146in}}{\pgfqpoint{0.207915in}{0.292085in}}%
\pgfpathcurveto{\pgfqpoint{0.218854in}{0.303025in}}{\pgfqpoint{0.225000in}{0.317863in}}{\pgfqpoint{0.225000in}{0.333333in}}%
\pgfpathcurveto{\pgfqpoint{0.225000in}{0.348804in}}{\pgfqpoint{0.218854in}{0.363642in}}{\pgfqpoint{0.207915in}{0.374581in}}%
\pgfpathcurveto{\pgfqpoint{0.196975in}{0.385520in}}{\pgfqpoint{0.182137in}{0.391667in}}{\pgfqpoint{0.166667in}{0.391667in}}%
\pgfpathcurveto{\pgfqpoint{0.151196in}{0.391667in}}{\pgfqpoint{0.136358in}{0.385520in}}{\pgfqpoint{0.125419in}{0.374581in}}%
\pgfpathcurveto{\pgfqpoint{0.114480in}{0.363642in}}{\pgfqpoint{0.108333in}{0.348804in}}{\pgfqpoint{0.108333in}{0.333333in}}%
\pgfpathcurveto{\pgfqpoint{0.108333in}{0.317863in}}{\pgfqpoint{0.114480in}{0.303025in}}{\pgfqpoint{0.125419in}{0.292085in}}%
\pgfpathcurveto{\pgfqpoint{0.136358in}{0.281146in}}{\pgfqpoint{0.151196in}{0.275000in}}{\pgfqpoint{0.166667in}{0.275000in}}%
\pgfpathclose%
\pgfpathmoveto{\pgfqpoint{0.166667in}{0.280833in}}%
\pgfpathcurveto{\pgfqpoint{0.166667in}{0.280833in}}{\pgfqpoint{0.152744in}{0.280833in}}{\pgfqpoint{0.139389in}{0.286365in}}%
\pgfpathcurveto{\pgfqpoint{0.129544in}{0.296210in}}{\pgfqpoint{0.119698in}{0.306055in}}{\pgfqpoint{0.114167in}{0.319410in}}%
\pgfpathcurveto{\pgfqpoint{0.114167in}{0.333333in}}{\pgfqpoint{0.114167in}{0.347256in}}{\pgfqpoint{0.119698in}{0.360611in}}%
\pgfpathcurveto{\pgfqpoint{0.129544in}{0.370456in}}{\pgfqpoint{0.139389in}{0.380302in}}{\pgfqpoint{0.152744in}{0.385833in}}%
\pgfpathcurveto{\pgfqpoint{0.166667in}{0.385833in}}{\pgfqpoint{0.180590in}{0.385833in}}{\pgfqpoint{0.193945in}{0.380302in}}%
\pgfpathcurveto{\pgfqpoint{0.203790in}{0.370456in}}{\pgfqpoint{0.213635in}{0.360611in}}{\pgfqpoint{0.219167in}{0.347256in}}%
\pgfpathcurveto{\pgfqpoint{0.219167in}{0.333333in}}{\pgfqpoint{0.219167in}{0.319410in}}{\pgfqpoint{0.213635in}{0.306055in}}%
\pgfpathcurveto{\pgfqpoint{0.203790in}{0.296210in}}{\pgfqpoint{0.193945in}{0.286365in}}{\pgfqpoint{0.180590in}{0.280833in}}%
\pgfpathclose%
\pgfpathmoveto{\pgfqpoint{0.333333in}{0.275000in}}%
\pgfpathcurveto{\pgfqpoint{0.348804in}{0.275000in}}{\pgfqpoint{0.363642in}{0.281146in}}{\pgfqpoint{0.374581in}{0.292085in}}%
\pgfpathcurveto{\pgfqpoint{0.385520in}{0.303025in}}{\pgfqpoint{0.391667in}{0.317863in}}{\pgfqpoint{0.391667in}{0.333333in}}%
\pgfpathcurveto{\pgfqpoint{0.391667in}{0.348804in}}{\pgfqpoint{0.385520in}{0.363642in}}{\pgfqpoint{0.374581in}{0.374581in}}%
\pgfpathcurveto{\pgfqpoint{0.363642in}{0.385520in}}{\pgfqpoint{0.348804in}{0.391667in}}{\pgfqpoint{0.333333in}{0.391667in}}%
\pgfpathcurveto{\pgfqpoint{0.317863in}{0.391667in}}{\pgfqpoint{0.303025in}{0.385520in}}{\pgfqpoint{0.292085in}{0.374581in}}%
\pgfpathcurveto{\pgfqpoint{0.281146in}{0.363642in}}{\pgfqpoint{0.275000in}{0.348804in}}{\pgfqpoint{0.275000in}{0.333333in}}%
\pgfpathcurveto{\pgfqpoint{0.275000in}{0.317863in}}{\pgfqpoint{0.281146in}{0.303025in}}{\pgfqpoint{0.292085in}{0.292085in}}%
\pgfpathcurveto{\pgfqpoint{0.303025in}{0.281146in}}{\pgfqpoint{0.317863in}{0.275000in}}{\pgfqpoint{0.333333in}{0.275000in}}%
\pgfpathclose%
\pgfpathmoveto{\pgfqpoint{0.333333in}{0.280833in}}%
\pgfpathcurveto{\pgfqpoint{0.333333in}{0.280833in}}{\pgfqpoint{0.319410in}{0.280833in}}{\pgfqpoint{0.306055in}{0.286365in}}%
\pgfpathcurveto{\pgfqpoint{0.296210in}{0.296210in}}{\pgfqpoint{0.286365in}{0.306055in}}{\pgfqpoint{0.280833in}{0.319410in}}%
\pgfpathcurveto{\pgfqpoint{0.280833in}{0.333333in}}{\pgfqpoint{0.280833in}{0.347256in}}{\pgfqpoint{0.286365in}{0.360611in}}%
\pgfpathcurveto{\pgfqpoint{0.296210in}{0.370456in}}{\pgfqpoint{0.306055in}{0.380302in}}{\pgfqpoint{0.319410in}{0.385833in}}%
\pgfpathcurveto{\pgfqpoint{0.333333in}{0.385833in}}{\pgfqpoint{0.347256in}{0.385833in}}{\pgfqpoint{0.360611in}{0.380302in}}%
\pgfpathcurveto{\pgfqpoint{0.370456in}{0.370456in}}{\pgfqpoint{0.380302in}{0.360611in}}{\pgfqpoint{0.385833in}{0.347256in}}%
\pgfpathcurveto{\pgfqpoint{0.385833in}{0.333333in}}{\pgfqpoint{0.385833in}{0.319410in}}{\pgfqpoint{0.380302in}{0.306055in}}%
\pgfpathcurveto{\pgfqpoint{0.370456in}{0.296210in}}{\pgfqpoint{0.360611in}{0.286365in}}{\pgfqpoint{0.347256in}{0.280833in}}%
\pgfpathclose%
\pgfpathmoveto{\pgfqpoint{0.500000in}{0.275000in}}%
\pgfpathcurveto{\pgfqpoint{0.515470in}{0.275000in}}{\pgfqpoint{0.530309in}{0.281146in}}{\pgfqpoint{0.541248in}{0.292085in}}%
\pgfpathcurveto{\pgfqpoint{0.552187in}{0.303025in}}{\pgfqpoint{0.558333in}{0.317863in}}{\pgfqpoint{0.558333in}{0.333333in}}%
\pgfpathcurveto{\pgfqpoint{0.558333in}{0.348804in}}{\pgfqpoint{0.552187in}{0.363642in}}{\pgfqpoint{0.541248in}{0.374581in}}%
\pgfpathcurveto{\pgfqpoint{0.530309in}{0.385520in}}{\pgfqpoint{0.515470in}{0.391667in}}{\pgfqpoint{0.500000in}{0.391667in}}%
\pgfpathcurveto{\pgfqpoint{0.484530in}{0.391667in}}{\pgfqpoint{0.469691in}{0.385520in}}{\pgfqpoint{0.458752in}{0.374581in}}%
\pgfpathcurveto{\pgfqpoint{0.447813in}{0.363642in}}{\pgfqpoint{0.441667in}{0.348804in}}{\pgfqpoint{0.441667in}{0.333333in}}%
\pgfpathcurveto{\pgfqpoint{0.441667in}{0.317863in}}{\pgfqpoint{0.447813in}{0.303025in}}{\pgfqpoint{0.458752in}{0.292085in}}%
\pgfpathcurveto{\pgfqpoint{0.469691in}{0.281146in}}{\pgfqpoint{0.484530in}{0.275000in}}{\pgfqpoint{0.500000in}{0.275000in}}%
\pgfpathclose%
\pgfpathmoveto{\pgfqpoint{0.500000in}{0.280833in}}%
\pgfpathcurveto{\pgfqpoint{0.500000in}{0.280833in}}{\pgfqpoint{0.486077in}{0.280833in}}{\pgfqpoint{0.472722in}{0.286365in}}%
\pgfpathcurveto{\pgfqpoint{0.462877in}{0.296210in}}{\pgfqpoint{0.453032in}{0.306055in}}{\pgfqpoint{0.447500in}{0.319410in}}%
\pgfpathcurveto{\pgfqpoint{0.447500in}{0.333333in}}{\pgfqpoint{0.447500in}{0.347256in}}{\pgfqpoint{0.453032in}{0.360611in}}%
\pgfpathcurveto{\pgfqpoint{0.462877in}{0.370456in}}{\pgfqpoint{0.472722in}{0.380302in}}{\pgfqpoint{0.486077in}{0.385833in}}%
\pgfpathcurveto{\pgfqpoint{0.500000in}{0.385833in}}{\pgfqpoint{0.513923in}{0.385833in}}{\pgfqpoint{0.527278in}{0.380302in}}%
\pgfpathcurveto{\pgfqpoint{0.537123in}{0.370456in}}{\pgfqpoint{0.546968in}{0.360611in}}{\pgfqpoint{0.552500in}{0.347256in}}%
\pgfpathcurveto{\pgfqpoint{0.552500in}{0.333333in}}{\pgfqpoint{0.552500in}{0.319410in}}{\pgfqpoint{0.546968in}{0.306055in}}%
\pgfpathcurveto{\pgfqpoint{0.537123in}{0.296210in}}{\pgfqpoint{0.527278in}{0.286365in}}{\pgfqpoint{0.513923in}{0.280833in}}%
\pgfpathclose%
\pgfpathmoveto{\pgfqpoint{0.666667in}{0.275000in}}%
\pgfpathcurveto{\pgfqpoint{0.682137in}{0.275000in}}{\pgfqpoint{0.696975in}{0.281146in}}{\pgfqpoint{0.707915in}{0.292085in}}%
\pgfpathcurveto{\pgfqpoint{0.718854in}{0.303025in}}{\pgfqpoint{0.725000in}{0.317863in}}{\pgfqpoint{0.725000in}{0.333333in}}%
\pgfpathcurveto{\pgfqpoint{0.725000in}{0.348804in}}{\pgfqpoint{0.718854in}{0.363642in}}{\pgfqpoint{0.707915in}{0.374581in}}%
\pgfpathcurveto{\pgfqpoint{0.696975in}{0.385520in}}{\pgfqpoint{0.682137in}{0.391667in}}{\pgfqpoint{0.666667in}{0.391667in}}%
\pgfpathcurveto{\pgfqpoint{0.651196in}{0.391667in}}{\pgfqpoint{0.636358in}{0.385520in}}{\pgfqpoint{0.625419in}{0.374581in}}%
\pgfpathcurveto{\pgfqpoint{0.614480in}{0.363642in}}{\pgfqpoint{0.608333in}{0.348804in}}{\pgfqpoint{0.608333in}{0.333333in}}%
\pgfpathcurveto{\pgfqpoint{0.608333in}{0.317863in}}{\pgfqpoint{0.614480in}{0.303025in}}{\pgfqpoint{0.625419in}{0.292085in}}%
\pgfpathcurveto{\pgfqpoint{0.636358in}{0.281146in}}{\pgfqpoint{0.651196in}{0.275000in}}{\pgfqpoint{0.666667in}{0.275000in}}%
\pgfpathclose%
\pgfpathmoveto{\pgfqpoint{0.666667in}{0.280833in}}%
\pgfpathcurveto{\pgfqpoint{0.666667in}{0.280833in}}{\pgfqpoint{0.652744in}{0.280833in}}{\pgfqpoint{0.639389in}{0.286365in}}%
\pgfpathcurveto{\pgfqpoint{0.629544in}{0.296210in}}{\pgfqpoint{0.619698in}{0.306055in}}{\pgfqpoint{0.614167in}{0.319410in}}%
\pgfpathcurveto{\pgfqpoint{0.614167in}{0.333333in}}{\pgfqpoint{0.614167in}{0.347256in}}{\pgfqpoint{0.619698in}{0.360611in}}%
\pgfpathcurveto{\pgfqpoint{0.629544in}{0.370456in}}{\pgfqpoint{0.639389in}{0.380302in}}{\pgfqpoint{0.652744in}{0.385833in}}%
\pgfpathcurveto{\pgfqpoint{0.666667in}{0.385833in}}{\pgfqpoint{0.680590in}{0.385833in}}{\pgfqpoint{0.693945in}{0.380302in}}%
\pgfpathcurveto{\pgfqpoint{0.703790in}{0.370456in}}{\pgfqpoint{0.713635in}{0.360611in}}{\pgfqpoint{0.719167in}{0.347256in}}%
\pgfpathcurveto{\pgfqpoint{0.719167in}{0.333333in}}{\pgfqpoint{0.719167in}{0.319410in}}{\pgfqpoint{0.713635in}{0.306055in}}%
\pgfpathcurveto{\pgfqpoint{0.703790in}{0.296210in}}{\pgfqpoint{0.693945in}{0.286365in}}{\pgfqpoint{0.680590in}{0.280833in}}%
\pgfpathclose%
\pgfpathmoveto{\pgfqpoint{0.833333in}{0.275000in}}%
\pgfpathcurveto{\pgfqpoint{0.848804in}{0.275000in}}{\pgfqpoint{0.863642in}{0.281146in}}{\pgfqpoint{0.874581in}{0.292085in}}%
\pgfpathcurveto{\pgfqpoint{0.885520in}{0.303025in}}{\pgfqpoint{0.891667in}{0.317863in}}{\pgfqpoint{0.891667in}{0.333333in}}%
\pgfpathcurveto{\pgfqpoint{0.891667in}{0.348804in}}{\pgfqpoint{0.885520in}{0.363642in}}{\pgfqpoint{0.874581in}{0.374581in}}%
\pgfpathcurveto{\pgfqpoint{0.863642in}{0.385520in}}{\pgfqpoint{0.848804in}{0.391667in}}{\pgfqpoint{0.833333in}{0.391667in}}%
\pgfpathcurveto{\pgfqpoint{0.817863in}{0.391667in}}{\pgfqpoint{0.803025in}{0.385520in}}{\pgfqpoint{0.792085in}{0.374581in}}%
\pgfpathcurveto{\pgfqpoint{0.781146in}{0.363642in}}{\pgfqpoint{0.775000in}{0.348804in}}{\pgfqpoint{0.775000in}{0.333333in}}%
\pgfpathcurveto{\pgfqpoint{0.775000in}{0.317863in}}{\pgfqpoint{0.781146in}{0.303025in}}{\pgfqpoint{0.792085in}{0.292085in}}%
\pgfpathcurveto{\pgfqpoint{0.803025in}{0.281146in}}{\pgfqpoint{0.817863in}{0.275000in}}{\pgfqpoint{0.833333in}{0.275000in}}%
\pgfpathclose%
\pgfpathmoveto{\pgfqpoint{0.833333in}{0.280833in}}%
\pgfpathcurveto{\pgfqpoint{0.833333in}{0.280833in}}{\pgfqpoint{0.819410in}{0.280833in}}{\pgfqpoint{0.806055in}{0.286365in}}%
\pgfpathcurveto{\pgfqpoint{0.796210in}{0.296210in}}{\pgfqpoint{0.786365in}{0.306055in}}{\pgfqpoint{0.780833in}{0.319410in}}%
\pgfpathcurveto{\pgfqpoint{0.780833in}{0.333333in}}{\pgfqpoint{0.780833in}{0.347256in}}{\pgfqpoint{0.786365in}{0.360611in}}%
\pgfpathcurveto{\pgfqpoint{0.796210in}{0.370456in}}{\pgfqpoint{0.806055in}{0.380302in}}{\pgfqpoint{0.819410in}{0.385833in}}%
\pgfpathcurveto{\pgfqpoint{0.833333in}{0.385833in}}{\pgfqpoint{0.847256in}{0.385833in}}{\pgfqpoint{0.860611in}{0.380302in}}%
\pgfpathcurveto{\pgfqpoint{0.870456in}{0.370456in}}{\pgfqpoint{0.880302in}{0.360611in}}{\pgfqpoint{0.885833in}{0.347256in}}%
\pgfpathcurveto{\pgfqpoint{0.885833in}{0.333333in}}{\pgfqpoint{0.885833in}{0.319410in}}{\pgfqpoint{0.880302in}{0.306055in}}%
\pgfpathcurveto{\pgfqpoint{0.870456in}{0.296210in}}{\pgfqpoint{0.860611in}{0.286365in}}{\pgfqpoint{0.847256in}{0.280833in}}%
\pgfpathclose%
\pgfpathmoveto{\pgfqpoint{1.000000in}{0.275000in}}%
\pgfpathcurveto{\pgfqpoint{1.015470in}{0.275000in}}{\pgfqpoint{1.030309in}{0.281146in}}{\pgfqpoint{1.041248in}{0.292085in}}%
\pgfpathcurveto{\pgfqpoint{1.052187in}{0.303025in}}{\pgfqpoint{1.058333in}{0.317863in}}{\pgfqpoint{1.058333in}{0.333333in}}%
\pgfpathcurveto{\pgfqpoint{1.058333in}{0.348804in}}{\pgfqpoint{1.052187in}{0.363642in}}{\pgfqpoint{1.041248in}{0.374581in}}%
\pgfpathcurveto{\pgfqpoint{1.030309in}{0.385520in}}{\pgfqpoint{1.015470in}{0.391667in}}{\pgfqpoint{1.000000in}{0.391667in}}%
\pgfpathcurveto{\pgfqpoint{0.984530in}{0.391667in}}{\pgfqpoint{0.969691in}{0.385520in}}{\pgfqpoint{0.958752in}{0.374581in}}%
\pgfpathcurveto{\pgfqpoint{0.947813in}{0.363642in}}{\pgfqpoint{0.941667in}{0.348804in}}{\pgfqpoint{0.941667in}{0.333333in}}%
\pgfpathcurveto{\pgfqpoint{0.941667in}{0.317863in}}{\pgfqpoint{0.947813in}{0.303025in}}{\pgfqpoint{0.958752in}{0.292085in}}%
\pgfpathcurveto{\pgfqpoint{0.969691in}{0.281146in}}{\pgfqpoint{0.984530in}{0.275000in}}{\pgfqpoint{1.000000in}{0.275000in}}%
\pgfpathclose%
\pgfpathmoveto{\pgfqpoint{1.000000in}{0.280833in}}%
\pgfpathcurveto{\pgfqpoint{1.000000in}{0.280833in}}{\pgfqpoint{0.986077in}{0.280833in}}{\pgfqpoint{0.972722in}{0.286365in}}%
\pgfpathcurveto{\pgfqpoint{0.962877in}{0.296210in}}{\pgfqpoint{0.953032in}{0.306055in}}{\pgfqpoint{0.947500in}{0.319410in}}%
\pgfpathcurveto{\pgfqpoint{0.947500in}{0.333333in}}{\pgfqpoint{0.947500in}{0.347256in}}{\pgfqpoint{0.953032in}{0.360611in}}%
\pgfpathcurveto{\pgfqpoint{0.962877in}{0.370456in}}{\pgfqpoint{0.972722in}{0.380302in}}{\pgfqpoint{0.986077in}{0.385833in}}%
\pgfpathcurveto{\pgfqpoint{1.000000in}{0.385833in}}{\pgfqpoint{1.013923in}{0.385833in}}{\pgfqpoint{1.027278in}{0.380302in}}%
\pgfpathcurveto{\pgfqpoint{1.037123in}{0.370456in}}{\pgfqpoint{1.046968in}{0.360611in}}{\pgfqpoint{1.052500in}{0.347256in}}%
\pgfpathcurveto{\pgfqpoint{1.052500in}{0.333333in}}{\pgfqpoint{1.052500in}{0.319410in}}{\pgfqpoint{1.046968in}{0.306055in}}%
\pgfpathcurveto{\pgfqpoint{1.037123in}{0.296210in}}{\pgfqpoint{1.027278in}{0.286365in}}{\pgfqpoint{1.013923in}{0.280833in}}%
\pgfpathclose%
\pgfpathmoveto{\pgfqpoint{0.083333in}{0.441667in}}%
\pgfpathcurveto{\pgfqpoint{0.098804in}{0.441667in}}{\pgfqpoint{0.113642in}{0.447813in}}{\pgfqpoint{0.124581in}{0.458752in}}%
\pgfpathcurveto{\pgfqpoint{0.135520in}{0.469691in}}{\pgfqpoint{0.141667in}{0.484530in}}{\pgfqpoint{0.141667in}{0.500000in}}%
\pgfpathcurveto{\pgfqpoint{0.141667in}{0.515470in}}{\pgfqpoint{0.135520in}{0.530309in}}{\pgfqpoint{0.124581in}{0.541248in}}%
\pgfpathcurveto{\pgfqpoint{0.113642in}{0.552187in}}{\pgfqpoint{0.098804in}{0.558333in}}{\pgfqpoint{0.083333in}{0.558333in}}%
\pgfpathcurveto{\pgfqpoint{0.067863in}{0.558333in}}{\pgfqpoint{0.053025in}{0.552187in}}{\pgfqpoint{0.042085in}{0.541248in}}%
\pgfpathcurveto{\pgfqpoint{0.031146in}{0.530309in}}{\pgfqpoint{0.025000in}{0.515470in}}{\pgfqpoint{0.025000in}{0.500000in}}%
\pgfpathcurveto{\pgfqpoint{0.025000in}{0.484530in}}{\pgfqpoint{0.031146in}{0.469691in}}{\pgfqpoint{0.042085in}{0.458752in}}%
\pgfpathcurveto{\pgfqpoint{0.053025in}{0.447813in}}{\pgfqpoint{0.067863in}{0.441667in}}{\pgfqpoint{0.083333in}{0.441667in}}%
\pgfpathclose%
\pgfpathmoveto{\pgfqpoint{0.083333in}{0.447500in}}%
\pgfpathcurveto{\pgfqpoint{0.083333in}{0.447500in}}{\pgfqpoint{0.069410in}{0.447500in}}{\pgfqpoint{0.056055in}{0.453032in}}%
\pgfpathcurveto{\pgfqpoint{0.046210in}{0.462877in}}{\pgfqpoint{0.036365in}{0.472722in}}{\pgfqpoint{0.030833in}{0.486077in}}%
\pgfpathcurveto{\pgfqpoint{0.030833in}{0.500000in}}{\pgfqpoint{0.030833in}{0.513923in}}{\pgfqpoint{0.036365in}{0.527278in}}%
\pgfpathcurveto{\pgfqpoint{0.046210in}{0.537123in}}{\pgfqpoint{0.056055in}{0.546968in}}{\pgfqpoint{0.069410in}{0.552500in}}%
\pgfpathcurveto{\pgfqpoint{0.083333in}{0.552500in}}{\pgfqpoint{0.097256in}{0.552500in}}{\pgfqpoint{0.110611in}{0.546968in}}%
\pgfpathcurveto{\pgfqpoint{0.120456in}{0.537123in}}{\pgfqpoint{0.130302in}{0.527278in}}{\pgfqpoint{0.135833in}{0.513923in}}%
\pgfpathcurveto{\pgfqpoint{0.135833in}{0.500000in}}{\pgfqpoint{0.135833in}{0.486077in}}{\pgfqpoint{0.130302in}{0.472722in}}%
\pgfpathcurveto{\pgfqpoint{0.120456in}{0.462877in}}{\pgfqpoint{0.110611in}{0.453032in}}{\pgfqpoint{0.097256in}{0.447500in}}%
\pgfpathclose%
\pgfpathmoveto{\pgfqpoint{0.250000in}{0.441667in}}%
\pgfpathcurveto{\pgfqpoint{0.265470in}{0.441667in}}{\pgfqpoint{0.280309in}{0.447813in}}{\pgfqpoint{0.291248in}{0.458752in}}%
\pgfpathcurveto{\pgfqpoint{0.302187in}{0.469691in}}{\pgfqpoint{0.308333in}{0.484530in}}{\pgfqpoint{0.308333in}{0.500000in}}%
\pgfpathcurveto{\pgfqpoint{0.308333in}{0.515470in}}{\pgfqpoint{0.302187in}{0.530309in}}{\pgfqpoint{0.291248in}{0.541248in}}%
\pgfpathcurveto{\pgfqpoint{0.280309in}{0.552187in}}{\pgfqpoint{0.265470in}{0.558333in}}{\pgfqpoint{0.250000in}{0.558333in}}%
\pgfpathcurveto{\pgfqpoint{0.234530in}{0.558333in}}{\pgfqpoint{0.219691in}{0.552187in}}{\pgfqpoint{0.208752in}{0.541248in}}%
\pgfpathcurveto{\pgfqpoint{0.197813in}{0.530309in}}{\pgfqpoint{0.191667in}{0.515470in}}{\pgfqpoint{0.191667in}{0.500000in}}%
\pgfpathcurveto{\pgfqpoint{0.191667in}{0.484530in}}{\pgfqpoint{0.197813in}{0.469691in}}{\pgfqpoint{0.208752in}{0.458752in}}%
\pgfpathcurveto{\pgfqpoint{0.219691in}{0.447813in}}{\pgfqpoint{0.234530in}{0.441667in}}{\pgfqpoint{0.250000in}{0.441667in}}%
\pgfpathclose%
\pgfpathmoveto{\pgfqpoint{0.250000in}{0.447500in}}%
\pgfpathcurveto{\pgfqpoint{0.250000in}{0.447500in}}{\pgfqpoint{0.236077in}{0.447500in}}{\pgfqpoint{0.222722in}{0.453032in}}%
\pgfpathcurveto{\pgfqpoint{0.212877in}{0.462877in}}{\pgfqpoint{0.203032in}{0.472722in}}{\pgfqpoint{0.197500in}{0.486077in}}%
\pgfpathcurveto{\pgfqpoint{0.197500in}{0.500000in}}{\pgfqpoint{0.197500in}{0.513923in}}{\pgfqpoint{0.203032in}{0.527278in}}%
\pgfpathcurveto{\pgfqpoint{0.212877in}{0.537123in}}{\pgfqpoint{0.222722in}{0.546968in}}{\pgfqpoint{0.236077in}{0.552500in}}%
\pgfpathcurveto{\pgfqpoint{0.250000in}{0.552500in}}{\pgfqpoint{0.263923in}{0.552500in}}{\pgfqpoint{0.277278in}{0.546968in}}%
\pgfpathcurveto{\pgfqpoint{0.287123in}{0.537123in}}{\pgfqpoint{0.296968in}{0.527278in}}{\pgfqpoint{0.302500in}{0.513923in}}%
\pgfpathcurveto{\pgfqpoint{0.302500in}{0.500000in}}{\pgfqpoint{0.302500in}{0.486077in}}{\pgfqpoint{0.296968in}{0.472722in}}%
\pgfpathcurveto{\pgfqpoint{0.287123in}{0.462877in}}{\pgfqpoint{0.277278in}{0.453032in}}{\pgfqpoint{0.263923in}{0.447500in}}%
\pgfpathclose%
\pgfpathmoveto{\pgfqpoint{0.416667in}{0.441667in}}%
\pgfpathcurveto{\pgfqpoint{0.432137in}{0.441667in}}{\pgfqpoint{0.446975in}{0.447813in}}{\pgfqpoint{0.457915in}{0.458752in}}%
\pgfpathcurveto{\pgfqpoint{0.468854in}{0.469691in}}{\pgfqpoint{0.475000in}{0.484530in}}{\pgfqpoint{0.475000in}{0.500000in}}%
\pgfpathcurveto{\pgfqpoint{0.475000in}{0.515470in}}{\pgfqpoint{0.468854in}{0.530309in}}{\pgfqpoint{0.457915in}{0.541248in}}%
\pgfpathcurveto{\pgfqpoint{0.446975in}{0.552187in}}{\pgfqpoint{0.432137in}{0.558333in}}{\pgfqpoint{0.416667in}{0.558333in}}%
\pgfpathcurveto{\pgfqpoint{0.401196in}{0.558333in}}{\pgfqpoint{0.386358in}{0.552187in}}{\pgfqpoint{0.375419in}{0.541248in}}%
\pgfpathcurveto{\pgfqpoint{0.364480in}{0.530309in}}{\pgfqpoint{0.358333in}{0.515470in}}{\pgfqpoint{0.358333in}{0.500000in}}%
\pgfpathcurveto{\pgfqpoint{0.358333in}{0.484530in}}{\pgfqpoint{0.364480in}{0.469691in}}{\pgfqpoint{0.375419in}{0.458752in}}%
\pgfpathcurveto{\pgfqpoint{0.386358in}{0.447813in}}{\pgfqpoint{0.401196in}{0.441667in}}{\pgfqpoint{0.416667in}{0.441667in}}%
\pgfpathclose%
\pgfpathmoveto{\pgfqpoint{0.416667in}{0.447500in}}%
\pgfpathcurveto{\pgfqpoint{0.416667in}{0.447500in}}{\pgfqpoint{0.402744in}{0.447500in}}{\pgfqpoint{0.389389in}{0.453032in}}%
\pgfpathcurveto{\pgfqpoint{0.379544in}{0.462877in}}{\pgfqpoint{0.369698in}{0.472722in}}{\pgfqpoint{0.364167in}{0.486077in}}%
\pgfpathcurveto{\pgfqpoint{0.364167in}{0.500000in}}{\pgfqpoint{0.364167in}{0.513923in}}{\pgfqpoint{0.369698in}{0.527278in}}%
\pgfpathcurveto{\pgfqpoint{0.379544in}{0.537123in}}{\pgfqpoint{0.389389in}{0.546968in}}{\pgfqpoint{0.402744in}{0.552500in}}%
\pgfpathcurveto{\pgfqpoint{0.416667in}{0.552500in}}{\pgfqpoint{0.430590in}{0.552500in}}{\pgfqpoint{0.443945in}{0.546968in}}%
\pgfpathcurveto{\pgfqpoint{0.453790in}{0.537123in}}{\pgfqpoint{0.463635in}{0.527278in}}{\pgfqpoint{0.469167in}{0.513923in}}%
\pgfpathcurveto{\pgfqpoint{0.469167in}{0.500000in}}{\pgfqpoint{0.469167in}{0.486077in}}{\pgfqpoint{0.463635in}{0.472722in}}%
\pgfpathcurveto{\pgfqpoint{0.453790in}{0.462877in}}{\pgfqpoint{0.443945in}{0.453032in}}{\pgfqpoint{0.430590in}{0.447500in}}%
\pgfpathclose%
\pgfpathmoveto{\pgfqpoint{0.583333in}{0.441667in}}%
\pgfpathcurveto{\pgfqpoint{0.598804in}{0.441667in}}{\pgfqpoint{0.613642in}{0.447813in}}{\pgfqpoint{0.624581in}{0.458752in}}%
\pgfpathcurveto{\pgfqpoint{0.635520in}{0.469691in}}{\pgfqpoint{0.641667in}{0.484530in}}{\pgfqpoint{0.641667in}{0.500000in}}%
\pgfpathcurveto{\pgfqpoint{0.641667in}{0.515470in}}{\pgfqpoint{0.635520in}{0.530309in}}{\pgfqpoint{0.624581in}{0.541248in}}%
\pgfpathcurveto{\pgfqpoint{0.613642in}{0.552187in}}{\pgfqpoint{0.598804in}{0.558333in}}{\pgfqpoint{0.583333in}{0.558333in}}%
\pgfpathcurveto{\pgfqpoint{0.567863in}{0.558333in}}{\pgfqpoint{0.553025in}{0.552187in}}{\pgfqpoint{0.542085in}{0.541248in}}%
\pgfpathcurveto{\pgfqpoint{0.531146in}{0.530309in}}{\pgfqpoint{0.525000in}{0.515470in}}{\pgfqpoint{0.525000in}{0.500000in}}%
\pgfpathcurveto{\pgfqpoint{0.525000in}{0.484530in}}{\pgfqpoint{0.531146in}{0.469691in}}{\pgfqpoint{0.542085in}{0.458752in}}%
\pgfpathcurveto{\pgfqpoint{0.553025in}{0.447813in}}{\pgfqpoint{0.567863in}{0.441667in}}{\pgfqpoint{0.583333in}{0.441667in}}%
\pgfpathclose%
\pgfpathmoveto{\pgfqpoint{0.583333in}{0.447500in}}%
\pgfpathcurveto{\pgfqpoint{0.583333in}{0.447500in}}{\pgfqpoint{0.569410in}{0.447500in}}{\pgfqpoint{0.556055in}{0.453032in}}%
\pgfpathcurveto{\pgfqpoint{0.546210in}{0.462877in}}{\pgfqpoint{0.536365in}{0.472722in}}{\pgfqpoint{0.530833in}{0.486077in}}%
\pgfpathcurveto{\pgfqpoint{0.530833in}{0.500000in}}{\pgfqpoint{0.530833in}{0.513923in}}{\pgfqpoint{0.536365in}{0.527278in}}%
\pgfpathcurveto{\pgfqpoint{0.546210in}{0.537123in}}{\pgfqpoint{0.556055in}{0.546968in}}{\pgfqpoint{0.569410in}{0.552500in}}%
\pgfpathcurveto{\pgfqpoint{0.583333in}{0.552500in}}{\pgfqpoint{0.597256in}{0.552500in}}{\pgfqpoint{0.610611in}{0.546968in}}%
\pgfpathcurveto{\pgfqpoint{0.620456in}{0.537123in}}{\pgfqpoint{0.630302in}{0.527278in}}{\pgfqpoint{0.635833in}{0.513923in}}%
\pgfpathcurveto{\pgfqpoint{0.635833in}{0.500000in}}{\pgfqpoint{0.635833in}{0.486077in}}{\pgfqpoint{0.630302in}{0.472722in}}%
\pgfpathcurveto{\pgfqpoint{0.620456in}{0.462877in}}{\pgfqpoint{0.610611in}{0.453032in}}{\pgfqpoint{0.597256in}{0.447500in}}%
\pgfpathclose%
\pgfpathmoveto{\pgfqpoint{0.750000in}{0.441667in}}%
\pgfpathcurveto{\pgfqpoint{0.765470in}{0.441667in}}{\pgfqpoint{0.780309in}{0.447813in}}{\pgfqpoint{0.791248in}{0.458752in}}%
\pgfpathcurveto{\pgfqpoint{0.802187in}{0.469691in}}{\pgfqpoint{0.808333in}{0.484530in}}{\pgfqpoint{0.808333in}{0.500000in}}%
\pgfpathcurveto{\pgfqpoint{0.808333in}{0.515470in}}{\pgfqpoint{0.802187in}{0.530309in}}{\pgfqpoint{0.791248in}{0.541248in}}%
\pgfpathcurveto{\pgfqpoint{0.780309in}{0.552187in}}{\pgfqpoint{0.765470in}{0.558333in}}{\pgfqpoint{0.750000in}{0.558333in}}%
\pgfpathcurveto{\pgfqpoint{0.734530in}{0.558333in}}{\pgfqpoint{0.719691in}{0.552187in}}{\pgfqpoint{0.708752in}{0.541248in}}%
\pgfpathcurveto{\pgfqpoint{0.697813in}{0.530309in}}{\pgfqpoint{0.691667in}{0.515470in}}{\pgfqpoint{0.691667in}{0.500000in}}%
\pgfpathcurveto{\pgfqpoint{0.691667in}{0.484530in}}{\pgfqpoint{0.697813in}{0.469691in}}{\pgfqpoint{0.708752in}{0.458752in}}%
\pgfpathcurveto{\pgfqpoint{0.719691in}{0.447813in}}{\pgfqpoint{0.734530in}{0.441667in}}{\pgfqpoint{0.750000in}{0.441667in}}%
\pgfpathclose%
\pgfpathmoveto{\pgfqpoint{0.750000in}{0.447500in}}%
\pgfpathcurveto{\pgfqpoint{0.750000in}{0.447500in}}{\pgfqpoint{0.736077in}{0.447500in}}{\pgfqpoint{0.722722in}{0.453032in}}%
\pgfpathcurveto{\pgfqpoint{0.712877in}{0.462877in}}{\pgfqpoint{0.703032in}{0.472722in}}{\pgfqpoint{0.697500in}{0.486077in}}%
\pgfpathcurveto{\pgfqpoint{0.697500in}{0.500000in}}{\pgfqpoint{0.697500in}{0.513923in}}{\pgfqpoint{0.703032in}{0.527278in}}%
\pgfpathcurveto{\pgfqpoint{0.712877in}{0.537123in}}{\pgfqpoint{0.722722in}{0.546968in}}{\pgfqpoint{0.736077in}{0.552500in}}%
\pgfpathcurveto{\pgfqpoint{0.750000in}{0.552500in}}{\pgfqpoint{0.763923in}{0.552500in}}{\pgfqpoint{0.777278in}{0.546968in}}%
\pgfpathcurveto{\pgfqpoint{0.787123in}{0.537123in}}{\pgfqpoint{0.796968in}{0.527278in}}{\pgfqpoint{0.802500in}{0.513923in}}%
\pgfpathcurveto{\pgfqpoint{0.802500in}{0.500000in}}{\pgfqpoint{0.802500in}{0.486077in}}{\pgfqpoint{0.796968in}{0.472722in}}%
\pgfpathcurveto{\pgfqpoint{0.787123in}{0.462877in}}{\pgfqpoint{0.777278in}{0.453032in}}{\pgfqpoint{0.763923in}{0.447500in}}%
\pgfpathclose%
\pgfpathmoveto{\pgfqpoint{0.916667in}{0.441667in}}%
\pgfpathcurveto{\pgfqpoint{0.932137in}{0.441667in}}{\pgfqpoint{0.946975in}{0.447813in}}{\pgfqpoint{0.957915in}{0.458752in}}%
\pgfpathcurveto{\pgfqpoint{0.968854in}{0.469691in}}{\pgfqpoint{0.975000in}{0.484530in}}{\pgfqpoint{0.975000in}{0.500000in}}%
\pgfpathcurveto{\pgfqpoint{0.975000in}{0.515470in}}{\pgfqpoint{0.968854in}{0.530309in}}{\pgfqpoint{0.957915in}{0.541248in}}%
\pgfpathcurveto{\pgfqpoint{0.946975in}{0.552187in}}{\pgfqpoint{0.932137in}{0.558333in}}{\pgfqpoint{0.916667in}{0.558333in}}%
\pgfpathcurveto{\pgfqpoint{0.901196in}{0.558333in}}{\pgfqpoint{0.886358in}{0.552187in}}{\pgfqpoint{0.875419in}{0.541248in}}%
\pgfpathcurveto{\pgfqpoint{0.864480in}{0.530309in}}{\pgfqpoint{0.858333in}{0.515470in}}{\pgfqpoint{0.858333in}{0.500000in}}%
\pgfpathcurveto{\pgfqpoint{0.858333in}{0.484530in}}{\pgfqpoint{0.864480in}{0.469691in}}{\pgfqpoint{0.875419in}{0.458752in}}%
\pgfpathcurveto{\pgfqpoint{0.886358in}{0.447813in}}{\pgfqpoint{0.901196in}{0.441667in}}{\pgfqpoint{0.916667in}{0.441667in}}%
\pgfpathclose%
\pgfpathmoveto{\pgfqpoint{0.916667in}{0.447500in}}%
\pgfpathcurveto{\pgfqpoint{0.916667in}{0.447500in}}{\pgfqpoint{0.902744in}{0.447500in}}{\pgfqpoint{0.889389in}{0.453032in}}%
\pgfpathcurveto{\pgfqpoint{0.879544in}{0.462877in}}{\pgfqpoint{0.869698in}{0.472722in}}{\pgfqpoint{0.864167in}{0.486077in}}%
\pgfpathcurveto{\pgfqpoint{0.864167in}{0.500000in}}{\pgfqpoint{0.864167in}{0.513923in}}{\pgfqpoint{0.869698in}{0.527278in}}%
\pgfpathcurveto{\pgfqpoint{0.879544in}{0.537123in}}{\pgfqpoint{0.889389in}{0.546968in}}{\pgfqpoint{0.902744in}{0.552500in}}%
\pgfpathcurveto{\pgfqpoint{0.916667in}{0.552500in}}{\pgfqpoint{0.930590in}{0.552500in}}{\pgfqpoint{0.943945in}{0.546968in}}%
\pgfpathcurveto{\pgfqpoint{0.953790in}{0.537123in}}{\pgfqpoint{0.963635in}{0.527278in}}{\pgfqpoint{0.969167in}{0.513923in}}%
\pgfpathcurveto{\pgfqpoint{0.969167in}{0.500000in}}{\pgfqpoint{0.969167in}{0.486077in}}{\pgfqpoint{0.963635in}{0.472722in}}%
\pgfpathcurveto{\pgfqpoint{0.953790in}{0.462877in}}{\pgfqpoint{0.943945in}{0.453032in}}{\pgfqpoint{0.930590in}{0.447500in}}%
\pgfpathclose%
\pgfpathmoveto{\pgfqpoint{0.000000in}{0.608333in}}%
\pgfpathcurveto{\pgfqpoint{0.015470in}{0.608333in}}{\pgfqpoint{0.030309in}{0.614480in}}{\pgfqpoint{0.041248in}{0.625419in}}%
\pgfpathcurveto{\pgfqpoint{0.052187in}{0.636358in}}{\pgfqpoint{0.058333in}{0.651196in}}{\pgfqpoint{0.058333in}{0.666667in}}%
\pgfpathcurveto{\pgfqpoint{0.058333in}{0.682137in}}{\pgfqpoint{0.052187in}{0.696975in}}{\pgfqpoint{0.041248in}{0.707915in}}%
\pgfpathcurveto{\pgfqpoint{0.030309in}{0.718854in}}{\pgfqpoint{0.015470in}{0.725000in}}{\pgfqpoint{0.000000in}{0.725000in}}%
\pgfpathcurveto{\pgfqpoint{-0.015470in}{0.725000in}}{\pgfqpoint{-0.030309in}{0.718854in}}{\pgfqpoint{-0.041248in}{0.707915in}}%
\pgfpathcurveto{\pgfqpoint{-0.052187in}{0.696975in}}{\pgfqpoint{-0.058333in}{0.682137in}}{\pgfqpoint{-0.058333in}{0.666667in}}%
\pgfpathcurveto{\pgfqpoint{-0.058333in}{0.651196in}}{\pgfqpoint{-0.052187in}{0.636358in}}{\pgfqpoint{-0.041248in}{0.625419in}}%
\pgfpathcurveto{\pgfqpoint{-0.030309in}{0.614480in}}{\pgfqpoint{-0.015470in}{0.608333in}}{\pgfqpoint{0.000000in}{0.608333in}}%
\pgfpathclose%
\pgfpathmoveto{\pgfqpoint{0.000000in}{0.614167in}}%
\pgfpathcurveto{\pgfqpoint{0.000000in}{0.614167in}}{\pgfqpoint{-0.013923in}{0.614167in}}{\pgfqpoint{-0.027278in}{0.619698in}}%
\pgfpathcurveto{\pgfqpoint{-0.037123in}{0.629544in}}{\pgfqpoint{-0.046968in}{0.639389in}}{\pgfqpoint{-0.052500in}{0.652744in}}%
\pgfpathcurveto{\pgfqpoint{-0.052500in}{0.666667in}}{\pgfqpoint{-0.052500in}{0.680590in}}{\pgfqpoint{-0.046968in}{0.693945in}}%
\pgfpathcurveto{\pgfqpoint{-0.037123in}{0.703790in}}{\pgfqpoint{-0.027278in}{0.713635in}}{\pgfqpoint{-0.013923in}{0.719167in}}%
\pgfpathcurveto{\pgfqpoint{0.000000in}{0.719167in}}{\pgfqpoint{0.013923in}{0.719167in}}{\pgfqpoint{0.027278in}{0.713635in}}%
\pgfpathcurveto{\pgfqpoint{0.037123in}{0.703790in}}{\pgfqpoint{0.046968in}{0.693945in}}{\pgfqpoint{0.052500in}{0.680590in}}%
\pgfpathcurveto{\pgfqpoint{0.052500in}{0.666667in}}{\pgfqpoint{0.052500in}{0.652744in}}{\pgfqpoint{0.046968in}{0.639389in}}%
\pgfpathcurveto{\pgfqpoint{0.037123in}{0.629544in}}{\pgfqpoint{0.027278in}{0.619698in}}{\pgfqpoint{0.013923in}{0.614167in}}%
\pgfpathclose%
\pgfpathmoveto{\pgfqpoint{0.166667in}{0.608333in}}%
\pgfpathcurveto{\pgfqpoint{0.182137in}{0.608333in}}{\pgfqpoint{0.196975in}{0.614480in}}{\pgfqpoint{0.207915in}{0.625419in}}%
\pgfpathcurveto{\pgfqpoint{0.218854in}{0.636358in}}{\pgfqpoint{0.225000in}{0.651196in}}{\pgfqpoint{0.225000in}{0.666667in}}%
\pgfpathcurveto{\pgfqpoint{0.225000in}{0.682137in}}{\pgfqpoint{0.218854in}{0.696975in}}{\pgfqpoint{0.207915in}{0.707915in}}%
\pgfpathcurveto{\pgfqpoint{0.196975in}{0.718854in}}{\pgfqpoint{0.182137in}{0.725000in}}{\pgfqpoint{0.166667in}{0.725000in}}%
\pgfpathcurveto{\pgfqpoint{0.151196in}{0.725000in}}{\pgfqpoint{0.136358in}{0.718854in}}{\pgfqpoint{0.125419in}{0.707915in}}%
\pgfpathcurveto{\pgfqpoint{0.114480in}{0.696975in}}{\pgfqpoint{0.108333in}{0.682137in}}{\pgfqpoint{0.108333in}{0.666667in}}%
\pgfpathcurveto{\pgfqpoint{0.108333in}{0.651196in}}{\pgfqpoint{0.114480in}{0.636358in}}{\pgfqpoint{0.125419in}{0.625419in}}%
\pgfpathcurveto{\pgfqpoint{0.136358in}{0.614480in}}{\pgfqpoint{0.151196in}{0.608333in}}{\pgfqpoint{0.166667in}{0.608333in}}%
\pgfpathclose%
\pgfpathmoveto{\pgfqpoint{0.166667in}{0.614167in}}%
\pgfpathcurveto{\pgfqpoint{0.166667in}{0.614167in}}{\pgfqpoint{0.152744in}{0.614167in}}{\pgfqpoint{0.139389in}{0.619698in}}%
\pgfpathcurveto{\pgfqpoint{0.129544in}{0.629544in}}{\pgfqpoint{0.119698in}{0.639389in}}{\pgfqpoint{0.114167in}{0.652744in}}%
\pgfpathcurveto{\pgfqpoint{0.114167in}{0.666667in}}{\pgfqpoint{0.114167in}{0.680590in}}{\pgfqpoint{0.119698in}{0.693945in}}%
\pgfpathcurveto{\pgfqpoint{0.129544in}{0.703790in}}{\pgfqpoint{0.139389in}{0.713635in}}{\pgfqpoint{0.152744in}{0.719167in}}%
\pgfpathcurveto{\pgfqpoint{0.166667in}{0.719167in}}{\pgfqpoint{0.180590in}{0.719167in}}{\pgfqpoint{0.193945in}{0.713635in}}%
\pgfpathcurveto{\pgfqpoint{0.203790in}{0.703790in}}{\pgfqpoint{0.213635in}{0.693945in}}{\pgfqpoint{0.219167in}{0.680590in}}%
\pgfpathcurveto{\pgfqpoint{0.219167in}{0.666667in}}{\pgfqpoint{0.219167in}{0.652744in}}{\pgfqpoint{0.213635in}{0.639389in}}%
\pgfpathcurveto{\pgfqpoint{0.203790in}{0.629544in}}{\pgfqpoint{0.193945in}{0.619698in}}{\pgfqpoint{0.180590in}{0.614167in}}%
\pgfpathclose%
\pgfpathmoveto{\pgfqpoint{0.333333in}{0.608333in}}%
\pgfpathcurveto{\pgfqpoint{0.348804in}{0.608333in}}{\pgfqpoint{0.363642in}{0.614480in}}{\pgfqpoint{0.374581in}{0.625419in}}%
\pgfpathcurveto{\pgfqpoint{0.385520in}{0.636358in}}{\pgfqpoint{0.391667in}{0.651196in}}{\pgfqpoint{0.391667in}{0.666667in}}%
\pgfpathcurveto{\pgfqpoint{0.391667in}{0.682137in}}{\pgfqpoint{0.385520in}{0.696975in}}{\pgfqpoint{0.374581in}{0.707915in}}%
\pgfpathcurveto{\pgfqpoint{0.363642in}{0.718854in}}{\pgfqpoint{0.348804in}{0.725000in}}{\pgfqpoint{0.333333in}{0.725000in}}%
\pgfpathcurveto{\pgfqpoint{0.317863in}{0.725000in}}{\pgfqpoint{0.303025in}{0.718854in}}{\pgfqpoint{0.292085in}{0.707915in}}%
\pgfpathcurveto{\pgfqpoint{0.281146in}{0.696975in}}{\pgfqpoint{0.275000in}{0.682137in}}{\pgfqpoint{0.275000in}{0.666667in}}%
\pgfpathcurveto{\pgfqpoint{0.275000in}{0.651196in}}{\pgfqpoint{0.281146in}{0.636358in}}{\pgfqpoint{0.292085in}{0.625419in}}%
\pgfpathcurveto{\pgfqpoint{0.303025in}{0.614480in}}{\pgfqpoint{0.317863in}{0.608333in}}{\pgfqpoint{0.333333in}{0.608333in}}%
\pgfpathclose%
\pgfpathmoveto{\pgfqpoint{0.333333in}{0.614167in}}%
\pgfpathcurveto{\pgfqpoint{0.333333in}{0.614167in}}{\pgfqpoint{0.319410in}{0.614167in}}{\pgfqpoint{0.306055in}{0.619698in}}%
\pgfpathcurveto{\pgfqpoint{0.296210in}{0.629544in}}{\pgfqpoint{0.286365in}{0.639389in}}{\pgfqpoint{0.280833in}{0.652744in}}%
\pgfpathcurveto{\pgfqpoint{0.280833in}{0.666667in}}{\pgfqpoint{0.280833in}{0.680590in}}{\pgfqpoint{0.286365in}{0.693945in}}%
\pgfpathcurveto{\pgfqpoint{0.296210in}{0.703790in}}{\pgfqpoint{0.306055in}{0.713635in}}{\pgfqpoint{0.319410in}{0.719167in}}%
\pgfpathcurveto{\pgfqpoint{0.333333in}{0.719167in}}{\pgfqpoint{0.347256in}{0.719167in}}{\pgfqpoint{0.360611in}{0.713635in}}%
\pgfpathcurveto{\pgfqpoint{0.370456in}{0.703790in}}{\pgfqpoint{0.380302in}{0.693945in}}{\pgfqpoint{0.385833in}{0.680590in}}%
\pgfpathcurveto{\pgfqpoint{0.385833in}{0.666667in}}{\pgfqpoint{0.385833in}{0.652744in}}{\pgfqpoint{0.380302in}{0.639389in}}%
\pgfpathcurveto{\pgfqpoint{0.370456in}{0.629544in}}{\pgfqpoint{0.360611in}{0.619698in}}{\pgfqpoint{0.347256in}{0.614167in}}%
\pgfpathclose%
\pgfpathmoveto{\pgfqpoint{0.500000in}{0.608333in}}%
\pgfpathcurveto{\pgfqpoint{0.515470in}{0.608333in}}{\pgfqpoint{0.530309in}{0.614480in}}{\pgfqpoint{0.541248in}{0.625419in}}%
\pgfpathcurveto{\pgfqpoint{0.552187in}{0.636358in}}{\pgfqpoint{0.558333in}{0.651196in}}{\pgfqpoint{0.558333in}{0.666667in}}%
\pgfpathcurveto{\pgfqpoint{0.558333in}{0.682137in}}{\pgfqpoint{0.552187in}{0.696975in}}{\pgfqpoint{0.541248in}{0.707915in}}%
\pgfpathcurveto{\pgfqpoint{0.530309in}{0.718854in}}{\pgfqpoint{0.515470in}{0.725000in}}{\pgfqpoint{0.500000in}{0.725000in}}%
\pgfpathcurveto{\pgfqpoint{0.484530in}{0.725000in}}{\pgfqpoint{0.469691in}{0.718854in}}{\pgfqpoint{0.458752in}{0.707915in}}%
\pgfpathcurveto{\pgfqpoint{0.447813in}{0.696975in}}{\pgfqpoint{0.441667in}{0.682137in}}{\pgfqpoint{0.441667in}{0.666667in}}%
\pgfpathcurveto{\pgfqpoint{0.441667in}{0.651196in}}{\pgfqpoint{0.447813in}{0.636358in}}{\pgfqpoint{0.458752in}{0.625419in}}%
\pgfpathcurveto{\pgfqpoint{0.469691in}{0.614480in}}{\pgfqpoint{0.484530in}{0.608333in}}{\pgfqpoint{0.500000in}{0.608333in}}%
\pgfpathclose%
\pgfpathmoveto{\pgfqpoint{0.500000in}{0.614167in}}%
\pgfpathcurveto{\pgfqpoint{0.500000in}{0.614167in}}{\pgfqpoint{0.486077in}{0.614167in}}{\pgfqpoint{0.472722in}{0.619698in}}%
\pgfpathcurveto{\pgfqpoint{0.462877in}{0.629544in}}{\pgfqpoint{0.453032in}{0.639389in}}{\pgfqpoint{0.447500in}{0.652744in}}%
\pgfpathcurveto{\pgfqpoint{0.447500in}{0.666667in}}{\pgfqpoint{0.447500in}{0.680590in}}{\pgfqpoint{0.453032in}{0.693945in}}%
\pgfpathcurveto{\pgfqpoint{0.462877in}{0.703790in}}{\pgfqpoint{0.472722in}{0.713635in}}{\pgfqpoint{0.486077in}{0.719167in}}%
\pgfpathcurveto{\pgfqpoint{0.500000in}{0.719167in}}{\pgfqpoint{0.513923in}{0.719167in}}{\pgfqpoint{0.527278in}{0.713635in}}%
\pgfpathcurveto{\pgfqpoint{0.537123in}{0.703790in}}{\pgfqpoint{0.546968in}{0.693945in}}{\pgfqpoint{0.552500in}{0.680590in}}%
\pgfpathcurveto{\pgfqpoint{0.552500in}{0.666667in}}{\pgfqpoint{0.552500in}{0.652744in}}{\pgfqpoint{0.546968in}{0.639389in}}%
\pgfpathcurveto{\pgfqpoint{0.537123in}{0.629544in}}{\pgfqpoint{0.527278in}{0.619698in}}{\pgfqpoint{0.513923in}{0.614167in}}%
\pgfpathclose%
\pgfpathmoveto{\pgfqpoint{0.666667in}{0.608333in}}%
\pgfpathcurveto{\pgfqpoint{0.682137in}{0.608333in}}{\pgfqpoint{0.696975in}{0.614480in}}{\pgfqpoint{0.707915in}{0.625419in}}%
\pgfpathcurveto{\pgfqpoint{0.718854in}{0.636358in}}{\pgfqpoint{0.725000in}{0.651196in}}{\pgfqpoint{0.725000in}{0.666667in}}%
\pgfpathcurveto{\pgfqpoint{0.725000in}{0.682137in}}{\pgfqpoint{0.718854in}{0.696975in}}{\pgfqpoint{0.707915in}{0.707915in}}%
\pgfpathcurveto{\pgfqpoint{0.696975in}{0.718854in}}{\pgfqpoint{0.682137in}{0.725000in}}{\pgfqpoint{0.666667in}{0.725000in}}%
\pgfpathcurveto{\pgfqpoint{0.651196in}{0.725000in}}{\pgfqpoint{0.636358in}{0.718854in}}{\pgfqpoint{0.625419in}{0.707915in}}%
\pgfpathcurveto{\pgfqpoint{0.614480in}{0.696975in}}{\pgfqpoint{0.608333in}{0.682137in}}{\pgfqpoint{0.608333in}{0.666667in}}%
\pgfpathcurveto{\pgfqpoint{0.608333in}{0.651196in}}{\pgfqpoint{0.614480in}{0.636358in}}{\pgfqpoint{0.625419in}{0.625419in}}%
\pgfpathcurveto{\pgfqpoint{0.636358in}{0.614480in}}{\pgfqpoint{0.651196in}{0.608333in}}{\pgfqpoint{0.666667in}{0.608333in}}%
\pgfpathclose%
\pgfpathmoveto{\pgfqpoint{0.666667in}{0.614167in}}%
\pgfpathcurveto{\pgfqpoint{0.666667in}{0.614167in}}{\pgfqpoint{0.652744in}{0.614167in}}{\pgfqpoint{0.639389in}{0.619698in}}%
\pgfpathcurveto{\pgfqpoint{0.629544in}{0.629544in}}{\pgfqpoint{0.619698in}{0.639389in}}{\pgfqpoint{0.614167in}{0.652744in}}%
\pgfpathcurveto{\pgfqpoint{0.614167in}{0.666667in}}{\pgfqpoint{0.614167in}{0.680590in}}{\pgfqpoint{0.619698in}{0.693945in}}%
\pgfpathcurveto{\pgfqpoint{0.629544in}{0.703790in}}{\pgfqpoint{0.639389in}{0.713635in}}{\pgfqpoint{0.652744in}{0.719167in}}%
\pgfpathcurveto{\pgfqpoint{0.666667in}{0.719167in}}{\pgfqpoint{0.680590in}{0.719167in}}{\pgfqpoint{0.693945in}{0.713635in}}%
\pgfpathcurveto{\pgfqpoint{0.703790in}{0.703790in}}{\pgfqpoint{0.713635in}{0.693945in}}{\pgfqpoint{0.719167in}{0.680590in}}%
\pgfpathcurveto{\pgfqpoint{0.719167in}{0.666667in}}{\pgfqpoint{0.719167in}{0.652744in}}{\pgfqpoint{0.713635in}{0.639389in}}%
\pgfpathcurveto{\pgfqpoint{0.703790in}{0.629544in}}{\pgfqpoint{0.693945in}{0.619698in}}{\pgfqpoint{0.680590in}{0.614167in}}%
\pgfpathclose%
\pgfpathmoveto{\pgfqpoint{0.833333in}{0.608333in}}%
\pgfpathcurveto{\pgfqpoint{0.848804in}{0.608333in}}{\pgfqpoint{0.863642in}{0.614480in}}{\pgfqpoint{0.874581in}{0.625419in}}%
\pgfpathcurveto{\pgfqpoint{0.885520in}{0.636358in}}{\pgfqpoint{0.891667in}{0.651196in}}{\pgfqpoint{0.891667in}{0.666667in}}%
\pgfpathcurveto{\pgfqpoint{0.891667in}{0.682137in}}{\pgfqpoint{0.885520in}{0.696975in}}{\pgfqpoint{0.874581in}{0.707915in}}%
\pgfpathcurveto{\pgfqpoint{0.863642in}{0.718854in}}{\pgfqpoint{0.848804in}{0.725000in}}{\pgfqpoint{0.833333in}{0.725000in}}%
\pgfpathcurveto{\pgfqpoint{0.817863in}{0.725000in}}{\pgfqpoint{0.803025in}{0.718854in}}{\pgfqpoint{0.792085in}{0.707915in}}%
\pgfpathcurveto{\pgfqpoint{0.781146in}{0.696975in}}{\pgfqpoint{0.775000in}{0.682137in}}{\pgfqpoint{0.775000in}{0.666667in}}%
\pgfpathcurveto{\pgfqpoint{0.775000in}{0.651196in}}{\pgfqpoint{0.781146in}{0.636358in}}{\pgfqpoint{0.792085in}{0.625419in}}%
\pgfpathcurveto{\pgfqpoint{0.803025in}{0.614480in}}{\pgfqpoint{0.817863in}{0.608333in}}{\pgfqpoint{0.833333in}{0.608333in}}%
\pgfpathclose%
\pgfpathmoveto{\pgfqpoint{0.833333in}{0.614167in}}%
\pgfpathcurveto{\pgfqpoint{0.833333in}{0.614167in}}{\pgfqpoint{0.819410in}{0.614167in}}{\pgfqpoint{0.806055in}{0.619698in}}%
\pgfpathcurveto{\pgfqpoint{0.796210in}{0.629544in}}{\pgfqpoint{0.786365in}{0.639389in}}{\pgfqpoint{0.780833in}{0.652744in}}%
\pgfpathcurveto{\pgfqpoint{0.780833in}{0.666667in}}{\pgfqpoint{0.780833in}{0.680590in}}{\pgfqpoint{0.786365in}{0.693945in}}%
\pgfpathcurveto{\pgfqpoint{0.796210in}{0.703790in}}{\pgfqpoint{0.806055in}{0.713635in}}{\pgfqpoint{0.819410in}{0.719167in}}%
\pgfpathcurveto{\pgfqpoint{0.833333in}{0.719167in}}{\pgfqpoint{0.847256in}{0.719167in}}{\pgfqpoint{0.860611in}{0.713635in}}%
\pgfpathcurveto{\pgfqpoint{0.870456in}{0.703790in}}{\pgfqpoint{0.880302in}{0.693945in}}{\pgfqpoint{0.885833in}{0.680590in}}%
\pgfpathcurveto{\pgfqpoint{0.885833in}{0.666667in}}{\pgfqpoint{0.885833in}{0.652744in}}{\pgfqpoint{0.880302in}{0.639389in}}%
\pgfpathcurveto{\pgfqpoint{0.870456in}{0.629544in}}{\pgfqpoint{0.860611in}{0.619698in}}{\pgfqpoint{0.847256in}{0.614167in}}%
\pgfpathclose%
\pgfpathmoveto{\pgfqpoint{1.000000in}{0.608333in}}%
\pgfpathcurveto{\pgfqpoint{1.015470in}{0.608333in}}{\pgfqpoint{1.030309in}{0.614480in}}{\pgfqpoint{1.041248in}{0.625419in}}%
\pgfpathcurveto{\pgfqpoint{1.052187in}{0.636358in}}{\pgfqpoint{1.058333in}{0.651196in}}{\pgfqpoint{1.058333in}{0.666667in}}%
\pgfpathcurveto{\pgfqpoint{1.058333in}{0.682137in}}{\pgfqpoint{1.052187in}{0.696975in}}{\pgfqpoint{1.041248in}{0.707915in}}%
\pgfpathcurveto{\pgfqpoint{1.030309in}{0.718854in}}{\pgfqpoint{1.015470in}{0.725000in}}{\pgfqpoint{1.000000in}{0.725000in}}%
\pgfpathcurveto{\pgfqpoint{0.984530in}{0.725000in}}{\pgfqpoint{0.969691in}{0.718854in}}{\pgfqpoint{0.958752in}{0.707915in}}%
\pgfpathcurveto{\pgfqpoint{0.947813in}{0.696975in}}{\pgfqpoint{0.941667in}{0.682137in}}{\pgfqpoint{0.941667in}{0.666667in}}%
\pgfpathcurveto{\pgfqpoint{0.941667in}{0.651196in}}{\pgfqpoint{0.947813in}{0.636358in}}{\pgfqpoint{0.958752in}{0.625419in}}%
\pgfpathcurveto{\pgfqpoint{0.969691in}{0.614480in}}{\pgfqpoint{0.984530in}{0.608333in}}{\pgfqpoint{1.000000in}{0.608333in}}%
\pgfpathclose%
\pgfpathmoveto{\pgfqpoint{1.000000in}{0.614167in}}%
\pgfpathcurveto{\pgfqpoint{1.000000in}{0.614167in}}{\pgfqpoint{0.986077in}{0.614167in}}{\pgfqpoint{0.972722in}{0.619698in}}%
\pgfpathcurveto{\pgfqpoint{0.962877in}{0.629544in}}{\pgfqpoint{0.953032in}{0.639389in}}{\pgfqpoint{0.947500in}{0.652744in}}%
\pgfpathcurveto{\pgfqpoint{0.947500in}{0.666667in}}{\pgfqpoint{0.947500in}{0.680590in}}{\pgfqpoint{0.953032in}{0.693945in}}%
\pgfpathcurveto{\pgfqpoint{0.962877in}{0.703790in}}{\pgfqpoint{0.972722in}{0.713635in}}{\pgfqpoint{0.986077in}{0.719167in}}%
\pgfpathcurveto{\pgfqpoint{1.000000in}{0.719167in}}{\pgfqpoint{1.013923in}{0.719167in}}{\pgfqpoint{1.027278in}{0.713635in}}%
\pgfpathcurveto{\pgfqpoint{1.037123in}{0.703790in}}{\pgfqpoint{1.046968in}{0.693945in}}{\pgfqpoint{1.052500in}{0.680590in}}%
\pgfpathcurveto{\pgfqpoint{1.052500in}{0.666667in}}{\pgfqpoint{1.052500in}{0.652744in}}{\pgfqpoint{1.046968in}{0.639389in}}%
\pgfpathcurveto{\pgfqpoint{1.037123in}{0.629544in}}{\pgfqpoint{1.027278in}{0.619698in}}{\pgfqpoint{1.013923in}{0.614167in}}%
\pgfpathclose%
\pgfpathmoveto{\pgfqpoint{0.083333in}{0.775000in}}%
\pgfpathcurveto{\pgfqpoint{0.098804in}{0.775000in}}{\pgfqpoint{0.113642in}{0.781146in}}{\pgfqpoint{0.124581in}{0.792085in}}%
\pgfpathcurveto{\pgfqpoint{0.135520in}{0.803025in}}{\pgfqpoint{0.141667in}{0.817863in}}{\pgfqpoint{0.141667in}{0.833333in}}%
\pgfpathcurveto{\pgfqpoint{0.141667in}{0.848804in}}{\pgfqpoint{0.135520in}{0.863642in}}{\pgfqpoint{0.124581in}{0.874581in}}%
\pgfpathcurveto{\pgfqpoint{0.113642in}{0.885520in}}{\pgfqpoint{0.098804in}{0.891667in}}{\pgfqpoint{0.083333in}{0.891667in}}%
\pgfpathcurveto{\pgfqpoint{0.067863in}{0.891667in}}{\pgfqpoint{0.053025in}{0.885520in}}{\pgfqpoint{0.042085in}{0.874581in}}%
\pgfpathcurveto{\pgfqpoint{0.031146in}{0.863642in}}{\pgfqpoint{0.025000in}{0.848804in}}{\pgfqpoint{0.025000in}{0.833333in}}%
\pgfpathcurveto{\pgfqpoint{0.025000in}{0.817863in}}{\pgfqpoint{0.031146in}{0.803025in}}{\pgfqpoint{0.042085in}{0.792085in}}%
\pgfpathcurveto{\pgfqpoint{0.053025in}{0.781146in}}{\pgfqpoint{0.067863in}{0.775000in}}{\pgfqpoint{0.083333in}{0.775000in}}%
\pgfpathclose%
\pgfpathmoveto{\pgfqpoint{0.083333in}{0.780833in}}%
\pgfpathcurveto{\pgfqpoint{0.083333in}{0.780833in}}{\pgfqpoint{0.069410in}{0.780833in}}{\pgfqpoint{0.056055in}{0.786365in}}%
\pgfpathcurveto{\pgfqpoint{0.046210in}{0.796210in}}{\pgfqpoint{0.036365in}{0.806055in}}{\pgfqpoint{0.030833in}{0.819410in}}%
\pgfpathcurveto{\pgfqpoint{0.030833in}{0.833333in}}{\pgfqpoint{0.030833in}{0.847256in}}{\pgfqpoint{0.036365in}{0.860611in}}%
\pgfpathcurveto{\pgfqpoint{0.046210in}{0.870456in}}{\pgfqpoint{0.056055in}{0.880302in}}{\pgfqpoint{0.069410in}{0.885833in}}%
\pgfpathcurveto{\pgfqpoint{0.083333in}{0.885833in}}{\pgfqpoint{0.097256in}{0.885833in}}{\pgfqpoint{0.110611in}{0.880302in}}%
\pgfpathcurveto{\pgfqpoint{0.120456in}{0.870456in}}{\pgfqpoint{0.130302in}{0.860611in}}{\pgfqpoint{0.135833in}{0.847256in}}%
\pgfpathcurveto{\pgfqpoint{0.135833in}{0.833333in}}{\pgfqpoint{0.135833in}{0.819410in}}{\pgfqpoint{0.130302in}{0.806055in}}%
\pgfpathcurveto{\pgfqpoint{0.120456in}{0.796210in}}{\pgfqpoint{0.110611in}{0.786365in}}{\pgfqpoint{0.097256in}{0.780833in}}%
\pgfpathclose%
\pgfpathmoveto{\pgfqpoint{0.250000in}{0.775000in}}%
\pgfpathcurveto{\pgfqpoint{0.265470in}{0.775000in}}{\pgfqpoint{0.280309in}{0.781146in}}{\pgfqpoint{0.291248in}{0.792085in}}%
\pgfpathcurveto{\pgfqpoint{0.302187in}{0.803025in}}{\pgfqpoint{0.308333in}{0.817863in}}{\pgfqpoint{0.308333in}{0.833333in}}%
\pgfpathcurveto{\pgfqpoint{0.308333in}{0.848804in}}{\pgfqpoint{0.302187in}{0.863642in}}{\pgfqpoint{0.291248in}{0.874581in}}%
\pgfpathcurveto{\pgfqpoint{0.280309in}{0.885520in}}{\pgfqpoint{0.265470in}{0.891667in}}{\pgfqpoint{0.250000in}{0.891667in}}%
\pgfpathcurveto{\pgfqpoint{0.234530in}{0.891667in}}{\pgfqpoint{0.219691in}{0.885520in}}{\pgfqpoint{0.208752in}{0.874581in}}%
\pgfpathcurveto{\pgfqpoint{0.197813in}{0.863642in}}{\pgfqpoint{0.191667in}{0.848804in}}{\pgfqpoint{0.191667in}{0.833333in}}%
\pgfpathcurveto{\pgfqpoint{0.191667in}{0.817863in}}{\pgfqpoint{0.197813in}{0.803025in}}{\pgfqpoint{0.208752in}{0.792085in}}%
\pgfpathcurveto{\pgfqpoint{0.219691in}{0.781146in}}{\pgfqpoint{0.234530in}{0.775000in}}{\pgfqpoint{0.250000in}{0.775000in}}%
\pgfpathclose%
\pgfpathmoveto{\pgfqpoint{0.250000in}{0.780833in}}%
\pgfpathcurveto{\pgfqpoint{0.250000in}{0.780833in}}{\pgfqpoint{0.236077in}{0.780833in}}{\pgfqpoint{0.222722in}{0.786365in}}%
\pgfpathcurveto{\pgfqpoint{0.212877in}{0.796210in}}{\pgfqpoint{0.203032in}{0.806055in}}{\pgfqpoint{0.197500in}{0.819410in}}%
\pgfpathcurveto{\pgfqpoint{0.197500in}{0.833333in}}{\pgfqpoint{0.197500in}{0.847256in}}{\pgfqpoint{0.203032in}{0.860611in}}%
\pgfpathcurveto{\pgfqpoint{0.212877in}{0.870456in}}{\pgfqpoint{0.222722in}{0.880302in}}{\pgfqpoint{0.236077in}{0.885833in}}%
\pgfpathcurveto{\pgfqpoint{0.250000in}{0.885833in}}{\pgfqpoint{0.263923in}{0.885833in}}{\pgfqpoint{0.277278in}{0.880302in}}%
\pgfpathcurveto{\pgfqpoint{0.287123in}{0.870456in}}{\pgfqpoint{0.296968in}{0.860611in}}{\pgfqpoint{0.302500in}{0.847256in}}%
\pgfpathcurveto{\pgfqpoint{0.302500in}{0.833333in}}{\pgfqpoint{0.302500in}{0.819410in}}{\pgfqpoint{0.296968in}{0.806055in}}%
\pgfpathcurveto{\pgfqpoint{0.287123in}{0.796210in}}{\pgfqpoint{0.277278in}{0.786365in}}{\pgfqpoint{0.263923in}{0.780833in}}%
\pgfpathclose%
\pgfpathmoveto{\pgfqpoint{0.416667in}{0.775000in}}%
\pgfpathcurveto{\pgfqpoint{0.432137in}{0.775000in}}{\pgfqpoint{0.446975in}{0.781146in}}{\pgfqpoint{0.457915in}{0.792085in}}%
\pgfpathcurveto{\pgfqpoint{0.468854in}{0.803025in}}{\pgfqpoint{0.475000in}{0.817863in}}{\pgfqpoint{0.475000in}{0.833333in}}%
\pgfpathcurveto{\pgfqpoint{0.475000in}{0.848804in}}{\pgfqpoint{0.468854in}{0.863642in}}{\pgfqpoint{0.457915in}{0.874581in}}%
\pgfpathcurveto{\pgfqpoint{0.446975in}{0.885520in}}{\pgfqpoint{0.432137in}{0.891667in}}{\pgfqpoint{0.416667in}{0.891667in}}%
\pgfpathcurveto{\pgfqpoint{0.401196in}{0.891667in}}{\pgfqpoint{0.386358in}{0.885520in}}{\pgfqpoint{0.375419in}{0.874581in}}%
\pgfpathcurveto{\pgfqpoint{0.364480in}{0.863642in}}{\pgfqpoint{0.358333in}{0.848804in}}{\pgfqpoint{0.358333in}{0.833333in}}%
\pgfpathcurveto{\pgfqpoint{0.358333in}{0.817863in}}{\pgfqpoint{0.364480in}{0.803025in}}{\pgfqpoint{0.375419in}{0.792085in}}%
\pgfpathcurveto{\pgfqpoint{0.386358in}{0.781146in}}{\pgfqpoint{0.401196in}{0.775000in}}{\pgfqpoint{0.416667in}{0.775000in}}%
\pgfpathclose%
\pgfpathmoveto{\pgfqpoint{0.416667in}{0.780833in}}%
\pgfpathcurveto{\pgfqpoint{0.416667in}{0.780833in}}{\pgfqpoint{0.402744in}{0.780833in}}{\pgfqpoint{0.389389in}{0.786365in}}%
\pgfpathcurveto{\pgfqpoint{0.379544in}{0.796210in}}{\pgfqpoint{0.369698in}{0.806055in}}{\pgfqpoint{0.364167in}{0.819410in}}%
\pgfpathcurveto{\pgfqpoint{0.364167in}{0.833333in}}{\pgfqpoint{0.364167in}{0.847256in}}{\pgfqpoint{0.369698in}{0.860611in}}%
\pgfpathcurveto{\pgfqpoint{0.379544in}{0.870456in}}{\pgfqpoint{0.389389in}{0.880302in}}{\pgfqpoint{0.402744in}{0.885833in}}%
\pgfpathcurveto{\pgfqpoint{0.416667in}{0.885833in}}{\pgfqpoint{0.430590in}{0.885833in}}{\pgfqpoint{0.443945in}{0.880302in}}%
\pgfpathcurveto{\pgfqpoint{0.453790in}{0.870456in}}{\pgfqpoint{0.463635in}{0.860611in}}{\pgfqpoint{0.469167in}{0.847256in}}%
\pgfpathcurveto{\pgfqpoint{0.469167in}{0.833333in}}{\pgfqpoint{0.469167in}{0.819410in}}{\pgfqpoint{0.463635in}{0.806055in}}%
\pgfpathcurveto{\pgfqpoint{0.453790in}{0.796210in}}{\pgfqpoint{0.443945in}{0.786365in}}{\pgfqpoint{0.430590in}{0.780833in}}%
\pgfpathclose%
\pgfpathmoveto{\pgfqpoint{0.583333in}{0.775000in}}%
\pgfpathcurveto{\pgfqpoint{0.598804in}{0.775000in}}{\pgfqpoint{0.613642in}{0.781146in}}{\pgfqpoint{0.624581in}{0.792085in}}%
\pgfpathcurveto{\pgfqpoint{0.635520in}{0.803025in}}{\pgfqpoint{0.641667in}{0.817863in}}{\pgfqpoint{0.641667in}{0.833333in}}%
\pgfpathcurveto{\pgfqpoint{0.641667in}{0.848804in}}{\pgfqpoint{0.635520in}{0.863642in}}{\pgfqpoint{0.624581in}{0.874581in}}%
\pgfpathcurveto{\pgfqpoint{0.613642in}{0.885520in}}{\pgfqpoint{0.598804in}{0.891667in}}{\pgfqpoint{0.583333in}{0.891667in}}%
\pgfpathcurveto{\pgfqpoint{0.567863in}{0.891667in}}{\pgfqpoint{0.553025in}{0.885520in}}{\pgfqpoint{0.542085in}{0.874581in}}%
\pgfpathcurveto{\pgfqpoint{0.531146in}{0.863642in}}{\pgfqpoint{0.525000in}{0.848804in}}{\pgfqpoint{0.525000in}{0.833333in}}%
\pgfpathcurveto{\pgfqpoint{0.525000in}{0.817863in}}{\pgfqpoint{0.531146in}{0.803025in}}{\pgfqpoint{0.542085in}{0.792085in}}%
\pgfpathcurveto{\pgfqpoint{0.553025in}{0.781146in}}{\pgfqpoint{0.567863in}{0.775000in}}{\pgfqpoint{0.583333in}{0.775000in}}%
\pgfpathclose%
\pgfpathmoveto{\pgfqpoint{0.583333in}{0.780833in}}%
\pgfpathcurveto{\pgfqpoint{0.583333in}{0.780833in}}{\pgfqpoint{0.569410in}{0.780833in}}{\pgfqpoint{0.556055in}{0.786365in}}%
\pgfpathcurveto{\pgfqpoint{0.546210in}{0.796210in}}{\pgfqpoint{0.536365in}{0.806055in}}{\pgfqpoint{0.530833in}{0.819410in}}%
\pgfpathcurveto{\pgfqpoint{0.530833in}{0.833333in}}{\pgfqpoint{0.530833in}{0.847256in}}{\pgfqpoint{0.536365in}{0.860611in}}%
\pgfpathcurveto{\pgfqpoint{0.546210in}{0.870456in}}{\pgfqpoint{0.556055in}{0.880302in}}{\pgfqpoint{0.569410in}{0.885833in}}%
\pgfpathcurveto{\pgfqpoint{0.583333in}{0.885833in}}{\pgfqpoint{0.597256in}{0.885833in}}{\pgfqpoint{0.610611in}{0.880302in}}%
\pgfpathcurveto{\pgfqpoint{0.620456in}{0.870456in}}{\pgfqpoint{0.630302in}{0.860611in}}{\pgfqpoint{0.635833in}{0.847256in}}%
\pgfpathcurveto{\pgfqpoint{0.635833in}{0.833333in}}{\pgfqpoint{0.635833in}{0.819410in}}{\pgfqpoint{0.630302in}{0.806055in}}%
\pgfpathcurveto{\pgfqpoint{0.620456in}{0.796210in}}{\pgfqpoint{0.610611in}{0.786365in}}{\pgfqpoint{0.597256in}{0.780833in}}%
\pgfpathclose%
\pgfpathmoveto{\pgfqpoint{0.750000in}{0.775000in}}%
\pgfpathcurveto{\pgfqpoint{0.765470in}{0.775000in}}{\pgfqpoint{0.780309in}{0.781146in}}{\pgfqpoint{0.791248in}{0.792085in}}%
\pgfpathcurveto{\pgfqpoint{0.802187in}{0.803025in}}{\pgfqpoint{0.808333in}{0.817863in}}{\pgfqpoint{0.808333in}{0.833333in}}%
\pgfpathcurveto{\pgfqpoint{0.808333in}{0.848804in}}{\pgfqpoint{0.802187in}{0.863642in}}{\pgfqpoint{0.791248in}{0.874581in}}%
\pgfpathcurveto{\pgfqpoint{0.780309in}{0.885520in}}{\pgfqpoint{0.765470in}{0.891667in}}{\pgfqpoint{0.750000in}{0.891667in}}%
\pgfpathcurveto{\pgfqpoint{0.734530in}{0.891667in}}{\pgfqpoint{0.719691in}{0.885520in}}{\pgfqpoint{0.708752in}{0.874581in}}%
\pgfpathcurveto{\pgfqpoint{0.697813in}{0.863642in}}{\pgfqpoint{0.691667in}{0.848804in}}{\pgfqpoint{0.691667in}{0.833333in}}%
\pgfpathcurveto{\pgfqpoint{0.691667in}{0.817863in}}{\pgfqpoint{0.697813in}{0.803025in}}{\pgfqpoint{0.708752in}{0.792085in}}%
\pgfpathcurveto{\pgfqpoint{0.719691in}{0.781146in}}{\pgfqpoint{0.734530in}{0.775000in}}{\pgfqpoint{0.750000in}{0.775000in}}%
\pgfpathclose%
\pgfpathmoveto{\pgfqpoint{0.750000in}{0.780833in}}%
\pgfpathcurveto{\pgfqpoint{0.750000in}{0.780833in}}{\pgfqpoint{0.736077in}{0.780833in}}{\pgfqpoint{0.722722in}{0.786365in}}%
\pgfpathcurveto{\pgfqpoint{0.712877in}{0.796210in}}{\pgfqpoint{0.703032in}{0.806055in}}{\pgfqpoint{0.697500in}{0.819410in}}%
\pgfpathcurveto{\pgfqpoint{0.697500in}{0.833333in}}{\pgfqpoint{0.697500in}{0.847256in}}{\pgfqpoint{0.703032in}{0.860611in}}%
\pgfpathcurveto{\pgfqpoint{0.712877in}{0.870456in}}{\pgfqpoint{0.722722in}{0.880302in}}{\pgfqpoint{0.736077in}{0.885833in}}%
\pgfpathcurveto{\pgfqpoint{0.750000in}{0.885833in}}{\pgfqpoint{0.763923in}{0.885833in}}{\pgfqpoint{0.777278in}{0.880302in}}%
\pgfpathcurveto{\pgfqpoint{0.787123in}{0.870456in}}{\pgfqpoint{0.796968in}{0.860611in}}{\pgfqpoint{0.802500in}{0.847256in}}%
\pgfpathcurveto{\pgfqpoint{0.802500in}{0.833333in}}{\pgfqpoint{0.802500in}{0.819410in}}{\pgfqpoint{0.796968in}{0.806055in}}%
\pgfpathcurveto{\pgfqpoint{0.787123in}{0.796210in}}{\pgfqpoint{0.777278in}{0.786365in}}{\pgfqpoint{0.763923in}{0.780833in}}%
\pgfpathclose%
\pgfpathmoveto{\pgfqpoint{0.916667in}{0.775000in}}%
\pgfpathcurveto{\pgfqpoint{0.932137in}{0.775000in}}{\pgfqpoint{0.946975in}{0.781146in}}{\pgfqpoint{0.957915in}{0.792085in}}%
\pgfpathcurveto{\pgfqpoint{0.968854in}{0.803025in}}{\pgfqpoint{0.975000in}{0.817863in}}{\pgfqpoint{0.975000in}{0.833333in}}%
\pgfpathcurveto{\pgfqpoint{0.975000in}{0.848804in}}{\pgfqpoint{0.968854in}{0.863642in}}{\pgfqpoint{0.957915in}{0.874581in}}%
\pgfpathcurveto{\pgfqpoint{0.946975in}{0.885520in}}{\pgfqpoint{0.932137in}{0.891667in}}{\pgfqpoint{0.916667in}{0.891667in}}%
\pgfpathcurveto{\pgfqpoint{0.901196in}{0.891667in}}{\pgfqpoint{0.886358in}{0.885520in}}{\pgfqpoint{0.875419in}{0.874581in}}%
\pgfpathcurveto{\pgfqpoint{0.864480in}{0.863642in}}{\pgfqpoint{0.858333in}{0.848804in}}{\pgfqpoint{0.858333in}{0.833333in}}%
\pgfpathcurveto{\pgfqpoint{0.858333in}{0.817863in}}{\pgfqpoint{0.864480in}{0.803025in}}{\pgfqpoint{0.875419in}{0.792085in}}%
\pgfpathcurveto{\pgfqpoint{0.886358in}{0.781146in}}{\pgfqpoint{0.901196in}{0.775000in}}{\pgfqpoint{0.916667in}{0.775000in}}%
\pgfpathclose%
\pgfpathmoveto{\pgfqpoint{0.916667in}{0.780833in}}%
\pgfpathcurveto{\pgfqpoint{0.916667in}{0.780833in}}{\pgfqpoint{0.902744in}{0.780833in}}{\pgfqpoint{0.889389in}{0.786365in}}%
\pgfpathcurveto{\pgfqpoint{0.879544in}{0.796210in}}{\pgfqpoint{0.869698in}{0.806055in}}{\pgfqpoint{0.864167in}{0.819410in}}%
\pgfpathcurveto{\pgfqpoint{0.864167in}{0.833333in}}{\pgfqpoint{0.864167in}{0.847256in}}{\pgfqpoint{0.869698in}{0.860611in}}%
\pgfpathcurveto{\pgfqpoint{0.879544in}{0.870456in}}{\pgfqpoint{0.889389in}{0.880302in}}{\pgfqpoint{0.902744in}{0.885833in}}%
\pgfpathcurveto{\pgfqpoint{0.916667in}{0.885833in}}{\pgfqpoint{0.930590in}{0.885833in}}{\pgfqpoint{0.943945in}{0.880302in}}%
\pgfpathcurveto{\pgfqpoint{0.953790in}{0.870456in}}{\pgfqpoint{0.963635in}{0.860611in}}{\pgfqpoint{0.969167in}{0.847256in}}%
\pgfpathcurveto{\pgfqpoint{0.969167in}{0.833333in}}{\pgfqpoint{0.969167in}{0.819410in}}{\pgfqpoint{0.963635in}{0.806055in}}%
\pgfpathcurveto{\pgfqpoint{0.953790in}{0.796210in}}{\pgfqpoint{0.943945in}{0.786365in}}{\pgfqpoint{0.930590in}{0.780833in}}%
\pgfpathclose%
\pgfpathmoveto{\pgfqpoint{0.000000in}{0.941667in}}%
\pgfpathcurveto{\pgfqpoint{0.015470in}{0.941667in}}{\pgfqpoint{0.030309in}{0.947813in}}{\pgfqpoint{0.041248in}{0.958752in}}%
\pgfpathcurveto{\pgfqpoint{0.052187in}{0.969691in}}{\pgfqpoint{0.058333in}{0.984530in}}{\pgfqpoint{0.058333in}{1.000000in}}%
\pgfpathcurveto{\pgfqpoint{0.058333in}{1.015470in}}{\pgfqpoint{0.052187in}{1.030309in}}{\pgfqpoint{0.041248in}{1.041248in}}%
\pgfpathcurveto{\pgfqpoint{0.030309in}{1.052187in}}{\pgfqpoint{0.015470in}{1.058333in}}{\pgfqpoint{0.000000in}{1.058333in}}%
\pgfpathcurveto{\pgfqpoint{-0.015470in}{1.058333in}}{\pgfqpoint{-0.030309in}{1.052187in}}{\pgfqpoint{-0.041248in}{1.041248in}}%
\pgfpathcurveto{\pgfqpoint{-0.052187in}{1.030309in}}{\pgfqpoint{-0.058333in}{1.015470in}}{\pgfqpoint{-0.058333in}{1.000000in}}%
\pgfpathcurveto{\pgfqpoint{-0.058333in}{0.984530in}}{\pgfqpoint{-0.052187in}{0.969691in}}{\pgfqpoint{-0.041248in}{0.958752in}}%
\pgfpathcurveto{\pgfqpoint{-0.030309in}{0.947813in}}{\pgfqpoint{-0.015470in}{0.941667in}}{\pgfqpoint{0.000000in}{0.941667in}}%
\pgfpathclose%
\pgfpathmoveto{\pgfqpoint{0.000000in}{0.947500in}}%
\pgfpathcurveto{\pgfqpoint{0.000000in}{0.947500in}}{\pgfqpoint{-0.013923in}{0.947500in}}{\pgfqpoint{-0.027278in}{0.953032in}}%
\pgfpathcurveto{\pgfqpoint{-0.037123in}{0.962877in}}{\pgfqpoint{-0.046968in}{0.972722in}}{\pgfqpoint{-0.052500in}{0.986077in}}%
\pgfpathcurveto{\pgfqpoint{-0.052500in}{1.000000in}}{\pgfqpoint{-0.052500in}{1.013923in}}{\pgfqpoint{-0.046968in}{1.027278in}}%
\pgfpathcurveto{\pgfqpoint{-0.037123in}{1.037123in}}{\pgfqpoint{-0.027278in}{1.046968in}}{\pgfqpoint{-0.013923in}{1.052500in}}%
\pgfpathcurveto{\pgfqpoint{0.000000in}{1.052500in}}{\pgfqpoint{0.013923in}{1.052500in}}{\pgfqpoint{0.027278in}{1.046968in}}%
\pgfpathcurveto{\pgfqpoint{0.037123in}{1.037123in}}{\pgfqpoint{0.046968in}{1.027278in}}{\pgfqpoint{0.052500in}{1.013923in}}%
\pgfpathcurveto{\pgfqpoint{0.052500in}{1.000000in}}{\pgfqpoint{0.052500in}{0.986077in}}{\pgfqpoint{0.046968in}{0.972722in}}%
\pgfpathcurveto{\pgfqpoint{0.037123in}{0.962877in}}{\pgfqpoint{0.027278in}{0.953032in}}{\pgfqpoint{0.013923in}{0.947500in}}%
\pgfpathclose%
\pgfpathmoveto{\pgfqpoint{0.166667in}{0.941667in}}%
\pgfpathcurveto{\pgfqpoint{0.182137in}{0.941667in}}{\pgfqpoint{0.196975in}{0.947813in}}{\pgfqpoint{0.207915in}{0.958752in}}%
\pgfpathcurveto{\pgfqpoint{0.218854in}{0.969691in}}{\pgfqpoint{0.225000in}{0.984530in}}{\pgfqpoint{0.225000in}{1.000000in}}%
\pgfpathcurveto{\pgfqpoint{0.225000in}{1.015470in}}{\pgfqpoint{0.218854in}{1.030309in}}{\pgfqpoint{0.207915in}{1.041248in}}%
\pgfpathcurveto{\pgfqpoint{0.196975in}{1.052187in}}{\pgfqpoint{0.182137in}{1.058333in}}{\pgfqpoint{0.166667in}{1.058333in}}%
\pgfpathcurveto{\pgfqpoint{0.151196in}{1.058333in}}{\pgfqpoint{0.136358in}{1.052187in}}{\pgfqpoint{0.125419in}{1.041248in}}%
\pgfpathcurveto{\pgfqpoint{0.114480in}{1.030309in}}{\pgfqpoint{0.108333in}{1.015470in}}{\pgfqpoint{0.108333in}{1.000000in}}%
\pgfpathcurveto{\pgfqpoint{0.108333in}{0.984530in}}{\pgfqpoint{0.114480in}{0.969691in}}{\pgfqpoint{0.125419in}{0.958752in}}%
\pgfpathcurveto{\pgfqpoint{0.136358in}{0.947813in}}{\pgfqpoint{0.151196in}{0.941667in}}{\pgfqpoint{0.166667in}{0.941667in}}%
\pgfpathclose%
\pgfpathmoveto{\pgfqpoint{0.166667in}{0.947500in}}%
\pgfpathcurveto{\pgfqpoint{0.166667in}{0.947500in}}{\pgfqpoint{0.152744in}{0.947500in}}{\pgfqpoint{0.139389in}{0.953032in}}%
\pgfpathcurveto{\pgfqpoint{0.129544in}{0.962877in}}{\pgfqpoint{0.119698in}{0.972722in}}{\pgfqpoint{0.114167in}{0.986077in}}%
\pgfpathcurveto{\pgfqpoint{0.114167in}{1.000000in}}{\pgfqpoint{0.114167in}{1.013923in}}{\pgfqpoint{0.119698in}{1.027278in}}%
\pgfpathcurveto{\pgfqpoint{0.129544in}{1.037123in}}{\pgfqpoint{0.139389in}{1.046968in}}{\pgfqpoint{0.152744in}{1.052500in}}%
\pgfpathcurveto{\pgfqpoint{0.166667in}{1.052500in}}{\pgfqpoint{0.180590in}{1.052500in}}{\pgfqpoint{0.193945in}{1.046968in}}%
\pgfpathcurveto{\pgfqpoint{0.203790in}{1.037123in}}{\pgfqpoint{0.213635in}{1.027278in}}{\pgfqpoint{0.219167in}{1.013923in}}%
\pgfpathcurveto{\pgfqpoint{0.219167in}{1.000000in}}{\pgfqpoint{0.219167in}{0.986077in}}{\pgfqpoint{0.213635in}{0.972722in}}%
\pgfpathcurveto{\pgfqpoint{0.203790in}{0.962877in}}{\pgfqpoint{0.193945in}{0.953032in}}{\pgfqpoint{0.180590in}{0.947500in}}%
\pgfpathclose%
\pgfpathmoveto{\pgfqpoint{0.333333in}{0.941667in}}%
\pgfpathcurveto{\pgfqpoint{0.348804in}{0.941667in}}{\pgfqpoint{0.363642in}{0.947813in}}{\pgfqpoint{0.374581in}{0.958752in}}%
\pgfpathcurveto{\pgfqpoint{0.385520in}{0.969691in}}{\pgfqpoint{0.391667in}{0.984530in}}{\pgfqpoint{0.391667in}{1.000000in}}%
\pgfpathcurveto{\pgfqpoint{0.391667in}{1.015470in}}{\pgfqpoint{0.385520in}{1.030309in}}{\pgfqpoint{0.374581in}{1.041248in}}%
\pgfpathcurveto{\pgfqpoint{0.363642in}{1.052187in}}{\pgfqpoint{0.348804in}{1.058333in}}{\pgfqpoint{0.333333in}{1.058333in}}%
\pgfpathcurveto{\pgfqpoint{0.317863in}{1.058333in}}{\pgfqpoint{0.303025in}{1.052187in}}{\pgfqpoint{0.292085in}{1.041248in}}%
\pgfpathcurveto{\pgfqpoint{0.281146in}{1.030309in}}{\pgfqpoint{0.275000in}{1.015470in}}{\pgfqpoint{0.275000in}{1.000000in}}%
\pgfpathcurveto{\pgfqpoint{0.275000in}{0.984530in}}{\pgfqpoint{0.281146in}{0.969691in}}{\pgfqpoint{0.292085in}{0.958752in}}%
\pgfpathcurveto{\pgfqpoint{0.303025in}{0.947813in}}{\pgfqpoint{0.317863in}{0.941667in}}{\pgfqpoint{0.333333in}{0.941667in}}%
\pgfpathclose%
\pgfpathmoveto{\pgfqpoint{0.333333in}{0.947500in}}%
\pgfpathcurveto{\pgfqpoint{0.333333in}{0.947500in}}{\pgfqpoint{0.319410in}{0.947500in}}{\pgfqpoint{0.306055in}{0.953032in}}%
\pgfpathcurveto{\pgfqpoint{0.296210in}{0.962877in}}{\pgfqpoint{0.286365in}{0.972722in}}{\pgfqpoint{0.280833in}{0.986077in}}%
\pgfpathcurveto{\pgfqpoint{0.280833in}{1.000000in}}{\pgfqpoint{0.280833in}{1.013923in}}{\pgfqpoint{0.286365in}{1.027278in}}%
\pgfpathcurveto{\pgfqpoint{0.296210in}{1.037123in}}{\pgfqpoint{0.306055in}{1.046968in}}{\pgfqpoint{0.319410in}{1.052500in}}%
\pgfpathcurveto{\pgfqpoint{0.333333in}{1.052500in}}{\pgfqpoint{0.347256in}{1.052500in}}{\pgfqpoint{0.360611in}{1.046968in}}%
\pgfpathcurveto{\pgfqpoint{0.370456in}{1.037123in}}{\pgfqpoint{0.380302in}{1.027278in}}{\pgfqpoint{0.385833in}{1.013923in}}%
\pgfpathcurveto{\pgfqpoint{0.385833in}{1.000000in}}{\pgfqpoint{0.385833in}{0.986077in}}{\pgfqpoint{0.380302in}{0.972722in}}%
\pgfpathcurveto{\pgfqpoint{0.370456in}{0.962877in}}{\pgfqpoint{0.360611in}{0.953032in}}{\pgfqpoint{0.347256in}{0.947500in}}%
\pgfpathclose%
\pgfpathmoveto{\pgfqpoint{0.500000in}{0.941667in}}%
\pgfpathcurveto{\pgfqpoint{0.515470in}{0.941667in}}{\pgfqpoint{0.530309in}{0.947813in}}{\pgfqpoint{0.541248in}{0.958752in}}%
\pgfpathcurveto{\pgfqpoint{0.552187in}{0.969691in}}{\pgfqpoint{0.558333in}{0.984530in}}{\pgfqpoint{0.558333in}{1.000000in}}%
\pgfpathcurveto{\pgfqpoint{0.558333in}{1.015470in}}{\pgfqpoint{0.552187in}{1.030309in}}{\pgfqpoint{0.541248in}{1.041248in}}%
\pgfpathcurveto{\pgfqpoint{0.530309in}{1.052187in}}{\pgfqpoint{0.515470in}{1.058333in}}{\pgfqpoint{0.500000in}{1.058333in}}%
\pgfpathcurveto{\pgfqpoint{0.484530in}{1.058333in}}{\pgfqpoint{0.469691in}{1.052187in}}{\pgfqpoint{0.458752in}{1.041248in}}%
\pgfpathcurveto{\pgfqpoint{0.447813in}{1.030309in}}{\pgfqpoint{0.441667in}{1.015470in}}{\pgfqpoint{0.441667in}{1.000000in}}%
\pgfpathcurveto{\pgfqpoint{0.441667in}{0.984530in}}{\pgfqpoint{0.447813in}{0.969691in}}{\pgfqpoint{0.458752in}{0.958752in}}%
\pgfpathcurveto{\pgfqpoint{0.469691in}{0.947813in}}{\pgfqpoint{0.484530in}{0.941667in}}{\pgfqpoint{0.500000in}{0.941667in}}%
\pgfpathclose%
\pgfpathmoveto{\pgfqpoint{0.500000in}{0.947500in}}%
\pgfpathcurveto{\pgfqpoint{0.500000in}{0.947500in}}{\pgfqpoint{0.486077in}{0.947500in}}{\pgfqpoint{0.472722in}{0.953032in}}%
\pgfpathcurveto{\pgfqpoint{0.462877in}{0.962877in}}{\pgfqpoint{0.453032in}{0.972722in}}{\pgfqpoint{0.447500in}{0.986077in}}%
\pgfpathcurveto{\pgfqpoint{0.447500in}{1.000000in}}{\pgfqpoint{0.447500in}{1.013923in}}{\pgfqpoint{0.453032in}{1.027278in}}%
\pgfpathcurveto{\pgfqpoint{0.462877in}{1.037123in}}{\pgfqpoint{0.472722in}{1.046968in}}{\pgfqpoint{0.486077in}{1.052500in}}%
\pgfpathcurveto{\pgfqpoint{0.500000in}{1.052500in}}{\pgfqpoint{0.513923in}{1.052500in}}{\pgfqpoint{0.527278in}{1.046968in}}%
\pgfpathcurveto{\pgfqpoint{0.537123in}{1.037123in}}{\pgfqpoint{0.546968in}{1.027278in}}{\pgfqpoint{0.552500in}{1.013923in}}%
\pgfpathcurveto{\pgfqpoint{0.552500in}{1.000000in}}{\pgfqpoint{0.552500in}{0.986077in}}{\pgfqpoint{0.546968in}{0.972722in}}%
\pgfpathcurveto{\pgfqpoint{0.537123in}{0.962877in}}{\pgfqpoint{0.527278in}{0.953032in}}{\pgfqpoint{0.513923in}{0.947500in}}%
\pgfpathclose%
\pgfpathmoveto{\pgfqpoint{0.666667in}{0.941667in}}%
\pgfpathcurveto{\pgfqpoint{0.682137in}{0.941667in}}{\pgfqpoint{0.696975in}{0.947813in}}{\pgfqpoint{0.707915in}{0.958752in}}%
\pgfpathcurveto{\pgfqpoint{0.718854in}{0.969691in}}{\pgfqpoint{0.725000in}{0.984530in}}{\pgfqpoint{0.725000in}{1.000000in}}%
\pgfpathcurveto{\pgfqpoint{0.725000in}{1.015470in}}{\pgfqpoint{0.718854in}{1.030309in}}{\pgfqpoint{0.707915in}{1.041248in}}%
\pgfpathcurveto{\pgfqpoint{0.696975in}{1.052187in}}{\pgfqpoint{0.682137in}{1.058333in}}{\pgfqpoint{0.666667in}{1.058333in}}%
\pgfpathcurveto{\pgfqpoint{0.651196in}{1.058333in}}{\pgfqpoint{0.636358in}{1.052187in}}{\pgfqpoint{0.625419in}{1.041248in}}%
\pgfpathcurveto{\pgfqpoint{0.614480in}{1.030309in}}{\pgfqpoint{0.608333in}{1.015470in}}{\pgfqpoint{0.608333in}{1.000000in}}%
\pgfpathcurveto{\pgfqpoint{0.608333in}{0.984530in}}{\pgfqpoint{0.614480in}{0.969691in}}{\pgfqpoint{0.625419in}{0.958752in}}%
\pgfpathcurveto{\pgfqpoint{0.636358in}{0.947813in}}{\pgfqpoint{0.651196in}{0.941667in}}{\pgfqpoint{0.666667in}{0.941667in}}%
\pgfpathclose%
\pgfpathmoveto{\pgfqpoint{0.666667in}{0.947500in}}%
\pgfpathcurveto{\pgfqpoint{0.666667in}{0.947500in}}{\pgfqpoint{0.652744in}{0.947500in}}{\pgfqpoint{0.639389in}{0.953032in}}%
\pgfpathcurveto{\pgfqpoint{0.629544in}{0.962877in}}{\pgfqpoint{0.619698in}{0.972722in}}{\pgfqpoint{0.614167in}{0.986077in}}%
\pgfpathcurveto{\pgfqpoint{0.614167in}{1.000000in}}{\pgfqpoint{0.614167in}{1.013923in}}{\pgfqpoint{0.619698in}{1.027278in}}%
\pgfpathcurveto{\pgfqpoint{0.629544in}{1.037123in}}{\pgfqpoint{0.639389in}{1.046968in}}{\pgfqpoint{0.652744in}{1.052500in}}%
\pgfpathcurveto{\pgfqpoint{0.666667in}{1.052500in}}{\pgfqpoint{0.680590in}{1.052500in}}{\pgfqpoint{0.693945in}{1.046968in}}%
\pgfpathcurveto{\pgfqpoint{0.703790in}{1.037123in}}{\pgfqpoint{0.713635in}{1.027278in}}{\pgfqpoint{0.719167in}{1.013923in}}%
\pgfpathcurveto{\pgfqpoint{0.719167in}{1.000000in}}{\pgfqpoint{0.719167in}{0.986077in}}{\pgfqpoint{0.713635in}{0.972722in}}%
\pgfpathcurveto{\pgfqpoint{0.703790in}{0.962877in}}{\pgfqpoint{0.693945in}{0.953032in}}{\pgfqpoint{0.680590in}{0.947500in}}%
\pgfpathclose%
\pgfpathmoveto{\pgfqpoint{0.833333in}{0.941667in}}%
\pgfpathcurveto{\pgfqpoint{0.848804in}{0.941667in}}{\pgfqpoint{0.863642in}{0.947813in}}{\pgfqpoint{0.874581in}{0.958752in}}%
\pgfpathcurveto{\pgfqpoint{0.885520in}{0.969691in}}{\pgfqpoint{0.891667in}{0.984530in}}{\pgfqpoint{0.891667in}{1.000000in}}%
\pgfpathcurveto{\pgfqpoint{0.891667in}{1.015470in}}{\pgfqpoint{0.885520in}{1.030309in}}{\pgfqpoint{0.874581in}{1.041248in}}%
\pgfpathcurveto{\pgfqpoint{0.863642in}{1.052187in}}{\pgfqpoint{0.848804in}{1.058333in}}{\pgfqpoint{0.833333in}{1.058333in}}%
\pgfpathcurveto{\pgfqpoint{0.817863in}{1.058333in}}{\pgfqpoint{0.803025in}{1.052187in}}{\pgfqpoint{0.792085in}{1.041248in}}%
\pgfpathcurveto{\pgfqpoint{0.781146in}{1.030309in}}{\pgfqpoint{0.775000in}{1.015470in}}{\pgfqpoint{0.775000in}{1.000000in}}%
\pgfpathcurveto{\pgfqpoint{0.775000in}{0.984530in}}{\pgfqpoint{0.781146in}{0.969691in}}{\pgfqpoint{0.792085in}{0.958752in}}%
\pgfpathcurveto{\pgfqpoint{0.803025in}{0.947813in}}{\pgfqpoint{0.817863in}{0.941667in}}{\pgfqpoint{0.833333in}{0.941667in}}%
\pgfpathclose%
\pgfpathmoveto{\pgfqpoint{0.833333in}{0.947500in}}%
\pgfpathcurveto{\pgfqpoint{0.833333in}{0.947500in}}{\pgfqpoint{0.819410in}{0.947500in}}{\pgfqpoint{0.806055in}{0.953032in}}%
\pgfpathcurveto{\pgfqpoint{0.796210in}{0.962877in}}{\pgfqpoint{0.786365in}{0.972722in}}{\pgfqpoint{0.780833in}{0.986077in}}%
\pgfpathcurveto{\pgfqpoint{0.780833in}{1.000000in}}{\pgfqpoint{0.780833in}{1.013923in}}{\pgfqpoint{0.786365in}{1.027278in}}%
\pgfpathcurveto{\pgfqpoint{0.796210in}{1.037123in}}{\pgfqpoint{0.806055in}{1.046968in}}{\pgfqpoint{0.819410in}{1.052500in}}%
\pgfpathcurveto{\pgfqpoint{0.833333in}{1.052500in}}{\pgfqpoint{0.847256in}{1.052500in}}{\pgfqpoint{0.860611in}{1.046968in}}%
\pgfpathcurveto{\pgfqpoint{0.870456in}{1.037123in}}{\pgfqpoint{0.880302in}{1.027278in}}{\pgfqpoint{0.885833in}{1.013923in}}%
\pgfpathcurveto{\pgfqpoint{0.885833in}{1.000000in}}{\pgfqpoint{0.885833in}{0.986077in}}{\pgfqpoint{0.880302in}{0.972722in}}%
\pgfpathcurveto{\pgfqpoint{0.870456in}{0.962877in}}{\pgfqpoint{0.860611in}{0.953032in}}{\pgfqpoint{0.847256in}{0.947500in}}%
\pgfpathclose%
\pgfpathmoveto{\pgfqpoint{1.000000in}{0.941667in}}%
\pgfpathcurveto{\pgfqpoint{1.015470in}{0.941667in}}{\pgfqpoint{1.030309in}{0.947813in}}{\pgfqpoint{1.041248in}{0.958752in}}%
\pgfpathcurveto{\pgfqpoint{1.052187in}{0.969691in}}{\pgfqpoint{1.058333in}{0.984530in}}{\pgfqpoint{1.058333in}{1.000000in}}%
\pgfpathcurveto{\pgfqpoint{1.058333in}{1.015470in}}{\pgfqpoint{1.052187in}{1.030309in}}{\pgfqpoint{1.041248in}{1.041248in}}%
\pgfpathcurveto{\pgfqpoint{1.030309in}{1.052187in}}{\pgfqpoint{1.015470in}{1.058333in}}{\pgfqpoint{1.000000in}{1.058333in}}%
\pgfpathcurveto{\pgfqpoint{0.984530in}{1.058333in}}{\pgfqpoint{0.969691in}{1.052187in}}{\pgfqpoint{0.958752in}{1.041248in}}%
\pgfpathcurveto{\pgfqpoint{0.947813in}{1.030309in}}{\pgfqpoint{0.941667in}{1.015470in}}{\pgfqpoint{0.941667in}{1.000000in}}%
\pgfpathcurveto{\pgfqpoint{0.941667in}{0.984530in}}{\pgfqpoint{0.947813in}{0.969691in}}{\pgfqpoint{0.958752in}{0.958752in}}%
\pgfpathcurveto{\pgfqpoint{0.969691in}{0.947813in}}{\pgfqpoint{0.984530in}{0.941667in}}{\pgfqpoint{1.000000in}{0.941667in}}%
\pgfpathclose%
\pgfpathmoveto{\pgfqpoint{1.000000in}{0.947500in}}%
\pgfpathcurveto{\pgfqpoint{1.000000in}{0.947500in}}{\pgfqpoint{0.986077in}{0.947500in}}{\pgfqpoint{0.972722in}{0.953032in}}%
\pgfpathcurveto{\pgfqpoint{0.962877in}{0.962877in}}{\pgfqpoint{0.953032in}{0.972722in}}{\pgfqpoint{0.947500in}{0.986077in}}%
\pgfpathcurveto{\pgfqpoint{0.947500in}{1.000000in}}{\pgfqpoint{0.947500in}{1.013923in}}{\pgfqpoint{0.953032in}{1.027278in}}%
\pgfpathcurveto{\pgfqpoint{0.962877in}{1.037123in}}{\pgfqpoint{0.972722in}{1.046968in}}{\pgfqpoint{0.986077in}{1.052500in}}%
\pgfpathcurveto{\pgfqpoint{1.000000in}{1.052500in}}{\pgfqpoint{1.013923in}{1.052500in}}{\pgfqpoint{1.027278in}{1.046968in}}%
\pgfpathcurveto{\pgfqpoint{1.037123in}{1.037123in}}{\pgfqpoint{1.046968in}{1.027278in}}{\pgfqpoint{1.052500in}{1.013923in}}%
\pgfpathcurveto{\pgfqpoint{1.052500in}{1.000000in}}{\pgfqpoint{1.052500in}{0.986077in}}{\pgfqpoint{1.046968in}{0.972722in}}%
\pgfpathcurveto{\pgfqpoint{1.037123in}{0.962877in}}{\pgfqpoint{1.027278in}{0.953032in}}{\pgfqpoint{1.013923in}{0.947500in}}%
\pgfpathclose%
\pgfusepath{stroke}%
\end{pgfscope}%
}%
\pgfsys@transformshift{2.873315in}{2.180049in}%
\pgfsys@useobject{currentpattern}{}%
\pgfsys@transformshift{1in}{0in}%
\pgfsys@transformshift{-1in}{0in}%
\pgfsys@transformshift{0in}{1in}%
\pgfsys@useobject{currentpattern}{}%
\pgfsys@transformshift{1in}{0in}%
\pgfsys@transformshift{-1in}{0in}%
\pgfsys@transformshift{0in}{1in}%
\end{pgfscope}%
\begin{pgfscope}%
\pgfpathrectangle{\pgfqpoint{0.935815in}{0.637495in}}{\pgfqpoint{9.300000in}{9.060000in}}%
\pgfusepath{clip}%
\pgfsetbuttcap%
\pgfsetmiterjoin%
\definecolor{currentfill}{rgb}{0.549020,0.337255,0.294118}%
\pgfsetfillcolor{currentfill}%
\pgfsetfillopacity{0.990000}%
\pgfsetlinewidth{0.000000pt}%
\definecolor{currentstroke}{rgb}{0.000000,0.000000,0.000000}%
\pgfsetstrokecolor{currentstroke}%
\pgfsetstrokeopacity{0.990000}%
\pgfsetdash{}{0pt}%
\pgfpathmoveto{\pgfqpoint{4.423315in}{2.758071in}}%
\pgfpathlineto{\pgfqpoint{5.198315in}{2.758071in}}%
\pgfpathlineto{\pgfqpoint{5.198315in}{4.380181in}}%
\pgfpathlineto{\pgfqpoint{4.423315in}{4.380181in}}%
\pgfpathclose%
\pgfusepath{fill}%
\end{pgfscope}%
\begin{pgfscope}%
\pgfsetbuttcap%
\pgfsetmiterjoin%
\definecolor{currentfill}{rgb}{0.549020,0.337255,0.294118}%
\pgfsetfillcolor{currentfill}%
\pgfsetfillopacity{0.990000}%
\pgfsetlinewidth{0.000000pt}%
\definecolor{currentstroke}{rgb}{0.000000,0.000000,0.000000}%
\pgfsetstrokecolor{currentstroke}%
\pgfsetstrokeopacity{0.990000}%
\pgfsetdash{}{0pt}%
\pgfpathrectangle{\pgfqpoint{0.935815in}{0.637495in}}{\pgfqpoint{9.300000in}{9.060000in}}%
\pgfusepath{clip}%
\pgfpathmoveto{\pgfqpoint{4.423315in}{2.758071in}}%
\pgfpathlineto{\pgfqpoint{5.198315in}{2.758071in}}%
\pgfpathlineto{\pgfqpoint{5.198315in}{4.380181in}}%
\pgfpathlineto{\pgfqpoint{4.423315in}{4.380181in}}%
\pgfpathclose%
\pgfusepath{clip}%
\pgfsys@defobject{currentpattern}{\pgfqpoint{0in}{0in}}{\pgfqpoint{1in}{1in}}{%
\begin{pgfscope}%
\pgfpathrectangle{\pgfqpoint{0in}{0in}}{\pgfqpoint{1in}{1in}}%
\pgfusepath{clip}%
\pgfpathmoveto{\pgfqpoint{0.000000in}{-0.058333in}}%
\pgfpathcurveto{\pgfqpoint{0.015470in}{-0.058333in}}{\pgfqpoint{0.030309in}{-0.052187in}}{\pgfqpoint{0.041248in}{-0.041248in}}%
\pgfpathcurveto{\pgfqpoint{0.052187in}{-0.030309in}}{\pgfqpoint{0.058333in}{-0.015470in}}{\pgfqpoint{0.058333in}{0.000000in}}%
\pgfpathcurveto{\pgfqpoint{0.058333in}{0.015470in}}{\pgfqpoint{0.052187in}{0.030309in}}{\pgfqpoint{0.041248in}{0.041248in}}%
\pgfpathcurveto{\pgfqpoint{0.030309in}{0.052187in}}{\pgfqpoint{0.015470in}{0.058333in}}{\pgfqpoint{0.000000in}{0.058333in}}%
\pgfpathcurveto{\pgfqpoint{-0.015470in}{0.058333in}}{\pgfqpoint{-0.030309in}{0.052187in}}{\pgfqpoint{-0.041248in}{0.041248in}}%
\pgfpathcurveto{\pgfqpoint{-0.052187in}{0.030309in}}{\pgfqpoint{-0.058333in}{0.015470in}}{\pgfqpoint{-0.058333in}{0.000000in}}%
\pgfpathcurveto{\pgfqpoint{-0.058333in}{-0.015470in}}{\pgfqpoint{-0.052187in}{-0.030309in}}{\pgfqpoint{-0.041248in}{-0.041248in}}%
\pgfpathcurveto{\pgfqpoint{-0.030309in}{-0.052187in}}{\pgfqpoint{-0.015470in}{-0.058333in}}{\pgfqpoint{0.000000in}{-0.058333in}}%
\pgfpathclose%
\pgfpathmoveto{\pgfqpoint{0.000000in}{-0.052500in}}%
\pgfpathcurveto{\pgfqpoint{0.000000in}{-0.052500in}}{\pgfqpoint{-0.013923in}{-0.052500in}}{\pgfqpoint{-0.027278in}{-0.046968in}}%
\pgfpathcurveto{\pgfqpoint{-0.037123in}{-0.037123in}}{\pgfqpoint{-0.046968in}{-0.027278in}}{\pgfqpoint{-0.052500in}{-0.013923in}}%
\pgfpathcurveto{\pgfqpoint{-0.052500in}{0.000000in}}{\pgfqpoint{-0.052500in}{0.013923in}}{\pgfqpoint{-0.046968in}{0.027278in}}%
\pgfpathcurveto{\pgfqpoint{-0.037123in}{0.037123in}}{\pgfqpoint{-0.027278in}{0.046968in}}{\pgfqpoint{-0.013923in}{0.052500in}}%
\pgfpathcurveto{\pgfqpoint{0.000000in}{0.052500in}}{\pgfqpoint{0.013923in}{0.052500in}}{\pgfqpoint{0.027278in}{0.046968in}}%
\pgfpathcurveto{\pgfqpoint{0.037123in}{0.037123in}}{\pgfqpoint{0.046968in}{0.027278in}}{\pgfqpoint{0.052500in}{0.013923in}}%
\pgfpathcurveto{\pgfqpoint{0.052500in}{0.000000in}}{\pgfqpoint{0.052500in}{-0.013923in}}{\pgfqpoint{0.046968in}{-0.027278in}}%
\pgfpathcurveto{\pgfqpoint{0.037123in}{-0.037123in}}{\pgfqpoint{0.027278in}{-0.046968in}}{\pgfqpoint{0.013923in}{-0.052500in}}%
\pgfpathclose%
\pgfpathmoveto{\pgfqpoint{0.166667in}{-0.058333in}}%
\pgfpathcurveto{\pgfqpoint{0.182137in}{-0.058333in}}{\pgfqpoint{0.196975in}{-0.052187in}}{\pgfqpoint{0.207915in}{-0.041248in}}%
\pgfpathcurveto{\pgfqpoint{0.218854in}{-0.030309in}}{\pgfqpoint{0.225000in}{-0.015470in}}{\pgfqpoint{0.225000in}{0.000000in}}%
\pgfpathcurveto{\pgfqpoint{0.225000in}{0.015470in}}{\pgfqpoint{0.218854in}{0.030309in}}{\pgfqpoint{0.207915in}{0.041248in}}%
\pgfpathcurveto{\pgfqpoint{0.196975in}{0.052187in}}{\pgfqpoint{0.182137in}{0.058333in}}{\pgfqpoint{0.166667in}{0.058333in}}%
\pgfpathcurveto{\pgfqpoint{0.151196in}{0.058333in}}{\pgfqpoint{0.136358in}{0.052187in}}{\pgfqpoint{0.125419in}{0.041248in}}%
\pgfpathcurveto{\pgfqpoint{0.114480in}{0.030309in}}{\pgfqpoint{0.108333in}{0.015470in}}{\pgfqpoint{0.108333in}{0.000000in}}%
\pgfpathcurveto{\pgfqpoint{0.108333in}{-0.015470in}}{\pgfqpoint{0.114480in}{-0.030309in}}{\pgfqpoint{0.125419in}{-0.041248in}}%
\pgfpathcurveto{\pgfqpoint{0.136358in}{-0.052187in}}{\pgfqpoint{0.151196in}{-0.058333in}}{\pgfqpoint{0.166667in}{-0.058333in}}%
\pgfpathclose%
\pgfpathmoveto{\pgfqpoint{0.166667in}{-0.052500in}}%
\pgfpathcurveto{\pgfqpoint{0.166667in}{-0.052500in}}{\pgfqpoint{0.152744in}{-0.052500in}}{\pgfqpoint{0.139389in}{-0.046968in}}%
\pgfpathcurveto{\pgfqpoint{0.129544in}{-0.037123in}}{\pgfqpoint{0.119698in}{-0.027278in}}{\pgfqpoint{0.114167in}{-0.013923in}}%
\pgfpathcurveto{\pgfqpoint{0.114167in}{0.000000in}}{\pgfqpoint{0.114167in}{0.013923in}}{\pgfqpoint{0.119698in}{0.027278in}}%
\pgfpathcurveto{\pgfqpoint{0.129544in}{0.037123in}}{\pgfqpoint{0.139389in}{0.046968in}}{\pgfqpoint{0.152744in}{0.052500in}}%
\pgfpathcurveto{\pgfqpoint{0.166667in}{0.052500in}}{\pgfqpoint{0.180590in}{0.052500in}}{\pgfqpoint{0.193945in}{0.046968in}}%
\pgfpathcurveto{\pgfqpoint{0.203790in}{0.037123in}}{\pgfqpoint{0.213635in}{0.027278in}}{\pgfqpoint{0.219167in}{0.013923in}}%
\pgfpathcurveto{\pgfqpoint{0.219167in}{0.000000in}}{\pgfqpoint{0.219167in}{-0.013923in}}{\pgfqpoint{0.213635in}{-0.027278in}}%
\pgfpathcurveto{\pgfqpoint{0.203790in}{-0.037123in}}{\pgfqpoint{0.193945in}{-0.046968in}}{\pgfqpoint{0.180590in}{-0.052500in}}%
\pgfpathclose%
\pgfpathmoveto{\pgfqpoint{0.333333in}{-0.058333in}}%
\pgfpathcurveto{\pgfqpoint{0.348804in}{-0.058333in}}{\pgfqpoint{0.363642in}{-0.052187in}}{\pgfqpoint{0.374581in}{-0.041248in}}%
\pgfpathcurveto{\pgfqpoint{0.385520in}{-0.030309in}}{\pgfqpoint{0.391667in}{-0.015470in}}{\pgfqpoint{0.391667in}{0.000000in}}%
\pgfpathcurveto{\pgfqpoint{0.391667in}{0.015470in}}{\pgfqpoint{0.385520in}{0.030309in}}{\pgfqpoint{0.374581in}{0.041248in}}%
\pgfpathcurveto{\pgfqpoint{0.363642in}{0.052187in}}{\pgfqpoint{0.348804in}{0.058333in}}{\pgfqpoint{0.333333in}{0.058333in}}%
\pgfpathcurveto{\pgfqpoint{0.317863in}{0.058333in}}{\pgfqpoint{0.303025in}{0.052187in}}{\pgfqpoint{0.292085in}{0.041248in}}%
\pgfpathcurveto{\pgfqpoint{0.281146in}{0.030309in}}{\pgfqpoint{0.275000in}{0.015470in}}{\pgfqpoint{0.275000in}{0.000000in}}%
\pgfpathcurveto{\pgfqpoint{0.275000in}{-0.015470in}}{\pgfqpoint{0.281146in}{-0.030309in}}{\pgfqpoint{0.292085in}{-0.041248in}}%
\pgfpathcurveto{\pgfqpoint{0.303025in}{-0.052187in}}{\pgfqpoint{0.317863in}{-0.058333in}}{\pgfqpoint{0.333333in}{-0.058333in}}%
\pgfpathclose%
\pgfpathmoveto{\pgfqpoint{0.333333in}{-0.052500in}}%
\pgfpathcurveto{\pgfqpoint{0.333333in}{-0.052500in}}{\pgfqpoint{0.319410in}{-0.052500in}}{\pgfqpoint{0.306055in}{-0.046968in}}%
\pgfpathcurveto{\pgfqpoint{0.296210in}{-0.037123in}}{\pgfqpoint{0.286365in}{-0.027278in}}{\pgfqpoint{0.280833in}{-0.013923in}}%
\pgfpathcurveto{\pgfqpoint{0.280833in}{0.000000in}}{\pgfqpoint{0.280833in}{0.013923in}}{\pgfqpoint{0.286365in}{0.027278in}}%
\pgfpathcurveto{\pgfqpoint{0.296210in}{0.037123in}}{\pgfqpoint{0.306055in}{0.046968in}}{\pgfqpoint{0.319410in}{0.052500in}}%
\pgfpathcurveto{\pgfqpoint{0.333333in}{0.052500in}}{\pgfqpoint{0.347256in}{0.052500in}}{\pgfqpoint{0.360611in}{0.046968in}}%
\pgfpathcurveto{\pgfqpoint{0.370456in}{0.037123in}}{\pgfqpoint{0.380302in}{0.027278in}}{\pgfqpoint{0.385833in}{0.013923in}}%
\pgfpathcurveto{\pgfqpoint{0.385833in}{0.000000in}}{\pgfqpoint{0.385833in}{-0.013923in}}{\pgfqpoint{0.380302in}{-0.027278in}}%
\pgfpathcurveto{\pgfqpoint{0.370456in}{-0.037123in}}{\pgfqpoint{0.360611in}{-0.046968in}}{\pgfqpoint{0.347256in}{-0.052500in}}%
\pgfpathclose%
\pgfpathmoveto{\pgfqpoint{0.500000in}{-0.058333in}}%
\pgfpathcurveto{\pgfqpoint{0.515470in}{-0.058333in}}{\pgfqpoint{0.530309in}{-0.052187in}}{\pgfqpoint{0.541248in}{-0.041248in}}%
\pgfpathcurveto{\pgfqpoint{0.552187in}{-0.030309in}}{\pgfqpoint{0.558333in}{-0.015470in}}{\pgfqpoint{0.558333in}{0.000000in}}%
\pgfpathcurveto{\pgfqpoint{0.558333in}{0.015470in}}{\pgfqpoint{0.552187in}{0.030309in}}{\pgfqpoint{0.541248in}{0.041248in}}%
\pgfpathcurveto{\pgfqpoint{0.530309in}{0.052187in}}{\pgfqpoint{0.515470in}{0.058333in}}{\pgfqpoint{0.500000in}{0.058333in}}%
\pgfpathcurveto{\pgfqpoint{0.484530in}{0.058333in}}{\pgfqpoint{0.469691in}{0.052187in}}{\pgfqpoint{0.458752in}{0.041248in}}%
\pgfpathcurveto{\pgfqpoint{0.447813in}{0.030309in}}{\pgfqpoint{0.441667in}{0.015470in}}{\pgfqpoint{0.441667in}{0.000000in}}%
\pgfpathcurveto{\pgfqpoint{0.441667in}{-0.015470in}}{\pgfqpoint{0.447813in}{-0.030309in}}{\pgfqpoint{0.458752in}{-0.041248in}}%
\pgfpathcurveto{\pgfqpoint{0.469691in}{-0.052187in}}{\pgfqpoint{0.484530in}{-0.058333in}}{\pgfqpoint{0.500000in}{-0.058333in}}%
\pgfpathclose%
\pgfpathmoveto{\pgfqpoint{0.500000in}{-0.052500in}}%
\pgfpathcurveto{\pgfqpoint{0.500000in}{-0.052500in}}{\pgfqpoint{0.486077in}{-0.052500in}}{\pgfqpoint{0.472722in}{-0.046968in}}%
\pgfpathcurveto{\pgfqpoint{0.462877in}{-0.037123in}}{\pgfqpoint{0.453032in}{-0.027278in}}{\pgfqpoint{0.447500in}{-0.013923in}}%
\pgfpathcurveto{\pgfqpoint{0.447500in}{0.000000in}}{\pgfqpoint{0.447500in}{0.013923in}}{\pgfqpoint{0.453032in}{0.027278in}}%
\pgfpathcurveto{\pgfqpoint{0.462877in}{0.037123in}}{\pgfqpoint{0.472722in}{0.046968in}}{\pgfqpoint{0.486077in}{0.052500in}}%
\pgfpathcurveto{\pgfqpoint{0.500000in}{0.052500in}}{\pgfqpoint{0.513923in}{0.052500in}}{\pgfqpoint{0.527278in}{0.046968in}}%
\pgfpathcurveto{\pgfqpoint{0.537123in}{0.037123in}}{\pgfqpoint{0.546968in}{0.027278in}}{\pgfqpoint{0.552500in}{0.013923in}}%
\pgfpathcurveto{\pgfqpoint{0.552500in}{0.000000in}}{\pgfqpoint{0.552500in}{-0.013923in}}{\pgfqpoint{0.546968in}{-0.027278in}}%
\pgfpathcurveto{\pgfqpoint{0.537123in}{-0.037123in}}{\pgfqpoint{0.527278in}{-0.046968in}}{\pgfqpoint{0.513923in}{-0.052500in}}%
\pgfpathclose%
\pgfpathmoveto{\pgfqpoint{0.666667in}{-0.058333in}}%
\pgfpathcurveto{\pgfqpoint{0.682137in}{-0.058333in}}{\pgfqpoint{0.696975in}{-0.052187in}}{\pgfqpoint{0.707915in}{-0.041248in}}%
\pgfpathcurveto{\pgfqpoint{0.718854in}{-0.030309in}}{\pgfqpoint{0.725000in}{-0.015470in}}{\pgfqpoint{0.725000in}{0.000000in}}%
\pgfpathcurveto{\pgfqpoint{0.725000in}{0.015470in}}{\pgfqpoint{0.718854in}{0.030309in}}{\pgfqpoint{0.707915in}{0.041248in}}%
\pgfpathcurveto{\pgfqpoint{0.696975in}{0.052187in}}{\pgfqpoint{0.682137in}{0.058333in}}{\pgfqpoint{0.666667in}{0.058333in}}%
\pgfpathcurveto{\pgfqpoint{0.651196in}{0.058333in}}{\pgfqpoint{0.636358in}{0.052187in}}{\pgfqpoint{0.625419in}{0.041248in}}%
\pgfpathcurveto{\pgfqpoint{0.614480in}{0.030309in}}{\pgfqpoint{0.608333in}{0.015470in}}{\pgfqpoint{0.608333in}{0.000000in}}%
\pgfpathcurveto{\pgfqpoint{0.608333in}{-0.015470in}}{\pgfqpoint{0.614480in}{-0.030309in}}{\pgfqpoint{0.625419in}{-0.041248in}}%
\pgfpathcurveto{\pgfqpoint{0.636358in}{-0.052187in}}{\pgfqpoint{0.651196in}{-0.058333in}}{\pgfqpoint{0.666667in}{-0.058333in}}%
\pgfpathclose%
\pgfpathmoveto{\pgfqpoint{0.666667in}{-0.052500in}}%
\pgfpathcurveto{\pgfqpoint{0.666667in}{-0.052500in}}{\pgfqpoint{0.652744in}{-0.052500in}}{\pgfqpoint{0.639389in}{-0.046968in}}%
\pgfpathcurveto{\pgfqpoint{0.629544in}{-0.037123in}}{\pgfqpoint{0.619698in}{-0.027278in}}{\pgfqpoint{0.614167in}{-0.013923in}}%
\pgfpathcurveto{\pgfqpoint{0.614167in}{0.000000in}}{\pgfqpoint{0.614167in}{0.013923in}}{\pgfqpoint{0.619698in}{0.027278in}}%
\pgfpathcurveto{\pgfqpoint{0.629544in}{0.037123in}}{\pgfqpoint{0.639389in}{0.046968in}}{\pgfqpoint{0.652744in}{0.052500in}}%
\pgfpathcurveto{\pgfqpoint{0.666667in}{0.052500in}}{\pgfqpoint{0.680590in}{0.052500in}}{\pgfqpoint{0.693945in}{0.046968in}}%
\pgfpathcurveto{\pgfqpoint{0.703790in}{0.037123in}}{\pgfqpoint{0.713635in}{0.027278in}}{\pgfqpoint{0.719167in}{0.013923in}}%
\pgfpathcurveto{\pgfqpoint{0.719167in}{0.000000in}}{\pgfqpoint{0.719167in}{-0.013923in}}{\pgfqpoint{0.713635in}{-0.027278in}}%
\pgfpathcurveto{\pgfqpoint{0.703790in}{-0.037123in}}{\pgfqpoint{0.693945in}{-0.046968in}}{\pgfqpoint{0.680590in}{-0.052500in}}%
\pgfpathclose%
\pgfpathmoveto{\pgfqpoint{0.833333in}{-0.058333in}}%
\pgfpathcurveto{\pgfqpoint{0.848804in}{-0.058333in}}{\pgfqpoint{0.863642in}{-0.052187in}}{\pgfqpoint{0.874581in}{-0.041248in}}%
\pgfpathcurveto{\pgfqpoint{0.885520in}{-0.030309in}}{\pgfqpoint{0.891667in}{-0.015470in}}{\pgfqpoint{0.891667in}{0.000000in}}%
\pgfpathcurveto{\pgfqpoint{0.891667in}{0.015470in}}{\pgfqpoint{0.885520in}{0.030309in}}{\pgfqpoint{0.874581in}{0.041248in}}%
\pgfpathcurveto{\pgfqpoint{0.863642in}{0.052187in}}{\pgfqpoint{0.848804in}{0.058333in}}{\pgfqpoint{0.833333in}{0.058333in}}%
\pgfpathcurveto{\pgfqpoint{0.817863in}{0.058333in}}{\pgfqpoint{0.803025in}{0.052187in}}{\pgfqpoint{0.792085in}{0.041248in}}%
\pgfpathcurveto{\pgfqpoint{0.781146in}{0.030309in}}{\pgfqpoint{0.775000in}{0.015470in}}{\pgfqpoint{0.775000in}{0.000000in}}%
\pgfpathcurveto{\pgfqpoint{0.775000in}{-0.015470in}}{\pgfqpoint{0.781146in}{-0.030309in}}{\pgfqpoint{0.792085in}{-0.041248in}}%
\pgfpathcurveto{\pgfqpoint{0.803025in}{-0.052187in}}{\pgfqpoint{0.817863in}{-0.058333in}}{\pgfqpoint{0.833333in}{-0.058333in}}%
\pgfpathclose%
\pgfpathmoveto{\pgfqpoint{0.833333in}{-0.052500in}}%
\pgfpathcurveto{\pgfqpoint{0.833333in}{-0.052500in}}{\pgfqpoint{0.819410in}{-0.052500in}}{\pgfqpoint{0.806055in}{-0.046968in}}%
\pgfpathcurveto{\pgfqpoint{0.796210in}{-0.037123in}}{\pgfqpoint{0.786365in}{-0.027278in}}{\pgfqpoint{0.780833in}{-0.013923in}}%
\pgfpathcurveto{\pgfqpoint{0.780833in}{0.000000in}}{\pgfqpoint{0.780833in}{0.013923in}}{\pgfqpoint{0.786365in}{0.027278in}}%
\pgfpathcurveto{\pgfqpoint{0.796210in}{0.037123in}}{\pgfqpoint{0.806055in}{0.046968in}}{\pgfqpoint{0.819410in}{0.052500in}}%
\pgfpathcurveto{\pgfqpoint{0.833333in}{0.052500in}}{\pgfqpoint{0.847256in}{0.052500in}}{\pgfqpoint{0.860611in}{0.046968in}}%
\pgfpathcurveto{\pgfqpoint{0.870456in}{0.037123in}}{\pgfqpoint{0.880302in}{0.027278in}}{\pgfqpoint{0.885833in}{0.013923in}}%
\pgfpathcurveto{\pgfqpoint{0.885833in}{0.000000in}}{\pgfqpoint{0.885833in}{-0.013923in}}{\pgfqpoint{0.880302in}{-0.027278in}}%
\pgfpathcurveto{\pgfqpoint{0.870456in}{-0.037123in}}{\pgfqpoint{0.860611in}{-0.046968in}}{\pgfqpoint{0.847256in}{-0.052500in}}%
\pgfpathclose%
\pgfpathmoveto{\pgfqpoint{1.000000in}{-0.058333in}}%
\pgfpathcurveto{\pgfqpoint{1.015470in}{-0.058333in}}{\pgfqpoint{1.030309in}{-0.052187in}}{\pgfqpoint{1.041248in}{-0.041248in}}%
\pgfpathcurveto{\pgfqpoint{1.052187in}{-0.030309in}}{\pgfqpoint{1.058333in}{-0.015470in}}{\pgfqpoint{1.058333in}{0.000000in}}%
\pgfpathcurveto{\pgfqpoint{1.058333in}{0.015470in}}{\pgfqpoint{1.052187in}{0.030309in}}{\pgfqpoint{1.041248in}{0.041248in}}%
\pgfpathcurveto{\pgfqpoint{1.030309in}{0.052187in}}{\pgfqpoint{1.015470in}{0.058333in}}{\pgfqpoint{1.000000in}{0.058333in}}%
\pgfpathcurveto{\pgfqpoint{0.984530in}{0.058333in}}{\pgfqpoint{0.969691in}{0.052187in}}{\pgfqpoint{0.958752in}{0.041248in}}%
\pgfpathcurveto{\pgfqpoint{0.947813in}{0.030309in}}{\pgfqpoint{0.941667in}{0.015470in}}{\pgfqpoint{0.941667in}{0.000000in}}%
\pgfpathcurveto{\pgfqpoint{0.941667in}{-0.015470in}}{\pgfqpoint{0.947813in}{-0.030309in}}{\pgfqpoint{0.958752in}{-0.041248in}}%
\pgfpathcurveto{\pgfqpoint{0.969691in}{-0.052187in}}{\pgfqpoint{0.984530in}{-0.058333in}}{\pgfqpoint{1.000000in}{-0.058333in}}%
\pgfpathclose%
\pgfpathmoveto{\pgfqpoint{1.000000in}{-0.052500in}}%
\pgfpathcurveto{\pgfqpoint{1.000000in}{-0.052500in}}{\pgfqpoint{0.986077in}{-0.052500in}}{\pgfqpoint{0.972722in}{-0.046968in}}%
\pgfpathcurveto{\pgfqpoint{0.962877in}{-0.037123in}}{\pgfqpoint{0.953032in}{-0.027278in}}{\pgfqpoint{0.947500in}{-0.013923in}}%
\pgfpathcurveto{\pgfqpoint{0.947500in}{0.000000in}}{\pgfqpoint{0.947500in}{0.013923in}}{\pgfqpoint{0.953032in}{0.027278in}}%
\pgfpathcurveto{\pgfqpoint{0.962877in}{0.037123in}}{\pgfqpoint{0.972722in}{0.046968in}}{\pgfqpoint{0.986077in}{0.052500in}}%
\pgfpathcurveto{\pgfqpoint{1.000000in}{0.052500in}}{\pgfqpoint{1.013923in}{0.052500in}}{\pgfqpoint{1.027278in}{0.046968in}}%
\pgfpathcurveto{\pgfqpoint{1.037123in}{0.037123in}}{\pgfqpoint{1.046968in}{0.027278in}}{\pgfqpoint{1.052500in}{0.013923in}}%
\pgfpathcurveto{\pgfqpoint{1.052500in}{0.000000in}}{\pgfqpoint{1.052500in}{-0.013923in}}{\pgfqpoint{1.046968in}{-0.027278in}}%
\pgfpathcurveto{\pgfqpoint{1.037123in}{-0.037123in}}{\pgfqpoint{1.027278in}{-0.046968in}}{\pgfqpoint{1.013923in}{-0.052500in}}%
\pgfpathclose%
\pgfpathmoveto{\pgfqpoint{0.083333in}{0.108333in}}%
\pgfpathcurveto{\pgfqpoint{0.098804in}{0.108333in}}{\pgfqpoint{0.113642in}{0.114480in}}{\pgfqpoint{0.124581in}{0.125419in}}%
\pgfpathcurveto{\pgfqpoint{0.135520in}{0.136358in}}{\pgfqpoint{0.141667in}{0.151196in}}{\pgfqpoint{0.141667in}{0.166667in}}%
\pgfpathcurveto{\pgfqpoint{0.141667in}{0.182137in}}{\pgfqpoint{0.135520in}{0.196975in}}{\pgfqpoint{0.124581in}{0.207915in}}%
\pgfpathcurveto{\pgfqpoint{0.113642in}{0.218854in}}{\pgfqpoint{0.098804in}{0.225000in}}{\pgfqpoint{0.083333in}{0.225000in}}%
\pgfpathcurveto{\pgfqpoint{0.067863in}{0.225000in}}{\pgfqpoint{0.053025in}{0.218854in}}{\pgfqpoint{0.042085in}{0.207915in}}%
\pgfpathcurveto{\pgfqpoint{0.031146in}{0.196975in}}{\pgfqpoint{0.025000in}{0.182137in}}{\pgfqpoint{0.025000in}{0.166667in}}%
\pgfpathcurveto{\pgfqpoint{0.025000in}{0.151196in}}{\pgfqpoint{0.031146in}{0.136358in}}{\pgfqpoint{0.042085in}{0.125419in}}%
\pgfpathcurveto{\pgfqpoint{0.053025in}{0.114480in}}{\pgfqpoint{0.067863in}{0.108333in}}{\pgfqpoint{0.083333in}{0.108333in}}%
\pgfpathclose%
\pgfpathmoveto{\pgfqpoint{0.083333in}{0.114167in}}%
\pgfpathcurveto{\pgfqpoint{0.083333in}{0.114167in}}{\pgfqpoint{0.069410in}{0.114167in}}{\pgfqpoint{0.056055in}{0.119698in}}%
\pgfpathcurveto{\pgfqpoint{0.046210in}{0.129544in}}{\pgfqpoint{0.036365in}{0.139389in}}{\pgfqpoint{0.030833in}{0.152744in}}%
\pgfpathcurveto{\pgfqpoint{0.030833in}{0.166667in}}{\pgfqpoint{0.030833in}{0.180590in}}{\pgfqpoint{0.036365in}{0.193945in}}%
\pgfpathcurveto{\pgfqpoint{0.046210in}{0.203790in}}{\pgfqpoint{0.056055in}{0.213635in}}{\pgfqpoint{0.069410in}{0.219167in}}%
\pgfpathcurveto{\pgfqpoint{0.083333in}{0.219167in}}{\pgfqpoint{0.097256in}{0.219167in}}{\pgfqpoint{0.110611in}{0.213635in}}%
\pgfpathcurveto{\pgfqpoint{0.120456in}{0.203790in}}{\pgfqpoint{0.130302in}{0.193945in}}{\pgfqpoint{0.135833in}{0.180590in}}%
\pgfpathcurveto{\pgfqpoint{0.135833in}{0.166667in}}{\pgfqpoint{0.135833in}{0.152744in}}{\pgfqpoint{0.130302in}{0.139389in}}%
\pgfpathcurveto{\pgfqpoint{0.120456in}{0.129544in}}{\pgfqpoint{0.110611in}{0.119698in}}{\pgfqpoint{0.097256in}{0.114167in}}%
\pgfpathclose%
\pgfpathmoveto{\pgfqpoint{0.250000in}{0.108333in}}%
\pgfpathcurveto{\pgfqpoint{0.265470in}{0.108333in}}{\pgfqpoint{0.280309in}{0.114480in}}{\pgfqpoint{0.291248in}{0.125419in}}%
\pgfpathcurveto{\pgfqpoint{0.302187in}{0.136358in}}{\pgfqpoint{0.308333in}{0.151196in}}{\pgfqpoint{0.308333in}{0.166667in}}%
\pgfpathcurveto{\pgfqpoint{0.308333in}{0.182137in}}{\pgfqpoint{0.302187in}{0.196975in}}{\pgfqpoint{0.291248in}{0.207915in}}%
\pgfpathcurveto{\pgfqpoint{0.280309in}{0.218854in}}{\pgfqpoint{0.265470in}{0.225000in}}{\pgfqpoint{0.250000in}{0.225000in}}%
\pgfpathcurveto{\pgfqpoint{0.234530in}{0.225000in}}{\pgfqpoint{0.219691in}{0.218854in}}{\pgfqpoint{0.208752in}{0.207915in}}%
\pgfpathcurveto{\pgfqpoint{0.197813in}{0.196975in}}{\pgfqpoint{0.191667in}{0.182137in}}{\pgfqpoint{0.191667in}{0.166667in}}%
\pgfpathcurveto{\pgfqpoint{0.191667in}{0.151196in}}{\pgfqpoint{0.197813in}{0.136358in}}{\pgfqpoint{0.208752in}{0.125419in}}%
\pgfpathcurveto{\pgfqpoint{0.219691in}{0.114480in}}{\pgfqpoint{0.234530in}{0.108333in}}{\pgfqpoint{0.250000in}{0.108333in}}%
\pgfpathclose%
\pgfpathmoveto{\pgfqpoint{0.250000in}{0.114167in}}%
\pgfpathcurveto{\pgfqpoint{0.250000in}{0.114167in}}{\pgfqpoint{0.236077in}{0.114167in}}{\pgfqpoint{0.222722in}{0.119698in}}%
\pgfpathcurveto{\pgfqpoint{0.212877in}{0.129544in}}{\pgfqpoint{0.203032in}{0.139389in}}{\pgfqpoint{0.197500in}{0.152744in}}%
\pgfpathcurveto{\pgfqpoint{0.197500in}{0.166667in}}{\pgfqpoint{0.197500in}{0.180590in}}{\pgfqpoint{0.203032in}{0.193945in}}%
\pgfpathcurveto{\pgfqpoint{0.212877in}{0.203790in}}{\pgfqpoint{0.222722in}{0.213635in}}{\pgfqpoint{0.236077in}{0.219167in}}%
\pgfpathcurveto{\pgfqpoint{0.250000in}{0.219167in}}{\pgfqpoint{0.263923in}{0.219167in}}{\pgfqpoint{0.277278in}{0.213635in}}%
\pgfpathcurveto{\pgfqpoint{0.287123in}{0.203790in}}{\pgfqpoint{0.296968in}{0.193945in}}{\pgfqpoint{0.302500in}{0.180590in}}%
\pgfpathcurveto{\pgfqpoint{0.302500in}{0.166667in}}{\pgfqpoint{0.302500in}{0.152744in}}{\pgfqpoint{0.296968in}{0.139389in}}%
\pgfpathcurveto{\pgfqpoint{0.287123in}{0.129544in}}{\pgfqpoint{0.277278in}{0.119698in}}{\pgfqpoint{0.263923in}{0.114167in}}%
\pgfpathclose%
\pgfpathmoveto{\pgfqpoint{0.416667in}{0.108333in}}%
\pgfpathcurveto{\pgfqpoint{0.432137in}{0.108333in}}{\pgfqpoint{0.446975in}{0.114480in}}{\pgfqpoint{0.457915in}{0.125419in}}%
\pgfpathcurveto{\pgfqpoint{0.468854in}{0.136358in}}{\pgfqpoint{0.475000in}{0.151196in}}{\pgfqpoint{0.475000in}{0.166667in}}%
\pgfpathcurveto{\pgfqpoint{0.475000in}{0.182137in}}{\pgfqpoint{0.468854in}{0.196975in}}{\pgfqpoint{0.457915in}{0.207915in}}%
\pgfpathcurveto{\pgfqpoint{0.446975in}{0.218854in}}{\pgfqpoint{0.432137in}{0.225000in}}{\pgfqpoint{0.416667in}{0.225000in}}%
\pgfpathcurveto{\pgfqpoint{0.401196in}{0.225000in}}{\pgfqpoint{0.386358in}{0.218854in}}{\pgfqpoint{0.375419in}{0.207915in}}%
\pgfpathcurveto{\pgfqpoint{0.364480in}{0.196975in}}{\pgfqpoint{0.358333in}{0.182137in}}{\pgfqpoint{0.358333in}{0.166667in}}%
\pgfpathcurveto{\pgfqpoint{0.358333in}{0.151196in}}{\pgfqpoint{0.364480in}{0.136358in}}{\pgfqpoint{0.375419in}{0.125419in}}%
\pgfpathcurveto{\pgfqpoint{0.386358in}{0.114480in}}{\pgfqpoint{0.401196in}{0.108333in}}{\pgfqpoint{0.416667in}{0.108333in}}%
\pgfpathclose%
\pgfpathmoveto{\pgfqpoint{0.416667in}{0.114167in}}%
\pgfpathcurveto{\pgfqpoint{0.416667in}{0.114167in}}{\pgfqpoint{0.402744in}{0.114167in}}{\pgfqpoint{0.389389in}{0.119698in}}%
\pgfpathcurveto{\pgfqpoint{0.379544in}{0.129544in}}{\pgfqpoint{0.369698in}{0.139389in}}{\pgfqpoint{0.364167in}{0.152744in}}%
\pgfpathcurveto{\pgfqpoint{0.364167in}{0.166667in}}{\pgfqpoint{0.364167in}{0.180590in}}{\pgfqpoint{0.369698in}{0.193945in}}%
\pgfpathcurveto{\pgfqpoint{0.379544in}{0.203790in}}{\pgfqpoint{0.389389in}{0.213635in}}{\pgfqpoint{0.402744in}{0.219167in}}%
\pgfpathcurveto{\pgfqpoint{0.416667in}{0.219167in}}{\pgfqpoint{0.430590in}{0.219167in}}{\pgfqpoint{0.443945in}{0.213635in}}%
\pgfpathcurveto{\pgfqpoint{0.453790in}{0.203790in}}{\pgfqpoint{0.463635in}{0.193945in}}{\pgfqpoint{0.469167in}{0.180590in}}%
\pgfpathcurveto{\pgfqpoint{0.469167in}{0.166667in}}{\pgfqpoint{0.469167in}{0.152744in}}{\pgfqpoint{0.463635in}{0.139389in}}%
\pgfpathcurveto{\pgfqpoint{0.453790in}{0.129544in}}{\pgfqpoint{0.443945in}{0.119698in}}{\pgfqpoint{0.430590in}{0.114167in}}%
\pgfpathclose%
\pgfpathmoveto{\pgfqpoint{0.583333in}{0.108333in}}%
\pgfpathcurveto{\pgfqpoint{0.598804in}{0.108333in}}{\pgfqpoint{0.613642in}{0.114480in}}{\pgfqpoint{0.624581in}{0.125419in}}%
\pgfpathcurveto{\pgfqpoint{0.635520in}{0.136358in}}{\pgfqpoint{0.641667in}{0.151196in}}{\pgfqpoint{0.641667in}{0.166667in}}%
\pgfpathcurveto{\pgfqpoint{0.641667in}{0.182137in}}{\pgfqpoint{0.635520in}{0.196975in}}{\pgfqpoint{0.624581in}{0.207915in}}%
\pgfpathcurveto{\pgfqpoint{0.613642in}{0.218854in}}{\pgfqpoint{0.598804in}{0.225000in}}{\pgfqpoint{0.583333in}{0.225000in}}%
\pgfpathcurveto{\pgfqpoint{0.567863in}{0.225000in}}{\pgfqpoint{0.553025in}{0.218854in}}{\pgfqpoint{0.542085in}{0.207915in}}%
\pgfpathcurveto{\pgfqpoint{0.531146in}{0.196975in}}{\pgfqpoint{0.525000in}{0.182137in}}{\pgfqpoint{0.525000in}{0.166667in}}%
\pgfpathcurveto{\pgfqpoint{0.525000in}{0.151196in}}{\pgfqpoint{0.531146in}{0.136358in}}{\pgfqpoint{0.542085in}{0.125419in}}%
\pgfpathcurveto{\pgfqpoint{0.553025in}{0.114480in}}{\pgfqpoint{0.567863in}{0.108333in}}{\pgfqpoint{0.583333in}{0.108333in}}%
\pgfpathclose%
\pgfpathmoveto{\pgfqpoint{0.583333in}{0.114167in}}%
\pgfpathcurveto{\pgfqpoint{0.583333in}{0.114167in}}{\pgfqpoint{0.569410in}{0.114167in}}{\pgfqpoint{0.556055in}{0.119698in}}%
\pgfpathcurveto{\pgfqpoint{0.546210in}{0.129544in}}{\pgfqpoint{0.536365in}{0.139389in}}{\pgfqpoint{0.530833in}{0.152744in}}%
\pgfpathcurveto{\pgfqpoint{0.530833in}{0.166667in}}{\pgfqpoint{0.530833in}{0.180590in}}{\pgfqpoint{0.536365in}{0.193945in}}%
\pgfpathcurveto{\pgfqpoint{0.546210in}{0.203790in}}{\pgfqpoint{0.556055in}{0.213635in}}{\pgfqpoint{0.569410in}{0.219167in}}%
\pgfpathcurveto{\pgfqpoint{0.583333in}{0.219167in}}{\pgfqpoint{0.597256in}{0.219167in}}{\pgfqpoint{0.610611in}{0.213635in}}%
\pgfpathcurveto{\pgfqpoint{0.620456in}{0.203790in}}{\pgfqpoint{0.630302in}{0.193945in}}{\pgfqpoint{0.635833in}{0.180590in}}%
\pgfpathcurveto{\pgfqpoint{0.635833in}{0.166667in}}{\pgfqpoint{0.635833in}{0.152744in}}{\pgfqpoint{0.630302in}{0.139389in}}%
\pgfpathcurveto{\pgfqpoint{0.620456in}{0.129544in}}{\pgfqpoint{0.610611in}{0.119698in}}{\pgfqpoint{0.597256in}{0.114167in}}%
\pgfpathclose%
\pgfpathmoveto{\pgfqpoint{0.750000in}{0.108333in}}%
\pgfpathcurveto{\pgfqpoint{0.765470in}{0.108333in}}{\pgfqpoint{0.780309in}{0.114480in}}{\pgfqpoint{0.791248in}{0.125419in}}%
\pgfpathcurveto{\pgfqpoint{0.802187in}{0.136358in}}{\pgfqpoint{0.808333in}{0.151196in}}{\pgfqpoint{0.808333in}{0.166667in}}%
\pgfpathcurveto{\pgfqpoint{0.808333in}{0.182137in}}{\pgfqpoint{0.802187in}{0.196975in}}{\pgfqpoint{0.791248in}{0.207915in}}%
\pgfpathcurveto{\pgfqpoint{0.780309in}{0.218854in}}{\pgfqpoint{0.765470in}{0.225000in}}{\pgfqpoint{0.750000in}{0.225000in}}%
\pgfpathcurveto{\pgfqpoint{0.734530in}{0.225000in}}{\pgfqpoint{0.719691in}{0.218854in}}{\pgfqpoint{0.708752in}{0.207915in}}%
\pgfpathcurveto{\pgfqpoint{0.697813in}{0.196975in}}{\pgfqpoint{0.691667in}{0.182137in}}{\pgfqpoint{0.691667in}{0.166667in}}%
\pgfpathcurveto{\pgfqpoint{0.691667in}{0.151196in}}{\pgfqpoint{0.697813in}{0.136358in}}{\pgfqpoint{0.708752in}{0.125419in}}%
\pgfpathcurveto{\pgfqpoint{0.719691in}{0.114480in}}{\pgfqpoint{0.734530in}{0.108333in}}{\pgfqpoint{0.750000in}{0.108333in}}%
\pgfpathclose%
\pgfpathmoveto{\pgfqpoint{0.750000in}{0.114167in}}%
\pgfpathcurveto{\pgfqpoint{0.750000in}{0.114167in}}{\pgfqpoint{0.736077in}{0.114167in}}{\pgfqpoint{0.722722in}{0.119698in}}%
\pgfpathcurveto{\pgfqpoint{0.712877in}{0.129544in}}{\pgfqpoint{0.703032in}{0.139389in}}{\pgfqpoint{0.697500in}{0.152744in}}%
\pgfpathcurveto{\pgfqpoint{0.697500in}{0.166667in}}{\pgfqpoint{0.697500in}{0.180590in}}{\pgfqpoint{0.703032in}{0.193945in}}%
\pgfpathcurveto{\pgfqpoint{0.712877in}{0.203790in}}{\pgfqpoint{0.722722in}{0.213635in}}{\pgfqpoint{0.736077in}{0.219167in}}%
\pgfpathcurveto{\pgfqpoint{0.750000in}{0.219167in}}{\pgfqpoint{0.763923in}{0.219167in}}{\pgfqpoint{0.777278in}{0.213635in}}%
\pgfpathcurveto{\pgfqpoint{0.787123in}{0.203790in}}{\pgfqpoint{0.796968in}{0.193945in}}{\pgfqpoint{0.802500in}{0.180590in}}%
\pgfpathcurveto{\pgfqpoint{0.802500in}{0.166667in}}{\pgfqpoint{0.802500in}{0.152744in}}{\pgfqpoint{0.796968in}{0.139389in}}%
\pgfpathcurveto{\pgfqpoint{0.787123in}{0.129544in}}{\pgfqpoint{0.777278in}{0.119698in}}{\pgfqpoint{0.763923in}{0.114167in}}%
\pgfpathclose%
\pgfpathmoveto{\pgfqpoint{0.916667in}{0.108333in}}%
\pgfpathcurveto{\pgfqpoint{0.932137in}{0.108333in}}{\pgfqpoint{0.946975in}{0.114480in}}{\pgfqpoint{0.957915in}{0.125419in}}%
\pgfpathcurveto{\pgfqpoint{0.968854in}{0.136358in}}{\pgfqpoint{0.975000in}{0.151196in}}{\pgfqpoint{0.975000in}{0.166667in}}%
\pgfpathcurveto{\pgfqpoint{0.975000in}{0.182137in}}{\pgfqpoint{0.968854in}{0.196975in}}{\pgfqpoint{0.957915in}{0.207915in}}%
\pgfpathcurveto{\pgfqpoint{0.946975in}{0.218854in}}{\pgfqpoint{0.932137in}{0.225000in}}{\pgfqpoint{0.916667in}{0.225000in}}%
\pgfpathcurveto{\pgfqpoint{0.901196in}{0.225000in}}{\pgfqpoint{0.886358in}{0.218854in}}{\pgfqpoint{0.875419in}{0.207915in}}%
\pgfpathcurveto{\pgfqpoint{0.864480in}{0.196975in}}{\pgfqpoint{0.858333in}{0.182137in}}{\pgfqpoint{0.858333in}{0.166667in}}%
\pgfpathcurveto{\pgfqpoint{0.858333in}{0.151196in}}{\pgfqpoint{0.864480in}{0.136358in}}{\pgfqpoint{0.875419in}{0.125419in}}%
\pgfpathcurveto{\pgfqpoint{0.886358in}{0.114480in}}{\pgfqpoint{0.901196in}{0.108333in}}{\pgfqpoint{0.916667in}{0.108333in}}%
\pgfpathclose%
\pgfpathmoveto{\pgfqpoint{0.916667in}{0.114167in}}%
\pgfpathcurveto{\pgfqpoint{0.916667in}{0.114167in}}{\pgfqpoint{0.902744in}{0.114167in}}{\pgfqpoint{0.889389in}{0.119698in}}%
\pgfpathcurveto{\pgfqpoint{0.879544in}{0.129544in}}{\pgfqpoint{0.869698in}{0.139389in}}{\pgfqpoint{0.864167in}{0.152744in}}%
\pgfpathcurveto{\pgfqpoint{0.864167in}{0.166667in}}{\pgfqpoint{0.864167in}{0.180590in}}{\pgfqpoint{0.869698in}{0.193945in}}%
\pgfpathcurveto{\pgfqpoint{0.879544in}{0.203790in}}{\pgfqpoint{0.889389in}{0.213635in}}{\pgfqpoint{0.902744in}{0.219167in}}%
\pgfpathcurveto{\pgfqpoint{0.916667in}{0.219167in}}{\pgfqpoint{0.930590in}{0.219167in}}{\pgfqpoint{0.943945in}{0.213635in}}%
\pgfpathcurveto{\pgfqpoint{0.953790in}{0.203790in}}{\pgfqpoint{0.963635in}{0.193945in}}{\pgfqpoint{0.969167in}{0.180590in}}%
\pgfpathcurveto{\pgfqpoint{0.969167in}{0.166667in}}{\pgfqpoint{0.969167in}{0.152744in}}{\pgfqpoint{0.963635in}{0.139389in}}%
\pgfpathcurveto{\pgfqpoint{0.953790in}{0.129544in}}{\pgfqpoint{0.943945in}{0.119698in}}{\pgfqpoint{0.930590in}{0.114167in}}%
\pgfpathclose%
\pgfpathmoveto{\pgfqpoint{0.000000in}{0.275000in}}%
\pgfpathcurveto{\pgfqpoint{0.015470in}{0.275000in}}{\pgfqpoint{0.030309in}{0.281146in}}{\pgfqpoint{0.041248in}{0.292085in}}%
\pgfpathcurveto{\pgfqpoint{0.052187in}{0.303025in}}{\pgfqpoint{0.058333in}{0.317863in}}{\pgfqpoint{0.058333in}{0.333333in}}%
\pgfpathcurveto{\pgfqpoint{0.058333in}{0.348804in}}{\pgfqpoint{0.052187in}{0.363642in}}{\pgfqpoint{0.041248in}{0.374581in}}%
\pgfpathcurveto{\pgfqpoint{0.030309in}{0.385520in}}{\pgfqpoint{0.015470in}{0.391667in}}{\pgfqpoint{0.000000in}{0.391667in}}%
\pgfpathcurveto{\pgfqpoint{-0.015470in}{0.391667in}}{\pgfqpoint{-0.030309in}{0.385520in}}{\pgfqpoint{-0.041248in}{0.374581in}}%
\pgfpathcurveto{\pgfqpoint{-0.052187in}{0.363642in}}{\pgfqpoint{-0.058333in}{0.348804in}}{\pgfqpoint{-0.058333in}{0.333333in}}%
\pgfpathcurveto{\pgfqpoint{-0.058333in}{0.317863in}}{\pgfqpoint{-0.052187in}{0.303025in}}{\pgfqpoint{-0.041248in}{0.292085in}}%
\pgfpathcurveto{\pgfqpoint{-0.030309in}{0.281146in}}{\pgfqpoint{-0.015470in}{0.275000in}}{\pgfqpoint{0.000000in}{0.275000in}}%
\pgfpathclose%
\pgfpathmoveto{\pgfqpoint{0.000000in}{0.280833in}}%
\pgfpathcurveto{\pgfqpoint{0.000000in}{0.280833in}}{\pgfqpoint{-0.013923in}{0.280833in}}{\pgfqpoint{-0.027278in}{0.286365in}}%
\pgfpathcurveto{\pgfqpoint{-0.037123in}{0.296210in}}{\pgfqpoint{-0.046968in}{0.306055in}}{\pgfqpoint{-0.052500in}{0.319410in}}%
\pgfpathcurveto{\pgfqpoint{-0.052500in}{0.333333in}}{\pgfqpoint{-0.052500in}{0.347256in}}{\pgfqpoint{-0.046968in}{0.360611in}}%
\pgfpathcurveto{\pgfqpoint{-0.037123in}{0.370456in}}{\pgfqpoint{-0.027278in}{0.380302in}}{\pgfqpoint{-0.013923in}{0.385833in}}%
\pgfpathcurveto{\pgfqpoint{0.000000in}{0.385833in}}{\pgfqpoint{0.013923in}{0.385833in}}{\pgfqpoint{0.027278in}{0.380302in}}%
\pgfpathcurveto{\pgfqpoint{0.037123in}{0.370456in}}{\pgfqpoint{0.046968in}{0.360611in}}{\pgfqpoint{0.052500in}{0.347256in}}%
\pgfpathcurveto{\pgfqpoint{0.052500in}{0.333333in}}{\pgfqpoint{0.052500in}{0.319410in}}{\pgfqpoint{0.046968in}{0.306055in}}%
\pgfpathcurveto{\pgfqpoint{0.037123in}{0.296210in}}{\pgfqpoint{0.027278in}{0.286365in}}{\pgfqpoint{0.013923in}{0.280833in}}%
\pgfpathclose%
\pgfpathmoveto{\pgfqpoint{0.166667in}{0.275000in}}%
\pgfpathcurveto{\pgfqpoint{0.182137in}{0.275000in}}{\pgfqpoint{0.196975in}{0.281146in}}{\pgfqpoint{0.207915in}{0.292085in}}%
\pgfpathcurveto{\pgfqpoint{0.218854in}{0.303025in}}{\pgfqpoint{0.225000in}{0.317863in}}{\pgfqpoint{0.225000in}{0.333333in}}%
\pgfpathcurveto{\pgfqpoint{0.225000in}{0.348804in}}{\pgfqpoint{0.218854in}{0.363642in}}{\pgfqpoint{0.207915in}{0.374581in}}%
\pgfpathcurveto{\pgfqpoint{0.196975in}{0.385520in}}{\pgfqpoint{0.182137in}{0.391667in}}{\pgfqpoint{0.166667in}{0.391667in}}%
\pgfpathcurveto{\pgfqpoint{0.151196in}{0.391667in}}{\pgfqpoint{0.136358in}{0.385520in}}{\pgfqpoint{0.125419in}{0.374581in}}%
\pgfpathcurveto{\pgfqpoint{0.114480in}{0.363642in}}{\pgfqpoint{0.108333in}{0.348804in}}{\pgfqpoint{0.108333in}{0.333333in}}%
\pgfpathcurveto{\pgfqpoint{0.108333in}{0.317863in}}{\pgfqpoint{0.114480in}{0.303025in}}{\pgfqpoint{0.125419in}{0.292085in}}%
\pgfpathcurveto{\pgfqpoint{0.136358in}{0.281146in}}{\pgfqpoint{0.151196in}{0.275000in}}{\pgfqpoint{0.166667in}{0.275000in}}%
\pgfpathclose%
\pgfpathmoveto{\pgfqpoint{0.166667in}{0.280833in}}%
\pgfpathcurveto{\pgfqpoint{0.166667in}{0.280833in}}{\pgfqpoint{0.152744in}{0.280833in}}{\pgfqpoint{0.139389in}{0.286365in}}%
\pgfpathcurveto{\pgfqpoint{0.129544in}{0.296210in}}{\pgfqpoint{0.119698in}{0.306055in}}{\pgfqpoint{0.114167in}{0.319410in}}%
\pgfpathcurveto{\pgfqpoint{0.114167in}{0.333333in}}{\pgfqpoint{0.114167in}{0.347256in}}{\pgfqpoint{0.119698in}{0.360611in}}%
\pgfpathcurveto{\pgfqpoint{0.129544in}{0.370456in}}{\pgfqpoint{0.139389in}{0.380302in}}{\pgfqpoint{0.152744in}{0.385833in}}%
\pgfpathcurveto{\pgfqpoint{0.166667in}{0.385833in}}{\pgfqpoint{0.180590in}{0.385833in}}{\pgfqpoint{0.193945in}{0.380302in}}%
\pgfpathcurveto{\pgfqpoint{0.203790in}{0.370456in}}{\pgfqpoint{0.213635in}{0.360611in}}{\pgfqpoint{0.219167in}{0.347256in}}%
\pgfpathcurveto{\pgfqpoint{0.219167in}{0.333333in}}{\pgfqpoint{0.219167in}{0.319410in}}{\pgfqpoint{0.213635in}{0.306055in}}%
\pgfpathcurveto{\pgfqpoint{0.203790in}{0.296210in}}{\pgfqpoint{0.193945in}{0.286365in}}{\pgfqpoint{0.180590in}{0.280833in}}%
\pgfpathclose%
\pgfpathmoveto{\pgfqpoint{0.333333in}{0.275000in}}%
\pgfpathcurveto{\pgfqpoint{0.348804in}{0.275000in}}{\pgfqpoint{0.363642in}{0.281146in}}{\pgfqpoint{0.374581in}{0.292085in}}%
\pgfpathcurveto{\pgfqpoint{0.385520in}{0.303025in}}{\pgfqpoint{0.391667in}{0.317863in}}{\pgfqpoint{0.391667in}{0.333333in}}%
\pgfpathcurveto{\pgfqpoint{0.391667in}{0.348804in}}{\pgfqpoint{0.385520in}{0.363642in}}{\pgfqpoint{0.374581in}{0.374581in}}%
\pgfpathcurveto{\pgfqpoint{0.363642in}{0.385520in}}{\pgfqpoint{0.348804in}{0.391667in}}{\pgfqpoint{0.333333in}{0.391667in}}%
\pgfpathcurveto{\pgfqpoint{0.317863in}{0.391667in}}{\pgfqpoint{0.303025in}{0.385520in}}{\pgfqpoint{0.292085in}{0.374581in}}%
\pgfpathcurveto{\pgfqpoint{0.281146in}{0.363642in}}{\pgfqpoint{0.275000in}{0.348804in}}{\pgfqpoint{0.275000in}{0.333333in}}%
\pgfpathcurveto{\pgfqpoint{0.275000in}{0.317863in}}{\pgfqpoint{0.281146in}{0.303025in}}{\pgfqpoint{0.292085in}{0.292085in}}%
\pgfpathcurveto{\pgfqpoint{0.303025in}{0.281146in}}{\pgfqpoint{0.317863in}{0.275000in}}{\pgfqpoint{0.333333in}{0.275000in}}%
\pgfpathclose%
\pgfpathmoveto{\pgfqpoint{0.333333in}{0.280833in}}%
\pgfpathcurveto{\pgfqpoint{0.333333in}{0.280833in}}{\pgfqpoint{0.319410in}{0.280833in}}{\pgfqpoint{0.306055in}{0.286365in}}%
\pgfpathcurveto{\pgfqpoint{0.296210in}{0.296210in}}{\pgfqpoint{0.286365in}{0.306055in}}{\pgfqpoint{0.280833in}{0.319410in}}%
\pgfpathcurveto{\pgfqpoint{0.280833in}{0.333333in}}{\pgfqpoint{0.280833in}{0.347256in}}{\pgfqpoint{0.286365in}{0.360611in}}%
\pgfpathcurveto{\pgfqpoint{0.296210in}{0.370456in}}{\pgfqpoint{0.306055in}{0.380302in}}{\pgfqpoint{0.319410in}{0.385833in}}%
\pgfpathcurveto{\pgfqpoint{0.333333in}{0.385833in}}{\pgfqpoint{0.347256in}{0.385833in}}{\pgfqpoint{0.360611in}{0.380302in}}%
\pgfpathcurveto{\pgfqpoint{0.370456in}{0.370456in}}{\pgfqpoint{0.380302in}{0.360611in}}{\pgfqpoint{0.385833in}{0.347256in}}%
\pgfpathcurveto{\pgfqpoint{0.385833in}{0.333333in}}{\pgfqpoint{0.385833in}{0.319410in}}{\pgfqpoint{0.380302in}{0.306055in}}%
\pgfpathcurveto{\pgfqpoint{0.370456in}{0.296210in}}{\pgfqpoint{0.360611in}{0.286365in}}{\pgfqpoint{0.347256in}{0.280833in}}%
\pgfpathclose%
\pgfpathmoveto{\pgfqpoint{0.500000in}{0.275000in}}%
\pgfpathcurveto{\pgfqpoint{0.515470in}{0.275000in}}{\pgfqpoint{0.530309in}{0.281146in}}{\pgfqpoint{0.541248in}{0.292085in}}%
\pgfpathcurveto{\pgfqpoint{0.552187in}{0.303025in}}{\pgfqpoint{0.558333in}{0.317863in}}{\pgfqpoint{0.558333in}{0.333333in}}%
\pgfpathcurveto{\pgfqpoint{0.558333in}{0.348804in}}{\pgfqpoint{0.552187in}{0.363642in}}{\pgfqpoint{0.541248in}{0.374581in}}%
\pgfpathcurveto{\pgfqpoint{0.530309in}{0.385520in}}{\pgfqpoint{0.515470in}{0.391667in}}{\pgfqpoint{0.500000in}{0.391667in}}%
\pgfpathcurveto{\pgfqpoint{0.484530in}{0.391667in}}{\pgfqpoint{0.469691in}{0.385520in}}{\pgfqpoint{0.458752in}{0.374581in}}%
\pgfpathcurveto{\pgfqpoint{0.447813in}{0.363642in}}{\pgfqpoint{0.441667in}{0.348804in}}{\pgfqpoint{0.441667in}{0.333333in}}%
\pgfpathcurveto{\pgfqpoint{0.441667in}{0.317863in}}{\pgfqpoint{0.447813in}{0.303025in}}{\pgfqpoint{0.458752in}{0.292085in}}%
\pgfpathcurveto{\pgfqpoint{0.469691in}{0.281146in}}{\pgfqpoint{0.484530in}{0.275000in}}{\pgfqpoint{0.500000in}{0.275000in}}%
\pgfpathclose%
\pgfpathmoveto{\pgfqpoint{0.500000in}{0.280833in}}%
\pgfpathcurveto{\pgfqpoint{0.500000in}{0.280833in}}{\pgfqpoint{0.486077in}{0.280833in}}{\pgfqpoint{0.472722in}{0.286365in}}%
\pgfpathcurveto{\pgfqpoint{0.462877in}{0.296210in}}{\pgfqpoint{0.453032in}{0.306055in}}{\pgfqpoint{0.447500in}{0.319410in}}%
\pgfpathcurveto{\pgfqpoint{0.447500in}{0.333333in}}{\pgfqpoint{0.447500in}{0.347256in}}{\pgfqpoint{0.453032in}{0.360611in}}%
\pgfpathcurveto{\pgfqpoint{0.462877in}{0.370456in}}{\pgfqpoint{0.472722in}{0.380302in}}{\pgfqpoint{0.486077in}{0.385833in}}%
\pgfpathcurveto{\pgfqpoint{0.500000in}{0.385833in}}{\pgfqpoint{0.513923in}{0.385833in}}{\pgfqpoint{0.527278in}{0.380302in}}%
\pgfpathcurveto{\pgfqpoint{0.537123in}{0.370456in}}{\pgfqpoint{0.546968in}{0.360611in}}{\pgfqpoint{0.552500in}{0.347256in}}%
\pgfpathcurveto{\pgfqpoint{0.552500in}{0.333333in}}{\pgfqpoint{0.552500in}{0.319410in}}{\pgfqpoint{0.546968in}{0.306055in}}%
\pgfpathcurveto{\pgfqpoint{0.537123in}{0.296210in}}{\pgfqpoint{0.527278in}{0.286365in}}{\pgfqpoint{0.513923in}{0.280833in}}%
\pgfpathclose%
\pgfpathmoveto{\pgfqpoint{0.666667in}{0.275000in}}%
\pgfpathcurveto{\pgfqpoint{0.682137in}{0.275000in}}{\pgfqpoint{0.696975in}{0.281146in}}{\pgfqpoint{0.707915in}{0.292085in}}%
\pgfpathcurveto{\pgfqpoint{0.718854in}{0.303025in}}{\pgfqpoint{0.725000in}{0.317863in}}{\pgfqpoint{0.725000in}{0.333333in}}%
\pgfpathcurveto{\pgfqpoint{0.725000in}{0.348804in}}{\pgfqpoint{0.718854in}{0.363642in}}{\pgfqpoint{0.707915in}{0.374581in}}%
\pgfpathcurveto{\pgfqpoint{0.696975in}{0.385520in}}{\pgfqpoint{0.682137in}{0.391667in}}{\pgfqpoint{0.666667in}{0.391667in}}%
\pgfpathcurveto{\pgfqpoint{0.651196in}{0.391667in}}{\pgfqpoint{0.636358in}{0.385520in}}{\pgfqpoint{0.625419in}{0.374581in}}%
\pgfpathcurveto{\pgfqpoint{0.614480in}{0.363642in}}{\pgfqpoint{0.608333in}{0.348804in}}{\pgfqpoint{0.608333in}{0.333333in}}%
\pgfpathcurveto{\pgfqpoint{0.608333in}{0.317863in}}{\pgfqpoint{0.614480in}{0.303025in}}{\pgfqpoint{0.625419in}{0.292085in}}%
\pgfpathcurveto{\pgfqpoint{0.636358in}{0.281146in}}{\pgfqpoint{0.651196in}{0.275000in}}{\pgfqpoint{0.666667in}{0.275000in}}%
\pgfpathclose%
\pgfpathmoveto{\pgfqpoint{0.666667in}{0.280833in}}%
\pgfpathcurveto{\pgfqpoint{0.666667in}{0.280833in}}{\pgfqpoint{0.652744in}{0.280833in}}{\pgfqpoint{0.639389in}{0.286365in}}%
\pgfpathcurveto{\pgfqpoint{0.629544in}{0.296210in}}{\pgfqpoint{0.619698in}{0.306055in}}{\pgfqpoint{0.614167in}{0.319410in}}%
\pgfpathcurveto{\pgfqpoint{0.614167in}{0.333333in}}{\pgfqpoint{0.614167in}{0.347256in}}{\pgfqpoint{0.619698in}{0.360611in}}%
\pgfpathcurveto{\pgfqpoint{0.629544in}{0.370456in}}{\pgfqpoint{0.639389in}{0.380302in}}{\pgfqpoint{0.652744in}{0.385833in}}%
\pgfpathcurveto{\pgfqpoint{0.666667in}{0.385833in}}{\pgfqpoint{0.680590in}{0.385833in}}{\pgfqpoint{0.693945in}{0.380302in}}%
\pgfpathcurveto{\pgfqpoint{0.703790in}{0.370456in}}{\pgfqpoint{0.713635in}{0.360611in}}{\pgfqpoint{0.719167in}{0.347256in}}%
\pgfpathcurveto{\pgfqpoint{0.719167in}{0.333333in}}{\pgfqpoint{0.719167in}{0.319410in}}{\pgfqpoint{0.713635in}{0.306055in}}%
\pgfpathcurveto{\pgfqpoint{0.703790in}{0.296210in}}{\pgfqpoint{0.693945in}{0.286365in}}{\pgfqpoint{0.680590in}{0.280833in}}%
\pgfpathclose%
\pgfpathmoveto{\pgfqpoint{0.833333in}{0.275000in}}%
\pgfpathcurveto{\pgfqpoint{0.848804in}{0.275000in}}{\pgfqpoint{0.863642in}{0.281146in}}{\pgfqpoint{0.874581in}{0.292085in}}%
\pgfpathcurveto{\pgfqpoint{0.885520in}{0.303025in}}{\pgfqpoint{0.891667in}{0.317863in}}{\pgfqpoint{0.891667in}{0.333333in}}%
\pgfpathcurveto{\pgfqpoint{0.891667in}{0.348804in}}{\pgfqpoint{0.885520in}{0.363642in}}{\pgfqpoint{0.874581in}{0.374581in}}%
\pgfpathcurveto{\pgfqpoint{0.863642in}{0.385520in}}{\pgfqpoint{0.848804in}{0.391667in}}{\pgfqpoint{0.833333in}{0.391667in}}%
\pgfpathcurveto{\pgfqpoint{0.817863in}{0.391667in}}{\pgfqpoint{0.803025in}{0.385520in}}{\pgfqpoint{0.792085in}{0.374581in}}%
\pgfpathcurveto{\pgfqpoint{0.781146in}{0.363642in}}{\pgfqpoint{0.775000in}{0.348804in}}{\pgfqpoint{0.775000in}{0.333333in}}%
\pgfpathcurveto{\pgfqpoint{0.775000in}{0.317863in}}{\pgfqpoint{0.781146in}{0.303025in}}{\pgfqpoint{0.792085in}{0.292085in}}%
\pgfpathcurveto{\pgfqpoint{0.803025in}{0.281146in}}{\pgfqpoint{0.817863in}{0.275000in}}{\pgfqpoint{0.833333in}{0.275000in}}%
\pgfpathclose%
\pgfpathmoveto{\pgfqpoint{0.833333in}{0.280833in}}%
\pgfpathcurveto{\pgfqpoint{0.833333in}{0.280833in}}{\pgfqpoint{0.819410in}{0.280833in}}{\pgfqpoint{0.806055in}{0.286365in}}%
\pgfpathcurveto{\pgfqpoint{0.796210in}{0.296210in}}{\pgfqpoint{0.786365in}{0.306055in}}{\pgfqpoint{0.780833in}{0.319410in}}%
\pgfpathcurveto{\pgfqpoint{0.780833in}{0.333333in}}{\pgfqpoint{0.780833in}{0.347256in}}{\pgfqpoint{0.786365in}{0.360611in}}%
\pgfpathcurveto{\pgfqpoint{0.796210in}{0.370456in}}{\pgfqpoint{0.806055in}{0.380302in}}{\pgfqpoint{0.819410in}{0.385833in}}%
\pgfpathcurveto{\pgfqpoint{0.833333in}{0.385833in}}{\pgfqpoint{0.847256in}{0.385833in}}{\pgfqpoint{0.860611in}{0.380302in}}%
\pgfpathcurveto{\pgfqpoint{0.870456in}{0.370456in}}{\pgfqpoint{0.880302in}{0.360611in}}{\pgfqpoint{0.885833in}{0.347256in}}%
\pgfpathcurveto{\pgfqpoint{0.885833in}{0.333333in}}{\pgfqpoint{0.885833in}{0.319410in}}{\pgfqpoint{0.880302in}{0.306055in}}%
\pgfpathcurveto{\pgfqpoint{0.870456in}{0.296210in}}{\pgfqpoint{0.860611in}{0.286365in}}{\pgfqpoint{0.847256in}{0.280833in}}%
\pgfpathclose%
\pgfpathmoveto{\pgfqpoint{1.000000in}{0.275000in}}%
\pgfpathcurveto{\pgfqpoint{1.015470in}{0.275000in}}{\pgfqpoint{1.030309in}{0.281146in}}{\pgfqpoint{1.041248in}{0.292085in}}%
\pgfpathcurveto{\pgfqpoint{1.052187in}{0.303025in}}{\pgfqpoint{1.058333in}{0.317863in}}{\pgfqpoint{1.058333in}{0.333333in}}%
\pgfpathcurveto{\pgfqpoint{1.058333in}{0.348804in}}{\pgfqpoint{1.052187in}{0.363642in}}{\pgfqpoint{1.041248in}{0.374581in}}%
\pgfpathcurveto{\pgfqpoint{1.030309in}{0.385520in}}{\pgfqpoint{1.015470in}{0.391667in}}{\pgfqpoint{1.000000in}{0.391667in}}%
\pgfpathcurveto{\pgfqpoint{0.984530in}{0.391667in}}{\pgfqpoint{0.969691in}{0.385520in}}{\pgfqpoint{0.958752in}{0.374581in}}%
\pgfpathcurveto{\pgfqpoint{0.947813in}{0.363642in}}{\pgfqpoint{0.941667in}{0.348804in}}{\pgfqpoint{0.941667in}{0.333333in}}%
\pgfpathcurveto{\pgfqpoint{0.941667in}{0.317863in}}{\pgfqpoint{0.947813in}{0.303025in}}{\pgfqpoint{0.958752in}{0.292085in}}%
\pgfpathcurveto{\pgfqpoint{0.969691in}{0.281146in}}{\pgfqpoint{0.984530in}{0.275000in}}{\pgfqpoint{1.000000in}{0.275000in}}%
\pgfpathclose%
\pgfpathmoveto{\pgfqpoint{1.000000in}{0.280833in}}%
\pgfpathcurveto{\pgfqpoint{1.000000in}{0.280833in}}{\pgfqpoint{0.986077in}{0.280833in}}{\pgfqpoint{0.972722in}{0.286365in}}%
\pgfpathcurveto{\pgfqpoint{0.962877in}{0.296210in}}{\pgfqpoint{0.953032in}{0.306055in}}{\pgfqpoint{0.947500in}{0.319410in}}%
\pgfpathcurveto{\pgfqpoint{0.947500in}{0.333333in}}{\pgfqpoint{0.947500in}{0.347256in}}{\pgfqpoint{0.953032in}{0.360611in}}%
\pgfpathcurveto{\pgfqpoint{0.962877in}{0.370456in}}{\pgfqpoint{0.972722in}{0.380302in}}{\pgfqpoint{0.986077in}{0.385833in}}%
\pgfpathcurveto{\pgfqpoint{1.000000in}{0.385833in}}{\pgfqpoint{1.013923in}{0.385833in}}{\pgfqpoint{1.027278in}{0.380302in}}%
\pgfpathcurveto{\pgfqpoint{1.037123in}{0.370456in}}{\pgfqpoint{1.046968in}{0.360611in}}{\pgfqpoint{1.052500in}{0.347256in}}%
\pgfpathcurveto{\pgfqpoint{1.052500in}{0.333333in}}{\pgfqpoint{1.052500in}{0.319410in}}{\pgfqpoint{1.046968in}{0.306055in}}%
\pgfpathcurveto{\pgfqpoint{1.037123in}{0.296210in}}{\pgfqpoint{1.027278in}{0.286365in}}{\pgfqpoint{1.013923in}{0.280833in}}%
\pgfpathclose%
\pgfpathmoveto{\pgfqpoint{0.083333in}{0.441667in}}%
\pgfpathcurveto{\pgfqpoint{0.098804in}{0.441667in}}{\pgfqpoint{0.113642in}{0.447813in}}{\pgfqpoint{0.124581in}{0.458752in}}%
\pgfpathcurveto{\pgfqpoint{0.135520in}{0.469691in}}{\pgfqpoint{0.141667in}{0.484530in}}{\pgfqpoint{0.141667in}{0.500000in}}%
\pgfpathcurveto{\pgfqpoint{0.141667in}{0.515470in}}{\pgfqpoint{0.135520in}{0.530309in}}{\pgfqpoint{0.124581in}{0.541248in}}%
\pgfpathcurveto{\pgfqpoint{0.113642in}{0.552187in}}{\pgfqpoint{0.098804in}{0.558333in}}{\pgfqpoint{0.083333in}{0.558333in}}%
\pgfpathcurveto{\pgfqpoint{0.067863in}{0.558333in}}{\pgfqpoint{0.053025in}{0.552187in}}{\pgfqpoint{0.042085in}{0.541248in}}%
\pgfpathcurveto{\pgfqpoint{0.031146in}{0.530309in}}{\pgfqpoint{0.025000in}{0.515470in}}{\pgfqpoint{0.025000in}{0.500000in}}%
\pgfpathcurveto{\pgfqpoint{0.025000in}{0.484530in}}{\pgfqpoint{0.031146in}{0.469691in}}{\pgfqpoint{0.042085in}{0.458752in}}%
\pgfpathcurveto{\pgfqpoint{0.053025in}{0.447813in}}{\pgfqpoint{0.067863in}{0.441667in}}{\pgfqpoint{0.083333in}{0.441667in}}%
\pgfpathclose%
\pgfpathmoveto{\pgfqpoint{0.083333in}{0.447500in}}%
\pgfpathcurveto{\pgfqpoint{0.083333in}{0.447500in}}{\pgfqpoint{0.069410in}{0.447500in}}{\pgfqpoint{0.056055in}{0.453032in}}%
\pgfpathcurveto{\pgfqpoint{0.046210in}{0.462877in}}{\pgfqpoint{0.036365in}{0.472722in}}{\pgfqpoint{0.030833in}{0.486077in}}%
\pgfpathcurveto{\pgfqpoint{0.030833in}{0.500000in}}{\pgfqpoint{0.030833in}{0.513923in}}{\pgfqpoint{0.036365in}{0.527278in}}%
\pgfpathcurveto{\pgfqpoint{0.046210in}{0.537123in}}{\pgfqpoint{0.056055in}{0.546968in}}{\pgfqpoint{0.069410in}{0.552500in}}%
\pgfpathcurveto{\pgfqpoint{0.083333in}{0.552500in}}{\pgfqpoint{0.097256in}{0.552500in}}{\pgfqpoint{0.110611in}{0.546968in}}%
\pgfpathcurveto{\pgfqpoint{0.120456in}{0.537123in}}{\pgfqpoint{0.130302in}{0.527278in}}{\pgfqpoint{0.135833in}{0.513923in}}%
\pgfpathcurveto{\pgfqpoint{0.135833in}{0.500000in}}{\pgfqpoint{0.135833in}{0.486077in}}{\pgfqpoint{0.130302in}{0.472722in}}%
\pgfpathcurveto{\pgfqpoint{0.120456in}{0.462877in}}{\pgfqpoint{0.110611in}{0.453032in}}{\pgfqpoint{0.097256in}{0.447500in}}%
\pgfpathclose%
\pgfpathmoveto{\pgfqpoint{0.250000in}{0.441667in}}%
\pgfpathcurveto{\pgfqpoint{0.265470in}{0.441667in}}{\pgfqpoint{0.280309in}{0.447813in}}{\pgfqpoint{0.291248in}{0.458752in}}%
\pgfpathcurveto{\pgfqpoint{0.302187in}{0.469691in}}{\pgfqpoint{0.308333in}{0.484530in}}{\pgfqpoint{0.308333in}{0.500000in}}%
\pgfpathcurveto{\pgfqpoint{0.308333in}{0.515470in}}{\pgfqpoint{0.302187in}{0.530309in}}{\pgfqpoint{0.291248in}{0.541248in}}%
\pgfpathcurveto{\pgfqpoint{0.280309in}{0.552187in}}{\pgfqpoint{0.265470in}{0.558333in}}{\pgfqpoint{0.250000in}{0.558333in}}%
\pgfpathcurveto{\pgfqpoint{0.234530in}{0.558333in}}{\pgfqpoint{0.219691in}{0.552187in}}{\pgfqpoint{0.208752in}{0.541248in}}%
\pgfpathcurveto{\pgfqpoint{0.197813in}{0.530309in}}{\pgfqpoint{0.191667in}{0.515470in}}{\pgfqpoint{0.191667in}{0.500000in}}%
\pgfpathcurveto{\pgfqpoint{0.191667in}{0.484530in}}{\pgfqpoint{0.197813in}{0.469691in}}{\pgfqpoint{0.208752in}{0.458752in}}%
\pgfpathcurveto{\pgfqpoint{0.219691in}{0.447813in}}{\pgfqpoint{0.234530in}{0.441667in}}{\pgfqpoint{0.250000in}{0.441667in}}%
\pgfpathclose%
\pgfpathmoveto{\pgfqpoint{0.250000in}{0.447500in}}%
\pgfpathcurveto{\pgfqpoint{0.250000in}{0.447500in}}{\pgfqpoint{0.236077in}{0.447500in}}{\pgfqpoint{0.222722in}{0.453032in}}%
\pgfpathcurveto{\pgfqpoint{0.212877in}{0.462877in}}{\pgfqpoint{0.203032in}{0.472722in}}{\pgfqpoint{0.197500in}{0.486077in}}%
\pgfpathcurveto{\pgfqpoint{0.197500in}{0.500000in}}{\pgfqpoint{0.197500in}{0.513923in}}{\pgfqpoint{0.203032in}{0.527278in}}%
\pgfpathcurveto{\pgfqpoint{0.212877in}{0.537123in}}{\pgfqpoint{0.222722in}{0.546968in}}{\pgfqpoint{0.236077in}{0.552500in}}%
\pgfpathcurveto{\pgfqpoint{0.250000in}{0.552500in}}{\pgfqpoint{0.263923in}{0.552500in}}{\pgfqpoint{0.277278in}{0.546968in}}%
\pgfpathcurveto{\pgfqpoint{0.287123in}{0.537123in}}{\pgfqpoint{0.296968in}{0.527278in}}{\pgfqpoint{0.302500in}{0.513923in}}%
\pgfpathcurveto{\pgfqpoint{0.302500in}{0.500000in}}{\pgfqpoint{0.302500in}{0.486077in}}{\pgfqpoint{0.296968in}{0.472722in}}%
\pgfpathcurveto{\pgfqpoint{0.287123in}{0.462877in}}{\pgfqpoint{0.277278in}{0.453032in}}{\pgfqpoint{0.263923in}{0.447500in}}%
\pgfpathclose%
\pgfpathmoveto{\pgfqpoint{0.416667in}{0.441667in}}%
\pgfpathcurveto{\pgfqpoint{0.432137in}{0.441667in}}{\pgfqpoint{0.446975in}{0.447813in}}{\pgfqpoint{0.457915in}{0.458752in}}%
\pgfpathcurveto{\pgfqpoint{0.468854in}{0.469691in}}{\pgfqpoint{0.475000in}{0.484530in}}{\pgfqpoint{0.475000in}{0.500000in}}%
\pgfpathcurveto{\pgfqpoint{0.475000in}{0.515470in}}{\pgfqpoint{0.468854in}{0.530309in}}{\pgfqpoint{0.457915in}{0.541248in}}%
\pgfpathcurveto{\pgfqpoint{0.446975in}{0.552187in}}{\pgfqpoint{0.432137in}{0.558333in}}{\pgfqpoint{0.416667in}{0.558333in}}%
\pgfpathcurveto{\pgfqpoint{0.401196in}{0.558333in}}{\pgfqpoint{0.386358in}{0.552187in}}{\pgfqpoint{0.375419in}{0.541248in}}%
\pgfpathcurveto{\pgfqpoint{0.364480in}{0.530309in}}{\pgfqpoint{0.358333in}{0.515470in}}{\pgfqpoint{0.358333in}{0.500000in}}%
\pgfpathcurveto{\pgfqpoint{0.358333in}{0.484530in}}{\pgfqpoint{0.364480in}{0.469691in}}{\pgfqpoint{0.375419in}{0.458752in}}%
\pgfpathcurveto{\pgfqpoint{0.386358in}{0.447813in}}{\pgfqpoint{0.401196in}{0.441667in}}{\pgfqpoint{0.416667in}{0.441667in}}%
\pgfpathclose%
\pgfpathmoveto{\pgfqpoint{0.416667in}{0.447500in}}%
\pgfpathcurveto{\pgfqpoint{0.416667in}{0.447500in}}{\pgfqpoint{0.402744in}{0.447500in}}{\pgfqpoint{0.389389in}{0.453032in}}%
\pgfpathcurveto{\pgfqpoint{0.379544in}{0.462877in}}{\pgfqpoint{0.369698in}{0.472722in}}{\pgfqpoint{0.364167in}{0.486077in}}%
\pgfpathcurveto{\pgfqpoint{0.364167in}{0.500000in}}{\pgfqpoint{0.364167in}{0.513923in}}{\pgfqpoint{0.369698in}{0.527278in}}%
\pgfpathcurveto{\pgfqpoint{0.379544in}{0.537123in}}{\pgfqpoint{0.389389in}{0.546968in}}{\pgfqpoint{0.402744in}{0.552500in}}%
\pgfpathcurveto{\pgfqpoint{0.416667in}{0.552500in}}{\pgfqpoint{0.430590in}{0.552500in}}{\pgfqpoint{0.443945in}{0.546968in}}%
\pgfpathcurveto{\pgfqpoint{0.453790in}{0.537123in}}{\pgfqpoint{0.463635in}{0.527278in}}{\pgfqpoint{0.469167in}{0.513923in}}%
\pgfpathcurveto{\pgfqpoint{0.469167in}{0.500000in}}{\pgfqpoint{0.469167in}{0.486077in}}{\pgfqpoint{0.463635in}{0.472722in}}%
\pgfpathcurveto{\pgfqpoint{0.453790in}{0.462877in}}{\pgfqpoint{0.443945in}{0.453032in}}{\pgfqpoint{0.430590in}{0.447500in}}%
\pgfpathclose%
\pgfpathmoveto{\pgfqpoint{0.583333in}{0.441667in}}%
\pgfpathcurveto{\pgfqpoint{0.598804in}{0.441667in}}{\pgfqpoint{0.613642in}{0.447813in}}{\pgfqpoint{0.624581in}{0.458752in}}%
\pgfpathcurveto{\pgfqpoint{0.635520in}{0.469691in}}{\pgfqpoint{0.641667in}{0.484530in}}{\pgfqpoint{0.641667in}{0.500000in}}%
\pgfpathcurveto{\pgfqpoint{0.641667in}{0.515470in}}{\pgfqpoint{0.635520in}{0.530309in}}{\pgfqpoint{0.624581in}{0.541248in}}%
\pgfpathcurveto{\pgfqpoint{0.613642in}{0.552187in}}{\pgfqpoint{0.598804in}{0.558333in}}{\pgfqpoint{0.583333in}{0.558333in}}%
\pgfpathcurveto{\pgfqpoint{0.567863in}{0.558333in}}{\pgfqpoint{0.553025in}{0.552187in}}{\pgfqpoint{0.542085in}{0.541248in}}%
\pgfpathcurveto{\pgfqpoint{0.531146in}{0.530309in}}{\pgfqpoint{0.525000in}{0.515470in}}{\pgfqpoint{0.525000in}{0.500000in}}%
\pgfpathcurveto{\pgfqpoint{0.525000in}{0.484530in}}{\pgfqpoint{0.531146in}{0.469691in}}{\pgfqpoint{0.542085in}{0.458752in}}%
\pgfpathcurveto{\pgfqpoint{0.553025in}{0.447813in}}{\pgfqpoint{0.567863in}{0.441667in}}{\pgfqpoint{0.583333in}{0.441667in}}%
\pgfpathclose%
\pgfpathmoveto{\pgfqpoint{0.583333in}{0.447500in}}%
\pgfpathcurveto{\pgfqpoint{0.583333in}{0.447500in}}{\pgfqpoint{0.569410in}{0.447500in}}{\pgfqpoint{0.556055in}{0.453032in}}%
\pgfpathcurveto{\pgfqpoint{0.546210in}{0.462877in}}{\pgfqpoint{0.536365in}{0.472722in}}{\pgfqpoint{0.530833in}{0.486077in}}%
\pgfpathcurveto{\pgfqpoint{0.530833in}{0.500000in}}{\pgfqpoint{0.530833in}{0.513923in}}{\pgfqpoint{0.536365in}{0.527278in}}%
\pgfpathcurveto{\pgfqpoint{0.546210in}{0.537123in}}{\pgfqpoint{0.556055in}{0.546968in}}{\pgfqpoint{0.569410in}{0.552500in}}%
\pgfpathcurveto{\pgfqpoint{0.583333in}{0.552500in}}{\pgfqpoint{0.597256in}{0.552500in}}{\pgfqpoint{0.610611in}{0.546968in}}%
\pgfpathcurveto{\pgfqpoint{0.620456in}{0.537123in}}{\pgfqpoint{0.630302in}{0.527278in}}{\pgfqpoint{0.635833in}{0.513923in}}%
\pgfpathcurveto{\pgfqpoint{0.635833in}{0.500000in}}{\pgfqpoint{0.635833in}{0.486077in}}{\pgfqpoint{0.630302in}{0.472722in}}%
\pgfpathcurveto{\pgfqpoint{0.620456in}{0.462877in}}{\pgfqpoint{0.610611in}{0.453032in}}{\pgfqpoint{0.597256in}{0.447500in}}%
\pgfpathclose%
\pgfpathmoveto{\pgfqpoint{0.750000in}{0.441667in}}%
\pgfpathcurveto{\pgfqpoint{0.765470in}{0.441667in}}{\pgfqpoint{0.780309in}{0.447813in}}{\pgfqpoint{0.791248in}{0.458752in}}%
\pgfpathcurveto{\pgfqpoint{0.802187in}{0.469691in}}{\pgfqpoint{0.808333in}{0.484530in}}{\pgfqpoint{0.808333in}{0.500000in}}%
\pgfpathcurveto{\pgfqpoint{0.808333in}{0.515470in}}{\pgfqpoint{0.802187in}{0.530309in}}{\pgfqpoint{0.791248in}{0.541248in}}%
\pgfpathcurveto{\pgfqpoint{0.780309in}{0.552187in}}{\pgfqpoint{0.765470in}{0.558333in}}{\pgfqpoint{0.750000in}{0.558333in}}%
\pgfpathcurveto{\pgfqpoint{0.734530in}{0.558333in}}{\pgfqpoint{0.719691in}{0.552187in}}{\pgfqpoint{0.708752in}{0.541248in}}%
\pgfpathcurveto{\pgfqpoint{0.697813in}{0.530309in}}{\pgfqpoint{0.691667in}{0.515470in}}{\pgfqpoint{0.691667in}{0.500000in}}%
\pgfpathcurveto{\pgfqpoint{0.691667in}{0.484530in}}{\pgfqpoint{0.697813in}{0.469691in}}{\pgfqpoint{0.708752in}{0.458752in}}%
\pgfpathcurveto{\pgfqpoint{0.719691in}{0.447813in}}{\pgfqpoint{0.734530in}{0.441667in}}{\pgfqpoint{0.750000in}{0.441667in}}%
\pgfpathclose%
\pgfpathmoveto{\pgfqpoint{0.750000in}{0.447500in}}%
\pgfpathcurveto{\pgfqpoint{0.750000in}{0.447500in}}{\pgfqpoint{0.736077in}{0.447500in}}{\pgfqpoint{0.722722in}{0.453032in}}%
\pgfpathcurveto{\pgfqpoint{0.712877in}{0.462877in}}{\pgfqpoint{0.703032in}{0.472722in}}{\pgfqpoint{0.697500in}{0.486077in}}%
\pgfpathcurveto{\pgfqpoint{0.697500in}{0.500000in}}{\pgfqpoint{0.697500in}{0.513923in}}{\pgfqpoint{0.703032in}{0.527278in}}%
\pgfpathcurveto{\pgfqpoint{0.712877in}{0.537123in}}{\pgfqpoint{0.722722in}{0.546968in}}{\pgfqpoint{0.736077in}{0.552500in}}%
\pgfpathcurveto{\pgfqpoint{0.750000in}{0.552500in}}{\pgfqpoint{0.763923in}{0.552500in}}{\pgfqpoint{0.777278in}{0.546968in}}%
\pgfpathcurveto{\pgfqpoint{0.787123in}{0.537123in}}{\pgfqpoint{0.796968in}{0.527278in}}{\pgfqpoint{0.802500in}{0.513923in}}%
\pgfpathcurveto{\pgfqpoint{0.802500in}{0.500000in}}{\pgfqpoint{0.802500in}{0.486077in}}{\pgfqpoint{0.796968in}{0.472722in}}%
\pgfpathcurveto{\pgfqpoint{0.787123in}{0.462877in}}{\pgfqpoint{0.777278in}{0.453032in}}{\pgfqpoint{0.763923in}{0.447500in}}%
\pgfpathclose%
\pgfpathmoveto{\pgfqpoint{0.916667in}{0.441667in}}%
\pgfpathcurveto{\pgfqpoint{0.932137in}{0.441667in}}{\pgfqpoint{0.946975in}{0.447813in}}{\pgfqpoint{0.957915in}{0.458752in}}%
\pgfpathcurveto{\pgfqpoint{0.968854in}{0.469691in}}{\pgfqpoint{0.975000in}{0.484530in}}{\pgfqpoint{0.975000in}{0.500000in}}%
\pgfpathcurveto{\pgfqpoint{0.975000in}{0.515470in}}{\pgfqpoint{0.968854in}{0.530309in}}{\pgfqpoint{0.957915in}{0.541248in}}%
\pgfpathcurveto{\pgfqpoint{0.946975in}{0.552187in}}{\pgfqpoint{0.932137in}{0.558333in}}{\pgfqpoint{0.916667in}{0.558333in}}%
\pgfpathcurveto{\pgfqpoint{0.901196in}{0.558333in}}{\pgfqpoint{0.886358in}{0.552187in}}{\pgfqpoint{0.875419in}{0.541248in}}%
\pgfpathcurveto{\pgfqpoint{0.864480in}{0.530309in}}{\pgfqpoint{0.858333in}{0.515470in}}{\pgfqpoint{0.858333in}{0.500000in}}%
\pgfpathcurveto{\pgfqpoint{0.858333in}{0.484530in}}{\pgfqpoint{0.864480in}{0.469691in}}{\pgfqpoint{0.875419in}{0.458752in}}%
\pgfpathcurveto{\pgfqpoint{0.886358in}{0.447813in}}{\pgfqpoint{0.901196in}{0.441667in}}{\pgfqpoint{0.916667in}{0.441667in}}%
\pgfpathclose%
\pgfpathmoveto{\pgfqpoint{0.916667in}{0.447500in}}%
\pgfpathcurveto{\pgfqpoint{0.916667in}{0.447500in}}{\pgfqpoint{0.902744in}{0.447500in}}{\pgfqpoint{0.889389in}{0.453032in}}%
\pgfpathcurveto{\pgfqpoint{0.879544in}{0.462877in}}{\pgfqpoint{0.869698in}{0.472722in}}{\pgfqpoint{0.864167in}{0.486077in}}%
\pgfpathcurveto{\pgfqpoint{0.864167in}{0.500000in}}{\pgfqpoint{0.864167in}{0.513923in}}{\pgfqpoint{0.869698in}{0.527278in}}%
\pgfpathcurveto{\pgfqpoint{0.879544in}{0.537123in}}{\pgfqpoint{0.889389in}{0.546968in}}{\pgfqpoint{0.902744in}{0.552500in}}%
\pgfpathcurveto{\pgfqpoint{0.916667in}{0.552500in}}{\pgfqpoint{0.930590in}{0.552500in}}{\pgfqpoint{0.943945in}{0.546968in}}%
\pgfpathcurveto{\pgfqpoint{0.953790in}{0.537123in}}{\pgfqpoint{0.963635in}{0.527278in}}{\pgfqpoint{0.969167in}{0.513923in}}%
\pgfpathcurveto{\pgfqpoint{0.969167in}{0.500000in}}{\pgfqpoint{0.969167in}{0.486077in}}{\pgfqpoint{0.963635in}{0.472722in}}%
\pgfpathcurveto{\pgfqpoint{0.953790in}{0.462877in}}{\pgfqpoint{0.943945in}{0.453032in}}{\pgfqpoint{0.930590in}{0.447500in}}%
\pgfpathclose%
\pgfpathmoveto{\pgfqpoint{0.000000in}{0.608333in}}%
\pgfpathcurveto{\pgfqpoint{0.015470in}{0.608333in}}{\pgfqpoint{0.030309in}{0.614480in}}{\pgfqpoint{0.041248in}{0.625419in}}%
\pgfpathcurveto{\pgfqpoint{0.052187in}{0.636358in}}{\pgfqpoint{0.058333in}{0.651196in}}{\pgfqpoint{0.058333in}{0.666667in}}%
\pgfpathcurveto{\pgfqpoint{0.058333in}{0.682137in}}{\pgfqpoint{0.052187in}{0.696975in}}{\pgfqpoint{0.041248in}{0.707915in}}%
\pgfpathcurveto{\pgfqpoint{0.030309in}{0.718854in}}{\pgfqpoint{0.015470in}{0.725000in}}{\pgfqpoint{0.000000in}{0.725000in}}%
\pgfpathcurveto{\pgfqpoint{-0.015470in}{0.725000in}}{\pgfqpoint{-0.030309in}{0.718854in}}{\pgfqpoint{-0.041248in}{0.707915in}}%
\pgfpathcurveto{\pgfqpoint{-0.052187in}{0.696975in}}{\pgfqpoint{-0.058333in}{0.682137in}}{\pgfqpoint{-0.058333in}{0.666667in}}%
\pgfpathcurveto{\pgfqpoint{-0.058333in}{0.651196in}}{\pgfqpoint{-0.052187in}{0.636358in}}{\pgfqpoint{-0.041248in}{0.625419in}}%
\pgfpathcurveto{\pgfqpoint{-0.030309in}{0.614480in}}{\pgfqpoint{-0.015470in}{0.608333in}}{\pgfqpoint{0.000000in}{0.608333in}}%
\pgfpathclose%
\pgfpathmoveto{\pgfqpoint{0.000000in}{0.614167in}}%
\pgfpathcurveto{\pgfqpoint{0.000000in}{0.614167in}}{\pgfqpoint{-0.013923in}{0.614167in}}{\pgfqpoint{-0.027278in}{0.619698in}}%
\pgfpathcurveto{\pgfqpoint{-0.037123in}{0.629544in}}{\pgfqpoint{-0.046968in}{0.639389in}}{\pgfqpoint{-0.052500in}{0.652744in}}%
\pgfpathcurveto{\pgfqpoint{-0.052500in}{0.666667in}}{\pgfqpoint{-0.052500in}{0.680590in}}{\pgfqpoint{-0.046968in}{0.693945in}}%
\pgfpathcurveto{\pgfqpoint{-0.037123in}{0.703790in}}{\pgfqpoint{-0.027278in}{0.713635in}}{\pgfqpoint{-0.013923in}{0.719167in}}%
\pgfpathcurveto{\pgfqpoint{0.000000in}{0.719167in}}{\pgfqpoint{0.013923in}{0.719167in}}{\pgfqpoint{0.027278in}{0.713635in}}%
\pgfpathcurveto{\pgfqpoint{0.037123in}{0.703790in}}{\pgfqpoint{0.046968in}{0.693945in}}{\pgfqpoint{0.052500in}{0.680590in}}%
\pgfpathcurveto{\pgfqpoint{0.052500in}{0.666667in}}{\pgfqpoint{0.052500in}{0.652744in}}{\pgfqpoint{0.046968in}{0.639389in}}%
\pgfpathcurveto{\pgfqpoint{0.037123in}{0.629544in}}{\pgfqpoint{0.027278in}{0.619698in}}{\pgfqpoint{0.013923in}{0.614167in}}%
\pgfpathclose%
\pgfpathmoveto{\pgfqpoint{0.166667in}{0.608333in}}%
\pgfpathcurveto{\pgfqpoint{0.182137in}{0.608333in}}{\pgfqpoint{0.196975in}{0.614480in}}{\pgfqpoint{0.207915in}{0.625419in}}%
\pgfpathcurveto{\pgfqpoint{0.218854in}{0.636358in}}{\pgfqpoint{0.225000in}{0.651196in}}{\pgfqpoint{0.225000in}{0.666667in}}%
\pgfpathcurveto{\pgfqpoint{0.225000in}{0.682137in}}{\pgfqpoint{0.218854in}{0.696975in}}{\pgfqpoint{0.207915in}{0.707915in}}%
\pgfpathcurveto{\pgfqpoint{0.196975in}{0.718854in}}{\pgfqpoint{0.182137in}{0.725000in}}{\pgfqpoint{0.166667in}{0.725000in}}%
\pgfpathcurveto{\pgfqpoint{0.151196in}{0.725000in}}{\pgfqpoint{0.136358in}{0.718854in}}{\pgfqpoint{0.125419in}{0.707915in}}%
\pgfpathcurveto{\pgfqpoint{0.114480in}{0.696975in}}{\pgfqpoint{0.108333in}{0.682137in}}{\pgfqpoint{0.108333in}{0.666667in}}%
\pgfpathcurveto{\pgfqpoint{0.108333in}{0.651196in}}{\pgfqpoint{0.114480in}{0.636358in}}{\pgfqpoint{0.125419in}{0.625419in}}%
\pgfpathcurveto{\pgfqpoint{0.136358in}{0.614480in}}{\pgfqpoint{0.151196in}{0.608333in}}{\pgfqpoint{0.166667in}{0.608333in}}%
\pgfpathclose%
\pgfpathmoveto{\pgfqpoint{0.166667in}{0.614167in}}%
\pgfpathcurveto{\pgfqpoint{0.166667in}{0.614167in}}{\pgfqpoint{0.152744in}{0.614167in}}{\pgfqpoint{0.139389in}{0.619698in}}%
\pgfpathcurveto{\pgfqpoint{0.129544in}{0.629544in}}{\pgfqpoint{0.119698in}{0.639389in}}{\pgfqpoint{0.114167in}{0.652744in}}%
\pgfpathcurveto{\pgfqpoint{0.114167in}{0.666667in}}{\pgfqpoint{0.114167in}{0.680590in}}{\pgfqpoint{0.119698in}{0.693945in}}%
\pgfpathcurveto{\pgfqpoint{0.129544in}{0.703790in}}{\pgfqpoint{0.139389in}{0.713635in}}{\pgfqpoint{0.152744in}{0.719167in}}%
\pgfpathcurveto{\pgfqpoint{0.166667in}{0.719167in}}{\pgfqpoint{0.180590in}{0.719167in}}{\pgfqpoint{0.193945in}{0.713635in}}%
\pgfpathcurveto{\pgfqpoint{0.203790in}{0.703790in}}{\pgfqpoint{0.213635in}{0.693945in}}{\pgfqpoint{0.219167in}{0.680590in}}%
\pgfpathcurveto{\pgfqpoint{0.219167in}{0.666667in}}{\pgfqpoint{0.219167in}{0.652744in}}{\pgfqpoint{0.213635in}{0.639389in}}%
\pgfpathcurveto{\pgfqpoint{0.203790in}{0.629544in}}{\pgfqpoint{0.193945in}{0.619698in}}{\pgfqpoint{0.180590in}{0.614167in}}%
\pgfpathclose%
\pgfpathmoveto{\pgfqpoint{0.333333in}{0.608333in}}%
\pgfpathcurveto{\pgfqpoint{0.348804in}{0.608333in}}{\pgfqpoint{0.363642in}{0.614480in}}{\pgfqpoint{0.374581in}{0.625419in}}%
\pgfpathcurveto{\pgfqpoint{0.385520in}{0.636358in}}{\pgfqpoint{0.391667in}{0.651196in}}{\pgfqpoint{0.391667in}{0.666667in}}%
\pgfpathcurveto{\pgfqpoint{0.391667in}{0.682137in}}{\pgfqpoint{0.385520in}{0.696975in}}{\pgfqpoint{0.374581in}{0.707915in}}%
\pgfpathcurveto{\pgfqpoint{0.363642in}{0.718854in}}{\pgfqpoint{0.348804in}{0.725000in}}{\pgfqpoint{0.333333in}{0.725000in}}%
\pgfpathcurveto{\pgfqpoint{0.317863in}{0.725000in}}{\pgfqpoint{0.303025in}{0.718854in}}{\pgfqpoint{0.292085in}{0.707915in}}%
\pgfpathcurveto{\pgfqpoint{0.281146in}{0.696975in}}{\pgfqpoint{0.275000in}{0.682137in}}{\pgfqpoint{0.275000in}{0.666667in}}%
\pgfpathcurveto{\pgfqpoint{0.275000in}{0.651196in}}{\pgfqpoint{0.281146in}{0.636358in}}{\pgfqpoint{0.292085in}{0.625419in}}%
\pgfpathcurveto{\pgfqpoint{0.303025in}{0.614480in}}{\pgfqpoint{0.317863in}{0.608333in}}{\pgfqpoint{0.333333in}{0.608333in}}%
\pgfpathclose%
\pgfpathmoveto{\pgfqpoint{0.333333in}{0.614167in}}%
\pgfpathcurveto{\pgfqpoint{0.333333in}{0.614167in}}{\pgfqpoint{0.319410in}{0.614167in}}{\pgfqpoint{0.306055in}{0.619698in}}%
\pgfpathcurveto{\pgfqpoint{0.296210in}{0.629544in}}{\pgfqpoint{0.286365in}{0.639389in}}{\pgfqpoint{0.280833in}{0.652744in}}%
\pgfpathcurveto{\pgfqpoint{0.280833in}{0.666667in}}{\pgfqpoint{0.280833in}{0.680590in}}{\pgfqpoint{0.286365in}{0.693945in}}%
\pgfpathcurveto{\pgfqpoint{0.296210in}{0.703790in}}{\pgfqpoint{0.306055in}{0.713635in}}{\pgfqpoint{0.319410in}{0.719167in}}%
\pgfpathcurveto{\pgfqpoint{0.333333in}{0.719167in}}{\pgfqpoint{0.347256in}{0.719167in}}{\pgfqpoint{0.360611in}{0.713635in}}%
\pgfpathcurveto{\pgfqpoint{0.370456in}{0.703790in}}{\pgfqpoint{0.380302in}{0.693945in}}{\pgfqpoint{0.385833in}{0.680590in}}%
\pgfpathcurveto{\pgfqpoint{0.385833in}{0.666667in}}{\pgfqpoint{0.385833in}{0.652744in}}{\pgfqpoint{0.380302in}{0.639389in}}%
\pgfpathcurveto{\pgfqpoint{0.370456in}{0.629544in}}{\pgfqpoint{0.360611in}{0.619698in}}{\pgfqpoint{0.347256in}{0.614167in}}%
\pgfpathclose%
\pgfpathmoveto{\pgfqpoint{0.500000in}{0.608333in}}%
\pgfpathcurveto{\pgfqpoint{0.515470in}{0.608333in}}{\pgfqpoint{0.530309in}{0.614480in}}{\pgfqpoint{0.541248in}{0.625419in}}%
\pgfpathcurveto{\pgfqpoint{0.552187in}{0.636358in}}{\pgfqpoint{0.558333in}{0.651196in}}{\pgfqpoint{0.558333in}{0.666667in}}%
\pgfpathcurveto{\pgfqpoint{0.558333in}{0.682137in}}{\pgfqpoint{0.552187in}{0.696975in}}{\pgfqpoint{0.541248in}{0.707915in}}%
\pgfpathcurveto{\pgfqpoint{0.530309in}{0.718854in}}{\pgfqpoint{0.515470in}{0.725000in}}{\pgfqpoint{0.500000in}{0.725000in}}%
\pgfpathcurveto{\pgfqpoint{0.484530in}{0.725000in}}{\pgfqpoint{0.469691in}{0.718854in}}{\pgfqpoint{0.458752in}{0.707915in}}%
\pgfpathcurveto{\pgfqpoint{0.447813in}{0.696975in}}{\pgfqpoint{0.441667in}{0.682137in}}{\pgfqpoint{0.441667in}{0.666667in}}%
\pgfpathcurveto{\pgfqpoint{0.441667in}{0.651196in}}{\pgfqpoint{0.447813in}{0.636358in}}{\pgfqpoint{0.458752in}{0.625419in}}%
\pgfpathcurveto{\pgfqpoint{0.469691in}{0.614480in}}{\pgfqpoint{0.484530in}{0.608333in}}{\pgfqpoint{0.500000in}{0.608333in}}%
\pgfpathclose%
\pgfpathmoveto{\pgfqpoint{0.500000in}{0.614167in}}%
\pgfpathcurveto{\pgfqpoint{0.500000in}{0.614167in}}{\pgfqpoint{0.486077in}{0.614167in}}{\pgfqpoint{0.472722in}{0.619698in}}%
\pgfpathcurveto{\pgfqpoint{0.462877in}{0.629544in}}{\pgfqpoint{0.453032in}{0.639389in}}{\pgfqpoint{0.447500in}{0.652744in}}%
\pgfpathcurveto{\pgfqpoint{0.447500in}{0.666667in}}{\pgfqpoint{0.447500in}{0.680590in}}{\pgfqpoint{0.453032in}{0.693945in}}%
\pgfpathcurveto{\pgfqpoint{0.462877in}{0.703790in}}{\pgfqpoint{0.472722in}{0.713635in}}{\pgfqpoint{0.486077in}{0.719167in}}%
\pgfpathcurveto{\pgfqpoint{0.500000in}{0.719167in}}{\pgfqpoint{0.513923in}{0.719167in}}{\pgfqpoint{0.527278in}{0.713635in}}%
\pgfpathcurveto{\pgfqpoint{0.537123in}{0.703790in}}{\pgfqpoint{0.546968in}{0.693945in}}{\pgfqpoint{0.552500in}{0.680590in}}%
\pgfpathcurveto{\pgfqpoint{0.552500in}{0.666667in}}{\pgfqpoint{0.552500in}{0.652744in}}{\pgfqpoint{0.546968in}{0.639389in}}%
\pgfpathcurveto{\pgfqpoint{0.537123in}{0.629544in}}{\pgfqpoint{0.527278in}{0.619698in}}{\pgfqpoint{0.513923in}{0.614167in}}%
\pgfpathclose%
\pgfpathmoveto{\pgfqpoint{0.666667in}{0.608333in}}%
\pgfpathcurveto{\pgfqpoint{0.682137in}{0.608333in}}{\pgfqpoint{0.696975in}{0.614480in}}{\pgfqpoint{0.707915in}{0.625419in}}%
\pgfpathcurveto{\pgfqpoint{0.718854in}{0.636358in}}{\pgfqpoint{0.725000in}{0.651196in}}{\pgfqpoint{0.725000in}{0.666667in}}%
\pgfpathcurveto{\pgfqpoint{0.725000in}{0.682137in}}{\pgfqpoint{0.718854in}{0.696975in}}{\pgfqpoint{0.707915in}{0.707915in}}%
\pgfpathcurveto{\pgfqpoint{0.696975in}{0.718854in}}{\pgfqpoint{0.682137in}{0.725000in}}{\pgfqpoint{0.666667in}{0.725000in}}%
\pgfpathcurveto{\pgfqpoint{0.651196in}{0.725000in}}{\pgfqpoint{0.636358in}{0.718854in}}{\pgfqpoint{0.625419in}{0.707915in}}%
\pgfpathcurveto{\pgfqpoint{0.614480in}{0.696975in}}{\pgfqpoint{0.608333in}{0.682137in}}{\pgfqpoint{0.608333in}{0.666667in}}%
\pgfpathcurveto{\pgfqpoint{0.608333in}{0.651196in}}{\pgfqpoint{0.614480in}{0.636358in}}{\pgfqpoint{0.625419in}{0.625419in}}%
\pgfpathcurveto{\pgfqpoint{0.636358in}{0.614480in}}{\pgfqpoint{0.651196in}{0.608333in}}{\pgfqpoint{0.666667in}{0.608333in}}%
\pgfpathclose%
\pgfpathmoveto{\pgfqpoint{0.666667in}{0.614167in}}%
\pgfpathcurveto{\pgfqpoint{0.666667in}{0.614167in}}{\pgfqpoint{0.652744in}{0.614167in}}{\pgfqpoint{0.639389in}{0.619698in}}%
\pgfpathcurveto{\pgfqpoint{0.629544in}{0.629544in}}{\pgfqpoint{0.619698in}{0.639389in}}{\pgfqpoint{0.614167in}{0.652744in}}%
\pgfpathcurveto{\pgfqpoint{0.614167in}{0.666667in}}{\pgfqpoint{0.614167in}{0.680590in}}{\pgfqpoint{0.619698in}{0.693945in}}%
\pgfpathcurveto{\pgfqpoint{0.629544in}{0.703790in}}{\pgfqpoint{0.639389in}{0.713635in}}{\pgfqpoint{0.652744in}{0.719167in}}%
\pgfpathcurveto{\pgfqpoint{0.666667in}{0.719167in}}{\pgfqpoint{0.680590in}{0.719167in}}{\pgfqpoint{0.693945in}{0.713635in}}%
\pgfpathcurveto{\pgfqpoint{0.703790in}{0.703790in}}{\pgfqpoint{0.713635in}{0.693945in}}{\pgfqpoint{0.719167in}{0.680590in}}%
\pgfpathcurveto{\pgfqpoint{0.719167in}{0.666667in}}{\pgfqpoint{0.719167in}{0.652744in}}{\pgfqpoint{0.713635in}{0.639389in}}%
\pgfpathcurveto{\pgfqpoint{0.703790in}{0.629544in}}{\pgfqpoint{0.693945in}{0.619698in}}{\pgfqpoint{0.680590in}{0.614167in}}%
\pgfpathclose%
\pgfpathmoveto{\pgfqpoint{0.833333in}{0.608333in}}%
\pgfpathcurveto{\pgfqpoint{0.848804in}{0.608333in}}{\pgfqpoint{0.863642in}{0.614480in}}{\pgfqpoint{0.874581in}{0.625419in}}%
\pgfpathcurveto{\pgfqpoint{0.885520in}{0.636358in}}{\pgfqpoint{0.891667in}{0.651196in}}{\pgfqpoint{0.891667in}{0.666667in}}%
\pgfpathcurveto{\pgfqpoint{0.891667in}{0.682137in}}{\pgfqpoint{0.885520in}{0.696975in}}{\pgfqpoint{0.874581in}{0.707915in}}%
\pgfpathcurveto{\pgfqpoint{0.863642in}{0.718854in}}{\pgfqpoint{0.848804in}{0.725000in}}{\pgfqpoint{0.833333in}{0.725000in}}%
\pgfpathcurveto{\pgfqpoint{0.817863in}{0.725000in}}{\pgfqpoint{0.803025in}{0.718854in}}{\pgfqpoint{0.792085in}{0.707915in}}%
\pgfpathcurveto{\pgfqpoint{0.781146in}{0.696975in}}{\pgfqpoint{0.775000in}{0.682137in}}{\pgfqpoint{0.775000in}{0.666667in}}%
\pgfpathcurveto{\pgfqpoint{0.775000in}{0.651196in}}{\pgfqpoint{0.781146in}{0.636358in}}{\pgfqpoint{0.792085in}{0.625419in}}%
\pgfpathcurveto{\pgfqpoint{0.803025in}{0.614480in}}{\pgfqpoint{0.817863in}{0.608333in}}{\pgfqpoint{0.833333in}{0.608333in}}%
\pgfpathclose%
\pgfpathmoveto{\pgfqpoint{0.833333in}{0.614167in}}%
\pgfpathcurveto{\pgfqpoint{0.833333in}{0.614167in}}{\pgfqpoint{0.819410in}{0.614167in}}{\pgfqpoint{0.806055in}{0.619698in}}%
\pgfpathcurveto{\pgfqpoint{0.796210in}{0.629544in}}{\pgfqpoint{0.786365in}{0.639389in}}{\pgfqpoint{0.780833in}{0.652744in}}%
\pgfpathcurveto{\pgfqpoint{0.780833in}{0.666667in}}{\pgfqpoint{0.780833in}{0.680590in}}{\pgfqpoint{0.786365in}{0.693945in}}%
\pgfpathcurveto{\pgfqpoint{0.796210in}{0.703790in}}{\pgfqpoint{0.806055in}{0.713635in}}{\pgfqpoint{0.819410in}{0.719167in}}%
\pgfpathcurveto{\pgfqpoint{0.833333in}{0.719167in}}{\pgfqpoint{0.847256in}{0.719167in}}{\pgfqpoint{0.860611in}{0.713635in}}%
\pgfpathcurveto{\pgfqpoint{0.870456in}{0.703790in}}{\pgfqpoint{0.880302in}{0.693945in}}{\pgfqpoint{0.885833in}{0.680590in}}%
\pgfpathcurveto{\pgfqpoint{0.885833in}{0.666667in}}{\pgfqpoint{0.885833in}{0.652744in}}{\pgfqpoint{0.880302in}{0.639389in}}%
\pgfpathcurveto{\pgfqpoint{0.870456in}{0.629544in}}{\pgfqpoint{0.860611in}{0.619698in}}{\pgfqpoint{0.847256in}{0.614167in}}%
\pgfpathclose%
\pgfpathmoveto{\pgfqpoint{1.000000in}{0.608333in}}%
\pgfpathcurveto{\pgfqpoint{1.015470in}{0.608333in}}{\pgfqpoint{1.030309in}{0.614480in}}{\pgfqpoint{1.041248in}{0.625419in}}%
\pgfpathcurveto{\pgfqpoint{1.052187in}{0.636358in}}{\pgfqpoint{1.058333in}{0.651196in}}{\pgfqpoint{1.058333in}{0.666667in}}%
\pgfpathcurveto{\pgfqpoint{1.058333in}{0.682137in}}{\pgfqpoint{1.052187in}{0.696975in}}{\pgfqpoint{1.041248in}{0.707915in}}%
\pgfpathcurveto{\pgfqpoint{1.030309in}{0.718854in}}{\pgfqpoint{1.015470in}{0.725000in}}{\pgfqpoint{1.000000in}{0.725000in}}%
\pgfpathcurveto{\pgfqpoint{0.984530in}{0.725000in}}{\pgfqpoint{0.969691in}{0.718854in}}{\pgfqpoint{0.958752in}{0.707915in}}%
\pgfpathcurveto{\pgfqpoint{0.947813in}{0.696975in}}{\pgfqpoint{0.941667in}{0.682137in}}{\pgfqpoint{0.941667in}{0.666667in}}%
\pgfpathcurveto{\pgfqpoint{0.941667in}{0.651196in}}{\pgfqpoint{0.947813in}{0.636358in}}{\pgfqpoint{0.958752in}{0.625419in}}%
\pgfpathcurveto{\pgfqpoint{0.969691in}{0.614480in}}{\pgfqpoint{0.984530in}{0.608333in}}{\pgfqpoint{1.000000in}{0.608333in}}%
\pgfpathclose%
\pgfpathmoveto{\pgfqpoint{1.000000in}{0.614167in}}%
\pgfpathcurveto{\pgfqpoint{1.000000in}{0.614167in}}{\pgfqpoint{0.986077in}{0.614167in}}{\pgfqpoint{0.972722in}{0.619698in}}%
\pgfpathcurveto{\pgfqpoint{0.962877in}{0.629544in}}{\pgfqpoint{0.953032in}{0.639389in}}{\pgfqpoint{0.947500in}{0.652744in}}%
\pgfpathcurveto{\pgfqpoint{0.947500in}{0.666667in}}{\pgfqpoint{0.947500in}{0.680590in}}{\pgfqpoint{0.953032in}{0.693945in}}%
\pgfpathcurveto{\pgfqpoint{0.962877in}{0.703790in}}{\pgfqpoint{0.972722in}{0.713635in}}{\pgfqpoint{0.986077in}{0.719167in}}%
\pgfpathcurveto{\pgfqpoint{1.000000in}{0.719167in}}{\pgfqpoint{1.013923in}{0.719167in}}{\pgfqpoint{1.027278in}{0.713635in}}%
\pgfpathcurveto{\pgfqpoint{1.037123in}{0.703790in}}{\pgfqpoint{1.046968in}{0.693945in}}{\pgfqpoint{1.052500in}{0.680590in}}%
\pgfpathcurveto{\pgfqpoint{1.052500in}{0.666667in}}{\pgfqpoint{1.052500in}{0.652744in}}{\pgfqpoint{1.046968in}{0.639389in}}%
\pgfpathcurveto{\pgfqpoint{1.037123in}{0.629544in}}{\pgfqpoint{1.027278in}{0.619698in}}{\pgfqpoint{1.013923in}{0.614167in}}%
\pgfpathclose%
\pgfpathmoveto{\pgfqpoint{0.083333in}{0.775000in}}%
\pgfpathcurveto{\pgfqpoint{0.098804in}{0.775000in}}{\pgfqpoint{0.113642in}{0.781146in}}{\pgfqpoint{0.124581in}{0.792085in}}%
\pgfpathcurveto{\pgfqpoint{0.135520in}{0.803025in}}{\pgfqpoint{0.141667in}{0.817863in}}{\pgfqpoint{0.141667in}{0.833333in}}%
\pgfpathcurveto{\pgfqpoint{0.141667in}{0.848804in}}{\pgfqpoint{0.135520in}{0.863642in}}{\pgfqpoint{0.124581in}{0.874581in}}%
\pgfpathcurveto{\pgfqpoint{0.113642in}{0.885520in}}{\pgfqpoint{0.098804in}{0.891667in}}{\pgfqpoint{0.083333in}{0.891667in}}%
\pgfpathcurveto{\pgfqpoint{0.067863in}{0.891667in}}{\pgfqpoint{0.053025in}{0.885520in}}{\pgfqpoint{0.042085in}{0.874581in}}%
\pgfpathcurveto{\pgfqpoint{0.031146in}{0.863642in}}{\pgfqpoint{0.025000in}{0.848804in}}{\pgfqpoint{0.025000in}{0.833333in}}%
\pgfpathcurveto{\pgfqpoint{0.025000in}{0.817863in}}{\pgfqpoint{0.031146in}{0.803025in}}{\pgfqpoint{0.042085in}{0.792085in}}%
\pgfpathcurveto{\pgfqpoint{0.053025in}{0.781146in}}{\pgfqpoint{0.067863in}{0.775000in}}{\pgfqpoint{0.083333in}{0.775000in}}%
\pgfpathclose%
\pgfpathmoveto{\pgfqpoint{0.083333in}{0.780833in}}%
\pgfpathcurveto{\pgfqpoint{0.083333in}{0.780833in}}{\pgfqpoint{0.069410in}{0.780833in}}{\pgfqpoint{0.056055in}{0.786365in}}%
\pgfpathcurveto{\pgfqpoint{0.046210in}{0.796210in}}{\pgfqpoint{0.036365in}{0.806055in}}{\pgfqpoint{0.030833in}{0.819410in}}%
\pgfpathcurveto{\pgfqpoint{0.030833in}{0.833333in}}{\pgfqpoint{0.030833in}{0.847256in}}{\pgfqpoint{0.036365in}{0.860611in}}%
\pgfpathcurveto{\pgfqpoint{0.046210in}{0.870456in}}{\pgfqpoint{0.056055in}{0.880302in}}{\pgfqpoint{0.069410in}{0.885833in}}%
\pgfpathcurveto{\pgfqpoint{0.083333in}{0.885833in}}{\pgfqpoint{0.097256in}{0.885833in}}{\pgfqpoint{0.110611in}{0.880302in}}%
\pgfpathcurveto{\pgfqpoint{0.120456in}{0.870456in}}{\pgfqpoint{0.130302in}{0.860611in}}{\pgfqpoint{0.135833in}{0.847256in}}%
\pgfpathcurveto{\pgfqpoint{0.135833in}{0.833333in}}{\pgfqpoint{0.135833in}{0.819410in}}{\pgfqpoint{0.130302in}{0.806055in}}%
\pgfpathcurveto{\pgfqpoint{0.120456in}{0.796210in}}{\pgfqpoint{0.110611in}{0.786365in}}{\pgfqpoint{0.097256in}{0.780833in}}%
\pgfpathclose%
\pgfpathmoveto{\pgfqpoint{0.250000in}{0.775000in}}%
\pgfpathcurveto{\pgfqpoint{0.265470in}{0.775000in}}{\pgfqpoint{0.280309in}{0.781146in}}{\pgfqpoint{0.291248in}{0.792085in}}%
\pgfpathcurveto{\pgfqpoint{0.302187in}{0.803025in}}{\pgfqpoint{0.308333in}{0.817863in}}{\pgfqpoint{0.308333in}{0.833333in}}%
\pgfpathcurveto{\pgfqpoint{0.308333in}{0.848804in}}{\pgfqpoint{0.302187in}{0.863642in}}{\pgfqpoint{0.291248in}{0.874581in}}%
\pgfpathcurveto{\pgfqpoint{0.280309in}{0.885520in}}{\pgfqpoint{0.265470in}{0.891667in}}{\pgfqpoint{0.250000in}{0.891667in}}%
\pgfpathcurveto{\pgfqpoint{0.234530in}{0.891667in}}{\pgfqpoint{0.219691in}{0.885520in}}{\pgfqpoint{0.208752in}{0.874581in}}%
\pgfpathcurveto{\pgfqpoint{0.197813in}{0.863642in}}{\pgfqpoint{0.191667in}{0.848804in}}{\pgfqpoint{0.191667in}{0.833333in}}%
\pgfpathcurveto{\pgfqpoint{0.191667in}{0.817863in}}{\pgfqpoint{0.197813in}{0.803025in}}{\pgfqpoint{0.208752in}{0.792085in}}%
\pgfpathcurveto{\pgfqpoint{0.219691in}{0.781146in}}{\pgfqpoint{0.234530in}{0.775000in}}{\pgfqpoint{0.250000in}{0.775000in}}%
\pgfpathclose%
\pgfpathmoveto{\pgfqpoint{0.250000in}{0.780833in}}%
\pgfpathcurveto{\pgfqpoint{0.250000in}{0.780833in}}{\pgfqpoint{0.236077in}{0.780833in}}{\pgfqpoint{0.222722in}{0.786365in}}%
\pgfpathcurveto{\pgfqpoint{0.212877in}{0.796210in}}{\pgfqpoint{0.203032in}{0.806055in}}{\pgfqpoint{0.197500in}{0.819410in}}%
\pgfpathcurveto{\pgfqpoint{0.197500in}{0.833333in}}{\pgfqpoint{0.197500in}{0.847256in}}{\pgfqpoint{0.203032in}{0.860611in}}%
\pgfpathcurveto{\pgfqpoint{0.212877in}{0.870456in}}{\pgfqpoint{0.222722in}{0.880302in}}{\pgfqpoint{0.236077in}{0.885833in}}%
\pgfpathcurveto{\pgfqpoint{0.250000in}{0.885833in}}{\pgfqpoint{0.263923in}{0.885833in}}{\pgfqpoint{0.277278in}{0.880302in}}%
\pgfpathcurveto{\pgfqpoint{0.287123in}{0.870456in}}{\pgfqpoint{0.296968in}{0.860611in}}{\pgfqpoint{0.302500in}{0.847256in}}%
\pgfpathcurveto{\pgfqpoint{0.302500in}{0.833333in}}{\pgfqpoint{0.302500in}{0.819410in}}{\pgfqpoint{0.296968in}{0.806055in}}%
\pgfpathcurveto{\pgfqpoint{0.287123in}{0.796210in}}{\pgfqpoint{0.277278in}{0.786365in}}{\pgfqpoint{0.263923in}{0.780833in}}%
\pgfpathclose%
\pgfpathmoveto{\pgfqpoint{0.416667in}{0.775000in}}%
\pgfpathcurveto{\pgfqpoint{0.432137in}{0.775000in}}{\pgfqpoint{0.446975in}{0.781146in}}{\pgfqpoint{0.457915in}{0.792085in}}%
\pgfpathcurveto{\pgfqpoint{0.468854in}{0.803025in}}{\pgfqpoint{0.475000in}{0.817863in}}{\pgfqpoint{0.475000in}{0.833333in}}%
\pgfpathcurveto{\pgfqpoint{0.475000in}{0.848804in}}{\pgfqpoint{0.468854in}{0.863642in}}{\pgfqpoint{0.457915in}{0.874581in}}%
\pgfpathcurveto{\pgfqpoint{0.446975in}{0.885520in}}{\pgfqpoint{0.432137in}{0.891667in}}{\pgfqpoint{0.416667in}{0.891667in}}%
\pgfpathcurveto{\pgfqpoint{0.401196in}{0.891667in}}{\pgfqpoint{0.386358in}{0.885520in}}{\pgfqpoint{0.375419in}{0.874581in}}%
\pgfpathcurveto{\pgfqpoint{0.364480in}{0.863642in}}{\pgfqpoint{0.358333in}{0.848804in}}{\pgfqpoint{0.358333in}{0.833333in}}%
\pgfpathcurveto{\pgfqpoint{0.358333in}{0.817863in}}{\pgfqpoint{0.364480in}{0.803025in}}{\pgfqpoint{0.375419in}{0.792085in}}%
\pgfpathcurveto{\pgfqpoint{0.386358in}{0.781146in}}{\pgfqpoint{0.401196in}{0.775000in}}{\pgfqpoint{0.416667in}{0.775000in}}%
\pgfpathclose%
\pgfpathmoveto{\pgfqpoint{0.416667in}{0.780833in}}%
\pgfpathcurveto{\pgfqpoint{0.416667in}{0.780833in}}{\pgfqpoint{0.402744in}{0.780833in}}{\pgfqpoint{0.389389in}{0.786365in}}%
\pgfpathcurveto{\pgfqpoint{0.379544in}{0.796210in}}{\pgfqpoint{0.369698in}{0.806055in}}{\pgfqpoint{0.364167in}{0.819410in}}%
\pgfpathcurveto{\pgfqpoint{0.364167in}{0.833333in}}{\pgfqpoint{0.364167in}{0.847256in}}{\pgfqpoint{0.369698in}{0.860611in}}%
\pgfpathcurveto{\pgfqpoint{0.379544in}{0.870456in}}{\pgfqpoint{0.389389in}{0.880302in}}{\pgfqpoint{0.402744in}{0.885833in}}%
\pgfpathcurveto{\pgfqpoint{0.416667in}{0.885833in}}{\pgfqpoint{0.430590in}{0.885833in}}{\pgfqpoint{0.443945in}{0.880302in}}%
\pgfpathcurveto{\pgfqpoint{0.453790in}{0.870456in}}{\pgfqpoint{0.463635in}{0.860611in}}{\pgfqpoint{0.469167in}{0.847256in}}%
\pgfpathcurveto{\pgfqpoint{0.469167in}{0.833333in}}{\pgfqpoint{0.469167in}{0.819410in}}{\pgfqpoint{0.463635in}{0.806055in}}%
\pgfpathcurveto{\pgfqpoint{0.453790in}{0.796210in}}{\pgfqpoint{0.443945in}{0.786365in}}{\pgfqpoint{0.430590in}{0.780833in}}%
\pgfpathclose%
\pgfpathmoveto{\pgfqpoint{0.583333in}{0.775000in}}%
\pgfpathcurveto{\pgfqpoint{0.598804in}{0.775000in}}{\pgfqpoint{0.613642in}{0.781146in}}{\pgfqpoint{0.624581in}{0.792085in}}%
\pgfpathcurveto{\pgfqpoint{0.635520in}{0.803025in}}{\pgfqpoint{0.641667in}{0.817863in}}{\pgfqpoint{0.641667in}{0.833333in}}%
\pgfpathcurveto{\pgfqpoint{0.641667in}{0.848804in}}{\pgfqpoint{0.635520in}{0.863642in}}{\pgfqpoint{0.624581in}{0.874581in}}%
\pgfpathcurveto{\pgfqpoint{0.613642in}{0.885520in}}{\pgfqpoint{0.598804in}{0.891667in}}{\pgfqpoint{0.583333in}{0.891667in}}%
\pgfpathcurveto{\pgfqpoint{0.567863in}{0.891667in}}{\pgfqpoint{0.553025in}{0.885520in}}{\pgfqpoint{0.542085in}{0.874581in}}%
\pgfpathcurveto{\pgfqpoint{0.531146in}{0.863642in}}{\pgfqpoint{0.525000in}{0.848804in}}{\pgfqpoint{0.525000in}{0.833333in}}%
\pgfpathcurveto{\pgfqpoint{0.525000in}{0.817863in}}{\pgfqpoint{0.531146in}{0.803025in}}{\pgfqpoint{0.542085in}{0.792085in}}%
\pgfpathcurveto{\pgfqpoint{0.553025in}{0.781146in}}{\pgfqpoint{0.567863in}{0.775000in}}{\pgfqpoint{0.583333in}{0.775000in}}%
\pgfpathclose%
\pgfpathmoveto{\pgfqpoint{0.583333in}{0.780833in}}%
\pgfpathcurveto{\pgfqpoint{0.583333in}{0.780833in}}{\pgfqpoint{0.569410in}{0.780833in}}{\pgfqpoint{0.556055in}{0.786365in}}%
\pgfpathcurveto{\pgfqpoint{0.546210in}{0.796210in}}{\pgfqpoint{0.536365in}{0.806055in}}{\pgfqpoint{0.530833in}{0.819410in}}%
\pgfpathcurveto{\pgfqpoint{0.530833in}{0.833333in}}{\pgfqpoint{0.530833in}{0.847256in}}{\pgfqpoint{0.536365in}{0.860611in}}%
\pgfpathcurveto{\pgfqpoint{0.546210in}{0.870456in}}{\pgfqpoint{0.556055in}{0.880302in}}{\pgfqpoint{0.569410in}{0.885833in}}%
\pgfpathcurveto{\pgfqpoint{0.583333in}{0.885833in}}{\pgfqpoint{0.597256in}{0.885833in}}{\pgfqpoint{0.610611in}{0.880302in}}%
\pgfpathcurveto{\pgfqpoint{0.620456in}{0.870456in}}{\pgfqpoint{0.630302in}{0.860611in}}{\pgfqpoint{0.635833in}{0.847256in}}%
\pgfpathcurveto{\pgfqpoint{0.635833in}{0.833333in}}{\pgfqpoint{0.635833in}{0.819410in}}{\pgfqpoint{0.630302in}{0.806055in}}%
\pgfpathcurveto{\pgfqpoint{0.620456in}{0.796210in}}{\pgfqpoint{0.610611in}{0.786365in}}{\pgfqpoint{0.597256in}{0.780833in}}%
\pgfpathclose%
\pgfpathmoveto{\pgfqpoint{0.750000in}{0.775000in}}%
\pgfpathcurveto{\pgfqpoint{0.765470in}{0.775000in}}{\pgfqpoint{0.780309in}{0.781146in}}{\pgfqpoint{0.791248in}{0.792085in}}%
\pgfpathcurveto{\pgfqpoint{0.802187in}{0.803025in}}{\pgfqpoint{0.808333in}{0.817863in}}{\pgfqpoint{0.808333in}{0.833333in}}%
\pgfpathcurveto{\pgfqpoint{0.808333in}{0.848804in}}{\pgfqpoint{0.802187in}{0.863642in}}{\pgfqpoint{0.791248in}{0.874581in}}%
\pgfpathcurveto{\pgfqpoint{0.780309in}{0.885520in}}{\pgfqpoint{0.765470in}{0.891667in}}{\pgfqpoint{0.750000in}{0.891667in}}%
\pgfpathcurveto{\pgfqpoint{0.734530in}{0.891667in}}{\pgfqpoint{0.719691in}{0.885520in}}{\pgfqpoint{0.708752in}{0.874581in}}%
\pgfpathcurveto{\pgfqpoint{0.697813in}{0.863642in}}{\pgfqpoint{0.691667in}{0.848804in}}{\pgfqpoint{0.691667in}{0.833333in}}%
\pgfpathcurveto{\pgfqpoint{0.691667in}{0.817863in}}{\pgfqpoint{0.697813in}{0.803025in}}{\pgfqpoint{0.708752in}{0.792085in}}%
\pgfpathcurveto{\pgfqpoint{0.719691in}{0.781146in}}{\pgfqpoint{0.734530in}{0.775000in}}{\pgfqpoint{0.750000in}{0.775000in}}%
\pgfpathclose%
\pgfpathmoveto{\pgfqpoint{0.750000in}{0.780833in}}%
\pgfpathcurveto{\pgfqpoint{0.750000in}{0.780833in}}{\pgfqpoint{0.736077in}{0.780833in}}{\pgfqpoint{0.722722in}{0.786365in}}%
\pgfpathcurveto{\pgfqpoint{0.712877in}{0.796210in}}{\pgfqpoint{0.703032in}{0.806055in}}{\pgfqpoint{0.697500in}{0.819410in}}%
\pgfpathcurveto{\pgfqpoint{0.697500in}{0.833333in}}{\pgfqpoint{0.697500in}{0.847256in}}{\pgfqpoint{0.703032in}{0.860611in}}%
\pgfpathcurveto{\pgfqpoint{0.712877in}{0.870456in}}{\pgfqpoint{0.722722in}{0.880302in}}{\pgfqpoint{0.736077in}{0.885833in}}%
\pgfpathcurveto{\pgfqpoint{0.750000in}{0.885833in}}{\pgfqpoint{0.763923in}{0.885833in}}{\pgfqpoint{0.777278in}{0.880302in}}%
\pgfpathcurveto{\pgfqpoint{0.787123in}{0.870456in}}{\pgfqpoint{0.796968in}{0.860611in}}{\pgfqpoint{0.802500in}{0.847256in}}%
\pgfpathcurveto{\pgfqpoint{0.802500in}{0.833333in}}{\pgfqpoint{0.802500in}{0.819410in}}{\pgfqpoint{0.796968in}{0.806055in}}%
\pgfpathcurveto{\pgfqpoint{0.787123in}{0.796210in}}{\pgfqpoint{0.777278in}{0.786365in}}{\pgfqpoint{0.763923in}{0.780833in}}%
\pgfpathclose%
\pgfpathmoveto{\pgfqpoint{0.916667in}{0.775000in}}%
\pgfpathcurveto{\pgfqpoint{0.932137in}{0.775000in}}{\pgfqpoint{0.946975in}{0.781146in}}{\pgfqpoint{0.957915in}{0.792085in}}%
\pgfpathcurveto{\pgfqpoint{0.968854in}{0.803025in}}{\pgfqpoint{0.975000in}{0.817863in}}{\pgfqpoint{0.975000in}{0.833333in}}%
\pgfpathcurveto{\pgfqpoint{0.975000in}{0.848804in}}{\pgfqpoint{0.968854in}{0.863642in}}{\pgfqpoint{0.957915in}{0.874581in}}%
\pgfpathcurveto{\pgfqpoint{0.946975in}{0.885520in}}{\pgfqpoint{0.932137in}{0.891667in}}{\pgfqpoint{0.916667in}{0.891667in}}%
\pgfpathcurveto{\pgfqpoint{0.901196in}{0.891667in}}{\pgfqpoint{0.886358in}{0.885520in}}{\pgfqpoint{0.875419in}{0.874581in}}%
\pgfpathcurveto{\pgfqpoint{0.864480in}{0.863642in}}{\pgfqpoint{0.858333in}{0.848804in}}{\pgfqpoint{0.858333in}{0.833333in}}%
\pgfpathcurveto{\pgfqpoint{0.858333in}{0.817863in}}{\pgfqpoint{0.864480in}{0.803025in}}{\pgfqpoint{0.875419in}{0.792085in}}%
\pgfpathcurveto{\pgfqpoint{0.886358in}{0.781146in}}{\pgfqpoint{0.901196in}{0.775000in}}{\pgfqpoint{0.916667in}{0.775000in}}%
\pgfpathclose%
\pgfpathmoveto{\pgfqpoint{0.916667in}{0.780833in}}%
\pgfpathcurveto{\pgfqpoint{0.916667in}{0.780833in}}{\pgfqpoint{0.902744in}{0.780833in}}{\pgfqpoint{0.889389in}{0.786365in}}%
\pgfpathcurveto{\pgfqpoint{0.879544in}{0.796210in}}{\pgfqpoint{0.869698in}{0.806055in}}{\pgfqpoint{0.864167in}{0.819410in}}%
\pgfpathcurveto{\pgfqpoint{0.864167in}{0.833333in}}{\pgfqpoint{0.864167in}{0.847256in}}{\pgfqpoint{0.869698in}{0.860611in}}%
\pgfpathcurveto{\pgfqpoint{0.879544in}{0.870456in}}{\pgfqpoint{0.889389in}{0.880302in}}{\pgfqpoint{0.902744in}{0.885833in}}%
\pgfpathcurveto{\pgfqpoint{0.916667in}{0.885833in}}{\pgfqpoint{0.930590in}{0.885833in}}{\pgfqpoint{0.943945in}{0.880302in}}%
\pgfpathcurveto{\pgfqpoint{0.953790in}{0.870456in}}{\pgfqpoint{0.963635in}{0.860611in}}{\pgfqpoint{0.969167in}{0.847256in}}%
\pgfpathcurveto{\pgfqpoint{0.969167in}{0.833333in}}{\pgfqpoint{0.969167in}{0.819410in}}{\pgfqpoint{0.963635in}{0.806055in}}%
\pgfpathcurveto{\pgfqpoint{0.953790in}{0.796210in}}{\pgfqpoint{0.943945in}{0.786365in}}{\pgfqpoint{0.930590in}{0.780833in}}%
\pgfpathclose%
\pgfpathmoveto{\pgfqpoint{0.000000in}{0.941667in}}%
\pgfpathcurveto{\pgfqpoint{0.015470in}{0.941667in}}{\pgfqpoint{0.030309in}{0.947813in}}{\pgfqpoint{0.041248in}{0.958752in}}%
\pgfpathcurveto{\pgfqpoint{0.052187in}{0.969691in}}{\pgfqpoint{0.058333in}{0.984530in}}{\pgfqpoint{0.058333in}{1.000000in}}%
\pgfpathcurveto{\pgfqpoint{0.058333in}{1.015470in}}{\pgfqpoint{0.052187in}{1.030309in}}{\pgfqpoint{0.041248in}{1.041248in}}%
\pgfpathcurveto{\pgfqpoint{0.030309in}{1.052187in}}{\pgfqpoint{0.015470in}{1.058333in}}{\pgfqpoint{0.000000in}{1.058333in}}%
\pgfpathcurveto{\pgfqpoint{-0.015470in}{1.058333in}}{\pgfqpoint{-0.030309in}{1.052187in}}{\pgfqpoint{-0.041248in}{1.041248in}}%
\pgfpathcurveto{\pgfqpoint{-0.052187in}{1.030309in}}{\pgfqpoint{-0.058333in}{1.015470in}}{\pgfqpoint{-0.058333in}{1.000000in}}%
\pgfpathcurveto{\pgfqpoint{-0.058333in}{0.984530in}}{\pgfqpoint{-0.052187in}{0.969691in}}{\pgfqpoint{-0.041248in}{0.958752in}}%
\pgfpathcurveto{\pgfqpoint{-0.030309in}{0.947813in}}{\pgfqpoint{-0.015470in}{0.941667in}}{\pgfqpoint{0.000000in}{0.941667in}}%
\pgfpathclose%
\pgfpathmoveto{\pgfqpoint{0.000000in}{0.947500in}}%
\pgfpathcurveto{\pgfqpoint{0.000000in}{0.947500in}}{\pgfqpoint{-0.013923in}{0.947500in}}{\pgfqpoint{-0.027278in}{0.953032in}}%
\pgfpathcurveto{\pgfqpoint{-0.037123in}{0.962877in}}{\pgfqpoint{-0.046968in}{0.972722in}}{\pgfqpoint{-0.052500in}{0.986077in}}%
\pgfpathcurveto{\pgfqpoint{-0.052500in}{1.000000in}}{\pgfqpoint{-0.052500in}{1.013923in}}{\pgfqpoint{-0.046968in}{1.027278in}}%
\pgfpathcurveto{\pgfqpoint{-0.037123in}{1.037123in}}{\pgfqpoint{-0.027278in}{1.046968in}}{\pgfqpoint{-0.013923in}{1.052500in}}%
\pgfpathcurveto{\pgfqpoint{0.000000in}{1.052500in}}{\pgfqpoint{0.013923in}{1.052500in}}{\pgfqpoint{0.027278in}{1.046968in}}%
\pgfpathcurveto{\pgfqpoint{0.037123in}{1.037123in}}{\pgfqpoint{0.046968in}{1.027278in}}{\pgfqpoint{0.052500in}{1.013923in}}%
\pgfpathcurveto{\pgfqpoint{0.052500in}{1.000000in}}{\pgfqpoint{0.052500in}{0.986077in}}{\pgfqpoint{0.046968in}{0.972722in}}%
\pgfpathcurveto{\pgfqpoint{0.037123in}{0.962877in}}{\pgfqpoint{0.027278in}{0.953032in}}{\pgfqpoint{0.013923in}{0.947500in}}%
\pgfpathclose%
\pgfpathmoveto{\pgfqpoint{0.166667in}{0.941667in}}%
\pgfpathcurveto{\pgfqpoint{0.182137in}{0.941667in}}{\pgfqpoint{0.196975in}{0.947813in}}{\pgfqpoint{0.207915in}{0.958752in}}%
\pgfpathcurveto{\pgfqpoint{0.218854in}{0.969691in}}{\pgfqpoint{0.225000in}{0.984530in}}{\pgfqpoint{0.225000in}{1.000000in}}%
\pgfpathcurveto{\pgfqpoint{0.225000in}{1.015470in}}{\pgfqpoint{0.218854in}{1.030309in}}{\pgfqpoint{0.207915in}{1.041248in}}%
\pgfpathcurveto{\pgfqpoint{0.196975in}{1.052187in}}{\pgfqpoint{0.182137in}{1.058333in}}{\pgfqpoint{0.166667in}{1.058333in}}%
\pgfpathcurveto{\pgfqpoint{0.151196in}{1.058333in}}{\pgfqpoint{0.136358in}{1.052187in}}{\pgfqpoint{0.125419in}{1.041248in}}%
\pgfpathcurveto{\pgfqpoint{0.114480in}{1.030309in}}{\pgfqpoint{0.108333in}{1.015470in}}{\pgfqpoint{0.108333in}{1.000000in}}%
\pgfpathcurveto{\pgfqpoint{0.108333in}{0.984530in}}{\pgfqpoint{0.114480in}{0.969691in}}{\pgfqpoint{0.125419in}{0.958752in}}%
\pgfpathcurveto{\pgfqpoint{0.136358in}{0.947813in}}{\pgfqpoint{0.151196in}{0.941667in}}{\pgfqpoint{0.166667in}{0.941667in}}%
\pgfpathclose%
\pgfpathmoveto{\pgfqpoint{0.166667in}{0.947500in}}%
\pgfpathcurveto{\pgfqpoint{0.166667in}{0.947500in}}{\pgfqpoint{0.152744in}{0.947500in}}{\pgfqpoint{0.139389in}{0.953032in}}%
\pgfpathcurveto{\pgfqpoint{0.129544in}{0.962877in}}{\pgfqpoint{0.119698in}{0.972722in}}{\pgfqpoint{0.114167in}{0.986077in}}%
\pgfpathcurveto{\pgfqpoint{0.114167in}{1.000000in}}{\pgfqpoint{0.114167in}{1.013923in}}{\pgfqpoint{0.119698in}{1.027278in}}%
\pgfpathcurveto{\pgfqpoint{0.129544in}{1.037123in}}{\pgfqpoint{0.139389in}{1.046968in}}{\pgfqpoint{0.152744in}{1.052500in}}%
\pgfpathcurveto{\pgfqpoint{0.166667in}{1.052500in}}{\pgfqpoint{0.180590in}{1.052500in}}{\pgfqpoint{0.193945in}{1.046968in}}%
\pgfpathcurveto{\pgfqpoint{0.203790in}{1.037123in}}{\pgfqpoint{0.213635in}{1.027278in}}{\pgfqpoint{0.219167in}{1.013923in}}%
\pgfpathcurveto{\pgfqpoint{0.219167in}{1.000000in}}{\pgfqpoint{0.219167in}{0.986077in}}{\pgfqpoint{0.213635in}{0.972722in}}%
\pgfpathcurveto{\pgfqpoint{0.203790in}{0.962877in}}{\pgfqpoint{0.193945in}{0.953032in}}{\pgfqpoint{0.180590in}{0.947500in}}%
\pgfpathclose%
\pgfpathmoveto{\pgfqpoint{0.333333in}{0.941667in}}%
\pgfpathcurveto{\pgfqpoint{0.348804in}{0.941667in}}{\pgfqpoint{0.363642in}{0.947813in}}{\pgfqpoint{0.374581in}{0.958752in}}%
\pgfpathcurveto{\pgfqpoint{0.385520in}{0.969691in}}{\pgfqpoint{0.391667in}{0.984530in}}{\pgfqpoint{0.391667in}{1.000000in}}%
\pgfpathcurveto{\pgfqpoint{0.391667in}{1.015470in}}{\pgfqpoint{0.385520in}{1.030309in}}{\pgfqpoint{0.374581in}{1.041248in}}%
\pgfpathcurveto{\pgfqpoint{0.363642in}{1.052187in}}{\pgfqpoint{0.348804in}{1.058333in}}{\pgfqpoint{0.333333in}{1.058333in}}%
\pgfpathcurveto{\pgfqpoint{0.317863in}{1.058333in}}{\pgfqpoint{0.303025in}{1.052187in}}{\pgfqpoint{0.292085in}{1.041248in}}%
\pgfpathcurveto{\pgfqpoint{0.281146in}{1.030309in}}{\pgfqpoint{0.275000in}{1.015470in}}{\pgfqpoint{0.275000in}{1.000000in}}%
\pgfpathcurveto{\pgfqpoint{0.275000in}{0.984530in}}{\pgfqpoint{0.281146in}{0.969691in}}{\pgfqpoint{0.292085in}{0.958752in}}%
\pgfpathcurveto{\pgfqpoint{0.303025in}{0.947813in}}{\pgfqpoint{0.317863in}{0.941667in}}{\pgfqpoint{0.333333in}{0.941667in}}%
\pgfpathclose%
\pgfpathmoveto{\pgfqpoint{0.333333in}{0.947500in}}%
\pgfpathcurveto{\pgfqpoint{0.333333in}{0.947500in}}{\pgfqpoint{0.319410in}{0.947500in}}{\pgfqpoint{0.306055in}{0.953032in}}%
\pgfpathcurveto{\pgfqpoint{0.296210in}{0.962877in}}{\pgfqpoint{0.286365in}{0.972722in}}{\pgfqpoint{0.280833in}{0.986077in}}%
\pgfpathcurveto{\pgfqpoint{0.280833in}{1.000000in}}{\pgfqpoint{0.280833in}{1.013923in}}{\pgfqpoint{0.286365in}{1.027278in}}%
\pgfpathcurveto{\pgfqpoint{0.296210in}{1.037123in}}{\pgfqpoint{0.306055in}{1.046968in}}{\pgfqpoint{0.319410in}{1.052500in}}%
\pgfpathcurveto{\pgfqpoint{0.333333in}{1.052500in}}{\pgfqpoint{0.347256in}{1.052500in}}{\pgfqpoint{0.360611in}{1.046968in}}%
\pgfpathcurveto{\pgfqpoint{0.370456in}{1.037123in}}{\pgfqpoint{0.380302in}{1.027278in}}{\pgfqpoint{0.385833in}{1.013923in}}%
\pgfpathcurveto{\pgfqpoint{0.385833in}{1.000000in}}{\pgfqpoint{0.385833in}{0.986077in}}{\pgfqpoint{0.380302in}{0.972722in}}%
\pgfpathcurveto{\pgfqpoint{0.370456in}{0.962877in}}{\pgfqpoint{0.360611in}{0.953032in}}{\pgfqpoint{0.347256in}{0.947500in}}%
\pgfpathclose%
\pgfpathmoveto{\pgfqpoint{0.500000in}{0.941667in}}%
\pgfpathcurveto{\pgfqpoint{0.515470in}{0.941667in}}{\pgfqpoint{0.530309in}{0.947813in}}{\pgfqpoint{0.541248in}{0.958752in}}%
\pgfpathcurveto{\pgfqpoint{0.552187in}{0.969691in}}{\pgfqpoint{0.558333in}{0.984530in}}{\pgfqpoint{0.558333in}{1.000000in}}%
\pgfpathcurveto{\pgfqpoint{0.558333in}{1.015470in}}{\pgfqpoint{0.552187in}{1.030309in}}{\pgfqpoint{0.541248in}{1.041248in}}%
\pgfpathcurveto{\pgfqpoint{0.530309in}{1.052187in}}{\pgfqpoint{0.515470in}{1.058333in}}{\pgfqpoint{0.500000in}{1.058333in}}%
\pgfpathcurveto{\pgfqpoint{0.484530in}{1.058333in}}{\pgfqpoint{0.469691in}{1.052187in}}{\pgfqpoint{0.458752in}{1.041248in}}%
\pgfpathcurveto{\pgfqpoint{0.447813in}{1.030309in}}{\pgfqpoint{0.441667in}{1.015470in}}{\pgfqpoint{0.441667in}{1.000000in}}%
\pgfpathcurveto{\pgfqpoint{0.441667in}{0.984530in}}{\pgfqpoint{0.447813in}{0.969691in}}{\pgfqpoint{0.458752in}{0.958752in}}%
\pgfpathcurveto{\pgfqpoint{0.469691in}{0.947813in}}{\pgfqpoint{0.484530in}{0.941667in}}{\pgfqpoint{0.500000in}{0.941667in}}%
\pgfpathclose%
\pgfpathmoveto{\pgfqpoint{0.500000in}{0.947500in}}%
\pgfpathcurveto{\pgfqpoint{0.500000in}{0.947500in}}{\pgfqpoint{0.486077in}{0.947500in}}{\pgfqpoint{0.472722in}{0.953032in}}%
\pgfpathcurveto{\pgfqpoint{0.462877in}{0.962877in}}{\pgfqpoint{0.453032in}{0.972722in}}{\pgfqpoint{0.447500in}{0.986077in}}%
\pgfpathcurveto{\pgfqpoint{0.447500in}{1.000000in}}{\pgfqpoint{0.447500in}{1.013923in}}{\pgfqpoint{0.453032in}{1.027278in}}%
\pgfpathcurveto{\pgfqpoint{0.462877in}{1.037123in}}{\pgfqpoint{0.472722in}{1.046968in}}{\pgfqpoint{0.486077in}{1.052500in}}%
\pgfpathcurveto{\pgfqpoint{0.500000in}{1.052500in}}{\pgfqpoint{0.513923in}{1.052500in}}{\pgfqpoint{0.527278in}{1.046968in}}%
\pgfpathcurveto{\pgfqpoint{0.537123in}{1.037123in}}{\pgfqpoint{0.546968in}{1.027278in}}{\pgfqpoint{0.552500in}{1.013923in}}%
\pgfpathcurveto{\pgfqpoint{0.552500in}{1.000000in}}{\pgfqpoint{0.552500in}{0.986077in}}{\pgfqpoint{0.546968in}{0.972722in}}%
\pgfpathcurveto{\pgfqpoint{0.537123in}{0.962877in}}{\pgfqpoint{0.527278in}{0.953032in}}{\pgfqpoint{0.513923in}{0.947500in}}%
\pgfpathclose%
\pgfpathmoveto{\pgfqpoint{0.666667in}{0.941667in}}%
\pgfpathcurveto{\pgfqpoint{0.682137in}{0.941667in}}{\pgfqpoint{0.696975in}{0.947813in}}{\pgfqpoint{0.707915in}{0.958752in}}%
\pgfpathcurveto{\pgfqpoint{0.718854in}{0.969691in}}{\pgfqpoint{0.725000in}{0.984530in}}{\pgfqpoint{0.725000in}{1.000000in}}%
\pgfpathcurveto{\pgfqpoint{0.725000in}{1.015470in}}{\pgfqpoint{0.718854in}{1.030309in}}{\pgfqpoint{0.707915in}{1.041248in}}%
\pgfpathcurveto{\pgfqpoint{0.696975in}{1.052187in}}{\pgfqpoint{0.682137in}{1.058333in}}{\pgfqpoint{0.666667in}{1.058333in}}%
\pgfpathcurveto{\pgfqpoint{0.651196in}{1.058333in}}{\pgfqpoint{0.636358in}{1.052187in}}{\pgfqpoint{0.625419in}{1.041248in}}%
\pgfpathcurveto{\pgfqpoint{0.614480in}{1.030309in}}{\pgfqpoint{0.608333in}{1.015470in}}{\pgfqpoint{0.608333in}{1.000000in}}%
\pgfpathcurveto{\pgfqpoint{0.608333in}{0.984530in}}{\pgfqpoint{0.614480in}{0.969691in}}{\pgfqpoint{0.625419in}{0.958752in}}%
\pgfpathcurveto{\pgfqpoint{0.636358in}{0.947813in}}{\pgfqpoint{0.651196in}{0.941667in}}{\pgfqpoint{0.666667in}{0.941667in}}%
\pgfpathclose%
\pgfpathmoveto{\pgfqpoint{0.666667in}{0.947500in}}%
\pgfpathcurveto{\pgfqpoint{0.666667in}{0.947500in}}{\pgfqpoint{0.652744in}{0.947500in}}{\pgfqpoint{0.639389in}{0.953032in}}%
\pgfpathcurveto{\pgfqpoint{0.629544in}{0.962877in}}{\pgfqpoint{0.619698in}{0.972722in}}{\pgfqpoint{0.614167in}{0.986077in}}%
\pgfpathcurveto{\pgfqpoint{0.614167in}{1.000000in}}{\pgfqpoint{0.614167in}{1.013923in}}{\pgfqpoint{0.619698in}{1.027278in}}%
\pgfpathcurveto{\pgfqpoint{0.629544in}{1.037123in}}{\pgfqpoint{0.639389in}{1.046968in}}{\pgfqpoint{0.652744in}{1.052500in}}%
\pgfpathcurveto{\pgfqpoint{0.666667in}{1.052500in}}{\pgfqpoint{0.680590in}{1.052500in}}{\pgfqpoint{0.693945in}{1.046968in}}%
\pgfpathcurveto{\pgfqpoint{0.703790in}{1.037123in}}{\pgfqpoint{0.713635in}{1.027278in}}{\pgfqpoint{0.719167in}{1.013923in}}%
\pgfpathcurveto{\pgfqpoint{0.719167in}{1.000000in}}{\pgfqpoint{0.719167in}{0.986077in}}{\pgfqpoint{0.713635in}{0.972722in}}%
\pgfpathcurveto{\pgfqpoint{0.703790in}{0.962877in}}{\pgfqpoint{0.693945in}{0.953032in}}{\pgfqpoint{0.680590in}{0.947500in}}%
\pgfpathclose%
\pgfpathmoveto{\pgfqpoint{0.833333in}{0.941667in}}%
\pgfpathcurveto{\pgfqpoint{0.848804in}{0.941667in}}{\pgfqpoint{0.863642in}{0.947813in}}{\pgfqpoint{0.874581in}{0.958752in}}%
\pgfpathcurveto{\pgfqpoint{0.885520in}{0.969691in}}{\pgfqpoint{0.891667in}{0.984530in}}{\pgfqpoint{0.891667in}{1.000000in}}%
\pgfpathcurveto{\pgfqpoint{0.891667in}{1.015470in}}{\pgfqpoint{0.885520in}{1.030309in}}{\pgfqpoint{0.874581in}{1.041248in}}%
\pgfpathcurveto{\pgfqpoint{0.863642in}{1.052187in}}{\pgfqpoint{0.848804in}{1.058333in}}{\pgfqpoint{0.833333in}{1.058333in}}%
\pgfpathcurveto{\pgfqpoint{0.817863in}{1.058333in}}{\pgfqpoint{0.803025in}{1.052187in}}{\pgfqpoint{0.792085in}{1.041248in}}%
\pgfpathcurveto{\pgfqpoint{0.781146in}{1.030309in}}{\pgfqpoint{0.775000in}{1.015470in}}{\pgfqpoint{0.775000in}{1.000000in}}%
\pgfpathcurveto{\pgfqpoint{0.775000in}{0.984530in}}{\pgfqpoint{0.781146in}{0.969691in}}{\pgfqpoint{0.792085in}{0.958752in}}%
\pgfpathcurveto{\pgfqpoint{0.803025in}{0.947813in}}{\pgfqpoint{0.817863in}{0.941667in}}{\pgfqpoint{0.833333in}{0.941667in}}%
\pgfpathclose%
\pgfpathmoveto{\pgfqpoint{0.833333in}{0.947500in}}%
\pgfpathcurveto{\pgfqpoint{0.833333in}{0.947500in}}{\pgfqpoint{0.819410in}{0.947500in}}{\pgfqpoint{0.806055in}{0.953032in}}%
\pgfpathcurveto{\pgfqpoint{0.796210in}{0.962877in}}{\pgfqpoint{0.786365in}{0.972722in}}{\pgfqpoint{0.780833in}{0.986077in}}%
\pgfpathcurveto{\pgfqpoint{0.780833in}{1.000000in}}{\pgfqpoint{0.780833in}{1.013923in}}{\pgfqpoint{0.786365in}{1.027278in}}%
\pgfpathcurveto{\pgfqpoint{0.796210in}{1.037123in}}{\pgfqpoint{0.806055in}{1.046968in}}{\pgfqpoint{0.819410in}{1.052500in}}%
\pgfpathcurveto{\pgfqpoint{0.833333in}{1.052500in}}{\pgfqpoint{0.847256in}{1.052500in}}{\pgfqpoint{0.860611in}{1.046968in}}%
\pgfpathcurveto{\pgfqpoint{0.870456in}{1.037123in}}{\pgfqpoint{0.880302in}{1.027278in}}{\pgfqpoint{0.885833in}{1.013923in}}%
\pgfpathcurveto{\pgfqpoint{0.885833in}{1.000000in}}{\pgfqpoint{0.885833in}{0.986077in}}{\pgfqpoint{0.880302in}{0.972722in}}%
\pgfpathcurveto{\pgfqpoint{0.870456in}{0.962877in}}{\pgfqpoint{0.860611in}{0.953032in}}{\pgfqpoint{0.847256in}{0.947500in}}%
\pgfpathclose%
\pgfpathmoveto{\pgfqpoint{1.000000in}{0.941667in}}%
\pgfpathcurveto{\pgfqpoint{1.015470in}{0.941667in}}{\pgfqpoint{1.030309in}{0.947813in}}{\pgfqpoint{1.041248in}{0.958752in}}%
\pgfpathcurveto{\pgfqpoint{1.052187in}{0.969691in}}{\pgfqpoint{1.058333in}{0.984530in}}{\pgfqpoint{1.058333in}{1.000000in}}%
\pgfpathcurveto{\pgfqpoint{1.058333in}{1.015470in}}{\pgfqpoint{1.052187in}{1.030309in}}{\pgfqpoint{1.041248in}{1.041248in}}%
\pgfpathcurveto{\pgfqpoint{1.030309in}{1.052187in}}{\pgfqpoint{1.015470in}{1.058333in}}{\pgfqpoint{1.000000in}{1.058333in}}%
\pgfpathcurveto{\pgfqpoint{0.984530in}{1.058333in}}{\pgfqpoint{0.969691in}{1.052187in}}{\pgfqpoint{0.958752in}{1.041248in}}%
\pgfpathcurveto{\pgfqpoint{0.947813in}{1.030309in}}{\pgfqpoint{0.941667in}{1.015470in}}{\pgfqpoint{0.941667in}{1.000000in}}%
\pgfpathcurveto{\pgfqpoint{0.941667in}{0.984530in}}{\pgfqpoint{0.947813in}{0.969691in}}{\pgfqpoint{0.958752in}{0.958752in}}%
\pgfpathcurveto{\pgfqpoint{0.969691in}{0.947813in}}{\pgfqpoint{0.984530in}{0.941667in}}{\pgfqpoint{1.000000in}{0.941667in}}%
\pgfpathclose%
\pgfpathmoveto{\pgfqpoint{1.000000in}{0.947500in}}%
\pgfpathcurveto{\pgfqpoint{1.000000in}{0.947500in}}{\pgfqpoint{0.986077in}{0.947500in}}{\pgfqpoint{0.972722in}{0.953032in}}%
\pgfpathcurveto{\pgfqpoint{0.962877in}{0.962877in}}{\pgfqpoint{0.953032in}{0.972722in}}{\pgfqpoint{0.947500in}{0.986077in}}%
\pgfpathcurveto{\pgfqpoint{0.947500in}{1.000000in}}{\pgfqpoint{0.947500in}{1.013923in}}{\pgfqpoint{0.953032in}{1.027278in}}%
\pgfpathcurveto{\pgfqpoint{0.962877in}{1.037123in}}{\pgfqpoint{0.972722in}{1.046968in}}{\pgfqpoint{0.986077in}{1.052500in}}%
\pgfpathcurveto{\pgfqpoint{1.000000in}{1.052500in}}{\pgfqpoint{1.013923in}{1.052500in}}{\pgfqpoint{1.027278in}{1.046968in}}%
\pgfpathcurveto{\pgfqpoint{1.037123in}{1.037123in}}{\pgfqpoint{1.046968in}{1.027278in}}{\pgfqpoint{1.052500in}{1.013923in}}%
\pgfpathcurveto{\pgfqpoint{1.052500in}{1.000000in}}{\pgfqpoint{1.052500in}{0.986077in}}{\pgfqpoint{1.046968in}{0.972722in}}%
\pgfpathcurveto{\pgfqpoint{1.037123in}{0.962877in}}{\pgfqpoint{1.027278in}{0.953032in}}{\pgfqpoint{1.013923in}{0.947500in}}%
\pgfpathclose%
\pgfusepath{stroke}%
\end{pgfscope}%
}%
\pgfsys@transformshift{4.423315in}{2.758071in}%
\pgfsys@useobject{currentpattern}{}%
\pgfsys@transformshift{1in}{0in}%
\pgfsys@transformshift{-1in}{0in}%
\pgfsys@transformshift{0in}{1in}%
\pgfsys@useobject{currentpattern}{}%
\pgfsys@transformshift{1in}{0in}%
\pgfsys@transformshift{-1in}{0in}%
\pgfsys@transformshift{0in}{1in}%
\end{pgfscope}%
\begin{pgfscope}%
\pgfpathrectangle{\pgfqpoint{0.935815in}{0.637495in}}{\pgfqpoint{9.300000in}{9.060000in}}%
\pgfusepath{clip}%
\pgfsetbuttcap%
\pgfsetmiterjoin%
\definecolor{currentfill}{rgb}{0.549020,0.337255,0.294118}%
\pgfsetfillcolor{currentfill}%
\pgfsetfillopacity{0.990000}%
\pgfsetlinewidth{0.000000pt}%
\definecolor{currentstroke}{rgb}{0.000000,0.000000,0.000000}%
\pgfsetstrokecolor{currentstroke}%
\pgfsetstrokeopacity{0.990000}%
\pgfsetdash{}{0pt}%
\pgfpathmoveto{\pgfqpoint{5.973315in}{2.878536in}}%
\pgfpathlineto{\pgfqpoint{6.748315in}{2.878536in}}%
\pgfpathlineto{\pgfqpoint{6.748315in}{4.921635in}}%
\pgfpathlineto{\pgfqpoint{5.973315in}{4.921635in}}%
\pgfpathclose%
\pgfusepath{fill}%
\end{pgfscope}%
\begin{pgfscope}%
\pgfsetbuttcap%
\pgfsetmiterjoin%
\definecolor{currentfill}{rgb}{0.549020,0.337255,0.294118}%
\pgfsetfillcolor{currentfill}%
\pgfsetfillopacity{0.990000}%
\pgfsetlinewidth{0.000000pt}%
\definecolor{currentstroke}{rgb}{0.000000,0.000000,0.000000}%
\pgfsetstrokecolor{currentstroke}%
\pgfsetstrokeopacity{0.990000}%
\pgfsetdash{}{0pt}%
\pgfpathrectangle{\pgfqpoint{0.935815in}{0.637495in}}{\pgfqpoint{9.300000in}{9.060000in}}%
\pgfusepath{clip}%
\pgfpathmoveto{\pgfqpoint{5.973315in}{2.878536in}}%
\pgfpathlineto{\pgfqpoint{6.748315in}{2.878536in}}%
\pgfpathlineto{\pgfqpoint{6.748315in}{4.921635in}}%
\pgfpathlineto{\pgfqpoint{5.973315in}{4.921635in}}%
\pgfpathclose%
\pgfusepath{clip}%
\pgfsys@defobject{currentpattern}{\pgfqpoint{0in}{0in}}{\pgfqpoint{1in}{1in}}{%
\begin{pgfscope}%
\pgfpathrectangle{\pgfqpoint{0in}{0in}}{\pgfqpoint{1in}{1in}}%
\pgfusepath{clip}%
\pgfpathmoveto{\pgfqpoint{0.000000in}{-0.058333in}}%
\pgfpathcurveto{\pgfqpoint{0.015470in}{-0.058333in}}{\pgfqpoint{0.030309in}{-0.052187in}}{\pgfqpoint{0.041248in}{-0.041248in}}%
\pgfpathcurveto{\pgfqpoint{0.052187in}{-0.030309in}}{\pgfqpoint{0.058333in}{-0.015470in}}{\pgfqpoint{0.058333in}{0.000000in}}%
\pgfpathcurveto{\pgfqpoint{0.058333in}{0.015470in}}{\pgfqpoint{0.052187in}{0.030309in}}{\pgfqpoint{0.041248in}{0.041248in}}%
\pgfpathcurveto{\pgfqpoint{0.030309in}{0.052187in}}{\pgfqpoint{0.015470in}{0.058333in}}{\pgfqpoint{0.000000in}{0.058333in}}%
\pgfpathcurveto{\pgfqpoint{-0.015470in}{0.058333in}}{\pgfqpoint{-0.030309in}{0.052187in}}{\pgfqpoint{-0.041248in}{0.041248in}}%
\pgfpathcurveto{\pgfqpoint{-0.052187in}{0.030309in}}{\pgfqpoint{-0.058333in}{0.015470in}}{\pgfqpoint{-0.058333in}{0.000000in}}%
\pgfpathcurveto{\pgfqpoint{-0.058333in}{-0.015470in}}{\pgfqpoint{-0.052187in}{-0.030309in}}{\pgfqpoint{-0.041248in}{-0.041248in}}%
\pgfpathcurveto{\pgfqpoint{-0.030309in}{-0.052187in}}{\pgfqpoint{-0.015470in}{-0.058333in}}{\pgfqpoint{0.000000in}{-0.058333in}}%
\pgfpathclose%
\pgfpathmoveto{\pgfqpoint{0.000000in}{-0.052500in}}%
\pgfpathcurveto{\pgfqpoint{0.000000in}{-0.052500in}}{\pgfqpoint{-0.013923in}{-0.052500in}}{\pgfqpoint{-0.027278in}{-0.046968in}}%
\pgfpathcurveto{\pgfqpoint{-0.037123in}{-0.037123in}}{\pgfqpoint{-0.046968in}{-0.027278in}}{\pgfqpoint{-0.052500in}{-0.013923in}}%
\pgfpathcurveto{\pgfqpoint{-0.052500in}{0.000000in}}{\pgfqpoint{-0.052500in}{0.013923in}}{\pgfqpoint{-0.046968in}{0.027278in}}%
\pgfpathcurveto{\pgfqpoint{-0.037123in}{0.037123in}}{\pgfqpoint{-0.027278in}{0.046968in}}{\pgfqpoint{-0.013923in}{0.052500in}}%
\pgfpathcurveto{\pgfqpoint{0.000000in}{0.052500in}}{\pgfqpoint{0.013923in}{0.052500in}}{\pgfqpoint{0.027278in}{0.046968in}}%
\pgfpathcurveto{\pgfqpoint{0.037123in}{0.037123in}}{\pgfqpoint{0.046968in}{0.027278in}}{\pgfqpoint{0.052500in}{0.013923in}}%
\pgfpathcurveto{\pgfqpoint{0.052500in}{0.000000in}}{\pgfqpoint{0.052500in}{-0.013923in}}{\pgfqpoint{0.046968in}{-0.027278in}}%
\pgfpathcurveto{\pgfqpoint{0.037123in}{-0.037123in}}{\pgfqpoint{0.027278in}{-0.046968in}}{\pgfqpoint{0.013923in}{-0.052500in}}%
\pgfpathclose%
\pgfpathmoveto{\pgfqpoint{0.166667in}{-0.058333in}}%
\pgfpathcurveto{\pgfqpoint{0.182137in}{-0.058333in}}{\pgfqpoint{0.196975in}{-0.052187in}}{\pgfqpoint{0.207915in}{-0.041248in}}%
\pgfpathcurveto{\pgfqpoint{0.218854in}{-0.030309in}}{\pgfqpoint{0.225000in}{-0.015470in}}{\pgfqpoint{0.225000in}{0.000000in}}%
\pgfpathcurveto{\pgfqpoint{0.225000in}{0.015470in}}{\pgfqpoint{0.218854in}{0.030309in}}{\pgfqpoint{0.207915in}{0.041248in}}%
\pgfpathcurveto{\pgfqpoint{0.196975in}{0.052187in}}{\pgfqpoint{0.182137in}{0.058333in}}{\pgfqpoint{0.166667in}{0.058333in}}%
\pgfpathcurveto{\pgfqpoint{0.151196in}{0.058333in}}{\pgfqpoint{0.136358in}{0.052187in}}{\pgfqpoint{0.125419in}{0.041248in}}%
\pgfpathcurveto{\pgfqpoint{0.114480in}{0.030309in}}{\pgfqpoint{0.108333in}{0.015470in}}{\pgfqpoint{0.108333in}{0.000000in}}%
\pgfpathcurveto{\pgfqpoint{0.108333in}{-0.015470in}}{\pgfqpoint{0.114480in}{-0.030309in}}{\pgfqpoint{0.125419in}{-0.041248in}}%
\pgfpathcurveto{\pgfqpoint{0.136358in}{-0.052187in}}{\pgfqpoint{0.151196in}{-0.058333in}}{\pgfqpoint{0.166667in}{-0.058333in}}%
\pgfpathclose%
\pgfpathmoveto{\pgfqpoint{0.166667in}{-0.052500in}}%
\pgfpathcurveto{\pgfqpoint{0.166667in}{-0.052500in}}{\pgfqpoint{0.152744in}{-0.052500in}}{\pgfqpoint{0.139389in}{-0.046968in}}%
\pgfpathcurveto{\pgfqpoint{0.129544in}{-0.037123in}}{\pgfqpoint{0.119698in}{-0.027278in}}{\pgfqpoint{0.114167in}{-0.013923in}}%
\pgfpathcurveto{\pgfqpoint{0.114167in}{0.000000in}}{\pgfqpoint{0.114167in}{0.013923in}}{\pgfqpoint{0.119698in}{0.027278in}}%
\pgfpathcurveto{\pgfqpoint{0.129544in}{0.037123in}}{\pgfqpoint{0.139389in}{0.046968in}}{\pgfqpoint{0.152744in}{0.052500in}}%
\pgfpathcurveto{\pgfqpoint{0.166667in}{0.052500in}}{\pgfqpoint{0.180590in}{0.052500in}}{\pgfqpoint{0.193945in}{0.046968in}}%
\pgfpathcurveto{\pgfqpoint{0.203790in}{0.037123in}}{\pgfqpoint{0.213635in}{0.027278in}}{\pgfqpoint{0.219167in}{0.013923in}}%
\pgfpathcurveto{\pgfqpoint{0.219167in}{0.000000in}}{\pgfqpoint{0.219167in}{-0.013923in}}{\pgfqpoint{0.213635in}{-0.027278in}}%
\pgfpathcurveto{\pgfqpoint{0.203790in}{-0.037123in}}{\pgfqpoint{0.193945in}{-0.046968in}}{\pgfqpoint{0.180590in}{-0.052500in}}%
\pgfpathclose%
\pgfpathmoveto{\pgfqpoint{0.333333in}{-0.058333in}}%
\pgfpathcurveto{\pgfqpoint{0.348804in}{-0.058333in}}{\pgfqpoint{0.363642in}{-0.052187in}}{\pgfqpoint{0.374581in}{-0.041248in}}%
\pgfpathcurveto{\pgfqpoint{0.385520in}{-0.030309in}}{\pgfqpoint{0.391667in}{-0.015470in}}{\pgfqpoint{0.391667in}{0.000000in}}%
\pgfpathcurveto{\pgfqpoint{0.391667in}{0.015470in}}{\pgfqpoint{0.385520in}{0.030309in}}{\pgfqpoint{0.374581in}{0.041248in}}%
\pgfpathcurveto{\pgfqpoint{0.363642in}{0.052187in}}{\pgfqpoint{0.348804in}{0.058333in}}{\pgfqpoint{0.333333in}{0.058333in}}%
\pgfpathcurveto{\pgfqpoint{0.317863in}{0.058333in}}{\pgfqpoint{0.303025in}{0.052187in}}{\pgfqpoint{0.292085in}{0.041248in}}%
\pgfpathcurveto{\pgfqpoint{0.281146in}{0.030309in}}{\pgfqpoint{0.275000in}{0.015470in}}{\pgfqpoint{0.275000in}{0.000000in}}%
\pgfpathcurveto{\pgfqpoint{0.275000in}{-0.015470in}}{\pgfqpoint{0.281146in}{-0.030309in}}{\pgfqpoint{0.292085in}{-0.041248in}}%
\pgfpathcurveto{\pgfqpoint{0.303025in}{-0.052187in}}{\pgfqpoint{0.317863in}{-0.058333in}}{\pgfqpoint{0.333333in}{-0.058333in}}%
\pgfpathclose%
\pgfpathmoveto{\pgfqpoint{0.333333in}{-0.052500in}}%
\pgfpathcurveto{\pgfqpoint{0.333333in}{-0.052500in}}{\pgfqpoint{0.319410in}{-0.052500in}}{\pgfqpoint{0.306055in}{-0.046968in}}%
\pgfpathcurveto{\pgfqpoint{0.296210in}{-0.037123in}}{\pgfqpoint{0.286365in}{-0.027278in}}{\pgfqpoint{0.280833in}{-0.013923in}}%
\pgfpathcurveto{\pgfqpoint{0.280833in}{0.000000in}}{\pgfqpoint{0.280833in}{0.013923in}}{\pgfqpoint{0.286365in}{0.027278in}}%
\pgfpathcurveto{\pgfqpoint{0.296210in}{0.037123in}}{\pgfqpoint{0.306055in}{0.046968in}}{\pgfqpoint{0.319410in}{0.052500in}}%
\pgfpathcurveto{\pgfqpoint{0.333333in}{0.052500in}}{\pgfqpoint{0.347256in}{0.052500in}}{\pgfqpoint{0.360611in}{0.046968in}}%
\pgfpathcurveto{\pgfqpoint{0.370456in}{0.037123in}}{\pgfqpoint{0.380302in}{0.027278in}}{\pgfqpoint{0.385833in}{0.013923in}}%
\pgfpathcurveto{\pgfqpoint{0.385833in}{0.000000in}}{\pgfqpoint{0.385833in}{-0.013923in}}{\pgfqpoint{0.380302in}{-0.027278in}}%
\pgfpathcurveto{\pgfqpoint{0.370456in}{-0.037123in}}{\pgfqpoint{0.360611in}{-0.046968in}}{\pgfqpoint{0.347256in}{-0.052500in}}%
\pgfpathclose%
\pgfpathmoveto{\pgfqpoint{0.500000in}{-0.058333in}}%
\pgfpathcurveto{\pgfqpoint{0.515470in}{-0.058333in}}{\pgfqpoint{0.530309in}{-0.052187in}}{\pgfqpoint{0.541248in}{-0.041248in}}%
\pgfpathcurveto{\pgfqpoint{0.552187in}{-0.030309in}}{\pgfqpoint{0.558333in}{-0.015470in}}{\pgfqpoint{0.558333in}{0.000000in}}%
\pgfpathcurveto{\pgfqpoint{0.558333in}{0.015470in}}{\pgfqpoint{0.552187in}{0.030309in}}{\pgfqpoint{0.541248in}{0.041248in}}%
\pgfpathcurveto{\pgfqpoint{0.530309in}{0.052187in}}{\pgfqpoint{0.515470in}{0.058333in}}{\pgfqpoint{0.500000in}{0.058333in}}%
\pgfpathcurveto{\pgfqpoint{0.484530in}{0.058333in}}{\pgfqpoint{0.469691in}{0.052187in}}{\pgfqpoint{0.458752in}{0.041248in}}%
\pgfpathcurveto{\pgfqpoint{0.447813in}{0.030309in}}{\pgfqpoint{0.441667in}{0.015470in}}{\pgfqpoint{0.441667in}{0.000000in}}%
\pgfpathcurveto{\pgfqpoint{0.441667in}{-0.015470in}}{\pgfqpoint{0.447813in}{-0.030309in}}{\pgfqpoint{0.458752in}{-0.041248in}}%
\pgfpathcurveto{\pgfqpoint{0.469691in}{-0.052187in}}{\pgfqpoint{0.484530in}{-0.058333in}}{\pgfqpoint{0.500000in}{-0.058333in}}%
\pgfpathclose%
\pgfpathmoveto{\pgfqpoint{0.500000in}{-0.052500in}}%
\pgfpathcurveto{\pgfqpoint{0.500000in}{-0.052500in}}{\pgfqpoint{0.486077in}{-0.052500in}}{\pgfqpoint{0.472722in}{-0.046968in}}%
\pgfpathcurveto{\pgfqpoint{0.462877in}{-0.037123in}}{\pgfqpoint{0.453032in}{-0.027278in}}{\pgfqpoint{0.447500in}{-0.013923in}}%
\pgfpathcurveto{\pgfqpoint{0.447500in}{0.000000in}}{\pgfqpoint{0.447500in}{0.013923in}}{\pgfqpoint{0.453032in}{0.027278in}}%
\pgfpathcurveto{\pgfqpoint{0.462877in}{0.037123in}}{\pgfqpoint{0.472722in}{0.046968in}}{\pgfqpoint{0.486077in}{0.052500in}}%
\pgfpathcurveto{\pgfqpoint{0.500000in}{0.052500in}}{\pgfqpoint{0.513923in}{0.052500in}}{\pgfqpoint{0.527278in}{0.046968in}}%
\pgfpathcurveto{\pgfqpoint{0.537123in}{0.037123in}}{\pgfqpoint{0.546968in}{0.027278in}}{\pgfqpoint{0.552500in}{0.013923in}}%
\pgfpathcurveto{\pgfqpoint{0.552500in}{0.000000in}}{\pgfqpoint{0.552500in}{-0.013923in}}{\pgfqpoint{0.546968in}{-0.027278in}}%
\pgfpathcurveto{\pgfqpoint{0.537123in}{-0.037123in}}{\pgfqpoint{0.527278in}{-0.046968in}}{\pgfqpoint{0.513923in}{-0.052500in}}%
\pgfpathclose%
\pgfpathmoveto{\pgfqpoint{0.666667in}{-0.058333in}}%
\pgfpathcurveto{\pgfqpoint{0.682137in}{-0.058333in}}{\pgfqpoint{0.696975in}{-0.052187in}}{\pgfqpoint{0.707915in}{-0.041248in}}%
\pgfpathcurveto{\pgfqpoint{0.718854in}{-0.030309in}}{\pgfqpoint{0.725000in}{-0.015470in}}{\pgfqpoint{0.725000in}{0.000000in}}%
\pgfpathcurveto{\pgfqpoint{0.725000in}{0.015470in}}{\pgfqpoint{0.718854in}{0.030309in}}{\pgfqpoint{0.707915in}{0.041248in}}%
\pgfpathcurveto{\pgfqpoint{0.696975in}{0.052187in}}{\pgfqpoint{0.682137in}{0.058333in}}{\pgfqpoint{0.666667in}{0.058333in}}%
\pgfpathcurveto{\pgfqpoint{0.651196in}{0.058333in}}{\pgfqpoint{0.636358in}{0.052187in}}{\pgfqpoint{0.625419in}{0.041248in}}%
\pgfpathcurveto{\pgfqpoint{0.614480in}{0.030309in}}{\pgfqpoint{0.608333in}{0.015470in}}{\pgfqpoint{0.608333in}{0.000000in}}%
\pgfpathcurveto{\pgfqpoint{0.608333in}{-0.015470in}}{\pgfqpoint{0.614480in}{-0.030309in}}{\pgfqpoint{0.625419in}{-0.041248in}}%
\pgfpathcurveto{\pgfqpoint{0.636358in}{-0.052187in}}{\pgfqpoint{0.651196in}{-0.058333in}}{\pgfqpoint{0.666667in}{-0.058333in}}%
\pgfpathclose%
\pgfpathmoveto{\pgfqpoint{0.666667in}{-0.052500in}}%
\pgfpathcurveto{\pgfqpoint{0.666667in}{-0.052500in}}{\pgfqpoint{0.652744in}{-0.052500in}}{\pgfqpoint{0.639389in}{-0.046968in}}%
\pgfpathcurveto{\pgfqpoint{0.629544in}{-0.037123in}}{\pgfqpoint{0.619698in}{-0.027278in}}{\pgfqpoint{0.614167in}{-0.013923in}}%
\pgfpathcurveto{\pgfqpoint{0.614167in}{0.000000in}}{\pgfqpoint{0.614167in}{0.013923in}}{\pgfqpoint{0.619698in}{0.027278in}}%
\pgfpathcurveto{\pgfqpoint{0.629544in}{0.037123in}}{\pgfqpoint{0.639389in}{0.046968in}}{\pgfqpoint{0.652744in}{0.052500in}}%
\pgfpathcurveto{\pgfqpoint{0.666667in}{0.052500in}}{\pgfqpoint{0.680590in}{0.052500in}}{\pgfqpoint{0.693945in}{0.046968in}}%
\pgfpathcurveto{\pgfqpoint{0.703790in}{0.037123in}}{\pgfqpoint{0.713635in}{0.027278in}}{\pgfqpoint{0.719167in}{0.013923in}}%
\pgfpathcurveto{\pgfqpoint{0.719167in}{0.000000in}}{\pgfqpoint{0.719167in}{-0.013923in}}{\pgfqpoint{0.713635in}{-0.027278in}}%
\pgfpathcurveto{\pgfqpoint{0.703790in}{-0.037123in}}{\pgfqpoint{0.693945in}{-0.046968in}}{\pgfqpoint{0.680590in}{-0.052500in}}%
\pgfpathclose%
\pgfpathmoveto{\pgfqpoint{0.833333in}{-0.058333in}}%
\pgfpathcurveto{\pgfqpoint{0.848804in}{-0.058333in}}{\pgfqpoint{0.863642in}{-0.052187in}}{\pgfqpoint{0.874581in}{-0.041248in}}%
\pgfpathcurveto{\pgfqpoint{0.885520in}{-0.030309in}}{\pgfqpoint{0.891667in}{-0.015470in}}{\pgfqpoint{0.891667in}{0.000000in}}%
\pgfpathcurveto{\pgfqpoint{0.891667in}{0.015470in}}{\pgfqpoint{0.885520in}{0.030309in}}{\pgfqpoint{0.874581in}{0.041248in}}%
\pgfpathcurveto{\pgfqpoint{0.863642in}{0.052187in}}{\pgfqpoint{0.848804in}{0.058333in}}{\pgfqpoint{0.833333in}{0.058333in}}%
\pgfpathcurveto{\pgfqpoint{0.817863in}{0.058333in}}{\pgfqpoint{0.803025in}{0.052187in}}{\pgfqpoint{0.792085in}{0.041248in}}%
\pgfpathcurveto{\pgfqpoint{0.781146in}{0.030309in}}{\pgfqpoint{0.775000in}{0.015470in}}{\pgfqpoint{0.775000in}{0.000000in}}%
\pgfpathcurveto{\pgfqpoint{0.775000in}{-0.015470in}}{\pgfqpoint{0.781146in}{-0.030309in}}{\pgfqpoint{0.792085in}{-0.041248in}}%
\pgfpathcurveto{\pgfqpoint{0.803025in}{-0.052187in}}{\pgfqpoint{0.817863in}{-0.058333in}}{\pgfqpoint{0.833333in}{-0.058333in}}%
\pgfpathclose%
\pgfpathmoveto{\pgfqpoint{0.833333in}{-0.052500in}}%
\pgfpathcurveto{\pgfqpoint{0.833333in}{-0.052500in}}{\pgfqpoint{0.819410in}{-0.052500in}}{\pgfqpoint{0.806055in}{-0.046968in}}%
\pgfpathcurveto{\pgfqpoint{0.796210in}{-0.037123in}}{\pgfqpoint{0.786365in}{-0.027278in}}{\pgfqpoint{0.780833in}{-0.013923in}}%
\pgfpathcurveto{\pgfqpoint{0.780833in}{0.000000in}}{\pgfqpoint{0.780833in}{0.013923in}}{\pgfqpoint{0.786365in}{0.027278in}}%
\pgfpathcurveto{\pgfqpoint{0.796210in}{0.037123in}}{\pgfqpoint{0.806055in}{0.046968in}}{\pgfqpoint{0.819410in}{0.052500in}}%
\pgfpathcurveto{\pgfqpoint{0.833333in}{0.052500in}}{\pgfqpoint{0.847256in}{0.052500in}}{\pgfqpoint{0.860611in}{0.046968in}}%
\pgfpathcurveto{\pgfqpoint{0.870456in}{0.037123in}}{\pgfqpoint{0.880302in}{0.027278in}}{\pgfqpoint{0.885833in}{0.013923in}}%
\pgfpathcurveto{\pgfqpoint{0.885833in}{0.000000in}}{\pgfqpoint{0.885833in}{-0.013923in}}{\pgfqpoint{0.880302in}{-0.027278in}}%
\pgfpathcurveto{\pgfqpoint{0.870456in}{-0.037123in}}{\pgfqpoint{0.860611in}{-0.046968in}}{\pgfqpoint{0.847256in}{-0.052500in}}%
\pgfpathclose%
\pgfpathmoveto{\pgfqpoint{1.000000in}{-0.058333in}}%
\pgfpathcurveto{\pgfqpoint{1.015470in}{-0.058333in}}{\pgfqpoint{1.030309in}{-0.052187in}}{\pgfqpoint{1.041248in}{-0.041248in}}%
\pgfpathcurveto{\pgfqpoint{1.052187in}{-0.030309in}}{\pgfqpoint{1.058333in}{-0.015470in}}{\pgfqpoint{1.058333in}{0.000000in}}%
\pgfpathcurveto{\pgfqpoint{1.058333in}{0.015470in}}{\pgfqpoint{1.052187in}{0.030309in}}{\pgfqpoint{1.041248in}{0.041248in}}%
\pgfpathcurveto{\pgfqpoint{1.030309in}{0.052187in}}{\pgfqpoint{1.015470in}{0.058333in}}{\pgfqpoint{1.000000in}{0.058333in}}%
\pgfpathcurveto{\pgfqpoint{0.984530in}{0.058333in}}{\pgfqpoint{0.969691in}{0.052187in}}{\pgfqpoint{0.958752in}{0.041248in}}%
\pgfpathcurveto{\pgfqpoint{0.947813in}{0.030309in}}{\pgfqpoint{0.941667in}{0.015470in}}{\pgfqpoint{0.941667in}{0.000000in}}%
\pgfpathcurveto{\pgfqpoint{0.941667in}{-0.015470in}}{\pgfqpoint{0.947813in}{-0.030309in}}{\pgfqpoint{0.958752in}{-0.041248in}}%
\pgfpathcurveto{\pgfqpoint{0.969691in}{-0.052187in}}{\pgfqpoint{0.984530in}{-0.058333in}}{\pgfqpoint{1.000000in}{-0.058333in}}%
\pgfpathclose%
\pgfpathmoveto{\pgfqpoint{1.000000in}{-0.052500in}}%
\pgfpathcurveto{\pgfqpoint{1.000000in}{-0.052500in}}{\pgfqpoint{0.986077in}{-0.052500in}}{\pgfqpoint{0.972722in}{-0.046968in}}%
\pgfpathcurveto{\pgfqpoint{0.962877in}{-0.037123in}}{\pgfqpoint{0.953032in}{-0.027278in}}{\pgfqpoint{0.947500in}{-0.013923in}}%
\pgfpathcurveto{\pgfqpoint{0.947500in}{0.000000in}}{\pgfqpoint{0.947500in}{0.013923in}}{\pgfqpoint{0.953032in}{0.027278in}}%
\pgfpathcurveto{\pgfqpoint{0.962877in}{0.037123in}}{\pgfqpoint{0.972722in}{0.046968in}}{\pgfqpoint{0.986077in}{0.052500in}}%
\pgfpathcurveto{\pgfqpoint{1.000000in}{0.052500in}}{\pgfqpoint{1.013923in}{0.052500in}}{\pgfqpoint{1.027278in}{0.046968in}}%
\pgfpathcurveto{\pgfqpoint{1.037123in}{0.037123in}}{\pgfqpoint{1.046968in}{0.027278in}}{\pgfqpoint{1.052500in}{0.013923in}}%
\pgfpathcurveto{\pgfqpoint{1.052500in}{0.000000in}}{\pgfqpoint{1.052500in}{-0.013923in}}{\pgfqpoint{1.046968in}{-0.027278in}}%
\pgfpathcurveto{\pgfqpoint{1.037123in}{-0.037123in}}{\pgfqpoint{1.027278in}{-0.046968in}}{\pgfqpoint{1.013923in}{-0.052500in}}%
\pgfpathclose%
\pgfpathmoveto{\pgfqpoint{0.083333in}{0.108333in}}%
\pgfpathcurveto{\pgfqpoint{0.098804in}{0.108333in}}{\pgfqpoint{0.113642in}{0.114480in}}{\pgfqpoint{0.124581in}{0.125419in}}%
\pgfpathcurveto{\pgfqpoint{0.135520in}{0.136358in}}{\pgfqpoint{0.141667in}{0.151196in}}{\pgfqpoint{0.141667in}{0.166667in}}%
\pgfpathcurveto{\pgfqpoint{0.141667in}{0.182137in}}{\pgfqpoint{0.135520in}{0.196975in}}{\pgfqpoint{0.124581in}{0.207915in}}%
\pgfpathcurveto{\pgfqpoint{0.113642in}{0.218854in}}{\pgfqpoint{0.098804in}{0.225000in}}{\pgfqpoint{0.083333in}{0.225000in}}%
\pgfpathcurveto{\pgfqpoint{0.067863in}{0.225000in}}{\pgfqpoint{0.053025in}{0.218854in}}{\pgfqpoint{0.042085in}{0.207915in}}%
\pgfpathcurveto{\pgfqpoint{0.031146in}{0.196975in}}{\pgfqpoint{0.025000in}{0.182137in}}{\pgfqpoint{0.025000in}{0.166667in}}%
\pgfpathcurveto{\pgfqpoint{0.025000in}{0.151196in}}{\pgfqpoint{0.031146in}{0.136358in}}{\pgfqpoint{0.042085in}{0.125419in}}%
\pgfpathcurveto{\pgfqpoint{0.053025in}{0.114480in}}{\pgfqpoint{0.067863in}{0.108333in}}{\pgfqpoint{0.083333in}{0.108333in}}%
\pgfpathclose%
\pgfpathmoveto{\pgfqpoint{0.083333in}{0.114167in}}%
\pgfpathcurveto{\pgfqpoint{0.083333in}{0.114167in}}{\pgfqpoint{0.069410in}{0.114167in}}{\pgfqpoint{0.056055in}{0.119698in}}%
\pgfpathcurveto{\pgfqpoint{0.046210in}{0.129544in}}{\pgfqpoint{0.036365in}{0.139389in}}{\pgfqpoint{0.030833in}{0.152744in}}%
\pgfpathcurveto{\pgfqpoint{0.030833in}{0.166667in}}{\pgfqpoint{0.030833in}{0.180590in}}{\pgfqpoint{0.036365in}{0.193945in}}%
\pgfpathcurveto{\pgfqpoint{0.046210in}{0.203790in}}{\pgfqpoint{0.056055in}{0.213635in}}{\pgfqpoint{0.069410in}{0.219167in}}%
\pgfpathcurveto{\pgfqpoint{0.083333in}{0.219167in}}{\pgfqpoint{0.097256in}{0.219167in}}{\pgfqpoint{0.110611in}{0.213635in}}%
\pgfpathcurveto{\pgfqpoint{0.120456in}{0.203790in}}{\pgfqpoint{0.130302in}{0.193945in}}{\pgfqpoint{0.135833in}{0.180590in}}%
\pgfpathcurveto{\pgfqpoint{0.135833in}{0.166667in}}{\pgfqpoint{0.135833in}{0.152744in}}{\pgfqpoint{0.130302in}{0.139389in}}%
\pgfpathcurveto{\pgfqpoint{0.120456in}{0.129544in}}{\pgfqpoint{0.110611in}{0.119698in}}{\pgfqpoint{0.097256in}{0.114167in}}%
\pgfpathclose%
\pgfpathmoveto{\pgfqpoint{0.250000in}{0.108333in}}%
\pgfpathcurveto{\pgfqpoint{0.265470in}{0.108333in}}{\pgfqpoint{0.280309in}{0.114480in}}{\pgfqpoint{0.291248in}{0.125419in}}%
\pgfpathcurveto{\pgfqpoint{0.302187in}{0.136358in}}{\pgfqpoint{0.308333in}{0.151196in}}{\pgfqpoint{0.308333in}{0.166667in}}%
\pgfpathcurveto{\pgfqpoint{0.308333in}{0.182137in}}{\pgfqpoint{0.302187in}{0.196975in}}{\pgfqpoint{0.291248in}{0.207915in}}%
\pgfpathcurveto{\pgfqpoint{0.280309in}{0.218854in}}{\pgfqpoint{0.265470in}{0.225000in}}{\pgfqpoint{0.250000in}{0.225000in}}%
\pgfpathcurveto{\pgfqpoint{0.234530in}{0.225000in}}{\pgfqpoint{0.219691in}{0.218854in}}{\pgfqpoint{0.208752in}{0.207915in}}%
\pgfpathcurveto{\pgfqpoint{0.197813in}{0.196975in}}{\pgfqpoint{0.191667in}{0.182137in}}{\pgfqpoint{0.191667in}{0.166667in}}%
\pgfpathcurveto{\pgfqpoint{0.191667in}{0.151196in}}{\pgfqpoint{0.197813in}{0.136358in}}{\pgfqpoint{0.208752in}{0.125419in}}%
\pgfpathcurveto{\pgfqpoint{0.219691in}{0.114480in}}{\pgfqpoint{0.234530in}{0.108333in}}{\pgfqpoint{0.250000in}{0.108333in}}%
\pgfpathclose%
\pgfpathmoveto{\pgfqpoint{0.250000in}{0.114167in}}%
\pgfpathcurveto{\pgfqpoint{0.250000in}{0.114167in}}{\pgfqpoint{0.236077in}{0.114167in}}{\pgfqpoint{0.222722in}{0.119698in}}%
\pgfpathcurveto{\pgfqpoint{0.212877in}{0.129544in}}{\pgfqpoint{0.203032in}{0.139389in}}{\pgfqpoint{0.197500in}{0.152744in}}%
\pgfpathcurveto{\pgfqpoint{0.197500in}{0.166667in}}{\pgfqpoint{0.197500in}{0.180590in}}{\pgfqpoint{0.203032in}{0.193945in}}%
\pgfpathcurveto{\pgfqpoint{0.212877in}{0.203790in}}{\pgfqpoint{0.222722in}{0.213635in}}{\pgfqpoint{0.236077in}{0.219167in}}%
\pgfpathcurveto{\pgfqpoint{0.250000in}{0.219167in}}{\pgfqpoint{0.263923in}{0.219167in}}{\pgfqpoint{0.277278in}{0.213635in}}%
\pgfpathcurveto{\pgfqpoint{0.287123in}{0.203790in}}{\pgfqpoint{0.296968in}{0.193945in}}{\pgfqpoint{0.302500in}{0.180590in}}%
\pgfpathcurveto{\pgfqpoint{0.302500in}{0.166667in}}{\pgfqpoint{0.302500in}{0.152744in}}{\pgfqpoint{0.296968in}{0.139389in}}%
\pgfpathcurveto{\pgfqpoint{0.287123in}{0.129544in}}{\pgfqpoint{0.277278in}{0.119698in}}{\pgfqpoint{0.263923in}{0.114167in}}%
\pgfpathclose%
\pgfpathmoveto{\pgfqpoint{0.416667in}{0.108333in}}%
\pgfpathcurveto{\pgfqpoint{0.432137in}{0.108333in}}{\pgfqpoint{0.446975in}{0.114480in}}{\pgfqpoint{0.457915in}{0.125419in}}%
\pgfpathcurveto{\pgfqpoint{0.468854in}{0.136358in}}{\pgfqpoint{0.475000in}{0.151196in}}{\pgfqpoint{0.475000in}{0.166667in}}%
\pgfpathcurveto{\pgfqpoint{0.475000in}{0.182137in}}{\pgfqpoint{0.468854in}{0.196975in}}{\pgfqpoint{0.457915in}{0.207915in}}%
\pgfpathcurveto{\pgfqpoint{0.446975in}{0.218854in}}{\pgfqpoint{0.432137in}{0.225000in}}{\pgfqpoint{0.416667in}{0.225000in}}%
\pgfpathcurveto{\pgfqpoint{0.401196in}{0.225000in}}{\pgfqpoint{0.386358in}{0.218854in}}{\pgfqpoint{0.375419in}{0.207915in}}%
\pgfpathcurveto{\pgfqpoint{0.364480in}{0.196975in}}{\pgfqpoint{0.358333in}{0.182137in}}{\pgfqpoint{0.358333in}{0.166667in}}%
\pgfpathcurveto{\pgfqpoint{0.358333in}{0.151196in}}{\pgfqpoint{0.364480in}{0.136358in}}{\pgfqpoint{0.375419in}{0.125419in}}%
\pgfpathcurveto{\pgfqpoint{0.386358in}{0.114480in}}{\pgfqpoint{0.401196in}{0.108333in}}{\pgfqpoint{0.416667in}{0.108333in}}%
\pgfpathclose%
\pgfpathmoveto{\pgfqpoint{0.416667in}{0.114167in}}%
\pgfpathcurveto{\pgfqpoint{0.416667in}{0.114167in}}{\pgfqpoint{0.402744in}{0.114167in}}{\pgfqpoint{0.389389in}{0.119698in}}%
\pgfpathcurveto{\pgfqpoint{0.379544in}{0.129544in}}{\pgfqpoint{0.369698in}{0.139389in}}{\pgfqpoint{0.364167in}{0.152744in}}%
\pgfpathcurveto{\pgfqpoint{0.364167in}{0.166667in}}{\pgfqpoint{0.364167in}{0.180590in}}{\pgfqpoint{0.369698in}{0.193945in}}%
\pgfpathcurveto{\pgfqpoint{0.379544in}{0.203790in}}{\pgfqpoint{0.389389in}{0.213635in}}{\pgfqpoint{0.402744in}{0.219167in}}%
\pgfpathcurveto{\pgfqpoint{0.416667in}{0.219167in}}{\pgfqpoint{0.430590in}{0.219167in}}{\pgfqpoint{0.443945in}{0.213635in}}%
\pgfpathcurveto{\pgfqpoint{0.453790in}{0.203790in}}{\pgfqpoint{0.463635in}{0.193945in}}{\pgfqpoint{0.469167in}{0.180590in}}%
\pgfpathcurveto{\pgfqpoint{0.469167in}{0.166667in}}{\pgfqpoint{0.469167in}{0.152744in}}{\pgfqpoint{0.463635in}{0.139389in}}%
\pgfpathcurveto{\pgfqpoint{0.453790in}{0.129544in}}{\pgfqpoint{0.443945in}{0.119698in}}{\pgfqpoint{0.430590in}{0.114167in}}%
\pgfpathclose%
\pgfpathmoveto{\pgfqpoint{0.583333in}{0.108333in}}%
\pgfpathcurveto{\pgfqpoint{0.598804in}{0.108333in}}{\pgfqpoint{0.613642in}{0.114480in}}{\pgfqpoint{0.624581in}{0.125419in}}%
\pgfpathcurveto{\pgfqpoint{0.635520in}{0.136358in}}{\pgfqpoint{0.641667in}{0.151196in}}{\pgfqpoint{0.641667in}{0.166667in}}%
\pgfpathcurveto{\pgfqpoint{0.641667in}{0.182137in}}{\pgfqpoint{0.635520in}{0.196975in}}{\pgfqpoint{0.624581in}{0.207915in}}%
\pgfpathcurveto{\pgfqpoint{0.613642in}{0.218854in}}{\pgfqpoint{0.598804in}{0.225000in}}{\pgfqpoint{0.583333in}{0.225000in}}%
\pgfpathcurveto{\pgfqpoint{0.567863in}{0.225000in}}{\pgfqpoint{0.553025in}{0.218854in}}{\pgfqpoint{0.542085in}{0.207915in}}%
\pgfpathcurveto{\pgfqpoint{0.531146in}{0.196975in}}{\pgfqpoint{0.525000in}{0.182137in}}{\pgfqpoint{0.525000in}{0.166667in}}%
\pgfpathcurveto{\pgfqpoint{0.525000in}{0.151196in}}{\pgfqpoint{0.531146in}{0.136358in}}{\pgfqpoint{0.542085in}{0.125419in}}%
\pgfpathcurveto{\pgfqpoint{0.553025in}{0.114480in}}{\pgfqpoint{0.567863in}{0.108333in}}{\pgfqpoint{0.583333in}{0.108333in}}%
\pgfpathclose%
\pgfpathmoveto{\pgfqpoint{0.583333in}{0.114167in}}%
\pgfpathcurveto{\pgfqpoint{0.583333in}{0.114167in}}{\pgfqpoint{0.569410in}{0.114167in}}{\pgfqpoint{0.556055in}{0.119698in}}%
\pgfpathcurveto{\pgfqpoint{0.546210in}{0.129544in}}{\pgfqpoint{0.536365in}{0.139389in}}{\pgfqpoint{0.530833in}{0.152744in}}%
\pgfpathcurveto{\pgfqpoint{0.530833in}{0.166667in}}{\pgfqpoint{0.530833in}{0.180590in}}{\pgfqpoint{0.536365in}{0.193945in}}%
\pgfpathcurveto{\pgfqpoint{0.546210in}{0.203790in}}{\pgfqpoint{0.556055in}{0.213635in}}{\pgfqpoint{0.569410in}{0.219167in}}%
\pgfpathcurveto{\pgfqpoint{0.583333in}{0.219167in}}{\pgfqpoint{0.597256in}{0.219167in}}{\pgfqpoint{0.610611in}{0.213635in}}%
\pgfpathcurveto{\pgfqpoint{0.620456in}{0.203790in}}{\pgfqpoint{0.630302in}{0.193945in}}{\pgfqpoint{0.635833in}{0.180590in}}%
\pgfpathcurveto{\pgfqpoint{0.635833in}{0.166667in}}{\pgfqpoint{0.635833in}{0.152744in}}{\pgfqpoint{0.630302in}{0.139389in}}%
\pgfpathcurveto{\pgfqpoint{0.620456in}{0.129544in}}{\pgfqpoint{0.610611in}{0.119698in}}{\pgfqpoint{0.597256in}{0.114167in}}%
\pgfpathclose%
\pgfpathmoveto{\pgfqpoint{0.750000in}{0.108333in}}%
\pgfpathcurveto{\pgfqpoint{0.765470in}{0.108333in}}{\pgfqpoint{0.780309in}{0.114480in}}{\pgfqpoint{0.791248in}{0.125419in}}%
\pgfpathcurveto{\pgfqpoint{0.802187in}{0.136358in}}{\pgfqpoint{0.808333in}{0.151196in}}{\pgfqpoint{0.808333in}{0.166667in}}%
\pgfpathcurveto{\pgfqpoint{0.808333in}{0.182137in}}{\pgfqpoint{0.802187in}{0.196975in}}{\pgfqpoint{0.791248in}{0.207915in}}%
\pgfpathcurveto{\pgfqpoint{0.780309in}{0.218854in}}{\pgfqpoint{0.765470in}{0.225000in}}{\pgfqpoint{0.750000in}{0.225000in}}%
\pgfpathcurveto{\pgfqpoint{0.734530in}{0.225000in}}{\pgfqpoint{0.719691in}{0.218854in}}{\pgfqpoint{0.708752in}{0.207915in}}%
\pgfpathcurveto{\pgfqpoint{0.697813in}{0.196975in}}{\pgfqpoint{0.691667in}{0.182137in}}{\pgfqpoint{0.691667in}{0.166667in}}%
\pgfpathcurveto{\pgfqpoint{0.691667in}{0.151196in}}{\pgfqpoint{0.697813in}{0.136358in}}{\pgfqpoint{0.708752in}{0.125419in}}%
\pgfpathcurveto{\pgfqpoint{0.719691in}{0.114480in}}{\pgfqpoint{0.734530in}{0.108333in}}{\pgfqpoint{0.750000in}{0.108333in}}%
\pgfpathclose%
\pgfpathmoveto{\pgfqpoint{0.750000in}{0.114167in}}%
\pgfpathcurveto{\pgfqpoint{0.750000in}{0.114167in}}{\pgfqpoint{0.736077in}{0.114167in}}{\pgfqpoint{0.722722in}{0.119698in}}%
\pgfpathcurveto{\pgfqpoint{0.712877in}{0.129544in}}{\pgfqpoint{0.703032in}{0.139389in}}{\pgfqpoint{0.697500in}{0.152744in}}%
\pgfpathcurveto{\pgfqpoint{0.697500in}{0.166667in}}{\pgfqpoint{0.697500in}{0.180590in}}{\pgfqpoint{0.703032in}{0.193945in}}%
\pgfpathcurveto{\pgfqpoint{0.712877in}{0.203790in}}{\pgfqpoint{0.722722in}{0.213635in}}{\pgfqpoint{0.736077in}{0.219167in}}%
\pgfpathcurveto{\pgfqpoint{0.750000in}{0.219167in}}{\pgfqpoint{0.763923in}{0.219167in}}{\pgfqpoint{0.777278in}{0.213635in}}%
\pgfpathcurveto{\pgfqpoint{0.787123in}{0.203790in}}{\pgfqpoint{0.796968in}{0.193945in}}{\pgfqpoint{0.802500in}{0.180590in}}%
\pgfpathcurveto{\pgfqpoint{0.802500in}{0.166667in}}{\pgfqpoint{0.802500in}{0.152744in}}{\pgfqpoint{0.796968in}{0.139389in}}%
\pgfpathcurveto{\pgfqpoint{0.787123in}{0.129544in}}{\pgfqpoint{0.777278in}{0.119698in}}{\pgfqpoint{0.763923in}{0.114167in}}%
\pgfpathclose%
\pgfpathmoveto{\pgfqpoint{0.916667in}{0.108333in}}%
\pgfpathcurveto{\pgfqpoint{0.932137in}{0.108333in}}{\pgfqpoint{0.946975in}{0.114480in}}{\pgfqpoint{0.957915in}{0.125419in}}%
\pgfpathcurveto{\pgfqpoint{0.968854in}{0.136358in}}{\pgfqpoint{0.975000in}{0.151196in}}{\pgfqpoint{0.975000in}{0.166667in}}%
\pgfpathcurveto{\pgfqpoint{0.975000in}{0.182137in}}{\pgfqpoint{0.968854in}{0.196975in}}{\pgfqpoint{0.957915in}{0.207915in}}%
\pgfpathcurveto{\pgfqpoint{0.946975in}{0.218854in}}{\pgfqpoint{0.932137in}{0.225000in}}{\pgfqpoint{0.916667in}{0.225000in}}%
\pgfpathcurveto{\pgfqpoint{0.901196in}{0.225000in}}{\pgfqpoint{0.886358in}{0.218854in}}{\pgfqpoint{0.875419in}{0.207915in}}%
\pgfpathcurveto{\pgfqpoint{0.864480in}{0.196975in}}{\pgfqpoint{0.858333in}{0.182137in}}{\pgfqpoint{0.858333in}{0.166667in}}%
\pgfpathcurveto{\pgfqpoint{0.858333in}{0.151196in}}{\pgfqpoint{0.864480in}{0.136358in}}{\pgfqpoint{0.875419in}{0.125419in}}%
\pgfpathcurveto{\pgfqpoint{0.886358in}{0.114480in}}{\pgfqpoint{0.901196in}{0.108333in}}{\pgfqpoint{0.916667in}{0.108333in}}%
\pgfpathclose%
\pgfpathmoveto{\pgfqpoint{0.916667in}{0.114167in}}%
\pgfpathcurveto{\pgfqpoint{0.916667in}{0.114167in}}{\pgfqpoint{0.902744in}{0.114167in}}{\pgfqpoint{0.889389in}{0.119698in}}%
\pgfpathcurveto{\pgfqpoint{0.879544in}{0.129544in}}{\pgfqpoint{0.869698in}{0.139389in}}{\pgfqpoint{0.864167in}{0.152744in}}%
\pgfpathcurveto{\pgfqpoint{0.864167in}{0.166667in}}{\pgfqpoint{0.864167in}{0.180590in}}{\pgfqpoint{0.869698in}{0.193945in}}%
\pgfpathcurveto{\pgfqpoint{0.879544in}{0.203790in}}{\pgfqpoint{0.889389in}{0.213635in}}{\pgfqpoint{0.902744in}{0.219167in}}%
\pgfpathcurveto{\pgfqpoint{0.916667in}{0.219167in}}{\pgfqpoint{0.930590in}{0.219167in}}{\pgfqpoint{0.943945in}{0.213635in}}%
\pgfpathcurveto{\pgfqpoint{0.953790in}{0.203790in}}{\pgfqpoint{0.963635in}{0.193945in}}{\pgfqpoint{0.969167in}{0.180590in}}%
\pgfpathcurveto{\pgfqpoint{0.969167in}{0.166667in}}{\pgfqpoint{0.969167in}{0.152744in}}{\pgfqpoint{0.963635in}{0.139389in}}%
\pgfpathcurveto{\pgfqpoint{0.953790in}{0.129544in}}{\pgfqpoint{0.943945in}{0.119698in}}{\pgfqpoint{0.930590in}{0.114167in}}%
\pgfpathclose%
\pgfpathmoveto{\pgfqpoint{0.000000in}{0.275000in}}%
\pgfpathcurveto{\pgfqpoint{0.015470in}{0.275000in}}{\pgfqpoint{0.030309in}{0.281146in}}{\pgfqpoint{0.041248in}{0.292085in}}%
\pgfpathcurveto{\pgfqpoint{0.052187in}{0.303025in}}{\pgfqpoint{0.058333in}{0.317863in}}{\pgfqpoint{0.058333in}{0.333333in}}%
\pgfpathcurveto{\pgfqpoint{0.058333in}{0.348804in}}{\pgfqpoint{0.052187in}{0.363642in}}{\pgfqpoint{0.041248in}{0.374581in}}%
\pgfpathcurveto{\pgfqpoint{0.030309in}{0.385520in}}{\pgfqpoint{0.015470in}{0.391667in}}{\pgfqpoint{0.000000in}{0.391667in}}%
\pgfpathcurveto{\pgfqpoint{-0.015470in}{0.391667in}}{\pgfqpoint{-0.030309in}{0.385520in}}{\pgfqpoint{-0.041248in}{0.374581in}}%
\pgfpathcurveto{\pgfqpoint{-0.052187in}{0.363642in}}{\pgfqpoint{-0.058333in}{0.348804in}}{\pgfqpoint{-0.058333in}{0.333333in}}%
\pgfpathcurveto{\pgfqpoint{-0.058333in}{0.317863in}}{\pgfqpoint{-0.052187in}{0.303025in}}{\pgfqpoint{-0.041248in}{0.292085in}}%
\pgfpathcurveto{\pgfqpoint{-0.030309in}{0.281146in}}{\pgfqpoint{-0.015470in}{0.275000in}}{\pgfqpoint{0.000000in}{0.275000in}}%
\pgfpathclose%
\pgfpathmoveto{\pgfqpoint{0.000000in}{0.280833in}}%
\pgfpathcurveto{\pgfqpoint{0.000000in}{0.280833in}}{\pgfqpoint{-0.013923in}{0.280833in}}{\pgfqpoint{-0.027278in}{0.286365in}}%
\pgfpathcurveto{\pgfqpoint{-0.037123in}{0.296210in}}{\pgfqpoint{-0.046968in}{0.306055in}}{\pgfqpoint{-0.052500in}{0.319410in}}%
\pgfpathcurveto{\pgfqpoint{-0.052500in}{0.333333in}}{\pgfqpoint{-0.052500in}{0.347256in}}{\pgfqpoint{-0.046968in}{0.360611in}}%
\pgfpathcurveto{\pgfqpoint{-0.037123in}{0.370456in}}{\pgfqpoint{-0.027278in}{0.380302in}}{\pgfqpoint{-0.013923in}{0.385833in}}%
\pgfpathcurveto{\pgfqpoint{0.000000in}{0.385833in}}{\pgfqpoint{0.013923in}{0.385833in}}{\pgfqpoint{0.027278in}{0.380302in}}%
\pgfpathcurveto{\pgfqpoint{0.037123in}{0.370456in}}{\pgfqpoint{0.046968in}{0.360611in}}{\pgfqpoint{0.052500in}{0.347256in}}%
\pgfpathcurveto{\pgfqpoint{0.052500in}{0.333333in}}{\pgfqpoint{0.052500in}{0.319410in}}{\pgfqpoint{0.046968in}{0.306055in}}%
\pgfpathcurveto{\pgfqpoint{0.037123in}{0.296210in}}{\pgfqpoint{0.027278in}{0.286365in}}{\pgfqpoint{0.013923in}{0.280833in}}%
\pgfpathclose%
\pgfpathmoveto{\pgfqpoint{0.166667in}{0.275000in}}%
\pgfpathcurveto{\pgfqpoint{0.182137in}{0.275000in}}{\pgfqpoint{0.196975in}{0.281146in}}{\pgfqpoint{0.207915in}{0.292085in}}%
\pgfpathcurveto{\pgfqpoint{0.218854in}{0.303025in}}{\pgfqpoint{0.225000in}{0.317863in}}{\pgfqpoint{0.225000in}{0.333333in}}%
\pgfpathcurveto{\pgfqpoint{0.225000in}{0.348804in}}{\pgfqpoint{0.218854in}{0.363642in}}{\pgfqpoint{0.207915in}{0.374581in}}%
\pgfpathcurveto{\pgfqpoint{0.196975in}{0.385520in}}{\pgfqpoint{0.182137in}{0.391667in}}{\pgfqpoint{0.166667in}{0.391667in}}%
\pgfpathcurveto{\pgfqpoint{0.151196in}{0.391667in}}{\pgfqpoint{0.136358in}{0.385520in}}{\pgfqpoint{0.125419in}{0.374581in}}%
\pgfpathcurveto{\pgfqpoint{0.114480in}{0.363642in}}{\pgfqpoint{0.108333in}{0.348804in}}{\pgfqpoint{0.108333in}{0.333333in}}%
\pgfpathcurveto{\pgfqpoint{0.108333in}{0.317863in}}{\pgfqpoint{0.114480in}{0.303025in}}{\pgfqpoint{0.125419in}{0.292085in}}%
\pgfpathcurveto{\pgfqpoint{0.136358in}{0.281146in}}{\pgfqpoint{0.151196in}{0.275000in}}{\pgfqpoint{0.166667in}{0.275000in}}%
\pgfpathclose%
\pgfpathmoveto{\pgfqpoint{0.166667in}{0.280833in}}%
\pgfpathcurveto{\pgfqpoint{0.166667in}{0.280833in}}{\pgfqpoint{0.152744in}{0.280833in}}{\pgfqpoint{0.139389in}{0.286365in}}%
\pgfpathcurveto{\pgfqpoint{0.129544in}{0.296210in}}{\pgfqpoint{0.119698in}{0.306055in}}{\pgfqpoint{0.114167in}{0.319410in}}%
\pgfpathcurveto{\pgfqpoint{0.114167in}{0.333333in}}{\pgfqpoint{0.114167in}{0.347256in}}{\pgfqpoint{0.119698in}{0.360611in}}%
\pgfpathcurveto{\pgfqpoint{0.129544in}{0.370456in}}{\pgfqpoint{0.139389in}{0.380302in}}{\pgfqpoint{0.152744in}{0.385833in}}%
\pgfpathcurveto{\pgfqpoint{0.166667in}{0.385833in}}{\pgfqpoint{0.180590in}{0.385833in}}{\pgfqpoint{0.193945in}{0.380302in}}%
\pgfpathcurveto{\pgfqpoint{0.203790in}{0.370456in}}{\pgfqpoint{0.213635in}{0.360611in}}{\pgfqpoint{0.219167in}{0.347256in}}%
\pgfpathcurveto{\pgfqpoint{0.219167in}{0.333333in}}{\pgfqpoint{0.219167in}{0.319410in}}{\pgfqpoint{0.213635in}{0.306055in}}%
\pgfpathcurveto{\pgfqpoint{0.203790in}{0.296210in}}{\pgfqpoint{0.193945in}{0.286365in}}{\pgfqpoint{0.180590in}{0.280833in}}%
\pgfpathclose%
\pgfpathmoveto{\pgfqpoint{0.333333in}{0.275000in}}%
\pgfpathcurveto{\pgfqpoint{0.348804in}{0.275000in}}{\pgfqpoint{0.363642in}{0.281146in}}{\pgfqpoint{0.374581in}{0.292085in}}%
\pgfpathcurveto{\pgfqpoint{0.385520in}{0.303025in}}{\pgfqpoint{0.391667in}{0.317863in}}{\pgfqpoint{0.391667in}{0.333333in}}%
\pgfpathcurveto{\pgfqpoint{0.391667in}{0.348804in}}{\pgfqpoint{0.385520in}{0.363642in}}{\pgfqpoint{0.374581in}{0.374581in}}%
\pgfpathcurveto{\pgfqpoint{0.363642in}{0.385520in}}{\pgfqpoint{0.348804in}{0.391667in}}{\pgfqpoint{0.333333in}{0.391667in}}%
\pgfpathcurveto{\pgfqpoint{0.317863in}{0.391667in}}{\pgfqpoint{0.303025in}{0.385520in}}{\pgfqpoint{0.292085in}{0.374581in}}%
\pgfpathcurveto{\pgfqpoint{0.281146in}{0.363642in}}{\pgfqpoint{0.275000in}{0.348804in}}{\pgfqpoint{0.275000in}{0.333333in}}%
\pgfpathcurveto{\pgfqpoint{0.275000in}{0.317863in}}{\pgfqpoint{0.281146in}{0.303025in}}{\pgfqpoint{0.292085in}{0.292085in}}%
\pgfpathcurveto{\pgfqpoint{0.303025in}{0.281146in}}{\pgfqpoint{0.317863in}{0.275000in}}{\pgfqpoint{0.333333in}{0.275000in}}%
\pgfpathclose%
\pgfpathmoveto{\pgfqpoint{0.333333in}{0.280833in}}%
\pgfpathcurveto{\pgfqpoint{0.333333in}{0.280833in}}{\pgfqpoint{0.319410in}{0.280833in}}{\pgfqpoint{0.306055in}{0.286365in}}%
\pgfpathcurveto{\pgfqpoint{0.296210in}{0.296210in}}{\pgfqpoint{0.286365in}{0.306055in}}{\pgfqpoint{0.280833in}{0.319410in}}%
\pgfpathcurveto{\pgfqpoint{0.280833in}{0.333333in}}{\pgfqpoint{0.280833in}{0.347256in}}{\pgfqpoint{0.286365in}{0.360611in}}%
\pgfpathcurveto{\pgfqpoint{0.296210in}{0.370456in}}{\pgfqpoint{0.306055in}{0.380302in}}{\pgfqpoint{0.319410in}{0.385833in}}%
\pgfpathcurveto{\pgfqpoint{0.333333in}{0.385833in}}{\pgfqpoint{0.347256in}{0.385833in}}{\pgfqpoint{0.360611in}{0.380302in}}%
\pgfpathcurveto{\pgfqpoint{0.370456in}{0.370456in}}{\pgfqpoint{0.380302in}{0.360611in}}{\pgfqpoint{0.385833in}{0.347256in}}%
\pgfpathcurveto{\pgfqpoint{0.385833in}{0.333333in}}{\pgfqpoint{0.385833in}{0.319410in}}{\pgfqpoint{0.380302in}{0.306055in}}%
\pgfpathcurveto{\pgfqpoint{0.370456in}{0.296210in}}{\pgfqpoint{0.360611in}{0.286365in}}{\pgfqpoint{0.347256in}{0.280833in}}%
\pgfpathclose%
\pgfpathmoveto{\pgfqpoint{0.500000in}{0.275000in}}%
\pgfpathcurveto{\pgfqpoint{0.515470in}{0.275000in}}{\pgfqpoint{0.530309in}{0.281146in}}{\pgfqpoint{0.541248in}{0.292085in}}%
\pgfpathcurveto{\pgfqpoint{0.552187in}{0.303025in}}{\pgfqpoint{0.558333in}{0.317863in}}{\pgfqpoint{0.558333in}{0.333333in}}%
\pgfpathcurveto{\pgfqpoint{0.558333in}{0.348804in}}{\pgfqpoint{0.552187in}{0.363642in}}{\pgfqpoint{0.541248in}{0.374581in}}%
\pgfpathcurveto{\pgfqpoint{0.530309in}{0.385520in}}{\pgfqpoint{0.515470in}{0.391667in}}{\pgfqpoint{0.500000in}{0.391667in}}%
\pgfpathcurveto{\pgfqpoint{0.484530in}{0.391667in}}{\pgfqpoint{0.469691in}{0.385520in}}{\pgfqpoint{0.458752in}{0.374581in}}%
\pgfpathcurveto{\pgfqpoint{0.447813in}{0.363642in}}{\pgfqpoint{0.441667in}{0.348804in}}{\pgfqpoint{0.441667in}{0.333333in}}%
\pgfpathcurveto{\pgfqpoint{0.441667in}{0.317863in}}{\pgfqpoint{0.447813in}{0.303025in}}{\pgfqpoint{0.458752in}{0.292085in}}%
\pgfpathcurveto{\pgfqpoint{0.469691in}{0.281146in}}{\pgfqpoint{0.484530in}{0.275000in}}{\pgfqpoint{0.500000in}{0.275000in}}%
\pgfpathclose%
\pgfpathmoveto{\pgfqpoint{0.500000in}{0.280833in}}%
\pgfpathcurveto{\pgfqpoint{0.500000in}{0.280833in}}{\pgfqpoint{0.486077in}{0.280833in}}{\pgfqpoint{0.472722in}{0.286365in}}%
\pgfpathcurveto{\pgfqpoint{0.462877in}{0.296210in}}{\pgfqpoint{0.453032in}{0.306055in}}{\pgfqpoint{0.447500in}{0.319410in}}%
\pgfpathcurveto{\pgfqpoint{0.447500in}{0.333333in}}{\pgfqpoint{0.447500in}{0.347256in}}{\pgfqpoint{0.453032in}{0.360611in}}%
\pgfpathcurveto{\pgfqpoint{0.462877in}{0.370456in}}{\pgfqpoint{0.472722in}{0.380302in}}{\pgfqpoint{0.486077in}{0.385833in}}%
\pgfpathcurveto{\pgfqpoint{0.500000in}{0.385833in}}{\pgfqpoint{0.513923in}{0.385833in}}{\pgfqpoint{0.527278in}{0.380302in}}%
\pgfpathcurveto{\pgfqpoint{0.537123in}{0.370456in}}{\pgfqpoint{0.546968in}{0.360611in}}{\pgfqpoint{0.552500in}{0.347256in}}%
\pgfpathcurveto{\pgfqpoint{0.552500in}{0.333333in}}{\pgfqpoint{0.552500in}{0.319410in}}{\pgfqpoint{0.546968in}{0.306055in}}%
\pgfpathcurveto{\pgfqpoint{0.537123in}{0.296210in}}{\pgfqpoint{0.527278in}{0.286365in}}{\pgfqpoint{0.513923in}{0.280833in}}%
\pgfpathclose%
\pgfpathmoveto{\pgfqpoint{0.666667in}{0.275000in}}%
\pgfpathcurveto{\pgfqpoint{0.682137in}{0.275000in}}{\pgfqpoint{0.696975in}{0.281146in}}{\pgfqpoint{0.707915in}{0.292085in}}%
\pgfpathcurveto{\pgfqpoint{0.718854in}{0.303025in}}{\pgfqpoint{0.725000in}{0.317863in}}{\pgfqpoint{0.725000in}{0.333333in}}%
\pgfpathcurveto{\pgfqpoint{0.725000in}{0.348804in}}{\pgfqpoint{0.718854in}{0.363642in}}{\pgfqpoint{0.707915in}{0.374581in}}%
\pgfpathcurveto{\pgfqpoint{0.696975in}{0.385520in}}{\pgfqpoint{0.682137in}{0.391667in}}{\pgfqpoint{0.666667in}{0.391667in}}%
\pgfpathcurveto{\pgfqpoint{0.651196in}{0.391667in}}{\pgfqpoint{0.636358in}{0.385520in}}{\pgfqpoint{0.625419in}{0.374581in}}%
\pgfpathcurveto{\pgfqpoint{0.614480in}{0.363642in}}{\pgfqpoint{0.608333in}{0.348804in}}{\pgfqpoint{0.608333in}{0.333333in}}%
\pgfpathcurveto{\pgfqpoint{0.608333in}{0.317863in}}{\pgfqpoint{0.614480in}{0.303025in}}{\pgfqpoint{0.625419in}{0.292085in}}%
\pgfpathcurveto{\pgfqpoint{0.636358in}{0.281146in}}{\pgfqpoint{0.651196in}{0.275000in}}{\pgfqpoint{0.666667in}{0.275000in}}%
\pgfpathclose%
\pgfpathmoveto{\pgfqpoint{0.666667in}{0.280833in}}%
\pgfpathcurveto{\pgfqpoint{0.666667in}{0.280833in}}{\pgfqpoint{0.652744in}{0.280833in}}{\pgfqpoint{0.639389in}{0.286365in}}%
\pgfpathcurveto{\pgfqpoint{0.629544in}{0.296210in}}{\pgfqpoint{0.619698in}{0.306055in}}{\pgfqpoint{0.614167in}{0.319410in}}%
\pgfpathcurveto{\pgfqpoint{0.614167in}{0.333333in}}{\pgfqpoint{0.614167in}{0.347256in}}{\pgfqpoint{0.619698in}{0.360611in}}%
\pgfpathcurveto{\pgfqpoint{0.629544in}{0.370456in}}{\pgfqpoint{0.639389in}{0.380302in}}{\pgfqpoint{0.652744in}{0.385833in}}%
\pgfpathcurveto{\pgfqpoint{0.666667in}{0.385833in}}{\pgfqpoint{0.680590in}{0.385833in}}{\pgfqpoint{0.693945in}{0.380302in}}%
\pgfpathcurveto{\pgfqpoint{0.703790in}{0.370456in}}{\pgfqpoint{0.713635in}{0.360611in}}{\pgfqpoint{0.719167in}{0.347256in}}%
\pgfpathcurveto{\pgfqpoint{0.719167in}{0.333333in}}{\pgfqpoint{0.719167in}{0.319410in}}{\pgfqpoint{0.713635in}{0.306055in}}%
\pgfpathcurveto{\pgfqpoint{0.703790in}{0.296210in}}{\pgfqpoint{0.693945in}{0.286365in}}{\pgfqpoint{0.680590in}{0.280833in}}%
\pgfpathclose%
\pgfpathmoveto{\pgfqpoint{0.833333in}{0.275000in}}%
\pgfpathcurveto{\pgfqpoint{0.848804in}{0.275000in}}{\pgfqpoint{0.863642in}{0.281146in}}{\pgfqpoint{0.874581in}{0.292085in}}%
\pgfpathcurveto{\pgfqpoint{0.885520in}{0.303025in}}{\pgfqpoint{0.891667in}{0.317863in}}{\pgfqpoint{0.891667in}{0.333333in}}%
\pgfpathcurveto{\pgfqpoint{0.891667in}{0.348804in}}{\pgfqpoint{0.885520in}{0.363642in}}{\pgfqpoint{0.874581in}{0.374581in}}%
\pgfpathcurveto{\pgfqpoint{0.863642in}{0.385520in}}{\pgfqpoint{0.848804in}{0.391667in}}{\pgfqpoint{0.833333in}{0.391667in}}%
\pgfpathcurveto{\pgfqpoint{0.817863in}{0.391667in}}{\pgfqpoint{0.803025in}{0.385520in}}{\pgfqpoint{0.792085in}{0.374581in}}%
\pgfpathcurveto{\pgfqpoint{0.781146in}{0.363642in}}{\pgfqpoint{0.775000in}{0.348804in}}{\pgfqpoint{0.775000in}{0.333333in}}%
\pgfpathcurveto{\pgfqpoint{0.775000in}{0.317863in}}{\pgfqpoint{0.781146in}{0.303025in}}{\pgfqpoint{0.792085in}{0.292085in}}%
\pgfpathcurveto{\pgfqpoint{0.803025in}{0.281146in}}{\pgfqpoint{0.817863in}{0.275000in}}{\pgfqpoint{0.833333in}{0.275000in}}%
\pgfpathclose%
\pgfpathmoveto{\pgfqpoint{0.833333in}{0.280833in}}%
\pgfpathcurveto{\pgfqpoint{0.833333in}{0.280833in}}{\pgfqpoint{0.819410in}{0.280833in}}{\pgfqpoint{0.806055in}{0.286365in}}%
\pgfpathcurveto{\pgfqpoint{0.796210in}{0.296210in}}{\pgfqpoint{0.786365in}{0.306055in}}{\pgfqpoint{0.780833in}{0.319410in}}%
\pgfpathcurveto{\pgfqpoint{0.780833in}{0.333333in}}{\pgfqpoint{0.780833in}{0.347256in}}{\pgfqpoint{0.786365in}{0.360611in}}%
\pgfpathcurveto{\pgfqpoint{0.796210in}{0.370456in}}{\pgfqpoint{0.806055in}{0.380302in}}{\pgfqpoint{0.819410in}{0.385833in}}%
\pgfpathcurveto{\pgfqpoint{0.833333in}{0.385833in}}{\pgfqpoint{0.847256in}{0.385833in}}{\pgfqpoint{0.860611in}{0.380302in}}%
\pgfpathcurveto{\pgfqpoint{0.870456in}{0.370456in}}{\pgfqpoint{0.880302in}{0.360611in}}{\pgfqpoint{0.885833in}{0.347256in}}%
\pgfpathcurveto{\pgfqpoint{0.885833in}{0.333333in}}{\pgfqpoint{0.885833in}{0.319410in}}{\pgfqpoint{0.880302in}{0.306055in}}%
\pgfpathcurveto{\pgfqpoint{0.870456in}{0.296210in}}{\pgfqpoint{0.860611in}{0.286365in}}{\pgfqpoint{0.847256in}{0.280833in}}%
\pgfpathclose%
\pgfpathmoveto{\pgfqpoint{1.000000in}{0.275000in}}%
\pgfpathcurveto{\pgfqpoint{1.015470in}{0.275000in}}{\pgfqpoint{1.030309in}{0.281146in}}{\pgfqpoint{1.041248in}{0.292085in}}%
\pgfpathcurveto{\pgfqpoint{1.052187in}{0.303025in}}{\pgfqpoint{1.058333in}{0.317863in}}{\pgfqpoint{1.058333in}{0.333333in}}%
\pgfpathcurveto{\pgfqpoint{1.058333in}{0.348804in}}{\pgfqpoint{1.052187in}{0.363642in}}{\pgfqpoint{1.041248in}{0.374581in}}%
\pgfpathcurveto{\pgfqpoint{1.030309in}{0.385520in}}{\pgfqpoint{1.015470in}{0.391667in}}{\pgfqpoint{1.000000in}{0.391667in}}%
\pgfpathcurveto{\pgfqpoint{0.984530in}{0.391667in}}{\pgfqpoint{0.969691in}{0.385520in}}{\pgfqpoint{0.958752in}{0.374581in}}%
\pgfpathcurveto{\pgfqpoint{0.947813in}{0.363642in}}{\pgfqpoint{0.941667in}{0.348804in}}{\pgfqpoint{0.941667in}{0.333333in}}%
\pgfpathcurveto{\pgfqpoint{0.941667in}{0.317863in}}{\pgfqpoint{0.947813in}{0.303025in}}{\pgfqpoint{0.958752in}{0.292085in}}%
\pgfpathcurveto{\pgfqpoint{0.969691in}{0.281146in}}{\pgfqpoint{0.984530in}{0.275000in}}{\pgfqpoint{1.000000in}{0.275000in}}%
\pgfpathclose%
\pgfpathmoveto{\pgfqpoint{1.000000in}{0.280833in}}%
\pgfpathcurveto{\pgfqpoint{1.000000in}{0.280833in}}{\pgfqpoint{0.986077in}{0.280833in}}{\pgfqpoint{0.972722in}{0.286365in}}%
\pgfpathcurveto{\pgfqpoint{0.962877in}{0.296210in}}{\pgfqpoint{0.953032in}{0.306055in}}{\pgfqpoint{0.947500in}{0.319410in}}%
\pgfpathcurveto{\pgfqpoint{0.947500in}{0.333333in}}{\pgfqpoint{0.947500in}{0.347256in}}{\pgfqpoint{0.953032in}{0.360611in}}%
\pgfpathcurveto{\pgfqpoint{0.962877in}{0.370456in}}{\pgfqpoint{0.972722in}{0.380302in}}{\pgfqpoint{0.986077in}{0.385833in}}%
\pgfpathcurveto{\pgfqpoint{1.000000in}{0.385833in}}{\pgfqpoint{1.013923in}{0.385833in}}{\pgfqpoint{1.027278in}{0.380302in}}%
\pgfpathcurveto{\pgfqpoint{1.037123in}{0.370456in}}{\pgfqpoint{1.046968in}{0.360611in}}{\pgfqpoint{1.052500in}{0.347256in}}%
\pgfpathcurveto{\pgfqpoint{1.052500in}{0.333333in}}{\pgfqpoint{1.052500in}{0.319410in}}{\pgfqpoint{1.046968in}{0.306055in}}%
\pgfpathcurveto{\pgfqpoint{1.037123in}{0.296210in}}{\pgfqpoint{1.027278in}{0.286365in}}{\pgfqpoint{1.013923in}{0.280833in}}%
\pgfpathclose%
\pgfpathmoveto{\pgfqpoint{0.083333in}{0.441667in}}%
\pgfpathcurveto{\pgfqpoint{0.098804in}{0.441667in}}{\pgfqpoint{0.113642in}{0.447813in}}{\pgfqpoint{0.124581in}{0.458752in}}%
\pgfpathcurveto{\pgfqpoint{0.135520in}{0.469691in}}{\pgfqpoint{0.141667in}{0.484530in}}{\pgfqpoint{0.141667in}{0.500000in}}%
\pgfpathcurveto{\pgfqpoint{0.141667in}{0.515470in}}{\pgfqpoint{0.135520in}{0.530309in}}{\pgfqpoint{0.124581in}{0.541248in}}%
\pgfpathcurveto{\pgfqpoint{0.113642in}{0.552187in}}{\pgfqpoint{0.098804in}{0.558333in}}{\pgfqpoint{0.083333in}{0.558333in}}%
\pgfpathcurveto{\pgfqpoint{0.067863in}{0.558333in}}{\pgfqpoint{0.053025in}{0.552187in}}{\pgfqpoint{0.042085in}{0.541248in}}%
\pgfpathcurveto{\pgfqpoint{0.031146in}{0.530309in}}{\pgfqpoint{0.025000in}{0.515470in}}{\pgfqpoint{0.025000in}{0.500000in}}%
\pgfpathcurveto{\pgfqpoint{0.025000in}{0.484530in}}{\pgfqpoint{0.031146in}{0.469691in}}{\pgfqpoint{0.042085in}{0.458752in}}%
\pgfpathcurveto{\pgfqpoint{0.053025in}{0.447813in}}{\pgfqpoint{0.067863in}{0.441667in}}{\pgfqpoint{0.083333in}{0.441667in}}%
\pgfpathclose%
\pgfpathmoveto{\pgfqpoint{0.083333in}{0.447500in}}%
\pgfpathcurveto{\pgfqpoint{0.083333in}{0.447500in}}{\pgfqpoint{0.069410in}{0.447500in}}{\pgfqpoint{0.056055in}{0.453032in}}%
\pgfpathcurveto{\pgfqpoint{0.046210in}{0.462877in}}{\pgfqpoint{0.036365in}{0.472722in}}{\pgfqpoint{0.030833in}{0.486077in}}%
\pgfpathcurveto{\pgfqpoint{0.030833in}{0.500000in}}{\pgfqpoint{0.030833in}{0.513923in}}{\pgfqpoint{0.036365in}{0.527278in}}%
\pgfpathcurveto{\pgfqpoint{0.046210in}{0.537123in}}{\pgfqpoint{0.056055in}{0.546968in}}{\pgfqpoint{0.069410in}{0.552500in}}%
\pgfpathcurveto{\pgfqpoint{0.083333in}{0.552500in}}{\pgfqpoint{0.097256in}{0.552500in}}{\pgfqpoint{0.110611in}{0.546968in}}%
\pgfpathcurveto{\pgfqpoint{0.120456in}{0.537123in}}{\pgfqpoint{0.130302in}{0.527278in}}{\pgfqpoint{0.135833in}{0.513923in}}%
\pgfpathcurveto{\pgfqpoint{0.135833in}{0.500000in}}{\pgfqpoint{0.135833in}{0.486077in}}{\pgfqpoint{0.130302in}{0.472722in}}%
\pgfpathcurveto{\pgfqpoint{0.120456in}{0.462877in}}{\pgfqpoint{0.110611in}{0.453032in}}{\pgfqpoint{0.097256in}{0.447500in}}%
\pgfpathclose%
\pgfpathmoveto{\pgfqpoint{0.250000in}{0.441667in}}%
\pgfpathcurveto{\pgfqpoint{0.265470in}{0.441667in}}{\pgfqpoint{0.280309in}{0.447813in}}{\pgfqpoint{0.291248in}{0.458752in}}%
\pgfpathcurveto{\pgfqpoint{0.302187in}{0.469691in}}{\pgfqpoint{0.308333in}{0.484530in}}{\pgfqpoint{0.308333in}{0.500000in}}%
\pgfpathcurveto{\pgfqpoint{0.308333in}{0.515470in}}{\pgfqpoint{0.302187in}{0.530309in}}{\pgfqpoint{0.291248in}{0.541248in}}%
\pgfpathcurveto{\pgfqpoint{0.280309in}{0.552187in}}{\pgfqpoint{0.265470in}{0.558333in}}{\pgfqpoint{0.250000in}{0.558333in}}%
\pgfpathcurveto{\pgfqpoint{0.234530in}{0.558333in}}{\pgfqpoint{0.219691in}{0.552187in}}{\pgfqpoint{0.208752in}{0.541248in}}%
\pgfpathcurveto{\pgfqpoint{0.197813in}{0.530309in}}{\pgfqpoint{0.191667in}{0.515470in}}{\pgfqpoint{0.191667in}{0.500000in}}%
\pgfpathcurveto{\pgfqpoint{0.191667in}{0.484530in}}{\pgfqpoint{0.197813in}{0.469691in}}{\pgfqpoint{0.208752in}{0.458752in}}%
\pgfpathcurveto{\pgfqpoint{0.219691in}{0.447813in}}{\pgfqpoint{0.234530in}{0.441667in}}{\pgfqpoint{0.250000in}{0.441667in}}%
\pgfpathclose%
\pgfpathmoveto{\pgfqpoint{0.250000in}{0.447500in}}%
\pgfpathcurveto{\pgfqpoint{0.250000in}{0.447500in}}{\pgfqpoint{0.236077in}{0.447500in}}{\pgfqpoint{0.222722in}{0.453032in}}%
\pgfpathcurveto{\pgfqpoint{0.212877in}{0.462877in}}{\pgfqpoint{0.203032in}{0.472722in}}{\pgfqpoint{0.197500in}{0.486077in}}%
\pgfpathcurveto{\pgfqpoint{0.197500in}{0.500000in}}{\pgfqpoint{0.197500in}{0.513923in}}{\pgfqpoint{0.203032in}{0.527278in}}%
\pgfpathcurveto{\pgfqpoint{0.212877in}{0.537123in}}{\pgfqpoint{0.222722in}{0.546968in}}{\pgfqpoint{0.236077in}{0.552500in}}%
\pgfpathcurveto{\pgfqpoint{0.250000in}{0.552500in}}{\pgfqpoint{0.263923in}{0.552500in}}{\pgfqpoint{0.277278in}{0.546968in}}%
\pgfpathcurveto{\pgfqpoint{0.287123in}{0.537123in}}{\pgfqpoint{0.296968in}{0.527278in}}{\pgfqpoint{0.302500in}{0.513923in}}%
\pgfpathcurveto{\pgfqpoint{0.302500in}{0.500000in}}{\pgfqpoint{0.302500in}{0.486077in}}{\pgfqpoint{0.296968in}{0.472722in}}%
\pgfpathcurveto{\pgfqpoint{0.287123in}{0.462877in}}{\pgfqpoint{0.277278in}{0.453032in}}{\pgfqpoint{0.263923in}{0.447500in}}%
\pgfpathclose%
\pgfpathmoveto{\pgfqpoint{0.416667in}{0.441667in}}%
\pgfpathcurveto{\pgfqpoint{0.432137in}{0.441667in}}{\pgfqpoint{0.446975in}{0.447813in}}{\pgfqpoint{0.457915in}{0.458752in}}%
\pgfpathcurveto{\pgfqpoint{0.468854in}{0.469691in}}{\pgfqpoint{0.475000in}{0.484530in}}{\pgfqpoint{0.475000in}{0.500000in}}%
\pgfpathcurveto{\pgfqpoint{0.475000in}{0.515470in}}{\pgfqpoint{0.468854in}{0.530309in}}{\pgfqpoint{0.457915in}{0.541248in}}%
\pgfpathcurveto{\pgfqpoint{0.446975in}{0.552187in}}{\pgfqpoint{0.432137in}{0.558333in}}{\pgfqpoint{0.416667in}{0.558333in}}%
\pgfpathcurveto{\pgfqpoint{0.401196in}{0.558333in}}{\pgfqpoint{0.386358in}{0.552187in}}{\pgfqpoint{0.375419in}{0.541248in}}%
\pgfpathcurveto{\pgfqpoint{0.364480in}{0.530309in}}{\pgfqpoint{0.358333in}{0.515470in}}{\pgfqpoint{0.358333in}{0.500000in}}%
\pgfpathcurveto{\pgfqpoint{0.358333in}{0.484530in}}{\pgfqpoint{0.364480in}{0.469691in}}{\pgfqpoint{0.375419in}{0.458752in}}%
\pgfpathcurveto{\pgfqpoint{0.386358in}{0.447813in}}{\pgfqpoint{0.401196in}{0.441667in}}{\pgfqpoint{0.416667in}{0.441667in}}%
\pgfpathclose%
\pgfpathmoveto{\pgfqpoint{0.416667in}{0.447500in}}%
\pgfpathcurveto{\pgfqpoint{0.416667in}{0.447500in}}{\pgfqpoint{0.402744in}{0.447500in}}{\pgfqpoint{0.389389in}{0.453032in}}%
\pgfpathcurveto{\pgfqpoint{0.379544in}{0.462877in}}{\pgfqpoint{0.369698in}{0.472722in}}{\pgfqpoint{0.364167in}{0.486077in}}%
\pgfpathcurveto{\pgfqpoint{0.364167in}{0.500000in}}{\pgfqpoint{0.364167in}{0.513923in}}{\pgfqpoint{0.369698in}{0.527278in}}%
\pgfpathcurveto{\pgfqpoint{0.379544in}{0.537123in}}{\pgfqpoint{0.389389in}{0.546968in}}{\pgfqpoint{0.402744in}{0.552500in}}%
\pgfpathcurveto{\pgfqpoint{0.416667in}{0.552500in}}{\pgfqpoint{0.430590in}{0.552500in}}{\pgfqpoint{0.443945in}{0.546968in}}%
\pgfpathcurveto{\pgfqpoint{0.453790in}{0.537123in}}{\pgfqpoint{0.463635in}{0.527278in}}{\pgfqpoint{0.469167in}{0.513923in}}%
\pgfpathcurveto{\pgfqpoint{0.469167in}{0.500000in}}{\pgfqpoint{0.469167in}{0.486077in}}{\pgfqpoint{0.463635in}{0.472722in}}%
\pgfpathcurveto{\pgfqpoint{0.453790in}{0.462877in}}{\pgfqpoint{0.443945in}{0.453032in}}{\pgfqpoint{0.430590in}{0.447500in}}%
\pgfpathclose%
\pgfpathmoveto{\pgfqpoint{0.583333in}{0.441667in}}%
\pgfpathcurveto{\pgfqpoint{0.598804in}{0.441667in}}{\pgfqpoint{0.613642in}{0.447813in}}{\pgfqpoint{0.624581in}{0.458752in}}%
\pgfpathcurveto{\pgfqpoint{0.635520in}{0.469691in}}{\pgfqpoint{0.641667in}{0.484530in}}{\pgfqpoint{0.641667in}{0.500000in}}%
\pgfpathcurveto{\pgfqpoint{0.641667in}{0.515470in}}{\pgfqpoint{0.635520in}{0.530309in}}{\pgfqpoint{0.624581in}{0.541248in}}%
\pgfpathcurveto{\pgfqpoint{0.613642in}{0.552187in}}{\pgfqpoint{0.598804in}{0.558333in}}{\pgfqpoint{0.583333in}{0.558333in}}%
\pgfpathcurveto{\pgfqpoint{0.567863in}{0.558333in}}{\pgfqpoint{0.553025in}{0.552187in}}{\pgfqpoint{0.542085in}{0.541248in}}%
\pgfpathcurveto{\pgfqpoint{0.531146in}{0.530309in}}{\pgfqpoint{0.525000in}{0.515470in}}{\pgfqpoint{0.525000in}{0.500000in}}%
\pgfpathcurveto{\pgfqpoint{0.525000in}{0.484530in}}{\pgfqpoint{0.531146in}{0.469691in}}{\pgfqpoint{0.542085in}{0.458752in}}%
\pgfpathcurveto{\pgfqpoint{0.553025in}{0.447813in}}{\pgfqpoint{0.567863in}{0.441667in}}{\pgfqpoint{0.583333in}{0.441667in}}%
\pgfpathclose%
\pgfpathmoveto{\pgfqpoint{0.583333in}{0.447500in}}%
\pgfpathcurveto{\pgfqpoint{0.583333in}{0.447500in}}{\pgfqpoint{0.569410in}{0.447500in}}{\pgfqpoint{0.556055in}{0.453032in}}%
\pgfpathcurveto{\pgfqpoint{0.546210in}{0.462877in}}{\pgfqpoint{0.536365in}{0.472722in}}{\pgfqpoint{0.530833in}{0.486077in}}%
\pgfpathcurveto{\pgfqpoint{0.530833in}{0.500000in}}{\pgfqpoint{0.530833in}{0.513923in}}{\pgfqpoint{0.536365in}{0.527278in}}%
\pgfpathcurveto{\pgfqpoint{0.546210in}{0.537123in}}{\pgfqpoint{0.556055in}{0.546968in}}{\pgfqpoint{0.569410in}{0.552500in}}%
\pgfpathcurveto{\pgfqpoint{0.583333in}{0.552500in}}{\pgfqpoint{0.597256in}{0.552500in}}{\pgfqpoint{0.610611in}{0.546968in}}%
\pgfpathcurveto{\pgfqpoint{0.620456in}{0.537123in}}{\pgfqpoint{0.630302in}{0.527278in}}{\pgfqpoint{0.635833in}{0.513923in}}%
\pgfpathcurveto{\pgfqpoint{0.635833in}{0.500000in}}{\pgfqpoint{0.635833in}{0.486077in}}{\pgfqpoint{0.630302in}{0.472722in}}%
\pgfpathcurveto{\pgfqpoint{0.620456in}{0.462877in}}{\pgfqpoint{0.610611in}{0.453032in}}{\pgfqpoint{0.597256in}{0.447500in}}%
\pgfpathclose%
\pgfpathmoveto{\pgfqpoint{0.750000in}{0.441667in}}%
\pgfpathcurveto{\pgfqpoint{0.765470in}{0.441667in}}{\pgfqpoint{0.780309in}{0.447813in}}{\pgfqpoint{0.791248in}{0.458752in}}%
\pgfpathcurveto{\pgfqpoint{0.802187in}{0.469691in}}{\pgfqpoint{0.808333in}{0.484530in}}{\pgfqpoint{0.808333in}{0.500000in}}%
\pgfpathcurveto{\pgfqpoint{0.808333in}{0.515470in}}{\pgfqpoint{0.802187in}{0.530309in}}{\pgfqpoint{0.791248in}{0.541248in}}%
\pgfpathcurveto{\pgfqpoint{0.780309in}{0.552187in}}{\pgfqpoint{0.765470in}{0.558333in}}{\pgfqpoint{0.750000in}{0.558333in}}%
\pgfpathcurveto{\pgfqpoint{0.734530in}{0.558333in}}{\pgfqpoint{0.719691in}{0.552187in}}{\pgfqpoint{0.708752in}{0.541248in}}%
\pgfpathcurveto{\pgfqpoint{0.697813in}{0.530309in}}{\pgfqpoint{0.691667in}{0.515470in}}{\pgfqpoint{0.691667in}{0.500000in}}%
\pgfpathcurveto{\pgfqpoint{0.691667in}{0.484530in}}{\pgfqpoint{0.697813in}{0.469691in}}{\pgfqpoint{0.708752in}{0.458752in}}%
\pgfpathcurveto{\pgfqpoint{0.719691in}{0.447813in}}{\pgfqpoint{0.734530in}{0.441667in}}{\pgfqpoint{0.750000in}{0.441667in}}%
\pgfpathclose%
\pgfpathmoveto{\pgfqpoint{0.750000in}{0.447500in}}%
\pgfpathcurveto{\pgfqpoint{0.750000in}{0.447500in}}{\pgfqpoint{0.736077in}{0.447500in}}{\pgfqpoint{0.722722in}{0.453032in}}%
\pgfpathcurveto{\pgfqpoint{0.712877in}{0.462877in}}{\pgfqpoint{0.703032in}{0.472722in}}{\pgfqpoint{0.697500in}{0.486077in}}%
\pgfpathcurveto{\pgfqpoint{0.697500in}{0.500000in}}{\pgfqpoint{0.697500in}{0.513923in}}{\pgfqpoint{0.703032in}{0.527278in}}%
\pgfpathcurveto{\pgfqpoint{0.712877in}{0.537123in}}{\pgfqpoint{0.722722in}{0.546968in}}{\pgfqpoint{0.736077in}{0.552500in}}%
\pgfpathcurveto{\pgfqpoint{0.750000in}{0.552500in}}{\pgfqpoint{0.763923in}{0.552500in}}{\pgfqpoint{0.777278in}{0.546968in}}%
\pgfpathcurveto{\pgfqpoint{0.787123in}{0.537123in}}{\pgfqpoint{0.796968in}{0.527278in}}{\pgfqpoint{0.802500in}{0.513923in}}%
\pgfpathcurveto{\pgfqpoint{0.802500in}{0.500000in}}{\pgfqpoint{0.802500in}{0.486077in}}{\pgfqpoint{0.796968in}{0.472722in}}%
\pgfpathcurveto{\pgfqpoint{0.787123in}{0.462877in}}{\pgfqpoint{0.777278in}{0.453032in}}{\pgfqpoint{0.763923in}{0.447500in}}%
\pgfpathclose%
\pgfpathmoveto{\pgfqpoint{0.916667in}{0.441667in}}%
\pgfpathcurveto{\pgfqpoint{0.932137in}{0.441667in}}{\pgfqpoint{0.946975in}{0.447813in}}{\pgfqpoint{0.957915in}{0.458752in}}%
\pgfpathcurveto{\pgfqpoint{0.968854in}{0.469691in}}{\pgfqpoint{0.975000in}{0.484530in}}{\pgfqpoint{0.975000in}{0.500000in}}%
\pgfpathcurveto{\pgfqpoint{0.975000in}{0.515470in}}{\pgfqpoint{0.968854in}{0.530309in}}{\pgfqpoint{0.957915in}{0.541248in}}%
\pgfpathcurveto{\pgfqpoint{0.946975in}{0.552187in}}{\pgfqpoint{0.932137in}{0.558333in}}{\pgfqpoint{0.916667in}{0.558333in}}%
\pgfpathcurveto{\pgfqpoint{0.901196in}{0.558333in}}{\pgfqpoint{0.886358in}{0.552187in}}{\pgfqpoint{0.875419in}{0.541248in}}%
\pgfpathcurveto{\pgfqpoint{0.864480in}{0.530309in}}{\pgfqpoint{0.858333in}{0.515470in}}{\pgfqpoint{0.858333in}{0.500000in}}%
\pgfpathcurveto{\pgfqpoint{0.858333in}{0.484530in}}{\pgfqpoint{0.864480in}{0.469691in}}{\pgfqpoint{0.875419in}{0.458752in}}%
\pgfpathcurveto{\pgfqpoint{0.886358in}{0.447813in}}{\pgfqpoint{0.901196in}{0.441667in}}{\pgfqpoint{0.916667in}{0.441667in}}%
\pgfpathclose%
\pgfpathmoveto{\pgfqpoint{0.916667in}{0.447500in}}%
\pgfpathcurveto{\pgfqpoint{0.916667in}{0.447500in}}{\pgfqpoint{0.902744in}{0.447500in}}{\pgfqpoint{0.889389in}{0.453032in}}%
\pgfpathcurveto{\pgfqpoint{0.879544in}{0.462877in}}{\pgfqpoint{0.869698in}{0.472722in}}{\pgfqpoint{0.864167in}{0.486077in}}%
\pgfpathcurveto{\pgfqpoint{0.864167in}{0.500000in}}{\pgfqpoint{0.864167in}{0.513923in}}{\pgfqpoint{0.869698in}{0.527278in}}%
\pgfpathcurveto{\pgfqpoint{0.879544in}{0.537123in}}{\pgfqpoint{0.889389in}{0.546968in}}{\pgfqpoint{0.902744in}{0.552500in}}%
\pgfpathcurveto{\pgfqpoint{0.916667in}{0.552500in}}{\pgfqpoint{0.930590in}{0.552500in}}{\pgfqpoint{0.943945in}{0.546968in}}%
\pgfpathcurveto{\pgfqpoint{0.953790in}{0.537123in}}{\pgfqpoint{0.963635in}{0.527278in}}{\pgfqpoint{0.969167in}{0.513923in}}%
\pgfpathcurveto{\pgfqpoint{0.969167in}{0.500000in}}{\pgfqpoint{0.969167in}{0.486077in}}{\pgfqpoint{0.963635in}{0.472722in}}%
\pgfpathcurveto{\pgfqpoint{0.953790in}{0.462877in}}{\pgfqpoint{0.943945in}{0.453032in}}{\pgfqpoint{0.930590in}{0.447500in}}%
\pgfpathclose%
\pgfpathmoveto{\pgfqpoint{0.000000in}{0.608333in}}%
\pgfpathcurveto{\pgfqpoint{0.015470in}{0.608333in}}{\pgfqpoint{0.030309in}{0.614480in}}{\pgfqpoint{0.041248in}{0.625419in}}%
\pgfpathcurveto{\pgfqpoint{0.052187in}{0.636358in}}{\pgfqpoint{0.058333in}{0.651196in}}{\pgfqpoint{0.058333in}{0.666667in}}%
\pgfpathcurveto{\pgfqpoint{0.058333in}{0.682137in}}{\pgfqpoint{0.052187in}{0.696975in}}{\pgfqpoint{0.041248in}{0.707915in}}%
\pgfpathcurveto{\pgfqpoint{0.030309in}{0.718854in}}{\pgfqpoint{0.015470in}{0.725000in}}{\pgfqpoint{0.000000in}{0.725000in}}%
\pgfpathcurveto{\pgfqpoint{-0.015470in}{0.725000in}}{\pgfqpoint{-0.030309in}{0.718854in}}{\pgfqpoint{-0.041248in}{0.707915in}}%
\pgfpathcurveto{\pgfqpoint{-0.052187in}{0.696975in}}{\pgfqpoint{-0.058333in}{0.682137in}}{\pgfqpoint{-0.058333in}{0.666667in}}%
\pgfpathcurveto{\pgfqpoint{-0.058333in}{0.651196in}}{\pgfqpoint{-0.052187in}{0.636358in}}{\pgfqpoint{-0.041248in}{0.625419in}}%
\pgfpathcurveto{\pgfqpoint{-0.030309in}{0.614480in}}{\pgfqpoint{-0.015470in}{0.608333in}}{\pgfqpoint{0.000000in}{0.608333in}}%
\pgfpathclose%
\pgfpathmoveto{\pgfqpoint{0.000000in}{0.614167in}}%
\pgfpathcurveto{\pgfqpoint{0.000000in}{0.614167in}}{\pgfqpoint{-0.013923in}{0.614167in}}{\pgfqpoint{-0.027278in}{0.619698in}}%
\pgfpathcurveto{\pgfqpoint{-0.037123in}{0.629544in}}{\pgfqpoint{-0.046968in}{0.639389in}}{\pgfqpoint{-0.052500in}{0.652744in}}%
\pgfpathcurveto{\pgfqpoint{-0.052500in}{0.666667in}}{\pgfqpoint{-0.052500in}{0.680590in}}{\pgfqpoint{-0.046968in}{0.693945in}}%
\pgfpathcurveto{\pgfqpoint{-0.037123in}{0.703790in}}{\pgfqpoint{-0.027278in}{0.713635in}}{\pgfqpoint{-0.013923in}{0.719167in}}%
\pgfpathcurveto{\pgfqpoint{0.000000in}{0.719167in}}{\pgfqpoint{0.013923in}{0.719167in}}{\pgfqpoint{0.027278in}{0.713635in}}%
\pgfpathcurveto{\pgfqpoint{0.037123in}{0.703790in}}{\pgfqpoint{0.046968in}{0.693945in}}{\pgfqpoint{0.052500in}{0.680590in}}%
\pgfpathcurveto{\pgfqpoint{0.052500in}{0.666667in}}{\pgfqpoint{0.052500in}{0.652744in}}{\pgfqpoint{0.046968in}{0.639389in}}%
\pgfpathcurveto{\pgfqpoint{0.037123in}{0.629544in}}{\pgfqpoint{0.027278in}{0.619698in}}{\pgfqpoint{0.013923in}{0.614167in}}%
\pgfpathclose%
\pgfpathmoveto{\pgfqpoint{0.166667in}{0.608333in}}%
\pgfpathcurveto{\pgfqpoint{0.182137in}{0.608333in}}{\pgfqpoint{0.196975in}{0.614480in}}{\pgfqpoint{0.207915in}{0.625419in}}%
\pgfpathcurveto{\pgfqpoint{0.218854in}{0.636358in}}{\pgfqpoint{0.225000in}{0.651196in}}{\pgfqpoint{0.225000in}{0.666667in}}%
\pgfpathcurveto{\pgfqpoint{0.225000in}{0.682137in}}{\pgfqpoint{0.218854in}{0.696975in}}{\pgfqpoint{0.207915in}{0.707915in}}%
\pgfpathcurveto{\pgfqpoint{0.196975in}{0.718854in}}{\pgfqpoint{0.182137in}{0.725000in}}{\pgfqpoint{0.166667in}{0.725000in}}%
\pgfpathcurveto{\pgfqpoint{0.151196in}{0.725000in}}{\pgfqpoint{0.136358in}{0.718854in}}{\pgfqpoint{0.125419in}{0.707915in}}%
\pgfpathcurveto{\pgfqpoint{0.114480in}{0.696975in}}{\pgfqpoint{0.108333in}{0.682137in}}{\pgfqpoint{0.108333in}{0.666667in}}%
\pgfpathcurveto{\pgfqpoint{0.108333in}{0.651196in}}{\pgfqpoint{0.114480in}{0.636358in}}{\pgfqpoint{0.125419in}{0.625419in}}%
\pgfpathcurveto{\pgfqpoint{0.136358in}{0.614480in}}{\pgfqpoint{0.151196in}{0.608333in}}{\pgfqpoint{0.166667in}{0.608333in}}%
\pgfpathclose%
\pgfpathmoveto{\pgfqpoint{0.166667in}{0.614167in}}%
\pgfpathcurveto{\pgfqpoint{0.166667in}{0.614167in}}{\pgfqpoint{0.152744in}{0.614167in}}{\pgfqpoint{0.139389in}{0.619698in}}%
\pgfpathcurveto{\pgfqpoint{0.129544in}{0.629544in}}{\pgfqpoint{0.119698in}{0.639389in}}{\pgfqpoint{0.114167in}{0.652744in}}%
\pgfpathcurveto{\pgfqpoint{0.114167in}{0.666667in}}{\pgfqpoint{0.114167in}{0.680590in}}{\pgfqpoint{0.119698in}{0.693945in}}%
\pgfpathcurveto{\pgfqpoint{0.129544in}{0.703790in}}{\pgfqpoint{0.139389in}{0.713635in}}{\pgfqpoint{0.152744in}{0.719167in}}%
\pgfpathcurveto{\pgfqpoint{0.166667in}{0.719167in}}{\pgfqpoint{0.180590in}{0.719167in}}{\pgfqpoint{0.193945in}{0.713635in}}%
\pgfpathcurveto{\pgfqpoint{0.203790in}{0.703790in}}{\pgfqpoint{0.213635in}{0.693945in}}{\pgfqpoint{0.219167in}{0.680590in}}%
\pgfpathcurveto{\pgfqpoint{0.219167in}{0.666667in}}{\pgfqpoint{0.219167in}{0.652744in}}{\pgfqpoint{0.213635in}{0.639389in}}%
\pgfpathcurveto{\pgfqpoint{0.203790in}{0.629544in}}{\pgfqpoint{0.193945in}{0.619698in}}{\pgfqpoint{0.180590in}{0.614167in}}%
\pgfpathclose%
\pgfpathmoveto{\pgfqpoint{0.333333in}{0.608333in}}%
\pgfpathcurveto{\pgfqpoint{0.348804in}{0.608333in}}{\pgfqpoint{0.363642in}{0.614480in}}{\pgfqpoint{0.374581in}{0.625419in}}%
\pgfpathcurveto{\pgfqpoint{0.385520in}{0.636358in}}{\pgfqpoint{0.391667in}{0.651196in}}{\pgfqpoint{0.391667in}{0.666667in}}%
\pgfpathcurveto{\pgfqpoint{0.391667in}{0.682137in}}{\pgfqpoint{0.385520in}{0.696975in}}{\pgfqpoint{0.374581in}{0.707915in}}%
\pgfpathcurveto{\pgfqpoint{0.363642in}{0.718854in}}{\pgfqpoint{0.348804in}{0.725000in}}{\pgfqpoint{0.333333in}{0.725000in}}%
\pgfpathcurveto{\pgfqpoint{0.317863in}{0.725000in}}{\pgfqpoint{0.303025in}{0.718854in}}{\pgfqpoint{0.292085in}{0.707915in}}%
\pgfpathcurveto{\pgfqpoint{0.281146in}{0.696975in}}{\pgfqpoint{0.275000in}{0.682137in}}{\pgfqpoint{0.275000in}{0.666667in}}%
\pgfpathcurveto{\pgfqpoint{0.275000in}{0.651196in}}{\pgfqpoint{0.281146in}{0.636358in}}{\pgfqpoint{0.292085in}{0.625419in}}%
\pgfpathcurveto{\pgfqpoint{0.303025in}{0.614480in}}{\pgfqpoint{0.317863in}{0.608333in}}{\pgfqpoint{0.333333in}{0.608333in}}%
\pgfpathclose%
\pgfpathmoveto{\pgfqpoint{0.333333in}{0.614167in}}%
\pgfpathcurveto{\pgfqpoint{0.333333in}{0.614167in}}{\pgfqpoint{0.319410in}{0.614167in}}{\pgfqpoint{0.306055in}{0.619698in}}%
\pgfpathcurveto{\pgfqpoint{0.296210in}{0.629544in}}{\pgfqpoint{0.286365in}{0.639389in}}{\pgfqpoint{0.280833in}{0.652744in}}%
\pgfpathcurveto{\pgfqpoint{0.280833in}{0.666667in}}{\pgfqpoint{0.280833in}{0.680590in}}{\pgfqpoint{0.286365in}{0.693945in}}%
\pgfpathcurveto{\pgfqpoint{0.296210in}{0.703790in}}{\pgfqpoint{0.306055in}{0.713635in}}{\pgfqpoint{0.319410in}{0.719167in}}%
\pgfpathcurveto{\pgfqpoint{0.333333in}{0.719167in}}{\pgfqpoint{0.347256in}{0.719167in}}{\pgfqpoint{0.360611in}{0.713635in}}%
\pgfpathcurveto{\pgfqpoint{0.370456in}{0.703790in}}{\pgfqpoint{0.380302in}{0.693945in}}{\pgfqpoint{0.385833in}{0.680590in}}%
\pgfpathcurveto{\pgfqpoint{0.385833in}{0.666667in}}{\pgfqpoint{0.385833in}{0.652744in}}{\pgfqpoint{0.380302in}{0.639389in}}%
\pgfpathcurveto{\pgfqpoint{0.370456in}{0.629544in}}{\pgfqpoint{0.360611in}{0.619698in}}{\pgfqpoint{0.347256in}{0.614167in}}%
\pgfpathclose%
\pgfpathmoveto{\pgfqpoint{0.500000in}{0.608333in}}%
\pgfpathcurveto{\pgfqpoint{0.515470in}{0.608333in}}{\pgfqpoint{0.530309in}{0.614480in}}{\pgfqpoint{0.541248in}{0.625419in}}%
\pgfpathcurveto{\pgfqpoint{0.552187in}{0.636358in}}{\pgfqpoint{0.558333in}{0.651196in}}{\pgfqpoint{0.558333in}{0.666667in}}%
\pgfpathcurveto{\pgfqpoint{0.558333in}{0.682137in}}{\pgfqpoint{0.552187in}{0.696975in}}{\pgfqpoint{0.541248in}{0.707915in}}%
\pgfpathcurveto{\pgfqpoint{0.530309in}{0.718854in}}{\pgfqpoint{0.515470in}{0.725000in}}{\pgfqpoint{0.500000in}{0.725000in}}%
\pgfpathcurveto{\pgfqpoint{0.484530in}{0.725000in}}{\pgfqpoint{0.469691in}{0.718854in}}{\pgfqpoint{0.458752in}{0.707915in}}%
\pgfpathcurveto{\pgfqpoint{0.447813in}{0.696975in}}{\pgfqpoint{0.441667in}{0.682137in}}{\pgfqpoint{0.441667in}{0.666667in}}%
\pgfpathcurveto{\pgfqpoint{0.441667in}{0.651196in}}{\pgfqpoint{0.447813in}{0.636358in}}{\pgfqpoint{0.458752in}{0.625419in}}%
\pgfpathcurveto{\pgfqpoint{0.469691in}{0.614480in}}{\pgfqpoint{0.484530in}{0.608333in}}{\pgfqpoint{0.500000in}{0.608333in}}%
\pgfpathclose%
\pgfpathmoveto{\pgfqpoint{0.500000in}{0.614167in}}%
\pgfpathcurveto{\pgfqpoint{0.500000in}{0.614167in}}{\pgfqpoint{0.486077in}{0.614167in}}{\pgfqpoint{0.472722in}{0.619698in}}%
\pgfpathcurveto{\pgfqpoint{0.462877in}{0.629544in}}{\pgfqpoint{0.453032in}{0.639389in}}{\pgfqpoint{0.447500in}{0.652744in}}%
\pgfpathcurveto{\pgfqpoint{0.447500in}{0.666667in}}{\pgfqpoint{0.447500in}{0.680590in}}{\pgfqpoint{0.453032in}{0.693945in}}%
\pgfpathcurveto{\pgfqpoint{0.462877in}{0.703790in}}{\pgfqpoint{0.472722in}{0.713635in}}{\pgfqpoint{0.486077in}{0.719167in}}%
\pgfpathcurveto{\pgfqpoint{0.500000in}{0.719167in}}{\pgfqpoint{0.513923in}{0.719167in}}{\pgfqpoint{0.527278in}{0.713635in}}%
\pgfpathcurveto{\pgfqpoint{0.537123in}{0.703790in}}{\pgfqpoint{0.546968in}{0.693945in}}{\pgfqpoint{0.552500in}{0.680590in}}%
\pgfpathcurveto{\pgfqpoint{0.552500in}{0.666667in}}{\pgfqpoint{0.552500in}{0.652744in}}{\pgfqpoint{0.546968in}{0.639389in}}%
\pgfpathcurveto{\pgfqpoint{0.537123in}{0.629544in}}{\pgfqpoint{0.527278in}{0.619698in}}{\pgfqpoint{0.513923in}{0.614167in}}%
\pgfpathclose%
\pgfpathmoveto{\pgfqpoint{0.666667in}{0.608333in}}%
\pgfpathcurveto{\pgfqpoint{0.682137in}{0.608333in}}{\pgfqpoint{0.696975in}{0.614480in}}{\pgfqpoint{0.707915in}{0.625419in}}%
\pgfpathcurveto{\pgfqpoint{0.718854in}{0.636358in}}{\pgfqpoint{0.725000in}{0.651196in}}{\pgfqpoint{0.725000in}{0.666667in}}%
\pgfpathcurveto{\pgfqpoint{0.725000in}{0.682137in}}{\pgfqpoint{0.718854in}{0.696975in}}{\pgfqpoint{0.707915in}{0.707915in}}%
\pgfpathcurveto{\pgfqpoint{0.696975in}{0.718854in}}{\pgfqpoint{0.682137in}{0.725000in}}{\pgfqpoint{0.666667in}{0.725000in}}%
\pgfpathcurveto{\pgfqpoint{0.651196in}{0.725000in}}{\pgfqpoint{0.636358in}{0.718854in}}{\pgfqpoint{0.625419in}{0.707915in}}%
\pgfpathcurveto{\pgfqpoint{0.614480in}{0.696975in}}{\pgfqpoint{0.608333in}{0.682137in}}{\pgfqpoint{0.608333in}{0.666667in}}%
\pgfpathcurveto{\pgfqpoint{0.608333in}{0.651196in}}{\pgfqpoint{0.614480in}{0.636358in}}{\pgfqpoint{0.625419in}{0.625419in}}%
\pgfpathcurveto{\pgfqpoint{0.636358in}{0.614480in}}{\pgfqpoint{0.651196in}{0.608333in}}{\pgfqpoint{0.666667in}{0.608333in}}%
\pgfpathclose%
\pgfpathmoveto{\pgfqpoint{0.666667in}{0.614167in}}%
\pgfpathcurveto{\pgfqpoint{0.666667in}{0.614167in}}{\pgfqpoint{0.652744in}{0.614167in}}{\pgfqpoint{0.639389in}{0.619698in}}%
\pgfpathcurveto{\pgfqpoint{0.629544in}{0.629544in}}{\pgfqpoint{0.619698in}{0.639389in}}{\pgfqpoint{0.614167in}{0.652744in}}%
\pgfpathcurveto{\pgfqpoint{0.614167in}{0.666667in}}{\pgfqpoint{0.614167in}{0.680590in}}{\pgfqpoint{0.619698in}{0.693945in}}%
\pgfpathcurveto{\pgfqpoint{0.629544in}{0.703790in}}{\pgfqpoint{0.639389in}{0.713635in}}{\pgfqpoint{0.652744in}{0.719167in}}%
\pgfpathcurveto{\pgfqpoint{0.666667in}{0.719167in}}{\pgfqpoint{0.680590in}{0.719167in}}{\pgfqpoint{0.693945in}{0.713635in}}%
\pgfpathcurveto{\pgfqpoint{0.703790in}{0.703790in}}{\pgfqpoint{0.713635in}{0.693945in}}{\pgfqpoint{0.719167in}{0.680590in}}%
\pgfpathcurveto{\pgfqpoint{0.719167in}{0.666667in}}{\pgfqpoint{0.719167in}{0.652744in}}{\pgfqpoint{0.713635in}{0.639389in}}%
\pgfpathcurveto{\pgfqpoint{0.703790in}{0.629544in}}{\pgfqpoint{0.693945in}{0.619698in}}{\pgfqpoint{0.680590in}{0.614167in}}%
\pgfpathclose%
\pgfpathmoveto{\pgfqpoint{0.833333in}{0.608333in}}%
\pgfpathcurveto{\pgfqpoint{0.848804in}{0.608333in}}{\pgfqpoint{0.863642in}{0.614480in}}{\pgfqpoint{0.874581in}{0.625419in}}%
\pgfpathcurveto{\pgfqpoint{0.885520in}{0.636358in}}{\pgfqpoint{0.891667in}{0.651196in}}{\pgfqpoint{0.891667in}{0.666667in}}%
\pgfpathcurveto{\pgfqpoint{0.891667in}{0.682137in}}{\pgfqpoint{0.885520in}{0.696975in}}{\pgfqpoint{0.874581in}{0.707915in}}%
\pgfpathcurveto{\pgfqpoint{0.863642in}{0.718854in}}{\pgfqpoint{0.848804in}{0.725000in}}{\pgfqpoint{0.833333in}{0.725000in}}%
\pgfpathcurveto{\pgfqpoint{0.817863in}{0.725000in}}{\pgfqpoint{0.803025in}{0.718854in}}{\pgfqpoint{0.792085in}{0.707915in}}%
\pgfpathcurveto{\pgfqpoint{0.781146in}{0.696975in}}{\pgfqpoint{0.775000in}{0.682137in}}{\pgfqpoint{0.775000in}{0.666667in}}%
\pgfpathcurveto{\pgfqpoint{0.775000in}{0.651196in}}{\pgfqpoint{0.781146in}{0.636358in}}{\pgfqpoint{0.792085in}{0.625419in}}%
\pgfpathcurveto{\pgfqpoint{0.803025in}{0.614480in}}{\pgfqpoint{0.817863in}{0.608333in}}{\pgfqpoint{0.833333in}{0.608333in}}%
\pgfpathclose%
\pgfpathmoveto{\pgfqpoint{0.833333in}{0.614167in}}%
\pgfpathcurveto{\pgfqpoint{0.833333in}{0.614167in}}{\pgfqpoint{0.819410in}{0.614167in}}{\pgfqpoint{0.806055in}{0.619698in}}%
\pgfpathcurveto{\pgfqpoint{0.796210in}{0.629544in}}{\pgfqpoint{0.786365in}{0.639389in}}{\pgfqpoint{0.780833in}{0.652744in}}%
\pgfpathcurveto{\pgfqpoint{0.780833in}{0.666667in}}{\pgfqpoint{0.780833in}{0.680590in}}{\pgfqpoint{0.786365in}{0.693945in}}%
\pgfpathcurveto{\pgfqpoint{0.796210in}{0.703790in}}{\pgfqpoint{0.806055in}{0.713635in}}{\pgfqpoint{0.819410in}{0.719167in}}%
\pgfpathcurveto{\pgfqpoint{0.833333in}{0.719167in}}{\pgfqpoint{0.847256in}{0.719167in}}{\pgfqpoint{0.860611in}{0.713635in}}%
\pgfpathcurveto{\pgfqpoint{0.870456in}{0.703790in}}{\pgfqpoint{0.880302in}{0.693945in}}{\pgfqpoint{0.885833in}{0.680590in}}%
\pgfpathcurveto{\pgfqpoint{0.885833in}{0.666667in}}{\pgfqpoint{0.885833in}{0.652744in}}{\pgfqpoint{0.880302in}{0.639389in}}%
\pgfpathcurveto{\pgfqpoint{0.870456in}{0.629544in}}{\pgfqpoint{0.860611in}{0.619698in}}{\pgfqpoint{0.847256in}{0.614167in}}%
\pgfpathclose%
\pgfpathmoveto{\pgfqpoint{1.000000in}{0.608333in}}%
\pgfpathcurveto{\pgfqpoint{1.015470in}{0.608333in}}{\pgfqpoint{1.030309in}{0.614480in}}{\pgfqpoint{1.041248in}{0.625419in}}%
\pgfpathcurveto{\pgfqpoint{1.052187in}{0.636358in}}{\pgfqpoint{1.058333in}{0.651196in}}{\pgfqpoint{1.058333in}{0.666667in}}%
\pgfpathcurveto{\pgfqpoint{1.058333in}{0.682137in}}{\pgfqpoint{1.052187in}{0.696975in}}{\pgfqpoint{1.041248in}{0.707915in}}%
\pgfpathcurveto{\pgfqpoint{1.030309in}{0.718854in}}{\pgfqpoint{1.015470in}{0.725000in}}{\pgfqpoint{1.000000in}{0.725000in}}%
\pgfpathcurveto{\pgfqpoint{0.984530in}{0.725000in}}{\pgfqpoint{0.969691in}{0.718854in}}{\pgfqpoint{0.958752in}{0.707915in}}%
\pgfpathcurveto{\pgfqpoint{0.947813in}{0.696975in}}{\pgfqpoint{0.941667in}{0.682137in}}{\pgfqpoint{0.941667in}{0.666667in}}%
\pgfpathcurveto{\pgfqpoint{0.941667in}{0.651196in}}{\pgfqpoint{0.947813in}{0.636358in}}{\pgfqpoint{0.958752in}{0.625419in}}%
\pgfpathcurveto{\pgfqpoint{0.969691in}{0.614480in}}{\pgfqpoint{0.984530in}{0.608333in}}{\pgfqpoint{1.000000in}{0.608333in}}%
\pgfpathclose%
\pgfpathmoveto{\pgfqpoint{1.000000in}{0.614167in}}%
\pgfpathcurveto{\pgfqpoint{1.000000in}{0.614167in}}{\pgfqpoint{0.986077in}{0.614167in}}{\pgfqpoint{0.972722in}{0.619698in}}%
\pgfpathcurveto{\pgfqpoint{0.962877in}{0.629544in}}{\pgfqpoint{0.953032in}{0.639389in}}{\pgfqpoint{0.947500in}{0.652744in}}%
\pgfpathcurveto{\pgfqpoint{0.947500in}{0.666667in}}{\pgfqpoint{0.947500in}{0.680590in}}{\pgfqpoint{0.953032in}{0.693945in}}%
\pgfpathcurveto{\pgfqpoint{0.962877in}{0.703790in}}{\pgfqpoint{0.972722in}{0.713635in}}{\pgfqpoint{0.986077in}{0.719167in}}%
\pgfpathcurveto{\pgfqpoint{1.000000in}{0.719167in}}{\pgfqpoint{1.013923in}{0.719167in}}{\pgfqpoint{1.027278in}{0.713635in}}%
\pgfpathcurveto{\pgfqpoint{1.037123in}{0.703790in}}{\pgfqpoint{1.046968in}{0.693945in}}{\pgfqpoint{1.052500in}{0.680590in}}%
\pgfpathcurveto{\pgfqpoint{1.052500in}{0.666667in}}{\pgfqpoint{1.052500in}{0.652744in}}{\pgfqpoint{1.046968in}{0.639389in}}%
\pgfpathcurveto{\pgfqpoint{1.037123in}{0.629544in}}{\pgfqpoint{1.027278in}{0.619698in}}{\pgfqpoint{1.013923in}{0.614167in}}%
\pgfpathclose%
\pgfpathmoveto{\pgfqpoint{0.083333in}{0.775000in}}%
\pgfpathcurveto{\pgfqpoint{0.098804in}{0.775000in}}{\pgfqpoint{0.113642in}{0.781146in}}{\pgfqpoint{0.124581in}{0.792085in}}%
\pgfpathcurveto{\pgfqpoint{0.135520in}{0.803025in}}{\pgfqpoint{0.141667in}{0.817863in}}{\pgfqpoint{0.141667in}{0.833333in}}%
\pgfpathcurveto{\pgfqpoint{0.141667in}{0.848804in}}{\pgfqpoint{0.135520in}{0.863642in}}{\pgfqpoint{0.124581in}{0.874581in}}%
\pgfpathcurveto{\pgfqpoint{0.113642in}{0.885520in}}{\pgfqpoint{0.098804in}{0.891667in}}{\pgfqpoint{0.083333in}{0.891667in}}%
\pgfpathcurveto{\pgfqpoint{0.067863in}{0.891667in}}{\pgfqpoint{0.053025in}{0.885520in}}{\pgfqpoint{0.042085in}{0.874581in}}%
\pgfpathcurveto{\pgfqpoint{0.031146in}{0.863642in}}{\pgfqpoint{0.025000in}{0.848804in}}{\pgfqpoint{0.025000in}{0.833333in}}%
\pgfpathcurveto{\pgfqpoint{0.025000in}{0.817863in}}{\pgfqpoint{0.031146in}{0.803025in}}{\pgfqpoint{0.042085in}{0.792085in}}%
\pgfpathcurveto{\pgfqpoint{0.053025in}{0.781146in}}{\pgfqpoint{0.067863in}{0.775000in}}{\pgfqpoint{0.083333in}{0.775000in}}%
\pgfpathclose%
\pgfpathmoveto{\pgfqpoint{0.083333in}{0.780833in}}%
\pgfpathcurveto{\pgfqpoint{0.083333in}{0.780833in}}{\pgfqpoint{0.069410in}{0.780833in}}{\pgfqpoint{0.056055in}{0.786365in}}%
\pgfpathcurveto{\pgfqpoint{0.046210in}{0.796210in}}{\pgfqpoint{0.036365in}{0.806055in}}{\pgfqpoint{0.030833in}{0.819410in}}%
\pgfpathcurveto{\pgfqpoint{0.030833in}{0.833333in}}{\pgfqpoint{0.030833in}{0.847256in}}{\pgfqpoint{0.036365in}{0.860611in}}%
\pgfpathcurveto{\pgfqpoint{0.046210in}{0.870456in}}{\pgfqpoint{0.056055in}{0.880302in}}{\pgfqpoint{0.069410in}{0.885833in}}%
\pgfpathcurveto{\pgfqpoint{0.083333in}{0.885833in}}{\pgfqpoint{0.097256in}{0.885833in}}{\pgfqpoint{0.110611in}{0.880302in}}%
\pgfpathcurveto{\pgfqpoint{0.120456in}{0.870456in}}{\pgfqpoint{0.130302in}{0.860611in}}{\pgfqpoint{0.135833in}{0.847256in}}%
\pgfpathcurveto{\pgfqpoint{0.135833in}{0.833333in}}{\pgfqpoint{0.135833in}{0.819410in}}{\pgfqpoint{0.130302in}{0.806055in}}%
\pgfpathcurveto{\pgfqpoint{0.120456in}{0.796210in}}{\pgfqpoint{0.110611in}{0.786365in}}{\pgfqpoint{0.097256in}{0.780833in}}%
\pgfpathclose%
\pgfpathmoveto{\pgfqpoint{0.250000in}{0.775000in}}%
\pgfpathcurveto{\pgfqpoint{0.265470in}{0.775000in}}{\pgfqpoint{0.280309in}{0.781146in}}{\pgfqpoint{0.291248in}{0.792085in}}%
\pgfpathcurveto{\pgfqpoint{0.302187in}{0.803025in}}{\pgfqpoint{0.308333in}{0.817863in}}{\pgfqpoint{0.308333in}{0.833333in}}%
\pgfpathcurveto{\pgfqpoint{0.308333in}{0.848804in}}{\pgfqpoint{0.302187in}{0.863642in}}{\pgfqpoint{0.291248in}{0.874581in}}%
\pgfpathcurveto{\pgfqpoint{0.280309in}{0.885520in}}{\pgfqpoint{0.265470in}{0.891667in}}{\pgfqpoint{0.250000in}{0.891667in}}%
\pgfpathcurveto{\pgfqpoint{0.234530in}{0.891667in}}{\pgfqpoint{0.219691in}{0.885520in}}{\pgfqpoint{0.208752in}{0.874581in}}%
\pgfpathcurveto{\pgfqpoint{0.197813in}{0.863642in}}{\pgfqpoint{0.191667in}{0.848804in}}{\pgfqpoint{0.191667in}{0.833333in}}%
\pgfpathcurveto{\pgfqpoint{0.191667in}{0.817863in}}{\pgfqpoint{0.197813in}{0.803025in}}{\pgfqpoint{0.208752in}{0.792085in}}%
\pgfpathcurveto{\pgfqpoint{0.219691in}{0.781146in}}{\pgfqpoint{0.234530in}{0.775000in}}{\pgfqpoint{0.250000in}{0.775000in}}%
\pgfpathclose%
\pgfpathmoveto{\pgfqpoint{0.250000in}{0.780833in}}%
\pgfpathcurveto{\pgfqpoint{0.250000in}{0.780833in}}{\pgfqpoint{0.236077in}{0.780833in}}{\pgfqpoint{0.222722in}{0.786365in}}%
\pgfpathcurveto{\pgfqpoint{0.212877in}{0.796210in}}{\pgfqpoint{0.203032in}{0.806055in}}{\pgfqpoint{0.197500in}{0.819410in}}%
\pgfpathcurveto{\pgfqpoint{0.197500in}{0.833333in}}{\pgfqpoint{0.197500in}{0.847256in}}{\pgfqpoint{0.203032in}{0.860611in}}%
\pgfpathcurveto{\pgfqpoint{0.212877in}{0.870456in}}{\pgfqpoint{0.222722in}{0.880302in}}{\pgfqpoint{0.236077in}{0.885833in}}%
\pgfpathcurveto{\pgfqpoint{0.250000in}{0.885833in}}{\pgfqpoint{0.263923in}{0.885833in}}{\pgfqpoint{0.277278in}{0.880302in}}%
\pgfpathcurveto{\pgfqpoint{0.287123in}{0.870456in}}{\pgfqpoint{0.296968in}{0.860611in}}{\pgfqpoint{0.302500in}{0.847256in}}%
\pgfpathcurveto{\pgfqpoint{0.302500in}{0.833333in}}{\pgfqpoint{0.302500in}{0.819410in}}{\pgfqpoint{0.296968in}{0.806055in}}%
\pgfpathcurveto{\pgfqpoint{0.287123in}{0.796210in}}{\pgfqpoint{0.277278in}{0.786365in}}{\pgfqpoint{0.263923in}{0.780833in}}%
\pgfpathclose%
\pgfpathmoveto{\pgfqpoint{0.416667in}{0.775000in}}%
\pgfpathcurveto{\pgfqpoint{0.432137in}{0.775000in}}{\pgfqpoint{0.446975in}{0.781146in}}{\pgfqpoint{0.457915in}{0.792085in}}%
\pgfpathcurveto{\pgfqpoint{0.468854in}{0.803025in}}{\pgfqpoint{0.475000in}{0.817863in}}{\pgfqpoint{0.475000in}{0.833333in}}%
\pgfpathcurveto{\pgfqpoint{0.475000in}{0.848804in}}{\pgfqpoint{0.468854in}{0.863642in}}{\pgfqpoint{0.457915in}{0.874581in}}%
\pgfpathcurveto{\pgfqpoint{0.446975in}{0.885520in}}{\pgfqpoint{0.432137in}{0.891667in}}{\pgfqpoint{0.416667in}{0.891667in}}%
\pgfpathcurveto{\pgfqpoint{0.401196in}{0.891667in}}{\pgfqpoint{0.386358in}{0.885520in}}{\pgfqpoint{0.375419in}{0.874581in}}%
\pgfpathcurveto{\pgfqpoint{0.364480in}{0.863642in}}{\pgfqpoint{0.358333in}{0.848804in}}{\pgfqpoint{0.358333in}{0.833333in}}%
\pgfpathcurveto{\pgfqpoint{0.358333in}{0.817863in}}{\pgfqpoint{0.364480in}{0.803025in}}{\pgfqpoint{0.375419in}{0.792085in}}%
\pgfpathcurveto{\pgfqpoint{0.386358in}{0.781146in}}{\pgfqpoint{0.401196in}{0.775000in}}{\pgfqpoint{0.416667in}{0.775000in}}%
\pgfpathclose%
\pgfpathmoveto{\pgfqpoint{0.416667in}{0.780833in}}%
\pgfpathcurveto{\pgfqpoint{0.416667in}{0.780833in}}{\pgfqpoint{0.402744in}{0.780833in}}{\pgfqpoint{0.389389in}{0.786365in}}%
\pgfpathcurveto{\pgfqpoint{0.379544in}{0.796210in}}{\pgfqpoint{0.369698in}{0.806055in}}{\pgfqpoint{0.364167in}{0.819410in}}%
\pgfpathcurveto{\pgfqpoint{0.364167in}{0.833333in}}{\pgfqpoint{0.364167in}{0.847256in}}{\pgfqpoint{0.369698in}{0.860611in}}%
\pgfpathcurveto{\pgfqpoint{0.379544in}{0.870456in}}{\pgfqpoint{0.389389in}{0.880302in}}{\pgfqpoint{0.402744in}{0.885833in}}%
\pgfpathcurveto{\pgfqpoint{0.416667in}{0.885833in}}{\pgfqpoint{0.430590in}{0.885833in}}{\pgfqpoint{0.443945in}{0.880302in}}%
\pgfpathcurveto{\pgfqpoint{0.453790in}{0.870456in}}{\pgfqpoint{0.463635in}{0.860611in}}{\pgfqpoint{0.469167in}{0.847256in}}%
\pgfpathcurveto{\pgfqpoint{0.469167in}{0.833333in}}{\pgfqpoint{0.469167in}{0.819410in}}{\pgfqpoint{0.463635in}{0.806055in}}%
\pgfpathcurveto{\pgfqpoint{0.453790in}{0.796210in}}{\pgfqpoint{0.443945in}{0.786365in}}{\pgfqpoint{0.430590in}{0.780833in}}%
\pgfpathclose%
\pgfpathmoveto{\pgfqpoint{0.583333in}{0.775000in}}%
\pgfpathcurveto{\pgfqpoint{0.598804in}{0.775000in}}{\pgfqpoint{0.613642in}{0.781146in}}{\pgfqpoint{0.624581in}{0.792085in}}%
\pgfpathcurveto{\pgfqpoint{0.635520in}{0.803025in}}{\pgfqpoint{0.641667in}{0.817863in}}{\pgfqpoint{0.641667in}{0.833333in}}%
\pgfpathcurveto{\pgfqpoint{0.641667in}{0.848804in}}{\pgfqpoint{0.635520in}{0.863642in}}{\pgfqpoint{0.624581in}{0.874581in}}%
\pgfpathcurveto{\pgfqpoint{0.613642in}{0.885520in}}{\pgfqpoint{0.598804in}{0.891667in}}{\pgfqpoint{0.583333in}{0.891667in}}%
\pgfpathcurveto{\pgfqpoint{0.567863in}{0.891667in}}{\pgfqpoint{0.553025in}{0.885520in}}{\pgfqpoint{0.542085in}{0.874581in}}%
\pgfpathcurveto{\pgfqpoint{0.531146in}{0.863642in}}{\pgfqpoint{0.525000in}{0.848804in}}{\pgfqpoint{0.525000in}{0.833333in}}%
\pgfpathcurveto{\pgfqpoint{0.525000in}{0.817863in}}{\pgfqpoint{0.531146in}{0.803025in}}{\pgfqpoint{0.542085in}{0.792085in}}%
\pgfpathcurveto{\pgfqpoint{0.553025in}{0.781146in}}{\pgfqpoint{0.567863in}{0.775000in}}{\pgfqpoint{0.583333in}{0.775000in}}%
\pgfpathclose%
\pgfpathmoveto{\pgfqpoint{0.583333in}{0.780833in}}%
\pgfpathcurveto{\pgfqpoint{0.583333in}{0.780833in}}{\pgfqpoint{0.569410in}{0.780833in}}{\pgfqpoint{0.556055in}{0.786365in}}%
\pgfpathcurveto{\pgfqpoint{0.546210in}{0.796210in}}{\pgfqpoint{0.536365in}{0.806055in}}{\pgfqpoint{0.530833in}{0.819410in}}%
\pgfpathcurveto{\pgfqpoint{0.530833in}{0.833333in}}{\pgfqpoint{0.530833in}{0.847256in}}{\pgfqpoint{0.536365in}{0.860611in}}%
\pgfpathcurveto{\pgfqpoint{0.546210in}{0.870456in}}{\pgfqpoint{0.556055in}{0.880302in}}{\pgfqpoint{0.569410in}{0.885833in}}%
\pgfpathcurveto{\pgfqpoint{0.583333in}{0.885833in}}{\pgfqpoint{0.597256in}{0.885833in}}{\pgfqpoint{0.610611in}{0.880302in}}%
\pgfpathcurveto{\pgfqpoint{0.620456in}{0.870456in}}{\pgfqpoint{0.630302in}{0.860611in}}{\pgfqpoint{0.635833in}{0.847256in}}%
\pgfpathcurveto{\pgfqpoint{0.635833in}{0.833333in}}{\pgfqpoint{0.635833in}{0.819410in}}{\pgfqpoint{0.630302in}{0.806055in}}%
\pgfpathcurveto{\pgfqpoint{0.620456in}{0.796210in}}{\pgfqpoint{0.610611in}{0.786365in}}{\pgfqpoint{0.597256in}{0.780833in}}%
\pgfpathclose%
\pgfpathmoveto{\pgfqpoint{0.750000in}{0.775000in}}%
\pgfpathcurveto{\pgfqpoint{0.765470in}{0.775000in}}{\pgfqpoint{0.780309in}{0.781146in}}{\pgfqpoint{0.791248in}{0.792085in}}%
\pgfpathcurveto{\pgfqpoint{0.802187in}{0.803025in}}{\pgfqpoint{0.808333in}{0.817863in}}{\pgfqpoint{0.808333in}{0.833333in}}%
\pgfpathcurveto{\pgfqpoint{0.808333in}{0.848804in}}{\pgfqpoint{0.802187in}{0.863642in}}{\pgfqpoint{0.791248in}{0.874581in}}%
\pgfpathcurveto{\pgfqpoint{0.780309in}{0.885520in}}{\pgfqpoint{0.765470in}{0.891667in}}{\pgfqpoint{0.750000in}{0.891667in}}%
\pgfpathcurveto{\pgfqpoint{0.734530in}{0.891667in}}{\pgfqpoint{0.719691in}{0.885520in}}{\pgfqpoint{0.708752in}{0.874581in}}%
\pgfpathcurveto{\pgfqpoint{0.697813in}{0.863642in}}{\pgfqpoint{0.691667in}{0.848804in}}{\pgfqpoint{0.691667in}{0.833333in}}%
\pgfpathcurveto{\pgfqpoint{0.691667in}{0.817863in}}{\pgfqpoint{0.697813in}{0.803025in}}{\pgfqpoint{0.708752in}{0.792085in}}%
\pgfpathcurveto{\pgfqpoint{0.719691in}{0.781146in}}{\pgfqpoint{0.734530in}{0.775000in}}{\pgfqpoint{0.750000in}{0.775000in}}%
\pgfpathclose%
\pgfpathmoveto{\pgfqpoint{0.750000in}{0.780833in}}%
\pgfpathcurveto{\pgfqpoint{0.750000in}{0.780833in}}{\pgfqpoint{0.736077in}{0.780833in}}{\pgfqpoint{0.722722in}{0.786365in}}%
\pgfpathcurveto{\pgfqpoint{0.712877in}{0.796210in}}{\pgfqpoint{0.703032in}{0.806055in}}{\pgfqpoint{0.697500in}{0.819410in}}%
\pgfpathcurveto{\pgfqpoint{0.697500in}{0.833333in}}{\pgfqpoint{0.697500in}{0.847256in}}{\pgfqpoint{0.703032in}{0.860611in}}%
\pgfpathcurveto{\pgfqpoint{0.712877in}{0.870456in}}{\pgfqpoint{0.722722in}{0.880302in}}{\pgfqpoint{0.736077in}{0.885833in}}%
\pgfpathcurveto{\pgfqpoint{0.750000in}{0.885833in}}{\pgfqpoint{0.763923in}{0.885833in}}{\pgfqpoint{0.777278in}{0.880302in}}%
\pgfpathcurveto{\pgfqpoint{0.787123in}{0.870456in}}{\pgfqpoint{0.796968in}{0.860611in}}{\pgfqpoint{0.802500in}{0.847256in}}%
\pgfpathcurveto{\pgfqpoint{0.802500in}{0.833333in}}{\pgfqpoint{0.802500in}{0.819410in}}{\pgfqpoint{0.796968in}{0.806055in}}%
\pgfpathcurveto{\pgfqpoint{0.787123in}{0.796210in}}{\pgfqpoint{0.777278in}{0.786365in}}{\pgfqpoint{0.763923in}{0.780833in}}%
\pgfpathclose%
\pgfpathmoveto{\pgfqpoint{0.916667in}{0.775000in}}%
\pgfpathcurveto{\pgfqpoint{0.932137in}{0.775000in}}{\pgfqpoint{0.946975in}{0.781146in}}{\pgfqpoint{0.957915in}{0.792085in}}%
\pgfpathcurveto{\pgfqpoint{0.968854in}{0.803025in}}{\pgfqpoint{0.975000in}{0.817863in}}{\pgfqpoint{0.975000in}{0.833333in}}%
\pgfpathcurveto{\pgfqpoint{0.975000in}{0.848804in}}{\pgfqpoint{0.968854in}{0.863642in}}{\pgfqpoint{0.957915in}{0.874581in}}%
\pgfpathcurveto{\pgfqpoint{0.946975in}{0.885520in}}{\pgfqpoint{0.932137in}{0.891667in}}{\pgfqpoint{0.916667in}{0.891667in}}%
\pgfpathcurveto{\pgfqpoint{0.901196in}{0.891667in}}{\pgfqpoint{0.886358in}{0.885520in}}{\pgfqpoint{0.875419in}{0.874581in}}%
\pgfpathcurveto{\pgfqpoint{0.864480in}{0.863642in}}{\pgfqpoint{0.858333in}{0.848804in}}{\pgfqpoint{0.858333in}{0.833333in}}%
\pgfpathcurveto{\pgfqpoint{0.858333in}{0.817863in}}{\pgfqpoint{0.864480in}{0.803025in}}{\pgfqpoint{0.875419in}{0.792085in}}%
\pgfpathcurveto{\pgfqpoint{0.886358in}{0.781146in}}{\pgfqpoint{0.901196in}{0.775000in}}{\pgfqpoint{0.916667in}{0.775000in}}%
\pgfpathclose%
\pgfpathmoveto{\pgfqpoint{0.916667in}{0.780833in}}%
\pgfpathcurveto{\pgfqpoint{0.916667in}{0.780833in}}{\pgfqpoint{0.902744in}{0.780833in}}{\pgfqpoint{0.889389in}{0.786365in}}%
\pgfpathcurveto{\pgfqpoint{0.879544in}{0.796210in}}{\pgfqpoint{0.869698in}{0.806055in}}{\pgfqpoint{0.864167in}{0.819410in}}%
\pgfpathcurveto{\pgfqpoint{0.864167in}{0.833333in}}{\pgfqpoint{0.864167in}{0.847256in}}{\pgfqpoint{0.869698in}{0.860611in}}%
\pgfpathcurveto{\pgfqpoint{0.879544in}{0.870456in}}{\pgfqpoint{0.889389in}{0.880302in}}{\pgfqpoint{0.902744in}{0.885833in}}%
\pgfpathcurveto{\pgfqpoint{0.916667in}{0.885833in}}{\pgfqpoint{0.930590in}{0.885833in}}{\pgfqpoint{0.943945in}{0.880302in}}%
\pgfpathcurveto{\pgfqpoint{0.953790in}{0.870456in}}{\pgfqpoint{0.963635in}{0.860611in}}{\pgfqpoint{0.969167in}{0.847256in}}%
\pgfpathcurveto{\pgfqpoint{0.969167in}{0.833333in}}{\pgfqpoint{0.969167in}{0.819410in}}{\pgfqpoint{0.963635in}{0.806055in}}%
\pgfpathcurveto{\pgfqpoint{0.953790in}{0.796210in}}{\pgfqpoint{0.943945in}{0.786365in}}{\pgfqpoint{0.930590in}{0.780833in}}%
\pgfpathclose%
\pgfpathmoveto{\pgfqpoint{0.000000in}{0.941667in}}%
\pgfpathcurveto{\pgfqpoint{0.015470in}{0.941667in}}{\pgfqpoint{0.030309in}{0.947813in}}{\pgfqpoint{0.041248in}{0.958752in}}%
\pgfpathcurveto{\pgfqpoint{0.052187in}{0.969691in}}{\pgfqpoint{0.058333in}{0.984530in}}{\pgfqpoint{0.058333in}{1.000000in}}%
\pgfpathcurveto{\pgfqpoint{0.058333in}{1.015470in}}{\pgfqpoint{0.052187in}{1.030309in}}{\pgfqpoint{0.041248in}{1.041248in}}%
\pgfpathcurveto{\pgfqpoint{0.030309in}{1.052187in}}{\pgfqpoint{0.015470in}{1.058333in}}{\pgfqpoint{0.000000in}{1.058333in}}%
\pgfpathcurveto{\pgfqpoint{-0.015470in}{1.058333in}}{\pgfqpoint{-0.030309in}{1.052187in}}{\pgfqpoint{-0.041248in}{1.041248in}}%
\pgfpathcurveto{\pgfqpoint{-0.052187in}{1.030309in}}{\pgfqpoint{-0.058333in}{1.015470in}}{\pgfqpoint{-0.058333in}{1.000000in}}%
\pgfpathcurveto{\pgfqpoint{-0.058333in}{0.984530in}}{\pgfqpoint{-0.052187in}{0.969691in}}{\pgfqpoint{-0.041248in}{0.958752in}}%
\pgfpathcurveto{\pgfqpoint{-0.030309in}{0.947813in}}{\pgfqpoint{-0.015470in}{0.941667in}}{\pgfqpoint{0.000000in}{0.941667in}}%
\pgfpathclose%
\pgfpathmoveto{\pgfqpoint{0.000000in}{0.947500in}}%
\pgfpathcurveto{\pgfqpoint{0.000000in}{0.947500in}}{\pgfqpoint{-0.013923in}{0.947500in}}{\pgfqpoint{-0.027278in}{0.953032in}}%
\pgfpathcurveto{\pgfqpoint{-0.037123in}{0.962877in}}{\pgfqpoint{-0.046968in}{0.972722in}}{\pgfqpoint{-0.052500in}{0.986077in}}%
\pgfpathcurveto{\pgfqpoint{-0.052500in}{1.000000in}}{\pgfqpoint{-0.052500in}{1.013923in}}{\pgfqpoint{-0.046968in}{1.027278in}}%
\pgfpathcurveto{\pgfqpoint{-0.037123in}{1.037123in}}{\pgfqpoint{-0.027278in}{1.046968in}}{\pgfqpoint{-0.013923in}{1.052500in}}%
\pgfpathcurveto{\pgfqpoint{0.000000in}{1.052500in}}{\pgfqpoint{0.013923in}{1.052500in}}{\pgfqpoint{0.027278in}{1.046968in}}%
\pgfpathcurveto{\pgfqpoint{0.037123in}{1.037123in}}{\pgfqpoint{0.046968in}{1.027278in}}{\pgfqpoint{0.052500in}{1.013923in}}%
\pgfpathcurveto{\pgfqpoint{0.052500in}{1.000000in}}{\pgfqpoint{0.052500in}{0.986077in}}{\pgfqpoint{0.046968in}{0.972722in}}%
\pgfpathcurveto{\pgfqpoint{0.037123in}{0.962877in}}{\pgfqpoint{0.027278in}{0.953032in}}{\pgfqpoint{0.013923in}{0.947500in}}%
\pgfpathclose%
\pgfpathmoveto{\pgfqpoint{0.166667in}{0.941667in}}%
\pgfpathcurveto{\pgfqpoint{0.182137in}{0.941667in}}{\pgfqpoint{0.196975in}{0.947813in}}{\pgfqpoint{0.207915in}{0.958752in}}%
\pgfpathcurveto{\pgfqpoint{0.218854in}{0.969691in}}{\pgfqpoint{0.225000in}{0.984530in}}{\pgfqpoint{0.225000in}{1.000000in}}%
\pgfpathcurveto{\pgfqpoint{0.225000in}{1.015470in}}{\pgfqpoint{0.218854in}{1.030309in}}{\pgfqpoint{0.207915in}{1.041248in}}%
\pgfpathcurveto{\pgfqpoint{0.196975in}{1.052187in}}{\pgfqpoint{0.182137in}{1.058333in}}{\pgfqpoint{0.166667in}{1.058333in}}%
\pgfpathcurveto{\pgfqpoint{0.151196in}{1.058333in}}{\pgfqpoint{0.136358in}{1.052187in}}{\pgfqpoint{0.125419in}{1.041248in}}%
\pgfpathcurveto{\pgfqpoint{0.114480in}{1.030309in}}{\pgfqpoint{0.108333in}{1.015470in}}{\pgfqpoint{0.108333in}{1.000000in}}%
\pgfpathcurveto{\pgfqpoint{0.108333in}{0.984530in}}{\pgfqpoint{0.114480in}{0.969691in}}{\pgfqpoint{0.125419in}{0.958752in}}%
\pgfpathcurveto{\pgfqpoint{0.136358in}{0.947813in}}{\pgfqpoint{0.151196in}{0.941667in}}{\pgfqpoint{0.166667in}{0.941667in}}%
\pgfpathclose%
\pgfpathmoveto{\pgfqpoint{0.166667in}{0.947500in}}%
\pgfpathcurveto{\pgfqpoint{0.166667in}{0.947500in}}{\pgfqpoint{0.152744in}{0.947500in}}{\pgfqpoint{0.139389in}{0.953032in}}%
\pgfpathcurveto{\pgfqpoint{0.129544in}{0.962877in}}{\pgfqpoint{0.119698in}{0.972722in}}{\pgfqpoint{0.114167in}{0.986077in}}%
\pgfpathcurveto{\pgfqpoint{0.114167in}{1.000000in}}{\pgfqpoint{0.114167in}{1.013923in}}{\pgfqpoint{0.119698in}{1.027278in}}%
\pgfpathcurveto{\pgfqpoint{0.129544in}{1.037123in}}{\pgfqpoint{0.139389in}{1.046968in}}{\pgfqpoint{0.152744in}{1.052500in}}%
\pgfpathcurveto{\pgfqpoint{0.166667in}{1.052500in}}{\pgfqpoint{0.180590in}{1.052500in}}{\pgfqpoint{0.193945in}{1.046968in}}%
\pgfpathcurveto{\pgfqpoint{0.203790in}{1.037123in}}{\pgfqpoint{0.213635in}{1.027278in}}{\pgfqpoint{0.219167in}{1.013923in}}%
\pgfpathcurveto{\pgfqpoint{0.219167in}{1.000000in}}{\pgfqpoint{0.219167in}{0.986077in}}{\pgfqpoint{0.213635in}{0.972722in}}%
\pgfpathcurveto{\pgfqpoint{0.203790in}{0.962877in}}{\pgfqpoint{0.193945in}{0.953032in}}{\pgfqpoint{0.180590in}{0.947500in}}%
\pgfpathclose%
\pgfpathmoveto{\pgfqpoint{0.333333in}{0.941667in}}%
\pgfpathcurveto{\pgfqpoint{0.348804in}{0.941667in}}{\pgfqpoint{0.363642in}{0.947813in}}{\pgfqpoint{0.374581in}{0.958752in}}%
\pgfpathcurveto{\pgfqpoint{0.385520in}{0.969691in}}{\pgfqpoint{0.391667in}{0.984530in}}{\pgfqpoint{0.391667in}{1.000000in}}%
\pgfpathcurveto{\pgfqpoint{0.391667in}{1.015470in}}{\pgfqpoint{0.385520in}{1.030309in}}{\pgfqpoint{0.374581in}{1.041248in}}%
\pgfpathcurveto{\pgfqpoint{0.363642in}{1.052187in}}{\pgfqpoint{0.348804in}{1.058333in}}{\pgfqpoint{0.333333in}{1.058333in}}%
\pgfpathcurveto{\pgfqpoint{0.317863in}{1.058333in}}{\pgfqpoint{0.303025in}{1.052187in}}{\pgfqpoint{0.292085in}{1.041248in}}%
\pgfpathcurveto{\pgfqpoint{0.281146in}{1.030309in}}{\pgfqpoint{0.275000in}{1.015470in}}{\pgfqpoint{0.275000in}{1.000000in}}%
\pgfpathcurveto{\pgfqpoint{0.275000in}{0.984530in}}{\pgfqpoint{0.281146in}{0.969691in}}{\pgfqpoint{0.292085in}{0.958752in}}%
\pgfpathcurveto{\pgfqpoint{0.303025in}{0.947813in}}{\pgfqpoint{0.317863in}{0.941667in}}{\pgfqpoint{0.333333in}{0.941667in}}%
\pgfpathclose%
\pgfpathmoveto{\pgfqpoint{0.333333in}{0.947500in}}%
\pgfpathcurveto{\pgfqpoint{0.333333in}{0.947500in}}{\pgfqpoint{0.319410in}{0.947500in}}{\pgfqpoint{0.306055in}{0.953032in}}%
\pgfpathcurveto{\pgfqpoint{0.296210in}{0.962877in}}{\pgfqpoint{0.286365in}{0.972722in}}{\pgfqpoint{0.280833in}{0.986077in}}%
\pgfpathcurveto{\pgfqpoint{0.280833in}{1.000000in}}{\pgfqpoint{0.280833in}{1.013923in}}{\pgfqpoint{0.286365in}{1.027278in}}%
\pgfpathcurveto{\pgfqpoint{0.296210in}{1.037123in}}{\pgfqpoint{0.306055in}{1.046968in}}{\pgfqpoint{0.319410in}{1.052500in}}%
\pgfpathcurveto{\pgfqpoint{0.333333in}{1.052500in}}{\pgfqpoint{0.347256in}{1.052500in}}{\pgfqpoint{0.360611in}{1.046968in}}%
\pgfpathcurveto{\pgfqpoint{0.370456in}{1.037123in}}{\pgfqpoint{0.380302in}{1.027278in}}{\pgfqpoint{0.385833in}{1.013923in}}%
\pgfpathcurveto{\pgfqpoint{0.385833in}{1.000000in}}{\pgfqpoint{0.385833in}{0.986077in}}{\pgfqpoint{0.380302in}{0.972722in}}%
\pgfpathcurveto{\pgfqpoint{0.370456in}{0.962877in}}{\pgfqpoint{0.360611in}{0.953032in}}{\pgfqpoint{0.347256in}{0.947500in}}%
\pgfpathclose%
\pgfpathmoveto{\pgfqpoint{0.500000in}{0.941667in}}%
\pgfpathcurveto{\pgfqpoint{0.515470in}{0.941667in}}{\pgfqpoint{0.530309in}{0.947813in}}{\pgfqpoint{0.541248in}{0.958752in}}%
\pgfpathcurveto{\pgfqpoint{0.552187in}{0.969691in}}{\pgfqpoint{0.558333in}{0.984530in}}{\pgfqpoint{0.558333in}{1.000000in}}%
\pgfpathcurveto{\pgfqpoint{0.558333in}{1.015470in}}{\pgfqpoint{0.552187in}{1.030309in}}{\pgfqpoint{0.541248in}{1.041248in}}%
\pgfpathcurveto{\pgfqpoint{0.530309in}{1.052187in}}{\pgfqpoint{0.515470in}{1.058333in}}{\pgfqpoint{0.500000in}{1.058333in}}%
\pgfpathcurveto{\pgfqpoint{0.484530in}{1.058333in}}{\pgfqpoint{0.469691in}{1.052187in}}{\pgfqpoint{0.458752in}{1.041248in}}%
\pgfpathcurveto{\pgfqpoint{0.447813in}{1.030309in}}{\pgfqpoint{0.441667in}{1.015470in}}{\pgfqpoint{0.441667in}{1.000000in}}%
\pgfpathcurveto{\pgfqpoint{0.441667in}{0.984530in}}{\pgfqpoint{0.447813in}{0.969691in}}{\pgfqpoint{0.458752in}{0.958752in}}%
\pgfpathcurveto{\pgfqpoint{0.469691in}{0.947813in}}{\pgfqpoint{0.484530in}{0.941667in}}{\pgfqpoint{0.500000in}{0.941667in}}%
\pgfpathclose%
\pgfpathmoveto{\pgfqpoint{0.500000in}{0.947500in}}%
\pgfpathcurveto{\pgfqpoint{0.500000in}{0.947500in}}{\pgfqpoint{0.486077in}{0.947500in}}{\pgfqpoint{0.472722in}{0.953032in}}%
\pgfpathcurveto{\pgfqpoint{0.462877in}{0.962877in}}{\pgfqpoint{0.453032in}{0.972722in}}{\pgfqpoint{0.447500in}{0.986077in}}%
\pgfpathcurveto{\pgfqpoint{0.447500in}{1.000000in}}{\pgfqpoint{0.447500in}{1.013923in}}{\pgfqpoint{0.453032in}{1.027278in}}%
\pgfpathcurveto{\pgfqpoint{0.462877in}{1.037123in}}{\pgfqpoint{0.472722in}{1.046968in}}{\pgfqpoint{0.486077in}{1.052500in}}%
\pgfpathcurveto{\pgfqpoint{0.500000in}{1.052500in}}{\pgfqpoint{0.513923in}{1.052500in}}{\pgfqpoint{0.527278in}{1.046968in}}%
\pgfpathcurveto{\pgfqpoint{0.537123in}{1.037123in}}{\pgfqpoint{0.546968in}{1.027278in}}{\pgfqpoint{0.552500in}{1.013923in}}%
\pgfpathcurveto{\pgfqpoint{0.552500in}{1.000000in}}{\pgfqpoint{0.552500in}{0.986077in}}{\pgfqpoint{0.546968in}{0.972722in}}%
\pgfpathcurveto{\pgfqpoint{0.537123in}{0.962877in}}{\pgfqpoint{0.527278in}{0.953032in}}{\pgfqpoint{0.513923in}{0.947500in}}%
\pgfpathclose%
\pgfpathmoveto{\pgfqpoint{0.666667in}{0.941667in}}%
\pgfpathcurveto{\pgfqpoint{0.682137in}{0.941667in}}{\pgfqpoint{0.696975in}{0.947813in}}{\pgfqpoint{0.707915in}{0.958752in}}%
\pgfpathcurveto{\pgfqpoint{0.718854in}{0.969691in}}{\pgfqpoint{0.725000in}{0.984530in}}{\pgfqpoint{0.725000in}{1.000000in}}%
\pgfpathcurveto{\pgfqpoint{0.725000in}{1.015470in}}{\pgfqpoint{0.718854in}{1.030309in}}{\pgfqpoint{0.707915in}{1.041248in}}%
\pgfpathcurveto{\pgfqpoint{0.696975in}{1.052187in}}{\pgfqpoint{0.682137in}{1.058333in}}{\pgfqpoint{0.666667in}{1.058333in}}%
\pgfpathcurveto{\pgfqpoint{0.651196in}{1.058333in}}{\pgfqpoint{0.636358in}{1.052187in}}{\pgfqpoint{0.625419in}{1.041248in}}%
\pgfpathcurveto{\pgfqpoint{0.614480in}{1.030309in}}{\pgfqpoint{0.608333in}{1.015470in}}{\pgfqpoint{0.608333in}{1.000000in}}%
\pgfpathcurveto{\pgfqpoint{0.608333in}{0.984530in}}{\pgfqpoint{0.614480in}{0.969691in}}{\pgfqpoint{0.625419in}{0.958752in}}%
\pgfpathcurveto{\pgfqpoint{0.636358in}{0.947813in}}{\pgfqpoint{0.651196in}{0.941667in}}{\pgfqpoint{0.666667in}{0.941667in}}%
\pgfpathclose%
\pgfpathmoveto{\pgfqpoint{0.666667in}{0.947500in}}%
\pgfpathcurveto{\pgfqpoint{0.666667in}{0.947500in}}{\pgfqpoint{0.652744in}{0.947500in}}{\pgfqpoint{0.639389in}{0.953032in}}%
\pgfpathcurveto{\pgfqpoint{0.629544in}{0.962877in}}{\pgfqpoint{0.619698in}{0.972722in}}{\pgfqpoint{0.614167in}{0.986077in}}%
\pgfpathcurveto{\pgfqpoint{0.614167in}{1.000000in}}{\pgfqpoint{0.614167in}{1.013923in}}{\pgfqpoint{0.619698in}{1.027278in}}%
\pgfpathcurveto{\pgfqpoint{0.629544in}{1.037123in}}{\pgfqpoint{0.639389in}{1.046968in}}{\pgfqpoint{0.652744in}{1.052500in}}%
\pgfpathcurveto{\pgfqpoint{0.666667in}{1.052500in}}{\pgfqpoint{0.680590in}{1.052500in}}{\pgfqpoint{0.693945in}{1.046968in}}%
\pgfpathcurveto{\pgfqpoint{0.703790in}{1.037123in}}{\pgfqpoint{0.713635in}{1.027278in}}{\pgfqpoint{0.719167in}{1.013923in}}%
\pgfpathcurveto{\pgfqpoint{0.719167in}{1.000000in}}{\pgfqpoint{0.719167in}{0.986077in}}{\pgfqpoint{0.713635in}{0.972722in}}%
\pgfpathcurveto{\pgfqpoint{0.703790in}{0.962877in}}{\pgfqpoint{0.693945in}{0.953032in}}{\pgfqpoint{0.680590in}{0.947500in}}%
\pgfpathclose%
\pgfpathmoveto{\pgfqpoint{0.833333in}{0.941667in}}%
\pgfpathcurveto{\pgfqpoint{0.848804in}{0.941667in}}{\pgfqpoint{0.863642in}{0.947813in}}{\pgfqpoint{0.874581in}{0.958752in}}%
\pgfpathcurveto{\pgfqpoint{0.885520in}{0.969691in}}{\pgfqpoint{0.891667in}{0.984530in}}{\pgfqpoint{0.891667in}{1.000000in}}%
\pgfpathcurveto{\pgfqpoint{0.891667in}{1.015470in}}{\pgfqpoint{0.885520in}{1.030309in}}{\pgfqpoint{0.874581in}{1.041248in}}%
\pgfpathcurveto{\pgfqpoint{0.863642in}{1.052187in}}{\pgfqpoint{0.848804in}{1.058333in}}{\pgfqpoint{0.833333in}{1.058333in}}%
\pgfpathcurveto{\pgfqpoint{0.817863in}{1.058333in}}{\pgfqpoint{0.803025in}{1.052187in}}{\pgfqpoint{0.792085in}{1.041248in}}%
\pgfpathcurveto{\pgfqpoint{0.781146in}{1.030309in}}{\pgfqpoint{0.775000in}{1.015470in}}{\pgfqpoint{0.775000in}{1.000000in}}%
\pgfpathcurveto{\pgfqpoint{0.775000in}{0.984530in}}{\pgfqpoint{0.781146in}{0.969691in}}{\pgfqpoint{0.792085in}{0.958752in}}%
\pgfpathcurveto{\pgfqpoint{0.803025in}{0.947813in}}{\pgfqpoint{0.817863in}{0.941667in}}{\pgfqpoint{0.833333in}{0.941667in}}%
\pgfpathclose%
\pgfpathmoveto{\pgfqpoint{0.833333in}{0.947500in}}%
\pgfpathcurveto{\pgfqpoint{0.833333in}{0.947500in}}{\pgfqpoint{0.819410in}{0.947500in}}{\pgfqpoint{0.806055in}{0.953032in}}%
\pgfpathcurveto{\pgfqpoint{0.796210in}{0.962877in}}{\pgfqpoint{0.786365in}{0.972722in}}{\pgfqpoint{0.780833in}{0.986077in}}%
\pgfpathcurveto{\pgfqpoint{0.780833in}{1.000000in}}{\pgfqpoint{0.780833in}{1.013923in}}{\pgfqpoint{0.786365in}{1.027278in}}%
\pgfpathcurveto{\pgfqpoint{0.796210in}{1.037123in}}{\pgfqpoint{0.806055in}{1.046968in}}{\pgfqpoint{0.819410in}{1.052500in}}%
\pgfpathcurveto{\pgfqpoint{0.833333in}{1.052500in}}{\pgfqpoint{0.847256in}{1.052500in}}{\pgfqpoint{0.860611in}{1.046968in}}%
\pgfpathcurveto{\pgfqpoint{0.870456in}{1.037123in}}{\pgfqpoint{0.880302in}{1.027278in}}{\pgfqpoint{0.885833in}{1.013923in}}%
\pgfpathcurveto{\pgfqpoint{0.885833in}{1.000000in}}{\pgfqpoint{0.885833in}{0.986077in}}{\pgfqpoint{0.880302in}{0.972722in}}%
\pgfpathcurveto{\pgfqpoint{0.870456in}{0.962877in}}{\pgfqpoint{0.860611in}{0.953032in}}{\pgfqpoint{0.847256in}{0.947500in}}%
\pgfpathclose%
\pgfpathmoveto{\pgfqpoint{1.000000in}{0.941667in}}%
\pgfpathcurveto{\pgfqpoint{1.015470in}{0.941667in}}{\pgfqpoint{1.030309in}{0.947813in}}{\pgfqpoint{1.041248in}{0.958752in}}%
\pgfpathcurveto{\pgfqpoint{1.052187in}{0.969691in}}{\pgfqpoint{1.058333in}{0.984530in}}{\pgfqpoint{1.058333in}{1.000000in}}%
\pgfpathcurveto{\pgfqpoint{1.058333in}{1.015470in}}{\pgfqpoint{1.052187in}{1.030309in}}{\pgfqpoint{1.041248in}{1.041248in}}%
\pgfpathcurveto{\pgfqpoint{1.030309in}{1.052187in}}{\pgfqpoint{1.015470in}{1.058333in}}{\pgfqpoint{1.000000in}{1.058333in}}%
\pgfpathcurveto{\pgfqpoint{0.984530in}{1.058333in}}{\pgfqpoint{0.969691in}{1.052187in}}{\pgfqpoint{0.958752in}{1.041248in}}%
\pgfpathcurveto{\pgfqpoint{0.947813in}{1.030309in}}{\pgfqpoint{0.941667in}{1.015470in}}{\pgfqpoint{0.941667in}{1.000000in}}%
\pgfpathcurveto{\pgfqpoint{0.941667in}{0.984530in}}{\pgfqpoint{0.947813in}{0.969691in}}{\pgfqpoint{0.958752in}{0.958752in}}%
\pgfpathcurveto{\pgfqpoint{0.969691in}{0.947813in}}{\pgfqpoint{0.984530in}{0.941667in}}{\pgfqpoint{1.000000in}{0.941667in}}%
\pgfpathclose%
\pgfpathmoveto{\pgfqpoint{1.000000in}{0.947500in}}%
\pgfpathcurveto{\pgfqpoint{1.000000in}{0.947500in}}{\pgfqpoint{0.986077in}{0.947500in}}{\pgfqpoint{0.972722in}{0.953032in}}%
\pgfpathcurveto{\pgfqpoint{0.962877in}{0.962877in}}{\pgfqpoint{0.953032in}{0.972722in}}{\pgfqpoint{0.947500in}{0.986077in}}%
\pgfpathcurveto{\pgfqpoint{0.947500in}{1.000000in}}{\pgfqpoint{0.947500in}{1.013923in}}{\pgfqpoint{0.953032in}{1.027278in}}%
\pgfpathcurveto{\pgfqpoint{0.962877in}{1.037123in}}{\pgfqpoint{0.972722in}{1.046968in}}{\pgfqpoint{0.986077in}{1.052500in}}%
\pgfpathcurveto{\pgfqpoint{1.000000in}{1.052500in}}{\pgfqpoint{1.013923in}{1.052500in}}{\pgfqpoint{1.027278in}{1.046968in}}%
\pgfpathcurveto{\pgfqpoint{1.037123in}{1.037123in}}{\pgfqpoint{1.046968in}{1.027278in}}{\pgfqpoint{1.052500in}{1.013923in}}%
\pgfpathcurveto{\pgfqpoint{1.052500in}{1.000000in}}{\pgfqpoint{1.052500in}{0.986077in}}{\pgfqpoint{1.046968in}{0.972722in}}%
\pgfpathcurveto{\pgfqpoint{1.037123in}{0.962877in}}{\pgfqpoint{1.027278in}{0.953032in}}{\pgfqpoint{1.013923in}{0.947500in}}%
\pgfpathclose%
\pgfusepath{stroke}%
\end{pgfscope}%
}%
\pgfsys@transformshift{5.973315in}{2.878536in}%
\pgfsys@useobject{currentpattern}{}%
\pgfsys@transformshift{1in}{0in}%
\pgfsys@transformshift{-1in}{0in}%
\pgfsys@transformshift{0in}{1in}%
\pgfsys@useobject{currentpattern}{}%
\pgfsys@transformshift{1in}{0in}%
\pgfsys@transformshift{-1in}{0in}%
\pgfsys@transformshift{0in}{1in}%
\pgfsys@useobject{currentpattern}{}%
\pgfsys@transformshift{1in}{0in}%
\pgfsys@transformshift{-1in}{0in}%
\pgfsys@transformshift{0in}{1in}%
\end{pgfscope}%
\begin{pgfscope}%
\pgfpathrectangle{\pgfqpoint{0.935815in}{0.637495in}}{\pgfqpoint{9.300000in}{9.060000in}}%
\pgfusepath{clip}%
\pgfsetbuttcap%
\pgfsetmiterjoin%
\definecolor{currentfill}{rgb}{0.549020,0.337255,0.294118}%
\pgfsetfillcolor{currentfill}%
\pgfsetfillopacity{0.990000}%
\pgfsetlinewidth{0.000000pt}%
\definecolor{currentstroke}{rgb}{0.000000,0.000000,0.000000}%
\pgfsetstrokecolor{currentstroke}%
\pgfsetstrokeopacity{0.990000}%
\pgfsetdash{}{0pt}%
\pgfpathmoveto{\pgfqpoint{7.523315in}{3.003473in}}%
\pgfpathlineto{\pgfqpoint{8.298315in}{3.003473in}}%
\pgfpathlineto{\pgfqpoint{8.298315in}{5.046572in}}%
\pgfpathlineto{\pgfqpoint{7.523315in}{5.046572in}}%
\pgfpathclose%
\pgfusepath{fill}%
\end{pgfscope}%
\begin{pgfscope}%
\pgfsetbuttcap%
\pgfsetmiterjoin%
\definecolor{currentfill}{rgb}{0.549020,0.337255,0.294118}%
\pgfsetfillcolor{currentfill}%
\pgfsetfillopacity{0.990000}%
\pgfsetlinewidth{0.000000pt}%
\definecolor{currentstroke}{rgb}{0.000000,0.000000,0.000000}%
\pgfsetstrokecolor{currentstroke}%
\pgfsetstrokeopacity{0.990000}%
\pgfsetdash{}{0pt}%
\pgfpathrectangle{\pgfqpoint{0.935815in}{0.637495in}}{\pgfqpoint{9.300000in}{9.060000in}}%
\pgfusepath{clip}%
\pgfpathmoveto{\pgfqpoint{7.523315in}{3.003473in}}%
\pgfpathlineto{\pgfqpoint{8.298315in}{3.003473in}}%
\pgfpathlineto{\pgfqpoint{8.298315in}{5.046572in}}%
\pgfpathlineto{\pgfqpoint{7.523315in}{5.046572in}}%
\pgfpathclose%
\pgfusepath{clip}%
\pgfsys@defobject{currentpattern}{\pgfqpoint{0in}{0in}}{\pgfqpoint{1in}{1in}}{%
\begin{pgfscope}%
\pgfpathrectangle{\pgfqpoint{0in}{0in}}{\pgfqpoint{1in}{1in}}%
\pgfusepath{clip}%
\pgfpathmoveto{\pgfqpoint{0.000000in}{-0.058333in}}%
\pgfpathcurveto{\pgfqpoint{0.015470in}{-0.058333in}}{\pgfqpoint{0.030309in}{-0.052187in}}{\pgfqpoint{0.041248in}{-0.041248in}}%
\pgfpathcurveto{\pgfqpoint{0.052187in}{-0.030309in}}{\pgfqpoint{0.058333in}{-0.015470in}}{\pgfqpoint{0.058333in}{0.000000in}}%
\pgfpathcurveto{\pgfqpoint{0.058333in}{0.015470in}}{\pgfqpoint{0.052187in}{0.030309in}}{\pgfqpoint{0.041248in}{0.041248in}}%
\pgfpathcurveto{\pgfqpoint{0.030309in}{0.052187in}}{\pgfqpoint{0.015470in}{0.058333in}}{\pgfqpoint{0.000000in}{0.058333in}}%
\pgfpathcurveto{\pgfqpoint{-0.015470in}{0.058333in}}{\pgfqpoint{-0.030309in}{0.052187in}}{\pgfqpoint{-0.041248in}{0.041248in}}%
\pgfpathcurveto{\pgfqpoint{-0.052187in}{0.030309in}}{\pgfqpoint{-0.058333in}{0.015470in}}{\pgfqpoint{-0.058333in}{0.000000in}}%
\pgfpathcurveto{\pgfqpoint{-0.058333in}{-0.015470in}}{\pgfqpoint{-0.052187in}{-0.030309in}}{\pgfqpoint{-0.041248in}{-0.041248in}}%
\pgfpathcurveto{\pgfqpoint{-0.030309in}{-0.052187in}}{\pgfqpoint{-0.015470in}{-0.058333in}}{\pgfqpoint{0.000000in}{-0.058333in}}%
\pgfpathclose%
\pgfpathmoveto{\pgfqpoint{0.000000in}{-0.052500in}}%
\pgfpathcurveto{\pgfqpoint{0.000000in}{-0.052500in}}{\pgfqpoint{-0.013923in}{-0.052500in}}{\pgfqpoint{-0.027278in}{-0.046968in}}%
\pgfpathcurveto{\pgfqpoint{-0.037123in}{-0.037123in}}{\pgfqpoint{-0.046968in}{-0.027278in}}{\pgfqpoint{-0.052500in}{-0.013923in}}%
\pgfpathcurveto{\pgfqpoint{-0.052500in}{0.000000in}}{\pgfqpoint{-0.052500in}{0.013923in}}{\pgfqpoint{-0.046968in}{0.027278in}}%
\pgfpathcurveto{\pgfqpoint{-0.037123in}{0.037123in}}{\pgfqpoint{-0.027278in}{0.046968in}}{\pgfqpoint{-0.013923in}{0.052500in}}%
\pgfpathcurveto{\pgfqpoint{0.000000in}{0.052500in}}{\pgfqpoint{0.013923in}{0.052500in}}{\pgfqpoint{0.027278in}{0.046968in}}%
\pgfpathcurveto{\pgfqpoint{0.037123in}{0.037123in}}{\pgfqpoint{0.046968in}{0.027278in}}{\pgfqpoint{0.052500in}{0.013923in}}%
\pgfpathcurveto{\pgfqpoint{0.052500in}{0.000000in}}{\pgfqpoint{0.052500in}{-0.013923in}}{\pgfqpoint{0.046968in}{-0.027278in}}%
\pgfpathcurveto{\pgfqpoint{0.037123in}{-0.037123in}}{\pgfqpoint{0.027278in}{-0.046968in}}{\pgfqpoint{0.013923in}{-0.052500in}}%
\pgfpathclose%
\pgfpathmoveto{\pgfqpoint{0.166667in}{-0.058333in}}%
\pgfpathcurveto{\pgfqpoint{0.182137in}{-0.058333in}}{\pgfqpoint{0.196975in}{-0.052187in}}{\pgfqpoint{0.207915in}{-0.041248in}}%
\pgfpathcurveto{\pgfqpoint{0.218854in}{-0.030309in}}{\pgfqpoint{0.225000in}{-0.015470in}}{\pgfqpoint{0.225000in}{0.000000in}}%
\pgfpathcurveto{\pgfqpoint{0.225000in}{0.015470in}}{\pgfqpoint{0.218854in}{0.030309in}}{\pgfqpoint{0.207915in}{0.041248in}}%
\pgfpathcurveto{\pgfqpoint{0.196975in}{0.052187in}}{\pgfqpoint{0.182137in}{0.058333in}}{\pgfqpoint{0.166667in}{0.058333in}}%
\pgfpathcurveto{\pgfqpoint{0.151196in}{0.058333in}}{\pgfqpoint{0.136358in}{0.052187in}}{\pgfqpoint{0.125419in}{0.041248in}}%
\pgfpathcurveto{\pgfqpoint{0.114480in}{0.030309in}}{\pgfqpoint{0.108333in}{0.015470in}}{\pgfqpoint{0.108333in}{0.000000in}}%
\pgfpathcurveto{\pgfqpoint{0.108333in}{-0.015470in}}{\pgfqpoint{0.114480in}{-0.030309in}}{\pgfqpoint{0.125419in}{-0.041248in}}%
\pgfpathcurveto{\pgfqpoint{0.136358in}{-0.052187in}}{\pgfqpoint{0.151196in}{-0.058333in}}{\pgfqpoint{0.166667in}{-0.058333in}}%
\pgfpathclose%
\pgfpathmoveto{\pgfqpoint{0.166667in}{-0.052500in}}%
\pgfpathcurveto{\pgfqpoint{0.166667in}{-0.052500in}}{\pgfqpoint{0.152744in}{-0.052500in}}{\pgfqpoint{0.139389in}{-0.046968in}}%
\pgfpathcurveto{\pgfqpoint{0.129544in}{-0.037123in}}{\pgfqpoint{0.119698in}{-0.027278in}}{\pgfqpoint{0.114167in}{-0.013923in}}%
\pgfpathcurveto{\pgfqpoint{0.114167in}{0.000000in}}{\pgfqpoint{0.114167in}{0.013923in}}{\pgfqpoint{0.119698in}{0.027278in}}%
\pgfpathcurveto{\pgfqpoint{0.129544in}{0.037123in}}{\pgfqpoint{0.139389in}{0.046968in}}{\pgfqpoint{0.152744in}{0.052500in}}%
\pgfpathcurveto{\pgfqpoint{0.166667in}{0.052500in}}{\pgfqpoint{0.180590in}{0.052500in}}{\pgfqpoint{0.193945in}{0.046968in}}%
\pgfpathcurveto{\pgfqpoint{0.203790in}{0.037123in}}{\pgfqpoint{0.213635in}{0.027278in}}{\pgfqpoint{0.219167in}{0.013923in}}%
\pgfpathcurveto{\pgfqpoint{0.219167in}{0.000000in}}{\pgfqpoint{0.219167in}{-0.013923in}}{\pgfqpoint{0.213635in}{-0.027278in}}%
\pgfpathcurveto{\pgfqpoint{0.203790in}{-0.037123in}}{\pgfqpoint{0.193945in}{-0.046968in}}{\pgfqpoint{0.180590in}{-0.052500in}}%
\pgfpathclose%
\pgfpathmoveto{\pgfqpoint{0.333333in}{-0.058333in}}%
\pgfpathcurveto{\pgfqpoint{0.348804in}{-0.058333in}}{\pgfqpoint{0.363642in}{-0.052187in}}{\pgfqpoint{0.374581in}{-0.041248in}}%
\pgfpathcurveto{\pgfqpoint{0.385520in}{-0.030309in}}{\pgfqpoint{0.391667in}{-0.015470in}}{\pgfqpoint{0.391667in}{0.000000in}}%
\pgfpathcurveto{\pgfqpoint{0.391667in}{0.015470in}}{\pgfqpoint{0.385520in}{0.030309in}}{\pgfqpoint{0.374581in}{0.041248in}}%
\pgfpathcurveto{\pgfqpoint{0.363642in}{0.052187in}}{\pgfqpoint{0.348804in}{0.058333in}}{\pgfqpoint{0.333333in}{0.058333in}}%
\pgfpathcurveto{\pgfqpoint{0.317863in}{0.058333in}}{\pgfqpoint{0.303025in}{0.052187in}}{\pgfqpoint{0.292085in}{0.041248in}}%
\pgfpathcurveto{\pgfqpoint{0.281146in}{0.030309in}}{\pgfqpoint{0.275000in}{0.015470in}}{\pgfqpoint{0.275000in}{0.000000in}}%
\pgfpathcurveto{\pgfqpoint{0.275000in}{-0.015470in}}{\pgfqpoint{0.281146in}{-0.030309in}}{\pgfqpoint{0.292085in}{-0.041248in}}%
\pgfpathcurveto{\pgfqpoint{0.303025in}{-0.052187in}}{\pgfqpoint{0.317863in}{-0.058333in}}{\pgfqpoint{0.333333in}{-0.058333in}}%
\pgfpathclose%
\pgfpathmoveto{\pgfqpoint{0.333333in}{-0.052500in}}%
\pgfpathcurveto{\pgfqpoint{0.333333in}{-0.052500in}}{\pgfqpoint{0.319410in}{-0.052500in}}{\pgfqpoint{0.306055in}{-0.046968in}}%
\pgfpathcurveto{\pgfqpoint{0.296210in}{-0.037123in}}{\pgfqpoint{0.286365in}{-0.027278in}}{\pgfqpoint{0.280833in}{-0.013923in}}%
\pgfpathcurveto{\pgfqpoint{0.280833in}{0.000000in}}{\pgfqpoint{0.280833in}{0.013923in}}{\pgfqpoint{0.286365in}{0.027278in}}%
\pgfpathcurveto{\pgfqpoint{0.296210in}{0.037123in}}{\pgfqpoint{0.306055in}{0.046968in}}{\pgfqpoint{0.319410in}{0.052500in}}%
\pgfpathcurveto{\pgfqpoint{0.333333in}{0.052500in}}{\pgfqpoint{0.347256in}{0.052500in}}{\pgfqpoint{0.360611in}{0.046968in}}%
\pgfpathcurveto{\pgfqpoint{0.370456in}{0.037123in}}{\pgfqpoint{0.380302in}{0.027278in}}{\pgfqpoint{0.385833in}{0.013923in}}%
\pgfpathcurveto{\pgfqpoint{0.385833in}{0.000000in}}{\pgfqpoint{0.385833in}{-0.013923in}}{\pgfqpoint{0.380302in}{-0.027278in}}%
\pgfpathcurveto{\pgfqpoint{0.370456in}{-0.037123in}}{\pgfqpoint{0.360611in}{-0.046968in}}{\pgfqpoint{0.347256in}{-0.052500in}}%
\pgfpathclose%
\pgfpathmoveto{\pgfqpoint{0.500000in}{-0.058333in}}%
\pgfpathcurveto{\pgfqpoint{0.515470in}{-0.058333in}}{\pgfqpoint{0.530309in}{-0.052187in}}{\pgfqpoint{0.541248in}{-0.041248in}}%
\pgfpathcurveto{\pgfqpoint{0.552187in}{-0.030309in}}{\pgfqpoint{0.558333in}{-0.015470in}}{\pgfqpoint{0.558333in}{0.000000in}}%
\pgfpathcurveto{\pgfqpoint{0.558333in}{0.015470in}}{\pgfqpoint{0.552187in}{0.030309in}}{\pgfqpoint{0.541248in}{0.041248in}}%
\pgfpathcurveto{\pgfqpoint{0.530309in}{0.052187in}}{\pgfqpoint{0.515470in}{0.058333in}}{\pgfqpoint{0.500000in}{0.058333in}}%
\pgfpathcurveto{\pgfqpoint{0.484530in}{0.058333in}}{\pgfqpoint{0.469691in}{0.052187in}}{\pgfqpoint{0.458752in}{0.041248in}}%
\pgfpathcurveto{\pgfqpoint{0.447813in}{0.030309in}}{\pgfqpoint{0.441667in}{0.015470in}}{\pgfqpoint{0.441667in}{0.000000in}}%
\pgfpathcurveto{\pgfqpoint{0.441667in}{-0.015470in}}{\pgfqpoint{0.447813in}{-0.030309in}}{\pgfqpoint{0.458752in}{-0.041248in}}%
\pgfpathcurveto{\pgfqpoint{0.469691in}{-0.052187in}}{\pgfqpoint{0.484530in}{-0.058333in}}{\pgfqpoint{0.500000in}{-0.058333in}}%
\pgfpathclose%
\pgfpathmoveto{\pgfqpoint{0.500000in}{-0.052500in}}%
\pgfpathcurveto{\pgfqpoint{0.500000in}{-0.052500in}}{\pgfqpoint{0.486077in}{-0.052500in}}{\pgfqpoint{0.472722in}{-0.046968in}}%
\pgfpathcurveto{\pgfqpoint{0.462877in}{-0.037123in}}{\pgfqpoint{0.453032in}{-0.027278in}}{\pgfqpoint{0.447500in}{-0.013923in}}%
\pgfpathcurveto{\pgfqpoint{0.447500in}{0.000000in}}{\pgfqpoint{0.447500in}{0.013923in}}{\pgfqpoint{0.453032in}{0.027278in}}%
\pgfpathcurveto{\pgfqpoint{0.462877in}{0.037123in}}{\pgfqpoint{0.472722in}{0.046968in}}{\pgfqpoint{0.486077in}{0.052500in}}%
\pgfpathcurveto{\pgfqpoint{0.500000in}{0.052500in}}{\pgfqpoint{0.513923in}{0.052500in}}{\pgfqpoint{0.527278in}{0.046968in}}%
\pgfpathcurveto{\pgfqpoint{0.537123in}{0.037123in}}{\pgfqpoint{0.546968in}{0.027278in}}{\pgfqpoint{0.552500in}{0.013923in}}%
\pgfpathcurveto{\pgfqpoint{0.552500in}{0.000000in}}{\pgfqpoint{0.552500in}{-0.013923in}}{\pgfqpoint{0.546968in}{-0.027278in}}%
\pgfpathcurveto{\pgfqpoint{0.537123in}{-0.037123in}}{\pgfqpoint{0.527278in}{-0.046968in}}{\pgfqpoint{0.513923in}{-0.052500in}}%
\pgfpathclose%
\pgfpathmoveto{\pgfqpoint{0.666667in}{-0.058333in}}%
\pgfpathcurveto{\pgfqpoint{0.682137in}{-0.058333in}}{\pgfqpoint{0.696975in}{-0.052187in}}{\pgfqpoint{0.707915in}{-0.041248in}}%
\pgfpathcurveto{\pgfqpoint{0.718854in}{-0.030309in}}{\pgfqpoint{0.725000in}{-0.015470in}}{\pgfqpoint{0.725000in}{0.000000in}}%
\pgfpathcurveto{\pgfqpoint{0.725000in}{0.015470in}}{\pgfqpoint{0.718854in}{0.030309in}}{\pgfqpoint{0.707915in}{0.041248in}}%
\pgfpathcurveto{\pgfqpoint{0.696975in}{0.052187in}}{\pgfqpoint{0.682137in}{0.058333in}}{\pgfqpoint{0.666667in}{0.058333in}}%
\pgfpathcurveto{\pgfqpoint{0.651196in}{0.058333in}}{\pgfqpoint{0.636358in}{0.052187in}}{\pgfqpoint{0.625419in}{0.041248in}}%
\pgfpathcurveto{\pgfqpoint{0.614480in}{0.030309in}}{\pgfqpoint{0.608333in}{0.015470in}}{\pgfqpoint{0.608333in}{0.000000in}}%
\pgfpathcurveto{\pgfqpoint{0.608333in}{-0.015470in}}{\pgfqpoint{0.614480in}{-0.030309in}}{\pgfqpoint{0.625419in}{-0.041248in}}%
\pgfpathcurveto{\pgfqpoint{0.636358in}{-0.052187in}}{\pgfqpoint{0.651196in}{-0.058333in}}{\pgfqpoint{0.666667in}{-0.058333in}}%
\pgfpathclose%
\pgfpathmoveto{\pgfqpoint{0.666667in}{-0.052500in}}%
\pgfpathcurveto{\pgfqpoint{0.666667in}{-0.052500in}}{\pgfqpoint{0.652744in}{-0.052500in}}{\pgfqpoint{0.639389in}{-0.046968in}}%
\pgfpathcurveto{\pgfqpoint{0.629544in}{-0.037123in}}{\pgfqpoint{0.619698in}{-0.027278in}}{\pgfqpoint{0.614167in}{-0.013923in}}%
\pgfpathcurveto{\pgfqpoint{0.614167in}{0.000000in}}{\pgfqpoint{0.614167in}{0.013923in}}{\pgfqpoint{0.619698in}{0.027278in}}%
\pgfpathcurveto{\pgfqpoint{0.629544in}{0.037123in}}{\pgfqpoint{0.639389in}{0.046968in}}{\pgfqpoint{0.652744in}{0.052500in}}%
\pgfpathcurveto{\pgfqpoint{0.666667in}{0.052500in}}{\pgfqpoint{0.680590in}{0.052500in}}{\pgfqpoint{0.693945in}{0.046968in}}%
\pgfpathcurveto{\pgfqpoint{0.703790in}{0.037123in}}{\pgfqpoint{0.713635in}{0.027278in}}{\pgfqpoint{0.719167in}{0.013923in}}%
\pgfpathcurveto{\pgfqpoint{0.719167in}{0.000000in}}{\pgfqpoint{0.719167in}{-0.013923in}}{\pgfqpoint{0.713635in}{-0.027278in}}%
\pgfpathcurveto{\pgfqpoint{0.703790in}{-0.037123in}}{\pgfqpoint{0.693945in}{-0.046968in}}{\pgfqpoint{0.680590in}{-0.052500in}}%
\pgfpathclose%
\pgfpathmoveto{\pgfqpoint{0.833333in}{-0.058333in}}%
\pgfpathcurveto{\pgfqpoint{0.848804in}{-0.058333in}}{\pgfqpoint{0.863642in}{-0.052187in}}{\pgfqpoint{0.874581in}{-0.041248in}}%
\pgfpathcurveto{\pgfqpoint{0.885520in}{-0.030309in}}{\pgfqpoint{0.891667in}{-0.015470in}}{\pgfqpoint{0.891667in}{0.000000in}}%
\pgfpathcurveto{\pgfqpoint{0.891667in}{0.015470in}}{\pgfqpoint{0.885520in}{0.030309in}}{\pgfqpoint{0.874581in}{0.041248in}}%
\pgfpathcurveto{\pgfqpoint{0.863642in}{0.052187in}}{\pgfqpoint{0.848804in}{0.058333in}}{\pgfqpoint{0.833333in}{0.058333in}}%
\pgfpathcurveto{\pgfqpoint{0.817863in}{0.058333in}}{\pgfqpoint{0.803025in}{0.052187in}}{\pgfqpoint{0.792085in}{0.041248in}}%
\pgfpathcurveto{\pgfqpoint{0.781146in}{0.030309in}}{\pgfqpoint{0.775000in}{0.015470in}}{\pgfqpoint{0.775000in}{0.000000in}}%
\pgfpathcurveto{\pgfqpoint{0.775000in}{-0.015470in}}{\pgfqpoint{0.781146in}{-0.030309in}}{\pgfqpoint{0.792085in}{-0.041248in}}%
\pgfpathcurveto{\pgfqpoint{0.803025in}{-0.052187in}}{\pgfqpoint{0.817863in}{-0.058333in}}{\pgfqpoint{0.833333in}{-0.058333in}}%
\pgfpathclose%
\pgfpathmoveto{\pgfqpoint{0.833333in}{-0.052500in}}%
\pgfpathcurveto{\pgfqpoint{0.833333in}{-0.052500in}}{\pgfqpoint{0.819410in}{-0.052500in}}{\pgfqpoint{0.806055in}{-0.046968in}}%
\pgfpathcurveto{\pgfqpoint{0.796210in}{-0.037123in}}{\pgfqpoint{0.786365in}{-0.027278in}}{\pgfqpoint{0.780833in}{-0.013923in}}%
\pgfpathcurveto{\pgfqpoint{0.780833in}{0.000000in}}{\pgfqpoint{0.780833in}{0.013923in}}{\pgfqpoint{0.786365in}{0.027278in}}%
\pgfpathcurveto{\pgfqpoint{0.796210in}{0.037123in}}{\pgfqpoint{0.806055in}{0.046968in}}{\pgfqpoint{0.819410in}{0.052500in}}%
\pgfpathcurveto{\pgfqpoint{0.833333in}{0.052500in}}{\pgfqpoint{0.847256in}{0.052500in}}{\pgfqpoint{0.860611in}{0.046968in}}%
\pgfpathcurveto{\pgfqpoint{0.870456in}{0.037123in}}{\pgfqpoint{0.880302in}{0.027278in}}{\pgfqpoint{0.885833in}{0.013923in}}%
\pgfpathcurveto{\pgfqpoint{0.885833in}{0.000000in}}{\pgfqpoint{0.885833in}{-0.013923in}}{\pgfqpoint{0.880302in}{-0.027278in}}%
\pgfpathcurveto{\pgfqpoint{0.870456in}{-0.037123in}}{\pgfqpoint{0.860611in}{-0.046968in}}{\pgfqpoint{0.847256in}{-0.052500in}}%
\pgfpathclose%
\pgfpathmoveto{\pgfqpoint{1.000000in}{-0.058333in}}%
\pgfpathcurveto{\pgfqpoint{1.015470in}{-0.058333in}}{\pgfqpoint{1.030309in}{-0.052187in}}{\pgfqpoint{1.041248in}{-0.041248in}}%
\pgfpathcurveto{\pgfqpoint{1.052187in}{-0.030309in}}{\pgfqpoint{1.058333in}{-0.015470in}}{\pgfqpoint{1.058333in}{0.000000in}}%
\pgfpathcurveto{\pgfqpoint{1.058333in}{0.015470in}}{\pgfqpoint{1.052187in}{0.030309in}}{\pgfqpoint{1.041248in}{0.041248in}}%
\pgfpathcurveto{\pgfqpoint{1.030309in}{0.052187in}}{\pgfqpoint{1.015470in}{0.058333in}}{\pgfqpoint{1.000000in}{0.058333in}}%
\pgfpathcurveto{\pgfqpoint{0.984530in}{0.058333in}}{\pgfqpoint{0.969691in}{0.052187in}}{\pgfqpoint{0.958752in}{0.041248in}}%
\pgfpathcurveto{\pgfqpoint{0.947813in}{0.030309in}}{\pgfqpoint{0.941667in}{0.015470in}}{\pgfqpoint{0.941667in}{0.000000in}}%
\pgfpathcurveto{\pgfqpoint{0.941667in}{-0.015470in}}{\pgfqpoint{0.947813in}{-0.030309in}}{\pgfqpoint{0.958752in}{-0.041248in}}%
\pgfpathcurveto{\pgfqpoint{0.969691in}{-0.052187in}}{\pgfqpoint{0.984530in}{-0.058333in}}{\pgfqpoint{1.000000in}{-0.058333in}}%
\pgfpathclose%
\pgfpathmoveto{\pgfqpoint{1.000000in}{-0.052500in}}%
\pgfpathcurveto{\pgfqpoint{1.000000in}{-0.052500in}}{\pgfqpoint{0.986077in}{-0.052500in}}{\pgfqpoint{0.972722in}{-0.046968in}}%
\pgfpathcurveto{\pgfqpoint{0.962877in}{-0.037123in}}{\pgfqpoint{0.953032in}{-0.027278in}}{\pgfqpoint{0.947500in}{-0.013923in}}%
\pgfpathcurveto{\pgfqpoint{0.947500in}{0.000000in}}{\pgfqpoint{0.947500in}{0.013923in}}{\pgfqpoint{0.953032in}{0.027278in}}%
\pgfpathcurveto{\pgfqpoint{0.962877in}{0.037123in}}{\pgfqpoint{0.972722in}{0.046968in}}{\pgfqpoint{0.986077in}{0.052500in}}%
\pgfpathcurveto{\pgfqpoint{1.000000in}{0.052500in}}{\pgfqpoint{1.013923in}{0.052500in}}{\pgfqpoint{1.027278in}{0.046968in}}%
\pgfpathcurveto{\pgfqpoint{1.037123in}{0.037123in}}{\pgfqpoint{1.046968in}{0.027278in}}{\pgfqpoint{1.052500in}{0.013923in}}%
\pgfpathcurveto{\pgfqpoint{1.052500in}{0.000000in}}{\pgfqpoint{1.052500in}{-0.013923in}}{\pgfqpoint{1.046968in}{-0.027278in}}%
\pgfpathcurveto{\pgfqpoint{1.037123in}{-0.037123in}}{\pgfqpoint{1.027278in}{-0.046968in}}{\pgfqpoint{1.013923in}{-0.052500in}}%
\pgfpathclose%
\pgfpathmoveto{\pgfqpoint{0.083333in}{0.108333in}}%
\pgfpathcurveto{\pgfqpoint{0.098804in}{0.108333in}}{\pgfqpoint{0.113642in}{0.114480in}}{\pgfqpoint{0.124581in}{0.125419in}}%
\pgfpathcurveto{\pgfqpoint{0.135520in}{0.136358in}}{\pgfqpoint{0.141667in}{0.151196in}}{\pgfqpoint{0.141667in}{0.166667in}}%
\pgfpathcurveto{\pgfqpoint{0.141667in}{0.182137in}}{\pgfqpoint{0.135520in}{0.196975in}}{\pgfqpoint{0.124581in}{0.207915in}}%
\pgfpathcurveto{\pgfqpoint{0.113642in}{0.218854in}}{\pgfqpoint{0.098804in}{0.225000in}}{\pgfqpoint{0.083333in}{0.225000in}}%
\pgfpathcurveto{\pgfqpoint{0.067863in}{0.225000in}}{\pgfqpoint{0.053025in}{0.218854in}}{\pgfqpoint{0.042085in}{0.207915in}}%
\pgfpathcurveto{\pgfqpoint{0.031146in}{0.196975in}}{\pgfqpoint{0.025000in}{0.182137in}}{\pgfqpoint{0.025000in}{0.166667in}}%
\pgfpathcurveto{\pgfqpoint{0.025000in}{0.151196in}}{\pgfqpoint{0.031146in}{0.136358in}}{\pgfqpoint{0.042085in}{0.125419in}}%
\pgfpathcurveto{\pgfqpoint{0.053025in}{0.114480in}}{\pgfqpoint{0.067863in}{0.108333in}}{\pgfqpoint{0.083333in}{0.108333in}}%
\pgfpathclose%
\pgfpathmoveto{\pgfqpoint{0.083333in}{0.114167in}}%
\pgfpathcurveto{\pgfqpoint{0.083333in}{0.114167in}}{\pgfqpoint{0.069410in}{0.114167in}}{\pgfqpoint{0.056055in}{0.119698in}}%
\pgfpathcurveto{\pgfqpoint{0.046210in}{0.129544in}}{\pgfqpoint{0.036365in}{0.139389in}}{\pgfqpoint{0.030833in}{0.152744in}}%
\pgfpathcurveto{\pgfqpoint{0.030833in}{0.166667in}}{\pgfqpoint{0.030833in}{0.180590in}}{\pgfqpoint{0.036365in}{0.193945in}}%
\pgfpathcurveto{\pgfqpoint{0.046210in}{0.203790in}}{\pgfqpoint{0.056055in}{0.213635in}}{\pgfqpoint{0.069410in}{0.219167in}}%
\pgfpathcurveto{\pgfqpoint{0.083333in}{0.219167in}}{\pgfqpoint{0.097256in}{0.219167in}}{\pgfqpoint{0.110611in}{0.213635in}}%
\pgfpathcurveto{\pgfqpoint{0.120456in}{0.203790in}}{\pgfqpoint{0.130302in}{0.193945in}}{\pgfqpoint{0.135833in}{0.180590in}}%
\pgfpathcurveto{\pgfqpoint{0.135833in}{0.166667in}}{\pgfqpoint{0.135833in}{0.152744in}}{\pgfqpoint{0.130302in}{0.139389in}}%
\pgfpathcurveto{\pgfqpoint{0.120456in}{0.129544in}}{\pgfqpoint{0.110611in}{0.119698in}}{\pgfqpoint{0.097256in}{0.114167in}}%
\pgfpathclose%
\pgfpathmoveto{\pgfqpoint{0.250000in}{0.108333in}}%
\pgfpathcurveto{\pgfqpoint{0.265470in}{0.108333in}}{\pgfqpoint{0.280309in}{0.114480in}}{\pgfqpoint{0.291248in}{0.125419in}}%
\pgfpathcurveto{\pgfqpoint{0.302187in}{0.136358in}}{\pgfqpoint{0.308333in}{0.151196in}}{\pgfqpoint{0.308333in}{0.166667in}}%
\pgfpathcurveto{\pgfqpoint{0.308333in}{0.182137in}}{\pgfqpoint{0.302187in}{0.196975in}}{\pgfqpoint{0.291248in}{0.207915in}}%
\pgfpathcurveto{\pgfqpoint{0.280309in}{0.218854in}}{\pgfqpoint{0.265470in}{0.225000in}}{\pgfqpoint{0.250000in}{0.225000in}}%
\pgfpathcurveto{\pgfqpoint{0.234530in}{0.225000in}}{\pgfqpoint{0.219691in}{0.218854in}}{\pgfqpoint{0.208752in}{0.207915in}}%
\pgfpathcurveto{\pgfqpoint{0.197813in}{0.196975in}}{\pgfqpoint{0.191667in}{0.182137in}}{\pgfqpoint{0.191667in}{0.166667in}}%
\pgfpathcurveto{\pgfqpoint{0.191667in}{0.151196in}}{\pgfqpoint{0.197813in}{0.136358in}}{\pgfqpoint{0.208752in}{0.125419in}}%
\pgfpathcurveto{\pgfqpoint{0.219691in}{0.114480in}}{\pgfqpoint{0.234530in}{0.108333in}}{\pgfqpoint{0.250000in}{0.108333in}}%
\pgfpathclose%
\pgfpathmoveto{\pgfqpoint{0.250000in}{0.114167in}}%
\pgfpathcurveto{\pgfqpoint{0.250000in}{0.114167in}}{\pgfqpoint{0.236077in}{0.114167in}}{\pgfqpoint{0.222722in}{0.119698in}}%
\pgfpathcurveto{\pgfqpoint{0.212877in}{0.129544in}}{\pgfqpoint{0.203032in}{0.139389in}}{\pgfqpoint{0.197500in}{0.152744in}}%
\pgfpathcurveto{\pgfqpoint{0.197500in}{0.166667in}}{\pgfqpoint{0.197500in}{0.180590in}}{\pgfqpoint{0.203032in}{0.193945in}}%
\pgfpathcurveto{\pgfqpoint{0.212877in}{0.203790in}}{\pgfqpoint{0.222722in}{0.213635in}}{\pgfqpoint{0.236077in}{0.219167in}}%
\pgfpathcurveto{\pgfqpoint{0.250000in}{0.219167in}}{\pgfqpoint{0.263923in}{0.219167in}}{\pgfqpoint{0.277278in}{0.213635in}}%
\pgfpathcurveto{\pgfqpoint{0.287123in}{0.203790in}}{\pgfqpoint{0.296968in}{0.193945in}}{\pgfqpoint{0.302500in}{0.180590in}}%
\pgfpathcurveto{\pgfqpoint{0.302500in}{0.166667in}}{\pgfqpoint{0.302500in}{0.152744in}}{\pgfqpoint{0.296968in}{0.139389in}}%
\pgfpathcurveto{\pgfqpoint{0.287123in}{0.129544in}}{\pgfqpoint{0.277278in}{0.119698in}}{\pgfqpoint{0.263923in}{0.114167in}}%
\pgfpathclose%
\pgfpathmoveto{\pgfqpoint{0.416667in}{0.108333in}}%
\pgfpathcurveto{\pgfqpoint{0.432137in}{0.108333in}}{\pgfqpoint{0.446975in}{0.114480in}}{\pgfqpoint{0.457915in}{0.125419in}}%
\pgfpathcurveto{\pgfqpoint{0.468854in}{0.136358in}}{\pgfqpoint{0.475000in}{0.151196in}}{\pgfqpoint{0.475000in}{0.166667in}}%
\pgfpathcurveto{\pgfqpoint{0.475000in}{0.182137in}}{\pgfqpoint{0.468854in}{0.196975in}}{\pgfqpoint{0.457915in}{0.207915in}}%
\pgfpathcurveto{\pgfqpoint{0.446975in}{0.218854in}}{\pgfqpoint{0.432137in}{0.225000in}}{\pgfqpoint{0.416667in}{0.225000in}}%
\pgfpathcurveto{\pgfqpoint{0.401196in}{0.225000in}}{\pgfqpoint{0.386358in}{0.218854in}}{\pgfqpoint{0.375419in}{0.207915in}}%
\pgfpathcurveto{\pgfqpoint{0.364480in}{0.196975in}}{\pgfqpoint{0.358333in}{0.182137in}}{\pgfqpoint{0.358333in}{0.166667in}}%
\pgfpathcurveto{\pgfqpoint{0.358333in}{0.151196in}}{\pgfqpoint{0.364480in}{0.136358in}}{\pgfqpoint{0.375419in}{0.125419in}}%
\pgfpathcurveto{\pgfqpoint{0.386358in}{0.114480in}}{\pgfqpoint{0.401196in}{0.108333in}}{\pgfqpoint{0.416667in}{0.108333in}}%
\pgfpathclose%
\pgfpathmoveto{\pgfqpoint{0.416667in}{0.114167in}}%
\pgfpathcurveto{\pgfqpoint{0.416667in}{0.114167in}}{\pgfqpoint{0.402744in}{0.114167in}}{\pgfqpoint{0.389389in}{0.119698in}}%
\pgfpathcurveto{\pgfqpoint{0.379544in}{0.129544in}}{\pgfqpoint{0.369698in}{0.139389in}}{\pgfqpoint{0.364167in}{0.152744in}}%
\pgfpathcurveto{\pgfqpoint{0.364167in}{0.166667in}}{\pgfqpoint{0.364167in}{0.180590in}}{\pgfqpoint{0.369698in}{0.193945in}}%
\pgfpathcurveto{\pgfqpoint{0.379544in}{0.203790in}}{\pgfqpoint{0.389389in}{0.213635in}}{\pgfqpoint{0.402744in}{0.219167in}}%
\pgfpathcurveto{\pgfqpoint{0.416667in}{0.219167in}}{\pgfqpoint{0.430590in}{0.219167in}}{\pgfqpoint{0.443945in}{0.213635in}}%
\pgfpathcurveto{\pgfqpoint{0.453790in}{0.203790in}}{\pgfqpoint{0.463635in}{0.193945in}}{\pgfqpoint{0.469167in}{0.180590in}}%
\pgfpathcurveto{\pgfqpoint{0.469167in}{0.166667in}}{\pgfqpoint{0.469167in}{0.152744in}}{\pgfqpoint{0.463635in}{0.139389in}}%
\pgfpathcurveto{\pgfqpoint{0.453790in}{0.129544in}}{\pgfqpoint{0.443945in}{0.119698in}}{\pgfqpoint{0.430590in}{0.114167in}}%
\pgfpathclose%
\pgfpathmoveto{\pgfqpoint{0.583333in}{0.108333in}}%
\pgfpathcurveto{\pgfqpoint{0.598804in}{0.108333in}}{\pgfqpoint{0.613642in}{0.114480in}}{\pgfqpoint{0.624581in}{0.125419in}}%
\pgfpathcurveto{\pgfqpoint{0.635520in}{0.136358in}}{\pgfqpoint{0.641667in}{0.151196in}}{\pgfqpoint{0.641667in}{0.166667in}}%
\pgfpathcurveto{\pgfqpoint{0.641667in}{0.182137in}}{\pgfqpoint{0.635520in}{0.196975in}}{\pgfqpoint{0.624581in}{0.207915in}}%
\pgfpathcurveto{\pgfqpoint{0.613642in}{0.218854in}}{\pgfqpoint{0.598804in}{0.225000in}}{\pgfqpoint{0.583333in}{0.225000in}}%
\pgfpathcurveto{\pgfqpoint{0.567863in}{0.225000in}}{\pgfqpoint{0.553025in}{0.218854in}}{\pgfqpoint{0.542085in}{0.207915in}}%
\pgfpathcurveto{\pgfqpoint{0.531146in}{0.196975in}}{\pgfqpoint{0.525000in}{0.182137in}}{\pgfqpoint{0.525000in}{0.166667in}}%
\pgfpathcurveto{\pgfqpoint{0.525000in}{0.151196in}}{\pgfqpoint{0.531146in}{0.136358in}}{\pgfqpoint{0.542085in}{0.125419in}}%
\pgfpathcurveto{\pgfqpoint{0.553025in}{0.114480in}}{\pgfqpoint{0.567863in}{0.108333in}}{\pgfqpoint{0.583333in}{0.108333in}}%
\pgfpathclose%
\pgfpathmoveto{\pgfqpoint{0.583333in}{0.114167in}}%
\pgfpathcurveto{\pgfqpoint{0.583333in}{0.114167in}}{\pgfqpoint{0.569410in}{0.114167in}}{\pgfqpoint{0.556055in}{0.119698in}}%
\pgfpathcurveto{\pgfqpoint{0.546210in}{0.129544in}}{\pgfqpoint{0.536365in}{0.139389in}}{\pgfqpoint{0.530833in}{0.152744in}}%
\pgfpathcurveto{\pgfqpoint{0.530833in}{0.166667in}}{\pgfqpoint{0.530833in}{0.180590in}}{\pgfqpoint{0.536365in}{0.193945in}}%
\pgfpathcurveto{\pgfqpoint{0.546210in}{0.203790in}}{\pgfqpoint{0.556055in}{0.213635in}}{\pgfqpoint{0.569410in}{0.219167in}}%
\pgfpathcurveto{\pgfqpoint{0.583333in}{0.219167in}}{\pgfqpoint{0.597256in}{0.219167in}}{\pgfqpoint{0.610611in}{0.213635in}}%
\pgfpathcurveto{\pgfqpoint{0.620456in}{0.203790in}}{\pgfqpoint{0.630302in}{0.193945in}}{\pgfqpoint{0.635833in}{0.180590in}}%
\pgfpathcurveto{\pgfqpoint{0.635833in}{0.166667in}}{\pgfqpoint{0.635833in}{0.152744in}}{\pgfqpoint{0.630302in}{0.139389in}}%
\pgfpathcurveto{\pgfqpoint{0.620456in}{0.129544in}}{\pgfqpoint{0.610611in}{0.119698in}}{\pgfqpoint{0.597256in}{0.114167in}}%
\pgfpathclose%
\pgfpathmoveto{\pgfqpoint{0.750000in}{0.108333in}}%
\pgfpathcurveto{\pgfqpoint{0.765470in}{0.108333in}}{\pgfqpoint{0.780309in}{0.114480in}}{\pgfqpoint{0.791248in}{0.125419in}}%
\pgfpathcurveto{\pgfqpoint{0.802187in}{0.136358in}}{\pgfqpoint{0.808333in}{0.151196in}}{\pgfqpoint{0.808333in}{0.166667in}}%
\pgfpathcurveto{\pgfqpoint{0.808333in}{0.182137in}}{\pgfqpoint{0.802187in}{0.196975in}}{\pgfqpoint{0.791248in}{0.207915in}}%
\pgfpathcurveto{\pgfqpoint{0.780309in}{0.218854in}}{\pgfqpoint{0.765470in}{0.225000in}}{\pgfqpoint{0.750000in}{0.225000in}}%
\pgfpathcurveto{\pgfqpoint{0.734530in}{0.225000in}}{\pgfqpoint{0.719691in}{0.218854in}}{\pgfqpoint{0.708752in}{0.207915in}}%
\pgfpathcurveto{\pgfqpoint{0.697813in}{0.196975in}}{\pgfqpoint{0.691667in}{0.182137in}}{\pgfqpoint{0.691667in}{0.166667in}}%
\pgfpathcurveto{\pgfqpoint{0.691667in}{0.151196in}}{\pgfqpoint{0.697813in}{0.136358in}}{\pgfqpoint{0.708752in}{0.125419in}}%
\pgfpathcurveto{\pgfqpoint{0.719691in}{0.114480in}}{\pgfqpoint{0.734530in}{0.108333in}}{\pgfqpoint{0.750000in}{0.108333in}}%
\pgfpathclose%
\pgfpathmoveto{\pgfqpoint{0.750000in}{0.114167in}}%
\pgfpathcurveto{\pgfqpoint{0.750000in}{0.114167in}}{\pgfqpoint{0.736077in}{0.114167in}}{\pgfqpoint{0.722722in}{0.119698in}}%
\pgfpathcurveto{\pgfqpoint{0.712877in}{0.129544in}}{\pgfqpoint{0.703032in}{0.139389in}}{\pgfqpoint{0.697500in}{0.152744in}}%
\pgfpathcurveto{\pgfqpoint{0.697500in}{0.166667in}}{\pgfqpoint{0.697500in}{0.180590in}}{\pgfqpoint{0.703032in}{0.193945in}}%
\pgfpathcurveto{\pgfqpoint{0.712877in}{0.203790in}}{\pgfqpoint{0.722722in}{0.213635in}}{\pgfqpoint{0.736077in}{0.219167in}}%
\pgfpathcurveto{\pgfqpoint{0.750000in}{0.219167in}}{\pgfqpoint{0.763923in}{0.219167in}}{\pgfqpoint{0.777278in}{0.213635in}}%
\pgfpathcurveto{\pgfqpoint{0.787123in}{0.203790in}}{\pgfqpoint{0.796968in}{0.193945in}}{\pgfqpoint{0.802500in}{0.180590in}}%
\pgfpathcurveto{\pgfqpoint{0.802500in}{0.166667in}}{\pgfqpoint{0.802500in}{0.152744in}}{\pgfqpoint{0.796968in}{0.139389in}}%
\pgfpathcurveto{\pgfqpoint{0.787123in}{0.129544in}}{\pgfqpoint{0.777278in}{0.119698in}}{\pgfqpoint{0.763923in}{0.114167in}}%
\pgfpathclose%
\pgfpathmoveto{\pgfqpoint{0.916667in}{0.108333in}}%
\pgfpathcurveto{\pgfqpoint{0.932137in}{0.108333in}}{\pgfqpoint{0.946975in}{0.114480in}}{\pgfqpoint{0.957915in}{0.125419in}}%
\pgfpathcurveto{\pgfqpoint{0.968854in}{0.136358in}}{\pgfqpoint{0.975000in}{0.151196in}}{\pgfqpoint{0.975000in}{0.166667in}}%
\pgfpathcurveto{\pgfqpoint{0.975000in}{0.182137in}}{\pgfqpoint{0.968854in}{0.196975in}}{\pgfqpoint{0.957915in}{0.207915in}}%
\pgfpathcurveto{\pgfqpoint{0.946975in}{0.218854in}}{\pgfqpoint{0.932137in}{0.225000in}}{\pgfqpoint{0.916667in}{0.225000in}}%
\pgfpathcurveto{\pgfqpoint{0.901196in}{0.225000in}}{\pgfqpoint{0.886358in}{0.218854in}}{\pgfqpoint{0.875419in}{0.207915in}}%
\pgfpathcurveto{\pgfqpoint{0.864480in}{0.196975in}}{\pgfqpoint{0.858333in}{0.182137in}}{\pgfqpoint{0.858333in}{0.166667in}}%
\pgfpathcurveto{\pgfqpoint{0.858333in}{0.151196in}}{\pgfqpoint{0.864480in}{0.136358in}}{\pgfqpoint{0.875419in}{0.125419in}}%
\pgfpathcurveto{\pgfqpoint{0.886358in}{0.114480in}}{\pgfqpoint{0.901196in}{0.108333in}}{\pgfqpoint{0.916667in}{0.108333in}}%
\pgfpathclose%
\pgfpathmoveto{\pgfqpoint{0.916667in}{0.114167in}}%
\pgfpathcurveto{\pgfqpoint{0.916667in}{0.114167in}}{\pgfqpoint{0.902744in}{0.114167in}}{\pgfqpoint{0.889389in}{0.119698in}}%
\pgfpathcurveto{\pgfqpoint{0.879544in}{0.129544in}}{\pgfqpoint{0.869698in}{0.139389in}}{\pgfqpoint{0.864167in}{0.152744in}}%
\pgfpathcurveto{\pgfqpoint{0.864167in}{0.166667in}}{\pgfqpoint{0.864167in}{0.180590in}}{\pgfqpoint{0.869698in}{0.193945in}}%
\pgfpathcurveto{\pgfqpoint{0.879544in}{0.203790in}}{\pgfqpoint{0.889389in}{0.213635in}}{\pgfqpoint{0.902744in}{0.219167in}}%
\pgfpathcurveto{\pgfqpoint{0.916667in}{0.219167in}}{\pgfqpoint{0.930590in}{0.219167in}}{\pgfqpoint{0.943945in}{0.213635in}}%
\pgfpathcurveto{\pgfqpoint{0.953790in}{0.203790in}}{\pgfqpoint{0.963635in}{0.193945in}}{\pgfqpoint{0.969167in}{0.180590in}}%
\pgfpathcurveto{\pgfqpoint{0.969167in}{0.166667in}}{\pgfqpoint{0.969167in}{0.152744in}}{\pgfqpoint{0.963635in}{0.139389in}}%
\pgfpathcurveto{\pgfqpoint{0.953790in}{0.129544in}}{\pgfqpoint{0.943945in}{0.119698in}}{\pgfqpoint{0.930590in}{0.114167in}}%
\pgfpathclose%
\pgfpathmoveto{\pgfqpoint{0.000000in}{0.275000in}}%
\pgfpathcurveto{\pgfqpoint{0.015470in}{0.275000in}}{\pgfqpoint{0.030309in}{0.281146in}}{\pgfqpoint{0.041248in}{0.292085in}}%
\pgfpathcurveto{\pgfqpoint{0.052187in}{0.303025in}}{\pgfqpoint{0.058333in}{0.317863in}}{\pgfqpoint{0.058333in}{0.333333in}}%
\pgfpathcurveto{\pgfqpoint{0.058333in}{0.348804in}}{\pgfqpoint{0.052187in}{0.363642in}}{\pgfqpoint{0.041248in}{0.374581in}}%
\pgfpathcurveto{\pgfqpoint{0.030309in}{0.385520in}}{\pgfqpoint{0.015470in}{0.391667in}}{\pgfqpoint{0.000000in}{0.391667in}}%
\pgfpathcurveto{\pgfqpoint{-0.015470in}{0.391667in}}{\pgfqpoint{-0.030309in}{0.385520in}}{\pgfqpoint{-0.041248in}{0.374581in}}%
\pgfpathcurveto{\pgfqpoint{-0.052187in}{0.363642in}}{\pgfqpoint{-0.058333in}{0.348804in}}{\pgfqpoint{-0.058333in}{0.333333in}}%
\pgfpathcurveto{\pgfqpoint{-0.058333in}{0.317863in}}{\pgfqpoint{-0.052187in}{0.303025in}}{\pgfqpoint{-0.041248in}{0.292085in}}%
\pgfpathcurveto{\pgfqpoint{-0.030309in}{0.281146in}}{\pgfqpoint{-0.015470in}{0.275000in}}{\pgfqpoint{0.000000in}{0.275000in}}%
\pgfpathclose%
\pgfpathmoveto{\pgfqpoint{0.000000in}{0.280833in}}%
\pgfpathcurveto{\pgfqpoint{0.000000in}{0.280833in}}{\pgfqpoint{-0.013923in}{0.280833in}}{\pgfqpoint{-0.027278in}{0.286365in}}%
\pgfpathcurveto{\pgfqpoint{-0.037123in}{0.296210in}}{\pgfqpoint{-0.046968in}{0.306055in}}{\pgfqpoint{-0.052500in}{0.319410in}}%
\pgfpathcurveto{\pgfqpoint{-0.052500in}{0.333333in}}{\pgfqpoint{-0.052500in}{0.347256in}}{\pgfqpoint{-0.046968in}{0.360611in}}%
\pgfpathcurveto{\pgfqpoint{-0.037123in}{0.370456in}}{\pgfqpoint{-0.027278in}{0.380302in}}{\pgfqpoint{-0.013923in}{0.385833in}}%
\pgfpathcurveto{\pgfqpoint{0.000000in}{0.385833in}}{\pgfqpoint{0.013923in}{0.385833in}}{\pgfqpoint{0.027278in}{0.380302in}}%
\pgfpathcurveto{\pgfqpoint{0.037123in}{0.370456in}}{\pgfqpoint{0.046968in}{0.360611in}}{\pgfqpoint{0.052500in}{0.347256in}}%
\pgfpathcurveto{\pgfqpoint{0.052500in}{0.333333in}}{\pgfqpoint{0.052500in}{0.319410in}}{\pgfqpoint{0.046968in}{0.306055in}}%
\pgfpathcurveto{\pgfqpoint{0.037123in}{0.296210in}}{\pgfqpoint{0.027278in}{0.286365in}}{\pgfqpoint{0.013923in}{0.280833in}}%
\pgfpathclose%
\pgfpathmoveto{\pgfqpoint{0.166667in}{0.275000in}}%
\pgfpathcurveto{\pgfqpoint{0.182137in}{0.275000in}}{\pgfqpoint{0.196975in}{0.281146in}}{\pgfqpoint{0.207915in}{0.292085in}}%
\pgfpathcurveto{\pgfqpoint{0.218854in}{0.303025in}}{\pgfqpoint{0.225000in}{0.317863in}}{\pgfqpoint{0.225000in}{0.333333in}}%
\pgfpathcurveto{\pgfqpoint{0.225000in}{0.348804in}}{\pgfqpoint{0.218854in}{0.363642in}}{\pgfqpoint{0.207915in}{0.374581in}}%
\pgfpathcurveto{\pgfqpoint{0.196975in}{0.385520in}}{\pgfqpoint{0.182137in}{0.391667in}}{\pgfqpoint{0.166667in}{0.391667in}}%
\pgfpathcurveto{\pgfqpoint{0.151196in}{0.391667in}}{\pgfqpoint{0.136358in}{0.385520in}}{\pgfqpoint{0.125419in}{0.374581in}}%
\pgfpathcurveto{\pgfqpoint{0.114480in}{0.363642in}}{\pgfqpoint{0.108333in}{0.348804in}}{\pgfqpoint{0.108333in}{0.333333in}}%
\pgfpathcurveto{\pgfqpoint{0.108333in}{0.317863in}}{\pgfqpoint{0.114480in}{0.303025in}}{\pgfqpoint{0.125419in}{0.292085in}}%
\pgfpathcurveto{\pgfqpoint{0.136358in}{0.281146in}}{\pgfqpoint{0.151196in}{0.275000in}}{\pgfqpoint{0.166667in}{0.275000in}}%
\pgfpathclose%
\pgfpathmoveto{\pgfqpoint{0.166667in}{0.280833in}}%
\pgfpathcurveto{\pgfqpoint{0.166667in}{0.280833in}}{\pgfqpoint{0.152744in}{0.280833in}}{\pgfqpoint{0.139389in}{0.286365in}}%
\pgfpathcurveto{\pgfqpoint{0.129544in}{0.296210in}}{\pgfqpoint{0.119698in}{0.306055in}}{\pgfqpoint{0.114167in}{0.319410in}}%
\pgfpathcurveto{\pgfqpoint{0.114167in}{0.333333in}}{\pgfqpoint{0.114167in}{0.347256in}}{\pgfqpoint{0.119698in}{0.360611in}}%
\pgfpathcurveto{\pgfqpoint{0.129544in}{0.370456in}}{\pgfqpoint{0.139389in}{0.380302in}}{\pgfqpoint{0.152744in}{0.385833in}}%
\pgfpathcurveto{\pgfqpoint{0.166667in}{0.385833in}}{\pgfqpoint{0.180590in}{0.385833in}}{\pgfqpoint{0.193945in}{0.380302in}}%
\pgfpathcurveto{\pgfqpoint{0.203790in}{0.370456in}}{\pgfqpoint{0.213635in}{0.360611in}}{\pgfqpoint{0.219167in}{0.347256in}}%
\pgfpathcurveto{\pgfqpoint{0.219167in}{0.333333in}}{\pgfqpoint{0.219167in}{0.319410in}}{\pgfqpoint{0.213635in}{0.306055in}}%
\pgfpathcurveto{\pgfqpoint{0.203790in}{0.296210in}}{\pgfqpoint{0.193945in}{0.286365in}}{\pgfqpoint{0.180590in}{0.280833in}}%
\pgfpathclose%
\pgfpathmoveto{\pgfqpoint{0.333333in}{0.275000in}}%
\pgfpathcurveto{\pgfqpoint{0.348804in}{0.275000in}}{\pgfqpoint{0.363642in}{0.281146in}}{\pgfqpoint{0.374581in}{0.292085in}}%
\pgfpathcurveto{\pgfqpoint{0.385520in}{0.303025in}}{\pgfqpoint{0.391667in}{0.317863in}}{\pgfqpoint{0.391667in}{0.333333in}}%
\pgfpathcurveto{\pgfqpoint{0.391667in}{0.348804in}}{\pgfqpoint{0.385520in}{0.363642in}}{\pgfqpoint{0.374581in}{0.374581in}}%
\pgfpathcurveto{\pgfqpoint{0.363642in}{0.385520in}}{\pgfqpoint{0.348804in}{0.391667in}}{\pgfqpoint{0.333333in}{0.391667in}}%
\pgfpathcurveto{\pgfqpoint{0.317863in}{0.391667in}}{\pgfqpoint{0.303025in}{0.385520in}}{\pgfqpoint{0.292085in}{0.374581in}}%
\pgfpathcurveto{\pgfqpoint{0.281146in}{0.363642in}}{\pgfqpoint{0.275000in}{0.348804in}}{\pgfqpoint{0.275000in}{0.333333in}}%
\pgfpathcurveto{\pgfqpoint{0.275000in}{0.317863in}}{\pgfqpoint{0.281146in}{0.303025in}}{\pgfqpoint{0.292085in}{0.292085in}}%
\pgfpathcurveto{\pgfqpoint{0.303025in}{0.281146in}}{\pgfqpoint{0.317863in}{0.275000in}}{\pgfqpoint{0.333333in}{0.275000in}}%
\pgfpathclose%
\pgfpathmoveto{\pgfqpoint{0.333333in}{0.280833in}}%
\pgfpathcurveto{\pgfqpoint{0.333333in}{0.280833in}}{\pgfqpoint{0.319410in}{0.280833in}}{\pgfqpoint{0.306055in}{0.286365in}}%
\pgfpathcurveto{\pgfqpoint{0.296210in}{0.296210in}}{\pgfqpoint{0.286365in}{0.306055in}}{\pgfqpoint{0.280833in}{0.319410in}}%
\pgfpathcurveto{\pgfqpoint{0.280833in}{0.333333in}}{\pgfqpoint{0.280833in}{0.347256in}}{\pgfqpoint{0.286365in}{0.360611in}}%
\pgfpathcurveto{\pgfqpoint{0.296210in}{0.370456in}}{\pgfqpoint{0.306055in}{0.380302in}}{\pgfqpoint{0.319410in}{0.385833in}}%
\pgfpathcurveto{\pgfqpoint{0.333333in}{0.385833in}}{\pgfqpoint{0.347256in}{0.385833in}}{\pgfqpoint{0.360611in}{0.380302in}}%
\pgfpathcurveto{\pgfqpoint{0.370456in}{0.370456in}}{\pgfqpoint{0.380302in}{0.360611in}}{\pgfqpoint{0.385833in}{0.347256in}}%
\pgfpathcurveto{\pgfqpoint{0.385833in}{0.333333in}}{\pgfqpoint{0.385833in}{0.319410in}}{\pgfqpoint{0.380302in}{0.306055in}}%
\pgfpathcurveto{\pgfqpoint{0.370456in}{0.296210in}}{\pgfqpoint{0.360611in}{0.286365in}}{\pgfqpoint{0.347256in}{0.280833in}}%
\pgfpathclose%
\pgfpathmoveto{\pgfqpoint{0.500000in}{0.275000in}}%
\pgfpathcurveto{\pgfqpoint{0.515470in}{0.275000in}}{\pgfqpoint{0.530309in}{0.281146in}}{\pgfqpoint{0.541248in}{0.292085in}}%
\pgfpathcurveto{\pgfqpoint{0.552187in}{0.303025in}}{\pgfqpoint{0.558333in}{0.317863in}}{\pgfqpoint{0.558333in}{0.333333in}}%
\pgfpathcurveto{\pgfqpoint{0.558333in}{0.348804in}}{\pgfqpoint{0.552187in}{0.363642in}}{\pgfqpoint{0.541248in}{0.374581in}}%
\pgfpathcurveto{\pgfqpoint{0.530309in}{0.385520in}}{\pgfqpoint{0.515470in}{0.391667in}}{\pgfqpoint{0.500000in}{0.391667in}}%
\pgfpathcurveto{\pgfqpoint{0.484530in}{0.391667in}}{\pgfqpoint{0.469691in}{0.385520in}}{\pgfqpoint{0.458752in}{0.374581in}}%
\pgfpathcurveto{\pgfqpoint{0.447813in}{0.363642in}}{\pgfqpoint{0.441667in}{0.348804in}}{\pgfqpoint{0.441667in}{0.333333in}}%
\pgfpathcurveto{\pgfqpoint{0.441667in}{0.317863in}}{\pgfqpoint{0.447813in}{0.303025in}}{\pgfqpoint{0.458752in}{0.292085in}}%
\pgfpathcurveto{\pgfqpoint{0.469691in}{0.281146in}}{\pgfqpoint{0.484530in}{0.275000in}}{\pgfqpoint{0.500000in}{0.275000in}}%
\pgfpathclose%
\pgfpathmoveto{\pgfqpoint{0.500000in}{0.280833in}}%
\pgfpathcurveto{\pgfqpoint{0.500000in}{0.280833in}}{\pgfqpoint{0.486077in}{0.280833in}}{\pgfqpoint{0.472722in}{0.286365in}}%
\pgfpathcurveto{\pgfqpoint{0.462877in}{0.296210in}}{\pgfqpoint{0.453032in}{0.306055in}}{\pgfqpoint{0.447500in}{0.319410in}}%
\pgfpathcurveto{\pgfqpoint{0.447500in}{0.333333in}}{\pgfqpoint{0.447500in}{0.347256in}}{\pgfqpoint{0.453032in}{0.360611in}}%
\pgfpathcurveto{\pgfqpoint{0.462877in}{0.370456in}}{\pgfqpoint{0.472722in}{0.380302in}}{\pgfqpoint{0.486077in}{0.385833in}}%
\pgfpathcurveto{\pgfqpoint{0.500000in}{0.385833in}}{\pgfqpoint{0.513923in}{0.385833in}}{\pgfqpoint{0.527278in}{0.380302in}}%
\pgfpathcurveto{\pgfqpoint{0.537123in}{0.370456in}}{\pgfqpoint{0.546968in}{0.360611in}}{\pgfqpoint{0.552500in}{0.347256in}}%
\pgfpathcurveto{\pgfqpoint{0.552500in}{0.333333in}}{\pgfqpoint{0.552500in}{0.319410in}}{\pgfqpoint{0.546968in}{0.306055in}}%
\pgfpathcurveto{\pgfqpoint{0.537123in}{0.296210in}}{\pgfqpoint{0.527278in}{0.286365in}}{\pgfqpoint{0.513923in}{0.280833in}}%
\pgfpathclose%
\pgfpathmoveto{\pgfqpoint{0.666667in}{0.275000in}}%
\pgfpathcurveto{\pgfqpoint{0.682137in}{0.275000in}}{\pgfqpoint{0.696975in}{0.281146in}}{\pgfqpoint{0.707915in}{0.292085in}}%
\pgfpathcurveto{\pgfqpoint{0.718854in}{0.303025in}}{\pgfqpoint{0.725000in}{0.317863in}}{\pgfqpoint{0.725000in}{0.333333in}}%
\pgfpathcurveto{\pgfqpoint{0.725000in}{0.348804in}}{\pgfqpoint{0.718854in}{0.363642in}}{\pgfqpoint{0.707915in}{0.374581in}}%
\pgfpathcurveto{\pgfqpoint{0.696975in}{0.385520in}}{\pgfqpoint{0.682137in}{0.391667in}}{\pgfqpoint{0.666667in}{0.391667in}}%
\pgfpathcurveto{\pgfqpoint{0.651196in}{0.391667in}}{\pgfqpoint{0.636358in}{0.385520in}}{\pgfqpoint{0.625419in}{0.374581in}}%
\pgfpathcurveto{\pgfqpoint{0.614480in}{0.363642in}}{\pgfqpoint{0.608333in}{0.348804in}}{\pgfqpoint{0.608333in}{0.333333in}}%
\pgfpathcurveto{\pgfqpoint{0.608333in}{0.317863in}}{\pgfqpoint{0.614480in}{0.303025in}}{\pgfqpoint{0.625419in}{0.292085in}}%
\pgfpathcurveto{\pgfqpoint{0.636358in}{0.281146in}}{\pgfqpoint{0.651196in}{0.275000in}}{\pgfqpoint{0.666667in}{0.275000in}}%
\pgfpathclose%
\pgfpathmoveto{\pgfqpoint{0.666667in}{0.280833in}}%
\pgfpathcurveto{\pgfqpoint{0.666667in}{0.280833in}}{\pgfqpoint{0.652744in}{0.280833in}}{\pgfqpoint{0.639389in}{0.286365in}}%
\pgfpathcurveto{\pgfqpoint{0.629544in}{0.296210in}}{\pgfqpoint{0.619698in}{0.306055in}}{\pgfqpoint{0.614167in}{0.319410in}}%
\pgfpathcurveto{\pgfqpoint{0.614167in}{0.333333in}}{\pgfqpoint{0.614167in}{0.347256in}}{\pgfqpoint{0.619698in}{0.360611in}}%
\pgfpathcurveto{\pgfqpoint{0.629544in}{0.370456in}}{\pgfqpoint{0.639389in}{0.380302in}}{\pgfqpoint{0.652744in}{0.385833in}}%
\pgfpathcurveto{\pgfqpoint{0.666667in}{0.385833in}}{\pgfqpoint{0.680590in}{0.385833in}}{\pgfqpoint{0.693945in}{0.380302in}}%
\pgfpathcurveto{\pgfqpoint{0.703790in}{0.370456in}}{\pgfqpoint{0.713635in}{0.360611in}}{\pgfqpoint{0.719167in}{0.347256in}}%
\pgfpathcurveto{\pgfqpoint{0.719167in}{0.333333in}}{\pgfqpoint{0.719167in}{0.319410in}}{\pgfqpoint{0.713635in}{0.306055in}}%
\pgfpathcurveto{\pgfqpoint{0.703790in}{0.296210in}}{\pgfqpoint{0.693945in}{0.286365in}}{\pgfqpoint{0.680590in}{0.280833in}}%
\pgfpathclose%
\pgfpathmoveto{\pgfqpoint{0.833333in}{0.275000in}}%
\pgfpathcurveto{\pgfqpoint{0.848804in}{0.275000in}}{\pgfqpoint{0.863642in}{0.281146in}}{\pgfqpoint{0.874581in}{0.292085in}}%
\pgfpathcurveto{\pgfqpoint{0.885520in}{0.303025in}}{\pgfqpoint{0.891667in}{0.317863in}}{\pgfqpoint{0.891667in}{0.333333in}}%
\pgfpathcurveto{\pgfqpoint{0.891667in}{0.348804in}}{\pgfqpoint{0.885520in}{0.363642in}}{\pgfqpoint{0.874581in}{0.374581in}}%
\pgfpathcurveto{\pgfqpoint{0.863642in}{0.385520in}}{\pgfqpoint{0.848804in}{0.391667in}}{\pgfqpoint{0.833333in}{0.391667in}}%
\pgfpathcurveto{\pgfqpoint{0.817863in}{0.391667in}}{\pgfqpoint{0.803025in}{0.385520in}}{\pgfqpoint{0.792085in}{0.374581in}}%
\pgfpathcurveto{\pgfqpoint{0.781146in}{0.363642in}}{\pgfqpoint{0.775000in}{0.348804in}}{\pgfqpoint{0.775000in}{0.333333in}}%
\pgfpathcurveto{\pgfqpoint{0.775000in}{0.317863in}}{\pgfqpoint{0.781146in}{0.303025in}}{\pgfqpoint{0.792085in}{0.292085in}}%
\pgfpathcurveto{\pgfqpoint{0.803025in}{0.281146in}}{\pgfqpoint{0.817863in}{0.275000in}}{\pgfqpoint{0.833333in}{0.275000in}}%
\pgfpathclose%
\pgfpathmoveto{\pgfqpoint{0.833333in}{0.280833in}}%
\pgfpathcurveto{\pgfqpoint{0.833333in}{0.280833in}}{\pgfqpoint{0.819410in}{0.280833in}}{\pgfqpoint{0.806055in}{0.286365in}}%
\pgfpathcurveto{\pgfqpoint{0.796210in}{0.296210in}}{\pgfqpoint{0.786365in}{0.306055in}}{\pgfqpoint{0.780833in}{0.319410in}}%
\pgfpathcurveto{\pgfqpoint{0.780833in}{0.333333in}}{\pgfqpoint{0.780833in}{0.347256in}}{\pgfqpoint{0.786365in}{0.360611in}}%
\pgfpathcurveto{\pgfqpoint{0.796210in}{0.370456in}}{\pgfqpoint{0.806055in}{0.380302in}}{\pgfqpoint{0.819410in}{0.385833in}}%
\pgfpathcurveto{\pgfqpoint{0.833333in}{0.385833in}}{\pgfqpoint{0.847256in}{0.385833in}}{\pgfqpoint{0.860611in}{0.380302in}}%
\pgfpathcurveto{\pgfqpoint{0.870456in}{0.370456in}}{\pgfqpoint{0.880302in}{0.360611in}}{\pgfqpoint{0.885833in}{0.347256in}}%
\pgfpathcurveto{\pgfqpoint{0.885833in}{0.333333in}}{\pgfqpoint{0.885833in}{0.319410in}}{\pgfqpoint{0.880302in}{0.306055in}}%
\pgfpathcurveto{\pgfqpoint{0.870456in}{0.296210in}}{\pgfqpoint{0.860611in}{0.286365in}}{\pgfqpoint{0.847256in}{0.280833in}}%
\pgfpathclose%
\pgfpathmoveto{\pgfqpoint{1.000000in}{0.275000in}}%
\pgfpathcurveto{\pgfqpoint{1.015470in}{0.275000in}}{\pgfqpoint{1.030309in}{0.281146in}}{\pgfqpoint{1.041248in}{0.292085in}}%
\pgfpathcurveto{\pgfqpoint{1.052187in}{0.303025in}}{\pgfqpoint{1.058333in}{0.317863in}}{\pgfqpoint{1.058333in}{0.333333in}}%
\pgfpathcurveto{\pgfqpoint{1.058333in}{0.348804in}}{\pgfqpoint{1.052187in}{0.363642in}}{\pgfqpoint{1.041248in}{0.374581in}}%
\pgfpathcurveto{\pgfqpoint{1.030309in}{0.385520in}}{\pgfqpoint{1.015470in}{0.391667in}}{\pgfqpoint{1.000000in}{0.391667in}}%
\pgfpathcurveto{\pgfqpoint{0.984530in}{0.391667in}}{\pgfqpoint{0.969691in}{0.385520in}}{\pgfqpoint{0.958752in}{0.374581in}}%
\pgfpathcurveto{\pgfqpoint{0.947813in}{0.363642in}}{\pgfqpoint{0.941667in}{0.348804in}}{\pgfqpoint{0.941667in}{0.333333in}}%
\pgfpathcurveto{\pgfqpoint{0.941667in}{0.317863in}}{\pgfqpoint{0.947813in}{0.303025in}}{\pgfqpoint{0.958752in}{0.292085in}}%
\pgfpathcurveto{\pgfqpoint{0.969691in}{0.281146in}}{\pgfqpoint{0.984530in}{0.275000in}}{\pgfqpoint{1.000000in}{0.275000in}}%
\pgfpathclose%
\pgfpathmoveto{\pgfqpoint{1.000000in}{0.280833in}}%
\pgfpathcurveto{\pgfqpoint{1.000000in}{0.280833in}}{\pgfqpoint{0.986077in}{0.280833in}}{\pgfqpoint{0.972722in}{0.286365in}}%
\pgfpathcurveto{\pgfqpoint{0.962877in}{0.296210in}}{\pgfqpoint{0.953032in}{0.306055in}}{\pgfqpoint{0.947500in}{0.319410in}}%
\pgfpathcurveto{\pgfqpoint{0.947500in}{0.333333in}}{\pgfqpoint{0.947500in}{0.347256in}}{\pgfqpoint{0.953032in}{0.360611in}}%
\pgfpathcurveto{\pgfqpoint{0.962877in}{0.370456in}}{\pgfqpoint{0.972722in}{0.380302in}}{\pgfqpoint{0.986077in}{0.385833in}}%
\pgfpathcurveto{\pgfqpoint{1.000000in}{0.385833in}}{\pgfqpoint{1.013923in}{0.385833in}}{\pgfqpoint{1.027278in}{0.380302in}}%
\pgfpathcurveto{\pgfqpoint{1.037123in}{0.370456in}}{\pgfqpoint{1.046968in}{0.360611in}}{\pgfqpoint{1.052500in}{0.347256in}}%
\pgfpathcurveto{\pgfqpoint{1.052500in}{0.333333in}}{\pgfqpoint{1.052500in}{0.319410in}}{\pgfqpoint{1.046968in}{0.306055in}}%
\pgfpathcurveto{\pgfqpoint{1.037123in}{0.296210in}}{\pgfqpoint{1.027278in}{0.286365in}}{\pgfqpoint{1.013923in}{0.280833in}}%
\pgfpathclose%
\pgfpathmoveto{\pgfqpoint{0.083333in}{0.441667in}}%
\pgfpathcurveto{\pgfqpoint{0.098804in}{0.441667in}}{\pgfqpoint{0.113642in}{0.447813in}}{\pgfqpoint{0.124581in}{0.458752in}}%
\pgfpathcurveto{\pgfqpoint{0.135520in}{0.469691in}}{\pgfqpoint{0.141667in}{0.484530in}}{\pgfqpoint{0.141667in}{0.500000in}}%
\pgfpathcurveto{\pgfqpoint{0.141667in}{0.515470in}}{\pgfqpoint{0.135520in}{0.530309in}}{\pgfqpoint{0.124581in}{0.541248in}}%
\pgfpathcurveto{\pgfqpoint{0.113642in}{0.552187in}}{\pgfqpoint{0.098804in}{0.558333in}}{\pgfqpoint{0.083333in}{0.558333in}}%
\pgfpathcurveto{\pgfqpoint{0.067863in}{0.558333in}}{\pgfqpoint{0.053025in}{0.552187in}}{\pgfqpoint{0.042085in}{0.541248in}}%
\pgfpathcurveto{\pgfqpoint{0.031146in}{0.530309in}}{\pgfqpoint{0.025000in}{0.515470in}}{\pgfqpoint{0.025000in}{0.500000in}}%
\pgfpathcurveto{\pgfqpoint{0.025000in}{0.484530in}}{\pgfqpoint{0.031146in}{0.469691in}}{\pgfqpoint{0.042085in}{0.458752in}}%
\pgfpathcurveto{\pgfqpoint{0.053025in}{0.447813in}}{\pgfqpoint{0.067863in}{0.441667in}}{\pgfqpoint{0.083333in}{0.441667in}}%
\pgfpathclose%
\pgfpathmoveto{\pgfqpoint{0.083333in}{0.447500in}}%
\pgfpathcurveto{\pgfqpoint{0.083333in}{0.447500in}}{\pgfqpoint{0.069410in}{0.447500in}}{\pgfqpoint{0.056055in}{0.453032in}}%
\pgfpathcurveto{\pgfqpoint{0.046210in}{0.462877in}}{\pgfqpoint{0.036365in}{0.472722in}}{\pgfqpoint{0.030833in}{0.486077in}}%
\pgfpathcurveto{\pgfqpoint{0.030833in}{0.500000in}}{\pgfqpoint{0.030833in}{0.513923in}}{\pgfqpoint{0.036365in}{0.527278in}}%
\pgfpathcurveto{\pgfqpoint{0.046210in}{0.537123in}}{\pgfqpoint{0.056055in}{0.546968in}}{\pgfqpoint{0.069410in}{0.552500in}}%
\pgfpathcurveto{\pgfqpoint{0.083333in}{0.552500in}}{\pgfqpoint{0.097256in}{0.552500in}}{\pgfqpoint{0.110611in}{0.546968in}}%
\pgfpathcurveto{\pgfqpoint{0.120456in}{0.537123in}}{\pgfqpoint{0.130302in}{0.527278in}}{\pgfqpoint{0.135833in}{0.513923in}}%
\pgfpathcurveto{\pgfqpoint{0.135833in}{0.500000in}}{\pgfqpoint{0.135833in}{0.486077in}}{\pgfqpoint{0.130302in}{0.472722in}}%
\pgfpathcurveto{\pgfqpoint{0.120456in}{0.462877in}}{\pgfqpoint{0.110611in}{0.453032in}}{\pgfqpoint{0.097256in}{0.447500in}}%
\pgfpathclose%
\pgfpathmoveto{\pgfqpoint{0.250000in}{0.441667in}}%
\pgfpathcurveto{\pgfqpoint{0.265470in}{0.441667in}}{\pgfqpoint{0.280309in}{0.447813in}}{\pgfqpoint{0.291248in}{0.458752in}}%
\pgfpathcurveto{\pgfqpoint{0.302187in}{0.469691in}}{\pgfqpoint{0.308333in}{0.484530in}}{\pgfqpoint{0.308333in}{0.500000in}}%
\pgfpathcurveto{\pgfqpoint{0.308333in}{0.515470in}}{\pgfqpoint{0.302187in}{0.530309in}}{\pgfqpoint{0.291248in}{0.541248in}}%
\pgfpathcurveto{\pgfqpoint{0.280309in}{0.552187in}}{\pgfqpoint{0.265470in}{0.558333in}}{\pgfqpoint{0.250000in}{0.558333in}}%
\pgfpathcurveto{\pgfqpoint{0.234530in}{0.558333in}}{\pgfqpoint{0.219691in}{0.552187in}}{\pgfqpoint{0.208752in}{0.541248in}}%
\pgfpathcurveto{\pgfqpoint{0.197813in}{0.530309in}}{\pgfqpoint{0.191667in}{0.515470in}}{\pgfqpoint{0.191667in}{0.500000in}}%
\pgfpathcurveto{\pgfqpoint{0.191667in}{0.484530in}}{\pgfqpoint{0.197813in}{0.469691in}}{\pgfqpoint{0.208752in}{0.458752in}}%
\pgfpathcurveto{\pgfqpoint{0.219691in}{0.447813in}}{\pgfqpoint{0.234530in}{0.441667in}}{\pgfqpoint{0.250000in}{0.441667in}}%
\pgfpathclose%
\pgfpathmoveto{\pgfqpoint{0.250000in}{0.447500in}}%
\pgfpathcurveto{\pgfqpoint{0.250000in}{0.447500in}}{\pgfqpoint{0.236077in}{0.447500in}}{\pgfqpoint{0.222722in}{0.453032in}}%
\pgfpathcurveto{\pgfqpoint{0.212877in}{0.462877in}}{\pgfqpoint{0.203032in}{0.472722in}}{\pgfqpoint{0.197500in}{0.486077in}}%
\pgfpathcurveto{\pgfqpoint{0.197500in}{0.500000in}}{\pgfqpoint{0.197500in}{0.513923in}}{\pgfqpoint{0.203032in}{0.527278in}}%
\pgfpathcurveto{\pgfqpoint{0.212877in}{0.537123in}}{\pgfqpoint{0.222722in}{0.546968in}}{\pgfqpoint{0.236077in}{0.552500in}}%
\pgfpathcurveto{\pgfqpoint{0.250000in}{0.552500in}}{\pgfqpoint{0.263923in}{0.552500in}}{\pgfqpoint{0.277278in}{0.546968in}}%
\pgfpathcurveto{\pgfqpoint{0.287123in}{0.537123in}}{\pgfqpoint{0.296968in}{0.527278in}}{\pgfqpoint{0.302500in}{0.513923in}}%
\pgfpathcurveto{\pgfqpoint{0.302500in}{0.500000in}}{\pgfqpoint{0.302500in}{0.486077in}}{\pgfqpoint{0.296968in}{0.472722in}}%
\pgfpathcurveto{\pgfqpoint{0.287123in}{0.462877in}}{\pgfqpoint{0.277278in}{0.453032in}}{\pgfqpoint{0.263923in}{0.447500in}}%
\pgfpathclose%
\pgfpathmoveto{\pgfqpoint{0.416667in}{0.441667in}}%
\pgfpathcurveto{\pgfqpoint{0.432137in}{0.441667in}}{\pgfqpoint{0.446975in}{0.447813in}}{\pgfqpoint{0.457915in}{0.458752in}}%
\pgfpathcurveto{\pgfqpoint{0.468854in}{0.469691in}}{\pgfqpoint{0.475000in}{0.484530in}}{\pgfqpoint{0.475000in}{0.500000in}}%
\pgfpathcurveto{\pgfqpoint{0.475000in}{0.515470in}}{\pgfqpoint{0.468854in}{0.530309in}}{\pgfqpoint{0.457915in}{0.541248in}}%
\pgfpathcurveto{\pgfqpoint{0.446975in}{0.552187in}}{\pgfqpoint{0.432137in}{0.558333in}}{\pgfqpoint{0.416667in}{0.558333in}}%
\pgfpathcurveto{\pgfqpoint{0.401196in}{0.558333in}}{\pgfqpoint{0.386358in}{0.552187in}}{\pgfqpoint{0.375419in}{0.541248in}}%
\pgfpathcurveto{\pgfqpoint{0.364480in}{0.530309in}}{\pgfqpoint{0.358333in}{0.515470in}}{\pgfqpoint{0.358333in}{0.500000in}}%
\pgfpathcurveto{\pgfqpoint{0.358333in}{0.484530in}}{\pgfqpoint{0.364480in}{0.469691in}}{\pgfqpoint{0.375419in}{0.458752in}}%
\pgfpathcurveto{\pgfqpoint{0.386358in}{0.447813in}}{\pgfqpoint{0.401196in}{0.441667in}}{\pgfqpoint{0.416667in}{0.441667in}}%
\pgfpathclose%
\pgfpathmoveto{\pgfqpoint{0.416667in}{0.447500in}}%
\pgfpathcurveto{\pgfqpoint{0.416667in}{0.447500in}}{\pgfqpoint{0.402744in}{0.447500in}}{\pgfqpoint{0.389389in}{0.453032in}}%
\pgfpathcurveto{\pgfqpoint{0.379544in}{0.462877in}}{\pgfqpoint{0.369698in}{0.472722in}}{\pgfqpoint{0.364167in}{0.486077in}}%
\pgfpathcurveto{\pgfqpoint{0.364167in}{0.500000in}}{\pgfqpoint{0.364167in}{0.513923in}}{\pgfqpoint{0.369698in}{0.527278in}}%
\pgfpathcurveto{\pgfqpoint{0.379544in}{0.537123in}}{\pgfqpoint{0.389389in}{0.546968in}}{\pgfqpoint{0.402744in}{0.552500in}}%
\pgfpathcurveto{\pgfqpoint{0.416667in}{0.552500in}}{\pgfqpoint{0.430590in}{0.552500in}}{\pgfqpoint{0.443945in}{0.546968in}}%
\pgfpathcurveto{\pgfqpoint{0.453790in}{0.537123in}}{\pgfqpoint{0.463635in}{0.527278in}}{\pgfqpoint{0.469167in}{0.513923in}}%
\pgfpathcurveto{\pgfqpoint{0.469167in}{0.500000in}}{\pgfqpoint{0.469167in}{0.486077in}}{\pgfqpoint{0.463635in}{0.472722in}}%
\pgfpathcurveto{\pgfqpoint{0.453790in}{0.462877in}}{\pgfqpoint{0.443945in}{0.453032in}}{\pgfqpoint{0.430590in}{0.447500in}}%
\pgfpathclose%
\pgfpathmoveto{\pgfqpoint{0.583333in}{0.441667in}}%
\pgfpathcurveto{\pgfqpoint{0.598804in}{0.441667in}}{\pgfqpoint{0.613642in}{0.447813in}}{\pgfqpoint{0.624581in}{0.458752in}}%
\pgfpathcurveto{\pgfqpoint{0.635520in}{0.469691in}}{\pgfqpoint{0.641667in}{0.484530in}}{\pgfqpoint{0.641667in}{0.500000in}}%
\pgfpathcurveto{\pgfqpoint{0.641667in}{0.515470in}}{\pgfqpoint{0.635520in}{0.530309in}}{\pgfqpoint{0.624581in}{0.541248in}}%
\pgfpathcurveto{\pgfqpoint{0.613642in}{0.552187in}}{\pgfqpoint{0.598804in}{0.558333in}}{\pgfqpoint{0.583333in}{0.558333in}}%
\pgfpathcurveto{\pgfqpoint{0.567863in}{0.558333in}}{\pgfqpoint{0.553025in}{0.552187in}}{\pgfqpoint{0.542085in}{0.541248in}}%
\pgfpathcurveto{\pgfqpoint{0.531146in}{0.530309in}}{\pgfqpoint{0.525000in}{0.515470in}}{\pgfqpoint{0.525000in}{0.500000in}}%
\pgfpathcurveto{\pgfqpoint{0.525000in}{0.484530in}}{\pgfqpoint{0.531146in}{0.469691in}}{\pgfqpoint{0.542085in}{0.458752in}}%
\pgfpathcurveto{\pgfqpoint{0.553025in}{0.447813in}}{\pgfqpoint{0.567863in}{0.441667in}}{\pgfqpoint{0.583333in}{0.441667in}}%
\pgfpathclose%
\pgfpathmoveto{\pgfqpoint{0.583333in}{0.447500in}}%
\pgfpathcurveto{\pgfqpoint{0.583333in}{0.447500in}}{\pgfqpoint{0.569410in}{0.447500in}}{\pgfqpoint{0.556055in}{0.453032in}}%
\pgfpathcurveto{\pgfqpoint{0.546210in}{0.462877in}}{\pgfqpoint{0.536365in}{0.472722in}}{\pgfqpoint{0.530833in}{0.486077in}}%
\pgfpathcurveto{\pgfqpoint{0.530833in}{0.500000in}}{\pgfqpoint{0.530833in}{0.513923in}}{\pgfqpoint{0.536365in}{0.527278in}}%
\pgfpathcurveto{\pgfqpoint{0.546210in}{0.537123in}}{\pgfqpoint{0.556055in}{0.546968in}}{\pgfqpoint{0.569410in}{0.552500in}}%
\pgfpathcurveto{\pgfqpoint{0.583333in}{0.552500in}}{\pgfqpoint{0.597256in}{0.552500in}}{\pgfqpoint{0.610611in}{0.546968in}}%
\pgfpathcurveto{\pgfqpoint{0.620456in}{0.537123in}}{\pgfqpoint{0.630302in}{0.527278in}}{\pgfqpoint{0.635833in}{0.513923in}}%
\pgfpathcurveto{\pgfqpoint{0.635833in}{0.500000in}}{\pgfqpoint{0.635833in}{0.486077in}}{\pgfqpoint{0.630302in}{0.472722in}}%
\pgfpathcurveto{\pgfqpoint{0.620456in}{0.462877in}}{\pgfqpoint{0.610611in}{0.453032in}}{\pgfqpoint{0.597256in}{0.447500in}}%
\pgfpathclose%
\pgfpathmoveto{\pgfqpoint{0.750000in}{0.441667in}}%
\pgfpathcurveto{\pgfqpoint{0.765470in}{0.441667in}}{\pgfqpoint{0.780309in}{0.447813in}}{\pgfqpoint{0.791248in}{0.458752in}}%
\pgfpathcurveto{\pgfqpoint{0.802187in}{0.469691in}}{\pgfqpoint{0.808333in}{0.484530in}}{\pgfqpoint{0.808333in}{0.500000in}}%
\pgfpathcurveto{\pgfqpoint{0.808333in}{0.515470in}}{\pgfqpoint{0.802187in}{0.530309in}}{\pgfqpoint{0.791248in}{0.541248in}}%
\pgfpathcurveto{\pgfqpoint{0.780309in}{0.552187in}}{\pgfqpoint{0.765470in}{0.558333in}}{\pgfqpoint{0.750000in}{0.558333in}}%
\pgfpathcurveto{\pgfqpoint{0.734530in}{0.558333in}}{\pgfqpoint{0.719691in}{0.552187in}}{\pgfqpoint{0.708752in}{0.541248in}}%
\pgfpathcurveto{\pgfqpoint{0.697813in}{0.530309in}}{\pgfqpoint{0.691667in}{0.515470in}}{\pgfqpoint{0.691667in}{0.500000in}}%
\pgfpathcurveto{\pgfqpoint{0.691667in}{0.484530in}}{\pgfqpoint{0.697813in}{0.469691in}}{\pgfqpoint{0.708752in}{0.458752in}}%
\pgfpathcurveto{\pgfqpoint{0.719691in}{0.447813in}}{\pgfqpoint{0.734530in}{0.441667in}}{\pgfqpoint{0.750000in}{0.441667in}}%
\pgfpathclose%
\pgfpathmoveto{\pgfqpoint{0.750000in}{0.447500in}}%
\pgfpathcurveto{\pgfqpoint{0.750000in}{0.447500in}}{\pgfqpoint{0.736077in}{0.447500in}}{\pgfqpoint{0.722722in}{0.453032in}}%
\pgfpathcurveto{\pgfqpoint{0.712877in}{0.462877in}}{\pgfqpoint{0.703032in}{0.472722in}}{\pgfqpoint{0.697500in}{0.486077in}}%
\pgfpathcurveto{\pgfqpoint{0.697500in}{0.500000in}}{\pgfqpoint{0.697500in}{0.513923in}}{\pgfqpoint{0.703032in}{0.527278in}}%
\pgfpathcurveto{\pgfqpoint{0.712877in}{0.537123in}}{\pgfqpoint{0.722722in}{0.546968in}}{\pgfqpoint{0.736077in}{0.552500in}}%
\pgfpathcurveto{\pgfqpoint{0.750000in}{0.552500in}}{\pgfqpoint{0.763923in}{0.552500in}}{\pgfqpoint{0.777278in}{0.546968in}}%
\pgfpathcurveto{\pgfqpoint{0.787123in}{0.537123in}}{\pgfqpoint{0.796968in}{0.527278in}}{\pgfqpoint{0.802500in}{0.513923in}}%
\pgfpathcurveto{\pgfqpoint{0.802500in}{0.500000in}}{\pgfqpoint{0.802500in}{0.486077in}}{\pgfqpoint{0.796968in}{0.472722in}}%
\pgfpathcurveto{\pgfqpoint{0.787123in}{0.462877in}}{\pgfqpoint{0.777278in}{0.453032in}}{\pgfqpoint{0.763923in}{0.447500in}}%
\pgfpathclose%
\pgfpathmoveto{\pgfqpoint{0.916667in}{0.441667in}}%
\pgfpathcurveto{\pgfqpoint{0.932137in}{0.441667in}}{\pgfqpoint{0.946975in}{0.447813in}}{\pgfqpoint{0.957915in}{0.458752in}}%
\pgfpathcurveto{\pgfqpoint{0.968854in}{0.469691in}}{\pgfqpoint{0.975000in}{0.484530in}}{\pgfqpoint{0.975000in}{0.500000in}}%
\pgfpathcurveto{\pgfqpoint{0.975000in}{0.515470in}}{\pgfqpoint{0.968854in}{0.530309in}}{\pgfqpoint{0.957915in}{0.541248in}}%
\pgfpathcurveto{\pgfqpoint{0.946975in}{0.552187in}}{\pgfqpoint{0.932137in}{0.558333in}}{\pgfqpoint{0.916667in}{0.558333in}}%
\pgfpathcurveto{\pgfqpoint{0.901196in}{0.558333in}}{\pgfqpoint{0.886358in}{0.552187in}}{\pgfqpoint{0.875419in}{0.541248in}}%
\pgfpathcurveto{\pgfqpoint{0.864480in}{0.530309in}}{\pgfqpoint{0.858333in}{0.515470in}}{\pgfqpoint{0.858333in}{0.500000in}}%
\pgfpathcurveto{\pgfqpoint{0.858333in}{0.484530in}}{\pgfqpoint{0.864480in}{0.469691in}}{\pgfqpoint{0.875419in}{0.458752in}}%
\pgfpathcurveto{\pgfqpoint{0.886358in}{0.447813in}}{\pgfqpoint{0.901196in}{0.441667in}}{\pgfqpoint{0.916667in}{0.441667in}}%
\pgfpathclose%
\pgfpathmoveto{\pgfqpoint{0.916667in}{0.447500in}}%
\pgfpathcurveto{\pgfqpoint{0.916667in}{0.447500in}}{\pgfqpoint{0.902744in}{0.447500in}}{\pgfqpoint{0.889389in}{0.453032in}}%
\pgfpathcurveto{\pgfqpoint{0.879544in}{0.462877in}}{\pgfqpoint{0.869698in}{0.472722in}}{\pgfqpoint{0.864167in}{0.486077in}}%
\pgfpathcurveto{\pgfqpoint{0.864167in}{0.500000in}}{\pgfqpoint{0.864167in}{0.513923in}}{\pgfqpoint{0.869698in}{0.527278in}}%
\pgfpathcurveto{\pgfqpoint{0.879544in}{0.537123in}}{\pgfqpoint{0.889389in}{0.546968in}}{\pgfqpoint{0.902744in}{0.552500in}}%
\pgfpathcurveto{\pgfqpoint{0.916667in}{0.552500in}}{\pgfqpoint{0.930590in}{0.552500in}}{\pgfqpoint{0.943945in}{0.546968in}}%
\pgfpathcurveto{\pgfqpoint{0.953790in}{0.537123in}}{\pgfqpoint{0.963635in}{0.527278in}}{\pgfqpoint{0.969167in}{0.513923in}}%
\pgfpathcurveto{\pgfqpoint{0.969167in}{0.500000in}}{\pgfqpoint{0.969167in}{0.486077in}}{\pgfqpoint{0.963635in}{0.472722in}}%
\pgfpathcurveto{\pgfqpoint{0.953790in}{0.462877in}}{\pgfqpoint{0.943945in}{0.453032in}}{\pgfqpoint{0.930590in}{0.447500in}}%
\pgfpathclose%
\pgfpathmoveto{\pgfqpoint{0.000000in}{0.608333in}}%
\pgfpathcurveto{\pgfqpoint{0.015470in}{0.608333in}}{\pgfqpoint{0.030309in}{0.614480in}}{\pgfqpoint{0.041248in}{0.625419in}}%
\pgfpathcurveto{\pgfqpoint{0.052187in}{0.636358in}}{\pgfqpoint{0.058333in}{0.651196in}}{\pgfqpoint{0.058333in}{0.666667in}}%
\pgfpathcurveto{\pgfqpoint{0.058333in}{0.682137in}}{\pgfqpoint{0.052187in}{0.696975in}}{\pgfqpoint{0.041248in}{0.707915in}}%
\pgfpathcurveto{\pgfqpoint{0.030309in}{0.718854in}}{\pgfqpoint{0.015470in}{0.725000in}}{\pgfqpoint{0.000000in}{0.725000in}}%
\pgfpathcurveto{\pgfqpoint{-0.015470in}{0.725000in}}{\pgfqpoint{-0.030309in}{0.718854in}}{\pgfqpoint{-0.041248in}{0.707915in}}%
\pgfpathcurveto{\pgfqpoint{-0.052187in}{0.696975in}}{\pgfqpoint{-0.058333in}{0.682137in}}{\pgfqpoint{-0.058333in}{0.666667in}}%
\pgfpathcurveto{\pgfqpoint{-0.058333in}{0.651196in}}{\pgfqpoint{-0.052187in}{0.636358in}}{\pgfqpoint{-0.041248in}{0.625419in}}%
\pgfpathcurveto{\pgfqpoint{-0.030309in}{0.614480in}}{\pgfqpoint{-0.015470in}{0.608333in}}{\pgfqpoint{0.000000in}{0.608333in}}%
\pgfpathclose%
\pgfpathmoveto{\pgfqpoint{0.000000in}{0.614167in}}%
\pgfpathcurveto{\pgfqpoint{0.000000in}{0.614167in}}{\pgfqpoint{-0.013923in}{0.614167in}}{\pgfqpoint{-0.027278in}{0.619698in}}%
\pgfpathcurveto{\pgfqpoint{-0.037123in}{0.629544in}}{\pgfqpoint{-0.046968in}{0.639389in}}{\pgfqpoint{-0.052500in}{0.652744in}}%
\pgfpathcurveto{\pgfqpoint{-0.052500in}{0.666667in}}{\pgfqpoint{-0.052500in}{0.680590in}}{\pgfqpoint{-0.046968in}{0.693945in}}%
\pgfpathcurveto{\pgfqpoint{-0.037123in}{0.703790in}}{\pgfqpoint{-0.027278in}{0.713635in}}{\pgfqpoint{-0.013923in}{0.719167in}}%
\pgfpathcurveto{\pgfqpoint{0.000000in}{0.719167in}}{\pgfqpoint{0.013923in}{0.719167in}}{\pgfqpoint{0.027278in}{0.713635in}}%
\pgfpathcurveto{\pgfqpoint{0.037123in}{0.703790in}}{\pgfqpoint{0.046968in}{0.693945in}}{\pgfqpoint{0.052500in}{0.680590in}}%
\pgfpathcurveto{\pgfqpoint{0.052500in}{0.666667in}}{\pgfqpoint{0.052500in}{0.652744in}}{\pgfqpoint{0.046968in}{0.639389in}}%
\pgfpathcurveto{\pgfqpoint{0.037123in}{0.629544in}}{\pgfqpoint{0.027278in}{0.619698in}}{\pgfqpoint{0.013923in}{0.614167in}}%
\pgfpathclose%
\pgfpathmoveto{\pgfqpoint{0.166667in}{0.608333in}}%
\pgfpathcurveto{\pgfqpoint{0.182137in}{0.608333in}}{\pgfqpoint{0.196975in}{0.614480in}}{\pgfqpoint{0.207915in}{0.625419in}}%
\pgfpathcurveto{\pgfqpoint{0.218854in}{0.636358in}}{\pgfqpoint{0.225000in}{0.651196in}}{\pgfqpoint{0.225000in}{0.666667in}}%
\pgfpathcurveto{\pgfqpoint{0.225000in}{0.682137in}}{\pgfqpoint{0.218854in}{0.696975in}}{\pgfqpoint{0.207915in}{0.707915in}}%
\pgfpathcurveto{\pgfqpoint{0.196975in}{0.718854in}}{\pgfqpoint{0.182137in}{0.725000in}}{\pgfqpoint{0.166667in}{0.725000in}}%
\pgfpathcurveto{\pgfqpoint{0.151196in}{0.725000in}}{\pgfqpoint{0.136358in}{0.718854in}}{\pgfqpoint{0.125419in}{0.707915in}}%
\pgfpathcurveto{\pgfqpoint{0.114480in}{0.696975in}}{\pgfqpoint{0.108333in}{0.682137in}}{\pgfqpoint{0.108333in}{0.666667in}}%
\pgfpathcurveto{\pgfqpoint{0.108333in}{0.651196in}}{\pgfqpoint{0.114480in}{0.636358in}}{\pgfqpoint{0.125419in}{0.625419in}}%
\pgfpathcurveto{\pgfqpoint{0.136358in}{0.614480in}}{\pgfqpoint{0.151196in}{0.608333in}}{\pgfqpoint{0.166667in}{0.608333in}}%
\pgfpathclose%
\pgfpathmoveto{\pgfqpoint{0.166667in}{0.614167in}}%
\pgfpathcurveto{\pgfqpoint{0.166667in}{0.614167in}}{\pgfqpoint{0.152744in}{0.614167in}}{\pgfqpoint{0.139389in}{0.619698in}}%
\pgfpathcurveto{\pgfqpoint{0.129544in}{0.629544in}}{\pgfqpoint{0.119698in}{0.639389in}}{\pgfqpoint{0.114167in}{0.652744in}}%
\pgfpathcurveto{\pgfqpoint{0.114167in}{0.666667in}}{\pgfqpoint{0.114167in}{0.680590in}}{\pgfqpoint{0.119698in}{0.693945in}}%
\pgfpathcurveto{\pgfqpoint{0.129544in}{0.703790in}}{\pgfqpoint{0.139389in}{0.713635in}}{\pgfqpoint{0.152744in}{0.719167in}}%
\pgfpathcurveto{\pgfqpoint{0.166667in}{0.719167in}}{\pgfqpoint{0.180590in}{0.719167in}}{\pgfqpoint{0.193945in}{0.713635in}}%
\pgfpathcurveto{\pgfqpoint{0.203790in}{0.703790in}}{\pgfqpoint{0.213635in}{0.693945in}}{\pgfqpoint{0.219167in}{0.680590in}}%
\pgfpathcurveto{\pgfqpoint{0.219167in}{0.666667in}}{\pgfqpoint{0.219167in}{0.652744in}}{\pgfqpoint{0.213635in}{0.639389in}}%
\pgfpathcurveto{\pgfqpoint{0.203790in}{0.629544in}}{\pgfqpoint{0.193945in}{0.619698in}}{\pgfqpoint{0.180590in}{0.614167in}}%
\pgfpathclose%
\pgfpathmoveto{\pgfqpoint{0.333333in}{0.608333in}}%
\pgfpathcurveto{\pgfqpoint{0.348804in}{0.608333in}}{\pgfqpoint{0.363642in}{0.614480in}}{\pgfqpoint{0.374581in}{0.625419in}}%
\pgfpathcurveto{\pgfqpoint{0.385520in}{0.636358in}}{\pgfqpoint{0.391667in}{0.651196in}}{\pgfqpoint{0.391667in}{0.666667in}}%
\pgfpathcurveto{\pgfqpoint{0.391667in}{0.682137in}}{\pgfqpoint{0.385520in}{0.696975in}}{\pgfqpoint{0.374581in}{0.707915in}}%
\pgfpathcurveto{\pgfqpoint{0.363642in}{0.718854in}}{\pgfqpoint{0.348804in}{0.725000in}}{\pgfqpoint{0.333333in}{0.725000in}}%
\pgfpathcurveto{\pgfqpoint{0.317863in}{0.725000in}}{\pgfqpoint{0.303025in}{0.718854in}}{\pgfqpoint{0.292085in}{0.707915in}}%
\pgfpathcurveto{\pgfqpoint{0.281146in}{0.696975in}}{\pgfqpoint{0.275000in}{0.682137in}}{\pgfqpoint{0.275000in}{0.666667in}}%
\pgfpathcurveto{\pgfqpoint{0.275000in}{0.651196in}}{\pgfqpoint{0.281146in}{0.636358in}}{\pgfqpoint{0.292085in}{0.625419in}}%
\pgfpathcurveto{\pgfqpoint{0.303025in}{0.614480in}}{\pgfqpoint{0.317863in}{0.608333in}}{\pgfqpoint{0.333333in}{0.608333in}}%
\pgfpathclose%
\pgfpathmoveto{\pgfqpoint{0.333333in}{0.614167in}}%
\pgfpathcurveto{\pgfqpoint{0.333333in}{0.614167in}}{\pgfqpoint{0.319410in}{0.614167in}}{\pgfqpoint{0.306055in}{0.619698in}}%
\pgfpathcurveto{\pgfqpoint{0.296210in}{0.629544in}}{\pgfqpoint{0.286365in}{0.639389in}}{\pgfqpoint{0.280833in}{0.652744in}}%
\pgfpathcurveto{\pgfqpoint{0.280833in}{0.666667in}}{\pgfqpoint{0.280833in}{0.680590in}}{\pgfqpoint{0.286365in}{0.693945in}}%
\pgfpathcurveto{\pgfqpoint{0.296210in}{0.703790in}}{\pgfqpoint{0.306055in}{0.713635in}}{\pgfqpoint{0.319410in}{0.719167in}}%
\pgfpathcurveto{\pgfqpoint{0.333333in}{0.719167in}}{\pgfqpoint{0.347256in}{0.719167in}}{\pgfqpoint{0.360611in}{0.713635in}}%
\pgfpathcurveto{\pgfqpoint{0.370456in}{0.703790in}}{\pgfqpoint{0.380302in}{0.693945in}}{\pgfqpoint{0.385833in}{0.680590in}}%
\pgfpathcurveto{\pgfqpoint{0.385833in}{0.666667in}}{\pgfqpoint{0.385833in}{0.652744in}}{\pgfqpoint{0.380302in}{0.639389in}}%
\pgfpathcurveto{\pgfqpoint{0.370456in}{0.629544in}}{\pgfqpoint{0.360611in}{0.619698in}}{\pgfqpoint{0.347256in}{0.614167in}}%
\pgfpathclose%
\pgfpathmoveto{\pgfqpoint{0.500000in}{0.608333in}}%
\pgfpathcurveto{\pgfqpoint{0.515470in}{0.608333in}}{\pgfqpoint{0.530309in}{0.614480in}}{\pgfqpoint{0.541248in}{0.625419in}}%
\pgfpathcurveto{\pgfqpoint{0.552187in}{0.636358in}}{\pgfqpoint{0.558333in}{0.651196in}}{\pgfqpoint{0.558333in}{0.666667in}}%
\pgfpathcurveto{\pgfqpoint{0.558333in}{0.682137in}}{\pgfqpoint{0.552187in}{0.696975in}}{\pgfqpoint{0.541248in}{0.707915in}}%
\pgfpathcurveto{\pgfqpoint{0.530309in}{0.718854in}}{\pgfqpoint{0.515470in}{0.725000in}}{\pgfqpoint{0.500000in}{0.725000in}}%
\pgfpathcurveto{\pgfqpoint{0.484530in}{0.725000in}}{\pgfqpoint{0.469691in}{0.718854in}}{\pgfqpoint{0.458752in}{0.707915in}}%
\pgfpathcurveto{\pgfqpoint{0.447813in}{0.696975in}}{\pgfqpoint{0.441667in}{0.682137in}}{\pgfqpoint{0.441667in}{0.666667in}}%
\pgfpathcurveto{\pgfqpoint{0.441667in}{0.651196in}}{\pgfqpoint{0.447813in}{0.636358in}}{\pgfqpoint{0.458752in}{0.625419in}}%
\pgfpathcurveto{\pgfqpoint{0.469691in}{0.614480in}}{\pgfqpoint{0.484530in}{0.608333in}}{\pgfqpoint{0.500000in}{0.608333in}}%
\pgfpathclose%
\pgfpathmoveto{\pgfqpoint{0.500000in}{0.614167in}}%
\pgfpathcurveto{\pgfqpoint{0.500000in}{0.614167in}}{\pgfqpoint{0.486077in}{0.614167in}}{\pgfqpoint{0.472722in}{0.619698in}}%
\pgfpathcurveto{\pgfqpoint{0.462877in}{0.629544in}}{\pgfqpoint{0.453032in}{0.639389in}}{\pgfqpoint{0.447500in}{0.652744in}}%
\pgfpathcurveto{\pgfqpoint{0.447500in}{0.666667in}}{\pgfqpoint{0.447500in}{0.680590in}}{\pgfqpoint{0.453032in}{0.693945in}}%
\pgfpathcurveto{\pgfqpoint{0.462877in}{0.703790in}}{\pgfqpoint{0.472722in}{0.713635in}}{\pgfqpoint{0.486077in}{0.719167in}}%
\pgfpathcurveto{\pgfqpoint{0.500000in}{0.719167in}}{\pgfqpoint{0.513923in}{0.719167in}}{\pgfqpoint{0.527278in}{0.713635in}}%
\pgfpathcurveto{\pgfqpoint{0.537123in}{0.703790in}}{\pgfqpoint{0.546968in}{0.693945in}}{\pgfqpoint{0.552500in}{0.680590in}}%
\pgfpathcurveto{\pgfqpoint{0.552500in}{0.666667in}}{\pgfqpoint{0.552500in}{0.652744in}}{\pgfqpoint{0.546968in}{0.639389in}}%
\pgfpathcurveto{\pgfqpoint{0.537123in}{0.629544in}}{\pgfqpoint{0.527278in}{0.619698in}}{\pgfqpoint{0.513923in}{0.614167in}}%
\pgfpathclose%
\pgfpathmoveto{\pgfqpoint{0.666667in}{0.608333in}}%
\pgfpathcurveto{\pgfqpoint{0.682137in}{0.608333in}}{\pgfqpoint{0.696975in}{0.614480in}}{\pgfqpoint{0.707915in}{0.625419in}}%
\pgfpathcurveto{\pgfqpoint{0.718854in}{0.636358in}}{\pgfqpoint{0.725000in}{0.651196in}}{\pgfqpoint{0.725000in}{0.666667in}}%
\pgfpathcurveto{\pgfqpoint{0.725000in}{0.682137in}}{\pgfqpoint{0.718854in}{0.696975in}}{\pgfqpoint{0.707915in}{0.707915in}}%
\pgfpathcurveto{\pgfqpoint{0.696975in}{0.718854in}}{\pgfqpoint{0.682137in}{0.725000in}}{\pgfqpoint{0.666667in}{0.725000in}}%
\pgfpathcurveto{\pgfqpoint{0.651196in}{0.725000in}}{\pgfqpoint{0.636358in}{0.718854in}}{\pgfqpoint{0.625419in}{0.707915in}}%
\pgfpathcurveto{\pgfqpoint{0.614480in}{0.696975in}}{\pgfqpoint{0.608333in}{0.682137in}}{\pgfqpoint{0.608333in}{0.666667in}}%
\pgfpathcurveto{\pgfqpoint{0.608333in}{0.651196in}}{\pgfqpoint{0.614480in}{0.636358in}}{\pgfqpoint{0.625419in}{0.625419in}}%
\pgfpathcurveto{\pgfqpoint{0.636358in}{0.614480in}}{\pgfqpoint{0.651196in}{0.608333in}}{\pgfqpoint{0.666667in}{0.608333in}}%
\pgfpathclose%
\pgfpathmoveto{\pgfqpoint{0.666667in}{0.614167in}}%
\pgfpathcurveto{\pgfqpoint{0.666667in}{0.614167in}}{\pgfqpoint{0.652744in}{0.614167in}}{\pgfqpoint{0.639389in}{0.619698in}}%
\pgfpathcurveto{\pgfqpoint{0.629544in}{0.629544in}}{\pgfqpoint{0.619698in}{0.639389in}}{\pgfqpoint{0.614167in}{0.652744in}}%
\pgfpathcurveto{\pgfqpoint{0.614167in}{0.666667in}}{\pgfqpoint{0.614167in}{0.680590in}}{\pgfqpoint{0.619698in}{0.693945in}}%
\pgfpathcurveto{\pgfqpoint{0.629544in}{0.703790in}}{\pgfqpoint{0.639389in}{0.713635in}}{\pgfqpoint{0.652744in}{0.719167in}}%
\pgfpathcurveto{\pgfqpoint{0.666667in}{0.719167in}}{\pgfqpoint{0.680590in}{0.719167in}}{\pgfqpoint{0.693945in}{0.713635in}}%
\pgfpathcurveto{\pgfqpoint{0.703790in}{0.703790in}}{\pgfqpoint{0.713635in}{0.693945in}}{\pgfqpoint{0.719167in}{0.680590in}}%
\pgfpathcurveto{\pgfqpoint{0.719167in}{0.666667in}}{\pgfqpoint{0.719167in}{0.652744in}}{\pgfqpoint{0.713635in}{0.639389in}}%
\pgfpathcurveto{\pgfqpoint{0.703790in}{0.629544in}}{\pgfqpoint{0.693945in}{0.619698in}}{\pgfqpoint{0.680590in}{0.614167in}}%
\pgfpathclose%
\pgfpathmoveto{\pgfqpoint{0.833333in}{0.608333in}}%
\pgfpathcurveto{\pgfqpoint{0.848804in}{0.608333in}}{\pgfqpoint{0.863642in}{0.614480in}}{\pgfqpoint{0.874581in}{0.625419in}}%
\pgfpathcurveto{\pgfqpoint{0.885520in}{0.636358in}}{\pgfqpoint{0.891667in}{0.651196in}}{\pgfqpoint{0.891667in}{0.666667in}}%
\pgfpathcurveto{\pgfqpoint{0.891667in}{0.682137in}}{\pgfqpoint{0.885520in}{0.696975in}}{\pgfqpoint{0.874581in}{0.707915in}}%
\pgfpathcurveto{\pgfqpoint{0.863642in}{0.718854in}}{\pgfqpoint{0.848804in}{0.725000in}}{\pgfqpoint{0.833333in}{0.725000in}}%
\pgfpathcurveto{\pgfqpoint{0.817863in}{0.725000in}}{\pgfqpoint{0.803025in}{0.718854in}}{\pgfqpoint{0.792085in}{0.707915in}}%
\pgfpathcurveto{\pgfqpoint{0.781146in}{0.696975in}}{\pgfqpoint{0.775000in}{0.682137in}}{\pgfqpoint{0.775000in}{0.666667in}}%
\pgfpathcurveto{\pgfqpoint{0.775000in}{0.651196in}}{\pgfqpoint{0.781146in}{0.636358in}}{\pgfqpoint{0.792085in}{0.625419in}}%
\pgfpathcurveto{\pgfqpoint{0.803025in}{0.614480in}}{\pgfqpoint{0.817863in}{0.608333in}}{\pgfqpoint{0.833333in}{0.608333in}}%
\pgfpathclose%
\pgfpathmoveto{\pgfqpoint{0.833333in}{0.614167in}}%
\pgfpathcurveto{\pgfqpoint{0.833333in}{0.614167in}}{\pgfqpoint{0.819410in}{0.614167in}}{\pgfqpoint{0.806055in}{0.619698in}}%
\pgfpathcurveto{\pgfqpoint{0.796210in}{0.629544in}}{\pgfqpoint{0.786365in}{0.639389in}}{\pgfqpoint{0.780833in}{0.652744in}}%
\pgfpathcurveto{\pgfqpoint{0.780833in}{0.666667in}}{\pgfqpoint{0.780833in}{0.680590in}}{\pgfqpoint{0.786365in}{0.693945in}}%
\pgfpathcurveto{\pgfqpoint{0.796210in}{0.703790in}}{\pgfqpoint{0.806055in}{0.713635in}}{\pgfqpoint{0.819410in}{0.719167in}}%
\pgfpathcurveto{\pgfqpoint{0.833333in}{0.719167in}}{\pgfqpoint{0.847256in}{0.719167in}}{\pgfqpoint{0.860611in}{0.713635in}}%
\pgfpathcurveto{\pgfqpoint{0.870456in}{0.703790in}}{\pgfqpoint{0.880302in}{0.693945in}}{\pgfqpoint{0.885833in}{0.680590in}}%
\pgfpathcurveto{\pgfqpoint{0.885833in}{0.666667in}}{\pgfqpoint{0.885833in}{0.652744in}}{\pgfqpoint{0.880302in}{0.639389in}}%
\pgfpathcurveto{\pgfqpoint{0.870456in}{0.629544in}}{\pgfqpoint{0.860611in}{0.619698in}}{\pgfqpoint{0.847256in}{0.614167in}}%
\pgfpathclose%
\pgfpathmoveto{\pgfqpoint{1.000000in}{0.608333in}}%
\pgfpathcurveto{\pgfqpoint{1.015470in}{0.608333in}}{\pgfqpoint{1.030309in}{0.614480in}}{\pgfqpoint{1.041248in}{0.625419in}}%
\pgfpathcurveto{\pgfqpoint{1.052187in}{0.636358in}}{\pgfqpoint{1.058333in}{0.651196in}}{\pgfqpoint{1.058333in}{0.666667in}}%
\pgfpathcurveto{\pgfqpoint{1.058333in}{0.682137in}}{\pgfqpoint{1.052187in}{0.696975in}}{\pgfqpoint{1.041248in}{0.707915in}}%
\pgfpathcurveto{\pgfqpoint{1.030309in}{0.718854in}}{\pgfqpoint{1.015470in}{0.725000in}}{\pgfqpoint{1.000000in}{0.725000in}}%
\pgfpathcurveto{\pgfqpoint{0.984530in}{0.725000in}}{\pgfqpoint{0.969691in}{0.718854in}}{\pgfqpoint{0.958752in}{0.707915in}}%
\pgfpathcurveto{\pgfqpoint{0.947813in}{0.696975in}}{\pgfqpoint{0.941667in}{0.682137in}}{\pgfqpoint{0.941667in}{0.666667in}}%
\pgfpathcurveto{\pgfqpoint{0.941667in}{0.651196in}}{\pgfqpoint{0.947813in}{0.636358in}}{\pgfqpoint{0.958752in}{0.625419in}}%
\pgfpathcurveto{\pgfqpoint{0.969691in}{0.614480in}}{\pgfqpoint{0.984530in}{0.608333in}}{\pgfqpoint{1.000000in}{0.608333in}}%
\pgfpathclose%
\pgfpathmoveto{\pgfqpoint{1.000000in}{0.614167in}}%
\pgfpathcurveto{\pgfqpoint{1.000000in}{0.614167in}}{\pgfqpoint{0.986077in}{0.614167in}}{\pgfqpoint{0.972722in}{0.619698in}}%
\pgfpathcurveto{\pgfqpoint{0.962877in}{0.629544in}}{\pgfqpoint{0.953032in}{0.639389in}}{\pgfqpoint{0.947500in}{0.652744in}}%
\pgfpathcurveto{\pgfqpoint{0.947500in}{0.666667in}}{\pgfqpoint{0.947500in}{0.680590in}}{\pgfqpoint{0.953032in}{0.693945in}}%
\pgfpathcurveto{\pgfqpoint{0.962877in}{0.703790in}}{\pgfqpoint{0.972722in}{0.713635in}}{\pgfqpoint{0.986077in}{0.719167in}}%
\pgfpathcurveto{\pgfqpoint{1.000000in}{0.719167in}}{\pgfqpoint{1.013923in}{0.719167in}}{\pgfqpoint{1.027278in}{0.713635in}}%
\pgfpathcurveto{\pgfqpoint{1.037123in}{0.703790in}}{\pgfqpoint{1.046968in}{0.693945in}}{\pgfqpoint{1.052500in}{0.680590in}}%
\pgfpathcurveto{\pgfqpoint{1.052500in}{0.666667in}}{\pgfqpoint{1.052500in}{0.652744in}}{\pgfqpoint{1.046968in}{0.639389in}}%
\pgfpathcurveto{\pgfqpoint{1.037123in}{0.629544in}}{\pgfqpoint{1.027278in}{0.619698in}}{\pgfqpoint{1.013923in}{0.614167in}}%
\pgfpathclose%
\pgfpathmoveto{\pgfqpoint{0.083333in}{0.775000in}}%
\pgfpathcurveto{\pgfqpoint{0.098804in}{0.775000in}}{\pgfqpoint{0.113642in}{0.781146in}}{\pgfqpoint{0.124581in}{0.792085in}}%
\pgfpathcurveto{\pgfqpoint{0.135520in}{0.803025in}}{\pgfqpoint{0.141667in}{0.817863in}}{\pgfqpoint{0.141667in}{0.833333in}}%
\pgfpathcurveto{\pgfqpoint{0.141667in}{0.848804in}}{\pgfqpoint{0.135520in}{0.863642in}}{\pgfqpoint{0.124581in}{0.874581in}}%
\pgfpathcurveto{\pgfqpoint{0.113642in}{0.885520in}}{\pgfqpoint{0.098804in}{0.891667in}}{\pgfqpoint{0.083333in}{0.891667in}}%
\pgfpathcurveto{\pgfqpoint{0.067863in}{0.891667in}}{\pgfqpoint{0.053025in}{0.885520in}}{\pgfqpoint{0.042085in}{0.874581in}}%
\pgfpathcurveto{\pgfqpoint{0.031146in}{0.863642in}}{\pgfqpoint{0.025000in}{0.848804in}}{\pgfqpoint{0.025000in}{0.833333in}}%
\pgfpathcurveto{\pgfqpoint{0.025000in}{0.817863in}}{\pgfqpoint{0.031146in}{0.803025in}}{\pgfqpoint{0.042085in}{0.792085in}}%
\pgfpathcurveto{\pgfqpoint{0.053025in}{0.781146in}}{\pgfqpoint{0.067863in}{0.775000in}}{\pgfqpoint{0.083333in}{0.775000in}}%
\pgfpathclose%
\pgfpathmoveto{\pgfqpoint{0.083333in}{0.780833in}}%
\pgfpathcurveto{\pgfqpoint{0.083333in}{0.780833in}}{\pgfqpoint{0.069410in}{0.780833in}}{\pgfqpoint{0.056055in}{0.786365in}}%
\pgfpathcurveto{\pgfqpoint{0.046210in}{0.796210in}}{\pgfqpoint{0.036365in}{0.806055in}}{\pgfqpoint{0.030833in}{0.819410in}}%
\pgfpathcurveto{\pgfqpoint{0.030833in}{0.833333in}}{\pgfqpoint{0.030833in}{0.847256in}}{\pgfqpoint{0.036365in}{0.860611in}}%
\pgfpathcurveto{\pgfqpoint{0.046210in}{0.870456in}}{\pgfqpoint{0.056055in}{0.880302in}}{\pgfqpoint{0.069410in}{0.885833in}}%
\pgfpathcurveto{\pgfqpoint{0.083333in}{0.885833in}}{\pgfqpoint{0.097256in}{0.885833in}}{\pgfqpoint{0.110611in}{0.880302in}}%
\pgfpathcurveto{\pgfqpoint{0.120456in}{0.870456in}}{\pgfqpoint{0.130302in}{0.860611in}}{\pgfqpoint{0.135833in}{0.847256in}}%
\pgfpathcurveto{\pgfqpoint{0.135833in}{0.833333in}}{\pgfqpoint{0.135833in}{0.819410in}}{\pgfqpoint{0.130302in}{0.806055in}}%
\pgfpathcurveto{\pgfqpoint{0.120456in}{0.796210in}}{\pgfqpoint{0.110611in}{0.786365in}}{\pgfqpoint{0.097256in}{0.780833in}}%
\pgfpathclose%
\pgfpathmoveto{\pgfqpoint{0.250000in}{0.775000in}}%
\pgfpathcurveto{\pgfqpoint{0.265470in}{0.775000in}}{\pgfqpoint{0.280309in}{0.781146in}}{\pgfqpoint{0.291248in}{0.792085in}}%
\pgfpathcurveto{\pgfqpoint{0.302187in}{0.803025in}}{\pgfqpoint{0.308333in}{0.817863in}}{\pgfqpoint{0.308333in}{0.833333in}}%
\pgfpathcurveto{\pgfqpoint{0.308333in}{0.848804in}}{\pgfqpoint{0.302187in}{0.863642in}}{\pgfqpoint{0.291248in}{0.874581in}}%
\pgfpathcurveto{\pgfqpoint{0.280309in}{0.885520in}}{\pgfqpoint{0.265470in}{0.891667in}}{\pgfqpoint{0.250000in}{0.891667in}}%
\pgfpathcurveto{\pgfqpoint{0.234530in}{0.891667in}}{\pgfqpoint{0.219691in}{0.885520in}}{\pgfqpoint{0.208752in}{0.874581in}}%
\pgfpathcurveto{\pgfqpoint{0.197813in}{0.863642in}}{\pgfqpoint{0.191667in}{0.848804in}}{\pgfqpoint{0.191667in}{0.833333in}}%
\pgfpathcurveto{\pgfqpoint{0.191667in}{0.817863in}}{\pgfqpoint{0.197813in}{0.803025in}}{\pgfqpoint{0.208752in}{0.792085in}}%
\pgfpathcurveto{\pgfqpoint{0.219691in}{0.781146in}}{\pgfqpoint{0.234530in}{0.775000in}}{\pgfqpoint{0.250000in}{0.775000in}}%
\pgfpathclose%
\pgfpathmoveto{\pgfqpoint{0.250000in}{0.780833in}}%
\pgfpathcurveto{\pgfqpoint{0.250000in}{0.780833in}}{\pgfqpoint{0.236077in}{0.780833in}}{\pgfqpoint{0.222722in}{0.786365in}}%
\pgfpathcurveto{\pgfqpoint{0.212877in}{0.796210in}}{\pgfqpoint{0.203032in}{0.806055in}}{\pgfqpoint{0.197500in}{0.819410in}}%
\pgfpathcurveto{\pgfqpoint{0.197500in}{0.833333in}}{\pgfqpoint{0.197500in}{0.847256in}}{\pgfqpoint{0.203032in}{0.860611in}}%
\pgfpathcurveto{\pgfqpoint{0.212877in}{0.870456in}}{\pgfqpoint{0.222722in}{0.880302in}}{\pgfqpoint{0.236077in}{0.885833in}}%
\pgfpathcurveto{\pgfqpoint{0.250000in}{0.885833in}}{\pgfqpoint{0.263923in}{0.885833in}}{\pgfqpoint{0.277278in}{0.880302in}}%
\pgfpathcurveto{\pgfqpoint{0.287123in}{0.870456in}}{\pgfqpoint{0.296968in}{0.860611in}}{\pgfqpoint{0.302500in}{0.847256in}}%
\pgfpathcurveto{\pgfqpoint{0.302500in}{0.833333in}}{\pgfqpoint{0.302500in}{0.819410in}}{\pgfqpoint{0.296968in}{0.806055in}}%
\pgfpathcurveto{\pgfqpoint{0.287123in}{0.796210in}}{\pgfqpoint{0.277278in}{0.786365in}}{\pgfqpoint{0.263923in}{0.780833in}}%
\pgfpathclose%
\pgfpathmoveto{\pgfqpoint{0.416667in}{0.775000in}}%
\pgfpathcurveto{\pgfqpoint{0.432137in}{0.775000in}}{\pgfqpoint{0.446975in}{0.781146in}}{\pgfqpoint{0.457915in}{0.792085in}}%
\pgfpathcurveto{\pgfqpoint{0.468854in}{0.803025in}}{\pgfqpoint{0.475000in}{0.817863in}}{\pgfqpoint{0.475000in}{0.833333in}}%
\pgfpathcurveto{\pgfqpoint{0.475000in}{0.848804in}}{\pgfqpoint{0.468854in}{0.863642in}}{\pgfqpoint{0.457915in}{0.874581in}}%
\pgfpathcurveto{\pgfqpoint{0.446975in}{0.885520in}}{\pgfqpoint{0.432137in}{0.891667in}}{\pgfqpoint{0.416667in}{0.891667in}}%
\pgfpathcurveto{\pgfqpoint{0.401196in}{0.891667in}}{\pgfqpoint{0.386358in}{0.885520in}}{\pgfqpoint{0.375419in}{0.874581in}}%
\pgfpathcurveto{\pgfqpoint{0.364480in}{0.863642in}}{\pgfqpoint{0.358333in}{0.848804in}}{\pgfqpoint{0.358333in}{0.833333in}}%
\pgfpathcurveto{\pgfqpoint{0.358333in}{0.817863in}}{\pgfqpoint{0.364480in}{0.803025in}}{\pgfqpoint{0.375419in}{0.792085in}}%
\pgfpathcurveto{\pgfqpoint{0.386358in}{0.781146in}}{\pgfqpoint{0.401196in}{0.775000in}}{\pgfqpoint{0.416667in}{0.775000in}}%
\pgfpathclose%
\pgfpathmoveto{\pgfqpoint{0.416667in}{0.780833in}}%
\pgfpathcurveto{\pgfqpoint{0.416667in}{0.780833in}}{\pgfqpoint{0.402744in}{0.780833in}}{\pgfqpoint{0.389389in}{0.786365in}}%
\pgfpathcurveto{\pgfqpoint{0.379544in}{0.796210in}}{\pgfqpoint{0.369698in}{0.806055in}}{\pgfqpoint{0.364167in}{0.819410in}}%
\pgfpathcurveto{\pgfqpoint{0.364167in}{0.833333in}}{\pgfqpoint{0.364167in}{0.847256in}}{\pgfqpoint{0.369698in}{0.860611in}}%
\pgfpathcurveto{\pgfqpoint{0.379544in}{0.870456in}}{\pgfqpoint{0.389389in}{0.880302in}}{\pgfqpoint{0.402744in}{0.885833in}}%
\pgfpathcurveto{\pgfqpoint{0.416667in}{0.885833in}}{\pgfqpoint{0.430590in}{0.885833in}}{\pgfqpoint{0.443945in}{0.880302in}}%
\pgfpathcurveto{\pgfqpoint{0.453790in}{0.870456in}}{\pgfqpoint{0.463635in}{0.860611in}}{\pgfqpoint{0.469167in}{0.847256in}}%
\pgfpathcurveto{\pgfqpoint{0.469167in}{0.833333in}}{\pgfqpoint{0.469167in}{0.819410in}}{\pgfqpoint{0.463635in}{0.806055in}}%
\pgfpathcurveto{\pgfqpoint{0.453790in}{0.796210in}}{\pgfqpoint{0.443945in}{0.786365in}}{\pgfqpoint{0.430590in}{0.780833in}}%
\pgfpathclose%
\pgfpathmoveto{\pgfqpoint{0.583333in}{0.775000in}}%
\pgfpathcurveto{\pgfqpoint{0.598804in}{0.775000in}}{\pgfqpoint{0.613642in}{0.781146in}}{\pgfqpoint{0.624581in}{0.792085in}}%
\pgfpathcurveto{\pgfqpoint{0.635520in}{0.803025in}}{\pgfqpoint{0.641667in}{0.817863in}}{\pgfqpoint{0.641667in}{0.833333in}}%
\pgfpathcurveto{\pgfqpoint{0.641667in}{0.848804in}}{\pgfqpoint{0.635520in}{0.863642in}}{\pgfqpoint{0.624581in}{0.874581in}}%
\pgfpathcurveto{\pgfqpoint{0.613642in}{0.885520in}}{\pgfqpoint{0.598804in}{0.891667in}}{\pgfqpoint{0.583333in}{0.891667in}}%
\pgfpathcurveto{\pgfqpoint{0.567863in}{0.891667in}}{\pgfqpoint{0.553025in}{0.885520in}}{\pgfqpoint{0.542085in}{0.874581in}}%
\pgfpathcurveto{\pgfqpoint{0.531146in}{0.863642in}}{\pgfqpoint{0.525000in}{0.848804in}}{\pgfqpoint{0.525000in}{0.833333in}}%
\pgfpathcurveto{\pgfqpoint{0.525000in}{0.817863in}}{\pgfqpoint{0.531146in}{0.803025in}}{\pgfqpoint{0.542085in}{0.792085in}}%
\pgfpathcurveto{\pgfqpoint{0.553025in}{0.781146in}}{\pgfqpoint{0.567863in}{0.775000in}}{\pgfqpoint{0.583333in}{0.775000in}}%
\pgfpathclose%
\pgfpathmoveto{\pgfqpoint{0.583333in}{0.780833in}}%
\pgfpathcurveto{\pgfqpoint{0.583333in}{0.780833in}}{\pgfqpoint{0.569410in}{0.780833in}}{\pgfqpoint{0.556055in}{0.786365in}}%
\pgfpathcurveto{\pgfqpoint{0.546210in}{0.796210in}}{\pgfqpoint{0.536365in}{0.806055in}}{\pgfqpoint{0.530833in}{0.819410in}}%
\pgfpathcurveto{\pgfqpoint{0.530833in}{0.833333in}}{\pgfqpoint{0.530833in}{0.847256in}}{\pgfqpoint{0.536365in}{0.860611in}}%
\pgfpathcurveto{\pgfqpoint{0.546210in}{0.870456in}}{\pgfqpoint{0.556055in}{0.880302in}}{\pgfqpoint{0.569410in}{0.885833in}}%
\pgfpathcurveto{\pgfqpoint{0.583333in}{0.885833in}}{\pgfqpoint{0.597256in}{0.885833in}}{\pgfqpoint{0.610611in}{0.880302in}}%
\pgfpathcurveto{\pgfqpoint{0.620456in}{0.870456in}}{\pgfqpoint{0.630302in}{0.860611in}}{\pgfqpoint{0.635833in}{0.847256in}}%
\pgfpathcurveto{\pgfqpoint{0.635833in}{0.833333in}}{\pgfqpoint{0.635833in}{0.819410in}}{\pgfqpoint{0.630302in}{0.806055in}}%
\pgfpathcurveto{\pgfqpoint{0.620456in}{0.796210in}}{\pgfqpoint{0.610611in}{0.786365in}}{\pgfqpoint{0.597256in}{0.780833in}}%
\pgfpathclose%
\pgfpathmoveto{\pgfqpoint{0.750000in}{0.775000in}}%
\pgfpathcurveto{\pgfqpoint{0.765470in}{0.775000in}}{\pgfqpoint{0.780309in}{0.781146in}}{\pgfqpoint{0.791248in}{0.792085in}}%
\pgfpathcurveto{\pgfqpoint{0.802187in}{0.803025in}}{\pgfqpoint{0.808333in}{0.817863in}}{\pgfqpoint{0.808333in}{0.833333in}}%
\pgfpathcurveto{\pgfqpoint{0.808333in}{0.848804in}}{\pgfqpoint{0.802187in}{0.863642in}}{\pgfqpoint{0.791248in}{0.874581in}}%
\pgfpathcurveto{\pgfqpoint{0.780309in}{0.885520in}}{\pgfqpoint{0.765470in}{0.891667in}}{\pgfqpoint{0.750000in}{0.891667in}}%
\pgfpathcurveto{\pgfqpoint{0.734530in}{0.891667in}}{\pgfqpoint{0.719691in}{0.885520in}}{\pgfqpoint{0.708752in}{0.874581in}}%
\pgfpathcurveto{\pgfqpoint{0.697813in}{0.863642in}}{\pgfqpoint{0.691667in}{0.848804in}}{\pgfqpoint{0.691667in}{0.833333in}}%
\pgfpathcurveto{\pgfqpoint{0.691667in}{0.817863in}}{\pgfqpoint{0.697813in}{0.803025in}}{\pgfqpoint{0.708752in}{0.792085in}}%
\pgfpathcurveto{\pgfqpoint{0.719691in}{0.781146in}}{\pgfqpoint{0.734530in}{0.775000in}}{\pgfqpoint{0.750000in}{0.775000in}}%
\pgfpathclose%
\pgfpathmoveto{\pgfqpoint{0.750000in}{0.780833in}}%
\pgfpathcurveto{\pgfqpoint{0.750000in}{0.780833in}}{\pgfqpoint{0.736077in}{0.780833in}}{\pgfqpoint{0.722722in}{0.786365in}}%
\pgfpathcurveto{\pgfqpoint{0.712877in}{0.796210in}}{\pgfqpoint{0.703032in}{0.806055in}}{\pgfqpoint{0.697500in}{0.819410in}}%
\pgfpathcurveto{\pgfqpoint{0.697500in}{0.833333in}}{\pgfqpoint{0.697500in}{0.847256in}}{\pgfqpoint{0.703032in}{0.860611in}}%
\pgfpathcurveto{\pgfqpoint{0.712877in}{0.870456in}}{\pgfqpoint{0.722722in}{0.880302in}}{\pgfqpoint{0.736077in}{0.885833in}}%
\pgfpathcurveto{\pgfqpoint{0.750000in}{0.885833in}}{\pgfqpoint{0.763923in}{0.885833in}}{\pgfqpoint{0.777278in}{0.880302in}}%
\pgfpathcurveto{\pgfqpoint{0.787123in}{0.870456in}}{\pgfqpoint{0.796968in}{0.860611in}}{\pgfqpoint{0.802500in}{0.847256in}}%
\pgfpathcurveto{\pgfqpoint{0.802500in}{0.833333in}}{\pgfqpoint{0.802500in}{0.819410in}}{\pgfqpoint{0.796968in}{0.806055in}}%
\pgfpathcurveto{\pgfqpoint{0.787123in}{0.796210in}}{\pgfqpoint{0.777278in}{0.786365in}}{\pgfqpoint{0.763923in}{0.780833in}}%
\pgfpathclose%
\pgfpathmoveto{\pgfqpoint{0.916667in}{0.775000in}}%
\pgfpathcurveto{\pgfqpoint{0.932137in}{0.775000in}}{\pgfqpoint{0.946975in}{0.781146in}}{\pgfqpoint{0.957915in}{0.792085in}}%
\pgfpathcurveto{\pgfqpoint{0.968854in}{0.803025in}}{\pgfqpoint{0.975000in}{0.817863in}}{\pgfqpoint{0.975000in}{0.833333in}}%
\pgfpathcurveto{\pgfqpoint{0.975000in}{0.848804in}}{\pgfqpoint{0.968854in}{0.863642in}}{\pgfqpoint{0.957915in}{0.874581in}}%
\pgfpathcurveto{\pgfqpoint{0.946975in}{0.885520in}}{\pgfqpoint{0.932137in}{0.891667in}}{\pgfqpoint{0.916667in}{0.891667in}}%
\pgfpathcurveto{\pgfqpoint{0.901196in}{0.891667in}}{\pgfqpoint{0.886358in}{0.885520in}}{\pgfqpoint{0.875419in}{0.874581in}}%
\pgfpathcurveto{\pgfqpoint{0.864480in}{0.863642in}}{\pgfqpoint{0.858333in}{0.848804in}}{\pgfqpoint{0.858333in}{0.833333in}}%
\pgfpathcurveto{\pgfqpoint{0.858333in}{0.817863in}}{\pgfqpoint{0.864480in}{0.803025in}}{\pgfqpoint{0.875419in}{0.792085in}}%
\pgfpathcurveto{\pgfqpoint{0.886358in}{0.781146in}}{\pgfqpoint{0.901196in}{0.775000in}}{\pgfqpoint{0.916667in}{0.775000in}}%
\pgfpathclose%
\pgfpathmoveto{\pgfqpoint{0.916667in}{0.780833in}}%
\pgfpathcurveto{\pgfqpoint{0.916667in}{0.780833in}}{\pgfqpoint{0.902744in}{0.780833in}}{\pgfqpoint{0.889389in}{0.786365in}}%
\pgfpathcurveto{\pgfqpoint{0.879544in}{0.796210in}}{\pgfqpoint{0.869698in}{0.806055in}}{\pgfqpoint{0.864167in}{0.819410in}}%
\pgfpathcurveto{\pgfqpoint{0.864167in}{0.833333in}}{\pgfqpoint{0.864167in}{0.847256in}}{\pgfqpoint{0.869698in}{0.860611in}}%
\pgfpathcurveto{\pgfqpoint{0.879544in}{0.870456in}}{\pgfqpoint{0.889389in}{0.880302in}}{\pgfqpoint{0.902744in}{0.885833in}}%
\pgfpathcurveto{\pgfqpoint{0.916667in}{0.885833in}}{\pgfqpoint{0.930590in}{0.885833in}}{\pgfqpoint{0.943945in}{0.880302in}}%
\pgfpathcurveto{\pgfqpoint{0.953790in}{0.870456in}}{\pgfqpoint{0.963635in}{0.860611in}}{\pgfqpoint{0.969167in}{0.847256in}}%
\pgfpathcurveto{\pgfqpoint{0.969167in}{0.833333in}}{\pgfqpoint{0.969167in}{0.819410in}}{\pgfqpoint{0.963635in}{0.806055in}}%
\pgfpathcurveto{\pgfqpoint{0.953790in}{0.796210in}}{\pgfqpoint{0.943945in}{0.786365in}}{\pgfqpoint{0.930590in}{0.780833in}}%
\pgfpathclose%
\pgfpathmoveto{\pgfqpoint{0.000000in}{0.941667in}}%
\pgfpathcurveto{\pgfqpoint{0.015470in}{0.941667in}}{\pgfqpoint{0.030309in}{0.947813in}}{\pgfqpoint{0.041248in}{0.958752in}}%
\pgfpathcurveto{\pgfqpoint{0.052187in}{0.969691in}}{\pgfqpoint{0.058333in}{0.984530in}}{\pgfqpoint{0.058333in}{1.000000in}}%
\pgfpathcurveto{\pgfqpoint{0.058333in}{1.015470in}}{\pgfqpoint{0.052187in}{1.030309in}}{\pgfqpoint{0.041248in}{1.041248in}}%
\pgfpathcurveto{\pgfqpoint{0.030309in}{1.052187in}}{\pgfqpoint{0.015470in}{1.058333in}}{\pgfqpoint{0.000000in}{1.058333in}}%
\pgfpathcurveto{\pgfqpoint{-0.015470in}{1.058333in}}{\pgfqpoint{-0.030309in}{1.052187in}}{\pgfqpoint{-0.041248in}{1.041248in}}%
\pgfpathcurveto{\pgfqpoint{-0.052187in}{1.030309in}}{\pgfqpoint{-0.058333in}{1.015470in}}{\pgfqpoint{-0.058333in}{1.000000in}}%
\pgfpathcurveto{\pgfqpoint{-0.058333in}{0.984530in}}{\pgfqpoint{-0.052187in}{0.969691in}}{\pgfqpoint{-0.041248in}{0.958752in}}%
\pgfpathcurveto{\pgfqpoint{-0.030309in}{0.947813in}}{\pgfqpoint{-0.015470in}{0.941667in}}{\pgfqpoint{0.000000in}{0.941667in}}%
\pgfpathclose%
\pgfpathmoveto{\pgfqpoint{0.000000in}{0.947500in}}%
\pgfpathcurveto{\pgfqpoint{0.000000in}{0.947500in}}{\pgfqpoint{-0.013923in}{0.947500in}}{\pgfqpoint{-0.027278in}{0.953032in}}%
\pgfpathcurveto{\pgfqpoint{-0.037123in}{0.962877in}}{\pgfqpoint{-0.046968in}{0.972722in}}{\pgfqpoint{-0.052500in}{0.986077in}}%
\pgfpathcurveto{\pgfqpoint{-0.052500in}{1.000000in}}{\pgfqpoint{-0.052500in}{1.013923in}}{\pgfqpoint{-0.046968in}{1.027278in}}%
\pgfpathcurveto{\pgfqpoint{-0.037123in}{1.037123in}}{\pgfqpoint{-0.027278in}{1.046968in}}{\pgfqpoint{-0.013923in}{1.052500in}}%
\pgfpathcurveto{\pgfqpoint{0.000000in}{1.052500in}}{\pgfqpoint{0.013923in}{1.052500in}}{\pgfqpoint{0.027278in}{1.046968in}}%
\pgfpathcurveto{\pgfqpoint{0.037123in}{1.037123in}}{\pgfqpoint{0.046968in}{1.027278in}}{\pgfqpoint{0.052500in}{1.013923in}}%
\pgfpathcurveto{\pgfqpoint{0.052500in}{1.000000in}}{\pgfqpoint{0.052500in}{0.986077in}}{\pgfqpoint{0.046968in}{0.972722in}}%
\pgfpathcurveto{\pgfqpoint{0.037123in}{0.962877in}}{\pgfqpoint{0.027278in}{0.953032in}}{\pgfqpoint{0.013923in}{0.947500in}}%
\pgfpathclose%
\pgfpathmoveto{\pgfqpoint{0.166667in}{0.941667in}}%
\pgfpathcurveto{\pgfqpoint{0.182137in}{0.941667in}}{\pgfqpoint{0.196975in}{0.947813in}}{\pgfqpoint{0.207915in}{0.958752in}}%
\pgfpathcurveto{\pgfqpoint{0.218854in}{0.969691in}}{\pgfqpoint{0.225000in}{0.984530in}}{\pgfqpoint{0.225000in}{1.000000in}}%
\pgfpathcurveto{\pgfqpoint{0.225000in}{1.015470in}}{\pgfqpoint{0.218854in}{1.030309in}}{\pgfqpoint{0.207915in}{1.041248in}}%
\pgfpathcurveto{\pgfqpoint{0.196975in}{1.052187in}}{\pgfqpoint{0.182137in}{1.058333in}}{\pgfqpoint{0.166667in}{1.058333in}}%
\pgfpathcurveto{\pgfqpoint{0.151196in}{1.058333in}}{\pgfqpoint{0.136358in}{1.052187in}}{\pgfqpoint{0.125419in}{1.041248in}}%
\pgfpathcurveto{\pgfqpoint{0.114480in}{1.030309in}}{\pgfqpoint{0.108333in}{1.015470in}}{\pgfqpoint{0.108333in}{1.000000in}}%
\pgfpathcurveto{\pgfqpoint{0.108333in}{0.984530in}}{\pgfqpoint{0.114480in}{0.969691in}}{\pgfqpoint{0.125419in}{0.958752in}}%
\pgfpathcurveto{\pgfqpoint{0.136358in}{0.947813in}}{\pgfqpoint{0.151196in}{0.941667in}}{\pgfqpoint{0.166667in}{0.941667in}}%
\pgfpathclose%
\pgfpathmoveto{\pgfqpoint{0.166667in}{0.947500in}}%
\pgfpathcurveto{\pgfqpoint{0.166667in}{0.947500in}}{\pgfqpoint{0.152744in}{0.947500in}}{\pgfqpoint{0.139389in}{0.953032in}}%
\pgfpathcurveto{\pgfqpoint{0.129544in}{0.962877in}}{\pgfqpoint{0.119698in}{0.972722in}}{\pgfqpoint{0.114167in}{0.986077in}}%
\pgfpathcurveto{\pgfqpoint{0.114167in}{1.000000in}}{\pgfqpoint{0.114167in}{1.013923in}}{\pgfqpoint{0.119698in}{1.027278in}}%
\pgfpathcurveto{\pgfqpoint{0.129544in}{1.037123in}}{\pgfqpoint{0.139389in}{1.046968in}}{\pgfqpoint{0.152744in}{1.052500in}}%
\pgfpathcurveto{\pgfqpoint{0.166667in}{1.052500in}}{\pgfqpoint{0.180590in}{1.052500in}}{\pgfqpoint{0.193945in}{1.046968in}}%
\pgfpathcurveto{\pgfqpoint{0.203790in}{1.037123in}}{\pgfqpoint{0.213635in}{1.027278in}}{\pgfqpoint{0.219167in}{1.013923in}}%
\pgfpathcurveto{\pgfqpoint{0.219167in}{1.000000in}}{\pgfqpoint{0.219167in}{0.986077in}}{\pgfqpoint{0.213635in}{0.972722in}}%
\pgfpathcurveto{\pgfqpoint{0.203790in}{0.962877in}}{\pgfqpoint{0.193945in}{0.953032in}}{\pgfqpoint{0.180590in}{0.947500in}}%
\pgfpathclose%
\pgfpathmoveto{\pgfqpoint{0.333333in}{0.941667in}}%
\pgfpathcurveto{\pgfqpoint{0.348804in}{0.941667in}}{\pgfqpoint{0.363642in}{0.947813in}}{\pgfqpoint{0.374581in}{0.958752in}}%
\pgfpathcurveto{\pgfqpoint{0.385520in}{0.969691in}}{\pgfqpoint{0.391667in}{0.984530in}}{\pgfqpoint{0.391667in}{1.000000in}}%
\pgfpathcurveto{\pgfqpoint{0.391667in}{1.015470in}}{\pgfqpoint{0.385520in}{1.030309in}}{\pgfqpoint{0.374581in}{1.041248in}}%
\pgfpathcurveto{\pgfqpoint{0.363642in}{1.052187in}}{\pgfqpoint{0.348804in}{1.058333in}}{\pgfqpoint{0.333333in}{1.058333in}}%
\pgfpathcurveto{\pgfqpoint{0.317863in}{1.058333in}}{\pgfqpoint{0.303025in}{1.052187in}}{\pgfqpoint{0.292085in}{1.041248in}}%
\pgfpathcurveto{\pgfqpoint{0.281146in}{1.030309in}}{\pgfqpoint{0.275000in}{1.015470in}}{\pgfqpoint{0.275000in}{1.000000in}}%
\pgfpathcurveto{\pgfqpoint{0.275000in}{0.984530in}}{\pgfqpoint{0.281146in}{0.969691in}}{\pgfqpoint{0.292085in}{0.958752in}}%
\pgfpathcurveto{\pgfqpoint{0.303025in}{0.947813in}}{\pgfqpoint{0.317863in}{0.941667in}}{\pgfqpoint{0.333333in}{0.941667in}}%
\pgfpathclose%
\pgfpathmoveto{\pgfqpoint{0.333333in}{0.947500in}}%
\pgfpathcurveto{\pgfqpoint{0.333333in}{0.947500in}}{\pgfqpoint{0.319410in}{0.947500in}}{\pgfqpoint{0.306055in}{0.953032in}}%
\pgfpathcurveto{\pgfqpoint{0.296210in}{0.962877in}}{\pgfqpoint{0.286365in}{0.972722in}}{\pgfqpoint{0.280833in}{0.986077in}}%
\pgfpathcurveto{\pgfqpoint{0.280833in}{1.000000in}}{\pgfqpoint{0.280833in}{1.013923in}}{\pgfqpoint{0.286365in}{1.027278in}}%
\pgfpathcurveto{\pgfqpoint{0.296210in}{1.037123in}}{\pgfqpoint{0.306055in}{1.046968in}}{\pgfqpoint{0.319410in}{1.052500in}}%
\pgfpathcurveto{\pgfqpoint{0.333333in}{1.052500in}}{\pgfqpoint{0.347256in}{1.052500in}}{\pgfqpoint{0.360611in}{1.046968in}}%
\pgfpathcurveto{\pgfqpoint{0.370456in}{1.037123in}}{\pgfqpoint{0.380302in}{1.027278in}}{\pgfqpoint{0.385833in}{1.013923in}}%
\pgfpathcurveto{\pgfqpoint{0.385833in}{1.000000in}}{\pgfqpoint{0.385833in}{0.986077in}}{\pgfqpoint{0.380302in}{0.972722in}}%
\pgfpathcurveto{\pgfqpoint{0.370456in}{0.962877in}}{\pgfqpoint{0.360611in}{0.953032in}}{\pgfqpoint{0.347256in}{0.947500in}}%
\pgfpathclose%
\pgfpathmoveto{\pgfqpoint{0.500000in}{0.941667in}}%
\pgfpathcurveto{\pgfqpoint{0.515470in}{0.941667in}}{\pgfqpoint{0.530309in}{0.947813in}}{\pgfqpoint{0.541248in}{0.958752in}}%
\pgfpathcurveto{\pgfqpoint{0.552187in}{0.969691in}}{\pgfqpoint{0.558333in}{0.984530in}}{\pgfqpoint{0.558333in}{1.000000in}}%
\pgfpathcurveto{\pgfqpoint{0.558333in}{1.015470in}}{\pgfqpoint{0.552187in}{1.030309in}}{\pgfqpoint{0.541248in}{1.041248in}}%
\pgfpathcurveto{\pgfqpoint{0.530309in}{1.052187in}}{\pgfqpoint{0.515470in}{1.058333in}}{\pgfqpoint{0.500000in}{1.058333in}}%
\pgfpathcurveto{\pgfqpoint{0.484530in}{1.058333in}}{\pgfqpoint{0.469691in}{1.052187in}}{\pgfqpoint{0.458752in}{1.041248in}}%
\pgfpathcurveto{\pgfqpoint{0.447813in}{1.030309in}}{\pgfqpoint{0.441667in}{1.015470in}}{\pgfqpoint{0.441667in}{1.000000in}}%
\pgfpathcurveto{\pgfqpoint{0.441667in}{0.984530in}}{\pgfqpoint{0.447813in}{0.969691in}}{\pgfqpoint{0.458752in}{0.958752in}}%
\pgfpathcurveto{\pgfqpoint{0.469691in}{0.947813in}}{\pgfqpoint{0.484530in}{0.941667in}}{\pgfqpoint{0.500000in}{0.941667in}}%
\pgfpathclose%
\pgfpathmoveto{\pgfqpoint{0.500000in}{0.947500in}}%
\pgfpathcurveto{\pgfqpoint{0.500000in}{0.947500in}}{\pgfqpoint{0.486077in}{0.947500in}}{\pgfqpoint{0.472722in}{0.953032in}}%
\pgfpathcurveto{\pgfqpoint{0.462877in}{0.962877in}}{\pgfqpoint{0.453032in}{0.972722in}}{\pgfqpoint{0.447500in}{0.986077in}}%
\pgfpathcurveto{\pgfqpoint{0.447500in}{1.000000in}}{\pgfqpoint{0.447500in}{1.013923in}}{\pgfqpoint{0.453032in}{1.027278in}}%
\pgfpathcurveto{\pgfqpoint{0.462877in}{1.037123in}}{\pgfqpoint{0.472722in}{1.046968in}}{\pgfqpoint{0.486077in}{1.052500in}}%
\pgfpathcurveto{\pgfqpoint{0.500000in}{1.052500in}}{\pgfqpoint{0.513923in}{1.052500in}}{\pgfqpoint{0.527278in}{1.046968in}}%
\pgfpathcurveto{\pgfqpoint{0.537123in}{1.037123in}}{\pgfqpoint{0.546968in}{1.027278in}}{\pgfqpoint{0.552500in}{1.013923in}}%
\pgfpathcurveto{\pgfqpoint{0.552500in}{1.000000in}}{\pgfqpoint{0.552500in}{0.986077in}}{\pgfqpoint{0.546968in}{0.972722in}}%
\pgfpathcurveto{\pgfqpoint{0.537123in}{0.962877in}}{\pgfqpoint{0.527278in}{0.953032in}}{\pgfqpoint{0.513923in}{0.947500in}}%
\pgfpathclose%
\pgfpathmoveto{\pgfqpoint{0.666667in}{0.941667in}}%
\pgfpathcurveto{\pgfqpoint{0.682137in}{0.941667in}}{\pgfqpoint{0.696975in}{0.947813in}}{\pgfqpoint{0.707915in}{0.958752in}}%
\pgfpathcurveto{\pgfqpoint{0.718854in}{0.969691in}}{\pgfqpoint{0.725000in}{0.984530in}}{\pgfqpoint{0.725000in}{1.000000in}}%
\pgfpathcurveto{\pgfqpoint{0.725000in}{1.015470in}}{\pgfqpoint{0.718854in}{1.030309in}}{\pgfqpoint{0.707915in}{1.041248in}}%
\pgfpathcurveto{\pgfqpoint{0.696975in}{1.052187in}}{\pgfqpoint{0.682137in}{1.058333in}}{\pgfqpoint{0.666667in}{1.058333in}}%
\pgfpathcurveto{\pgfqpoint{0.651196in}{1.058333in}}{\pgfqpoint{0.636358in}{1.052187in}}{\pgfqpoint{0.625419in}{1.041248in}}%
\pgfpathcurveto{\pgfqpoint{0.614480in}{1.030309in}}{\pgfqpoint{0.608333in}{1.015470in}}{\pgfqpoint{0.608333in}{1.000000in}}%
\pgfpathcurveto{\pgfqpoint{0.608333in}{0.984530in}}{\pgfqpoint{0.614480in}{0.969691in}}{\pgfqpoint{0.625419in}{0.958752in}}%
\pgfpathcurveto{\pgfqpoint{0.636358in}{0.947813in}}{\pgfqpoint{0.651196in}{0.941667in}}{\pgfqpoint{0.666667in}{0.941667in}}%
\pgfpathclose%
\pgfpathmoveto{\pgfqpoint{0.666667in}{0.947500in}}%
\pgfpathcurveto{\pgfqpoint{0.666667in}{0.947500in}}{\pgfqpoint{0.652744in}{0.947500in}}{\pgfqpoint{0.639389in}{0.953032in}}%
\pgfpathcurveto{\pgfqpoint{0.629544in}{0.962877in}}{\pgfqpoint{0.619698in}{0.972722in}}{\pgfqpoint{0.614167in}{0.986077in}}%
\pgfpathcurveto{\pgfqpoint{0.614167in}{1.000000in}}{\pgfqpoint{0.614167in}{1.013923in}}{\pgfqpoint{0.619698in}{1.027278in}}%
\pgfpathcurveto{\pgfqpoint{0.629544in}{1.037123in}}{\pgfqpoint{0.639389in}{1.046968in}}{\pgfqpoint{0.652744in}{1.052500in}}%
\pgfpathcurveto{\pgfqpoint{0.666667in}{1.052500in}}{\pgfqpoint{0.680590in}{1.052500in}}{\pgfqpoint{0.693945in}{1.046968in}}%
\pgfpathcurveto{\pgfqpoint{0.703790in}{1.037123in}}{\pgfqpoint{0.713635in}{1.027278in}}{\pgfqpoint{0.719167in}{1.013923in}}%
\pgfpathcurveto{\pgfqpoint{0.719167in}{1.000000in}}{\pgfqpoint{0.719167in}{0.986077in}}{\pgfqpoint{0.713635in}{0.972722in}}%
\pgfpathcurveto{\pgfqpoint{0.703790in}{0.962877in}}{\pgfqpoint{0.693945in}{0.953032in}}{\pgfqpoint{0.680590in}{0.947500in}}%
\pgfpathclose%
\pgfpathmoveto{\pgfqpoint{0.833333in}{0.941667in}}%
\pgfpathcurveto{\pgfqpoint{0.848804in}{0.941667in}}{\pgfqpoint{0.863642in}{0.947813in}}{\pgfqpoint{0.874581in}{0.958752in}}%
\pgfpathcurveto{\pgfqpoint{0.885520in}{0.969691in}}{\pgfqpoint{0.891667in}{0.984530in}}{\pgfqpoint{0.891667in}{1.000000in}}%
\pgfpathcurveto{\pgfqpoint{0.891667in}{1.015470in}}{\pgfqpoint{0.885520in}{1.030309in}}{\pgfqpoint{0.874581in}{1.041248in}}%
\pgfpathcurveto{\pgfqpoint{0.863642in}{1.052187in}}{\pgfqpoint{0.848804in}{1.058333in}}{\pgfqpoint{0.833333in}{1.058333in}}%
\pgfpathcurveto{\pgfqpoint{0.817863in}{1.058333in}}{\pgfqpoint{0.803025in}{1.052187in}}{\pgfqpoint{0.792085in}{1.041248in}}%
\pgfpathcurveto{\pgfqpoint{0.781146in}{1.030309in}}{\pgfqpoint{0.775000in}{1.015470in}}{\pgfqpoint{0.775000in}{1.000000in}}%
\pgfpathcurveto{\pgfqpoint{0.775000in}{0.984530in}}{\pgfqpoint{0.781146in}{0.969691in}}{\pgfqpoint{0.792085in}{0.958752in}}%
\pgfpathcurveto{\pgfqpoint{0.803025in}{0.947813in}}{\pgfqpoint{0.817863in}{0.941667in}}{\pgfqpoint{0.833333in}{0.941667in}}%
\pgfpathclose%
\pgfpathmoveto{\pgfqpoint{0.833333in}{0.947500in}}%
\pgfpathcurveto{\pgfqpoint{0.833333in}{0.947500in}}{\pgfqpoint{0.819410in}{0.947500in}}{\pgfqpoint{0.806055in}{0.953032in}}%
\pgfpathcurveto{\pgfqpoint{0.796210in}{0.962877in}}{\pgfqpoint{0.786365in}{0.972722in}}{\pgfqpoint{0.780833in}{0.986077in}}%
\pgfpathcurveto{\pgfqpoint{0.780833in}{1.000000in}}{\pgfqpoint{0.780833in}{1.013923in}}{\pgfqpoint{0.786365in}{1.027278in}}%
\pgfpathcurveto{\pgfqpoint{0.796210in}{1.037123in}}{\pgfqpoint{0.806055in}{1.046968in}}{\pgfqpoint{0.819410in}{1.052500in}}%
\pgfpathcurveto{\pgfqpoint{0.833333in}{1.052500in}}{\pgfqpoint{0.847256in}{1.052500in}}{\pgfqpoint{0.860611in}{1.046968in}}%
\pgfpathcurveto{\pgfqpoint{0.870456in}{1.037123in}}{\pgfqpoint{0.880302in}{1.027278in}}{\pgfqpoint{0.885833in}{1.013923in}}%
\pgfpathcurveto{\pgfqpoint{0.885833in}{1.000000in}}{\pgfqpoint{0.885833in}{0.986077in}}{\pgfqpoint{0.880302in}{0.972722in}}%
\pgfpathcurveto{\pgfqpoint{0.870456in}{0.962877in}}{\pgfqpoint{0.860611in}{0.953032in}}{\pgfqpoint{0.847256in}{0.947500in}}%
\pgfpathclose%
\pgfpathmoveto{\pgfqpoint{1.000000in}{0.941667in}}%
\pgfpathcurveto{\pgfqpoint{1.015470in}{0.941667in}}{\pgfqpoint{1.030309in}{0.947813in}}{\pgfqpoint{1.041248in}{0.958752in}}%
\pgfpathcurveto{\pgfqpoint{1.052187in}{0.969691in}}{\pgfqpoint{1.058333in}{0.984530in}}{\pgfqpoint{1.058333in}{1.000000in}}%
\pgfpathcurveto{\pgfqpoint{1.058333in}{1.015470in}}{\pgfqpoint{1.052187in}{1.030309in}}{\pgfqpoint{1.041248in}{1.041248in}}%
\pgfpathcurveto{\pgfqpoint{1.030309in}{1.052187in}}{\pgfqpoint{1.015470in}{1.058333in}}{\pgfqpoint{1.000000in}{1.058333in}}%
\pgfpathcurveto{\pgfqpoint{0.984530in}{1.058333in}}{\pgfqpoint{0.969691in}{1.052187in}}{\pgfqpoint{0.958752in}{1.041248in}}%
\pgfpathcurveto{\pgfqpoint{0.947813in}{1.030309in}}{\pgfqpoint{0.941667in}{1.015470in}}{\pgfqpoint{0.941667in}{1.000000in}}%
\pgfpathcurveto{\pgfqpoint{0.941667in}{0.984530in}}{\pgfqpoint{0.947813in}{0.969691in}}{\pgfqpoint{0.958752in}{0.958752in}}%
\pgfpathcurveto{\pgfqpoint{0.969691in}{0.947813in}}{\pgfqpoint{0.984530in}{0.941667in}}{\pgfqpoint{1.000000in}{0.941667in}}%
\pgfpathclose%
\pgfpathmoveto{\pgfqpoint{1.000000in}{0.947500in}}%
\pgfpathcurveto{\pgfqpoint{1.000000in}{0.947500in}}{\pgfqpoint{0.986077in}{0.947500in}}{\pgfqpoint{0.972722in}{0.953032in}}%
\pgfpathcurveto{\pgfqpoint{0.962877in}{0.962877in}}{\pgfqpoint{0.953032in}{0.972722in}}{\pgfqpoint{0.947500in}{0.986077in}}%
\pgfpathcurveto{\pgfqpoint{0.947500in}{1.000000in}}{\pgfqpoint{0.947500in}{1.013923in}}{\pgfqpoint{0.953032in}{1.027278in}}%
\pgfpathcurveto{\pgfqpoint{0.962877in}{1.037123in}}{\pgfqpoint{0.972722in}{1.046968in}}{\pgfqpoint{0.986077in}{1.052500in}}%
\pgfpathcurveto{\pgfqpoint{1.000000in}{1.052500in}}{\pgfqpoint{1.013923in}{1.052500in}}{\pgfqpoint{1.027278in}{1.046968in}}%
\pgfpathcurveto{\pgfqpoint{1.037123in}{1.037123in}}{\pgfqpoint{1.046968in}{1.027278in}}{\pgfqpoint{1.052500in}{1.013923in}}%
\pgfpathcurveto{\pgfqpoint{1.052500in}{1.000000in}}{\pgfqpoint{1.052500in}{0.986077in}}{\pgfqpoint{1.046968in}{0.972722in}}%
\pgfpathcurveto{\pgfqpoint{1.037123in}{0.962877in}}{\pgfqpoint{1.027278in}{0.953032in}}{\pgfqpoint{1.013923in}{0.947500in}}%
\pgfpathclose%
\pgfusepath{stroke}%
\end{pgfscope}%
}%
\pgfsys@transformshift{7.523315in}{3.003473in}%
\pgfsys@useobject{currentpattern}{}%
\pgfsys@transformshift{1in}{0in}%
\pgfsys@transformshift{-1in}{0in}%
\pgfsys@transformshift{0in}{1in}%
\pgfsys@useobject{currentpattern}{}%
\pgfsys@transformshift{1in}{0in}%
\pgfsys@transformshift{-1in}{0in}%
\pgfsys@transformshift{0in}{1in}%
\pgfsys@useobject{currentpattern}{}%
\pgfsys@transformshift{1in}{0in}%
\pgfsys@transformshift{-1in}{0in}%
\pgfsys@transformshift{0in}{1in}%
\end{pgfscope}%
\begin{pgfscope}%
\pgfpathrectangle{\pgfqpoint{0.935815in}{0.637495in}}{\pgfqpoint{9.300000in}{9.060000in}}%
\pgfusepath{clip}%
\pgfsetbuttcap%
\pgfsetmiterjoin%
\definecolor{currentfill}{rgb}{0.549020,0.337255,0.294118}%
\pgfsetfillcolor{currentfill}%
\pgfsetfillopacity{0.990000}%
\pgfsetlinewidth{0.000000pt}%
\definecolor{currentstroke}{rgb}{0.000000,0.000000,0.000000}%
\pgfsetstrokecolor{currentstroke}%
\pgfsetstrokeopacity{0.990000}%
\pgfsetdash{}{0pt}%
\pgfpathmoveto{\pgfqpoint{9.073315in}{3.003473in}}%
\pgfpathlineto{\pgfqpoint{9.848315in}{3.003473in}}%
\pgfpathlineto{\pgfqpoint{9.848315in}{5.392542in}}%
\pgfpathlineto{\pgfqpoint{9.073315in}{5.392542in}}%
\pgfpathclose%
\pgfusepath{fill}%
\end{pgfscope}%
\begin{pgfscope}%
\pgfsetbuttcap%
\pgfsetmiterjoin%
\definecolor{currentfill}{rgb}{0.549020,0.337255,0.294118}%
\pgfsetfillcolor{currentfill}%
\pgfsetfillopacity{0.990000}%
\pgfsetlinewidth{0.000000pt}%
\definecolor{currentstroke}{rgb}{0.000000,0.000000,0.000000}%
\pgfsetstrokecolor{currentstroke}%
\pgfsetstrokeopacity{0.990000}%
\pgfsetdash{}{0pt}%
\pgfpathrectangle{\pgfqpoint{0.935815in}{0.637495in}}{\pgfqpoint{9.300000in}{9.060000in}}%
\pgfusepath{clip}%
\pgfpathmoveto{\pgfqpoint{9.073315in}{3.003473in}}%
\pgfpathlineto{\pgfqpoint{9.848315in}{3.003473in}}%
\pgfpathlineto{\pgfqpoint{9.848315in}{5.392542in}}%
\pgfpathlineto{\pgfqpoint{9.073315in}{5.392542in}}%
\pgfpathclose%
\pgfusepath{clip}%
\pgfsys@defobject{currentpattern}{\pgfqpoint{0in}{0in}}{\pgfqpoint{1in}{1in}}{%
\begin{pgfscope}%
\pgfpathrectangle{\pgfqpoint{0in}{0in}}{\pgfqpoint{1in}{1in}}%
\pgfusepath{clip}%
\pgfpathmoveto{\pgfqpoint{0.000000in}{-0.058333in}}%
\pgfpathcurveto{\pgfqpoint{0.015470in}{-0.058333in}}{\pgfqpoint{0.030309in}{-0.052187in}}{\pgfqpoint{0.041248in}{-0.041248in}}%
\pgfpathcurveto{\pgfqpoint{0.052187in}{-0.030309in}}{\pgfqpoint{0.058333in}{-0.015470in}}{\pgfqpoint{0.058333in}{0.000000in}}%
\pgfpathcurveto{\pgfqpoint{0.058333in}{0.015470in}}{\pgfqpoint{0.052187in}{0.030309in}}{\pgfqpoint{0.041248in}{0.041248in}}%
\pgfpathcurveto{\pgfqpoint{0.030309in}{0.052187in}}{\pgfqpoint{0.015470in}{0.058333in}}{\pgfqpoint{0.000000in}{0.058333in}}%
\pgfpathcurveto{\pgfqpoint{-0.015470in}{0.058333in}}{\pgfqpoint{-0.030309in}{0.052187in}}{\pgfqpoint{-0.041248in}{0.041248in}}%
\pgfpathcurveto{\pgfqpoint{-0.052187in}{0.030309in}}{\pgfqpoint{-0.058333in}{0.015470in}}{\pgfqpoint{-0.058333in}{0.000000in}}%
\pgfpathcurveto{\pgfqpoint{-0.058333in}{-0.015470in}}{\pgfqpoint{-0.052187in}{-0.030309in}}{\pgfqpoint{-0.041248in}{-0.041248in}}%
\pgfpathcurveto{\pgfqpoint{-0.030309in}{-0.052187in}}{\pgfqpoint{-0.015470in}{-0.058333in}}{\pgfqpoint{0.000000in}{-0.058333in}}%
\pgfpathclose%
\pgfpathmoveto{\pgfqpoint{0.000000in}{-0.052500in}}%
\pgfpathcurveto{\pgfqpoint{0.000000in}{-0.052500in}}{\pgfqpoint{-0.013923in}{-0.052500in}}{\pgfqpoint{-0.027278in}{-0.046968in}}%
\pgfpathcurveto{\pgfqpoint{-0.037123in}{-0.037123in}}{\pgfqpoint{-0.046968in}{-0.027278in}}{\pgfqpoint{-0.052500in}{-0.013923in}}%
\pgfpathcurveto{\pgfqpoint{-0.052500in}{0.000000in}}{\pgfqpoint{-0.052500in}{0.013923in}}{\pgfqpoint{-0.046968in}{0.027278in}}%
\pgfpathcurveto{\pgfqpoint{-0.037123in}{0.037123in}}{\pgfqpoint{-0.027278in}{0.046968in}}{\pgfqpoint{-0.013923in}{0.052500in}}%
\pgfpathcurveto{\pgfqpoint{0.000000in}{0.052500in}}{\pgfqpoint{0.013923in}{0.052500in}}{\pgfqpoint{0.027278in}{0.046968in}}%
\pgfpathcurveto{\pgfqpoint{0.037123in}{0.037123in}}{\pgfqpoint{0.046968in}{0.027278in}}{\pgfqpoint{0.052500in}{0.013923in}}%
\pgfpathcurveto{\pgfqpoint{0.052500in}{0.000000in}}{\pgfqpoint{0.052500in}{-0.013923in}}{\pgfqpoint{0.046968in}{-0.027278in}}%
\pgfpathcurveto{\pgfqpoint{0.037123in}{-0.037123in}}{\pgfqpoint{0.027278in}{-0.046968in}}{\pgfqpoint{0.013923in}{-0.052500in}}%
\pgfpathclose%
\pgfpathmoveto{\pgfqpoint{0.166667in}{-0.058333in}}%
\pgfpathcurveto{\pgfqpoint{0.182137in}{-0.058333in}}{\pgfqpoint{0.196975in}{-0.052187in}}{\pgfqpoint{0.207915in}{-0.041248in}}%
\pgfpathcurveto{\pgfqpoint{0.218854in}{-0.030309in}}{\pgfqpoint{0.225000in}{-0.015470in}}{\pgfqpoint{0.225000in}{0.000000in}}%
\pgfpathcurveto{\pgfqpoint{0.225000in}{0.015470in}}{\pgfqpoint{0.218854in}{0.030309in}}{\pgfqpoint{0.207915in}{0.041248in}}%
\pgfpathcurveto{\pgfqpoint{0.196975in}{0.052187in}}{\pgfqpoint{0.182137in}{0.058333in}}{\pgfqpoint{0.166667in}{0.058333in}}%
\pgfpathcurveto{\pgfqpoint{0.151196in}{0.058333in}}{\pgfqpoint{0.136358in}{0.052187in}}{\pgfqpoint{0.125419in}{0.041248in}}%
\pgfpathcurveto{\pgfqpoint{0.114480in}{0.030309in}}{\pgfqpoint{0.108333in}{0.015470in}}{\pgfqpoint{0.108333in}{0.000000in}}%
\pgfpathcurveto{\pgfqpoint{0.108333in}{-0.015470in}}{\pgfqpoint{0.114480in}{-0.030309in}}{\pgfqpoint{0.125419in}{-0.041248in}}%
\pgfpathcurveto{\pgfqpoint{0.136358in}{-0.052187in}}{\pgfqpoint{0.151196in}{-0.058333in}}{\pgfqpoint{0.166667in}{-0.058333in}}%
\pgfpathclose%
\pgfpathmoveto{\pgfqpoint{0.166667in}{-0.052500in}}%
\pgfpathcurveto{\pgfqpoint{0.166667in}{-0.052500in}}{\pgfqpoint{0.152744in}{-0.052500in}}{\pgfqpoint{0.139389in}{-0.046968in}}%
\pgfpathcurveto{\pgfqpoint{0.129544in}{-0.037123in}}{\pgfqpoint{0.119698in}{-0.027278in}}{\pgfqpoint{0.114167in}{-0.013923in}}%
\pgfpathcurveto{\pgfqpoint{0.114167in}{0.000000in}}{\pgfqpoint{0.114167in}{0.013923in}}{\pgfqpoint{0.119698in}{0.027278in}}%
\pgfpathcurveto{\pgfqpoint{0.129544in}{0.037123in}}{\pgfqpoint{0.139389in}{0.046968in}}{\pgfqpoint{0.152744in}{0.052500in}}%
\pgfpathcurveto{\pgfqpoint{0.166667in}{0.052500in}}{\pgfqpoint{0.180590in}{0.052500in}}{\pgfqpoint{0.193945in}{0.046968in}}%
\pgfpathcurveto{\pgfqpoint{0.203790in}{0.037123in}}{\pgfqpoint{0.213635in}{0.027278in}}{\pgfqpoint{0.219167in}{0.013923in}}%
\pgfpathcurveto{\pgfqpoint{0.219167in}{0.000000in}}{\pgfqpoint{0.219167in}{-0.013923in}}{\pgfqpoint{0.213635in}{-0.027278in}}%
\pgfpathcurveto{\pgfqpoint{0.203790in}{-0.037123in}}{\pgfqpoint{0.193945in}{-0.046968in}}{\pgfqpoint{0.180590in}{-0.052500in}}%
\pgfpathclose%
\pgfpathmoveto{\pgfqpoint{0.333333in}{-0.058333in}}%
\pgfpathcurveto{\pgfqpoint{0.348804in}{-0.058333in}}{\pgfqpoint{0.363642in}{-0.052187in}}{\pgfqpoint{0.374581in}{-0.041248in}}%
\pgfpathcurveto{\pgfqpoint{0.385520in}{-0.030309in}}{\pgfqpoint{0.391667in}{-0.015470in}}{\pgfqpoint{0.391667in}{0.000000in}}%
\pgfpathcurveto{\pgfqpoint{0.391667in}{0.015470in}}{\pgfqpoint{0.385520in}{0.030309in}}{\pgfqpoint{0.374581in}{0.041248in}}%
\pgfpathcurveto{\pgfqpoint{0.363642in}{0.052187in}}{\pgfqpoint{0.348804in}{0.058333in}}{\pgfqpoint{0.333333in}{0.058333in}}%
\pgfpathcurveto{\pgfqpoint{0.317863in}{0.058333in}}{\pgfqpoint{0.303025in}{0.052187in}}{\pgfqpoint{0.292085in}{0.041248in}}%
\pgfpathcurveto{\pgfqpoint{0.281146in}{0.030309in}}{\pgfqpoint{0.275000in}{0.015470in}}{\pgfqpoint{0.275000in}{0.000000in}}%
\pgfpathcurveto{\pgfqpoint{0.275000in}{-0.015470in}}{\pgfqpoint{0.281146in}{-0.030309in}}{\pgfqpoint{0.292085in}{-0.041248in}}%
\pgfpathcurveto{\pgfqpoint{0.303025in}{-0.052187in}}{\pgfqpoint{0.317863in}{-0.058333in}}{\pgfqpoint{0.333333in}{-0.058333in}}%
\pgfpathclose%
\pgfpathmoveto{\pgfqpoint{0.333333in}{-0.052500in}}%
\pgfpathcurveto{\pgfqpoint{0.333333in}{-0.052500in}}{\pgfqpoint{0.319410in}{-0.052500in}}{\pgfqpoint{0.306055in}{-0.046968in}}%
\pgfpathcurveto{\pgfqpoint{0.296210in}{-0.037123in}}{\pgfqpoint{0.286365in}{-0.027278in}}{\pgfqpoint{0.280833in}{-0.013923in}}%
\pgfpathcurveto{\pgfqpoint{0.280833in}{0.000000in}}{\pgfqpoint{0.280833in}{0.013923in}}{\pgfqpoint{0.286365in}{0.027278in}}%
\pgfpathcurveto{\pgfqpoint{0.296210in}{0.037123in}}{\pgfqpoint{0.306055in}{0.046968in}}{\pgfqpoint{0.319410in}{0.052500in}}%
\pgfpathcurveto{\pgfqpoint{0.333333in}{0.052500in}}{\pgfqpoint{0.347256in}{0.052500in}}{\pgfqpoint{0.360611in}{0.046968in}}%
\pgfpathcurveto{\pgfqpoint{0.370456in}{0.037123in}}{\pgfqpoint{0.380302in}{0.027278in}}{\pgfqpoint{0.385833in}{0.013923in}}%
\pgfpathcurveto{\pgfqpoint{0.385833in}{0.000000in}}{\pgfqpoint{0.385833in}{-0.013923in}}{\pgfqpoint{0.380302in}{-0.027278in}}%
\pgfpathcurveto{\pgfqpoint{0.370456in}{-0.037123in}}{\pgfqpoint{0.360611in}{-0.046968in}}{\pgfqpoint{0.347256in}{-0.052500in}}%
\pgfpathclose%
\pgfpathmoveto{\pgfqpoint{0.500000in}{-0.058333in}}%
\pgfpathcurveto{\pgfqpoint{0.515470in}{-0.058333in}}{\pgfqpoint{0.530309in}{-0.052187in}}{\pgfqpoint{0.541248in}{-0.041248in}}%
\pgfpathcurveto{\pgfqpoint{0.552187in}{-0.030309in}}{\pgfqpoint{0.558333in}{-0.015470in}}{\pgfqpoint{0.558333in}{0.000000in}}%
\pgfpathcurveto{\pgfqpoint{0.558333in}{0.015470in}}{\pgfqpoint{0.552187in}{0.030309in}}{\pgfqpoint{0.541248in}{0.041248in}}%
\pgfpathcurveto{\pgfqpoint{0.530309in}{0.052187in}}{\pgfqpoint{0.515470in}{0.058333in}}{\pgfqpoint{0.500000in}{0.058333in}}%
\pgfpathcurveto{\pgfqpoint{0.484530in}{0.058333in}}{\pgfqpoint{0.469691in}{0.052187in}}{\pgfqpoint{0.458752in}{0.041248in}}%
\pgfpathcurveto{\pgfqpoint{0.447813in}{0.030309in}}{\pgfqpoint{0.441667in}{0.015470in}}{\pgfqpoint{0.441667in}{0.000000in}}%
\pgfpathcurveto{\pgfqpoint{0.441667in}{-0.015470in}}{\pgfqpoint{0.447813in}{-0.030309in}}{\pgfqpoint{0.458752in}{-0.041248in}}%
\pgfpathcurveto{\pgfqpoint{0.469691in}{-0.052187in}}{\pgfqpoint{0.484530in}{-0.058333in}}{\pgfqpoint{0.500000in}{-0.058333in}}%
\pgfpathclose%
\pgfpathmoveto{\pgfqpoint{0.500000in}{-0.052500in}}%
\pgfpathcurveto{\pgfqpoint{0.500000in}{-0.052500in}}{\pgfqpoint{0.486077in}{-0.052500in}}{\pgfqpoint{0.472722in}{-0.046968in}}%
\pgfpathcurveto{\pgfqpoint{0.462877in}{-0.037123in}}{\pgfqpoint{0.453032in}{-0.027278in}}{\pgfqpoint{0.447500in}{-0.013923in}}%
\pgfpathcurveto{\pgfqpoint{0.447500in}{0.000000in}}{\pgfqpoint{0.447500in}{0.013923in}}{\pgfqpoint{0.453032in}{0.027278in}}%
\pgfpathcurveto{\pgfqpoint{0.462877in}{0.037123in}}{\pgfqpoint{0.472722in}{0.046968in}}{\pgfqpoint{0.486077in}{0.052500in}}%
\pgfpathcurveto{\pgfqpoint{0.500000in}{0.052500in}}{\pgfqpoint{0.513923in}{0.052500in}}{\pgfqpoint{0.527278in}{0.046968in}}%
\pgfpathcurveto{\pgfqpoint{0.537123in}{0.037123in}}{\pgfqpoint{0.546968in}{0.027278in}}{\pgfqpoint{0.552500in}{0.013923in}}%
\pgfpathcurveto{\pgfqpoint{0.552500in}{0.000000in}}{\pgfqpoint{0.552500in}{-0.013923in}}{\pgfqpoint{0.546968in}{-0.027278in}}%
\pgfpathcurveto{\pgfqpoint{0.537123in}{-0.037123in}}{\pgfqpoint{0.527278in}{-0.046968in}}{\pgfqpoint{0.513923in}{-0.052500in}}%
\pgfpathclose%
\pgfpathmoveto{\pgfqpoint{0.666667in}{-0.058333in}}%
\pgfpathcurveto{\pgfqpoint{0.682137in}{-0.058333in}}{\pgfqpoint{0.696975in}{-0.052187in}}{\pgfqpoint{0.707915in}{-0.041248in}}%
\pgfpathcurveto{\pgfqpoint{0.718854in}{-0.030309in}}{\pgfqpoint{0.725000in}{-0.015470in}}{\pgfqpoint{0.725000in}{0.000000in}}%
\pgfpathcurveto{\pgfqpoint{0.725000in}{0.015470in}}{\pgfqpoint{0.718854in}{0.030309in}}{\pgfqpoint{0.707915in}{0.041248in}}%
\pgfpathcurveto{\pgfqpoint{0.696975in}{0.052187in}}{\pgfqpoint{0.682137in}{0.058333in}}{\pgfqpoint{0.666667in}{0.058333in}}%
\pgfpathcurveto{\pgfqpoint{0.651196in}{0.058333in}}{\pgfqpoint{0.636358in}{0.052187in}}{\pgfqpoint{0.625419in}{0.041248in}}%
\pgfpathcurveto{\pgfqpoint{0.614480in}{0.030309in}}{\pgfqpoint{0.608333in}{0.015470in}}{\pgfqpoint{0.608333in}{0.000000in}}%
\pgfpathcurveto{\pgfqpoint{0.608333in}{-0.015470in}}{\pgfqpoint{0.614480in}{-0.030309in}}{\pgfqpoint{0.625419in}{-0.041248in}}%
\pgfpathcurveto{\pgfqpoint{0.636358in}{-0.052187in}}{\pgfqpoint{0.651196in}{-0.058333in}}{\pgfqpoint{0.666667in}{-0.058333in}}%
\pgfpathclose%
\pgfpathmoveto{\pgfqpoint{0.666667in}{-0.052500in}}%
\pgfpathcurveto{\pgfqpoint{0.666667in}{-0.052500in}}{\pgfqpoint{0.652744in}{-0.052500in}}{\pgfqpoint{0.639389in}{-0.046968in}}%
\pgfpathcurveto{\pgfqpoint{0.629544in}{-0.037123in}}{\pgfqpoint{0.619698in}{-0.027278in}}{\pgfqpoint{0.614167in}{-0.013923in}}%
\pgfpathcurveto{\pgfqpoint{0.614167in}{0.000000in}}{\pgfqpoint{0.614167in}{0.013923in}}{\pgfqpoint{0.619698in}{0.027278in}}%
\pgfpathcurveto{\pgfqpoint{0.629544in}{0.037123in}}{\pgfqpoint{0.639389in}{0.046968in}}{\pgfqpoint{0.652744in}{0.052500in}}%
\pgfpathcurveto{\pgfqpoint{0.666667in}{0.052500in}}{\pgfqpoint{0.680590in}{0.052500in}}{\pgfqpoint{0.693945in}{0.046968in}}%
\pgfpathcurveto{\pgfqpoint{0.703790in}{0.037123in}}{\pgfqpoint{0.713635in}{0.027278in}}{\pgfqpoint{0.719167in}{0.013923in}}%
\pgfpathcurveto{\pgfqpoint{0.719167in}{0.000000in}}{\pgfqpoint{0.719167in}{-0.013923in}}{\pgfqpoint{0.713635in}{-0.027278in}}%
\pgfpathcurveto{\pgfqpoint{0.703790in}{-0.037123in}}{\pgfqpoint{0.693945in}{-0.046968in}}{\pgfqpoint{0.680590in}{-0.052500in}}%
\pgfpathclose%
\pgfpathmoveto{\pgfqpoint{0.833333in}{-0.058333in}}%
\pgfpathcurveto{\pgfqpoint{0.848804in}{-0.058333in}}{\pgfqpoint{0.863642in}{-0.052187in}}{\pgfqpoint{0.874581in}{-0.041248in}}%
\pgfpathcurveto{\pgfqpoint{0.885520in}{-0.030309in}}{\pgfqpoint{0.891667in}{-0.015470in}}{\pgfqpoint{0.891667in}{0.000000in}}%
\pgfpathcurveto{\pgfqpoint{0.891667in}{0.015470in}}{\pgfqpoint{0.885520in}{0.030309in}}{\pgfqpoint{0.874581in}{0.041248in}}%
\pgfpathcurveto{\pgfqpoint{0.863642in}{0.052187in}}{\pgfqpoint{0.848804in}{0.058333in}}{\pgfqpoint{0.833333in}{0.058333in}}%
\pgfpathcurveto{\pgfqpoint{0.817863in}{0.058333in}}{\pgfqpoint{0.803025in}{0.052187in}}{\pgfqpoint{0.792085in}{0.041248in}}%
\pgfpathcurveto{\pgfqpoint{0.781146in}{0.030309in}}{\pgfqpoint{0.775000in}{0.015470in}}{\pgfqpoint{0.775000in}{0.000000in}}%
\pgfpathcurveto{\pgfqpoint{0.775000in}{-0.015470in}}{\pgfqpoint{0.781146in}{-0.030309in}}{\pgfqpoint{0.792085in}{-0.041248in}}%
\pgfpathcurveto{\pgfqpoint{0.803025in}{-0.052187in}}{\pgfqpoint{0.817863in}{-0.058333in}}{\pgfqpoint{0.833333in}{-0.058333in}}%
\pgfpathclose%
\pgfpathmoveto{\pgfqpoint{0.833333in}{-0.052500in}}%
\pgfpathcurveto{\pgfqpoint{0.833333in}{-0.052500in}}{\pgfqpoint{0.819410in}{-0.052500in}}{\pgfqpoint{0.806055in}{-0.046968in}}%
\pgfpathcurveto{\pgfqpoint{0.796210in}{-0.037123in}}{\pgfqpoint{0.786365in}{-0.027278in}}{\pgfqpoint{0.780833in}{-0.013923in}}%
\pgfpathcurveto{\pgfqpoint{0.780833in}{0.000000in}}{\pgfqpoint{0.780833in}{0.013923in}}{\pgfqpoint{0.786365in}{0.027278in}}%
\pgfpathcurveto{\pgfqpoint{0.796210in}{0.037123in}}{\pgfqpoint{0.806055in}{0.046968in}}{\pgfqpoint{0.819410in}{0.052500in}}%
\pgfpathcurveto{\pgfqpoint{0.833333in}{0.052500in}}{\pgfqpoint{0.847256in}{0.052500in}}{\pgfqpoint{0.860611in}{0.046968in}}%
\pgfpathcurveto{\pgfqpoint{0.870456in}{0.037123in}}{\pgfqpoint{0.880302in}{0.027278in}}{\pgfqpoint{0.885833in}{0.013923in}}%
\pgfpathcurveto{\pgfqpoint{0.885833in}{0.000000in}}{\pgfqpoint{0.885833in}{-0.013923in}}{\pgfqpoint{0.880302in}{-0.027278in}}%
\pgfpathcurveto{\pgfqpoint{0.870456in}{-0.037123in}}{\pgfqpoint{0.860611in}{-0.046968in}}{\pgfqpoint{0.847256in}{-0.052500in}}%
\pgfpathclose%
\pgfpathmoveto{\pgfqpoint{1.000000in}{-0.058333in}}%
\pgfpathcurveto{\pgfqpoint{1.015470in}{-0.058333in}}{\pgfqpoint{1.030309in}{-0.052187in}}{\pgfqpoint{1.041248in}{-0.041248in}}%
\pgfpathcurveto{\pgfqpoint{1.052187in}{-0.030309in}}{\pgfqpoint{1.058333in}{-0.015470in}}{\pgfqpoint{1.058333in}{0.000000in}}%
\pgfpathcurveto{\pgfqpoint{1.058333in}{0.015470in}}{\pgfqpoint{1.052187in}{0.030309in}}{\pgfqpoint{1.041248in}{0.041248in}}%
\pgfpathcurveto{\pgfqpoint{1.030309in}{0.052187in}}{\pgfqpoint{1.015470in}{0.058333in}}{\pgfqpoint{1.000000in}{0.058333in}}%
\pgfpathcurveto{\pgfqpoint{0.984530in}{0.058333in}}{\pgfqpoint{0.969691in}{0.052187in}}{\pgfqpoint{0.958752in}{0.041248in}}%
\pgfpathcurveto{\pgfqpoint{0.947813in}{0.030309in}}{\pgfqpoint{0.941667in}{0.015470in}}{\pgfqpoint{0.941667in}{0.000000in}}%
\pgfpathcurveto{\pgfqpoint{0.941667in}{-0.015470in}}{\pgfqpoint{0.947813in}{-0.030309in}}{\pgfqpoint{0.958752in}{-0.041248in}}%
\pgfpathcurveto{\pgfqpoint{0.969691in}{-0.052187in}}{\pgfqpoint{0.984530in}{-0.058333in}}{\pgfqpoint{1.000000in}{-0.058333in}}%
\pgfpathclose%
\pgfpathmoveto{\pgfqpoint{1.000000in}{-0.052500in}}%
\pgfpathcurveto{\pgfqpoint{1.000000in}{-0.052500in}}{\pgfqpoint{0.986077in}{-0.052500in}}{\pgfqpoint{0.972722in}{-0.046968in}}%
\pgfpathcurveto{\pgfqpoint{0.962877in}{-0.037123in}}{\pgfqpoint{0.953032in}{-0.027278in}}{\pgfqpoint{0.947500in}{-0.013923in}}%
\pgfpathcurveto{\pgfqpoint{0.947500in}{0.000000in}}{\pgfqpoint{0.947500in}{0.013923in}}{\pgfqpoint{0.953032in}{0.027278in}}%
\pgfpathcurveto{\pgfqpoint{0.962877in}{0.037123in}}{\pgfqpoint{0.972722in}{0.046968in}}{\pgfqpoint{0.986077in}{0.052500in}}%
\pgfpathcurveto{\pgfqpoint{1.000000in}{0.052500in}}{\pgfqpoint{1.013923in}{0.052500in}}{\pgfqpoint{1.027278in}{0.046968in}}%
\pgfpathcurveto{\pgfqpoint{1.037123in}{0.037123in}}{\pgfqpoint{1.046968in}{0.027278in}}{\pgfqpoint{1.052500in}{0.013923in}}%
\pgfpathcurveto{\pgfqpoint{1.052500in}{0.000000in}}{\pgfqpoint{1.052500in}{-0.013923in}}{\pgfqpoint{1.046968in}{-0.027278in}}%
\pgfpathcurveto{\pgfqpoint{1.037123in}{-0.037123in}}{\pgfqpoint{1.027278in}{-0.046968in}}{\pgfqpoint{1.013923in}{-0.052500in}}%
\pgfpathclose%
\pgfpathmoveto{\pgfqpoint{0.083333in}{0.108333in}}%
\pgfpathcurveto{\pgfqpoint{0.098804in}{0.108333in}}{\pgfqpoint{0.113642in}{0.114480in}}{\pgfqpoint{0.124581in}{0.125419in}}%
\pgfpathcurveto{\pgfqpoint{0.135520in}{0.136358in}}{\pgfqpoint{0.141667in}{0.151196in}}{\pgfqpoint{0.141667in}{0.166667in}}%
\pgfpathcurveto{\pgfqpoint{0.141667in}{0.182137in}}{\pgfqpoint{0.135520in}{0.196975in}}{\pgfqpoint{0.124581in}{0.207915in}}%
\pgfpathcurveto{\pgfqpoint{0.113642in}{0.218854in}}{\pgfqpoint{0.098804in}{0.225000in}}{\pgfqpoint{0.083333in}{0.225000in}}%
\pgfpathcurveto{\pgfqpoint{0.067863in}{0.225000in}}{\pgfqpoint{0.053025in}{0.218854in}}{\pgfqpoint{0.042085in}{0.207915in}}%
\pgfpathcurveto{\pgfqpoint{0.031146in}{0.196975in}}{\pgfqpoint{0.025000in}{0.182137in}}{\pgfqpoint{0.025000in}{0.166667in}}%
\pgfpathcurveto{\pgfqpoint{0.025000in}{0.151196in}}{\pgfqpoint{0.031146in}{0.136358in}}{\pgfqpoint{0.042085in}{0.125419in}}%
\pgfpathcurveto{\pgfqpoint{0.053025in}{0.114480in}}{\pgfqpoint{0.067863in}{0.108333in}}{\pgfqpoint{0.083333in}{0.108333in}}%
\pgfpathclose%
\pgfpathmoveto{\pgfqpoint{0.083333in}{0.114167in}}%
\pgfpathcurveto{\pgfqpoint{0.083333in}{0.114167in}}{\pgfqpoint{0.069410in}{0.114167in}}{\pgfqpoint{0.056055in}{0.119698in}}%
\pgfpathcurveto{\pgfqpoint{0.046210in}{0.129544in}}{\pgfqpoint{0.036365in}{0.139389in}}{\pgfqpoint{0.030833in}{0.152744in}}%
\pgfpathcurveto{\pgfqpoint{0.030833in}{0.166667in}}{\pgfqpoint{0.030833in}{0.180590in}}{\pgfqpoint{0.036365in}{0.193945in}}%
\pgfpathcurveto{\pgfqpoint{0.046210in}{0.203790in}}{\pgfqpoint{0.056055in}{0.213635in}}{\pgfqpoint{0.069410in}{0.219167in}}%
\pgfpathcurveto{\pgfqpoint{0.083333in}{0.219167in}}{\pgfqpoint{0.097256in}{0.219167in}}{\pgfqpoint{0.110611in}{0.213635in}}%
\pgfpathcurveto{\pgfqpoint{0.120456in}{0.203790in}}{\pgfqpoint{0.130302in}{0.193945in}}{\pgfqpoint{0.135833in}{0.180590in}}%
\pgfpathcurveto{\pgfqpoint{0.135833in}{0.166667in}}{\pgfqpoint{0.135833in}{0.152744in}}{\pgfqpoint{0.130302in}{0.139389in}}%
\pgfpathcurveto{\pgfqpoint{0.120456in}{0.129544in}}{\pgfqpoint{0.110611in}{0.119698in}}{\pgfqpoint{0.097256in}{0.114167in}}%
\pgfpathclose%
\pgfpathmoveto{\pgfqpoint{0.250000in}{0.108333in}}%
\pgfpathcurveto{\pgfqpoint{0.265470in}{0.108333in}}{\pgfqpoint{0.280309in}{0.114480in}}{\pgfqpoint{0.291248in}{0.125419in}}%
\pgfpathcurveto{\pgfqpoint{0.302187in}{0.136358in}}{\pgfqpoint{0.308333in}{0.151196in}}{\pgfqpoint{0.308333in}{0.166667in}}%
\pgfpathcurveto{\pgfqpoint{0.308333in}{0.182137in}}{\pgfqpoint{0.302187in}{0.196975in}}{\pgfqpoint{0.291248in}{0.207915in}}%
\pgfpathcurveto{\pgfqpoint{0.280309in}{0.218854in}}{\pgfqpoint{0.265470in}{0.225000in}}{\pgfqpoint{0.250000in}{0.225000in}}%
\pgfpathcurveto{\pgfqpoint{0.234530in}{0.225000in}}{\pgfqpoint{0.219691in}{0.218854in}}{\pgfqpoint{0.208752in}{0.207915in}}%
\pgfpathcurveto{\pgfqpoint{0.197813in}{0.196975in}}{\pgfqpoint{0.191667in}{0.182137in}}{\pgfqpoint{0.191667in}{0.166667in}}%
\pgfpathcurveto{\pgfqpoint{0.191667in}{0.151196in}}{\pgfqpoint{0.197813in}{0.136358in}}{\pgfqpoint{0.208752in}{0.125419in}}%
\pgfpathcurveto{\pgfqpoint{0.219691in}{0.114480in}}{\pgfqpoint{0.234530in}{0.108333in}}{\pgfqpoint{0.250000in}{0.108333in}}%
\pgfpathclose%
\pgfpathmoveto{\pgfqpoint{0.250000in}{0.114167in}}%
\pgfpathcurveto{\pgfqpoint{0.250000in}{0.114167in}}{\pgfqpoint{0.236077in}{0.114167in}}{\pgfqpoint{0.222722in}{0.119698in}}%
\pgfpathcurveto{\pgfqpoint{0.212877in}{0.129544in}}{\pgfqpoint{0.203032in}{0.139389in}}{\pgfqpoint{0.197500in}{0.152744in}}%
\pgfpathcurveto{\pgfqpoint{0.197500in}{0.166667in}}{\pgfqpoint{0.197500in}{0.180590in}}{\pgfqpoint{0.203032in}{0.193945in}}%
\pgfpathcurveto{\pgfqpoint{0.212877in}{0.203790in}}{\pgfqpoint{0.222722in}{0.213635in}}{\pgfqpoint{0.236077in}{0.219167in}}%
\pgfpathcurveto{\pgfqpoint{0.250000in}{0.219167in}}{\pgfqpoint{0.263923in}{0.219167in}}{\pgfqpoint{0.277278in}{0.213635in}}%
\pgfpathcurveto{\pgfqpoint{0.287123in}{0.203790in}}{\pgfqpoint{0.296968in}{0.193945in}}{\pgfqpoint{0.302500in}{0.180590in}}%
\pgfpathcurveto{\pgfqpoint{0.302500in}{0.166667in}}{\pgfqpoint{0.302500in}{0.152744in}}{\pgfqpoint{0.296968in}{0.139389in}}%
\pgfpathcurveto{\pgfqpoint{0.287123in}{0.129544in}}{\pgfqpoint{0.277278in}{0.119698in}}{\pgfqpoint{0.263923in}{0.114167in}}%
\pgfpathclose%
\pgfpathmoveto{\pgfqpoint{0.416667in}{0.108333in}}%
\pgfpathcurveto{\pgfqpoint{0.432137in}{0.108333in}}{\pgfqpoint{0.446975in}{0.114480in}}{\pgfqpoint{0.457915in}{0.125419in}}%
\pgfpathcurveto{\pgfqpoint{0.468854in}{0.136358in}}{\pgfqpoint{0.475000in}{0.151196in}}{\pgfqpoint{0.475000in}{0.166667in}}%
\pgfpathcurveto{\pgfqpoint{0.475000in}{0.182137in}}{\pgfqpoint{0.468854in}{0.196975in}}{\pgfqpoint{0.457915in}{0.207915in}}%
\pgfpathcurveto{\pgfqpoint{0.446975in}{0.218854in}}{\pgfqpoint{0.432137in}{0.225000in}}{\pgfqpoint{0.416667in}{0.225000in}}%
\pgfpathcurveto{\pgfqpoint{0.401196in}{0.225000in}}{\pgfqpoint{0.386358in}{0.218854in}}{\pgfqpoint{0.375419in}{0.207915in}}%
\pgfpathcurveto{\pgfqpoint{0.364480in}{0.196975in}}{\pgfqpoint{0.358333in}{0.182137in}}{\pgfqpoint{0.358333in}{0.166667in}}%
\pgfpathcurveto{\pgfqpoint{0.358333in}{0.151196in}}{\pgfqpoint{0.364480in}{0.136358in}}{\pgfqpoint{0.375419in}{0.125419in}}%
\pgfpathcurveto{\pgfqpoint{0.386358in}{0.114480in}}{\pgfqpoint{0.401196in}{0.108333in}}{\pgfqpoint{0.416667in}{0.108333in}}%
\pgfpathclose%
\pgfpathmoveto{\pgfqpoint{0.416667in}{0.114167in}}%
\pgfpathcurveto{\pgfqpoint{0.416667in}{0.114167in}}{\pgfqpoint{0.402744in}{0.114167in}}{\pgfqpoint{0.389389in}{0.119698in}}%
\pgfpathcurveto{\pgfqpoint{0.379544in}{0.129544in}}{\pgfqpoint{0.369698in}{0.139389in}}{\pgfqpoint{0.364167in}{0.152744in}}%
\pgfpathcurveto{\pgfqpoint{0.364167in}{0.166667in}}{\pgfqpoint{0.364167in}{0.180590in}}{\pgfqpoint{0.369698in}{0.193945in}}%
\pgfpathcurveto{\pgfqpoint{0.379544in}{0.203790in}}{\pgfqpoint{0.389389in}{0.213635in}}{\pgfqpoint{0.402744in}{0.219167in}}%
\pgfpathcurveto{\pgfqpoint{0.416667in}{0.219167in}}{\pgfqpoint{0.430590in}{0.219167in}}{\pgfqpoint{0.443945in}{0.213635in}}%
\pgfpathcurveto{\pgfqpoint{0.453790in}{0.203790in}}{\pgfqpoint{0.463635in}{0.193945in}}{\pgfqpoint{0.469167in}{0.180590in}}%
\pgfpathcurveto{\pgfqpoint{0.469167in}{0.166667in}}{\pgfqpoint{0.469167in}{0.152744in}}{\pgfqpoint{0.463635in}{0.139389in}}%
\pgfpathcurveto{\pgfqpoint{0.453790in}{0.129544in}}{\pgfqpoint{0.443945in}{0.119698in}}{\pgfqpoint{0.430590in}{0.114167in}}%
\pgfpathclose%
\pgfpathmoveto{\pgfqpoint{0.583333in}{0.108333in}}%
\pgfpathcurveto{\pgfqpoint{0.598804in}{0.108333in}}{\pgfqpoint{0.613642in}{0.114480in}}{\pgfqpoint{0.624581in}{0.125419in}}%
\pgfpathcurveto{\pgfqpoint{0.635520in}{0.136358in}}{\pgfqpoint{0.641667in}{0.151196in}}{\pgfqpoint{0.641667in}{0.166667in}}%
\pgfpathcurveto{\pgfqpoint{0.641667in}{0.182137in}}{\pgfqpoint{0.635520in}{0.196975in}}{\pgfqpoint{0.624581in}{0.207915in}}%
\pgfpathcurveto{\pgfqpoint{0.613642in}{0.218854in}}{\pgfqpoint{0.598804in}{0.225000in}}{\pgfqpoint{0.583333in}{0.225000in}}%
\pgfpathcurveto{\pgfqpoint{0.567863in}{0.225000in}}{\pgfqpoint{0.553025in}{0.218854in}}{\pgfqpoint{0.542085in}{0.207915in}}%
\pgfpathcurveto{\pgfqpoint{0.531146in}{0.196975in}}{\pgfqpoint{0.525000in}{0.182137in}}{\pgfqpoint{0.525000in}{0.166667in}}%
\pgfpathcurveto{\pgfqpoint{0.525000in}{0.151196in}}{\pgfqpoint{0.531146in}{0.136358in}}{\pgfqpoint{0.542085in}{0.125419in}}%
\pgfpathcurveto{\pgfqpoint{0.553025in}{0.114480in}}{\pgfqpoint{0.567863in}{0.108333in}}{\pgfqpoint{0.583333in}{0.108333in}}%
\pgfpathclose%
\pgfpathmoveto{\pgfqpoint{0.583333in}{0.114167in}}%
\pgfpathcurveto{\pgfqpoint{0.583333in}{0.114167in}}{\pgfqpoint{0.569410in}{0.114167in}}{\pgfqpoint{0.556055in}{0.119698in}}%
\pgfpathcurveto{\pgfqpoint{0.546210in}{0.129544in}}{\pgfqpoint{0.536365in}{0.139389in}}{\pgfqpoint{0.530833in}{0.152744in}}%
\pgfpathcurveto{\pgfqpoint{0.530833in}{0.166667in}}{\pgfqpoint{0.530833in}{0.180590in}}{\pgfqpoint{0.536365in}{0.193945in}}%
\pgfpathcurveto{\pgfqpoint{0.546210in}{0.203790in}}{\pgfqpoint{0.556055in}{0.213635in}}{\pgfqpoint{0.569410in}{0.219167in}}%
\pgfpathcurveto{\pgfqpoint{0.583333in}{0.219167in}}{\pgfqpoint{0.597256in}{0.219167in}}{\pgfqpoint{0.610611in}{0.213635in}}%
\pgfpathcurveto{\pgfqpoint{0.620456in}{0.203790in}}{\pgfqpoint{0.630302in}{0.193945in}}{\pgfqpoint{0.635833in}{0.180590in}}%
\pgfpathcurveto{\pgfqpoint{0.635833in}{0.166667in}}{\pgfqpoint{0.635833in}{0.152744in}}{\pgfqpoint{0.630302in}{0.139389in}}%
\pgfpathcurveto{\pgfqpoint{0.620456in}{0.129544in}}{\pgfqpoint{0.610611in}{0.119698in}}{\pgfqpoint{0.597256in}{0.114167in}}%
\pgfpathclose%
\pgfpathmoveto{\pgfqpoint{0.750000in}{0.108333in}}%
\pgfpathcurveto{\pgfqpoint{0.765470in}{0.108333in}}{\pgfqpoint{0.780309in}{0.114480in}}{\pgfqpoint{0.791248in}{0.125419in}}%
\pgfpathcurveto{\pgfqpoint{0.802187in}{0.136358in}}{\pgfqpoint{0.808333in}{0.151196in}}{\pgfqpoint{0.808333in}{0.166667in}}%
\pgfpathcurveto{\pgfqpoint{0.808333in}{0.182137in}}{\pgfqpoint{0.802187in}{0.196975in}}{\pgfqpoint{0.791248in}{0.207915in}}%
\pgfpathcurveto{\pgfqpoint{0.780309in}{0.218854in}}{\pgfqpoint{0.765470in}{0.225000in}}{\pgfqpoint{0.750000in}{0.225000in}}%
\pgfpathcurveto{\pgfqpoint{0.734530in}{0.225000in}}{\pgfqpoint{0.719691in}{0.218854in}}{\pgfqpoint{0.708752in}{0.207915in}}%
\pgfpathcurveto{\pgfqpoint{0.697813in}{0.196975in}}{\pgfqpoint{0.691667in}{0.182137in}}{\pgfqpoint{0.691667in}{0.166667in}}%
\pgfpathcurveto{\pgfqpoint{0.691667in}{0.151196in}}{\pgfqpoint{0.697813in}{0.136358in}}{\pgfqpoint{0.708752in}{0.125419in}}%
\pgfpathcurveto{\pgfqpoint{0.719691in}{0.114480in}}{\pgfqpoint{0.734530in}{0.108333in}}{\pgfqpoint{0.750000in}{0.108333in}}%
\pgfpathclose%
\pgfpathmoveto{\pgfqpoint{0.750000in}{0.114167in}}%
\pgfpathcurveto{\pgfqpoint{0.750000in}{0.114167in}}{\pgfqpoint{0.736077in}{0.114167in}}{\pgfqpoint{0.722722in}{0.119698in}}%
\pgfpathcurveto{\pgfqpoint{0.712877in}{0.129544in}}{\pgfqpoint{0.703032in}{0.139389in}}{\pgfqpoint{0.697500in}{0.152744in}}%
\pgfpathcurveto{\pgfqpoint{0.697500in}{0.166667in}}{\pgfqpoint{0.697500in}{0.180590in}}{\pgfqpoint{0.703032in}{0.193945in}}%
\pgfpathcurveto{\pgfqpoint{0.712877in}{0.203790in}}{\pgfqpoint{0.722722in}{0.213635in}}{\pgfqpoint{0.736077in}{0.219167in}}%
\pgfpathcurveto{\pgfqpoint{0.750000in}{0.219167in}}{\pgfqpoint{0.763923in}{0.219167in}}{\pgfqpoint{0.777278in}{0.213635in}}%
\pgfpathcurveto{\pgfqpoint{0.787123in}{0.203790in}}{\pgfqpoint{0.796968in}{0.193945in}}{\pgfqpoint{0.802500in}{0.180590in}}%
\pgfpathcurveto{\pgfqpoint{0.802500in}{0.166667in}}{\pgfqpoint{0.802500in}{0.152744in}}{\pgfqpoint{0.796968in}{0.139389in}}%
\pgfpathcurveto{\pgfqpoint{0.787123in}{0.129544in}}{\pgfqpoint{0.777278in}{0.119698in}}{\pgfqpoint{0.763923in}{0.114167in}}%
\pgfpathclose%
\pgfpathmoveto{\pgfqpoint{0.916667in}{0.108333in}}%
\pgfpathcurveto{\pgfqpoint{0.932137in}{0.108333in}}{\pgfqpoint{0.946975in}{0.114480in}}{\pgfqpoint{0.957915in}{0.125419in}}%
\pgfpathcurveto{\pgfqpoint{0.968854in}{0.136358in}}{\pgfqpoint{0.975000in}{0.151196in}}{\pgfqpoint{0.975000in}{0.166667in}}%
\pgfpathcurveto{\pgfqpoint{0.975000in}{0.182137in}}{\pgfqpoint{0.968854in}{0.196975in}}{\pgfqpoint{0.957915in}{0.207915in}}%
\pgfpathcurveto{\pgfqpoint{0.946975in}{0.218854in}}{\pgfqpoint{0.932137in}{0.225000in}}{\pgfqpoint{0.916667in}{0.225000in}}%
\pgfpathcurveto{\pgfqpoint{0.901196in}{0.225000in}}{\pgfqpoint{0.886358in}{0.218854in}}{\pgfqpoint{0.875419in}{0.207915in}}%
\pgfpathcurveto{\pgfqpoint{0.864480in}{0.196975in}}{\pgfqpoint{0.858333in}{0.182137in}}{\pgfqpoint{0.858333in}{0.166667in}}%
\pgfpathcurveto{\pgfqpoint{0.858333in}{0.151196in}}{\pgfqpoint{0.864480in}{0.136358in}}{\pgfqpoint{0.875419in}{0.125419in}}%
\pgfpathcurveto{\pgfqpoint{0.886358in}{0.114480in}}{\pgfqpoint{0.901196in}{0.108333in}}{\pgfqpoint{0.916667in}{0.108333in}}%
\pgfpathclose%
\pgfpathmoveto{\pgfqpoint{0.916667in}{0.114167in}}%
\pgfpathcurveto{\pgfqpoint{0.916667in}{0.114167in}}{\pgfqpoint{0.902744in}{0.114167in}}{\pgfqpoint{0.889389in}{0.119698in}}%
\pgfpathcurveto{\pgfqpoint{0.879544in}{0.129544in}}{\pgfqpoint{0.869698in}{0.139389in}}{\pgfqpoint{0.864167in}{0.152744in}}%
\pgfpathcurveto{\pgfqpoint{0.864167in}{0.166667in}}{\pgfqpoint{0.864167in}{0.180590in}}{\pgfqpoint{0.869698in}{0.193945in}}%
\pgfpathcurveto{\pgfqpoint{0.879544in}{0.203790in}}{\pgfqpoint{0.889389in}{0.213635in}}{\pgfqpoint{0.902744in}{0.219167in}}%
\pgfpathcurveto{\pgfqpoint{0.916667in}{0.219167in}}{\pgfqpoint{0.930590in}{0.219167in}}{\pgfqpoint{0.943945in}{0.213635in}}%
\pgfpathcurveto{\pgfqpoint{0.953790in}{0.203790in}}{\pgfqpoint{0.963635in}{0.193945in}}{\pgfqpoint{0.969167in}{0.180590in}}%
\pgfpathcurveto{\pgfqpoint{0.969167in}{0.166667in}}{\pgfqpoint{0.969167in}{0.152744in}}{\pgfqpoint{0.963635in}{0.139389in}}%
\pgfpathcurveto{\pgfqpoint{0.953790in}{0.129544in}}{\pgfqpoint{0.943945in}{0.119698in}}{\pgfqpoint{0.930590in}{0.114167in}}%
\pgfpathclose%
\pgfpathmoveto{\pgfqpoint{0.000000in}{0.275000in}}%
\pgfpathcurveto{\pgfqpoint{0.015470in}{0.275000in}}{\pgfqpoint{0.030309in}{0.281146in}}{\pgfqpoint{0.041248in}{0.292085in}}%
\pgfpathcurveto{\pgfqpoint{0.052187in}{0.303025in}}{\pgfqpoint{0.058333in}{0.317863in}}{\pgfqpoint{0.058333in}{0.333333in}}%
\pgfpathcurveto{\pgfqpoint{0.058333in}{0.348804in}}{\pgfqpoint{0.052187in}{0.363642in}}{\pgfqpoint{0.041248in}{0.374581in}}%
\pgfpathcurveto{\pgfqpoint{0.030309in}{0.385520in}}{\pgfqpoint{0.015470in}{0.391667in}}{\pgfqpoint{0.000000in}{0.391667in}}%
\pgfpathcurveto{\pgfqpoint{-0.015470in}{0.391667in}}{\pgfqpoint{-0.030309in}{0.385520in}}{\pgfqpoint{-0.041248in}{0.374581in}}%
\pgfpathcurveto{\pgfqpoint{-0.052187in}{0.363642in}}{\pgfqpoint{-0.058333in}{0.348804in}}{\pgfqpoint{-0.058333in}{0.333333in}}%
\pgfpathcurveto{\pgfqpoint{-0.058333in}{0.317863in}}{\pgfqpoint{-0.052187in}{0.303025in}}{\pgfqpoint{-0.041248in}{0.292085in}}%
\pgfpathcurveto{\pgfqpoint{-0.030309in}{0.281146in}}{\pgfqpoint{-0.015470in}{0.275000in}}{\pgfqpoint{0.000000in}{0.275000in}}%
\pgfpathclose%
\pgfpathmoveto{\pgfqpoint{0.000000in}{0.280833in}}%
\pgfpathcurveto{\pgfqpoint{0.000000in}{0.280833in}}{\pgfqpoint{-0.013923in}{0.280833in}}{\pgfqpoint{-0.027278in}{0.286365in}}%
\pgfpathcurveto{\pgfqpoint{-0.037123in}{0.296210in}}{\pgfqpoint{-0.046968in}{0.306055in}}{\pgfqpoint{-0.052500in}{0.319410in}}%
\pgfpathcurveto{\pgfqpoint{-0.052500in}{0.333333in}}{\pgfqpoint{-0.052500in}{0.347256in}}{\pgfqpoint{-0.046968in}{0.360611in}}%
\pgfpathcurveto{\pgfqpoint{-0.037123in}{0.370456in}}{\pgfqpoint{-0.027278in}{0.380302in}}{\pgfqpoint{-0.013923in}{0.385833in}}%
\pgfpathcurveto{\pgfqpoint{0.000000in}{0.385833in}}{\pgfqpoint{0.013923in}{0.385833in}}{\pgfqpoint{0.027278in}{0.380302in}}%
\pgfpathcurveto{\pgfqpoint{0.037123in}{0.370456in}}{\pgfqpoint{0.046968in}{0.360611in}}{\pgfqpoint{0.052500in}{0.347256in}}%
\pgfpathcurveto{\pgfqpoint{0.052500in}{0.333333in}}{\pgfqpoint{0.052500in}{0.319410in}}{\pgfqpoint{0.046968in}{0.306055in}}%
\pgfpathcurveto{\pgfqpoint{0.037123in}{0.296210in}}{\pgfqpoint{0.027278in}{0.286365in}}{\pgfqpoint{0.013923in}{0.280833in}}%
\pgfpathclose%
\pgfpathmoveto{\pgfqpoint{0.166667in}{0.275000in}}%
\pgfpathcurveto{\pgfqpoint{0.182137in}{0.275000in}}{\pgfqpoint{0.196975in}{0.281146in}}{\pgfqpoint{0.207915in}{0.292085in}}%
\pgfpathcurveto{\pgfqpoint{0.218854in}{0.303025in}}{\pgfqpoint{0.225000in}{0.317863in}}{\pgfqpoint{0.225000in}{0.333333in}}%
\pgfpathcurveto{\pgfqpoint{0.225000in}{0.348804in}}{\pgfqpoint{0.218854in}{0.363642in}}{\pgfqpoint{0.207915in}{0.374581in}}%
\pgfpathcurveto{\pgfqpoint{0.196975in}{0.385520in}}{\pgfqpoint{0.182137in}{0.391667in}}{\pgfqpoint{0.166667in}{0.391667in}}%
\pgfpathcurveto{\pgfqpoint{0.151196in}{0.391667in}}{\pgfqpoint{0.136358in}{0.385520in}}{\pgfqpoint{0.125419in}{0.374581in}}%
\pgfpathcurveto{\pgfqpoint{0.114480in}{0.363642in}}{\pgfqpoint{0.108333in}{0.348804in}}{\pgfqpoint{0.108333in}{0.333333in}}%
\pgfpathcurveto{\pgfqpoint{0.108333in}{0.317863in}}{\pgfqpoint{0.114480in}{0.303025in}}{\pgfqpoint{0.125419in}{0.292085in}}%
\pgfpathcurveto{\pgfqpoint{0.136358in}{0.281146in}}{\pgfqpoint{0.151196in}{0.275000in}}{\pgfqpoint{0.166667in}{0.275000in}}%
\pgfpathclose%
\pgfpathmoveto{\pgfqpoint{0.166667in}{0.280833in}}%
\pgfpathcurveto{\pgfqpoint{0.166667in}{0.280833in}}{\pgfqpoint{0.152744in}{0.280833in}}{\pgfqpoint{0.139389in}{0.286365in}}%
\pgfpathcurveto{\pgfqpoint{0.129544in}{0.296210in}}{\pgfqpoint{0.119698in}{0.306055in}}{\pgfqpoint{0.114167in}{0.319410in}}%
\pgfpathcurveto{\pgfqpoint{0.114167in}{0.333333in}}{\pgfqpoint{0.114167in}{0.347256in}}{\pgfqpoint{0.119698in}{0.360611in}}%
\pgfpathcurveto{\pgfqpoint{0.129544in}{0.370456in}}{\pgfqpoint{0.139389in}{0.380302in}}{\pgfqpoint{0.152744in}{0.385833in}}%
\pgfpathcurveto{\pgfqpoint{0.166667in}{0.385833in}}{\pgfqpoint{0.180590in}{0.385833in}}{\pgfqpoint{0.193945in}{0.380302in}}%
\pgfpathcurveto{\pgfqpoint{0.203790in}{0.370456in}}{\pgfqpoint{0.213635in}{0.360611in}}{\pgfqpoint{0.219167in}{0.347256in}}%
\pgfpathcurveto{\pgfqpoint{0.219167in}{0.333333in}}{\pgfqpoint{0.219167in}{0.319410in}}{\pgfqpoint{0.213635in}{0.306055in}}%
\pgfpathcurveto{\pgfqpoint{0.203790in}{0.296210in}}{\pgfqpoint{0.193945in}{0.286365in}}{\pgfqpoint{0.180590in}{0.280833in}}%
\pgfpathclose%
\pgfpathmoveto{\pgfqpoint{0.333333in}{0.275000in}}%
\pgfpathcurveto{\pgfqpoint{0.348804in}{0.275000in}}{\pgfqpoint{0.363642in}{0.281146in}}{\pgfqpoint{0.374581in}{0.292085in}}%
\pgfpathcurveto{\pgfqpoint{0.385520in}{0.303025in}}{\pgfqpoint{0.391667in}{0.317863in}}{\pgfqpoint{0.391667in}{0.333333in}}%
\pgfpathcurveto{\pgfqpoint{0.391667in}{0.348804in}}{\pgfqpoint{0.385520in}{0.363642in}}{\pgfqpoint{0.374581in}{0.374581in}}%
\pgfpathcurveto{\pgfqpoint{0.363642in}{0.385520in}}{\pgfqpoint{0.348804in}{0.391667in}}{\pgfqpoint{0.333333in}{0.391667in}}%
\pgfpathcurveto{\pgfqpoint{0.317863in}{0.391667in}}{\pgfqpoint{0.303025in}{0.385520in}}{\pgfqpoint{0.292085in}{0.374581in}}%
\pgfpathcurveto{\pgfqpoint{0.281146in}{0.363642in}}{\pgfqpoint{0.275000in}{0.348804in}}{\pgfqpoint{0.275000in}{0.333333in}}%
\pgfpathcurveto{\pgfqpoint{0.275000in}{0.317863in}}{\pgfqpoint{0.281146in}{0.303025in}}{\pgfqpoint{0.292085in}{0.292085in}}%
\pgfpathcurveto{\pgfqpoint{0.303025in}{0.281146in}}{\pgfqpoint{0.317863in}{0.275000in}}{\pgfqpoint{0.333333in}{0.275000in}}%
\pgfpathclose%
\pgfpathmoveto{\pgfqpoint{0.333333in}{0.280833in}}%
\pgfpathcurveto{\pgfqpoint{0.333333in}{0.280833in}}{\pgfqpoint{0.319410in}{0.280833in}}{\pgfqpoint{0.306055in}{0.286365in}}%
\pgfpathcurveto{\pgfqpoint{0.296210in}{0.296210in}}{\pgfqpoint{0.286365in}{0.306055in}}{\pgfqpoint{0.280833in}{0.319410in}}%
\pgfpathcurveto{\pgfqpoint{0.280833in}{0.333333in}}{\pgfqpoint{0.280833in}{0.347256in}}{\pgfqpoint{0.286365in}{0.360611in}}%
\pgfpathcurveto{\pgfqpoint{0.296210in}{0.370456in}}{\pgfqpoint{0.306055in}{0.380302in}}{\pgfqpoint{0.319410in}{0.385833in}}%
\pgfpathcurveto{\pgfqpoint{0.333333in}{0.385833in}}{\pgfqpoint{0.347256in}{0.385833in}}{\pgfqpoint{0.360611in}{0.380302in}}%
\pgfpathcurveto{\pgfqpoint{0.370456in}{0.370456in}}{\pgfqpoint{0.380302in}{0.360611in}}{\pgfqpoint{0.385833in}{0.347256in}}%
\pgfpathcurveto{\pgfqpoint{0.385833in}{0.333333in}}{\pgfqpoint{0.385833in}{0.319410in}}{\pgfqpoint{0.380302in}{0.306055in}}%
\pgfpathcurveto{\pgfqpoint{0.370456in}{0.296210in}}{\pgfqpoint{0.360611in}{0.286365in}}{\pgfqpoint{0.347256in}{0.280833in}}%
\pgfpathclose%
\pgfpathmoveto{\pgfqpoint{0.500000in}{0.275000in}}%
\pgfpathcurveto{\pgfqpoint{0.515470in}{0.275000in}}{\pgfqpoint{0.530309in}{0.281146in}}{\pgfqpoint{0.541248in}{0.292085in}}%
\pgfpathcurveto{\pgfqpoint{0.552187in}{0.303025in}}{\pgfqpoint{0.558333in}{0.317863in}}{\pgfqpoint{0.558333in}{0.333333in}}%
\pgfpathcurveto{\pgfqpoint{0.558333in}{0.348804in}}{\pgfqpoint{0.552187in}{0.363642in}}{\pgfqpoint{0.541248in}{0.374581in}}%
\pgfpathcurveto{\pgfqpoint{0.530309in}{0.385520in}}{\pgfqpoint{0.515470in}{0.391667in}}{\pgfqpoint{0.500000in}{0.391667in}}%
\pgfpathcurveto{\pgfqpoint{0.484530in}{0.391667in}}{\pgfqpoint{0.469691in}{0.385520in}}{\pgfqpoint{0.458752in}{0.374581in}}%
\pgfpathcurveto{\pgfqpoint{0.447813in}{0.363642in}}{\pgfqpoint{0.441667in}{0.348804in}}{\pgfqpoint{0.441667in}{0.333333in}}%
\pgfpathcurveto{\pgfqpoint{0.441667in}{0.317863in}}{\pgfqpoint{0.447813in}{0.303025in}}{\pgfqpoint{0.458752in}{0.292085in}}%
\pgfpathcurveto{\pgfqpoint{0.469691in}{0.281146in}}{\pgfqpoint{0.484530in}{0.275000in}}{\pgfqpoint{0.500000in}{0.275000in}}%
\pgfpathclose%
\pgfpathmoveto{\pgfqpoint{0.500000in}{0.280833in}}%
\pgfpathcurveto{\pgfqpoint{0.500000in}{0.280833in}}{\pgfqpoint{0.486077in}{0.280833in}}{\pgfqpoint{0.472722in}{0.286365in}}%
\pgfpathcurveto{\pgfqpoint{0.462877in}{0.296210in}}{\pgfqpoint{0.453032in}{0.306055in}}{\pgfqpoint{0.447500in}{0.319410in}}%
\pgfpathcurveto{\pgfqpoint{0.447500in}{0.333333in}}{\pgfqpoint{0.447500in}{0.347256in}}{\pgfqpoint{0.453032in}{0.360611in}}%
\pgfpathcurveto{\pgfqpoint{0.462877in}{0.370456in}}{\pgfqpoint{0.472722in}{0.380302in}}{\pgfqpoint{0.486077in}{0.385833in}}%
\pgfpathcurveto{\pgfqpoint{0.500000in}{0.385833in}}{\pgfqpoint{0.513923in}{0.385833in}}{\pgfqpoint{0.527278in}{0.380302in}}%
\pgfpathcurveto{\pgfqpoint{0.537123in}{0.370456in}}{\pgfqpoint{0.546968in}{0.360611in}}{\pgfqpoint{0.552500in}{0.347256in}}%
\pgfpathcurveto{\pgfqpoint{0.552500in}{0.333333in}}{\pgfqpoint{0.552500in}{0.319410in}}{\pgfqpoint{0.546968in}{0.306055in}}%
\pgfpathcurveto{\pgfqpoint{0.537123in}{0.296210in}}{\pgfqpoint{0.527278in}{0.286365in}}{\pgfqpoint{0.513923in}{0.280833in}}%
\pgfpathclose%
\pgfpathmoveto{\pgfqpoint{0.666667in}{0.275000in}}%
\pgfpathcurveto{\pgfqpoint{0.682137in}{0.275000in}}{\pgfqpoint{0.696975in}{0.281146in}}{\pgfqpoint{0.707915in}{0.292085in}}%
\pgfpathcurveto{\pgfqpoint{0.718854in}{0.303025in}}{\pgfqpoint{0.725000in}{0.317863in}}{\pgfqpoint{0.725000in}{0.333333in}}%
\pgfpathcurveto{\pgfqpoint{0.725000in}{0.348804in}}{\pgfqpoint{0.718854in}{0.363642in}}{\pgfqpoint{0.707915in}{0.374581in}}%
\pgfpathcurveto{\pgfqpoint{0.696975in}{0.385520in}}{\pgfqpoint{0.682137in}{0.391667in}}{\pgfqpoint{0.666667in}{0.391667in}}%
\pgfpathcurveto{\pgfqpoint{0.651196in}{0.391667in}}{\pgfqpoint{0.636358in}{0.385520in}}{\pgfqpoint{0.625419in}{0.374581in}}%
\pgfpathcurveto{\pgfqpoint{0.614480in}{0.363642in}}{\pgfqpoint{0.608333in}{0.348804in}}{\pgfqpoint{0.608333in}{0.333333in}}%
\pgfpathcurveto{\pgfqpoint{0.608333in}{0.317863in}}{\pgfqpoint{0.614480in}{0.303025in}}{\pgfqpoint{0.625419in}{0.292085in}}%
\pgfpathcurveto{\pgfqpoint{0.636358in}{0.281146in}}{\pgfqpoint{0.651196in}{0.275000in}}{\pgfqpoint{0.666667in}{0.275000in}}%
\pgfpathclose%
\pgfpathmoveto{\pgfqpoint{0.666667in}{0.280833in}}%
\pgfpathcurveto{\pgfqpoint{0.666667in}{0.280833in}}{\pgfqpoint{0.652744in}{0.280833in}}{\pgfqpoint{0.639389in}{0.286365in}}%
\pgfpathcurveto{\pgfqpoint{0.629544in}{0.296210in}}{\pgfqpoint{0.619698in}{0.306055in}}{\pgfqpoint{0.614167in}{0.319410in}}%
\pgfpathcurveto{\pgfqpoint{0.614167in}{0.333333in}}{\pgfqpoint{0.614167in}{0.347256in}}{\pgfqpoint{0.619698in}{0.360611in}}%
\pgfpathcurveto{\pgfqpoint{0.629544in}{0.370456in}}{\pgfqpoint{0.639389in}{0.380302in}}{\pgfqpoint{0.652744in}{0.385833in}}%
\pgfpathcurveto{\pgfqpoint{0.666667in}{0.385833in}}{\pgfqpoint{0.680590in}{0.385833in}}{\pgfqpoint{0.693945in}{0.380302in}}%
\pgfpathcurveto{\pgfqpoint{0.703790in}{0.370456in}}{\pgfqpoint{0.713635in}{0.360611in}}{\pgfqpoint{0.719167in}{0.347256in}}%
\pgfpathcurveto{\pgfqpoint{0.719167in}{0.333333in}}{\pgfqpoint{0.719167in}{0.319410in}}{\pgfqpoint{0.713635in}{0.306055in}}%
\pgfpathcurveto{\pgfqpoint{0.703790in}{0.296210in}}{\pgfqpoint{0.693945in}{0.286365in}}{\pgfqpoint{0.680590in}{0.280833in}}%
\pgfpathclose%
\pgfpathmoveto{\pgfqpoint{0.833333in}{0.275000in}}%
\pgfpathcurveto{\pgfqpoint{0.848804in}{0.275000in}}{\pgfqpoint{0.863642in}{0.281146in}}{\pgfqpoint{0.874581in}{0.292085in}}%
\pgfpathcurveto{\pgfqpoint{0.885520in}{0.303025in}}{\pgfqpoint{0.891667in}{0.317863in}}{\pgfqpoint{0.891667in}{0.333333in}}%
\pgfpathcurveto{\pgfqpoint{0.891667in}{0.348804in}}{\pgfqpoint{0.885520in}{0.363642in}}{\pgfqpoint{0.874581in}{0.374581in}}%
\pgfpathcurveto{\pgfqpoint{0.863642in}{0.385520in}}{\pgfqpoint{0.848804in}{0.391667in}}{\pgfqpoint{0.833333in}{0.391667in}}%
\pgfpathcurveto{\pgfqpoint{0.817863in}{0.391667in}}{\pgfqpoint{0.803025in}{0.385520in}}{\pgfqpoint{0.792085in}{0.374581in}}%
\pgfpathcurveto{\pgfqpoint{0.781146in}{0.363642in}}{\pgfqpoint{0.775000in}{0.348804in}}{\pgfqpoint{0.775000in}{0.333333in}}%
\pgfpathcurveto{\pgfqpoint{0.775000in}{0.317863in}}{\pgfqpoint{0.781146in}{0.303025in}}{\pgfqpoint{0.792085in}{0.292085in}}%
\pgfpathcurveto{\pgfqpoint{0.803025in}{0.281146in}}{\pgfqpoint{0.817863in}{0.275000in}}{\pgfqpoint{0.833333in}{0.275000in}}%
\pgfpathclose%
\pgfpathmoveto{\pgfqpoint{0.833333in}{0.280833in}}%
\pgfpathcurveto{\pgfqpoint{0.833333in}{0.280833in}}{\pgfqpoint{0.819410in}{0.280833in}}{\pgfqpoint{0.806055in}{0.286365in}}%
\pgfpathcurveto{\pgfqpoint{0.796210in}{0.296210in}}{\pgfqpoint{0.786365in}{0.306055in}}{\pgfqpoint{0.780833in}{0.319410in}}%
\pgfpathcurveto{\pgfqpoint{0.780833in}{0.333333in}}{\pgfqpoint{0.780833in}{0.347256in}}{\pgfqpoint{0.786365in}{0.360611in}}%
\pgfpathcurveto{\pgfqpoint{0.796210in}{0.370456in}}{\pgfqpoint{0.806055in}{0.380302in}}{\pgfqpoint{0.819410in}{0.385833in}}%
\pgfpathcurveto{\pgfqpoint{0.833333in}{0.385833in}}{\pgfqpoint{0.847256in}{0.385833in}}{\pgfqpoint{0.860611in}{0.380302in}}%
\pgfpathcurveto{\pgfqpoint{0.870456in}{0.370456in}}{\pgfqpoint{0.880302in}{0.360611in}}{\pgfqpoint{0.885833in}{0.347256in}}%
\pgfpathcurveto{\pgfqpoint{0.885833in}{0.333333in}}{\pgfqpoint{0.885833in}{0.319410in}}{\pgfqpoint{0.880302in}{0.306055in}}%
\pgfpathcurveto{\pgfqpoint{0.870456in}{0.296210in}}{\pgfqpoint{0.860611in}{0.286365in}}{\pgfqpoint{0.847256in}{0.280833in}}%
\pgfpathclose%
\pgfpathmoveto{\pgfqpoint{1.000000in}{0.275000in}}%
\pgfpathcurveto{\pgfqpoint{1.015470in}{0.275000in}}{\pgfqpoint{1.030309in}{0.281146in}}{\pgfqpoint{1.041248in}{0.292085in}}%
\pgfpathcurveto{\pgfqpoint{1.052187in}{0.303025in}}{\pgfqpoint{1.058333in}{0.317863in}}{\pgfqpoint{1.058333in}{0.333333in}}%
\pgfpathcurveto{\pgfqpoint{1.058333in}{0.348804in}}{\pgfqpoint{1.052187in}{0.363642in}}{\pgfqpoint{1.041248in}{0.374581in}}%
\pgfpathcurveto{\pgfqpoint{1.030309in}{0.385520in}}{\pgfqpoint{1.015470in}{0.391667in}}{\pgfqpoint{1.000000in}{0.391667in}}%
\pgfpathcurveto{\pgfqpoint{0.984530in}{0.391667in}}{\pgfqpoint{0.969691in}{0.385520in}}{\pgfqpoint{0.958752in}{0.374581in}}%
\pgfpathcurveto{\pgfqpoint{0.947813in}{0.363642in}}{\pgfqpoint{0.941667in}{0.348804in}}{\pgfqpoint{0.941667in}{0.333333in}}%
\pgfpathcurveto{\pgfqpoint{0.941667in}{0.317863in}}{\pgfqpoint{0.947813in}{0.303025in}}{\pgfqpoint{0.958752in}{0.292085in}}%
\pgfpathcurveto{\pgfqpoint{0.969691in}{0.281146in}}{\pgfqpoint{0.984530in}{0.275000in}}{\pgfqpoint{1.000000in}{0.275000in}}%
\pgfpathclose%
\pgfpathmoveto{\pgfqpoint{1.000000in}{0.280833in}}%
\pgfpathcurveto{\pgfqpoint{1.000000in}{0.280833in}}{\pgfqpoint{0.986077in}{0.280833in}}{\pgfqpoint{0.972722in}{0.286365in}}%
\pgfpathcurveto{\pgfqpoint{0.962877in}{0.296210in}}{\pgfqpoint{0.953032in}{0.306055in}}{\pgfqpoint{0.947500in}{0.319410in}}%
\pgfpathcurveto{\pgfqpoint{0.947500in}{0.333333in}}{\pgfqpoint{0.947500in}{0.347256in}}{\pgfqpoint{0.953032in}{0.360611in}}%
\pgfpathcurveto{\pgfqpoint{0.962877in}{0.370456in}}{\pgfqpoint{0.972722in}{0.380302in}}{\pgfqpoint{0.986077in}{0.385833in}}%
\pgfpathcurveto{\pgfqpoint{1.000000in}{0.385833in}}{\pgfqpoint{1.013923in}{0.385833in}}{\pgfqpoint{1.027278in}{0.380302in}}%
\pgfpathcurveto{\pgfqpoint{1.037123in}{0.370456in}}{\pgfqpoint{1.046968in}{0.360611in}}{\pgfqpoint{1.052500in}{0.347256in}}%
\pgfpathcurveto{\pgfqpoint{1.052500in}{0.333333in}}{\pgfqpoint{1.052500in}{0.319410in}}{\pgfqpoint{1.046968in}{0.306055in}}%
\pgfpathcurveto{\pgfqpoint{1.037123in}{0.296210in}}{\pgfqpoint{1.027278in}{0.286365in}}{\pgfqpoint{1.013923in}{0.280833in}}%
\pgfpathclose%
\pgfpathmoveto{\pgfqpoint{0.083333in}{0.441667in}}%
\pgfpathcurveto{\pgfqpoint{0.098804in}{0.441667in}}{\pgfqpoint{0.113642in}{0.447813in}}{\pgfqpoint{0.124581in}{0.458752in}}%
\pgfpathcurveto{\pgfqpoint{0.135520in}{0.469691in}}{\pgfqpoint{0.141667in}{0.484530in}}{\pgfqpoint{0.141667in}{0.500000in}}%
\pgfpathcurveto{\pgfqpoint{0.141667in}{0.515470in}}{\pgfqpoint{0.135520in}{0.530309in}}{\pgfqpoint{0.124581in}{0.541248in}}%
\pgfpathcurveto{\pgfqpoint{0.113642in}{0.552187in}}{\pgfqpoint{0.098804in}{0.558333in}}{\pgfqpoint{0.083333in}{0.558333in}}%
\pgfpathcurveto{\pgfqpoint{0.067863in}{0.558333in}}{\pgfqpoint{0.053025in}{0.552187in}}{\pgfqpoint{0.042085in}{0.541248in}}%
\pgfpathcurveto{\pgfqpoint{0.031146in}{0.530309in}}{\pgfqpoint{0.025000in}{0.515470in}}{\pgfqpoint{0.025000in}{0.500000in}}%
\pgfpathcurveto{\pgfqpoint{0.025000in}{0.484530in}}{\pgfqpoint{0.031146in}{0.469691in}}{\pgfqpoint{0.042085in}{0.458752in}}%
\pgfpathcurveto{\pgfqpoint{0.053025in}{0.447813in}}{\pgfqpoint{0.067863in}{0.441667in}}{\pgfqpoint{0.083333in}{0.441667in}}%
\pgfpathclose%
\pgfpathmoveto{\pgfqpoint{0.083333in}{0.447500in}}%
\pgfpathcurveto{\pgfqpoint{0.083333in}{0.447500in}}{\pgfqpoint{0.069410in}{0.447500in}}{\pgfqpoint{0.056055in}{0.453032in}}%
\pgfpathcurveto{\pgfqpoint{0.046210in}{0.462877in}}{\pgfqpoint{0.036365in}{0.472722in}}{\pgfqpoint{0.030833in}{0.486077in}}%
\pgfpathcurveto{\pgfqpoint{0.030833in}{0.500000in}}{\pgfqpoint{0.030833in}{0.513923in}}{\pgfqpoint{0.036365in}{0.527278in}}%
\pgfpathcurveto{\pgfqpoint{0.046210in}{0.537123in}}{\pgfqpoint{0.056055in}{0.546968in}}{\pgfqpoint{0.069410in}{0.552500in}}%
\pgfpathcurveto{\pgfqpoint{0.083333in}{0.552500in}}{\pgfqpoint{0.097256in}{0.552500in}}{\pgfqpoint{0.110611in}{0.546968in}}%
\pgfpathcurveto{\pgfqpoint{0.120456in}{0.537123in}}{\pgfqpoint{0.130302in}{0.527278in}}{\pgfqpoint{0.135833in}{0.513923in}}%
\pgfpathcurveto{\pgfqpoint{0.135833in}{0.500000in}}{\pgfqpoint{0.135833in}{0.486077in}}{\pgfqpoint{0.130302in}{0.472722in}}%
\pgfpathcurveto{\pgfqpoint{0.120456in}{0.462877in}}{\pgfqpoint{0.110611in}{0.453032in}}{\pgfqpoint{0.097256in}{0.447500in}}%
\pgfpathclose%
\pgfpathmoveto{\pgfqpoint{0.250000in}{0.441667in}}%
\pgfpathcurveto{\pgfqpoint{0.265470in}{0.441667in}}{\pgfqpoint{0.280309in}{0.447813in}}{\pgfqpoint{0.291248in}{0.458752in}}%
\pgfpathcurveto{\pgfqpoint{0.302187in}{0.469691in}}{\pgfqpoint{0.308333in}{0.484530in}}{\pgfqpoint{0.308333in}{0.500000in}}%
\pgfpathcurveto{\pgfqpoint{0.308333in}{0.515470in}}{\pgfqpoint{0.302187in}{0.530309in}}{\pgfqpoint{0.291248in}{0.541248in}}%
\pgfpathcurveto{\pgfqpoint{0.280309in}{0.552187in}}{\pgfqpoint{0.265470in}{0.558333in}}{\pgfqpoint{0.250000in}{0.558333in}}%
\pgfpathcurveto{\pgfqpoint{0.234530in}{0.558333in}}{\pgfqpoint{0.219691in}{0.552187in}}{\pgfqpoint{0.208752in}{0.541248in}}%
\pgfpathcurveto{\pgfqpoint{0.197813in}{0.530309in}}{\pgfqpoint{0.191667in}{0.515470in}}{\pgfqpoint{0.191667in}{0.500000in}}%
\pgfpathcurveto{\pgfqpoint{0.191667in}{0.484530in}}{\pgfqpoint{0.197813in}{0.469691in}}{\pgfqpoint{0.208752in}{0.458752in}}%
\pgfpathcurveto{\pgfqpoint{0.219691in}{0.447813in}}{\pgfqpoint{0.234530in}{0.441667in}}{\pgfqpoint{0.250000in}{0.441667in}}%
\pgfpathclose%
\pgfpathmoveto{\pgfqpoint{0.250000in}{0.447500in}}%
\pgfpathcurveto{\pgfqpoint{0.250000in}{0.447500in}}{\pgfqpoint{0.236077in}{0.447500in}}{\pgfqpoint{0.222722in}{0.453032in}}%
\pgfpathcurveto{\pgfqpoint{0.212877in}{0.462877in}}{\pgfqpoint{0.203032in}{0.472722in}}{\pgfqpoint{0.197500in}{0.486077in}}%
\pgfpathcurveto{\pgfqpoint{0.197500in}{0.500000in}}{\pgfqpoint{0.197500in}{0.513923in}}{\pgfqpoint{0.203032in}{0.527278in}}%
\pgfpathcurveto{\pgfqpoint{0.212877in}{0.537123in}}{\pgfqpoint{0.222722in}{0.546968in}}{\pgfqpoint{0.236077in}{0.552500in}}%
\pgfpathcurveto{\pgfqpoint{0.250000in}{0.552500in}}{\pgfqpoint{0.263923in}{0.552500in}}{\pgfqpoint{0.277278in}{0.546968in}}%
\pgfpathcurveto{\pgfqpoint{0.287123in}{0.537123in}}{\pgfqpoint{0.296968in}{0.527278in}}{\pgfqpoint{0.302500in}{0.513923in}}%
\pgfpathcurveto{\pgfqpoint{0.302500in}{0.500000in}}{\pgfqpoint{0.302500in}{0.486077in}}{\pgfqpoint{0.296968in}{0.472722in}}%
\pgfpathcurveto{\pgfqpoint{0.287123in}{0.462877in}}{\pgfqpoint{0.277278in}{0.453032in}}{\pgfqpoint{0.263923in}{0.447500in}}%
\pgfpathclose%
\pgfpathmoveto{\pgfqpoint{0.416667in}{0.441667in}}%
\pgfpathcurveto{\pgfqpoint{0.432137in}{0.441667in}}{\pgfqpoint{0.446975in}{0.447813in}}{\pgfqpoint{0.457915in}{0.458752in}}%
\pgfpathcurveto{\pgfqpoint{0.468854in}{0.469691in}}{\pgfqpoint{0.475000in}{0.484530in}}{\pgfqpoint{0.475000in}{0.500000in}}%
\pgfpathcurveto{\pgfqpoint{0.475000in}{0.515470in}}{\pgfqpoint{0.468854in}{0.530309in}}{\pgfqpoint{0.457915in}{0.541248in}}%
\pgfpathcurveto{\pgfqpoint{0.446975in}{0.552187in}}{\pgfqpoint{0.432137in}{0.558333in}}{\pgfqpoint{0.416667in}{0.558333in}}%
\pgfpathcurveto{\pgfqpoint{0.401196in}{0.558333in}}{\pgfqpoint{0.386358in}{0.552187in}}{\pgfqpoint{0.375419in}{0.541248in}}%
\pgfpathcurveto{\pgfqpoint{0.364480in}{0.530309in}}{\pgfqpoint{0.358333in}{0.515470in}}{\pgfqpoint{0.358333in}{0.500000in}}%
\pgfpathcurveto{\pgfqpoint{0.358333in}{0.484530in}}{\pgfqpoint{0.364480in}{0.469691in}}{\pgfqpoint{0.375419in}{0.458752in}}%
\pgfpathcurveto{\pgfqpoint{0.386358in}{0.447813in}}{\pgfqpoint{0.401196in}{0.441667in}}{\pgfqpoint{0.416667in}{0.441667in}}%
\pgfpathclose%
\pgfpathmoveto{\pgfqpoint{0.416667in}{0.447500in}}%
\pgfpathcurveto{\pgfqpoint{0.416667in}{0.447500in}}{\pgfqpoint{0.402744in}{0.447500in}}{\pgfqpoint{0.389389in}{0.453032in}}%
\pgfpathcurveto{\pgfqpoint{0.379544in}{0.462877in}}{\pgfqpoint{0.369698in}{0.472722in}}{\pgfqpoint{0.364167in}{0.486077in}}%
\pgfpathcurveto{\pgfqpoint{0.364167in}{0.500000in}}{\pgfqpoint{0.364167in}{0.513923in}}{\pgfqpoint{0.369698in}{0.527278in}}%
\pgfpathcurveto{\pgfqpoint{0.379544in}{0.537123in}}{\pgfqpoint{0.389389in}{0.546968in}}{\pgfqpoint{0.402744in}{0.552500in}}%
\pgfpathcurveto{\pgfqpoint{0.416667in}{0.552500in}}{\pgfqpoint{0.430590in}{0.552500in}}{\pgfqpoint{0.443945in}{0.546968in}}%
\pgfpathcurveto{\pgfqpoint{0.453790in}{0.537123in}}{\pgfqpoint{0.463635in}{0.527278in}}{\pgfqpoint{0.469167in}{0.513923in}}%
\pgfpathcurveto{\pgfqpoint{0.469167in}{0.500000in}}{\pgfqpoint{0.469167in}{0.486077in}}{\pgfqpoint{0.463635in}{0.472722in}}%
\pgfpathcurveto{\pgfqpoint{0.453790in}{0.462877in}}{\pgfqpoint{0.443945in}{0.453032in}}{\pgfqpoint{0.430590in}{0.447500in}}%
\pgfpathclose%
\pgfpathmoveto{\pgfqpoint{0.583333in}{0.441667in}}%
\pgfpathcurveto{\pgfqpoint{0.598804in}{0.441667in}}{\pgfqpoint{0.613642in}{0.447813in}}{\pgfqpoint{0.624581in}{0.458752in}}%
\pgfpathcurveto{\pgfqpoint{0.635520in}{0.469691in}}{\pgfqpoint{0.641667in}{0.484530in}}{\pgfqpoint{0.641667in}{0.500000in}}%
\pgfpathcurveto{\pgfqpoint{0.641667in}{0.515470in}}{\pgfqpoint{0.635520in}{0.530309in}}{\pgfqpoint{0.624581in}{0.541248in}}%
\pgfpathcurveto{\pgfqpoint{0.613642in}{0.552187in}}{\pgfqpoint{0.598804in}{0.558333in}}{\pgfqpoint{0.583333in}{0.558333in}}%
\pgfpathcurveto{\pgfqpoint{0.567863in}{0.558333in}}{\pgfqpoint{0.553025in}{0.552187in}}{\pgfqpoint{0.542085in}{0.541248in}}%
\pgfpathcurveto{\pgfqpoint{0.531146in}{0.530309in}}{\pgfqpoint{0.525000in}{0.515470in}}{\pgfqpoint{0.525000in}{0.500000in}}%
\pgfpathcurveto{\pgfqpoint{0.525000in}{0.484530in}}{\pgfqpoint{0.531146in}{0.469691in}}{\pgfqpoint{0.542085in}{0.458752in}}%
\pgfpathcurveto{\pgfqpoint{0.553025in}{0.447813in}}{\pgfqpoint{0.567863in}{0.441667in}}{\pgfqpoint{0.583333in}{0.441667in}}%
\pgfpathclose%
\pgfpathmoveto{\pgfqpoint{0.583333in}{0.447500in}}%
\pgfpathcurveto{\pgfqpoint{0.583333in}{0.447500in}}{\pgfqpoint{0.569410in}{0.447500in}}{\pgfqpoint{0.556055in}{0.453032in}}%
\pgfpathcurveto{\pgfqpoint{0.546210in}{0.462877in}}{\pgfqpoint{0.536365in}{0.472722in}}{\pgfqpoint{0.530833in}{0.486077in}}%
\pgfpathcurveto{\pgfqpoint{0.530833in}{0.500000in}}{\pgfqpoint{0.530833in}{0.513923in}}{\pgfqpoint{0.536365in}{0.527278in}}%
\pgfpathcurveto{\pgfqpoint{0.546210in}{0.537123in}}{\pgfqpoint{0.556055in}{0.546968in}}{\pgfqpoint{0.569410in}{0.552500in}}%
\pgfpathcurveto{\pgfqpoint{0.583333in}{0.552500in}}{\pgfqpoint{0.597256in}{0.552500in}}{\pgfqpoint{0.610611in}{0.546968in}}%
\pgfpathcurveto{\pgfqpoint{0.620456in}{0.537123in}}{\pgfqpoint{0.630302in}{0.527278in}}{\pgfqpoint{0.635833in}{0.513923in}}%
\pgfpathcurveto{\pgfqpoint{0.635833in}{0.500000in}}{\pgfqpoint{0.635833in}{0.486077in}}{\pgfqpoint{0.630302in}{0.472722in}}%
\pgfpathcurveto{\pgfqpoint{0.620456in}{0.462877in}}{\pgfqpoint{0.610611in}{0.453032in}}{\pgfqpoint{0.597256in}{0.447500in}}%
\pgfpathclose%
\pgfpathmoveto{\pgfqpoint{0.750000in}{0.441667in}}%
\pgfpathcurveto{\pgfqpoint{0.765470in}{0.441667in}}{\pgfqpoint{0.780309in}{0.447813in}}{\pgfqpoint{0.791248in}{0.458752in}}%
\pgfpathcurveto{\pgfqpoint{0.802187in}{0.469691in}}{\pgfqpoint{0.808333in}{0.484530in}}{\pgfqpoint{0.808333in}{0.500000in}}%
\pgfpathcurveto{\pgfqpoint{0.808333in}{0.515470in}}{\pgfqpoint{0.802187in}{0.530309in}}{\pgfqpoint{0.791248in}{0.541248in}}%
\pgfpathcurveto{\pgfqpoint{0.780309in}{0.552187in}}{\pgfqpoint{0.765470in}{0.558333in}}{\pgfqpoint{0.750000in}{0.558333in}}%
\pgfpathcurveto{\pgfqpoint{0.734530in}{0.558333in}}{\pgfqpoint{0.719691in}{0.552187in}}{\pgfqpoint{0.708752in}{0.541248in}}%
\pgfpathcurveto{\pgfqpoint{0.697813in}{0.530309in}}{\pgfqpoint{0.691667in}{0.515470in}}{\pgfqpoint{0.691667in}{0.500000in}}%
\pgfpathcurveto{\pgfqpoint{0.691667in}{0.484530in}}{\pgfqpoint{0.697813in}{0.469691in}}{\pgfqpoint{0.708752in}{0.458752in}}%
\pgfpathcurveto{\pgfqpoint{0.719691in}{0.447813in}}{\pgfqpoint{0.734530in}{0.441667in}}{\pgfqpoint{0.750000in}{0.441667in}}%
\pgfpathclose%
\pgfpathmoveto{\pgfqpoint{0.750000in}{0.447500in}}%
\pgfpathcurveto{\pgfqpoint{0.750000in}{0.447500in}}{\pgfqpoint{0.736077in}{0.447500in}}{\pgfqpoint{0.722722in}{0.453032in}}%
\pgfpathcurveto{\pgfqpoint{0.712877in}{0.462877in}}{\pgfqpoint{0.703032in}{0.472722in}}{\pgfqpoint{0.697500in}{0.486077in}}%
\pgfpathcurveto{\pgfqpoint{0.697500in}{0.500000in}}{\pgfqpoint{0.697500in}{0.513923in}}{\pgfqpoint{0.703032in}{0.527278in}}%
\pgfpathcurveto{\pgfqpoint{0.712877in}{0.537123in}}{\pgfqpoint{0.722722in}{0.546968in}}{\pgfqpoint{0.736077in}{0.552500in}}%
\pgfpathcurveto{\pgfqpoint{0.750000in}{0.552500in}}{\pgfqpoint{0.763923in}{0.552500in}}{\pgfqpoint{0.777278in}{0.546968in}}%
\pgfpathcurveto{\pgfqpoint{0.787123in}{0.537123in}}{\pgfqpoint{0.796968in}{0.527278in}}{\pgfqpoint{0.802500in}{0.513923in}}%
\pgfpathcurveto{\pgfqpoint{0.802500in}{0.500000in}}{\pgfqpoint{0.802500in}{0.486077in}}{\pgfqpoint{0.796968in}{0.472722in}}%
\pgfpathcurveto{\pgfqpoint{0.787123in}{0.462877in}}{\pgfqpoint{0.777278in}{0.453032in}}{\pgfqpoint{0.763923in}{0.447500in}}%
\pgfpathclose%
\pgfpathmoveto{\pgfqpoint{0.916667in}{0.441667in}}%
\pgfpathcurveto{\pgfqpoint{0.932137in}{0.441667in}}{\pgfqpoint{0.946975in}{0.447813in}}{\pgfqpoint{0.957915in}{0.458752in}}%
\pgfpathcurveto{\pgfqpoint{0.968854in}{0.469691in}}{\pgfqpoint{0.975000in}{0.484530in}}{\pgfqpoint{0.975000in}{0.500000in}}%
\pgfpathcurveto{\pgfqpoint{0.975000in}{0.515470in}}{\pgfqpoint{0.968854in}{0.530309in}}{\pgfqpoint{0.957915in}{0.541248in}}%
\pgfpathcurveto{\pgfqpoint{0.946975in}{0.552187in}}{\pgfqpoint{0.932137in}{0.558333in}}{\pgfqpoint{0.916667in}{0.558333in}}%
\pgfpathcurveto{\pgfqpoint{0.901196in}{0.558333in}}{\pgfqpoint{0.886358in}{0.552187in}}{\pgfqpoint{0.875419in}{0.541248in}}%
\pgfpathcurveto{\pgfqpoint{0.864480in}{0.530309in}}{\pgfqpoint{0.858333in}{0.515470in}}{\pgfqpoint{0.858333in}{0.500000in}}%
\pgfpathcurveto{\pgfqpoint{0.858333in}{0.484530in}}{\pgfqpoint{0.864480in}{0.469691in}}{\pgfqpoint{0.875419in}{0.458752in}}%
\pgfpathcurveto{\pgfqpoint{0.886358in}{0.447813in}}{\pgfqpoint{0.901196in}{0.441667in}}{\pgfqpoint{0.916667in}{0.441667in}}%
\pgfpathclose%
\pgfpathmoveto{\pgfqpoint{0.916667in}{0.447500in}}%
\pgfpathcurveto{\pgfqpoint{0.916667in}{0.447500in}}{\pgfqpoint{0.902744in}{0.447500in}}{\pgfqpoint{0.889389in}{0.453032in}}%
\pgfpathcurveto{\pgfqpoint{0.879544in}{0.462877in}}{\pgfqpoint{0.869698in}{0.472722in}}{\pgfqpoint{0.864167in}{0.486077in}}%
\pgfpathcurveto{\pgfqpoint{0.864167in}{0.500000in}}{\pgfqpoint{0.864167in}{0.513923in}}{\pgfqpoint{0.869698in}{0.527278in}}%
\pgfpathcurveto{\pgfqpoint{0.879544in}{0.537123in}}{\pgfqpoint{0.889389in}{0.546968in}}{\pgfqpoint{0.902744in}{0.552500in}}%
\pgfpathcurveto{\pgfqpoint{0.916667in}{0.552500in}}{\pgfqpoint{0.930590in}{0.552500in}}{\pgfqpoint{0.943945in}{0.546968in}}%
\pgfpathcurveto{\pgfqpoint{0.953790in}{0.537123in}}{\pgfqpoint{0.963635in}{0.527278in}}{\pgfqpoint{0.969167in}{0.513923in}}%
\pgfpathcurveto{\pgfqpoint{0.969167in}{0.500000in}}{\pgfqpoint{0.969167in}{0.486077in}}{\pgfqpoint{0.963635in}{0.472722in}}%
\pgfpathcurveto{\pgfqpoint{0.953790in}{0.462877in}}{\pgfqpoint{0.943945in}{0.453032in}}{\pgfqpoint{0.930590in}{0.447500in}}%
\pgfpathclose%
\pgfpathmoveto{\pgfqpoint{0.000000in}{0.608333in}}%
\pgfpathcurveto{\pgfqpoint{0.015470in}{0.608333in}}{\pgfqpoint{0.030309in}{0.614480in}}{\pgfqpoint{0.041248in}{0.625419in}}%
\pgfpathcurveto{\pgfqpoint{0.052187in}{0.636358in}}{\pgfqpoint{0.058333in}{0.651196in}}{\pgfqpoint{0.058333in}{0.666667in}}%
\pgfpathcurveto{\pgfqpoint{0.058333in}{0.682137in}}{\pgfqpoint{0.052187in}{0.696975in}}{\pgfqpoint{0.041248in}{0.707915in}}%
\pgfpathcurveto{\pgfqpoint{0.030309in}{0.718854in}}{\pgfqpoint{0.015470in}{0.725000in}}{\pgfqpoint{0.000000in}{0.725000in}}%
\pgfpathcurveto{\pgfqpoint{-0.015470in}{0.725000in}}{\pgfqpoint{-0.030309in}{0.718854in}}{\pgfqpoint{-0.041248in}{0.707915in}}%
\pgfpathcurveto{\pgfqpoint{-0.052187in}{0.696975in}}{\pgfqpoint{-0.058333in}{0.682137in}}{\pgfqpoint{-0.058333in}{0.666667in}}%
\pgfpathcurveto{\pgfqpoint{-0.058333in}{0.651196in}}{\pgfqpoint{-0.052187in}{0.636358in}}{\pgfqpoint{-0.041248in}{0.625419in}}%
\pgfpathcurveto{\pgfqpoint{-0.030309in}{0.614480in}}{\pgfqpoint{-0.015470in}{0.608333in}}{\pgfqpoint{0.000000in}{0.608333in}}%
\pgfpathclose%
\pgfpathmoveto{\pgfqpoint{0.000000in}{0.614167in}}%
\pgfpathcurveto{\pgfqpoint{0.000000in}{0.614167in}}{\pgfqpoint{-0.013923in}{0.614167in}}{\pgfqpoint{-0.027278in}{0.619698in}}%
\pgfpathcurveto{\pgfqpoint{-0.037123in}{0.629544in}}{\pgfqpoint{-0.046968in}{0.639389in}}{\pgfqpoint{-0.052500in}{0.652744in}}%
\pgfpathcurveto{\pgfqpoint{-0.052500in}{0.666667in}}{\pgfqpoint{-0.052500in}{0.680590in}}{\pgfqpoint{-0.046968in}{0.693945in}}%
\pgfpathcurveto{\pgfqpoint{-0.037123in}{0.703790in}}{\pgfqpoint{-0.027278in}{0.713635in}}{\pgfqpoint{-0.013923in}{0.719167in}}%
\pgfpathcurveto{\pgfqpoint{0.000000in}{0.719167in}}{\pgfqpoint{0.013923in}{0.719167in}}{\pgfqpoint{0.027278in}{0.713635in}}%
\pgfpathcurveto{\pgfqpoint{0.037123in}{0.703790in}}{\pgfqpoint{0.046968in}{0.693945in}}{\pgfqpoint{0.052500in}{0.680590in}}%
\pgfpathcurveto{\pgfqpoint{0.052500in}{0.666667in}}{\pgfqpoint{0.052500in}{0.652744in}}{\pgfqpoint{0.046968in}{0.639389in}}%
\pgfpathcurveto{\pgfqpoint{0.037123in}{0.629544in}}{\pgfqpoint{0.027278in}{0.619698in}}{\pgfqpoint{0.013923in}{0.614167in}}%
\pgfpathclose%
\pgfpathmoveto{\pgfqpoint{0.166667in}{0.608333in}}%
\pgfpathcurveto{\pgfqpoint{0.182137in}{0.608333in}}{\pgfqpoint{0.196975in}{0.614480in}}{\pgfqpoint{0.207915in}{0.625419in}}%
\pgfpathcurveto{\pgfqpoint{0.218854in}{0.636358in}}{\pgfqpoint{0.225000in}{0.651196in}}{\pgfqpoint{0.225000in}{0.666667in}}%
\pgfpathcurveto{\pgfqpoint{0.225000in}{0.682137in}}{\pgfqpoint{0.218854in}{0.696975in}}{\pgfqpoint{0.207915in}{0.707915in}}%
\pgfpathcurveto{\pgfqpoint{0.196975in}{0.718854in}}{\pgfqpoint{0.182137in}{0.725000in}}{\pgfqpoint{0.166667in}{0.725000in}}%
\pgfpathcurveto{\pgfqpoint{0.151196in}{0.725000in}}{\pgfqpoint{0.136358in}{0.718854in}}{\pgfqpoint{0.125419in}{0.707915in}}%
\pgfpathcurveto{\pgfqpoint{0.114480in}{0.696975in}}{\pgfqpoint{0.108333in}{0.682137in}}{\pgfqpoint{0.108333in}{0.666667in}}%
\pgfpathcurveto{\pgfqpoint{0.108333in}{0.651196in}}{\pgfqpoint{0.114480in}{0.636358in}}{\pgfqpoint{0.125419in}{0.625419in}}%
\pgfpathcurveto{\pgfqpoint{0.136358in}{0.614480in}}{\pgfqpoint{0.151196in}{0.608333in}}{\pgfqpoint{0.166667in}{0.608333in}}%
\pgfpathclose%
\pgfpathmoveto{\pgfqpoint{0.166667in}{0.614167in}}%
\pgfpathcurveto{\pgfqpoint{0.166667in}{0.614167in}}{\pgfqpoint{0.152744in}{0.614167in}}{\pgfqpoint{0.139389in}{0.619698in}}%
\pgfpathcurveto{\pgfqpoint{0.129544in}{0.629544in}}{\pgfqpoint{0.119698in}{0.639389in}}{\pgfqpoint{0.114167in}{0.652744in}}%
\pgfpathcurveto{\pgfqpoint{0.114167in}{0.666667in}}{\pgfqpoint{0.114167in}{0.680590in}}{\pgfqpoint{0.119698in}{0.693945in}}%
\pgfpathcurveto{\pgfqpoint{0.129544in}{0.703790in}}{\pgfqpoint{0.139389in}{0.713635in}}{\pgfqpoint{0.152744in}{0.719167in}}%
\pgfpathcurveto{\pgfqpoint{0.166667in}{0.719167in}}{\pgfqpoint{0.180590in}{0.719167in}}{\pgfqpoint{0.193945in}{0.713635in}}%
\pgfpathcurveto{\pgfqpoint{0.203790in}{0.703790in}}{\pgfqpoint{0.213635in}{0.693945in}}{\pgfqpoint{0.219167in}{0.680590in}}%
\pgfpathcurveto{\pgfqpoint{0.219167in}{0.666667in}}{\pgfqpoint{0.219167in}{0.652744in}}{\pgfqpoint{0.213635in}{0.639389in}}%
\pgfpathcurveto{\pgfqpoint{0.203790in}{0.629544in}}{\pgfqpoint{0.193945in}{0.619698in}}{\pgfqpoint{0.180590in}{0.614167in}}%
\pgfpathclose%
\pgfpathmoveto{\pgfqpoint{0.333333in}{0.608333in}}%
\pgfpathcurveto{\pgfqpoint{0.348804in}{0.608333in}}{\pgfqpoint{0.363642in}{0.614480in}}{\pgfqpoint{0.374581in}{0.625419in}}%
\pgfpathcurveto{\pgfqpoint{0.385520in}{0.636358in}}{\pgfqpoint{0.391667in}{0.651196in}}{\pgfqpoint{0.391667in}{0.666667in}}%
\pgfpathcurveto{\pgfqpoint{0.391667in}{0.682137in}}{\pgfqpoint{0.385520in}{0.696975in}}{\pgfqpoint{0.374581in}{0.707915in}}%
\pgfpathcurveto{\pgfqpoint{0.363642in}{0.718854in}}{\pgfqpoint{0.348804in}{0.725000in}}{\pgfqpoint{0.333333in}{0.725000in}}%
\pgfpathcurveto{\pgfqpoint{0.317863in}{0.725000in}}{\pgfqpoint{0.303025in}{0.718854in}}{\pgfqpoint{0.292085in}{0.707915in}}%
\pgfpathcurveto{\pgfqpoint{0.281146in}{0.696975in}}{\pgfqpoint{0.275000in}{0.682137in}}{\pgfqpoint{0.275000in}{0.666667in}}%
\pgfpathcurveto{\pgfqpoint{0.275000in}{0.651196in}}{\pgfqpoint{0.281146in}{0.636358in}}{\pgfqpoint{0.292085in}{0.625419in}}%
\pgfpathcurveto{\pgfqpoint{0.303025in}{0.614480in}}{\pgfqpoint{0.317863in}{0.608333in}}{\pgfqpoint{0.333333in}{0.608333in}}%
\pgfpathclose%
\pgfpathmoveto{\pgfqpoint{0.333333in}{0.614167in}}%
\pgfpathcurveto{\pgfqpoint{0.333333in}{0.614167in}}{\pgfqpoint{0.319410in}{0.614167in}}{\pgfqpoint{0.306055in}{0.619698in}}%
\pgfpathcurveto{\pgfqpoint{0.296210in}{0.629544in}}{\pgfqpoint{0.286365in}{0.639389in}}{\pgfqpoint{0.280833in}{0.652744in}}%
\pgfpathcurveto{\pgfqpoint{0.280833in}{0.666667in}}{\pgfqpoint{0.280833in}{0.680590in}}{\pgfqpoint{0.286365in}{0.693945in}}%
\pgfpathcurveto{\pgfqpoint{0.296210in}{0.703790in}}{\pgfqpoint{0.306055in}{0.713635in}}{\pgfqpoint{0.319410in}{0.719167in}}%
\pgfpathcurveto{\pgfqpoint{0.333333in}{0.719167in}}{\pgfqpoint{0.347256in}{0.719167in}}{\pgfqpoint{0.360611in}{0.713635in}}%
\pgfpathcurveto{\pgfqpoint{0.370456in}{0.703790in}}{\pgfqpoint{0.380302in}{0.693945in}}{\pgfqpoint{0.385833in}{0.680590in}}%
\pgfpathcurveto{\pgfqpoint{0.385833in}{0.666667in}}{\pgfqpoint{0.385833in}{0.652744in}}{\pgfqpoint{0.380302in}{0.639389in}}%
\pgfpathcurveto{\pgfqpoint{0.370456in}{0.629544in}}{\pgfqpoint{0.360611in}{0.619698in}}{\pgfqpoint{0.347256in}{0.614167in}}%
\pgfpathclose%
\pgfpathmoveto{\pgfqpoint{0.500000in}{0.608333in}}%
\pgfpathcurveto{\pgfqpoint{0.515470in}{0.608333in}}{\pgfqpoint{0.530309in}{0.614480in}}{\pgfqpoint{0.541248in}{0.625419in}}%
\pgfpathcurveto{\pgfqpoint{0.552187in}{0.636358in}}{\pgfqpoint{0.558333in}{0.651196in}}{\pgfqpoint{0.558333in}{0.666667in}}%
\pgfpathcurveto{\pgfqpoint{0.558333in}{0.682137in}}{\pgfqpoint{0.552187in}{0.696975in}}{\pgfqpoint{0.541248in}{0.707915in}}%
\pgfpathcurveto{\pgfqpoint{0.530309in}{0.718854in}}{\pgfqpoint{0.515470in}{0.725000in}}{\pgfqpoint{0.500000in}{0.725000in}}%
\pgfpathcurveto{\pgfqpoint{0.484530in}{0.725000in}}{\pgfqpoint{0.469691in}{0.718854in}}{\pgfqpoint{0.458752in}{0.707915in}}%
\pgfpathcurveto{\pgfqpoint{0.447813in}{0.696975in}}{\pgfqpoint{0.441667in}{0.682137in}}{\pgfqpoint{0.441667in}{0.666667in}}%
\pgfpathcurveto{\pgfqpoint{0.441667in}{0.651196in}}{\pgfqpoint{0.447813in}{0.636358in}}{\pgfqpoint{0.458752in}{0.625419in}}%
\pgfpathcurveto{\pgfqpoint{0.469691in}{0.614480in}}{\pgfqpoint{0.484530in}{0.608333in}}{\pgfqpoint{0.500000in}{0.608333in}}%
\pgfpathclose%
\pgfpathmoveto{\pgfqpoint{0.500000in}{0.614167in}}%
\pgfpathcurveto{\pgfqpoint{0.500000in}{0.614167in}}{\pgfqpoint{0.486077in}{0.614167in}}{\pgfqpoint{0.472722in}{0.619698in}}%
\pgfpathcurveto{\pgfqpoint{0.462877in}{0.629544in}}{\pgfqpoint{0.453032in}{0.639389in}}{\pgfqpoint{0.447500in}{0.652744in}}%
\pgfpathcurveto{\pgfqpoint{0.447500in}{0.666667in}}{\pgfqpoint{0.447500in}{0.680590in}}{\pgfqpoint{0.453032in}{0.693945in}}%
\pgfpathcurveto{\pgfqpoint{0.462877in}{0.703790in}}{\pgfqpoint{0.472722in}{0.713635in}}{\pgfqpoint{0.486077in}{0.719167in}}%
\pgfpathcurveto{\pgfqpoint{0.500000in}{0.719167in}}{\pgfqpoint{0.513923in}{0.719167in}}{\pgfqpoint{0.527278in}{0.713635in}}%
\pgfpathcurveto{\pgfqpoint{0.537123in}{0.703790in}}{\pgfqpoint{0.546968in}{0.693945in}}{\pgfqpoint{0.552500in}{0.680590in}}%
\pgfpathcurveto{\pgfqpoint{0.552500in}{0.666667in}}{\pgfqpoint{0.552500in}{0.652744in}}{\pgfqpoint{0.546968in}{0.639389in}}%
\pgfpathcurveto{\pgfqpoint{0.537123in}{0.629544in}}{\pgfqpoint{0.527278in}{0.619698in}}{\pgfqpoint{0.513923in}{0.614167in}}%
\pgfpathclose%
\pgfpathmoveto{\pgfqpoint{0.666667in}{0.608333in}}%
\pgfpathcurveto{\pgfqpoint{0.682137in}{0.608333in}}{\pgfqpoint{0.696975in}{0.614480in}}{\pgfqpoint{0.707915in}{0.625419in}}%
\pgfpathcurveto{\pgfqpoint{0.718854in}{0.636358in}}{\pgfqpoint{0.725000in}{0.651196in}}{\pgfqpoint{0.725000in}{0.666667in}}%
\pgfpathcurveto{\pgfqpoint{0.725000in}{0.682137in}}{\pgfqpoint{0.718854in}{0.696975in}}{\pgfqpoint{0.707915in}{0.707915in}}%
\pgfpathcurveto{\pgfqpoint{0.696975in}{0.718854in}}{\pgfqpoint{0.682137in}{0.725000in}}{\pgfqpoint{0.666667in}{0.725000in}}%
\pgfpathcurveto{\pgfqpoint{0.651196in}{0.725000in}}{\pgfqpoint{0.636358in}{0.718854in}}{\pgfqpoint{0.625419in}{0.707915in}}%
\pgfpathcurveto{\pgfqpoint{0.614480in}{0.696975in}}{\pgfqpoint{0.608333in}{0.682137in}}{\pgfqpoint{0.608333in}{0.666667in}}%
\pgfpathcurveto{\pgfqpoint{0.608333in}{0.651196in}}{\pgfqpoint{0.614480in}{0.636358in}}{\pgfqpoint{0.625419in}{0.625419in}}%
\pgfpathcurveto{\pgfqpoint{0.636358in}{0.614480in}}{\pgfqpoint{0.651196in}{0.608333in}}{\pgfqpoint{0.666667in}{0.608333in}}%
\pgfpathclose%
\pgfpathmoveto{\pgfqpoint{0.666667in}{0.614167in}}%
\pgfpathcurveto{\pgfqpoint{0.666667in}{0.614167in}}{\pgfqpoint{0.652744in}{0.614167in}}{\pgfqpoint{0.639389in}{0.619698in}}%
\pgfpathcurveto{\pgfqpoint{0.629544in}{0.629544in}}{\pgfqpoint{0.619698in}{0.639389in}}{\pgfqpoint{0.614167in}{0.652744in}}%
\pgfpathcurveto{\pgfqpoint{0.614167in}{0.666667in}}{\pgfqpoint{0.614167in}{0.680590in}}{\pgfqpoint{0.619698in}{0.693945in}}%
\pgfpathcurveto{\pgfqpoint{0.629544in}{0.703790in}}{\pgfqpoint{0.639389in}{0.713635in}}{\pgfqpoint{0.652744in}{0.719167in}}%
\pgfpathcurveto{\pgfqpoint{0.666667in}{0.719167in}}{\pgfqpoint{0.680590in}{0.719167in}}{\pgfqpoint{0.693945in}{0.713635in}}%
\pgfpathcurveto{\pgfqpoint{0.703790in}{0.703790in}}{\pgfqpoint{0.713635in}{0.693945in}}{\pgfqpoint{0.719167in}{0.680590in}}%
\pgfpathcurveto{\pgfqpoint{0.719167in}{0.666667in}}{\pgfqpoint{0.719167in}{0.652744in}}{\pgfqpoint{0.713635in}{0.639389in}}%
\pgfpathcurveto{\pgfqpoint{0.703790in}{0.629544in}}{\pgfqpoint{0.693945in}{0.619698in}}{\pgfqpoint{0.680590in}{0.614167in}}%
\pgfpathclose%
\pgfpathmoveto{\pgfqpoint{0.833333in}{0.608333in}}%
\pgfpathcurveto{\pgfqpoint{0.848804in}{0.608333in}}{\pgfqpoint{0.863642in}{0.614480in}}{\pgfqpoint{0.874581in}{0.625419in}}%
\pgfpathcurveto{\pgfqpoint{0.885520in}{0.636358in}}{\pgfqpoint{0.891667in}{0.651196in}}{\pgfqpoint{0.891667in}{0.666667in}}%
\pgfpathcurveto{\pgfqpoint{0.891667in}{0.682137in}}{\pgfqpoint{0.885520in}{0.696975in}}{\pgfqpoint{0.874581in}{0.707915in}}%
\pgfpathcurveto{\pgfqpoint{0.863642in}{0.718854in}}{\pgfqpoint{0.848804in}{0.725000in}}{\pgfqpoint{0.833333in}{0.725000in}}%
\pgfpathcurveto{\pgfqpoint{0.817863in}{0.725000in}}{\pgfqpoint{0.803025in}{0.718854in}}{\pgfqpoint{0.792085in}{0.707915in}}%
\pgfpathcurveto{\pgfqpoint{0.781146in}{0.696975in}}{\pgfqpoint{0.775000in}{0.682137in}}{\pgfqpoint{0.775000in}{0.666667in}}%
\pgfpathcurveto{\pgfqpoint{0.775000in}{0.651196in}}{\pgfqpoint{0.781146in}{0.636358in}}{\pgfqpoint{0.792085in}{0.625419in}}%
\pgfpathcurveto{\pgfqpoint{0.803025in}{0.614480in}}{\pgfqpoint{0.817863in}{0.608333in}}{\pgfqpoint{0.833333in}{0.608333in}}%
\pgfpathclose%
\pgfpathmoveto{\pgfqpoint{0.833333in}{0.614167in}}%
\pgfpathcurveto{\pgfqpoint{0.833333in}{0.614167in}}{\pgfqpoint{0.819410in}{0.614167in}}{\pgfqpoint{0.806055in}{0.619698in}}%
\pgfpathcurveto{\pgfqpoint{0.796210in}{0.629544in}}{\pgfqpoint{0.786365in}{0.639389in}}{\pgfqpoint{0.780833in}{0.652744in}}%
\pgfpathcurveto{\pgfqpoint{0.780833in}{0.666667in}}{\pgfqpoint{0.780833in}{0.680590in}}{\pgfqpoint{0.786365in}{0.693945in}}%
\pgfpathcurveto{\pgfqpoint{0.796210in}{0.703790in}}{\pgfqpoint{0.806055in}{0.713635in}}{\pgfqpoint{0.819410in}{0.719167in}}%
\pgfpathcurveto{\pgfqpoint{0.833333in}{0.719167in}}{\pgfqpoint{0.847256in}{0.719167in}}{\pgfqpoint{0.860611in}{0.713635in}}%
\pgfpathcurveto{\pgfqpoint{0.870456in}{0.703790in}}{\pgfqpoint{0.880302in}{0.693945in}}{\pgfqpoint{0.885833in}{0.680590in}}%
\pgfpathcurveto{\pgfqpoint{0.885833in}{0.666667in}}{\pgfqpoint{0.885833in}{0.652744in}}{\pgfqpoint{0.880302in}{0.639389in}}%
\pgfpathcurveto{\pgfqpoint{0.870456in}{0.629544in}}{\pgfqpoint{0.860611in}{0.619698in}}{\pgfqpoint{0.847256in}{0.614167in}}%
\pgfpathclose%
\pgfpathmoveto{\pgfqpoint{1.000000in}{0.608333in}}%
\pgfpathcurveto{\pgfqpoint{1.015470in}{0.608333in}}{\pgfqpoint{1.030309in}{0.614480in}}{\pgfqpoint{1.041248in}{0.625419in}}%
\pgfpathcurveto{\pgfqpoint{1.052187in}{0.636358in}}{\pgfqpoint{1.058333in}{0.651196in}}{\pgfqpoint{1.058333in}{0.666667in}}%
\pgfpathcurveto{\pgfqpoint{1.058333in}{0.682137in}}{\pgfqpoint{1.052187in}{0.696975in}}{\pgfqpoint{1.041248in}{0.707915in}}%
\pgfpathcurveto{\pgfqpoint{1.030309in}{0.718854in}}{\pgfqpoint{1.015470in}{0.725000in}}{\pgfqpoint{1.000000in}{0.725000in}}%
\pgfpathcurveto{\pgfqpoint{0.984530in}{0.725000in}}{\pgfqpoint{0.969691in}{0.718854in}}{\pgfqpoint{0.958752in}{0.707915in}}%
\pgfpathcurveto{\pgfqpoint{0.947813in}{0.696975in}}{\pgfqpoint{0.941667in}{0.682137in}}{\pgfqpoint{0.941667in}{0.666667in}}%
\pgfpathcurveto{\pgfqpoint{0.941667in}{0.651196in}}{\pgfqpoint{0.947813in}{0.636358in}}{\pgfqpoint{0.958752in}{0.625419in}}%
\pgfpathcurveto{\pgfqpoint{0.969691in}{0.614480in}}{\pgfqpoint{0.984530in}{0.608333in}}{\pgfqpoint{1.000000in}{0.608333in}}%
\pgfpathclose%
\pgfpathmoveto{\pgfqpoint{1.000000in}{0.614167in}}%
\pgfpathcurveto{\pgfqpoint{1.000000in}{0.614167in}}{\pgfqpoint{0.986077in}{0.614167in}}{\pgfqpoint{0.972722in}{0.619698in}}%
\pgfpathcurveto{\pgfqpoint{0.962877in}{0.629544in}}{\pgfqpoint{0.953032in}{0.639389in}}{\pgfqpoint{0.947500in}{0.652744in}}%
\pgfpathcurveto{\pgfqpoint{0.947500in}{0.666667in}}{\pgfqpoint{0.947500in}{0.680590in}}{\pgfqpoint{0.953032in}{0.693945in}}%
\pgfpathcurveto{\pgfqpoint{0.962877in}{0.703790in}}{\pgfqpoint{0.972722in}{0.713635in}}{\pgfqpoint{0.986077in}{0.719167in}}%
\pgfpathcurveto{\pgfqpoint{1.000000in}{0.719167in}}{\pgfqpoint{1.013923in}{0.719167in}}{\pgfqpoint{1.027278in}{0.713635in}}%
\pgfpathcurveto{\pgfqpoint{1.037123in}{0.703790in}}{\pgfqpoint{1.046968in}{0.693945in}}{\pgfqpoint{1.052500in}{0.680590in}}%
\pgfpathcurveto{\pgfqpoint{1.052500in}{0.666667in}}{\pgfqpoint{1.052500in}{0.652744in}}{\pgfqpoint{1.046968in}{0.639389in}}%
\pgfpathcurveto{\pgfqpoint{1.037123in}{0.629544in}}{\pgfqpoint{1.027278in}{0.619698in}}{\pgfqpoint{1.013923in}{0.614167in}}%
\pgfpathclose%
\pgfpathmoveto{\pgfqpoint{0.083333in}{0.775000in}}%
\pgfpathcurveto{\pgfqpoint{0.098804in}{0.775000in}}{\pgfqpoint{0.113642in}{0.781146in}}{\pgfqpoint{0.124581in}{0.792085in}}%
\pgfpathcurveto{\pgfqpoint{0.135520in}{0.803025in}}{\pgfqpoint{0.141667in}{0.817863in}}{\pgfqpoint{0.141667in}{0.833333in}}%
\pgfpathcurveto{\pgfqpoint{0.141667in}{0.848804in}}{\pgfqpoint{0.135520in}{0.863642in}}{\pgfqpoint{0.124581in}{0.874581in}}%
\pgfpathcurveto{\pgfqpoint{0.113642in}{0.885520in}}{\pgfqpoint{0.098804in}{0.891667in}}{\pgfqpoint{0.083333in}{0.891667in}}%
\pgfpathcurveto{\pgfqpoint{0.067863in}{0.891667in}}{\pgfqpoint{0.053025in}{0.885520in}}{\pgfqpoint{0.042085in}{0.874581in}}%
\pgfpathcurveto{\pgfqpoint{0.031146in}{0.863642in}}{\pgfqpoint{0.025000in}{0.848804in}}{\pgfqpoint{0.025000in}{0.833333in}}%
\pgfpathcurveto{\pgfqpoint{0.025000in}{0.817863in}}{\pgfqpoint{0.031146in}{0.803025in}}{\pgfqpoint{0.042085in}{0.792085in}}%
\pgfpathcurveto{\pgfqpoint{0.053025in}{0.781146in}}{\pgfqpoint{0.067863in}{0.775000in}}{\pgfqpoint{0.083333in}{0.775000in}}%
\pgfpathclose%
\pgfpathmoveto{\pgfqpoint{0.083333in}{0.780833in}}%
\pgfpathcurveto{\pgfqpoint{0.083333in}{0.780833in}}{\pgfqpoint{0.069410in}{0.780833in}}{\pgfqpoint{0.056055in}{0.786365in}}%
\pgfpathcurveto{\pgfqpoint{0.046210in}{0.796210in}}{\pgfqpoint{0.036365in}{0.806055in}}{\pgfqpoint{0.030833in}{0.819410in}}%
\pgfpathcurveto{\pgfqpoint{0.030833in}{0.833333in}}{\pgfqpoint{0.030833in}{0.847256in}}{\pgfqpoint{0.036365in}{0.860611in}}%
\pgfpathcurveto{\pgfqpoint{0.046210in}{0.870456in}}{\pgfqpoint{0.056055in}{0.880302in}}{\pgfqpoint{0.069410in}{0.885833in}}%
\pgfpathcurveto{\pgfqpoint{0.083333in}{0.885833in}}{\pgfqpoint{0.097256in}{0.885833in}}{\pgfqpoint{0.110611in}{0.880302in}}%
\pgfpathcurveto{\pgfqpoint{0.120456in}{0.870456in}}{\pgfqpoint{0.130302in}{0.860611in}}{\pgfqpoint{0.135833in}{0.847256in}}%
\pgfpathcurveto{\pgfqpoint{0.135833in}{0.833333in}}{\pgfqpoint{0.135833in}{0.819410in}}{\pgfqpoint{0.130302in}{0.806055in}}%
\pgfpathcurveto{\pgfqpoint{0.120456in}{0.796210in}}{\pgfqpoint{0.110611in}{0.786365in}}{\pgfqpoint{0.097256in}{0.780833in}}%
\pgfpathclose%
\pgfpathmoveto{\pgfqpoint{0.250000in}{0.775000in}}%
\pgfpathcurveto{\pgfqpoint{0.265470in}{0.775000in}}{\pgfqpoint{0.280309in}{0.781146in}}{\pgfqpoint{0.291248in}{0.792085in}}%
\pgfpathcurveto{\pgfqpoint{0.302187in}{0.803025in}}{\pgfqpoint{0.308333in}{0.817863in}}{\pgfqpoint{0.308333in}{0.833333in}}%
\pgfpathcurveto{\pgfqpoint{0.308333in}{0.848804in}}{\pgfqpoint{0.302187in}{0.863642in}}{\pgfqpoint{0.291248in}{0.874581in}}%
\pgfpathcurveto{\pgfqpoint{0.280309in}{0.885520in}}{\pgfqpoint{0.265470in}{0.891667in}}{\pgfqpoint{0.250000in}{0.891667in}}%
\pgfpathcurveto{\pgfqpoint{0.234530in}{0.891667in}}{\pgfqpoint{0.219691in}{0.885520in}}{\pgfqpoint{0.208752in}{0.874581in}}%
\pgfpathcurveto{\pgfqpoint{0.197813in}{0.863642in}}{\pgfqpoint{0.191667in}{0.848804in}}{\pgfqpoint{0.191667in}{0.833333in}}%
\pgfpathcurveto{\pgfqpoint{0.191667in}{0.817863in}}{\pgfqpoint{0.197813in}{0.803025in}}{\pgfqpoint{0.208752in}{0.792085in}}%
\pgfpathcurveto{\pgfqpoint{0.219691in}{0.781146in}}{\pgfqpoint{0.234530in}{0.775000in}}{\pgfqpoint{0.250000in}{0.775000in}}%
\pgfpathclose%
\pgfpathmoveto{\pgfqpoint{0.250000in}{0.780833in}}%
\pgfpathcurveto{\pgfqpoint{0.250000in}{0.780833in}}{\pgfqpoint{0.236077in}{0.780833in}}{\pgfqpoint{0.222722in}{0.786365in}}%
\pgfpathcurveto{\pgfqpoint{0.212877in}{0.796210in}}{\pgfqpoint{0.203032in}{0.806055in}}{\pgfqpoint{0.197500in}{0.819410in}}%
\pgfpathcurveto{\pgfqpoint{0.197500in}{0.833333in}}{\pgfqpoint{0.197500in}{0.847256in}}{\pgfqpoint{0.203032in}{0.860611in}}%
\pgfpathcurveto{\pgfqpoint{0.212877in}{0.870456in}}{\pgfqpoint{0.222722in}{0.880302in}}{\pgfqpoint{0.236077in}{0.885833in}}%
\pgfpathcurveto{\pgfqpoint{0.250000in}{0.885833in}}{\pgfqpoint{0.263923in}{0.885833in}}{\pgfqpoint{0.277278in}{0.880302in}}%
\pgfpathcurveto{\pgfqpoint{0.287123in}{0.870456in}}{\pgfqpoint{0.296968in}{0.860611in}}{\pgfqpoint{0.302500in}{0.847256in}}%
\pgfpathcurveto{\pgfqpoint{0.302500in}{0.833333in}}{\pgfqpoint{0.302500in}{0.819410in}}{\pgfqpoint{0.296968in}{0.806055in}}%
\pgfpathcurveto{\pgfqpoint{0.287123in}{0.796210in}}{\pgfqpoint{0.277278in}{0.786365in}}{\pgfqpoint{0.263923in}{0.780833in}}%
\pgfpathclose%
\pgfpathmoveto{\pgfqpoint{0.416667in}{0.775000in}}%
\pgfpathcurveto{\pgfqpoint{0.432137in}{0.775000in}}{\pgfqpoint{0.446975in}{0.781146in}}{\pgfqpoint{0.457915in}{0.792085in}}%
\pgfpathcurveto{\pgfqpoint{0.468854in}{0.803025in}}{\pgfqpoint{0.475000in}{0.817863in}}{\pgfqpoint{0.475000in}{0.833333in}}%
\pgfpathcurveto{\pgfqpoint{0.475000in}{0.848804in}}{\pgfqpoint{0.468854in}{0.863642in}}{\pgfqpoint{0.457915in}{0.874581in}}%
\pgfpathcurveto{\pgfqpoint{0.446975in}{0.885520in}}{\pgfqpoint{0.432137in}{0.891667in}}{\pgfqpoint{0.416667in}{0.891667in}}%
\pgfpathcurveto{\pgfqpoint{0.401196in}{0.891667in}}{\pgfqpoint{0.386358in}{0.885520in}}{\pgfqpoint{0.375419in}{0.874581in}}%
\pgfpathcurveto{\pgfqpoint{0.364480in}{0.863642in}}{\pgfqpoint{0.358333in}{0.848804in}}{\pgfqpoint{0.358333in}{0.833333in}}%
\pgfpathcurveto{\pgfqpoint{0.358333in}{0.817863in}}{\pgfqpoint{0.364480in}{0.803025in}}{\pgfqpoint{0.375419in}{0.792085in}}%
\pgfpathcurveto{\pgfqpoint{0.386358in}{0.781146in}}{\pgfqpoint{0.401196in}{0.775000in}}{\pgfqpoint{0.416667in}{0.775000in}}%
\pgfpathclose%
\pgfpathmoveto{\pgfqpoint{0.416667in}{0.780833in}}%
\pgfpathcurveto{\pgfqpoint{0.416667in}{0.780833in}}{\pgfqpoint{0.402744in}{0.780833in}}{\pgfqpoint{0.389389in}{0.786365in}}%
\pgfpathcurveto{\pgfqpoint{0.379544in}{0.796210in}}{\pgfqpoint{0.369698in}{0.806055in}}{\pgfqpoint{0.364167in}{0.819410in}}%
\pgfpathcurveto{\pgfqpoint{0.364167in}{0.833333in}}{\pgfqpoint{0.364167in}{0.847256in}}{\pgfqpoint{0.369698in}{0.860611in}}%
\pgfpathcurveto{\pgfqpoint{0.379544in}{0.870456in}}{\pgfqpoint{0.389389in}{0.880302in}}{\pgfqpoint{0.402744in}{0.885833in}}%
\pgfpathcurveto{\pgfqpoint{0.416667in}{0.885833in}}{\pgfqpoint{0.430590in}{0.885833in}}{\pgfqpoint{0.443945in}{0.880302in}}%
\pgfpathcurveto{\pgfqpoint{0.453790in}{0.870456in}}{\pgfqpoint{0.463635in}{0.860611in}}{\pgfqpoint{0.469167in}{0.847256in}}%
\pgfpathcurveto{\pgfqpoint{0.469167in}{0.833333in}}{\pgfqpoint{0.469167in}{0.819410in}}{\pgfqpoint{0.463635in}{0.806055in}}%
\pgfpathcurveto{\pgfqpoint{0.453790in}{0.796210in}}{\pgfqpoint{0.443945in}{0.786365in}}{\pgfqpoint{0.430590in}{0.780833in}}%
\pgfpathclose%
\pgfpathmoveto{\pgfqpoint{0.583333in}{0.775000in}}%
\pgfpathcurveto{\pgfqpoint{0.598804in}{0.775000in}}{\pgfqpoint{0.613642in}{0.781146in}}{\pgfqpoint{0.624581in}{0.792085in}}%
\pgfpathcurveto{\pgfqpoint{0.635520in}{0.803025in}}{\pgfqpoint{0.641667in}{0.817863in}}{\pgfqpoint{0.641667in}{0.833333in}}%
\pgfpathcurveto{\pgfqpoint{0.641667in}{0.848804in}}{\pgfqpoint{0.635520in}{0.863642in}}{\pgfqpoint{0.624581in}{0.874581in}}%
\pgfpathcurveto{\pgfqpoint{0.613642in}{0.885520in}}{\pgfqpoint{0.598804in}{0.891667in}}{\pgfqpoint{0.583333in}{0.891667in}}%
\pgfpathcurveto{\pgfqpoint{0.567863in}{0.891667in}}{\pgfqpoint{0.553025in}{0.885520in}}{\pgfqpoint{0.542085in}{0.874581in}}%
\pgfpathcurveto{\pgfqpoint{0.531146in}{0.863642in}}{\pgfqpoint{0.525000in}{0.848804in}}{\pgfqpoint{0.525000in}{0.833333in}}%
\pgfpathcurveto{\pgfqpoint{0.525000in}{0.817863in}}{\pgfqpoint{0.531146in}{0.803025in}}{\pgfqpoint{0.542085in}{0.792085in}}%
\pgfpathcurveto{\pgfqpoint{0.553025in}{0.781146in}}{\pgfqpoint{0.567863in}{0.775000in}}{\pgfqpoint{0.583333in}{0.775000in}}%
\pgfpathclose%
\pgfpathmoveto{\pgfqpoint{0.583333in}{0.780833in}}%
\pgfpathcurveto{\pgfqpoint{0.583333in}{0.780833in}}{\pgfqpoint{0.569410in}{0.780833in}}{\pgfqpoint{0.556055in}{0.786365in}}%
\pgfpathcurveto{\pgfqpoint{0.546210in}{0.796210in}}{\pgfqpoint{0.536365in}{0.806055in}}{\pgfqpoint{0.530833in}{0.819410in}}%
\pgfpathcurveto{\pgfqpoint{0.530833in}{0.833333in}}{\pgfqpoint{0.530833in}{0.847256in}}{\pgfqpoint{0.536365in}{0.860611in}}%
\pgfpathcurveto{\pgfqpoint{0.546210in}{0.870456in}}{\pgfqpoint{0.556055in}{0.880302in}}{\pgfqpoint{0.569410in}{0.885833in}}%
\pgfpathcurveto{\pgfqpoint{0.583333in}{0.885833in}}{\pgfqpoint{0.597256in}{0.885833in}}{\pgfqpoint{0.610611in}{0.880302in}}%
\pgfpathcurveto{\pgfqpoint{0.620456in}{0.870456in}}{\pgfqpoint{0.630302in}{0.860611in}}{\pgfqpoint{0.635833in}{0.847256in}}%
\pgfpathcurveto{\pgfqpoint{0.635833in}{0.833333in}}{\pgfqpoint{0.635833in}{0.819410in}}{\pgfqpoint{0.630302in}{0.806055in}}%
\pgfpathcurveto{\pgfqpoint{0.620456in}{0.796210in}}{\pgfqpoint{0.610611in}{0.786365in}}{\pgfqpoint{0.597256in}{0.780833in}}%
\pgfpathclose%
\pgfpathmoveto{\pgfqpoint{0.750000in}{0.775000in}}%
\pgfpathcurveto{\pgfqpoint{0.765470in}{0.775000in}}{\pgfqpoint{0.780309in}{0.781146in}}{\pgfqpoint{0.791248in}{0.792085in}}%
\pgfpathcurveto{\pgfqpoint{0.802187in}{0.803025in}}{\pgfqpoint{0.808333in}{0.817863in}}{\pgfqpoint{0.808333in}{0.833333in}}%
\pgfpathcurveto{\pgfqpoint{0.808333in}{0.848804in}}{\pgfqpoint{0.802187in}{0.863642in}}{\pgfqpoint{0.791248in}{0.874581in}}%
\pgfpathcurveto{\pgfqpoint{0.780309in}{0.885520in}}{\pgfqpoint{0.765470in}{0.891667in}}{\pgfqpoint{0.750000in}{0.891667in}}%
\pgfpathcurveto{\pgfqpoint{0.734530in}{0.891667in}}{\pgfqpoint{0.719691in}{0.885520in}}{\pgfqpoint{0.708752in}{0.874581in}}%
\pgfpathcurveto{\pgfqpoint{0.697813in}{0.863642in}}{\pgfqpoint{0.691667in}{0.848804in}}{\pgfqpoint{0.691667in}{0.833333in}}%
\pgfpathcurveto{\pgfqpoint{0.691667in}{0.817863in}}{\pgfqpoint{0.697813in}{0.803025in}}{\pgfqpoint{0.708752in}{0.792085in}}%
\pgfpathcurveto{\pgfqpoint{0.719691in}{0.781146in}}{\pgfqpoint{0.734530in}{0.775000in}}{\pgfqpoint{0.750000in}{0.775000in}}%
\pgfpathclose%
\pgfpathmoveto{\pgfqpoint{0.750000in}{0.780833in}}%
\pgfpathcurveto{\pgfqpoint{0.750000in}{0.780833in}}{\pgfqpoint{0.736077in}{0.780833in}}{\pgfqpoint{0.722722in}{0.786365in}}%
\pgfpathcurveto{\pgfqpoint{0.712877in}{0.796210in}}{\pgfqpoint{0.703032in}{0.806055in}}{\pgfqpoint{0.697500in}{0.819410in}}%
\pgfpathcurveto{\pgfqpoint{0.697500in}{0.833333in}}{\pgfqpoint{0.697500in}{0.847256in}}{\pgfqpoint{0.703032in}{0.860611in}}%
\pgfpathcurveto{\pgfqpoint{0.712877in}{0.870456in}}{\pgfqpoint{0.722722in}{0.880302in}}{\pgfqpoint{0.736077in}{0.885833in}}%
\pgfpathcurveto{\pgfqpoint{0.750000in}{0.885833in}}{\pgfqpoint{0.763923in}{0.885833in}}{\pgfqpoint{0.777278in}{0.880302in}}%
\pgfpathcurveto{\pgfqpoint{0.787123in}{0.870456in}}{\pgfqpoint{0.796968in}{0.860611in}}{\pgfqpoint{0.802500in}{0.847256in}}%
\pgfpathcurveto{\pgfqpoint{0.802500in}{0.833333in}}{\pgfqpoint{0.802500in}{0.819410in}}{\pgfqpoint{0.796968in}{0.806055in}}%
\pgfpathcurveto{\pgfqpoint{0.787123in}{0.796210in}}{\pgfqpoint{0.777278in}{0.786365in}}{\pgfqpoint{0.763923in}{0.780833in}}%
\pgfpathclose%
\pgfpathmoveto{\pgfqpoint{0.916667in}{0.775000in}}%
\pgfpathcurveto{\pgfqpoint{0.932137in}{0.775000in}}{\pgfqpoint{0.946975in}{0.781146in}}{\pgfqpoint{0.957915in}{0.792085in}}%
\pgfpathcurveto{\pgfqpoint{0.968854in}{0.803025in}}{\pgfqpoint{0.975000in}{0.817863in}}{\pgfqpoint{0.975000in}{0.833333in}}%
\pgfpathcurveto{\pgfqpoint{0.975000in}{0.848804in}}{\pgfqpoint{0.968854in}{0.863642in}}{\pgfqpoint{0.957915in}{0.874581in}}%
\pgfpathcurveto{\pgfqpoint{0.946975in}{0.885520in}}{\pgfqpoint{0.932137in}{0.891667in}}{\pgfqpoint{0.916667in}{0.891667in}}%
\pgfpathcurveto{\pgfqpoint{0.901196in}{0.891667in}}{\pgfqpoint{0.886358in}{0.885520in}}{\pgfqpoint{0.875419in}{0.874581in}}%
\pgfpathcurveto{\pgfqpoint{0.864480in}{0.863642in}}{\pgfqpoint{0.858333in}{0.848804in}}{\pgfqpoint{0.858333in}{0.833333in}}%
\pgfpathcurveto{\pgfqpoint{0.858333in}{0.817863in}}{\pgfqpoint{0.864480in}{0.803025in}}{\pgfqpoint{0.875419in}{0.792085in}}%
\pgfpathcurveto{\pgfqpoint{0.886358in}{0.781146in}}{\pgfqpoint{0.901196in}{0.775000in}}{\pgfqpoint{0.916667in}{0.775000in}}%
\pgfpathclose%
\pgfpathmoveto{\pgfqpoint{0.916667in}{0.780833in}}%
\pgfpathcurveto{\pgfqpoint{0.916667in}{0.780833in}}{\pgfqpoint{0.902744in}{0.780833in}}{\pgfqpoint{0.889389in}{0.786365in}}%
\pgfpathcurveto{\pgfqpoint{0.879544in}{0.796210in}}{\pgfqpoint{0.869698in}{0.806055in}}{\pgfqpoint{0.864167in}{0.819410in}}%
\pgfpathcurveto{\pgfqpoint{0.864167in}{0.833333in}}{\pgfqpoint{0.864167in}{0.847256in}}{\pgfqpoint{0.869698in}{0.860611in}}%
\pgfpathcurveto{\pgfqpoint{0.879544in}{0.870456in}}{\pgfqpoint{0.889389in}{0.880302in}}{\pgfqpoint{0.902744in}{0.885833in}}%
\pgfpathcurveto{\pgfqpoint{0.916667in}{0.885833in}}{\pgfqpoint{0.930590in}{0.885833in}}{\pgfqpoint{0.943945in}{0.880302in}}%
\pgfpathcurveto{\pgfqpoint{0.953790in}{0.870456in}}{\pgfqpoint{0.963635in}{0.860611in}}{\pgfqpoint{0.969167in}{0.847256in}}%
\pgfpathcurveto{\pgfqpoint{0.969167in}{0.833333in}}{\pgfqpoint{0.969167in}{0.819410in}}{\pgfqpoint{0.963635in}{0.806055in}}%
\pgfpathcurveto{\pgfqpoint{0.953790in}{0.796210in}}{\pgfqpoint{0.943945in}{0.786365in}}{\pgfqpoint{0.930590in}{0.780833in}}%
\pgfpathclose%
\pgfpathmoveto{\pgfqpoint{0.000000in}{0.941667in}}%
\pgfpathcurveto{\pgfqpoint{0.015470in}{0.941667in}}{\pgfqpoint{0.030309in}{0.947813in}}{\pgfqpoint{0.041248in}{0.958752in}}%
\pgfpathcurveto{\pgfqpoint{0.052187in}{0.969691in}}{\pgfqpoint{0.058333in}{0.984530in}}{\pgfqpoint{0.058333in}{1.000000in}}%
\pgfpathcurveto{\pgfqpoint{0.058333in}{1.015470in}}{\pgfqpoint{0.052187in}{1.030309in}}{\pgfqpoint{0.041248in}{1.041248in}}%
\pgfpathcurveto{\pgfqpoint{0.030309in}{1.052187in}}{\pgfqpoint{0.015470in}{1.058333in}}{\pgfqpoint{0.000000in}{1.058333in}}%
\pgfpathcurveto{\pgfqpoint{-0.015470in}{1.058333in}}{\pgfqpoint{-0.030309in}{1.052187in}}{\pgfqpoint{-0.041248in}{1.041248in}}%
\pgfpathcurveto{\pgfqpoint{-0.052187in}{1.030309in}}{\pgfqpoint{-0.058333in}{1.015470in}}{\pgfqpoint{-0.058333in}{1.000000in}}%
\pgfpathcurveto{\pgfqpoint{-0.058333in}{0.984530in}}{\pgfqpoint{-0.052187in}{0.969691in}}{\pgfqpoint{-0.041248in}{0.958752in}}%
\pgfpathcurveto{\pgfqpoint{-0.030309in}{0.947813in}}{\pgfqpoint{-0.015470in}{0.941667in}}{\pgfqpoint{0.000000in}{0.941667in}}%
\pgfpathclose%
\pgfpathmoveto{\pgfqpoint{0.000000in}{0.947500in}}%
\pgfpathcurveto{\pgfqpoint{0.000000in}{0.947500in}}{\pgfqpoint{-0.013923in}{0.947500in}}{\pgfqpoint{-0.027278in}{0.953032in}}%
\pgfpathcurveto{\pgfqpoint{-0.037123in}{0.962877in}}{\pgfqpoint{-0.046968in}{0.972722in}}{\pgfqpoint{-0.052500in}{0.986077in}}%
\pgfpathcurveto{\pgfqpoint{-0.052500in}{1.000000in}}{\pgfqpoint{-0.052500in}{1.013923in}}{\pgfqpoint{-0.046968in}{1.027278in}}%
\pgfpathcurveto{\pgfqpoint{-0.037123in}{1.037123in}}{\pgfqpoint{-0.027278in}{1.046968in}}{\pgfqpoint{-0.013923in}{1.052500in}}%
\pgfpathcurveto{\pgfqpoint{0.000000in}{1.052500in}}{\pgfqpoint{0.013923in}{1.052500in}}{\pgfqpoint{0.027278in}{1.046968in}}%
\pgfpathcurveto{\pgfqpoint{0.037123in}{1.037123in}}{\pgfqpoint{0.046968in}{1.027278in}}{\pgfqpoint{0.052500in}{1.013923in}}%
\pgfpathcurveto{\pgfqpoint{0.052500in}{1.000000in}}{\pgfqpoint{0.052500in}{0.986077in}}{\pgfqpoint{0.046968in}{0.972722in}}%
\pgfpathcurveto{\pgfqpoint{0.037123in}{0.962877in}}{\pgfqpoint{0.027278in}{0.953032in}}{\pgfqpoint{0.013923in}{0.947500in}}%
\pgfpathclose%
\pgfpathmoveto{\pgfqpoint{0.166667in}{0.941667in}}%
\pgfpathcurveto{\pgfqpoint{0.182137in}{0.941667in}}{\pgfqpoint{0.196975in}{0.947813in}}{\pgfqpoint{0.207915in}{0.958752in}}%
\pgfpathcurveto{\pgfqpoint{0.218854in}{0.969691in}}{\pgfqpoint{0.225000in}{0.984530in}}{\pgfqpoint{0.225000in}{1.000000in}}%
\pgfpathcurveto{\pgfqpoint{0.225000in}{1.015470in}}{\pgfqpoint{0.218854in}{1.030309in}}{\pgfqpoint{0.207915in}{1.041248in}}%
\pgfpathcurveto{\pgfqpoint{0.196975in}{1.052187in}}{\pgfqpoint{0.182137in}{1.058333in}}{\pgfqpoint{0.166667in}{1.058333in}}%
\pgfpathcurveto{\pgfqpoint{0.151196in}{1.058333in}}{\pgfqpoint{0.136358in}{1.052187in}}{\pgfqpoint{0.125419in}{1.041248in}}%
\pgfpathcurveto{\pgfqpoint{0.114480in}{1.030309in}}{\pgfqpoint{0.108333in}{1.015470in}}{\pgfqpoint{0.108333in}{1.000000in}}%
\pgfpathcurveto{\pgfqpoint{0.108333in}{0.984530in}}{\pgfqpoint{0.114480in}{0.969691in}}{\pgfqpoint{0.125419in}{0.958752in}}%
\pgfpathcurveto{\pgfqpoint{0.136358in}{0.947813in}}{\pgfqpoint{0.151196in}{0.941667in}}{\pgfqpoint{0.166667in}{0.941667in}}%
\pgfpathclose%
\pgfpathmoveto{\pgfqpoint{0.166667in}{0.947500in}}%
\pgfpathcurveto{\pgfqpoint{0.166667in}{0.947500in}}{\pgfqpoint{0.152744in}{0.947500in}}{\pgfqpoint{0.139389in}{0.953032in}}%
\pgfpathcurveto{\pgfqpoint{0.129544in}{0.962877in}}{\pgfqpoint{0.119698in}{0.972722in}}{\pgfqpoint{0.114167in}{0.986077in}}%
\pgfpathcurveto{\pgfqpoint{0.114167in}{1.000000in}}{\pgfqpoint{0.114167in}{1.013923in}}{\pgfqpoint{0.119698in}{1.027278in}}%
\pgfpathcurveto{\pgfqpoint{0.129544in}{1.037123in}}{\pgfqpoint{0.139389in}{1.046968in}}{\pgfqpoint{0.152744in}{1.052500in}}%
\pgfpathcurveto{\pgfqpoint{0.166667in}{1.052500in}}{\pgfqpoint{0.180590in}{1.052500in}}{\pgfqpoint{0.193945in}{1.046968in}}%
\pgfpathcurveto{\pgfqpoint{0.203790in}{1.037123in}}{\pgfqpoint{0.213635in}{1.027278in}}{\pgfqpoint{0.219167in}{1.013923in}}%
\pgfpathcurveto{\pgfqpoint{0.219167in}{1.000000in}}{\pgfqpoint{0.219167in}{0.986077in}}{\pgfqpoint{0.213635in}{0.972722in}}%
\pgfpathcurveto{\pgfqpoint{0.203790in}{0.962877in}}{\pgfqpoint{0.193945in}{0.953032in}}{\pgfqpoint{0.180590in}{0.947500in}}%
\pgfpathclose%
\pgfpathmoveto{\pgfqpoint{0.333333in}{0.941667in}}%
\pgfpathcurveto{\pgfqpoint{0.348804in}{0.941667in}}{\pgfqpoint{0.363642in}{0.947813in}}{\pgfqpoint{0.374581in}{0.958752in}}%
\pgfpathcurveto{\pgfqpoint{0.385520in}{0.969691in}}{\pgfqpoint{0.391667in}{0.984530in}}{\pgfqpoint{0.391667in}{1.000000in}}%
\pgfpathcurveto{\pgfqpoint{0.391667in}{1.015470in}}{\pgfqpoint{0.385520in}{1.030309in}}{\pgfqpoint{0.374581in}{1.041248in}}%
\pgfpathcurveto{\pgfqpoint{0.363642in}{1.052187in}}{\pgfqpoint{0.348804in}{1.058333in}}{\pgfqpoint{0.333333in}{1.058333in}}%
\pgfpathcurveto{\pgfqpoint{0.317863in}{1.058333in}}{\pgfqpoint{0.303025in}{1.052187in}}{\pgfqpoint{0.292085in}{1.041248in}}%
\pgfpathcurveto{\pgfqpoint{0.281146in}{1.030309in}}{\pgfqpoint{0.275000in}{1.015470in}}{\pgfqpoint{0.275000in}{1.000000in}}%
\pgfpathcurveto{\pgfqpoint{0.275000in}{0.984530in}}{\pgfqpoint{0.281146in}{0.969691in}}{\pgfqpoint{0.292085in}{0.958752in}}%
\pgfpathcurveto{\pgfqpoint{0.303025in}{0.947813in}}{\pgfqpoint{0.317863in}{0.941667in}}{\pgfqpoint{0.333333in}{0.941667in}}%
\pgfpathclose%
\pgfpathmoveto{\pgfqpoint{0.333333in}{0.947500in}}%
\pgfpathcurveto{\pgfqpoint{0.333333in}{0.947500in}}{\pgfqpoint{0.319410in}{0.947500in}}{\pgfqpoint{0.306055in}{0.953032in}}%
\pgfpathcurveto{\pgfqpoint{0.296210in}{0.962877in}}{\pgfqpoint{0.286365in}{0.972722in}}{\pgfqpoint{0.280833in}{0.986077in}}%
\pgfpathcurveto{\pgfqpoint{0.280833in}{1.000000in}}{\pgfqpoint{0.280833in}{1.013923in}}{\pgfqpoint{0.286365in}{1.027278in}}%
\pgfpathcurveto{\pgfqpoint{0.296210in}{1.037123in}}{\pgfqpoint{0.306055in}{1.046968in}}{\pgfqpoint{0.319410in}{1.052500in}}%
\pgfpathcurveto{\pgfqpoint{0.333333in}{1.052500in}}{\pgfqpoint{0.347256in}{1.052500in}}{\pgfqpoint{0.360611in}{1.046968in}}%
\pgfpathcurveto{\pgfqpoint{0.370456in}{1.037123in}}{\pgfqpoint{0.380302in}{1.027278in}}{\pgfqpoint{0.385833in}{1.013923in}}%
\pgfpathcurveto{\pgfqpoint{0.385833in}{1.000000in}}{\pgfqpoint{0.385833in}{0.986077in}}{\pgfqpoint{0.380302in}{0.972722in}}%
\pgfpathcurveto{\pgfqpoint{0.370456in}{0.962877in}}{\pgfqpoint{0.360611in}{0.953032in}}{\pgfqpoint{0.347256in}{0.947500in}}%
\pgfpathclose%
\pgfpathmoveto{\pgfqpoint{0.500000in}{0.941667in}}%
\pgfpathcurveto{\pgfqpoint{0.515470in}{0.941667in}}{\pgfqpoint{0.530309in}{0.947813in}}{\pgfqpoint{0.541248in}{0.958752in}}%
\pgfpathcurveto{\pgfqpoint{0.552187in}{0.969691in}}{\pgfqpoint{0.558333in}{0.984530in}}{\pgfqpoint{0.558333in}{1.000000in}}%
\pgfpathcurveto{\pgfqpoint{0.558333in}{1.015470in}}{\pgfqpoint{0.552187in}{1.030309in}}{\pgfqpoint{0.541248in}{1.041248in}}%
\pgfpathcurveto{\pgfqpoint{0.530309in}{1.052187in}}{\pgfqpoint{0.515470in}{1.058333in}}{\pgfqpoint{0.500000in}{1.058333in}}%
\pgfpathcurveto{\pgfqpoint{0.484530in}{1.058333in}}{\pgfqpoint{0.469691in}{1.052187in}}{\pgfqpoint{0.458752in}{1.041248in}}%
\pgfpathcurveto{\pgfqpoint{0.447813in}{1.030309in}}{\pgfqpoint{0.441667in}{1.015470in}}{\pgfqpoint{0.441667in}{1.000000in}}%
\pgfpathcurveto{\pgfqpoint{0.441667in}{0.984530in}}{\pgfqpoint{0.447813in}{0.969691in}}{\pgfqpoint{0.458752in}{0.958752in}}%
\pgfpathcurveto{\pgfqpoint{0.469691in}{0.947813in}}{\pgfqpoint{0.484530in}{0.941667in}}{\pgfqpoint{0.500000in}{0.941667in}}%
\pgfpathclose%
\pgfpathmoveto{\pgfqpoint{0.500000in}{0.947500in}}%
\pgfpathcurveto{\pgfqpoint{0.500000in}{0.947500in}}{\pgfqpoint{0.486077in}{0.947500in}}{\pgfqpoint{0.472722in}{0.953032in}}%
\pgfpathcurveto{\pgfqpoint{0.462877in}{0.962877in}}{\pgfqpoint{0.453032in}{0.972722in}}{\pgfqpoint{0.447500in}{0.986077in}}%
\pgfpathcurveto{\pgfqpoint{0.447500in}{1.000000in}}{\pgfqpoint{0.447500in}{1.013923in}}{\pgfqpoint{0.453032in}{1.027278in}}%
\pgfpathcurveto{\pgfqpoint{0.462877in}{1.037123in}}{\pgfqpoint{0.472722in}{1.046968in}}{\pgfqpoint{0.486077in}{1.052500in}}%
\pgfpathcurveto{\pgfqpoint{0.500000in}{1.052500in}}{\pgfqpoint{0.513923in}{1.052500in}}{\pgfqpoint{0.527278in}{1.046968in}}%
\pgfpathcurveto{\pgfqpoint{0.537123in}{1.037123in}}{\pgfqpoint{0.546968in}{1.027278in}}{\pgfqpoint{0.552500in}{1.013923in}}%
\pgfpathcurveto{\pgfqpoint{0.552500in}{1.000000in}}{\pgfqpoint{0.552500in}{0.986077in}}{\pgfqpoint{0.546968in}{0.972722in}}%
\pgfpathcurveto{\pgfqpoint{0.537123in}{0.962877in}}{\pgfqpoint{0.527278in}{0.953032in}}{\pgfqpoint{0.513923in}{0.947500in}}%
\pgfpathclose%
\pgfpathmoveto{\pgfqpoint{0.666667in}{0.941667in}}%
\pgfpathcurveto{\pgfqpoint{0.682137in}{0.941667in}}{\pgfqpoint{0.696975in}{0.947813in}}{\pgfqpoint{0.707915in}{0.958752in}}%
\pgfpathcurveto{\pgfqpoint{0.718854in}{0.969691in}}{\pgfqpoint{0.725000in}{0.984530in}}{\pgfqpoint{0.725000in}{1.000000in}}%
\pgfpathcurveto{\pgfqpoint{0.725000in}{1.015470in}}{\pgfqpoint{0.718854in}{1.030309in}}{\pgfqpoint{0.707915in}{1.041248in}}%
\pgfpathcurveto{\pgfqpoint{0.696975in}{1.052187in}}{\pgfqpoint{0.682137in}{1.058333in}}{\pgfqpoint{0.666667in}{1.058333in}}%
\pgfpathcurveto{\pgfqpoint{0.651196in}{1.058333in}}{\pgfqpoint{0.636358in}{1.052187in}}{\pgfqpoint{0.625419in}{1.041248in}}%
\pgfpathcurveto{\pgfqpoint{0.614480in}{1.030309in}}{\pgfqpoint{0.608333in}{1.015470in}}{\pgfqpoint{0.608333in}{1.000000in}}%
\pgfpathcurveto{\pgfqpoint{0.608333in}{0.984530in}}{\pgfqpoint{0.614480in}{0.969691in}}{\pgfqpoint{0.625419in}{0.958752in}}%
\pgfpathcurveto{\pgfqpoint{0.636358in}{0.947813in}}{\pgfqpoint{0.651196in}{0.941667in}}{\pgfqpoint{0.666667in}{0.941667in}}%
\pgfpathclose%
\pgfpathmoveto{\pgfqpoint{0.666667in}{0.947500in}}%
\pgfpathcurveto{\pgfqpoint{0.666667in}{0.947500in}}{\pgfqpoint{0.652744in}{0.947500in}}{\pgfqpoint{0.639389in}{0.953032in}}%
\pgfpathcurveto{\pgfqpoint{0.629544in}{0.962877in}}{\pgfqpoint{0.619698in}{0.972722in}}{\pgfqpoint{0.614167in}{0.986077in}}%
\pgfpathcurveto{\pgfqpoint{0.614167in}{1.000000in}}{\pgfqpoint{0.614167in}{1.013923in}}{\pgfqpoint{0.619698in}{1.027278in}}%
\pgfpathcurveto{\pgfqpoint{0.629544in}{1.037123in}}{\pgfqpoint{0.639389in}{1.046968in}}{\pgfqpoint{0.652744in}{1.052500in}}%
\pgfpathcurveto{\pgfqpoint{0.666667in}{1.052500in}}{\pgfqpoint{0.680590in}{1.052500in}}{\pgfqpoint{0.693945in}{1.046968in}}%
\pgfpathcurveto{\pgfqpoint{0.703790in}{1.037123in}}{\pgfqpoint{0.713635in}{1.027278in}}{\pgfqpoint{0.719167in}{1.013923in}}%
\pgfpathcurveto{\pgfqpoint{0.719167in}{1.000000in}}{\pgfqpoint{0.719167in}{0.986077in}}{\pgfqpoint{0.713635in}{0.972722in}}%
\pgfpathcurveto{\pgfqpoint{0.703790in}{0.962877in}}{\pgfqpoint{0.693945in}{0.953032in}}{\pgfqpoint{0.680590in}{0.947500in}}%
\pgfpathclose%
\pgfpathmoveto{\pgfqpoint{0.833333in}{0.941667in}}%
\pgfpathcurveto{\pgfqpoint{0.848804in}{0.941667in}}{\pgfqpoint{0.863642in}{0.947813in}}{\pgfqpoint{0.874581in}{0.958752in}}%
\pgfpathcurveto{\pgfqpoint{0.885520in}{0.969691in}}{\pgfqpoint{0.891667in}{0.984530in}}{\pgfqpoint{0.891667in}{1.000000in}}%
\pgfpathcurveto{\pgfqpoint{0.891667in}{1.015470in}}{\pgfqpoint{0.885520in}{1.030309in}}{\pgfqpoint{0.874581in}{1.041248in}}%
\pgfpathcurveto{\pgfqpoint{0.863642in}{1.052187in}}{\pgfqpoint{0.848804in}{1.058333in}}{\pgfqpoint{0.833333in}{1.058333in}}%
\pgfpathcurveto{\pgfqpoint{0.817863in}{1.058333in}}{\pgfqpoint{0.803025in}{1.052187in}}{\pgfqpoint{0.792085in}{1.041248in}}%
\pgfpathcurveto{\pgfqpoint{0.781146in}{1.030309in}}{\pgfqpoint{0.775000in}{1.015470in}}{\pgfqpoint{0.775000in}{1.000000in}}%
\pgfpathcurveto{\pgfqpoint{0.775000in}{0.984530in}}{\pgfqpoint{0.781146in}{0.969691in}}{\pgfqpoint{0.792085in}{0.958752in}}%
\pgfpathcurveto{\pgfqpoint{0.803025in}{0.947813in}}{\pgfqpoint{0.817863in}{0.941667in}}{\pgfqpoint{0.833333in}{0.941667in}}%
\pgfpathclose%
\pgfpathmoveto{\pgfqpoint{0.833333in}{0.947500in}}%
\pgfpathcurveto{\pgfqpoint{0.833333in}{0.947500in}}{\pgfqpoint{0.819410in}{0.947500in}}{\pgfqpoint{0.806055in}{0.953032in}}%
\pgfpathcurveto{\pgfqpoint{0.796210in}{0.962877in}}{\pgfqpoint{0.786365in}{0.972722in}}{\pgfqpoint{0.780833in}{0.986077in}}%
\pgfpathcurveto{\pgfqpoint{0.780833in}{1.000000in}}{\pgfqpoint{0.780833in}{1.013923in}}{\pgfqpoint{0.786365in}{1.027278in}}%
\pgfpathcurveto{\pgfqpoint{0.796210in}{1.037123in}}{\pgfqpoint{0.806055in}{1.046968in}}{\pgfqpoint{0.819410in}{1.052500in}}%
\pgfpathcurveto{\pgfqpoint{0.833333in}{1.052500in}}{\pgfqpoint{0.847256in}{1.052500in}}{\pgfqpoint{0.860611in}{1.046968in}}%
\pgfpathcurveto{\pgfqpoint{0.870456in}{1.037123in}}{\pgfqpoint{0.880302in}{1.027278in}}{\pgfqpoint{0.885833in}{1.013923in}}%
\pgfpathcurveto{\pgfqpoint{0.885833in}{1.000000in}}{\pgfqpoint{0.885833in}{0.986077in}}{\pgfqpoint{0.880302in}{0.972722in}}%
\pgfpathcurveto{\pgfqpoint{0.870456in}{0.962877in}}{\pgfqpoint{0.860611in}{0.953032in}}{\pgfqpoint{0.847256in}{0.947500in}}%
\pgfpathclose%
\pgfpathmoveto{\pgfqpoint{1.000000in}{0.941667in}}%
\pgfpathcurveto{\pgfqpoint{1.015470in}{0.941667in}}{\pgfqpoint{1.030309in}{0.947813in}}{\pgfqpoint{1.041248in}{0.958752in}}%
\pgfpathcurveto{\pgfqpoint{1.052187in}{0.969691in}}{\pgfqpoint{1.058333in}{0.984530in}}{\pgfqpoint{1.058333in}{1.000000in}}%
\pgfpathcurveto{\pgfqpoint{1.058333in}{1.015470in}}{\pgfqpoint{1.052187in}{1.030309in}}{\pgfqpoint{1.041248in}{1.041248in}}%
\pgfpathcurveto{\pgfqpoint{1.030309in}{1.052187in}}{\pgfqpoint{1.015470in}{1.058333in}}{\pgfqpoint{1.000000in}{1.058333in}}%
\pgfpathcurveto{\pgfqpoint{0.984530in}{1.058333in}}{\pgfqpoint{0.969691in}{1.052187in}}{\pgfqpoint{0.958752in}{1.041248in}}%
\pgfpathcurveto{\pgfqpoint{0.947813in}{1.030309in}}{\pgfqpoint{0.941667in}{1.015470in}}{\pgfqpoint{0.941667in}{1.000000in}}%
\pgfpathcurveto{\pgfqpoint{0.941667in}{0.984530in}}{\pgfqpoint{0.947813in}{0.969691in}}{\pgfqpoint{0.958752in}{0.958752in}}%
\pgfpathcurveto{\pgfqpoint{0.969691in}{0.947813in}}{\pgfqpoint{0.984530in}{0.941667in}}{\pgfqpoint{1.000000in}{0.941667in}}%
\pgfpathclose%
\pgfpathmoveto{\pgfqpoint{1.000000in}{0.947500in}}%
\pgfpathcurveto{\pgfqpoint{1.000000in}{0.947500in}}{\pgfqpoint{0.986077in}{0.947500in}}{\pgfqpoint{0.972722in}{0.953032in}}%
\pgfpathcurveto{\pgfqpoint{0.962877in}{0.962877in}}{\pgfqpoint{0.953032in}{0.972722in}}{\pgfqpoint{0.947500in}{0.986077in}}%
\pgfpathcurveto{\pgfqpoint{0.947500in}{1.000000in}}{\pgfqpoint{0.947500in}{1.013923in}}{\pgfqpoint{0.953032in}{1.027278in}}%
\pgfpathcurveto{\pgfqpoint{0.962877in}{1.037123in}}{\pgfqpoint{0.972722in}{1.046968in}}{\pgfqpoint{0.986077in}{1.052500in}}%
\pgfpathcurveto{\pgfqpoint{1.000000in}{1.052500in}}{\pgfqpoint{1.013923in}{1.052500in}}{\pgfqpoint{1.027278in}{1.046968in}}%
\pgfpathcurveto{\pgfqpoint{1.037123in}{1.037123in}}{\pgfqpoint{1.046968in}{1.027278in}}{\pgfqpoint{1.052500in}{1.013923in}}%
\pgfpathcurveto{\pgfqpoint{1.052500in}{1.000000in}}{\pgfqpoint{1.052500in}{0.986077in}}{\pgfqpoint{1.046968in}{0.972722in}}%
\pgfpathcurveto{\pgfqpoint{1.037123in}{0.962877in}}{\pgfqpoint{1.027278in}{0.953032in}}{\pgfqpoint{1.013923in}{0.947500in}}%
\pgfpathclose%
\pgfusepath{stroke}%
\end{pgfscope}%
}%
\pgfsys@transformshift{9.073315in}{3.003473in}%
\pgfsys@useobject{currentpattern}{}%
\pgfsys@transformshift{1in}{0in}%
\pgfsys@transformshift{-1in}{0in}%
\pgfsys@transformshift{0in}{1in}%
\pgfsys@useobject{currentpattern}{}%
\pgfsys@transformshift{1in}{0in}%
\pgfsys@transformshift{-1in}{0in}%
\pgfsys@transformshift{0in}{1in}%
\pgfsys@useobject{currentpattern}{}%
\pgfsys@transformshift{1in}{0in}%
\pgfsys@transformshift{-1in}{0in}%
\pgfsys@transformshift{0in}{1in}%
\end{pgfscope}%
\begin{pgfscope}%
\pgfpathrectangle{\pgfqpoint{0.935815in}{0.637495in}}{\pgfqpoint{9.300000in}{9.060000in}}%
\pgfusepath{clip}%
\pgfsetbuttcap%
\pgfsetmiterjoin%
\definecolor{currentfill}{rgb}{0.411765,0.411765,0.411765}%
\pgfsetfillcolor{currentfill}%
\pgfsetfillopacity{0.990000}%
\pgfsetlinewidth{0.000000pt}%
\definecolor{currentstroke}{rgb}{0.000000,0.000000,0.000000}%
\pgfsetstrokecolor{currentstroke}%
\pgfsetstrokeopacity{0.990000}%
\pgfsetdash{}{0pt}%
\pgfpathmoveto{\pgfqpoint{1.323315in}{0.637495in}}%
\pgfpathlineto{\pgfqpoint{2.098315in}{0.637495in}}%
\pgfpathlineto{\pgfqpoint{2.098315in}{0.637495in}}%
\pgfpathlineto{\pgfqpoint{1.323315in}{0.637495in}}%
\pgfpathclose%
\pgfusepath{fill}%
\end{pgfscope}%
\begin{pgfscope}%
\pgfsetbuttcap%
\pgfsetmiterjoin%
\definecolor{currentfill}{rgb}{0.411765,0.411765,0.411765}%
\pgfsetfillcolor{currentfill}%
\pgfsetfillopacity{0.990000}%
\pgfsetlinewidth{0.000000pt}%
\definecolor{currentstroke}{rgb}{0.000000,0.000000,0.000000}%
\pgfsetstrokecolor{currentstroke}%
\pgfsetstrokeopacity{0.990000}%
\pgfsetdash{}{0pt}%
\pgfpathrectangle{\pgfqpoint{0.935815in}{0.637495in}}{\pgfqpoint{9.300000in}{9.060000in}}%
\pgfusepath{clip}%
\pgfpathmoveto{\pgfqpoint{1.323315in}{0.637495in}}%
\pgfpathlineto{\pgfqpoint{2.098315in}{0.637495in}}%
\pgfpathlineto{\pgfqpoint{2.098315in}{0.637495in}}%
\pgfpathlineto{\pgfqpoint{1.323315in}{0.637495in}}%
\pgfpathclose%
\pgfusepath{clip}%
\pgfsys@defobject{currentpattern}{\pgfqpoint{0in}{0in}}{\pgfqpoint{1in}{1in}}{%
\begin{pgfscope}%
\pgfpathrectangle{\pgfqpoint{0in}{0in}}{\pgfqpoint{1in}{1in}}%
\pgfusepath{clip}%
\pgfpathmoveto{\pgfqpoint{-0.500000in}{0.500000in}}%
\pgfpathlineto{\pgfqpoint{0.500000in}{1.500000in}}%
\pgfpathmoveto{\pgfqpoint{-0.333333in}{0.333333in}}%
\pgfpathlineto{\pgfqpoint{0.666667in}{1.333333in}}%
\pgfpathmoveto{\pgfqpoint{-0.166667in}{0.166667in}}%
\pgfpathlineto{\pgfqpoint{0.833333in}{1.166667in}}%
\pgfpathmoveto{\pgfqpoint{0.000000in}{0.000000in}}%
\pgfpathlineto{\pgfqpoint{1.000000in}{1.000000in}}%
\pgfpathmoveto{\pgfqpoint{0.166667in}{-0.166667in}}%
\pgfpathlineto{\pgfqpoint{1.166667in}{0.833333in}}%
\pgfpathmoveto{\pgfqpoint{0.333333in}{-0.333333in}}%
\pgfpathlineto{\pgfqpoint{1.333333in}{0.666667in}}%
\pgfpathmoveto{\pgfqpoint{0.500000in}{-0.500000in}}%
\pgfpathlineto{\pgfqpoint{1.500000in}{0.500000in}}%
\pgfusepath{stroke}%
\end{pgfscope}%
}%
\pgfsys@transformshift{1.323315in}{0.637495in}%
\end{pgfscope}%
\begin{pgfscope}%
\pgfpathrectangle{\pgfqpoint{0.935815in}{0.637495in}}{\pgfqpoint{9.300000in}{9.060000in}}%
\pgfusepath{clip}%
\pgfsetbuttcap%
\pgfsetmiterjoin%
\definecolor{currentfill}{rgb}{0.411765,0.411765,0.411765}%
\pgfsetfillcolor{currentfill}%
\pgfsetfillopacity{0.990000}%
\pgfsetlinewidth{0.000000pt}%
\definecolor{currentstroke}{rgb}{0.000000,0.000000,0.000000}%
\pgfsetstrokecolor{currentstroke}%
\pgfsetstrokeopacity{0.990000}%
\pgfsetdash{}{0pt}%
\pgfpathmoveto{\pgfqpoint{2.873315in}{0.637495in}}%
\pgfpathlineto{\pgfqpoint{3.648315in}{0.637495in}}%
\pgfpathlineto{\pgfqpoint{3.648315in}{0.637495in}}%
\pgfpathlineto{\pgfqpoint{2.873315in}{0.637495in}}%
\pgfpathclose%
\pgfusepath{fill}%
\end{pgfscope}%
\begin{pgfscope}%
\pgfsetbuttcap%
\pgfsetmiterjoin%
\definecolor{currentfill}{rgb}{0.411765,0.411765,0.411765}%
\pgfsetfillcolor{currentfill}%
\pgfsetfillopacity{0.990000}%
\pgfsetlinewidth{0.000000pt}%
\definecolor{currentstroke}{rgb}{0.000000,0.000000,0.000000}%
\pgfsetstrokecolor{currentstroke}%
\pgfsetstrokeopacity{0.990000}%
\pgfsetdash{}{0pt}%
\pgfpathrectangle{\pgfqpoint{0.935815in}{0.637495in}}{\pgfqpoint{9.300000in}{9.060000in}}%
\pgfusepath{clip}%
\pgfpathmoveto{\pgfqpoint{2.873315in}{0.637495in}}%
\pgfpathlineto{\pgfqpoint{3.648315in}{0.637495in}}%
\pgfpathlineto{\pgfqpoint{3.648315in}{0.637495in}}%
\pgfpathlineto{\pgfqpoint{2.873315in}{0.637495in}}%
\pgfpathclose%
\pgfusepath{clip}%
\pgfsys@defobject{currentpattern}{\pgfqpoint{0in}{0in}}{\pgfqpoint{1in}{1in}}{%
\begin{pgfscope}%
\pgfpathrectangle{\pgfqpoint{0in}{0in}}{\pgfqpoint{1in}{1in}}%
\pgfusepath{clip}%
\pgfpathmoveto{\pgfqpoint{-0.500000in}{0.500000in}}%
\pgfpathlineto{\pgfqpoint{0.500000in}{1.500000in}}%
\pgfpathmoveto{\pgfqpoint{-0.333333in}{0.333333in}}%
\pgfpathlineto{\pgfqpoint{0.666667in}{1.333333in}}%
\pgfpathmoveto{\pgfqpoint{-0.166667in}{0.166667in}}%
\pgfpathlineto{\pgfqpoint{0.833333in}{1.166667in}}%
\pgfpathmoveto{\pgfqpoint{0.000000in}{0.000000in}}%
\pgfpathlineto{\pgfqpoint{1.000000in}{1.000000in}}%
\pgfpathmoveto{\pgfqpoint{0.166667in}{-0.166667in}}%
\pgfpathlineto{\pgfqpoint{1.166667in}{0.833333in}}%
\pgfpathmoveto{\pgfqpoint{0.333333in}{-0.333333in}}%
\pgfpathlineto{\pgfqpoint{1.333333in}{0.666667in}}%
\pgfpathmoveto{\pgfqpoint{0.500000in}{-0.500000in}}%
\pgfpathlineto{\pgfqpoint{1.500000in}{0.500000in}}%
\pgfusepath{stroke}%
\end{pgfscope}%
}%
\pgfsys@transformshift{2.873315in}{0.637495in}%
\end{pgfscope}%
\begin{pgfscope}%
\pgfpathrectangle{\pgfqpoint{0.935815in}{0.637495in}}{\pgfqpoint{9.300000in}{9.060000in}}%
\pgfusepath{clip}%
\pgfsetbuttcap%
\pgfsetmiterjoin%
\definecolor{currentfill}{rgb}{0.411765,0.411765,0.411765}%
\pgfsetfillcolor{currentfill}%
\pgfsetfillopacity{0.990000}%
\pgfsetlinewidth{0.000000pt}%
\definecolor{currentstroke}{rgb}{0.000000,0.000000,0.000000}%
\pgfsetstrokecolor{currentstroke}%
\pgfsetstrokeopacity{0.990000}%
\pgfsetdash{}{0pt}%
\pgfpathmoveto{\pgfqpoint{4.423315in}{0.637495in}}%
\pgfpathlineto{\pgfqpoint{5.198315in}{0.637495in}}%
\pgfpathlineto{\pgfqpoint{5.198315in}{0.637495in}}%
\pgfpathlineto{\pgfqpoint{4.423315in}{0.637495in}}%
\pgfpathclose%
\pgfusepath{fill}%
\end{pgfscope}%
\begin{pgfscope}%
\pgfsetbuttcap%
\pgfsetmiterjoin%
\definecolor{currentfill}{rgb}{0.411765,0.411765,0.411765}%
\pgfsetfillcolor{currentfill}%
\pgfsetfillopacity{0.990000}%
\pgfsetlinewidth{0.000000pt}%
\definecolor{currentstroke}{rgb}{0.000000,0.000000,0.000000}%
\pgfsetstrokecolor{currentstroke}%
\pgfsetstrokeopacity{0.990000}%
\pgfsetdash{}{0pt}%
\pgfpathrectangle{\pgfqpoint{0.935815in}{0.637495in}}{\pgfqpoint{9.300000in}{9.060000in}}%
\pgfusepath{clip}%
\pgfpathmoveto{\pgfqpoint{4.423315in}{0.637495in}}%
\pgfpathlineto{\pgfqpoint{5.198315in}{0.637495in}}%
\pgfpathlineto{\pgfqpoint{5.198315in}{0.637495in}}%
\pgfpathlineto{\pgfqpoint{4.423315in}{0.637495in}}%
\pgfpathclose%
\pgfusepath{clip}%
\pgfsys@defobject{currentpattern}{\pgfqpoint{0in}{0in}}{\pgfqpoint{1in}{1in}}{%
\begin{pgfscope}%
\pgfpathrectangle{\pgfqpoint{0in}{0in}}{\pgfqpoint{1in}{1in}}%
\pgfusepath{clip}%
\pgfpathmoveto{\pgfqpoint{-0.500000in}{0.500000in}}%
\pgfpathlineto{\pgfqpoint{0.500000in}{1.500000in}}%
\pgfpathmoveto{\pgfqpoint{-0.333333in}{0.333333in}}%
\pgfpathlineto{\pgfqpoint{0.666667in}{1.333333in}}%
\pgfpathmoveto{\pgfqpoint{-0.166667in}{0.166667in}}%
\pgfpathlineto{\pgfqpoint{0.833333in}{1.166667in}}%
\pgfpathmoveto{\pgfqpoint{0.000000in}{0.000000in}}%
\pgfpathlineto{\pgfqpoint{1.000000in}{1.000000in}}%
\pgfpathmoveto{\pgfqpoint{0.166667in}{-0.166667in}}%
\pgfpathlineto{\pgfqpoint{1.166667in}{0.833333in}}%
\pgfpathmoveto{\pgfqpoint{0.333333in}{-0.333333in}}%
\pgfpathlineto{\pgfqpoint{1.333333in}{0.666667in}}%
\pgfpathmoveto{\pgfqpoint{0.500000in}{-0.500000in}}%
\pgfpathlineto{\pgfqpoint{1.500000in}{0.500000in}}%
\pgfusepath{stroke}%
\end{pgfscope}%
}%
\pgfsys@transformshift{4.423315in}{0.637495in}%
\end{pgfscope}%
\begin{pgfscope}%
\pgfpathrectangle{\pgfqpoint{0.935815in}{0.637495in}}{\pgfqpoint{9.300000in}{9.060000in}}%
\pgfusepath{clip}%
\pgfsetbuttcap%
\pgfsetmiterjoin%
\definecolor{currentfill}{rgb}{0.411765,0.411765,0.411765}%
\pgfsetfillcolor{currentfill}%
\pgfsetfillopacity{0.990000}%
\pgfsetlinewidth{0.000000pt}%
\definecolor{currentstroke}{rgb}{0.000000,0.000000,0.000000}%
\pgfsetstrokecolor{currentstroke}%
\pgfsetstrokeopacity{0.990000}%
\pgfsetdash{}{0pt}%
\pgfpathmoveto{\pgfqpoint{5.973315in}{0.637495in}}%
\pgfpathlineto{\pgfqpoint{6.748315in}{0.637495in}}%
\pgfpathlineto{\pgfqpoint{6.748315in}{0.637495in}}%
\pgfpathlineto{\pgfqpoint{5.973315in}{0.637495in}}%
\pgfpathclose%
\pgfusepath{fill}%
\end{pgfscope}%
\begin{pgfscope}%
\pgfsetbuttcap%
\pgfsetmiterjoin%
\definecolor{currentfill}{rgb}{0.411765,0.411765,0.411765}%
\pgfsetfillcolor{currentfill}%
\pgfsetfillopacity{0.990000}%
\pgfsetlinewidth{0.000000pt}%
\definecolor{currentstroke}{rgb}{0.000000,0.000000,0.000000}%
\pgfsetstrokecolor{currentstroke}%
\pgfsetstrokeopacity{0.990000}%
\pgfsetdash{}{0pt}%
\pgfpathrectangle{\pgfqpoint{0.935815in}{0.637495in}}{\pgfqpoint{9.300000in}{9.060000in}}%
\pgfusepath{clip}%
\pgfpathmoveto{\pgfqpoint{5.973315in}{0.637495in}}%
\pgfpathlineto{\pgfqpoint{6.748315in}{0.637495in}}%
\pgfpathlineto{\pgfqpoint{6.748315in}{0.637495in}}%
\pgfpathlineto{\pgfqpoint{5.973315in}{0.637495in}}%
\pgfpathclose%
\pgfusepath{clip}%
\pgfsys@defobject{currentpattern}{\pgfqpoint{0in}{0in}}{\pgfqpoint{1in}{1in}}{%
\begin{pgfscope}%
\pgfpathrectangle{\pgfqpoint{0in}{0in}}{\pgfqpoint{1in}{1in}}%
\pgfusepath{clip}%
\pgfpathmoveto{\pgfqpoint{-0.500000in}{0.500000in}}%
\pgfpathlineto{\pgfqpoint{0.500000in}{1.500000in}}%
\pgfpathmoveto{\pgfqpoint{-0.333333in}{0.333333in}}%
\pgfpathlineto{\pgfqpoint{0.666667in}{1.333333in}}%
\pgfpathmoveto{\pgfqpoint{-0.166667in}{0.166667in}}%
\pgfpathlineto{\pgfqpoint{0.833333in}{1.166667in}}%
\pgfpathmoveto{\pgfqpoint{0.000000in}{0.000000in}}%
\pgfpathlineto{\pgfqpoint{1.000000in}{1.000000in}}%
\pgfpathmoveto{\pgfqpoint{0.166667in}{-0.166667in}}%
\pgfpathlineto{\pgfqpoint{1.166667in}{0.833333in}}%
\pgfpathmoveto{\pgfqpoint{0.333333in}{-0.333333in}}%
\pgfpathlineto{\pgfqpoint{1.333333in}{0.666667in}}%
\pgfpathmoveto{\pgfqpoint{0.500000in}{-0.500000in}}%
\pgfpathlineto{\pgfqpoint{1.500000in}{0.500000in}}%
\pgfusepath{stroke}%
\end{pgfscope}%
}%
\pgfsys@transformshift{5.973315in}{0.637495in}%
\end{pgfscope}%
\begin{pgfscope}%
\pgfpathrectangle{\pgfqpoint{0.935815in}{0.637495in}}{\pgfqpoint{9.300000in}{9.060000in}}%
\pgfusepath{clip}%
\pgfsetbuttcap%
\pgfsetmiterjoin%
\definecolor{currentfill}{rgb}{0.411765,0.411765,0.411765}%
\pgfsetfillcolor{currentfill}%
\pgfsetfillopacity{0.990000}%
\pgfsetlinewidth{0.000000pt}%
\definecolor{currentstroke}{rgb}{0.000000,0.000000,0.000000}%
\pgfsetstrokecolor{currentstroke}%
\pgfsetstrokeopacity{0.990000}%
\pgfsetdash{}{0pt}%
\pgfpathmoveto{\pgfqpoint{7.523315in}{0.637495in}}%
\pgfpathlineto{\pgfqpoint{8.298315in}{0.637495in}}%
\pgfpathlineto{\pgfqpoint{8.298315in}{0.637495in}}%
\pgfpathlineto{\pgfqpoint{7.523315in}{0.637495in}}%
\pgfpathclose%
\pgfusepath{fill}%
\end{pgfscope}%
\begin{pgfscope}%
\pgfsetbuttcap%
\pgfsetmiterjoin%
\definecolor{currentfill}{rgb}{0.411765,0.411765,0.411765}%
\pgfsetfillcolor{currentfill}%
\pgfsetfillopacity{0.990000}%
\pgfsetlinewidth{0.000000pt}%
\definecolor{currentstroke}{rgb}{0.000000,0.000000,0.000000}%
\pgfsetstrokecolor{currentstroke}%
\pgfsetstrokeopacity{0.990000}%
\pgfsetdash{}{0pt}%
\pgfpathrectangle{\pgfqpoint{0.935815in}{0.637495in}}{\pgfqpoint{9.300000in}{9.060000in}}%
\pgfusepath{clip}%
\pgfpathmoveto{\pgfqpoint{7.523315in}{0.637495in}}%
\pgfpathlineto{\pgfqpoint{8.298315in}{0.637495in}}%
\pgfpathlineto{\pgfqpoint{8.298315in}{0.637495in}}%
\pgfpathlineto{\pgfqpoint{7.523315in}{0.637495in}}%
\pgfpathclose%
\pgfusepath{clip}%
\pgfsys@defobject{currentpattern}{\pgfqpoint{0in}{0in}}{\pgfqpoint{1in}{1in}}{%
\begin{pgfscope}%
\pgfpathrectangle{\pgfqpoint{0in}{0in}}{\pgfqpoint{1in}{1in}}%
\pgfusepath{clip}%
\pgfpathmoveto{\pgfqpoint{-0.500000in}{0.500000in}}%
\pgfpathlineto{\pgfqpoint{0.500000in}{1.500000in}}%
\pgfpathmoveto{\pgfqpoint{-0.333333in}{0.333333in}}%
\pgfpathlineto{\pgfqpoint{0.666667in}{1.333333in}}%
\pgfpathmoveto{\pgfqpoint{-0.166667in}{0.166667in}}%
\pgfpathlineto{\pgfqpoint{0.833333in}{1.166667in}}%
\pgfpathmoveto{\pgfqpoint{0.000000in}{0.000000in}}%
\pgfpathlineto{\pgfqpoint{1.000000in}{1.000000in}}%
\pgfpathmoveto{\pgfqpoint{0.166667in}{-0.166667in}}%
\pgfpathlineto{\pgfqpoint{1.166667in}{0.833333in}}%
\pgfpathmoveto{\pgfqpoint{0.333333in}{-0.333333in}}%
\pgfpathlineto{\pgfqpoint{1.333333in}{0.666667in}}%
\pgfpathmoveto{\pgfqpoint{0.500000in}{-0.500000in}}%
\pgfpathlineto{\pgfqpoint{1.500000in}{0.500000in}}%
\pgfusepath{stroke}%
\end{pgfscope}%
}%
\pgfsys@transformshift{7.523315in}{0.637495in}%
\end{pgfscope}%
\begin{pgfscope}%
\pgfpathrectangle{\pgfqpoint{0.935815in}{0.637495in}}{\pgfqpoint{9.300000in}{9.060000in}}%
\pgfusepath{clip}%
\pgfsetbuttcap%
\pgfsetmiterjoin%
\definecolor{currentfill}{rgb}{0.411765,0.411765,0.411765}%
\pgfsetfillcolor{currentfill}%
\pgfsetfillopacity{0.990000}%
\pgfsetlinewidth{0.000000pt}%
\definecolor{currentstroke}{rgb}{0.000000,0.000000,0.000000}%
\pgfsetstrokecolor{currentstroke}%
\pgfsetstrokeopacity{0.990000}%
\pgfsetdash{}{0pt}%
\pgfpathmoveto{\pgfqpoint{9.073315in}{0.637495in}}%
\pgfpathlineto{\pgfqpoint{9.848315in}{0.637495in}}%
\pgfpathlineto{\pgfqpoint{9.848315in}{0.637495in}}%
\pgfpathlineto{\pgfqpoint{9.073315in}{0.637495in}}%
\pgfpathclose%
\pgfusepath{fill}%
\end{pgfscope}%
\begin{pgfscope}%
\pgfsetbuttcap%
\pgfsetmiterjoin%
\definecolor{currentfill}{rgb}{0.411765,0.411765,0.411765}%
\pgfsetfillcolor{currentfill}%
\pgfsetfillopacity{0.990000}%
\pgfsetlinewidth{0.000000pt}%
\definecolor{currentstroke}{rgb}{0.000000,0.000000,0.000000}%
\pgfsetstrokecolor{currentstroke}%
\pgfsetstrokeopacity{0.990000}%
\pgfsetdash{}{0pt}%
\pgfpathrectangle{\pgfqpoint{0.935815in}{0.637495in}}{\pgfqpoint{9.300000in}{9.060000in}}%
\pgfusepath{clip}%
\pgfpathmoveto{\pgfqpoint{9.073315in}{0.637495in}}%
\pgfpathlineto{\pgfqpoint{9.848315in}{0.637495in}}%
\pgfpathlineto{\pgfqpoint{9.848315in}{0.637495in}}%
\pgfpathlineto{\pgfqpoint{9.073315in}{0.637495in}}%
\pgfpathclose%
\pgfusepath{clip}%
\pgfsys@defobject{currentpattern}{\pgfqpoint{0in}{0in}}{\pgfqpoint{1in}{1in}}{%
\begin{pgfscope}%
\pgfpathrectangle{\pgfqpoint{0in}{0in}}{\pgfqpoint{1in}{1in}}%
\pgfusepath{clip}%
\pgfpathmoveto{\pgfqpoint{-0.500000in}{0.500000in}}%
\pgfpathlineto{\pgfqpoint{0.500000in}{1.500000in}}%
\pgfpathmoveto{\pgfqpoint{-0.333333in}{0.333333in}}%
\pgfpathlineto{\pgfqpoint{0.666667in}{1.333333in}}%
\pgfpathmoveto{\pgfqpoint{-0.166667in}{0.166667in}}%
\pgfpathlineto{\pgfqpoint{0.833333in}{1.166667in}}%
\pgfpathmoveto{\pgfqpoint{0.000000in}{0.000000in}}%
\pgfpathlineto{\pgfqpoint{1.000000in}{1.000000in}}%
\pgfpathmoveto{\pgfqpoint{0.166667in}{-0.166667in}}%
\pgfpathlineto{\pgfqpoint{1.166667in}{0.833333in}}%
\pgfpathmoveto{\pgfqpoint{0.333333in}{-0.333333in}}%
\pgfpathlineto{\pgfqpoint{1.333333in}{0.666667in}}%
\pgfpathmoveto{\pgfqpoint{0.500000in}{-0.500000in}}%
\pgfpathlineto{\pgfqpoint{1.500000in}{0.500000in}}%
\pgfusepath{stroke}%
\end{pgfscope}%
}%
\pgfsys@transformshift{9.073315in}{0.637495in}%
\end{pgfscope}%
\begin{pgfscope}%
\pgfpathrectangle{\pgfqpoint{0.935815in}{0.637495in}}{\pgfqpoint{9.300000in}{9.060000in}}%
\pgfusepath{clip}%
\pgfsetbuttcap%
\pgfsetmiterjoin%
\definecolor{currentfill}{rgb}{0.172549,0.627451,0.172549}%
\pgfsetfillcolor{currentfill}%
\pgfsetfillopacity{0.990000}%
\pgfsetlinewidth{0.000000pt}%
\definecolor{currentstroke}{rgb}{0.000000,0.000000,0.000000}%
\pgfsetstrokecolor{currentstroke}%
\pgfsetstrokeopacity{0.990000}%
\pgfsetdash{}{0pt}%
\pgfpathmoveto{\pgfqpoint{1.323315in}{2.460694in}}%
\pgfpathlineto{\pgfqpoint{2.098315in}{2.460694in}}%
\pgfpathlineto{\pgfqpoint{2.098315in}{2.557723in}}%
\pgfpathlineto{\pgfqpoint{1.323315in}{2.557723in}}%
\pgfpathclose%
\pgfusepath{fill}%
\end{pgfscope}%
\begin{pgfscope}%
\pgfsetbuttcap%
\pgfsetmiterjoin%
\definecolor{currentfill}{rgb}{0.172549,0.627451,0.172549}%
\pgfsetfillcolor{currentfill}%
\pgfsetfillopacity{0.990000}%
\pgfsetlinewidth{0.000000pt}%
\definecolor{currentstroke}{rgb}{0.000000,0.000000,0.000000}%
\pgfsetstrokecolor{currentstroke}%
\pgfsetstrokeopacity{0.990000}%
\pgfsetdash{}{0pt}%
\pgfpathrectangle{\pgfqpoint{0.935815in}{0.637495in}}{\pgfqpoint{9.300000in}{9.060000in}}%
\pgfusepath{clip}%
\pgfpathmoveto{\pgfqpoint{1.323315in}{2.460694in}}%
\pgfpathlineto{\pgfqpoint{2.098315in}{2.460694in}}%
\pgfpathlineto{\pgfqpoint{2.098315in}{2.557723in}}%
\pgfpathlineto{\pgfqpoint{1.323315in}{2.557723in}}%
\pgfpathclose%
\pgfusepath{clip}%
\pgfsys@defobject{currentpattern}{\pgfqpoint{0in}{0in}}{\pgfqpoint{1in}{1in}}{%
\begin{pgfscope}%
\pgfpathrectangle{\pgfqpoint{0in}{0in}}{\pgfqpoint{1in}{1in}}%
\pgfusepath{clip}%
\pgfpathmoveto{\pgfqpoint{0.000000in}{-0.016667in}}%
\pgfpathcurveto{\pgfqpoint{0.004420in}{-0.016667in}}{\pgfqpoint{0.008660in}{-0.014911in}}{\pgfqpoint{0.011785in}{-0.011785in}}%
\pgfpathcurveto{\pgfqpoint{0.014911in}{-0.008660in}}{\pgfqpoint{0.016667in}{-0.004420in}}{\pgfqpoint{0.016667in}{0.000000in}}%
\pgfpathcurveto{\pgfqpoint{0.016667in}{0.004420in}}{\pgfqpoint{0.014911in}{0.008660in}}{\pgfqpoint{0.011785in}{0.011785in}}%
\pgfpathcurveto{\pgfqpoint{0.008660in}{0.014911in}}{\pgfqpoint{0.004420in}{0.016667in}}{\pgfqpoint{0.000000in}{0.016667in}}%
\pgfpathcurveto{\pgfqpoint{-0.004420in}{0.016667in}}{\pgfqpoint{-0.008660in}{0.014911in}}{\pgfqpoint{-0.011785in}{0.011785in}}%
\pgfpathcurveto{\pgfqpoint{-0.014911in}{0.008660in}}{\pgfqpoint{-0.016667in}{0.004420in}}{\pgfqpoint{-0.016667in}{0.000000in}}%
\pgfpathcurveto{\pgfqpoint{-0.016667in}{-0.004420in}}{\pgfqpoint{-0.014911in}{-0.008660in}}{\pgfqpoint{-0.011785in}{-0.011785in}}%
\pgfpathcurveto{\pgfqpoint{-0.008660in}{-0.014911in}}{\pgfqpoint{-0.004420in}{-0.016667in}}{\pgfqpoint{0.000000in}{-0.016667in}}%
\pgfpathclose%
\pgfpathmoveto{\pgfqpoint{0.166667in}{-0.016667in}}%
\pgfpathcurveto{\pgfqpoint{0.171087in}{-0.016667in}}{\pgfqpoint{0.175326in}{-0.014911in}}{\pgfqpoint{0.178452in}{-0.011785in}}%
\pgfpathcurveto{\pgfqpoint{0.181577in}{-0.008660in}}{\pgfqpoint{0.183333in}{-0.004420in}}{\pgfqpoint{0.183333in}{0.000000in}}%
\pgfpathcurveto{\pgfqpoint{0.183333in}{0.004420in}}{\pgfqpoint{0.181577in}{0.008660in}}{\pgfqpoint{0.178452in}{0.011785in}}%
\pgfpathcurveto{\pgfqpoint{0.175326in}{0.014911in}}{\pgfqpoint{0.171087in}{0.016667in}}{\pgfqpoint{0.166667in}{0.016667in}}%
\pgfpathcurveto{\pgfqpoint{0.162247in}{0.016667in}}{\pgfqpoint{0.158007in}{0.014911in}}{\pgfqpoint{0.154882in}{0.011785in}}%
\pgfpathcurveto{\pgfqpoint{0.151756in}{0.008660in}}{\pgfqpoint{0.150000in}{0.004420in}}{\pgfqpoint{0.150000in}{0.000000in}}%
\pgfpathcurveto{\pgfqpoint{0.150000in}{-0.004420in}}{\pgfqpoint{0.151756in}{-0.008660in}}{\pgfqpoint{0.154882in}{-0.011785in}}%
\pgfpathcurveto{\pgfqpoint{0.158007in}{-0.014911in}}{\pgfqpoint{0.162247in}{-0.016667in}}{\pgfqpoint{0.166667in}{-0.016667in}}%
\pgfpathclose%
\pgfpathmoveto{\pgfqpoint{0.333333in}{-0.016667in}}%
\pgfpathcurveto{\pgfqpoint{0.337753in}{-0.016667in}}{\pgfqpoint{0.341993in}{-0.014911in}}{\pgfqpoint{0.345118in}{-0.011785in}}%
\pgfpathcurveto{\pgfqpoint{0.348244in}{-0.008660in}}{\pgfqpoint{0.350000in}{-0.004420in}}{\pgfqpoint{0.350000in}{0.000000in}}%
\pgfpathcurveto{\pgfqpoint{0.350000in}{0.004420in}}{\pgfqpoint{0.348244in}{0.008660in}}{\pgfqpoint{0.345118in}{0.011785in}}%
\pgfpathcurveto{\pgfqpoint{0.341993in}{0.014911in}}{\pgfqpoint{0.337753in}{0.016667in}}{\pgfqpoint{0.333333in}{0.016667in}}%
\pgfpathcurveto{\pgfqpoint{0.328913in}{0.016667in}}{\pgfqpoint{0.324674in}{0.014911in}}{\pgfqpoint{0.321548in}{0.011785in}}%
\pgfpathcurveto{\pgfqpoint{0.318423in}{0.008660in}}{\pgfqpoint{0.316667in}{0.004420in}}{\pgfqpoint{0.316667in}{0.000000in}}%
\pgfpathcurveto{\pgfqpoint{0.316667in}{-0.004420in}}{\pgfqpoint{0.318423in}{-0.008660in}}{\pgfqpoint{0.321548in}{-0.011785in}}%
\pgfpathcurveto{\pgfqpoint{0.324674in}{-0.014911in}}{\pgfqpoint{0.328913in}{-0.016667in}}{\pgfqpoint{0.333333in}{-0.016667in}}%
\pgfpathclose%
\pgfpathmoveto{\pgfqpoint{0.500000in}{-0.016667in}}%
\pgfpathcurveto{\pgfqpoint{0.504420in}{-0.016667in}}{\pgfqpoint{0.508660in}{-0.014911in}}{\pgfqpoint{0.511785in}{-0.011785in}}%
\pgfpathcurveto{\pgfqpoint{0.514911in}{-0.008660in}}{\pgfqpoint{0.516667in}{-0.004420in}}{\pgfqpoint{0.516667in}{0.000000in}}%
\pgfpathcurveto{\pgfqpoint{0.516667in}{0.004420in}}{\pgfqpoint{0.514911in}{0.008660in}}{\pgfqpoint{0.511785in}{0.011785in}}%
\pgfpathcurveto{\pgfqpoint{0.508660in}{0.014911in}}{\pgfqpoint{0.504420in}{0.016667in}}{\pgfqpoint{0.500000in}{0.016667in}}%
\pgfpathcurveto{\pgfqpoint{0.495580in}{0.016667in}}{\pgfqpoint{0.491340in}{0.014911in}}{\pgfqpoint{0.488215in}{0.011785in}}%
\pgfpathcurveto{\pgfqpoint{0.485089in}{0.008660in}}{\pgfqpoint{0.483333in}{0.004420in}}{\pgfqpoint{0.483333in}{0.000000in}}%
\pgfpathcurveto{\pgfqpoint{0.483333in}{-0.004420in}}{\pgfqpoint{0.485089in}{-0.008660in}}{\pgfqpoint{0.488215in}{-0.011785in}}%
\pgfpathcurveto{\pgfqpoint{0.491340in}{-0.014911in}}{\pgfqpoint{0.495580in}{-0.016667in}}{\pgfqpoint{0.500000in}{-0.016667in}}%
\pgfpathclose%
\pgfpathmoveto{\pgfqpoint{0.666667in}{-0.016667in}}%
\pgfpathcurveto{\pgfqpoint{0.671087in}{-0.016667in}}{\pgfqpoint{0.675326in}{-0.014911in}}{\pgfqpoint{0.678452in}{-0.011785in}}%
\pgfpathcurveto{\pgfqpoint{0.681577in}{-0.008660in}}{\pgfqpoint{0.683333in}{-0.004420in}}{\pgfqpoint{0.683333in}{0.000000in}}%
\pgfpathcurveto{\pgfqpoint{0.683333in}{0.004420in}}{\pgfqpoint{0.681577in}{0.008660in}}{\pgfqpoint{0.678452in}{0.011785in}}%
\pgfpathcurveto{\pgfqpoint{0.675326in}{0.014911in}}{\pgfqpoint{0.671087in}{0.016667in}}{\pgfqpoint{0.666667in}{0.016667in}}%
\pgfpathcurveto{\pgfqpoint{0.662247in}{0.016667in}}{\pgfqpoint{0.658007in}{0.014911in}}{\pgfqpoint{0.654882in}{0.011785in}}%
\pgfpathcurveto{\pgfqpoint{0.651756in}{0.008660in}}{\pgfqpoint{0.650000in}{0.004420in}}{\pgfqpoint{0.650000in}{0.000000in}}%
\pgfpathcurveto{\pgfqpoint{0.650000in}{-0.004420in}}{\pgfqpoint{0.651756in}{-0.008660in}}{\pgfqpoint{0.654882in}{-0.011785in}}%
\pgfpathcurveto{\pgfqpoint{0.658007in}{-0.014911in}}{\pgfqpoint{0.662247in}{-0.016667in}}{\pgfqpoint{0.666667in}{-0.016667in}}%
\pgfpathclose%
\pgfpathmoveto{\pgfqpoint{0.833333in}{-0.016667in}}%
\pgfpathcurveto{\pgfqpoint{0.837753in}{-0.016667in}}{\pgfqpoint{0.841993in}{-0.014911in}}{\pgfqpoint{0.845118in}{-0.011785in}}%
\pgfpathcurveto{\pgfqpoint{0.848244in}{-0.008660in}}{\pgfqpoint{0.850000in}{-0.004420in}}{\pgfqpoint{0.850000in}{0.000000in}}%
\pgfpathcurveto{\pgfqpoint{0.850000in}{0.004420in}}{\pgfqpoint{0.848244in}{0.008660in}}{\pgfqpoint{0.845118in}{0.011785in}}%
\pgfpathcurveto{\pgfqpoint{0.841993in}{0.014911in}}{\pgfqpoint{0.837753in}{0.016667in}}{\pgfqpoint{0.833333in}{0.016667in}}%
\pgfpathcurveto{\pgfqpoint{0.828913in}{0.016667in}}{\pgfqpoint{0.824674in}{0.014911in}}{\pgfqpoint{0.821548in}{0.011785in}}%
\pgfpathcurveto{\pgfqpoint{0.818423in}{0.008660in}}{\pgfqpoint{0.816667in}{0.004420in}}{\pgfqpoint{0.816667in}{0.000000in}}%
\pgfpathcurveto{\pgfqpoint{0.816667in}{-0.004420in}}{\pgfqpoint{0.818423in}{-0.008660in}}{\pgfqpoint{0.821548in}{-0.011785in}}%
\pgfpathcurveto{\pgfqpoint{0.824674in}{-0.014911in}}{\pgfqpoint{0.828913in}{-0.016667in}}{\pgfqpoint{0.833333in}{-0.016667in}}%
\pgfpathclose%
\pgfpathmoveto{\pgfqpoint{1.000000in}{-0.016667in}}%
\pgfpathcurveto{\pgfqpoint{1.004420in}{-0.016667in}}{\pgfqpoint{1.008660in}{-0.014911in}}{\pgfqpoint{1.011785in}{-0.011785in}}%
\pgfpathcurveto{\pgfqpoint{1.014911in}{-0.008660in}}{\pgfqpoint{1.016667in}{-0.004420in}}{\pgfqpoint{1.016667in}{0.000000in}}%
\pgfpathcurveto{\pgfqpoint{1.016667in}{0.004420in}}{\pgfqpoint{1.014911in}{0.008660in}}{\pgfqpoint{1.011785in}{0.011785in}}%
\pgfpathcurveto{\pgfqpoint{1.008660in}{0.014911in}}{\pgfqpoint{1.004420in}{0.016667in}}{\pgfqpoint{1.000000in}{0.016667in}}%
\pgfpathcurveto{\pgfqpoint{0.995580in}{0.016667in}}{\pgfqpoint{0.991340in}{0.014911in}}{\pgfqpoint{0.988215in}{0.011785in}}%
\pgfpathcurveto{\pgfqpoint{0.985089in}{0.008660in}}{\pgfqpoint{0.983333in}{0.004420in}}{\pgfqpoint{0.983333in}{0.000000in}}%
\pgfpathcurveto{\pgfqpoint{0.983333in}{-0.004420in}}{\pgfqpoint{0.985089in}{-0.008660in}}{\pgfqpoint{0.988215in}{-0.011785in}}%
\pgfpathcurveto{\pgfqpoint{0.991340in}{-0.014911in}}{\pgfqpoint{0.995580in}{-0.016667in}}{\pgfqpoint{1.000000in}{-0.016667in}}%
\pgfpathclose%
\pgfpathmoveto{\pgfqpoint{0.083333in}{0.150000in}}%
\pgfpathcurveto{\pgfqpoint{0.087753in}{0.150000in}}{\pgfqpoint{0.091993in}{0.151756in}}{\pgfqpoint{0.095118in}{0.154882in}}%
\pgfpathcurveto{\pgfqpoint{0.098244in}{0.158007in}}{\pgfqpoint{0.100000in}{0.162247in}}{\pgfqpoint{0.100000in}{0.166667in}}%
\pgfpathcurveto{\pgfqpoint{0.100000in}{0.171087in}}{\pgfqpoint{0.098244in}{0.175326in}}{\pgfqpoint{0.095118in}{0.178452in}}%
\pgfpathcurveto{\pgfqpoint{0.091993in}{0.181577in}}{\pgfqpoint{0.087753in}{0.183333in}}{\pgfqpoint{0.083333in}{0.183333in}}%
\pgfpathcurveto{\pgfqpoint{0.078913in}{0.183333in}}{\pgfqpoint{0.074674in}{0.181577in}}{\pgfqpoint{0.071548in}{0.178452in}}%
\pgfpathcurveto{\pgfqpoint{0.068423in}{0.175326in}}{\pgfqpoint{0.066667in}{0.171087in}}{\pgfqpoint{0.066667in}{0.166667in}}%
\pgfpathcurveto{\pgfqpoint{0.066667in}{0.162247in}}{\pgfqpoint{0.068423in}{0.158007in}}{\pgfqpoint{0.071548in}{0.154882in}}%
\pgfpathcurveto{\pgfqpoint{0.074674in}{0.151756in}}{\pgfqpoint{0.078913in}{0.150000in}}{\pgfqpoint{0.083333in}{0.150000in}}%
\pgfpathclose%
\pgfpathmoveto{\pgfqpoint{0.250000in}{0.150000in}}%
\pgfpathcurveto{\pgfqpoint{0.254420in}{0.150000in}}{\pgfqpoint{0.258660in}{0.151756in}}{\pgfqpoint{0.261785in}{0.154882in}}%
\pgfpathcurveto{\pgfqpoint{0.264911in}{0.158007in}}{\pgfqpoint{0.266667in}{0.162247in}}{\pgfqpoint{0.266667in}{0.166667in}}%
\pgfpathcurveto{\pgfqpoint{0.266667in}{0.171087in}}{\pgfqpoint{0.264911in}{0.175326in}}{\pgfqpoint{0.261785in}{0.178452in}}%
\pgfpathcurveto{\pgfqpoint{0.258660in}{0.181577in}}{\pgfqpoint{0.254420in}{0.183333in}}{\pgfqpoint{0.250000in}{0.183333in}}%
\pgfpathcurveto{\pgfqpoint{0.245580in}{0.183333in}}{\pgfqpoint{0.241340in}{0.181577in}}{\pgfqpoint{0.238215in}{0.178452in}}%
\pgfpathcurveto{\pgfqpoint{0.235089in}{0.175326in}}{\pgfqpoint{0.233333in}{0.171087in}}{\pgfqpoint{0.233333in}{0.166667in}}%
\pgfpathcurveto{\pgfqpoint{0.233333in}{0.162247in}}{\pgfqpoint{0.235089in}{0.158007in}}{\pgfqpoint{0.238215in}{0.154882in}}%
\pgfpathcurveto{\pgfqpoint{0.241340in}{0.151756in}}{\pgfqpoint{0.245580in}{0.150000in}}{\pgfqpoint{0.250000in}{0.150000in}}%
\pgfpathclose%
\pgfpathmoveto{\pgfqpoint{0.416667in}{0.150000in}}%
\pgfpathcurveto{\pgfqpoint{0.421087in}{0.150000in}}{\pgfqpoint{0.425326in}{0.151756in}}{\pgfqpoint{0.428452in}{0.154882in}}%
\pgfpathcurveto{\pgfqpoint{0.431577in}{0.158007in}}{\pgfqpoint{0.433333in}{0.162247in}}{\pgfqpoint{0.433333in}{0.166667in}}%
\pgfpathcurveto{\pgfqpoint{0.433333in}{0.171087in}}{\pgfqpoint{0.431577in}{0.175326in}}{\pgfqpoint{0.428452in}{0.178452in}}%
\pgfpathcurveto{\pgfqpoint{0.425326in}{0.181577in}}{\pgfqpoint{0.421087in}{0.183333in}}{\pgfqpoint{0.416667in}{0.183333in}}%
\pgfpathcurveto{\pgfqpoint{0.412247in}{0.183333in}}{\pgfqpoint{0.408007in}{0.181577in}}{\pgfqpoint{0.404882in}{0.178452in}}%
\pgfpathcurveto{\pgfqpoint{0.401756in}{0.175326in}}{\pgfqpoint{0.400000in}{0.171087in}}{\pgfqpoint{0.400000in}{0.166667in}}%
\pgfpathcurveto{\pgfqpoint{0.400000in}{0.162247in}}{\pgfqpoint{0.401756in}{0.158007in}}{\pgfqpoint{0.404882in}{0.154882in}}%
\pgfpathcurveto{\pgfqpoint{0.408007in}{0.151756in}}{\pgfqpoint{0.412247in}{0.150000in}}{\pgfqpoint{0.416667in}{0.150000in}}%
\pgfpathclose%
\pgfpathmoveto{\pgfqpoint{0.583333in}{0.150000in}}%
\pgfpathcurveto{\pgfqpoint{0.587753in}{0.150000in}}{\pgfqpoint{0.591993in}{0.151756in}}{\pgfqpoint{0.595118in}{0.154882in}}%
\pgfpathcurveto{\pgfqpoint{0.598244in}{0.158007in}}{\pgfqpoint{0.600000in}{0.162247in}}{\pgfqpoint{0.600000in}{0.166667in}}%
\pgfpathcurveto{\pgfqpoint{0.600000in}{0.171087in}}{\pgfqpoint{0.598244in}{0.175326in}}{\pgfqpoint{0.595118in}{0.178452in}}%
\pgfpathcurveto{\pgfqpoint{0.591993in}{0.181577in}}{\pgfqpoint{0.587753in}{0.183333in}}{\pgfqpoint{0.583333in}{0.183333in}}%
\pgfpathcurveto{\pgfqpoint{0.578913in}{0.183333in}}{\pgfqpoint{0.574674in}{0.181577in}}{\pgfqpoint{0.571548in}{0.178452in}}%
\pgfpathcurveto{\pgfqpoint{0.568423in}{0.175326in}}{\pgfqpoint{0.566667in}{0.171087in}}{\pgfqpoint{0.566667in}{0.166667in}}%
\pgfpathcurveto{\pgfqpoint{0.566667in}{0.162247in}}{\pgfqpoint{0.568423in}{0.158007in}}{\pgfqpoint{0.571548in}{0.154882in}}%
\pgfpathcurveto{\pgfqpoint{0.574674in}{0.151756in}}{\pgfqpoint{0.578913in}{0.150000in}}{\pgfqpoint{0.583333in}{0.150000in}}%
\pgfpathclose%
\pgfpathmoveto{\pgfqpoint{0.750000in}{0.150000in}}%
\pgfpathcurveto{\pgfqpoint{0.754420in}{0.150000in}}{\pgfqpoint{0.758660in}{0.151756in}}{\pgfqpoint{0.761785in}{0.154882in}}%
\pgfpathcurveto{\pgfqpoint{0.764911in}{0.158007in}}{\pgfqpoint{0.766667in}{0.162247in}}{\pgfqpoint{0.766667in}{0.166667in}}%
\pgfpathcurveto{\pgfqpoint{0.766667in}{0.171087in}}{\pgfqpoint{0.764911in}{0.175326in}}{\pgfqpoint{0.761785in}{0.178452in}}%
\pgfpathcurveto{\pgfqpoint{0.758660in}{0.181577in}}{\pgfqpoint{0.754420in}{0.183333in}}{\pgfqpoint{0.750000in}{0.183333in}}%
\pgfpathcurveto{\pgfqpoint{0.745580in}{0.183333in}}{\pgfqpoint{0.741340in}{0.181577in}}{\pgfqpoint{0.738215in}{0.178452in}}%
\pgfpathcurveto{\pgfqpoint{0.735089in}{0.175326in}}{\pgfqpoint{0.733333in}{0.171087in}}{\pgfqpoint{0.733333in}{0.166667in}}%
\pgfpathcurveto{\pgfqpoint{0.733333in}{0.162247in}}{\pgfqpoint{0.735089in}{0.158007in}}{\pgfqpoint{0.738215in}{0.154882in}}%
\pgfpathcurveto{\pgfqpoint{0.741340in}{0.151756in}}{\pgfqpoint{0.745580in}{0.150000in}}{\pgfqpoint{0.750000in}{0.150000in}}%
\pgfpathclose%
\pgfpathmoveto{\pgfqpoint{0.916667in}{0.150000in}}%
\pgfpathcurveto{\pgfqpoint{0.921087in}{0.150000in}}{\pgfqpoint{0.925326in}{0.151756in}}{\pgfqpoint{0.928452in}{0.154882in}}%
\pgfpathcurveto{\pgfqpoint{0.931577in}{0.158007in}}{\pgfqpoint{0.933333in}{0.162247in}}{\pgfqpoint{0.933333in}{0.166667in}}%
\pgfpathcurveto{\pgfqpoint{0.933333in}{0.171087in}}{\pgfqpoint{0.931577in}{0.175326in}}{\pgfqpoint{0.928452in}{0.178452in}}%
\pgfpathcurveto{\pgfqpoint{0.925326in}{0.181577in}}{\pgfqpoint{0.921087in}{0.183333in}}{\pgfqpoint{0.916667in}{0.183333in}}%
\pgfpathcurveto{\pgfqpoint{0.912247in}{0.183333in}}{\pgfqpoint{0.908007in}{0.181577in}}{\pgfqpoint{0.904882in}{0.178452in}}%
\pgfpathcurveto{\pgfqpoint{0.901756in}{0.175326in}}{\pgfqpoint{0.900000in}{0.171087in}}{\pgfqpoint{0.900000in}{0.166667in}}%
\pgfpathcurveto{\pgfqpoint{0.900000in}{0.162247in}}{\pgfqpoint{0.901756in}{0.158007in}}{\pgfqpoint{0.904882in}{0.154882in}}%
\pgfpathcurveto{\pgfqpoint{0.908007in}{0.151756in}}{\pgfqpoint{0.912247in}{0.150000in}}{\pgfqpoint{0.916667in}{0.150000in}}%
\pgfpathclose%
\pgfpathmoveto{\pgfqpoint{0.000000in}{0.316667in}}%
\pgfpathcurveto{\pgfqpoint{0.004420in}{0.316667in}}{\pgfqpoint{0.008660in}{0.318423in}}{\pgfqpoint{0.011785in}{0.321548in}}%
\pgfpathcurveto{\pgfqpoint{0.014911in}{0.324674in}}{\pgfqpoint{0.016667in}{0.328913in}}{\pgfqpoint{0.016667in}{0.333333in}}%
\pgfpathcurveto{\pgfqpoint{0.016667in}{0.337753in}}{\pgfqpoint{0.014911in}{0.341993in}}{\pgfqpoint{0.011785in}{0.345118in}}%
\pgfpathcurveto{\pgfqpoint{0.008660in}{0.348244in}}{\pgfqpoint{0.004420in}{0.350000in}}{\pgfqpoint{0.000000in}{0.350000in}}%
\pgfpathcurveto{\pgfqpoint{-0.004420in}{0.350000in}}{\pgfqpoint{-0.008660in}{0.348244in}}{\pgfqpoint{-0.011785in}{0.345118in}}%
\pgfpathcurveto{\pgfqpoint{-0.014911in}{0.341993in}}{\pgfqpoint{-0.016667in}{0.337753in}}{\pgfqpoint{-0.016667in}{0.333333in}}%
\pgfpathcurveto{\pgfqpoint{-0.016667in}{0.328913in}}{\pgfqpoint{-0.014911in}{0.324674in}}{\pgfqpoint{-0.011785in}{0.321548in}}%
\pgfpathcurveto{\pgfqpoint{-0.008660in}{0.318423in}}{\pgfqpoint{-0.004420in}{0.316667in}}{\pgfqpoint{0.000000in}{0.316667in}}%
\pgfpathclose%
\pgfpathmoveto{\pgfqpoint{0.166667in}{0.316667in}}%
\pgfpathcurveto{\pgfqpoint{0.171087in}{0.316667in}}{\pgfqpoint{0.175326in}{0.318423in}}{\pgfqpoint{0.178452in}{0.321548in}}%
\pgfpathcurveto{\pgfqpoint{0.181577in}{0.324674in}}{\pgfqpoint{0.183333in}{0.328913in}}{\pgfqpoint{0.183333in}{0.333333in}}%
\pgfpathcurveto{\pgfqpoint{0.183333in}{0.337753in}}{\pgfqpoint{0.181577in}{0.341993in}}{\pgfqpoint{0.178452in}{0.345118in}}%
\pgfpathcurveto{\pgfqpoint{0.175326in}{0.348244in}}{\pgfqpoint{0.171087in}{0.350000in}}{\pgfqpoint{0.166667in}{0.350000in}}%
\pgfpathcurveto{\pgfqpoint{0.162247in}{0.350000in}}{\pgfqpoint{0.158007in}{0.348244in}}{\pgfqpoint{0.154882in}{0.345118in}}%
\pgfpathcurveto{\pgfqpoint{0.151756in}{0.341993in}}{\pgfqpoint{0.150000in}{0.337753in}}{\pgfqpoint{0.150000in}{0.333333in}}%
\pgfpathcurveto{\pgfqpoint{0.150000in}{0.328913in}}{\pgfqpoint{0.151756in}{0.324674in}}{\pgfqpoint{0.154882in}{0.321548in}}%
\pgfpathcurveto{\pgfqpoint{0.158007in}{0.318423in}}{\pgfqpoint{0.162247in}{0.316667in}}{\pgfqpoint{0.166667in}{0.316667in}}%
\pgfpathclose%
\pgfpathmoveto{\pgfqpoint{0.333333in}{0.316667in}}%
\pgfpathcurveto{\pgfqpoint{0.337753in}{0.316667in}}{\pgfqpoint{0.341993in}{0.318423in}}{\pgfqpoint{0.345118in}{0.321548in}}%
\pgfpathcurveto{\pgfqpoint{0.348244in}{0.324674in}}{\pgfqpoint{0.350000in}{0.328913in}}{\pgfqpoint{0.350000in}{0.333333in}}%
\pgfpathcurveto{\pgfqpoint{0.350000in}{0.337753in}}{\pgfqpoint{0.348244in}{0.341993in}}{\pgfqpoint{0.345118in}{0.345118in}}%
\pgfpathcurveto{\pgfqpoint{0.341993in}{0.348244in}}{\pgfqpoint{0.337753in}{0.350000in}}{\pgfqpoint{0.333333in}{0.350000in}}%
\pgfpathcurveto{\pgfqpoint{0.328913in}{0.350000in}}{\pgfqpoint{0.324674in}{0.348244in}}{\pgfqpoint{0.321548in}{0.345118in}}%
\pgfpathcurveto{\pgfqpoint{0.318423in}{0.341993in}}{\pgfqpoint{0.316667in}{0.337753in}}{\pgfqpoint{0.316667in}{0.333333in}}%
\pgfpathcurveto{\pgfqpoint{0.316667in}{0.328913in}}{\pgfqpoint{0.318423in}{0.324674in}}{\pgfqpoint{0.321548in}{0.321548in}}%
\pgfpathcurveto{\pgfqpoint{0.324674in}{0.318423in}}{\pgfqpoint{0.328913in}{0.316667in}}{\pgfqpoint{0.333333in}{0.316667in}}%
\pgfpathclose%
\pgfpathmoveto{\pgfqpoint{0.500000in}{0.316667in}}%
\pgfpathcurveto{\pgfqpoint{0.504420in}{0.316667in}}{\pgfqpoint{0.508660in}{0.318423in}}{\pgfqpoint{0.511785in}{0.321548in}}%
\pgfpathcurveto{\pgfqpoint{0.514911in}{0.324674in}}{\pgfqpoint{0.516667in}{0.328913in}}{\pgfqpoint{0.516667in}{0.333333in}}%
\pgfpathcurveto{\pgfqpoint{0.516667in}{0.337753in}}{\pgfqpoint{0.514911in}{0.341993in}}{\pgfqpoint{0.511785in}{0.345118in}}%
\pgfpathcurveto{\pgfqpoint{0.508660in}{0.348244in}}{\pgfqpoint{0.504420in}{0.350000in}}{\pgfqpoint{0.500000in}{0.350000in}}%
\pgfpathcurveto{\pgfqpoint{0.495580in}{0.350000in}}{\pgfqpoint{0.491340in}{0.348244in}}{\pgfqpoint{0.488215in}{0.345118in}}%
\pgfpathcurveto{\pgfqpoint{0.485089in}{0.341993in}}{\pgfqpoint{0.483333in}{0.337753in}}{\pgfqpoint{0.483333in}{0.333333in}}%
\pgfpathcurveto{\pgfqpoint{0.483333in}{0.328913in}}{\pgfqpoint{0.485089in}{0.324674in}}{\pgfqpoint{0.488215in}{0.321548in}}%
\pgfpathcurveto{\pgfqpoint{0.491340in}{0.318423in}}{\pgfqpoint{0.495580in}{0.316667in}}{\pgfqpoint{0.500000in}{0.316667in}}%
\pgfpathclose%
\pgfpathmoveto{\pgfqpoint{0.666667in}{0.316667in}}%
\pgfpathcurveto{\pgfqpoint{0.671087in}{0.316667in}}{\pgfqpoint{0.675326in}{0.318423in}}{\pgfqpoint{0.678452in}{0.321548in}}%
\pgfpathcurveto{\pgfqpoint{0.681577in}{0.324674in}}{\pgfqpoint{0.683333in}{0.328913in}}{\pgfqpoint{0.683333in}{0.333333in}}%
\pgfpathcurveto{\pgfqpoint{0.683333in}{0.337753in}}{\pgfqpoint{0.681577in}{0.341993in}}{\pgfqpoint{0.678452in}{0.345118in}}%
\pgfpathcurveto{\pgfqpoint{0.675326in}{0.348244in}}{\pgfqpoint{0.671087in}{0.350000in}}{\pgfqpoint{0.666667in}{0.350000in}}%
\pgfpathcurveto{\pgfqpoint{0.662247in}{0.350000in}}{\pgfqpoint{0.658007in}{0.348244in}}{\pgfqpoint{0.654882in}{0.345118in}}%
\pgfpathcurveto{\pgfqpoint{0.651756in}{0.341993in}}{\pgfqpoint{0.650000in}{0.337753in}}{\pgfqpoint{0.650000in}{0.333333in}}%
\pgfpathcurveto{\pgfqpoint{0.650000in}{0.328913in}}{\pgfqpoint{0.651756in}{0.324674in}}{\pgfqpoint{0.654882in}{0.321548in}}%
\pgfpathcurveto{\pgfqpoint{0.658007in}{0.318423in}}{\pgfqpoint{0.662247in}{0.316667in}}{\pgfqpoint{0.666667in}{0.316667in}}%
\pgfpathclose%
\pgfpathmoveto{\pgfqpoint{0.833333in}{0.316667in}}%
\pgfpathcurveto{\pgfqpoint{0.837753in}{0.316667in}}{\pgfqpoint{0.841993in}{0.318423in}}{\pgfqpoint{0.845118in}{0.321548in}}%
\pgfpathcurveto{\pgfqpoint{0.848244in}{0.324674in}}{\pgfqpoint{0.850000in}{0.328913in}}{\pgfqpoint{0.850000in}{0.333333in}}%
\pgfpathcurveto{\pgfqpoint{0.850000in}{0.337753in}}{\pgfqpoint{0.848244in}{0.341993in}}{\pgfqpoint{0.845118in}{0.345118in}}%
\pgfpathcurveto{\pgfqpoint{0.841993in}{0.348244in}}{\pgfqpoint{0.837753in}{0.350000in}}{\pgfqpoint{0.833333in}{0.350000in}}%
\pgfpathcurveto{\pgfqpoint{0.828913in}{0.350000in}}{\pgfqpoint{0.824674in}{0.348244in}}{\pgfqpoint{0.821548in}{0.345118in}}%
\pgfpathcurveto{\pgfqpoint{0.818423in}{0.341993in}}{\pgfqpoint{0.816667in}{0.337753in}}{\pgfqpoint{0.816667in}{0.333333in}}%
\pgfpathcurveto{\pgfqpoint{0.816667in}{0.328913in}}{\pgfqpoint{0.818423in}{0.324674in}}{\pgfqpoint{0.821548in}{0.321548in}}%
\pgfpathcurveto{\pgfqpoint{0.824674in}{0.318423in}}{\pgfqpoint{0.828913in}{0.316667in}}{\pgfqpoint{0.833333in}{0.316667in}}%
\pgfpathclose%
\pgfpathmoveto{\pgfqpoint{1.000000in}{0.316667in}}%
\pgfpathcurveto{\pgfqpoint{1.004420in}{0.316667in}}{\pgfqpoint{1.008660in}{0.318423in}}{\pgfqpoint{1.011785in}{0.321548in}}%
\pgfpathcurveto{\pgfqpoint{1.014911in}{0.324674in}}{\pgfqpoint{1.016667in}{0.328913in}}{\pgfqpoint{1.016667in}{0.333333in}}%
\pgfpathcurveto{\pgfqpoint{1.016667in}{0.337753in}}{\pgfqpoint{1.014911in}{0.341993in}}{\pgfqpoint{1.011785in}{0.345118in}}%
\pgfpathcurveto{\pgfqpoint{1.008660in}{0.348244in}}{\pgfqpoint{1.004420in}{0.350000in}}{\pgfqpoint{1.000000in}{0.350000in}}%
\pgfpathcurveto{\pgfqpoint{0.995580in}{0.350000in}}{\pgfqpoint{0.991340in}{0.348244in}}{\pgfqpoint{0.988215in}{0.345118in}}%
\pgfpathcurveto{\pgfqpoint{0.985089in}{0.341993in}}{\pgfqpoint{0.983333in}{0.337753in}}{\pgfqpoint{0.983333in}{0.333333in}}%
\pgfpathcurveto{\pgfqpoint{0.983333in}{0.328913in}}{\pgfqpoint{0.985089in}{0.324674in}}{\pgfqpoint{0.988215in}{0.321548in}}%
\pgfpathcurveto{\pgfqpoint{0.991340in}{0.318423in}}{\pgfqpoint{0.995580in}{0.316667in}}{\pgfqpoint{1.000000in}{0.316667in}}%
\pgfpathclose%
\pgfpathmoveto{\pgfqpoint{0.083333in}{0.483333in}}%
\pgfpathcurveto{\pgfqpoint{0.087753in}{0.483333in}}{\pgfqpoint{0.091993in}{0.485089in}}{\pgfqpoint{0.095118in}{0.488215in}}%
\pgfpathcurveto{\pgfqpoint{0.098244in}{0.491340in}}{\pgfqpoint{0.100000in}{0.495580in}}{\pgfqpoint{0.100000in}{0.500000in}}%
\pgfpathcurveto{\pgfqpoint{0.100000in}{0.504420in}}{\pgfqpoint{0.098244in}{0.508660in}}{\pgfqpoint{0.095118in}{0.511785in}}%
\pgfpathcurveto{\pgfqpoint{0.091993in}{0.514911in}}{\pgfqpoint{0.087753in}{0.516667in}}{\pgfqpoint{0.083333in}{0.516667in}}%
\pgfpathcurveto{\pgfqpoint{0.078913in}{0.516667in}}{\pgfqpoint{0.074674in}{0.514911in}}{\pgfqpoint{0.071548in}{0.511785in}}%
\pgfpathcurveto{\pgfqpoint{0.068423in}{0.508660in}}{\pgfqpoint{0.066667in}{0.504420in}}{\pgfqpoint{0.066667in}{0.500000in}}%
\pgfpathcurveto{\pgfqpoint{0.066667in}{0.495580in}}{\pgfqpoint{0.068423in}{0.491340in}}{\pgfqpoint{0.071548in}{0.488215in}}%
\pgfpathcurveto{\pgfqpoint{0.074674in}{0.485089in}}{\pgfqpoint{0.078913in}{0.483333in}}{\pgfqpoint{0.083333in}{0.483333in}}%
\pgfpathclose%
\pgfpathmoveto{\pgfqpoint{0.250000in}{0.483333in}}%
\pgfpathcurveto{\pgfqpoint{0.254420in}{0.483333in}}{\pgfqpoint{0.258660in}{0.485089in}}{\pgfqpoint{0.261785in}{0.488215in}}%
\pgfpathcurveto{\pgfqpoint{0.264911in}{0.491340in}}{\pgfqpoint{0.266667in}{0.495580in}}{\pgfqpoint{0.266667in}{0.500000in}}%
\pgfpathcurveto{\pgfqpoint{0.266667in}{0.504420in}}{\pgfqpoint{0.264911in}{0.508660in}}{\pgfqpoint{0.261785in}{0.511785in}}%
\pgfpathcurveto{\pgfqpoint{0.258660in}{0.514911in}}{\pgfqpoint{0.254420in}{0.516667in}}{\pgfqpoint{0.250000in}{0.516667in}}%
\pgfpathcurveto{\pgfqpoint{0.245580in}{0.516667in}}{\pgfqpoint{0.241340in}{0.514911in}}{\pgfqpoint{0.238215in}{0.511785in}}%
\pgfpathcurveto{\pgfqpoint{0.235089in}{0.508660in}}{\pgfqpoint{0.233333in}{0.504420in}}{\pgfqpoint{0.233333in}{0.500000in}}%
\pgfpathcurveto{\pgfqpoint{0.233333in}{0.495580in}}{\pgfqpoint{0.235089in}{0.491340in}}{\pgfqpoint{0.238215in}{0.488215in}}%
\pgfpathcurveto{\pgfqpoint{0.241340in}{0.485089in}}{\pgfqpoint{0.245580in}{0.483333in}}{\pgfqpoint{0.250000in}{0.483333in}}%
\pgfpathclose%
\pgfpathmoveto{\pgfqpoint{0.416667in}{0.483333in}}%
\pgfpathcurveto{\pgfqpoint{0.421087in}{0.483333in}}{\pgfqpoint{0.425326in}{0.485089in}}{\pgfqpoint{0.428452in}{0.488215in}}%
\pgfpathcurveto{\pgfqpoint{0.431577in}{0.491340in}}{\pgfqpoint{0.433333in}{0.495580in}}{\pgfqpoint{0.433333in}{0.500000in}}%
\pgfpathcurveto{\pgfqpoint{0.433333in}{0.504420in}}{\pgfqpoint{0.431577in}{0.508660in}}{\pgfqpoint{0.428452in}{0.511785in}}%
\pgfpathcurveto{\pgfqpoint{0.425326in}{0.514911in}}{\pgfqpoint{0.421087in}{0.516667in}}{\pgfqpoint{0.416667in}{0.516667in}}%
\pgfpathcurveto{\pgfqpoint{0.412247in}{0.516667in}}{\pgfqpoint{0.408007in}{0.514911in}}{\pgfqpoint{0.404882in}{0.511785in}}%
\pgfpathcurveto{\pgfqpoint{0.401756in}{0.508660in}}{\pgfqpoint{0.400000in}{0.504420in}}{\pgfqpoint{0.400000in}{0.500000in}}%
\pgfpathcurveto{\pgfqpoint{0.400000in}{0.495580in}}{\pgfqpoint{0.401756in}{0.491340in}}{\pgfqpoint{0.404882in}{0.488215in}}%
\pgfpathcurveto{\pgfqpoint{0.408007in}{0.485089in}}{\pgfqpoint{0.412247in}{0.483333in}}{\pgfqpoint{0.416667in}{0.483333in}}%
\pgfpathclose%
\pgfpathmoveto{\pgfqpoint{0.583333in}{0.483333in}}%
\pgfpathcurveto{\pgfqpoint{0.587753in}{0.483333in}}{\pgfqpoint{0.591993in}{0.485089in}}{\pgfqpoint{0.595118in}{0.488215in}}%
\pgfpathcurveto{\pgfqpoint{0.598244in}{0.491340in}}{\pgfqpoint{0.600000in}{0.495580in}}{\pgfqpoint{0.600000in}{0.500000in}}%
\pgfpathcurveto{\pgfqpoint{0.600000in}{0.504420in}}{\pgfqpoint{0.598244in}{0.508660in}}{\pgfqpoint{0.595118in}{0.511785in}}%
\pgfpathcurveto{\pgfqpoint{0.591993in}{0.514911in}}{\pgfqpoint{0.587753in}{0.516667in}}{\pgfqpoint{0.583333in}{0.516667in}}%
\pgfpathcurveto{\pgfqpoint{0.578913in}{0.516667in}}{\pgfqpoint{0.574674in}{0.514911in}}{\pgfqpoint{0.571548in}{0.511785in}}%
\pgfpathcurveto{\pgfqpoint{0.568423in}{0.508660in}}{\pgfqpoint{0.566667in}{0.504420in}}{\pgfqpoint{0.566667in}{0.500000in}}%
\pgfpathcurveto{\pgfqpoint{0.566667in}{0.495580in}}{\pgfqpoint{0.568423in}{0.491340in}}{\pgfqpoint{0.571548in}{0.488215in}}%
\pgfpathcurveto{\pgfqpoint{0.574674in}{0.485089in}}{\pgfqpoint{0.578913in}{0.483333in}}{\pgfqpoint{0.583333in}{0.483333in}}%
\pgfpathclose%
\pgfpathmoveto{\pgfqpoint{0.750000in}{0.483333in}}%
\pgfpathcurveto{\pgfqpoint{0.754420in}{0.483333in}}{\pgfqpoint{0.758660in}{0.485089in}}{\pgfqpoint{0.761785in}{0.488215in}}%
\pgfpathcurveto{\pgfqpoint{0.764911in}{0.491340in}}{\pgfqpoint{0.766667in}{0.495580in}}{\pgfqpoint{0.766667in}{0.500000in}}%
\pgfpathcurveto{\pgfqpoint{0.766667in}{0.504420in}}{\pgfqpoint{0.764911in}{0.508660in}}{\pgfqpoint{0.761785in}{0.511785in}}%
\pgfpathcurveto{\pgfqpoint{0.758660in}{0.514911in}}{\pgfqpoint{0.754420in}{0.516667in}}{\pgfqpoint{0.750000in}{0.516667in}}%
\pgfpathcurveto{\pgfqpoint{0.745580in}{0.516667in}}{\pgfqpoint{0.741340in}{0.514911in}}{\pgfqpoint{0.738215in}{0.511785in}}%
\pgfpathcurveto{\pgfqpoint{0.735089in}{0.508660in}}{\pgfqpoint{0.733333in}{0.504420in}}{\pgfqpoint{0.733333in}{0.500000in}}%
\pgfpathcurveto{\pgfqpoint{0.733333in}{0.495580in}}{\pgfqpoint{0.735089in}{0.491340in}}{\pgfqpoint{0.738215in}{0.488215in}}%
\pgfpathcurveto{\pgfqpoint{0.741340in}{0.485089in}}{\pgfqpoint{0.745580in}{0.483333in}}{\pgfqpoint{0.750000in}{0.483333in}}%
\pgfpathclose%
\pgfpathmoveto{\pgfqpoint{0.916667in}{0.483333in}}%
\pgfpathcurveto{\pgfqpoint{0.921087in}{0.483333in}}{\pgfqpoint{0.925326in}{0.485089in}}{\pgfqpoint{0.928452in}{0.488215in}}%
\pgfpathcurveto{\pgfqpoint{0.931577in}{0.491340in}}{\pgfqpoint{0.933333in}{0.495580in}}{\pgfqpoint{0.933333in}{0.500000in}}%
\pgfpathcurveto{\pgfqpoint{0.933333in}{0.504420in}}{\pgfqpoint{0.931577in}{0.508660in}}{\pgfqpoint{0.928452in}{0.511785in}}%
\pgfpathcurveto{\pgfqpoint{0.925326in}{0.514911in}}{\pgfqpoint{0.921087in}{0.516667in}}{\pgfqpoint{0.916667in}{0.516667in}}%
\pgfpathcurveto{\pgfqpoint{0.912247in}{0.516667in}}{\pgfqpoint{0.908007in}{0.514911in}}{\pgfqpoint{0.904882in}{0.511785in}}%
\pgfpathcurveto{\pgfqpoint{0.901756in}{0.508660in}}{\pgfqpoint{0.900000in}{0.504420in}}{\pgfqpoint{0.900000in}{0.500000in}}%
\pgfpathcurveto{\pgfqpoint{0.900000in}{0.495580in}}{\pgfqpoint{0.901756in}{0.491340in}}{\pgfqpoint{0.904882in}{0.488215in}}%
\pgfpathcurveto{\pgfqpoint{0.908007in}{0.485089in}}{\pgfqpoint{0.912247in}{0.483333in}}{\pgfqpoint{0.916667in}{0.483333in}}%
\pgfpathclose%
\pgfpathmoveto{\pgfqpoint{0.000000in}{0.650000in}}%
\pgfpathcurveto{\pgfqpoint{0.004420in}{0.650000in}}{\pgfqpoint{0.008660in}{0.651756in}}{\pgfqpoint{0.011785in}{0.654882in}}%
\pgfpathcurveto{\pgfqpoint{0.014911in}{0.658007in}}{\pgfqpoint{0.016667in}{0.662247in}}{\pgfqpoint{0.016667in}{0.666667in}}%
\pgfpathcurveto{\pgfqpoint{0.016667in}{0.671087in}}{\pgfqpoint{0.014911in}{0.675326in}}{\pgfqpoint{0.011785in}{0.678452in}}%
\pgfpathcurveto{\pgfqpoint{0.008660in}{0.681577in}}{\pgfqpoint{0.004420in}{0.683333in}}{\pgfqpoint{0.000000in}{0.683333in}}%
\pgfpathcurveto{\pgfqpoint{-0.004420in}{0.683333in}}{\pgfqpoint{-0.008660in}{0.681577in}}{\pgfqpoint{-0.011785in}{0.678452in}}%
\pgfpathcurveto{\pgfqpoint{-0.014911in}{0.675326in}}{\pgfqpoint{-0.016667in}{0.671087in}}{\pgfqpoint{-0.016667in}{0.666667in}}%
\pgfpathcurveto{\pgfqpoint{-0.016667in}{0.662247in}}{\pgfqpoint{-0.014911in}{0.658007in}}{\pgfqpoint{-0.011785in}{0.654882in}}%
\pgfpathcurveto{\pgfqpoint{-0.008660in}{0.651756in}}{\pgfqpoint{-0.004420in}{0.650000in}}{\pgfqpoint{0.000000in}{0.650000in}}%
\pgfpathclose%
\pgfpathmoveto{\pgfqpoint{0.166667in}{0.650000in}}%
\pgfpathcurveto{\pgfqpoint{0.171087in}{0.650000in}}{\pgfqpoint{0.175326in}{0.651756in}}{\pgfqpoint{0.178452in}{0.654882in}}%
\pgfpathcurveto{\pgfqpoint{0.181577in}{0.658007in}}{\pgfqpoint{0.183333in}{0.662247in}}{\pgfqpoint{0.183333in}{0.666667in}}%
\pgfpathcurveto{\pgfqpoint{0.183333in}{0.671087in}}{\pgfqpoint{0.181577in}{0.675326in}}{\pgfqpoint{0.178452in}{0.678452in}}%
\pgfpathcurveto{\pgfqpoint{0.175326in}{0.681577in}}{\pgfqpoint{0.171087in}{0.683333in}}{\pgfqpoint{0.166667in}{0.683333in}}%
\pgfpathcurveto{\pgfqpoint{0.162247in}{0.683333in}}{\pgfqpoint{0.158007in}{0.681577in}}{\pgfqpoint{0.154882in}{0.678452in}}%
\pgfpathcurveto{\pgfqpoint{0.151756in}{0.675326in}}{\pgfqpoint{0.150000in}{0.671087in}}{\pgfqpoint{0.150000in}{0.666667in}}%
\pgfpathcurveto{\pgfqpoint{0.150000in}{0.662247in}}{\pgfqpoint{0.151756in}{0.658007in}}{\pgfqpoint{0.154882in}{0.654882in}}%
\pgfpathcurveto{\pgfqpoint{0.158007in}{0.651756in}}{\pgfqpoint{0.162247in}{0.650000in}}{\pgfqpoint{0.166667in}{0.650000in}}%
\pgfpathclose%
\pgfpathmoveto{\pgfqpoint{0.333333in}{0.650000in}}%
\pgfpathcurveto{\pgfqpoint{0.337753in}{0.650000in}}{\pgfqpoint{0.341993in}{0.651756in}}{\pgfqpoint{0.345118in}{0.654882in}}%
\pgfpathcurveto{\pgfqpoint{0.348244in}{0.658007in}}{\pgfqpoint{0.350000in}{0.662247in}}{\pgfqpoint{0.350000in}{0.666667in}}%
\pgfpathcurveto{\pgfqpoint{0.350000in}{0.671087in}}{\pgfqpoint{0.348244in}{0.675326in}}{\pgfqpoint{0.345118in}{0.678452in}}%
\pgfpathcurveto{\pgfqpoint{0.341993in}{0.681577in}}{\pgfqpoint{0.337753in}{0.683333in}}{\pgfqpoint{0.333333in}{0.683333in}}%
\pgfpathcurveto{\pgfqpoint{0.328913in}{0.683333in}}{\pgfqpoint{0.324674in}{0.681577in}}{\pgfqpoint{0.321548in}{0.678452in}}%
\pgfpathcurveto{\pgfqpoint{0.318423in}{0.675326in}}{\pgfqpoint{0.316667in}{0.671087in}}{\pgfqpoint{0.316667in}{0.666667in}}%
\pgfpathcurveto{\pgfqpoint{0.316667in}{0.662247in}}{\pgfqpoint{0.318423in}{0.658007in}}{\pgfqpoint{0.321548in}{0.654882in}}%
\pgfpathcurveto{\pgfqpoint{0.324674in}{0.651756in}}{\pgfqpoint{0.328913in}{0.650000in}}{\pgfqpoint{0.333333in}{0.650000in}}%
\pgfpathclose%
\pgfpathmoveto{\pgfqpoint{0.500000in}{0.650000in}}%
\pgfpathcurveto{\pgfqpoint{0.504420in}{0.650000in}}{\pgfqpoint{0.508660in}{0.651756in}}{\pgfqpoint{0.511785in}{0.654882in}}%
\pgfpathcurveto{\pgfqpoint{0.514911in}{0.658007in}}{\pgfqpoint{0.516667in}{0.662247in}}{\pgfqpoint{0.516667in}{0.666667in}}%
\pgfpathcurveto{\pgfqpoint{0.516667in}{0.671087in}}{\pgfqpoint{0.514911in}{0.675326in}}{\pgfqpoint{0.511785in}{0.678452in}}%
\pgfpathcurveto{\pgfqpoint{0.508660in}{0.681577in}}{\pgfqpoint{0.504420in}{0.683333in}}{\pgfqpoint{0.500000in}{0.683333in}}%
\pgfpathcurveto{\pgfqpoint{0.495580in}{0.683333in}}{\pgfqpoint{0.491340in}{0.681577in}}{\pgfqpoint{0.488215in}{0.678452in}}%
\pgfpathcurveto{\pgfqpoint{0.485089in}{0.675326in}}{\pgfqpoint{0.483333in}{0.671087in}}{\pgfqpoint{0.483333in}{0.666667in}}%
\pgfpathcurveto{\pgfqpoint{0.483333in}{0.662247in}}{\pgfqpoint{0.485089in}{0.658007in}}{\pgfqpoint{0.488215in}{0.654882in}}%
\pgfpathcurveto{\pgfqpoint{0.491340in}{0.651756in}}{\pgfqpoint{0.495580in}{0.650000in}}{\pgfqpoint{0.500000in}{0.650000in}}%
\pgfpathclose%
\pgfpathmoveto{\pgfqpoint{0.666667in}{0.650000in}}%
\pgfpathcurveto{\pgfqpoint{0.671087in}{0.650000in}}{\pgfqpoint{0.675326in}{0.651756in}}{\pgfqpoint{0.678452in}{0.654882in}}%
\pgfpathcurveto{\pgfqpoint{0.681577in}{0.658007in}}{\pgfqpoint{0.683333in}{0.662247in}}{\pgfqpoint{0.683333in}{0.666667in}}%
\pgfpathcurveto{\pgfqpoint{0.683333in}{0.671087in}}{\pgfqpoint{0.681577in}{0.675326in}}{\pgfqpoint{0.678452in}{0.678452in}}%
\pgfpathcurveto{\pgfqpoint{0.675326in}{0.681577in}}{\pgfqpoint{0.671087in}{0.683333in}}{\pgfqpoint{0.666667in}{0.683333in}}%
\pgfpathcurveto{\pgfqpoint{0.662247in}{0.683333in}}{\pgfqpoint{0.658007in}{0.681577in}}{\pgfqpoint{0.654882in}{0.678452in}}%
\pgfpathcurveto{\pgfqpoint{0.651756in}{0.675326in}}{\pgfqpoint{0.650000in}{0.671087in}}{\pgfqpoint{0.650000in}{0.666667in}}%
\pgfpathcurveto{\pgfqpoint{0.650000in}{0.662247in}}{\pgfqpoint{0.651756in}{0.658007in}}{\pgfqpoint{0.654882in}{0.654882in}}%
\pgfpathcurveto{\pgfqpoint{0.658007in}{0.651756in}}{\pgfqpoint{0.662247in}{0.650000in}}{\pgfqpoint{0.666667in}{0.650000in}}%
\pgfpathclose%
\pgfpathmoveto{\pgfqpoint{0.833333in}{0.650000in}}%
\pgfpathcurveto{\pgfqpoint{0.837753in}{0.650000in}}{\pgfqpoint{0.841993in}{0.651756in}}{\pgfqpoint{0.845118in}{0.654882in}}%
\pgfpathcurveto{\pgfqpoint{0.848244in}{0.658007in}}{\pgfqpoint{0.850000in}{0.662247in}}{\pgfqpoint{0.850000in}{0.666667in}}%
\pgfpathcurveto{\pgfqpoint{0.850000in}{0.671087in}}{\pgfqpoint{0.848244in}{0.675326in}}{\pgfqpoint{0.845118in}{0.678452in}}%
\pgfpathcurveto{\pgfqpoint{0.841993in}{0.681577in}}{\pgfqpoint{0.837753in}{0.683333in}}{\pgfqpoint{0.833333in}{0.683333in}}%
\pgfpathcurveto{\pgfqpoint{0.828913in}{0.683333in}}{\pgfqpoint{0.824674in}{0.681577in}}{\pgfqpoint{0.821548in}{0.678452in}}%
\pgfpathcurveto{\pgfqpoint{0.818423in}{0.675326in}}{\pgfqpoint{0.816667in}{0.671087in}}{\pgfqpoint{0.816667in}{0.666667in}}%
\pgfpathcurveto{\pgfqpoint{0.816667in}{0.662247in}}{\pgfqpoint{0.818423in}{0.658007in}}{\pgfqpoint{0.821548in}{0.654882in}}%
\pgfpathcurveto{\pgfqpoint{0.824674in}{0.651756in}}{\pgfqpoint{0.828913in}{0.650000in}}{\pgfqpoint{0.833333in}{0.650000in}}%
\pgfpathclose%
\pgfpathmoveto{\pgfqpoint{1.000000in}{0.650000in}}%
\pgfpathcurveto{\pgfqpoint{1.004420in}{0.650000in}}{\pgfqpoint{1.008660in}{0.651756in}}{\pgfqpoint{1.011785in}{0.654882in}}%
\pgfpathcurveto{\pgfqpoint{1.014911in}{0.658007in}}{\pgfqpoint{1.016667in}{0.662247in}}{\pgfqpoint{1.016667in}{0.666667in}}%
\pgfpathcurveto{\pgfqpoint{1.016667in}{0.671087in}}{\pgfqpoint{1.014911in}{0.675326in}}{\pgfqpoint{1.011785in}{0.678452in}}%
\pgfpathcurveto{\pgfqpoint{1.008660in}{0.681577in}}{\pgfqpoint{1.004420in}{0.683333in}}{\pgfqpoint{1.000000in}{0.683333in}}%
\pgfpathcurveto{\pgfqpoint{0.995580in}{0.683333in}}{\pgfqpoint{0.991340in}{0.681577in}}{\pgfqpoint{0.988215in}{0.678452in}}%
\pgfpathcurveto{\pgfqpoint{0.985089in}{0.675326in}}{\pgfqpoint{0.983333in}{0.671087in}}{\pgfqpoint{0.983333in}{0.666667in}}%
\pgfpathcurveto{\pgfqpoint{0.983333in}{0.662247in}}{\pgfqpoint{0.985089in}{0.658007in}}{\pgfqpoint{0.988215in}{0.654882in}}%
\pgfpathcurveto{\pgfqpoint{0.991340in}{0.651756in}}{\pgfqpoint{0.995580in}{0.650000in}}{\pgfqpoint{1.000000in}{0.650000in}}%
\pgfpathclose%
\pgfpathmoveto{\pgfqpoint{0.083333in}{0.816667in}}%
\pgfpathcurveto{\pgfqpoint{0.087753in}{0.816667in}}{\pgfqpoint{0.091993in}{0.818423in}}{\pgfqpoint{0.095118in}{0.821548in}}%
\pgfpathcurveto{\pgfqpoint{0.098244in}{0.824674in}}{\pgfqpoint{0.100000in}{0.828913in}}{\pgfqpoint{0.100000in}{0.833333in}}%
\pgfpathcurveto{\pgfqpoint{0.100000in}{0.837753in}}{\pgfqpoint{0.098244in}{0.841993in}}{\pgfqpoint{0.095118in}{0.845118in}}%
\pgfpathcurveto{\pgfqpoint{0.091993in}{0.848244in}}{\pgfqpoint{0.087753in}{0.850000in}}{\pgfqpoint{0.083333in}{0.850000in}}%
\pgfpathcurveto{\pgfqpoint{0.078913in}{0.850000in}}{\pgfqpoint{0.074674in}{0.848244in}}{\pgfqpoint{0.071548in}{0.845118in}}%
\pgfpathcurveto{\pgfqpoint{0.068423in}{0.841993in}}{\pgfqpoint{0.066667in}{0.837753in}}{\pgfqpoint{0.066667in}{0.833333in}}%
\pgfpathcurveto{\pgfqpoint{0.066667in}{0.828913in}}{\pgfqpoint{0.068423in}{0.824674in}}{\pgfqpoint{0.071548in}{0.821548in}}%
\pgfpathcurveto{\pgfqpoint{0.074674in}{0.818423in}}{\pgfqpoint{0.078913in}{0.816667in}}{\pgfqpoint{0.083333in}{0.816667in}}%
\pgfpathclose%
\pgfpathmoveto{\pgfqpoint{0.250000in}{0.816667in}}%
\pgfpathcurveto{\pgfqpoint{0.254420in}{0.816667in}}{\pgfqpoint{0.258660in}{0.818423in}}{\pgfqpoint{0.261785in}{0.821548in}}%
\pgfpathcurveto{\pgfqpoint{0.264911in}{0.824674in}}{\pgfqpoint{0.266667in}{0.828913in}}{\pgfqpoint{0.266667in}{0.833333in}}%
\pgfpathcurveto{\pgfqpoint{0.266667in}{0.837753in}}{\pgfqpoint{0.264911in}{0.841993in}}{\pgfqpoint{0.261785in}{0.845118in}}%
\pgfpathcurveto{\pgfqpoint{0.258660in}{0.848244in}}{\pgfqpoint{0.254420in}{0.850000in}}{\pgfqpoint{0.250000in}{0.850000in}}%
\pgfpathcurveto{\pgfqpoint{0.245580in}{0.850000in}}{\pgfqpoint{0.241340in}{0.848244in}}{\pgfqpoint{0.238215in}{0.845118in}}%
\pgfpathcurveto{\pgfqpoint{0.235089in}{0.841993in}}{\pgfqpoint{0.233333in}{0.837753in}}{\pgfqpoint{0.233333in}{0.833333in}}%
\pgfpathcurveto{\pgfqpoint{0.233333in}{0.828913in}}{\pgfqpoint{0.235089in}{0.824674in}}{\pgfqpoint{0.238215in}{0.821548in}}%
\pgfpathcurveto{\pgfqpoint{0.241340in}{0.818423in}}{\pgfqpoint{0.245580in}{0.816667in}}{\pgfqpoint{0.250000in}{0.816667in}}%
\pgfpathclose%
\pgfpathmoveto{\pgfqpoint{0.416667in}{0.816667in}}%
\pgfpathcurveto{\pgfqpoint{0.421087in}{0.816667in}}{\pgfqpoint{0.425326in}{0.818423in}}{\pgfqpoint{0.428452in}{0.821548in}}%
\pgfpathcurveto{\pgfqpoint{0.431577in}{0.824674in}}{\pgfqpoint{0.433333in}{0.828913in}}{\pgfqpoint{0.433333in}{0.833333in}}%
\pgfpathcurveto{\pgfqpoint{0.433333in}{0.837753in}}{\pgfqpoint{0.431577in}{0.841993in}}{\pgfqpoint{0.428452in}{0.845118in}}%
\pgfpathcurveto{\pgfqpoint{0.425326in}{0.848244in}}{\pgfqpoint{0.421087in}{0.850000in}}{\pgfqpoint{0.416667in}{0.850000in}}%
\pgfpathcurveto{\pgfqpoint{0.412247in}{0.850000in}}{\pgfqpoint{0.408007in}{0.848244in}}{\pgfqpoint{0.404882in}{0.845118in}}%
\pgfpathcurveto{\pgfqpoint{0.401756in}{0.841993in}}{\pgfqpoint{0.400000in}{0.837753in}}{\pgfqpoint{0.400000in}{0.833333in}}%
\pgfpathcurveto{\pgfqpoint{0.400000in}{0.828913in}}{\pgfqpoint{0.401756in}{0.824674in}}{\pgfqpoint{0.404882in}{0.821548in}}%
\pgfpathcurveto{\pgfqpoint{0.408007in}{0.818423in}}{\pgfqpoint{0.412247in}{0.816667in}}{\pgfqpoint{0.416667in}{0.816667in}}%
\pgfpathclose%
\pgfpathmoveto{\pgfqpoint{0.583333in}{0.816667in}}%
\pgfpathcurveto{\pgfqpoint{0.587753in}{0.816667in}}{\pgfqpoint{0.591993in}{0.818423in}}{\pgfqpoint{0.595118in}{0.821548in}}%
\pgfpathcurveto{\pgfqpoint{0.598244in}{0.824674in}}{\pgfqpoint{0.600000in}{0.828913in}}{\pgfqpoint{0.600000in}{0.833333in}}%
\pgfpathcurveto{\pgfqpoint{0.600000in}{0.837753in}}{\pgfqpoint{0.598244in}{0.841993in}}{\pgfqpoint{0.595118in}{0.845118in}}%
\pgfpathcurveto{\pgfqpoint{0.591993in}{0.848244in}}{\pgfqpoint{0.587753in}{0.850000in}}{\pgfqpoint{0.583333in}{0.850000in}}%
\pgfpathcurveto{\pgfqpoint{0.578913in}{0.850000in}}{\pgfqpoint{0.574674in}{0.848244in}}{\pgfqpoint{0.571548in}{0.845118in}}%
\pgfpathcurveto{\pgfqpoint{0.568423in}{0.841993in}}{\pgfqpoint{0.566667in}{0.837753in}}{\pgfqpoint{0.566667in}{0.833333in}}%
\pgfpathcurveto{\pgfqpoint{0.566667in}{0.828913in}}{\pgfqpoint{0.568423in}{0.824674in}}{\pgfqpoint{0.571548in}{0.821548in}}%
\pgfpathcurveto{\pgfqpoint{0.574674in}{0.818423in}}{\pgfqpoint{0.578913in}{0.816667in}}{\pgfqpoint{0.583333in}{0.816667in}}%
\pgfpathclose%
\pgfpathmoveto{\pgfqpoint{0.750000in}{0.816667in}}%
\pgfpathcurveto{\pgfqpoint{0.754420in}{0.816667in}}{\pgfqpoint{0.758660in}{0.818423in}}{\pgfqpoint{0.761785in}{0.821548in}}%
\pgfpathcurveto{\pgfqpoint{0.764911in}{0.824674in}}{\pgfqpoint{0.766667in}{0.828913in}}{\pgfqpoint{0.766667in}{0.833333in}}%
\pgfpathcurveto{\pgfqpoint{0.766667in}{0.837753in}}{\pgfqpoint{0.764911in}{0.841993in}}{\pgfqpoint{0.761785in}{0.845118in}}%
\pgfpathcurveto{\pgfqpoint{0.758660in}{0.848244in}}{\pgfqpoint{0.754420in}{0.850000in}}{\pgfqpoint{0.750000in}{0.850000in}}%
\pgfpathcurveto{\pgfqpoint{0.745580in}{0.850000in}}{\pgfqpoint{0.741340in}{0.848244in}}{\pgfqpoint{0.738215in}{0.845118in}}%
\pgfpathcurveto{\pgfqpoint{0.735089in}{0.841993in}}{\pgfqpoint{0.733333in}{0.837753in}}{\pgfqpoint{0.733333in}{0.833333in}}%
\pgfpathcurveto{\pgfqpoint{0.733333in}{0.828913in}}{\pgfqpoint{0.735089in}{0.824674in}}{\pgfqpoint{0.738215in}{0.821548in}}%
\pgfpathcurveto{\pgfqpoint{0.741340in}{0.818423in}}{\pgfqpoint{0.745580in}{0.816667in}}{\pgfqpoint{0.750000in}{0.816667in}}%
\pgfpathclose%
\pgfpathmoveto{\pgfqpoint{0.916667in}{0.816667in}}%
\pgfpathcurveto{\pgfqpoint{0.921087in}{0.816667in}}{\pgfqpoint{0.925326in}{0.818423in}}{\pgfqpoint{0.928452in}{0.821548in}}%
\pgfpathcurveto{\pgfqpoint{0.931577in}{0.824674in}}{\pgfqpoint{0.933333in}{0.828913in}}{\pgfqpoint{0.933333in}{0.833333in}}%
\pgfpathcurveto{\pgfqpoint{0.933333in}{0.837753in}}{\pgfqpoint{0.931577in}{0.841993in}}{\pgfqpoint{0.928452in}{0.845118in}}%
\pgfpathcurveto{\pgfqpoint{0.925326in}{0.848244in}}{\pgfqpoint{0.921087in}{0.850000in}}{\pgfqpoint{0.916667in}{0.850000in}}%
\pgfpathcurveto{\pgfqpoint{0.912247in}{0.850000in}}{\pgfqpoint{0.908007in}{0.848244in}}{\pgfqpoint{0.904882in}{0.845118in}}%
\pgfpathcurveto{\pgfqpoint{0.901756in}{0.841993in}}{\pgfqpoint{0.900000in}{0.837753in}}{\pgfqpoint{0.900000in}{0.833333in}}%
\pgfpathcurveto{\pgfqpoint{0.900000in}{0.828913in}}{\pgfqpoint{0.901756in}{0.824674in}}{\pgfqpoint{0.904882in}{0.821548in}}%
\pgfpathcurveto{\pgfqpoint{0.908007in}{0.818423in}}{\pgfqpoint{0.912247in}{0.816667in}}{\pgfqpoint{0.916667in}{0.816667in}}%
\pgfpathclose%
\pgfpathmoveto{\pgfqpoint{0.000000in}{0.983333in}}%
\pgfpathcurveto{\pgfqpoint{0.004420in}{0.983333in}}{\pgfqpoint{0.008660in}{0.985089in}}{\pgfqpoint{0.011785in}{0.988215in}}%
\pgfpathcurveto{\pgfqpoint{0.014911in}{0.991340in}}{\pgfqpoint{0.016667in}{0.995580in}}{\pgfqpoint{0.016667in}{1.000000in}}%
\pgfpathcurveto{\pgfqpoint{0.016667in}{1.004420in}}{\pgfqpoint{0.014911in}{1.008660in}}{\pgfqpoint{0.011785in}{1.011785in}}%
\pgfpathcurveto{\pgfqpoint{0.008660in}{1.014911in}}{\pgfqpoint{0.004420in}{1.016667in}}{\pgfqpoint{0.000000in}{1.016667in}}%
\pgfpathcurveto{\pgfqpoint{-0.004420in}{1.016667in}}{\pgfqpoint{-0.008660in}{1.014911in}}{\pgfqpoint{-0.011785in}{1.011785in}}%
\pgfpathcurveto{\pgfqpoint{-0.014911in}{1.008660in}}{\pgfqpoint{-0.016667in}{1.004420in}}{\pgfqpoint{-0.016667in}{1.000000in}}%
\pgfpathcurveto{\pgfqpoint{-0.016667in}{0.995580in}}{\pgfqpoint{-0.014911in}{0.991340in}}{\pgfqpoint{-0.011785in}{0.988215in}}%
\pgfpathcurveto{\pgfqpoint{-0.008660in}{0.985089in}}{\pgfqpoint{-0.004420in}{0.983333in}}{\pgfqpoint{0.000000in}{0.983333in}}%
\pgfpathclose%
\pgfpathmoveto{\pgfqpoint{0.166667in}{0.983333in}}%
\pgfpathcurveto{\pgfqpoint{0.171087in}{0.983333in}}{\pgfqpoint{0.175326in}{0.985089in}}{\pgfqpoint{0.178452in}{0.988215in}}%
\pgfpathcurveto{\pgfqpoint{0.181577in}{0.991340in}}{\pgfqpoint{0.183333in}{0.995580in}}{\pgfqpoint{0.183333in}{1.000000in}}%
\pgfpathcurveto{\pgfqpoint{0.183333in}{1.004420in}}{\pgfqpoint{0.181577in}{1.008660in}}{\pgfqpoint{0.178452in}{1.011785in}}%
\pgfpathcurveto{\pgfqpoint{0.175326in}{1.014911in}}{\pgfqpoint{0.171087in}{1.016667in}}{\pgfqpoint{0.166667in}{1.016667in}}%
\pgfpathcurveto{\pgfqpoint{0.162247in}{1.016667in}}{\pgfqpoint{0.158007in}{1.014911in}}{\pgfqpoint{0.154882in}{1.011785in}}%
\pgfpathcurveto{\pgfqpoint{0.151756in}{1.008660in}}{\pgfqpoint{0.150000in}{1.004420in}}{\pgfqpoint{0.150000in}{1.000000in}}%
\pgfpathcurveto{\pgfqpoint{0.150000in}{0.995580in}}{\pgfqpoint{0.151756in}{0.991340in}}{\pgfqpoint{0.154882in}{0.988215in}}%
\pgfpathcurveto{\pgfqpoint{0.158007in}{0.985089in}}{\pgfqpoint{0.162247in}{0.983333in}}{\pgfqpoint{0.166667in}{0.983333in}}%
\pgfpathclose%
\pgfpathmoveto{\pgfqpoint{0.333333in}{0.983333in}}%
\pgfpathcurveto{\pgfqpoint{0.337753in}{0.983333in}}{\pgfqpoint{0.341993in}{0.985089in}}{\pgfqpoint{0.345118in}{0.988215in}}%
\pgfpathcurveto{\pgfqpoint{0.348244in}{0.991340in}}{\pgfqpoint{0.350000in}{0.995580in}}{\pgfqpoint{0.350000in}{1.000000in}}%
\pgfpathcurveto{\pgfqpoint{0.350000in}{1.004420in}}{\pgfqpoint{0.348244in}{1.008660in}}{\pgfqpoint{0.345118in}{1.011785in}}%
\pgfpathcurveto{\pgfqpoint{0.341993in}{1.014911in}}{\pgfqpoint{0.337753in}{1.016667in}}{\pgfqpoint{0.333333in}{1.016667in}}%
\pgfpathcurveto{\pgfqpoint{0.328913in}{1.016667in}}{\pgfqpoint{0.324674in}{1.014911in}}{\pgfqpoint{0.321548in}{1.011785in}}%
\pgfpathcurveto{\pgfqpoint{0.318423in}{1.008660in}}{\pgfqpoint{0.316667in}{1.004420in}}{\pgfqpoint{0.316667in}{1.000000in}}%
\pgfpathcurveto{\pgfqpoint{0.316667in}{0.995580in}}{\pgfqpoint{0.318423in}{0.991340in}}{\pgfqpoint{0.321548in}{0.988215in}}%
\pgfpathcurveto{\pgfqpoint{0.324674in}{0.985089in}}{\pgfqpoint{0.328913in}{0.983333in}}{\pgfqpoint{0.333333in}{0.983333in}}%
\pgfpathclose%
\pgfpathmoveto{\pgfqpoint{0.500000in}{0.983333in}}%
\pgfpathcurveto{\pgfqpoint{0.504420in}{0.983333in}}{\pgfqpoint{0.508660in}{0.985089in}}{\pgfqpoint{0.511785in}{0.988215in}}%
\pgfpathcurveto{\pgfqpoint{0.514911in}{0.991340in}}{\pgfqpoint{0.516667in}{0.995580in}}{\pgfqpoint{0.516667in}{1.000000in}}%
\pgfpathcurveto{\pgfqpoint{0.516667in}{1.004420in}}{\pgfqpoint{0.514911in}{1.008660in}}{\pgfqpoint{0.511785in}{1.011785in}}%
\pgfpathcurveto{\pgfqpoint{0.508660in}{1.014911in}}{\pgfqpoint{0.504420in}{1.016667in}}{\pgfqpoint{0.500000in}{1.016667in}}%
\pgfpathcurveto{\pgfqpoint{0.495580in}{1.016667in}}{\pgfqpoint{0.491340in}{1.014911in}}{\pgfqpoint{0.488215in}{1.011785in}}%
\pgfpathcurveto{\pgfqpoint{0.485089in}{1.008660in}}{\pgfqpoint{0.483333in}{1.004420in}}{\pgfqpoint{0.483333in}{1.000000in}}%
\pgfpathcurveto{\pgfqpoint{0.483333in}{0.995580in}}{\pgfqpoint{0.485089in}{0.991340in}}{\pgfqpoint{0.488215in}{0.988215in}}%
\pgfpathcurveto{\pgfqpoint{0.491340in}{0.985089in}}{\pgfqpoint{0.495580in}{0.983333in}}{\pgfqpoint{0.500000in}{0.983333in}}%
\pgfpathclose%
\pgfpathmoveto{\pgfqpoint{0.666667in}{0.983333in}}%
\pgfpathcurveto{\pgfqpoint{0.671087in}{0.983333in}}{\pgfqpoint{0.675326in}{0.985089in}}{\pgfqpoint{0.678452in}{0.988215in}}%
\pgfpathcurveto{\pgfqpoint{0.681577in}{0.991340in}}{\pgfqpoint{0.683333in}{0.995580in}}{\pgfqpoint{0.683333in}{1.000000in}}%
\pgfpathcurveto{\pgfqpoint{0.683333in}{1.004420in}}{\pgfqpoint{0.681577in}{1.008660in}}{\pgfqpoint{0.678452in}{1.011785in}}%
\pgfpathcurveto{\pgfqpoint{0.675326in}{1.014911in}}{\pgfqpoint{0.671087in}{1.016667in}}{\pgfqpoint{0.666667in}{1.016667in}}%
\pgfpathcurveto{\pgfqpoint{0.662247in}{1.016667in}}{\pgfqpoint{0.658007in}{1.014911in}}{\pgfqpoint{0.654882in}{1.011785in}}%
\pgfpathcurveto{\pgfqpoint{0.651756in}{1.008660in}}{\pgfqpoint{0.650000in}{1.004420in}}{\pgfqpoint{0.650000in}{1.000000in}}%
\pgfpathcurveto{\pgfqpoint{0.650000in}{0.995580in}}{\pgfqpoint{0.651756in}{0.991340in}}{\pgfqpoint{0.654882in}{0.988215in}}%
\pgfpathcurveto{\pgfqpoint{0.658007in}{0.985089in}}{\pgfqpoint{0.662247in}{0.983333in}}{\pgfqpoint{0.666667in}{0.983333in}}%
\pgfpathclose%
\pgfpathmoveto{\pgfqpoint{0.833333in}{0.983333in}}%
\pgfpathcurveto{\pgfqpoint{0.837753in}{0.983333in}}{\pgfqpoint{0.841993in}{0.985089in}}{\pgfqpoint{0.845118in}{0.988215in}}%
\pgfpathcurveto{\pgfqpoint{0.848244in}{0.991340in}}{\pgfqpoint{0.850000in}{0.995580in}}{\pgfqpoint{0.850000in}{1.000000in}}%
\pgfpathcurveto{\pgfqpoint{0.850000in}{1.004420in}}{\pgfqpoint{0.848244in}{1.008660in}}{\pgfqpoint{0.845118in}{1.011785in}}%
\pgfpathcurveto{\pgfqpoint{0.841993in}{1.014911in}}{\pgfqpoint{0.837753in}{1.016667in}}{\pgfqpoint{0.833333in}{1.016667in}}%
\pgfpathcurveto{\pgfqpoint{0.828913in}{1.016667in}}{\pgfqpoint{0.824674in}{1.014911in}}{\pgfqpoint{0.821548in}{1.011785in}}%
\pgfpathcurveto{\pgfqpoint{0.818423in}{1.008660in}}{\pgfqpoint{0.816667in}{1.004420in}}{\pgfqpoint{0.816667in}{1.000000in}}%
\pgfpathcurveto{\pgfqpoint{0.816667in}{0.995580in}}{\pgfqpoint{0.818423in}{0.991340in}}{\pgfqpoint{0.821548in}{0.988215in}}%
\pgfpathcurveto{\pgfqpoint{0.824674in}{0.985089in}}{\pgfqpoint{0.828913in}{0.983333in}}{\pgfqpoint{0.833333in}{0.983333in}}%
\pgfpathclose%
\pgfpathmoveto{\pgfqpoint{1.000000in}{0.983333in}}%
\pgfpathcurveto{\pgfqpoint{1.004420in}{0.983333in}}{\pgfqpoint{1.008660in}{0.985089in}}{\pgfqpoint{1.011785in}{0.988215in}}%
\pgfpathcurveto{\pgfqpoint{1.014911in}{0.991340in}}{\pgfqpoint{1.016667in}{0.995580in}}{\pgfqpoint{1.016667in}{1.000000in}}%
\pgfpathcurveto{\pgfqpoint{1.016667in}{1.004420in}}{\pgfqpoint{1.014911in}{1.008660in}}{\pgfqpoint{1.011785in}{1.011785in}}%
\pgfpathcurveto{\pgfqpoint{1.008660in}{1.014911in}}{\pgfqpoint{1.004420in}{1.016667in}}{\pgfqpoint{1.000000in}{1.016667in}}%
\pgfpathcurveto{\pgfqpoint{0.995580in}{1.016667in}}{\pgfqpoint{0.991340in}{1.014911in}}{\pgfqpoint{0.988215in}{1.011785in}}%
\pgfpathcurveto{\pgfqpoint{0.985089in}{1.008660in}}{\pgfqpoint{0.983333in}{1.004420in}}{\pgfqpoint{0.983333in}{1.000000in}}%
\pgfpathcurveto{\pgfqpoint{0.983333in}{0.995580in}}{\pgfqpoint{0.985089in}{0.991340in}}{\pgfqpoint{0.988215in}{0.988215in}}%
\pgfpathcurveto{\pgfqpoint{0.991340in}{0.985089in}}{\pgfqpoint{0.995580in}{0.983333in}}{\pgfqpoint{1.000000in}{0.983333in}}%
\pgfpathclose%
\pgfusepath{stroke}%
\end{pgfscope}%
}%
\pgfsys@transformshift{1.323315in}{2.460694in}%
\pgfsys@useobject{currentpattern}{}%
\pgfsys@transformshift{1in}{0in}%
\pgfsys@transformshift{-1in}{0in}%
\pgfsys@transformshift{0in}{1in}%
\end{pgfscope}%
\begin{pgfscope}%
\pgfpathrectangle{\pgfqpoint{0.935815in}{0.637495in}}{\pgfqpoint{9.300000in}{9.060000in}}%
\pgfusepath{clip}%
\pgfsetbuttcap%
\pgfsetmiterjoin%
\definecolor{currentfill}{rgb}{0.172549,0.627451,0.172549}%
\pgfsetfillcolor{currentfill}%
\pgfsetfillopacity{0.990000}%
\pgfsetlinewidth{0.000000pt}%
\definecolor{currentstroke}{rgb}{0.000000,0.000000,0.000000}%
\pgfsetstrokecolor{currentstroke}%
\pgfsetstrokeopacity{0.990000}%
\pgfsetdash{}{0pt}%
\pgfpathmoveto{\pgfqpoint{2.873315in}{3.551179in}}%
\pgfpathlineto{\pgfqpoint{3.648315in}{3.551179in}}%
\pgfpathlineto{\pgfqpoint{3.648315in}{3.755575in}}%
\pgfpathlineto{\pgfqpoint{2.873315in}{3.755575in}}%
\pgfpathclose%
\pgfusepath{fill}%
\end{pgfscope}%
\begin{pgfscope}%
\pgfsetbuttcap%
\pgfsetmiterjoin%
\definecolor{currentfill}{rgb}{0.172549,0.627451,0.172549}%
\pgfsetfillcolor{currentfill}%
\pgfsetfillopacity{0.990000}%
\pgfsetlinewidth{0.000000pt}%
\definecolor{currentstroke}{rgb}{0.000000,0.000000,0.000000}%
\pgfsetstrokecolor{currentstroke}%
\pgfsetstrokeopacity{0.990000}%
\pgfsetdash{}{0pt}%
\pgfpathrectangle{\pgfqpoint{0.935815in}{0.637495in}}{\pgfqpoint{9.300000in}{9.060000in}}%
\pgfusepath{clip}%
\pgfpathmoveto{\pgfqpoint{2.873315in}{3.551179in}}%
\pgfpathlineto{\pgfqpoint{3.648315in}{3.551179in}}%
\pgfpathlineto{\pgfqpoint{3.648315in}{3.755575in}}%
\pgfpathlineto{\pgfqpoint{2.873315in}{3.755575in}}%
\pgfpathclose%
\pgfusepath{clip}%
\pgfsys@defobject{currentpattern}{\pgfqpoint{0in}{0in}}{\pgfqpoint{1in}{1in}}{%
\begin{pgfscope}%
\pgfpathrectangle{\pgfqpoint{0in}{0in}}{\pgfqpoint{1in}{1in}}%
\pgfusepath{clip}%
\pgfpathmoveto{\pgfqpoint{0.000000in}{-0.016667in}}%
\pgfpathcurveto{\pgfqpoint{0.004420in}{-0.016667in}}{\pgfqpoint{0.008660in}{-0.014911in}}{\pgfqpoint{0.011785in}{-0.011785in}}%
\pgfpathcurveto{\pgfqpoint{0.014911in}{-0.008660in}}{\pgfqpoint{0.016667in}{-0.004420in}}{\pgfqpoint{0.016667in}{0.000000in}}%
\pgfpathcurveto{\pgfqpoint{0.016667in}{0.004420in}}{\pgfqpoint{0.014911in}{0.008660in}}{\pgfqpoint{0.011785in}{0.011785in}}%
\pgfpathcurveto{\pgfqpoint{0.008660in}{0.014911in}}{\pgfqpoint{0.004420in}{0.016667in}}{\pgfqpoint{0.000000in}{0.016667in}}%
\pgfpathcurveto{\pgfqpoint{-0.004420in}{0.016667in}}{\pgfqpoint{-0.008660in}{0.014911in}}{\pgfqpoint{-0.011785in}{0.011785in}}%
\pgfpathcurveto{\pgfqpoint{-0.014911in}{0.008660in}}{\pgfqpoint{-0.016667in}{0.004420in}}{\pgfqpoint{-0.016667in}{0.000000in}}%
\pgfpathcurveto{\pgfqpoint{-0.016667in}{-0.004420in}}{\pgfqpoint{-0.014911in}{-0.008660in}}{\pgfqpoint{-0.011785in}{-0.011785in}}%
\pgfpathcurveto{\pgfqpoint{-0.008660in}{-0.014911in}}{\pgfqpoint{-0.004420in}{-0.016667in}}{\pgfqpoint{0.000000in}{-0.016667in}}%
\pgfpathclose%
\pgfpathmoveto{\pgfqpoint{0.166667in}{-0.016667in}}%
\pgfpathcurveto{\pgfqpoint{0.171087in}{-0.016667in}}{\pgfqpoint{0.175326in}{-0.014911in}}{\pgfqpoint{0.178452in}{-0.011785in}}%
\pgfpathcurveto{\pgfqpoint{0.181577in}{-0.008660in}}{\pgfqpoint{0.183333in}{-0.004420in}}{\pgfqpoint{0.183333in}{0.000000in}}%
\pgfpathcurveto{\pgfqpoint{0.183333in}{0.004420in}}{\pgfqpoint{0.181577in}{0.008660in}}{\pgfqpoint{0.178452in}{0.011785in}}%
\pgfpathcurveto{\pgfqpoint{0.175326in}{0.014911in}}{\pgfqpoint{0.171087in}{0.016667in}}{\pgfqpoint{0.166667in}{0.016667in}}%
\pgfpathcurveto{\pgfqpoint{0.162247in}{0.016667in}}{\pgfqpoint{0.158007in}{0.014911in}}{\pgfqpoint{0.154882in}{0.011785in}}%
\pgfpathcurveto{\pgfqpoint{0.151756in}{0.008660in}}{\pgfqpoint{0.150000in}{0.004420in}}{\pgfqpoint{0.150000in}{0.000000in}}%
\pgfpathcurveto{\pgfqpoint{0.150000in}{-0.004420in}}{\pgfqpoint{0.151756in}{-0.008660in}}{\pgfqpoint{0.154882in}{-0.011785in}}%
\pgfpathcurveto{\pgfqpoint{0.158007in}{-0.014911in}}{\pgfqpoint{0.162247in}{-0.016667in}}{\pgfqpoint{0.166667in}{-0.016667in}}%
\pgfpathclose%
\pgfpathmoveto{\pgfqpoint{0.333333in}{-0.016667in}}%
\pgfpathcurveto{\pgfqpoint{0.337753in}{-0.016667in}}{\pgfqpoint{0.341993in}{-0.014911in}}{\pgfqpoint{0.345118in}{-0.011785in}}%
\pgfpathcurveto{\pgfqpoint{0.348244in}{-0.008660in}}{\pgfqpoint{0.350000in}{-0.004420in}}{\pgfqpoint{0.350000in}{0.000000in}}%
\pgfpathcurveto{\pgfqpoint{0.350000in}{0.004420in}}{\pgfqpoint{0.348244in}{0.008660in}}{\pgfqpoint{0.345118in}{0.011785in}}%
\pgfpathcurveto{\pgfqpoint{0.341993in}{0.014911in}}{\pgfqpoint{0.337753in}{0.016667in}}{\pgfqpoint{0.333333in}{0.016667in}}%
\pgfpathcurveto{\pgfqpoint{0.328913in}{0.016667in}}{\pgfqpoint{0.324674in}{0.014911in}}{\pgfqpoint{0.321548in}{0.011785in}}%
\pgfpathcurveto{\pgfqpoint{0.318423in}{0.008660in}}{\pgfqpoint{0.316667in}{0.004420in}}{\pgfqpoint{0.316667in}{0.000000in}}%
\pgfpathcurveto{\pgfqpoint{0.316667in}{-0.004420in}}{\pgfqpoint{0.318423in}{-0.008660in}}{\pgfqpoint{0.321548in}{-0.011785in}}%
\pgfpathcurveto{\pgfqpoint{0.324674in}{-0.014911in}}{\pgfqpoint{0.328913in}{-0.016667in}}{\pgfqpoint{0.333333in}{-0.016667in}}%
\pgfpathclose%
\pgfpathmoveto{\pgfqpoint{0.500000in}{-0.016667in}}%
\pgfpathcurveto{\pgfqpoint{0.504420in}{-0.016667in}}{\pgfqpoint{0.508660in}{-0.014911in}}{\pgfqpoint{0.511785in}{-0.011785in}}%
\pgfpathcurveto{\pgfqpoint{0.514911in}{-0.008660in}}{\pgfqpoint{0.516667in}{-0.004420in}}{\pgfqpoint{0.516667in}{0.000000in}}%
\pgfpathcurveto{\pgfqpoint{0.516667in}{0.004420in}}{\pgfqpoint{0.514911in}{0.008660in}}{\pgfqpoint{0.511785in}{0.011785in}}%
\pgfpathcurveto{\pgfqpoint{0.508660in}{0.014911in}}{\pgfqpoint{0.504420in}{0.016667in}}{\pgfqpoint{0.500000in}{0.016667in}}%
\pgfpathcurveto{\pgfqpoint{0.495580in}{0.016667in}}{\pgfqpoint{0.491340in}{0.014911in}}{\pgfqpoint{0.488215in}{0.011785in}}%
\pgfpathcurveto{\pgfqpoint{0.485089in}{0.008660in}}{\pgfqpoint{0.483333in}{0.004420in}}{\pgfqpoint{0.483333in}{0.000000in}}%
\pgfpathcurveto{\pgfqpoint{0.483333in}{-0.004420in}}{\pgfqpoint{0.485089in}{-0.008660in}}{\pgfqpoint{0.488215in}{-0.011785in}}%
\pgfpathcurveto{\pgfqpoint{0.491340in}{-0.014911in}}{\pgfqpoint{0.495580in}{-0.016667in}}{\pgfqpoint{0.500000in}{-0.016667in}}%
\pgfpathclose%
\pgfpathmoveto{\pgfqpoint{0.666667in}{-0.016667in}}%
\pgfpathcurveto{\pgfqpoint{0.671087in}{-0.016667in}}{\pgfqpoint{0.675326in}{-0.014911in}}{\pgfqpoint{0.678452in}{-0.011785in}}%
\pgfpathcurveto{\pgfqpoint{0.681577in}{-0.008660in}}{\pgfqpoint{0.683333in}{-0.004420in}}{\pgfqpoint{0.683333in}{0.000000in}}%
\pgfpathcurveto{\pgfqpoint{0.683333in}{0.004420in}}{\pgfqpoint{0.681577in}{0.008660in}}{\pgfqpoint{0.678452in}{0.011785in}}%
\pgfpathcurveto{\pgfqpoint{0.675326in}{0.014911in}}{\pgfqpoint{0.671087in}{0.016667in}}{\pgfqpoint{0.666667in}{0.016667in}}%
\pgfpathcurveto{\pgfqpoint{0.662247in}{0.016667in}}{\pgfqpoint{0.658007in}{0.014911in}}{\pgfqpoint{0.654882in}{0.011785in}}%
\pgfpathcurveto{\pgfqpoint{0.651756in}{0.008660in}}{\pgfqpoint{0.650000in}{0.004420in}}{\pgfqpoint{0.650000in}{0.000000in}}%
\pgfpathcurveto{\pgfqpoint{0.650000in}{-0.004420in}}{\pgfqpoint{0.651756in}{-0.008660in}}{\pgfqpoint{0.654882in}{-0.011785in}}%
\pgfpathcurveto{\pgfqpoint{0.658007in}{-0.014911in}}{\pgfqpoint{0.662247in}{-0.016667in}}{\pgfqpoint{0.666667in}{-0.016667in}}%
\pgfpathclose%
\pgfpathmoveto{\pgfqpoint{0.833333in}{-0.016667in}}%
\pgfpathcurveto{\pgfqpoint{0.837753in}{-0.016667in}}{\pgfqpoint{0.841993in}{-0.014911in}}{\pgfqpoint{0.845118in}{-0.011785in}}%
\pgfpathcurveto{\pgfqpoint{0.848244in}{-0.008660in}}{\pgfqpoint{0.850000in}{-0.004420in}}{\pgfqpoint{0.850000in}{0.000000in}}%
\pgfpathcurveto{\pgfqpoint{0.850000in}{0.004420in}}{\pgfqpoint{0.848244in}{0.008660in}}{\pgfqpoint{0.845118in}{0.011785in}}%
\pgfpathcurveto{\pgfqpoint{0.841993in}{0.014911in}}{\pgfqpoint{0.837753in}{0.016667in}}{\pgfqpoint{0.833333in}{0.016667in}}%
\pgfpathcurveto{\pgfqpoint{0.828913in}{0.016667in}}{\pgfqpoint{0.824674in}{0.014911in}}{\pgfqpoint{0.821548in}{0.011785in}}%
\pgfpathcurveto{\pgfqpoint{0.818423in}{0.008660in}}{\pgfqpoint{0.816667in}{0.004420in}}{\pgfqpoint{0.816667in}{0.000000in}}%
\pgfpathcurveto{\pgfqpoint{0.816667in}{-0.004420in}}{\pgfqpoint{0.818423in}{-0.008660in}}{\pgfqpoint{0.821548in}{-0.011785in}}%
\pgfpathcurveto{\pgfqpoint{0.824674in}{-0.014911in}}{\pgfqpoint{0.828913in}{-0.016667in}}{\pgfqpoint{0.833333in}{-0.016667in}}%
\pgfpathclose%
\pgfpathmoveto{\pgfqpoint{1.000000in}{-0.016667in}}%
\pgfpathcurveto{\pgfqpoint{1.004420in}{-0.016667in}}{\pgfqpoint{1.008660in}{-0.014911in}}{\pgfqpoint{1.011785in}{-0.011785in}}%
\pgfpathcurveto{\pgfqpoint{1.014911in}{-0.008660in}}{\pgfqpoint{1.016667in}{-0.004420in}}{\pgfqpoint{1.016667in}{0.000000in}}%
\pgfpathcurveto{\pgfqpoint{1.016667in}{0.004420in}}{\pgfqpoint{1.014911in}{0.008660in}}{\pgfqpoint{1.011785in}{0.011785in}}%
\pgfpathcurveto{\pgfqpoint{1.008660in}{0.014911in}}{\pgfqpoint{1.004420in}{0.016667in}}{\pgfqpoint{1.000000in}{0.016667in}}%
\pgfpathcurveto{\pgfqpoint{0.995580in}{0.016667in}}{\pgfqpoint{0.991340in}{0.014911in}}{\pgfqpoint{0.988215in}{0.011785in}}%
\pgfpathcurveto{\pgfqpoint{0.985089in}{0.008660in}}{\pgfqpoint{0.983333in}{0.004420in}}{\pgfqpoint{0.983333in}{0.000000in}}%
\pgfpathcurveto{\pgfqpoint{0.983333in}{-0.004420in}}{\pgfqpoint{0.985089in}{-0.008660in}}{\pgfqpoint{0.988215in}{-0.011785in}}%
\pgfpathcurveto{\pgfqpoint{0.991340in}{-0.014911in}}{\pgfqpoint{0.995580in}{-0.016667in}}{\pgfqpoint{1.000000in}{-0.016667in}}%
\pgfpathclose%
\pgfpathmoveto{\pgfqpoint{0.083333in}{0.150000in}}%
\pgfpathcurveto{\pgfqpoint{0.087753in}{0.150000in}}{\pgfqpoint{0.091993in}{0.151756in}}{\pgfqpoint{0.095118in}{0.154882in}}%
\pgfpathcurveto{\pgfqpoint{0.098244in}{0.158007in}}{\pgfqpoint{0.100000in}{0.162247in}}{\pgfqpoint{0.100000in}{0.166667in}}%
\pgfpathcurveto{\pgfqpoint{0.100000in}{0.171087in}}{\pgfqpoint{0.098244in}{0.175326in}}{\pgfqpoint{0.095118in}{0.178452in}}%
\pgfpathcurveto{\pgfqpoint{0.091993in}{0.181577in}}{\pgfqpoint{0.087753in}{0.183333in}}{\pgfqpoint{0.083333in}{0.183333in}}%
\pgfpathcurveto{\pgfqpoint{0.078913in}{0.183333in}}{\pgfqpoint{0.074674in}{0.181577in}}{\pgfqpoint{0.071548in}{0.178452in}}%
\pgfpathcurveto{\pgfqpoint{0.068423in}{0.175326in}}{\pgfqpoint{0.066667in}{0.171087in}}{\pgfqpoint{0.066667in}{0.166667in}}%
\pgfpathcurveto{\pgfqpoint{0.066667in}{0.162247in}}{\pgfqpoint{0.068423in}{0.158007in}}{\pgfqpoint{0.071548in}{0.154882in}}%
\pgfpathcurveto{\pgfqpoint{0.074674in}{0.151756in}}{\pgfqpoint{0.078913in}{0.150000in}}{\pgfqpoint{0.083333in}{0.150000in}}%
\pgfpathclose%
\pgfpathmoveto{\pgfqpoint{0.250000in}{0.150000in}}%
\pgfpathcurveto{\pgfqpoint{0.254420in}{0.150000in}}{\pgfqpoint{0.258660in}{0.151756in}}{\pgfqpoint{0.261785in}{0.154882in}}%
\pgfpathcurveto{\pgfqpoint{0.264911in}{0.158007in}}{\pgfqpoint{0.266667in}{0.162247in}}{\pgfqpoint{0.266667in}{0.166667in}}%
\pgfpathcurveto{\pgfqpoint{0.266667in}{0.171087in}}{\pgfqpoint{0.264911in}{0.175326in}}{\pgfqpoint{0.261785in}{0.178452in}}%
\pgfpathcurveto{\pgfqpoint{0.258660in}{0.181577in}}{\pgfqpoint{0.254420in}{0.183333in}}{\pgfqpoint{0.250000in}{0.183333in}}%
\pgfpathcurveto{\pgfqpoint{0.245580in}{0.183333in}}{\pgfqpoint{0.241340in}{0.181577in}}{\pgfqpoint{0.238215in}{0.178452in}}%
\pgfpathcurveto{\pgfqpoint{0.235089in}{0.175326in}}{\pgfqpoint{0.233333in}{0.171087in}}{\pgfqpoint{0.233333in}{0.166667in}}%
\pgfpathcurveto{\pgfqpoint{0.233333in}{0.162247in}}{\pgfqpoint{0.235089in}{0.158007in}}{\pgfqpoint{0.238215in}{0.154882in}}%
\pgfpathcurveto{\pgfqpoint{0.241340in}{0.151756in}}{\pgfqpoint{0.245580in}{0.150000in}}{\pgfqpoint{0.250000in}{0.150000in}}%
\pgfpathclose%
\pgfpathmoveto{\pgfqpoint{0.416667in}{0.150000in}}%
\pgfpathcurveto{\pgfqpoint{0.421087in}{0.150000in}}{\pgfqpoint{0.425326in}{0.151756in}}{\pgfqpoint{0.428452in}{0.154882in}}%
\pgfpathcurveto{\pgfqpoint{0.431577in}{0.158007in}}{\pgfqpoint{0.433333in}{0.162247in}}{\pgfqpoint{0.433333in}{0.166667in}}%
\pgfpathcurveto{\pgfqpoint{0.433333in}{0.171087in}}{\pgfqpoint{0.431577in}{0.175326in}}{\pgfqpoint{0.428452in}{0.178452in}}%
\pgfpathcurveto{\pgfqpoint{0.425326in}{0.181577in}}{\pgfqpoint{0.421087in}{0.183333in}}{\pgfqpoint{0.416667in}{0.183333in}}%
\pgfpathcurveto{\pgfqpoint{0.412247in}{0.183333in}}{\pgfqpoint{0.408007in}{0.181577in}}{\pgfqpoint{0.404882in}{0.178452in}}%
\pgfpathcurveto{\pgfqpoint{0.401756in}{0.175326in}}{\pgfqpoint{0.400000in}{0.171087in}}{\pgfqpoint{0.400000in}{0.166667in}}%
\pgfpathcurveto{\pgfqpoint{0.400000in}{0.162247in}}{\pgfqpoint{0.401756in}{0.158007in}}{\pgfqpoint{0.404882in}{0.154882in}}%
\pgfpathcurveto{\pgfqpoint{0.408007in}{0.151756in}}{\pgfqpoint{0.412247in}{0.150000in}}{\pgfqpoint{0.416667in}{0.150000in}}%
\pgfpathclose%
\pgfpathmoveto{\pgfqpoint{0.583333in}{0.150000in}}%
\pgfpathcurveto{\pgfqpoint{0.587753in}{0.150000in}}{\pgfqpoint{0.591993in}{0.151756in}}{\pgfqpoint{0.595118in}{0.154882in}}%
\pgfpathcurveto{\pgfqpoint{0.598244in}{0.158007in}}{\pgfqpoint{0.600000in}{0.162247in}}{\pgfqpoint{0.600000in}{0.166667in}}%
\pgfpathcurveto{\pgfqpoint{0.600000in}{0.171087in}}{\pgfqpoint{0.598244in}{0.175326in}}{\pgfqpoint{0.595118in}{0.178452in}}%
\pgfpathcurveto{\pgfqpoint{0.591993in}{0.181577in}}{\pgfqpoint{0.587753in}{0.183333in}}{\pgfqpoint{0.583333in}{0.183333in}}%
\pgfpathcurveto{\pgfqpoint{0.578913in}{0.183333in}}{\pgfqpoint{0.574674in}{0.181577in}}{\pgfqpoint{0.571548in}{0.178452in}}%
\pgfpathcurveto{\pgfqpoint{0.568423in}{0.175326in}}{\pgfqpoint{0.566667in}{0.171087in}}{\pgfqpoint{0.566667in}{0.166667in}}%
\pgfpathcurveto{\pgfqpoint{0.566667in}{0.162247in}}{\pgfqpoint{0.568423in}{0.158007in}}{\pgfqpoint{0.571548in}{0.154882in}}%
\pgfpathcurveto{\pgfqpoint{0.574674in}{0.151756in}}{\pgfqpoint{0.578913in}{0.150000in}}{\pgfqpoint{0.583333in}{0.150000in}}%
\pgfpathclose%
\pgfpathmoveto{\pgfqpoint{0.750000in}{0.150000in}}%
\pgfpathcurveto{\pgfqpoint{0.754420in}{0.150000in}}{\pgfqpoint{0.758660in}{0.151756in}}{\pgfqpoint{0.761785in}{0.154882in}}%
\pgfpathcurveto{\pgfqpoint{0.764911in}{0.158007in}}{\pgfqpoint{0.766667in}{0.162247in}}{\pgfqpoint{0.766667in}{0.166667in}}%
\pgfpathcurveto{\pgfqpoint{0.766667in}{0.171087in}}{\pgfqpoint{0.764911in}{0.175326in}}{\pgfqpoint{0.761785in}{0.178452in}}%
\pgfpathcurveto{\pgfqpoint{0.758660in}{0.181577in}}{\pgfqpoint{0.754420in}{0.183333in}}{\pgfqpoint{0.750000in}{0.183333in}}%
\pgfpathcurveto{\pgfqpoint{0.745580in}{0.183333in}}{\pgfqpoint{0.741340in}{0.181577in}}{\pgfqpoint{0.738215in}{0.178452in}}%
\pgfpathcurveto{\pgfqpoint{0.735089in}{0.175326in}}{\pgfqpoint{0.733333in}{0.171087in}}{\pgfqpoint{0.733333in}{0.166667in}}%
\pgfpathcurveto{\pgfqpoint{0.733333in}{0.162247in}}{\pgfqpoint{0.735089in}{0.158007in}}{\pgfqpoint{0.738215in}{0.154882in}}%
\pgfpathcurveto{\pgfqpoint{0.741340in}{0.151756in}}{\pgfqpoint{0.745580in}{0.150000in}}{\pgfqpoint{0.750000in}{0.150000in}}%
\pgfpathclose%
\pgfpathmoveto{\pgfqpoint{0.916667in}{0.150000in}}%
\pgfpathcurveto{\pgfqpoint{0.921087in}{0.150000in}}{\pgfqpoint{0.925326in}{0.151756in}}{\pgfqpoint{0.928452in}{0.154882in}}%
\pgfpathcurveto{\pgfqpoint{0.931577in}{0.158007in}}{\pgfqpoint{0.933333in}{0.162247in}}{\pgfqpoint{0.933333in}{0.166667in}}%
\pgfpathcurveto{\pgfqpoint{0.933333in}{0.171087in}}{\pgfqpoint{0.931577in}{0.175326in}}{\pgfqpoint{0.928452in}{0.178452in}}%
\pgfpathcurveto{\pgfqpoint{0.925326in}{0.181577in}}{\pgfqpoint{0.921087in}{0.183333in}}{\pgfqpoint{0.916667in}{0.183333in}}%
\pgfpathcurveto{\pgfqpoint{0.912247in}{0.183333in}}{\pgfqpoint{0.908007in}{0.181577in}}{\pgfqpoint{0.904882in}{0.178452in}}%
\pgfpathcurveto{\pgfqpoint{0.901756in}{0.175326in}}{\pgfqpoint{0.900000in}{0.171087in}}{\pgfqpoint{0.900000in}{0.166667in}}%
\pgfpathcurveto{\pgfqpoint{0.900000in}{0.162247in}}{\pgfqpoint{0.901756in}{0.158007in}}{\pgfqpoint{0.904882in}{0.154882in}}%
\pgfpathcurveto{\pgfqpoint{0.908007in}{0.151756in}}{\pgfqpoint{0.912247in}{0.150000in}}{\pgfqpoint{0.916667in}{0.150000in}}%
\pgfpathclose%
\pgfpathmoveto{\pgfqpoint{0.000000in}{0.316667in}}%
\pgfpathcurveto{\pgfqpoint{0.004420in}{0.316667in}}{\pgfqpoint{0.008660in}{0.318423in}}{\pgfqpoint{0.011785in}{0.321548in}}%
\pgfpathcurveto{\pgfqpoint{0.014911in}{0.324674in}}{\pgfqpoint{0.016667in}{0.328913in}}{\pgfqpoint{0.016667in}{0.333333in}}%
\pgfpathcurveto{\pgfqpoint{0.016667in}{0.337753in}}{\pgfqpoint{0.014911in}{0.341993in}}{\pgfqpoint{0.011785in}{0.345118in}}%
\pgfpathcurveto{\pgfqpoint{0.008660in}{0.348244in}}{\pgfqpoint{0.004420in}{0.350000in}}{\pgfqpoint{0.000000in}{0.350000in}}%
\pgfpathcurveto{\pgfqpoint{-0.004420in}{0.350000in}}{\pgfqpoint{-0.008660in}{0.348244in}}{\pgfqpoint{-0.011785in}{0.345118in}}%
\pgfpathcurveto{\pgfqpoint{-0.014911in}{0.341993in}}{\pgfqpoint{-0.016667in}{0.337753in}}{\pgfqpoint{-0.016667in}{0.333333in}}%
\pgfpathcurveto{\pgfqpoint{-0.016667in}{0.328913in}}{\pgfqpoint{-0.014911in}{0.324674in}}{\pgfqpoint{-0.011785in}{0.321548in}}%
\pgfpathcurveto{\pgfqpoint{-0.008660in}{0.318423in}}{\pgfqpoint{-0.004420in}{0.316667in}}{\pgfqpoint{0.000000in}{0.316667in}}%
\pgfpathclose%
\pgfpathmoveto{\pgfqpoint{0.166667in}{0.316667in}}%
\pgfpathcurveto{\pgfqpoint{0.171087in}{0.316667in}}{\pgfqpoint{0.175326in}{0.318423in}}{\pgfqpoint{0.178452in}{0.321548in}}%
\pgfpathcurveto{\pgfqpoint{0.181577in}{0.324674in}}{\pgfqpoint{0.183333in}{0.328913in}}{\pgfqpoint{0.183333in}{0.333333in}}%
\pgfpathcurveto{\pgfqpoint{0.183333in}{0.337753in}}{\pgfqpoint{0.181577in}{0.341993in}}{\pgfqpoint{0.178452in}{0.345118in}}%
\pgfpathcurveto{\pgfqpoint{0.175326in}{0.348244in}}{\pgfqpoint{0.171087in}{0.350000in}}{\pgfqpoint{0.166667in}{0.350000in}}%
\pgfpathcurveto{\pgfqpoint{0.162247in}{0.350000in}}{\pgfqpoint{0.158007in}{0.348244in}}{\pgfqpoint{0.154882in}{0.345118in}}%
\pgfpathcurveto{\pgfqpoint{0.151756in}{0.341993in}}{\pgfqpoint{0.150000in}{0.337753in}}{\pgfqpoint{0.150000in}{0.333333in}}%
\pgfpathcurveto{\pgfqpoint{0.150000in}{0.328913in}}{\pgfqpoint{0.151756in}{0.324674in}}{\pgfqpoint{0.154882in}{0.321548in}}%
\pgfpathcurveto{\pgfqpoint{0.158007in}{0.318423in}}{\pgfqpoint{0.162247in}{0.316667in}}{\pgfqpoint{0.166667in}{0.316667in}}%
\pgfpathclose%
\pgfpathmoveto{\pgfqpoint{0.333333in}{0.316667in}}%
\pgfpathcurveto{\pgfqpoint{0.337753in}{0.316667in}}{\pgfqpoint{0.341993in}{0.318423in}}{\pgfqpoint{0.345118in}{0.321548in}}%
\pgfpathcurveto{\pgfqpoint{0.348244in}{0.324674in}}{\pgfqpoint{0.350000in}{0.328913in}}{\pgfqpoint{0.350000in}{0.333333in}}%
\pgfpathcurveto{\pgfqpoint{0.350000in}{0.337753in}}{\pgfqpoint{0.348244in}{0.341993in}}{\pgfqpoint{0.345118in}{0.345118in}}%
\pgfpathcurveto{\pgfqpoint{0.341993in}{0.348244in}}{\pgfqpoint{0.337753in}{0.350000in}}{\pgfqpoint{0.333333in}{0.350000in}}%
\pgfpathcurveto{\pgfqpoint{0.328913in}{0.350000in}}{\pgfqpoint{0.324674in}{0.348244in}}{\pgfqpoint{0.321548in}{0.345118in}}%
\pgfpathcurveto{\pgfqpoint{0.318423in}{0.341993in}}{\pgfqpoint{0.316667in}{0.337753in}}{\pgfqpoint{0.316667in}{0.333333in}}%
\pgfpathcurveto{\pgfqpoint{0.316667in}{0.328913in}}{\pgfqpoint{0.318423in}{0.324674in}}{\pgfqpoint{0.321548in}{0.321548in}}%
\pgfpathcurveto{\pgfqpoint{0.324674in}{0.318423in}}{\pgfqpoint{0.328913in}{0.316667in}}{\pgfqpoint{0.333333in}{0.316667in}}%
\pgfpathclose%
\pgfpathmoveto{\pgfqpoint{0.500000in}{0.316667in}}%
\pgfpathcurveto{\pgfqpoint{0.504420in}{0.316667in}}{\pgfqpoint{0.508660in}{0.318423in}}{\pgfqpoint{0.511785in}{0.321548in}}%
\pgfpathcurveto{\pgfqpoint{0.514911in}{0.324674in}}{\pgfqpoint{0.516667in}{0.328913in}}{\pgfqpoint{0.516667in}{0.333333in}}%
\pgfpathcurveto{\pgfqpoint{0.516667in}{0.337753in}}{\pgfqpoint{0.514911in}{0.341993in}}{\pgfqpoint{0.511785in}{0.345118in}}%
\pgfpathcurveto{\pgfqpoint{0.508660in}{0.348244in}}{\pgfqpoint{0.504420in}{0.350000in}}{\pgfqpoint{0.500000in}{0.350000in}}%
\pgfpathcurveto{\pgfqpoint{0.495580in}{0.350000in}}{\pgfqpoint{0.491340in}{0.348244in}}{\pgfqpoint{0.488215in}{0.345118in}}%
\pgfpathcurveto{\pgfqpoint{0.485089in}{0.341993in}}{\pgfqpoint{0.483333in}{0.337753in}}{\pgfqpoint{0.483333in}{0.333333in}}%
\pgfpathcurveto{\pgfqpoint{0.483333in}{0.328913in}}{\pgfqpoint{0.485089in}{0.324674in}}{\pgfqpoint{0.488215in}{0.321548in}}%
\pgfpathcurveto{\pgfqpoint{0.491340in}{0.318423in}}{\pgfqpoint{0.495580in}{0.316667in}}{\pgfqpoint{0.500000in}{0.316667in}}%
\pgfpathclose%
\pgfpathmoveto{\pgfqpoint{0.666667in}{0.316667in}}%
\pgfpathcurveto{\pgfqpoint{0.671087in}{0.316667in}}{\pgfqpoint{0.675326in}{0.318423in}}{\pgfqpoint{0.678452in}{0.321548in}}%
\pgfpathcurveto{\pgfqpoint{0.681577in}{0.324674in}}{\pgfqpoint{0.683333in}{0.328913in}}{\pgfqpoint{0.683333in}{0.333333in}}%
\pgfpathcurveto{\pgfqpoint{0.683333in}{0.337753in}}{\pgfqpoint{0.681577in}{0.341993in}}{\pgfqpoint{0.678452in}{0.345118in}}%
\pgfpathcurveto{\pgfqpoint{0.675326in}{0.348244in}}{\pgfqpoint{0.671087in}{0.350000in}}{\pgfqpoint{0.666667in}{0.350000in}}%
\pgfpathcurveto{\pgfqpoint{0.662247in}{0.350000in}}{\pgfqpoint{0.658007in}{0.348244in}}{\pgfqpoint{0.654882in}{0.345118in}}%
\pgfpathcurveto{\pgfqpoint{0.651756in}{0.341993in}}{\pgfqpoint{0.650000in}{0.337753in}}{\pgfqpoint{0.650000in}{0.333333in}}%
\pgfpathcurveto{\pgfqpoint{0.650000in}{0.328913in}}{\pgfqpoint{0.651756in}{0.324674in}}{\pgfqpoint{0.654882in}{0.321548in}}%
\pgfpathcurveto{\pgfqpoint{0.658007in}{0.318423in}}{\pgfqpoint{0.662247in}{0.316667in}}{\pgfqpoint{0.666667in}{0.316667in}}%
\pgfpathclose%
\pgfpathmoveto{\pgfqpoint{0.833333in}{0.316667in}}%
\pgfpathcurveto{\pgfqpoint{0.837753in}{0.316667in}}{\pgfqpoint{0.841993in}{0.318423in}}{\pgfqpoint{0.845118in}{0.321548in}}%
\pgfpathcurveto{\pgfqpoint{0.848244in}{0.324674in}}{\pgfqpoint{0.850000in}{0.328913in}}{\pgfqpoint{0.850000in}{0.333333in}}%
\pgfpathcurveto{\pgfqpoint{0.850000in}{0.337753in}}{\pgfqpoint{0.848244in}{0.341993in}}{\pgfqpoint{0.845118in}{0.345118in}}%
\pgfpathcurveto{\pgfqpoint{0.841993in}{0.348244in}}{\pgfqpoint{0.837753in}{0.350000in}}{\pgfqpoint{0.833333in}{0.350000in}}%
\pgfpathcurveto{\pgfqpoint{0.828913in}{0.350000in}}{\pgfqpoint{0.824674in}{0.348244in}}{\pgfqpoint{0.821548in}{0.345118in}}%
\pgfpathcurveto{\pgfqpoint{0.818423in}{0.341993in}}{\pgfqpoint{0.816667in}{0.337753in}}{\pgfqpoint{0.816667in}{0.333333in}}%
\pgfpathcurveto{\pgfqpoint{0.816667in}{0.328913in}}{\pgfqpoint{0.818423in}{0.324674in}}{\pgfqpoint{0.821548in}{0.321548in}}%
\pgfpathcurveto{\pgfqpoint{0.824674in}{0.318423in}}{\pgfqpoint{0.828913in}{0.316667in}}{\pgfqpoint{0.833333in}{0.316667in}}%
\pgfpathclose%
\pgfpathmoveto{\pgfqpoint{1.000000in}{0.316667in}}%
\pgfpathcurveto{\pgfqpoint{1.004420in}{0.316667in}}{\pgfqpoint{1.008660in}{0.318423in}}{\pgfqpoint{1.011785in}{0.321548in}}%
\pgfpathcurveto{\pgfqpoint{1.014911in}{0.324674in}}{\pgfqpoint{1.016667in}{0.328913in}}{\pgfqpoint{1.016667in}{0.333333in}}%
\pgfpathcurveto{\pgfqpoint{1.016667in}{0.337753in}}{\pgfqpoint{1.014911in}{0.341993in}}{\pgfqpoint{1.011785in}{0.345118in}}%
\pgfpathcurveto{\pgfqpoint{1.008660in}{0.348244in}}{\pgfqpoint{1.004420in}{0.350000in}}{\pgfqpoint{1.000000in}{0.350000in}}%
\pgfpathcurveto{\pgfqpoint{0.995580in}{0.350000in}}{\pgfqpoint{0.991340in}{0.348244in}}{\pgfqpoint{0.988215in}{0.345118in}}%
\pgfpathcurveto{\pgfqpoint{0.985089in}{0.341993in}}{\pgfqpoint{0.983333in}{0.337753in}}{\pgfqpoint{0.983333in}{0.333333in}}%
\pgfpathcurveto{\pgfqpoint{0.983333in}{0.328913in}}{\pgfqpoint{0.985089in}{0.324674in}}{\pgfqpoint{0.988215in}{0.321548in}}%
\pgfpathcurveto{\pgfqpoint{0.991340in}{0.318423in}}{\pgfqpoint{0.995580in}{0.316667in}}{\pgfqpoint{1.000000in}{0.316667in}}%
\pgfpathclose%
\pgfpathmoveto{\pgfqpoint{0.083333in}{0.483333in}}%
\pgfpathcurveto{\pgfqpoint{0.087753in}{0.483333in}}{\pgfqpoint{0.091993in}{0.485089in}}{\pgfqpoint{0.095118in}{0.488215in}}%
\pgfpathcurveto{\pgfqpoint{0.098244in}{0.491340in}}{\pgfqpoint{0.100000in}{0.495580in}}{\pgfqpoint{0.100000in}{0.500000in}}%
\pgfpathcurveto{\pgfqpoint{0.100000in}{0.504420in}}{\pgfqpoint{0.098244in}{0.508660in}}{\pgfqpoint{0.095118in}{0.511785in}}%
\pgfpathcurveto{\pgfqpoint{0.091993in}{0.514911in}}{\pgfqpoint{0.087753in}{0.516667in}}{\pgfqpoint{0.083333in}{0.516667in}}%
\pgfpathcurveto{\pgfqpoint{0.078913in}{0.516667in}}{\pgfqpoint{0.074674in}{0.514911in}}{\pgfqpoint{0.071548in}{0.511785in}}%
\pgfpathcurveto{\pgfqpoint{0.068423in}{0.508660in}}{\pgfqpoint{0.066667in}{0.504420in}}{\pgfqpoint{0.066667in}{0.500000in}}%
\pgfpathcurveto{\pgfqpoint{0.066667in}{0.495580in}}{\pgfqpoint{0.068423in}{0.491340in}}{\pgfqpoint{0.071548in}{0.488215in}}%
\pgfpathcurveto{\pgfqpoint{0.074674in}{0.485089in}}{\pgfqpoint{0.078913in}{0.483333in}}{\pgfqpoint{0.083333in}{0.483333in}}%
\pgfpathclose%
\pgfpathmoveto{\pgfqpoint{0.250000in}{0.483333in}}%
\pgfpathcurveto{\pgfqpoint{0.254420in}{0.483333in}}{\pgfqpoint{0.258660in}{0.485089in}}{\pgfqpoint{0.261785in}{0.488215in}}%
\pgfpathcurveto{\pgfqpoint{0.264911in}{0.491340in}}{\pgfqpoint{0.266667in}{0.495580in}}{\pgfqpoint{0.266667in}{0.500000in}}%
\pgfpathcurveto{\pgfqpoint{0.266667in}{0.504420in}}{\pgfqpoint{0.264911in}{0.508660in}}{\pgfqpoint{0.261785in}{0.511785in}}%
\pgfpathcurveto{\pgfqpoint{0.258660in}{0.514911in}}{\pgfqpoint{0.254420in}{0.516667in}}{\pgfqpoint{0.250000in}{0.516667in}}%
\pgfpathcurveto{\pgfqpoint{0.245580in}{0.516667in}}{\pgfqpoint{0.241340in}{0.514911in}}{\pgfqpoint{0.238215in}{0.511785in}}%
\pgfpathcurveto{\pgfqpoint{0.235089in}{0.508660in}}{\pgfqpoint{0.233333in}{0.504420in}}{\pgfqpoint{0.233333in}{0.500000in}}%
\pgfpathcurveto{\pgfqpoint{0.233333in}{0.495580in}}{\pgfqpoint{0.235089in}{0.491340in}}{\pgfqpoint{0.238215in}{0.488215in}}%
\pgfpathcurveto{\pgfqpoint{0.241340in}{0.485089in}}{\pgfqpoint{0.245580in}{0.483333in}}{\pgfqpoint{0.250000in}{0.483333in}}%
\pgfpathclose%
\pgfpathmoveto{\pgfqpoint{0.416667in}{0.483333in}}%
\pgfpathcurveto{\pgfqpoint{0.421087in}{0.483333in}}{\pgfqpoint{0.425326in}{0.485089in}}{\pgfqpoint{0.428452in}{0.488215in}}%
\pgfpathcurveto{\pgfqpoint{0.431577in}{0.491340in}}{\pgfqpoint{0.433333in}{0.495580in}}{\pgfqpoint{0.433333in}{0.500000in}}%
\pgfpathcurveto{\pgfqpoint{0.433333in}{0.504420in}}{\pgfqpoint{0.431577in}{0.508660in}}{\pgfqpoint{0.428452in}{0.511785in}}%
\pgfpathcurveto{\pgfqpoint{0.425326in}{0.514911in}}{\pgfqpoint{0.421087in}{0.516667in}}{\pgfqpoint{0.416667in}{0.516667in}}%
\pgfpathcurveto{\pgfqpoint{0.412247in}{0.516667in}}{\pgfqpoint{0.408007in}{0.514911in}}{\pgfqpoint{0.404882in}{0.511785in}}%
\pgfpathcurveto{\pgfqpoint{0.401756in}{0.508660in}}{\pgfqpoint{0.400000in}{0.504420in}}{\pgfqpoint{0.400000in}{0.500000in}}%
\pgfpathcurveto{\pgfqpoint{0.400000in}{0.495580in}}{\pgfqpoint{0.401756in}{0.491340in}}{\pgfqpoint{0.404882in}{0.488215in}}%
\pgfpathcurveto{\pgfqpoint{0.408007in}{0.485089in}}{\pgfqpoint{0.412247in}{0.483333in}}{\pgfqpoint{0.416667in}{0.483333in}}%
\pgfpathclose%
\pgfpathmoveto{\pgfqpoint{0.583333in}{0.483333in}}%
\pgfpathcurveto{\pgfqpoint{0.587753in}{0.483333in}}{\pgfqpoint{0.591993in}{0.485089in}}{\pgfqpoint{0.595118in}{0.488215in}}%
\pgfpathcurveto{\pgfqpoint{0.598244in}{0.491340in}}{\pgfqpoint{0.600000in}{0.495580in}}{\pgfqpoint{0.600000in}{0.500000in}}%
\pgfpathcurveto{\pgfqpoint{0.600000in}{0.504420in}}{\pgfqpoint{0.598244in}{0.508660in}}{\pgfqpoint{0.595118in}{0.511785in}}%
\pgfpathcurveto{\pgfqpoint{0.591993in}{0.514911in}}{\pgfqpoint{0.587753in}{0.516667in}}{\pgfqpoint{0.583333in}{0.516667in}}%
\pgfpathcurveto{\pgfqpoint{0.578913in}{0.516667in}}{\pgfqpoint{0.574674in}{0.514911in}}{\pgfqpoint{0.571548in}{0.511785in}}%
\pgfpathcurveto{\pgfqpoint{0.568423in}{0.508660in}}{\pgfqpoint{0.566667in}{0.504420in}}{\pgfqpoint{0.566667in}{0.500000in}}%
\pgfpathcurveto{\pgfqpoint{0.566667in}{0.495580in}}{\pgfqpoint{0.568423in}{0.491340in}}{\pgfqpoint{0.571548in}{0.488215in}}%
\pgfpathcurveto{\pgfqpoint{0.574674in}{0.485089in}}{\pgfqpoint{0.578913in}{0.483333in}}{\pgfqpoint{0.583333in}{0.483333in}}%
\pgfpathclose%
\pgfpathmoveto{\pgfqpoint{0.750000in}{0.483333in}}%
\pgfpathcurveto{\pgfqpoint{0.754420in}{0.483333in}}{\pgfqpoint{0.758660in}{0.485089in}}{\pgfqpoint{0.761785in}{0.488215in}}%
\pgfpathcurveto{\pgfqpoint{0.764911in}{0.491340in}}{\pgfqpoint{0.766667in}{0.495580in}}{\pgfqpoint{0.766667in}{0.500000in}}%
\pgfpathcurveto{\pgfqpoint{0.766667in}{0.504420in}}{\pgfqpoint{0.764911in}{0.508660in}}{\pgfqpoint{0.761785in}{0.511785in}}%
\pgfpathcurveto{\pgfqpoint{0.758660in}{0.514911in}}{\pgfqpoint{0.754420in}{0.516667in}}{\pgfqpoint{0.750000in}{0.516667in}}%
\pgfpathcurveto{\pgfqpoint{0.745580in}{0.516667in}}{\pgfqpoint{0.741340in}{0.514911in}}{\pgfqpoint{0.738215in}{0.511785in}}%
\pgfpathcurveto{\pgfqpoint{0.735089in}{0.508660in}}{\pgfqpoint{0.733333in}{0.504420in}}{\pgfqpoint{0.733333in}{0.500000in}}%
\pgfpathcurveto{\pgfqpoint{0.733333in}{0.495580in}}{\pgfqpoint{0.735089in}{0.491340in}}{\pgfqpoint{0.738215in}{0.488215in}}%
\pgfpathcurveto{\pgfqpoint{0.741340in}{0.485089in}}{\pgfqpoint{0.745580in}{0.483333in}}{\pgfqpoint{0.750000in}{0.483333in}}%
\pgfpathclose%
\pgfpathmoveto{\pgfqpoint{0.916667in}{0.483333in}}%
\pgfpathcurveto{\pgfqpoint{0.921087in}{0.483333in}}{\pgfqpoint{0.925326in}{0.485089in}}{\pgfqpoint{0.928452in}{0.488215in}}%
\pgfpathcurveto{\pgfqpoint{0.931577in}{0.491340in}}{\pgfqpoint{0.933333in}{0.495580in}}{\pgfqpoint{0.933333in}{0.500000in}}%
\pgfpathcurveto{\pgfqpoint{0.933333in}{0.504420in}}{\pgfqpoint{0.931577in}{0.508660in}}{\pgfqpoint{0.928452in}{0.511785in}}%
\pgfpathcurveto{\pgfqpoint{0.925326in}{0.514911in}}{\pgfqpoint{0.921087in}{0.516667in}}{\pgfqpoint{0.916667in}{0.516667in}}%
\pgfpathcurveto{\pgfqpoint{0.912247in}{0.516667in}}{\pgfqpoint{0.908007in}{0.514911in}}{\pgfqpoint{0.904882in}{0.511785in}}%
\pgfpathcurveto{\pgfqpoint{0.901756in}{0.508660in}}{\pgfqpoint{0.900000in}{0.504420in}}{\pgfqpoint{0.900000in}{0.500000in}}%
\pgfpathcurveto{\pgfqpoint{0.900000in}{0.495580in}}{\pgfqpoint{0.901756in}{0.491340in}}{\pgfqpoint{0.904882in}{0.488215in}}%
\pgfpathcurveto{\pgfqpoint{0.908007in}{0.485089in}}{\pgfqpoint{0.912247in}{0.483333in}}{\pgfqpoint{0.916667in}{0.483333in}}%
\pgfpathclose%
\pgfpathmoveto{\pgfqpoint{0.000000in}{0.650000in}}%
\pgfpathcurveto{\pgfqpoint{0.004420in}{0.650000in}}{\pgfqpoint{0.008660in}{0.651756in}}{\pgfqpoint{0.011785in}{0.654882in}}%
\pgfpathcurveto{\pgfqpoint{0.014911in}{0.658007in}}{\pgfqpoint{0.016667in}{0.662247in}}{\pgfqpoint{0.016667in}{0.666667in}}%
\pgfpathcurveto{\pgfqpoint{0.016667in}{0.671087in}}{\pgfqpoint{0.014911in}{0.675326in}}{\pgfqpoint{0.011785in}{0.678452in}}%
\pgfpathcurveto{\pgfqpoint{0.008660in}{0.681577in}}{\pgfqpoint{0.004420in}{0.683333in}}{\pgfqpoint{0.000000in}{0.683333in}}%
\pgfpathcurveto{\pgfqpoint{-0.004420in}{0.683333in}}{\pgfqpoint{-0.008660in}{0.681577in}}{\pgfqpoint{-0.011785in}{0.678452in}}%
\pgfpathcurveto{\pgfqpoint{-0.014911in}{0.675326in}}{\pgfqpoint{-0.016667in}{0.671087in}}{\pgfqpoint{-0.016667in}{0.666667in}}%
\pgfpathcurveto{\pgfqpoint{-0.016667in}{0.662247in}}{\pgfqpoint{-0.014911in}{0.658007in}}{\pgfqpoint{-0.011785in}{0.654882in}}%
\pgfpathcurveto{\pgfqpoint{-0.008660in}{0.651756in}}{\pgfqpoint{-0.004420in}{0.650000in}}{\pgfqpoint{0.000000in}{0.650000in}}%
\pgfpathclose%
\pgfpathmoveto{\pgfqpoint{0.166667in}{0.650000in}}%
\pgfpathcurveto{\pgfqpoint{0.171087in}{0.650000in}}{\pgfqpoint{0.175326in}{0.651756in}}{\pgfqpoint{0.178452in}{0.654882in}}%
\pgfpathcurveto{\pgfqpoint{0.181577in}{0.658007in}}{\pgfqpoint{0.183333in}{0.662247in}}{\pgfqpoint{0.183333in}{0.666667in}}%
\pgfpathcurveto{\pgfqpoint{0.183333in}{0.671087in}}{\pgfqpoint{0.181577in}{0.675326in}}{\pgfqpoint{0.178452in}{0.678452in}}%
\pgfpathcurveto{\pgfqpoint{0.175326in}{0.681577in}}{\pgfqpoint{0.171087in}{0.683333in}}{\pgfqpoint{0.166667in}{0.683333in}}%
\pgfpathcurveto{\pgfqpoint{0.162247in}{0.683333in}}{\pgfqpoint{0.158007in}{0.681577in}}{\pgfqpoint{0.154882in}{0.678452in}}%
\pgfpathcurveto{\pgfqpoint{0.151756in}{0.675326in}}{\pgfqpoint{0.150000in}{0.671087in}}{\pgfqpoint{0.150000in}{0.666667in}}%
\pgfpathcurveto{\pgfqpoint{0.150000in}{0.662247in}}{\pgfqpoint{0.151756in}{0.658007in}}{\pgfqpoint{0.154882in}{0.654882in}}%
\pgfpathcurveto{\pgfqpoint{0.158007in}{0.651756in}}{\pgfqpoint{0.162247in}{0.650000in}}{\pgfqpoint{0.166667in}{0.650000in}}%
\pgfpathclose%
\pgfpathmoveto{\pgfqpoint{0.333333in}{0.650000in}}%
\pgfpathcurveto{\pgfqpoint{0.337753in}{0.650000in}}{\pgfqpoint{0.341993in}{0.651756in}}{\pgfqpoint{0.345118in}{0.654882in}}%
\pgfpathcurveto{\pgfqpoint{0.348244in}{0.658007in}}{\pgfqpoint{0.350000in}{0.662247in}}{\pgfqpoint{0.350000in}{0.666667in}}%
\pgfpathcurveto{\pgfqpoint{0.350000in}{0.671087in}}{\pgfqpoint{0.348244in}{0.675326in}}{\pgfqpoint{0.345118in}{0.678452in}}%
\pgfpathcurveto{\pgfqpoint{0.341993in}{0.681577in}}{\pgfqpoint{0.337753in}{0.683333in}}{\pgfqpoint{0.333333in}{0.683333in}}%
\pgfpathcurveto{\pgfqpoint{0.328913in}{0.683333in}}{\pgfqpoint{0.324674in}{0.681577in}}{\pgfqpoint{0.321548in}{0.678452in}}%
\pgfpathcurveto{\pgfqpoint{0.318423in}{0.675326in}}{\pgfqpoint{0.316667in}{0.671087in}}{\pgfqpoint{0.316667in}{0.666667in}}%
\pgfpathcurveto{\pgfqpoint{0.316667in}{0.662247in}}{\pgfqpoint{0.318423in}{0.658007in}}{\pgfqpoint{0.321548in}{0.654882in}}%
\pgfpathcurveto{\pgfqpoint{0.324674in}{0.651756in}}{\pgfqpoint{0.328913in}{0.650000in}}{\pgfqpoint{0.333333in}{0.650000in}}%
\pgfpathclose%
\pgfpathmoveto{\pgfqpoint{0.500000in}{0.650000in}}%
\pgfpathcurveto{\pgfqpoint{0.504420in}{0.650000in}}{\pgfqpoint{0.508660in}{0.651756in}}{\pgfqpoint{0.511785in}{0.654882in}}%
\pgfpathcurveto{\pgfqpoint{0.514911in}{0.658007in}}{\pgfqpoint{0.516667in}{0.662247in}}{\pgfqpoint{0.516667in}{0.666667in}}%
\pgfpathcurveto{\pgfqpoint{0.516667in}{0.671087in}}{\pgfqpoint{0.514911in}{0.675326in}}{\pgfqpoint{0.511785in}{0.678452in}}%
\pgfpathcurveto{\pgfqpoint{0.508660in}{0.681577in}}{\pgfqpoint{0.504420in}{0.683333in}}{\pgfqpoint{0.500000in}{0.683333in}}%
\pgfpathcurveto{\pgfqpoint{0.495580in}{0.683333in}}{\pgfqpoint{0.491340in}{0.681577in}}{\pgfqpoint{0.488215in}{0.678452in}}%
\pgfpathcurveto{\pgfqpoint{0.485089in}{0.675326in}}{\pgfqpoint{0.483333in}{0.671087in}}{\pgfqpoint{0.483333in}{0.666667in}}%
\pgfpathcurveto{\pgfqpoint{0.483333in}{0.662247in}}{\pgfqpoint{0.485089in}{0.658007in}}{\pgfqpoint{0.488215in}{0.654882in}}%
\pgfpathcurveto{\pgfqpoint{0.491340in}{0.651756in}}{\pgfqpoint{0.495580in}{0.650000in}}{\pgfqpoint{0.500000in}{0.650000in}}%
\pgfpathclose%
\pgfpathmoveto{\pgfqpoint{0.666667in}{0.650000in}}%
\pgfpathcurveto{\pgfqpoint{0.671087in}{0.650000in}}{\pgfqpoint{0.675326in}{0.651756in}}{\pgfqpoint{0.678452in}{0.654882in}}%
\pgfpathcurveto{\pgfqpoint{0.681577in}{0.658007in}}{\pgfqpoint{0.683333in}{0.662247in}}{\pgfqpoint{0.683333in}{0.666667in}}%
\pgfpathcurveto{\pgfqpoint{0.683333in}{0.671087in}}{\pgfqpoint{0.681577in}{0.675326in}}{\pgfqpoint{0.678452in}{0.678452in}}%
\pgfpathcurveto{\pgfqpoint{0.675326in}{0.681577in}}{\pgfqpoint{0.671087in}{0.683333in}}{\pgfqpoint{0.666667in}{0.683333in}}%
\pgfpathcurveto{\pgfqpoint{0.662247in}{0.683333in}}{\pgfqpoint{0.658007in}{0.681577in}}{\pgfqpoint{0.654882in}{0.678452in}}%
\pgfpathcurveto{\pgfqpoint{0.651756in}{0.675326in}}{\pgfqpoint{0.650000in}{0.671087in}}{\pgfqpoint{0.650000in}{0.666667in}}%
\pgfpathcurveto{\pgfqpoint{0.650000in}{0.662247in}}{\pgfqpoint{0.651756in}{0.658007in}}{\pgfqpoint{0.654882in}{0.654882in}}%
\pgfpathcurveto{\pgfqpoint{0.658007in}{0.651756in}}{\pgfqpoint{0.662247in}{0.650000in}}{\pgfqpoint{0.666667in}{0.650000in}}%
\pgfpathclose%
\pgfpathmoveto{\pgfqpoint{0.833333in}{0.650000in}}%
\pgfpathcurveto{\pgfqpoint{0.837753in}{0.650000in}}{\pgfqpoint{0.841993in}{0.651756in}}{\pgfqpoint{0.845118in}{0.654882in}}%
\pgfpathcurveto{\pgfqpoint{0.848244in}{0.658007in}}{\pgfqpoint{0.850000in}{0.662247in}}{\pgfqpoint{0.850000in}{0.666667in}}%
\pgfpathcurveto{\pgfqpoint{0.850000in}{0.671087in}}{\pgfqpoint{0.848244in}{0.675326in}}{\pgfqpoint{0.845118in}{0.678452in}}%
\pgfpathcurveto{\pgfqpoint{0.841993in}{0.681577in}}{\pgfqpoint{0.837753in}{0.683333in}}{\pgfqpoint{0.833333in}{0.683333in}}%
\pgfpathcurveto{\pgfqpoint{0.828913in}{0.683333in}}{\pgfqpoint{0.824674in}{0.681577in}}{\pgfqpoint{0.821548in}{0.678452in}}%
\pgfpathcurveto{\pgfqpoint{0.818423in}{0.675326in}}{\pgfqpoint{0.816667in}{0.671087in}}{\pgfqpoint{0.816667in}{0.666667in}}%
\pgfpathcurveto{\pgfqpoint{0.816667in}{0.662247in}}{\pgfqpoint{0.818423in}{0.658007in}}{\pgfqpoint{0.821548in}{0.654882in}}%
\pgfpathcurveto{\pgfqpoint{0.824674in}{0.651756in}}{\pgfqpoint{0.828913in}{0.650000in}}{\pgfqpoint{0.833333in}{0.650000in}}%
\pgfpathclose%
\pgfpathmoveto{\pgfqpoint{1.000000in}{0.650000in}}%
\pgfpathcurveto{\pgfqpoint{1.004420in}{0.650000in}}{\pgfqpoint{1.008660in}{0.651756in}}{\pgfqpoint{1.011785in}{0.654882in}}%
\pgfpathcurveto{\pgfqpoint{1.014911in}{0.658007in}}{\pgfqpoint{1.016667in}{0.662247in}}{\pgfqpoint{1.016667in}{0.666667in}}%
\pgfpathcurveto{\pgfqpoint{1.016667in}{0.671087in}}{\pgfqpoint{1.014911in}{0.675326in}}{\pgfqpoint{1.011785in}{0.678452in}}%
\pgfpathcurveto{\pgfqpoint{1.008660in}{0.681577in}}{\pgfqpoint{1.004420in}{0.683333in}}{\pgfqpoint{1.000000in}{0.683333in}}%
\pgfpathcurveto{\pgfqpoint{0.995580in}{0.683333in}}{\pgfqpoint{0.991340in}{0.681577in}}{\pgfqpoint{0.988215in}{0.678452in}}%
\pgfpathcurveto{\pgfqpoint{0.985089in}{0.675326in}}{\pgfqpoint{0.983333in}{0.671087in}}{\pgfqpoint{0.983333in}{0.666667in}}%
\pgfpathcurveto{\pgfqpoint{0.983333in}{0.662247in}}{\pgfqpoint{0.985089in}{0.658007in}}{\pgfqpoint{0.988215in}{0.654882in}}%
\pgfpathcurveto{\pgfqpoint{0.991340in}{0.651756in}}{\pgfqpoint{0.995580in}{0.650000in}}{\pgfqpoint{1.000000in}{0.650000in}}%
\pgfpathclose%
\pgfpathmoveto{\pgfqpoint{0.083333in}{0.816667in}}%
\pgfpathcurveto{\pgfqpoint{0.087753in}{0.816667in}}{\pgfqpoint{0.091993in}{0.818423in}}{\pgfqpoint{0.095118in}{0.821548in}}%
\pgfpathcurveto{\pgfqpoint{0.098244in}{0.824674in}}{\pgfqpoint{0.100000in}{0.828913in}}{\pgfqpoint{0.100000in}{0.833333in}}%
\pgfpathcurveto{\pgfqpoint{0.100000in}{0.837753in}}{\pgfqpoint{0.098244in}{0.841993in}}{\pgfqpoint{0.095118in}{0.845118in}}%
\pgfpathcurveto{\pgfqpoint{0.091993in}{0.848244in}}{\pgfqpoint{0.087753in}{0.850000in}}{\pgfqpoint{0.083333in}{0.850000in}}%
\pgfpathcurveto{\pgfqpoint{0.078913in}{0.850000in}}{\pgfqpoint{0.074674in}{0.848244in}}{\pgfqpoint{0.071548in}{0.845118in}}%
\pgfpathcurveto{\pgfqpoint{0.068423in}{0.841993in}}{\pgfqpoint{0.066667in}{0.837753in}}{\pgfqpoint{0.066667in}{0.833333in}}%
\pgfpathcurveto{\pgfqpoint{0.066667in}{0.828913in}}{\pgfqpoint{0.068423in}{0.824674in}}{\pgfqpoint{0.071548in}{0.821548in}}%
\pgfpathcurveto{\pgfqpoint{0.074674in}{0.818423in}}{\pgfqpoint{0.078913in}{0.816667in}}{\pgfqpoint{0.083333in}{0.816667in}}%
\pgfpathclose%
\pgfpathmoveto{\pgfqpoint{0.250000in}{0.816667in}}%
\pgfpathcurveto{\pgfqpoint{0.254420in}{0.816667in}}{\pgfqpoint{0.258660in}{0.818423in}}{\pgfqpoint{0.261785in}{0.821548in}}%
\pgfpathcurveto{\pgfqpoint{0.264911in}{0.824674in}}{\pgfqpoint{0.266667in}{0.828913in}}{\pgfqpoint{0.266667in}{0.833333in}}%
\pgfpathcurveto{\pgfqpoint{0.266667in}{0.837753in}}{\pgfqpoint{0.264911in}{0.841993in}}{\pgfqpoint{0.261785in}{0.845118in}}%
\pgfpathcurveto{\pgfqpoint{0.258660in}{0.848244in}}{\pgfqpoint{0.254420in}{0.850000in}}{\pgfqpoint{0.250000in}{0.850000in}}%
\pgfpathcurveto{\pgfqpoint{0.245580in}{0.850000in}}{\pgfqpoint{0.241340in}{0.848244in}}{\pgfqpoint{0.238215in}{0.845118in}}%
\pgfpathcurveto{\pgfqpoint{0.235089in}{0.841993in}}{\pgfqpoint{0.233333in}{0.837753in}}{\pgfqpoint{0.233333in}{0.833333in}}%
\pgfpathcurveto{\pgfqpoint{0.233333in}{0.828913in}}{\pgfqpoint{0.235089in}{0.824674in}}{\pgfqpoint{0.238215in}{0.821548in}}%
\pgfpathcurveto{\pgfqpoint{0.241340in}{0.818423in}}{\pgfqpoint{0.245580in}{0.816667in}}{\pgfqpoint{0.250000in}{0.816667in}}%
\pgfpathclose%
\pgfpathmoveto{\pgfqpoint{0.416667in}{0.816667in}}%
\pgfpathcurveto{\pgfqpoint{0.421087in}{0.816667in}}{\pgfqpoint{0.425326in}{0.818423in}}{\pgfqpoint{0.428452in}{0.821548in}}%
\pgfpathcurveto{\pgfqpoint{0.431577in}{0.824674in}}{\pgfqpoint{0.433333in}{0.828913in}}{\pgfqpoint{0.433333in}{0.833333in}}%
\pgfpathcurveto{\pgfqpoint{0.433333in}{0.837753in}}{\pgfqpoint{0.431577in}{0.841993in}}{\pgfqpoint{0.428452in}{0.845118in}}%
\pgfpathcurveto{\pgfqpoint{0.425326in}{0.848244in}}{\pgfqpoint{0.421087in}{0.850000in}}{\pgfqpoint{0.416667in}{0.850000in}}%
\pgfpathcurveto{\pgfqpoint{0.412247in}{0.850000in}}{\pgfqpoint{0.408007in}{0.848244in}}{\pgfqpoint{0.404882in}{0.845118in}}%
\pgfpathcurveto{\pgfqpoint{0.401756in}{0.841993in}}{\pgfqpoint{0.400000in}{0.837753in}}{\pgfqpoint{0.400000in}{0.833333in}}%
\pgfpathcurveto{\pgfqpoint{0.400000in}{0.828913in}}{\pgfqpoint{0.401756in}{0.824674in}}{\pgfqpoint{0.404882in}{0.821548in}}%
\pgfpathcurveto{\pgfqpoint{0.408007in}{0.818423in}}{\pgfqpoint{0.412247in}{0.816667in}}{\pgfqpoint{0.416667in}{0.816667in}}%
\pgfpathclose%
\pgfpathmoveto{\pgfqpoint{0.583333in}{0.816667in}}%
\pgfpathcurveto{\pgfqpoint{0.587753in}{0.816667in}}{\pgfqpoint{0.591993in}{0.818423in}}{\pgfqpoint{0.595118in}{0.821548in}}%
\pgfpathcurveto{\pgfqpoint{0.598244in}{0.824674in}}{\pgfqpoint{0.600000in}{0.828913in}}{\pgfqpoint{0.600000in}{0.833333in}}%
\pgfpathcurveto{\pgfqpoint{0.600000in}{0.837753in}}{\pgfqpoint{0.598244in}{0.841993in}}{\pgfqpoint{0.595118in}{0.845118in}}%
\pgfpathcurveto{\pgfqpoint{0.591993in}{0.848244in}}{\pgfqpoint{0.587753in}{0.850000in}}{\pgfqpoint{0.583333in}{0.850000in}}%
\pgfpathcurveto{\pgfqpoint{0.578913in}{0.850000in}}{\pgfqpoint{0.574674in}{0.848244in}}{\pgfqpoint{0.571548in}{0.845118in}}%
\pgfpathcurveto{\pgfqpoint{0.568423in}{0.841993in}}{\pgfqpoint{0.566667in}{0.837753in}}{\pgfqpoint{0.566667in}{0.833333in}}%
\pgfpathcurveto{\pgfqpoint{0.566667in}{0.828913in}}{\pgfqpoint{0.568423in}{0.824674in}}{\pgfqpoint{0.571548in}{0.821548in}}%
\pgfpathcurveto{\pgfqpoint{0.574674in}{0.818423in}}{\pgfqpoint{0.578913in}{0.816667in}}{\pgfqpoint{0.583333in}{0.816667in}}%
\pgfpathclose%
\pgfpathmoveto{\pgfqpoint{0.750000in}{0.816667in}}%
\pgfpathcurveto{\pgfqpoint{0.754420in}{0.816667in}}{\pgfqpoint{0.758660in}{0.818423in}}{\pgfqpoint{0.761785in}{0.821548in}}%
\pgfpathcurveto{\pgfqpoint{0.764911in}{0.824674in}}{\pgfqpoint{0.766667in}{0.828913in}}{\pgfqpoint{0.766667in}{0.833333in}}%
\pgfpathcurveto{\pgfqpoint{0.766667in}{0.837753in}}{\pgfqpoint{0.764911in}{0.841993in}}{\pgfqpoint{0.761785in}{0.845118in}}%
\pgfpathcurveto{\pgfqpoint{0.758660in}{0.848244in}}{\pgfqpoint{0.754420in}{0.850000in}}{\pgfqpoint{0.750000in}{0.850000in}}%
\pgfpathcurveto{\pgfqpoint{0.745580in}{0.850000in}}{\pgfqpoint{0.741340in}{0.848244in}}{\pgfqpoint{0.738215in}{0.845118in}}%
\pgfpathcurveto{\pgfqpoint{0.735089in}{0.841993in}}{\pgfqpoint{0.733333in}{0.837753in}}{\pgfqpoint{0.733333in}{0.833333in}}%
\pgfpathcurveto{\pgfqpoint{0.733333in}{0.828913in}}{\pgfqpoint{0.735089in}{0.824674in}}{\pgfqpoint{0.738215in}{0.821548in}}%
\pgfpathcurveto{\pgfqpoint{0.741340in}{0.818423in}}{\pgfqpoint{0.745580in}{0.816667in}}{\pgfqpoint{0.750000in}{0.816667in}}%
\pgfpathclose%
\pgfpathmoveto{\pgfqpoint{0.916667in}{0.816667in}}%
\pgfpathcurveto{\pgfqpoint{0.921087in}{0.816667in}}{\pgfqpoint{0.925326in}{0.818423in}}{\pgfqpoint{0.928452in}{0.821548in}}%
\pgfpathcurveto{\pgfqpoint{0.931577in}{0.824674in}}{\pgfqpoint{0.933333in}{0.828913in}}{\pgfqpoint{0.933333in}{0.833333in}}%
\pgfpathcurveto{\pgfqpoint{0.933333in}{0.837753in}}{\pgfqpoint{0.931577in}{0.841993in}}{\pgfqpoint{0.928452in}{0.845118in}}%
\pgfpathcurveto{\pgfqpoint{0.925326in}{0.848244in}}{\pgfqpoint{0.921087in}{0.850000in}}{\pgfqpoint{0.916667in}{0.850000in}}%
\pgfpathcurveto{\pgfqpoint{0.912247in}{0.850000in}}{\pgfqpoint{0.908007in}{0.848244in}}{\pgfqpoint{0.904882in}{0.845118in}}%
\pgfpathcurveto{\pgfqpoint{0.901756in}{0.841993in}}{\pgfqpoint{0.900000in}{0.837753in}}{\pgfqpoint{0.900000in}{0.833333in}}%
\pgfpathcurveto{\pgfqpoint{0.900000in}{0.828913in}}{\pgfqpoint{0.901756in}{0.824674in}}{\pgfqpoint{0.904882in}{0.821548in}}%
\pgfpathcurveto{\pgfqpoint{0.908007in}{0.818423in}}{\pgfqpoint{0.912247in}{0.816667in}}{\pgfqpoint{0.916667in}{0.816667in}}%
\pgfpathclose%
\pgfpathmoveto{\pgfqpoint{0.000000in}{0.983333in}}%
\pgfpathcurveto{\pgfqpoint{0.004420in}{0.983333in}}{\pgfqpoint{0.008660in}{0.985089in}}{\pgfqpoint{0.011785in}{0.988215in}}%
\pgfpathcurveto{\pgfqpoint{0.014911in}{0.991340in}}{\pgfqpoint{0.016667in}{0.995580in}}{\pgfqpoint{0.016667in}{1.000000in}}%
\pgfpathcurveto{\pgfqpoint{0.016667in}{1.004420in}}{\pgfqpoint{0.014911in}{1.008660in}}{\pgfqpoint{0.011785in}{1.011785in}}%
\pgfpathcurveto{\pgfqpoint{0.008660in}{1.014911in}}{\pgfqpoint{0.004420in}{1.016667in}}{\pgfqpoint{0.000000in}{1.016667in}}%
\pgfpathcurveto{\pgfqpoint{-0.004420in}{1.016667in}}{\pgfqpoint{-0.008660in}{1.014911in}}{\pgfqpoint{-0.011785in}{1.011785in}}%
\pgfpathcurveto{\pgfqpoint{-0.014911in}{1.008660in}}{\pgfqpoint{-0.016667in}{1.004420in}}{\pgfqpoint{-0.016667in}{1.000000in}}%
\pgfpathcurveto{\pgfqpoint{-0.016667in}{0.995580in}}{\pgfqpoint{-0.014911in}{0.991340in}}{\pgfqpoint{-0.011785in}{0.988215in}}%
\pgfpathcurveto{\pgfqpoint{-0.008660in}{0.985089in}}{\pgfqpoint{-0.004420in}{0.983333in}}{\pgfqpoint{0.000000in}{0.983333in}}%
\pgfpathclose%
\pgfpathmoveto{\pgfqpoint{0.166667in}{0.983333in}}%
\pgfpathcurveto{\pgfqpoint{0.171087in}{0.983333in}}{\pgfqpoint{0.175326in}{0.985089in}}{\pgfqpoint{0.178452in}{0.988215in}}%
\pgfpathcurveto{\pgfqpoint{0.181577in}{0.991340in}}{\pgfqpoint{0.183333in}{0.995580in}}{\pgfqpoint{0.183333in}{1.000000in}}%
\pgfpathcurveto{\pgfqpoint{0.183333in}{1.004420in}}{\pgfqpoint{0.181577in}{1.008660in}}{\pgfqpoint{0.178452in}{1.011785in}}%
\pgfpathcurveto{\pgfqpoint{0.175326in}{1.014911in}}{\pgfqpoint{0.171087in}{1.016667in}}{\pgfqpoint{0.166667in}{1.016667in}}%
\pgfpathcurveto{\pgfqpoint{0.162247in}{1.016667in}}{\pgfqpoint{0.158007in}{1.014911in}}{\pgfqpoint{0.154882in}{1.011785in}}%
\pgfpathcurveto{\pgfqpoint{0.151756in}{1.008660in}}{\pgfqpoint{0.150000in}{1.004420in}}{\pgfqpoint{0.150000in}{1.000000in}}%
\pgfpathcurveto{\pgfqpoint{0.150000in}{0.995580in}}{\pgfqpoint{0.151756in}{0.991340in}}{\pgfqpoint{0.154882in}{0.988215in}}%
\pgfpathcurveto{\pgfqpoint{0.158007in}{0.985089in}}{\pgfqpoint{0.162247in}{0.983333in}}{\pgfqpoint{0.166667in}{0.983333in}}%
\pgfpathclose%
\pgfpathmoveto{\pgfqpoint{0.333333in}{0.983333in}}%
\pgfpathcurveto{\pgfqpoint{0.337753in}{0.983333in}}{\pgfqpoint{0.341993in}{0.985089in}}{\pgfqpoint{0.345118in}{0.988215in}}%
\pgfpathcurveto{\pgfqpoint{0.348244in}{0.991340in}}{\pgfqpoint{0.350000in}{0.995580in}}{\pgfqpoint{0.350000in}{1.000000in}}%
\pgfpathcurveto{\pgfqpoint{0.350000in}{1.004420in}}{\pgfqpoint{0.348244in}{1.008660in}}{\pgfqpoint{0.345118in}{1.011785in}}%
\pgfpathcurveto{\pgfqpoint{0.341993in}{1.014911in}}{\pgfqpoint{0.337753in}{1.016667in}}{\pgfqpoint{0.333333in}{1.016667in}}%
\pgfpathcurveto{\pgfqpoint{0.328913in}{1.016667in}}{\pgfqpoint{0.324674in}{1.014911in}}{\pgfqpoint{0.321548in}{1.011785in}}%
\pgfpathcurveto{\pgfqpoint{0.318423in}{1.008660in}}{\pgfqpoint{0.316667in}{1.004420in}}{\pgfqpoint{0.316667in}{1.000000in}}%
\pgfpathcurveto{\pgfqpoint{0.316667in}{0.995580in}}{\pgfqpoint{0.318423in}{0.991340in}}{\pgfqpoint{0.321548in}{0.988215in}}%
\pgfpathcurveto{\pgfqpoint{0.324674in}{0.985089in}}{\pgfqpoint{0.328913in}{0.983333in}}{\pgfqpoint{0.333333in}{0.983333in}}%
\pgfpathclose%
\pgfpathmoveto{\pgfqpoint{0.500000in}{0.983333in}}%
\pgfpathcurveto{\pgfqpoint{0.504420in}{0.983333in}}{\pgfqpoint{0.508660in}{0.985089in}}{\pgfqpoint{0.511785in}{0.988215in}}%
\pgfpathcurveto{\pgfqpoint{0.514911in}{0.991340in}}{\pgfqpoint{0.516667in}{0.995580in}}{\pgfqpoint{0.516667in}{1.000000in}}%
\pgfpathcurveto{\pgfqpoint{0.516667in}{1.004420in}}{\pgfqpoint{0.514911in}{1.008660in}}{\pgfqpoint{0.511785in}{1.011785in}}%
\pgfpathcurveto{\pgfqpoint{0.508660in}{1.014911in}}{\pgfqpoint{0.504420in}{1.016667in}}{\pgfqpoint{0.500000in}{1.016667in}}%
\pgfpathcurveto{\pgfqpoint{0.495580in}{1.016667in}}{\pgfqpoint{0.491340in}{1.014911in}}{\pgfqpoint{0.488215in}{1.011785in}}%
\pgfpathcurveto{\pgfqpoint{0.485089in}{1.008660in}}{\pgfqpoint{0.483333in}{1.004420in}}{\pgfqpoint{0.483333in}{1.000000in}}%
\pgfpathcurveto{\pgfqpoint{0.483333in}{0.995580in}}{\pgfqpoint{0.485089in}{0.991340in}}{\pgfqpoint{0.488215in}{0.988215in}}%
\pgfpathcurveto{\pgfqpoint{0.491340in}{0.985089in}}{\pgfqpoint{0.495580in}{0.983333in}}{\pgfqpoint{0.500000in}{0.983333in}}%
\pgfpathclose%
\pgfpathmoveto{\pgfqpoint{0.666667in}{0.983333in}}%
\pgfpathcurveto{\pgfqpoint{0.671087in}{0.983333in}}{\pgfqpoint{0.675326in}{0.985089in}}{\pgfqpoint{0.678452in}{0.988215in}}%
\pgfpathcurveto{\pgfqpoint{0.681577in}{0.991340in}}{\pgfqpoint{0.683333in}{0.995580in}}{\pgfqpoint{0.683333in}{1.000000in}}%
\pgfpathcurveto{\pgfqpoint{0.683333in}{1.004420in}}{\pgfqpoint{0.681577in}{1.008660in}}{\pgfqpoint{0.678452in}{1.011785in}}%
\pgfpathcurveto{\pgfqpoint{0.675326in}{1.014911in}}{\pgfqpoint{0.671087in}{1.016667in}}{\pgfqpoint{0.666667in}{1.016667in}}%
\pgfpathcurveto{\pgfqpoint{0.662247in}{1.016667in}}{\pgfqpoint{0.658007in}{1.014911in}}{\pgfqpoint{0.654882in}{1.011785in}}%
\pgfpathcurveto{\pgfqpoint{0.651756in}{1.008660in}}{\pgfqpoint{0.650000in}{1.004420in}}{\pgfqpoint{0.650000in}{1.000000in}}%
\pgfpathcurveto{\pgfqpoint{0.650000in}{0.995580in}}{\pgfqpoint{0.651756in}{0.991340in}}{\pgfqpoint{0.654882in}{0.988215in}}%
\pgfpathcurveto{\pgfqpoint{0.658007in}{0.985089in}}{\pgfqpoint{0.662247in}{0.983333in}}{\pgfqpoint{0.666667in}{0.983333in}}%
\pgfpathclose%
\pgfpathmoveto{\pgfqpoint{0.833333in}{0.983333in}}%
\pgfpathcurveto{\pgfqpoint{0.837753in}{0.983333in}}{\pgfqpoint{0.841993in}{0.985089in}}{\pgfqpoint{0.845118in}{0.988215in}}%
\pgfpathcurveto{\pgfqpoint{0.848244in}{0.991340in}}{\pgfqpoint{0.850000in}{0.995580in}}{\pgfqpoint{0.850000in}{1.000000in}}%
\pgfpathcurveto{\pgfqpoint{0.850000in}{1.004420in}}{\pgfqpoint{0.848244in}{1.008660in}}{\pgfqpoint{0.845118in}{1.011785in}}%
\pgfpathcurveto{\pgfqpoint{0.841993in}{1.014911in}}{\pgfqpoint{0.837753in}{1.016667in}}{\pgfqpoint{0.833333in}{1.016667in}}%
\pgfpathcurveto{\pgfqpoint{0.828913in}{1.016667in}}{\pgfqpoint{0.824674in}{1.014911in}}{\pgfqpoint{0.821548in}{1.011785in}}%
\pgfpathcurveto{\pgfqpoint{0.818423in}{1.008660in}}{\pgfqpoint{0.816667in}{1.004420in}}{\pgfqpoint{0.816667in}{1.000000in}}%
\pgfpathcurveto{\pgfqpoint{0.816667in}{0.995580in}}{\pgfqpoint{0.818423in}{0.991340in}}{\pgfqpoint{0.821548in}{0.988215in}}%
\pgfpathcurveto{\pgfqpoint{0.824674in}{0.985089in}}{\pgfqpoint{0.828913in}{0.983333in}}{\pgfqpoint{0.833333in}{0.983333in}}%
\pgfpathclose%
\pgfpathmoveto{\pgfqpoint{1.000000in}{0.983333in}}%
\pgfpathcurveto{\pgfqpoint{1.004420in}{0.983333in}}{\pgfqpoint{1.008660in}{0.985089in}}{\pgfqpoint{1.011785in}{0.988215in}}%
\pgfpathcurveto{\pgfqpoint{1.014911in}{0.991340in}}{\pgfqpoint{1.016667in}{0.995580in}}{\pgfqpoint{1.016667in}{1.000000in}}%
\pgfpathcurveto{\pgfqpoint{1.016667in}{1.004420in}}{\pgfqpoint{1.014911in}{1.008660in}}{\pgfqpoint{1.011785in}{1.011785in}}%
\pgfpathcurveto{\pgfqpoint{1.008660in}{1.014911in}}{\pgfqpoint{1.004420in}{1.016667in}}{\pgfqpoint{1.000000in}{1.016667in}}%
\pgfpathcurveto{\pgfqpoint{0.995580in}{1.016667in}}{\pgfqpoint{0.991340in}{1.014911in}}{\pgfqpoint{0.988215in}{1.011785in}}%
\pgfpathcurveto{\pgfqpoint{0.985089in}{1.008660in}}{\pgfqpoint{0.983333in}{1.004420in}}{\pgfqpoint{0.983333in}{1.000000in}}%
\pgfpathcurveto{\pgfqpoint{0.983333in}{0.995580in}}{\pgfqpoint{0.985089in}{0.991340in}}{\pgfqpoint{0.988215in}{0.988215in}}%
\pgfpathcurveto{\pgfqpoint{0.991340in}{0.985089in}}{\pgfqpoint{0.995580in}{0.983333in}}{\pgfqpoint{1.000000in}{0.983333in}}%
\pgfpathclose%
\pgfusepath{stroke}%
\end{pgfscope}%
}%
\pgfsys@transformshift{2.873315in}{3.551179in}%
\pgfsys@useobject{currentpattern}{}%
\pgfsys@transformshift{1in}{0in}%
\pgfsys@transformshift{-1in}{0in}%
\pgfsys@transformshift{0in}{1in}%
\end{pgfscope}%
\begin{pgfscope}%
\pgfpathrectangle{\pgfqpoint{0.935815in}{0.637495in}}{\pgfqpoint{9.300000in}{9.060000in}}%
\pgfusepath{clip}%
\pgfsetbuttcap%
\pgfsetmiterjoin%
\definecolor{currentfill}{rgb}{0.172549,0.627451,0.172549}%
\pgfsetfillcolor{currentfill}%
\pgfsetfillopacity{0.990000}%
\pgfsetlinewidth{0.000000pt}%
\definecolor{currentstroke}{rgb}{0.000000,0.000000,0.000000}%
\pgfsetstrokecolor{currentstroke}%
\pgfsetstrokeopacity{0.990000}%
\pgfsetdash{}{0pt}%
\pgfpathmoveto{\pgfqpoint{4.423315in}{4.380181in}}%
\pgfpathlineto{\pgfqpoint{5.198315in}{4.380181in}}%
\pgfpathlineto{\pgfqpoint{5.198315in}{4.688801in}}%
\pgfpathlineto{\pgfqpoint{4.423315in}{4.688801in}}%
\pgfpathclose%
\pgfusepath{fill}%
\end{pgfscope}%
\begin{pgfscope}%
\pgfsetbuttcap%
\pgfsetmiterjoin%
\definecolor{currentfill}{rgb}{0.172549,0.627451,0.172549}%
\pgfsetfillcolor{currentfill}%
\pgfsetfillopacity{0.990000}%
\pgfsetlinewidth{0.000000pt}%
\definecolor{currentstroke}{rgb}{0.000000,0.000000,0.000000}%
\pgfsetstrokecolor{currentstroke}%
\pgfsetstrokeopacity{0.990000}%
\pgfsetdash{}{0pt}%
\pgfpathrectangle{\pgfqpoint{0.935815in}{0.637495in}}{\pgfqpoint{9.300000in}{9.060000in}}%
\pgfusepath{clip}%
\pgfpathmoveto{\pgfqpoint{4.423315in}{4.380181in}}%
\pgfpathlineto{\pgfqpoint{5.198315in}{4.380181in}}%
\pgfpathlineto{\pgfqpoint{5.198315in}{4.688801in}}%
\pgfpathlineto{\pgfqpoint{4.423315in}{4.688801in}}%
\pgfpathclose%
\pgfusepath{clip}%
\pgfsys@defobject{currentpattern}{\pgfqpoint{0in}{0in}}{\pgfqpoint{1in}{1in}}{%
\begin{pgfscope}%
\pgfpathrectangle{\pgfqpoint{0in}{0in}}{\pgfqpoint{1in}{1in}}%
\pgfusepath{clip}%
\pgfpathmoveto{\pgfqpoint{0.000000in}{-0.016667in}}%
\pgfpathcurveto{\pgfqpoint{0.004420in}{-0.016667in}}{\pgfqpoint{0.008660in}{-0.014911in}}{\pgfqpoint{0.011785in}{-0.011785in}}%
\pgfpathcurveto{\pgfqpoint{0.014911in}{-0.008660in}}{\pgfqpoint{0.016667in}{-0.004420in}}{\pgfqpoint{0.016667in}{0.000000in}}%
\pgfpathcurveto{\pgfqpoint{0.016667in}{0.004420in}}{\pgfqpoint{0.014911in}{0.008660in}}{\pgfqpoint{0.011785in}{0.011785in}}%
\pgfpathcurveto{\pgfqpoint{0.008660in}{0.014911in}}{\pgfqpoint{0.004420in}{0.016667in}}{\pgfqpoint{0.000000in}{0.016667in}}%
\pgfpathcurveto{\pgfqpoint{-0.004420in}{0.016667in}}{\pgfqpoint{-0.008660in}{0.014911in}}{\pgfqpoint{-0.011785in}{0.011785in}}%
\pgfpathcurveto{\pgfqpoint{-0.014911in}{0.008660in}}{\pgfqpoint{-0.016667in}{0.004420in}}{\pgfqpoint{-0.016667in}{0.000000in}}%
\pgfpathcurveto{\pgfqpoint{-0.016667in}{-0.004420in}}{\pgfqpoint{-0.014911in}{-0.008660in}}{\pgfqpoint{-0.011785in}{-0.011785in}}%
\pgfpathcurveto{\pgfqpoint{-0.008660in}{-0.014911in}}{\pgfqpoint{-0.004420in}{-0.016667in}}{\pgfqpoint{0.000000in}{-0.016667in}}%
\pgfpathclose%
\pgfpathmoveto{\pgfqpoint{0.166667in}{-0.016667in}}%
\pgfpathcurveto{\pgfqpoint{0.171087in}{-0.016667in}}{\pgfqpoint{0.175326in}{-0.014911in}}{\pgfqpoint{0.178452in}{-0.011785in}}%
\pgfpathcurveto{\pgfqpoint{0.181577in}{-0.008660in}}{\pgfqpoint{0.183333in}{-0.004420in}}{\pgfqpoint{0.183333in}{0.000000in}}%
\pgfpathcurveto{\pgfqpoint{0.183333in}{0.004420in}}{\pgfqpoint{0.181577in}{0.008660in}}{\pgfqpoint{0.178452in}{0.011785in}}%
\pgfpathcurveto{\pgfqpoint{0.175326in}{0.014911in}}{\pgfqpoint{0.171087in}{0.016667in}}{\pgfqpoint{0.166667in}{0.016667in}}%
\pgfpathcurveto{\pgfqpoint{0.162247in}{0.016667in}}{\pgfqpoint{0.158007in}{0.014911in}}{\pgfqpoint{0.154882in}{0.011785in}}%
\pgfpathcurveto{\pgfqpoint{0.151756in}{0.008660in}}{\pgfqpoint{0.150000in}{0.004420in}}{\pgfqpoint{0.150000in}{0.000000in}}%
\pgfpathcurveto{\pgfqpoint{0.150000in}{-0.004420in}}{\pgfqpoint{0.151756in}{-0.008660in}}{\pgfqpoint{0.154882in}{-0.011785in}}%
\pgfpathcurveto{\pgfqpoint{0.158007in}{-0.014911in}}{\pgfqpoint{0.162247in}{-0.016667in}}{\pgfqpoint{0.166667in}{-0.016667in}}%
\pgfpathclose%
\pgfpathmoveto{\pgfqpoint{0.333333in}{-0.016667in}}%
\pgfpathcurveto{\pgfqpoint{0.337753in}{-0.016667in}}{\pgfqpoint{0.341993in}{-0.014911in}}{\pgfqpoint{0.345118in}{-0.011785in}}%
\pgfpathcurveto{\pgfqpoint{0.348244in}{-0.008660in}}{\pgfqpoint{0.350000in}{-0.004420in}}{\pgfqpoint{0.350000in}{0.000000in}}%
\pgfpathcurveto{\pgfqpoint{0.350000in}{0.004420in}}{\pgfqpoint{0.348244in}{0.008660in}}{\pgfqpoint{0.345118in}{0.011785in}}%
\pgfpathcurveto{\pgfqpoint{0.341993in}{0.014911in}}{\pgfqpoint{0.337753in}{0.016667in}}{\pgfqpoint{0.333333in}{0.016667in}}%
\pgfpathcurveto{\pgfqpoint{0.328913in}{0.016667in}}{\pgfqpoint{0.324674in}{0.014911in}}{\pgfqpoint{0.321548in}{0.011785in}}%
\pgfpathcurveto{\pgfqpoint{0.318423in}{0.008660in}}{\pgfqpoint{0.316667in}{0.004420in}}{\pgfqpoint{0.316667in}{0.000000in}}%
\pgfpathcurveto{\pgfqpoint{0.316667in}{-0.004420in}}{\pgfqpoint{0.318423in}{-0.008660in}}{\pgfqpoint{0.321548in}{-0.011785in}}%
\pgfpathcurveto{\pgfqpoint{0.324674in}{-0.014911in}}{\pgfqpoint{0.328913in}{-0.016667in}}{\pgfqpoint{0.333333in}{-0.016667in}}%
\pgfpathclose%
\pgfpathmoveto{\pgfqpoint{0.500000in}{-0.016667in}}%
\pgfpathcurveto{\pgfqpoint{0.504420in}{-0.016667in}}{\pgfqpoint{0.508660in}{-0.014911in}}{\pgfqpoint{0.511785in}{-0.011785in}}%
\pgfpathcurveto{\pgfqpoint{0.514911in}{-0.008660in}}{\pgfqpoint{0.516667in}{-0.004420in}}{\pgfqpoint{0.516667in}{0.000000in}}%
\pgfpathcurveto{\pgfqpoint{0.516667in}{0.004420in}}{\pgfqpoint{0.514911in}{0.008660in}}{\pgfqpoint{0.511785in}{0.011785in}}%
\pgfpathcurveto{\pgfqpoint{0.508660in}{0.014911in}}{\pgfqpoint{0.504420in}{0.016667in}}{\pgfqpoint{0.500000in}{0.016667in}}%
\pgfpathcurveto{\pgfqpoint{0.495580in}{0.016667in}}{\pgfqpoint{0.491340in}{0.014911in}}{\pgfqpoint{0.488215in}{0.011785in}}%
\pgfpathcurveto{\pgfqpoint{0.485089in}{0.008660in}}{\pgfqpoint{0.483333in}{0.004420in}}{\pgfqpoint{0.483333in}{0.000000in}}%
\pgfpathcurveto{\pgfqpoint{0.483333in}{-0.004420in}}{\pgfqpoint{0.485089in}{-0.008660in}}{\pgfqpoint{0.488215in}{-0.011785in}}%
\pgfpathcurveto{\pgfqpoint{0.491340in}{-0.014911in}}{\pgfqpoint{0.495580in}{-0.016667in}}{\pgfqpoint{0.500000in}{-0.016667in}}%
\pgfpathclose%
\pgfpathmoveto{\pgfqpoint{0.666667in}{-0.016667in}}%
\pgfpathcurveto{\pgfqpoint{0.671087in}{-0.016667in}}{\pgfqpoint{0.675326in}{-0.014911in}}{\pgfqpoint{0.678452in}{-0.011785in}}%
\pgfpathcurveto{\pgfqpoint{0.681577in}{-0.008660in}}{\pgfqpoint{0.683333in}{-0.004420in}}{\pgfqpoint{0.683333in}{0.000000in}}%
\pgfpathcurveto{\pgfqpoint{0.683333in}{0.004420in}}{\pgfqpoint{0.681577in}{0.008660in}}{\pgfqpoint{0.678452in}{0.011785in}}%
\pgfpathcurveto{\pgfqpoint{0.675326in}{0.014911in}}{\pgfqpoint{0.671087in}{0.016667in}}{\pgfqpoint{0.666667in}{0.016667in}}%
\pgfpathcurveto{\pgfqpoint{0.662247in}{0.016667in}}{\pgfqpoint{0.658007in}{0.014911in}}{\pgfqpoint{0.654882in}{0.011785in}}%
\pgfpathcurveto{\pgfqpoint{0.651756in}{0.008660in}}{\pgfqpoint{0.650000in}{0.004420in}}{\pgfqpoint{0.650000in}{0.000000in}}%
\pgfpathcurveto{\pgfqpoint{0.650000in}{-0.004420in}}{\pgfqpoint{0.651756in}{-0.008660in}}{\pgfqpoint{0.654882in}{-0.011785in}}%
\pgfpathcurveto{\pgfqpoint{0.658007in}{-0.014911in}}{\pgfqpoint{0.662247in}{-0.016667in}}{\pgfqpoint{0.666667in}{-0.016667in}}%
\pgfpathclose%
\pgfpathmoveto{\pgfqpoint{0.833333in}{-0.016667in}}%
\pgfpathcurveto{\pgfqpoint{0.837753in}{-0.016667in}}{\pgfqpoint{0.841993in}{-0.014911in}}{\pgfqpoint{0.845118in}{-0.011785in}}%
\pgfpathcurveto{\pgfqpoint{0.848244in}{-0.008660in}}{\pgfqpoint{0.850000in}{-0.004420in}}{\pgfqpoint{0.850000in}{0.000000in}}%
\pgfpathcurveto{\pgfqpoint{0.850000in}{0.004420in}}{\pgfqpoint{0.848244in}{0.008660in}}{\pgfqpoint{0.845118in}{0.011785in}}%
\pgfpathcurveto{\pgfqpoint{0.841993in}{0.014911in}}{\pgfqpoint{0.837753in}{0.016667in}}{\pgfqpoint{0.833333in}{0.016667in}}%
\pgfpathcurveto{\pgfqpoint{0.828913in}{0.016667in}}{\pgfqpoint{0.824674in}{0.014911in}}{\pgfqpoint{0.821548in}{0.011785in}}%
\pgfpathcurveto{\pgfqpoint{0.818423in}{0.008660in}}{\pgfqpoint{0.816667in}{0.004420in}}{\pgfqpoint{0.816667in}{0.000000in}}%
\pgfpathcurveto{\pgfqpoint{0.816667in}{-0.004420in}}{\pgfqpoint{0.818423in}{-0.008660in}}{\pgfqpoint{0.821548in}{-0.011785in}}%
\pgfpathcurveto{\pgfqpoint{0.824674in}{-0.014911in}}{\pgfqpoint{0.828913in}{-0.016667in}}{\pgfqpoint{0.833333in}{-0.016667in}}%
\pgfpathclose%
\pgfpathmoveto{\pgfqpoint{1.000000in}{-0.016667in}}%
\pgfpathcurveto{\pgfqpoint{1.004420in}{-0.016667in}}{\pgfqpoint{1.008660in}{-0.014911in}}{\pgfqpoint{1.011785in}{-0.011785in}}%
\pgfpathcurveto{\pgfqpoint{1.014911in}{-0.008660in}}{\pgfqpoint{1.016667in}{-0.004420in}}{\pgfqpoint{1.016667in}{0.000000in}}%
\pgfpathcurveto{\pgfqpoint{1.016667in}{0.004420in}}{\pgfqpoint{1.014911in}{0.008660in}}{\pgfqpoint{1.011785in}{0.011785in}}%
\pgfpathcurveto{\pgfqpoint{1.008660in}{0.014911in}}{\pgfqpoint{1.004420in}{0.016667in}}{\pgfqpoint{1.000000in}{0.016667in}}%
\pgfpathcurveto{\pgfqpoint{0.995580in}{0.016667in}}{\pgfqpoint{0.991340in}{0.014911in}}{\pgfqpoint{0.988215in}{0.011785in}}%
\pgfpathcurveto{\pgfqpoint{0.985089in}{0.008660in}}{\pgfqpoint{0.983333in}{0.004420in}}{\pgfqpoint{0.983333in}{0.000000in}}%
\pgfpathcurveto{\pgfqpoint{0.983333in}{-0.004420in}}{\pgfqpoint{0.985089in}{-0.008660in}}{\pgfqpoint{0.988215in}{-0.011785in}}%
\pgfpathcurveto{\pgfqpoint{0.991340in}{-0.014911in}}{\pgfqpoint{0.995580in}{-0.016667in}}{\pgfqpoint{1.000000in}{-0.016667in}}%
\pgfpathclose%
\pgfpathmoveto{\pgfqpoint{0.083333in}{0.150000in}}%
\pgfpathcurveto{\pgfqpoint{0.087753in}{0.150000in}}{\pgfqpoint{0.091993in}{0.151756in}}{\pgfqpoint{0.095118in}{0.154882in}}%
\pgfpathcurveto{\pgfqpoint{0.098244in}{0.158007in}}{\pgfqpoint{0.100000in}{0.162247in}}{\pgfqpoint{0.100000in}{0.166667in}}%
\pgfpathcurveto{\pgfqpoint{0.100000in}{0.171087in}}{\pgfqpoint{0.098244in}{0.175326in}}{\pgfqpoint{0.095118in}{0.178452in}}%
\pgfpathcurveto{\pgfqpoint{0.091993in}{0.181577in}}{\pgfqpoint{0.087753in}{0.183333in}}{\pgfqpoint{0.083333in}{0.183333in}}%
\pgfpathcurveto{\pgfqpoint{0.078913in}{0.183333in}}{\pgfqpoint{0.074674in}{0.181577in}}{\pgfqpoint{0.071548in}{0.178452in}}%
\pgfpathcurveto{\pgfqpoint{0.068423in}{0.175326in}}{\pgfqpoint{0.066667in}{0.171087in}}{\pgfqpoint{0.066667in}{0.166667in}}%
\pgfpathcurveto{\pgfqpoint{0.066667in}{0.162247in}}{\pgfqpoint{0.068423in}{0.158007in}}{\pgfqpoint{0.071548in}{0.154882in}}%
\pgfpathcurveto{\pgfqpoint{0.074674in}{0.151756in}}{\pgfqpoint{0.078913in}{0.150000in}}{\pgfqpoint{0.083333in}{0.150000in}}%
\pgfpathclose%
\pgfpathmoveto{\pgfqpoint{0.250000in}{0.150000in}}%
\pgfpathcurveto{\pgfqpoint{0.254420in}{0.150000in}}{\pgfqpoint{0.258660in}{0.151756in}}{\pgfqpoint{0.261785in}{0.154882in}}%
\pgfpathcurveto{\pgfqpoint{0.264911in}{0.158007in}}{\pgfqpoint{0.266667in}{0.162247in}}{\pgfqpoint{0.266667in}{0.166667in}}%
\pgfpathcurveto{\pgfqpoint{0.266667in}{0.171087in}}{\pgfqpoint{0.264911in}{0.175326in}}{\pgfqpoint{0.261785in}{0.178452in}}%
\pgfpathcurveto{\pgfqpoint{0.258660in}{0.181577in}}{\pgfqpoint{0.254420in}{0.183333in}}{\pgfqpoint{0.250000in}{0.183333in}}%
\pgfpathcurveto{\pgfqpoint{0.245580in}{0.183333in}}{\pgfqpoint{0.241340in}{0.181577in}}{\pgfqpoint{0.238215in}{0.178452in}}%
\pgfpathcurveto{\pgfqpoint{0.235089in}{0.175326in}}{\pgfqpoint{0.233333in}{0.171087in}}{\pgfqpoint{0.233333in}{0.166667in}}%
\pgfpathcurveto{\pgfqpoint{0.233333in}{0.162247in}}{\pgfqpoint{0.235089in}{0.158007in}}{\pgfqpoint{0.238215in}{0.154882in}}%
\pgfpathcurveto{\pgfqpoint{0.241340in}{0.151756in}}{\pgfqpoint{0.245580in}{0.150000in}}{\pgfqpoint{0.250000in}{0.150000in}}%
\pgfpathclose%
\pgfpathmoveto{\pgfqpoint{0.416667in}{0.150000in}}%
\pgfpathcurveto{\pgfqpoint{0.421087in}{0.150000in}}{\pgfqpoint{0.425326in}{0.151756in}}{\pgfqpoint{0.428452in}{0.154882in}}%
\pgfpathcurveto{\pgfqpoint{0.431577in}{0.158007in}}{\pgfqpoint{0.433333in}{0.162247in}}{\pgfqpoint{0.433333in}{0.166667in}}%
\pgfpathcurveto{\pgfqpoint{0.433333in}{0.171087in}}{\pgfqpoint{0.431577in}{0.175326in}}{\pgfqpoint{0.428452in}{0.178452in}}%
\pgfpathcurveto{\pgfqpoint{0.425326in}{0.181577in}}{\pgfqpoint{0.421087in}{0.183333in}}{\pgfqpoint{0.416667in}{0.183333in}}%
\pgfpathcurveto{\pgfqpoint{0.412247in}{0.183333in}}{\pgfqpoint{0.408007in}{0.181577in}}{\pgfqpoint{0.404882in}{0.178452in}}%
\pgfpathcurveto{\pgfqpoint{0.401756in}{0.175326in}}{\pgfqpoint{0.400000in}{0.171087in}}{\pgfqpoint{0.400000in}{0.166667in}}%
\pgfpathcurveto{\pgfqpoint{0.400000in}{0.162247in}}{\pgfqpoint{0.401756in}{0.158007in}}{\pgfqpoint{0.404882in}{0.154882in}}%
\pgfpathcurveto{\pgfqpoint{0.408007in}{0.151756in}}{\pgfqpoint{0.412247in}{0.150000in}}{\pgfqpoint{0.416667in}{0.150000in}}%
\pgfpathclose%
\pgfpathmoveto{\pgfqpoint{0.583333in}{0.150000in}}%
\pgfpathcurveto{\pgfqpoint{0.587753in}{0.150000in}}{\pgfqpoint{0.591993in}{0.151756in}}{\pgfqpoint{0.595118in}{0.154882in}}%
\pgfpathcurveto{\pgfqpoint{0.598244in}{0.158007in}}{\pgfqpoint{0.600000in}{0.162247in}}{\pgfqpoint{0.600000in}{0.166667in}}%
\pgfpathcurveto{\pgfqpoint{0.600000in}{0.171087in}}{\pgfqpoint{0.598244in}{0.175326in}}{\pgfqpoint{0.595118in}{0.178452in}}%
\pgfpathcurveto{\pgfqpoint{0.591993in}{0.181577in}}{\pgfqpoint{0.587753in}{0.183333in}}{\pgfqpoint{0.583333in}{0.183333in}}%
\pgfpathcurveto{\pgfqpoint{0.578913in}{0.183333in}}{\pgfqpoint{0.574674in}{0.181577in}}{\pgfqpoint{0.571548in}{0.178452in}}%
\pgfpathcurveto{\pgfqpoint{0.568423in}{0.175326in}}{\pgfqpoint{0.566667in}{0.171087in}}{\pgfqpoint{0.566667in}{0.166667in}}%
\pgfpathcurveto{\pgfqpoint{0.566667in}{0.162247in}}{\pgfqpoint{0.568423in}{0.158007in}}{\pgfqpoint{0.571548in}{0.154882in}}%
\pgfpathcurveto{\pgfqpoint{0.574674in}{0.151756in}}{\pgfqpoint{0.578913in}{0.150000in}}{\pgfqpoint{0.583333in}{0.150000in}}%
\pgfpathclose%
\pgfpathmoveto{\pgfqpoint{0.750000in}{0.150000in}}%
\pgfpathcurveto{\pgfqpoint{0.754420in}{0.150000in}}{\pgfqpoint{0.758660in}{0.151756in}}{\pgfqpoint{0.761785in}{0.154882in}}%
\pgfpathcurveto{\pgfqpoint{0.764911in}{0.158007in}}{\pgfqpoint{0.766667in}{0.162247in}}{\pgfqpoint{0.766667in}{0.166667in}}%
\pgfpathcurveto{\pgfqpoint{0.766667in}{0.171087in}}{\pgfqpoint{0.764911in}{0.175326in}}{\pgfqpoint{0.761785in}{0.178452in}}%
\pgfpathcurveto{\pgfqpoint{0.758660in}{0.181577in}}{\pgfqpoint{0.754420in}{0.183333in}}{\pgfqpoint{0.750000in}{0.183333in}}%
\pgfpathcurveto{\pgfqpoint{0.745580in}{0.183333in}}{\pgfqpoint{0.741340in}{0.181577in}}{\pgfqpoint{0.738215in}{0.178452in}}%
\pgfpathcurveto{\pgfqpoint{0.735089in}{0.175326in}}{\pgfqpoint{0.733333in}{0.171087in}}{\pgfqpoint{0.733333in}{0.166667in}}%
\pgfpathcurveto{\pgfqpoint{0.733333in}{0.162247in}}{\pgfqpoint{0.735089in}{0.158007in}}{\pgfqpoint{0.738215in}{0.154882in}}%
\pgfpathcurveto{\pgfqpoint{0.741340in}{0.151756in}}{\pgfqpoint{0.745580in}{0.150000in}}{\pgfqpoint{0.750000in}{0.150000in}}%
\pgfpathclose%
\pgfpathmoveto{\pgfqpoint{0.916667in}{0.150000in}}%
\pgfpathcurveto{\pgfqpoint{0.921087in}{0.150000in}}{\pgfqpoint{0.925326in}{0.151756in}}{\pgfqpoint{0.928452in}{0.154882in}}%
\pgfpathcurveto{\pgfqpoint{0.931577in}{0.158007in}}{\pgfqpoint{0.933333in}{0.162247in}}{\pgfqpoint{0.933333in}{0.166667in}}%
\pgfpathcurveto{\pgfqpoint{0.933333in}{0.171087in}}{\pgfqpoint{0.931577in}{0.175326in}}{\pgfqpoint{0.928452in}{0.178452in}}%
\pgfpathcurveto{\pgfqpoint{0.925326in}{0.181577in}}{\pgfqpoint{0.921087in}{0.183333in}}{\pgfqpoint{0.916667in}{0.183333in}}%
\pgfpathcurveto{\pgfqpoint{0.912247in}{0.183333in}}{\pgfqpoint{0.908007in}{0.181577in}}{\pgfqpoint{0.904882in}{0.178452in}}%
\pgfpathcurveto{\pgfqpoint{0.901756in}{0.175326in}}{\pgfqpoint{0.900000in}{0.171087in}}{\pgfqpoint{0.900000in}{0.166667in}}%
\pgfpathcurveto{\pgfqpoint{0.900000in}{0.162247in}}{\pgfqpoint{0.901756in}{0.158007in}}{\pgfqpoint{0.904882in}{0.154882in}}%
\pgfpathcurveto{\pgfqpoint{0.908007in}{0.151756in}}{\pgfqpoint{0.912247in}{0.150000in}}{\pgfqpoint{0.916667in}{0.150000in}}%
\pgfpathclose%
\pgfpathmoveto{\pgfqpoint{0.000000in}{0.316667in}}%
\pgfpathcurveto{\pgfqpoint{0.004420in}{0.316667in}}{\pgfqpoint{0.008660in}{0.318423in}}{\pgfqpoint{0.011785in}{0.321548in}}%
\pgfpathcurveto{\pgfqpoint{0.014911in}{0.324674in}}{\pgfqpoint{0.016667in}{0.328913in}}{\pgfqpoint{0.016667in}{0.333333in}}%
\pgfpathcurveto{\pgfqpoint{0.016667in}{0.337753in}}{\pgfqpoint{0.014911in}{0.341993in}}{\pgfqpoint{0.011785in}{0.345118in}}%
\pgfpathcurveto{\pgfqpoint{0.008660in}{0.348244in}}{\pgfqpoint{0.004420in}{0.350000in}}{\pgfqpoint{0.000000in}{0.350000in}}%
\pgfpathcurveto{\pgfqpoint{-0.004420in}{0.350000in}}{\pgfqpoint{-0.008660in}{0.348244in}}{\pgfqpoint{-0.011785in}{0.345118in}}%
\pgfpathcurveto{\pgfqpoint{-0.014911in}{0.341993in}}{\pgfqpoint{-0.016667in}{0.337753in}}{\pgfqpoint{-0.016667in}{0.333333in}}%
\pgfpathcurveto{\pgfqpoint{-0.016667in}{0.328913in}}{\pgfqpoint{-0.014911in}{0.324674in}}{\pgfqpoint{-0.011785in}{0.321548in}}%
\pgfpathcurveto{\pgfqpoint{-0.008660in}{0.318423in}}{\pgfqpoint{-0.004420in}{0.316667in}}{\pgfqpoint{0.000000in}{0.316667in}}%
\pgfpathclose%
\pgfpathmoveto{\pgfqpoint{0.166667in}{0.316667in}}%
\pgfpathcurveto{\pgfqpoint{0.171087in}{0.316667in}}{\pgfqpoint{0.175326in}{0.318423in}}{\pgfqpoint{0.178452in}{0.321548in}}%
\pgfpathcurveto{\pgfqpoint{0.181577in}{0.324674in}}{\pgfqpoint{0.183333in}{0.328913in}}{\pgfqpoint{0.183333in}{0.333333in}}%
\pgfpathcurveto{\pgfqpoint{0.183333in}{0.337753in}}{\pgfqpoint{0.181577in}{0.341993in}}{\pgfqpoint{0.178452in}{0.345118in}}%
\pgfpathcurveto{\pgfqpoint{0.175326in}{0.348244in}}{\pgfqpoint{0.171087in}{0.350000in}}{\pgfqpoint{0.166667in}{0.350000in}}%
\pgfpathcurveto{\pgfqpoint{0.162247in}{0.350000in}}{\pgfqpoint{0.158007in}{0.348244in}}{\pgfqpoint{0.154882in}{0.345118in}}%
\pgfpathcurveto{\pgfqpoint{0.151756in}{0.341993in}}{\pgfqpoint{0.150000in}{0.337753in}}{\pgfqpoint{0.150000in}{0.333333in}}%
\pgfpathcurveto{\pgfqpoint{0.150000in}{0.328913in}}{\pgfqpoint{0.151756in}{0.324674in}}{\pgfqpoint{0.154882in}{0.321548in}}%
\pgfpathcurveto{\pgfqpoint{0.158007in}{0.318423in}}{\pgfqpoint{0.162247in}{0.316667in}}{\pgfqpoint{0.166667in}{0.316667in}}%
\pgfpathclose%
\pgfpathmoveto{\pgfqpoint{0.333333in}{0.316667in}}%
\pgfpathcurveto{\pgfqpoint{0.337753in}{0.316667in}}{\pgfqpoint{0.341993in}{0.318423in}}{\pgfqpoint{0.345118in}{0.321548in}}%
\pgfpathcurveto{\pgfqpoint{0.348244in}{0.324674in}}{\pgfqpoint{0.350000in}{0.328913in}}{\pgfqpoint{0.350000in}{0.333333in}}%
\pgfpathcurveto{\pgfqpoint{0.350000in}{0.337753in}}{\pgfqpoint{0.348244in}{0.341993in}}{\pgfqpoint{0.345118in}{0.345118in}}%
\pgfpathcurveto{\pgfqpoint{0.341993in}{0.348244in}}{\pgfqpoint{0.337753in}{0.350000in}}{\pgfqpoint{0.333333in}{0.350000in}}%
\pgfpathcurveto{\pgfqpoint{0.328913in}{0.350000in}}{\pgfqpoint{0.324674in}{0.348244in}}{\pgfqpoint{0.321548in}{0.345118in}}%
\pgfpathcurveto{\pgfqpoint{0.318423in}{0.341993in}}{\pgfqpoint{0.316667in}{0.337753in}}{\pgfqpoint{0.316667in}{0.333333in}}%
\pgfpathcurveto{\pgfqpoint{0.316667in}{0.328913in}}{\pgfqpoint{0.318423in}{0.324674in}}{\pgfqpoint{0.321548in}{0.321548in}}%
\pgfpathcurveto{\pgfqpoint{0.324674in}{0.318423in}}{\pgfqpoint{0.328913in}{0.316667in}}{\pgfqpoint{0.333333in}{0.316667in}}%
\pgfpathclose%
\pgfpathmoveto{\pgfqpoint{0.500000in}{0.316667in}}%
\pgfpathcurveto{\pgfqpoint{0.504420in}{0.316667in}}{\pgfqpoint{0.508660in}{0.318423in}}{\pgfqpoint{0.511785in}{0.321548in}}%
\pgfpathcurveto{\pgfqpoint{0.514911in}{0.324674in}}{\pgfqpoint{0.516667in}{0.328913in}}{\pgfqpoint{0.516667in}{0.333333in}}%
\pgfpathcurveto{\pgfqpoint{0.516667in}{0.337753in}}{\pgfqpoint{0.514911in}{0.341993in}}{\pgfqpoint{0.511785in}{0.345118in}}%
\pgfpathcurveto{\pgfqpoint{0.508660in}{0.348244in}}{\pgfqpoint{0.504420in}{0.350000in}}{\pgfqpoint{0.500000in}{0.350000in}}%
\pgfpathcurveto{\pgfqpoint{0.495580in}{0.350000in}}{\pgfqpoint{0.491340in}{0.348244in}}{\pgfqpoint{0.488215in}{0.345118in}}%
\pgfpathcurveto{\pgfqpoint{0.485089in}{0.341993in}}{\pgfqpoint{0.483333in}{0.337753in}}{\pgfqpoint{0.483333in}{0.333333in}}%
\pgfpathcurveto{\pgfqpoint{0.483333in}{0.328913in}}{\pgfqpoint{0.485089in}{0.324674in}}{\pgfqpoint{0.488215in}{0.321548in}}%
\pgfpathcurveto{\pgfqpoint{0.491340in}{0.318423in}}{\pgfqpoint{0.495580in}{0.316667in}}{\pgfqpoint{0.500000in}{0.316667in}}%
\pgfpathclose%
\pgfpathmoveto{\pgfqpoint{0.666667in}{0.316667in}}%
\pgfpathcurveto{\pgfqpoint{0.671087in}{0.316667in}}{\pgfqpoint{0.675326in}{0.318423in}}{\pgfqpoint{0.678452in}{0.321548in}}%
\pgfpathcurveto{\pgfqpoint{0.681577in}{0.324674in}}{\pgfqpoint{0.683333in}{0.328913in}}{\pgfqpoint{0.683333in}{0.333333in}}%
\pgfpathcurveto{\pgfqpoint{0.683333in}{0.337753in}}{\pgfqpoint{0.681577in}{0.341993in}}{\pgfqpoint{0.678452in}{0.345118in}}%
\pgfpathcurveto{\pgfqpoint{0.675326in}{0.348244in}}{\pgfqpoint{0.671087in}{0.350000in}}{\pgfqpoint{0.666667in}{0.350000in}}%
\pgfpathcurveto{\pgfqpoint{0.662247in}{0.350000in}}{\pgfqpoint{0.658007in}{0.348244in}}{\pgfqpoint{0.654882in}{0.345118in}}%
\pgfpathcurveto{\pgfqpoint{0.651756in}{0.341993in}}{\pgfqpoint{0.650000in}{0.337753in}}{\pgfqpoint{0.650000in}{0.333333in}}%
\pgfpathcurveto{\pgfqpoint{0.650000in}{0.328913in}}{\pgfqpoint{0.651756in}{0.324674in}}{\pgfqpoint{0.654882in}{0.321548in}}%
\pgfpathcurveto{\pgfqpoint{0.658007in}{0.318423in}}{\pgfqpoint{0.662247in}{0.316667in}}{\pgfqpoint{0.666667in}{0.316667in}}%
\pgfpathclose%
\pgfpathmoveto{\pgfqpoint{0.833333in}{0.316667in}}%
\pgfpathcurveto{\pgfqpoint{0.837753in}{0.316667in}}{\pgfqpoint{0.841993in}{0.318423in}}{\pgfqpoint{0.845118in}{0.321548in}}%
\pgfpathcurveto{\pgfqpoint{0.848244in}{0.324674in}}{\pgfqpoint{0.850000in}{0.328913in}}{\pgfqpoint{0.850000in}{0.333333in}}%
\pgfpathcurveto{\pgfqpoint{0.850000in}{0.337753in}}{\pgfqpoint{0.848244in}{0.341993in}}{\pgfqpoint{0.845118in}{0.345118in}}%
\pgfpathcurveto{\pgfqpoint{0.841993in}{0.348244in}}{\pgfqpoint{0.837753in}{0.350000in}}{\pgfqpoint{0.833333in}{0.350000in}}%
\pgfpathcurveto{\pgfqpoint{0.828913in}{0.350000in}}{\pgfqpoint{0.824674in}{0.348244in}}{\pgfqpoint{0.821548in}{0.345118in}}%
\pgfpathcurveto{\pgfqpoint{0.818423in}{0.341993in}}{\pgfqpoint{0.816667in}{0.337753in}}{\pgfqpoint{0.816667in}{0.333333in}}%
\pgfpathcurveto{\pgfqpoint{0.816667in}{0.328913in}}{\pgfqpoint{0.818423in}{0.324674in}}{\pgfqpoint{0.821548in}{0.321548in}}%
\pgfpathcurveto{\pgfqpoint{0.824674in}{0.318423in}}{\pgfqpoint{0.828913in}{0.316667in}}{\pgfqpoint{0.833333in}{0.316667in}}%
\pgfpathclose%
\pgfpathmoveto{\pgfqpoint{1.000000in}{0.316667in}}%
\pgfpathcurveto{\pgfqpoint{1.004420in}{0.316667in}}{\pgfqpoint{1.008660in}{0.318423in}}{\pgfqpoint{1.011785in}{0.321548in}}%
\pgfpathcurveto{\pgfqpoint{1.014911in}{0.324674in}}{\pgfqpoint{1.016667in}{0.328913in}}{\pgfqpoint{1.016667in}{0.333333in}}%
\pgfpathcurveto{\pgfqpoint{1.016667in}{0.337753in}}{\pgfqpoint{1.014911in}{0.341993in}}{\pgfqpoint{1.011785in}{0.345118in}}%
\pgfpathcurveto{\pgfqpoint{1.008660in}{0.348244in}}{\pgfqpoint{1.004420in}{0.350000in}}{\pgfqpoint{1.000000in}{0.350000in}}%
\pgfpathcurveto{\pgfqpoint{0.995580in}{0.350000in}}{\pgfqpoint{0.991340in}{0.348244in}}{\pgfqpoint{0.988215in}{0.345118in}}%
\pgfpathcurveto{\pgfqpoint{0.985089in}{0.341993in}}{\pgfqpoint{0.983333in}{0.337753in}}{\pgfqpoint{0.983333in}{0.333333in}}%
\pgfpathcurveto{\pgfqpoint{0.983333in}{0.328913in}}{\pgfqpoint{0.985089in}{0.324674in}}{\pgfqpoint{0.988215in}{0.321548in}}%
\pgfpathcurveto{\pgfqpoint{0.991340in}{0.318423in}}{\pgfqpoint{0.995580in}{0.316667in}}{\pgfqpoint{1.000000in}{0.316667in}}%
\pgfpathclose%
\pgfpathmoveto{\pgfqpoint{0.083333in}{0.483333in}}%
\pgfpathcurveto{\pgfqpoint{0.087753in}{0.483333in}}{\pgfqpoint{0.091993in}{0.485089in}}{\pgfqpoint{0.095118in}{0.488215in}}%
\pgfpathcurveto{\pgfqpoint{0.098244in}{0.491340in}}{\pgfqpoint{0.100000in}{0.495580in}}{\pgfqpoint{0.100000in}{0.500000in}}%
\pgfpathcurveto{\pgfqpoint{0.100000in}{0.504420in}}{\pgfqpoint{0.098244in}{0.508660in}}{\pgfqpoint{0.095118in}{0.511785in}}%
\pgfpathcurveto{\pgfqpoint{0.091993in}{0.514911in}}{\pgfqpoint{0.087753in}{0.516667in}}{\pgfqpoint{0.083333in}{0.516667in}}%
\pgfpathcurveto{\pgfqpoint{0.078913in}{0.516667in}}{\pgfqpoint{0.074674in}{0.514911in}}{\pgfqpoint{0.071548in}{0.511785in}}%
\pgfpathcurveto{\pgfqpoint{0.068423in}{0.508660in}}{\pgfqpoint{0.066667in}{0.504420in}}{\pgfqpoint{0.066667in}{0.500000in}}%
\pgfpathcurveto{\pgfqpoint{0.066667in}{0.495580in}}{\pgfqpoint{0.068423in}{0.491340in}}{\pgfqpoint{0.071548in}{0.488215in}}%
\pgfpathcurveto{\pgfqpoint{0.074674in}{0.485089in}}{\pgfqpoint{0.078913in}{0.483333in}}{\pgfqpoint{0.083333in}{0.483333in}}%
\pgfpathclose%
\pgfpathmoveto{\pgfqpoint{0.250000in}{0.483333in}}%
\pgfpathcurveto{\pgfqpoint{0.254420in}{0.483333in}}{\pgfqpoint{0.258660in}{0.485089in}}{\pgfqpoint{0.261785in}{0.488215in}}%
\pgfpathcurveto{\pgfqpoint{0.264911in}{0.491340in}}{\pgfqpoint{0.266667in}{0.495580in}}{\pgfqpoint{0.266667in}{0.500000in}}%
\pgfpathcurveto{\pgfqpoint{0.266667in}{0.504420in}}{\pgfqpoint{0.264911in}{0.508660in}}{\pgfqpoint{0.261785in}{0.511785in}}%
\pgfpathcurveto{\pgfqpoint{0.258660in}{0.514911in}}{\pgfqpoint{0.254420in}{0.516667in}}{\pgfqpoint{0.250000in}{0.516667in}}%
\pgfpathcurveto{\pgfqpoint{0.245580in}{0.516667in}}{\pgfqpoint{0.241340in}{0.514911in}}{\pgfqpoint{0.238215in}{0.511785in}}%
\pgfpathcurveto{\pgfqpoint{0.235089in}{0.508660in}}{\pgfqpoint{0.233333in}{0.504420in}}{\pgfqpoint{0.233333in}{0.500000in}}%
\pgfpathcurveto{\pgfqpoint{0.233333in}{0.495580in}}{\pgfqpoint{0.235089in}{0.491340in}}{\pgfqpoint{0.238215in}{0.488215in}}%
\pgfpathcurveto{\pgfqpoint{0.241340in}{0.485089in}}{\pgfqpoint{0.245580in}{0.483333in}}{\pgfqpoint{0.250000in}{0.483333in}}%
\pgfpathclose%
\pgfpathmoveto{\pgfqpoint{0.416667in}{0.483333in}}%
\pgfpathcurveto{\pgfqpoint{0.421087in}{0.483333in}}{\pgfqpoint{0.425326in}{0.485089in}}{\pgfqpoint{0.428452in}{0.488215in}}%
\pgfpathcurveto{\pgfqpoint{0.431577in}{0.491340in}}{\pgfqpoint{0.433333in}{0.495580in}}{\pgfqpoint{0.433333in}{0.500000in}}%
\pgfpathcurveto{\pgfqpoint{0.433333in}{0.504420in}}{\pgfqpoint{0.431577in}{0.508660in}}{\pgfqpoint{0.428452in}{0.511785in}}%
\pgfpathcurveto{\pgfqpoint{0.425326in}{0.514911in}}{\pgfqpoint{0.421087in}{0.516667in}}{\pgfqpoint{0.416667in}{0.516667in}}%
\pgfpathcurveto{\pgfqpoint{0.412247in}{0.516667in}}{\pgfqpoint{0.408007in}{0.514911in}}{\pgfqpoint{0.404882in}{0.511785in}}%
\pgfpathcurveto{\pgfqpoint{0.401756in}{0.508660in}}{\pgfqpoint{0.400000in}{0.504420in}}{\pgfqpoint{0.400000in}{0.500000in}}%
\pgfpathcurveto{\pgfqpoint{0.400000in}{0.495580in}}{\pgfqpoint{0.401756in}{0.491340in}}{\pgfqpoint{0.404882in}{0.488215in}}%
\pgfpathcurveto{\pgfqpoint{0.408007in}{0.485089in}}{\pgfqpoint{0.412247in}{0.483333in}}{\pgfqpoint{0.416667in}{0.483333in}}%
\pgfpathclose%
\pgfpathmoveto{\pgfqpoint{0.583333in}{0.483333in}}%
\pgfpathcurveto{\pgfqpoint{0.587753in}{0.483333in}}{\pgfqpoint{0.591993in}{0.485089in}}{\pgfqpoint{0.595118in}{0.488215in}}%
\pgfpathcurveto{\pgfqpoint{0.598244in}{0.491340in}}{\pgfqpoint{0.600000in}{0.495580in}}{\pgfqpoint{0.600000in}{0.500000in}}%
\pgfpathcurveto{\pgfqpoint{0.600000in}{0.504420in}}{\pgfqpoint{0.598244in}{0.508660in}}{\pgfqpoint{0.595118in}{0.511785in}}%
\pgfpathcurveto{\pgfqpoint{0.591993in}{0.514911in}}{\pgfqpoint{0.587753in}{0.516667in}}{\pgfqpoint{0.583333in}{0.516667in}}%
\pgfpathcurveto{\pgfqpoint{0.578913in}{0.516667in}}{\pgfqpoint{0.574674in}{0.514911in}}{\pgfqpoint{0.571548in}{0.511785in}}%
\pgfpathcurveto{\pgfqpoint{0.568423in}{0.508660in}}{\pgfqpoint{0.566667in}{0.504420in}}{\pgfqpoint{0.566667in}{0.500000in}}%
\pgfpathcurveto{\pgfqpoint{0.566667in}{0.495580in}}{\pgfqpoint{0.568423in}{0.491340in}}{\pgfqpoint{0.571548in}{0.488215in}}%
\pgfpathcurveto{\pgfqpoint{0.574674in}{0.485089in}}{\pgfqpoint{0.578913in}{0.483333in}}{\pgfqpoint{0.583333in}{0.483333in}}%
\pgfpathclose%
\pgfpathmoveto{\pgfqpoint{0.750000in}{0.483333in}}%
\pgfpathcurveto{\pgfqpoint{0.754420in}{0.483333in}}{\pgfqpoint{0.758660in}{0.485089in}}{\pgfqpoint{0.761785in}{0.488215in}}%
\pgfpathcurveto{\pgfqpoint{0.764911in}{0.491340in}}{\pgfqpoint{0.766667in}{0.495580in}}{\pgfqpoint{0.766667in}{0.500000in}}%
\pgfpathcurveto{\pgfqpoint{0.766667in}{0.504420in}}{\pgfqpoint{0.764911in}{0.508660in}}{\pgfqpoint{0.761785in}{0.511785in}}%
\pgfpathcurveto{\pgfqpoint{0.758660in}{0.514911in}}{\pgfqpoint{0.754420in}{0.516667in}}{\pgfqpoint{0.750000in}{0.516667in}}%
\pgfpathcurveto{\pgfqpoint{0.745580in}{0.516667in}}{\pgfqpoint{0.741340in}{0.514911in}}{\pgfqpoint{0.738215in}{0.511785in}}%
\pgfpathcurveto{\pgfqpoint{0.735089in}{0.508660in}}{\pgfqpoint{0.733333in}{0.504420in}}{\pgfqpoint{0.733333in}{0.500000in}}%
\pgfpathcurveto{\pgfqpoint{0.733333in}{0.495580in}}{\pgfqpoint{0.735089in}{0.491340in}}{\pgfqpoint{0.738215in}{0.488215in}}%
\pgfpathcurveto{\pgfqpoint{0.741340in}{0.485089in}}{\pgfqpoint{0.745580in}{0.483333in}}{\pgfqpoint{0.750000in}{0.483333in}}%
\pgfpathclose%
\pgfpathmoveto{\pgfqpoint{0.916667in}{0.483333in}}%
\pgfpathcurveto{\pgfqpoint{0.921087in}{0.483333in}}{\pgfqpoint{0.925326in}{0.485089in}}{\pgfqpoint{0.928452in}{0.488215in}}%
\pgfpathcurveto{\pgfqpoint{0.931577in}{0.491340in}}{\pgfqpoint{0.933333in}{0.495580in}}{\pgfqpoint{0.933333in}{0.500000in}}%
\pgfpathcurveto{\pgfqpoint{0.933333in}{0.504420in}}{\pgfqpoint{0.931577in}{0.508660in}}{\pgfqpoint{0.928452in}{0.511785in}}%
\pgfpathcurveto{\pgfqpoint{0.925326in}{0.514911in}}{\pgfqpoint{0.921087in}{0.516667in}}{\pgfqpoint{0.916667in}{0.516667in}}%
\pgfpathcurveto{\pgfqpoint{0.912247in}{0.516667in}}{\pgfqpoint{0.908007in}{0.514911in}}{\pgfqpoint{0.904882in}{0.511785in}}%
\pgfpathcurveto{\pgfqpoint{0.901756in}{0.508660in}}{\pgfqpoint{0.900000in}{0.504420in}}{\pgfqpoint{0.900000in}{0.500000in}}%
\pgfpathcurveto{\pgfqpoint{0.900000in}{0.495580in}}{\pgfqpoint{0.901756in}{0.491340in}}{\pgfqpoint{0.904882in}{0.488215in}}%
\pgfpathcurveto{\pgfqpoint{0.908007in}{0.485089in}}{\pgfqpoint{0.912247in}{0.483333in}}{\pgfqpoint{0.916667in}{0.483333in}}%
\pgfpathclose%
\pgfpathmoveto{\pgfqpoint{0.000000in}{0.650000in}}%
\pgfpathcurveto{\pgfqpoint{0.004420in}{0.650000in}}{\pgfqpoint{0.008660in}{0.651756in}}{\pgfqpoint{0.011785in}{0.654882in}}%
\pgfpathcurveto{\pgfqpoint{0.014911in}{0.658007in}}{\pgfqpoint{0.016667in}{0.662247in}}{\pgfqpoint{0.016667in}{0.666667in}}%
\pgfpathcurveto{\pgfqpoint{0.016667in}{0.671087in}}{\pgfqpoint{0.014911in}{0.675326in}}{\pgfqpoint{0.011785in}{0.678452in}}%
\pgfpathcurveto{\pgfqpoint{0.008660in}{0.681577in}}{\pgfqpoint{0.004420in}{0.683333in}}{\pgfqpoint{0.000000in}{0.683333in}}%
\pgfpathcurveto{\pgfqpoint{-0.004420in}{0.683333in}}{\pgfqpoint{-0.008660in}{0.681577in}}{\pgfqpoint{-0.011785in}{0.678452in}}%
\pgfpathcurveto{\pgfqpoint{-0.014911in}{0.675326in}}{\pgfqpoint{-0.016667in}{0.671087in}}{\pgfqpoint{-0.016667in}{0.666667in}}%
\pgfpathcurveto{\pgfqpoint{-0.016667in}{0.662247in}}{\pgfqpoint{-0.014911in}{0.658007in}}{\pgfqpoint{-0.011785in}{0.654882in}}%
\pgfpathcurveto{\pgfqpoint{-0.008660in}{0.651756in}}{\pgfqpoint{-0.004420in}{0.650000in}}{\pgfqpoint{0.000000in}{0.650000in}}%
\pgfpathclose%
\pgfpathmoveto{\pgfqpoint{0.166667in}{0.650000in}}%
\pgfpathcurveto{\pgfqpoint{0.171087in}{0.650000in}}{\pgfqpoint{0.175326in}{0.651756in}}{\pgfqpoint{0.178452in}{0.654882in}}%
\pgfpathcurveto{\pgfqpoint{0.181577in}{0.658007in}}{\pgfqpoint{0.183333in}{0.662247in}}{\pgfqpoint{0.183333in}{0.666667in}}%
\pgfpathcurveto{\pgfqpoint{0.183333in}{0.671087in}}{\pgfqpoint{0.181577in}{0.675326in}}{\pgfqpoint{0.178452in}{0.678452in}}%
\pgfpathcurveto{\pgfqpoint{0.175326in}{0.681577in}}{\pgfqpoint{0.171087in}{0.683333in}}{\pgfqpoint{0.166667in}{0.683333in}}%
\pgfpathcurveto{\pgfqpoint{0.162247in}{0.683333in}}{\pgfqpoint{0.158007in}{0.681577in}}{\pgfqpoint{0.154882in}{0.678452in}}%
\pgfpathcurveto{\pgfqpoint{0.151756in}{0.675326in}}{\pgfqpoint{0.150000in}{0.671087in}}{\pgfqpoint{0.150000in}{0.666667in}}%
\pgfpathcurveto{\pgfqpoint{0.150000in}{0.662247in}}{\pgfqpoint{0.151756in}{0.658007in}}{\pgfqpoint{0.154882in}{0.654882in}}%
\pgfpathcurveto{\pgfqpoint{0.158007in}{0.651756in}}{\pgfqpoint{0.162247in}{0.650000in}}{\pgfqpoint{0.166667in}{0.650000in}}%
\pgfpathclose%
\pgfpathmoveto{\pgfqpoint{0.333333in}{0.650000in}}%
\pgfpathcurveto{\pgfqpoint{0.337753in}{0.650000in}}{\pgfqpoint{0.341993in}{0.651756in}}{\pgfqpoint{0.345118in}{0.654882in}}%
\pgfpathcurveto{\pgfqpoint{0.348244in}{0.658007in}}{\pgfqpoint{0.350000in}{0.662247in}}{\pgfqpoint{0.350000in}{0.666667in}}%
\pgfpathcurveto{\pgfqpoint{0.350000in}{0.671087in}}{\pgfqpoint{0.348244in}{0.675326in}}{\pgfqpoint{0.345118in}{0.678452in}}%
\pgfpathcurveto{\pgfqpoint{0.341993in}{0.681577in}}{\pgfqpoint{0.337753in}{0.683333in}}{\pgfqpoint{0.333333in}{0.683333in}}%
\pgfpathcurveto{\pgfqpoint{0.328913in}{0.683333in}}{\pgfqpoint{0.324674in}{0.681577in}}{\pgfqpoint{0.321548in}{0.678452in}}%
\pgfpathcurveto{\pgfqpoint{0.318423in}{0.675326in}}{\pgfqpoint{0.316667in}{0.671087in}}{\pgfqpoint{0.316667in}{0.666667in}}%
\pgfpathcurveto{\pgfqpoint{0.316667in}{0.662247in}}{\pgfqpoint{0.318423in}{0.658007in}}{\pgfqpoint{0.321548in}{0.654882in}}%
\pgfpathcurveto{\pgfqpoint{0.324674in}{0.651756in}}{\pgfqpoint{0.328913in}{0.650000in}}{\pgfqpoint{0.333333in}{0.650000in}}%
\pgfpathclose%
\pgfpathmoveto{\pgfqpoint{0.500000in}{0.650000in}}%
\pgfpathcurveto{\pgfqpoint{0.504420in}{0.650000in}}{\pgfqpoint{0.508660in}{0.651756in}}{\pgfqpoint{0.511785in}{0.654882in}}%
\pgfpathcurveto{\pgfqpoint{0.514911in}{0.658007in}}{\pgfqpoint{0.516667in}{0.662247in}}{\pgfqpoint{0.516667in}{0.666667in}}%
\pgfpathcurveto{\pgfqpoint{0.516667in}{0.671087in}}{\pgfqpoint{0.514911in}{0.675326in}}{\pgfqpoint{0.511785in}{0.678452in}}%
\pgfpathcurveto{\pgfqpoint{0.508660in}{0.681577in}}{\pgfqpoint{0.504420in}{0.683333in}}{\pgfqpoint{0.500000in}{0.683333in}}%
\pgfpathcurveto{\pgfqpoint{0.495580in}{0.683333in}}{\pgfqpoint{0.491340in}{0.681577in}}{\pgfqpoint{0.488215in}{0.678452in}}%
\pgfpathcurveto{\pgfqpoint{0.485089in}{0.675326in}}{\pgfqpoint{0.483333in}{0.671087in}}{\pgfqpoint{0.483333in}{0.666667in}}%
\pgfpathcurveto{\pgfqpoint{0.483333in}{0.662247in}}{\pgfqpoint{0.485089in}{0.658007in}}{\pgfqpoint{0.488215in}{0.654882in}}%
\pgfpathcurveto{\pgfqpoint{0.491340in}{0.651756in}}{\pgfqpoint{0.495580in}{0.650000in}}{\pgfqpoint{0.500000in}{0.650000in}}%
\pgfpathclose%
\pgfpathmoveto{\pgfqpoint{0.666667in}{0.650000in}}%
\pgfpathcurveto{\pgfqpoint{0.671087in}{0.650000in}}{\pgfqpoint{0.675326in}{0.651756in}}{\pgfqpoint{0.678452in}{0.654882in}}%
\pgfpathcurveto{\pgfqpoint{0.681577in}{0.658007in}}{\pgfqpoint{0.683333in}{0.662247in}}{\pgfqpoint{0.683333in}{0.666667in}}%
\pgfpathcurveto{\pgfqpoint{0.683333in}{0.671087in}}{\pgfqpoint{0.681577in}{0.675326in}}{\pgfqpoint{0.678452in}{0.678452in}}%
\pgfpathcurveto{\pgfqpoint{0.675326in}{0.681577in}}{\pgfqpoint{0.671087in}{0.683333in}}{\pgfqpoint{0.666667in}{0.683333in}}%
\pgfpathcurveto{\pgfqpoint{0.662247in}{0.683333in}}{\pgfqpoint{0.658007in}{0.681577in}}{\pgfqpoint{0.654882in}{0.678452in}}%
\pgfpathcurveto{\pgfqpoint{0.651756in}{0.675326in}}{\pgfqpoint{0.650000in}{0.671087in}}{\pgfqpoint{0.650000in}{0.666667in}}%
\pgfpathcurveto{\pgfqpoint{0.650000in}{0.662247in}}{\pgfqpoint{0.651756in}{0.658007in}}{\pgfqpoint{0.654882in}{0.654882in}}%
\pgfpathcurveto{\pgfqpoint{0.658007in}{0.651756in}}{\pgfqpoint{0.662247in}{0.650000in}}{\pgfqpoint{0.666667in}{0.650000in}}%
\pgfpathclose%
\pgfpathmoveto{\pgfqpoint{0.833333in}{0.650000in}}%
\pgfpathcurveto{\pgfqpoint{0.837753in}{0.650000in}}{\pgfqpoint{0.841993in}{0.651756in}}{\pgfqpoint{0.845118in}{0.654882in}}%
\pgfpathcurveto{\pgfqpoint{0.848244in}{0.658007in}}{\pgfqpoint{0.850000in}{0.662247in}}{\pgfqpoint{0.850000in}{0.666667in}}%
\pgfpathcurveto{\pgfqpoint{0.850000in}{0.671087in}}{\pgfqpoint{0.848244in}{0.675326in}}{\pgfqpoint{0.845118in}{0.678452in}}%
\pgfpathcurveto{\pgfqpoint{0.841993in}{0.681577in}}{\pgfqpoint{0.837753in}{0.683333in}}{\pgfqpoint{0.833333in}{0.683333in}}%
\pgfpathcurveto{\pgfqpoint{0.828913in}{0.683333in}}{\pgfqpoint{0.824674in}{0.681577in}}{\pgfqpoint{0.821548in}{0.678452in}}%
\pgfpathcurveto{\pgfqpoint{0.818423in}{0.675326in}}{\pgfqpoint{0.816667in}{0.671087in}}{\pgfqpoint{0.816667in}{0.666667in}}%
\pgfpathcurveto{\pgfqpoint{0.816667in}{0.662247in}}{\pgfqpoint{0.818423in}{0.658007in}}{\pgfqpoint{0.821548in}{0.654882in}}%
\pgfpathcurveto{\pgfqpoint{0.824674in}{0.651756in}}{\pgfqpoint{0.828913in}{0.650000in}}{\pgfqpoint{0.833333in}{0.650000in}}%
\pgfpathclose%
\pgfpathmoveto{\pgfqpoint{1.000000in}{0.650000in}}%
\pgfpathcurveto{\pgfqpoint{1.004420in}{0.650000in}}{\pgfqpoint{1.008660in}{0.651756in}}{\pgfqpoint{1.011785in}{0.654882in}}%
\pgfpathcurveto{\pgfqpoint{1.014911in}{0.658007in}}{\pgfqpoint{1.016667in}{0.662247in}}{\pgfqpoint{1.016667in}{0.666667in}}%
\pgfpathcurveto{\pgfqpoint{1.016667in}{0.671087in}}{\pgfqpoint{1.014911in}{0.675326in}}{\pgfqpoint{1.011785in}{0.678452in}}%
\pgfpathcurveto{\pgfqpoint{1.008660in}{0.681577in}}{\pgfqpoint{1.004420in}{0.683333in}}{\pgfqpoint{1.000000in}{0.683333in}}%
\pgfpathcurveto{\pgfqpoint{0.995580in}{0.683333in}}{\pgfqpoint{0.991340in}{0.681577in}}{\pgfqpoint{0.988215in}{0.678452in}}%
\pgfpathcurveto{\pgfqpoint{0.985089in}{0.675326in}}{\pgfqpoint{0.983333in}{0.671087in}}{\pgfqpoint{0.983333in}{0.666667in}}%
\pgfpathcurveto{\pgfqpoint{0.983333in}{0.662247in}}{\pgfqpoint{0.985089in}{0.658007in}}{\pgfqpoint{0.988215in}{0.654882in}}%
\pgfpathcurveto{\pgfqpoint{0.991340in}{0.651756in}}{\pgfqpoint{0.995580in}{0.650000in}}{\pgfqpoint{1.000000in}{0.650000in}}%
\pgfpathclose%
\pgfpathmoveto{\pgfqpoint{0.083333in}{0.816667in}}%
\pgfpathcurveto{\pgfqpoint{0.087753in}{0.816667in}}{\pgfqpoint{0.091993in}{0.818423in}}{\pgfqpoint{0.095118in}{0.821548in}}%
\pgfpathcurveto{\pgfqpoint{0.098244in}{0.824674in}}{\pgfqpoint{0.100000in}{0.828913in}}{\pgfqpoint{0.100000in}{0.833333in}}%
\pgfpathcurveto{\pgfqpoint{0.100000in}{0.837753in}}{\pgfqpoint{0.098244in}{0.841993in}}{\pgfqpoint{0.095118in}{0.845118in}}%
\pgfpathcurveto{\pgfqpoint{0.091993in}{0.848244in}}{\pgfqpoint{0.087753in}{0.850000in}}{\pgfqpoint{0.083333in}{0.850000in}}%
\pgfpathcurveto{\pgfqpoint{0.078913in}{0.850000in}}{\pgfqpoint{0.074674in}{0.848244in}}{\pgfqpoint{0.071548in}{0.845118in}}%
\pgfpathcurveto{\pgfqpoint{0.068423in}{0.841993in}}{\pgfqpoint{0.066667in}{0.837753in}}{\pgfqpoint{0.066667in}{0.833333in}}%
\pgfpathcurveto{\pgfqpoint{0.066667in}{0.828913in}}{\pgfqpoint{0.068423in}{0.824674in}}{\pgfqpoint{0.071548in}{0.821548in}}%
\pgfpathcurveto{\pgfqpoint{0.074674in}{0.818423in}}{\pgfqpoint{0.078913in}{0.816667in}}{\pgfqpoint{0.083333in}{0.816667in}}%
\pgfpathclose%
\pgfpathmoveto{\pgfqpoint{0.250000in}{0.816667in}}%
\pgfpathcurveto{\pgfqpoint{0.254420in}{0.816667in}}{\pgfqpoint{0.258660in}{0.818423in}}{\pgfqpoint{0.261785in}{0.821548in}}%
\pgfpathcurveto{\pgfqpoint{0.264911in}{0.824674in}}{\pgfqpoint{0.266667in}{0.828913in}}{\pgfqpoint{0.266667in}{0.833333in}}%
\pgfpathcurveto{\pgfqpoint{0.266667in}{0.837753in}}{\pgfqpoint{0.264911in}{0.841993in}}{\pgfqpoint{0.261785in}{0.845118in}}%
\pgfpathcurveto{\pgfqpoint{0.258660in}{0.848244in}}{\pgfqpoint{0.254420in}{0.850000in}}{\pgfqpoint{0.250000in}{0.850000in}}%
\pgfpathcurveto{\pgfqpoint{0.245580in}{0.850000in}}{\pgfqpoint{0.241340in}{0.848244in}}{\pgfqpoint{0.238215in}{0.845118in}}%
\pgfpathcurveto{\pgfqpoint{0.235089in}{0.841993in}}{\pgfqpoint{0.233333in}{0.837753in}}{\pgfqpoint{0.233333in}{0.833333in}}%
\pgfpathcurveto{\pgfqpoint{0.233333in}{0.828913in}}{\pgfqpoint{0.235089in}{0.824674in}}{\pgfqpoint{0.238215in}{0.821548in}}%
\pgfpathcurveto{\pgfqpoint{0.241340in}{0.818423in}}{\pgfqpoint{0.245580in}{0.816667in}}{\pgfqpoint{0.250000in}{0.816667in}}%
\pgfpathclose%
\pgfpathmoveto{\pgfqpoint{0.416667in}{0.816667in}}%
\pgfpathcurveto{\pgfqpoint{0.421087in}{0.816667in}}{\pgfqpoint{0.425326in}{0.818423in}}{\pgfqpoint{0.428452in}{0.821548in}}%
\pgfpathcurveto{\pgfqpoint{0.431577in}{0.824674in}}{\pgfqpoint{0.433333in}{0.828913in}}{\pgfqpoint{0.433333in}{0.833333in}}%
\pgfpathcurveto{\pgfqpoint{0.433333in}{0.837753in}}{\pgfqpoint{0.431577in}{0.841993in}}{\pgfqpoint{0.428452in}{0.845118in}}%
\pgfpathcurveto{\pgfqpoint{0.425326in}{0.848244in}}{\pgfqpoint{0.421087in}{0.850000in}}{\pgfqpoint{0.416667in}{0.850000in}}%
\pgfpathcurveto{\pgfqpoint{0.412247in}{0.850000in}}{\pgfqpoint{0.408007in}{0.848244in}}{\pgfqpoint{0.404882in}{0.845118in}}%
\pgfpathcurveto{\pgfqpoint{0.401756in}{0.841993in}}{\pgfqpoint{0.400000in}{0.837753in}}{\pgfqpoint{0.400000in}{0.833333in}}%
\pgfpathcurveto{\pgfqpoint{0.400000in}{0.828913in}}{\pgfqpoint{0.401756in}{0.824674in}}{\pgfqpoint{0.404882in}{0.821548in}}%
\pgfpathcurveto{\pgfqpoint{0.408007in}{0.818423in}}{\pgfqpoint{0.412247in}{0.816667in}}{\pgfqpoint{0.416667in}{0.816667in}}%
\pgfpathclose%
\pgfpathmoveto{\pgfqpoint{0.583333in}{0.816667in}}%
\pgfpathcurveto{\pgfqpoint{0.587753in}{0.816667in}}{\pgfqpoint{0.591993in}{0.818423in}}{\pgfqpoint{0.595118in}{0.821548in}}%
\pgfpathcurveto{\pgfqpoint{0.598244in}{0.824674in}}{\pgfqpoint{0.600000in}{0.828913in}}{\pgfqpoint{0.600000in}{0.833333in}}%
\pgfpathcurveto{\pgfqpoint{0.600000in}{0.837753in}}{\pgfqpoint{0.598244in}{0.841993in}}{\pgfqpoint{0.595118in}{0.845118in}}%
\pgfpathcurveto{\pgfqpoint{0.591993in}{0.848244in}}{\pgfqpoint{0.587753in}{0.850000in}}{\pgfqpoint{0.583333in}{0.850000in}}%
\pgfpathcurveto{\pgfqpoint{0.578913in}{0.850000in}}{\pgfqpoint{0.574674in}{0.848244in}}{\pgfqpoint{0.571548in}{0.845118in}}%
\pgfpathcurveto{\pgfqpoint{0.568423in}{0.841993in}}{\pgfqpoint{0.566667in}{0.837753in}}{\pgfqpoint{0.566667in}{0.833333in}}%
\pgfpathcurveto{\pgfqpoint{0.566667in}{0.828913in}}{\pgfqpoint{0.568423in}{0.824674in}}{\pgfqpoint{0.571548in}{0.821548in}}%
\pgfpathcurveto{\pgfqpoint{0.574674in}{0.818423in}}{\pgfqpoint{0.578913in}{0.816667in}}{\pgfqpoint{0.583333in}{0.816667in}}%
\pgfpathclose%
\pgfpathmoveto{\pgfqpoint{0.750000in}{0.816667in}}%
\pgfpathcurveto{\pgfqpoint{0.754420in}{0.816667in}}{\pgfqpoint{0.758660in}{0.818423in}}{\pgfqpoint{0.761785in}{0.821548in}}%
\pgfpathcurveto{\pgfqpoint{0.764911in}{0.824674in}}{\pgfqpoint{0.766667in}{0.828913in}}{\pgfqpoint{0.766667in}{0.833333in}}%
\pgfpathcurveto{\pgfqpoint{0.766667in}{0.837753in}}{\pgfqpoint{0.764911in}{0.841993in}}{\pgfqpoint{0.761785in}{0.845118in}}%
\pgfpathcurveto{\pgfqpoint{0.758660in}{0.848244in}}{\pgfqpoint{0.754420in}{0.850000in}}{\pgfqpoint{0.750000in}{0.850000in}}%
\pgfpathcurveto{\pgfqpoint{0.745580in}{0.850000in}}{\pgfqpoint{0.741340in}{0.848244in}}{\pgfqpoint{0.738215in}{0.845118in}}%
\pgfpathcurveto{\pgfqpoint{0.735089in}{0.841993in}}{\pgfqpoint{0.733333in}{0.837753in}}{\pgfqpoint{0.733333in}{0.833333in}}%
\pgfpathcurveto{\pgfqpoint{0.733333in}{0.828913in}}{\pgfqpoint{0.735089in}{0.824674in}}{\pgfqpoint{0.738215in}{0.821548in}}%
\pgfpathcurveto{\pgfqpoint{0.741340in}{0.818423in}}{\pgfqpoint{0.745580in}{0.816667in}}{\pgfqpoint{0.750000in}{0.816667in}}%
\pgfpathclose%
\pgfpathmoveto{\pgfqpoint{0.916667in}{0.816667in}}%
\pgfpathcurveto{\pgfqpoint{0.921087in}{0.816667in}}{\pgfqpoint{0.925326in}{0.818423in}}{\pgfqpoint{0.928452in}{0.821548in}}%
\pgfpathcurveto{\pgfqpoint{0.931577in}{0.824674in}}{\pgfqpoint{0.933333in}{0.828913in}}{\pgfqpoint{0.933333in}{0.833333in}}%
\pgfpathcurveto{\pgfqpoint{0.933333in}{0.837753in}}{\pgfqpoint{0.931577in}{0.841993in}}{\pgfqpoint{0.928452in}{0.845118in}}%
\pgfpathcurveto{\pgfqpoint{0.925326in}{0.848244in}}{\pgfqpoint{0.921087in}{0.850000in}}{\pgfqpoint{0.916667in}{0.850000in}}%
\pgfpathcurveto{\pgfqpoint{0.912247in}{0.850000in}}{\pgfqpoint{0.908007in}{0.848244in}}{\pgfqpoint{0.904882in}{0.845118in}}%
\pgfpathcurveto{\pgfqpoint{0.901756in}{0.841993in}}{\pgfqpoint{0.900000in}{0.837753in}}{\pgfqpoint{0.900000in}{0.833333in}}%
\pgfpathcurveto{\pgfqpoint{0.900000in}{0.828913in}}{\pgfqpoint{0.901756in}{0.824674in}}{\pgfqpoint{0.904882in}{0.821548in}}%
\pgfpathcurveto{\pgfqpoint{0.908007in}{0.818423in}}{\pgfqpoint{0.912247in}{0.816667in}}{\pgfqpoint{0.916667in}{0.816667in}}%
\pgfpathclose%
\pgfpathmoveto{\pgfqpoint{0.000000in}{0.983333in}}%
\pgfpathcurveto{\pgfqpoint{0.004420in}{0.983333in}}{\pgfqpoint{0.008660in}{0.985089in}}{\pgfqpoint{0.011785in}{0.988215in}}%
\pgfpathcurveto{\pgfqpoint{0.014911in}{0.991340in}}{\pgfqpoint{0.016667in}{0.995580in}}{\pgfqpoint{0.016667in}{1.000000in}}%
\pgfpathcurveto{\pgfqpoint{0.016667in}{1.004420in}}{\pgfqpoint{0.014911in}{1.008660in}}{\pgfqpoint{0.011785in}{1.011785in}}%
\pgfpathcurveto{\pgfqpoint{0.008660in}{1.014911in}}{\pgfqpoint{0.004420in}{1.016667in}}{\pgfqpoint{0.000000in}{1.016667in}}%
\pgfpathcurveto{\pgfqpoint{-0.004420in}{1.016667in}}{\pgfqpoint{-0.008660in}{1.014911in}}{\pgfqpoint{-0.011785in}{1.011785in}}%
\pgfpathcurveto{\pgfqpoint{-0.014911in}{1.008660in}}{\pgfqpoint{-0.016667in}{1.004420in}}{\pgfqpoint{-0.016667in}{1.000000in}}%
\pgfpathcurveto{\pgfqpoint{-0.016667in}{0.995580in}}{\pgfqpoint{-0.014911in}{0.991340in}}{\pgfqpoint{-0.011785in}{0.988215in}}%
\pgfpathcurveto{\pgfqpoint{-0.008660in}{0.985089in}}{\pgfqpoint{-0.004420in}{0.983333in}}{\pgfqpoint{0.000000in}{0.983333in}}%
\pgfpathclose%
\pgfpathmoveto{\pgfqpoint{0.166667in}{0.983333in}}%
\pgfpathcurveto{\pgfqpoint{0.171087in}{0.983333in}}{\pgfqpoint{0.175326in}{0.985089in}}{\pgfqpoint{0.178452in}{0.988215in}}%
\pgfpathcurveto{\pgfqpoint{0.181577in}{0.991340in}}{\pgfqpoint{0.183333in}{0.995580in}}{\pgfqpoint{0.183333in}{1.000000in}}%
\pgfpathcurveto{\pgfqpoint{0.183333in}{1.004420in}}{\pgfqpoint{0.181577in}{1.008660in}}{\pgfqpoint{0.178452in}{1.011785in}}%
\pgfpathcurveto{\pgfqpoint{0.175326in}{1.014911in}}{\pgfqpoint{0.171087in}{1.016667in}}{\pgfqpoint{0.166667in}{1.016667in}}%
\pgfpathcurveto{\pgfqpoint{0.162247in}{1.016667in}}{\pgfqpoint{0.158007in}{1.014911in}}{\pgfqpoint{0.154882in}{1.011785in}}%
\pgfpathcurveto{\pgfqpoint{0.151756in}{1.008660in}}{\pgfqpoint{0.150000in}{1.004420in}}{\pgfqpoint{0.150000in}{1.000000in}}%
\pgfpathcurveto{\pgfqpoint{0.150000in}{0.995580in}}{\pgfqpoint{0.151756in}{0.991340in}}{\pgfqpoint{0.154882in}{0.988215in}}%
\pgfpathcurveto{\pgfqpoint{0.158007in}{0.985089in}}{\pgfqpoint{0.162247in}{0.983333in}}{\pgfqpoint{0.166667in}{0.983333in}}%
\pgfpathclose%
\pgfpathmoveto{\pgfqpoint{0.333333in}{0.983333in}}%
\pgfpathcurveto{\pgfqpoint{0.337753in}{0.983333in}}{\pgfqpoint{0.341993in}{0.985089in}}{\pgfqpoint{0.345118in}{0.988215in}}%
\pgfpathcurveto{\pgfqpoint{0.348244in}{0.991340in}}{\pgfqpoint{0.350000in}{0.995580in}}{\pgfqpoint{0.350000in}{1.000000in}}%
\pgfpathcurveto{\pgfqpoint{0.350000in}{1.004420in}}{\pgfqpoint{0.348244in}{1.008660in}}{\pgfqpoint{0.345118in}{1.011785in}}%
\pgfpathcurveto{\pgfqpoint{0.341993in}{1.014911in}}{\pgfqpoint{0.337753in}{1.016667in}}{\pgfqpoint{0.333333in}{1.016667in}}%
\pgfpathcurveto{\pgfqpoint{0.328913in}{1.016667in}}{\pgfqpoint{0.324674in}{1.014911in}}{\pgfqpoint{0.321548in}{1.011785in}}%
\pgfpathcurveto{\pgfqpoint{0.318423in}{1.008660in}}{\pgfqpoint{0.316667in}{1.004420in}}{\pgfqpoint{0.316667in}{1.000000in}}%
\pgfpathcurveto{\pgfqpoint{0.316667in}{0.995580in}}{\pgfqpoint{0.318423in}{0.991340in}}{\pgfqpoint{0.321548in}{0.988215in}}%
\pgfpathcurveto{\pgfqpoint{0.324674in}{0.985089in}}{\pgfqpoint{0.328913in}{0.983333in}}{\pgfqpoint{0.333333in}{0.983333in}}%
\pgfpathclose%
\pgfpathmoveto{\pgfqpoint{0.500000in}{0.983333in}}%
\pgfpathcurveto{\pgfqpoint{0.504420in}{0.983333in}}{\pgfqpoint{0.508660in}{0.985089in}}{\pgfqpoint{0.511785in}{0.988215in}}%
\pgfpathcurveto{\pgfqpoint{0.514911in}{0.991340in}}{\pgfqpoint{0.516667in}{0.995580in}}{\pgfqpoint{0.516667in}{1.000000in}}%
\pgfpathcurveto{\pgfqpoint{0.516667in}{1.004420in}}{\pgfqpoint{0.514911in}{1.008660in}}{\pgfqpoint{0.511785in}{1.011785in}}%
\pgfpathcurveto{\pgfqpoint{0.508660in}{1.014911in}}{\pgfqpoint{0.504420in}{1.016667in}}{\pgfqpoint{0.500000in}{1.016667in}}%
\pgfpathcurveto{\pgfqpoint{0.495580in}{1.016667in}}{\pgfqpoint{0.491340in}{1.014911in}}{\pgfqpoint{0.488215in}{1.011785in}}%
\pgfpathcurveto{\pgfqpoint{0.485089in}{1.008660in}}{\pgfqpoint{0.483333in}{1.004420in}}{\pgfqpoint{0.483333in}{1.000000in}}%
\pgfpathcurveto{\pgfqpoint{0.483333in}{0.995580in}}{\pgfqpoint{0.485089in}{0.991340in}}{\pgfqpoint{0.488215in}{0.988215in}}%
\pgfpathcurveto{\pgfqpoint{0.491340in}{0.985089in}}{\pgfqpoint{0.495580in}{0.983333in}}{\pgfqpoint{0.500000in}{0.983333in}}%
\pgfpathclose%
\pgfpathmoveto{\pgfqpoint{0.666667in}{0.983333in}}%
\pgfpathcurveto{\pgfqpoint{0.671087in}{0.983333in}}{\pgfqpoint{0.675326in}{0.985089in}}{\pgfqpoint{0.678452in}{0.988215in}}%
\pgfpathcurveto{\pgfqpoint{0.681577in}{0.991340in}}{\pgfqpoint{0.683333in}{0.995580in}}{\pgfqpoint{0.683333in}{1.000000in}}%
\pgfpathcurveto{\pgfqpoint{0.683333in}{1.004420in}}{\pgfqpoint{0.681577in}{1.008660in}}{\pgfqpoint{0.678452in}{1.011785in}}%
\pgfpathcurveto{\pgfqpoint{0.675326in}{1.014911in}}{\pgfqpoint{0.671087in}{1.016667in}}{\pgfqpoint{0.666667in}{1.016667in}}%
\pgfpathcurveto{\pgfqpoint{0.662247in}{1.016667in}}{\pgfqpoint{0.658007in}{1.014911in}}{\pgfqpoint{0.654882in}{1.011785in}}%
\pgfpathcurveto{\pgfqpoint{0.651756in}{1.008660in}}{\pgfqpoint{0.650000in}{1.004420in}}{\pgfqpoint{0.650000in}{1.000000in}}%
\pgfpathcurveto{\pgfqpoint{0.650000in}{0.995580in}}{\pgfqpoint{0.651756in}{0.991340in}}{\pgfqpoint{0.654882in}{0.988215in}}%
\pgfpathcurveto{\pgfqpoint{0.658007in}{0.985089in}}{\pgfqpoint{0.662247in}{0.983333in}}{\pgfqpoint{0.666667in}{0.983333in}}%
\pgfpathclose%
\pgfpathmoveto{\pgfqpoint{0.833333in}{0.983333in}}%
\pgfpathcurveto{\pgfqpoint{0.837753in}{0.983333in}}{\pgfqpoint{0.841993in}{0.985089in}}{\pgfqpoint{0.845118in}{0.988215in}}%
\pgfpathcurveto{\pgfqpoint{0.848244in}{0.991340in}}{\pgfqpoint{0.850000in}{0.995580in}}{\pgfqpoint{0.850000in}{1.000000in}}%
\pgfpathcurveto{\pgfqpoint{0.850000in}{1.004420in}}{\pgfqpoint{0.848244in}{1.008660in}}{\pgfqpoint{0.845118in}{1.011785in}}%
\pgfpathcurveto{\pgfqpoint{0.841993in}{1.014911in}}{\pgfqpoint{0.837753in}{1.016667in}}{\pgfqpoint{0.833333in}{1.016667in}}%
\pgfpathcurveto{\pgfqpoint{0.828913in}{1.016667in}}{\pgfqpoint{0.824674in}{1.014911in}}{\pgfqpoint{0.821548in}{1.011785in}}%
\pgfpathcurveto{\pgfqpoint{0.818423in}{1.008660in}}{\pgfqpoint{0.816667in}{1.004420in}}{\pgfqpoint{0.816667in}{1.000000in}}%
\pgfpathcurveto{\pgfqpoint{0.816667in}{0.995580in}}{\pgfqpoint{0.818423in}{0.991340in}}{\pgfqpoint{0.821548in}{0.988215in}}%
\pgfpathcurveto{\pgfqpoint{0.824674in}{0.985089in}}{\pgfqpoint{0.828913in}{0.983333in}}{\pgfqpoint{0.833333in}{0.983333in}}%
\pgfpathclose%
\pgfpathmoveto{\pgfqpoint{1.000000in}{0.983333in}}%
\pgfpathcurveto{\pgfqpoint{1.004420in}{0.983333in}}{\pgfqpoint{1.008660in}{0.985089in}}{\pgfqpoint{1.011785in}{0.988215in}}%
\pgfpathcurveto{\pgfqpoint{1.014911in}{0.991340in}}{\pgfqpoint{1.016667in}{0.995580in}}{\pgfqpoint{1.016667in}{1.000000in}}%
\pgfpathcurveto{\pgfqpoint{1.016667in}{1.004420in}}{\pgfqpoint{1.014911in}{1.008660in}}{\pgfqpoint{1.011785in}{1.011785in}}%
\pgfpathcurveto{\pgfqpoint{1.008660in}{1.014911in}}{\pgfqpoint{1.004420in}{1.016667in}}{\pgfqpoint{1.000000in}{1.016667in}}%
\pgfpathcurveto{\pgfqpoint{0.995580in}{1.016667in}}{\pgfqpoint{0.991340in}{1.014911in}}{\pgfqpoint{0.988215in}{1.011785in}}%
\pgfpathcurveto{\pgfqpoint{0.985089in}{1.008660in}}{\pgfqpoint{0.983333in}{1.004420in}}{\pgfqpoint{0.983333in}{1.000000in}}%
\pgfpathcurveto{\pgfqpoint{0.983333in}{0.995580in}}{\pgfqpoint{0.985089in}{0.991340in}}{\pgfqpoint{0.988215in}{0.988215in}}%
\pgfpathcurveto{\pgfqpoint{0.991340in}{0.985089in}}{\pgfqpoint{0.995580in}{0.983333in}}{\pgfqpoint{1.000000in}{0.983333in}}%
\pgfpathclose%
\pgfusepath{stroke}%
\end{pgfscope}%
}%
\pgfsys@transformshift{4.423315in}{4.380181in}%
\pgfsys@useobject{currentpattern}{}%
\pgfsys@transformshift{1in}{0in}%
\pgfsys@transformshift{-1in}{0in}%
\pgfsys@transformshift{0in}{1in}%
\end{pgfscope}%
\begin{pgfscope}%
\pgfpathrectangle{\pgfqpoint{0.935815in}{0.637495in}}{\pgfqpoint{9.300000in}{9.060000in}}%
\pgfusepath{clip}%
\pgfsetbuttcap%
\pgfsetmiterjoin%
\definecolor{currentfill}{rgb}{0.172549,0.627451,0.172549}%
\pgfsetfillcolor{currentfill}%
\pgfsetfillopacity{0.990000}%
\pgfsetlinewidth{0.000000pt}%
\definecolor{currentstroke}{rgb}{0.000000,0.000000,0.000000}%
\pgfsetstrokecolor{currentstroke}%
\pgfsetstrokeopacity{0.990000}%
\pgfsetdash{}{0pt}%
\pgfpathmoveto{\pgfqpoint{5.973315in}{4.921635in}}%
\pgfpathlineto{\pgfqpoint{6.748315in}{4.921635in}}%
\pgfpathlineto{\pgfqpoint{6.748315in}{5.330636in}}%
\pgfpathlineto{\pgfqpoint{5.973315in}{5.330636in}}%
\pgfpathclose%
\pgfusepath{fill}%
\end{pgfscope}%
\begin{pgfscope}%
\pgfsetbuttcap%
\pgfsetmiterjoin%
\definecolor{currentfill}{rgb}{0.172549,0.627451,0.172549}%
\pgfsetfillcolor{currentfill}%
\pgfsetfillopacity{0.990000}%
\pgfsetlinewidth{0.000000pt}%
\definecolor{currentstroke}{rgb}{0.000000,0.000000,0.000000}%
\pgfsetstrokecolor{currentstroke}%
\pgfsetstrokeopacity{0.990000}%
\pgfsetdash{}{0pt}%
\pgfpathrectangle{\pgfqpoint{0.935815in}{0.637495in}}{\pgfqpoint{9.300000in}{9.060000in}}%
\pgfusepath{clip}%
\pgfpathmoveto{\pgfqpoint{5.973315in}{4.921635in}}%
\pgfpathlineto{\pgfqpoint{6.748315in}{4.921635in}}%
\pgfpathlineto{\pgfqpoint{6.748315in}{5.330636in}}%
\pgfpathlineto{\pgfqpoint{5.973315in}{5.330636in}}%
\pgfpathclose%
\pgfusepath{clip}%
\pgfsys@defobject{currentpattern}{\pgfqpoint{0in}{0in}}{\pgfqpoint{1in}{1in}}{%
\begin{pgfscope}%
\pgfpathrectangle{\pgfqpoint{0in}{0in}}{\pgfqpoint{1in}{1in}}%
\pgfusepath{clip}%
\pgfpathmoveto{\pgfqpoint{0.000000in}{-0.016667in}}%
\pgfpathcurveto{\pgfqpoint{0.004420in}{-0.016667in}}{\pgfqpoint{0.008660in}{-0.014911in}}{\pgfqpoint{0.011785in}{-0.011785in}}%
\pgfpathcurveto{\pgfqpoint{0.014911in}{-0.008660in}}{\pgfqpoint{0.016667in}{-0.004420in}}{\pgfqpoint{0.016667in}{0.000000in}}%
\pgfpathcurveto{\pgfqpoint{0.016667in}{0.004420in}}{\pgfqpoint{0.014911in}{0.008660in}}{\pgfqpoint{0.011785in}{0.011785in}}%
\pgfpathcurveto{\pgfqpoint{0.008660in}{0.014911in}}{\pgfqpoint{0.004420in}{0.016667in}}{\pgfqpoint{0.000000in}{0.016667in}}%
\pgfpathcurveto{\pgfqpoint{-0.004420in}{0.016667in}}{\pgfqpoint{-0.008660in}{0.014911in}}{\pgfqpoint{-0.011785in}{0.011785in}}%
\pgfpathcurveto{\pgfqpoint{-0.014911in}{0.008660in}}{\pgfqpoint{-0.016667in}{0.004420in}}{\pgfqpoint{-0.016667in}{0.000000in}}%
\pgfpathcurveto{\pgfqpoint{-0.016667in}{-0.004420in}}{\pgfqpoint{-0.014911in}{-0.008660in}}{\pgfqpoint{-0.011785in}{-0.011785in}}%
\pgfpathcurveto{\pgfqpoint{-0.008660in}{-0.014911in}}{\pgfqpoint{-0.004420in}{-0.016667in}}{\pgfqpoint{0.000000in}{-0.016667in}}%
\pgfpathclose%
\pgfpathmoveto{\pgfqpoint{0.166667in}{-0.016667in}}%
\pgfpathcurveto{\pgfqpoint{0.171087in}{-0.016667in}}{\pgfqpoint{0.175326in}{-0.014911in}}{\pgfqpoint{0.178452in}{-0.011785in}}%
\pgfpathcurveto{\pgfqpoint{0.181577in}{-0.008660in}}{\pgfqpoint{0.183333in}{-0.004420in}}{\pgfqpoint{0.183333in}{0.000000in}}%
\pgfpathcurveto{\pgfqpoint{0.183333in}{0.004420in}}{\pgfqpoint{0.181577in}{0.008660in}}{\pgfqpoint{0.178452in}{0.011785in}}%
\pgfpathcurveto{\pgfqpoint{0.175326in}{0.014911in}}{\pgfqpoint{0.171087in}{0.016667in}}{\pgfqpoint{0.166667in}{0.016667in}}%
\pgfpathcurveto{\pgfqpoint{0.162247in}{0.016667in}}{\pgfqpoint{0.158007in}{0.014911in}}{\pgfqpoint{0.154882in}{0.011785in}}%
\pgfpathcurveto{\pgfqpoint{0.151756in}{0.008660in}}{\pgfqpoint{0.150000in}{0.004420in}}{\pgfqpoint{0.150000in}{0.000000in}}%
\pgfpathcurveto{\pgfqpoint{0.150000in}{-0.004420in}}{\pgfqpoint{0.151756in}{-0.008660in}}{\pgfqpoint{0.154882in}{-0.011785in}}%
\pgfpathcurveto{\pgfqpoint{0.158007in}{-0.014911in}}{\pgfqpoint{0.162247in}{-0.016667in}}{\pgfqpoint{0.166667in}{-0.016667in}}%
\pgfpathclose%
\pgfpathmoveto{\pgfqpoint{0.333333in}{-0.016667in}}%
\pgfpathcurveto{\pgfqpoint{0.337753in}{-0.016667in}}{\pgfqpoint{0.341993in}{-0.014911in}}{\pgfqpoint{0.345118in}{-0.011785in}}%
\pgfpathcurveto{\pgfqpoint{0.348244in}{-0.008660in}}{\pgfqpoint{0.350000in}{-0.004420in}}{\pgfqpoint{0.350000in}{0.000000in}}%
\pgfpathcurveto{\pgfqpoint{0.350000in}{0.004420in}}{\pgfqpoint{0.348244in}{0.008660in}}{\pgfqpoint{0.345118in}{0.011785in}}%
\pgfpathcurveto{\pgfqpoint{0.341993in}{0.014911in}}{\pgfqpoint{0.337753in}{0.016667in}}{\pgfqpoint{0.333333in}{0.016667in}}%
\pgfpathcurveto{\pgfqpoint{0.328913in}{0.016667in}}{\pgfqpoint{0.324674in}{0.014911in}}{\pgfqpoint{0.321548in}{0.011785in}}%
\pgfpathcurveto{\pgfqpoint{0.318423in}{0.008660in}}{\pgfqpoint{0.316667in}{0.004420in}}{\pgfqpoint{0.316667in}{0.000000in}}%
\pgfpathcurveto{\pgfqpoint{0.316667in}{-0.004420in}}{\pgfqpoint{0.318423in}{-0.008660in}}{\pgfqpoint{0.321548in}{-0.011785in}}%
\pgfpathcurveto{\pgfqpoint{0.324674in}{-0.014911in}}{\pgfqpoint{0.328913in}{-0.016667in}}{\pgfqpoint{0.333333in}{-0.016667in}}%
\pgfpathclose%
\pgfpathmoveto{\pgfqpoint{0.500000in}{-0.016667in}}%
\pgfpathcurveto{\pgfqpoint{0.504420in}{-0.016667in}}{\pgfqpoint{0.508660in}{-0.014911in}}{\pgfqpoint{0.511785in}{-0.011785in}}%
\pgfpathcurveto{\pgfqpoint{0.514911in}{-0.008660in}}{\pgfqpoint{0.516667in}{-0.004420in}}{\pgfqpoint{0.516667in}{0.000000in}}%
\pgfpathcurveto{\pgfqpoint{0.516667in}{0.004420in}}{\pgfqpoint{0.514911in}{0.008660in}}{\pgfqpoint{0.511785in}{0.011785in}}%
\pgfpathcurveto{\pgfqpoint{0.508660in}{0.014911in}}{\pgfqpoint{0.504420in}{0.016667in}}{\pgfqpoint{0.500000in}{0.016667in}}%
\pgfpathcurveto{\pgfqpoint{0.495580in}{0.016667in}}{\pgfqpoint{0.491340in}{0.014911in}}{\pgfqpoint{0.488215in}{0.011785in}}%
\pgfpathcurveto{\pgfqpoint{0.485089in}{0.008660in}}{\pgfqpoint{0.483333in}{0.004420in}}{\pgfqpoint{0.483333in}{0.000000in}}%
\pgfpathcurveto{\pgfqpoint{0.483333in}{-0.004420in}}{\pgfqpoint{0.485089in}{-0.008660in}}{\pgfqpoint{0.488215in}{-0.011785in}}%
\pgfpathcurveto{\pgfqpoint{0.491340in}{-0.014911in}}{\pgfqpoint{0.495580in}{-0.016667in}}{\pgfqpoint{0.500000in}{-0.016667in}}%
\pgfpathclose%
\pgfpathmoveto{\pgfqpoint{0.666667in}{-0.016667in}}%
\pgfpathcurveto{\pgfqpoint{0.671087in}{-0.016667in}}{\pgfqpoint{0.675326in}{-0.014911in}}{\pgfqpoint{0.678452in}{-0.011785in}}%
\pgfpathcurveto{\pgfqpoint{0.681577in}{-0.008660in}}{\pgfqpoint{0.683333in}{-0.004420in}}{\pgfqpoint{0.683333in}{0.000000in}}%
\pgfpathcurveto{\pgfqpoint{0.683333in}{0.004420in}}{\pgfqpoint{0.681577in}{0.008660in}}{\pgfqpoint{0.678452in}{0.011785in}}%
\pgfpathcurveto{\pgfqpoint{0.675326in}{0.014911in}}{\pgfqpoint{0.671087in}{0.016667in}}{\pgfqpoint{0.666667in}{0.016667in}}%
\pgfpathcurveto{\pgfqpoint{0.662247in}{0.016667in}}{\pgfqpoint{0.658007in}{0.014911in}}{\pgfqpoint{0.654882in}{0.011785in}}%
\pgfpathcurveto{\pgfqpoint{0.651756in}{0.008660in}}{\pgfqpoint{0.650000in}{0.004420in}}{\pgfqpoint{0.650000in}{0.000000in}}%
\pgfpathcurveto{\pgfqpoint{0.650000in}{-0.004420in}}{\pgfqpoint{0.651756in}{-0.008660in}}{\pgfqpoint{0.654882in}{-0.011785in}}%
\pgfpathcurveto{\pgfqpoint{0.658007in}{-0.014911in}}{\pgfqpoint{0.662247in}{-0.016667in}}{\pgfqpoint{0.666667in}{-0.016667in}}%
\pgfpathclose%
\pgfpathmoveto{\pgfqpoint{0.833333in}{-0.016667in}}%
\pgfpathcurveto{\pgfqpoint{0.837753in}{-0.016667in}}{\pgfqpoint{0.841993in}{-0.014911in}}{\pgfqpoint{0.845118in}{-0.011785in}}%
\pgfpathcurveto{\pgfqpoint{0.848244in}{-0.008660in}}{\pgfqpoint{0.850000in}{-0.004420in}}{\pgfqpoint{0.850000in}{0.000000in}}%
\pgfpathcurveto{\pgfqpoint{0.850000in}{0.004420in}}{\pgfqpoint{0.848244in}{0.008660in}}{\pgfqpoint{0.845118in}{0.011785in}}%
\pgfpathcurveto{\pgfqpoint{0.841993in}{0.014911in}}{\pgfqpoint{0.837753in}{0.016667in}}{\pgfqpoint{0.833333in}{0.016667in}}%
\pgfpathcurveto{\pgfqpoint{0.828913in}{0.016667in}}{\pgfqpoint{0.824674in}{0.014911in}}{\pgfqpoint{0.821548in}{0.011785in}}%
\pgfpathcurveto{\pgfqpoint{0.818423in}{0.008660in}}{\pgfqpoint{0.816667in}{0.004420in}}{\pgfqpoint{0.816667in}{0.000000in}}%
\pgfpathcurveto{\pgfqpoint{0.816667in}{-0.004420in}}{\pgfqpoint{0.818423in}{-0.008660in}}{\pgfqpoint{0.821548in}{-0.011785in}}%
\pgfpathcurveto{\pgfqpoint{0.824674in}{-0.014911in}}{\pgfqpoint{0.828913in}{-0.016667in}}{\pgfqpoint{0.833333in}{-0.016667in}}%
\pgfpathclose%
\pgfpathmoveto{\pgfqpoint{1.000000in}{-0.016667in}}%
\pgfpathcurveto{\pgfqpoint{1.004420in}{-0.016667in}}{\pgfqpoint{1.008660in}{-0.014911in}}{\pgfqpoint{1.011785in}{-0.011785in}}%
\pgfpathcurveto{\pgfqpoint{1.014911in}{-0.008660in}}{\pgfqpoint{1.016667in}{-0.004420in}}{\pgfqpoint{1.016667in}{0.000000in}}%
\pgfpathcurveto{\pgfqpoint{1.016667in}{0.004420in}}{\pgfqpoint{1.014911in}{0.008660in}}{\pgfqpoint{1.011785in}{0.011785in}}%
\pgfpathcurveto{\pgfqpoint{1.008660in}{0.014911in}}{\pgfqpoint{1.004420in}{0.016667in}}{\pgfqpoint{1.000000in}{0.016667in}}%
\pgfpathcurveto{\pgfqpoint{0.995580in}{0.016667in}}{\pgfqpoint{0.991340in}{0.014911in}}{\pgfqpoint{0.988215in}{0.011785in}}%
\pgfpathcurveto{\pgfqpoint{0.985089in}{0.008660in}}{\pgfqpoint{0.983333in}{0.004420in}}{\pgfqpoint{0.983333in}{0.000000in}}%
\pgfpathcurveto{\pgfqpoint{0.983333in}{-0.004420in}}{\pgfqpoint{0.985089in}{-0.008660in}}{\pgfqpoint{0.988215in}{-0.011785in}}%
\pgfpathcurveto{\pgfqpoint{0.991340in}{-0.014911in}}{\pgfqpoint{0.995580in}{-0.016667in}}{\pgfqpoint{1.000000in}{-0.016667in}}%
\pgfpathclose%
\pgfpathmoveto{\pgfqpoint{0.083333in}{0.150000in}}%
\pgfpathcurveto{\pgfqpoint{0.087753in}{0.150000in}}{\pgfqpoint{0.091993in}{0.151756in}}{\pgfqpoint{0.095118in}{0.154882in}}%
\pgfpathcurveto{\pgfqpoint{0.098244in}{0.158007in}}{\pgfqpoint{0.100000in}{0.162247in}}{\pgfqpoint{0.100000in}{0.166667in}}%
\pgfpathcurveto{\pgfqpoint{0.100000in}{0.171087in}}{\pgfqpoint{0.098244in}{0.175326in}}{\pgfqpoint{0.095118in}{0.178452in}}%
\pgfpathcurveto{\pgfqpoint{0.091993in}{0.181577in}}{\pgfqpoint{0.087753in}{0.183333in}}{\pgfqpoint{0.083333in}{0.183333in}}%
\pgfpathcurveto{\pgfqpoint{0.078913in}{0.183333in}}{\pgfqpoint{0.074674in}{0.181577in}}{\pgfqpoint{0.071548in}{0.178452in}}%
\pgfpathcurveto{\pgfqpoint{0.068423in}{0.175326in}}{\pgfqpoint{0.066667in}{0.171087in}}{\pgfqpoint{0.066667in}{0.166667in}}%
\pgfpathcurveto{\pgfqpoint{0.066667in}{0.162247in}}{\pgfqpoint{0.068423in}{0.158007in}}{\pgfqpoint{0.071548in}{0.154882in}}%
\pgfpathcurveto{\pgfqpoint{0.074674in}{0.151756in}}{\pgfqpoint{0.078913in}{0.150000in}}{\pgfqpoint{0.083333in}{0.150000in}}%
\pgfpathclose%
\pgfpathmoveto{\pgfqpoint{0.250000in}{0.150000in}}%
\pgfpathcurveto{\pgfqpoint{0.254420in}{0.150000in}}{\pgfqpoint{0.258660in}{0.151756in}}{\pgfqpoint{0.261785in}{0.154882in}}%
\pgfpathcurveto{\pgfqpoint{0.264911in}{0.158007in}}{\pgfqpoint{0.266667in}{0.162247in}}{\pgfqpoint{0.266667in}{0.166667in}}%
\pgfpathcurveto{\pgfqpoint{0.266667in}{0.171087in}}{\pgfqpoint{0.264911in}{0.175326in}}{\pgfqpoint{0.261785in}{0.178452in}}%
\pgfpathcurveto{\pgfqpoint{0.258660in}{0.181577in}}{\pgfqpoint{0.254420in}{0.183333in}}{\pgfqpoint{0.250000in}{0.183333in}}%
\pgfpathcurveto{\pgfqpoint{0.245580in}{0.183333in}}{\pgfqpoint{0.241340in}{0.181577in}}{\pgfqpoint{0.238215in}{0.178452in}}%
\pgfpathcurveto{\pgfqpoint{0.235089in}{0.175326in}}{\pgfqpoint{0.233333in}{0.171087in}}{\pgfqpoint{0.233333in}{0.166667in}}%
\pgfpathcurveto{\pgfqpoint{0.233333in}{0.162247in}}{\pgfqpoint{0.235089in}{0.158007in}}{\pgfqpoint{0.238215in}{0.154882in}}%
\pgfpathcurveto{\pgfqpoint{0.241340in}{0.151756in}}{\pgfqpoint{0.245580in}{0.150000in}}{\pgfqpoint{0.250000in}{0.150000in}}%
\pgfpathclose%
\pgfpathmoveto{\pgfqpoint{0.416667in}{0.150000in}}%
\pgfpathcurveto{\pgfqpoint{0.421087in}{0.150000in}}{\pgfqpoint{0.425326in}{0.151756in}}{\pgfqpoint{0.428452in}{0.154882in}}%
\pgfpathcurveto{\pgfqpoint{0.431577in}{0.158007in}}{\pgfqpoint{0.433333in}{0.162247in}}{\pgfqpoint{0.433333in}{0.166667in}}%
\pgfpathcurveto{\pgfqpoint{0.433333in}{0.171087in}}{\pgfqpoint{0.431577in}{0.175326in}}{\pgfqpoint{0.428452in}{0.178452in}}%
\pgfpathcurveto{\pgfqpoint{0.425326in}{0.181577in}}{\pgfqpoint{0.421087in}{0.183333in}}{\pgfqpoint{0.416667in}{0.183333in}}%
\pgfpathcurveto{\pgfqpoint{0.412247in}{0.183333in}}{\pgfqpoint{0.408007in}{0.181577in}}{\pgfqpoint{0.404882in}{0.178452in}}%
\pgfpathcurveto{\pgfqpoint{0.401756in}{0.175326in}}{\pgfqpoint{0.400000in}{0.171087in}}{\pgfqpoint{0.400000in}{0.166667in}}%
\pgfpathcurveto{\pgfqpoint{0.400000in}{0.162247in}}{\pgfqpoint{0.401756in}{0.158007in}}{\pgfqpoint{0.404882in}{0.154882in}}%
\pgfpathcurveto{\pgfqpoint{0.408007in}{0.151756in}}{\pgfqpoint{0.412247in}{0.150000in}}{\pgfqpoint{0.416667in}{0.150000in}}%
\pgfpathclose%
\pgfpathmoveto{\pgfqpoint{0.583333in}{0.150000in}}%
\pgfpathcurveto{\pgfqpoint{0.587753in}{0.150000in}}{\pgfqpoint{0.591993in}{0.151756in}}{\pgfqpoint{0.595118in}{0.154882in}}%
\pgfpathcurveto{\pgfqpoint{0.598244in}{0.158007in}}{\pgfqpoint{0.600000in}{0.162247in}}{\pgfqpoint{0.600000in}{0.166667in}}%
\pgfpathcurveto{\pgfqpoint{0.600000in}{0.171087in}}{\pgfqpoint{0.598244in}{0.175326in}}{\pgfqpoint{0.595118in}{0.178452in}}%
\pgfpathcurveto{\pgfqpoint{0.591993in}{0.181577in}}{\pgfqpoint{0.587753in}{0.183333in}}{\pgfqpoint{0.583333in}{0.183333in}}%
\pgfpathcurveto{\pgfqpoint{0.578913in}{0.183333in}}{\pgfqpoint{0.574674in}{0.181577in}}{\pgfqpoint{0.571548in}{0.178452in}}%
\pgfpathcurveto{\pgfqpoint{0.568423in}{0.175326in}}{\pgfqpoint{0.566667in}{0.171087in}}{\pgfqpoint{0.566667in}{0.166667in}}%
\pgfpathcurveto{\pgfqpoint{0.566667in}{0.162247in}}{\pgfqpoint{0.568423in}{0.158007in}}{\pgfqpoint{0.571548in}{0.154882in}}%
\pgfpathcurveto{\pgfqpoint{0.574674in}{0.151756in}}{\pgfqpoint{0.578913in}{0.150000in}}{\pgfqpoint{0.583333in}{0.150000in}}%
\pgfpathclose%
\pgfpathmoveto{\pgfqpoint{0.750000in}{0.150000in}}%
\pgfpathcurveto{\pgfqpoint{0.754420in}{0.150000in}}{\pgfqpoint{0.758660in}{0.151756in}}{\pgfqpoint{0.761785in}{0.154882in}}%
\pgfpathcurveto{\pgfqpoint{0.764911in}{0.158007in}}{\pgfqpoint{0.766667in}{0.162247in}}{\pgfqpoint{0.766667in}{0.166667in}}%
\pgfpathcurveto{\pgfqpoint{0.766667in}{0.171087in}}{\pgfqpoint{0.764911in}{0.175326in}}{\pgfqpoint{0.761785in}{0.178452in}}%
\pgfpathcurveto{\pgfqpoint{0.758660in}{0.181577in}}{\pgfqpoint{0.754420in}{0.183333in}}{\pgfqpoint{0.750000in}{0.183333in}}%
\pgfpathcurveto{\pgfqpoint{0.745580in}{0.183333in}}{\pgfqpoint{0.741340in}{0.181577in}}{\pgfqpoint{0.738215in}{0.178452in}}%
\pgfpathcurveto{\pgfqpoint{0.735089in}{0.175326in}}{\pgfqpoint{0.733333in}{0.171087in}}{\pgfqpoint{0.733333in}{0.166667in}}%
\pgfpathcurveto{\pgfqpoint{0.733333in}{0.162247in}}{\pgfqpoint{0.735089in}{0.158007in}}{\pgfqpoint{0.738215in}{0.154882in}}%
\pgfpathcurveto{\pgfqpoint{0.741340in}{0.151756in}}{\pgfqpoint{0.745580in}{0.150000in}}{\pgfqpoint{0.750000in}{0.150000in}}%
\pgfpathclose%
\pgfpathmoveto{\pgfqpoint{0.916667in}{0.150000in}}%
\pgfpathcurveto{\pgfqpoint{0.921087in}{0.150000in}}{\pgfqpoint{0.925326in}{0.151756in}}{\pgfqpoint{0.928452in}{0.154882in}}%
\pgfpathcurveto{\pgfqpoint{0.931577in}{0.158007in}}{\pgfqpoint{0.933333in}{0.162247in}}{\pgfqpoint{0.933333in}{0.166667in}}%
\pgfpathcurveto{\pgfqpoint{0.933333in}{0.171087in}}{\pgfqpoint{0.931577in}{0.175326in}}{\pgfqpoint{0.928452in}{0.178452in}}%
\pgfpathcurveto{\pgfqpoint{0.925326in}{0.181577in}}{\pgfqpoint{0.921087in}{0.183333in}}{\pgfqpoint{0.916667in}{0.183333in}}%
\pgfpathcurveto{\pgfqpoint{0.912247in}{0.183333in}}{\pgfqpoint{0.908007in}{0.181577in}}{\pgfqpoint{0.904882in}{0.178452in}}%
\pgfpathcurveto{\pgfqpoint{0.901756in}{0.175326in}}{\pgfqpoint{0.900000in}{0.171087in}}{\pgfqpoint{0.900000in}{0.166667in}}%
\pgfpathcurveto{\pgfqpoint{0.900000in}{0.162247in}}{\pgfqpoint{0.901756in}{0.158007in}}{\pgfqpoint{0.904882in}{0.154882in}}%
\pgfpathcurveto{\pgfqpoint{0.908007in}{0.151756in}}{\pgfqpoint{0.912247in}{0.150000in}}{\pgfqpoint{0.916667in}{0.150000in}}%
\pgfpathclose%
\pgfpathmoveto{\pgfqpoint{0.000000in}{0.316667in}}%
\pgfpathcurveto{\pgfqpoint{0.004420in}{0.316667in}}{\pgfqpoint{0.008660in}{0.318423in}}{\pgfqpoint{0.011785in}{0.321548in}}%
\pgfpathcurveto{\pgfqpoint{0.014911in}{0.324674in}}{\pgfqpoint{0.016667in}{0.328913in}}{\pgfqpoint{0.016667in}{0.333333in}}%
\pgfpathcurveto{\pgfqpoint{0.016667in}{0.337753in}}{\pgfqpoint{0.014911in}{0.341993in}}{\pgfqpoint{0.011785in}{0.345118in}}%
\pgfpathcurveto{\pgfqpoint{0.008660in}{0.348244in}}{\pgfqpoint{0.004420in}{0.350000in}}{\pgfqpoint{0.000000in}{0.350000in}}%
\pgfpathcurveto{\pgfqpoint{-0.004420in}{0.350000in}}{\pgfqpoint{-0.008660in}{0.348244in}}{\pgfqpoint{-0.011785in}{0.345118in}}%
\pgfpathcurveto{\pgfqpoint{-0.014911in}{0.341993in}}{\pgfqpoint{-0.016667in}{0.337753in}}{\pgfqpoint{-0.016667in}{0.333333in}}%
\pgfpathcurveto{\pgfqpoint{-0.016667in}{0.328913in}}{\pgfqpoint{-0.014911in}{0.324674in}}{\pgfqpoint{-0.011785in}{0.321548in}}%
\pgfpathcurveto{\pgfqpoint{-0.008660in}{0.318423in}}{\pgfqpoint{-0.004420in}{0.316667in}}{\pgfqpoint{0.000000in}{0.316667in}}%
\pgfpathclose%
\pgfpathmoveto{\pgfqpoint{0.166667in}{0.316667in}}%
\pgfpathcurveto{\pgfqpoint{0.171087in}{0.316667in}}{\pgfqpoint{0.175326in}{0.318423in}}{\pgfqpoint{0.178452in}{0.321548in}}%
\pgfpathcurveto{\pgfqpoint{0.181577in}{0.324674in}}{\pgfqpoint{0.183333in}{0.328913in}}{\pgfqpoint{0.183333in}{0.333333in}}%
\pgfpathcurveto{\pgfqpoint{0.183333in}{0.337753in}}{\pgfqpoint{0.181577in}{0.341993in}}{\pgfqpoint{0.178452in}{0.345118in}}%
\pgfpathcurveto{\pgfqpoint{0.175326in}{0.348244in}}{\pgfqpoint{0.171087in}{0.350000in}}{\pgfqpoint{0.166667in}{0.350000in}}%
\pgfpathcurveto{\pgfqpoint{0.162247in}{0.350000in}}{\pgfqpoint{0.158007in}{0.348244in}}{\pgfqpoint{0.154882in}{0.345118in}}%
\pgfpathcurveto{\pgfqpoint{0.151756in}{0.341993in}}{\pgfqpoint{0.150000in}{0.337753in}}{\pgfqpoint{0.150000in}{0.333333in}}%
\pgfpathcurveto{\pgfqpoint{0.150000in}{0.328913in}}{\pgfqpoint{0.151756in}{0.324674in}}{\pgfqpoint{0.154882in}{0.321548in}}%
\pgfpathcurveto{\pgfqpoint{0.158007in}{0.318423in}}{\pgfqpoint{0.162247in}{0.316667in}}{\pgfqpoint{0.166667in}{0.316667in}}%
\pgfpathclose%
\pgfpathmoveto{\pgfqpoint{0.333333in}{0.316667in}}%
\pgfpathcurveto{\pgfqpoint{0.337753in}{0.316667in}}{\pgfqpoint{0.341993in}{0.318423in}}{\pgfqpoint{0.345118in}{0.321548in}}%
\pgfpathcurveto{\pgfqpoint{0.348244in}{0.324674in}}{\pgfqpoint{0.350000in}{0.328913in}}{\pgfqpoint{0.350000in}{0.333333in}}%
\pgfpathcurveto{\pgfqpoint{0.350000in}{0.337753in}}{\pgfqpoint{0.348244in}{0.341993in}}{\pgfqpoint{0.345118in}{0.345118in}}%
\pgfpathcurveto{\pgfqpoint{0.341993in}{0.348244in}}{\pgfqpoint{0.337753in}{0.350000in}}{\pgfqpoint{0.333333in}{0.350000in}}%
\pgfpathcurveto{\pgfqpoint{0.328913in}{0.350000in}}{\pgfqpoint{0.324674in}{0.348244in}}{\pgfqpoint{0.321548in}{0.345118in}}%
\pgfpathcurveto{\pgfqpoint{0.318423in}{0.341993in}}{\pgfqpoint{0.316667in}{0.337753in}}{\pgfqpoint{0.316667in}{0.333333in}}%
\pgfpathcurveto{\pgfqpoint{0.316667in}{0.328913in}}{\pgfqpoint{0.318423in}{0.324674in}}{\pgfqpoint{0.321548in}{0.321548in}}%
\pgfpathcurveto{\pgfqpoint{0.324674in}{0.318423in}}{\pgfqpoint{0.328913in}{0.316667in}}{\pgfqpoint{0.333333in}{0.316667in}}%
\pgfpathclose%
\pgfpathmoveto{\pgfqpoint{0.500000in}{0.316667in}}%
\pgfpathcurveto{\pgfqpoint{0.504420in}{0.316667in}}{\pgfqpoint{0.508660in}{0.318423in}}{\pgfqpoint{0.511785in}{0.321548in}}%
\pgfpathcurveto{\pgfqpoint{0.514911in}{0.324674in}}{\pgfqpoint{0.516667in}{0.328913in}}{\pgfqpoint{0.516667in}{0.333333in}}%
\pgfpathcurveto{\pgfqpoint{0.516667in}{0.337753in}}{\pgfqpoint{0.514911in}{0.341993in}}{\pgfqpoint{0.511785in}{0.345118in}}%
\pgfpathcurveto{\pgfqpoint{0.508660in}{0.348244in}}{\pgfqpoint{0.504420in}{0.350000in}}{\pgfqpoint{0.500000in}{0.350000in}}%
\pgfpathcurveto{\pgfqpoint{0.495580in}{0.350000in}}{\pgfqpoint{0.491340in}{0.348244in}}{\pgfqpoint{0.488215in}{0.345118in}}%
\pgfpathcurveto{\pgfqpoint{0.485089in}{0.341993in}}{\pgfqpoint{0.483333in}{0.337753in}}{\pgfqpoint{0.483333in}{0.333333in}}%
\pgfpathcurveto{\pgfqpoint{0.483333in}{0.328913in}}{\pgfqpoint{0.485089in}{0.324674in}}{\pgfqpoint{0.488215in}{0.321548in}}%
\pgfpathcurveto{\pgfqpoint{0.491340in}{0.318423in}}{\pgfqpoint{0.495580in}{0.316667in}}{\pgfqpoint{0.500000in}{0.316667in}}%
\pgfpathclose%
\pgfpathmoveto{\pgfqpoint{0.666667in}{0.316667in}}%
\pgfpathcurveto{\pgfqpoint{0.671087in}{0.316667in}}{\pgfqpoint{0.675326in}{0.318423in}}{\pgfqpoint{0.678452in}{0.321548in}}%
\pgfpathcurveto{\pgfqpoint{0.681577in}{0.324674in}}{\pgfqpoint{0.683333in}{0.328913in}}{\pgfqpoint{0.683333in}{0.333333in}}%
\pgfpathcurveto{\pgfqpoint{0.683333in}{0.337753in}}{\pgfqpoint{0.681577in}{0.341993in}}{\pgfqpoint{0.678452in}{0.345118in}}%
\pgfpathcurveto{\pgfqpoint{0.675326in}{0.348244in}}{\pgfqpoint{0.671087in}{0.350000in}}{\pgfqpoint{0.666667in}{0.350000in}}%
\pgfpathcurveto{\pgfqpoint{0.662247in}{0.350000in}}{\pgfqpoint{0.658007in}{0.348244in}}{\pgfqpoint{0.654882in}{0.345118in}}%
\pgfpathcurveto{\pgfqpoint{0.651756in}{0.341993in}}{\pgfqpoint{0.650000in}{0.337753in}}{\pgfqpoint{0.650000in}{0.333333in}}%
\pgfpathcurveto{\pgfqpoint{0.650000in}{0.328913in}}{\pgfqpoint{0.651756in}{0.324674in}}{\pgfqpoint{0.654882in}{0.321548in}}%
\pgfpathcurveto{\pgfqpoint{0.658007in}{0.318423in}}{\pgfqpoint{0.662247in}{0.316667in}}{\pgfqpoint{0.666667in}{0.316667in}}%
\pgfpathclose%
\pgfpathmoveto{\pgfqpoint{0.833333in}{0.316667in}}%
\pgfpathcurveto{\pgfqpoint{0.837753in}{0.316667in}}{\pgfqpoint{0.841993in}{0.318423in}}{\pgfqpoint{0.845118in}{0.321548in}}%
\pgfpathcurveto{\pgfqpoint{0.848244in}{0.324674in}}{\pgfqpoint{0.850000in}{0.328913in}}{\pgfqpoint{0.850000in}{0.333333in}}%
\pgfpathcurveto{\pgfqpoint{0.850000in}{0.337753in}}{\pgfqpoint{0.848244in}{0.341993in}}{\pgfqpoint{0.845118in}{0.345118in}}%
\pgfpathcurveto{\pgfqpoint{0.841993in}{0.348244in}}{\pgfqpoint{0.837753in}{0.350000in}}{\pgfqpoint{0.833333in}{0.350000in}}%
\pgfpathcurveto{\pgfqpoint{0.828913in}{0.350000in}}{\pgfqpoint{0.824674in}{0.348244in}}{\pgfqpoint{0.821548in}{0.345118in}}%
\pgfpathcurveto{\pgfqpoint{0.818423in}{0.341993in}}{\pgfqpoint{0.816667in}{0.337753in}}{\pgfqpoint{0.816667in}{0.333333in}}%
\pgfpathcurveto{\pgfqpoint{0.816667in}{0.328913in}}{\pgfqpoint{0.818423in}{0.324674in}}{\pgfqpoint{0.821548in}{0.321548in}}%
\pgfpathcurveto{\pgfqpoint{0.824674in}{0.318423in}}{\pgfqpoint{0.828913in}{0.316667in}}{\pgfqpoint{0.833333in}{0.316667in}}%
\pgfpathclose%
\pgfpathmoveto{\pgfqpoint{1.000000in}{0.316667in}}%
\pgfpathcurveto{\pgfqpoint{1.004420in}{0.316667in}}{\pgfqpoint{1.008660in}{0.318423in}}{\pgfqpoint{1.011785in}{0.321548in}}%
\pgfpathcurveto{\pgfqpoint{1.014911in}{0.324674in}}{\pgfqpoint{1.016667in}{0.328913in}}{\pgfqpoint{1.016667in}{0.333333in}}%
\pgfpathcurveto{\pgfqpoint{1.016667in}{0.337753in}}{\pgfqpoint{1.014911in}{0.341993in}}{\pgfqpoint{1.011785in}{0.345118in}}%
\pgfpathcurveto{\pgfqpoint{1.008660in}{0.348244in}}{\pgfqpoint{1.004420in}{0.350000in}}{\pgfqpoint{1.000000in}{0.350000in}}%
\pgfpathcurveto{\pgfqpoint{0.995580in}{0.350000in}}{\pgfqpoint{0.991340in}{0.348244in}}{\pgfqpoint{0.988215in}{0.345118in}}%
\pgfpathcurveto{\pgfqpoint{0.985089in}{0.341993in}}{\pgfqpoint{0.983333in}{0.337753in}}{\pgfqpoint{0.983333in}{0.333333in}}%
\pgfpathcurveto{\pgfqpoint{0.983333in}{0.328913in}}{\pgfqpoint{0.985089in}{0.324674in}}{\pgfqpoint{0.988215in}{0.321548in}}%
\pgfpathcurveto{\pgfqpoint{0.991340in}{0.318423in}}{\pgfqpoint{0.995580in}{0.316667in}}{\pgfqpoint{1.000000in}{0.316667in}}%
\pgfpathclose%
\pgfpathmoveto{\pgfqpoint{0.083333in}{0.483333in}}%
\pgfpathcurveto{\pgfqpoint{0.087753in}{0.483333in}}{\pgfqpoint{0.091993in}{0.485089in}}{\pgfqpoint{0.095118in}{0.488215in}}%
\pgfpathcurveto{\pgfqpoint{0.098244in}{0.491340in}}{\pgfqpoint{0.100000in}{0.495580in}}{\pgfqpoint{0.100000in}{0.500000in}}%
\pgfpathcurveto{\pgfqpoint{0.100000in}{0.504420in}}{\pgfqpoint{0.098244in}{0.508660in}}{\pgfqpoint{0.095118in}{0.511785in}}%
\pgfpathcurveto{\pgfqpoint{0.091993in}{0.514911in}}{\pgfqpoint{0.087753in}{0.516667in}}{\pgfqpoint{0.083333in}{0.516667in}}%
\pgfpathcurveto{\pgfqpoint{0.078913in}{0.516667in}}{\pgfqpoint{0.074674in}{0.514911in}}{\pgfqpoint{0.071548in}{0.511785in}}%
\pgfpathcurveto{\pgfqpoint{0.068423in}{0.508660in}}{\pgfqpoint{0.066667in}{0.504420in}}{\pgfqpoint{0.066667in}{0.500000in}}%
\pgfpathcurveto{\pgfqpoint{0.066667in}{0.495580in}}{\pgfqpoint{0.068423in}{0.491340in}}{\pgfqpoint{0.071548in}{0.488215in}}%
\pgfpathcurveto{\pgfqpoint{0.074674in}{0.485089in}}{\pgfqpoint{0.078913in}{0.483333in}}{\pgfqpoint{0.083333in}{0.483333in}}%
\pgfpathclose%
\pgfpathmoveto{\pgfqpoint{0.250000in}{0.483333in}}%
\pgfpathcurveto{\pgfqpoint{0.254420in}{0.483333in}}{\pgfqpoint{0.258660in}{0.485089in}}{\pgfqpoint{0.261785in}{0.488215in}}%
\pgfpathcurveto{\pgfqpoint{0.264911in}{0.491340in}}{\pgfqpoint{0.266667in}{0.495580in}}{\pgfqpoint{0.266667in}{0.500000in}}%
\pgfpathcurveto{\pgfqpoint{0.266667in}{0.504420in}}{\pgfqpoint{0.264911in}{0.508660in}}{\pgfqpoint{0.261785in}{0.511785in}}%
\pgfpathcurveto{\pgfqpoint{0.258660in}{0.514911in}}{\pgfqpoint{0.254420in}{0.516667in}}{\pgfqpoint{0.250000in}{0.516667in}}%
\pgfpathcurveto{\pgfqpoint{0.245580in}{0.516667in}}{\pgfqpoint{0.241340in}{0.514911in}}{\pgfqpoint{0.238215in}{0.511785in}}%
\pgfpathcurveto{\pgfqpoint{0.235089in}{0.508660in}}{\pgfqpoint{0.233333in}{0.504420in}}{\pgfqpoint{0.233333in}{0.500000in}}%
\pgfpathcurveto{\pgfqpoint{0.233333in}{0.495580in}}{\pgfqpoint{0.235089in}{0.491340in}}{\pgfqpoint{0.238215in}{0.488215in}}%
\pgfpathcurveto{\pgfqpoint{0.241340in}{0.485089in}}{\pgfqpoint{0.245580in}{0.483333in}}{\pgfqpoint{0.250000in}{0.483333in}}%
\pgfpathclose%
\pgfpathmoveto{\pgfqpoint{0.416667in}{0.483333in}}%
\pgfpathcurveto{\pgfqpoint{0.421087in}{0.483333in}}{\pgfqpoint{0.425326in}{0.485089in}}{\pgfqpoint{0.428452in}{0.488215in}}%
\pgfpathcurveto{\pgfqpoint{0.431577in}{0.491340in}}{\pgfqpoint{0.433333in}{0.495580in}}{\pgfqpoint{0.433333in}{0.500000in}}%
\pgfpathcurveto{\pgfqpoint{0.433333in}{0.504420in}}{\pgfqpoint{0.431577in}{0.508660in}}{\pgfqpoint{0.428452in}{0.511785in}}%
\pgfpathcurveto{\pgfqpoint{0.425326in}{0.514911in}}{\pgfqpoint{0.421087in}{0.516667in}}{\pgfqpoint{0.416667in}{0.516667in}}%
\pgfpathcurveto{\pgfqpoint{0.412247in}{0.516667in}}{\pgfqpoint{0.408007in}{0.514911in}}{\pgfqpoint{0.404882in}{0.511785in}}%
\pgfpathcurveto{\pgfqpoint{0.401756in}{0.508660in}}{\pgfqpoint{0.400000in}{0.504420in}}{\pgfqpoint{0.400000in}{0.500000in}}%
\pgfpathcurveto{\pgfqpoint{0.400000in}{0.495580in}}{\pgfqpoint{0.401756in}{0.491340in}}{\pgfqpoint{0.404882in}{0.488215in}}%
\pgfpathcurveto{\pgfqpoint{0.408007in}{0.485089in}}{\pgfqpoint{0.412247in}{0.483333in}}{\pgfqpoint{0.416667in}{0.483333in}}%
\pgfpathclose%
\pgfpathmoveto{\pgfqpoint{0.583333in}{0.483333in}}%
\pgfpathcurveto{\pgfqpoint{0.587753in}{0.483333in}}{\pgfqpoint{0.591993in}{0.485089in}}{\pgfqpoint{0.595118in}{0.488215in}}%
\pgfpathcurveto{\pgfqpoint{0.598244in}{0.491340in}}{\pgfqpoint{0.600000in}{0.495580in}}{\pgfqpoint{0.600000in}{0.500000in}}%
\pgfpathcurveto{\pgfqpoint{0.600000in}{0.504420in}}{\pgfqpoint{0.598244in}{0.508660in}}{\pgfqpoint{0.595118in}{0.511785in}}%
\pgfpathcurveto{\pgfqpoint{0.591993in}{0.514911in}}{\pgfqpoint{0.587753in}{0.516667in}}{\pgfqpoint{0.583333in}{0.516667in}}%
\pgfpathcurveto{\pgfqpoint{0.578913in}{0.516667in}}{\pgfqpoint{0.574674in}{0.514911in}}{\pgfqpoint{0.571548in}{0.511785in}}%
\pgfpathcurveto{\pgfqpoint{0.568423in}{0.508660in}}{\pgfqpoint{0.566667in}{0.504420in}}{\pgfqpoint{0.566667in}{0.500000in}}%
\pgfpathcurveto{\pgfqpoint{0.566667in}{0.495580in}}{\pgfqpoint{0.568423in}{0.491340in}}{\pgfqpoint{0.571548in}{0.488215in}}%
\pgfpathcurveto{\pgfqpoint{0.574674in}{0.485089in}}{\pgfqpoint{0.578913in}{0.483333in}}{\pgfqpoint{0.583333in}{0.483333in}}%
\pgfpathclose%
\pgfpathmoveto{\pgfqpoint{0.750000in}{0.483333in}}%
\pgfpathcurveto{\pgfqpoint{0.754420in}{0.483333in}}{\pgfqpoint{0.758660in}{0.485089in}}{\pgfqpoint{0.761785in}{0.488215in}}%
\pgfpathcurveto{\pgfqpoint{0.764911in}{0.491340in}}{\pgfqpoint{0.766667in}{0.495580in}}{\pgfqpoint{0.766667in}{0.500000in}}%
\pgfpathcurveto{\pgfqpoint{0.766667in}{0.504420in}}{\pgfqpoint{0.764911in}{0.508660in}}{\pgfqpoint{0.761785in}{0.511785in}}%
\pgfpathcurveto{\pgfqpoint{0.758660in}{0.514911in}}{\pgfqpoint{0.754420in}{0.516667in}}{\pgfqpoint{0.750000in}{0.516667in}}%
\pgfpathcurveto{\pgfqpoint{0.745580in}{0.516667in}}{\pgfqpoint{0.741340in}{0.514911in}}{\pgfqpoint{0.738215in}{0.511785in}}%
\pgfpathcurveto{\pgfqpoint{0.735089in}{0.508660in}}{\pgfqpoint{0.733333in}{0.504420in}}{\pgfqpoint{0.733333in}{0.500000in}}%
\pgfpathcurveto{\pgfqpoint{0.733333in}{0.495580in}}{\pgfqpoint{0.735089in}{0.491340in}}{\pgfqpoint{0.738215in}{0.488215in}}%
\pgfpathcurveto{\pgfqpoint{0.741340in}{0.485089in}}{\pgfqpoint{0.745580in}{0.483333in}}{\pgfqpoint{0.750000in}{0.483333in}}%
\pgfpathclose%
\pgfpathmoveto{\pgfqpoint{0.916667in}{0.483333in}}%
\pgfpathcurveto{\pgfqpoint{0.921087in}{0.483333in}}{\pgfqpoint{0.925326in}{0.485089in}}{\pgfqpoint{0.928452in}{0.488215in}}%
\pgfpathcurveto{\pgfqpoint{0.931577in}{0.491340in}}{\pgfqpoint{0.933333in}{0.495580in}}{\pgfqpoint{0.933333in}{0.500000in}}%
\pgfpathcurveto{\pgfqpoint{0.933333in}{0.504420in}}{\pgfqpoint{0.931577in}{0.508660in}}{\pgfqpoint{0.928452in}{0.511785in}}%
\pgfpathcurveto{\pgfqpoint{0.925326in}{0.514911in}}{\pgfqpoint{0.921087in}{0.516667in}}{\pgfqpoint{0.916667in}{0.516667in}}%
\pgfpathcurveto{\pgfqpoint{0.912247in}{0.516667in}}{\pgfqpoint{0.908007in}{0.514911in}}{\pgfqpoint{0.904882in}{0.511785in}}%
\pgfpathcurveto{\pgfqpoint{0.901756in}{0.508660in}}{\pgfqpoint{0.900000in}{0.504420in}}{\pgfqpoint{0.900000in}{0.500000in}}%
\pgfpathcurveto{\pgfqpoint{0.900000in}{0.495580in}}{\pgfqpoint{0.901756in}{0.491340in}}{\pgfqpoint{0.904882in}{0.488215in}}%
\pgfpathcurveto{\pgfqpoint{0.908007in}{0.485089in}}{\pgfqpoint{0.912247in}{0.483333in}}{\pgfqpoint{0.916667in}{0.483333in}}%
\pgfpathclose%
\pgfpathmoveto{\pgfqpoint{0.000000in}{0.650000in}}%
\pgfpathcurveto{\pgfqpoint{0.004420in}{0.650000in}}{\pgfqpoint{0.008660in}{0.651756in}}{\pgfqpoint{0.011785in}{0.654882in}}%
\pgfpathcurveto{\pgfqpoint{0.014911in}{0.658007in}}{\pgfqpoint{0.016667in}{0.662247in}}{\pgfqpoint{0.016667in}{0.666667in}}%
\pgfpathcurveto{\pgfqpoint{0.016667in}{0.671087in}}{\pgfqpoint{0.014911in}{0.675326in}}{\pgfqpoint{0.011785in}{0.678452in}}%
\pgfpathcurveto{\pgfqpoint{0.008660in}{0.681577in}}{\pgfqpoint{0.004420in}{0.683333in}}{\pgfqpoint{0.000000in}{0.683333in}}%
\pgfpathcurveto{\pgfqpoint{-0.004420in}{0.683333in}}{\pgfqpoint{-0.008660in}{0.681577in}}{\pgfqpoint{-0.011785in}{0.678452in}}%
\pgfpathcurveto{\pgfqpoint{-0.014911in}{0.675326in}}{\pgfqpoint{-0.016667in}{0.671087in}}{\pgfqpoint{-0.016667in}{0.666667in}}%
\pgfpathcurveto{\pgfqpoint{-0.016667in}{0.662247in}}{\pgfqpoint{-0.014911in}{0.658007in}}{\pgfqpoint{-0.011785in}{0.654882in}}%
\pgfpathcurveto{\pgfqpoint{-0.008660in}{0.651756in}}{\pgfqpoint{-0.004420in}{0.650000in}}{\pgfqpoint{0.000000in}{0.650000in}}%
\pgfpathclose%
\pgfpathmoveto{\pgfqpoint{0.166667in}{0.650000in}}%
\pgfpathcurveto{\pgfqpoint{0.171087in}{0.650000in}}{\pgfqpoint{0.175326in}{0.651756in}}{\pgfqpoint{0.178452in}{0.654882in}}%
\pgfpathcurveto{\pgfqpoint{0.181577in}{0.658007in}}{\pgfqpoint{0.183333in}{0.662247in}}{\pgfqpoint{0.183333in}{0.666667in}}%
\pgfpathcurveto{\pgfqpoint{0.183333in}{0.671087in}}{\pgfqpoint{0.181577in}{0.675326in}}{\pgfqpoint{0.178452in}{0.678452in}}%
\pgfpathcurveto{\pgfqpoint{0.175326in}{0.681577in}}{\pgfqpoint{0.171087in}{0.683333in}}{\pgfqpoint{0.166667in}{0.683333in}}%
\pgfpathcurveto{\pgfqpoint{0.162247in}{0.683333in}}{\pgfqpoint{0.158007in}{0.681577in}}{\pgfqpoint{0.154882in}{0.678452in}}%
\pgfpathcurveto{\pgfqpoint{0.151756in}{0.675326in}}{\pgfqpoint{0.150000in}{0.671087in}}{\pgfqpoint{0.150000in}{0.666667in}}%
\pgfpathcurveto{\pgfqpoint{0.150000in}{0.662247in}}{\pgfqpoint{0.151756in}{0.658007in}}{\pgfqpoint{0.154882in}{0.654882in}}%
\pgfpathcurveto{\pgfqpoint{0.158007in}{0.651756in}}{\pgfqpoint{0.162247in}{0.650000in}}{\pgfqpoint{0.166667in}{0.650000in}}%
\pgfpathclose%
\pgfpathmoveto{\pgfqpoint{0.333333in}{0.650000in}}%
\pgfpathcurveto{\pgfqpoint{0.337753in}{0.650000in}}{\pgfqpoint{0.341993in}{0.651756in}}{\pgfqpoint{0.345118in}{0.654882in}}%
\pgfpathcurveto{\pgfqpoint{0.348244in}{0.658007in}}{\pgfqpoint{0.350000in}{0.662247in}}{\pgfqpoint{0.350000in}{0.666667in}}%
\pgfpathcurveto{\pgfqpoint{0.350000in}{0.671087in}}{\pgfqpoint{0.348244in}{0.675326in}}{\pgfqpoint{0.345118in}{0.678452in}}%
\pgfpathcurveto{\pgfqpoint{0.341993in}{0.681577in}}{\pgfqpoint{0.337753in}{0.683333in}}{\pgfqpoint{0.333333in}{0.683333in}}%
\pgfpathcurveto{\pgfqpoint{0.328913in}{0.683333in}}{\pgfqpoint{0.324674in}{0.681577in}}{\pgfqpoint{0.321548in}{0.678452in}}%
\pgfpathcurveto{\pgfqpoint{0.318423in}{0.675326in}}{\pgfqpoint{0.316667in}{0.671087in}}{\pgfqpoint{0.316667in}{0.666667in}}%
\pgfpathcurveto{\pgfqpoint{0.316667in}{0.662247in}}{\pgfqpoint{0.318423in}{0.658007in}}{\pgfqpoint{0.321548in}{0.654882in}}%
\pgfpathcurveto{\pgfqpoint{0.324674in}{0.651756in}}{\pgfqpoint{0.328913in}{0.650000in}}{\pgfqpoint{0.333333in}{0.650000in}}%
\pgfpathclose%
\pgfpathmoveto{\pgfqpoint{0.500000in}{0.650000in}}%
\pgfpathcurveto{\pgfqpoint{0.504420in}{0.650000in}}{\pgfqpoint{0.508660in}{0.651756in}}{\pgfqpoint{0.511785in}{0.654882in}}%
\pgfpathcurveto{\pgfqpoint{0.514911in}{0.658007in}}{\pgfqpoint{0.516667in}{0.662247in}}{\pgfqpoint{0.516667in}{0.666667in}}%
\pgfpathcurveto{\pgfqpoint{0.516667in}{0.671087in}}{\pgfqpoint{0.514911in}{0.675326in}}{\pgfqpoint{0.511785in}{0.678452in}}%
\pgfpathcurveto{\pgfqpoint{0.508660in}{0.681577in}}{\pgfqpoint{0.504420in}{0.683333in}}{\pgfqpoint{0.500000in}{0.683333in}}%
\pgfpathcurveto{\pgfqpoint{0.495580in}{0.683333in}}{\pgfqpoint{0.491340in}{0.681577in}}{\pgfqpoint{0.488215in}{0.678452in}}%
\pgfpathcurveto{\pgfqpoint{0.485089in}{0.675326in}}{\pgfqpoint{0.483333in}{0.671087in}}{\pgfqpoint{0.483333in}{0.666667in}}%
\pgfpathcurveto{\pgfqpoint{0.483333in}{0.662247in}}{\pgfqpoint{0.485089in}{0.658007in}}{\pgfqpoint{0.488215in}{0.654882in}}%
\pgfpathcurveto{\pgfqpoint{0.491340in}{0.651756in}}{\pgfqpoint{0.495580in}{0.650000in}}{\pgfqpoint{0.500000in}{0.650000in}}%
\pgfpathclose%
\pgfpathmoveto{\pgfqpoint{0.666667in}{0.650000in}}%
\pgfpathcurveto{\pgfqpoint{0.671087in}{0.650000in}}{\pgfqpoint{0.675326in}{0.651756in}}{\pgfqpoint{0.678452in}{0.654882in}}%
\pgfpathcurveto{\pgfqpoint{0.681577in}{0.658007in}}{\pgfqpoint{0.683333in}{0.662247in}}{\pgfqpoint{0.683333in}{0.666667in}}%
\pgfpathcurveto{\pgfqpoint{0.683333in}{0.671087in}}{\pgfqpoint{0.681577in}{0.675326in}}{\pgfqpoint{0.678452in}{0.678452in}}%
\pgfpathcurveto{\pgfqpoint{0.675326in}{0.681577in}}{\pgfqpoint{0.671087in}{0.683333in}}{\pgfqpoint{0.666667in}{0.683333in}}%
\pgfpathcurveto{\pgfqpoint{0.662247in}{0.683333in}}{\pgfqpoint{0.658007in}{0.681577in}}{\pgfqpoint{0.654882in}{0.678452in}}%
\pgfpathcurveto{\pgfqpoint{0.651756in}{0.675326in}}{\pgfqpoint{0.650000in}{0.671087in}}{\pgfqpoint{0.650000in}{0.666667in}}%
\pgfpathcurveto{\pgfqpoint{0.650000in}{0.662247in}}{\pgfqpoint{0.651756in}{0.658007in}}{\pgfqpoint{0.654882in}{0.654882in}}%
\pgfpathcurveto{\pgfqpoint{0.658007in}{0.651756in}}{\pgfqpoint{0.662247in}{0.650000in}}{\pgfqpoint{0.666667in}{0.650000in}}%
\pgfpathclose%
\pgfpathmoveto{\pgfqpoint{0.833333in}{0.650000in}}%
\pgfpathcurveto{\pgfqpoint{0.837753in}{0.650000in}}{\pgfqpoint{0.841993in}{0.651756in}}{\pgfqpoint{0.845118in}{0.654882in}}%
\pgfpathcurveto{\pgfqpoint{0.848244in}{0.658007in}}{\pgfqpoint{0.850000in}{0.662247in}}{\pgfqpoint{0.850000in}{0.666667in}}%
\pgfpathcurveto{\pgfqpoint{0.850000in}{0.671087in}}{\pgfqpoint{0.848244in}{0.675326in}}{\pgfqpoint{0.845118in}{0.678452in}}%
\pgfpathcurveto{\pgfqpoint{0.841993in}{0.681577in}}{\pgfqpoint{0.837753in}{0.683333in}}{\pgfqpoint{0.833333in}{0.683333in}}%
\pgfpathcurveto{\pgfqpoint{0.828913in}{0.683333in}}{\pgfqpoint{0.824674in}{0.681577in}}{\pgfqpoint{0.821548in}{0.678452in}}%
\pgfpathcurveto{\pgfqpoint{0.818423in}{0.675326in}}{\pgfqpoint{0.816667in}{0.671087in}}{\pgfqpoint{0.816667in}{0.666667in}}%
\pgfpathcurveto{\pgfqpoint{0.816667in}{0.662247in}}{\pgfqpoint{0.818423in}{0.658007in}}{\pgfqpoint{0.821548in}{0.654882in}}%
\pgfpathcurveto{\pgfqpoint{0.824674in}{0.651756in}}{\pgfqpoint{0.828913in}{0.650000in}}{\pgfqpoint{0.833333in}{0.650000in}}%
\pgfpathclose%
\pgfpathmoveto{\pgfqpoint{1.000000in}{0.650000in}}%
\pgfpathcurveto{\pgfqpoint{1.004420in}{0.650000in}}{\pgfqpoint{1.008660in}{0.651756in}}{\pgfqpoint{1.011785in}{0.654882in}}%
\pgfpathcurveto{\pgfqpoint{1.014911in}{0.658007in}}{\pgfqpoint{1.016667in}{0.662247in}}{\pgfqpoint{1.016667in}{0.666667in}}%
\pgfpathcurveto{\pgfqpoint{1.016667in}{0.671087in}}{\pgfqpoint{1.014911in}{0.675326in}}{\pgfqpoint{1.011785in}{0.678452in}}%
\pgfpathcurveto{\pgfqpoint{1.008660in}{0.681577in}}{\pgfqpoint{1.004420in}{0.683333in}}{\pgfqpoint{1.000000in}{0.683333in}}%
\pgfpathcurveto{\pgfqpoint{0.995580in}{0.683333in}}{\pgfqpoint{0.991340in}{0.681577in}}{\pgfqpoint{0.988215in}{0.678452in}}%
\pgfpathcurveto{\pgfqpoint{0.985089in}{0.675326in}}{\pgfqpoint{0.983333in}{0.671087in}}{\pgfqpoint{0.983333in}{0.666667in}}%
\pgfpathcurveto{\pgfqpoint{0.983333in}{0.662247in}}{\pgfqpoint{0.985089in}{0.658007in}}{\pgfqpoint{0.988215in}{0.654882in}}%
\pgfpathcurveto{\pgfqpoint{0.991340in}{0.651756in}}{\pgfqpoint{0.995580in}{0.650000in}}{\pgfqpoint{1.000000in}{0.650000in}}%
\pgfpathclose%
\pgfpathmoveto{\pgfqpoint{0.083333in}{0.816667in}}%
\pgfpathcurveto{\pgfqpoint{0.087753in}{0.816667in}}{\pgfqpoint{0.091993in}{0.818423in}}{\pgfqpoint{0.095118in}{0.821548in}}%
\pgfpathcurveto{\pgfqpoint{0.098244in}{0.824674in}}{\pgfqpoint{0.100000in}{0.828913in}}{\pgfqpoint{0.100000in}{0.833333in}}%
\pgfpathcurveto{\pgfqpoint{0.100000in}{0.837753in}}{\pgfqpoint{0.098244in}{0.841993in}}{\pgfqpoint{0.095118in}{0.845118in}}%
\pgfpathcurveto{\pgfqpoint{0.091993in}{0.848244in}}{\pgfqpoint{0.087753in}{0.850000in}}{\pgfqpoint{0.083333in}{0.850000in}}%
\pgfpathcurveto{\pgfqpoint{0.078913in}{0.850000in}}{\pgfqpoint{0.074674in}{0.848244in}}{\pgfqpoint{0.071548in}{0.845118in}}%
\pgfpathcurveto{\pgfqpoint{0.068423in}{0.841993in}}{\pgfqpoint{0.066667in}{0.837753in}}{\pgfqpoint{0.066667in}{0.833333in}}%
\pgfpathcurveto{\pgfqpoint{0.066667in}{0.828913in}}{\pgfqpoint{0.068423in}{0.824674in}}{\pgfqpoint{0.071548in}{0.821548in}}%
\pgfpathcurveto{\pgfqpoint{0.074674in}{0.818423in}}{\pgfqpoint{0.078913in}{0.816667in}}{\pgfqpoint{0.083333in}{0.816667in}}%
\pgfpathclose%
\pgfpathmoveto{\pgfqpoint{0.250000in}{0.816667in}}%
\pgfpathcurveto{\pgfqpoint{0.254420in}{0.816667in}}{\pgfqpoint{0.258660in}{0.818423in}}{\pgfqpoint{0.261785in}{0.821548in}}%
\pgfpathcurveto{\pgfqpoint{0.264911in}{0.824674in}}{\pgfqpoint{0.266667in}{0.828913in}}{\pgfqpoint{0.266667in}{0.833333in}}%
\pgfpathcurveto{\pgfqpoint{0.266667in}{0.837753in}}{\pgfqpoint{0.264911in}{0.841993in}}{\pgfqpoint{0.261785in}{0.845118in}}%
\pgfpathcurveto{\pgfqpoint{0.258660in}{0.848244in}}{\pgfqpoint{0.254420in}{0.850000in}}{\pgfqpoint{0.250000in}{0.850000in}}%
\pgfpathcurveto{\pgfqpoint{0.245580in}{0.850000in}}{\pgfqpoint{0.241340in}{0.848244in}}{\pgfqpoint{0.238215in}{0.845118in}}%
\pgfpathcurveto{\pgfqpoint{0.235089in}{0.841993in}}{\pgfqpoint{0.233333in}{0.837753in}}{\pgfqpoint{0.233333in}{0.833333in}}%
\pgfpathcurveto{\pgfqpoint{0.233333in}{0.828913in}}{\pgfqpoint{0.235089in}{0.824674in}}{\pgfqpoint{0.238215in}{0.821548in}}%
\pgfpathcurveto{\pgfqpoint{0.241340in}{0.818423in}}{\pgfqpoint{0.245580in}{0.816667in}}{\pgfqpoint{0.250000in}{0.816667in}}%
\pgfpathclose%
\pgfpathmoveto{\pgfqpoint{0.416667in}{0.816667in}}%
\pgfpathcurveto{\pgfqpoint{0.421087in}{0.816667in}}{\pgfqpoint{0.425326in}{0.818423in}}{\pgfqpoint{0.428452in}{0.821548in}}%
\pgfpathcurveto{\pgfqpoint{0.431577in}{0.824674in}}{\pgfqpoint{0.433333in}{0.828913in}}{\pgfqpoint{0.433333in}{0.833333in}}%
\pgfpathcurveto{\pgfqpoint{0.433333in}{0.837753in}}{\pgfqpoint{0.431577in}{0.841993in}}{\pgfqpoint{0.428452in}{0.845118in}}%
\pgfpathcurveto{\pgfqpoint{0.425326in}{0.848244in}}{\pgfqpoint{0.421087in}{0.850000in}}{\pgfqpoint{0.416667in}{0.850000in}}%
\pgfpathcurveto{\pgfqpoint{0.412247in}{0.850000in}}{\pgfqpoint{0.408007in}{0.848244in}}{\pgfqpoint{0.404882in}{0.845118in}}%
\pgfpathcurveto{\pgfqpoint{0.401756in}{0.841993in}}{\pgfqpoint{0.400000in}{0.837753in}}{\pgfqpoint{0.400000in}{0.833333in}}%
\pgfpathcurveto{\pgfqpoint{0.400000in}{0.828913in}}{\pgfqpoint{0.401756in}{0.824674in}}{\pgfqpoint{0.404882in}{0.821548in}}%
\pgfpathcurveto{\pgfqpoint{0.408007in}{0.818423in}}{\pgfqpoint{0.412247in}{0.816667in}}{\pgfqpoint{0.416667in}{0.816667in}}%
\pgfpathclose%
\pgfpathmoveto{\pgfqpoint{0.583333in}{0.816667in}}%
\pgfpathcurveto{\pgfqpoint{0.587753in}{0.816667in}}{\pgfqpoint{0.591993in}{0.818423in}}{\pgfqpoint{0.595118in}{0.821548in}}%
\pgfpathcurveto{\pgfqpoint{0.598244in}{0.824674in}}{\pgfqpoint{0.600000in}{0.828913in}}{\pgfqpoint{0.600000in}{0.833333in}}%
\pgfpathcurveto{\pgfqpoint{0.600000in}{0.837753in}}{\pgfqpoint{0.598244in}{0.841993in}}{\pgfqpoint{0.595118in}{0.845118in}}%
\pgfpathcurveto{\pgfqpoint{0.591993in}{0.848244in}}{\pgfqpoint{0.587753in}{0.850000in}}{\pgfqpoint{0.583333in}{0.850000in}}%
\pgfpathcurveto{\pgfqpoint{0.578913in}{0.850000in}}{\pgfqpoint{0.574674in}{0.848244in}}{\pgfqpoint{0.571548in}{0.845118in}}%
\pgfpathcurveto{\pgfqpoint{0.568423in}{0.841993in}}{\pgfqpoint{0.566667in}{0.837753in}}{\pgfqpoint{0.566667in}{0.833333in}}%
\pgfpathcurveto{\pgfqpoint{0.566667in}{0.828913in}}{\pgfqpoint{0.568423in}{0.824674in}}{\pgfqpoint{0.571548in}{0.821548in}}%
\pgfpathcurveto{\pgfqpoint{0.574674in}{0.818423in}}{\pgfqpoint{0.578913in}{0.816667in}}{\pgfqpoint{0.583333in}{0.816667in}}%
\pgfpathclose%
\pgfpathmoveto{\pgfqpoint{0.750000in}{0.816667in}}%
\pgfpathcurveto{\pgfqpoint{0.754420in}{0.816667in}}{\pgfqpoint{0.758660in}{0.818423in}}{\pgfqpoint{0.761785in}{0.821548in}}%
\pgfpathcurveto{\pgfqpoint{0.764911in}{0.824674in}}{\pgfqpoint{0.766667in}{0.828913in}}{\pgfqpoint{0.766667in}{0.833333in}}%
\pgfpathcurveto{\pgfqpoint{0.766667in}{0.837753in}}{\pgfqpoint{0.764911in}{0.841993in}}{\pgfqpoint{0.761785in}{0.845118in}}%
\pgfpathcurveto{\pgfqpoint{0.758660in}{0.848244in}}{\pgfqpoint{0.754420in}{0.850000in}}{\pgfqpoint{0.750000in}{0.850000in}}%
\pgfpathcurveto{\pgfqpoint{0.745580in}{0.850000in}}{\pgfqpoint{0.741340in}{0.848244in}}{\pgfqpoint{0.738215in}{0.845118in}}%
\pgfpathcurveto{\pgfqpoint{0.735089in}{0.841993in}}{\pgfqpoint{0.733333in}{0.837753in}}{\pgfqpoint{0.733333in}{0.833333in}}%
\pgfpathcurveto{\pgfqpoint{0.733333in}{0.828913in}}{\pgfqpoint{0.735089in}{0.824674in}}{\pgfqpoint{0.738215in}{0.821548in}}%
\pgfpathcurveto{\pgfqpoint{0.741340in}{0.818423in}}{\pgfqpoint{0.745580in}{0.816667in}}{\pgfqpoint{0.750000in}{0.816667in}}%
\pgfpathclose%
\pgfpathmoveto{\pgfqpoint{0.916667in}{0.816667in}}%
\pgfpathcurveto{\pgfqpoint{0.921087in}{0.816667in}}{\pgfqpoint{0.925326in}{0.818423in}}{\pgfqpoint{0.928452in}{0.821548in}}%
\pgfpathcurveto{\pgfqpoint{0.931577in}{0.824674in}}{\pgfqpoint{0.933333in}{0.828913in}}{\pgfqpoint{0.933333in}{0.833333in}}%
\pgfpathcurveto{\pgfqpoint{0.933333in}{0.837753in}}{\pgfqpoint{0.931577in}{0.841993in}}{\pgfqpoint{0.928452in}{0.845118in}}%
\pgfpathcurveto{\pgfqpoint{0.925326in}{0.848244in}}{\pgfqpoint{0.921087in}{0.850000in}}{\pgfqpoint{0.916667in}{0.850000in}}%
\pgfpathcurveto{\pgfqpoint{0.912247in}{0.850000in}}{\pgfqpoint{0.908007in}{0.848244in}}{\pgfqpoint{0.904882in}{0.845118in}}%
\pgfpathcurveto{\pgfqpoint{0.901756in}{0.841993in}}{\pgfqpoint{0.900000in}{0.837753in}}{\pgfqpoint{0.900000in}{0.833333in}}%
\pgfpathcurveto{\pgfqpoint{0.900000in}{0.828913in}}{\pgfqpoint{0.901756in}{0.824674in}}{\pgfqpoint{0.904882in}{0.821548in}}%
\pgfpathcurveto{\pgfqpoint{0.908007in}{0.818423in}}{\pgfqpoint{0.912247in}{0.816667in}}{\pgfqpoint{0.916667in}{0.816667in}}%
\pgfpathclose%
\pgfpathmoveto{\pgfqpoint{0.000000in}{0.983333in}}%
\pgfpathcurveto{\pgfqpoint{0.004420in}{0.983333in}}{\pgfqpoint{0.008660in}{0.985089in}}{\pgfqpoint{0.011785in}{0.988215in}}%
\pgfpathcurveto{\pgfqpoint{0.014911in}{0.991340in}}{\pgfqpoint{0.016667in}{0.995580in}}{\pgfqpoint{0.016667in}{1.000000in}}%
\pgfpathcurveto{\pgfqpoint{0.016667in}{1.004420in}}{\pgfqpoint{0.014911in}{1.008660in}}{\pgfqpoint{0.011785in}{1.011785in}}%
\pgfpathcurveto{\pgfqpoint{0.008660in}{1.014911in}}{\pgfqpoint{0.004420in}{1.016667in}}{\pgfqpoint{0.000000in}{1.016667in}}%
\pgfpathcurveto{\pgfqpoint{-0.004420in}{1.016667in}}{\pgfqpoint{-0.008660in}{1.014911in}}{\pgfqpoint{-0.011785in}{1.011785in}}%
\pgfpathcurveto{\pgfqpoint{-0.014911in}{1.008660in}}{\pgfqpoint{-0.016667in}{1.004420in}}{\pgfqpoint{-0.016667in}{1.000000in}}%
\pgfpathcurveto{\pgfqpoint{-0.016667in}{0.995580in}}{\pgfqpoint{-0.014911in}{0.991340in}}{\pgfqpoint{-0.011785in}{0.988215in}}%
\pgfpathcurveto{\pgfqpoint{-0.008660in}{0.985089in}}{\pgfqpoint{-0.004420in}{0.983333in}}{\pgfqpoint{0.000000in}{0.983333in}}%
\pgfpathclose%
\pgfpathmoveto{\pgfqpoint{0.166667in}{0.983333in}}%
\pgfpathcurveto{\pgfqpoint{0.171087in}{0.983333in}}{\pgfqpoint{0.175326in}{0.985089in}}{\pgfqpoint{0.178452in}{0.988215in}}%
\pgfpathcurveto{\pgfqpoint{0.181577in}{0.991340in}}{\pgfqpoint{0.183333in}{0.995580in}}{\pgfqpoint{0.183333in}{1.000000in}}%
\pgfpathcurveto{\pgfqpoint{0.183333in}{1.004420in}}{\pgfqpoint{0.181577in}{1.008660in}}{\pgfqpoint{0.178452in}{1.011785in}}%
\pgfpathcurveto{\pgfqpoint{0.175326in}{1.014911in}}{\pgfqpoint{0.171087in}{1.016667in}}{\pgfqpoint{0.166667in}{1.016667in}}%
\pgfpathcurveto{\pgfqpoint{0.162247in}{1.016667in}}{\pgfqpoint{0.158007in}{1.014911in}}{\pgfqpoint{0.154882in}{1.011785in}}%
\pgfpathcurveto{\pgfqpoint{0.151756in}{1.008660in}}{\pgfqpoint{0.150000in}{1.004420in}}{\pgfqpoint{0.150000in}{1.000000in}}%
\pgfpathcurveto{\pgfqpoint{0.150000in}{0.995580in}}{\pgfqpoint{0.151756in}{0.991340in}}{\pgfqpoint{0.154882in}{0.988215in}}%
\pgfpathcurveto{\pgfqpoint{0.158007in}{0.985089in}}{\pgfqpoint{0.162247in}{0.983333in}}{\pgfqpoint{0.166667in}{0.983333in}}%
\pgfpathclose%
\pgfpathmoveto{\pgfqpoint{0.333333in}{0.983333in}}%
\pgfpathcurveto{\pgfqpoint{0.337753in}{0.983333in}}{\pgfqpoint{0.341993in}{0.985089in}}{\pgfqpoint{0.345118in}{0.988215in}}%
\pgfpathcurveto{\pgfqpoint{0.348244in}{0.991340in}}{\pgfqpoint{0.350000in}{0.995580in}}{\pgfqpoint{0.350000in}{1.000000in}}%
\pgfpathcurveto{\pgfqpoint{0.350000in}{1.004420in}}{\pgfqpoint{0.348244in}{1.008660in}}{\pgfqpoint{0.345118in}{1.011785in}}%
\pgfpathcurveto{\pgfqpoint{0.341993in}{1.014911in}}{\pgfqpoint{0.337753in}{1.016667in}}{\pgfqpoint{0.333333in}{1.016667in}}%
\pgfpathcurveto{\pgfqpoint{0.328913in}{1.016667in}}{\pgfqpoint{0.324674in}{1.014911in}}{\pgfqpoint{0.321548in}{1.011785in}}%
\pgfpathcurveto{\pgfqpoint{0.318423in}{1.008660in}}{\pgfqpoint{0.316667in}{1.004420in}}{\pgfqpoint{0.316667in}{1.000000in}}%
\pgfpathcurveto{\pgfqpoint{0.316667in}{0.995580in}}{\pgfqpoint{0.318423in}{0.991340in}}{\pgfqpoint{0.321548in}{0.988215in}}%
\pgfpathcurveto{\pgfqpoint{0.324674in}{0.985089in}}{\pgfqpoint{0.328913in}{0.983333in}}{\pgfqpoint{0.333333in}{0.983333in}}%
\pgfpathclose%
\pgfpathmoveto{\pgfqpoint{0.500000in}{0.983333in}}%
\pgfpathcurveto{\pgfqpoint{0.504420in}{0.983333in}}{\pgfqpoint{0.508660in}{0.985089in}}{\pgfqpoint{0.511785in}{0.988215in}}%
\pgfpathcurveto{\pgfqpoint{0.514911in}{0.991340in}}{\pgfqpoint{0.516667in}{0.995580in}}{\pgfqpoint{0.516667in}{1.000000in}}%
\pgfpathcurveto{\pgfqpoint{0.516667in}{1.004420in}}{\pgfqpoint{0.514911in}{1.008660in}}{\pgfqpoint{0.511785in}{1.011785in}}%
\pgfpathcurveto{\pgfqpoint{0.508660in}{1.014911in}}{\pgfqpoint{0.504420in}{1.016667in}}{\pgfqpoint{0.500000in}{1.016667in}}%
\pgfpathcurveto{\pgfqpoint{0.495580in}{1.016667in}}{\pgfqpoint{0.491340in}{1.014911in}}{\pgfqpoint{0.488215in}{1.011785in}}%
\pgfpathcurveto{\pgfqpoint{0.485089in}{1.008660in}}{\pgfqpoint{0.483333in}{1.004420in}}{\pgfqpoint{0.483333in}{1.000000in}}%
\pgfpathcurveto{\pgfqpoint{0.483333in}{0.995580in}}{\pgfqpoint{0.485089in}{0.991340in}}{\pgfqpoint{0.488215in}{0.988215in}}%
\pgfpathcurveto{\pgfqpoint{0.491340in}{0.985089in}}{\pgfqpoint{0.495580in}{0.983333in}}{\pgfqpoint{0.500000in}{0.983333in}}%
\pgfpathclose%
\pgfpathmoveto{\pgfqpoint{0.666667in}{0.983333in}}%
\pgfpathcurveto{\pgfqpoint{0.671087in}{0.983333in}}{\pgfqpoint{0.675326in}{0.985089in}}{\pgfqpoint{0.678452in}{0.988215in}}%
\pgfpathcurveto{\pgfqpoint{0.681577in}{0.991340in}}{\pgfqpoint{0.683333in}{0.995580in}}{\pgfqpoint{0.683333in}{1.000000in}}%
\pgfpathcurveto{\pgfqpoint{0.683333in}{1.004420in}}{\pgfqpoint{0.681577in}{1.008660in}}{\pgfqpoint{0.678452in}{1.011785in}}%
\pgfpathcurveto{\pgfqpoint{0.675326in}{1.014911in}}{\pgfqpoint{0.671087in}{1.016667in}}{\pgfqpoint{0.666667in}{1.016667in}}%
\pgfpathcurveto{\pgfqpoint{0.662247in}{1.016667in}}{\pgfqpoint{0.658007in}{1.014911in}}{\pgfqpoint{0.654882in}{1.011785in}}%
\pgfpathcurveto{\pgfqpoint{0.651756in}{1.008660in}}{\pgfqpoint{0.650000in}{1.004420in}}{\pgfqpoint{0.650000in}{1.000000in}}%
\pgfpathcurveto{\pgfqpoint{0.650000in}{0.995580in}}{\pgfqpoint{0.651756in}{0.991340in}}{\pgfqpoint{0.654882in}{0.988215in}}%
\pgfpathcurveto{\pgfqpoint{0.658007in}{0.985089in}}{\pgfqpoint{0.662247in}{0.983333in}}{\pgfqpoint{0.666667in}{0.983333in}}%
\pgfpathclose%
\pgfpathmoveto{\pgfqpoint{0.833333in}{0.983333in}}%
\pgfpathcurveto{\pgfqpoint{0.837753in}{0.983333in}}{\pgfqpoint{0.841993in}{0.985089in}}{\pgfqpoint{0.845118in}{0.988215in}}%
\pgfpathcurveto{\pgfqpoint{0.848244in}{0.991340in}}{\pgfqpoint{0.850000in}{0.995580in}}{\pgfqpoint{0.850000in}{1.000000in}}%
\pgfpathcurveto{\pgfqpoint{0.850000in}{1.004420in}}{\pgfqpoint{0.848244in}{1.008660in}}{\pgfqpoint{0.845118in}{1.011785in}}%
\pgfpathcurveto{\pgfqpoint{0.841993in}{1.014911in}}{\pgfqpoint{0.837753in}{1.016667in}}{\pgfqpoint{0.833333in}{1.016667in}}%
\pgfpathcurveto{\pgfqpoint{0.828913in}{1.016667in}}{\pgfqpoint{0.824674in}{1.014911in}}{\pgfqpoint{0.821548in}{1.011785in}}%
\pgfpathcurveto{\pgfqpoint{0.818423in}{1.008660in}}{\pgfqpoint{0.816667in}{1.004420in}}{\pgfqpoint{0.816667in}{1.000000in}}%
\pgfpathcurveto{\pgfqpoint{0.816667in}{0.995580in}}{\pgfqpoint{0.818423in}{0.991340in}}{\pgfqpoint{0.821548in}{0.988215in}}%
\pgfpathcurveto{\pgfqpoint{0.824674in}{0.985089in}}{\pgfqpoint{0.828913in}{0.983333in}}{\pgfqpoint{0.833333in}{0.983333in}}%
\pgfpathclose%
\pgfpathmoveto{\pgfqpoint{1.000000in}{0.983333in}}%
\pgfpathcurveto{\pgfqpoint{1.004420in}{0.983333in}}{\pgfqpoint{1.008660in}{0.985089in}}{\pgfqpoint{1.011785in}{0.988215in}}%
\pgfpathcurveto{\pgfqpoint{1.014911in}{0.991340in}}{\pgfqpoint{1.016667in}{0.995580in}}{\pgfqpoint{1.016667in}{1.000000in}}%
\pgfpathcurveto{\pgfqpoint{1.016667in}{1.004420in}}{\pgfqpoint{1.014911in}{1.008660in}}{\pgfqpoint{1.011785in}{1.011785in}}%
\pgfpathcurveto{\pgfqpoint{1.008660in}{1.014911in}}{\pgfqpoint{1.004420in}{1.016667in}}{\pgfqpoint{1.000000in}{1.016667in}}%
\pgfpathcurveto{\pgfqpoint{0.995580in}{1.016667in}}{\pgfqpoint{0.991340in}{1.014911in}}{\pgfqpoint{0.988215in}{1.011785in}}%
\pgfpathcurveto{\pgfqpoint{0.985089in}{1.008660in}}{\pgfqpoint{0.983333in}{1.004420in}}{\pgfqpoint{0.983333in}{1.000000in}}%
\pgfpathcurveto{\pgfqpoint{0.983333in}{0.995580in}}{\pgfqpoint{0.985089in}{0.991340in}}{\pgfqpoint{0.988215in}{0.988215in}}%
\pgfpathcurveto{\pgfqpoint{0.991340in}{0.985089in}}{\pgfqpoint{0.995580in}{0.983333in}}{\pgfqpoint{1.000000in}{0.983333in}}%
\pgfpathclose%
\pgfusepath{stroke}%
\end{pgfscope}%
}%
\pgfsys@transformshift{5.973315in}{4.921635in}%
\pgfsys@useobject{currentpattern}{}%
\pgfsys@transformshift{1in}{0in}%
\pgfsys@transformshift{-1in}{0in}%
\pgfsys@transformshift{0in}{1in}%
\end{pgfscope}%
\begin{pgfscope}%
\pgfpathrectangle{\pgfqpoint{0.935815in}{0.637495in}}{\pgfqpoint{9.300000in}{9.060000in}}%
\pgfusepath{clip}%
\pgfsetbuttcap%
\pgfsetmiterjoin%
\definecolor{currentfill}{rgb}{0.172549,0.627451,0.172549}%
\pgfsetfillcolor{currentfill}%
\pgfsetfillopacity{0.990000}%
\pgfsetlinewidth{0.000000pt}%
\definecolor{currentstroke}{rgb}{0.000000,0.000000,0.000000}%
\pgfsetstrokecolor{currentstroke}%
\pgfsetstrokeopacity{0.990000}%
\pgfsetdash{}{0pt}%
\pgfpathmoveto{\pgfqpoint{7.523315in}{5.046572in}}%
\pgfpathlineto{\pgfqpoint{8.298315in}{5.046572in}}%
\pgfpathlineto{\pgfqpoint{8.298315in}{5.549528in}}%
\pgfpathlineto{\pgfqpoint{7.523315in}{5.549528in}}%
\pgfpathclose%
\pgfusepath{fill}%
\end{pgfscope}%
\begin{pgfscope}%
\pgfsetbuttcap%
\pgfsetmiterjoin%
\definecolor{currentfill}{rgb}{0.172549,0.627451,0.172549}%
\pgfsetfillcolor{currentfill}%
\pgfsetfillopacity{0.990000}%
\pgfsetlinewidth{0.000000pt}%
\definecolor{currentstroke}{rgb}{0.000000,0.000000,0.000000}%
\pgfsetstrokecolor{currentstroke}%
\pgfsetstrokeopacity{0.990000}%
\pgfsetdash{}{0pt}%
\pgfpathrectangle{\pgfqpoint{0.935815in}{0.637495in}}{\pgfqpoint{9.300000in}{9.060000in}}%
\pgfusepath{clip}%
\pgfpathmoveto{\pgfqpoint{7.523315in}{5.046572in}}%
\pgfpathlineto{\pgfqpoint{8.298315in}{5.046572in}}%
\pgfpathlineto{\pgfqpoint{8.298315in}{5.549528in}}%
\pgfpathlineto{\pgfqpoint{7.523315in}{5.549528in}}%
\pgfpathclose%
\pgfusepath{clip}%
\pgfsys@defobject{currentpattern}{\pgfqpoint{0in}{0in}}{\pgfqpoint{1in}{1in}}{%
\begin{pgfscope}%
\pgfpathrectangle{\pgfqpoint{0in}{0in}}{\pgfqpoint{1in}{1in}}%
\pgfusepath{clip}%
\pgfpathmoveto{\pgfqpoint{0.000000in}{-0.016667in}}%
\pgfpathcurveto{\pgfqpoint{0.004420in}{-0.016667in}}{\pgfqpoint{0.008660in}{-0.014911in}}{\pgfqpoint{0.011785in}{-0.011785in}}%
\pgfpathcurveto{\pgfqpoint{0.014911in}{-0.008660in}}{\pgfqpoint{0.016667in}{-0.004420in}}{\pgfqpoint{0.016667in}{0.000000in}}%
\pgfpathcurveto{\pgfqpoint{0.016667in}{0.004420in}}{\pgfqpoint{0.014911in}{0.008660in}}{\pgfqpoint{0.011785in}{0.011785in}}%
\pgfpathcurveto{\pgfqpoint{0.008660in}{0.014911in}}{\pgfqpoint{0.004420in}{0.016667in}}{\pgfqpoint{0.000000in}{0.016667in}}%
\pgfpathcurveto{\pgfqpoint{-0.004420in}{0.016667in}}{\pgfqpoint{-0.008660in}{0.014911in}}{\pgfqpoint{-0.011785in}{0.011785in}}%
\pgfpathcurveto{\pgfqpoint{-0.014911in}{0.008660in}}{\pgfqpoint{-0.016667in}{0.004420in}}{\pgfqpoint{-0.016667in}{0.000000in}}%
\pgfpathcurveto{\pgfqpoint{-0.016667in}{-0.004420in}}{\pgfqpoint{-0.014911in}{-0.008660in}}{\pgfqpoint{-0.011785in}{-0.011785in}}%
\pgfpathcurveto{\pgfqpoint{-0.008660in}{-0.014911in}}{\pgfqpoint{-0.004420in}{-0.016667in}}{\pgfqpoint{0.000000in}{-0.016667in}}%
\pgfpathclose%
\pgfpathmoveto{\pgfqpoint{0.166667in}{-0.016667in}}%
\pgfpathcurveto{\pgfqpoint{0.171087in}{-0.016667in}}{\pgfqpoint{0.175326in}{-0.014911in}}{\pgfqpoint{0.178452in}{-0.011785in}}%
\pgfpathcurveto{\pgfqpoint{0.181577in}{-0.008660in}}{\pgfqpoint{0.183333in}{-0.004420in}}{\pgfqpoint{0.183333in}{0.000000in}}%
\pgfpathcurveto{\pgfqpoint{0.183333in}{0.004420in}}{\pgfqpoint{0.181577in}{0.008660in}}{\pgfqpoint{0.178452in}{0.011785in}}%
\pgfpathcurveto{\pgfqpoint{0.175326in}{0.014911in}}{\pgfqpoint{0.171087in}{0.016667in}}{\pgfqpoint{0.166667in}{0.016667in}}%
\pgfpathcurveto{\pgfqpoint{0.162247in}{0.016667in}}{\pgfqpoint{0.158007in}{0.014911in}}{\pgfqpoint{0.154882in}{0.011785in}}%
\pgfpathcurveto{\pgfqpoint{0.151756in}{0.008660in}}{\pgfqpoint{0.150000in}{0.004420in}}{\pgfqpoint{0.150000in}{0.000000in}}%
\pgfpathcurveto{\pgfqpoint{0.150000in}{-0.004420in}}{\pgfqpoint{0.151756in}{-0.008660in}}{\pgfqpoint{0.154882in}{-0.011785in}}%
\pgfpathcurveto{\pgfqpoint{0.158007in}{-0.014911in}}{\pgfqpoint{0.162247in}{-0.016667in}}{\pgfqpoint{0.166667in}{-0.016667in}}%
\pgfpathclose%
\pgfpathmoveto{\pgfqpoint{0.333333in}{-0.016667in}}%
\pgfpathcurveto{\pgfqpoint{0.337753in}{-0.016667in}}{\pgfqpoint{0.341993in}{-0.014911in}}{\pgfqpoint{0.345118in}{-0.011785in}}%
\pgfpathcurveto{\pgfqpoint{0.348244in}{-0.008660in}}{\pgfqpoint{0.350000in}{-0.004420in}}{\pgfqpoint{0.350000in}{0.000000in}}%
\pgfpathcurveto{\pgfqpoint{0.350000in}{0.004420in}}{\pgfqpoint{0.348244in}{0.008660in}}{\pgfqpoint{0.345118in}{0.011785in}}%
\pgfpathcurveto{\pgfqpoint{0.341993in}{0.014911in}}{\pgfqpoint{0.337753in}{0.016667in}}{\pgfqpoint{0.333333in}{0.016667in}}%
\pgfpathcurveto{\pgfqpoint{0.328913in}{0.016667in}}{\pgfqpoint{0.324674in}{0.014911in}}{\pgfqpoint{0.321548in}{0.011785in}}%
\pgfpathcurveto{\pgfqpoint{0.318423in}{0.008660in}}{\pgfqpoint{0.316667in}{0.004420in}}{\pgfqpoint{0.316667in}{0.000000in}}%
\pgfpathcurveto{\pgfqpoint{0.316667in}{-0.004420in}}{\pgfqpoint{0.318423in}{-0.008660in}}{\pgfqpoint{0.321548in}{-0.011785in}}%
\pgfpathcurveto{\pgfqpoint{0.324674in}{-0.014911in}}{\pgfqpoint{0.328913in}{-0.016667in}}{\pgfqpoint{0.333333in}{-0.016667in}}%
\pgfpathclose%
\pgfpathmoveto{\pgfqpoint{0.500000in}{-0.016667in}}%
\pgfpathcurveto{\pgfqpoint{0.504420in}{-0.016667in}}{\pgfqpoint{0.508660in}{-0.014911in}}{\pgfqpoint{0.511785in}{-0.011785in}}%
\pgfpathcurveto{\pgfqpoint{0.514911in}{-0.008660in}}{\pgfqpoint{0.516667in}{-0.004420in}}{\pgfqpoint{0.516667in}{0.000000in}}%
\pgfpathcurveto{\pgfqpoint{0.516667in}{0.004420in}}{\pgfqpoint{0.514911in}{0.008660in}}{\pgfqpoint{0.511785in}{0.011785in}}%
\pgfpathcurveto{\pgfqpoint{0.508660in}{0.014911in}}{\pgfqpoint{0.504420in}{0.016667in}}{\pgfqpoint{0.500000in}{0.016667in}}%
\pgfpathcurveto{\pgfqpoint{0.495580in}{0.016667in}}{\pgfqpoint{0.491340in}{0.014911in}}{\pgfqpoint{0.488215in}{0.011785in}}%
\pgfpathcurveto{\pgfqpoint{0.485089in}{0.008660in}}{\pgfqpoint{0.483333in}{0.004420in}}{\pgfqpoint{0.483333in}{0.000000in}}%
\pgfpathcurveto{\pgfqpoint{0.483333in}{-0.004420in}}{\pgfqpoint{0.485089in}{-0.008660in}}{\pgfqpoint{0.488215in}{-0.011785in}}%
\pgfpathcurveto{\pgfqpoint{0.491340in}{-0.014911in}}{\pgfqpoint{0.495580in}{-0.016667in}}{\pgfqpoint{0.500000in}{-0.016667in}}%
\pgfpathclose%
\pgfpathmoveto{\pgfqpoint{0.666667in}{-0.016667in}}%
\pgfpathcurveto{\pgfqpoint{0.671087in}{-0.016667in}}{\pgfqpoint{0.675326in}{-0.014911in}}{\pgfqpoint{0.678452in}{-0.011785in}}%
\pgfpathcurveto{\pgfqpoint{0.681577in}{-0.008660in}}{\pgfqpoint{0.683333in}{-0.004420in}}{\pgfqpoint{0.683333in}{0.000000in}}%
\pgfpathcurveto{\pgfqpoint{0.683333in}{0.004420in}}{\pgfqpoint{0.681577in}{0.008660in}}{\pgfqpoint{0.678452in}{0.011785in}}%
\pgfpathcurveto{\pgfqpoint{0.675326in}{0.014911in}}{\pgfqpoint{0.671087in}{0.016667in}}{\pgfqpoint{0.666667in}{0.016667in}}%
\pgfpathcurveto{\pgfqpoint{0.662247in}{0.016667in}}{\pgfqpoint{0.658007in}{0.014911in}}{\pgfqpoint{0.654882in}{0.011785in}}%
\pgfpathcurveto{\pgfqpoint{0.651756in}{0.008660in}}{\pgfqpoint{0.650000in}{0.004420in}}{\pgfqpoint{0.650000in}{0.000000in}}%
\pgfpathcurveto{\pgfqpoint{0.650000in}{-0.004420in}}{\pgfqpoint{0.651756in}{-0.008660in}}{\pgfqpoint{0.654882in}{-0.011785in}}%
\pgfpathcurveto{\pgfqpoint{0.658007in}{-0.014911in}}{\pgfqpoint{0.662247in}{-0.016667in}}{\pgfqpoint{0.666667in}{-0.016667in}}%
\pgfpathclose%
\pgfpathmoveto{\pgfqpoint{0.833333in}{-0.016667in}}%
\pgfpathcurveto{\pgfqpoint{0.837753in}{-0.016667in}}{\pgfqpoint{0.841993in}{-0.014911in}}{\pgfqpoint{0.845118in}{-0.011785in}}%
\pgfpathcurveto{\pgfqpoint{0.848244in}{-0.008660in}}{\pgfqpoint{0.850000in}{-0.004420in}}{\pgfqpoint{0.850000in}{0.000000in}}%
\pgfpathcurveto{\pgfqpoint{0.850000in}{0.004420in}}{\pgfqpoint{0.848244in}{0.008660in}}{\pgfqpoint{0.845118in}{0.011785in}}%
\pgfpathcurveto{\pgfqpoint{0.841993in}{0.014911in}}{\pgfqpoint{0.837753in}{0.016667in}}{\pgfqpoint{0.833333in}{0.016667in}}%
\pgfpathcurveto{\pgfqpoint{0.828913in}{0.016667in}}{\pgfqpoint{0.824674in}{0.014911in}}{\pgfqpoint{0.821548in}{0.011785in}}%
\pgfpathcurveto{\pgfqpoint{0.818423in}{0.008660in}}{\pgfqpoint{0.816667in}{0.004420in}}{\pgfqpoint{0.816667in}{0.000000in}}%
\pgfpathcurveto{\pgfqpoint{0.816667in}{-0.004420in}}{\pgfqpoint{0.818423in}{-0.008660in}}{\pgfqpoint{0.821548in}{-0.011785in}}%
\pgfpathcurveto{\pgfqpoint{0.824674in}{-0.014911in}}{\pgfqpoint{0.828913in}{-0.016667in}}{\pgfqpoint{0.833333in}{-0.016667in}}%
\pgfpathclose%
\pgfpathmoveto{\pgfqpoint{1.000000in}{-0.016667in}}%
\pgfpathcurveto{\pgfqpoint{1.004420in}{-0.016667in}}{\pgfqpoint{1.008660in}{-0.014911in}}{\pgfqpoint{1.011785in}{-0.011785in}}%
\pgfpathcurveto{\pgfqpoint{1.014911in}{-0.008660in}}{\pgfqpoint{1.016667in}{-0.004420in}}{\pgfqpoint{1.016667in}{0.000000in}}%
\pgfpathcurveto{\pgfqpoint{1.016667in}{0.004420in}}{\pgfqpoint{1.014911in}{0.008660in}}{\pgfqpoint{1.011785in}{0.011785in}}%
\pgfpathcurveto{\pgfqpoint{1.008660in}{0.014911in}}{\pgfqpoint{1.004420in}{0.016667in}}{\pgfqpoint{1.000000in}{0.016667in}}%
\pgfpathcurveto{\pgfqpoint{0.995580in}{0.016667in}}{\pgfqpoint{0.991340in}{0.014911in}}{\pgfqpoint{0.988215in}{0.011785in}}%
\pgfpathcurveto{\pgfqpoint{0.985089in}{0.008660in}}{\pgfqpoint{0.983333in}{0.004420in}}{\pgfqpoint{0.983333in}{0.000000in}}%
\pgfpathcurveto{\pgfqpoint{0.983333in}{-0.004420in}}{\pgfqpoint{0.985089in}{-0.008660in}}{\pgfqpoint{0.988215in}{-0.011785in}}%
\pgfpathcurveto{\pgfqpoint{0.991340in}{-0.014911in}}{\pgfqpoint{0.995580in}{-0.016667in}}{\pgfqpoint{1.000000in}{-0.016667in}}%
\pgfpathclose%
\pgfpathmoveto{\pgfqpoint{0.083333in}{0.150000in}}%
\pgfpathcurveto{\pgfqpoint{0.087753in}{0.150000in}}{\pgfqpoint{0.091993in}{0.151756in}}{\pgfqpoint{0.095118in}{0.154882in}}%
\pgfpathcurveto{\pgfqpoint{0.098244in}{0.158007in}}{\pgfqpoint{0.100000in}{0.162247in}}{\pgfqpoint{0.100000in}{0.166667in}}%
\pgfpathcurveto{\pgfqpoint{0.100000in}{0.171087in}}{\pgfqpoint{0.098244in}{0.175326in}}{\pgfqpoint{0.095118in}{0.178452in}}%
\pgfpathcurveto{\pgfqpoint{0.091993in}{0.181577in}}{\pgfqpoint{0.087753in}{0.183333in}}{\pgfqpoint{0.083333in}{0.183333in}}%
\pgfpathcurveto{\pgfqpoint{0.078913in}{0.183333in}}{\pgfqpoint{0.074674in}{0.181577in}}{\pgfqpoint{0.071548in}{0.178452in}}%
\pgfpathcurveto{\pgfqpoint{0.068423in}{0.175326in}}{\pgfqpoint{0.066667in}{0.171087in}}{\pgfqpoint{0.066667in}{0.166667in}}%
\pgfpathcurveto{\pgfqpoint{0.066667in}{0.162247in}}{\pgfqpoint{0.068423in}{0.158007in}}{\pgfqpoint{0.071548in}{0.154882in}}%
\pgfpathcurveto{\pgfqpoint{0.074674in}{0.151756in}}{\pgfqpoint{0.078913in}{0.150000in}}{\pgfqpoint{0.083333in}{0.150000in}}%
\pgfpathclose%
\pgfpathmoveto{\pgfqpoint{0.250000in}{0.150000in}}%
\pgfpathcurveto{\pgfqpoint{0.254420in}{0.150000in}}{\pgfqpoint{0.258660in}{0.151756in}}{\pgfqpoint{0.261785in}{0.154882in}}%
\pgfpathcurveto{\pgfqpoint{0.264911in}{0.158007in}}{\pgfqpoint{0.266667in}{0.162247in}}{\pgfqpoint{0.266667in}{0.166667in}}%
\pgfpathcurveto{\pgfqpoint{0.266667in}{0.171087in}}{\pgfqpoint{0.264911in}{0.175326in}}{\pgfqpoint{0.261785in}{0.178452in}}%
\pgfpathcurveto{\pgfqpoint{0.258660in}{0.181577in}}{\pgfqpoint{0.254420in}{0.183333in}}{\pgfqpoint{0.250000in}{0.183333in}}%
\pgfpathcurveto{\pgfqpoint{0.245580in}{0.183333in}}{\pgfqpoint{0.241340in}{0.181577in}}{\pgfqpoint{0.238215in}{0.178452in}}%
\pgfpathcurveto{\pgfqpoint{0.235089in}{0.175326in}}{\pgfqpoint{0.233333in}{0.171087in}}{\pgfqpoint{0.233333in}{0.166667in}}%
\pgfpathcurveto{\pgfqpoint{0.233333in}{0.162247in}}{\pgfqpoint{0.235089in}{0.158007in}}{\pgfqpoint{0.238215in}{0.154882in}}%
\pgfpathcurveto{\pgfqpoint{0.241340in}{0.151756in}}{\pgfqpoint{0.245580in}{0.150000in}}{\pgfqpoint{0.250000in}{0.150000in}}%
\pgfpathclose%
\pgfpathmoveto{\pgfqpoint{0.416667in}{0.150000in}}%
\pgfpathcurveto{\pgfqpoint{0.421087in}{0.150000in}}{\pgfqpoint{0.425326in}{0.151756in}}{\pgfqpoint{0.428452in}{0.154882in}}%
\pgfpathcurveto{\pgfqpoint{0.431577in}{0.158007in}}{\pgfqpoint{0.433333in}{0.162247in}}{\pgfqpoint{0.433333in}{0.166667in}}%
\pgfpathcurveto{\pgfqpoint{0.433333in}{0.171087in}}{\pgfqpoint{0.431577in}{0.175326in}}{\pgfqpoint{0.428452in}{0.178452in}}%
\pgfpathcurveto{\pgfqpoint{0.425326in}{0.181577in}}{\pgfqpoint{0.421087in}{0.183333in}}{\pgfqpoint{0.416667in}{0.183333in}}%
\pgfpathcurveto{\pgfqpoint{0.412247in}{0.183333in}}{\pgfqpoint{0.408007in}{0.181577in}}{\pgfqpoint{0.404882in}{0.178452in}}%
\pgfpathcurveto{\pgfqpoint{0.401756in}{0.175326in}}{\pgfqpoint{0.400000in}{0.171087in}}{\pgfqpoint{0.400000in}{0.166667in}}%
\pgfpathcurveto{\pgfqpoint{0.400000in}{0.162247in}}{\pgfqpoint{0.401756in}{0.158007in}}{\pgfqpoint{0.404882in}{0.154882in}}%
\pgfpathcurveto{\pgfqpoint{0.408007in}{0.151756in}}{\pgfqpoint{0.412247in}{0.150000in}}{\pgfqpoint{0.416667in}{0.150000in}}%
\pgfpathclose%
\pgfpathmoveto{\pgfqpoint{0.583333in}{0.150000in}}%
\pgfpathcurveto{\pgfqpoint{0.587753in}{0.150000in}}{\pgfqpoint{0.591993in}{0.151756in}}{\pgfqpoint{0.595118in}{0.154882in}}%
\pgfpathcurveto{\pgfqpoint{0.598244in}{0.158007in}}{\pgfqpoint{0.600000in}{0.162247in}}{\pgfqpoint{0.600000in}{0.166667in}}%
\pgfpathcurveto{\pgfqpoint{0.600000in}{0.171087in}}{\pgfqpoint{0.598244in}{0.175326in}}{\pgfqpoint{0.595118in}{0.178452in}}%
\pgfpathcurveto{\pgfqpoint{0.591993in}{0.181577in}}{\pgfqpoint{0.587753in}{0.183333in}}{\pgfqpoint{0.583333in}{0.183333in}}%
\pgfpathcurveto{\pgfqpoint{0.578913in}{0.183333in}}{\pgfqpoint{0.574674in}{0.181577in}}{\pgfqpoint{0.571548in}{0.178452in}}%
\pgfpathcurveto{\pgfqpoint{0.568423in}{0.175326in}}{\pgfqpoint{0.566667in}{0.171087in}}{\pgfqpoint{0.566667in}{0.166667in}}%
\pgfpathcurveto{\pgfqpoint{0.566667in}{0.162247in}}{\pgfqpoint{0.568423in}{0.158007in}}{\pgfqpoint{0.571548in}{0.154882in}}%
\pgfpathcurveto{\pgfqpoint{0.574674in}{0.151756in}}{\pgfqpoint{0.578913in}{0.150000in}}{\pgfqpoint{0.583333in}{0.150000in}}%
\pgfpathclose%
\pgfpathmoveto{\pgfqpoint{0.750000in}{0.150000in}}%
\pgfpathcurveto{\pgfqpoint{0.754420in}{0.150000in}}{\pgfqpoint{0.758660in}{0.151756in}}{\pgfqpoint{0.761785in}{0.154882in}}%
\pgfpathcurveto{\pgfqpoint{0.764911in}{0.158007in}}{\pgfqpoint{0.766667in}{0.162247in}}{\pgfqpoint{0.766667in}{0.166667in}}%
\pgfpathcurveto{\pgfqpoint{0.766667in}{0.171087in}}{\pgfqpoint{0.764911in}{0.175326in}}{\pgfqpoint{0.761785in}{0.178452in}}%
\pgfpathcurveto{\pgfqpoint{0.758660in}{0.181577in}}{\pgfqpoint{0.754420in}{0.183333in}}{\pgfqpoint{0.750000in}{0.183333in}}%
\pgfpathcurveto{\pgfqpoint{0.745580in}{0.183333in}}{\pgfqpoint{0.741340in}{0.181577in}}{\pgfqpoint{0.738215in}{0.178452in}}%
\pgfpathcurveto{\pgfqpoint{0.735089in}{0.175326in}}{\pgfqpoint{0.733333in}{0.171087in}}{\pgfqpoint{0.733333in}{0.166667in}}%
\pgfpathcurveto{\pgfqpoint{0.733333in}{0.162247in}}{\pgfqpoint{0.735089in}{0.158007in}}{\pgfqpoint{0.738215in}{0.154882in}}%
\pgfpathcurveto{\pgfqpoint{0.741340in}{0.151756in}}{\pgfqpoint{0.745580in}{0.150000in}}{\pgfqpoint{0.750000in}{0.150000in}}%
\pgfpathclose%
\pgfpathmoveto{\pgfqpoint{0.916667in}{0.150000in}}%
\pgfpathcurveto{\pgfqpoint{0.921087in}{0.150000in}}{\pgfqpoint{0.925326in}{0.151756in}}{\pgfqpoint{0.928452in}{0.154882in}}%
\pgfpathcurveto{\pgfqpoint{0.931577in}{0.158007in}}{\pgfqpoint{0.933333in}{0.162247in}}{\pgfqpoint{0.933333in}{0.166667in}}%
\pgfpathcurveto{\pgfqpoint{0.933333in}{0.171087in}}{\pgfqpoint{0.931577in}{0.175326in}}{\pgfqpoint{0.928452in}{0.178452in}}%
\pgfpathcurveto{\pgfqpoint{0.925326in}{0.181577in}}{\pgfqpoint{0.921087in}{0.183333in}}{\pgfqpoint{0.916667in}{0.183333in}}%
\pgfpathcurveto{\pgfqpoint{0.912247in}{0.183333in}}{\pgfqpoint{0.908007in}{0.181577in}}{\pgfqpoint{0.904882in}{0.178452in}}%
\pgfpathcurveto{\pgfqpoint{0.901756in}{0.175326in}}{\pgfqpoint{0.900000in}{0.171087in}}{\pgfqpoint{0.900000in}{0.166667in}}%
\pgfpathcurveto{\pgfqpoint{0.900000in}{0.162247in}}{\pgfqpoint{0.901756in}{0.158007in}}{\pgfqpoint{0.904882in}{0.154882in}}%
\pgfpathcurveto{\pgfqpoint{0.908007in}{0.151756in}}{\pgfqpoint{0.912247in}{0.150000in}}{\pgfqpoint{0.916667in}{0.150000in}}%
\pgfpathclose%
\pgfpathmoveto{\pgfqpoint{0.000000in}{0.316667in}}%
\pgfpathcurveto{\pgfqpoint{0.004420in}{0.316667in}}{\pgfqpoint{0.008660in}{0.318423in}}{\pgfqpoint{0.011785in}{0.321548in}}%
\pgfpathcurveto{\pgfqpoint{0.014911in}{0.324674in}}{\pgfqpoint{0.016667in}{0.328913in}}{\pgfqpoint{0.016667in}{0.333333in}}%
\pgfpathcurveto{\pgfqpoint{0.016667in}{0.337753in}}{\pgfqpoint{0.014911in}{0.341993in}}{\pgfqpoint{0.011785in}{0.345118in}}%
\pgfpathcurveto{\pgfqpoint{0.008660in}{0.348244in}}{\pgfqpoint{0.004420in}{0.350000in}}{\pgfqpoint{0.000000in}{0.350000in}}%
\pgfpathcurveto{\pgfqpoint{-0.004420in}{0.350000in}}{\pgfqpoint{-0.008660in}{0.348244in}}{\pgfqpoint{-0.011785in}{0.345118in}}%
\pgfpathcurveto{\pgfqpoint{-0.014911in}{0.341993in}}{\pgfqpoint{-0.016667in}{0.337753in}}{\pgfqpoint{-0.016667in}{0.333333in}}%
\pgfpathcurveto{\pgfqpoint{-0.016667in}{0.328913in}}{\pgfqpoint{-0.014911in}{0.324674in}}{\pgfqpoint{-0.011785in}{0.321548in}}%
\pgfpathcurveto{\pgfqpoint{-0.008660in}{0.318423in}}{\pgfqpoint{-0.004420in}{0.316667in}}{\pgfqpoint{0.000000in}{0.316667in}}%
\pgfpathclose%
\pgfpathmoveto{\pgfqpoint{0.166667in}{0.316667in}}%
\pgfpathcurveto{\pgfqpoint{0.171087in}{0.316667in}}{\pgfqpoint{0.175326in}{0.318423in}}{\pgfqpoint{0.178452in}{0.321548in}}%
\pgfpathcurveto{\pgfqpoint{0.181577in}{0.324674in}}{\pgfqpoint{0.183333in}{0.328913in}}{\pgfqpoint{0.183333in}{0.333333in}}%
\pgfpathcurveto{\pgfqpoint{0.183333in}{0.337753in}}{\pgfqpoint{0.181577in}{0.341993in}}{\pgfqpoint{0.178452in}{0.345118in}}%
\pgfpathcurveto{\pgfqpoint{0.175326in}{0.348244in}}{\pgfqpoint{0.171087in}{0.350000in}}{\pgfqpoint{0.166667in}{0.350000in}}%
\pgfpathcurveto{\pgfqpoint{0.162247in}{0.350000in}}{\pgfqpoint{0.158007in}{0.348244in}}{\pgfqpoint{0.154882in}{0.345118in}}%
\pgfpathcurveto{\pgfqpoint{0.151756in}{0.341993in}}{\pgfqpoint{0.150000in}{0.337753in}}{\pgfqpoint{0.150000in}{0.333333in}}%
\pgfpathcurveto{\pgfqpoint{0.150000in}{0.328913in}}{\pgfqpoint{0.151756in}{0.324674in}}{\pgfqpoint{0.154882in}{0.321548in}}%
\pgfpathcurveto{\pgfqpoint{0.158007in}{0.318423in}}{\pgfqpoint{0.162247in}{0.316667in}}{\pgfqpoint{0.166667in}{0.316667in}}%
\pgfpathclose%
\pgfpathmoveto{\pgfqpoint{0.333333in}{0.316667in}}%
\pgfpathcurveto{\pgfqpoint{0.337753in}{0.316667in}}{\pgfqpoint{0.341993in}{0.318423in}}{\pgfqpoint{0.345118in}{0.321548in}}%
\pgfpathcurveto{\pgfqpoint{0.348244in}{0.324674in}}{\pgfqpoint{0.350000in}{0.328913in}}{\pgfqpoint{0.350000in}{0.333333in}}%
\pgfpathcurveto{\pgfqpoint{0.350000in}{0.337753in}}{\pgfqpoint{0.348244in}{0.341993in}}{\pgfqpoint{0.345118in}{0.345118in}}%
\pgfpathcurveto{\pgfqpoint{0.341993in}{0.348244in}}{\pgfqpoint{0.337753in}{0.350000in}}{\pgfqpoint{0.333333in}{0.350000in}}%
\pgfpathcurveto{\pgfqpoint{0.328913in}{0.350000in}}{\pgfqpoint{0.324674in}{0.348244in}}{\pgfqpoint{0.321548in}{0.345118in}}%
\pgfpathcurveto{\pgfqpoint{0.318423in}{0.341993in}}{\pgfqpoint{0.316667in}{0.337753in}}{\pgfqpoint{0.316667in}{0.333333in}}%
\pgfpathcurveto{\pgfqpoint{0.316667in}{0.328913in}}{\pgfqpoint{0.318423in}{0.324674in}}{\pgfqpoint{0.321548in}{0.321548in}}%
\pgfpathcurveto{\pgfqpoint{0.324674in}{0.318423in}}{\pgfqpoint{0.328913in}{0.316667in}}{\pgfqpoint{0.333333in}{0.316667in}}%
\pgfpathclose%
\pgfpathmoveto{\pgfqpoint{0.500000in}{0.316667in}}%
\pgfpathcurveto{\pgfqpoint{0.504420in}{0.316667in}}{\pgfqpoint{0.508660in}{0.318423in}}{\pgfqpoint{0.511785in}{0.321548in}}%
\pgfpathcurveto{\pgfqpoint{0.514911in}{0.324674in}}{\pgfqpoint{0.516667in}{0.328913in}}{\pgfqpoint{0.516667in}{0.333333in}}%
\pgfpathcurveto{\pgfqpoint{0.516667in}{0.337753in}}{\pgfqpoint{0.514911in}{0.341993in}}{\pgfqpoint{0.511785in}{0.345118in}}%
\pgfpathcurveto{\pgfqpoint{0.508660in}{0.348244in}}{\pgfqpoint{0.504420in}{0.350000in}}{\pgfqpoint{0.500000in}{0.350000in}}%
\pgfpathcurveto{\pgfqpoint{0.495580in}{0.350000in}}{\pgfqpoint{0.491340in}{0.348244in}}{\pgfqpoint{0.488215in}{0.345118in}}%
\pgfpathcurveto{\pgfqpoint{0.485089in}{0.341993in}}{\pgfqpoint{0.483333in}{0.337753in}}{\pgfqpoint{0.483333in}{0.333333in}}%
\pgfpathcurveto{\pgfqpoint{0.483333in}{0.328913in}}{\pgfqpoint{0.485089in}{0.324674in}}{\pgfqpoint{0.488215in}{0.321548in}}%
\pgfpathcurveto{\pgfqpoint{0.491340in}{0.318423in}}{\pgfqpoint{0.495580in}{0.316667in}}{\pgfqpoint{0.500000in}{0.316667in}}%
\pgfpathclose%
\pgfpathmoveto{\pgfqpoint{0.666667in}{0.316667in}}%
\pgfpathcurveto{\pgfqpoint{0.671087in}{0.316667in}}{\pgfqpoint{0.675326in}{0.318423in}}{\pgfqpoint{0.678452in}{0.321548in}}%
\pgfpathcurveto{\pgfqpoint{0.681577in}{0.324674in}}{\pgfqpoint{0.683333in}{0.328913in}}{\pgfqpoint{0.683333in}{0.333333in}}%
\pgfpathcurveto{\pgfqpoint{0.683333in}{0.337753in}}{\pgfqpoint{0.681577in}{0.341993in}}{\pgfqpoint{0.678452in}{0.345118in}}%
\pgfpathcurveto{\pgfqpoint{0.675326in}{0.348244in}}{\pgfqpoint{0.671087in}{0.350000in}}{\pgfqpoint{0.666667in}{0.350000in}}%
\pgfpathcurveto{\pgfqpoint{0.662247in}{0.350000in}}{\pgfqpoint{0.658007in}{0.348244in}}{\pgfqpoint{0.654882in}{0.345118in}}%
\pgfpathcurveto{\pgfqpoint{0.651756in}{0.341993in}}{\pgfqpoint{0.650000in}{0.337753in}}{\pgfqpoint{0.650000in}{0.333333in}}%
\pgfpathcurveto{\pgfqpoint{0.650000in}{0.328913in}}{\pgfqpoint{0.651756in}{0.324674in}}{\pgfqpoint{0.654882in}{0.321548in}}%
\pgfpathcurveto{\pgfqpoint{0.658007in}{0.318423in}}{\pgfqpoint{0.662247in}{0.316667in}}{\pgfqpoint{0.666667in}{0.316667in}}%
\pgfpathclose%
\pgfpathmoveto{\pgfqpoint{0.833333in}{0.316667in}}%
\pgfpathcurveto{\pgfqpoint{0.837753in}{0.316667in}}{\pgfqpoint{0.841993in}{0.318423in}}{\pgfqpoint{0.845118in}{0.321548in}}%
\pgfpathcurveto{\pgfqpoint{0.848244in}{0.324674in}}{\pgfqpoint{0.850000in}{0.328913in}}{\pgfqpoint{0.850000in}{0.333333in}}%
\pgfpathcurveto{\pgfqpoint{0.850000in}{0.337753in}}{\pgfqpoint{0.848244in}{0.341993in}}{\pgfqpoint{0.845118in}{0.345118in}}%
\pgfpathcurveto{\pgfqpoint{0.841993in}{0.348244in}}{\pgfqpoint{0.837753in}{0.350000in}}{\pgfqpoint{0.833333in}{0.350000in}}%
\pgfpathcurveto{\pgfqpoint{0.828913in}{0.350000in}}{\pgfqpoint{0.824674in}{0.348244in}}{\pgfqpoint{0.821548in}{0.345118in}}%
\pgfpathcurveto{\pgfqpoint{0.818423in}{0.341993in}}{\pgfqpoint{0.816667in}{0.337753in}}{\pgfqpoint{0.816667in}{0.333333in}}%
\pgfpathcurveto{\pgfqpoint{0.816667in}{0.328913in}}{\pgfqpoint{0.818423in}{0.324674in}}{\pgfqpoint{0.821548in}{0.321548in}}%
\pgfpathcurveto{\pgfqpoint{0.824674in}{0.318423in}}{\pgfqpoint{0.828913in}{0.316667in}}{\pgfqpoint{0.833333in}{0.316667in}}%
\pgfpathclose%
\pgfpathmoveto{\pgfqpoint{1.000000in}{0.316667in}}%
\pgfpathcurveto{\pgfqpoint{1.004420in}{0.316667in}}{\pgfqpoint{1.008660in}{0.318423in}}{\pgfqpoint{1.011785in}{0.321548in}}%
\pgfpathcurveto{\pgfqpoint{1.014911in}{0.324674in}}{\pgfqpoint{1.016667in}{0.328913in}}{\pgfqpoint{1.016667in}{0.333333in}}%
\pgfpathcurveto{\pgfqpoint{1.016667in}{0.337753in}}{\pgfqpoint{1.014911in}{0.341993in}}{\pgfqpoint{1.011785in}{0.345118in}}%
\pgfpathcurveto{\pgfqpoint{1.008660in}{0.348244in}}{\pgfqpoint{1.004420in}{0.350000in}}{\pgfqpoint{1.000000in}{0.350000in}}%
\pgfpathcurveto{\pgfqpoint{0.995580in}{0.350000in}}{\pgfqpoint{0.991340in}{0.348244in}}{\pgfqpoint{0.988215in}{0.345118in}}%
\pgfpathcurveto{\pgfqpoint{0.985089in}{0.341993in}}{\pgfqpoint{0.983333in}{0.337753in}}{\pgfqpoint{0.983333in}{0.333333in}}%
\pgfpathcurveto{\pgfqpoint{0.983333in}{0.328913in}}{\pgfqpoint{0.985089in}{0.324674in}}{\pgfqpoint{0.988215in}{0.321548in}}%
\pgfpathcurveto{\pgfqpoint{0.991340in}{0.318423in}}{\pgfqpoint{0.995580in}{0.316667in}}{\pgfqpoint{1.000000in}{0.316667in}}%
\pgfpathclose%
\pgfpathmoveto{\pgfqpoint{0.083333in}{0.483333in}}%
\pgfpathcurveto{\pgfqpoint{0.087753in}{0.483333in}}{\pgfqpoint{0.091993in}{0.485089in}}{\pgfqpoint{0.095118in}{0.488215in}}%
\pgfpathcurveto{\pgfqpoint{0.098244in}{0.491340in}}{\pgfqpoint{0.100000in}{0.495580in}}{\pgfqpoint{0.100000in}{0.500000in}}%
\pgfpathcurveto{\pgfqpoint{0.100000in}{0.504420in}}{\pgfqpoint{0.098244in}{0.508660in}}{\pgfqpoint{0.095118in}{0.511785in}}%
\pgfpathcurveto{\pgfqpoint{0.091993in}{0.514911in}}{\pgfqpoint{0.087753in}{0.516667in}}{\pgfqpoint{0.083333in}{0.516667in}}%
\pgfpathcurveto{\pgfqpoint{0.078913in}{0.516667in}}{\pgfqpoint{0.074674in}{0.514911in}}{\pgfqpoint{0.071548in}{0.511785in}}%
\pgfpathcurveto{\pgfqpoint{0.068423in}{0.508660in}}{\pgfqpoint{0.066667in}{0.504420in}}{\pgfqpoint{0.066667in}{0.500000in}}%
\pgfpathcurveto{\pgfqpoint{0.066667in}{0.495580in}}{\pgfqpoint{0.068423in}{0.491340in}}{\pgfqpoint{0.071548in}{0.488215in}}%
\pgfpathcurveto{\pgfqpoint{0.074674in}{0.485089in}}{\pgfqpoint{0.078913in}{0.483333in}}{\pgfqpoint{0.083333in}{0.483333in}}%
\pgfpathclose%
\pgfpathmoveto{\pgfqpoint{0.250000in}{0.483333in}}%
\pgfpathcurveto{\pgfqpoint{0.254420in}{0.483333in}}{\pgfqpoint{0.258660in}{0.485089in}}{\pgfqpoint{0.261785in}{0.488215in}}%
\pgfpathcurveto{\pgfqpoint{0.264911in}{0.491340in}}{\pgfqpoint{0.266667in}{0.495580in}}{\pgfqpoint{0.266667in}{0.500000in}}%
\pgfpathcurveto{\pgfqpoint{0.266667in}{0.504420in}}{\pgfqpoint{0.264911in}{0.508660in}}{\pgfqpoint{0.261785in}{0.511785in}}%
\pgfpathcurveto{\pgfqpoint{0.258660in}{0.514911in}}{\pgfqpoint{0.254420in}{0.516667in}}{\pgfqpoint{0.250000in}{0.516667in}}%
\pgfpathcurveto{\pgfqpoint{0.245580in}{0.516667in}}{\pgfqpoint{0.241340in}{0.514911in}}{\pgfqpoint{0.238215in}{0.511785in}}%
\pgfpathcurveto{\pgfqpoint{0.235089in}{0.508660in}}{\pgfqpoint{0.233333in}{0.504420in}}{\pgfqpoint{0.233333in}{0.500000in}}%
\pgfpathcurveto{\pgfqpoint{0.233333in}{0.495580in}}{\pgfqpoint{0.235089in}{0.491340in}}{\pgfqpoint{0.238215in}{0.488215in}}%
\pgfpathcurveto{\pgfqpoint{0.241340in}{0.485089in}}{\pgfqpoint{0.245580in}{0.483333in}}{\pgfqpoint{0.250000in}{0.483333in}}%
\pgfpathclose%
\pgfpathmoveto{\pgfqpoint{0.416667in}{0.483333in}}%
\pgfpathcurveto{\pgfqpoint{0.421087in}{0.483333in}}{\pgfqpoint{0.425326in}{0.485089in}}{\pgfqpoint{0.428452in}{0.488215in}}%
\pgfpathcurveto{\pgfqpoint{0.431577in}{0.491340in}}{\pgfqpoint{0.433333in}{0.495580in}}{\pgfqpoint{0.433333in}{0.500000in}}%
\pgfpathcurveto{\pgfqpoint{0.433333in}{0.504420in}}{\pgfqpoint{0.431577in}{0.508660in}}{\pgfqpoint{0.428452in}{0.511785in}}%
\pgfpathcurveto{\pgfqpoint{0.425326in}{0.514911in}}{\pgfqpoint{0.421087in}{0.516667in}}{\pgfqpoint{0.416667in}{0.516667in}}%
\pgfpathcurveto{\pgfqpoint{0.412247in}{0.516667in}}{\pgfqpoint{0.408007in}{0.514911in}}{\pgfqpoint{0.404882in}{0.511785in}}%
\pgfpathcurveto{\pgfqpoint{0.401756in}{0.508660in}}{\pgfqpoint{0.400000in}{0.504420in}}{\pgfqpoint{0.400000in}{0.500000in}}%
\pgfpathcurveto{\pgfqpoint{0.400000in}{0.495580in}}{\pgfqpoint{0.401756in}{0.491340in}}{\pgfqpoint{0.404882in}{0.488215in}}%
\pgfpathcurveto{\pgfqpoint{0.408007in}{0.485089in}}{\pgfqpoint{0.412247in}{0.483333in}}{\pgfqpoint{0.416667in}{0.483333in}}%
\pgfpathclose%
\pgfpathmoveto{\pgfqpoint{0.583333in}{0.483333in}}%
\pgfpathcurveto{\pgfqpoint{0.587753in}{0.483333in}}{\pgfqpoint{0.591993in}{0.485089in}}{\pgfqpoint{0.595118in}{0.488215in}}%
\pgfpathcurveto{\pgfqpoint{0.598244in}{0.491340in}}{\pgfqpoint{0.600000in}{0.495580in}}{\pgfqpoint{0.600000in}{0.500000in}}%
\pgfpathcurveto{\pgfqpoint{0.600000in}{0.504420in}}{\pgfqpoint{0.598244in}{0.508660in}}{\pgfqpoint{0.595118in}{0.511785in}}%
\pgfpathcurveto{\pgfqpoint{0.591993in}{0.514911in}}{\pgfqpoint{0.587753in}{0.516667in}}{\pgfqpoint{0.583333in}{0.516667in}}%
\pgfpathcurveto{\pgfqpoint{0.578913in}{0.516667in}}{\pgfqpoint{0.574674in}{0.514911in}}{\pgfqpoint{0.571548in}{0.511785in}}%
\pgfpathcurveto{\pgfqpoint{0.568423in}{0.508660in}}{\pgfqpoint{0.566667in}{0.504420in}}{\pgfqpoint{0.566667in}{0.500000in}}%
\pgfpathcurveto{\pgfqpoint{0.566667in}{0.495580in}}{\pgfqpoint{0.568423in}{0.491340in}}{\pgfqpoint{0.571548in}{0.488215in}}%
\pgfpathcurveto{\pgfqpoint{0.574674in}{0.485089in}}{\pgfqpoint{0.578913in}{0.483333in}}{\pgfqpoint{0.583333in}{0.483333in}}%
\pgfpathclose%
\pgfpathmoveto{\pgfqpoint{0.750000in}{0.483333in}}%
\pgfpathcurveto{\pgfqpoint{0.754420in}{0.483333in}}{\pgfqpoint{0.758660in}{0.485089in}}{\pgfqpoint{0.761785in}{0.488215in}}%
\pgfpathcurveto{\pgfqpoint{0.764911in}{0.491340in}}{\pgfqpoint{0.766667in}{0.495580in}}{\pgfqpoint{0.766667in}{0.500000in}}%
\pgfpathcurveto{\pgfqpoint{0.766667in}{0.504420in}}{\pgfqpoint{0.764911in}{0.508660in}}{\pgfqpoint{0.761785in}{0.511785in}}%
\pgfpathcurveto{\pgfqpoint{0.758660in}{0.514911in}}{\pgfqpoint{0.754420in}{0.516667in}}{\pgfqpoint{0.750000in}{0.516667in}}%
\pgfpathcurveto{\pgfqpoint{0.745580in}{0.516667in}}{\pgfqpoint{0.741340in}{0.514911in}}{\pgfqpoint{0.738215in}{0.511785in}}%
\pgfpathcurveto{\pgfqpoint{0.735089in}{0.508660in}}{\pgfqpoint{0.733333in}{0.504420in}}{\pgfqpoint{0.733333in}{0.500000in}}%
\pgfpathcurveto{\pgfqpoint{0.733333in}{0.495580in}}{\pgfqpoint{0.735089in}{0.491340in}}{\pgfqpoint{0.738215in}{0.488215in}}%
\pgfpathcurveto{\pgfqpoint{0.741340in}{0.485089in}}{\pgfqpoint{0.745580in}{0.483333in}}{\pgfqpoint{0.750000in}{0.483333in}}%
\pgfpathclose%
\pgfpathmoveto{\pgfqpoint{0.916667in}{0.483333in}}%
\pgfpathcurveto{\pgfqpoint{0.921087in}{0.483333in}}{\pgfqpoint{0.925326in}{0.485089in}}{\pgfqpoint{0.928452in}{0.488215in}}%
\pgfpathcurveto{\pgfqpoint{0.931577in}{0.491340in}}{\pgfqpoint{0.933333in}{0.495580in}}{\pgfqpoint{0.933333in}{0.500000in}}%
\pgfpathcurveto{\pgfqpoint{0.933333in}{0.504420in}}{\pgfqpoint{0.931577in}{0.508660in}}{\pgfqpoint{0.928452in}{0.511785in}}%
\pgfpathcurveto{\pgfqpoint{0.925326in}{0.514911in}}{\pgfqpoint{0.921087in}{0.516667in}}{\pgfqpoint{0.916667in}{0.516667in}}%
\pgfpathcurveto{\pgfqpoint{0.912247in}{0.516667in}}{\pgfqpoint{0.908007in}{0.514911in}}{\pgfqpoint{0.904882in}{0.511785in}}%
\pgfpathcurveto{\pgfqpoint{0.901756in}{0.508660in}}{\pgfqpoint{0.900000in}{0.504420in}}{\pgfqpoint{0.900000in}{0.500000in}}%
\pgfpathcurveto{\pgfqpoint{0.900000in}{0.495580in}}{\pgfqpoint{0.901756in}{0.491340in}}{\pgfqpoint{0.904882in}{0.488215in}}%
\pgfpathcurveto{\pgfqpoint{0.908007in}{0.485089in}}{\pgfqpoint{0.912247in}{0.483333in}}{\pgfqpoint{0.916667in}{0.483333in}}%
\pgfpathclose%
\pgfpathmoveto{\pgfqpoint{0.000000in}{0.650000in}}%
\pgfpathcurveto{\pgfqpoint{0.004420in}{0.650000in}}{\pgfqpoint{0.008660in}{0.651756in}}{\pgfqpoint{0.011785in}{0.654882in}}%
\pgfpathcurveto{\pgfqpoint{0.014911in}{0.658007in}}{\pgfqpoint{0.016667in}{0.662247in}}{\pgfqpoint{0.016667in}{0.666667in}}%
\pgfpathcurveto{\pgfqpoint{0.016667in}{0.671087in}}{\pgfqpoint{0.014911in}{0.675326in}}{\pgfqpoint{0.011785in}{0.678452in}}%
\pgfpathcurveto{\pgfqpoint{0.008660in}{0.681577in}}{\pgfqpoint{0.004420in}{0.683333in}}{\pgfqpoint{0.000000in}{0.683333in}}%
\pgfpathcurveto{\pgfqpoint{-0.004420in}{0.683333in}}{\pgfqpoint{-0.008660in}{0.681577in}}{\pgfqpoint{-0.011785in}{0.678452in}}%
\pgfpathcurveto{\pgfqpoint{-0.014911in}{0.675326in}}{\pgfqpoint{-0.016667in}{0.671087in}}{\pgfqpoint{-0.016667in}{0.666667in}}%
\pgfpathcurveto{\pgfqpoint{-0.016667in}{0.662247in}}{\pgfqpoint{-0.014911in}{0.658007in}}{\pgfqpoint{-0.011785in}{0.654882in}}%
\pgfpathcurveto{\pgfqpoint{-0.008660in}{0.651756in}}{\pgfqpoint{-0.004420in}{0.650000in}}{\pgfqpoint{0.000000in}{0.650000in}}%
\pgfpathclose%
\pgfpathmoveto{\pgfqpoint{0.166667in}{0.650000in}}%
\pgfpathcurveto{\pgfqpoint{0.171087in}{0.650000in}}{\pgfqpoint{0.175326in}{0.651756in}}{\pgfqpoint{0.178452in}{0.654882in}}%
\pgfpathcurveto{\pgfqpoint{0.181577in}{0.658007in}}{\pgfqpoint{0.183333in}{0.662247in}}{\pgfqpoint{0.183333in}{0.666667in}}%
\pgfpathcurveto{\pgfqpoint{0.183333in}{0.671087in}}{\pgfqpoint{0.181577in}{0.675326in}}{\pgfqpoint{0.178452in}{0.678452in}}%
\pgfpathcurveto{\pgfqpoint{0.175326in}{0.681577in}}{\pgfqpoint{0.171087in}{0.683333in}}{\pgfqpoint{0.166667in}{0.683333in}}%
\pgfpathcurveto{\pgfqpoint{0.162247in}{0.683333in}}{\pgfqpoint{0.158007in}{0.681577in}}{\pgfqpoint{0.154882in}{0.678452in}}%
\pgfpathcurveto{\pgfqpoint{0.151756in}{0.675326in}}{\pgfqpoint{0.150000in}{0.671087in}}{\pgfqpoint{0.150000in}{0.666667in}}%
\pgfpathcurveto{\pgfqpoint{0.150000in}{0.662247in}}{\pgfqpoint{0.151756in}{0.658007in}}{\pgfqpoint{0.154882in}{0.654882in}}%
\pgfpathcurveto{\pgfqpoint{0.158007in}{0.651756in}}{\pgfqpoint{0.162247in}{0.650000in}}{\pgfqpoint{0.166667in}{0.650000in}}%
\pgfpathclose%
\pgfpathmoveto{\pgfqpoint{0.333333in}{0.650000in}}%
\pgfpathcurveto{\pgfqpoint{0.337753in}{0.650000in}}{\pgfqpoint{0.341993in}{0.651756in}}{\pgfqpoint{0.345118in}{0.654882in}}%
\pgfpathcurveto{\pgfqpoint{0.348244in}{0.658007in}}{\pgfqpoint{0.350000in}{0.662247in}}{\pgfqpoint{0.350000in}{0.666667in}}%
\pgfpathcurveto{\pgfqpoint{0.350000in}{0.671087in}}{\pgfqpoint{0.348244in}{0.675326in}}{\pgfqpoint{0.345118in}{0.678452in}}%
\pgfpathcurveto{\pgfqpoint{0.341993in}{0.681577in}}{\pgfqpoint{0.337753in}{0.683333in}}{\pgfqpoint{0.333333in}{0.683333in}}%
\pgfpathcurveto{\pgfqpoint{0.328913in}{0.683333in}}{\pgfqpoint{0.324674in}{0.681577in}}{\pgfqpoint{0.321548in}{0.678452in}}%
\pgfpathcurveto{\pgfqpoint{0.318423in}{0.675326in}}{\pgfqpoint{0.316667in}{0.671087in}}{\pgfqpoint{0.316667in}{0.666667in}}%
\pgfpathcurveto{\pgfqpoint{0.316667in}{0.662247in}}{\pgfqpoint{0.318423in}{0.658007in}}{\pgfqpoint{0.321548in}{0.654882in}}%
\pgfpathcurveto{\pgfqpoint{0.324674in}{0.651756in}}{\pgfqpoint{0.328913in}{0.650000in}}{\pgfqpoint{0.333333in}{0.650000in}}%
\pgfpathclose%
\pgfpathmoveto{\pgfqpoint{0.500000in}{0.650000in}}%
\pgfpathcurveto{\pgfqpoint{0.504420in}{0.650000in}}{\pgfqpoint{0.508660in}{0.651756in}}{\pgfqpoint{0.511785in}{0.654882in}}%
\pgfpathcurveto{\pgfqpoint{0.514911in}{0.658007in}}{\pgfqpoint{0.516667in}{0.662247in}}{\pgfqpoint{0.516667in}{0.666667in}}%
\pgfpathcurveto{\pgfqpoint{0.516667in}{0.671087in}}{\pgfqpoint{0.514911in}{0.675326in}}{\pgfqpoint{0.511785in}{0.678452in}}%
\pgfpathcurveto{\pgfqpoint{0.508660in}{0.681577in}}{\pgfqpoint{0.504420in}{0.683333in}}{\pgfqpoint{0.500000in}{0.683333in}}%
\pgfpathcurveto{\pgfqpoint{0.495580in}{0.683333in}}{\pgfqpoint{0.491340in}{0.681577in}}{\pgfqpoint{0.488215in}{0.678452in}}%
\pgfpathcurveto{\pgfqpoint{0.485089in}{0.675326in}}{\pgfqpoint{0.483333in}{0.671087in}}{\pgfqpoint{0.483333in}{0.666667in}}%
\pgfpathcurveto{\pgfqpoint{0.483333in}{0.662247in}}{\pgfqpoint{0.485089in}{0.658007in}}{\pgfqpoint{0.488215in}{0.654882in}}%
\pgfpathcurveto{\pgfqpoint{0.491340in}{0.651756in}}{\pgfqpoint{0.495580in}{0.650000in}}{\pgfqpoint{0.500000in}{0.650000in}}%
\pgfpathclose%
\pgfpathmoveto{\pgfqpoint{0.666667in}{0.650000in}}%
\pgfpathcurveto{\pgfqpoint{0.671087in}{0.650000in}}{\pgfqpoint{0.675326in}{0.651756in}}{\pgfqpoint{0.678452in}{0.654882in}}%
\pgfpathcurveto{\pgfqpoint{0.681577in}{0.658007in}}{\pgfqpoint{0.683333in}{0.662247in}}{\pgfqpoint{0.683333in}{0.666667in}}%
\pgfpathcurveto{\pgfqpoint{0.683333in}{0.671087in}}{\pgfqpoint{0.681577in}{0.675326in}}{\pgfqpoint{0.678452in}{0.678452in}}%
\pgfpathcurveto{\pgfqpoint{0.675326in}{0.681577in}}{\pgfqpoint{0.671087in}{0.683333in}}{\pgfqpoint{0.666667in}{0.683333in}}%
\pgfpathcurveto{\pgfqpoint{0.662247in}{0.683333in}}{\pgfqpoint{0.658007in}{0.681577in}}{\pgfqpoint{0.654882in}{0.678452in}}%
\pgfpathcurveto{\pgfqpoint{0.651756in}{0.675326in}}{\pgfqpoint{0.650000in}{0.671087in}}{\pgfqpoint{0.650000in}{0.666667in}}%
\pgfpathcurveto{\pgfqpoint{0.650000in}{0.662247in}}{\pgfqpoint{0.651756in}{0.658007in}}{\pgfqpoint{0.654882in}{0.654882in}}%
\pgfpathcurveto{\pgfqpoint{0.658007in}{0.651756in}}{\pgfqpoint{0.662247in}{0.650000in}}{\pgfqpoint{0.666667in}{0.650000in}}%
\pgfpathclose%
\pgfpathmoveto{\pgfqpoint{0.833333in}{0.650000in}}%
\pgfpathcurveto{\pgfqpoint{0.837753in}{0.650000in}}{\pgfqpoint{0.841993in}{0.651756in}}{\pgfqpoint{0.845118in}{0.654882in}}%
\pgfpathcurveto{\pgfqpoint{0.848244in}{0.658007in}}{\pgfqpoint{0.850000in}{0.662247in}}{\pgfqpoint{0.850000in}{0.666667in}}%
\pgfpathcurveto{\pgfqpoint{0.850000in}{0.671087in}}{\pgfqpoint{0.848244in}{0.675326in}}{\pgfqpoint{0.845118in}{0.678452in}}%
\pgfpathcurveto{\pgfqpoint{0.841993in}{0.681577in}}{\pgfqpoint{0.837753in}{0.683333in}}{\pgfqpoint{0.833333in}{0.683333in}}%
\pgfpathcurveto{\pgfqpoint{0.828913in}{0.683333in}}{\pgfqpoint{0.824674in}{0.681577in}}{\pgfqpoint{0.821548in}{0.678452in}}%
\pgfpathcurveto{\pgfqpoint{0.818423in}{0.675326in}}{\pgfqpoint{0.816667in}{0.671087in}}{\pgfqpoint{0.816667in}{0.666667in}}%
\pgfpathcurveto{\pgfqpoint{0.816667in}{0.662247in}}{\pgfqpoint{0.818423in}{0.658007in}}{\pgfqpoint{0.821548in}{0.654882in}}%
\pgfpathcurveto{\pgfqpoint{0.824674in}{0.651756in}}{\pgfqpoint{0.828913in}{0.650000in}}{\pgfqpoint{0.833333in}{0.650000in}}%
\pgfpathclose%
\pgfpathmoveto{\pgfqpoint{1.000000in}{0.650000in}}%
\pgfpathcurveto{\pgfqpoint{1.004420in}{0.650000in}}{\pgfqpoint{1.008660in}{0.651756in}}{\pgfqpoint{1.011785in}{0.654882in}}%
\pgfpathcurveto{\pgfqpoint{1.014911in}{0.658007in}}{\pgfqpoint{1.016667in}{0.662247in}}{\pgfqpoint{1.016667in}{0.666667in}}%
\pgfpathcurveto{\pgfqpoint{1.016667in}{0.671087in}}{\pgfqpoint{1.014911in}{0.675326in}}{\pgfqpoint{1.011785in}{0.678452in}}%
\pgfpathcurveto{\pgfqpoint{1.008660in}{0.681577in}}{\pgfqpoint{1.004420in}{0.683333in}}{\pgfqpoint{1.000000in}{0.683333in}}%
\pgfpathcurveto{\pgfqpoint{0.995580in}{0.683333in}}{\pgfqpoint{0.991340in}{0.681577in}}{\pgfqpoint{0.988215in}{0.678452in}}%
\pgfpathcurveto{\pgfqpoint{0.985089in}{0.675326in}}{\pgfqpoint{0.983333in}{0.671087in}}{\pgfqpoint{0.983333in}{0.666667in}}%
\pgfpathcurveto{\pgfqpoint{0.983333in}{0.662247in}}{\pgfqpoint{0.985089in}{0.658007in}}{\pgfqpoint{0.988215in}{0.654882in}}%
\pgfpathcurveto{\pgfqpoint{0.991340in}{0.651756in}}{\pgfqpoint{0.995580in}{0.650000in}}{\pgfqpoint{1.000000in}{0.650000in}}%
\pgfpathclose%
\pgfpathmoveto{\pgfqpoint{0.083333in}{0.816667in}}%
\pgfpathcurveto{\pgfqpoint{0.087753in}{0.816667in}}{\pgfqpoint{0.091993in}{0.818423in}}{\pgfqpoint{0.095118in}{0.821548in}}%
\pgfpathcurveto{\pgfqpoint{0.098244in}{0.824674in}}{\pgfqpoint{0.100000in}{0.828913in}}{\pgfqpoint{0.100000in}{0.833333in}}%
\pgfpathcurveto{\pgfqpoint{0.100000in}{0.837753in}}{\pgfqpoint{0.098244in}{0.841993in}}{\pgfqpoint{0.095118in}{0.845118in}}%
\pgfpathcurveto{\pgfqpoint{0.091993in}{0.848244in}}{\pgfqpoint{0.087753in}{0.850000in}}{\pgfqpoint{0.083333in}{0.850000in}}%
\pgfpathcurveto{\pgfqpoint{0.078913in}{0.850000in}}{\pgfqpoint{0.074674in}{0.848244in}}{\pgfqpoint{0.071548in}{0.845118in}}%
\pgfpathcurveto{\pgfqpoint{0.068423in}{0.841993in}}{\pgfqpoint{0.066667in}{0.837753in}}{\pgfqpoint{0.066667in}{0.833333in}}%
\pgfpathcurveto{\pgfqpoint{0.066667in}{0.828913in}}{\pgfqpoint{0.068423in}{0.824674in}}{\pgfqpoint{0.071548in}{0.821548in}}%
\pgfpathcurveto{\pgfqpoint{0.074674in}{0.818423in}}{\pgfqpoint{0.078913in}{0.816667in}}{\pgfqpoint{0.083333in}{0.816667in}}%
\pgfpathclose%
\pgfpathmoveto{\pgfqpoint{0.250000in}{0.816667in}}%
\pgfpathcurveto{\pgfqpoint{0.254420in}{0.816667in}}{\pgfqpoint{0.258660in}{0.818423in}}{\pgfqpoint{0.261785in}{0.821548in}}%
\pgfpathcurveto{\pgfqpoint{0.264911in}{0.824674in}}{\pgfqpoint{0.266667in}{0.828913in}}{\pgfqpoint{0.266667in}{0.833333in}}%
\pgfpathcurveto{\pgfqpoint{0.266667in}{0.837753in}}{\pgfqpoint{0.264911in}{0.841993in}}{\pgfqpoint{0.261785in}{0.845118in}}%
\pgfpathcurveto{\pgfqpoint{0.258660in}{0.848244in}}{\pgfqpoint{0.254420in}{0.850000in}}{\pgfqpoint{0.250000in}{0.850000in}}%
\pgfpathcurveto{\pgfqpoint{0.245580in}{0.850000in}}{\pgfqpoint{0.241340in}{0.848244in}}{\pgfqpoint{0.238215in}{0.845118in}}%
\pgfpathcurveto{\pgfqpoint{0.235089in}{0.841993in}}{\pgfqpoint{0.233333in}{0.837753in}}{\pgfqpoint{0.233333in}{0.833333in}}%
\pgfpathcurveto{\pgfqpoint{0.233333in}{0.828913in}}{\pgfqpoint{0.235089in}{0.824674in}}{\pgfqpoint{0.238215in}{0.821548in}}%
\pgfpathcurveto{\pgfqpoint{0.241340in}{0.818423in}}{\pgfqpoint{0.245580in}{0.816667in}}{\pgfqpoint{0.250000in}{0.816667in}}%
\pgfpathclose%
\pgfpathmoveto{\pgfqpoint{0.416667in}{0.816667in}}%
\pgfpathcurveto{\pgfqpoint{0.421087in}{0.816667in}}{\pgfqpoint{0.425326in}{0.818423in}}{\pgfqpoint{0.428452in}{0.821548in}}%
\pgfpathcurveto{\pgfqpoint{0.431577in}{0.824674in}}{\pgfqpoint{0.433333in}{0.828913in}}{\pgfqpoint{0.433333in}{0.833333in}}%
\pgfpathcurveto{\pgfqpoint{0.433333in}{0.837753in}}{\pgfqpoint{0.431577in}{0.841993in}}{\pgfqpoint{0.428452in}{0.845118in}}%
\pgfpathcurveto{\pgfqpoint{0.425326in}{0.848244in}}{\pgfqpoint{0.421087in}{0.850000in}}{\pgfqpoint{0.416667in}{0.850000in}}%
\pgfpathcurveto{\pgfqpoint{0.412247in}{0.850000in}}{\pgfqpoint{0.408007in}{0.848244in}}{\pgfqpoint{0.404882in}{0.845118in}}%
\pgfpathcurveto{\pgfqpoint{0.401756in}{0.841993in}}{\pgfqpoint{0.400000in}{0.837753in}}{\pgfqpoint{0.400000in}{0.833333in}}%
\pgfpathcurveto{\pgfqpoint{0.400000in}{0.828913in}}{\pgfqpoint{0.401756in}{0.824674in}}{\pgfqpoint{0.404882in}{0.821548in}}%
\pgfpathcurveto{\pgfqpoint{0.408007in}{0.818423in}}{\pgfqpoint{0.412247in}{0.816667in}}{\pgfqpoint{0.416667in}{0.816667in}}%
\pgfpathclose%
\pgfpathmoveto{\pgfqpoint{0.583333in}{0.816667in}}%
\pgfpathcurveto{\pgfqpoint{0.587753in}{0.816667in}}{\pgfqpoint{0.591993in}{0.818423in}}{\pgfqpoint{0.595118in}{0.821548in}}%
\pgfpathcurveto{\pgfqpoint{0.598244in}{0.824674in}}{\pgfqpoint{0.600000in}{0.828913in}}{\pgfqpoint{0.600000in}{0.833333in}}%
\pgfpathcurveto{\pgfqpoint{0.600000in}{0.837753in}}{\pgfqpoint{0.598244in}{0.841993in}}{\pgfqpoint{0.595118in}{0.845118in}}%
\pgfpathcurveto{\pgfqpoint{0.591993in}{0.848244in}}{\pgfqpoint{0.587753in}{0.850000in}}{\pgfqpoint{0.583333in}{0.850000in}}%
\pgfpathcurveto{\pgfqpoint{0.578913in}{0.850000in}}{\pgfqpoint{0.574674in}{0.848244in}}{\pgfqpoint{0.571548in}{0.845118in}}%
\pgfpathcurveto{\pgfqpoint{0.568423in}{0.841993in}}{\pgfqpoint{0.566667in}{0.837753in}}{\pgfqpoint{0.566667in}{0.833333in}}%
\pgfpathcurveto{\pgfqpoint{0.566667in}{0.828913in}}{\pgfqpoint{0.568423in}{0.824674in}}{\pgfqpoint{0.571548in}{0.821548in}}%
\pgfpathcurveto{\pgfqpoint{0.574674in}{0.818423in}}{\pgfqpoint{0.578913in}{0.816667in}}{\pgfqpoint{0.583333in}{0.816667in}}%
\pgfpathclose%
\pgfpathmoveto{\pgfqpoint{0.750000in}{0.816667in}}%
\pgfpathcurveto{\pgfqpoint{0.754420in}{0.816667in}}{\pgfqpoint{0.758660in}{0.818423in}}{\pgfqpoint{0.761785in}{0.821548in}}%
\pgfpathcurveto{\pgfqpoint{0.764911in}{0.824674in}}{\pgfqpoint{0.766667in}{0.828913in}}{\pgfqpoint{0.766667in}{0.833333in}}%
\pgfpathcurveto{\pgfqpoint{0.766667in}{0.837753in}}{\pgfqpoint{0.764911in}{0.841993in}}{\pgfqpoint{0.761785in}{0.845118in}}%
\pgfpathcurveto{\pgfqpoint{0.758660in}{0.848244in}}{\pgfqpoint{0.754420in}{0.850000in}}{\pgfqpoint{0.750000in}{0.850000in}}%
\pgfpathcurveto{\pgfqpoint{0.745580in}{0.850000in}}{\pgfqpoint{0.741340in}{0.848244in}}{\pgfqpoint{0.738215in}{0.845118in}}%
\pgfpathcurveto{\pgfqpoint{0.735089in}{0.841993in}}{\pgfqpoint{0.733333in}{0.837753in}}{\pgfqpoint{0.733333in}{0.833333in}}%
\pgfpathcurveto{\pgfqpoint{0.733333in}{0.828913in}}{\pgfqpoint{0.735089in}{0.824674in}}{\pgfqpoint{0.738215in}{0.821548in}}%
\pgfpathcurveto{\pgfqpoint{0.741340in}{0.818423in}}{\pgfqpoint{0.745580in}{0.816667in}}{\pgfqpoint{0.750000in}{0.816667in}}%
\pgfpathclose%
\pgfpathmoveto{\pgfqpoint{0.916667in}{0.816667in}}%
\pgfpathcurveto{\pgfqpoint{0.921087in}{0.816667in}}{\pgfqpoint{0.925326in}{0.818423in}}{\pgfqpoint{0.928452in}{0.821548in}}%
\pgfpathcurveto{\pgfqpoint{0.931577in}{0.824674in}}{\pgfqpoint{0.933333in}{0.828913in}}{\pgfqpoint{0.933333in}{0.833333in}}%
\pgfpathcurveto{\pgfqpoint{0.933333in}{0.837753in}}{\pgfqpoint{0.931577in}{0.841993in}}{\pgfqpoint{0.928452in}{0.845118in}}%
\pgfpathcurveto{\pgfqpoint{0.925326in}{0.848244in}}{\pgfqpoint{0.921087in}{0.850000in}}{\pgfqpoint{0.916667in}{0.850000in}}%
\pgfpathcurveto{\pgfqpoint{0.912247in}{0.850000in}}{\pgfqpoint{0.908007in}{0.848244in}}{\pgfqpoint{0.904882in}{0.845118in}}%
\pgfpathcurveto{\pgfqpoint{0.901756in}{0.841993in}}{\pgfqpoint{0.900000in}{0.837753in}}{\pgfqpoint{0.900000in}{0.833333in}}%
\pgfpathcurveto{\pgfqpoint{0.900000in}{0.828913in}}{\pgfqpoint{0.901756in}{0.824674in}}{\pgfqpoint{0.904882in}{0.821548in}}%
\pgfpathcurveto{\pgfqpoint{0.908007in}{0.818423in}}{\pgfqpoint{0.912247in}{0.816667in}}{\pgfqpoint{0.916667in}{0.816667in}}%
\pgfpathclose%
\pgfpathmoveto{\pgfqpoint{0.000000in}{0.983333in}}%
\pgfpathcurveto{\pgfqpoint{0.004420in}{0.983333in}}{\pgfqpoint{0.008660in}{0.985089in}}{\pgfqpoint{0.011785in}{0.988215in}}%
\pgfpathcurveto{\pgfqpoint{0.014911in}{0.991340in}}{\pgfqpoint{0.016667in}{0.995580in}}{\pgfqpoint{0.016667in}{1.000000in}}%
\pgfpathcurveto{\pgfqpoint{0.016667in}{1.004420in}}{\pgfqpoint{0.014911in}{1.008660in}}{\pgfqpoint{0.011785in}{1.011785in}}%
\pgfpathcurveto{\pgfqpoint{0.008660in}{1.014911in}}{\pgfqpoint{0.004420in}{1.016667in}}{\pgfqpoint{0.000000in}{1.016667in}}%
\pgfpathcurveto{\pgfqpoint{-0.004420in}{1.016667in}}{\pgfqpoint{-0.008660in}{1.014911in}}{\pgfqpoint{-0.011785in}{1.011785in}}%
\pgfpathcurveto{\pgfqpoint{-0.014911in}{1.008660in}}{\pgfqpoint{-0.016667in}{1.004420in}}{\pgfqpoint{-0.016667in}{1.000000in}}%
\pgfpathcurveto{\pgfqpoint{-0.016667in}{0.995580in}}{\pgfqpoint{-0.014911in}{0.991340in}}{\pgfqpoint{-0.011785in}{0.988215in}}%
\pgfpathcurveto{\pgfqpoint{-0.008660in}{0.985089in}}{\pgfqpoint{-0.004420in}{0.983333in}}{\pgfqpoint{0.000000in}{0.983333in}}%
\pgfpathclose%
\pgfpathmoveto{\pgfqpoint{0.166667in}{0.983333in}}%
\pgfpathcurveto{\pgfqpoint{0.171087in}{0.983333in}}{\pgfqpoint{0.175326in}{0.985089in}}{\pgfqpoint{0.178452in}{0.988215in}}%
\pgfpathcurveto{\pgfqpoint{0.181577in}{0.991340in}}{\pgfqpoint{0.183333in}{0.995580in}}{\pgfqpoint{0.183333in}{1.000000in}}%
\pgfpathcurveto{\pgfqpoint{0.183333in}{1.004420in}}{\pgfqpoint{0.181577in}{1.008660in}}{\pgfqpoint{0.178452in}{1.011785in}}%
\pgfpathcurveto{\pgfqpoint{0.175326in}{1.014911in}}{\pgfqpoint{0.171087in}{1.016667in}}{\pgfqpoint{0.166667in}{1.016667in}}%
\pgfpathcurveto{\pgfqpoint{0.162247in}{1.016667in}}{\pgfqpoint{0.158007in}{1.014911in}}{\pgfqpoint{0.154882in}{1.011785in}}%
\pgfpathcurveto{\pgfqpoint{0.151756in}{1.008660in}}{\pgfqpoint{0.150000in}{1.004420in}}{\pgfqpoint{0.150000in}{1.000000in}}%
\pgfpathcurveto{\pgfqpoint{0.150000in}{0.995580in}}{\pgfqpoint{0.151756in}{0.991340in}}{\pgfqpoint{0.154882in}{0.988215in}}%
\pgfpathcurveto{\pgfqpoint{0.158007in}{0.985089in}}{\pgfqpoint{0.162247in}{0.983333in}}{\pgfqpoint{0.166667in}{0.983333in}}%
\pgfpathclose%
\pgfpathmoveto{\pgfqpoint{0.333333in}{0.983333in}}%
\pgfpathcurveto{\pgfqpoint{0.337753in}{0.983333in}}{\pgfqpoint{0.341993in}{0.985089in}}{\pgfqpoint{0.345118in}{0.988215in}}%
\pgfpathcurveto{\pgfqpoint{0.348244in}{0.991340in}}{\pgfqpoint{0.350000in}{0.995580in}}{\pgfqpoint{0.350000in}{1.000000in}}%
\pgfpathcurveto{\pgfqpoint{0.350000in}{1.004420in}}{\pgfqpoint{0.348244in}{1.008660in}}{\pgfqpoint{0.345118in}{1.011785in}}%
\pgfpathcurveto{\pgfqpoint{0.341993in}{1.014911in}}{\pgfqpoint{0.337753in}{1.016667in}}{\pgfqpoint{0.333333in}{1.016667in}}%
\pgfpathcurveto{\pgfqpoint{0.328913in}{1.016667in}}{\pgfqpoint{0.324674in}{1.014911in}}{\pgfqpoint{0.321548in}{1.011785in}}%
\pgfpathcurveto{\pgfqpoint{0.318423in}{1.008660in}}{\pgfqpoint{0.316667in}{1.004420in}}{\pgfqpoint{0.316667in}{1.000000in}}%
\pgfpathcurveto{\pgfqpoint{0.316667in}{0.995580in}}{\pgfqpoint{0.318423in}{0.991340in}}{\pgfqpoint{0.321548in}{0.988215in}}%
\pgfpathcurveto{\pgfqpoint{0.324674in}{0.985089in}}{\pgfqpoint{0.328913in}{0.983333in}}{\pgfqpoint{0.333333in}{0.983333in}}%
\pgfpathclose%
\pgfpathmoveto{\pgfqpoint{0.500000in}{0.983333in}}%
\pgfpathcurveto{\pgfqpoint{0.504420in}{0.983333in}}{\pgfqpoint{0.508660in}{0.985089in}}{\pgfqpoint{0.511785in}{0.988215in}}%
\pgfpathcurveto{\pgfqpoint{0.514911in}{0.991340in}}{\pgfqpoint{0.516667in}{0.995580in}}{\pgfqpoint{0.516667in}{1.000000in}}%
\pgfpathcurveto{\pgfqpoint{0.516667in}{1.004420in}}{\pgfqpoint{0.514911in}{1.008660in}}{\pgfqpoint{0.511785in}{1.011785in}}%
\pgfpathcurveto{\pgfqpoint{0.508660in}{1.014911in}}{\pgfqpoint{0.504420in}{1.016667in}}{\pgfqpoint{0.500000in}{1.016667in}}%
\pgfpathcurveto{\pgfqpoint{0.495580in}{1.016667in}}{\pgfqpoint{0.491340in}{1.014911in}}{\pgfqpoint{0.488215in}{1.011785in}}%
\pgfpathcurveto{\pgfqpoint{0.485089in}{1.008660in}}{\pgfqpoint{0.483333in}{1.004420in}}{\pgfqpoint{0.483333in}{1.000000in}}%
\pgfpathcurveto{\pgfqpoint{0.483333in}{0.995580in}}{\pgfqpoint{0.485089in}{0.991340in}}{\pgfqpoint{0.488215in}{0.988215in}}%
\pgfpathcurveto{\pgfqpoint{0.491340in}{0.985089in}}{\pgfqpoint{0.495580in}{0.983333in}}{\pgfqpoint{0.500000in}{0.983333in}}%
\pgfpathclose%
\pgfpathmoveto{\pgfqpoint{0.666667in}{0.983333in}}%
\pgfpathcurveto{\pgfqpoint{0.671087in}{0.983333in}}{\pgfqpoint{0.675326in}{0.985089in}}{\pgfqpoint{0.678452in}{0.988215in}}%
\pgfpathcurveto{\pgfqpoint{0.681577in}{0.991340in}}{\pgfqpoint{0.683333in}{0.995580in}}{\pgfqpoint{0.683333in}{1.000000in}}%
\pgfpathcurveto{\pgfqpoint{0.683333in}{1.004420in}}{\pgfqpoint{0.681577in}{1.008660in}}{\pgfqpoint{0.678452in}{1.011785in}}%
\pgfpathcurveto{\pgfqpoint{0.675326in}{1.014911in}}{\pgfqpoint{0.671087in}{1.016667in}}{\pgfqpoint{0.666667in}{1.016667in}}%
\pgfpathcurveto{\pgfqpoint{0.662247in}{1.016667in}}{\pgfqpoint{0.658007in}{1.014911in}}{\pgfqpoint{0.654882in}{1.011785in}}%
\pgfpathcurveto{\pgfqpoint{0.651756in}{1.008660in}}{\pgfqpoint{0.650000in}{1.004420in}}{\pgfqpoint{0.650000in}{1.000000in}}%
\pgfpathcurveto{\pgfqpoint{0.650000in}{0.995580in}}{\pgfqpoint{0.651756in}{0.991340in}}{\pgfqpoint{0.654882in}{0.988215in}}%
\pgfpathcurveto{\pgfqpoint{0.658007in}{0.985089in}}{\pgfqpoint{0.662247in}{0.983333in}}{\pgfqpoint{0.666667in}{0.983333in}}%
\pgfpathclose%
\pgfpathmoveto{\pgfqpoint{0.833333in}{0.983333in}}%
\pgfpathcurveto{\pgfqpoint{0.837753in}{0.983333in}}{\pgfqpoint{0.841993in}{0.985089in}}{\pgfqpoint{0.845118in}{0.988215in}}%
\pgfpathcurveto{\pgfqpoint{0.848244in}{0.991340in}}{\pgfqpoint{0.850000in}{0.995580in}}{\pgfqpoint{0.850000in}{1.000000in}}%
\pgfpathcurveto{\pgfqpoint{0.850000in}{1.004420in}}{\pgfqpoint{0.848244in}{1.008660in}}{\pgfqpoint{0.845118in}{1.011785in}}%
\pgfpathcurveto{\pgfqpoint{0.841993in}{1.014911in}}{\pgfqpoint{0.837753in}{1.016667in}}{\pgfqpoint{0.833333in}{1.016667in}}%
\pgfpathcurveto{\pgfqpoint{0.828913in}{1.016667in}}{\pgfqpoint{0.824674in}{1.014911in}}{\pgfqpoint{0.821548in}{1.011785in}}%
\pgfpathcurveto{\pgfqpoint{0.818423in}{1.008660in}}{\pgfqpoint{0.816667in}{1.004420in}}{\pgfqpoint{0.816667in}{1.000000in}}%
\pgfpathcurveto{\pgfqpoint{0.816667in}{0.995580in}}{\pgfqpoint{0.818423in}{0.991340in}}{\pgfqpoint{0.821548in}{0.988215in}}%
\pgfpathcurveto{\pgfqpoint{0.824674in}{0.985089in}}{\pgfqpoint{0.828913in}{0.983333in}}{\pgfqpoint{0.833333in}{0.983333in}}%
\pgfpathclose%
\pgfpathmoveto{\pgfqpoint{1.000000in}{0.983333in}}%
\pgfpathcurveto{\pgfqpoint{1.004420in}{0.983333in}}{\pgfqpoint{1.008660in}{0.985089in}}{\pgfqpoint{1.011785in}{0.988215in}}%
\pgfpathcurveto{\pgfqpoint{1.014911in}{0.991340in}}{\pgfqpoint{1.016667in}{0.995580in}}{\pgfqpoint{1.016667in}{1.000000in}}%
\pgfpathcurveto{\pgfqpoint{1.016667in}{1.004420in}}{\pgfqpoint{1.014911in}{1.008660in}}{\pgfqpoint{1.011785in}{1.011785in}}%
\pgfpathcurveto{\pgfqpoint{1.008660in}{1.014911in}}{\pgfqpoint{1.004420in}{1.016667in}}{\pgfqpoint{1.000000in}{1.016667in}}%
\pgfpathcurveto{\pgfqpoint{0.995580in}{1.016667in}}{\pgfqpoint{0.991340in}{1.014911in}}{\pgfqpoint{0.988215in}{1.011785in}}%
\pgfpathcurveto{\pgfqpoint{0.985089in}{1.008660in}}{\pgfqpoint{0.983333in}{1.004420in}}{\pgfqpoint{0.983333in}{1.000000in}}%
\pgfpathcurveto{\pgfqpoint{0.983333in}{0.995580in}}{\pgfqpoint{0.985089in}{0.991340in}}{\pgfqpoint{0.988215in}{0.988215in}}%
\pgfpathcurveto{\pgfqpoint{0.991340in}{0.985089in}}{\pgfqpoint{0.995580in}{0.983333in}}{\pgfqpoint{1.000000in}{0.983333in}}%
\pgfpathclose%
\pgfusepath{stroke}%
\end{pgfscope}%
}%
\pgfsys@transformshift{7.523315in}{5.046572in}%
\pgfsys@useobject{currentpattern}{}%
\pgfsys@transformshift{1in}{0in}%
\pgfsys@transformshift{-1in}{0in}%
\pgfsys@transformshift{0in}{1in}%
\end{pgfscope}%
\begin{pgfscope}%
\pgfpathrectangle{\pgfqpoint{0.935815in}{0.637495in}}{\pgfqpoint{9.300000in}{9.060000in}}%
\pgfusepath{clip}%
\pgfsetbuttcap%
\pgfsetmiterjoin%
\definecolor{currentfill}{rgb}{0.172549,0.627451,0.172549}%
\pgfsetfillcolor{currentfill}%
\pgfsetfillopacity{0.990000}%
\pgfsetlinewidth{0.000000pt}%
\definecolor{currentstroke}{rgb}{0.000000,0.000000,0.000000}%
\pgfsetstrokecolor{currentstroke}%
\pgfsetstrokeopacity{0.990000}%
\pgfsetdash{}{0pt}%
\pgfpathmoveto{\pgfqpoint{9.073315in}{5.392542in}}%
\pgfpathlineto{\pgfqpoint{9.848315in}{5.392542in}}%
\pgfpathlineto{\pgfqpoint{9.848315in}{6.003887in}}%
\pgfpathlineto{\pgfqpoint{9.073315in}{6.003887in}}%
\pgfpathclose%
\pgfusepath{fill}%
\end{pgfscope}%
\begin{pgfscope}%
\pgfsetbuttcap%
\pgfsetmiterjoin%
\definecolor{currentfill}{rgb}{0.172549,0.627451,0.172549}%
\pgfsetfillcolor{currentfill}%
\pgfsetfillopacity{0.990000}%
\pgfsetlinewidth{0.000000pt}%
\definecolor{currentstroke}{rgb}{0.000000,0.000000,0.000000}%
\pgfsetstrokecolor{currentstroke}%
\pgfsetstrokeopacity{0.990000}%
\pgfsetdash{}{0pt}%
\pgfpathrectangle{\pgfqpoint{0.935815in}{0.637495in}}{\pgfqpoint{9.300000in}{9.060000in}}%
\pgfusepath{clip}%
\pgfpathmoveto{\pgfqpoint{9.073315in}{5.392542in}}%
\pgfpathlineto{\pgfqpoint{9.848315in}{5.392542in}}%
\pgfpathlineto{\pgfqpoint{9.848315in}{6.003887in}}%
\pgfpathlineto{\pgfqpoint{9.073315in}{6.003887in}}%
\pgfpathclose%
\pgfusepath{clip}%
\pgfsys@defobject{currentpattern}{\pgfqpoint{0in}{0in}}{\pgfqpoint{1in}{1in}}{%
\begin{pgfscope}%
\pgfpathrectangle{\pgfqpoint{0in}{0in}}{\pgfqpoint{1in}{1in}}%
\pgfusepath{clip}%
\pgfpathmoveto{\pgfqpoint{0.000000in}{-0.016667in}}%
\pgfpathcurveto{\pgfqpoint{0.004420in}{-0.016667in}}{\pgfqpoint{0.008660in}{-0.014911in}}{\pgfqpoint{0.011785in}{-0.011785in}}%
\pgfpathcurveto{\pgfqpoint{0.014911in}{-0.008660in}}{\pgfqpoint{0.016667in}{-0.004420in}}{\pgfqpoint{0.016667in}{0.000000in}}%
\pgfpathcurveto{\pgfqpoint{0.016667in}{0.004420in}}{\pgfqpoint{0.014911in}{0.008660in}}{\pgfqpoint{0.011785in}{0.011785in}}%
\pgfpathcurveto{\pgfqpoint{0.008660in}{0.014911in}}{\pgfqpoint{0.004420in}{0.016667in}}{\pgfqpoint{0.000000in}{0.016667in}}%
\pgfpathcurveto{\pgfqpoint{-0.004420in}{0.016667in}}{\pgfqpoint{-0.008660in}{0.014911in}}{\pgfqpoint{-0.011785in}{0.011785in}}%
\pgfpathcurveto{\pgfqpoint{-0.014911in}{0.008660in}}{\pgfqpoint{-0.016667in}{0.004420in}}{\pgfqpoint{-0.016667in}{0.000000in}}%
\pgfpathcurveto{\pgfqpoint{-0.016667in}{-0.004420in}}{\pgfqpoint{-0.014911in}{-0.008660in}}{\pgfqpoint{-0.011785in}{-0.011785in}}%
\pgfpathcurveto{\pgfqpoint{-0.008660in}{-0.014911in}}{\pgfqpoint{-0.004420in}{-0.016667in}}{\pgfqpoint{0.000000in}{-0.016667in}}%
\pgfpathclose%
\pgfpathmoveto{\pgfqpoint{0.166667in}{-0.016667in}}%
\pgfpathcurveto{\pgfqpoint{0.171087in}{-0.016667in}}{\pgfqpoint{0.175326in}{-0.014911in}}{\pgfqpoint{0.178452in}{-0.011785in}}%
\pgfpathcurveto{\pgfqpoint{0.181577in}{-0.008660in}}{\pgfqpoint{0.183333in}{-0.004420in}}{\pgfqpoint{0.183333in}{0.000000in}}%
\pgfpathcurveto{\pgfqpoint{0.183333in}{0.004420in}}{\pgfqpoint{0.181577in}{0.008660in}}{\pgfqpoint{0.178452in}{0.011785in}}%
\pgfpathcurveto{\pgfqpoint{0.175326in}{0.014911in}}{\pgfqpoint{0.171087in}{0.016667in}}{\pgfqpoint{0.166667in}{0.016667in}}%
\pgfpathcurveto{\pgfqpoint{0.162247in}{0.016667in}}{\pgfqpoint{0.158007in}{0.014911in}}{\pgfqpoint{0.154882in}{0.011785in}}%
\pgfpathcurveto{\pgfqpoint{0.151756in}{0.008660in}}{\pgfqpoint{0.150000in}{0.004420in}}{\pgfqpoint{0.150000in}{0.000000in}}%
\pgfpathcurveto{\pgfqpoint{0.150000in}{-0.004420in}}{\pgfqpoint{0.151756in}{-0.008660in}}{\pgfqpoint{0.154882in}{-0.011785in}}%
\pgfpathcurveto{\pgfqpoint{0.158007in}{-0.014911in}}{\pgfqpoint{0.162247in}{-0.016667in}}{\pgfqpoint{0.166667in}{-0.016667in}}%
\pgfpathclose%
\pgfpathmoveto{\pgfqpoint{0.333333in}{-0.016667in}}%
\pgfpathcurveto{\pgfqpoint{0.337753in}{-0.016667in}}{\pgfqpoint{0.341993in}{-0.014911in}}{\pgfqpoint{0.345118in}{-0.011785in}}%
\pgfpathcurveto{\pgfqpoint{0.348244in}{-0.008660in}}{\pgfqpoint{0.350000in}{-0.004420in}}{\pgfqpoint{0.350000in}{0.000000in}}%
\pgfpathcurveto{\pgfqpoint{0.350000in}{0.004420in}}{\pgfqpoint{0.348244in}{0.008660in}}{\pgfqpoint{0.345118in}{0.011785in}}%
\pgfpathcurveto{\pgfqpoint{0.341993in}{0.014911in}}{\pgfqpoint{0.337753in}{0.016667in}}{\pgfqpoint{0.333333in}{0.016667in}}%
\pgfpathcurveto{\pgfqpoint{0.328913in}{0.016667in}}{\pgfqpoint{0.324674in}{0.014911in}}{\pgfqpoint{0.321548in}{0.011785in}}%
\pgfpathcurveto{\pgfqpoint{0.318423in}{0.008660in}}{\pgfqpoint{0.316667in}{0.004420in}}{\pgfqpoint{0.316667in}{0.000000in}}%
\pgfpathcurveto{\pgfqpoint{0.316667in}{-0.004420in}}{\pgfqpoint{0.318423in}{-0.008660in}}{\pgfqpoint{0.321548in}{-0.011785in}}%
\pgfpathcurveto{\pgfqpoint{0.324674in}{-0.014911in}}{\pgfqpoint{0.328913in}{-0.016667in}}{\pgfqpoint{0.333333in}{-0.016667in}}%
\pgfpathclose%
\pgfpathmoveto{\pgfqpoint{0.500000in}{-0.016667in}}%
\pgfpathcurveto{\pgfqpoint{0.504420in}{-0.016667in}}{\pgfqpoint{0.508660in}{-0.014911in}}{\pgfqpoint{0.511785in}{-0.011785in}}%
\pgfpathcurveto{\pgfqpoint{0.514911in}{-0.008660in}}{\pgfqpoint{0.516667in}{-0.004420in}}{\pgfqpoint{0.516667in}{0.000000in}}%
\pgfpathcurveto{\pgfqpoint{0.516667in}{0.004420in}}{\pgfqpoint{0.514911in}{0.008660in}}{\pgfqpoint{0.511785in}{0.011785in}}%
\pgfpathcurveto{\pgfqpoint{0.508660in}{0.014911in}}{\pgfqpoint{0.504420in}{0.016667in}}{\pgfqpoint{0.500000in}{0.016667in}}%
\pgfpathcurveto{\pgfqpoint{0.495580in}{0.016667in}}{\pgfqpoint{0.491340in}{0.014911in}}{\pgfqpoint{0.488215in}{0.011785in}}%
\pgfpathcurveto{\pgfqpoint{0.485089in}{0.008660in}}{\pgfqpoint{0.483333in}{0.004420in}}{\pgfqpoint{0.483333in}{0.000000in}}%
\pgfpathcurveto{\pgfqpoint{0.483333in}{-0.004420in}}{\pgfqpoint{0.485089in}{-0.008660in}}{\pgfqpoint{0.488215in}{-0.011785in}}%
\pgfpathcurveto{\pgfqpoint{0.491340in}{-0.014911in}}{\pgfqpoint{0.495580in}{-0.016667in}}{\pgfqpoint{0.500000in}{-0.016667in}}%
\pgfpathclose%
\pgfpathmoveto{\pgfqpoint{0.666667in}{-0.016667in}}%
\pgfpathcurveto{\pgfqpoint{0.671087in}{-0.016667in}}{\pgfqpoint{0.675326in}{-0.014911in}}{\pgfqpoint{0.678452in}{-0.011785in}}%
\pgfpathcurveto{\pgfqpoint{0.681577in}{-0.008660in}}{\pgfqpoint{0.683333in}{-0.004420in}}{\pgfqpoint{0.683333in}{0.000000in}}%
\pgfpathcurveto{\pgfqpoint{0.683333in}{0.004420in}}{\pgfqpoint{0.681577in}{0.008660in}}{\pgfqpoint{0.678452in}{0.011785in}}%
\pgfpathcurveto{\pgfqpoint{0.675326in}{0.014911in}}{\pgfqpoint{0.671087in}{0.016667in}}{\pgfqpoint{0.666667in}{0.016667in}}%
\pgfpathcurveto{\pgfqpoint{0.662247in}{0.016667in}}{\pgfqpoint{0.658007in}{0.014911in}}{\pgfqpoint{0.654882in}{0.011785in}}%
\pgfpathcurveto{\pgfqpoint{0.651756in}{0.008660in}}{\pgfqpoint{0.650000in}{0.004420in}}{\pgfqpoint{0.650000in}{0.000000in}}%
\pgfpathcurveto{\pgfqpoint{0.650000in}{-0.004420in}}{\pgfqpoint{0.651756in}{-0.008660in}}{\pgfqpoint{0.654882in}{-0.011785in}}%
\pgfpathcurveto{\pgfqpoint{0.658007in}{-0.014911in}}{\pgfqpoint{0.662247in}{-0.016667in}}{\pgfqpoint{0.666667in}{-0.016667in}}%
\pgfpathclose%
\pgfpathmoveto{\pgfqpoint{0.833333in}{-0.016667in}}%
\pgfpathcurveto{\pgfqpoint{0.837753in}{-0.016667in}}{\pgfqpoint{0.841993in}{-0.014911in}}{\pgfqpoint{0.845118in}{-0.011785in}}%
\pgfpathcurveto{\pgfqpoint{0.848244in}{-0.008660in}}{\pgfqpoint{0.850000in}{-0.004420in}}{\pgfqpoint{0.850000in}{0.000000in}}%
\pgfpathcurveto{\pgfqpoint{0.850000in}{0.004420in}}{\pgfqpoint{0.848244in}{0.008660in}}{\pgfqpoint{0.845118in}{0.011785in}}%
\pgfpathcurveto{\pgfqpoint{0.841993in}{0.014911in}}{\pgfqpoint{0.837753in}{0.016667in}}{\pgfqpoint{0.833333in}{0.016667in}}%
\pgfpathcurveto{\pgfqpoint{0.828913in}{0.016667in}}{\pgfqpoint{0.824674in}{0.014911in}}{\pgfqpoint{0.821548in}{0.011785in}}%
\pgfpathcurveto{\pgfqpoint{0.818423in}{0.008660in}}{\pgfqpoint{0.816667in}{0.004420in}}{\pgfqpoint{0.816667in}{0.000000in}}%
\pgfpathcurveto{\pgfqpoint{0.816667in}{-0.004420in}}{\pgfqpoint{0.818423in}{-0.008660in}}{\pgfqpoint{0.821548in}{-0.011785in}}%
\pgfpathcurveto{\pgfqpoint{0.824674in}{-0.014911in}}{\pgfqpoint{0.828913in}{-0.016667in}}{\pgfqpoint{0.833333in}{-0.016667in}}%
\pgfpathclose%
\pgfpathmoveto{\pgfqpoint{1.000000in}{-0.016667in}}%
\pgfpathcurveto{\pgfqpoint{1.004420in}{-0.016667in}}{\pgfqpoint{1.008660in}{-0.014911in}}{\pgfqpoint{1.011785in}{-0.011785in}}%
\pgfpathcurveto{\pgfqpoint{1.014911in}{-0.008660in}}{\pgfqpoint{1.016667in}{-0.004420in}}{\pgfqpoint{1.016667in}{0.000000in}}%
\pgfpathcurveto{\pgfqpoint{1.016667in}{0.004420in}}{\pgfqpoint{1.014911in}{0.008660in}}{\pgfqpoint{1.011785in}{0.011785in}}%
\pgfpathcurveto{\pgfqpoint{1.008660in}{0.014911in}}{\pgfqpoint{1.004420in}{0.016667in}}{\pgfqpoint{1.000000in}{0.016667in}}%
\pgfpathcurveto{\pgfqpoint{0.995580in}{0.016667in}}{\pgfqpoint{0.991340in}{0.014911in}}{\pgfqpoint{0.988215in}{0.011785in}}%
\pgfpathcurveto{\pgfqpoint{0.985089in}{0.008660in}}{\pgfqpoint{0.983333in}{0.004420in}}{\pgfqpoint{0.983333in}{0.000000in}}%
\pgfpathcurveto{\pgfqpoint{0.983333in}{-0.004420in}}{\pgfqpoint{0.985089in}{-0.008660in}}{\pgfqpoint{0.988215in}{-0.011785in}}%
\pgfpathcurveto{\pgfqpoint{0.991340in}{-0.014911in}}{\pgfqpoint{0.995580in}{-0.016667in}}{\pgfqpoint{1.000000in}{-0.016667in}}%
\pgfpathclose%
\pgfpathmoveto{\pgfqpoint{0.083333in}{0.150000in}}%
\pgfpathcurveto{\pgfqpoint{0.087753in}{0.150000in}}{\pgfqpoint{0.091993in}{0.151756in}}{\pgfqpoint{0.095118in}{0.154882in}}%
\pgfpathcurveto{\pgfqpoint{0.098244in}{0.158007in}}{\pgfqpoint{0.100000in}{0.162247in}}{\pgfqpoint{0.100000in}{0.166667in}}%
\pgfpathcurveto{\pgfqpoint{0.100000in}{0.171087in}}{\pgfqpoint{0.098244in}{0.175326in}}{\pgfqpoint{0.095118in}{0.178452in}}%
\pgfpathcurveto{\pgfqpoint{0.091993in}{0.181577in}}{\pgfqpoint{0.087753in}{0.183333in}}{\pgfqpoint{0.083333in}{0.183333in}}%
\pgfpathcurveto{\pgfqpoint{0.078913in}{0.183333in}}{\pgfqpoint{0.074674in}{0.181577in}}{\pgfqpoint{0.071548in}{0.178452in}}%
\pgfpathcurveto{\pgfqpoint{0.068423in}{0.175326in}}{\pgfqpoint{0.066667in}{0.171087in}}{\pgfqpoint{0.066667in}{0.166667in}}%
\pgfpathcurveto{\pgfqpoint{0.066667in}{0.162247in}}{\pgfqpoint{0.068423in}{0.158007in}}{\pgfqpoint{0.071548in}{0.154882in}}%
\pgfpathcurveto{\pgfqpoint{0.074674in}{0.151756in}}{\pgfqpoint{0.078913in}{0.150000in}}{\pgfqpoint{0.083333in}{0.150000in}}%
\pgfpathclose%
\pgfpathmoveto{\pgfqpoint{0.250000in}{0.150000in}}%
\pgfpathcurveto{\pgfqpoint{0.254420in}{0.150000in}}{\pgfqpoint{0.258660in}{0.151756in}}{\pgfqpoint{0.261785in}{0.154882in}}%
\pgfpathcurveto{\pgfqpoint{0.264911in}{0.158007in}}{\pgfqpoint{0.266667in}{0.162247in}}{\pgfqpoint{0.266667in}{0.166667in}}%
\pgfpathcurveto{\pgfqpoint{0.266667in}{0.171087in}}{\pgfqpoint{0.264911in}{0.175326in}}{\pgfqpoint{0.261785in}{0.178452in}}%
\pgfpathcurveto{\pgfqpoint{0.258660in}{0.181577in}}{\pgfqpoint{0.254420in}{0.183333in}}{\pgfqpoint{0.250000in}{0.183333in}}%
\pgfpathcurveto{\pgfqpoint{0.245580in}{0.183333in}}{\pgfqpoint{0.241340in}{0.181577in}}{\pgfqpoint{0.238215in}{0.178452in}}%
\pgfpathcurveto{\pgfqpoint{0.235089in}{0.175326in}}{\pgfqpoint{0.233333in}{0.171087in}}{\pgfqpoint{0.233333in}{0.166667in}}%
\pgfpathcurveto{\pgfqpoint{0.233333in}{0.162247in}}{\pgfqpoint{0.235089in}{0.158007in}}{\pgfqpoint{0.238215in}{0.154882in}}%
\pgfpathcurveto{\pgfqpoint{0.241340in}{0.151756in}}{\pgfqpoint{0.245580in}{0.150000in}}{\pgfqpoint{0.250000in}{0.150000in}}%
\pgfpathclose%
\pgfpathmoveto{\pgfqpoint{0.416667in}{0.150000in}}%
\pgfpathcurveto{\pgfqpoint{0.421087in}{0.150000in}}{\pgfqpoint{0.425326in}{0.151756in}}{\pgfqpoint{0.428452in}{0.154882in}}%
\pgfpathcurveto{\pgfqpoint{0.431577in}{0.158007in}}{\pgfqpoint{0.433333in}{0.162247in}}{\pgfqpoint{0.433333in}{0.166667in}}%
\pgfpathcurveto{\pgfqpoint{0.433333in}{0.171087in}}{\pgfqpoint{0.431577in}{0.175326in}}{\pgfqpoint{0.428452in}{0.178452in}}%
\pgfpathcurveto{\pgfqpoint{0.425326in}{0.181577in}}{\pgfqpoint{0.421087in}{0.183333in}}{\pgfqpoint{0.416667in}{0.183333in}}%
\pgfpathcurveto{\pgfqpoint{0.412247in}{0.183333in}}{\pgfqpoint{0.408007in}{0.181577in}}{\pgfqpoint{0.404882in}{0.178452in}}%
\pgfpathcurveto{\pgfqpoint{0.401756in}{0.175326in}}{\pgfqpoint{0.400000in}{0.171087in}}{\pgfqpoint{0.400000in}{0.166667in}}%
\pgfpathcurveto{\pgfqpoint{0.400000in}{0.162247in}}{\pgfqpoint{0.401756in}{0.158007in}}{\pgfqpoint{0.404882in}{0.154882in}}%
\pgfpathcurveto{\pgfqpoint{0.408007in}{0.151756in}}{\pgfqpoint{0.412247in}{0.150000in}}{\pgfqpoint{0.416667in}{0.150000in}}%
\pgfpathclose%
\pgfpathmoveto{\pgfqpoint{0.583333in}{0.150000in}}%
\pgfpathcurveto{\pgfqpoint{0.587753in}{0.150000in}}{\pgfqpoint{0.591993in}{0.151756in}}{\pgfqpoint{0.595118in}{0.154882in}}%
\pgfpathcurveto{\pgfqpoint{0.598244in}{0.158007in}}{\pgfqpoint{0.600000in}{0.162247in}}{\pgfqpoint{0.600000in}{0.166667in}}%
\pgfpathcurveto{\pgfqpoint{0.600000in}{0.171087in}}{\pgfqpoint{0.598244in}{0.175326in}}{\pgfqpoint{0.595118in}{0.178452in}}%
\pgfpathcurveto{\pgfqpoint{0.591993in}{0.181577in}}{\pgfqpoint{0.587753in}{0.183333in}}{\pgfqpoint{0.583333in}{0.183333in}}%
\pgfpathcurveto{\pgfqpoint{0.578913in}{0.183333in}}{\pgfqpoint{0.574674in}{0.181577in}}{\pgfqpoint{0.571548in}{0.178452in}}%
\pgfpathcurveto{\pgfqpoint{0.568423in}{0.175326in}}{\pgfqpoint{0.566667in}{0.171087in}}{\pgfqpoint{0.566667in}{0.166667in}}%
\pgfpathcurveto{\pgfqpoint{0.566667in}{0.162247in}}{\pgfqpoint{0.568423in}{0.158007in}}{\pgfqpoint{0.571548in}{0.154882in}}%
\pgfpathcurveto{\pgfqpoint{0.574674in}{0.151756in}}{\pgfqpoint{0.578913in}{0.150000in}}{\pgfqpoint{0.583333in}{0.150000in}}%
\pgfpathclose%
\pgfpathmoveto{\pgfqpoint{0.750000in}{0.150000in}}%
\pgfpathcurveto{\pgfqpoint{0.754420in}{0.150000in}}{\pgfqpoint{0.758660in}{0.151756in}}{\pgfqpoint{0.761785in}{0.154882in}}%
\pgfpathcurveto{\pgfqpoint{0.764911in}{0.158007in}}{\pgfqpoint{0.766667in}{0.162247in}}{\pgfqpoint{0.766667in}{0.166667in}}%
\pgfpathcurveto{\pgfqpoint{0.766667in}{0.171087in}}{\pgfqpoint{0.764911in}{0.175326in}}{\pgfqpoint{0.761785in}{0.178452in}}%
\pgfpathcurveto{\pgfqpoint{0.758660in}{0.181577in}}{\pgfqpoint{0.754420in}{0.183333in}}{\pgfqpoint{0.750000in}{0.183333in}}%
\pgfpathcurveto{\pgfqpoint{0.745580in}{0.183333in}}{\pgfqpoint{0.741340in}{0.181577in}}{\pgfqpoint{0.738215in}{0.178452in}}%
\pgfpathcurveto{\pgfqpoint{0.735089in}{0.175326in}}{\pgfqpoint{0.733333in}{0.171087in}}{\pgfqpoint{0.733333in}{0.166667in}}%
\pgfpathcurveto{\pgfqpoint{0.733333in}{0.162247in}}{\pgfqpoint{0.735089in}{0.158007in}}{\pgfqpoint{0.738215in}{0.154882in}}%
\pgfpathcurveto{\pgfqpoint{0.741340in}{0.151756in}}{\pgfqpoint{0.745580in}{0.150000in}}{\pgfqpoint{0.750000in}{0.150000in}}%
\pgfpathclose%
\pgfpathmoveto{\pgfqpoint{0.916667in}{0.150000in}}%
\pgfpathcurveto{\pgfqpoint{0.921087in}{0.150000in}}{\pgfqpoint{0.925326in}{0.151756in}}{\pgfqpoint{0.928452in}{0.154882in}}%
\pgfpathcurveto{\pgfqpoint{0.931577in}{0.158007in}}{\pgfqpoint{0.933333in}{0.162247in}}{\pgfqpoint{0.933333in}{0.166667in}}%
\pgfpathcurveto{\pgfqpoint{0.933333in}{0.171087in}}{\pgfqpoint{0.931577in}{0.175326in}}{\pgfqpoint{0.928452in}{0.178452in}}%
\pgfpathcurveto{\pgfqpoint{0.925326in}{0.181577in}}{\pgfqpoint{0.921087in}{0.183333in}}{\pgfqpoint{0.916667in}{0.183333in}}%
\pgfpathcurveto{\pgfqpoint{0.912247in}{0.183333in}}{\pgfqpoint{0.908007in}{0.181577in}}{\pgfqpoint{0.904882in}{0.178452in}}%
\pgfpathcurveto{\pgfqpoint{0.901756in}{0.175326in}}{\pgfqpoint{0.900000in}{0.171087in}}{\pgfqpoint{0.900000in}{0.166667in}}%
\pgfpathcurveto{\pgfqpoint{0.900000in}{0.162247in}}{\pgfqpoint{0.901756in}{0.158007in}}{\pgfqpoint{0.904882in}{0.154882in}}%
\pgfpathcurveto{\pgfqpoint{0.908007in}{0.151756in}}{\pgfqpoint{0.912247in}{0.150000in}}{\pgfqpoint{0.916667in}{0.150000in}}%
\pgfpathclose%
\pgfpathmoveto{\pgfqpoint{0.000000in}{0.316667in}}%
\pgfpathcurveto{\pgfqpoint{0.004420in}{0.316667in}}{\pgfqpoint{0.008660in}{0.318423in}}{\pgfqpoint{0.011785in}{0.321548in}}%
\pgfpathcurveto{\pgfqpoint{0.014911in}{0.324674in}}{\pgfqpoint{0.016667in}{0.328913in}}{\pgfqpoint{0.016667in}{0.333333in}}%
\pgfpathcurveto{\pgfqpoint{0.016667in}{0.337753in}}{\pgfqpoint{0.014911in}{0.341993in}}{\pgfqpoint{0.011785in}{0.345118in}}%
\pgfpathcurveto{\pgfqpoint{0.008660in}{0.348244in}}{\pgfqpoint{0.004420in}{0.350000in}}{\pgfqpoint{0.000000in}{0.350000in}}%
\pgfpathcurveto{\pgfqpoint{-0.004420in}{0.350000in}}{\pgfqpoint{-0.008660in}{0.348244in}}{\pgfqpoint{-0.011785in}{0.345118in}}%
\pgfpathcurveto{\pgfqpoint{-0.014911in}{0.341993in}}{\pgfqpoint{-0.016667in}{0.337753in}}{\pgfqpoint{-0.016667in}{0.333333in}}%
\pgfpathcurveto{\pgfqpoint{-0.016667in}{0.328913in}}{\pgfqpoint{-0.014911in}{0.324674in}}{\pgfqpoint{-0.011785in}{0.321548in}}%
\pgfpathcurveto{\pgfqpoint{-0.008660in}{0.318423in}}{\pgfqpoint{-0.004420in}{0.316667in}}{\pgfqpoint{0.000000in}{0.316667in}}%
\pgfpathclose%
\pgfpathmoveto{\pgfqpoint{0.166667in}{0.316667in}}%
\pgfpathcurveto{\pgfqpoint{0.171087in}{0.316667in}}{\pgfqpoint{0.175326in}{0.318423in}}{\pgfqpoint{0.178452in}{0.321548in}}%
\pgfpathcurveto{\pgfqpoint{0.181577in}{0.324674in}}{\pgfqpoint{0.183333in}{0.328913in}}{\pgfqpoint{0.183333in}{0.333333in}}%
\pgfpathcurveto{\pgfqpoint{0.183333in}{0.337753in}}{\pgfqpoint{0.181577in}{0.341993in}}{\pgfqpoint{0.178452in}{0.345118in}}%
\pgfpathcurveto{\pgfqpoint{0.175326in}{0.348244in}}{\pgfqpoint{0.171087in}{0.350000in}}{\pgfqpoint{0.166667in}{0.350000in}}%
\pgfpathcurveto{\pgfqpoint{0.162247in}{0.350000in}}{\pgfqpoint{0.158007in}{0.348244in}}{\pgfqpoint{0.154882in}{0.345118in}}%
\pgfpathcurveto{\pgfqpoint{0.151756in}{0.341993in}}{\pgfqpoint{0.150000in}{0.337753in}}{\pgfqpoint{0.150000in}{0.333333in}}%
\pgfpathcurveto{\pgfqpoint{0.150000in}{0.328913in}}{\pgfqpoint{0.151756in}{0.324674in}}{\pgfqpoint{0.154882in}{0.321548in}}%
\pgfpathcurveto{\pgfqpoint{0.158007in}{0.318423in}}{\pgfqpoint{0.162247in}{0.316667in}}{\pgfqpoint{0.166667in}{0.316667in}}%
\pgfpathclose%
\pgfpathmoveto{\pgfqpoint{0.333333in}{0.316667in}}%
\pgfpathcurveto{\pgfqpoint{0.337753in}{0.316667in}}{\pgfqpoint{0.341993in}{0.318423in}}{\pgfqpoint{0.345118in}{0.321548in}}%
\pgfpathcurveto{\pgfqpoint{0.348244in}{0.324674in}}{\pgfqpoint{0.350000in}{0.328913in}}{\pgfqpoint{0.350000in}{0.333333in}}%
\pgfpathcurveto{\pgfqpoint{0.350000in}{0.337753in}}{\pgfqpoint{0.348244in}{0.341993in}}{\pgfqpoint{0.345118in}{0.345118in}}%
\pgfpathcurveto{\pgfqpoint{0.341993in}{0.348244in}}{\pgfqpoint{0.337753in}{0.350000in}}{\pgfqpoint{0.333333in}{0.350000in}}%
\pgfpathcurveto{\pgfqpoint{0.328913in}{0.350000in}}{\pgfqpoint{0.324674in}{0.348244in}}{\pgfqpoint{0.321548in}{0.345118in}}%
\pgfpathcurveto{\pgfqpoint{0.318423in}{0.341993in}}{\pgfqpoint{0.316667in}{0.337753in}}{\pgfqpoint{0.316667in}{0.333333in}}%
\pgfpathcurveto{\pgfqpoint{0.316667in}{0.328913in}}{\pgfqpoint{0.318423in}{0.324674in}}{\pgfqpoint{0.321548in}{0.321548in}}%
\pgfpathcurveto{\pgfqpoint{0.324674in}{0.318423in}}{\pgfqpoint{0.328913in}{0.316667in}}{\pgfqpoint{0.333333in}{0.316667in}}%
\pgfpathclose%
\pgfpathmoveto{\pgfqpoint{0.500000in}{0.316667in}}%
\pgfpathcurveto{\pgfqpoint{0.504420in}{0.316667in}}{\pgfqpoint{0.508660in}{0.318423in}}{\pgfqpoint{0.511785in}{0.321548in}}%
\pgfpathcurveto{\pgfqpoint{0.514911in}{0.324674in}}{\pgfqpoint{0.516667in}{0.328913in}}{\pgfqpoint{0.516667in}{0.333333in}}%
\pgfpathcurveto{\pgfqpoint{0.516667in}{0.337753in}}{\pgfqpoint{0.514911in}{0.341993in}}{\pgfqpoint{0.511785in}{0.345118in}}%
\pgfpathcurveto{\pgfqpoint{0.508660in}{0.348244in}}{\pgfqpoint{0.504420in}{0.350000in}}{\pgfqpoint{0.500000in}{0.350000in}}%
\pgfpathcurveto{\pgfqpoint{0.495580in}{0.350000in}}{\pgfqpoint{0.491340in}{0.348244in}}{\pgfqpoint{0.488215in}{0.345118in}}%
\pgfpathcurveto{\pgfqpoint{0.485089in}{0.341993in}}{\pgfqpoint{0.483333in}{0.337753in}}{\pgfqpoint{0.483333in}{0.333333in}}%
\pgfpathcurveto{\pgfqpoint{0.483333in}{0.328913in}}{\pgfqpoint{0.485089in}{0.324674in}}{\pgfqpoint{0.488215in}{0.321548in}}%
\pgfpathcurveto{\pgfqpoint{0.491340in}{0.318423in}}{\pgfqpoint{0.495580in}{0.316667in}}{\pgfqpoint{0.500000in}{0.316667in}}%
\pgfpathclose%
\pgfpathmoveto{\pgfqpoint{0.666667in}{0.316667in}}%
\pgfpathcurveto{\pgfqpoint{0.671087in}{0.316667in}}{\pgfqpoint{0.675326in}{0.318423in}}{\pgfqpoint{0.678452in}{0.321548in}}%
\pgfpathcurveto{\pgfqpoint{0.681577in}{0.324674in}}{\pgfqpoint{0.683333in}{0.328913in}}{\pgfqpoint{0.683333in}{0.333333in}}%
\pgfpathcurveto{\pgfqpoint{0.683333in}{0.337753in}}{\pgfqpoint{0.681577in}{0.341993in}}{\pgfqpoint{0.678452in}{0.345118in}}%
\pgfpathcurveto{\pgfqpoint{0.675326in}{0.348244in}}{\pgfqpoint{0.671087in}{0.350000in}}{\pgfqpoint{0.666667in}{0.350000in}}%
\pgfpathcurveto{\pgfqpoint{0.662247in}{0.350000in}}{\pgfqpoint{0.658007in}{0.348244in}}{\pgfqpoint{0.654882in}{0.345118in}}%
\pgfpathcurveto{\pgfqpoint{0.651756in}{0.341993in}}{\pgfqpoint{0.650000in}{0.337753in}}{\pgfqpoint{0.650000in}{0.333333in}}%
\pgfpathcurveto{\pgfqpoint{0.650000in}{0.328913in}}{\pgfqpoint{0.651756in}{0.324674in}}{\pgfqpoint{0.654882in}{0.321548in}}%
\pgfpathcurveto{\pgfqpoint{0.658007in}{0.318423in}}{\pgfqpoint{0.662247in}{0.316667in}}{\pgfqpoint{0.666667in}{0.316667in}}%
\pgfpathclose%
\pgfpathmoveto{\pgfqpoint{0.833333in}{0.316667in}}%
\pgfpathcurveto{\pgfqpoint{0.837753in}{0.316667in}}{\pgfqpoint{0.841993in}{0.318423in}}{\pgfqpoint{0.845118in}{0.321548in}}%
\pgfpathcurveto{\pgfqpoint{0.848244in}{0.324674in}}{\pgfqpoint{0.850000in}{0.328913in}}{\pgfqpoint{0.850000in}{0.333333in}}%
\pgfpathcurveto{\pgfqpoint{0.850000in}{0.337753in}}{\pgfqpoint{0.848244in}{0.341993in}}{\pgfqpoint{0.845118in}{0.345118in}}%
\pgfpathcurveto{\pgfqpoint{0.841993in}{0.348244in}}{\pgfqpoint{0.837753in}{0.350000in}}{\pgfqpoint{0.833333in}{0.350000in}}%
\pgfpathcurveto{\pgfqpoint{0.828913in}{0.350000in}}{\pgfqpoint{0.824674in}{0.348244in}}{\pgfqpoint{0.821548in}{0.345118in}}%
\pgfpathcurveto{\pgfqpoint{0.818423in}{0.341993in}}{\pgfqpoint{0.816667in}{0.337753in}}{\pgfqpoint{0.816667in}{0.333333in}}%
\pgfpathcurveto{\pgfqpoint{0.816667in}{0.328913in}}{\pgfqpoint{0.818423in}{0.324674in}}{\pgfqpoint{0.821548in}{0.321548in}}%
\pgfpathcurveto{\pgfqpoint{0.824674in}{0.318423in}}{\pgfqpoint{0.828913in}{0.316667in}}{\pgfqpoint{0.833333in}{0.316667in}}%
\pgfpathclose%
\pgfpathmoveto{\pgfqpoint{1.000000in}{0.316667in}}%
\pgfpathcurveto{\pgfqpoint{1.004420in}{0.316667in}}{\pgfqpoint{1.008660in}{0.318423in}}{\pgfqpoint{1.011785in}{0.321548in}}%
\pgfpathcurveto{\pgfqpoint{1.014911in}{0.324674in}}{\pgfqpoint{1.016667in}{0.328913in}}{\pgfqpoint{1.016667in}{0.333333in}}%
\pgfpathcurveto{\pgfqpoint{1.016667in}{0.337753in}}{\pgfqpoint{1.014911in}{0.341993in}}{\pgfqpoint{1.011785in}{0.345118in}}%
\pgfpathcurveto{\pgfqpoint{1.008660in}{0.348244in}}{\pgfqpoint{1.004420in}{0.350000in}}{\pgfqpoint{1.000000in}{0.350000in}}%
\pgfpathcurveto{\pgfqpoint{0.995580in}{0.350000in}}{\pgfqpoint{0.991340in}{0.348244in}}{\pgfqpoint{0.988215in}{0.345118in}}%
\pgfpathcurveto{\pgfqpoint{0.985089in}{0.341993in}}{\pgfqpoint{0.983333in}{0.337753in}}{\pgfqpoint{0.983333in}{0.333333in}}%
\pgfpathcurveto{\pgfqpoint{0.983333in}{0.328913in}}{\pgfqpoint{0.985089in}{0.324674in}}{\pgfqpoint{0.988215in}{0.321548in}}%
\pgfpathcurveto{\pgfqpoint{0.991340in}{0.318423in}}{\pgfqpoint{0.995580in}{0.316667in}}{\pgfqpoint{1.000000in}{0.316667in}}%
\pgfpathclose%
\pgfpathmoveto{\pgfqpoint{0.083333in}{0.483333in}}%
\pgfpathcurveto{\pgfqpoint{0.087753in}{0.483333in}}{\pgfqpoint{0.091993in}{0.485089in}}{\pgfqpoint{0.095118in}{0.488215in}}%
\pgfpathcurveto{\pgfqpoint{0.098244in}{0.491340in}}{\pgfqpoint{0.100000in}{0.495580in}}{\pgfqpoint{0.100000in}{0.500000in}}%
\pgfpathcurveto{\pgfqpoint{0.100000in}{0.504420in}}{\pgfqpoint{0.098244in}{0.508660in}}{\pgfqpoint{0.095118in}{0.511785in}}%
\pgfpathcurveto{\pgfqpoint{0.091993in}{0.514911in}}{\pgfqpoint{0.087753in}{0.516667in}}{\pgfqpoint{0.083333in}{0.516667in}}%
\pgfpathcurveto{\pgfqpoint{0.078913in}{0.516667in}}{\pgfqpoint{0.074674in}{0.514911in}}{\pgfqpoint{0.071548in}{0.511785in}}%
\pgfpathcurveto{\pgfqpoint{0.068423in}{0.508660in}}{\pgfqpoint{0.066667in}{0.504420in}}{\pgfqpoint{0.066667in}{0.500000in}}%
\pgfpathcurveto{\pgfqpoint{0.066667in}{0.495580in}}{\pgfqpoint{0.068423in}{0.491340in}}{\pgfqpoint{0.071548in}{0.488215in}}%
\pgfpathcurveto{\pgfqpoint{0.074674in}{0.485089in}}{\pgfqpoint{0.078913in}{0.483333in}}{\pgfqpoint{0.083333in}{0.483333in}}%
\pgfpathclose%
\pgfpathmoveto{\pgfqpoint{0.250000in}{0.483333in}}%
\pgfpathcurveto{\pgfqpoint{0.254420in}{0.483333in}}{\pgfqpoint{0.258660in}{0.485089in}}{\pgfqpoint{0.261785in}{0.488215in}}%
\pgfpathcurveto{\pgfqpoint{0.264911in}{0.491340in}}{\pgfqpoint{0.266667in}{0.495580in}}{\pgfqpoint{0.266667in}{0.500000in}}%
\pgfpathcurveto{\pgfqpoint{0.266667in}{0.504420in}}{\pgfqpoint{0.264911in}{0.508660in}}{\pgfqpoint{0.261785in}{0.511785in}}%
\pgfpathcurveto{\pgfqpoint{0.258660in}{0.514911in}}{\pgfqpoint{0.254420in}{0.516667in}}{\pgfqpoint{0.250000in}{0.516667in}}%
\pgfpathcurveto{\pgfqpoint{0.245580in}{0.516667in}}{\pgfqpoint{0.241340in}{0.514911in}}{\pgfqpoint{0.238215in}{0.511785in}}%
\pgfpathcurveto{\pgfqpoint{0.235089in}{0.508660in}}{\pgfqpoint{0.233333in}{0.504420in}}{\pgfqpoint{0.233333in}{0.500000in}}%
\pgfpathcurveto{\pgfqpoint{0.233333in}{0.495580in}}{\pgfqpoint{0.235089in}{0.491340in}}{\pgfqpoint{0.238215in}{0.488215in}}%
\pgfpathcurveto{\pgfqpoint{0.241340in}{0.485089in}}{\pgfqpoint{0.245580in}{0.483333in}}{\pgfqpoint{0.250000in}{0.483333in}}%
\pgfpathclose%
\pgfpathmoveto{\pgfqpoint{0.416667in}{0.483333in}}%
\pgfpathcurveto{\pgfqpoint{0.421087in}{0.483333in}}{\pgfqpoint{0.425326in}{0.485089in}}{\pgfqpoint{0.428452in}{0.488215in}}%
\pgfpathcurveto{\pgfqpoint{0.431577in}{0.491340in}}{\pgfqpoint{0.433333in}{0.495580in}}{\pgfqpoint{0.433333in}{0.500000in}}%
\pgfpathcurveto{\pgfqpoint{0.433333in}{0.504420in}}{\pgfqpoint{0.431577in}{0.508660in}}{\pgfqpoint{0.428452in}{0.511785in}}%
\pgfpathcurveto{\pgfqpoint{0.425326in}{0.514911in}}{\pgfqpoint{0.421087in}{0.516667in}}{\pgfqpoint{0.416667in}{0.516667in}}%
\pgfpathcurveto{\pgfqpoint{0.412247in}{0.516667in}}{\pgfqpoint{0.408007in}{0.514911in}}{\pgfqpoint{0.404882in}{0.511785in}}%
\pgfpathcurveto{\pgfqpoint{0.401756in}{0.508660in}}{\pgfqpoint{0.400000in}{0.504420in}}{\pgfqpoint{0.400000in}{0.500000in}}%
\pgfpathcurveto{\pgfqpoint{0.400000in}{0.495580in}}{\pgfqpoint{0.401756in}{0.491340in}}{\pgfqpoint{0.404882in}{0.488215in}}%
\pgfpathcurveto{\pgfqpoint{0.408007in}{0.485089in}}{\pgfqpoint{0.412247in}{0.483333in}}{\pgfqpoint{0.416667in}{0.483333in}}%
\pgfpathclose%
\pgfpathmoveto{\pgfqpoint{0.583333in}{0.483333in}}%
\pgfpathcurveto{\pgfqpoint{0.587753in}{0.483333in}}{\pgfqpoint{0.591993in}{0.485089in}}{\pgfqpoint{0.595118in}{0.488215in}}%
\pgfpathcurveto{\pgfqpoint{0.598244in}{0.491340in}}{\pgfqpoint{0.600000in}{0.495580in}}{\pgfqpoint{0.600000in}{0.500000in}}%
\pgfpathcurveto{\pgfqpoint{0.600000in}{0.504420in}}{\pgfqpoint{0.598244in}{0.508660in}}{\pgfqpoint{0.595118in}{0.511785in}}%
\pgfpathcurveto{\pgfqpoint{0.591993in}{0.514911in}}{\pgfqpoint{0.587753in}{0.516667in}}{\pgfqpoint{0.583333in}{0.516667in}}%
\pgfpathcurveto{\pgfqpoint{0.578913in}{0.516667in}}{\pgfqpoint{0.574674in}{0.514911in}}{\pgfqpoint{0.571548in}{0.511785in}}%
\pgfpathcurveto{\pgfqpoint{0.568423in}{0.508660in}}{\pgfqpoint{0.566667in}{0.504420in}}{\pgfqpoint{0.566667in}{0.500000in}}%
\pgfpathcurveto{\pgfqpoint{0.566667in}{0.495580in}}{\pgfqpoint{0.568423in}{0.491340in}}{\pgfqpoint{0.571548in}{0.488215in}}%
\pgfpathcurveto{\pgfqpoint{0.574674in}{0.485089in}}{\pgfqpoint{0.578913in}{0.483333in}}{\pgfqpoint{0.583333in}{0.483333in}}%
\pgfpathclose%
\pgfpathmoveto{\pgfqpoint{0.750000in}{0.483333in}}%
\pgfpathcurveto{\pgfqpoint{0.754420in}{0.483333in}}{\pgfqpoint{0.758660in}{0.485089in}}{\pgfqpoint{0.761785in}{0.488215in}}%
\pgfpathcurveto{\pgfqpoint{0.764911in}{0.491340in}}{\pgfqpoint{0.766667in}{0.495580in}}{\pgfqpoint{0.766667in}{0.500000in}}%
\pgfpathcurveto{\pgfqpoint{0.766667in}{0.504420in}}{\pgfqpoint{0.764911in}{0.508660in}}{\pgfqpoint{0.761785in}{0.511785in}}%
\pgfpathcurveto{\pgfqpoint{0.758660in}{0.514911in}}{\pgfqpoint{0.754420in}{0.516667in}}{\pgfqpoint{0.750000in}{0.516667in}}%
\pgfpathcurveto{\pgfqpoint{0.745580in}{0.516667in}}{\pgfqpoint{0.741340in}{0.514911in}}{\pgfqpoint{0.738215in}{0.511785in}}%
\pgfpathcurveto{\pgfqpoint{0.735089in}{0.508660in}}{\pgfqpoint{0.733333in}{0.504420in}}{\pgfqpoint{0.733333in}{0.500000in}}%
\pgfpathcurveto{\pgfqpoint{0.733333in}{0.495580in}}{\pgfqpoint{0.735089in}{0.491340in}}{\pgfqpoint{0.738215in}{0.488215in}}%
\pgfpathcurveto{\pgfqpoint{0.741340in}{0.485089in}}{\pgfqpoint{0.745580in}{0.483333in}}{\pgfqpoint{0.750000in}{0.483333in}}%
\pgfpathclose%
\pgfpathmoveto{\pgfqpoint{0.916667in}{0.483333in}}%
\pgfpathcurveto{\pgfqpoint{0.921087in}{0.483333in}}{\pgfqpoint{0.925326in}{0.485089in}}{\pgfqpoint{0.928452in}{0.488215in}}%
\pgfpathcurveto{\pgfqpoint{0.931577in}{0.491340in}}{\pgfqpoint{0.933333in}{0.495580in}}{\pgfqpoint{0.933333in}{0.500000in}}%
\pgfpathcurveto{\pgfqpoint{0.933333in}{0.504420in}}{\pgfqpoint{0.931577in}{0.508660in}}{\pgfqpoint{0.928452in}{0.511785in}}%
\pgfpathcurveto{\pgfqpoint{0.925326in}{0.514911in}}{\pgfqpoint{0.921087in}{0.516667in}}{\pgfqpoint{0.916667in}{0.516667in}}%
\pgfpathcurveto{\pgfqpoint{0.912247in}{0.516667in}}{\pgfqpoint{0.908007in}{0.514911in}}{\pgfqpoint{0.904882in}{0.511785in}}%
\pgfpathcurveto{\pgfqpoint{0.901756in}{0.508660in}}{\pgfqpoint{0.900000in}{0.504420in}}{\pgfqpoint{0.900000in}{0.500000in}}%
\pgfpathcurveto{\pgfqpoint{0.900000in}{0.495580in}}{\pgfqpoint{0.901756in}{0.491340in}}{\pgfqpoint{0.904882in}{0.488215in}}%
\pgfpathcurveto{\pgfqpoint{0.908007in}{0.485089in}}{\pgfqpoint{0.912247in}{0.483333in}}{\pgfqpoint{0.916667in}{0.483333in}}%
\pgfpathclose%
\pgfpathmoveto{\pgfqpoint{0.000000in}{0.650000in}}%
\pgfpathcurveto{\pgfqpoint{0.004420in}{0.650000in}}{\pgfqpoint{0.008660in}{0.651756in}}{\pgfqpoint{0.011785in}{0.654882in}}%
\pgfpathcurveto{\pgfqpoint{0.014911in}{0.658007in}}{\pgfqpoint{0.016667in}{0.662247in}}{\pgfqpoint{0.016667in}{0.666667in}}%
\pgfpathcurveto{\pgfqpoint{0.016667in}{0.671087in}}{\pgfqpoint{0.014911in}{0.675326in}}{\pgfqpoint{0.011785in}{0.678452in}}%
\pgfpathcurveto{\pgfqpoint{0.008660in}{0.681577in}}{\pgfqpoint{0.004420in}{0.683333in}}{\pgfqpoint{0.000000in}{0.683333in}}%
\pgfpathcurveto{\pgfqpoint{-0.004420in}{0.683333in}}{\pgfqpoint{-0.008660in}{0.681577in}}{\pgfqpoint{-0.011785in}{0.678452in}}%
\pgfpathcurveto{\pgfqpoint{-0.014911in}{0.675326in}}{\pgfqpoint{-0.016667in}{0.671087in}}{\pgfqpoint{-0.016667in}{0.666667in}}%
\pgfpathcurveto{\pgfqpoint{-0.016667in}{0.662247in}}{\pgfqpoint{-0.014911in}{0.658007in}}{\pgfqpoint{-0.011785in}{0.654882in}}%
\pgfpathcurveto{\pgfqpoint{-0.008660in}{0.651756in}}{\pgfqpoint{-0.004420in}{0.650000in}}{\pgfqpoint{0.000000in}{0.650000in}}%
\pgfpathclose%
\pgfpathmoveto{\pgfqpoint{0.166667in}{0.650000in}}%
\pgfpathcurveto{\pgfqpoint{0.171087in}{0.650000in}}{\pgfqpoint{0.175326in}{0.651756in}}{\pgfqpoint{0.178452in}{0.654882in}}%
\pgfpathcurveto{\pgfqpoint{0.181577in}{0.658007in}}{\pgfqpoint{0.183333in}{0.662247in}}{\pgfqpoint{0.183333in}{0.666667in}}%
\pgfpathcurveto{\pgfqpoint{0.183333in}{0.671087in}}{\pgfqpoint{0.181577in}{0.675326in}}{\pgfqpoint{0.178452in}{0.678452in}}%
\pgfpathcurveto{\pgfqpoint{0.175326in}{0.681577in}}{\pgfqpoint{0.171087in}{0.683333in}}{\pgfqpoint{0.166667in}{0.683333in}}%
\pgfpathcurveto{\pgfqpoint{0.162247in}{0.683333in}}{\pgfqpoint{0.158007in}{0.681577in}}{\pgfqpoint{0.154882in}{0.678452in}}%
\pgfpathcurveto{\pgfqpoint{0.151756in}{0.675326in}}{\pgfqpoint{0.150000in}{0.671087in}}{\pgfqpoint{0.150000in}{0.666667in}}%
\pgfpathcurveto{\pgfqpoint{0.150000in}{0.662247in}}{\pgfqpoint{0.151756in}{0.658007in}}{\pgfqpoint{0.154882in}{0.654882in}}%
\pgfpathcurveto{\pgfqpoint{0.158007in}{0.651756in}}{\pgfqpoint{0.162247in}{0.650000in}}{\pgfqpoint{0.166667in}{0.650000in}}%
\pgfpathclose%
\pgfpathmoveto{\pgfqpoint{0.333333in}{0.650000in}}%
\pgfpathcurveto{\pgfqpoint{0.337753in}{0.650000in}}{\pgfqpoint{0.341993in}{0.651756in}}{\pgfqpoint{0.345118in}{0.654882in}}%
\pgfpathcurveto{\pgfqpoint{0.348244in}{0.658007in}}{\pgfqpoint{0.350000in}{0.662247in}}{\pgfqpoint{0.350000in}{0.666667in}}%
\pgfpathcurveto{\pgfqpoint{0.350000in}{0.671087in}}{\pgfqpoint{0.348244in}{0.675326in}}{\pgfqpoint{0.345118in}{0.678452in}}%
\pgfpathcurveto{\pgfqpoint{0.341993in}{0.681577in}}{\pgfqpoint{0.337753in}{0.683333in}}{\pgfqpoint{0.333333in}{0.683333in}}%
\pgfpathcurveto{\pgfqpoint{0.328913in}{0.683333in}}{\pgfqpoint{0.324674in}{0.681577in}}{\pgfqpoint{0.321548in}{0.678452in}}%
\pgfpathcurveto{\pgfqpoint{0.318423in}{0.675326in}}{\pgfqpoint{0.316667in}{0.671087in}}{\pgfqpoint{0.316667in}{0.666667in}}%
\pgfpathcurveto{\pgfqpoint{0.316667in}{0.662247in}}{\pgfqpoint{0.318423in}{0.658007in}}{\pgfqpoint{0.321548in}{0.654882in}}%
\pgfpathcurveto{\pgfqpoint{0.324674in}{0.651756in}}{\pgfqpoint{0.328913in}{0.650000in}}{\pgfqpoint{0.333333in}{0.650000in}}%
\pgfpathclose%
\pgfpathmoveto{\pgfqpoint{0.500000in}{0.650000in}}%
\pgfpathcurveto{\pgfqpoint{0.504420in}{0.650000in}}{\pgfqpoint{0.508660in}{0.651756in}}{\pgfqpoint{0.511785in}{0.654882in}}%
\pgfpathcurveto{\pgfqpoint{0.514911in}{0.658007in}}{\pgfqpoint{0.516667in}{0.662247in}}{\pgfqpoint{0.516667in}{0.666667in}}%
\pgfpathcurveto{\pgfqpoint{0.516667in}{0.671087in}}{\pgfqpoint{0.514911in}{0.675326in}}{\pgfqpoint{0.511785in}{0.678452in}}%
\pgfpathcurveto{\pgfqpoint{0.508660in}{0.681577in}}{\pgfqpoint{0.504420in}{0.683333in}}{\pgfqpoint{0.500000in}{0.683333in}}%
\pgfpathcurveto{\pgfqpoint{0.495580in}{0.683333in}}{\pgfqpoint{0.491340in}{0.681577in}}{\pgfqpoint{0.488215in}{0.678452in}}%
\pgfpathcurveto{\pgfqpoint{0.485089in}{0.675326in}}{\pgfqpoint{0.483333in}{0.671087in}}{\pgfqpoint{0.483333in}{0.666667in}}%
\pgfpathcurveto{\pgfqpoint{0.483333in}{0.662247in}}{\pgfqpoint{0.485089in}{0.658007in}}{\pgfqpoint{0.488215in}{0.654882in}}%
\pgfpathcurveto{\pgfqpoint{0.491340in}{0.651756in}}{\pgfqpoint{0.495580in}{0.650000in}}{\pgfqpoint{0.500000in}{0.650000in}}%
\pgfpathclose%
\pgfpathmoveto{\pgfqpoint{0.666667in}{0.650000in}}%
\pgfpathcurveto{\pgfqpoint{0.671087in}{0.650000in}}{\pgfqpoint{0.675326in}{0.651756in}}{\pgfqpoint{0.678452in}{0.654882in}}%
\pgfpathcurveto{\pgfqpoint{0.681577in}{0.658007in}}{\pgfqpoint{0.683333in}{0.662247in}}{\pgfqpoint{0.683333in}{0.666667in}}%
\pgfpathcurveto{\pgfqpoint{0.683333in}{0.671087in}}{\pgfqpoint{0.681577in}{0.675326in}}{\pgfqpoint{0.678452in}{0.678452in}}%
\pgfpathcurveto{\pgfqpoint{0.675326in}{0.681577in}}{\pgfqpoint{0.671087in}{0.683333in}}{\pgfqpoint{0.666667in}{0.683333in}}%
\pgfpathcurveto{\pgfqpoint{0.662247in}{0.683333in}}{\pgfqpoint{0.658007in}{0.681577in}}{\pgfqpoint{0.654882in}{0.678452in}}%
\pgfpathcurveto{\pgfqpoint{0.651756in}{0.675326in}}{\pgfqpoint{0.650000in}{0.671087in}}{\pgfqpoint{0.650000in}{0.666667in}}%
\pgfpathcurveto{\pgfqpoint{0.650000in}{0.662247in}}{\pgfqpoint{0.651756in}{0.658007in}}{\pgfqpoint{0.654882in}{0.654882in}}%
\pgfpathcurveto{\pgfqpoint{0.658007in}{0.651756in}}{\pgfqpoint{0.662247in}{0.650000in}}{\pgfqpoint{0.666667in}{0.650000in}}%
\pgfpathclose%
\pgfpathmoveto{\pgfqpoint{0.833333in}{0.650000in}}%
\pgfpathcurveto{\pgfqpoint{0.837753in}{0.650000in}}{\pgfqpoint{0.841993in}{0.651756in}}{\pgfqpoint{0.845118in}{0.654882in}}%
\pgfpathcurveto{\pgfqpoint{0.848244in}{0.658007in}}{\pgfqpoint{0.850000in}{0.662247in}}{\pgfqpoint{0.850000in}{0.666667in}}%
\pgfpathcurveto{\pgfqpoint{0.850000in}{0.671087in}}{\pgfqpoint{0.848244in}{0.675326in}}{\pgfqpoint{0.845118in}{0.678452in}}%
\pgfpathcurveto{\pgfqpoint{0.841993in}{0.681577in}}{\pgfqpoint{0.837753in}{0.683333in}}{\pgfqpoint{0.833333in}{0.683333in}}%
\pgfpathcurveto{\pgfqpoint{0.828913in}{0.683333in}}{\pgfqpoint{0.824674in}{0.681577in}}{\pgfqpoint{0.821548in}{0.678452in}}%
\pgfpathcurveto{\pgfqpoint{0.818423in}{0.675326in}}{\pgfqpoint{0.816667in}{0.671087in}}{\pgfqpoint{0.816667in}{0.666667in}}%
\pgfpathcurveto{\pgfqpoint{0.816667in}{0.662247in}}{\pgfqpoint{0.818423in}{0.658007in}}{\pgfqpoint{0.821548in}{0.654882in}}%
\pgfpathcurveto{\pgfqpoint{0.824674in}{0.651756in}}{\pgfqpoint{0.828913in}{0.650000in}}{\pgfqpoint{0.833333in}{0.650000in}}%
\pgfpathclose%
\pgfpathmoveto{\pgfqpoint{1.000000in}{0.650000in}}%
\pgfpathcurveto{\pgfqpoint{1.004420in}{0.650000in}}{\pgfqpoint{1.008660in}{0.651756in}}{\pgfqpoint{1.011785in}{0.654882in}}%
\pgfpathcurveto{\pgfqpoint{1.014911in}{0.658007in}}{\pgfqpoint{1.016667in}{0.662247in}}{\pgfqpoint{1.016667in}{0.666667in}}%
\pgfpathcurveto{\pgfqpoint{1.016667in}{0.671087in}}{\pgfqpoint{1.014911in}{0.675326in}}{\pgfqpoint{1.011785in}{0.678452in}}%
\pgfpathcurveto{\pgfqpoint{1.008660in}{0.681577in}}{\pgfqpoint{1.004420in}{0.683333in}}{\pgfqpoint{1.000000in}{0.683333in}}%
\pgfpathcurveto{\pgfqpoint{0.995580in}{0.683333in}}{\pgfqpoint{0.991340in}{0.681577in}}{\pgfqpoint{0.988215in}{0.678452in}}%
\pgfpathcurveto{\pgfqpoint{0.985089in}{0.675326in}}{\pgfqpoint{0.983333in}{0.671087in}}{\pgfqpoint{0.983333in}{0.666667in}}%
\pgfpathcurveto{\pgfqpoint{0.983333in}{0.662247in}}{\pgfqpoint{0.985089in}{0.658007in}}{\pgfqpoint{0.988215in}{0.654882in}}%
\pgfpathcurveto{\pgfqpoint{0.991340in}{0.651756in}}{\pgfqpoint{0.995580in}{0.650000in}}{\pgfqpoint{1.000000in}{0.650000in}}%
\pgfpathclose%
\pgfpathmoveto{\pgfqpoint{0.083333in}{0.816667in}}%
\pgfpathcurveto{\pgfqpoint{0.087753in}{0.816667in}}{\pgfqpoint{0.091993in}{0.818423in}}{\pgfqpoint{0.095118in}{0.821548in}}%
\pgfpathcurveto{\pgfqpoint{0.098244in}{0.824674in}}{\pgfqpoint{0.100000in}{0.828913in}}{\pgfqpoint{0.100000in}{0.833333in}}%
\pgfpathcurveto{\pgfqpoint{0.100000in}{0.837753in}}{\pgfqpoint{0.098244in}{0.841993in}}{\pgfqpoint{0.095118in}{0.845118in}}%
\pgfpathcurveto{\pgfqpoint{0.091993in}{0.848244in}}{\pgfqpoint{0.087753in}{0.850000in}}{\pgfqpoint{0.083333in}{0.850000in}}%
\pgfpathcurveto{\pgfqpoint{0.078913in}{0.850000in}}{\pgfqpoint{0.074674in}{0.848244in}}{\pgfqpoint{0.071548in}{0.845118in}}%
\pgfpathcurveto{\pgfqpoint{0.068423in}{0.841993in}}{\pgfqpoint{0.066667in}{0.837753in}}{\pgfqpoint{0.066667in}{0.833333in}}%
\pgfpathcurveto{\pgfqpoint{0.066667in}{0.828913in}}{\pgfqpoint{0.068423in}{0.824674in}}{\pgfqpoint{0.071548in}{0.821548in}}%
\pgfpathcurveto{\pgfqpoint{0.074674in}{0.818423in}}{\pgfqpoint{0.078913in}{0.816667in}}{\pgfqpoint{0.083333in}{0.816667in}}%
\pgfpathclose%
\pgfpathmoveto{\pgfqpoint{0.250000in}{0.816667in}}%
\pgfpathcurveto{\pgfqpoint{0.254420in}{0.816667in}}{\pgfqpoint{0.258660in}{0.818423in}}{\pgfqpoint{0.261785in}{0.821548in}}%
\pgfpathcurveto{\pgfqpoint{0.264911in}{0.824674in}}{\pgfqpoint{0.266667in}{0.828913in}}{\pgfqpoint{0.266667in}{0.833333in}}%
\pgfpathcurveto{\pgfqpoint{0.266667in}{0.837753in}}{\pgfqpoint{0.264911in}{0.841993in}}{\pgfqpoint{0.261785in}{0.845118in}}%
\pgfpathcurveto{\pgfqpoint{0.258660in}{0.848244in}}{\pgfqpoint{0.254420in}{0.850000in}}{\pgfqpoint{0.250000in}{0.850000in}}%
\pgfpathcurveto{\pgfqpoint{0.245580in}{0.850000in}}{\pgfqpoint{0.241340in}{0.848244in}}{\pgfqpoint{0.238215in}{0.845118in}}%
\pgfpathcurveto{\pgfqpoint{0.235089in}{0.841993in}}{\pgfqpoint{0.233333in}{0.837753in}}{\pgfqpoint{0.233333in}{0.833333in}}%
\pgfpathcurveto{\pgfqpoint{0.233333in}{0.828913in}}{\pgfqpoint{0.235089in}{0.824674in}}{\pgfqpoint{0.238215in}{0.821548in}}%
\pgfpathcurveto{\pgfqpoint{0.241340in}{0.818423in}}{\pgfqpoint{0.245580in}{0.816667in}}{\pgfqpoint{0.250000in}{0.816667in}}%
\pgfpathclose%
\pgfpathmoveto{\pgfqpoint{0.416667in}{0.816667in}}%
\pgfpathcurveto{\pgfqpoint{0.421087in}{0.816667in}}{\pgfqpoint{0.425326in}{0.818423in}}{\pgfqpoint{0.428452in}{0.821548in}}%
\pgfpathcurveto{\pgfqpoint{0.431577in}{0.824674in}}{\pgfqpoint{0.433333in}{0.828913in}}{\pgfqpoint{0.433333in}{0.833333in}}%
\pgfpathcurveto{\pgfqpoint{0.433333in}{0.837753in}}{\pgfqpoint{0.431577in}{0.841993in}}{\pgfqpoint{0.428452in}{0.845118in}}%
\pgfpathcurveto{\pgfqpoint{0.425326in}{0.848244in}}{\pgfqpoint{0.421087in}{0.850000in}}{\pgfqpoint{0.416667in}{0.850000in}}%
\pgfpathcurveto{\pgfqpoint{0.412247in}{0.850000in}}{\pgfqpoint{0.408007in}{0.848244in}}{\pgfqpoint{0.404882in}{0.845118in}}%
\pgfpathcurveto{\pgfqpoint{0.401756in}{0.841993in}}{\pgfqpoint{0.400000in}{0.837753in}}{\pgfqpoint{0.400000in}{0.833333in}}%
\pgfpathcurveto{\pgfqpoint{0.400000in}{0.828913in}}{\pgfqpoint{0.401756in}{0.824674in}}{\pgfqpoint{0.404882in}{0.821548in}}%
\pgfpathcurveto{\pgfqpoint{0.408007in}{0.818423in}}{\pgfqpoint{0.412247in}{0.816667in}}{\pgfqpoint{0.416667in}{0.816667in}}%
\pgfpathclose%
\pgfpathmoveto{\pgfqpoint{0.583333in}{0.816667in}}%
\pgfpathcurveto{\pgfqpoint{0.587753in}{0.816667in}}{\pgfqpoint{0.591993in}{0.818423in}}{\pgfqpoint{0.595118in}{0.821548in}}%
\pgfpathcurveto{\pgfqpoint{0.598244in}{0.824674in}}{\pgfqpoint{0.600000in}{0.828913in}}{\pgfqpoint{0.600000in}{0.833333in}}%
\pgfpathcurveto{\pgfqpoint{0.600000in}{0.837753in}}{\pgfqpoint{0.598244in}{0.841993in}}{\pgfqpoint{0.595118in}{0.845118in}}%
\pgfpathcurveto{\pgfqpoint{0.591993in}{0.848244in}}{\pgfqpoint{0.587753in}{0.850000in}}{\pgfqpoint{0.583333in}{0.850000in}}%
\pgfpathcurveto{\pgfqpoint{0.578913in}{0.850000in}}{\pgfqpoint{0.574674in}{0.848244in}}{\pgfqpoint{0.571548in}{0.845118in}}%
\pgfpathcurveto{\pgfqpoint{0.568423in}{0.841993in}}{\pgfqpoint{0.566667in}{0.837753in}}{\pgfqpoint{0.566667in}{0.833333in}}%
\pgfpathcurveto{\pgfqpoint{0.566667in}{0.828913in}}{\pgfqpoint{0.568423in}{0.824674in}}{\pgfqpoint{0.571548in}{0.821548in}}%
\pgfpathcurveto{\pgfqpoint{0.574674in}{0.818423in}}{\pgfqpoint{0.578913in}{0.816667in}}{\pgfqpoint{0.583333in}{0.816667in}}%
\pgfpathclose%
\pgfpathmoveto{\pgfqpoint{0.750000in}{0.816667in}}%
\pgfpathcurveto{\pgfqpoint{0.754420in}{0.816667in}}{\pgfqpoint{0.758660in}{0.818423in}}{\pgfqpoint{0.761785in}{0.821548in}}%
\pgfpathcurveto{\pgfqpoint{0.764911in}{0.824674in}}{\pgfqpoint{0.766667in}{0.828913in}}{\pgfqpoint{0.766667in}{0.833333in}}%
\pgfpathcurveto{\pgfqpoint{0.766667in}{0.837753in}}{\pgfqpoint{0.764911in}{0.841993in}}{\pgfqpoint{0.761785in}{0.845118in}}%
\pgfpathcurveto{\pgfqpoint{0.758660in}{0.848244in}}{\pgfqpoint{0.754420in}{0.850000in}}{\pgfqpoint{0.750000in}{0.850000in}}%
\pgfpathcurveto{\pgfqpoint{0.745580in}{0.850000in}}{\pgfqpoint{0.741340in}{0.848244in}}{\pgfqpoint{0.738215in}{0.845118in}}%
\pgfpathcurveto{\pgfqpoint{0.735089in}{0.841993in}}{\pgfqpoint{0.733333in}{0.837753in}}{\pgfqpoint{0.733333in}{0.833333in}}%
\pgfpathcurveto{\pgfqpoint{0.733333in}{0.828913in}}{\pgfqpoint{0.735089in}{0.824674in}}{\pgfqpoint{0.738215in}{0.821548in}}%
\pgfpathcurveto{\pgfqpoint{0.741340in}{0.818423in}}{\pgfqpoint{0.745580in}{0.816667in}}{\pgfqpoint{0.750000in}{0.816667in}}%
\pgfpathclose%
\pgfpathmoveto{\pgfqpoint{0.916667in}{0.816667in}}%
\pgfpathcurveto{\pgfqpoint{0.921087in}{0.816667in}}{\pgfqpoint{0.925326in}{0.818423in}}{\pgfqpoint{0.928452in}{0.821548in}}%
\pgfpathcurveto{\pgfqpoint{0.931577in}{0.824674in}}{\pgfqpoint{0.933333in}{0.828913in}}{\pgfqpoint{0.933333in}{0.833333in}}%
\pgfpathcurveto{\pgfqpoint{0.933333in}{0.837753in}}{\pgfqpoint{0.931577in}{0.841993in}}{\pgfqpoint{0.928452in}{0.845118in}}%
\pgfpathcurveto{\pgfqpoint{0.925326in}{0.848244in}}{\pgfqpoint{0.921087in}{0.850000in}}{\pgfqpoint{0.916667in}{0.850000in}}%
\pgfpathcurveto{\pgfqpoint{0.912247in}{0.850000in}}{\pgfqpoint{0.908007in}{0.848244in}}{\pgfqpoint{0.904882in}{0.845118in}}%
\pgfpathcurveto{\pgfqpoint{0.901756in}{0.841993in}}{\pgfqpoint{0.900000in}{0.837753in}}{\pgfqpoint{0.900000in}{0.833333in}}%
\pgfpathcurveto{\pgfqpoint{0.900000in}{0.828913in}}{\pgfqpoint{0.901756in}{0.824674in}}{\pgfqpoint{0.904882in}{0.821548in}}%
\pgfpathcurveto{\pgfqpoint{0.908007in}{0.818423in}}{\pgfqpoint{0.912247in}{0.816667in}}{\pgfqpoint{0.916667in}{0.816667in}}%
\pgfpathclose%
\pgfpathmoveto{\pgfqpoint{0.000000in}{0.983333in}}%
\pgfpathcurveto{\pgfqpoint{0.004420in}{0.983333in}}{\pgfqpoint{0.008660in}{0.985089in}}{\pgfqpoint{0.011785in}{0.988215in}}%
\pgfpathcurveto{\pgfqpoint{0.014911in}{0.991340in}}{\pgfqpoint{0.016667in}{0.995580in}}{\pgfqpoint{0.016667in}{1.000000in}}%
\pgfpathcurveto{\pgfqpoint{0.016667in}{1.004420in}}{\pgfqpoint{0.014911in}{1.008660in}}{\pgfqpoint{0.011785in}{1.011785in}}%
\pgfpathcurveto{\pgfqpoint{0.008660in}{1.014911in}}{\pgfqpoint{0.004420in}{1.016667in}}{\pgfqpoint{0.000000in}{1.016667in}}%
\pgfpathcurveto{\pgfqpoint{-0.004420in}{1.016667in}}{\pgfqpoint{-0.008660in}{1.014911in}}{\pgfqpoint{-0.011785in}{1.011785in}}%
\pgfpathcurveto{\pgfqpoint{-0.014911in}{1.008660in}}{\pgfqpoint{-0.016667in}{1.004420in}}{\pgfqpoint{-0.016667in}{1.000000in}}%
\pgfpathcurveto{\pgfqpoint{-0.016667in}{0.995580in}}{\pgfqpoint{-0.014911in}{0.991340in}}{\pgfqpoint{-0.011785in}{0.988215in}}%
\pgfpathcurveto{\pgfqpoint{-0.008660in}{0.985089in}}{\pgfqpoint{-0.004420in}{0.983333in}}{\pgfqpoint{0.000000in}{0.983333in}}%
\pgfpathclose%
\pgfpathmoveto{\pgfqpoint{0.166667in}{0.983333in}}%
\pgfpathcurveto{\pgfqpoint{0.171087in}{0.983333in}}{\pgfqpoint{0.175326in}{0.985089in}}{\pgfqpoint{0.178452in}{0.988215in}}%
\pgfpathcurveto{\pgfqpoint{0.181577in}{0.991340in}}{\pgfqpoint{0.183333in}{0.995580in}}{\pgfqpoint{0.183333in}{1.000000in}}%
\pgfpathcurveto{\pgfqpoint{0.183333in}{1.004420in}}{\pgfqpoint{0.181577in}{1.008660in}}{\pgfqpoint{0.178452in}{1.011785in}}%
\pgfpathcurveto{\pgfqpoint{0.175326in}{1.014911in}}{\pgfqpoint{0.171087in}{1.016667in}}{\pgfqpoint{0.166667in}{1.016667in}}%
\pgfpathcurveto{\pgfqpoint{0.162247in}{1.016667in}}{\pgfqpoint{0.158007in}{1.014911in}}{\pgfqpoint{0.154882in}{1.011785in}}%
\pgfpathcurveto{\pgfqpoint{0.151756in}{1.008660in}}{\pgfqpoint{0.150000in}{1.004420in}}{\pgfqpoint{0.150000in}{1.000000in}}%
\pgfpathcurveto{\pgfqpoint{0.150000in}{0.995580in}}{\pgfqpoint{0.151756in}{0.991340in}}{\pgfqpoint{0.154882in}{0.988215in}}%
\pgfpathcurveto{\pgfqpoint{0.158007in}{0.985089in}}{\pgfqpoint{0.162247in}{0.983333in}}{\pgfqpoint{0.166667in}{0.983333in}}%
\pgfpathclose%
\pgfpathmoveto{\pgfqpoint{0.333333in}{0.983333in}}%
\pgfpathcurveto{\pgfqpoint{0.337753in}{0.983333in}}{\pgfqpoint{0.341993in}{0.985089in}}{\pgfqpoint{0.345118in}{0.988215in}}%
\pgfpathcurveto{\pgfqpoint{0.348244in}{0.991340in}}{\pgfqpoint{0.350000in}{0.995580in}}{\pgfqpoint{0.350000in}{1.000000in}}%
\pgfpathcurveto{\pgfqpoint{0.350000in}{1.004420in}}{\pgfqpoint{0.348244in}{1.008660in}}{\pgfqpoint{0.345118in}{1.011785in}}%
\pgfpathcurveto{\pgfqpoint{0.341993in}{1.014911in}}{\pgfqpoint{0.337753in}{1.016667in}}{\pgfqpoint{0.333333in}{1.016667in}}%
\pgfpathcurveto{\pgfqpoint{0.328913in}{1.016667in}}{\pgfqpoint{0.324674in}{1.014911in}}{\pgfqpoint{0.321548in}{1.011785in}}%
\pgfpathcurveto{\pgfqpoint{0.318423in}{1.008660in}}{\pgfqpoint{0.316667in}{1.004420in}}{\pgfqpoint{0.316667in}{1.000000in}}%
\pgfpathcurveto{\pgfqpoint{0.316667in}{0.995580in}}{\pgfqpoint{0.318423in}{0.991340in}}{\pgfqpoint{0.321548in}{0.988215in}}%
\pgfpathcurveto{\pgfqpoint{0.324674in}{0.985089in}}{\pgfqpoint{0.328913in}{0.983333in}}{\pgfqpoint{0.333333in}{0.983333in}}%
\pgfpathclose%
\pgfpathmoveto{\pgfqpoint{0.500000in}{0.983333in}}%
\pgfpathcurveto{\pgfqpoint{0.504420in}{0.983333in}}{\pgfqpoint{0.508660in}{0.985089in}}{\pgfqpoint{0.511785in}{0.988215in}}%
\pgfpathcurveto{\pgfqpoint{0.514911in}{0.991340in}}{\pgfqpoint{0.516667in}{0.995580in}}{\pgfqpoint{0.516667in}{1.000000in}}%
\pgfpathcurveto{\pgfqpoint{0.516667in}{1.004420in}}{\pgfqpoint{0.514911in}{1.008660in}}{\pgfqpoint{0.511785in}{1.011785in}}%
\pgfpathcurveto{\pgfqpoint{0.508660in}{1.014911in}}{\pgfqpoint{0.504420in}{1.016667in}}{\pgfqpoint{0.500000in}{1.016667in}}%
\pgfpathcurveto{\pgfqpoint{0.495580in}{1.016667in}}{\pgfqpoint{0.491340in}{1.014911in}}{\pgfqpoint{0.488215in}{1.011785in}}%
\pgfpathcurveto{\pgfqpoint{0.485089in}{1.008660in}}{\pgfqpoint{0.483333in}{1.004420in}}{\pgfqpoint{0.483333in}{1.000000in}}%
\pgfpathcurveto{\pgfqpoint{0.483333in}{0.995580in}}{\pgfqpoint{0.485089in}{0.991340in}}{\pgfqpoint{0.488215in}{0.988215in}}%
\pgfpathcurveto{\pgfqpoint{0.491340in}{0.985089in}}{\pgfqpoint{0.495580in}{0.983333in}}{\pgfqpoint{0.500000in}{0.983333in}}%
\pgfpathclose%
\pgfpathmoveto{\pgfqpoint{0.666667in}{0.983333in}}%
\pgfpathcurveto{\pgfqpoint{0.671087in}{0.983333in}}{\pgfqpoint{0.675326in}{0.985089in}}{\pgfqpoint{0.678452in}{0.988215in}}%
\pgfpathcurveto{\pgfqpoint{0.681577in}{0.991340in}}{\pgfqpoint{0.683333in}{0.995580in}}{\pgfqpoint{0.683333in}{1.000000in}}%
\pgfpathcurveto{\pgfqpoint{0.683333in}{1.004420in}}{\pgfqpoint{0.681577in}{1.008660in}}{\pgfqpoint{0.678452in}{1.011785in}}%
\pgfpathcurveto{\pgfqpoint{0.675326in}{1.014911in}}{\pgfqpoint{0.671087in}{1.016667in}}{\pgfqpoint{0.666667in}{1.016667in}}%
\pgfpathcurveto{\pgfqpoint{0.662247in}{1.016667in}}{\pgfqpoint{0.658007in}{1.014911in}}{\pgfqpoint{0.654882in}{1.011785in}}%
\pgfpathcurveto{\pgfqpoint{0.651756in}{1.008660in}}{\pgfqpoint{0.650000in}{1.004420in}}{\pgfqpoint{0.650000in}{1.000000in}}%
\pgfpathcurveto{\pgfqpoint{0.650000in}{0.995580in}}{\pgfqpoint{0.651756in}{0.991340in}}{\pgfqpoint{0.654882in}{0.988215in}}%
\pgfpathcurveto{\pgfqpoint{0.658007in}{0.985089in}}{\pgfqpoint{0.662247in}{0.983333in}}{\pgfqpoint{0.666667in}{0.983333in}}%
\pgfpathclose%
\pgfpathmoveto{\pgfqpoint{0.833333in}{0.983333in}}%
\pgfpathcurveto{\pgfqpoint{0.837753in}{0.983333in}}{\pgfqpoint{0.841993in}{0.985089in}}{\pgfqpoint{0.845118in}{0.988215in}}%
\pgfpathcurveto{\pgfqpoint{0.848244in}{0.991340in}}{\pgfqpoint{0.850000in}{0.995580in}}{\pgfqpoint{0.850000in}{1.000000in}}%
\pgfpathcurveto{\pgfqpoint{0.850000in}{1.004420in}}{\pgfqpoint{0.848244in}{1.008660in}}{\pgfqpoint{0.845118in}{1.011785in}}%
\pgfpathcurveto{\pgfqpoint{0.841993in}{1.014911in}}{\pgfqpoint{0.837753in}{1.016667in}}{\pgfqpoint{0.833333in}{1.016667in}}%
\pgfpathcurveto{\pgfqpoint{0.828913in}{1.016667in}}{\pgfqpoint{0.824674in}{1.014911in}}{\pgfqpoint{0.821548in}{1.011785in}}%
\pgfpathcurveto{\pgfqpoint{0.818423in}{1.008660in}}{\pgfqpoint{0.816667in}{1.004420in}}{\pgfqpoint{0.816667in}{1.000000in}}%
\pgfpathcurveto{\pgfqpoint{0.816667in}{0.995580in}}{\pgfqpoint{0.818423in}{0.991340in}}{\pgfqpoint{0.821548in}{0.988215in}}%
\pgfpathcurveto{\pgfqpoint{0.824674in}{0.985089in}}{\pgfqpoint{0.828913in}{0.983333in}}{\pgfqpoint{0.833333in}{0.983333in}}%
\pgfpathclose%
\pgfpathmoveto{\pgfqpoint{1.000000in}{0.983333in}}%
\pgfpathcurveto{\pgfqpoint{1.004420in}{0.983333in}}{\pgfqpoint{1.008660in}{0.985089in}}{\pgfqpoint{1.011785in}{0.988215in}}%
\pgfpathcurveto{\pgfqpoint{1.014911in}{0.991340in}}{\pgfqpoint{1.016667in}{0.995580in}}{\pgfqpoint{1.016667in}{1.000000in}}%
\pgfpathcurveto{\pgfqpoint{1.016667in}{1.004420in}}{\pgfqpoint{1.014911in}{1.008660in}}{\pgfqpoint{1.011785in}{1.011785in}}%
\pgfpathcurveto{\pgfqpoint{1.008660in}{1.014911in}}{\pgfqpoint{1.004420in}{1.016667in}}{\pgfqpoint{1.000000in}{1.016667in}}%
\pgfpathcurveto{\pgfqpoint{0.995580in}{1.016667in}}{\pgfqpoint{0.991340in}{1.014911in}}{\pgfqpoint{0.988215in}{1.011785in}}%
\pgfpathcurveto{\pgfqpoint{0.985089in}{1.008660in}}{\pgfqpoint{0.983333in}{1.004420in}}{\pgfqpoint{0.983333in}{1.000000in}}%
\pgfpathcurveto{\pgfqpoint{0.983333in}{0.995580in}}{\pgfqpoint{0.985089in}{0.991340in}}{\pgfqpoint{0.988215in}{0.988215in}}%
\pgfpathcurveto{\pgfqpoint{0.991340in}{0.985089in}}{\pgfqpoint{0.995580in}{0.983333in}}{\pgfqpoint{1.000000in}{0.983333in}}%
\pgfpathclose%
\pgfusepath{stroke}%
\end{pgfscope}%
}%
\pgfsys@transformshift{9.073315in}{5.392542in}%
\pgfsys@useobject{currentpattern}{}%
\pgfsys@transformshift{1in}{0in}%
\pgfsys@transformshift{-1in}{0in}%
\pgfsys@transformshift{0in}{1in}%
\end{pgfscope}%
\begin{pgfscope}%
\pgfpathrectangle{\pgfqpoint{0.935815in}{0.637495in}}{\pgfqpoint{9.300000in}{9.060000in}}%
\pgfusepath{clip}%
\pgfsetbuttcap%
\pgfsetmiterjoin%
\definecolor{currentfill}{rgb}{1.000000,1.000000,0.000000}%
\pgfsetfillcolor{currentfill}%
\pgfsetfillopacity{0.990000}%
\pgfsetlinewidth{0.000000pt}%
\definecolor{currentstroke}{rgb}{0.000000,0.000000,0.000000}%
\pgfsetstrokecolor{currentstroke}%
\pgfsetstrokeopacity{0.990000}%
\pgfsetdash{}{0pt}%
\pgfpathmoveto{\pgfqpoint{1.323315in}{2.557723in}}%
\pgfpathlineto{\pgfqpoint{2.098315in}{2.557723in}}%
\pgfpathlineto{\pgfqpoint{2.098315in}{2.908608in}}%
\pgfpathlineto{\pgfqpoint{1.323315in}{2.908608in}}%
\pgfpathclose%
\pgfusepath{fill}%
\end{pgfscope}%
\begin{pgfscope}%
\pgfsetbuttcap%
\pgfsetmiterjoin%
\definecolor{currentfill}{rgb}{1.000000,1.000000,0.000000}%
\pgfsetfillcolor{currentfill}%
\pgfsetfillopacity{0.990000}%
\pgfsetlinewidth{0.000000pt}%
\definecolor{currentstroke}{rgb}{0.000000,0.000000,0.000000}%
\pgfsetstrokecolor{currentstroke}%
\pgfsetstrokeopacity{0.990000}%
\pgfsetdash{}{0pt}%
\pgfpathrectangle{\pgfqpoint{0.935815in}{0.637495in}}{\pgfqpoint{9.300000in}{9.060000in}}%
\pgfusepath{clip}%
\pgfpathmoveto{\pgfqpoint{1.323315in}{2.557723in}}%
\pgfpathlineto{\pgfqpoint{2.098315in}{2.557723in}}%
\pgfpathlineto{\pgfqpoint{2.098315in}{2.908608in}}%
\pgfpathlineto{\pgfqpoint{1.323315in}{2.908608in}}%
\pgfpathclose%
\pgfusepath{clip}%
\pgfsys@defobject{currentpattern}{\pgfqpoint{0in}{0in}}{\pgfqpoint{1in}{1in}}{%
\begin{pgfscope}%
\pgfpathrectangle{\pgfqpoint{0in}{0in}}{\pgfqpoint{1in}{1in}}%
\pgfusepath{clip}%
\pgfpathmoveto{\pgfqpoint{0.000000in}{0.055556in}}%
\pgfpathlineto{\pgfqpoint{-0.016327in}{0.022473in}}%
\pgfpathlineto{\pgfqpoint{-0.052836in}{0.017168in}}%
\pgfpathlineto{\pgfqpoint{-0.026418in}{-0.008584in}}%
\pgfpathlineto{\pgfqpoint{-0.032655in}{-0.044945in}}%
\pgfpathlineto{\pgfqpoint{-0.000000in}{-0.027778in}}%
\pgfpathlineto{\pgfqpoint{0.032655in}{-0.044945in}}%
\pgfpathlineto{\pgfqpoint{0.026418in}{-0.008584in}}%
\pgfpathlineto{\pgfqpoint{0.052836in}{0.017168in}}%
\pgfpathlineto{\pgfqpoint{0.016327in}{0.022473in}}%
\pgfpathlineto{\pgfqpoint{0.000000in}{0.055556in}}%
\pgfpathmoveto{\pgfqpoint{0.166667in}{0.055556in}}%
\pgfpathlineto{\pgfqpoint{0.150339in}{0.022473in}}%
\pgfpathlineto{\pgfqpoint{0.113830in}{0.017168in}}%
\pgfpathlineto{\pgfqpoint{0.140248in}{-0.008584in}}%
\pgfpathlineto{\pgfqpoint{0.134012in}{-0.044945in}}%
\pgfpathlineto{\pgfqpoint{0.166667in}{-0.027778in}}%
\pgfpathlineto{\pgfqpoint{0.199321in}{-0.044945in}}%
\pgfpathlineto{\pgfqpoint{0.193085in}{-0.008584in}}%
\pgfpathlineto{\pgfqpoint{0.219503in}{0.017168in}}%
\pgfpathlineto{\pgfqpoint{0.182994in}{0.022473in}}%
\pgfpathlineto{\pgfqpoint{0.166667in}{0.055556in}}%
\pgfpathmoveto{\pgfqpoint{0.333333in}{0.055556in}}%
\pgfpathlineto{\pgfqpoint{0.317006in}{0.022473in}}%
\pgfpathlineto{\pgfqpoint{0.280497in}{0.017168in}}%
\pgfpathlineto{\pgfqpoint{0.306915in}{-0.008584in}}%
\pgfpathlineto{\pgfqpoint{0.300679in}{-0.044945in}}%
\pgfpathlineto{\pgfqpoint{0.333333in}{-0.027778in}}%
\pgfpathlineto{\pgfqpoint{0.365988in}{-0.044945in}}%
\pgfpathlineto{\pgfqpoint{0.359752in}{-0.008584in}}%
\pgfpathlineto{\pgfqpoint{0.386170in}{0.017168in}}%
\pgfpathlineto{\pgfqpoint{0.349661in}{0.022473in}}%
\pgfpathlineto{\pgfqpoint{0.333333in}{0.055556in}}%
\pgfpathmoveto{\pgfqpoint{0.500000in}{0.055556in}}%
\pgfpathlineto{\pgfqpoint{0.483673in}{0.022473in}}%
\pgfpathlineto{\pgfqpoint{0.447164in}{0.017168in}}%
\pgfpathlineto{\pgfqpoint{0.473582in}{-0.008584in}}%
\pgfpathlineto{\pgfqpoint{0.467345in}{-0.044945in}}%
\pgfpathlineto{\pgfqpoint{0.500000in}{-0.027778in}}%
\pgfpathlineto{\pgfqpoint{0.532655in}{-0.044945in}}%
\pgfpathlineto{\pgfqpoint{0.526418in}{-0.008584in}}%
\pgfpathlineto{\pgfqpoint{0.552836in}{0.017168in}}%
\pgfpathlineto{\pgfqpoint{0.516327in}{0.022473in}}%
\pgfpathlineto{\pgfqpoint{0.500000in}{0.055556in}}%
\pgfpathmoveto{\pgfqpoint{0.666667in}{0.055556in}}%
\pgfpathlineto{\pgfqpoint{0.650339in}{0.022473in}}%
\pgfpathlineto{\pgfqpoint{0.613830in}{0.017168in}}%
\pgfpathlineto{\pgfqpoint{0.640248in}{-0.008584in}}%
\pgfpathlineto{\pgfqpoint{0.634012in}{-0.044945in}}%
\pgfpathlineto{\pgfqpoint{0.666667in}{-0.027778in}}%
\pgfpathlineto{\pgfqpoint{0.699321in}{-0.044945in}}%
\pgfpathlineto{\pgfqpoint{0.693085in}{-0.008584in}}%
\pgfpathlineto{\pgfqpoint{0.719503in}{0.017168in}}%
\pgfpathlineto{\pgfqpoint{0.682994in}{0.022473in}}%
\pgfpathlineto{\pgfqpoint{0.666667in}{0.055556in}}%
\pgfpathmoveto{\pgfqpoint{0.833333in}{0.055556in}}%
\pgfpathlineto{\pgfqpoint{0.817006in}{0.022473in}}%
\pgfpathlineto{\pgfqpoint{0.780497in}{0.017168in}}%
\pgfpathlineto{\pgfqpoint{0.806915in}{-0.008584in}}%
\pgfpathlineto{\pgfqpoint{0.800679in}{-0.044945in}}%
\pgfpathlineto{\pgfqpoint{0.833333in}{-0.027778in}}%
\pgfpathlineto{\pgfqpoint{0.865988in}{-0.044945in}}%
\pgfpathlineto{\pgfqpoint{0.859752in}{-0.008584in}}%
\pgfpathlineto{\pgfqpoint{0.886170in}{0.017168in}}%
\pgfpathlineto{\pgfqpoint{0.849661in}{0.022473in}}%
\pgfpathlineto{\pgfqpoint{0.833333in}{0.055556in}}%
\pgfpathmoveto{\pgfqpoint{1.000000in}{0.055556in}}%
\pgfpathlineto{\pgfqpoint{0.983673in}{0.022473in}}%
\pgfpathlineto{\pgfqpoint{0.947164in}{0.017168in}}%
\pgfpathlineto{\pgfqpoint{0.973582in}{-0.008584in}}%
\pgfpathlineto{\pgfqpoint{0.967345in}{-0.044945in}}%
\pgfpathlineto{\pgfqpoint{1.000000in}{-0.027778in}}%
\pgfpathlineto{\pgfqpoint{1.032655in}{-0.044945in}}%
\pgfpathlineto{\pgfqpoint{1.026418in}{-0.008584in}}%
\pgfpathlineto{\pgfqpoint{1.052836in}{0.017168in}}%
\pgfpathlineto{\pgfqpoint{1.016327in}{0.022473in}}%
\pgfpathlineto{\pgfqpoint{1.000000in}{0.055556in}}%
\pgfpathmoveto{\pgfqpoint{0.083333in}{0.222222in}}%
\pgfpathlineto{\pgfqpoint{0.067006in}{0.189139in}}%
\pgfpathlineto{\pgfqpoint{0.030497in}{0.183834in}}%
\pgfpathlineto{\pgfqpoint{0.056915in}{0.158083in}}%
\pgfpathlineto{\pgfqpoint{0.050679in}{0.121721in}}%
\pgfpathlineto{\pgfqpoint{0.083333in}{0.138889in}}%
\pgfpathlineto{\pgfqpoint{0.115988in}{0.121721in}}%
\pgfpathlineto{\pgfqpoint{0.109752in}{0.158083in}}%
\pgfpathlineto{\pgfqpoint{0.136170in}{0.183834in}}%
\pgfpathlineto{\pgfqpoint{0.099661in}{0.189139in}}%
\pgfpathlineto{\pgfqpoint{0.083333in}{0.222222in}}%
\pgfpathmoveto{\pgfqpoint{0.250000in}{0.222222in}}%
\pgfpathlineto{\pgfqpoint{0.233673in}{0.189139in}}%
\pgfpathlineto{\pgfqpoint{0.197164in}{0.183834in}}%
\pgfpathlineto{\pgfqpoint{0.223582in}{0.158083in}}%
\pgfpathlineto{\pgfqpoint{0.217345in}{0.121721in}}%
\pgfpathlineto{\pgfqpoint{0.250000in}{0.138889in}}%
\pgfpathlineto{\pgfqpoint{0.282655in}{0.121721in}}%
\pgfpathlineto{\pgfqpoint{0.276418in}{0.158083in}}%
\pgfpathlineto{\pgfqpoint{0.302836in}{0.183834in}}%
\pgfpathlineto{\pgfqpoint{0.266327in}{0.189139in}}%
\pgfpathlineto{\pgfqpoint{0.250000in}{0.222222in}}%
\pgfpathmoveto{\pgfqpoint{0.416667in}{0.222222in}}%
\pgfpathlineto{\pgfqpoint{0.400339in}{0.189139in}}%
\pgfpathlineto{\pgfqpoint{0.363830in}{0.183834in}}%
\pgfpathlineto{\pgfqpoint{0.390248in}{0.158083in}}%
\pgfpathlineto{\pgfqpoint{0.384012in}{0.121721in}}%
\pgfpathlineto{\pgfqpoint{0.416667in}{0.138889in}}%
\pgfpathlineto{\pgfqpoint{0.449321in}{0.121721in}}%
\pgfpathlineto{\pgfqpoint{0.443085in}{0.158083in}}%
\pgfpathlineto{\pgfqpoint{0.469503in}{0.183834in}}%
\pgfpathlineto{\pgfqpoint{0.432994in}{0.189139in}}%
\pgfpathlineto{\pgfqpoint{0.416667in}{0.222222in}}%
\pgfpathmoveto{\pgfqpoint{0.583333in}{0.222222in}}%
\pgfpathlineto{\pgfqpoint{0.567006in}{0.189139in}}%
\pgfpathlineto{\pgfqpoint{0.530497in}{0.183834in}}%
\pgfpathlineto{\pgfqpoint{0.556915in}{0.158083in}}%
\pgfpathlineto{\pgfqpoint{0.550679in}{0.121721in}}%
\pgfpathlineto{\pgfqpoint{0.583333in}{0.138889in}}%
\pgfpathlineto{\pgfqpoint{0.615988in}{0.121721in}}%
\pgfpathlineto{\pgfqpoint{0.609752in}{0.158083in}}%
\pgfpathlineto{\pgfqpoint{0.636170in}{0.183834in}}%
\pgfpathlineto{\pgfqpoint{0.599661in}{0.189139in}}%
\pgfpathlineto{\pgfqpoint{0.583333in}{0.222222in}}%
\pgfpathmoveto{\pgfqpoint{0.750000in}{0.222222in}}%
\pgfpathlineto{\pgfqpoint{0.733673in}{0.189139in}}%
\pgfpathlineto{\pgfqpoint{0.697164in}{0.183834in}}%
\pgfpathlineto{\pgfqpoint{0.723582in}{0.158083in}}%
\pgfpathlineto{\pgfqpoint{0.717345in}{0.121721in}}%
\pgfpathlineto{\pgfqpoint{0.750000in}{0.138889in}}%
\pgfpathlineto{\pgfqpoint{0.782655in}{0.121721in}}%
\pgfpathlineto{\pgfqpoint{0.776418in}{0.158083in}}%
\pgfpathlineto{\pgfqpoint{0.802836in}{0.183834in}}%
\pgfpathlineto{\pgfqpoint{0.766327in}{0.189139in}}%
\pgfpathlineto{\pgfqpoint{0.750000in}{0.222222in}}%
\pgfpathmoveto{\pgfqpoint{0.916667in}{0.222222in}}%
\pgfpathlineto{\pgfqpoint{0.900339in}{0.189139in}}%
\pgfpathlineto{\pgfqpoint{0.863830in}{0.183834in}}%
\pgfpathlineto{\pgfqpoint{0.890248in}{0.158083in}}%
\pgfpathlineto{\pgfqpoint{0.884012in}{0.121721in}}%
\pgfpathlineto{\pgfqpoint{0.916667in}{0.138889in}}%
\pgfpathlineto{\pgfqpoint{0.949321in}{0.121721in}}%
\pgfpathlineto{\pgfqpoint{0.943085in}{0.158083in}}%
\pgfpathlineto{\pgfqpoint{0.969503in}{0.183834in}}%
\pgfpathlineto{\pgfqpoint{0.932994in}{0.189139in}}%
\pgfpathlineto{\pgfqpoint{0.916667in}{0.222222in}}%
\pgfpathmoveto{\pgfqpoint{0.000000in}{0.388889in}}%
\pgfpathlineto{\pgfqpoint{-0.016327in}{0.355806in}}%
\pgfpathlineto{\pgfqpoint{-0.052836in}{0.350501in}}%
\pgfpathlineto{\pgfqpoint{-0.026418in}{0.324750in}}%
\pgfpathlineto{\pgfqpoint{-0.032655in}{0.288388in}}%
\pgfpathlineto{\pgfqpoint{-0.000000in}{0.305556in}}%
\pgfpathlineto{\pgfqpoint{0.032655in}{0.288388in}}%
\pgfpathlineto{\pgfqpoint{0.026418in}{0.324750in}}%
\pgfpathlineto{\pgfqpoint{0.052836in}{0.350501in}}%
\pgfpathlineto{\pgfqpoint{0.016327in}{0.355806in}}%
\pgfpathlineto{\pgfqpoint{0.000000in}{0.388889in}}%
\pgfpathmoveto{\pgfqpoint{0.166667in}{0.388889in}}%
\pgfpathlineto{\pgfqpoint{0.150339in}{0.355806in}}%
\pgfpathlineto{\pgfqpoint{0.113830in}{0.350501in}}%
\pgfpathlineto{\pgfqpoint{0.140248in}{0.324750in}}%
\pgfpathlineto{\pgfqpoint{0.134012in}{0.288388in}}%
\pgfpathlineto{\pgfqpoint{0.166667in}{0.305556in}}%
\pgfpathlineto{\pgfqpoint{0.199321in}{0.288388in}}%
\pgfpathlineto{\pgfqpoint{0.193085in}{0.324750in}}%
\pgfpathlineto{\pgfqpoint{0.219503in}{0.350501in}}%
\pgfpathlineto{\pgfqpoint{0.182994in}{0.355806in}}%
\pgfpathlineto{\pgfqpoint{0.166667in}{0.388889in}}%
\pgfpathmoveto{\pgfqpoint{0.333333in}{0.388889in}}%
\pgfpathlineto{\pgfqpoint{0.317006in}{0.355806in}}%
\pgfpathlineto{\pgfqpoint{0.280497in}{0.350501in}}%
\pgfpathlineto{\pgfqpoint{0.306915in}{0.324750in}}%
\pgfpathlineto{\pgfqpoint{0.300679in}{0.288388in}}%
\pgfpathlineto{\pgfqpoint{0.333333in}{0.305556in}}%
\pgfpathlineto{\pgfqpoint{0.365988in}{0.288388in}}%
\pgfpathlineto{\pgfqpoint{0.359752in}{0.324750in}}%
\pgfpathlineto{\pgfqpoint{0.386170in}{0.350501in}}%
\pgfpathlineto{\pgfqpoint{0.349661in}{0.355806in}}%
\pgfpathlineto{\pgfqpoint{0.333333in}{0.388889in}}%
\pgfpathmoveto{\pgfqpoint{0.500000in}{0.388889in}}%
\pgfpathlineto{\pgfqpoint{0.483673in}{0.355806in}}%
\pgfpathlineto{\pgfqpoint{0.447164in}{0.350501in}}%
\pgfpathlineto{\pgfqpoint{0.473582in}{0.324750in}}%
\pgfpathlineto{\pgfqpoint{0.467345in}{0.288388in}}%
\pgfpathlineto{\pgfqpoint{0.500000in}{0.305556in}}%
\pgfpathlineto{\pgfqpoint{0.532655in}{0.288388in}}%
\pgfpathlineto{\pgfqpoint{0.526418in}{0.324750in}}%
\pgfpathlineto{\pgfqpoint{0.552836in}{0.350501in}}%
\pgfpathlineto{\pgfqpoint{0.516327in}{0.355806in}}%
\pgfpathlineto{\pgfqpoint{0.500000in}{0.388889in}}%
\pgfpathmoveto{\pgfqpoint{0.666667in}{0.388889in}}%
\pgfpathlineto{\pgfqpoint{0.650339in}{0.355806in}}%
\pgfpathlineto{\pgfqpoint{0.613830in}{0.350501in}}%
\pgfpathlineto{\pgfqpoint{0.640248in}{0.324750in}}%
\pgfpathlineto{\pgfqpoint{0.634012in}{0.288388in}}%
\pgfpathlineto{\pgfqpoint{0.666667in}{0.305556in}}%
\pgfpathlineto{\pgfqpoint{0.699321in}{0.288388in}}%
\pgfpathlineto{\pgfqpoint{0.693085in}{0.324750in}}%
\pgfpathlineto{\pgfqpoint{0.719503in}{0.350501in}}%
\pgfpathlineto{\pgfqpoint{0.682994in}{0.355806in}}%
\pgfpathlineto{\pgfqpoint{0.666667in}{0.388889in}}%
\pgfpathmoveto{\pgfqpoint{0.833333in}{0.388889in}}%
\pgfpathlineto{\pgfqpoint{0.817006in}{0.355806in}}%
\pgfpathlineto{\pgfqpoint{0.780497in}{0.350501in}}%
\pgfpathlineto{\pgfqpoint{0.806915in}{0.324750in}}%
\pgfpathlineto{\pgfqpoint{0.800679in}{0.288388in}}%
\pgfpathlineto{\pgfqpoint{0.833333in}{0.305556in}}%
\pgfpathlineto{\pgfqpoint{0.865988in}{0.288388in}}%
\pgfpathlineto{\pgfqpoint{0.859752in}{0.324750in}}%
\pgfpathlineto{\pgfqpoint{0.886170in}{0.350501in}}%
\pgfpathlineto{\pgfqpoint{0.849661in}{0.355806in}}%
\pgfpathlineto{\pgfqpoint{0.833333in}{0.388889in}}%
\pgfpathmoveto{\pgfqpoint{1.000000in}{0.388889in}}%
\pgfpathlineto{\pgfqpoint{0.983673in}{0.355806in}}%
\pgfpathlineto{\pgfqpoint{0.947164in}{0.350501in}}%
\pgfpathlineto{\pgfqpoint{0.973582in}{0.324750in}}%
\pgfpathlineto{\pgfqpoint{0.967345in}{0.288388in}}%
\pgfpathlineto{\pgfqpoint{1.000000in}{0.305556in}}%
\pgfpathlineto{\pgfqpoint{1.032655in}{0.288388in}}%
\pgfpathlineto{\pgfqpoint{1.026418in}{0.324750in}}%
\pgfpathlineto{\pgfqpoint{1.052836in}{0.350501in}}%
\pgfpathlineto{\pgfqpoint{1.016327in}{0.355806in}}%
\pgfpathlineto{\pgfqpoint{1.000000in}{0.388889in}}%
\pgfpathmoveto{\pgfqpoint{0.083333in}{0.555556in}}%
\pgfpathlineto{\pgfqpoint{0.067006in}{0.522473in}}%
\pgfpathlineto{\pgfqpoint{0.030497in}{0.517168in}}%
\pgfpathlineto{\pgfqpoint{0.056915in}{0.491416in}}%
\pgfpathlineto{\pgfqpoint{0.050679in}{0.455055in}}%
\pgfpathlineto{\pgfqpoint{0.083333in}{0.472222in}}%
\pgfpathlineto{\pgfqpoint{0.115988in}{0.455055in}}%
\pgfpathlineto{\pgfqpoint{0.109752in}{0.491416in}}%
\pgfpathlineto{\pgfqpoint{0.136170in}{0.517168in}}%
\pgfpathlineto{\pgfqpoint{0.099661in}{0.522473in}}%
\pgfpathlineto{\pgfqpoint{0.083333in}{0.555556in}}%
\pgfpathmoveto{\pgfqpoint{0.250000in}{0.555556in}}%
\pgfpathlineto{\pgfqpoint{0.233673in}{0.522473in}}%
\pgfpathlineto{\pgfqpoint{0.197164in}{0.517168in}}%
\pgfpathlineto{\pgfqpoint{0.223582in}{0.491416in}}%
\pgfpathlineto{\pgfqpoint{0.217345in}{0.455055in}}%
\pgfpathlineto{\pgfqpoint{0.250000in}{0.472222in}}%
\pgfpathlineto{\pgfqpoint{0.282655in}{0.455055in}}%
\pgfpathlineto{\pgfqpoint{0.276418in}{0.491416in}}%
\pgfpathlineto{\pgfqpoint{0.302836in}{0.517168in}}%
\pgfpathlineto{\pgfqpoint{0.266327in}{0.522473in}}%
\pgfpathlineto{\pgfqpoint{0.250000in}{0.555556in}}%
\pgfpathmoveto{\pgfqpoint{0.416667in}{0.555556in}}%
\pgfpathlineto{\pgfqpoint{0.400339in}{0.522473in}}%
\pgfpathlineto{\pgfqpoint{0.363830in}{0.517168in}}%
\pgfpathlineto{\pgfqpoint{0.390248in}{0.491416in}}%
\pgfpathlineto{\pgfqpoint{0.384012in}{0.455055in}}%
\pgfpathlineto{\pgfqpoint{0.416667in}{0.472222in}}%
\pgfpathlineto{\pgfqpoint{0.449321in}{0.455055in}}%
\pgfpathlineto{\pgfqpoint{0.443085in}{0.491416in}}%
\pgfpathlineto{\pgfqpoint{0.469503in}{0.517168in}}%
\pgfpathlineto{\pgfqpoint{0.432994in}{0.522473in}}%
\pgfpathlineto{\pgfqpoint{0.416667in}{0.555556in}}%
\pgfpathmoveto{\pgfqpoint{0.583333in}{0.555556in}}%
\pgfpathlineto{\pgfqpoint{0.567006in}{0.522473in}}%
\pgfpathlineto{\pgfqpoint{0.530497in}{0.517168in}}%
\pgfpathlineto{\pgfqpoint{0.556915in}{0.491416in}}%
\pgfpathlineto{\pgfqpoint{0.550679in}{0.455055in}}%
\pgfpathlineto{\pgfqpoint{0.583333in}{0.472222in}}%
\pgfpathlineto{\pgfqpoint{0.615988in}{0.455055in}}%
\pgfpathlineto{\pgfqpoint{0.609752in}{0.491416in}}%
\pgfpathlineto{\pgfqpoint{0.636170in}{0.517168in}}%
\pgfpathlineto{\pgfqpoint{0.599661in}{0.522473in}}%
\pgfpathlineto{\pgfqpoint{0.583333in}{0.555556in}}%
\pgfpathmoveto{\pgfqpoint{0.750000in}{0.555556in}}%
\pgfpathlineto{\pgfqpoint{0.733673in}{0.522473in}}%
\pgfpathlineto{\pgfqpoint{0.697164in}{0.517168in}}%
\pgfpathlineto{\pgfqpoint{0.723582in}{0.491416in}}%
\pgfpathlineto{\pgfqpoint{0.717345in}{0.455055in}}%
\pgfpathlineto{\pgfqpoint{0.750000in}{0.472222in}}%
\pgfpathlineto{\pgfqpoint{0.782655in}{0.455055in}}%
\pgfpathlineto{\pgfqpoint{0.776418in}{0.491416in}}%
\pgfpathlineto{\pgfqpoint{0.802836in}{0.517168in}}%
\pgfpathlineto{\pgfqpoint{0.766327in}{0.522473in}}%
\pgfpathlineto{\pgfqpoint{0.750000in}{0.555556in}}%
\pgfpathmoveto{\pgfqpoint{0.916667in}{0.555556in}}%
\pgfpathlineto{\pgfqpoint{0.900339in}{0.522473in}}%
\pgfpathlineto{\pgfqpoint{0.863830in}{0.517168in}}%
\pgfpathlineto{\pgfqpoint{0.890248in}{0.491416in}}%
\pgfpathlineto{\pgfqpoint{0.884012in}{0.455055in}}%
\pgfpathlineto{\pgfqpoint{0.916667in}{0.472222in}}%
\pgfpathlineto{\pgfqpoint{0.949321in}{0.455055in}}%
\pgfpathlineto{\pgfqpoint{0.943085in}{0.491416in}}%
\pgfpathlineto{\pgfqpoint{0.969503in}{0.517168in}}%
\pgfpathlineto{\pgfqpoint{0.932994in}{0.522473in}}%
\pgfpathlineto{\pgfqpoint{0.916667in}{0.555556in}}%
\pgfpathmoveto{\pgfqpoint{0.000000in}{0.722222in}}%
\pgfpathlineto{\pgfqpoint{-0.016327in}{0.689139in}}%
\pgfpathlineto{\pgfqpoint{-0.052836in}{0.683834in}}%
\pgfpathlineto{\pgfqpoint{-0.026418in}{0.658083in}}%
\pgfpathlineto{\pgfqpoint{-0.032655in}{0.621721in}}%
\pgfpathlineto{\pgfqpoint{-0.000000in}{0.638889in}}%
\pgfpathlineto{\pgfqpoint{0.032655in}{0.621721in}}%
\pgfpathlineto{\pgfqpoint{0.026418in}{0.658083in}}%
\pgfpathlineto{\pgfqpoint{0.052836in}{0.683834in}}%
\pgfpathlineto{\pgfqpoint{0.016327in}{0.689139in}}%
\pgfpathlineto{\pgfqpoint{0.000000in}{0.722222in}}%
\pgfpathmoveto{\pgfqpoint{0.166667in}{0.722222in}}%
\pgfpathlineto{\pgfqpoint{0.150339in}{0.689139in}}%
\pgfpathlineto{\pgfqpoint{0.113830in}{0.683834in}}%
\pgfpathlineto{\pgfqpoint{0.140248in}{0.658083in}}%
\pgfpathlineto{\pgfqpoint{0.134012in}{0.621721in}}%
\pgfpathlineto{\pgfqpoint{0.166667in}{0.638889in}}%
\pgfpathlineto{\pgfqpoint{0.199321in}{0.621721in}}%
\pgfpathlineto{\pgfqpoint{0.193085in}{0.658083in}}%
\pgfpathlineto{\pgfqpoint{0.219503in}{0.683834in}}%
\pgfpathlineto{\pgfqpoint{0.182994in}{0.689139in}}%
\pgfpathlineto{\pgfqpoint{0.166667in}{0.722222in}}%
\pgfpathmoveto{\pgfqpoint{0.333333in}{0.722222in}}%
\pgfpathlineto{\pgfqpoint{0.317006in}{0.689139in}}%
\pgfpathlineto{\pgfqpoint{0.280497in}{0.683834in}}%
\pgfpathlineto{\pgfqpoint{0.306915in}{0.658083in}}%
\pgfpathlineto{\pgfqpoint{0.300679in}{0.621721in}}%
\pgfpathlineto{\pgfqpoint{0.333333in}{0.638889in}}%
\pgfpathlineto{\pgfqpoint{0.365988in}{0.621721in}}%
\pgfpathlineto{\pgfqpoint{0.359752in}{0.658083in}}%
\pgfpathlineto{\pgfqpoint{0.386170in}{0.683834in}}%
\pgfpathlineto{\pgfqpoint{0.349661in}{0.689139in}}%
\pgfpathlineto{\pgfqpoint{0.333333in}{0.722222in}}%
\pgfpathmoveto{\pgfqpoint{0.500000in}{0.722222in}}%
\pgfpathlineto{\pgfqpoint{0.483673in}{0.689139in}}%
\pgfpathlineto{\pgfqpoint{0.447164in}{0.683834in}}%
\pgfpathlineto{\pgfqpoint{0.473582in}{0.658083in}}%
\pgfpathlineto{\pgfqpoint{0.467345in}{0.621721in}}%
\pgfpathlineto{\pgfqpoint{0.500000in}{0.638889in}}%
\pgfpathlineto{\pgfqpoint{0.532655in}{0.621721in}}%
\pgfpathlineto{\pgfqpoint{0.526418in}{0.658083in}}%
\pgfpathlineto{\pgfqpoint{0.552836in}{0.683834in}}%
\pgfpathlineto{\pgfqpoint{0.516327in}{0.689139in}}%
\pgfpathlineto{\pgfqpoint{0.500000in}{0.722222in}}%
\pgfpathmoveto{\pgfqpoint{0.666667in}{0.722222in}}%
\pgfpathlineto{\pgfqpoint{0.650339in}{0.689139in}}%
\pgfpathlineto{\pgfqpoint{0.613830in}{0.683834in}}%
\pgfpathlineto{\pgfqpoint{0.640248in}{0.658083in}}%
\pgfpathlineto{\pgfqpoint{0.634012in}{0.621721in}}%
\pgfpathlineto{\pgfqpoint{0.666667in}{0.638889in}}%
\pgfpathlineto{\pgfqpoint{0.699321in}{0.621721in}}%
\pgfpathlineto{\pgfqpoint{0.693085in}{0.658083in}}%
\pgfpathlineto{\pgfqpoint{0.719503in}{0.683834in}}%
\pgfpathlineto{\pgfqpoint{0.682994in}{0.689139in}}%
\pgfpathlineto{\pgfqpoint{0.666667in}{0.722222in}}%
\pgfpathmoveto{\pgfqpoint{0.833333in}{0.722222in}}%
\pgfpathlineto{\pgfqpoint{0.817006in}{0.689139in}}%
\pgfpathlineto{\pgfqpoint{0.780497in}{0.683834in}}%
\pgfpathlineto{\pgfqpoint{0.806915in}{0.658083in}}%
\pgfpathlineto{\pgfqpoint{0.800679in}{0.621721in}}%
\pgfpathlineto{\pgfqpoint{0.833333in}{0.638889in}}%
\pgfpathlineto{\pgfqpoint{0.865988in}{0.621721in}}%
\pgfpathlineto{\pgfqpoint{0.859752in}{0.658083in}}%
\pgfpathlineto{\pgfqpoint{0.886170in}{0.683834in}}%
\pgfpathlineto{\pgfqpoint{0.849661in}{0.689139in}}%
\pgfpathlineto{\pgfqpoint{0.833333in}{0.722222in}}%
\pgfpathmoveto{\pgfqpoint{1.000000in}{0.722222in}}%
\pgfpathlineto{\pgfqpoint{0.983673in}{0.689139in}}%
\pgfpathlineto{\pgfqpoint{0.947164in}{0.683834in}}%
\pgfpathlineto{\pgfqpoint{0.973582in}{0.658083in}}%
\pgfpathlineto{\pgfqpoint{0.967345in}{0.621721in}}%
\pgfpathlineto{\pgfqpoint{1.000000in}{0.638889in}}%
\pgfpathlineto{\pgfqpoint{1.032655in}{0.621721in}}%
\pgfpathlineto{\pgfqpoint{1.026418in}{0.658083in}}%
\pgfpathlineto{\pgfqpoint{1.052836in}{0.683834in}}%
\pgfpathlineto{\pgfqpoint{1.016327in}{0.689139in}}%
\pgfpathlineto{\pgfqpoint{1.000000in}{0.722222in}}%
\pgfpathmoveto{\pgfqpoint{0.083333in}{0.888889in}}%
\pgfpathlineto{\pgfqpoint{0.067006in}{0.855806in}}%
\pgfpathlineto{\pgfqpoint{0.030497in}{0.850501in}}%
\pgfpathlineto{\pgfqpoint{0.056915in}{0.824750in}}%
\pgfpathlineto{\pgfqpoint{0.050679in}{0.788388in}}%
\pgfpathlineto{\pgfqpoint{0.083333in}{0.805556in}}%
\pgfpathlineto{\pgfqpoint{0.115988in}{0.788388in}}%
\pgfpathlineto{\pgfqpoint{0.109752in}{0.824750in}}%
\pgfpathlineto{\pgfqpoint{0.136170in}{0.850501in}}%
\pgfpathlineto{\pgfqpoint{0.099661in}{0.855806in}}%
\pgfpathlineto{\pgfqpoint{0.083333in}{0.888889in}}%
\pgfpathmoveto{\pgfqpoint{0.250000in}{0.888889in}}%
\pgfpathlineto{\pgfqpoint{0.233673in}{0.855806in}}%
\pgfpathlineto{\pgfqpoint{0.197164in}{0.850501in}}%
\pgfpathlineto{\pgfqpoint{0.223582in}{0.824750in}}%
\pgfpathlineto{\pgfqpoint{0.217345in}{0.788388in}}%
\pgfpathlineto{\pgfqpoint{0.250000in}{0.805556in}}%
\pgfpathlineto{\pgfqpoint{0.282655in}{0.788388in}}%
\pgfpathlineto{\pgfqpoint{0.276418in}{0.824750in}}%
\pgfpathlineto{\pgfqpoint{0.302836in}{0.850501in}}%
\pgfpathlineto{\pgfqpoint{0.266327in}{0.855806in}}%
\pgfpathlineto{\pgfqpoint{0.250000in}{0.888889in}}%
\pgfpathmoveto{\pgfqpoint{0.416667in}{0.888889in}}%
\pgfpathlineto{\pgfqpoint{0.400339in}{0.855806in}}%
\pgfpathlineto{\pgfqpoint{0.363830in}{0.850501in}}%
\pgfpathlineto{\pgfqpoint{0.390248in}{0.824750in}}%
\pgfpathlineto{\pgfqpoint{0.384012in}{0.788388in}}%
\pgfpathlineto{\pgfqpoint{0.416667in}{0.805556in}}%
\pgfpathlineto{\pgfqpoint{0.449321in}{0.788388in}}%
\pgfpathlineto{\pgfqpoint{0.443085in}{0.824750in}}%
\pgfpathlineto{\pgfqpoint{0.469503in}{0.850501in}}%
\pgfpathlineto{\pgfqpoint{0.432994in}{0.855806in}}%
\pgfpathlineto{\pgfqpoint{0.416667in}{0.888889in}}%
\pgfpathmoveto{\pgfqpoint{0.583333in}{0.888889in}}%
\pgfpathlineto{\pgfqpoint{0.567006in}{0.855806in}}%
\pgfpathlineto{\pgfqpoint{0.530497in}{0.850501in}}%
\pgfpathlineto{\pgfqpoint{0.556915in}{0.824750in}}%
\pgfpathlineto{\pgfqpoint{0.550679in}{0.788388in}}%
\pgfpathlineto{\pgfqpoint{0.583333in}{0.805556in}}%
\pgfpathlineto{\pgfqpoint{0.615988in}{0.788388in}}%
\pgfpathlineto{\pgfqpoint{0.609752in}{0.824750in}}%
\pgfpathlineto{\pgfqpoint{0.636170in}{0.850501in}}%
\pgfpathlineto{\pgfqpoint{0.599661in}{0.855806in}}%
\pgfpathlineto{\pgfqpoint{0.583333in}{0.888889in}}%
\pgfpathmoveto{\pgfqpoint{0.750000in}{0.888889in}}%
\pgfpathlineto{\pgfqpoint{0.733673in}{0.855806in}}%
\pgfpathlineto{\pgfqpoint{0.697164in}{0.850501in}}%
\pgfpathlineto{\pgfqpoint{0.723582in}{0.824750in}}%
\pgfpathlineto{\pgfqpoint{0.717345in}{0.788388in}}%
\pgfpathlineto{\pgfqpoint{0.750000in}{0.805556in}}%
\pgfpathlineto{\pgfqpoint{0.782655in}{0.788388in}}%
\pgfpathlineto{\pgfqpoint{0.776418in}{0.824750in}}%
\pgfpathlineto{\pgfqpoint{0.802836in}{0.850501in}}%
\pgfpathlineto{\pgfqpoint{0.766327in}{0.855806in}}%
\pgfpathlineto{\pgfqpoint{0.750000in}{0.888889in}}%
\pgfpathmoveto{\pgfqpoint{0.916667in}{0.888889in}}%
\pgfpathlineto{\pgfqpoint{0.900339in}{0.855806in}}%
\pgfpathlineto{\pgfqpoint{0.863830in}{0.850501in}}%
\pgfpathlineto{\pgfqpoint{0.890248in}{0.824750in}}%
\pgfpathlineto{\pgfqpoint{0.884012in}{0.788388in}}%
\pgfpathlineto{\pgfqpoint{0.916667in}{0.805556in}}%
\pgfpathlineto{\pgfqpoint{0.949321in}{0.788388in}}%
\pgfpathlineto{\pgfqpoint{0.943085in}{0.824750in}}%
\pgfpathlineto{\pgfqpoint{0.969503in}{0.850501in}}%
\pgfpathlineto{\pgfqpoint{0.932994in}{0.855806in}}%
\pgfpathlineto{\pgfqpoint{0.916667in}{0.888889in}}%
\pgfpathmoveto{\pgfqpoint{0.000000in}{1.055556in}}%
\pgfpathlineto{\pgfqpoint{-0.016327in}{1.022473in}}%
\pgfpathlineto{\pgfqpoint{-0.052836in}{1.017168in}}%
\pgfpathlineto{\pgfqpoint{-0.026418in}{0.991416in}}%
\pgfpathlineto{\pgfqpoint{-0.032655in}{0.955055in}}%
\pgfpathlineto{\pgfqpoint{-0.000000in}{0.972222in}}%
\pgfpathlineto{\pgfqpoint{0.032655in}{0.955055in}}%
\pgfpathlineto{\pgfqpoint{0.026418in}{0.991416in}}%
\pgfpathlineto{\pgfqpoint{0.052836in}{1.017168in}}%
\pgfpathlineto{\pgfqpoint{0.016327in}{1.022473in}}%
\pgfpathlineto{\pgfqpoint{0.000000in}{1.055556in}}%
\pgfpathmoveto{\pgfqpoint{0.166667in}{1.055556in}}%
\pgfpathlineto{\pgfqpoint{0.150339in}{1.022473in}}%
\pgfpathlineto{\pgfqpoint{0.113830in}{1.017168in}}%
\pgfpathlineto{\pgfqpoint{0.140248in}{0.991416in}}%
\pgfpathlineto{\pgfqpoint{0.134012in}{0.955055in}}%
\pgfpathlineto{\pgfqpoint{0.166667in}{0.972222in}}%
\pgfpathlineto{\pgfqpoint{0.199321in}{0.955055in}}%
\pgfpathlineto{\pgfqpoint{0.193085in}{0.991416in}}%
\pgfpathlineto{\pgfqpoint{0.219503in}{1.017168in}}%
\pgfpathlineto{\pgfqpoint{0.182994in}{1.022473in}}%
\pgfpathlineto{\pgfqpoint{0.166667in}{1.055556in}}%
\pgfpathmoveto{\pgfqpoint{0.333333in}{1.055556in}}%
\pgfpathlineto{\pgfqpoint{0.317006in}{1.022473in}}%
\pgfpathlineto{\pgfqpoint{0.280497in}{1.017168in}}%
\pgfpathlineto{\pgfqpoint{0.306915in}{0.991416in}}%
\pgfpathlineto{\pgfqpoint{0.300679in}{0.955055in}}%
\pgfpathlineto{\pgfqpoint{0.333333in}{0.972222in}}%
\pgfpathlineto{\pgfqpoint{0.365988in}{0.955055in}}%
\pgfpathlineto{\pgfqpoint{0.359752in}{0.991416in}}%
\pgfpathlineto{\pgfqpoint{0.386170in}{1.017168in}}%
\pgfpathlineto{\pgfqpoint{0.349661in}{1.022473in}}%
\pgfpathlineto{\pgfqpoint{0.333333in}{1.055556in}}%
\pgfpathmoveto{\pgfqpoint{0.500000in}{1.055556in}}%
\pgfpathlineto{\pgfqpoint{0.483673in}{1.022473in}}%
\pgfpathlineto{\pgfqpoint{0.447164in}{1.017168in}}%
\pgfpathlineto{\pgfqpoint{0.473582in}{0.991416in}}%
\pgfpathlineto{\pgfqpoint{0.467345in}{0.955055in}}%
\pgfpathlineto{\pgfqpoint{0.500000in}{0.972222in}}%
\pgfpathlineto{\pgfqpoint{0.532655in}{0.955055in}}%
\pgfpathlineto{\pgfqpoint{0.526418in}{0.991416in}}%
\pgfpathlineto{\pgfqpoint{0.552836in}{1.017168in}}%
\pgfpathlineto{\pgfqpoint{0.516327in}{1.022473in}}%
\pgfpathlineto{\pgfqpoint{0.500000in}{1.055556in}}%
\pgfpathmoveto{\pgfqpoint{0.666667in}{1.055556in}}%
\pgfpathlineto{\pgfqpoint{0.650339in}{1.022473in}}%
\pgfpathlineto{\pgfqpoint{0.613830in}{1.017168in}}%
\pgfpathlineto{\pgfqpoint{0.640248in}{0.991416in}}%
\pgfpathlineto{\pgfqpoint{0.634012in}{0.955055in}}%
\pgfpathlineto{\pgfqpoint{0.666667in}{0.972222in}}%
\pgfpathlineto{\pgfqpoint{0.699321in}{0.955055in}}%
\pgfpathlineto{\pgfqpoint{0.693085in}{0.991416in}}%
\pgfpathlineto{\pgfqpoint{0.719503in}{1.017168in}}%
\pgfpathlineto{\pgfqpoint{0.682994in}{1.022473in}}%
\pgfpathlineto{\pgfqpoint{0.666667in}{1.055556in}}%
\pgfpathmoveto{\pgfqpoint{0.833333in}{1.055556in}}%
\pgfpathlineto{\pgfqpoint{0.817006in}{1.022473in}}%
\pgfpathlineto{\pgfqpoint{0.780497in}{1.017168in}}%
\pgfpathlineto{\pgfqpoint{0.806915in}{0.991416in}}%
\pgfpathlineto{\pgfqpoint{0.800679in}{0.955055in}}%
\pgfpathlineto{\pgfqpoint{0.833333in}{0.972222in}}%
\pgfpathlineto{\pgfqpoint{0.865988in}{0.955055in}}%
\pgfpathlineto{\pgfqpoint{0.859752in}{0.991416in}}%
\pgfpathlineto{\pgfqpoint{0.886170in}{1.017168in}}%
\pgfpathlineto{\pgfqpoint{0.849661in}{1.022473in}}%
\pgfpathlineto{\pgfqpoint{0.833333in}{1.055556in}}%
\pgfpathmoveto{\pgfqpoint{1.000000in}{1.055556in}}%
\pgfpathlineto{\pgfqpoint{0.983673in}{1.022473in}}%
\pgfpathlineto{\pgfqpoint{0.947164in}{1.017168in}}%
\pgfpathlineto{\pgfqpoint{0.973582in}{0.991416in}}%
\pgfpathlineto{\pgfqpoint{0.967345in}{0.955055in}}%
\pgfpathlineto{\pgfqpoint{1.000000in}{0.972222in}}%
\pgfpathlineto{\pgfqpoint{1.032655in}{0.955055in}}%
\pgfpathlineto{\pgfqpoint{1.026418in}{0.991416in}}%
\pgfpathlineto{\pgfqpoint{1.052836in}{1.017168in}}%
\pgfpathlineto{\pgfqpoint{1.016327in}{1.022473in}}%
\pgfpathlineto{\pgfqpoint{1.000000in}{1.055556in}}%
\pgfpathlineto{\pgfqpoint{1.000000in}{1.055556in}}%
\pgfusepath{stroke}%
\end{pgfscope}%
}%
\pgfsys@transformshift{1.323315in}{2.557723in}%
\pgfsys@useobject{currentpattern}{}%
\pgfsys@transformshift{1in}{0in}%
\pgfsys@transformshift{-1in}{0in}%
\pgfsys@transformshift{0in}{1in}%
\end{pgfscope}%
\begin{pgfscope}%
\pgfpathrectangle{\pgfqpoint{0.935815in}{0.637495in}}{\pgfqpoint{9.300000in}{9.060000in}}%
\pgfusepath{clip}%
\pgfsetbuttcap%
\pgfsetmiterjoin%
\definecolor{currentfill}{rgb}{1.000000,1.000000,0.000000}%
\pgfsetfillcolor{currentfill}%
\pgfsetfillopacity{0.990000}%
\pgfsetlinewidth{0.000000pt}%
\definecolor{currentstroke}{rgb}{0.000000,0.000000,0.000000}%
\pgfsetstrokecolor{currentstroke}%
\pgfsetstrokeopacity{0.990000}%
\pgfsetdash{}{0pt}%
\pgfpathmoveto{\pgfqpoint{2.873315in}{3.755575in}}%
\pgfpathlineto{\pgfqpoint{3.648315in}{3.755575in}}%
\pgfpathlineto{\pgfqpoint{3.648315in}{4.265443in}}%
\pgfpathlineto{\pgfqpoint{2.873315in}{4.265443in}}%
\pgfpathclose%
\pgfusepath{fill}%
\end{pgfscope}%
\begin{pgfscope}%
\pgfsetbuttcap%
\pgfsetmiterjoin%
\definecolor{currentfill}{rgb}{1.000000,1.000000,0.000000}%
\pgfsetfillcolor{currentfill}%
\pgfsetfillopacity{0.990000}%
\pgfsetlinewidth{0.000000pt}%
\definecolor{currentstroke}{rgb}{0.000000,0.000000,0.000000}%
\pgfsetstrokecolor{currentstroke}%
\pgfsetstrokeopacity{0.990000}%
\pgfsetdash{}{0pt}%
\pgfpathrectangle{\pgfqpoint{0.935815in}{0.637495in}}{\pgfqpoint{9.300000in}{9.060000in}}%
\pgfusepath{clip}%
\pgfpathmoveto{\pgfqpoint{2.873315in}{3.755575in}}%
\pgfpathlineto{\pgfqpoint{3.648315in}{3.755575in}}%
\pgfpathlineto{\pgfqpoint{3.648315in}{4.265443in}}%
\pgfpathlineto{\pgfqpoint{2.873315in}{4.265443in}}%
\pgfpathclose%
\pgfusepath{clip}%
\pgfsys@defobject{currentpattern}{\pgfqpoint{0in}{0in}}{\pgfqpoint{1in}{1in}}{%
\begin{pgfscope}%
\pgfpathrectangle{\pgfqpoint{0in}{0in}}{\pgfqpoint{1in}{1in}}%
\pgfusepath{clip}%
\pgfpathmoveto{\pgfqpoint{0.000000in}{0.055556in}}%
\pgfpathlineto{\pgfqpoint{-0.016327in}{0.022473in}}%
\pgfpathlineto{\pgfqpoint{-0.052836in}{0.017168in}}%
\pgfpathlineto{\pgfqpoint{-0.026418in}{-0.008584in}}%
\pgfpathlineto{\pgfqpoint{-0.032655in}{-0.044945in}}%
\pgfpathlineto{\pgfqpoint{-0.000000in}{-0.027778in}}%
\pgfpathlineto{\pgfqpoint{0.032655in}{-0.044945in}}%
\pgfpathlineto{\pgfqpoint{0.026418in}{-0.008584in}}%
\pgfpathlineto{\pgfqpoint{0.052836in}{0.017168in}}%
\pgfpathlineto{\pgfqpoint{0.016327in}{0.022473in}}%
\pgfpathlineto{\pgfqpoint{0.000000in}{0.055556in}}%
\pgfpathmoveto{\pgfqpoint{0.166667in}{0.055556in}}%
\pgfpathlineto{\pgfqpoint{0.150339in}{0.022473in}}%
\pgfpathlineto{\pgfqpoint{0.113830in}{0.017168in}}%
\pgfpathlineto{\pgfqpoint{0.140248in}{-0.008584in}}%
\pgfpathlineto{\pgfqpoint{0.134012in}{-0.044945in}}%
\pgfpathlineto{\pgfqpoint{0.166667in}{-0.027778in}}%
\pgfpathlineto{\pgfqpoint{0.199321in}{-0.044945in}}%
\pgfpathlineto{\pgfqpoint{0.193085in}{-0.008584in}}%
\pgfpathlineto{\pgfqpoint{0.219503in}{0.017168in}}%
\pgfpathlineto{\pgfqpoint{0.182994in}{0.022473in}}%
\pgfpathlineto{\pgfqpoint{0.166667in}{0.055556in}}%
\pgfpathmoveto{\pgfqpoint{0.333333in}{0.055556in}}%
\pgfpathlineto{\pgfqpoint{0.317006in}{0.022473in}}%
\pgfpathlineto{\pgfqpoint{0.280497in}{0.017168in}}%
\pgfpathlineto{\pgfqpoint{0.306915in}{-0.008584in}}%
\pgfpathlineto{\pgfqpoint{0.300679in}{-0.044945in}}%
\pgfpathlineto{\pgfqpoint{0.333333in}{-0.027778in}}%
\pgfpathlineto{\pgfqpoint{0.365988in}{-0.044945in}}%
\pgfpathlineto{\pgfqpoint{0.359752in}{-0.008584in}}%
\pgfpathlineto{\pgfqpoint{0.386170in}{0.017168in}}%
\pgfpathlineto{\pgfqpoint{0.349661in}{0.022473in}}%
\pgfpathlineto{\pgfqpoint{0.333333in}{0.055556in}}%
\pgfpathmoveto{\pgfqpoint{0.500000in}{0.055556in}}%
\pgfpathlineto{\pgfqpoint{0.483673in}{0.022473in}}%
\pgfpathlineto{\pgfqpoint{0.447164in}{0.017168in}}%
\pgfpathlineto{\pgfqpoint{0.473582in}{-0.008584in}}%
\pgfpathlineto{\pgfqpoint{0.467345in}{-0.044945in}}%
\pgfpathlineto{\pgfqpoint{0.500000in}{-0.027778in}}%
\pgfpathlineto{\pgfqpoint{0.532655in}{-0.044945in}}%
\pgfpathlineto{\pgfqpoint{0.526418in}{-0.008584in}}%
\pgfpathlineto{\pgfqpoint{0.552836in}{0.017168in}}%
\pgfpathlineto{\pgfqpoint{0.516327in}{0.022473in}}%
\pgfpathlineto{\pgfqpoint{0.500000in}{0.055556in}}%
\pgfpathmoveto{\pgfqpoint{0.666667in}{0.055556in}}%
\pgfpathlineto{\pgfqpoint{0.650339in}{0.022473in}}%
\pgfpathlineto{\pgfqpoint{0.613830in}{0.017168in}}%
\pgfpathlineto{\pgfqpoint{0.640248in}{-0.008584in}}%
\pgfpathlineto{\pgfqpoint{0.634012in}{-0.044945in}}%
\pgfpathlineto{\pgfqpoint{0.666667in}{-0.027778in}}%
\pgfpathlineto{\pgfqpoint{0.699321in}{-0.044945in}}%
\pgfpathlineto{\pgfqpoint{0.693085in}{-0.008584in}}%
\pgfpathlineto{\pgfqpoint{0.719503in}{0.017168in}}%
\pgfpathlineto{\pgfqpoint{0.682994in}{0.022473in}}%
\pgfpathlineto{\pgfqpoint{0.666667in}{0.055556in}}%
\pgfpathmoveto{\pgfqpoint{0.833333in}{0.055556in}}%
\pgfpathlineto{\pgfqpoint{0.817006in}{0.022473in}}%
\pgfpathlineto{\pgfqpoint{0.780497in}{0.017168in}}%
\pgfpathlineto{\pgfqpoint{0.806915in}{-0.008584in}}%
\pgfpathlineto{\pgfqpoint{0.800679in}{-0.044945in}}%
\pgfpathlineto{\pgfqpoint{0.833333in}{-0.027778in}}%
\pgfpathlineto{\pgfqpoint{0.865988in}{-0.044945in}}%
\pgfpathlineto{\pgfqpoint{0.859752in}{-0.008584in}}%
\pgfpathlineto{\pgfqpoint{0.886170in}{0.017168in}}%
\pgfpathlineto{\pgfqpoint{0.849661in}{0.022473in}}%
\pgfpathlineto{\pgfqpoint{0.833333in}{0.055556in}}%
\pgfpathmoveto{\pgfqpoint{1.000000in}{0.055556in}}%
\pgfpathlineto{\pgfqpoint{0.983673in}{0.022473in}}%
\pgfpathlineto{\pgfqpoint{0.947164in}{0.017168in}}%
\pgfpathlineto{\pgfqpoint{0.973582in}{-0.008584in}}%
\pgfpathlineto{\pgfqpoint{0.967345in}{-0.044945in}}%
\pgfpathlineto{\pgfqpoint{1.000000in}{-0.027778in}}%
\pgfpathlineto{\pgfqpoint{1.032655in}{-0.044945in}}%
\pgfpathlineto{\pgfqpoint{1.026418in}{-0.008584in}}%
\pgfpathlineto{\pgfqpoint{1.052836in}{0.017168in}}%
\pgfpathlineto{\pgfqpoint{1.016327in}{0.022473in}}%
\pgfpathlineto{\pgfqpoint{1.000000in}{0.055556in}}%
\pgfpathmoveto{\pgfqpoint{0.083333in}{0.222222in}}%
\pgfpathlineto{\pgfqpoint{0.067006in}{0.189139in}}%
\pgfpathlineto{\pgfqpoint{0.030497in}{0.183834in}}%
\pgfpathlineto{\pgfqpoint{0.056915in}{0.158083in}}%
\pgfpathlineto{\pgfqpoint{0.050679in}{0.121721in}}%
\pgfpathlineto{\pgfqpoint{0.083333in}{0.138889in}}%
\pgfpathlineto{\pgfqpoint{0.115988in}{0.121721in}}%
\pgfpathlineto{\pgfqpoint{0.109752in}{0.158083in}}%
\pgfpathlineto{\pgfqpoint{0.136170in}{0.183834in}}%
\pgfpathlineto{\pgfqpoint{0.099661in}{0.189139in}}%
\pgfpathlineto{\pgfqpoint{0.083333in}{0.222222in}}%
\pgfpathmoveto{\pgfqpoint{0.250000in}{0.222222in}}%
\pgfpathlineto{\pgfqpoint{0.233673in}{0.189139in}}%
\pgfpathlineto{\pgfqpoint{0.197164in}{0.183834in}}%
\pgfpathlineto{\pgfqpoint{0.223582in}{0.158083in}}%
\pgfpathlineto{\pgfqpoint{0.217345in}{0.121721in}}%
\pgfpathlineto{\pgfqpoint{0.250000in}{0.138889in}}%
\pgfpathlineto{\pgfqpoint{0.282655in}{0.121721in}}%
\pgfpathlineto{\pgfqpoint{0.276418in}{0.158083in}}%
\pgfpathlineto{\pgfqpoint{0.302836in}{0.183834in}}%
\pgfpathlineto{\pgfqpoint{0.266327in}{0.189139in}}%
\pgfpathlineto{\pgfqpoint{0.250000in}{0.222222in}}%
\pgfpathmoveto{\pgfqpoint{0.416667in}{0.222222in}}%
\pgfpathlineto{\pgfqpoint{0.400339in}{0.189139in}}%
\pgfpathlineto{\pgfqpoint{0.363830in}{0.183834in}}%
\pgfpathlineto{\pgfqpoint{0.390248in}{0.158083in}}%
\pgfpathlineto{\pgfqpoint{0.384012in}{0.121721in}}%
\pgfpathlineto{\pgfqpoint{0.416667in}{0.138889in}}%
\pgfpathlineto{\pgfqpoint{0.449321in}{0.121721in}}%
\pgfpathlineto{\pgfqpoint{0.443085in}{0.158083in}}%
\pgfpathlineto{\pgfqpoint{0.469503in}{0.183834in}}%
\pgfpathlineto{\pgfqpoint{0.432994in}{0.189139in}}%
\pgfpathlineto{\pgfqpoint{0.416667in}{0.222222in}}%
\pgfpathmoveto{\pgfqpoint{0.583333in}{0.222222in}}%
\pgfpathlineto{\pgfqpoint{0.567006in}{0.189139in}}%
\pgfpathlineto{\pgfqpoint{0.530497in}{0.183834in}}%
\pgfpathlineto{\pgfqpoint{0.556915in}{0.158083in}}%
\pgfpathlineto{\pgfqpoint{0.550679in}{0.121721in}}%
\pgfpathlineto{\pgfqpoint{0.583333in}{0.138889in}}%
\pgfpathlineto{\pgfqpoint{0.615988in}{0.121721in}}%
\pgfpathlineto{\pgfqpoint{0.609752in}{0.158083in}}%
\pgfpathlineto{\pgfqpoint{0.636170in}{0.183834in}}%
\pgfpathlineto{\pgfqpoint{0.599661in}{0.189139in}}%
\pgfpathlineto{\pgfqpoint{0.583333in}{0.222222in}}%
\pgfpathmoveto{\pgfqpoint{0.750000in}{0.222222in}}%
\pgfpathlineto{\pgfqpoint{0.733673in}{0.189139in}}%
\pgfpathlineto{\pgfqpoint{0.697164in}{0.183834in}}%
\pgfpathlineto{\pgfqpoint{0.723582in}{0.158083in}}%
\pgfpathlineto{\pgfqpoint{0.717345in}{0.121721in}}%
\pgfpathlineto{\pgfqpoint{0.750000in}{0.138889in}}%
\pgfpathlineto{\pgfqpoint{0.782655in}{0.121721in}}%
\pgfpathlineto{\pgfqpoint{0.776418in}{0.158083in}}%
\pgfpathlineto{\pgfqpoint{0.802836in}{0.183834in}}%
\pgfpathlineto{\pgfqpoint{0.766327in}{0.189139in}}%
\pgfpathlineto{\pgfqpoint{0.750000in}{0.222222in}}%
\pgfpathmoveto{\pgfqpoint{0.916667in}{0.222222in}}%
\pgfpathlineto{\pgfqpoint{0.900339in}{0.189139in}}%
\pgfpathlineto{\pgfqpoint{0.863830in}{0.183834in}}%
\pgfpathlineto{\pgfqpoint{0.890248in}{0.158083in}}%
\pgfpathlineto{\pgfqpoint{0.884012in}{0.121721in}}%
\pgfpathlineto{\pgfqpoint{0.916667in}{0.138889in}}%
\pgfpathlineto{\pgfqpoint{0.949321in}{0.121721in}}%
\pgfpathlineto{\pgfqpoint{0.943085in}{0.158083in}}%
\pgfpathlineto{\pgfqpoint{0.969503in}{0.183834in}}%
\pgfpathlineto{\pgfqpoint{0.932994in}{0.189139in}}%
\pgfpathlineto{\pgfqpoint{0.916667in}{0.222222in}}%
\pgfpathmoveto{\pgfqpoint{0.000000in}{0.388889in}}%
\pgfpathlineto{\pgfqpoint{-0.016327in}{0.355806in}}%
\pgfpathlineto{\pgfqpoint{-0.052836in}{0.350501in}}%
\pgfpathlineto{\pgfqpoint{-0.026418in}{0.324750in}}%
\pgfpathlineto{\pgfqpoint{-0.032655in}{0.288388in}}%
\pgfpathlineto{\pgfqpoint{-0.000000in}{0.305556in}}%
\pgfpathlineto{\pgfqpoint{0.032655in}{0.288388in}}%
\pgfpathlineto{\pgfqpoint{0.026418in}{0.324750in}}%
\pgfpathlineto{\pgfqpoint{0.052836in}{0.350501in}}%
\pgfpathlineto{\pgfqpoint{0.016327in}{0.355806in}}%
\pgfpathlineto{\pgfqpoint{0.000000in}{0.388889in}}%
\pgfpathmoveto{\pgfqpoint{0.166667in}{0.388889in}}%
\pgfpathlineto{\pgfqpoint{0.150339in}{0.355806in}}%
\pgfpathlineto{\pgfqpoint{0.113830in}{0.350501in}}%
\pgfpathlineto{\pgfqpoint{0.140248in}{0.324750in}}%
\pgfpathlineto{\pgfqpoint{0.134012in}{0.288388in}}%
\pgfpathlineto{\pgfqpoint{0.166667in}{0.305556in}}%
\pgfpathlineto{\pgfqpoint{0.199321in}{0.288388in}}%
\pgfpathlineto{\pgfqpoint{0.193085in}{0.324750in}}%
\pgfpathlineto{\pgfqpoint{0.219503in}{0.350501in}}%
\pgfpathlineto{\pgfqpoint{0.182994in}{0.355806in}}%
\pgfpathlineto{\pgfqpoint{0.166667in}{0.388889in}}%
\pgfpathmoveto{\pgfqpoint{0.333333in}{0.388889in}}%
\pgfpathlineto{\pgfqpoint{0.317006in}{0.355806in}}%
\pgfpathlineto{\pgfqpoint{0.280497in}{0.350501in}}%
\pgfpathlineto{\pgfqpoint{0.306915in}{0.324750in}}%
\pgfpathlineto{\pgfqpoint{0.300679in}{0.288388in}}%
\pgfpathlineto{\pgfqpoint{0.333333in}{0.305556in}}%
\pgfpathlineto{\pgfqpoint{0.365988in}{0.288388in}}%
\pgfpathlineto{\pgfqpoint{0.359752in}{0.324750in}}%
\pgfpathlineto{\pgfqpoint{0.386170in}{0.350501in}}%
\pgfpathlineto{\pgfqpoint{0.349661in}{0.355806in}}%
\pgfpathlineto{\pgfqpoint{0.333333in}{0.388889in}}%
\pgfpathmoveto{\pgfqpoint{0.500000in}{0.388889in}}%
\pgfpathlineto{\pgfqpoint{0.483673in}{0.355806in}}%
\pgfpathlineto{\pgfqpoint{0.447164in}{0.350501in}}%
\pgfpathlineto{\pgfqpoint{0.473582in}{0.324750in}}%
\pgfpathlineto{\pgfqpoint{0.467345in}{0.288388in}}%
\pgfpathlineto{\pgfqpoint{0.500000in}{0.305556in}}%
\pgfpathlineto{\pgfqpoint{0.532655in}{0.288388in}}%
\pgfpathlineto{\pgfqpoint{0.526418in}{0.324750in}}%
\pgfpathlineto{\pgfqpoint{0.552836in}{0.350501in}}%
\pgfpathlineto{\pgfqpoint{0.516327in}{0.355806in}}%
\pgfpathlineto{\pgfqpoint{0.500000in}{0.388889in}}%
\pgfpathmoveto{\pgfqpoint{0.666667in}{0.388889in}}%
\pgfpathlineto{\pgfqpoint{0.650339in}{0.355806in}}%
\pgfpathlineto{\pgfqpoint{0.613830in}{0.350501in}}%
\pgfpathlineto{\pgfqpoint{0.640248in}{0.324750in}}%
\pgfpathlineto{\pgfqpoint{0.634012in}{0.288388in}}%
\pgfpathlineto{\pgfqpoint{0.666667in}{0.305556in}}%
\pgfpathlineto{\pgfqpoint{0.699321in}{0.288388in}}%
\pgfpathlineto{\pgfqpoint{0.693085in}{0.324750in}}%
\pgfpathlineto{\pgfqpoint{0.719503in}{0.350501in}}%
\pgfpathlineto{\pgfqpoint{0.682994in}{0.355806in}}%
\pgfpathlineto{\pgfqpoint{0.666667in}{0.388889in}}%
\pgfpathmoveto{\pgfqpoint{0.833333in}{0.388889in}}%
\pgfpathlineto{\pgfqpoint{0.817006in}{0.355806in}}%
\pgfpathlineto{\pgfqpoint{0.780497in}{0.350501in}}%
\pgfpathlineto{\pgfqpoint{0.806915in}{0.324750in}}%
\pgfpathlineto{\pgfqpoint{0.800679in}{0.288388in}}%
\pgfpathlineto{\pgfqpoint{0.833333in}{0.305556in}}%
\pgfpathlineto{\pgfqpoint{0.865988in}{0.288388in}}%
\pgfpathlineto{\pgfqpoint{0.859752in}{0.324750in}}%
\pgfpathlineto{\pgfqpoint{0.886170in}{0.350501in}}%
\pgfpathlineto{\pgfqpoint{0.849661in}{0.355806in}}%
\pgfpathlineto{\pgfqpoint{0.833333in}{0.388889in}}%
\pgfpathmoveto{\pgfqpoint{1.000000in}{0.388889in}}%
\pgfpathlineto{\pgfqpoint{0.983673in}{0.355806in}}%
\pgfpathlineto{\pgfqpoint{0.947164in}{0.350501in}}%
\pgfpathlineto{\pgfqpoint{0.973582in}{0.324750in}}%
\pgfpathlineto{\pgfqpoint{0.967345in}{0.288388in}}%
\pgfpathlineto{\pgfqpoint{1.000000in}{0.305556in}}%
\pgfpathlineto{\pgfqpoint{1.032655in}{0.288388in}}%
\pgfpathlineto{\pgfqpoint{1.026418in}{0.324750in}}%
\pgfpathlineto{\pgfqpoint{1.052836in}{0.350501in}}%
\pgfpathlineto{\pgfqpoint{1.016327in}{0.355806in}}%
\pgfpathlineto{\pgfqpoint{1.000000in}{0.388889in}}%
\pgfpathmoveto{\pgfqpoint{0.083333in}{0.555556in}}%
\pgfpathlineto{\pgfqpoint{0.067006in}{0.522473in}}%
\pgfpathlineto{\pgfqpoint{0.030497in}{0.517168in}}%
\pgfpathlineto{\pgfqpoint{0.056915in}{0.491416in}}%
\pgfpathlineto{\pgfqpoint{0.050679in}{0.455055in}}%
\pgfpathlineto{\pgfqpoint{0.083333in}{0.472222in}}%
\pgfpathlineto{\pgfqpoint{0.115988in}{0.455055in}}%
\pgfpathlineto{\pgfqpoint{0.109752in}{0.491416in}}%
\pgfpathlineto{\pgfqpoint{0.136170in}{0.517168in}}%
\pgfpathlineto{\pgfqpoint{0.099661in}{0.522473in}}%
\pgfpathlineto{\pgfqpoint{0.083333in}{0.555556in}}%
\pgfpathmoveto{\pgfqpoint{0.250000in}{0.555556in}}%
\pgfpathlineto{\pgfqpoint{0.233673in}{0.522473in}}%
\pgfpathlineto{\pgfqpoint{0.197164in}{0.517168in}}%
\pgfpathlineto{\pgfqpoint{0.223582in}{0.491416in}}%
\pgfpathlineto{\pgfqpoint{0.217345in}{0.455055in}}%
\pgfpathlineto{\pgfqpoint{0.250000in}{0.472222in}}%
\pgfpathlineto{\pgfqpoint{0.282655in}{0.455055in}}%
\pgfpathlineto{\pgfqpoint{0.276418in}{0.491416in}}%
\pgfpathlineto{\pgfqpoint{0.302836in}{0.517168in}}%
\pgfpathlineto{\pgfqpoint{0.266327in}{0.522473in}}%
\pgfpathlineto{\pgfqpoint{0.250000in}{0.555556in}}%
\pgfpathmoveto{\pgfqpoint{0.416667in}{0.555556in}}%
\pgfpathlineto{\pgfqpoint{0.400339in}{0.522473in}}%
\pgfpathlineto{\pgfqpoint{0.363830in}{0.517168in}}%
\pgfpathlineto{\pgfqpoint{0.390248in}{0.491416in}}%
\pgfpathlineto{\pgfqpoint{0.384012in}{0.455055in}}%
\pgfpathlineto{\pgfqpoint{0.416667in}{0.472222in}}%
\pgfpathlineto{\pgfqpoint{0.449321in}{0.455055in}}%
\pgfpathlineto{\pgfqpoint{0.443085in}{0.491416in}}%
\pgfpathlineto{\pgfqpoint{0.469503in}{0.517168in}}%
\pgfpathlineto{\pgfqpoint{0.432994in}{0.522473in}}%
\pgfpathlineto{\pgfqpoint{0.416667in}{0.555556in}}%
\pgfpathmoveto{\pgfqpoint{0.583333in}{0.555556in}}%
\pgfpathlineto{\pgfqpoint{0.567006in}{0.522473in}}%
\pgfpathlineto{\pgfqpoint{0.530497in}{0.517168in}}%
\pgfpathlineto{\pgfqpoint{0.556915in}{0.491416in}}%
\pgfpathlineto{\pgfqpoint{0.550679in}{0.455055in}}%
\pgfpathlineto{\pgfqpoint{0.583333in}{0.472222in}}%
\pgfpathlineto{\pgfqpoint{0.615988in}{0.455055in}}%
\pgfpathlineto{\pgfqpoint{0.609752in}{0.491416in}}%
\pgfpathlineto{\pgfqpoint{0.636170in}{0.517168in}}%
\pgfpathlineto{\pgfqpoint{0.599661in}{0.522473in}}%
\pgfpathlineto{\pgfqpoint{0.583333in}{0.555556in}}%
\pgfpathmoveto{\pgfqpoint{0.750000in}{0.555556in}}%
\pgfpathlineto{\pgfqpoint{0.733673in}{0.522473in}}%
\pgfpathlineto{\pgfqpoint{0.697164in}{0.517168in}}%
\pgfpathlineto{\pgfqpoint{0.723582in}{0.491416in}}%
\pgfpathlineto{\pgfqpoint{0.717345in}{0.455055in}}%
\pgfpathlineto{\pgfqpoint{0.750000in}{0.472222in}}%
\pgfpathlineto{\pgfqpoint{0.782655in}{0.455055in}}%
\pgfpathlineto{\pgfqpoint{0.776418in}{0.491416in}}%
\pgfpathlineto{\pgfqpoint{0.802836in}{0.517168in}}%
\pgfpathlineto{\pgfqpoint{0.766327in}{0.522473in}}%
\pgfpathlineto{\pgfqpoint{0.750000in}{0.555556in}}%
\pgfpathmoveto{\pgfqpoint{0.916667in}{0.555556in}}%
\pgfpathlineto{\pgfqpoint{0.900339in}{0.522473in}}%
\pgfpathlineto{\pgfqpoint{0.863830in}{0.517168in}}%
\pgfpathlineto{\pgfqpoint{0.890248in}{0.491416in}}%
\pgfpathlineto{\pgfqpoint{0.884012in}{0.455055in}}%
\pgfpathlineto{\pgfqpoint{0.916667in}{0.472222in}}%
\pgfpathlineto{\pgfqpoint{0.949321in}{0.455055in}}%
\pgfpathlineto{\pgfqpoint{0.943085in}{0.491416in}}%
\pgfpathlineto{\pgfqpoint{0.969503in}{0.517168in}}%
\pgfpathlineto{\pgfqpoint{0.932994in}{0.522473in}}%
\pgfpathlineto{\pgfqpoint{0.916667in}{0.555556in}}%
\pgfpathmoveto{\pgfqpoint{0.000000in}{0.722222in}}%
\pgfpathlineto{\pgfqpoint{-0.016327in}{0.689139in}}%
\pgfpathlineto{\pgfqpoint{-0.052836in}{0.683834in}}%
\pgfpathlineto{\pgfqpoint{-0.026418in}{0.658083in}}%
\pgfpathlineto{\pgfqpoint{-0.032655in}{0.621721in}}%
\pgfpathlineto{\pgfqpoint{-0.000000in}{0.638889in}}%
\pgfpathlineto{\pgfqpoint{0.032655in}{0.621721in}}%
\pgfpathlineto{\pgfqpoint{0.026418in}{0.658083in}}%
\pgfpathlineto{\pgfqpoint{0.052836in}{0.683834in}}%
\pgfpathlineto{\pgfqpoint{0.016327in}{0.689139in}}%
\pgfpathlineto{\pgfqpoint{0.000000in}{0.722222in}}%
\pgfpathmoveto{\pgfqpoint{0.166667in}{0.722222in}}%
\pgfpathlineto{\pgfqpoint{0.150339in}{0.689139in}}%
\pgfpathlineto{\pgfqpoint{0.113830in}{0.683834in}}%
\pgfpathlineto{\pgfqpoint{0.140248in}{0.658083in}}%
\pgfpathlineto{\pgfqpoint{0.134012in}{0.621721in}}%
\pgfpathlineto{\pgfqpoint{0.166667in}{0.638889in}}%
\pgfpathlineto{\pgfqpoint{0.199321in}{0.621721in}}%
\pgfpathlineto{\pgfqpoint{0.193085in}{0.658083in}}%
\pgfpathlineto{\pgfqpoint{0.219503in}{0.683834in}}%
\pgfpathlineto{\pgfqpoint{0.182994in}{0.689139in}}%
\pgfpathlineto{\pgfqpoint{0.166667in}{0.722222in}}%
\pgfpathmoveto{\pgfqpoint{0.333333in}{0.722222in}}%
\pgfpathlineto{\pgfqpoint{0.317006in}{0.689139in}}%
\pgfpathlineto{\pgfqpoint{0.280497in}{0.683834in}}%
\pgfpathlineto{\pgfqpoint{0.306915in}{0.658083in}}%
\pgfpathlineto{\pgfqpoint{0.300679in}{0.621721in}}%
\pgfpathlineto{\pgfqpoint{0.333333in}{0.638889in}}%
\pgfpathlineto{\pgfqpoint{0.365988in}{0.621721in}}%
\pgfpathlineto{\pgfqpoint{0.359752in}{0.658083in}}%
\pgfpathlineto{\pgfqpoint{0.386170in}{0.683834in}}%
\pgfpathlineto{\pgfqpoint{0.349661in}{0.689139in}}%
\pgfpathlineto{\pgfqpoint{0.333333in}{0.722222in}}%
\pgfpathmoveto{\pgfqpoint{0.500000in}{0.722222in}}%
\pgfpathlineto{\pgfqpoint{0.483673in}{0.689139in}}%
\pgfpathlineto{\pgfqpoint{0.447164in}{0.683834in}}%
\pgfpathlineto{\pgfqpoint{0.473582in}{0.658083in}}%
\pgfpathlineto{\pgfqpoint{0.467345in}{0.621721in}}%
\pgfpathlineto{\pgfqpoint{0.500000in}{0.638889in}}%
\pgfpathlineto{\pgfqpoint{0.532655in}{0.621721in}}%
\pgfpathlineto{\pgfqpoint{0.526418in}{0.658083in}}%
\pgfpathlineto{\pgfqpoint{0.552836in}{0.683834in}}%
\pgfpathlineto{\pgfqpoint{0.516327in}{0.689139in}}%
\pgfpathlineto{\pgfqpoint{0.500000in}{0.722222in}}%
\pgfpathmoveto{\pgfqpoint{0.666667in}{0.722222in}}%
\pgfpathlineto{\pgfqpoint{0.650339in}{0.689139in}}%
\pgfpathlineto{\pgfqpoint{0.613830in}{0.683834in}}%
\pgfpathlineto{\pgfqpoint{0.640248in}{0.658083in}}%
\pgfpathlineto{\pgfqpoint{0.634012in}{0.621721in}}%
\pgfpathlineto{\pgfqpoint{0.666667in}{0.638889in}}%
\pgfpathlineto{\pgfqpoint{0.699321in}{0.621721in}}%
\pgfpathlineto{\pgfqpoint{0.693085in}{0.658083in}}%
\pgfpathlineto{\pgfqpoint{0.719503in}{0.683834in}}%
\pgfpathlineto{\pgfqpoint{0.682994in}{0.689139in}}%
\pgfpathlineto{\pgfqpoint{0.666667in}{0.722222in}}%
\pgfpathmoveto{\pgfqpoint{0.833333in}{0.722222in}}%
\pgfpathlineto{\pgfqpoint{0.817006in}{0.689139in}}%
\pgfpathlineto{\pgfqpoint{0.780497in}{0.683834in}}%
\pgfpathlineto{\pgfqpoint{0.806915in}{0.658083in}}%
\pgfpathlineto{\pgfqpoint{0.800679in}{0.621721in}}%
\pgfpathlineto{\pgfqpoint{0.833333in}{0.638889in}}%
\pgfpathlineto{\pgfqpoint{0.865988in}{0.621721in}}%
\pgfpathlineto{\pgfqpoint{0.859752in}{0.658083in}}%
\pgfpathlineto{\pgfqpoint{0.886170in}{0.683834in}}%
\pgfpathlineto{\pgfqpoint{0.849661in}{0.689139in}}%
\pgfpathlineto{\pgfqpoint{0.833333in}{0.722222in}}%
\pgfpathmoveto{\pgfqpoint{1.000000in}{0.722222in}}%
\pgfpathlineto{\pgfqpoint{0.983673in}{0.689139in}}%
\pgfpathlineto{\pgfqpoint{0.947164in}{0.683834in}}%
\pgfpathlineto{\pgfqpoint{0.973582in}{0.658083in}}%
\pgfpathlineto{\pgfqpoint{0.967345in}{0.621721in}}%
\pgfpathlineto{\pgfqpoint{1.000000in}{0.638889in}}%
\pgfpathlineto{\pgfqpoint{1.032655in}{0.621721in}}%
\pgfpathlineto{\pgfqpoint{1.026418in}{0.658083in}}%
\pgfpathlineto{\pgfqpoint{1.052836in}{0.683834in}}%
\pgfpathlineto{\pgfqpoint{1.016327in}{0.689139in}}%
\pgfpathlineto{\pgfqpoint{1.000000in}{0.722222in}}%
\pgfpathmoveto{\pgfqpoint{0.083333in}{0.888889in}}%
\pgfpathlineto{\pgfqpoint{0.067006in}{0.855806in}}%
\pgfpathlineto{\pgfqpoint{0.030497in}{0.850501in}}%
\pgfpathlineto{\pgfqpoint{0.056915in}{0.824750in}}%
\pgfpathlineto{\pgfqpoint{0.050679in}{0.788388in}}%
\pgfpathlineto{\pgfqpoint{0.083333in}{0.805556in}}%
\pgfpathlineto{\pgfqpoint{0.115988in}{0.788388in}}%
\pgfpathlineto{\pgfqpoint{0.109752in}{0.824750in}}%
\pgfpathlineto{\pgfqpoint{0.136170in}{0.850501in}}%
\pgfpathlineto{\pgfqpoint{0.099661in}{0.855806in}}%
\pgfpathlineto{\pgfqpoint{0.083333in}{0.888889in}}%
\pgfpathmoveto{\pgfqpoint{0.250000in}{0.888889in}}%
\pgfpathlineto{\pgfqpoint{0.233673in}{0.855806in}}%
\pgfpathlineto{\pgfqpoint{0.197164in}{0.850501in}}%
\pgfpathlineto{\pgfqpoint{0.223582in}{0.824750in}}%
\pgfpathlineto{\pgfqpoint{0.217345in}{0.788388in}}%
\pgfpathlineto{\pgfqpoint{0.250000in}{0.805556in}}%
\pgfpathlineto{\pgfqpoint{0.282655in}{0.788388in}}%
\pgfpathlineto{\pgfqpoint{0.276418in}{0.824750in}}%
\pgfpathlineto{\pgfqpoint{0.302836in}{0.850501in}}%
\pgfpathlineto{\pgfqpoint{0.266327in}{0.855806in}}%
\pgfpathlineto{\pgfqpoint{0.250000in}{0.888889in}}%
\pgfpathmoveto{\pgfqpoint{0.416667in}{0.888889in}}%
\pgfpathlineto{\pgfqpoint{0.400339in}{0.855806in}}%
\pgfpathlineto{\pgfqpoint{0.363830in}{0.850501in}}%
\pgfpathlineto{\pgfqpoint{0.390248in}{0.824750in}}%
\pgfpathlineto{\pgfqpoint{0.384012in}{0.788388in}}%
\pgfpathlineto{\pgfqpoint{0.416667in}{0.805556in}}%
\pgfpathlineto{\pgfqpoint{0.449321in}{0.788388in}}%
\pgfpathlineto{\pgfqpoint{0.443085in}{0.824750in}}%
\pgfpathlineto{\pgfqpoint{0.469503in}{0.850501in}}%
\pgfpathlineto{\pgfqpoint{0.432994in}{0.855806in}}%
\pgfpathlineto{\pgfqpoint{0.416667in}{0.888889in}}%
\pgfpathmoveto{\pgfqpoint{0.583333in}{0.888889in}}%
\pgfpathlineto{\pgfqpoint{0.567006in}{0.855806in}}%
\pgfpathlineto{\pgfqpoint{0.530497in}{0.850501in}}%
\pgfpathlineto{\pgfqpoint{0.556915in}{0.824750in}}%
\pgfpathlineto{\pgfqpoint{0.550679in}{0.788388in}}%
\pgfpathlineto{\pgfqpoint{0.583333in}{0.805556in}}%
\pgfpathlineto{\pgfqpoint{0.615988in}{0.788388in}}%
\pgfpathlineto{\pgfqpoint{0.609752in}{0.824750in}}%
\pgfpathlineto{\pgfqpoint{0.636170in}{0.850501in}}%
\pgfpathlineto{\pgfqpoint{0.599661in}{0.855806in}}%
\pgfpathlineto{\pgfqpoint{0.583333in}{0.888889in}}%
\pgfpathmoveto{\pgfqpoint{0.750000in}{0.888889in}}%
\pgfpathlineto{\pgfqpoint{0.733673in}{0.855806in}}%
\pgfpathlineto{\pgfqpoint{0.697164in}{0.850501in}}%
\pgfpathlineto{\pgfqpoint{0.723582in}{0.824750in}}%
\pgfpathlineto{\pgfqpoint{0.717345in}{0.788388in}}%
\pgfpathlineto{\pgfqpoint{0.750000in}{0.805556in}}%
\pgfpathlineto{\pgfqpoint{0.782655in}{0.788388in}}%
\pgfpathlineto{\pgfqpoint{0.776418in}{0.824750in}}%
\pgfpathlineto{\pgfqpoint{0.802836in}{0.850501in}}%
\pgfpathlineto{\pgfqpoint{0.766327in}{0.855806in}}%
\pgfpathlineto{\pgfqpoint{0.750000in}{0.888889in}}%
\pgfpathmoveto{\pgfqpoint{0.916667in}{0.888889in}}%
\pgfpathlineto{\pgfqpoint{0.900339in}{0.855806in}}%
\pgfpathlineto{\pgfqpoint{0.863830in}{0.850501in}}%
\pgfpathlineto{\pgfqpoint{0.890248in}{0.824750in}}%
\pgfpathlineto{\pgfqpoint{0.884012in}{0.788388in}}%
\pgfpathlineto{\pgfqpoint{0.916667in}{0.805556in}}%
\pgfpathlineto{\pgfqpoint{0.949321in}{0.788388in}}%
\pgfpathlineto{\pgfqpoint{0.943085in}{0.824750in}}%
\pgfpathlineto{\pgfqpoint{0.969503in}{0.850501in}}%
\pgfpathlineto{\pgfqpoint{0.932994in}{0.855806in}}%
\pgfpathlineto{\pgfqpoint{0.916667in}{0.888889in}}%
\pgfpathmoveto{\pgfqpoint{0.000000in}{1.055556in}}%
\pgfpathlineto{\pgfqpoint{-0.016327in}{1.022473in}}%
\pgfpathlineto{\pgfqpoint{-0.052836in}{1.017168in}}%
\pgfpathlineto{\pgfqpoint{-0.026418in}{0.991416in}}%
\pgfpathlineto{\pgfqpoint{-0.032655in}{0.955055in}}%
\pgfpathlineto{\pgfqpoint{-0.000000in}{0.972222in}}%
\pgfpathlineto{\pgfqpoint{0.032655in}{0.955055in}}%
\pgfpathlineto{\pgfqpoint{0.026418in}{0.991416in}}%
\pgfpathlineto{\pgfqpoint{0.052836in}{1.017168in}}%
\pgfpathlineto{\pgfqpoint{0.016327in}{1.022473in}}%
\pgfpathlineto{\pgfqpoint{0.000000in}{1.055556in}}%
\pgfpathmoveto{\pgfqpoint{0.166667in}{1.055556in}}%
\pgfpathlineto{\pgfqpoint{0.150339in}{1.022473in}}%
\pgfpathlineto{\pgfqpoint{0.113830in}{1.017168in}}%
\pgfpathlineto{\pgfqpoint{0.140248in}{0.991416in}}%
\pgfpathlineto{\pgfqpoint{0.134012in}{0.955055in}}%
\pgfpathlineto{\pgfqpoint{0.166667in}{0.972222in}}%
\pgfpathlineto{\pgfqpoint{0.199321in}{0.955055in}}%
\pgfpathlineto{\pgfqpoint{0.193085in}{0.991416in}}%
\pgfpathlineto{\pgfqpoint{0.219503in}{1.017168in}}%
\pgfpathlineto{\pgfqpoint{0.182994in}{1.022473in}}%
\pgfpathlineto{\pgfqpoint{0.166667in}{1.055556in}}%
\pgfpathmoveto{\pgfqpoint{0.333333in}{1.055556in}}%
\pgfpathlineto{\pgfqpoint{0.317006in}{1.022473in}}%
\pgfpathlineto{\pgfqpoint{0.280497in}{1.017168in}}%
\pgfpathlineto{\pgfqpoint{0.306915in}{0.991416in}}%
\pgfpathlineto{\pgfqpoint{0.300679in}{0.955055in}}%
\pgfpathlineto{\pgfqpoint{0.333333in}{0.972222in}}%
\pgfpathlineto{\pgfqpoint{0.365988in}{0.955055in}}%
\pgfpathlineto{\pgfqpoint{0.359752in}{0.991416in}}%
\pgfpathlineto{\pgfqpoint{0.386170in}{1.017168in}}%
\pgfpathlineto{\pgfqpoint{0.349661in}{1.022473in}}%
\pgfpathlineto{\pgfqpoint{0.333333in}{1.055556in}}%
\pgfpathmoveto{\pgfqpoint{0.500000in}{1.055556in}}%
\pgfpathlineto{\pgfqpoint{0.483673in}{1.022473in}}%
\pgfpathlineto{\pgfqpoint{0.447164in}{1.017168in}}%
\pgfpathlineto{\pgfqpoint{0.473582in}{0.991416in}}%
\pgfpathlineto{\pgfqpoint{0.467345in}{0.955055in}}%
\pgfpathlineto{\pgfqpoint{0.500000in}{0.972222in}}%
\pgfpathlineto{\pgfqpoint{0.532655in}{0.955055in}}%
\pgfpathlineto{\pgfqpoint{0.526418in}{0.991416in}}%
\pgfpathlineto{\pgfqpoint{0.552836in}{1.017168in}}%
\pgfpathlineto{\pgfqpoint{0.516327in}{1.022473in}}%
\pgfpathlineto{\pgfqpoint{0.500000in}{1.055556in}}%
\pgfpathmoveto{\pgfqpoint{0.666667in}{1.055556in}}%
\pgfpathlineto{\pgfqpoint{0.650339in}{1.022473in}}%
\pgfpathlineto{\pgfqpoint{0.613830in}{1.017168in}}%
\pgfpathlineto{\pgfqpoint{0.640248in}{0.991416in}}%
\pgfpathlineto{\pgfqpoint{0.634012in}{0.955055in}}%
\pgfpathlineto{\pgfqpoint{0.666667in}{0.972222in}}%
\pgfpathlineto{\pgfqpoint{0.699321in}{0.955055in}}%
\pgfpathlineto{\pgfqpoint{0.693085in}{0.991416in}}%
\pgfpathlineto{\pgfqpoint{0.719503in}{1.017168in}}%
\pgfpathlineto{\pgfqpoint{0.682994in}{1.022473in}}%
\pgfpathlineto{\pgfqpoint{0.666667in}{1.055556in}}%
\pgfpathmoveto{\pgfqpoint{0.833333in}{1.055556in}}%
\pgfpathlineto{\pgfqpoint{0.817006in}{1.022473in}}%
\pgfpathlineto{\pgfqpoint{0.780497in}{1.017168in}}%
\pgfpathlineto{\pgfqpoint{0.806915in}{0.991416in}}%
\pgfpathlineto{\pgfqpoint{0.800679in}{0.955055in}}%
\pgfpathlineto{\pgfqpoint{0.833333in}{0.972222in}}%
\pgfpathlineto{\pgfqpoint{0.865988in}{0.955055in}}%
\pgfpathlineto{\pgfqpoint{0.859752in}{0.991416in}}%
\pgfpathlineto{\pgfqpoint{0.886170in}{1.017168in}}%
\pgfpathlineto{\pgfqpoint{0.849661in}{1.022473in}}%
\pgfpathlineto{\pgfqpoint{0.833333in}{1.055556in}}%
\pgfpathmoveto{\pgfqpoint{1.000000in}{1.055556in}}%
\pgfpathlineto{\pgfqpoint{0.983673in}{1.022473in}}%
\pgfpathlineto{\pgfqpoint{0.947164in}{1.017168in}}%
\pgfpathlineto{\pgfqpoint{0.973582in}{0.991416in}}%
\pgfpathlineto{\pgfqpoint{0.967345in}{0.955055in}}%
\pgfpathlineto{\pgfqpoint{1.000000in}{0.972222in}}%
\pgfpathlineto{\pgfqpoint{1.032655in}{0.955055in}}%
\pgfpathlineto{\pgfqpoint{1.026418in}{0.991416in}}%
\pgfpathlineto{\pgfqpoint{1.052836in}{1.017168in}}%
\pgfpathlineto{\pgfqpoint{1.016327in}{1.022473in}}%
\pgfpathlineto{\pgfqpoint{1.000000in}{1.055556in}}%
\pgfpathlineto{\pgfqpoint{1.000000in}{1.055556in}}%
\pgfusepath{stroke}%
\end{pgfscope}%
}%
\pgfsys@transformshift{2.873315in}{3.755575in}%
\pgfsys@useobject{currentpattern}{}%
\pgfsys@transformshift{1in}{0in}%
\pgfsys@transformshift{-1in}{0in}%
\pgfsys@transformshift{0in}{1in}%
\end{pgfscope}%
\begin{pgfscope}%
\pgfpathrectangle{\pgfqpoint{0.935815in}{0.637495in}}{\pgfqpoint{9.300000in}{9.060000in}}%
\pgfusepath{clip}%
\pgfsetbuttcap%
\pgfsetmiterjoin%
\definecolor{currentfill}{rgb}{1.000000,1.000000,0.000000}%
\pgfsetfillcolor{currentfill}%
\pgfsetfillopacity{0.990000}%
\pgfsetlinewidth{0.000000pt}%
\definecolor{currentstroke}{rgb}{0.000000,0.000000,0.000000}%
\pgfsetstrokecolor{currentstroke}%
\pgfsetstrokeopacity{0.990000}%
\pgfsetdash{}{0pt}%
\pgfpathmoveto{\pgfqpoint{4.423315in}{4.688801in}}%
\pgfpathlineto{\pgfqpoint{5.198315in}{4.688801in}}%
\pgfpathlineto{\pgfqpoint{5.198315in}{5.531339in}}%
\pgfpathlineto{\pgfqpoint{4.423315in}{5.531339in}}%
\pgfpathclose%
\pgfusepath{fill}%
\end{pgfscope}%
\begin{pgfscope}%
\pgfsetbuttcap%
\pgfsetmiterjoin%
\definecolor{currentfill}{rgb}{1.000000,1.000000,0.000000}%
\pgfsetfillcolor{currentfill}%
\pgfsetfillopacity{0.990000}%
\pgfsetlinewidth{0.000000pt}%
\definecolor{currentstroke}{rgb}{0.000000,0.000000,0.000000}%
\pgfsetstrokecolor{currentstroke}%
\pgfsetstrokeopacity{0.990000}%
\pgfsetdash{}{0pt}%
\pgfpathrectangle{\pgfqpoint{0.935815in}{0.637495in}}{\pgfqpoint{9.300000in}{9.060000in}}%
\pgfusepath{clip}%
\pgfpathmoveto{\pgfqpoint{4.423315in}{4.688801in}}%
\pgfpathlineto{\pgfqpoint{5.198315in}{4.688801in}}%
\pgfpathlineto{\pgfqpoint{5.198315in}{5.531339in}}%
\pgfpathlineto{\pgfqpoint{4.423315in}{5.531339in}}%
\pgfpathclose%
\pgfusepath{clip}%
\pgfsys@defobject{currentpattern}{\pgfqpoint{0in}{0in}}{\pgfqpoint{1in}{1in}}{%
\begin{pgfscope}%
\pgfpathrectangle{\pgfqpoint{0in}{0in}}{\pgfqpoint{1in}{1in}}%
\pgfusepath{clip}%
\pgfpathmoveto{\pgfqpoint{0.000000in}{0.055556in}}%
\pgfpathlineto{\pgfqpoint{-0.016327in}{0.022473in}}%
\pgfpathlineto{\pgfqpoint{-0.052836in}{0.017168in}}%
\pgfpathlineto{\pgfqpoint{-0.026418in}{-0.008584in}}%
\pgfpathlineto{\pgfqpoint{-0.032655in}{-0.044945in}}%
\pgfpathlineto{\pgfqpoint{-0.000000in}{-0.027778in}}%
\pgfpathlineto{\pgfqpoint{0.032655in}{-0.044945in}}%
\pgfpathlineto{\pgfqpoint{0.026418in}{-0.008584in}}%
\pgfpathlineto{\pgfqpoint{0.052836in}{0.017168in}}%
\pgfpathlineto{\pgfqpoint{0.016327in}{0.022473in}}%
\pgfpathlineto{\pgfqpoint{0.000000in}{0.055556in}}%
\pgfpathmoveto{\pgfqpoint{0.166667in}{0.055556in}}%
\pgfpathlineto{\pgfqpoint{0.150339in}{0.022473in}}%
\pgfpathlineto{\pgfqpoint{0.113830in}{0.017168in}}%
\pgfpathlineto{\pgfqpoint{0.140248in}{-0.008584in}}%
\pgfpathlineto{\pgfqpoint{0.134012in}{-0.044945in}}%
\pgfpathlineto{\pgfqpoint{0.166667in}{-0.027778in}}%
\pgfpathlineto{\pgfqpoint{0.199321in}{-0.044945in}}%
\pgfpathlineto{\pgfqpoint{0.193085in}{-0.008584in}}%
\pgfpathlineto{\pgfqpoint{0.219503in}{0.017168in}}%
\pgfpathlineto{\pgfqpoint{0.182994in}{0.022473in}}%
\pgfpathlineto{\pgfqpoint{0.166667in}{0.055556in}}%
\pgfpathmoveto{\pgfqpoint{0.333333in}{0.055556in}}%
\pgfpathlineto{\pgfqpoint{0.317006in}{0.022473in}}%
\pgfpathlineto{\pgfqpoint{0.280497in}{0.017168in}}%
\pgfpathlineto{\pgfqpoint{0.306915in}{-0.008584in}}%
\pgfpathlineto{\pgfqpoint{0.300679in}{-0.044945in}}%
\pgfpathlineto{\pgfqpoint{0.333333in}{-0.027778in}}%
\pgfpathlineto{\pgfqpoint{0.365988in}{-0.044945in}}%
\pgfpathlineto{\pgfqpoint{0.359752in}{-0.008584in}}%
\pgfpathlineto{\pgfqpoint{0.386170in}{0.017168in}}%
\pgfpathlineto{\pgfqpoint{0.349661in}{0.022473in}}%
\pgfpathlineto{\pgfqpoint{0.333333in}{0.055556in}}%
\pgfpathmoveto{\pgfqpoint{0.500000in}{0.055556in}}%
\pgfpathlineto{\pgfqpoint{0.483673in}{0.022473in}}%
\pgfpathlineto{\pgfqpoint{0.447164in}{0.017168in}}%
\pgfpathlineto{\pgfqpoint{0.473582in}{-0.008584in}}%
\pgfpathlineto{\pgfqpoint{0.467345in}{-0.044945in}}%
\pgfpathlineto{\pgfqpoint{0.500000in}{-0.027778in}}%
\pgfpathlineto{\pgfqpoint{0.532655in}{-0.044945in}}%
\pgfpathlineto{\pgfqpoint{0.526418in}{-0.008584in}}%
\pgfpathlineto{\pgfqpoint{0.552836in}{0.017168in}}%
\pgfpathlineto{\pgfqpoint{0.516327in}{0.022473in}}%
\pgfpathlineto{\pgfqpoint{0.500000in}{0.055556in}}%
\pgfpathmoveto{\pgfqpoint{0.666667in}{0.055556in}}%
\pgfpathlineto{\pgfqpoint{0.650339in}{0.022473in}}%
\pgfpathlineto{\pgfqpoint{0.613830in}{0.017168in}}%
\pgfpathlineto{\pgfqpoint{0.640248in}{-0.008584in}}%
\pgfpathlineto{\pgfqpoint{0.634012in}{-0.044945in}}%
\pgfpathlineto{\pgfqpoint{0.666667in}{-0.027778in}}%
\pgfpathlineto{\pgfqpoint{0.699321in}{-0.044945in}}%
\pgfpathlineto{\pgfqpoint{0.693085in}{-0.008584in}}%
\pgfpathlineto{\pgfqpoint{0.719503in}{0.017168in}}%
\pgfpathlineto{\pgfqpoint{0.682994in}{0.022473in}}%
\pgfpathlineto{\pgfqpoint{0.666667in}{0.055556in}}%
\pgfpathmoveto{\pgfqpoint{0.833333in}{0.055556in}}%
\pgfpathlineto{\pgfqpoint{0.817006in}{0.022473in}}%
\pgfpathlineto{\pgfqpoint{0.780497in}{0.017168in}}%
\pgfpathlineto{\pgfqpoint{0.806915in}{-0.008584in}}%
\pgfpathlineto{\pgfqpoint{0.800679in}{-0.044945in}}%
\pgfpathlineto{\pgfqpoint{0.833333in}{-0.027778in}}%
\pgfpathlineto{\pgfqpoint{0.865988in}{-0.044945in}}%
\pgfpathlineto{\pgfqpoint{0.859752in}{-0.008584in}}%
\pgfpathlineto{\pgfqpoint{0.886170in}{0.017168in}}%
\pgfpathlineto{\pgfqpoint{0.849661in}{0.022473in}}%
\pgfpathlineto{\pgfqpoint{0.833333in}{0.055556in}}%
\pgfpathmoveto{\pgfqpoint{1.000000in}{0.055556in}}%
\pgfpathlineto{\pgfqpoint{0.983673in}{0.022473in}}%
\pgfpathlineto{\pgfqpoint{0.947164in}{0.017168in}}%
\pgfpathlineto{\pgfqpoint{0.973582in}{-0.008584in}}%
\pgfpathlineto{\pgfqpoint{0.967345in}{-0.044945in}}%
\pgfpathlineto{\pgfqpoint{1.000000in}{-0.027778in}}%
\pgfpathlineto{\pgfqpoint{1.032655in}{-0.044945in}}%
\pgfpathlineto{\pgfqpoint{1.026418in}{-0.008584in}}%
\pgfpathlineto{\pgfqpoint{1.052836in}{0.017168in}}%
\pgfpathlineto{\pgfqpoint{1.016327in}{0.022473in}}%
\pgfpathlineto{\pgfqpoint{1.000000in}{0.055556in}}%
\pgfpathmoveto{\pgfqpoint{0.083333in}{0.222222in}}%
\pgfpathlineto{\pgfqpoint{0.067006in}{0.189139in}}%
\pgfpathlineto{\pgfqpoint{0.030497in}{0.183834in}}%
\pgfpathlineto{\pgfqpoint{0.056915in}{0.158083in}}%
\pgfpathlineto{\pgfqpoint{0.050679in}{0.121721in}}%
\pgfpathlineto{\pgfqpoint{0.083333in}{0.138889in}}%
\pgfpathlineto{\pgfqpoint{0.115988in}{0.121721in}}%
\pgfpathlineto{\pgfqpoint{0.109752in}{0.158083in}}%
\pgfpathlineto{\pgfqpoint{0.136170in}{0.183834in}}%
\pgfpathlineto{\pgfqpoint{0.099661in}{0.189139in}}%
\pgfpathlineto{\pgfqpoint{0.083333in}{0.222222in}}%
\pgfpathmoveto{\pgfqpoint{0.250000in}{0.222222in}}%
\pgfpathlineto{\pgfqpoint{0.233673in}{0.189139in}}%
\pgfpathlineto{\pgfqpoint{0.197164in}{0.183834in}}%
\pgfpathlineto{\pgfqpoint{0.223582in}{0.158083in}}%
\pgfpathlineto{\pgfqpoint{0.217345in}{0.121721in}}%
\pgfpathlineto{\pgfqpoint{0.250000in}{0.138889in}}%
\pgfpathlineto{\pgfqpoint{0.282655in}{0.121721in}}%
\pgfpathlineto{\pgfqpoint{0.276418in}{0.158083in}}%
\pgfpathlineto{\pgfqpoint{0.302836in}{0.183834in}}%
\pgfpathlineto{\pgfqpoint{0.266327in}{0.189139in}}%
\pgfpathlineto{\pgfqpoint{0.250000in}{0.222222in}}%
\pgfpathmoveto{\pgfqpoint{0.416667in}{0.222222in}}%
\pgfpathlineto{\pgfqpoint{0.400339in}{0.189139in}}%
\pgfpathlineto{\pgfqpoint{0.363830in}{0.183834in}}%
\pgfpathlineto{\pgfqpoint{0.390248in}{0.158083in}}%
\pgfpathlineto{\pgfqpoint{0.384012in}{0.121721in}}%
\pgfpathlineto{\pgfqpoint{0.416667in}{0.138889in}}%
\pgfpathlineto{\pgfqpoint{0.449321in}{0.121721in}}%
\pgfpathlineto{\pgfqpoint{0.443085in}{0.158083in}}%
\pgfpathlineto{\pgfqpoint{0.469503in}{0.183834in}}%
\pgfpathlineto{\pgfqpoint{0.432994in}{0.189139in}}%
\pgfpathlineto{\pgfqpoint{0.416667in}{0.222222in}}%
\pgfpathmoveto{\pgfqpoint{0.583333in}{0.222222in}}%
\pgfpathlineto{\pgfqpoint{0.567006in}{0.189139in}}%
\pgfpathlineto{\pgfqpoint{0.530497in}{0.183834in}}%
\pgfpathlineto{\pgfqpoint{0.556915in}{0.158083in}}%
\pgfpathlineto{\pgfqpoint{0.550679in}{0.121721in}}%
\pgfpathlineto{\pgfqpoint{0.583333in}{0.138889in}}%
\pgfpathlineto{\pgfqpoint{0.615988in}{0.121721in}}%
\pgfpathlineto{\pgfqpoint{0.609752in}{0.158083in}}%
\pgfpathlineto{\pgfqpoint{0.636170in}{0.183834in}}%
\pgfpathlineto{\pgfqpoint{0.599661in}{0.189139in}}%
\pgfpathlineto{\pgfqpoint{0.583333in}{0.222222in}}%
\pgfpathmoveto{\pgfqpoint{0.750000in}{0.222222in}}%
\pgfpathlineto{\pgfqpoint{0.733673in}{0.189139in}}%
\pgfpathlineto{\pgfqpoint{0.697164in}{0.183834in}}%
\pgfpathlineto{\pgfqpoint{0.723582in}{0.158083in}}%
\pgfpathlineto{\pgfqpoint{0.717345in}{0.121721in}}%
\pgfpathlineto{\pgfqpoint{0.750000in}{0.138889in}}%
\pgfpathlineto{\pgfqpoint{0.782655in}{0.121721in}}%
\pgfpathlineto{\pgfqpoint{0.776418in}{0.158083in}}%
\pgfpathlineto{\pgfqpoint{0.802836in}{0.183834in}}%
\pgfpathlineto{\pgfqpoint{0.766327in}{0.189139in}}%
\pgfpathlineto{\pgfqpoint{0.750000in}{0.222222in}}%
\pgfpathmoveto{\pgfqpoint{0.916667in}{0.222222in}}%
\pgfpathlineto{\pgfqpoint{0.900339in}{0.189139in}}%
\pgfpathlineto{\pgfqpoint{0.863830in}{0.183834in}}%
\pgfpathlineto{\pgfqpoint{0.890248in}{0.158083in}}%
\pgfpathlineto{\pgfqpoint{0.884012in}{0.121721in}}%
\pgfpathlineto{\pgfqpoint{0.916667in}{0.138889in}}%
\pgfpathlineto{\pgfqpoint{0.949321in}{0.121721in}}%
\pgfpathlineto{\pgfqpoint{0.943085in}{0.158083in}}%
\pgfpathlineto{\pgfqpoint{0.969503in}{0.183834in}}%
\pgfpathlineto{\pgfqpoint{0.932994in}{0.189139in}}%
\pgfpathlineto{\pgfqpoint{0.916667in}{0.222222in}}%
\pgfpathmoveto{\pgfqpoint{0.000000in}{0.388889in}}%
\pgfpathlineto{\pgfqpoint{-0.016327in}{0.355806in}}%
\pgfpathlineto{\pgfqpoint{-0.052836in}{0.350501in}}%
\pgfpathlineto{\pgfqpoint{-0.026418in}{0.324750in}}%
\pgfpathlineto{\pgfqpoint{-0.032655in}{0.288388in}}%
\pgfpathlineto{\pgfqpoint{-0.000000in}{0.305556in}}%
\pgfpathlineto{\pgfqpoint{0.032655in}{0.288388in}}%
\pgfpathlineto{\pgfqpoint{0.026418in}{0.324750in}}%
\pgfpathlineto{\pgfqpoint{0.052836in}{0.350501in}}%
\pgfpathlineto{\pgfqpoint{0.016327in}{0.355806in}}%
\pgfpathlineto{\pgfqpoint{0.000000in}{0.388889in}}%
\pgfpathmoveto{\pgfqpoint{0.166667in}{0.388889in}}%
\pgfpathlineto{\pgfqpoint{0.150339in}{0.355806in}}%
\pgfpathlineto{\pgfqpoint{0.113830in}{0.350501in}}%
\pgfpathlineto{\pgfqpoint{0.140248in}{0.324750in}}%
\pgfpathlineto{\pgfqpoint{0.134012in}{0.288388in}}%
\pgfpathlineto{\pgfqpoint{0.166667in}{0.305556in}}%
\pgfpathlineto{\pgfqpoint{0.199321in}{0.288388in}}%
\pgfpathlineto{\pgfqpoint{0.193085in}{0.324750in}}%
\pgfpathlineto{\pgfqpoint{0.219503in}{0.350501in}}%
\pgfpathlineto{\pgfqpoint{0.182994in}{0.355806in}}%
\pgfpathlineto{\pgfqpoint{0.166667in}{0.388889in}}%
\pgfpathmoveto{\pgfqpoint{0.333333in}{0.388889in}}%
\pgfpathlineto{\pgfqpoint{0.317006in}{0.355806in}}%
\pgfpathlineto{\pgfqpoint{0.280497in}{0.350501in}}%
\pgfpathlineto{\pgfqpoint{0.306915in}{0.324750in}}%
\pgfpathlineto{\pgfqpoint{0.300679in}{0.288388in}}%
\pgfpathlineto{\pgfqpoint{0.333333in}{0.305556in}}%
\pgfpathlineto{\pgfqpoint{0.365988in}{0.288388in}}%
\pgfpathlineto{\pgfqpoint{0.359752in}{0.324750in}}%
\pgfpathlineto{\pgfqpoint{0.386170in}{0.350501in}}%
\pgfpathlineto{\pgfqpoint{0.349661in}{0.355806in}}%
\pgfpathlineto{\pgfqpoint{0.333333in}{0.388889in}}%
\pgfpathmoveto{\pgfqpoint{0.500000in}{0.388889in}}%
\pgfpathlineto{\pgfqpoint{0.483673in}{0.355806in}}%
\pgfpathlineto{\pgfqpoint{0.447164in}{0.350501in}}%
\pgfpathlineto{\pgfqpoint{0.473582in}{0.324750in}}%
\pgfpathlineto{\pgfqpoint{0.467345in}{0.288388in}}%
\pgfpathlineto{\pgfqpoint{0.500000in}{0.305556in}}%
\pgfpathlineto{\pgfqpoint{0.532655in}{0.288388in}}%
\pgfpathlineto{\pgfqpoint{0.526418in}{0.324750in}}%
\pgfpathlineto{\pgfqpoint{0.552836in}{0.350501in}}%
\pgfpathlineto{\pgfqpoint{0.516327in}{0.355806in}}%
\pgfpathlineto{\pgfqpoint{0.500000in}{0.388889in}}%
\pgfpathmoveto{\pgfqpoint{0.666667in}{0.388889in}}%
\pgfpathlineto{\pgfqpoint{0.650339in}{0.355806in}}%
\pgfpathlineto{\pgfqpoint{0.613830in}{0.350501in}}%
\pgfpathlineto{\pgfqpoint{0.640248in}{0.324750in}}%
\pgfpathlineto{\pgfqpoint{0.634012in}{0.288388in}}%
\pgfpathlineto{\pgfqpoint{0.666667in}{0.305556in}}%
\pgfpathlineto{\pgfqpoint{0.699321in}{0.288388in}}%
\pgfpathlineto{\pgfqpoint{0.693085in}{0.324750in}}%
\pgfpathlineto{\pgfqpoint{0.719503in}{0.350501in}}%
\pgfpathlineto{\pgfqpoint{0.682994in}{0.355806in}}%
\pgfpathlineto{\pgfqpoint{0.666667in}{0.388889in}}%
\pgfpathmoveto{\pgfqpoint{0.833333in}{0.388889in}}%
\pgfpathlineto{\pgfqpoint{0.817006in}{0.355806in}}%
\pgfpathlineto{\pgfqpoint{0.780497in}{0.350501in}}%
\pgfpathlineto{\pgfqpoint{0.806915in}{0.324750in}}%
\pgfpathlineto{\pgfqpoint{0.800679in}{0.288388in}}%
\pgfpathlineto{\pgfqpoint{0.833333in}{0.305556in}}%
\pgfpathlineto{\pgfqpoint{0.865988in}{0.288388in}}%
\pgfpathlineto{\pgfqpoint{0.859752in}{0.324750in}}%
\pgfpathlineto{\pgfqpoint{0.886170in}{0.350501in}}%
\pgfpathlineto{\pgfqpoint{0.849661in}{0.355806in}}%
\pgfpathlineto{\pgfqpoint{0.833333in}{0.388889in}}%
\pgfpathmoveto{\pgfqpoint{1.000000in}{0.388889in}}%
\pgfpathlineto{\pgfqpoint{0.983673in}{0.355806in}}%
\pgfpathlineto{\pgfqpoint{0.947164in}{0.350501in}}%
\pgfpathlineto{\pgfqpoint{0.973582in}{0.324750in}}%
\pgfpathlineto{\pgfqpoint{0.967345in}{0.288388in}}%
\pgfpathlineto{\pgfqpoint{1.000000in}{0.305556in}}%
\pgfpathlineto{\pgfqpoint{1.032655in}{0.288388in}}%
\pgfpathlineto{\pgfqpoint{1.026418in}{0.324750in}}%
\pgfpathlineto{\pgfqpoint{1.052836in}{0.350501in}}%
\pgfpathlineto{\pgfqpoint{1.016327in}{0.355806in}}%
\pgfpathlineto{\pgfqpoint{1.000000in}{0.388889in}}%
\pgfpathmoveto{\pgfqpoint{0.083333in}{0.555556in}}%
\pgfpathlineto{\pgfqpoint{0.067006in}{0.522473in}}%
\pgfpathlineto{\pgfqpoint{0.030497in}{0.517168in}}%
\pgfpathlineto{\pgfqpoint{0.056915in}{0.491416in}}%
\pgfpathlineto{\pgfqpoint{0.050679in}{0.455055in}}%
\pgfpathlineto{\pgfqpoint{0.083333in}{0.472222in}}%
\pgfpathlineto{\pgfqpoint{0.115988in}{0.455055in}}%
\pgfpathlineto{\pgfqpoint{0.109752in}{0.491416in}}%
\pgfpathlineto{\pgfqpoint{0.136170in}{0.517168in}}%
\pgfpathlineto{\pgfqpoint{0.099661in}{0.522473in}}%
\pgfpathlineto{\pgfqpoint{0.083333in}{0.555556in}}%
\pgfpathmoveto{\pgfqpoint{0.250000in}{0.555556in}}%
\pgfpathlineto{\pgfqpoint{0.233673in}{0.522473in}}%
\pgfpathlineto{\pgfqpoint{0.197164in}{0.517168in}}%
\pgfpathlineto{\pgfqpoint{0.223582in}{0.491416in}}%
\pgfpathlineto{\pgfqpoint{0.217345in}{0.455055in}}%
\pgfpathlineto{\pgfqpoint{0.250000in}{0.472222in}}%
\pgfpathlineto{\pgfqpoint{0.282655in}{0.455055in}}%
\pgfpathlineto{\pgfqpoint{0.276418in}{0.491416in}}%
\pgfpathlineto{\pgfqpoint{0.302836in}{0.517168in}}%
\pgfpathlineto{\pgfqpoint{0.266327in}{0.522473in}}%
\pgfpathlineto{\pgfqpoint{0.250000in}{0.555556in}}%
\pgfpathmoveto{\pgfqpoint{0.416667in}{0.555556in}}%
\pgfpathlineto{\pgfqpoint{0.400339in}{0.522473in}}%
\pgfpathlineto{\pgfqpoint{0.363830in}{0.517168in}}%
\pgfpathlineto{\pgfqpoint{0.390248in}{0.491416in}}%
\pgfpathlineto{\pgfqpoint{0.384012in}{0.455055in}}%
\pgfpathlineto{\pgfqpoint{0.416667in}{0.472222in}}%
\pgfpathlineto{\pgfqpoint{0.449321in}{0.455055in}}%
\pgfpathlineto{\pgfqpoint{0.443085in}{0.491416in}}%
\pgfpathlineto{\pgfqpoint{0.469503in}{0.517168in}}%
\pgfpathlineto{\pgfqpoint{0.432994in}{0.522473in}}%
\pgfpathlineto{\pgfqpoint{0.416667in}{0.555556in}}%
\pgfpathmoveto{\pgfqpoint{0.583333in}{0.555556in}}%
\pgfpathlineto{\pgfqpoint{0.567006in}{0.522473in}}%
\pgfpathlineto{\pgfqpoint{0.530497in}{0.517168in}}%
\pgfpathlineto{\pgfqpoint{0.556915in}{0.491416in}}%
\pgfpathlineto{\pgfqpoint{0.550679in}{0.455055in}}%
\pgfpathlineto{\pgfqpoint{0.583333in}{0.472222in}}%
\pgfpathlineto{\pgfqpoint{0.615988in}{0.455055in}}%
\pgfpathlineto{\pgfqpoint{0.609752in}{0.491416in}}%
\pgfpathlineto{\pgfqpoint{0.636170in}{0.517168in}}%
\pgfpathlineto{\pgfqpoint{0.599661in}{0.522473in}}%
\pgfpathlineto{\pgfqpoint{0.583333in}{0.555556in}}%
\pgfpathmoveto{\pgfqpoint{0.750000in}{0.555556in}}%
\pgfpathlineto{\pgfqpoint{0.733673in}{0.522473in}}%
\pgfpathlineto{\pgfqpoint{0.697164in}{0.517168in}}%
\pgfpathlineto{\pgfqpoint{0.723582in}{0.491416in}}%
\pgfpathlineto{\pgfqpoint{0.717345in}{0.455055in}}%
\pgfpathlineto{\pgfqpoint{0.750000in}{0.472222in}}%
\pgfpathlineto{\pgfqpoint{0.782655in}{0.455055in}}%
\pgfpathlineto{\pgfqpoint{0.776418in}{0.491416in}}%
\pgfpathlineto{\pgfqpoint{0.802836in}{0.517168in}}%
\pgfpathlineto{\pgfqpoint{0.766327in}{0.522473in}}%
\pgfpathlineto{\pgfqpoint{0.750000in}{0.555556in}}%
\pgfpathmoveto{\pgfqpoint{0.916667in}{0.555556in}}%
\pgfpathlineto{\pgfqpoint{0.900339in}{0.522473in}}%
\pgfpathlineto{\pgfqpoint{0.863830in}{0.517168in}}%
\pgfpathlineto{\pgfqpoint{0.890248in}{0.491416in}}%
\pgfpathlineto{\pgfqpoint{0.884012in}{0.455055in}}%
\pgfpathlineto{\pgfqpoint{0.916667in}{0.472222in}}%
\pgfpathlineto{\pgfqpoint{0.949321in}{0.455055in}}%
\pgfpathlineto{\pgfqpoint{0.943085in}{0.491416in}}%
\pgfpathlineto{\pgfqpoint{0.969503in}{0.517168in}}%
\pgfpathlineto{\pgfqpoint{0.932994in}{0.522473in}}%
\pgfpathlineto{\pgfqpoint{0.916667in}{0.555556in}}%
\pgfpathmoveto{\pgfqpoint{0.000000in}{0.722222in}}%
\pgfpathlineto{\pgfqpoint{-0.016327in}{0.689139in}}%
\pgfpathlineto{\pgfqpoint{-0.052836in}{0.683834in}}%
\pgfpathlineto{\pgfqpoint{-0.026418in}{0.658083in}}%
\pgfpathlineto{\pgfqpoint{-0.032655in}{0.621721in}}%
\pgfpathlineto{\pgfqpoint{-0.000000in}{0.638889in}}%
\pgfpathlineto{\pgfqpoint{0.032655in}{0.621721in}}%
\pgfpathlineto{\pgfqpoint{0.026418in}{0.658083in}}%
\pgfpathlineto{\pgfqpoint{0.052836in}{0.683834in}}%
\pgfpathlineto{\pgfqpoint{0.016327in}{0.689139in}}%
\pgfpathlineto{\pgfqpoint{0.000000in}{0.722222in}}%
\pgfpathmoveto{\pgfqpoint{0.166667in}{0.722222in}}%
\pgfpathlineto{\pgfqpoint{0.150339in}{0.689139in}}%
\pgfpathlineto{\pgfqpoint{0.113830in}{0.683834in}}%
\pgfpathlineto{\pgfqpoint{0.140248in}{0.658083in}}%
\pgfpathlineto{\pgfqpoint{0.134012in}{0.621721in}}%
\pgfpathlineto{\pgfqpoint{0.166667in}{0.638889in}}%
\pgfpathlineto{\pgfqpoint{0.199321in}{0.621721in}}%
\pgfpathlineto{\pgfqpoint{0.193085in}{0.658083in}}%
\pgfpathlineto{\pgfqpoint{0.219503in}{0.683834in}}%
\pgfpathlineto{\pgfqpoint{0.182994in}{0.689139in}}%
\pgfpathlineto{\pgfqpoint{0.166667in}{0.722222in}}%
\pgfpathmoveto{\pgfqpoint{0.333333in}{0.722222in}}%
\pgfpathlineto{\pgfqpoint{0.317006in}{0.689139in}}%
\pgfpathlineto{\pgfqpoint{0.280497in}{0.683834in}}%
\pgfpathlineto{\pgfqpoint{0.306915in}{0.658083in}}%
\pgfpathlineto{\pgfqpoint{0.300679in}{0.621721in}}%
\pgfpathlineto{\pgfqpoint{0.333333in}{0.638889in}}%
\pgfpathlineto{\pgfqpoint{0.365988in}{0.621721in}}%
\pgfpathlineto{\pgfqpoint{0.359752in}{0.658083in}}%
\pgfpathlineto{\pgfqpoint{0.386170in}{0.683834in}}%
\pgfpathlineto{\pgfqpoint{0.349661in}{0.689139in}}%
\pgfpathlineto{\pgfqpoint{0.333333in}{0.722222in}}%
\pgfpathmoveto{\pgfqpoint{0.500000in}{0.722222in}}%
\pgfpathlineto{\pgfqpoint{0.483673in}{0.689139in}}%
\pgfpathlineto{\pgfqpoint{0.447164in}{0.683834in}}%
\pgfpathlineto{\pgfqpoint{0.473582in}{0.658083in}}%
\pgfpathlineto{\pgfqpoint{0.467345in}{0.621721in}}%
\pgfpathlineto{\pgfqpoint{0.500000in}{0.638889in}}%
\pgfpathlineto{\pgfqpoint{0.532655in}{0.621721in}}%
\pgfpathlineto{\pgfqpoint{0.526418in}{0.658083in}}%
\pgfpathlineto{\pgfqpoint{0.552836in}{0.683834in}}%
\pgfpathlineto{\pgfqpoint{0.516327in}{0.689139in}}%
\pgfpathlineto{\pgfqpoint{0.500000in}{0.722222in}}%
\pgfpathmoveto{\pgfqpoint{0.666667in}{0.722222in}}%
\pgfpathlineto{\pgfqpoint{0.650339in}{0.689139in}}%
\pgfpathlineto{\pgfqpoint{0.613830in}{0.683834in}}%
\pgfpathlineto{\pgfqpoint{0.640248in}{0.658083in}}%
\pgfpathlineto{\pgfqpoint{0.634012in}{0.621721in}}%
\pgfpathlineto{\pgfqpoint{0.666667in}{0.638889in}}%
\pgfpathlineto{\pgfqpoint{0.699321in}{0.621721in}}%
\pgfpathlineto{\pgfqpoint{0.693085in}{0.658083in}}%
\pgfpathlineto{\pgfqpoint{0.719503in}{0.683834in}}%
\pgfpathlineto{\pgfqpoint{0.682994in}{0.689139in}}%
\pgfpathlineto{\pgfqpoint{0.666667in}{0.722222in}}%
\pgfpathmoveto{\pgfqpoint{0.833333in}{0.722222in}}%
\pgfpathlineto{\pgfqpoint{0.817006in}{0.689139in}}%
\pgfpathlineto{\pgfqpoint{0.780497in}{0.683834in}}%
\pgfpathlineto{\pgfqpoint{0.806915in}{0.658083in}}%
\pgfpathlineto{\pgfqpoint{0.800679in}{0.621721in}}%
\pgfpathlineto{\pgfqpoint{0.833333in}{0.638889in}}%
\pgfpathlineto{\pgfqpoint{0.865988in}{0.621721in}}%
\pgfpathlineto{\pgfqpoint{0.859752in}{0.658083in}}%
\pgfpathlineto{\pgfqpoint{0.886170in}{0.683834in}}%
\pgfpathlineto{\pgfqpoint{0.849661in}{0.689139in}}%
\pgfpathlineto{\pgfqpoint{0.833333in}{0.722222in}}%
\pgfpathmoveto{\pgfqpoint{1.000000in}{0.722222in}}%
\pgfpathlineto{\pgfqpoint{0.983673in}{0.689139in}}%
\pgfpathlineto{\pgfqpoint{0.947164in}{0.683834in}}%
\pgfpathlineto{\pgfqpoint{0.973582in}{0.658083in}}%
\pgfpathlineto{\pgfqpoint{0.967345in}{0.621721in}}%
\pgfpathlineto{\pgfqpoint{1.000000in}{0.638889in}}%
\pgfpathlineto{\pgfqpoint{1.032655in}{0.621721in}}%
\pgfpathlineto{\pgfqpoint{1.026418in}{0.658083in}}%
\pgfpathlineto{\pgfqpoint{1.052836in}{0.683834in}}%
\pgfpathlineto{\pgfqpoint{1.016327in}{0.689139in}}%
\pgfpathlineto{\pgfqpoint{1.000000in}{0.722222in}}%
\pgfpathmoveto{\pgfqpoint{0.083333in}{0.888889in}}%
\pgfpathlineto{\pgfqpoint{0.067006in}{0.855806in}}%
\pgfpathlineto{\pgfqpoint{0.030497in}{0.850501in}}%
\pgfpathlineto{\pgfqpoint{0.056915in}{0.824750in}}%
\pgfpathlineto{\pgfqpoint{0.050679in}{0.788388in}}%
\pgfpathlineto{\pgfqpoint{0.083333in}{0.805556in}}%
\pgfpathlineto{\pgfqpoint{0.115988in}{0.788388in}}%
\pgfpathlineto{\pgfqpoint{0.109752in}{0.824750in}}%
\pgfpathlineto{\pgfqpoint{0.136170in}{0.850501in}}%
\pgfpathlineto{\pgfqpoint{0.099661in}{0.855806in}}%
\pgfpathlineto{\pgfqpoint{0.083333in}{0.888889in}}%
\pgfpathmoveto{\pgfqpoint{0.250000in}{0.888889in}}%
\pgfpathlineto{\pgfqpoint{0.233673in}{0.855806in}}%
\pgfpathlineto{\pgfqpoint{0.197164in}{0.850501in}}%
\pgfpathlineto{\pgfqpoint{0.223582in}{0.824750in}}%
\pgfpathlineto{\pgfqpoint{0.217345in}{0.788388in}}%
\pgfpathlineto{\pgfqpoint{0.250000in}{0.805556in}}%
\pgfpathlineto{\pgfqpoint{0.282655in}{0.788388in}}%
\pgfpathlineto{\pgfqpoint{0.276418in}{0.824750in}}%
\pgfpathlineto{\pgfqpoint{0.302836in}{0.850501in}}%
\pgfpathlineto{\pgfqpoint{0.266327in}{0.855806in}}%
\pgfpathlineto{\pgfqpoint{0.250000in}{0.888889in}}%
\pgfpathmoveto{\pgfqpoint{0.416667in}{0.888889in}}%
\pgfpathlineto{\pgfqpoint{0.400339in}{0.855806in}}%
\pgfpathlineto{\pgfqpoint{0.363830in}{0.850501in}}%
\pgfpathlineto{\pgfqpoint{0.390248in}{0.824750in}}%
\pgfpathlineto{\pgfqpoint{0.384012in}{0.788388in}}%
\pgfpathlineto{\pgfqpoint{0.416667in}{0.805556in}}%
\pgfpathlineto{\pgfqpoint{0.449321in}{0.788388in}}%
\pgfpathlineto{\pgfqpoint{0.443085in}{0.824750in}}%
\pgfpathlineto{\pgfqpoint{0.469503in}{0.850501in}}%
\pgfpathlineto{\pgfqpoint{0.432994in}{0.855806in}}%
\pgfpathlineto{\pgfqpoint{0.416667in}{0.888889in}}%
\pgfpathmoveto{\pgfqpoint{0.583333in}{0.888889in}}%
\pgfpathlineto{\pgfqpoint{0.567006in}{0.855806in}}%
\pgfpathlineto{\pgfqpoint{0.530497in}{0.850501in}}%
\pgfpathlineto{\pgfqpoint{0.556915in}{0.824750in}}%
\pgfpathlineto{\pgfqpoint{0.550679in}{0.788388in}}%
\pgfpathlineto{\pgfqpoint{0.583333in}{0.805556in}}%
\pgfpathlineto{\pgfqpoint{0.615988in}{0.788388in}}%
\pgfpathlineto{\pgfqpoint{0.609752in}{0.824750in}}%
\pgfpathlineto{\pgfqpoint{0.636170in}{0.850501in}}%
\pgfpathlineto{\pgfqpoint{0.599661in}{0.855806in}}%
\pgfpathlineto{\pgfqpoint{0.583333in}{0.888889in}}%
\pgfpathmoveto{\pgfqpoint{0.750000in}{0.888889in}}%
\pgfpathlineto{\pgfqpoint{0.733673in}{0.855806in}}%
\pgfpathlineto{\pgfqpoint{0.697164in}{0.850501in}}%
\pgfpathlineto{\pgfqpoint{0.723582in}{0.824750in}}%
\pgfpathlineto{\pgfqpoint{0.717345in}{0.788388in}}%
\pgfpathlineto{\pgfqpoint{0.750000in}{0.805556in}}%
\pgfpathlineto{\pgfqpoint{0.782655in}{0.788388in}}%
\pgfpathlineto{\pgfqpoint{0.776418in}{0.824750in}}%
\pgfpathlineto{\pgfqpoint{0.802836in}{0.850501in}}%
\pgfpathlineto{\pgfqpoint{0.766327in}{0.855806in}}%
\pgfpathlineto{\pgfqpoint{0.750000in}{0.888889in}}%
\pgfpathmoveto{\pgfqpoint{0.916667in}{0.888889in}}%
\pgfpathlineto{\pgfqpoint{0.900339in}{0.855806in}}%
\pgfpathlineto{\pgfqpoint{0.863830in}{0.850501in}}%
\pgfpathlineto{\pgfqpoint{0.890248in}{0.824750in}}%
\pgfpathlineto{\pgfqpoint{0.884012in}{0.788388in}}%
\pgfpathlineto{\pgfqpoint{0.916667in}{0.805556in}}%
\pgfpathlineto{\pgfqpoint{0.949321in}{0.788388in}}%
\pgfpathlineto{\pgfqpoint{0.943085in}{0.824750in}}%
\pgfpathlineto{\pgfqpoint{0.969503in}{0.850501in}}%
\pgfpathlineto{\pgfqpoint{0.932994in}{0.855806in}}%
\pgfpathlineto{\pgfqpoint{0.916667in}{0.888889in}}%
\pgfpathmoveto{\pgfqpoint{0.000000in}{1.055556in}}%
\pgfpathlineto{\pgfqpoint{-0.016327in}{1.022473in}}%
\pgfpathlineto{\pgfqpoint{-0.052836in}{1.017168in}}%
\pgfpathlineto{\pgfqpoint{-0.026418in}{0.991416in}}%
\pgfpathlineto{\pgfqpoint{-0.032655in}{0.955055in}}%
\pgfpathlineto{\pgfqpoint{-0.000000in}{0.972222in}}%
\pgfpathlineto{\pgfqpoint{0.032655in}{0.955055in}}%
\pgfpathlineto{\pgfqpoint{0.026418in}{0.991416in}}%
\pgfpathlineto{\pgfqpoint{0.052836in}{1.017168in}}%
\pgfpathlineto{\pgfqpoint{0.016327in}{1.022473in}}%
\pgfpathlineto{\pgfqpoint{0.000000in}{1.055556in}}%
\pgfpathmoveto{\pgfqpoint{0.166667in}{1.055556in}}%
\pgfpathlineto{\pgfqpoint{0.150339in}{1.022473in}}%
\pgfpathlineto{\pgfqpoint{0.113830in}{1.017168in}}%
\pgfpathlineto{\pgfqpoint{0.140248in}{0.991416in}}%
\pgfpathlineto{\pgfqpoint{0.134012in}{0.955055in}}%
\pgfpathlineto{\pgfqpoint{0.166667in}{0.972222in}}%
\pgfpathlineto{\pgfqpoint{0.199321in}{0.955055in}}%
\pgfpathlineto{\pgfqpoint{0.193085in}{0.991416in}}%
\pgfpathlineto{\pgfqpoint{0.219503in}{1.017168in}}%
\pgfpathlineto{\pgfqpoint{0.182994in}{1.022473in}}%
\pgfpathlineto{\pgfqpoint{0.166667in}{1.055556in}}%
\pgfpathmoveto{\pgfqpoint{0.333333in}{1.055556in}}%
\pgfpathlineto{\pgfqpoint{0.317006in}{1.022473in}}%
\pgfpathlineto{\pgfqpoint{0.280497in}{1.017168in}}%
\pgfpathlineto{\pgfqpoint{0.306915in}{0.991416in}}%
\pgfpathlineto{\pgfqpoint{0.300679in}{0.955055in}}%
\pgfpathlineto{\pgfqpoint{0.333333in}{0.972222in}}%
\pgfpathlineto{\pgfqpoint{0.365988in}{0.955055in}}%
\pgfpathlineto{\pgfqpoint{0.359752in}{0.991416in}}%
\pgfpathlineto{\pgfqpoint{0.386170in}{1.017168in}}%
\pgfpathlineto{\pgfqpoint{0.349661in}{1.022473in}}%
\pgfpathlineto{\pgfqpoint{0.333333in}{1.055556in}}%
\pgfpathmoveto{\pgfqpoint{0.500000in}{1.055556in}}%
\pgfpathlineto{\pgfqpoint{0.483673in}{1.022473in}}%
\pgfpathlineto{\pgfqpoint{0.447164in}{1.017168in}}%
\pgfpathlineto{\pgfqpoint{0.473582in}{0.991416in}}%
\pgfpathlineto{\pgfqpoint{0.467345in}{0.955055in}}%
\pgfpathlineto{\pgfqpoint{0.500000in}{0.972222in}}%
\pgfpathlineto{\pgfqpoint{0.532655in}{0.955055in}}%
\pgfpathlineto{\pgfqpoint{0.526418in}{0.991416in}}%
\pgfpathlineto{\pgfqpoint{0.552836in}{1.017168in}}%
\pgfpathlineto{\pgfqpoint{0.516327in}{1.022473in}}%
\pgfpathlineto{\pgfqpoint{0.500000in}{1.055556in}}%
\pgfpathmoveto{\pgfqpoint{0.666667in}{1.055556in}}%
\pgfpathlineto{\pgfqpoint{0.650339in}{1.022473in}}%
\pgfpathlineto{\pgfqpoint{0.613830in}{1.017168in}}%
\pgfpathlineto{\pgfqpoint{0.640248in}{0.991416in}}%
\pgfpathlineto{\pgfqpoint{0.634012in}{0.955055in}}%
\pgfpathlineto{\pgfqpoint{0.666667in}{0.972222in}}%
\pgfpathlineto{\pgfqpoint{0.699321in}{0.955055in}}%
\pgfpathlineto{\pgfqpoint{0.693085in}{0.991416in}}%
\pgfpathlineto{\pgfqpoint{0.719503in}{1.017168in}}%
\pgfpathlineto{\pgfqpoint{0.682994in}{1.022473in}}%
\pgfpathlineto{\pgfqpoint{0.666667in}{1.055556in}}%
\pgfpathmoveto{\pgfqpoint{0.833333in}{1.055556in}}%
\pgfpathlineto{\pgfqpoint{0.817006in}{1.022473in}}%
\pgfpathlineto{\pgfqpoint{0.780497in}{1.017168in}}%
\pgfpathlineto{\pgfqpoint{0.806915in}{0.991416in}}%
\pgfpathlineto{\pgfqpoint{0.800679in}{0.955055in}}%
\pgfpathlineto{\pgfqpoint{0.833333in}{0.972222in}}%
\pgfpathlineto{\pgfqpoint{0.865988in}{0.955055in}}%
\pgfpathlineto{\pgfqpoint{0.859752in}{0.991416in}}%
\pgfpathlineto{\pgfqpoint{0.886170in}{1.017168in}}%
\pgfpathlineto{\pgfqpoint{0.849661in}{1.022473in}}%
\pgfpathlineto{\pgfqpoint{0.833333in}{1.055556in}}%
\pgfpathmoveto{\pgfqpoint{1.000000in}{1.055556in}}%
\pgfpathlineto{\pgfqpoint{0.983673in}{1.022473in}}%
\pgfpathlineto{\pgfqpoint{0.947164in}{1.017168in}}%
\pgfpathlineto{\pgfqpoint{0.973582in}{0.991416in}}%
\pgfpathlineto{\pgfqpoint{0.967345in}{0.955055in}}%
\pgfpathlineto{\pgfqpoint{1.000000in}{0.972222in}}%
\pgfpathlineto{\pgfqpoint{1.032655in}{0.955055in}}%
\pgfpathlineto{\pgfqpoint{1.026418in}{0.991416in}}%
\pgfpathlineto{\pgfqpoint{1.052836in}{1.017168in}}%
\pgfpathlineto{\pgfqpoint{1.016327in}{1.022473in}}%
\pgfpathlineto{\pgfqpoint{1.000000in}{1.055556in}}%
\pgfpathlineto{\pgfqpoint{1.000000in}{1.055556in}}%
\pgfusepath{stroke}%
\end{pgfscope}%
}%
\pgfsys@transformshift{4.423315in}{4.688801in}%
\pgfsys@useobject{currentpattern}{}%
\pgfsys@transformshift{1in}{0in}%
\pgfsys@transformshift{-1in}{0in}%
\pgfsys@transformshift{0in}{1in}%
\end{pgfscope}%
\begin{pgfscope}%
\pgfpathrectangle{\pgfqpoint{0.935815in}{0.637495in}}{\pgfqpoint{9.300000in}{9.060000in}}%
\pgfusepath{clip}%
\pgfsetbuttcap%
\pgfsetmiterjoin%
\definecolor{currentfill}{rgb}{1.000000,1.000000,0.000000}%
\pgfsetfillcolor{currentfill}%
\pgfsetfillopacity{0.990000}%
\pgfsetlinewidth{0.000000pt}%
\definecolor{currentstroke}{rgb}{0.000000,0.000000,0.000000}%
\pgfsetstrokecolor{currentstroke}%
\pgfsetstrokeopacity{0.990000}%
\pgfsetdash{}{0pt}%
\pgfpathmoveto{\pgfqpoint{5.973315in}{5.330636in}}%
\pgfpathlineto{\pgfqpoint{6.748315in}{5.330636in}}%
\pgfpathlineto{\pgfqpoint{6.748315in}{6.327614in}}%
\pgfpathlineto{\pgfqpoint{5.973315in}{6.327614in}}%
\pgfpathclose%
\pgfusepath{fill}%
\end{pgfscope}%
\begin{pgfscope}%
\pgfsetbuttcap%
\pgfsetmiterjoin%
\definecolor{currentfill}{rgb}{1.000000,1.000000,0.000000}%
\pgfsetfillcolor{currentfill}%
\pgfsetfillopacity{0.990000}%
\pgfsetlinewidth{0.000000pt}%
\definecolor{currentstroke}{rgb}{0.000000,0.000000,0.000000}%
\pgfsetstrokecolor{currentstroke}%
\pgfsetstrokeopacity{0.990000}%
\pgfsetdash{}{0pt}%
\pgfpathrectangle{\pgfqpoint{0.935815in}{0.637495in}}{\pgfqpoint{9.300000in}{9.060000in}}%
\pgfusepath{clip}%
\pgfpathmoveto{\pgfqpoint{5.973315in}{5.330636in}}%
\pgfpathlineto{\pgfqpoint{6.748315in}{5.330636in}}%
\pgfpathlineto{\pgfqpoint{6.748315in}{6.327614in}}%
\pgfpathlineto{\pgfqpoint{5.973315in}{6.327614in}}%
\pgfpathclose%
\pgfusepath{clip}%
\pgfsys@defobject{currentpattern}{\pgfqpoint{0in}{0in}}{\pgfqpoint{1in}{1in}}{%
\begin{pgfscope}%
\pgfpathrectangle{\pgfqpoint{0in}{0in}}{\pgfqpoint{1in}{1in}}%
\pgfusepath{clip}%
\pgfpathmoveto{\pgfqpoint{0.000000in}{0.055556in}}%
\pgfpathlineto{\pgfqpoint{-0.016327in}{0.022473in}}%
\pgfpathlineto{\pgfqpoint{-0.052836in}{0.017168in}}%
\pgfpathlineto{\pgfqpoint{-0.026418in}{-0.008584in}}%
\pgfpathlineto{\pgfqpoint{-0.032655in}{-0.044945in}}%
\pgfpathlineto{\pgfqpoint{-0.000000in}{-0.027778in}}%
\pgfpathlineto{\pgfqpoint{0.032655in}{-0.044945in}}%
\pgfpathlineto{\pgfqpoint{0.026418in}{-0.008584in}}%
\pgfpathlineto{\pgfqpoint{0.052836in}{0.017168in}}%
\pgfpathlineto{\pgfqpoint{0.016327in}{0.022473in}}%
\pgfpathlineto{\pgfqpoint{0.000000in}{0.055556in}}%
\pgfpathmoveto{\pgfqpoint{0.166667in}{0.055556in}}%
\pgfpathlineto{\pgfqpoint{0.150339in}{0.022473in}}%
\pgfpathlineto{\pgfqpoint{0.113830in}{0.017168in}}%
\pgfpathlineto{\pgfqpoint{0.140248in}{-0.008584in}}%
\pgfpathlineto{\pgfqpoint{0.134012in}{-0.044945in}}%
\pgfpathlineto{\pgfqpoint{0.166667in}{-0.027778in}}%
\pgfpathlineto{\pgfqpoint{0.199321in}{-0.044945in}}%
\pgfpathlineto{\pgfqpoint{0.193085in}{-0.008584in}}%
\pgfpathlineto{\pgfqpoint{0.219503in}{0.017168in}}%
\pgfpathlineto{\pgfqpoint{0.182994in}{0.022473in}}%
\pgfpathlineto{\pgfqpoint{0.166667in}{0.055556in}}%
\pgfpathmoveto{\pgfqpoint{0.333333in}{0.055556in}}%
\pgfpathlineto{\pgfqpoint{0.317006in}{0.022473in}}%
\pgfpathlineto{\pgfqpoint{0.280497in}{0.017168in}}%
\pgfpathlineto{\pgfqpoint{0.306915in}{-0.008584in}}%
\pgfpathlineto{\pgfqpoint{0.300679in}{-0.044945in}}%
\pgfpathlineto{\pgfqpoint{0.333333in}{-0.027778in}}%
\pgfpathlineto{\pgfqpoint{0.365988in}{-0.044945in}}%
\pgfpathlineto{\pgfqpoint{0.359752in}{-0.008584in}}%
\pgfpathlineto{\pgfqpoint{0.386170in}{0.017168in}}%
\pgfpathlineto{\pgfqpoint{0.349661in}{0.022473in}}%
\pgfpathlineto{\pgfqpoint{0.333333in}{0.055556in}}%
\pgfpathmoveto{\pgfqpoint{0.500000in}{0.055556in}}%
\pgfpathlineto{\pgfqpoint{0.483673in}{0.022473in}}%
\pgfpathlineto{\pgfqpoint{0.447164in}{0.017168in}}%
\pgfpathlineto{\pgfqpoint{0.473582in}{-0.008584in}}%
\pgfpathlineto{\pgfqpoint{0.467345in}{-0.044945in}}%
\pgfpathlineto{\pgfqpoint{0.500000in}{-0.027778in}}%
\pgfpathlineto{\pgfqpoint{0.532655in}{-0.044945in}}%
\pgfpathlineto{\pgfqpoint{0.526418in}{-0.008584in}}%
\pgfpathlineto{\pgfqpoint{0.552836in}{0.017168in}}%
\pgfpathlineto{\pgfqpoint{0.516327in}{0.022473in}}%
\pgfpathlineto{\pgfqpoint{0.500000in}{0.055556in}}%
\pgfpathmoveto{\pgfqpoint{0.666667in}{0.055556in}}%
\pgfpathlineto{\pgfqpoint{0.650339in}{0.022473in}}%
\pgfpathlineto{\pgfqpoint{0.613830in}{0.017168in}}%
\pgfpathlineto{\pgfqpoint{0.640248in}{-0.008584in}}%
\pgfpathlineto{\pgfqpoint{0.634012in}{-0.044945in}}%
\pgfpathlineto{\pgfqpoint{0.666667in}{-0.027778in}}%
\pgfpathlineto{\pgfqpoint{0.699321in}{-0.044945in}}%
\pgfpathlineto{\pgfqpoint{0.693085in}{-0.008584in}}%
\pgfpathlineto{\pgfqpoint{0.719503in}{0.017168in}}%
\pgfpathlineto{\pgfqpoint{0.682994in}{0.022473in}}%
\pgfpathlineto{\pgfqpoint{0.666667in}{0.055556in}}%
\pgfpathmoveto{\pgfqpoint{0.833333in}{0.055556in}}%
\pgfpathlineto{\pgfqpoint{0.817006in}{0.022473in}}%
\pgfpathlineto{\pgfqpoint{0.780497in}{0.017168in}}%
\pgfpathlineto{\pgfqpoint{0.806915in}{-0.008584in}}%
\pgfpathlineto{\pgfqpoint{0.800679in}{-0.044945in}}%
\pgfpathlineto{\pgfqpoint{0.833333in}{-0.027778in}}%
\pgfpathlineto{\pgfqpoint{0.865988in}{-0.044945in}}%
\pgfpathlineto{\pgfqpoint{0.859752in}{-0.008584in}}%
\pgfpathlineto{\pgfqpoint{0.886170in}{0.017168in}}%
\pgfpathlineto{\pgfqpoint{0.849661in}{0.022473in}}%
\pgfpathlineto{\pgfqpoint{0.833333in}{0.055556in}}%
\pgfpathmoveto{\pgfqpoint{1.000000in}{0.055556in}}%
\pgfpathlineto{\pgfqpoint{0.983673in}{0.022473in}}%
\pgfpathlineto{\pgfqpoint{0.947164in}{0.017168in}}%
\pgfpathlineto{\pgfqpoint{0.973582in}{-0.008584in}}%
\pgfpathlineto{\pgfqpoint{0.967345in}{-0.044945in}}%
\pgfpathlineto{\pgfqpoint{1.000000in}{-0.027778in}}%
\pgfpathlineto{\pgfqpoint{1.032655in}{-0.044945in}}%
\pgfpathlineto{\pgfqpoint{1.026418in}{-0.008584in}}%
\pgfpathlineto{\pgfqpoint{1.052836in}{0.017168in}}%
\pgfpathlineto{\pgfqpoint{1.016327in}{0.022473in}}%
\pgfpathlineto{\pgfqpoint{1.000000in}{0.055556in}}%
\pgfpathmoveto{\pgfqpoint{0.083333in}{0.222222in}}%
\pgfpathlineto{\pgfqpoint{0.067006in}{0.189139in}}%
\pgfpathlineto{\pgfqpoint{0.030497in}{0.183834in}}%
\pgfpathlineto{\pgfqpoint{0.056915in}{0.158083in}}%
\pgfpathlineto{\pgfqpoint{0.050679in}{0.121721in}}%
\pgfpathlineto{\pgfqpoint{0.083333in}{0.138889in}}%
\pgfpathlineto{\pgfqpoint{0.115988in}{0.121721in}}%
\pgfpathlineto{\pgfqpoint{0.109752in}{0.158083in}}%
\pgfpathlineto{\pgfqpoint{0.136170in}{0.183834in}}%
\pgfpathlineto{\pgfqpoint{0.099661in}{0.189139in}}%
\pgfpathlineto{\pgfqpoint{0.083333in}{0.222222in}}%
\pgfpathmoveto{\pgfqpoint{0.250000in}{0.222222in}}%
\pgfpathlineto{\pgfqpoint{0.233673in}{0.189139in}}%
\pgfpathlineto{\pgfqpoint{0.197164in}{0.183834in}}%
\pgfpathlineto{\pgfqpoint{0.223582in}{0.158083in}}%
\pgfpathlineto{\pgfqpoint{0.217345in}{0.121721in}}%
\pgfpathlineto{\pgfqpoint{0.250000in}{0.138889in}}%
\pgfpathlineto{\pgfqpoint{0.282655in}{0.121721in}}%
\pgfpathlineto{\pgfqpoint{0.276418in}{0.158083in}}%
\pgfpathlineto{\pgfqpoint{0.302836in}{0.183834in}}%
\pgfpathlineto{\pgfqpoint{0.266327in}{0.189139in}}%
\pgfpathlineto{\pgfqpoint{0.250000in}{0.222222in}}%
\pgfpathmoveto{\pgfqpoint{0.416667in}{0.222222in}}%
\pgfpathlineto{\pgfqpoint{0.400339in}{0.189139in}}%
\pgfpathlineto{\pgfqpoint{0.363830in}{0.183834in}}%
\pgfpathlineto{\pgfqpoint{0.390248in}{0.158083in}}%
\pgfpathlineto{\pgfqpoint{0.384012in}{0.121721in}}%
\pgfpathlineto{\pgfqpoint{0.416667in}{0.138889in}}%
\pgfpathlineto{\pgfqpoint{0.449321in}{0.121721in}}%
\pgfpathlineto{\pgfqpoint{0.443085in}{0.158083in}}%
\pgfpathlineto{\pgfqpoint{0.469503in}{0.183834in}}%
\pgfpathlineto{\pgfqpoint{0.432994in}{0.189139in}}%
\pgfpathlineto{\pgfqpoint{0.416667in}{0.222222in}}%
\pgfpathmoveto{\pgfqpoint{0.583333in}{0.222222in}}%
\pgfpathlineto{\pgfqpoint{0.567006in}{0.189139in}}%
\pgfpathlineto{\pgfqpoint{0.530497in}{0.183834in}}%
\pgfpathlineto{\pgfqpoint{0.556915in}{0.158083in}}%
\pgfpathlineto{\pgfqpoint{0.550679in}{0.121721in}}%
\pgfpathlineto{\pgfqpoint{0.583333in}{0.138889in}}%
\pgfpathlineto{\pgfqpoint{0.615988in}{0.121721in}}%
\pgfpathlineto{\pgfqpoint{0.609752in}{0.158083in}}%
\pgfpathlineto{\pgfqpoint{0.636170in}{0.183834in}}%
\pgfpathlineto{\pgfqpoint{0.599661in}{0.189139in}}%
\pgfpathlineto{\pgfqpoint{0.583333in}{0.222222in}}%
\pgfpathmoveto{\pgfqpoint{0.750000in}{0.222222in}}%
\pgfpathlineto{\pgfqpoint{0.733673in}{0.189139in}}%
\pgfpathlineto{\pgfqpoint{0.697164in}{0.183834in}}%
\pgfpathlineto{\pgfqpoint{0.723582in}{0.158083in}}%
\pgfpathlineto{\pgfqpoint{0.717345in}{0.121721in}}%
\pgfpathlineto{\pgfqpoint{0.750000in}{0.138889in}}%
\pgfpathlineto{\pgfqpoint{0.782655in}{0.121721in}}%
\pgfpathlineto{\pgfqpoint{0.776418in}{0.158083in}}%
\pgfpathlineto{\pgfqpoint{0.802836in}{0.183834in}}%
\pgfpathlineto{\pgfqpoint{0.766327in}{0.189139in}}%
\pgfpathlineto{\pgfqpoint{0.750000in}{0.222222in}}%
\pgfpathmoveto{\pgfqpoint{0.916667in}{0.222222in}}%
\pgfpathlineto{\pgfqpoint{0.900339in}{0.189139in}}%
\pgfpathlineto{\pgfqpoint{0.863830in}{0.183834in}}%
\pgfpathlineto{\pgfqpoint{0.890248in}{0.158083in}}%
\pgfpathlineto{\pgfqpoint{0.884012in}{0.121721in}}%
\pgfpathlineto{\pgfqpoint{0.916667in}{0.138889in}}%
\pgfpathlineto{\pgfqpoint{0.949321in}{0.121721in}}%
\pgfpathlineto{\pgfqpoint{0.943085in}{0.158083in}}%
\pgfpathlineto{\pgfqpoint{0.969503in}{0.183834in}}%
\pgfpathlineto{\pgfqpoint{0.932994in}{0.189139in}}%
\pgfpathlineto{\pgfqpoint{0.916667in}{0.222222in}}%
\pgfpathmoveto{\pgfqpoint{0.000000in}{0.388889in}}%
\pgfpathlineto{\pgfqpoint{-0.016327in}{0.355806in}}%
\pgfpathlineto{\pgfqpoint{-0.052836in}{0.350501in}}%
\pgfpathlineto{\pgfqpoint{-0.026418in}{0.324750in}}%
\pgfpathlineto{\pgfqpoint{-0.032655in}{0.288388in}}%
\pgfpathlineto{\pgfqpoint{-0.000000in}{0.305556in}}%
\pgfpathlineto{\pgfqpoint{0.032655in}{0.288388in}}%
\pgfpathlineto{\pgfqpoint{0.026418in}{0.324750in}}%
\pgfpathlineto{\pgfqpoint{0.052836in}{0.350501in}}%
\pgfpathlineto{\pgfqpoint{0.016327in}{0.355806in}}%
\pgfpathlineto{\pgfqpoint{0.000000in}{0.388889in}}%
\pgfpathmoveto{\pgfqpoint{0.166667in}{0.388889in}}%
\pgfpathlineto{\pgfqpoint{0.150339in}{0.355806in}}%
\pgfpathlineto{\pgfqpoint{0.113830in}{0.350501in}}%
\pgfpathlineto{\pgfqpoint{0.140248in}{0.324750in}}%
\pgfpathlineto{\pgfqpoint{0.134012in}{0.288388in}}%
\pgfpathlineto{\pgfqpoint{0.166667in}{0.305556in}}%
\pgfpathlineto{\pgfqpoint{0.199321in}{0.288388in}}%
\pgfpathlineto{\pgfqpoint{0.193085in}{0.324750in}}%
\pgfpathlineto{\pgfqpoint{0.219503in}{0.350501in}}%
\pgfpathlineto{\pgfqpoint{0.182994in}{0.355806in}}%
\pgfpathlineto{\pgfqpoint{0.166667in}{0.388889in}}%
\pgfpathmoveto{\pgfqpoint{0.333333in}{0.388889in}}%
\pgfpathlineto{\pgfqpoint{0.317006in}{0.355806in}}%
\pgfpathlineto{\pgfqpoint{0.280497in}{0.350501in}}%
\pgfpathlineto{\pgfqpoint{0.306915in}{0.324750in}}%
\pgfpathlineto{\pgfqpoint{0.300679in}{0.288388in}}%
\pgfpathlineto{\pgfqpoint{0.333333in}{0.305556in}}%
\pgfpathlineto{\pgfqpoint{0.365988in}{0.288388in}}%
\pgfpathlineto{\pgfqpoint{0.359752in}{0.324750in}}%
\pgfpathlineto{\pgfqpoint{0.386170in}{0.350501in}}%
\pgfpathlineto{\pgfqpoint{0.349661in}{0.355806in}}%
\pgfpathlineto{\pgfqpoint{0.333333in}{0.388889in}}%
\pgfpathmoveto{\pgfqpoint{0.500000in}{0.388889in}}%
\pgfpathlineto{\pgfqpoint{0.483673in}{0.355806in}}%
\pgfpathlineto{\pgfqpoint{0.447164in}{0.350501in}}%
\pgfpathlineto{\pgfqpoint{0.473582in}{0.324750in}}%
\pgfpathlineto{\pgfqpoint{0.467345in}{0.288388in}}%
\pgfpathlineto{\pgfqpoint{0.500000in}{0.305556in}}%
\pgfpathlineto{\pgfqpoint{0.532655in}{0.288388in}}%
\pgfpathlineto{\pgfqpoint{0.526418in}{0.324750in}}%
\pgfpathlineto{\pgfqpoint{0.552836in}{0.350501in}}%
\pgfpathlineto{\pgfqpoint{0.516327in}{0.355806in}}%
\pgfpathlineto{\pgfqpoint{0.500000in}{0.388889in}}%
\pgfpathmoveto{\pgfqpoint{0.666667in}{0.388889in}}%
\pgfpathlineto{\pgfqpoint{0.650339in}{0.355806in}}%
\pgfpathlineto{\pgfqpoint{0.613830in}{0.350501in}}%
\pgfpathlineto{\pgfqpoint{0.640248in}{0.324750in}}%
\pgfpathlineto{\pgfqpoint{0.634012in}{0.288388in}}%
\pgfpathlineto{\pgfqpoint{0.666667in}{0.305556in}}%
\pgfpathlineto{\pgfqpoint{0.699321in}{0.288388in}}%
\pgfpathlineto{\pgfqpoint{0.693085in}{0.324750in}}%
\pgfpathlineto{\pgfqpoint{0.719503in}{0.350501in}}%
\pgfpathlineto{\pgfqpoint{0.682994in}{0.355806in}}%
\pgfpathlineto{\pgfqpoint{0.666667in}{0.388889in}}%
\pgfpathmoveto{\pgfqpoint{0.833333in}{0.388889in}}%
\pgfpathlineto{\pgfqpoint{0.817006in}{0.355806in}}%
\pgfpathlineto{\pgfqpoint{0.780497in}{0.350501in}}%
\pgfpathlineto{\pgfqpoint{0.806915in}{0.324750in}}%
\pgfpathlineto{\pgfqpoint{0.800679in}{0.288388in}}%
\pgfpathlineto{\pgfqpoint{0.833333in}{0.305556in}}%
\pgfpathlineto{\pgfqpoint{0.865988in}{0.288388in}}%
\pgfpathlineto{\pgfqpoint{0.859752in}{0.324750in}}%
\pgfpathlineto{\pgfqpoint{0.886170in}{0.350501in}}%
\pgfpathlineto{\pgfqpoint{0.849661in}{0.355806in}}%
\pgfpathlineto{\pgfqpoint{0.833333in}{0.388889in}}%
\pgfpathmoveto{\pgfqpoint{1.000000in}{0.388889in}}%
\pgfpathlineto{\pgfqpoint{0.983673in}{0.355806in}}%
\pgfpathlineto{\pgfqpoint{0.947164in}{0.350501in}}%
\pgfpathlineto{\pgfqpoint{0.973582in}{0.324750in}}%
\pgfpathlineto{\pgfqpoint{0.967345in}{0.288388in}}%
\pgfpathlineto{\pgfqpoint{1.000000in}{0.305556in}}%
\pgfpathlineto{\pgfqpoint{1.032655in}{0.288388in}}%
\pgfpathlineto{\pgfqpoint{1.026418in}{0.324750in}}%
\pgfpathlineto{\pgfqpoint{1.052836in}{0.350501in}}%
\pgfpathlineto{\pgfqpoint{1.016327in}{0.355806in}}%
\pgfpathlineto{\pgfqpoint{1.000000in}{0.388889in}}%
\pgfpathmoveto{\pgfqpoint{0.083333in}{0.555556in}}%
\pgfpathlineto{\pgfqpoint{0.067006in}{0.522473in}}%
\pgfpathlineto{\pgfqpoint{0.030497in}{0.517168in}}%
\pgfpathlineto{\pgfqpoint{0.056915in}{0.491416in}}%
\pgfpathlineto{\pgfqpoint{0.050679in}{0.455055in}}%
\pgfpathlineto{\pgfqpoint{0.083333in}{0.472222in}}%
\pgfpathlineto{\pgfqpoint{0.115988in}{0.455055in}}%
\pgfpathlineto{\pgfqpoint{0.109752in}{0.491416in}}%
\pgfpathlineto{\pgfqpoint{0.136170in}{0.517168in}}%
\pgfpathlineto{\pgfqpoint{0.099661in}{0.522473in}}%
\pgfpathlineto{\pgfqpoint{0.083333in}{0.555556in}}%
\pgfpathmoveto{\pgfqpoint{0.250000in}{0.555556in}}%
\pgfpathlineto{\pgfqpoint{0.233673in}{0.522473in}}%
\pgfpathlineto{\pgfqpoint{0.197164in}{0.517168in}}%
\pgfpathlineto{\pgfqpoint{0.223582in}{0.491416in}}%
\pgfpathlineto{\pgfqpoint{0.217345in}{0.455055in}}%
\pgfpathlineto{\pgfqpoint{0.250000in}{0.472222in}}%
\pgfpathlineto{\pgfqpoint{0.282655in}{0.455055in}}%
\pgfpathlineto{\pgfqpoint{0.276418in}{0.491416in}}%
\pgfpathlineto{\pgfqpoint{0.302836in}{0.517168in}}%
\pgfpathlineto{\pgfqpoint{0.266327in}{0.522473in}}%
\pgfpathlineto{\pgfqpoint{0.250000in}{0.555556in}}%
\pgfpathmoveto{\pgfqpoint{0.416667in}{0.555556in}}%
\pgfpathlineto{\pgfqpoint{0.400339in}{0.522473in}}%
\pgfpathlineto{\pgfqpoint{0.363830in}{0.517168in}}%
\pgfpathlineto{\pgfqpoint{0.390248in}{0.491416in}}%
\pgfpathlineto{\pgfqpoint{0.384012in}{0.455055in}}%
\pgfpathlineto{\pgfqpoint{0.416667in}{0.472222in}}%
\pgfpathlineto{\pgfqpoint{0.449321in}{0.455055in}}%
\pgfpathlineto{\pgfqpoint{0.443085in}{0.491416in}}%
\pgfpathlineto{\pgfqpoint{0.469503in}{0.517168in}}%
\pgfpathlineto{\pgfqpoint{0.432994in}{0.522473in}}%
\pgfpathlineto{\pgfqpoint{0.416667in}{0.555556in}}%
\pgfpathmoveto{\pgfqpoint{0.583333in}{0.555556in}}%
\pgfpathlineto{\pgfqpoint{0.567006in}{0.522473in}}%
\pgfpathlineto{\pgfqpoint{0.530497in}{0.517168in}}%
\pgfpathlineto{\pgfqpoint{0.556915in}{0.491416in}}%
\pgfpathlineto{\pgfqpoint{0.550679in}{0.455055in}}%
\pgfpathlineto{\pgfqpoint{0.583333in}{0.472222in}}%
\pgfpathlineto{\pgfqpoint{0.615988in}{0.455055in}}%
\pgfpathlineto{\pgfqpoint{0.609752in}{0.491416in}}%
\pgfpathlineto{\pgfqpoint{0.636170in}{0.517168in}}%
\pgfpathlineto{\pgfqpoint{0.599661in}{0.522473in}}%
\pgfpathlineto{\pgfqpoint{0.583333in}{0.555556in}}%
\pgfpathmoveto{\pgfqpoint{0.750000in}{0.555556in}}%
\pgfpathlineto{\pgfqpoint{0.733673in}{0.522473in}}%
\pgfpathlineto{\pgfqpoint{0.697164in}{0.517168in}}%
\pgfpathlineto{\pgfqpoint{0.723582in}{0.491416in}}%
\pgfpathlineto{\pgfqpoint{0.717345in}{0.455055in}}%
\pgfpathlineto{\pgfqpoint{0.750000in}{0.472222in}}%
\pgfpathlineto{\pgfqpoint{0.782655in}{0.455055in}}%
\pgfpathlineto{\pgfqpoint{0.776418in}{0.491416in}}%
\pgfpathlineto{\pgfqpoint{0.802836in}{0.517168in}}%
\pgfpathlineto{\pgfqpoint{0.766327in}{0.522473in}}%
\pgfpathlineto{\pgfqpoint{0.750000in}{0.555556in}}%
\pgfpathmoveto{\pgfqpoint{0.916667in}{0.555556in}}%
\pgfpathlineto{\pgfqpoint{0.900339in}{0.522473in}}%
\pgfpathlineto{\pgfqpoint{0.863830in}{0.517168in}}%
\pgfpathlineto{\pgfqpoint{0.890248in}{0.491416in}}%
\pgfpathlineto{\pgfqpoint{0.884012in}{0.455055in}}%
\pgfpathlineto{\pgfqpoint{0.916667in}{0.472222in}}%
\pgfpathlineto{\pgfqpoint{0.949321in}{0.455055in}}%
\pgfpathlineto{\pgfqpoint{0.943085in}{0.491416in}}%
\pgfpathlineto{\pgfqpoint{0.969503in}{0.517168in}}%
\pgfpathlineto{\pgfqpoint{0.932994in}{0.522473in}}%
\pgfpathlineto{\pgfqpoint{0.916667in}{0.555556in}}%
\pgfpathmoveto{\pgfqpoint{0.000000in}{0.722222in}}%
\pgfpathlineto{\pgfqpoint{-0.016327in}{0.689139in}}%
\pgfpathlineto{\pgfqpoint{-0.052836in}{0.683834in}}%
\pgfpathlineto{\pgfqpoint{-0.026418in}{0.658083in}}%
\pgfpathlineto{\pgfqpoint{-0.032655in}{0.621721in}}%
\pgfpathlineto{\pgfqpoint{-0.000000in}{0.638889in}}%
\pgfpathlineto{\pgfqpoint{0.032655in}{0.621721in}}%
\pgfpathlineto{\pgfqpoint{0.026418in}{0.658083in}}%
\pgfpathlineto{\pgfqpoint{0.052836in}{0.683834in}}%
\pgfpathlineto{\pgfqpoint{0.016327in}{0.689139in}}%
\pgfpathlineto{\pgfqpoint{0.000000in}{0.722222in}}%
\pgfpathmoveto{\pgfqpoint{0.166667in}{0.722222in}}%
\pgfpathlineto{\pgfqpoint{0.150339in}{0.689139in}}%
\pgfpathlineto{\pgfqpoint{0.113830in}{0.683834in}}%
\pgfpathlineto{\pgfqpoint{0.140248in}{0.658083in}}%
\pgfpathlineto{\pgfqpoint{0.134012in}{0.621721in}}%
\pgfpathlineto{\pgfqpoint{0.166667in}{0.638889in}}%
\pgfpathlineto{\pgfqpoint{0.199321in}{0.621721in}}%
\pgfpathlineto{\pgfqpoint{0.193085in}{0.658083in}}%
\pgfpathlineto{\pgfqpoint{0.219503in}{0.683834in}}%
\pgfpathlineto{\pgfqpoint{0.182994in}{0.689139in}}%
\pgfpathlineto{\pgfqpoint{0.166667in}{0.722222in}}%
\pgfpathmoveto{\pgfqpoint{0.333333in}{0.722222in}}%
\pgfpathlineto{\pgfqpoint{0.317006in}{0.689139in}}%
\pgfpathlineto{\pgfqpoint{0.280497in}{0.683834in}}%
\pgfpathlineto{\pgfqpoint{0.306915in}{0.658083in}}%
\pgfpathlineto{\pgfqpoint{0.300679in}{0.621721in}}%
\pgfpathlineto{\pgfqpoint{0.333333in}{0.638889in}}%
\pgfpathlineto{\pgfqpoint{0.365988in}{0.621721in}}%
\pgfpathlineto{\pgfqpoint{0.359752in}{0.658083in}}%
\pgfpathlineto{\pgfqpoint{0.386170in}{0.683834in}}%
\pgfpathlineto{\pgfqpoint{0.349661in}{0.689139in}}%
\pgfpathlineto{\pgfqpoint{0.333333in}{0.722222in}}%
\pgfpathmoveto{\pgfqpoint{0.500000in}{0.722222in}}%
\pgfpathlineto{\pgfqpoint{0.483673in}{0.689139in}}%
\pgfpathlineto{\pgfqpoint{0.447164in}{0.683834in}}%
\pgfpathlineto{\pgfqpoint{0.473582in}{0.658083in}}%
\pgfpathlineto{\pgfqpoint{0.467345in}{0.621721in}}%
\pgfpathlineto{\pgfqpoint{0.500000in}{0.638889in}}%
\pgfpathlineto{\pgfqpoint{0.532655in}{0.621721in}}%
\pgfpathlineto{\pgfqpoint{0.526418in}{0.658083in}}%
\pgfpathlineto{\pgfqpoint{0.552836in}{0.683834in}}%
\pgfpathlineto{\pgfqpoint{0.516327in}{0.689139in}}%
\pgfpathlineto{\pgfqpoint{0.500000in}{0.722222in}}%
\pgfpathmoveto{\pgfqpoint{0.666667in}{0.722222in}}%
\pgfpathlineto{\pgfqpoint{0.650339in}{0.689139in}}%
\pgfpathlineto{\pgfqpoint{0.613830in}{0.683834in}}%
\pgfpathlineto{\pgfqpoint{0.640248in}{0.658083in}}%
\pgfpathlineto{\pgfqpoint{0.634012in}{0.621721in}}%
\pgfpathlineto{\pgfqpoint{0.666667in}{0.638889in}}%
\pgfpathlineto{\pgfqpoint{0.699321in}{0.621721in}}%
\pgfpathlineto{\pgfqpoint{0.693085in}{0.658083in}}%
\pgfpathlineto{\pgfqpoint{0.719503in}{0.683834in}}%
\pgfpathlineto{\pgfqpoint{0.682994in}{0.689139in}}%
\pgfpathlineto{\pgfqpoint{0.666667in}{0.722222in}}%
\pgfpathmoveto{\pgfqpoint{0.833333in}{0.722222in}}%
\pgfpathlineto{\pgfqpoint{0.817006in}{0.689139in}}%
\pgfpathlineto{\pgfqpoint{0.780497in}{0.683834in}}%
\pgfpathlineto{\pgfqpoint{0.806915in}{0.658083in}}%
\pgfpathlineto{\pgfqpoint{0.800679in}{0.621721in}}%
\pgfpathlineto{\pgfqpoint{0.833333in}{0.638889in}}%
\pgfpathlineto{\pgfqpoint{0.865988in}{0.621721in}}%
\pgfpathlineto{\pgfqpoint{0.859752in}{0.658083in}}%
\pgfpathlineto{\pgfqpoint{0.886170in}{0.683834in}}%
\pgfpathlineto{\pgfqpoint{0.849661in}{0.689139in}}%
\pgfpathlineto{\pgfqpoint{0.833333in}{0.722222in}}%
\pgfpathmoveto{\pgfqpoint{1.000000in}{0.722222in}}%
\pgfpathlineto{\pgfqpoint{0.983673in}{0.689139in}}%
\pgfpathlineto{\pgfqpoint{0.947164in}{0.683834in}}%
\pgfpathlineto{\pgfqpoint{0.973582in}{0.658083in}}%
\pgfpathlineto{\pgfqpoint{0.967345in}{0.621721in}}%
\pgfpathlineto{\pgfqpoint{1.000000in}{0.638889in}}%
\pgfpathlineto{\pgfqpoint{1.032655in}{0.621721in}}%
\pgfpathlineto{\pgfqpoint{1.026418in}{0.658083in}}%
\pgfpathlineto{\pgfqpoint{1.052836in}{0.683834in}}%
\pgfpathlineto{\pgfqpoint{1.016327in}{0.689139in}}%
\pgfpathlineto{\pgfqpoint{1.000000in}{0.722222in}}%
\pgfpathmoveto{\pgfqpoint{0.083333in}{0.888889in}}%
\pgfpathlineto{\pgfqpoint{0.067006in}{0.855806in}}%
\pgfpathlineto{\pgfqpoint{0.030497in}{0.850501in}}%
\pgfpathlineto{\pgfqpoint{0.056915in}{0.824750in}}%
\pgfpathlineto{\pgfqpoint{0.050679in}{0.788388in}}%
\pgfpathlineto{\pgfqpoint{0.083333in}{0.805556in}}%
\pgfpathlineto{\pgfqpoint{0.115988in}{0.788388in}}%
\pgfpathlineto{\pgfqpoint{0.109752in}{0.824750in}}%
\pgfpathlineto{\pgfqpoint{0.136170in}{0.850501in}}%
\pgfpathlineto{\pgfqpoint{0.099661in}{0.855806in}}%
\pgfpathlineto{\pgfqpoint{0.083333in}{0.888889in}}%
\pgfpathmoveto{\pgfqpoint{0.250000in}{0.888889in}}%
\pgfpathlineto{\pgfqpoint{0.233673in}{0.855806in}}%
\pgfpathlineto{\pgfqpoint{0.197164in}{0.850501in}}%
\pgfpathlineto{\pgfqpoint{0.223582in}{0.824750in}}%
\pgfpathlineto{\pgfqpoint{0.217345in}{0.788388in}}%
\pgfpathlineto{\pgfqpoint{0.250000in}{0.805556in}}%
\pgfpathlineto{\pgfqpoint{0.282655in}{0.788388in}}%
\pgfpathlineto{\pgfqpoint{0.276418in}{0.824750in}}%
\pgfpathlineto{\pgfqpoint{0.302836in}{0.850501in}}%
\pgfpathlineto{\pgfqpoint{0.266327in}{0.855806in}}%
\pgfpathlineto{\pgfqpoint{0.250000in}{0.888889in}}%
\pgfpathmoveto{\pgfqpoint{0.416667in}{0.888889in}}%
\pgfpathlineto{\pgfqpoint{0.400339in}{0.855806in}}%
\pgfpathlineto{\pgfqpoint{0.363830in}{0.850501in}}%
\pgfpathlineto{\pgfqpoint{0.390248in}{0.824750in}}%
\pgfpathlineto{\pgfqpoint{0.384012in}{0.788388in}}%
\pgfpathlineto{\pgfqpoint{0.416667in}{0.805556in}}%
\pgfpathlineto{\pgfqpoint{0.449321in}{0.788388in}}%
\pgfpathlineto{\pgfqpoint{0.443085in}{0.824750in}}%
\pgfpathlineto{\pgfqpoint{0.469503in}{0.850501in}}%
\pgfpathlineto{\pgfqpoint{0.432994in}{0.855806in}}%
\pgfpathlineto{\pgfqpoint{0.416667in}{0.888889in}}%
\pgfpathmoveto{\pgfqpoint{0.583333in}{0.888889in}}%
\pgfpathlineto{\pgfqpoint{0.567006in}{0.855806in}}%
\pgfpathlineto{\pgfqpoint{0.530497in}{0.850501in}}%
\pgfpathlineto{\pgfqpoint{0.556915in}{0.824750in}}%
\pgfpathlineto{\pgfqpoint{0.550679in}{0.788388in}}%
\pgfpathlineto{\pgfqpoint{0.583333in}{0.805556in}}%
\pgfpathlineto{\pgfqpoint{0.615988in}{0.788388in}}%
\pgfpathlineto{\pgfqpoint{0.609752in}{0.824750in}}%
\pgfpathlineto{\pgfqpoint{0.636170in}{0.850501in}}%
\pgfpathlineto{\pgfqpoint{0.599661in}{0.855806in}}%
\pgfpathlineto{\pgfqpoint{0.583333in}{0.888889in}}%
\pgfpathmoveto{\pgfqpoint{0.750000in}{0.888889in}}%
\pgfpathlineto{\pgfqpoint{0.733673in}{0.855806in}}%
\pgfpathlineto{\pgfqpoint{0.697164in}{0.850501in}}%
\pgfpathlineto{\pgfqpoint{0.723582in}{0.824750in}}%
\pgfpathlineto{\pgfqpoint{0.717345in}{0.788388in}}%
\pgfpathlineto{\pgfqpoint{0.750000in}{0.805556in}}%
\pgfpathlineto{\pgfqpoint{0.782655in}{0.788388in}}%
\pgfpathlineto{\pgfqpoint{0.776418in}{0.824750in}}%
\pgfpathlineto{\pgfqpoint{0.802836in}{0.850501in}}%
\pgfpathlineto{\pgfqpoint{0.766327in}{0.855806in}}%
\pgfpathlineto{\pgfqpoint{0.750000in}{0.888889in}}%
\pgfpathmoveto{\pgfqpoint{0.916667in}{0.888889in}}%
\pgfpathlineto{\pgfqpoint{0.900339in}{0.855806in}}%
\pgfpathlineto{\pgfqpoint{0.863830in}{0.850501in}}%
\pgfpathlineto{\pgfqpoint{0.890248in}{0.824750in}}%
\pgfpathlineto{\pgfqpoint{0.884012in}{0.788388in}}%
\pgfpathlineto{\pgfqpoint{0.916667in}{0.805556in}}%
\pgfpathlineto{\pgfqpoint{0.949321in}{0.788388in}}%
\pgfpathlineto{\pgfqpoint{0.943085in}{0.824750in}}%
\pgfpathlineto{\pgfqpoint{0.969503in}{0.850501in}}%
\pgfpathlineto{\pgfqpoint{0.932994in}{0.855806in}}%
\pgfpathlineto{\pgfqpoint{0.916667in}{0.888889in}}%
\pgfpathmoveto{\pgfqpoint{0.000000in}{1.055556in}}%
\pgfpathlineto{\pgfqpoint{-0.016327in}{1.022473in}}%
\pgfpathlineto{\pgfqpoint{-0.052836in}{1.017168in}}%
\pgfpathlineto{\pgfqpoint{-0.026418in}{0.991416in}}%
\pgfpathlineto{\pgfqpoint{-0.032655in}{0.955055in}}%
\pgfpathlineto{\pgfqpoint{-0.000000in}{0.972222in}}%
\pgfpathlineto{\pgfqpoint{0.032655in}{0.955055in}}%
\pgfpathlineto{\pgfqpoint{0.026418in}{0.991416in}}%
\pgfpathlineto{\pgfqpoint{0.052836in}{1.017168in}}%
\pgfpathlineto{\pgfqpoint{0.016327in}{1.022473in}}%
\pgfpathlineto{\pgfqpoint{0.000000in}{1.055556in}}%
\pgfpathmoveto{\pgfqpoint{0.166667in}{1.055556in}}%
\pgfpathlineto{\pgfqpoint{0.150339in}{1.022473in}}%
\pgfpathlineto{\pgfqpoint{0.113830in}{1.017168in}}%
\pgfpathlineto{\pgfqpoint{0.140248in}{0.991416in}}%
\pgfpathlineto{\pgfqpoint{0.134012in}{0.955055in}}%
\pgfpathlineto{\pgfqpoint{0.166667in}{0.972222in}}%
\pgfpathlineto{\pgfqpoint{0.199321in}{0.955055in}}%
\pgfpathlineto{\pgfqpoint{0.193085in}{0.991416in}}%
\pgfpathlineto{\pgfqpoint{0.219503in}{1.017168in}}%
\pgfpathlineto{\pgfqpoint{0.182994in}{1.022473in}}%
\pgfpathlineto{\pgfqpoint{0.166667in}{1.055556in}}%
\pgfpathmoveto{\pgfqpoint{0.333333in}{1.055556in}}%
\pgfpathlineto{\pgfqpoint{0.317006in}{1.022473in}}%
\pgfpathlineto{\pgfqpoint{0.280497in}{1.017168in}}%
\pgfpathlineto{\pgfqpoint{0.306915in}{0.991416in}}%
\pgfpathlineto{\pgfqpoint{0.300679in}{0.955055in}}%
\pgfpathlineto{\pgfqpoint{0.333333in}{0.972222in}}%
\pgfpathlineto{\pgfqpoint{0.365988in}{0.955055in}}%
\pgfpathlineto{\pgfqpoint{0.359752in}{0.991416in}}%
\pgfpathlineto{\pgfqpoint{0.386170in}{1.017168in}}%
\pgfpathlineto{\pgfqpoint{0.349661in}{1.022473in}}%
\pgfpathlineto{\pgfqpoint{0.333333in}{1.055556in}}%
\pgfpathmoveto{\pgfqpoint{0.500000in}{1.055556in}}%
\pgfpathlineto{\pgfqpoint{0.483673in}{1.022473in}}%
\pgfpathlineto{\pgfqpoint{0.447164in}{1.017168in}}%
\pgfpathlineto{\pgfqpoint{0.473582in}{0.991416in}}%
\pgfpathlineto{\pgfqpoint{0.467345in}{0.955055in}}%
\pgfpathlineto{\pgfqpoint{0.500000in}{0.972222in}}%
\pgfpathlineto{\pgfqpoint{0.532655in}{0.955055in}}%
\pgfpathlineto{\pgfqpoint{0.526418in}{0.991416in}}%
\pgfpathlineto{\pgfqpoint{0.552836in}{1.017168in}}%
\pgfpathlineto{\pgfqpoint{0.516327in}{1.022473in}}%
\pgfpathlineto{\pgfqpoint{0.500000in}{1.055556in}}%
\pgfpathmoveto{\pgfqpoint{0.666667in}{1.055556in}}%
\pgfpathlineto{\pgfqpoint{0.650339in}{1.022473in}}%
\pgfpathlineto{\pgfqpoint{0.613830in}{1.017168in}}%
\pgfpathlineto{\pgfqpoint{0.640248in}{0.991416in}}%
\pgfpathlineto{\pgfqpoint{0.634012in}{0.955055in}}%
\pgfpathlineto{\pgfqpoint{0.666667in}{0.972222in}}%
\pgfpathlineto{\pgfqpoint{0.699321in}{0.955055in}}%
\pgfpathlineto{\pgfqpoint{0.693085in}{0.991416in}}%
\pgfpathlineto{\pgfqpoint{0.719503in}{1.017168in}}%
\pgfpathlineto{\pgfqpoint{0.682994in}{1.022473in}}%
\pgfpathlineto{\pgfqpoint{0.666667in}{1.055556in}}%
\pgfpathmoveto{\pgfqpoint{0.833333in}{1.055556in}}%
\pgfpathlineto{\pgfqpoint{0.817006in}{1.022473in}}%
\pgfpathlineto{\pgfqpoint{0.780497in}{1.017168in}}%
\pgfpathlineto{\pgfqpoint{0.806915in}{0.991416in}}%
\pgfpathlineto{\pgfqpoint{0.800679in}{0.955055in}}%
\pgfpathlineto{\pgfqpoint{0.833333in}{0.972222in}}%
\pgfpathlineto{\pgfqpoint{0.865988in}{0.955055in}}%
\pgfpathlineto{\pgfqpoint{0.859752in}{0.991416in}}%
\pgfpathlineto{\pgfqpoint{0.886170in}{1.017168in}}%
\pgfpathlineto{\pgfqpoint{0.849661in}{1.022473in}}%
\pgfpathlineto{\pgfqpoint{0.833333in}{1.055556in}}%
\pgfpathmoveto{\pgfqpoint{1.000000in}{1.055556in}}%
\pgfpathlineto{\pgfqpoint{0.983673in}{1.022473in}}%
\pgfpathlineto{\pgfqpoint{0.947164in}{1.017168in}}%
\pgfpathlineto{\pgfqpoint{0.973582in}{0.991416in}}%
\pgfpathlineto{\pgfqpoint{0.967345in}{0.955055in}}%
\pgfpathlineto{\pgfqpoint{1.000000in}{0.972222in}}%
\pgfpathlineto{\pgfqpoint{1.032655in}{0.955055in}}%
\pgfpathlineto{\pgfqpoint{1.026418in}{0.991416in}}%
\pgfpathlineto{\pgfqpoint{1.052836in}{1.017168in}}%
\pgfpathlineto{\pgfqpoint{1.016327in}{1.022473in}}%
\pgfpathlineto{\pgfqpoint{1.000000in}{1.055556in}}%
\pgfpathlineto{\pgfqpoint{1.000000in}{1.055556in}}%
\pgfusepath{stroke}%
\end{pgfscope}%
}%
\pgfsys@transformshift{5.973315in}{5.330636in}%
\pgfsys@useobject{currentpattern}{}%
\pgfsys@transformshift{1in}{0in}%
\pgfsys@transformshift{-1in}{0in}%
\pgfsys@transformshift{0in}{1in}%
\end{pgfscope}%
\begin{pgfscope}%
\pgfpathrectangle{\pgfqpoint{0.935815in}{0.637495in}}{\pgfqpoint{9.300000in}{9.060000in}}%
\pgfusepath{clip}%
\pgfsetbuttcap%
\pgfsetmiterjoin%
\definecolor{currentfill}{rgb}{1.000000,1.000000,0.000000}%
\pgfsetfillcolor{currentfill}%
\pgfsetfillopacity{0.990000}%
\pgfsetlinewidth{0.000000pt}%
\definecolor{currentstroke}{rgb}{0.000000,0.000000,0.000000}%
\pgfsetstrokecolor{currentstroke}%
\pgfsetstrokeopacity{0.990000}%
\pgfsetdash{}{0pt}%
\pgfpathmoveto{\pgfqpoint{7.523315in}{5.549528in}}%
\pgfpathlineto{\pgfqpoint{8.298315in}{5.549528in}}%
\pgfpathlineto{\pgfqpoint{8.298315in}{6.633305in}}%
\pgfpathlineto{\pgfqpoint{7.523315in}{6.633305in}}%
\pgfpathclose%
\pgfusepath{fill}%
\end{pgfscope}%
\begin{pgfscope}%
\pgfsetbuttcap%
\pgfsetmiterjoin%
\definecolor{currentfill}{rgb}{1.000000,1.000000,0.000000}%
\pgfsetfillcolor{currentfill}%
\pgfsetfillopacity{0.990000}%
\pgfsetlinewidth{0.000000pt}%
\definecolor{currentstroke}{rgb}{0.000000,0.000000,0.000000}%
\pgfsetstrokecolor{currentstroke}%
\pgfsetstrokeopacity{0.990000}%
\pgfsetdash{}{0pt}%
\pgfpathrectangle{\pgfqpoint{0.935815in}{0.637495in}}{\pgfqpoint{9.300000in}{9.060000in}}%
\pgfusepath{clip}%
\pgfpathmoveto{\pgfqpoint{7.523315in}{5.549528in}}%
\pgfpathlineto{\pgfqpoint{8.298315in}{5.549528in}}%
\pgfpathlineto{\pgfqpoint{8.298315in}{6.633305in}}%
\pgfpathlineto{\pgfqpoint{7.523315in}{6.633305in}}%
\pgfpathclose%
\pgfusepath{clip}%
\pgfsys@defobject{currentpattern}{\pgfqpoint{0in}{0in}}{\pgfqpoint{1in}{1in}}{%
\begin{pgfscope}%
\pgfpathrectangle{\pgfqpoint{0in}{0in}}{\pgfqpoint{1in}{1in}}%
\pgfusepath{clip}%
\pgfpathmoveto{\pgfqpoint{0.000000in}{0.055556in}}%
\pgfpathlineto{\pgfqpoint{-0.016327in}{0.022473in}}%
\pgfpathlineto{\pgfqpoint{-0.052836in}{0.017168in}}%
\pgfpathlineto{\pgfqpoint{-0.026418in}{-0.008584in}}%
\pgfpathlineto{\pgfqpoint{-0.032655in}{-0.044945in}}%
\pgfpathlineto{\pgfqpoint{-0.000000in}{-0.027778in}}%
\pgfpathlineto{\pgfqpoint{0.032655in}{-0.044945in}}%
\pgfpathlineto{\pgfqpoint{0.026418in}{-0.008584in}}%
\pgfpathlineto{\pgfqpoint{0.052836in}{0.017168in}}%
\pgfpathlineto{\pgfqpoint{0.016327in}{0.022473in}}%
\pgfpathlineto{\pgfqpoint{0.000000in}{0.055556in}}%
\pgfpathmoveto{\pgfqpoint{0.166667in}{0.055556in}}%
\pgfpathlineto{\pgfqpoint{0.150339in}{0.022473in}}%
\pgfpathlineto{\pgfqpoint{0.113830in}{0.017168in}}%
\pgfpathlineto{\pgfqpoint{0.140248in}{-0.008584in}}%
\pgfpathlineto{\pgfqpoint{0.134012in}{-0.044945in}}%
\pgfpathlineto{\pgfqpoint{0.166667in}{-0.027778in}}%
\pgfpathlineto{\pgfqpoint{0.199321in}{-0.044945in}}%
\pgfpathlineto{\pgfqpoint{0.193085in}{-0.008584in}}%
\pgfpathlineto{\pgfqpoint{0.219503in}{0.017168in}}%
\pgfpathlineto{\pgfqpoint{0.182994in}{0.022473in}}%
\pgfpathlineto{\pgfqpoint{0.166667in}{0.055556in}}%
\pgfpathmoveto{\pgfqpoint{0.333333in}{0.055556in}}%
\pgfpathlineto{\pgfqpoint{0.317006in}{0.022473in}}%
\pgfpathlineto{\pgfqpoint{0.280497in}{0.017168in}}%
\pgfpathlineto{\pgfqpoint{0.306915in}{-0.008584in}}%
\pgfpathlineto{\pgfqpoint{0.300679in}{-0.044945in}}%
\pgfpathlineto{\pgfqpoint{0.333333in}{-0.027778in}}%
\pgfpathlineto{\pgfqpoint{0.365988in}{-0.044945in}}%
\pgfpathlineto{\pgfqpoint{0.359752in}{-0.008584in}}%
\pgfpathlineto{\pgfqpoint{0.386170in}{0.017168in}}%
\pgfpathlineto{\pgfqpoint{0.349661in}{0.022473in}}%
\pgfpathlineto{\pgfqpoint{0.333333in}{0.055556in}}%
\pgfpathmoveto{\pgfqpoint{0.500000in}{0.055556in}}%
\pgfpathlineto{\pgfqpoint{0.483673in}{0.022473in}}%
\pgfpathlineto{\pgfqpoint{0.447164in}{0.017168in}}%
\pgfpathlineto{\pgfqpoint{0.473582in}{-0.008584in}}%
\pgfpathlineto{\pgfqpoint{0.467345in}{-0.044945in}}%
\pgfpathlineto{\pgfqpoint{0.500000in}{-0.027778in}}%
\pgfpathlineto{\pgfqpoint{0.532655in}{-0.044945in}}%
\pgfpathlineto{\pgfqpoint{0.526418in}{-0.008584in}}%
\pgfpathlineto{\pgfqpoint{0.552836in}{0.017168in}}%
\pgfpathlineto{\pgfqpoint{0.516327in}{0.022473in}}%
\pgfpathlineto{\pgfqpoint{0.500000in}{0.055556in}}%
\pgfpathmoveto{\pgfqpoint{0.666667in}{0.055556in}}%
\pgfpathlineto{\pgfqpoint{0.650339in}{0.022473in}}%
\pgfpathlineto{\pgfqpoint{0.613830in}{0.017168in}}%
\pgfpathlineto{\pgfqpoint{0.640248in}{-0.008584in}}%
\pgfpathlineto{\pgfqpoint{0.634012in}{-0.044945in}}%
\pgfpathlineto{\pgfqpoint{0.666667in}{-0.027778in}}%
\pgfpathlineto{\pgfqpoint{0.699321in}{-0.044945in}}%
\pgfpathlineto{\pgfqpoint{0.693085in}{-0.008584in}}%
\pgfpathlineto{\pgfqpoint{0.719503in}{0.017168in}}%
\pgfpathlineto{\pgfqpoint{0.682994in}{0.022473in}}%
\pgfpathlineto{\pgfqpoint{0.666667in}{0.055556in}}%
\pgfpathmoveto{\pgfqpoint{0.833333in}{0.055556in}}%
\pgfpathlineto{\pgfqpoint{0.817006in}{0.022473in}}%
\pgfpathlineto{\pgfqpoint{0.780497in}{0.017168in}}%
\pgfpathlineto{\pgfqpoint{0.806915in}{-0.008584in}}%
\pgfpathlineto{\pgfqpoint{0.800679in}{-0.044945in}}%
\pgfpathlineto{\pgfqpoint{0.833333in}{-0.027778in}}%
\pgfpathlineto{\pgfqpoint{0.865988in}{-0.044945in}}%
\pgfpathlineto{\pgfqpoint{0.859752in}{-0.008584in}}%
\pgfpathlineto{\pgfqpoint{0.886170in}{0.017168in}}%
\pgfpathlineto{\pgfqpoint{0.849661in}{0.022473in}}%
\pgfpathlineto{\pgfqpoint{0.833333in}{0.055556in}}%
\pgfpathmoveto{\pgfqpoint{1.000000in}{0.055556in}}%
\pgfpathlineto{\pgfqpoint{0.983673in}{0.022473in}}%
\pgfpathlineto{\pgfqpoint{0.947164in}{0.017168in}}%
\pgfpathlineto{\pgfqpoint{0.973582in}{-0.008584in}}%
\pgfpathlineto{\pgfqpoint{0.967345in}{-0.044945in}}%
\pgfpathlineto{\pgfqpoint{1.000000in}{-0.027778in}}%
\pgfpathlineto{\pgfqpoint{1.032655in}{-0.044945in}}%
\pgfpathlineto{\pgfqpoint{1.026418in}{-0.008584in}}%
\pgfpathlineto{\pgfqpoint{1.052836in}{0.017168in}}%
\pgfpathlineto{\pgfqpoint{1.016327in}{0.022473in}}%
\pgfpathlineto{\pgfqpoint{1.000000in}{0.055556in}}%
\pgfpathmoveto{\pgfqpoint{0.083333in}{0.222222in}}%
\pgfpathlineto{\pgfqpoint{0.067006in}{0.189139in}}%
\pgfpathlineto{\pgfqpoint{0.030497in}{0.183834in}}%
\pgfpathlineto{\pgfqpoint{0.056915in}{0.158083in}}%
\pgfpathlineto{\pgfqpoint{0.050679in}{0.121721in}}%
\pgfpathlineto{\pgfqpoint{0.083333in}{0.138889in}}%
\pgfpathlineto{\pgfqpoint{0.115988in}{0.121721in}}%
\pgfpathlineto{\pgfqpoint{0.109752in}{0.158083in}}%
\pgfpathlineto{\pgfqpoint{0.136170in}{0.183834in}}%
\pgfpathlineto{\pgfqpoint{0.099661in}{0.189139in}}%
\pgfpathlineto{\pgfqpoint{0.083333in}{0.222222in}}%
\pgfpathmoveto{\pgfqpoint{0.250000in}{0.222222in}}%
\pgfpathlineto{\pgfqpoint{0.233673in}{0.189139in}}%
\pgfpathlineto{\pgfqpoint{0.197164in}{0.183834in}}%
\pgfpathlineto{\pgfqpoint{0.223582in}{0.158083in}}%
\pgfpathlineto{\pgfqpoint{0.217345in}{0.121721in}}%
\pgfpathlineto{\pgfqpoint{0.250000in}{0.138889in}}%
\pgfpathlineto{\pgfqpoint{0.282655in}{0.121721in}}%
\pgfpathlineto{\pgfqpoint{0.276418in}{0.158083in}}%
\pgfpathlineto{\pgfqpoint{0.302836in}{0.183834in}}%
\pgfpathlineto{\pgfqpoint{0.266327in}{0.189139in}}%
\pgfpathlineto{\pgfqpoint{0.250000in}{0.222222in}}%
\pgfpathmoveto{\pgfqpoint{0.416667in}{0.222222in}}%
\pgfpathlineto{\pgfqpoint{0.400339in}{0.189139in}}%
\pgfpathlineto{\pgfqpoint{0.363830in}{0.183834in}}%
\pgfpathlineto{\pgfqpoint{0.390248in}{0.158083in}}%
\pgfpathlineto{\pgfqpoint{0.384012in}{0.121721in}}%
\pgfpathlineto{\pgfqpoint{0.416667in}{0.138889in}}%
\pgfpathlineto{\pgfqpoint{0.449321in}{0.121721in}}%
\pgfpathlineto{\pgfqpoint{0.443085in}{0.158083in}}%
\pgfpathlineto{\pgfqpoint{0.469503in}{0.183834in}}%
\pgfpathlineto{\pgfqpoint{0.432994in}{0.189139in}}%
\pgfpathlineto{\pgfqpoint{0.416667in}{0.222222in}}%
\pgfpathmoveto{\pgfqpoint{0.583333in}{0.222222in}}%
\pgfpathlineto{\pgfqpoint{0.567006in}{0.189139in}}%
\pgfpathlineto{\pgfqpoint{0.530497in}{0.183834in}}%
\pgfpathlineto{\pgfqpoint{0.556915in}{0.158083in}}%
\pgfpathlineto{\pgfqpoint{0.550679in}{0.121721in}}%
\pgfpathlineto{\pgfqpoint{0.583333in}{0.138889in}}%
\pgfpathlineto{\pgfqpoint{0.615988in}{0.121721in}}%
\pgfpathlineto{\pgfqpoint{0.609752in}{0.158083in}}%
\pgfpathlineto{\pgfqpoint{0.636170in}{0.183834in}}%
\pgfpathlineto{\pgfqpoint{0.599661in}{0.189139in}}%
\pgfpathlineto{\pgfqpoint{0.583333in}{0.222222in}}%
\pgfpathmoveto{\pgfqpoint{0.750000in}{0.222222in}}%
\pgfpathlineto{\pgfqpoint{0.733673in}{0.189139in}}%
\pgfpathlineto{\pgfqpoint{0.697164in}{0.183834in}}%
\pgfpathlineto{\pgfqpoint{0.723582in}{0.158083in}}%
\pgfpathlineto{\pgfqpoint{0.717345in}{0.121721in}}%
\pgfpathlineto{\pgfqpoint{0.750000in}{0.138889in}}%
\pgfpathlineto{\pgfqpoint{0.782655in}{0.121721in}}%
\pgfpathlineto{\pgfqpoint{0.776418in}{0.158083in}}%
\pgfpathlineto{\pgfqpoint{0.802836in}{0.183834in}}%
\pgfpathlineto{\pgfqpoint{0.766327in}{0.189139in}}%
\pgfpathlineto{\pgfqpoint{0.750000in}{0.222222in}}%
\pgfpathmoveto{\pgfqpoint{0.916667in}{0.222222in}}%
\pgfpathlineto{\pgfqpoint{0.900339in}{0.189139in}}%
\pgfpathlineto{\pgfqpoint{0.863830in}{0.183834in}}%
\pgfpathlineto{\pgfqpoint{0.890248in}{0.158083in}}%
\pgfpathlineto{\pgfqpoint{0.884012in}{0.121721in}}%
\pgfpathlineto{\pgfqpoint{0.916667in}{0.138889in}}%
\pgfpathlineto{\pgfqpoint{0.949321in}{0.121721in}}%
\pgfpathlineto{\pgfqpoint{0.943085in}{0.158083in}}%
\pgfpathlineto{\pgfqpoint{0.969503in}{0.183834in}}%
\pgfpathlineto{\pgfqpoint{0.932994in}{0.189139in}}%
\pgfpathlineto{\pgfqpoint{0.916667in}{0.222222in}}%
\pgfpathmoveto{\pgfqpoint{0.000000in}{0.388889in}}%
\pgfpathlineto{\pgfqpoint{-0.016327in}{0.355806in}}%
\pgfpathlineto{\pgfqpoint{-0.052836in}{0.350501in}}%
\pgfpathlineto{\pgfqpoint{-0.026418in}{0.324750in}}%
\pgfpathlineto{\pgfqpoint{-0.032655in}{0.288388in}}%
\pgfpathlineto{\pgfqpoint{-0.000000in}{0.305556in}}%
\pgfpathlineto{\pgfqpoint{0.032655in}{0.288388in}}%
\pgfpathlineto{\pgfqpoint{0.026418in}{0.324750in}}%
\pgfpathlineto{\pgfqpoint{0.052836in}{0.350501in}}%
\pgfpathlineto{\pgfqpoint{0.016327in}{0.355806in}}%
\pgfpathlineto{\pgfqpoint{0.000000in}{0.388889in}}%
\pgfpathmoveto{\pgfqpoint{0.166667in}{0.388889in}}%
\pgfpathlineto{\pgfqpoint{0.150339in}{0.355806in}}%
\pgfpathlineto{\pgfqpoint{0.113830in}{0.350501in}}%
\pgfpathlineto{\pgfqpoint{0.140248in}{0.324750in}}%
\pgfpathlineto{\pgfqpoint{0.134012in}{0.288388in}}%
\pgfpathlineto{\pgfqpoint{0.166667in}{0.305556in}}%
\pgfpathlineto{\pgfqpoint{0.199321in}{0.288388in}}%
\pgfpathlineto{\pgfqpoint{0.193085in}{0.324750in}}%
\pgfpathlineto{\pgfqpoint{0.219503in}{0.350501in}}%
\pgfpathlineto{\pgfqpoint{0.182994in}{0.355806in}}%
\pgfpathlineto{\pgfqpoint{0.166667in}{0.388889in}}%
\pgfpathmoveto{\pgfqpoint{0.333333in}{0.388889in}}%
\pgfpathlineto{\pgfqpoint{0.317006in}{0.355806in}}%
\pgfpathlineto{\pgfqpoint{0.280497in}{0.350501in}}%
\pgfpathlineto{\pgfqpoint{0.306915in}{0.324750in}}%
\pgfpathlineto{\pgfqpoint{0.300679in}{0.288388in}}%
\pgfpathlineto{\pgfqpoint{0.333333in}{0.305556in}}%
\pgfpathlineto{\pgfqpoint{0.365988in}{0.288388in}}%
\pgfpathlineto{\pgfqpoint{0.359752in}{0.324750in}}%
\pgfpathlineto{\pgfqpoint{0.386170in}{0.350501in}}%
\pgfpathlineto{\pgfqpoint{0.349661in}{0.355806in}}%
\pgfpathlineto{\pgfqpoint{0.333333in}{0.388889in}}%
\pgfpathmoveto{\pgfqpoint{0.500000in}{0.388889in}}%
\pgfpathlineto{\pgfqpoint{0.483673in}{0.355806in}}%
\pgfpathlineto{\pgfqpoint{0.447164in}{0.350501in}}%
\pgfpathlineto{\pgfqpoint{0.473582in}{0.324750in}}%
\pgfpathlineto{\pgfqpoint{0.467345in}{0.288388in}}%
\pgfpathlineto{\pgfqpoint{0.500000in}{0.305556in}}%
\pgfpathlineto{\pgfqpoint{0.532655in}{0.288388in}}%
\pgfpathlineto{\pgfqpoint{0.526418in}{0.324750in}}%
\pgfpathlineto{\pgfqpoint{0.552836in}{0.350501in}}%
\pgfpathlineto{\pgfqpoint{0.516327in}{0.355806in}}%
\pgfpathlineto{\pgfqpoint{0.500000in}{0.388889in}}%
\pgfpathmoveto{\pgfqpoint{0.666667in}{0.388889in}}%
\pgfpathlineto{\pgfqpoint{0.650339in}{0.355806in}}%
\pgfpathlineto{\pgfqpoint{0.613830in}{0.350501in}}%
\pgfpathlineto{\pgfqpoint{0.640248in}{0.324750in}}%
\pgfpathlineto{\pgfqpoint{0.634012in}{0.288388in}}%
\pgfpathlineto{\pgfqpoint{0.666667in}{0.305556in}}%
\pgfpathlineto{\pgfqpoint{0.699321in}{0.288388in}}%
\pgfpathlineto{\pgfqpoint{0.693085in}{0.324750in}}%
\pgfpathlineto{\pgfqpoint{0.719503in}{0.350501in}}%
\pgfpathlineto{\pgfqpoint{0.682994in}{0.355806in}}%
\pgfpathlineto{\pgfqpoint{0.666667in}{0.388889in}}%
\pgfpathmoveto{\pgfqpoint{0.833333in}{0.388889in}}%
\pgfpathlineto{\pgfqpoint{0.817006in}{0.355806in}}%
\pgfpathlineto{\pgfqpoint{0.780497in}{0.350501in}}%
\pgfpathlineto{\pgfqpoint{0.806915in}{0.324750in}}%
\pgfpathlineto{\pgfqpoint{0.800679in}{0.288388in}}%
\pgfpathlineto{\pgfqpoint{0.833333in}{0.305556in}}%
\pgfpathlineto{\pgfqpoint{0.865988in}{0.288388in}}%
\pgfpathlineto{\pgfqpoint{0.859752in}{0.324750in}}%
\pgfpathlineto{\pgfqpoint{0.886170in}{0.350501in}}%
\pgfpathlineto{\pgfqpoint{0.849661in}{0.355806in}}%
\pgfpathlineto{\pgfqpoint{0.833333in}{0.388889in}}%
\pgfpathmoveto{\pgfqpoint{1.000000in}{0.388889in}}%
\pgfpathlineto{\pgfqpoint{0.983673in}{0.355806in}}%
\pgfpathlineto{\pgfqpoint{0.947164in}{0.350501in}}%
\pgfpathlineto{\pgfqpoint{0.973582in}{0.324750in}}%
\pgfpathlineto{\pgfqpoint{0.967345in}{0.288388in}}%
\pgfpathlineto{\pgfqpoint{1.000000in}{0.305556in}}%
\pgfpathlineto{\pgfqpoint{1.032655in}{0.288388in}}%
\pgfpathlineto{\pgfqpoint{1.026418in}{0.324750in}}%
\pgfpathlineto{\pgfqpoint{1.052836in}{0.350501in}}%
\pgfpathlineto{\pgfqpoint{1.016327in}{0.355806in}}%
\pgfpathlineto{\pgfqpoint{1.000000in}{0.388889in}}%
\pgfpathmoveto{\pgfqpoint{0.083333in}{0.555556in}}%
\pgfpathlineto{\pgfqpoint{0.067006in}{0.522473in}}%
\pgfpathlineto{\pgfqpoint{0.030497in}{0.517168in}}%
\pgfpathlineto{\pgfqpoint{0.056915in}{0.491416in}}%
\pgfpathlineto{\pgfqpoint{0.050679in}{0.455055in}}%
\pgfpathlineto{\pgfqpoint{0.083333in}{0.472222in}}%
\pgfpathlineto{\pgfqpoint{0.115988in}{0.455055in}}%
\pgfpathlineto{\pgfqpoint{0.109752in}{0.491416in}}%
\pgfpathlineto{\pgfqpoint{0.136170in}{0.517168in}}%
\pgfpathlineto{\pgfqpoint{0.099661in}{0.522473in}}%
\pgfpathlineto{\pgfqpoint{0.083333in}{0.555556in}}%
\pgfpathmoveto{\pgfqpoint{0.250000in}{0.555556in}}%
\pgfpathlineto{\pgfqpoint{0.233673in}{0.522473in}}%
\pgfpathlineto{\pgfqpoint{0.197164in}{0.517168in}}%
\pgfpathlineto{\pgfqpoint{0.223582in}{0.491416in}}%
\pgfpathlineto{\pgfqpoint{0.217345in}{0.455055in}}%
\pgfpathlineto{\pgfqpoint{0.250000in}{0.472222in}}%
\pgfpathlineto{\pgfqpoint{0.282655in}{0.455055in}}%
\pgfpathlineto{\pgfqpoint{0.276418in}{0.491416in}}%
\pgfpathlineto{\pgfqpoint{0.302836in}{0.517168in}}%
\pgfpathlineto{\pgfqpoint{0.266327in}{0.522473in}}%
\pgfpathlineto{\pgfqpoint{0.250000in}{0.555556in}}%
\pgfpathmoveto{\pgfqpoint{0.416667in}{0.555556in}}%
\pgfpathlineto{\pgfqpoint{0.400339in}{0.522473in}}%
\pgfpathlineto{\pgfqpoint{0.363830in}{0.517168in}}%
\pgfpathlineto{\pgfqpoint{0.390248in}{0.491416in}}%
\pgfpathlineto{\pgfqpoint{0.384012in}{0.455055in}}%
\pgfpathlineto{\pgfqpoint{0.416667in}{0.472222in}}%
\pgfpathlineto{\pgfqpoint{0.449321in}{0.455055in}}%
\pgfpathlineto{\pgfqpoint{0.443085in}{0.491416in}}%
\pgfpathlineto{\pgfqpoint{0.469503in}{0.517168in}}%
\pgfpathlineto{\pgfqpoint{0.432994in}{0.522473in}}%
\pgfpathlineto{\pgfqpoint{0.416667in}{0.555556in}}%
\pgfpathmoveto{\pgfqpoint{0.583333in}{0.555556in}}%
\pgfpathlineto{\pgfqpoint{0.567006in}{0.522473in}}%
\pgfpathlineto{\pgfqpoint{0.530497in}{0.517168in}}%
\pgfpathlineto{\pgfqpoint{0.556915in}{0.491416in}}%
\pgfpathlineto{\pgfqpoint{0.550679in}{0.455055in}}%
\pgfpathlineto{\pgfqpoint{0.583333in}{0.472222in}}%
\pgfpathlineto{\pgfqpoint{0.615988in}{0.455055in}}%
\pgfpathlineto{\pgfqpoint{0.609752in}{0.491416in}}%
\pgfpathlineto{\pgfqpoint{0.636170in}{0.517168in}}%
\pgfpathlineto{\pgfqpoint{0.599661in}{0.522473in}}%
\pgfpathlineto{\pgfqpoint{0.583333in}{0.555556in}}%
\pgfpathmoveto{\pgfqpoint{0.750000in}{0.555556in}}%
\pgfpathlineto{\pgfqpoint{0.733673in}{0.522473in}}%
\pgfpathlineto{\pgfqpoint{0.697164in}{0.517168in}}%
\pgfpathlineto{\pgfqpoint{0.723582in}{0.491416in}}%
\pgfpathlineto{\pgfqpoint{0.717345in}{0.455055in}}%
\pgfpathlineto{\pgfqpoint{0.750000in}{0.472222in}}%
\pgfpathlineto{\pgfqpoint{0.782655in}{0.455055in}}%
\pgfpathlineto{\pgfqpoint{0.776418in}{0.491416in}}%
\pgfpathlineto{\pgfqpoint{0.802836in}{0.517168in}}%
\pgfpathlineto{\pgfqpoint{0.766327in}{0.522473in}}%
\pgfpathlineto{\pgfqpoint{0.750000in}{0.555556in}}%
\pgfpathmoveto{\pgfqpoint{0.916667in}{0.555556in}}%
\pgfpathlineto{\pgfqpoint{0.900339in}{0.522473in}}%
\pgfpathlineto{\pgfqpoint{0.863830in}{0.517168in}}%
\pgfpathlineto{\pgfqpoint{0.890248in}{0.491416in}}%
\pgfpathlineto{\pgfqpoint{0.884012in}{0.455055in}}%
\pgfpathlineto{\pgfqpoint{0.916667in}{0.472222in}}%
\pgfpathlineto{\pgfqpoint{0.949321in}{0.455055in}}%
\pgfpathlineto{\pgfqpoint{0.943085in}{0.491416in}}%
\pgfpathlineto{\pgfqpoint{0.969503in}{0.517168in}}%
\pgfpathlineto{\pgfqpoint{0.932994in}{0.522473in}}%
\pgfpathlineto{\pgfqpoint{0.916667in}{0.555556in}}%
\pgfpathmoveto{\pgfqpoint{0.000000in}{0.722222in}}%
\pgfpathlineto{\pgfqpoint{-0.016327in}{0.689139in}}%
\pgfpathlineto{\pgfqpoint{-0.052836in}{0.683834in}}%
\pgfpathlineto{\pgfqpoint{-0.026418in}{0.658083in}}%
\pgfpathlineto{\pgfqpoint{-0.032655in}{0.621721in}}%
\pgfpathlineto{\pgfqpoint{-0.000000in}{0.638889in}}%
\pgfpathlineto{\pgfqpoint{0.032655in}{0.621721in}}%
\pgfpathlineto{\pgfqpoint{0.026418in}{0.658083in}}%
\pgfpathlineto{\pgfqpoint{0.052836in}{0.683834in}}%
\pgfpathlineto{\pgfqpoint{0.016327in}{0.689139in}}%
\pgfpathlineto{\pgfqpoint{0.000000in}{0.722222in}}%
\pgfpathmoveto{\pgfqpoint{0.166667in}{0.722222in}}%
\pgfpathlineto{\pgfqpoint{0.150339in}{0.689139in}}%
\pgfpathlineto{\pgfqpoint{0.113830in}{0.683834in}}%
\pgfpathlineto{\pgfqpoint{0.140248in}{0.658083in}}%
\pgfpathlineto{\pgfqpoint{0.134012in}{0.621721in}}%
\pgfpathlineto{\pgfqpoint{0.166667in}{0.638889in}}%
\pgfpathlineto{\pgfqpoint{0.199321in}{0.621721in}}%
\pgfpathlineto{\pgfqpoint{0.193085in}{0.658083in}}%
\pgfpathlineto{\pgfqpoint{0.219503in}{0.683834in}}%
\pgfpathlineto{\pgfqpoint{0.182994in}{0.689139in}}%
\pgfpathlineto{\pgfqpoint{0.166667in}{0.722222in}}%
\pgfpathmoveto{\pgfqpoint{0.333333in}{0.722222in}}%
\pgfpathlineto{\pgfqpoint{0.317006in}{0.689139in}}%
\pgfpathlineto{\pgfqpoint{0.280497in}{0.683834in}}%
\pgfpathlineto{\pgfqpoint{0.306915in}{0.658083in}}%
\pgfpathlineto{\pgfqpoint{0.300679in}{0.621721in}}%
\pgfpathlineto{\pgfqpoint{0.333333in}{0.638889in}}%
\pgfpathlineto{\pgfqpoint{0.365988in}{0.621721in}}%
\pgfpathlineto{\pgfqpoint{0.359752in}{0.658083in}}%
\pgfpathlineto{\pgfqpoint{0.386170in}{0.683834in}}%
\pgfpathlineto{\pgfqpoint{0.349661in}{0.689139in}}%
\pgfpathlineto{\pgfqpoint{0.333333in}{0.722222in}}%
\pgfpathmoveto{\pgfqpoint{0.500000in}{0.722222in}}%
\pgfpathlineto{\pgfqpoint{0.483673in}{0.689139in}}%
\pgfpathlineto{\pgfqpoint{0.447164in}{0.683834in}}%
\pgfpathlineto{\pgfqpoint{0.473582in}{0.658083in}}%
\pgfpathlineto{\pgfqpoint{0.467345in}{0.621721in}}%
\pgfpathlineto{\pgfqpoint{0.500000in}{0.638889in}}%
\pgfpathlineto{\pgfqpoint{0.532655in}{0.621721in}}%
\pgfpathlineto{\pgfqpoint{0.526418in}{0.658083in}}%
\pgfpathlineto{\pgfqpoint{0.552836in}{0.683834in}}%
\pgfpathlineto{\pgfqpoint{0.516327in}{0.689139in}}%
\pgfpathlineto{\pgfqpoint{0.500000in}{0.722222in}}%
\pgfpathmoveto{\pgfqpoint{0.666667in}{0.722222in}}%
\pgfpathlineto{\pgfqpoint{0.650339in}{0.689139in}}%
\pgfpathlineto{\pgfqpoint{0.613830in}{0.683834in}}%
\pgfpathlineto{\pgfqpoint{0.640248in}{0.658083in}}%
\pgfpathlineto{\pgfqpoint{0.634012in}{0.621721in}}%
\pgfpathlineto{\pgfqpoint{0.666667in}{0.638889in}}%
\pgfpathlineto{\pgfqpoint{0.699321in}{0.621721in}}%
\pgfpathlineto{\pgfqpoint{0.693085in}{0.658083in}}%
\pgfpathlineto{\pgfqpoint{0.719503in}{0.683834in}}%
\pgfpathlineto{\pgfqpoint{0.682994in}{0.689139in}}%
\pgfpathlineto{\pgfqpoint{0.666667in}{0.722222in}}%
\pgfpathmoveto{\pgfqpoint{0.833333in}{0.722222in}}%
\pgfpathlineto{\pgfqpoint{0.817006in}{0.689139in}}%
\pgfpathlineto{\pgfqpoint{0.780497in}{0.683834in}}%
\pgfpathlineto{\pgfqpoint{0.806915in}{0.658083in}}%
\pgfpathlineto{\pgfqpoint{0.800679in}{0.621721in}}%
\pgfpathlineto{\pgfqpoint{0.833333in}{0.638889in}}%
\pgfpathlineto{\pgfqpoint{0.865988in}{0.621721in}}%
\pgfpathlineto{\pgfqpoint{0.859752in}{0.658083in}}%
\pgfpathlineto{\pgfqpoint{0.886170in}{0.683834in}}%
\pgfpathlineto{\pgfqpoint{0.849661in}{0.689139in}}%
\pgfpathlineto{\pgfqpoint{0.833333in}{0.722222in}}%
\pgfpathmoveto{\pgfqpoint{1.000000in}{0.722222in}}%
\pgfpathlineto{\pgfqpoint{0.983673in}{0.689139in}}%
\pgfpathlineto{\pgfqpoint{0.947164in}{0.683834in}}%
\pgfpathlineto{\pgfqpoint{0.973582in}{0.658083in}}%
\pgfpathlineto{\pgfqpoint{0.967345in}{0.621721in}}%
\pgfpathlineto{\pgfqpoint{1.000000in}{0.638889in}}%
\pgfpathlineto{\pgfqpoint{1.032655in}{0.621721in}}%
\pgfpathlineto{\pgfqpoint{1.026418in}{0.658083in}}%
\pgfpathlineto{\pgfqpoint{1.052836in}{0.683834in}}%
\pgfpathlineto{\pgfqpoint{1.016327in}{0.689139in}}%
\pgfpathlineto{\pgfqpoint{1.000000in}{0.722222in}}%
\pgfpathmoveto{\pgfqpoint{0.083333in}{0.888889in}}%
\pgfpathlineto{\pgfqpoint{0.067006in}{0.855806in}}%
\pgfpathlineto{\pgfqpoint{0.030497in}{0.850501in}}%
\pgfpathlineto{\pgfqpoint{0.056915in}{0.824750in}}%
\pgfpathlineto{\pgfqpoint{0.050679in}{0.788388in}}%
\pgfpathlineto{\pgfqpoint{0.083333in}{0.805556in}}%
\pgfpathlineto{\pgfqpoint{0.115988in}{0.788388in}}%
\pgfpathlineto{\pgfqpoint{0.109752in}{0.824750in}}%
\pgfpathlineto{\pgfqpoint{0.136170in}{0.850501in}}%
\pgfpathlineto{\pgfqpoint{0.099661in}{0.855806in}}%
\pgfpathlineto{\pgfqpoint{0.083333in}{0.888889in}}%
\pgfpathmoveto{\pgfqpoint{0.250000in}{0.888889in}}%
\pgfpathlineto{\pgfqpoint{0.233673in}{0.855806in}}%
\pgfpathlineto{\pgfqpoint{0.197164in}{0.850501in}}%
\pgfpathlineto{\pgfqpoint{0.223582in}{0.824750in}}%
\pgfpathlineto{\pgfqpoint{0.217345in}{0.788388in}}%
\pgfpathlineto{\pgfqpoint{0.250000in}{0.805556in}}%
\pgfpathlineto{\pgfqpoint{0.282655in}{0.788388in}}%
\pgfpathlineto{\pgfqpoint{0.276418in}{0.824750in}}%
\pgfpathlineto{\pgfqpoint{0.302836in}{0.850501in}}%
\pgfpathlineto{\pgfqpoint{0.266327in}{0.855806in}}%
\pgfpathlineto{\pgfqpoint{0.250000in}{0.888889in}}%
\pgfpathmoveto{\pgfqpoint{0.416667in}{0.888889in}}%
\pgfpathlineto{\pgfqpoint{0.400339in}{0.855806in}}%
\pgfpathlineto{\pgfqpoint{0.363830in}{0.850501in}}%
\pgfpathlineto{\pgfqpoint{0.390248in}{0.824750in}}%
\pgfpathlineto{\pgfqpoint{0.384012in}{0.788388in}}%
\pgfpathlineto{\pgfqpoint{0.416667in}{0.805556in}}%
\pgfpathlineto{\pgfqpoint{0.449321in}{0.788388in}}%
\pgfpathlineto{\pgfqpoint{0.443085in}{0.824750in}}%
\pgfpathlineto{\pgfqpoint{0.469503in}{0.850501in}}%
\pgfpathlineto{\pgfqpoint{0.432994in}{0.855806in}}%
\pgfpathlineto{\pgfqpoint{0.416667in}{0.888889in}}%
\pgfpathmoveto{\pgfqpoint{0.583333in}{0.888889in}}%
\pgfpathlineto{\pgfqpoint{0.567006in}{0.855806in}}%
\pgfpathlineto{\pgfqpoint{0.530497in}{0.850501in}}%
\pgfpathlineto{\pgfqpoint{0.556915in}{0.824750in}}%
\pgfpathlineto{\pgfqpoint{0.550679in}{0.788388in}}%
\pgfpathlineto{\pgfqpoint{0.583333in}{0.805556in}}%
\pgfpathlineto{\pgfqpoint{0.615988in}{0.788388in}}%
\pgfpathlineto{\pgfqpoint{0.609752in}{0.824750in}}%
\pgfpathlineto{\pgfqpoint{0.636170in}{0.850501in}}%
\pgfpathlineto{\pgfqpoint{0.599661in}{0.855806in}}%
\pgfpathlineto{\pgfqpoint{0.583333in}{0.888889in}}%
\pgfpathmoveto{\pgfqpoint{0.750000in}{0.888889in}}%
\pgfpathlineto{\pgfqpoint{0.733673in}{0.855806in}}%
\pgfpathlineto{\pgfqpoint{0.697164in}{0.850501in}}%
\pgfpathlineto{\pgfqpoint{0.723582in}{0.824750in}}%
\pgfpathlineto{\pgfqpoint{0.717345in}{0.788388in}}%
\pgfpathlineto{\pgfqpoint{0.750000in}{0.805556in}}%
\pgfpathlineto{\pgfqpoint{0.782655in}{0.788388in}}%
\pgfpathlineto{\pgfqpoint{0.776418in}{0.824750in}}%
\pgfpathlineto{\pgfqpoint{0.802836in}{0.850501in}}%
\pgfpathlineto{\pgfqpoint{0.766327in}{0.855806in}}%
\pgfpathlineto{\pgfqpoint{0.750000in}{0.888889in}}%
\pgfpathmoveto{\pgfqpoint{0.916667in}{0.888889in}}%
\pgfpathlineto{\pgfqpoint{0.900339in}{0.855806in}}%
\pgfpathlineto{\pgfqpoint{0.863830in}{0.850501in}}%
\pgfpathlineto{\pgfqpoint{0.890248in}{0.824750in}}%
\pgfpathlineto{\pgfqpoint{0.884012in}{0.788388in}}%
\pgfpathlineto{\pgfqpoint{0.916667in}{0.805556in}}%
\pgfpathlineto{\pgfqpoint{0.949321in}{0.788388in}}%
\pgfpathlineto{\pgfqpoint{0.943085in}{0.824750in}}%
\pgfpathlineto{\pgfqpoint{0.969503in}{0.850501in}}%
\pgfpathlineto{\pgfqpoint{0.932994in}{0.855806in}}%
\pgfpathlineto{\pgfqpoint{0.916667in}{0.888889in}}%
\pgfpathmoveto{\pgfqpoint{0.000000in}{1.055556in}}%
\pgfpathlineto{\pgfqpoint{-0.016327in}{1.022473in}}%
\pgfpathlineto{\pgfqpoint{-0.052836in}{1.017168in}}%
\pgfpathlineto{\pgfqpoint{-0.026418in}{0.991416in}}%
\pgfpathlineto{\pgfqpoint{-0.032655in}{0.955055in}}%
\pgfpathlineto{\pgfqpoint{-0.000000in}{0.972222in}}%
\pgfpathlineto{\pgfqpoint{0.032655in}{0.955055in}}%
\pgfpathlineto{\pgfqpoint{0.026418in}{0.991416in}}%
\pgfpathlineto{\pgfqpoint{0.052836in}{1.017168in}}%
\pgfpathlineto{\pgfqpoint{0.016327in}{1.022473in}}%
\pgfpathlineto{\pgfqpoint{0.000000in}{1.055556in}}%
\pgfpathmoveto{\pgfqpoint{0.166667in}{1.055556in}}%
\pgfpathlineto{\pgfqpoint{0.150339in}{1.022473in}}%
\pgfpathlineto{\pgfqpoint{0.113830in}{1.017168in}}%
\pgfpathlineto{\pgfqpoint{0.140248in}{0.991416in}}%
\pgfpathlineto{\pgfqpoint{0.134012in}{0.955055in}}%
\pgfpathlineto{\pgfqpoint{0.166667in}{0.972222in}}%
\pgfpathlineto{\pgfqpoint{0.199321in}{0.955055in}}%
\pgfpathlineto{\pgfqpoint{0.193085in}{0.991416in}}%
\pgfpathlineto{\pgfqpoint{0.219503in}{1.017168in}}%
\pgfpathlineto{\pgfqpoint{0.182994in}{1.022473in}}%
\pgfpathlineto{\pgfqpoint{0.166667in}{1.055556in}}%
\pgfpathmoveto{\pgfqpoint{0.333333in}{1.055556in}}%
\pgfpathlineto{\pgfqpoint{0.317006in}{1.022473in}}%
\pgfpathlineto{\pgfqpoint{0.280497in}{1.017168in}}%
\pgfpathlineto{\pgfqpoint{0.306915in}{0.991416in}}%
\pgfpathlineto{\pgfqpoint{0.300679in}{0.955055in}}%
\pgfpathlineto{\pgfqpoint{0.333333in}{0.972222in}}%
\pgfpathlineto{\pgfqpoint{0.365988in}{0.955055in}}%
\pgfpathlineto{\pgfqpoint{0.359752in}{0.991416in}}%
\pgfpathlineto{\pgfqpoint{0.386170in}{1.017168in}}%
\pgfpathlineto{\pgfqpoint{0.349661in}{1.022473in}}%
\pgfpathlineto{\pgfqpoint{0.333333in}{1.055556in}}%
\pgfpathmoveto{\pgfqpoint{0.500000in}{1.055556in}}%
\pgfpathlineto{\pgfqpoint{0.483673in}{1.022473in}}%
\pgfpathlineto{\pgfqpoint{0.447164in}{1.017168in}}%
\pgfpathlineto{\pgfqpoint{0.473582in}{0.991416in}}%
\pgfpathlineto{\pgfqpoint{0.467345in}{0.955055in}}%
\pgfpathlineto{\pgfqpoint{0.500000in}{0.972222in}}%
\pgfpathlineto{\pgfqpoint{0.532655in}{0.955055in}}%
\pgfpathlineto{\pgfqpoint{0.526418in}{0.991416in}}%
\pgfpathlineto{\pgfqpoint{0.552836in}{1.017168in}}%
\pgfpathlineto{\pgfqpoint{0.516327in}{1.022473in}}%
\pgfpathlineto{\pgfqpoint{0.500000in}{1.055556in}}%
\pgfpathmoveto{\pgfqpoint{0.666667in}{1.055556in}}%
\pgfpathlineto{\pgfqpoint{0.650339in}{1.022473in}}%
\pgfpathlineto{\pgfqpoint{0.613830in}{1.017168in}}%
\pgfpathlineto{\pgfqpoint{0.640248in}{0.991416in}}%
\pgfpathlineto{\pgfqpoint{0.634012in}{0.955055in}}%
\pgfpathlineto{\pgfqpoint{0.666667in}{0.972222in}}%
\pgfpathlineto{\pgfqpoint{0.699321in}{0.955055in}}%
\pgfpathlineto{\pgfqpoint{0.693085in}{0.991416in}}%
\pgfpathlineto{\pgfqpoint{0.719503in}{1.017168in}}%
\pgfpathlineto{\pgfqpoint{0.682994in}{1.022473in}}%
\pgfpathlineto{\pgfqpoint{0.666667in}{1.055556in}}%
\pgfpathmoveto{\pgfqpoint{0.833333in}{1.055556in}}%
\pgfpathlineto{\pgfqpoint{0.817006in}{1.022473in}}%
\pgfpathlineto{\pgfqpoint{0.780497in}{1.017168in}}%
\pgfpathlineto{\pgfqpoint{0.806915in}{0.991416in}}%
\pgfpathlineto{\pgfqpoint{0.800679in}{0.955055in}}%
\pgfpathlineto{\pgfqpoint{0.833333in}{0.972222in}}%
\pgfpathlineto{\pgfqpoint{0.865988in}{0.955055in}}%
\pgfpathlineto{\pgfqpoint{0.859752in}{0.991416in}}%
\pgfpathlineto{\pgfqpoint{0.886170in}{1.017168in}}%
\pgfpathlineto{\pgfqpoint{0.849661in}{1.022473in}}%
\pgfpathlineto{\pgfqpoint{0.833333in}{1.055556in}}%
\pgfpathmoveto{\pgfqpoint{1.000000in}{1.055556in}}%
\pgfpathlineto{\pgfqpoint{0.983673in}{1.022473in}}%
\pgfpathlineto{\pgfqpoint{0.947164in}{1.017168in}}%
\pgfpathlineto{\pgfqpoint{0.973582in}{0.991416in}}%
\pgfpathlineto{\pgfqpoint{0.967345in}{0.955055in}}%
\pgfpathlineto{\pgfqpoint{1.000000in}{0.972222in}}%
\pgfpathlineto{\pgfqpoint{1.032655in}{0.955055in}}%
\pgfpathlineto{\pgfqpoint{1.026418in}{0.991416in}}%
\pgfpathlineto{\pgfqpoint{1.052836in}{1.017168in}}%
\pgfpathlineto{\pgfqpoint{1.016327in}{1.022473in}}%
\pgfpathlineto{\pgfqpoint{1.000000in}{1.055556in}}%
\pgfpathlineto{\pgfqpoint{1.000000in}{1.055556in}}%
\pgfusepath{stroke}%
\end{pgfscope}%
}%
\pgfsys@transformshift{7.523315in}{5.549528in}%
\pgfsys@useobject{currentpattern}{}%
\pgfsys@transformshift{1in}{0in}%
\pgfsys@transformshift{-1in}{0in}%
\pgfsys@transformshift{0in}{1in}%
\pgfsys@useobject{currentpattern}{}%
\pgfsys@transformshift{1in}{0in}%
\pgfsys@transformshift{-1in}{0in}%
\pgfsys@transformshift{0in}{1in}%
\end{pgfscope}%
\begin{pgfscope}%
\pgfpathrectangle{\pgfqpoint{0.935815in}{0.637495in}}{\pgfqpoint{9.300000in}{9.060000in}}%
\pgfusepath{clip}%
\pgfsetbuttcap%
\pgfsetmiterjoin%
\definecolor{currentfill}{rgb}{1.000000,1.000000,0.000000}%
\pgfsetfillcolor{currentfill}%
\pgfsetfillopacity{0.990000}%
\pgfsetlinewidth{0.000000pt}%
\definecolor{currentstroke}{rgb}{0.000000,0.000000,0.000000}%
\pgfsetstrokecolor{currentstroke}%
\pgfsetstrokeopacity{0.990000}%
\pgfsetdash{}{0pt}%
\pgfpathmoveto{\pgfqpoint{9.073315in}{6.003887in}}%
\pgfpathlineto{\pgfqpoint{9.848315in}{6.003887in}}%
\pgfpathlineto{\pgfqpoint{9.848315in}{7.186823in}}%
\pgfpathlineto{\pgfqpoint{9.073315in}{7.186823in}}%
\pgfpathclose%
\pgfusepath{fill}%
\end{pgfscope}%
\begin{pgfscope}%
\pgfsetbuttcap%
\pgfsetmiterjoin%
\definecolor{currentfill}{rgb}{1.000000,1.000000,0.000000}%
\pgfsetfillcolor{currentfill}%
\pgfsetfillopacity{0.990000}%
\pgfsetlinewidth{0.000000pt}%
\definecolor{currentstroke}{rgb}{0.000000,0.000000,0.000000}%
\pgfsetstrokecolor{currentstroke}%
\pgfsetstrokeopacity{0.990000}%
\pgfsetdash{}{0pt}%
\pgfpathrectangle{\pgfqpoint{0.935815in}{0.637495in}}{\pgfqpoint{9.300000in}{9.060000in}}%
\pgfusepath{clip}%
\pgfpathmoveto{\pgfqpoint{9.073315in}{6.003887in}}%
\pgfpathlineto{\pgfqpoint{9.848315in}{6.003887in}}%
\pgfpathlineto{\pgfqpoint{9.848315in}{7.186823in}}%
\pgfpathlineto{\pgfqpoint{9.073315in}{7.186823in}}%
\pgfpathclose%
\pgfusepath{clip}%
\pgfsys@defobject{currentpattern}{\pgfqpoint{0in}{0in}}{\pgfqpoint{1in}{1in}}{%
\begin{pgfscope}%
\pgfpathrectangle{\pgfqpoint{0in}{0in}}{\pgfqpoint{1in}{1in}}%
\pgfusepath{clip}%
\pgfpathmoveto{\pgfqpoint{0.000000in}{0.055556in}}%
\pgfpathlineto{\pgfqpoint{-0.016327in}{0.022473in}}%
\pgfpathlineto{\pgfqpoint{-0.052836in}{0.017168in}}%
\pgfpathlineto{\pgfqpoint{-0.026418in}{-0.008584in}}%
\pgfpathlineto{\pgfqpoint{-0.032655in}{-0.044945in}}%
\pgfpathlineto{\pgfqpoint{-0.000000in}{-0.027778in}}%
\pgfpathlineto{\pgfqpoint{0.032655in}{-0.044945in}}%
\pgfpathlineto{\pgfqpoint{0.026418in}{-0.008584in}}%
\pgfpathlineto{\pgfqpoint{0.052836in}{0.017168in}}%
\pgfpathlineto{\pgfqpoint{0.016327in}{0.022473in}}%
\pgfpathlineto{\pgfqpoint{0.000000in}{0.055556in}}%
\pgfpathmoveto{\pgfqpoint{0.166667in}{0.055556in}}%
\pgfpathlineto{\pgfqpoint{0.150339in}{0.022473in}}%
\pgfpathlineto{\pgfqpoint{0.113830in}{0.017168in}}%
\pgfpathlineto{\pgfqpoint{0.140248in}{-0.008584in}}%
\pgfpathlineto{\pgfqpoint{0.134012in}{-0.044945in}}%
\pgfpathlineto{\pgfqpoint{0.166667in}{-0.027778in}}%
\pgfpathlineto{\pgfqpoint{0.199321in}{-0.044945in}}%
\pgfpathlineto{\pgfqpoint{0.193085in}{-0.008584in}}%
\pgfpathlineto{\pgfqpoint{0.219503in}{0.017168in}}%
\pgfpathlineto{\pgfqpoint{0.182994in}{0.022473in}}%
\pgfpathlineto{\pgfqpoint{0.166667in}{0.055556in}}%
\pgfpathmoveto{\pgfqpoint{0.333333in}{0.055556in}}%
\pgfpathlineto{\pgfqpoint{0.317006in}{0.022473in}}%
\pgfpathlineto{\pgfqpoint{0.280497in}{0.017168in}}%
\pgfpathlineto{\pgfqpoint{0.306915in}{-0.008584in}}%
\pgfpathlineto{\pgfqpoint{0.300679in}{-0.044945in}}%
\pgfpathlineto{\pgfqpoint{0.333333in}{-0.027778in}}%
\pgfpathlineto{\pgfqpoint{0.365988in}{-0.044945in}}%
\pgfpathlineto{\pgfqpoint{0.359752in}{-0.008584in}}%
\pgfpathlineto{\pgfqpoint{0.386170in}{0.017168in}}%
\pgfpathlineto{\pgfqpoint{0.349661in}{0.022473in}}%
\pgfpathlineto{\pgfqpoint{0.333333in}{0.055556in}}%
\pgfpathmoveto{\pgfqpoint{0.500000in}{0.055556in}}%
\pgfpathlineto{\pgfqpoint{0.483673in}{0.022473in}}%
\pgfpathlineto{\pgfqpoint{0.447164in}{0.017168in}}%
\pgfpathlineto{\pgfqpoint{0.473582in}{-0.008584in}}%
\pgfpathlineto{\pgfqpoint{0.467345in}{-0.044945in}}%
\pgfpathlineto{\pgfqpoint{0.500000in}{-0.027778in}}%
\pgfpathlineto{\pgfqpoint{0.532655in}{-0.044945in}}%
\pgfpathlineto{\pgfqpoint{0.526418in}{-0.008584in}}%
\pgfpathlineto{\pgfqpoint{0.552836in}{0.017168in}}%
\pgfpathlineto{\pgfqpoint{0.516327in}{0.022473in}}%
\pgfpathlineto{\pgfqpoint{0.500000in}{0.055556in}}%
\pgfpathmoveto{\pgfqpoint{0.666667in}{0.055556in}}%
\pgfpathlineto{\pgfqpoint{0.650339in}{0.022473in}}%
\pgfpathlineto{\pgfqpoint{0.613830in}{0.017168in}}%
\pgfpathlineto{\pgfqpoint{0.640248in}{-0.008584in}}%
\pgfpathlineto{\pgfqpoint{0.634012in}{-0.044945in}}%
\pgfpathlineto{\pgfqpoint{0.666667in}{-0.027778in}}%
\pgfpathlineto{\pgfqpoint{0.699321in}{-0.044945in}}%
\pgfpathlineto{\pgfqpoint{0.693085in}{-0.008584in}}%
\pgfpathlineto{\pgfqpoint{0.719503in}{0.017168in}}%
\pgfpathlineto{\pgfqpoint{0.682994in}{0.022473in}}%
\pgfpathlineto{\pgfqpoint{0.666667in}{0.055556in}}%
\pgfpathmoveto{\pgfqpoint{0.833333in}{0.055556in}}%
\pgfpathlineto{\pgfqpoint{0.817006in}{0.022473in}}%
\pgfpathlineto{\pgfqpoint{0.780497in}{0.017168in}}%
\pgfpathlineto{\pgfqpoint{0.806915in}{-0.008584in}}%
\pgfpathlineto{\pgfqpoint{0.800679in}{-0.044945in}}%
\pgfpathlineto{\pgfqpoint{0.833333in}{-0.027778in}}%
\pgfpathlineto{\pgfqpoint{0.865988in}{-0.044945in}}%
\pgfpathlineto{\pgfqpoint{0.859752in}{-0.008584in}}%
\pgfpathlineto{\pgfqpoint{0.886170in}{0.017168in}}%
\pgfpathlineto{\pgfqpoint{0.849661in}{0.022473in}}%
\pgfpathlineto{\pgfqpoint{0.833333in}{0.055556in}}%
\pgfpathmoveto{\pgfqpoint{1.000000in}{0.055556in}}%
\pgfpathlineto{\pgfqpoint{0.983673in}{0.022473in}}%
\pgfpathlineto{\pgfqpoint{0.947164in}{0.017168in}}%
\pgfpathlineto{\pgfqpoint{0.973582in}{-0.008584in}}%
\pgfpathlineto{\pgfqpoint{0.967345in}{-0.044945in}}%
\pgfpathlineto{\pgfqpoint{1.000000in}{-0.027778in}}%
\pgfpathlineto{\pgfqpoint{1.032655in}{-0.044945in}}%
\pgfpathlineto{\pgfqpoint{1.026418in}{-0.008584in}}%
\pgfpathlineto{\pgfqpoint{1.052836in}{0.017168in}}%
\pgfpathlineto{\pgfqpoint{1.016327in}{0.022473in}}%
\pgfpathlineto{\pgfqpoint{1.000000in}{0.055556in}}%
\pgfpathmoveto{\pgfqpoint{0.083333in}{0.222222in}}%
\pgfpathlineto{\pgfqpoint{0.067006in}{0.189139in}}%
\pgfpathlineto{\pgfqpoint{0.030497in}{0.183834in}}%
\pgfpathlineto{\pgfqpoint{0.056915in}{0.158083in}}%
\pgfpathlineto{\pgfqpoint{0.050679in}{0.121721in}}%
\pgfpathlineto{\pgfqpoint{0.083333in}{0.138889in}}%
\pgfpathlineto{\pgfqpoint{0.115988in}{0.121721in}}%
\pgfpathlineto{\pgfqpoint{0.109752in}{0.158083in}}%
\pgfpathlineto{\pgfqpoint{0.136170in}{0.183834in}}%
\pgfpathlineto{\pgfqpoint{0.099661in}{0.189139in}}%
\pgfpathlineto{\pgfqpoint{0.083333in}{0.222222in}}%
\pgfpathmoveto{\pgfqpoint{0.250000in}{0.222222in}}%
\pgfpathlineto{\pgfqpoint{0.233673in}{0.189139in}}%
\pgfpathlineto{\pgfqpoint{0.197164in}{0.183834in}}%
\pgfpathlineto{\pgfqpoint{0.223582in}{0.158083in}}%
\pgfpathlineto{\pgfqpoint{0.217345in}{0.121721in}}%
\pgfpathlineto{\pgfqpoint{0.250000in}{0.138889in}}%
\pgfpathlineto{\pgfqpoint{0.282655in}{0.121721in}}%
\pgfpathlineto{\pgfqpoint{0.276418in}{0.158083in}}%
\pgfpathlineto{\pgfqpoint{0.302836in}{0.183834in}}%
\pgfpathlineto{\pgfqpoint{0.266327in}{0.189139in}}%
\pgfpathlineto{\pgfqpoint{0.250000in}{0.222222in}}%
\pgfpathmoveto{\pgfqpoint{0.416667in}{0.222222in}}%
\pgfpathlineto{\pgfqpoint{0.400339in}{0.189139in}}%
\pgfpathlineto{\pgfqpoint{0.363830in}{0.183834in}}%
\pgfpathlineto{\pgfqpoint{0.390248in}{0.158083in}}%
\pgfpathlineto{\pgfqpoint{0.384012in}{0.121721in}}%
\pgfpathlineto{\pgfqpoint{0.416667in}{0.138889in}}%
\pgfpathlineto{\pgfqpoint{0.449321in}{0.121721in}}%
\pgfpathlineto{\pgfqpoint{0.443085in}{0.158083in}}%
\pgfpathlineto{\pgfqpoint{0.469503in}{0.183834in}}%
\pgfpathlineto{\pgfqpoint{0.432994in}{0.189139in}}%
\pgfpathlineto{\pgfqpoint{0.416667in}{0.222222in}}%
\pgfpathmoveto{\pgfqpoint{0.583333in}{0.222222in}}%
\pgfpathlineto{\pgfqpoint{0.567006in}{0.189139in}}%
\pgfpathlineto{\pgfqpoint{0.530497in}{0.183834in}}%
\pgfpathlineto{\pgfqpoint{0.556915in}{0.158083in}}%
\pgfpathlineto{\pgfqpoint{0.550679in}{0.121721in}}%
\pgfpathlineto{\pgfqpoint{0.583333in}{0.138889in}}%
\pgfpathlineto{\pgfqpoint{0.615988in}{0.121721in}}%
\pgfpathlineto{\pgfqpoint{0.609752in}{0.158083in}}%
\pgfpathlineto{\pgfqpoint{0.636170in}{0.183834in}}%
\pgfpathlineto{\pgfqpoint{0.599661in}{0.189139in}}%
\pgfpathlineto{\pgfqpoint{0.583333in}{0.222222in}}%
\pgfpathmoveto{\pgfqpoint{0.750000in}{0.222222in}}%
\pgfpathlineto{\pgfqpoint{0.733673in}{0.189139in}}%
\pgfpathlineto{\pgfqpoint{0.697164in}{0.183834in}}%
\pgfpathlineto{\pgfqpoint{0.723582in}{0.158083in}}%
\pgfpathlineto{\pgfqpoint{0.717345in}{0.121721in}}%
\pgfpathlineto{\pgfqpoint{0.750000in}{0.138889in}}%
\pgfpathlineto{\pgfqpoint{0.782655in}{0.121721in}}%
\pgfpathlineto{\pgfqpoint{0.776418in}{0.158083in}}%
\pgfpathlineto{\pgfqpoint{0.802836in}{0.183834in}}%
\pgfpathlineto{\pgfqpoint{0.766327in}{0.189139in}}%
\pgfpathlineto{\pgfqpoint{0.750000in}{0.222222in}}%
\pgfpathmoveto{\pgfqpoint{0.916667in}{0.222222in}}%
\pgfpathlineto{\pgfqpoint{0.900339in}{0.189139in}}%
\pgfpathlineto{\pgfqpoint{0.863830in}{0.183834in}}%
\pgfpathlineto{\pgfqpoint{0.890248in}{0.158083in}}%
\pgfpathlineto{\pgfqpoint{0.884012in}{0.121721in}}%
\pgfpathlineto{\pgfqpoint{0.916667in}{0.138889in}}%
\pgfpathlineto{\pgfqpoint{0.949321in}{0.121721in}}%
\pgfpathlineto{\pgfqpoint{0.943085in}{0.158083in}}%
\pgfpathlineto{\pgfqpoint{0.969503in}{0.183834in}}%
\pgfpathlineto{\pgfqpoint{0.932994in}{0.189139in}}%
\pgfpathlineto{\pgfqpoint{0.916667in}{0.222222in}}%
\pgfpathmoveto{\pgfqpoint{0.000000in}{0.388889in}}%
\pgfpathlineto{\pgfqpoint{-0.016327in}{0.355806in}}%
\pgfpathlineto{\pgfqpoint{-0.052836in}{0.350501in}}%
\pgfpathlineto{\pgfqpoint{-0.026418in}{0.324750in}}%
\pgfpathlineto{\pgfqpoint{-0.032655in}{0.288388in}}%
\pgfpathlineto{\pgfqpoint{-0.000000in}{0.305556in}}%
\pgfpathlineto{\pgfqpoint{0.032655in}{0.288388in}}%
\pgfpathlineto{\pgfqpoint{0.026418in}{0.324750in}}%
\pgfpathlineto{\pgfqpoint{0.052836in}{0.350501in}}%
\pgfpathlineto{\pgfqpoint{0.016327in}{0.355806in}}%
\pgfpathlineto{\pgfqpoint{0.000000in}{0.388889in}}%
\pgfpathmoveto{\pgfqpoint{0.166667in}{0.388889in}}%
\pgfpathlineto{\pgfqpoint{0.150339in}{0.355806in}}%
\pgfpathlineto{\pgfqpoint{0.113830in}{0.350501in}}%
\pgfpathlineto{\pgfqpoint{0.140248in}{0.324750in}}%
\pgfpathlineto{\pgfqpoint{0.134012in}{0.288388in}}%
\pgfpathlineto{\pgfqpoint{0.166667in}{0.305556in}}%
\pgfpathlineto{\pgfqpoint{0.199321in}{0.288388in}}%
\pgfpathlineto{\pgfqpoint{0.193085in}{0.324750in}}%
\pgfpathlineto{\pgfqpoint{0.219503in}{0.350501in}}%
\pgfpathlineto{\pgfqpoint{0.182994in}{0.355806in}}%
\pgfpathlineto{\pgfqpoint{0.166667in}{0.388889in}}%
\pgfpathmoveto{\pgfqpoint{0.333333in}{0.388889in}}%
\pgfpathlineto{\pgfqpoint{0.317006in}{0.355806in}}%
\pgfpathlineto{\pgfqpoint{0.280497in}{0.350501in}}%
\pgfpathlineto{\pgfqpoint{0.306915in}{0.324750in}}%
\pgfpathlineto{\pgfqpoint{0.300679in}{0.288388in}}%
\pgfpathlineto{\pgfqpoint{0.333333in}{0.305556in}}%
\pgfpathlineto{\pgfqpoint{0.365988in}{0.288388in}}%
\pgfpathlineto{\pgfqpoint{0.359752in}{0.324750in}}%
\pgfpathlineto{\pgfqpoint{0.386170in}{0.350501in}}%
\pgfpathlineto{\pgfqpoint{0.349661in}{0.355806in}}%
\pgfpathlineto{\pgfqpoint{0.333333in}{0.388889in}}%
\pgfpathmoveto{\pgfqpoint{0.500000in}{0.388889in}}%
\pgfpathlineto{\pgfqpoint{0.483673in}{0.355806in}}%
\pgfpathlineto{\pgfqpoint{0.447164in}{0.350501in}}%
\pgfpathlineto{\pgfqpoint{0.473582in}{0.324750in}}%
\pgfpathlineto{\pgfqpoint{0.467345in}{0.288388in}}%
\pgfpathlineto{\pgfqpoint{0.500000in}{0.305556in}}%
\pgfpathlineto{\pgfqpoint{0.532655in}{0.288388in}}%
\pgfpathlineto{\pgfqpoint{0.526418in}{0.324750in}}%
\pgfpathlineto{\pgfqpoint{0.552836in}{0.350501in}}%
\pgfpathlineto{\pgfqpoint{0.516327in}{0.355806in}}%
\pgfpathlineto{\pgfqpoint{0.500000in}{0.388889in}}%
\pgfpathmoveto{\pgfqpoint{0.666667in}{0.388889in}}%
\pgfpathlineto{\pgfqpoint{0.650339in}{0.355806in}}%
\pgfpathlineto{\pgfqpoint{0.613830in}{0.350501in}}%
\pgfpathlineto{\pgfqpoint{0.640248in}{0.324750in}}%
\pgfpathlineto{\pgfqpoint{0.634012in}{0.288388in}}%
\pgfpathlineto{\pgfqpoint{0.666667in}{0.305556in}}%
\pgfpathlineto{\pgfqpoint{0.699321in}{0.288388in}}%
\pgfpathlineto{\pgfqpoint{0.693085in}{0.324750in}}%
\pgfpathlineto{\pgfqpoint{0.719503in}{0.350501in}}%
\pgfpathlineto{\pgfqpoint{0.682994in}{0.355806in}}%
\pgfpathlineto{\pgfqpoint{0.666667in}{0.388889in}}%
\pgfpathmoveto{\pgfqpoint{0.833333in}{0.388889in}}%
\pgfpathlineto{\pgfqpoint{0.817006in}{0.355806in}}%
\pgfpathlineto{\pgfqpoint{0.780497in}{0.350501in}}%
\pgfpathlineto{\pgfqpoint{0.806915in}{0.324750in}}%
\pgfpathlineto{\pgfqpoint{0.800679in}{0.288388in}}%
\pgfpathlineto{\pgfqpoint{0.833333in}{0.305556in}}%
\pgfpathlineto{\pgfqpoint{0.865988in}{0.288388in}}%
\pgfpathlineto{\pgfqpoint{0.859752in}{0.324750in}}%
\pgfpathlineto{\pgfqpoint{0.886170in}{0.350501in}}%
\pgfpathlineto{\pgfqpoint{0.849661in}{0.355806in}}%
\pgfpathlineto{\pgfqpoint{0.833333in}{0.388889in}}%
\pgfpathmoveto{\pgfqpoint{1.000000in}{0.388889in}}%
\pgfpathlineto{\pgfqpoint{0.983673in}{0.355806in}}%
\pgfpathlineto{\pgfqpoint{0.947164in}{0.350501in}}%
\pgfpathlineto{\pgfqpoint{0.973582in}{0.324750in}}%
\pgfpathlineto{\pgfqpoint{0.967345in}{0.288388in}}%
\pgfpathlineto{\pgfqpoint{1.000000in}{0.305556in}}%
\pgfpathlineto{\pgfqpoint{1.032655in}{0.288388in}}%
\pgfpathlineto{\pgfqpoint{1.026418in}{0.324750in}}%
\pgfpathlineto{\pgfqpoint{1.052836in}{0.350501in}}%
\pgfpathlineto{\pgfqpoint{1.016327in}{0.355806in}}%
\pgfpathlineto{\pgfqpoint{1.000000in}{0.388889in}}%
\pgfpathmoveto{\pgfqpoint{0.083333in}{0.555556in}}%
\pgfpathlineto{\pgfqpoint{0.067006in}{0.522473in}}%
\pgfpathlineto{\pgfqpoint{0.030497in}{0.517168in}}%
\pgfpathlineto{\pgfqpoint{0.056915in}{0.491416in}}%
\pgfpathlineto{\pgfqpoint{0.050679in}{0.455055in}}%
\pgfpathlineto{\pgfqpoint{0.083333in}{0.472222in}}%
\pgfpathlineto{\pgfqpoint{0.115988in}{0.455055in}}%
\pgfpathlineto{\pgfqpoint{0.109752in}{0.491416in}}%
\pgfpathlineto{\pgfqpoint{0.136170in}{0.517168in}}%
\pgfpathlineto{\pgfqpoint{0.099661in}{0.522473in}}%
\pgfpathlineto{\pgfqpoint{0.083333in}{0.555556in}}%
\pgfpathmoveto{\pgfqpoint{0.250000in}{0.555556in}}%
\pgfpathlineto{\pgfqpoint{0.233673in}{0.522473in}}%
\pgfpathlineto{\pgfqpoint{0.197164in}{0.517168in}}%
\pgfpathlineto{\pgfqpoint{0.223582in}{0.491416in}}%
\pgfpathlineto{\pgfqpoint{0.217345in}{0.455055in}}%
\pgfpathlineto{\pgfqpoint{0.250000in}{0.472222in}}%
\pgfpathlineto{\pgfqpoint{0.282655in}{0.455055in}}%
\pgfpathlineto{\pgfqpoint{0.276418in}{0.491416in}}%
\pgfpathlineto{\pgfqpoint{0.302836in}{0.517168in}}%
\pgfpathlineto{\pgfqpoint{0.266327in}{0.522473in}}%
\pgfpathlineto{\pgfqpoint{0.250000in}{0.555556in}}%
\pgfpathmoveto{\pgfqpoint{0.416667in}{0.555556in}}%
\pgfpathlineto{\pgfqpoint{0.400339in}{0.522473in}}%
\pgfpathlineto{\pgfqpoint{0.363830in}{0.517168in}}%
\pgfpathlineto{\pgfqpoint{0.390248in}{0.491416in}}%
\pgfpathlineto{\pgfqpoint{0.384012in}{0.455055in}}%
\pgfpathlineto{\pgfqpoint{0.416667in}{0.472222in}}%
\pgfpathlineto{\pgfqpoint{0.449321in}{0.455055in}}%
\pgfpathlineto{\pgfqpoint{0.443085in}{0.491416in}}%
\pgfpathlineto{\pgfqpoint{0.469503in}{0.517168in}}%
\pgfpathlineto{\pgfqpoint{0.432994in}{0.522473in}}%
\pgfpathlineto{\pgfqpoint{0.416667in}{0.555556in}}%
\pgfpathmoveto{\pgfqpoint{0.583333in}{0.555556in}}%
\pgfpathlineto{\pgfqpoint{0.567006in}{0.522473in}}%
\pgfpathlineto{\pgfqpoint{0.530497in}{0.517168in}}%
\pgfpathlineto{\pgfqpoint{0.556915in}{0.491416in}}%
\pgfpathlineto{\pgfqpoint{0.550679in}{0.455055in}}%
\pgfpathlineto{\pgfqpoint{0.583333in}{0.472222in}}%
\pgfpathlineto{\pgfqpoint{0.615988in}{0.455055in}}%
\pgfpathlineto{\pgfqpoint{0.609752in}{0.491416in}}%
\pgfpathlineto{\pgfqpoint{0.636170in}{0.517168in}}%
\pgfpathlineto{\pgfqpoint{0.599661in}{0.522473in}}%
\pgfpathlineto{\pgfqpoint{0.583333in}{0.555556in}}%
\pgfpathmoveto{\pgfqpoint{0.750000in}{0.555556in}}%
\pgfpathlineto{\pgfqpoint{0.733673in}{0.522473in}}%
\pgfpathlineto{\pgfqpoint{0.697164in}{0.517168in}}%
\pgfpathlineto{\pgfqpoint{0.723582in}{0.491416in}}%
\pgfpathlineto{\pgfqpoint{0.717345in}{0.455055in}}%
\pgfpathlineto{\pgfqpoint{0.750000in}{0.472222in}}%
\pgfpathlineto{\pgfqpoint{0.782655in}{0.455055in}}%
\pgfpathlineto{\pgfqpoint{0.776418in}{0.491416in}}%
\pgfpathlineto{\pgfqpoint{0.802836in}{0.517168in}}%
\pgfpathlineto{\pgfqpoint{0.766327in}{0.522473in}}%
\pgfpathlineto{\pgfqpoint{0.750000in}{0.555556in}}%
\pgfpathmoveto{\pgfqpoint{0.916667in}{0.555556in}}%
\pgfpathlineto{\pgfqpoint{0.900339in}{0.522473in}}%
\pgfpathlineto{\pgfqpoint{0.863830in}{0.517168in}}%
\pgfpathlineto{\pgfqpoint{0.890248in}{0.491416in}}%
\pgfpathlineto{\pgfqpoint{0.884012in}{0.455055in}}%
\pgfpathlineto{\pgfqpoint{0.916667in}{0.472222in}}%
\pgfpathlineto{\pgfqpoint{0.949321in}{0.455055in}}%
\pgfpathlineto{\pgfqpoint{0.943085in}{0.491416in}}%
\pgfpathlineto{\pgfqpoint{0.969503in}{0.517168in}}%
\pgfpathlineto{\pgfqpoint{0.932994in}{0.522473in}}%
\pgfpathlineto{\pgfqpoint{0.916667in}{0.555556in}}%
\pgfpathmoveto{\pgfqpoint{0.000000in}{0.722222in}}%
\pgfpathlineto{\pgfqpoint{-0.016327in}{0.689139in}}%
\pgfpathlineto{\pgfqpoint{-0.052836in}{0.683834in}}%
\pgfpathlineto{\pgfqpoint{-0.026418in}{0.658083in}}%
\pgfpathlineto{\pgfqpoint{-0.032655in}{0.621721in}}%
\pgfpathlineto{\pgfqpoint{-0.000000in}{0.638889in}}%
\pgfpathlineto{\pgfqpoint{0.032655in}{0.621721in}}%
\pgfpathlineto{\pgfqpoint{0.026418in}{0.658083in}}%
\pgfpathlineto{\pgfqpoint{0.052836in}{0.683834in}}%
\pgfpathlineto{\pgfqpoint{0.016327in}{0.689139in}}%
\pgfpathlineto{\pgfqpoint{0.000000in}{0.722222in}}%
\pgfpathmoveto{\pgfqpoint{0.166667in}{0.722222in}}%
\pgfpathlineto{\pgfqpoint{0.150339in}{0.689139in}}%
\pgfpathlineto{\pgfqpoint{0.113830in}{0.683834in}}%
\pgfpathlineto{\pgfqpoint{0.140248in}{0.658083in}}%
\pgfpathlineto{\pgfqpoint{0.134012in}{0.621721in}}%
\pgfpathlineto{\pgfqpoint{0.166667in}{0.638889in}}%
\pgfpathlineto{\pgfqpoint{0.199321in}{0.621721in}}%
\pgfpathlineto{\pgfqpoint{0.193085in}{0.658083in}}%
\pgfpathlineto{\pgfqpoint{0.219503in}{0.683834in}}%
\pgfpathlineto{\pgfqpoint{0.182994in}{0.689139in}}%
\pgfpathlineto{\pgfqpoint{0.166667in}{0.722222in}}%
\pgfpathmoveto{\pgfqpoint{0.333333in}{0.722222in}}%
\pgfpathlineto{\pgfqpoint{0.317006in}{0.689139in}}%
\pgfpathlineto{\pgfqpoint{0.280497in}{0.683834in}}%
\pgfpathlineto{\pgfqpoint{0.306915in}{0.658083in}}%
\pgfpathlineto{\pgfqpoint{0.300679in}{0.621721in}}%
\pgfpathlineto{\pgfqpoint{0.333333in}{0.638889in}}%
\pgfpathlineto{\pgfqpoint{0.365988in}{0.621721in}}%
\pgfpathlineto{\pgfqpoint{0.359752in}{0.658083in}}%
\pgfpathlineto{\pgfqpoint{0.386170in}{0.683834in}}%
\pgfpathlineto{\pgfqpoint{0.349661in}{0.689139in}}%
\pgfpathlineto{\pgfqpoint{0.333333in}{0.722222in}}%
\pgfpathmoveto{\pgfqpoint{0.500000in}{0.722222in}}%
\pgfpathlineto{\pgfqpoint{0.483673in}{0.689139in}}%
\pgfpathlineto{\pgfqpoint{0.447164in}{0.683834in}}%
\pgfpathlineto{\pgfqpoint{0.473582in}{0.658083in}}%
\pgfpathlineto{\pgfqpoint{0.467345in}{0.621721in}}%
\pgfpathlineto{\pgfqpoint{0.500000in}{0.638889in}}%
\pgfpathlineto{\pgfqpoint{0.532655in}{0.621721in}}%
\pgfpathlineto{\pgfqpoint{0.526418in}{0.658083in}}%
\pgfpathlineto{\pgfqpoint{0.552836in}{0.683834in}}%
\pgfpathlineto{\pgfqpoint{0.516327in}{0.689139in}}%
\pgfpathlineto{\pgfqpoint{0.500000in}{0.722222in}}%
\pgfpathmoveto{\pgfqpoint{0.666667in}{0.722222in}}%
\pgfpathlineto{\pgfqpoint{0.650339in}{0.689139in}}%
\pgfpathlineto{\pgfqpoint{0.613830in}{0.683834in}}%
\pgfpathlineto{\pgfqpoint{0.640248in}{0.658083in}}%
\pgfpathlineto{\pgfqpoint{0.634012in}{0.621721in}}%
\pgfpathlineto{\pgfqpoint{0.666667in}{0.638889in}}%
\pgfpathlineto{\pgfqpoint{0.699321in}{0.621721in}}%
\pgfpathlineto{\pgfqpoint{0.693085in}{0.658083in}}%
\pgfpathlineto{\pgfqpoint{0.719503in}{0.683834in}}%
\pgfpathlineto{\pgfqpoint{0.682994in}{0.689139in}}%
\pgfpathlineto{\pgfqpoint{0.666667in}{0.722222in}}%
\pgfpathmoveto{\pgfqpoint{0.833333in}{0.722222in}}%
\pgfpathlineto{\pgfqpoint{0.817006in}{0.689139in}}%
\pgfpathlineto{\pgfqpoint{0.780497in}{0.683834in}}%
\pgfpathlineto{\pgfqpoint{0.806915in}{0.658083in}}%
\pgfpathlineto{\pgfqpoint{0.800679in}{0.621721in}}%
\pgfpathlineto{\pgfqpoint{0.833333in}{0.638889in}}%
\pgfpathlineto{\pgfqpoint{0.865988in}{0.621721in}}%
\pgfpathlineto{\pgfqpoint{0.859752in}{0.658083in}}%
\pgfpathlineto{\pgfqpoint{0.886170in}{0.683834in}}%
\pgfpathlineto{\pgfqpoint{0.849661in}{0.689139in}}%
\pgfpathlineto{\pgfqpoint{0.833333in}{0.722222in}}%
\pgfpathmoveto{\pgfqpoint{1.000000in}{0.722222in}}%
\pgfpathlineto{\pgfqpoint{0.983673in}{0.689139in}}%
\pgfpathlineto{\pgfqpoint{0.947164in}{0.683834in}}%
\pgfpathlineto{\pgfqpoint{0.973582in}{0.658083in}}%
\pgfpathlineto{\pgfqpoint{0.967345in}{0.621721in}}%
\pgfpathlineto{\pgfqpoint{1.000000in}{0.638889in}}%
\pgfpathlineto{\pgfqpoint{1.032655in}{0.621721in}}%
\pgfpathlineto{\pgfqpoint{1.026418in}{0.658083in}}%
\pgfpathlineto{\pgfqpoint{1.052836in}{0.683834in}}%
\pgfpathlineto{\pgfqpoint{1.016327in}{0.689139in}}%
\pgfpathlineto{\pgfqpoint{1.000000in}{0.722222in}}%
\pgfpathmoveto{\pgfqpoint{0.083333in}{0.888889in}}%
\pgfpathlineto{\pgfqpoint{0.067006in}{0.855806in}}%
\pgfpathlineto{\pgfqpoint{0.030497in}{0.850501in}}%
\pgfpathlineto{\pgfqpoint{0.056915in}{0.824750in}}%
\pgfpathlineto{\pgfqpoint{0.050679in}{0.788388in}}%
\pgfpathlineto{\pgfqpoint{0.083333in}{0.805556in}}%
\pgfpathlineto{\pgfqpoint{0.115988in}{0.788388in}}%
\pgfpathlineto{\pgfqpoint{0.109752in}{0.824750in}}%
\pgfpathlineto{\pgfqpoint{0.136170in}{0.850501in}}%
\pgfpathlineto{\pgfqpoint{0.099661in}{0.855806in}}%
\pgfpathlineto{\pgfqpoint{0.083333in}{0.888889in}}%
\pgfpathmoveto{\pgfqpoint{0.250000in}{0.888889in}}%
\pgfpathlineto{\pgfqpoint{0.233673in}{0.855806in}}%
\pgfpathlineto{\pgfqpoint{0.197164in}{0.850501in}}%
\pgfpathlineto{\pgfqpoint{0.223582in}{0.824750in}}%
\pgfpathlineto{\pgfqpoint{0.217345in}{0.788388in}}%
\pgfpathlineto{\pgfqpoint{0.250000in}{0.805556in}}%
\pgfpathlineto{\pgfqpoint{0.282655in}{0.788388in}}%
\pgfpathlineto{\pgfqpoint{0.276418in}{0.824750in}}%
\pgfpathlineto{\pgfqpoint{0.302836in}{0.850501in}}%
\pgfpathlineto{\pgfqpoint{0.266327in}{0.855806in}}%
\pgfpathlineto{\pgfqpoint{0.250000in}{0.888889in}}%
\pgfpathmoveto{\pgfqpoint{0.416667in}{0.888889in}}%
\pgfpathlineto{\pgfqpoint{0.400339in}{0.855806in}}%
\pgfpathlineto{\pgfqpoint{0.363830in}{0.850501in}}%
\pgfpathlineto{\pgfqpoint{0.390248in}{0.824750in}}%
\pgfpathlineto{\pgfqpoint{0.384012in}{0.788388in}}%
\pgfpathlineto{\pgfqpoint{0.416667in}{0.805556in}}%
\pgfpathlineto{\pgfqpoint{0.449321in}{0.788388in}}%
\pgfpathlineto{\pgfqpoint{0.443085in}{0.824750in}}%
\pgfpathlineto{\pgfqpoint{0.469503in}{0.850501in}}%
\pgfpathlineto{\pgfqpoint{0.432994in}{0.855806in}}%
\pgfpathlineto{\pgfqpoint{0.416667in}{0.888889in}}%
\pgfpathmoveto{\pgfqpoint{0.583333in}{0.888889in}}%
\pgfpathlineto{\pgfqpoint{0.567006in}{0.855806in}}%
\pgfpathlineto{\pgfqpoint{0.530497in}{0.850501in}}%
\pgfpathlineto{\pgfqpoint{0.556915in}{0.824750in}}%
\pgfpathlineto{\pgfqpoint{0.550679in}{0.788388in}}%
\pgfpathlineto{\pgfqpoint{0.583333in}{0.805556in}}%
\pgfpathlineto{\pgfqpoint{0.615988in}{0.788388in}}%
\pgfpathlineto{\pgfqpoint{0.609752in}{0.824750in}}%
\pgfpathlineto{\pgfqpoint{0.636170in}{0.850501in}}%
\pgfpathlineto{\pgfqpoint{0.599661in}{0.855806in}}%
\pgfpathlineto{\pgfqpoint{0.583333in}{0.888889in}}%
\pgfpathmoveto{\pgfqpoint{0.750000in}{0.888889in}}%
\pgfpathlineto{\pgfqpoint{0.733673in}{0.855806in}}%
\pgfpathlineto{\pgfqpoint{0.697164in}{0.850501in}}%
\pgfpathlineto{\pgfqpoint{0.723582in}{0.824750in}}%
\pgfpathlineto{\pgfqpoint{0.717345in}{0.788388in}}%
\pgfpathlineto{\pgfqpoint{0.750000in}{0.805556in}}%
\pgfpathlineto{\pgfqpoint{0.782655in}{0.788388in}}%
\pgfpathlineto{\pgfqpoint{0.776418in}{0.824750in}}%
\pgfpathlineto{\pgfqpoint{0.802836in}{0.850501in}}%
\pgfpathlineto{\pgfqpoint{0.766327in}{0.855806in}}%
\pgfpathlineto{\pgfqpoint{0.750000in}{0.888889in}}%
\pgfpathmoveto{\pgfqpoint{0.916667in}{0.888889in}}%
\pgfpathlineto{\pgfqpoint{0.900339in}{0.855806in}}%
\pgfpathlineto{\pgfqpoint{0.863830in}{0.850501in}}%
\pgfpathlineto{\pgfqpoint{0.890248in}{0.824750in}}%
\pgfpathlineto{\pgfqpoint{0.884012in}{0.788388in}}%
\pgfpathlineto{\pgfqpoint{0.916667in}{0.805556in}}%
\pgfpathlineto{\pgfqpoint{0.949321in}{0.788388in}}%
\pgfpathlineto{\pgfqpoint{0.943085in}{0.824750in}}%
\pgfpathlineto{\pgfqpoint{0.969503in}{0.850501in}}%
\pgfpathlineto{\pgfqpoint{0.932994in}{0.855806in}}%
\pgfpathlineto{\pgfqpoint{0.916667in}{0.888889in}}%
\pgfpathmoveto{\pgfqpoint{0.000000in}{1.055556in}}%
\pgfpathlineto{\pgfqpoint{-0.016327in}{1.022473in}}%
\pgfpathlineto{\pgfqpoint{-0.052836in}{1.017168in}}%
\pgfpathlineto{\pgfqpoint{-0.026418in}{0.991416in}}%
\pgfpathlineto{\pgfqpoint{-0.032655in}{0.955055in}}%
\pgfpathlineto{\pgfqpoint{-0.000000in}{0.972222in}}%
\pgfpathlineto{\pgfqpoint{0.032655in}{0.955055in}}%
\pgfpathlineto{\pgfqpoint{0.026418in}{0.991416in}}%
\pgfpathlineto{\pgfqpoint{0.052836in}{1.017168in}}%
\pgfpathlineto{\pgfqpoint{0.016327in}{1.022473in}}%
\pgfpathlineto{\pgfqpoint{0.000000in}{1.055556in}}%
\pgfpathmoveto{\pgfqpoint{0.166667in}{1.055556in}}%
\pgfpathlineto{\pgfqpoint{0.150339in}{1.022473in}}%
\pgfpathlineto{\pgfqpoint{0.113830in}{1.017168in}}%
\pgfpathlineto{\pgfqpoint{0.140248in}{0.991416in}}%
\pgfpathlineto{\pgfqpoint{0.134012in}{0.955055in}}%
\pgfpathlineto{\pgfqpoint{0.166667in}{0.972222in}}%
\pgfpathlineto{\pgfqpoint{0.199321in}{0.955055in}}%
\pgfpathlineto{\pgfqpoint{0.193085in}{0.991416in}}%
\pgfpathlineto{\pgfqpoint{0.219503in}{1.017168in}}%
\pgfpathlineto{\pgfqpoint{0.182994in}{1.022473in}}%
\pgfpathlineto{\pgfqpoint{0.166667in}{1.055556in}}%
\pgfpathmoveto{\pgfqpoint{0.333333in}{1.055556in}}%
\pgfpathlineto{\pgfqpoint{0.317006in}{1.022473in}}%
\pgfpathlineto{\pgfqpoint{0.280497in}{1.017168in}}%
\pgfpathlineto{\pgfqpoint{0.306915in}{0.991416in}}%
\pgfpathlineto{\pgfqpoint{0.300679in}{0.955055in}}%
\pgfpathlineto{\pgfqpoint{0.333333in}{0.972222in}}%
\pgfpathlineto{\pgfqpoint{0.365988in}{0.955055in}}%
\pgfpathlineto{\pgfqpoint{0.359752in}{0.991416in}}%
\pgfpathlineto{\pgfqpoint{0.386170in}{1.017168in}}%
\pgfpathlineto{\pgfqpoint{0.349661in}{1.022473in}}%
\pgfpathlineto{\pgfqpoint{0.333333in}{1.055556in}}%
\pgfpathmoveto{\pgfqpoint{0.500000in}{1.055556in}}%
\pgfpathlineto{\pgfqpoint{0.483673in}{1.022473in}}%
\pgfpathlineto{\pgfqpoint{0.447164in}{1.017168in}}%
\pgfpathlineto{\pgfqpoint{0.473582in}{0.991416in}}%
\pgfpathlineto{\pgfqpoint{0.467345in}{0.955055in}}%
\pgfpathlineto{\pgfqpoint{0.500000in}{0.972222in}}%
\pgfpathlineto{\pgfqpoint{0.532655in}{0.955055in}}%
\pgfpathlineto{\pgfqpoint{0.526418in}{0.991416in}}%
\pgfpathlineto{\pgfqpoint{0.552836in}{1.017168in}}%
\pgfpathlineto{\pgfqpoint{0.516327in}{1.022473in}}%
\pgfpathlineto{\pgfqpoint{0.500000in}{1.055556in}}%
\pgfpathmoveto{\pgfqpoint{0.666667in}{1.055556in}}%
\pgfpathlineto{\pgfqpoint{0.650339in}{1.022473in}}%
\pgfpathlineto{\pgfqpoint{0.613830in}{1.017168in}}%
\pgfpathlineto{\pgfqpoint{0.640248in}{0.991416in}}%
\pgfpathlineto{\pgfqpoint{0.634012in}{0.955055in}}%
\pgfpathlineto{\pgfqpoint{0.666667in}{0.972222in}}%
\pgfpathlineto{\pgfqpoint{0.699321in}{0.955055in}}%
\pgfpathlineto{\pgfqpoint{0.693085in}{0.991416in}}%
\pgfpathlineto{\pgfqpoint{0.719503in}{1.017168in}}%
\pgfpathlineto{\pgfqpoint{0.682994in}{1.022473in}}%
\pgfpathlineto{\pgfqpoint{0.666667in}{1.055556in}}%
\pgfpathmoveto{\pgfqpoint{0.833333in}{1.055556in}}%
\pgfpathlineto{\pgfqpoint{0.817006in}{1.022473in}}%
\pgfpathlineto{\pgfqpoint{0.780497in}{1.017168in}}%
\pgfpathlineto{\pgfqpoint{0.806915in}{0.991416in}}%
\pgfpathlineto{\pgfqpoint{0.800679in}{0.955055in}}%
\pgfpathlineto{\pgfqpoint{0.833333in}{0.972222in}}%
\pgfpathlineto{\pgfqpoint{0.865988in}{0.955055in}}%
\pgfpathlineto{\pgfqpoint{0.859752in}{0.991416in}}%
\pgfpathlineto{\pgfqpoint{0.886170in}{1.017168in}}%
\pgfpathlineto{\pgfqpoint{0.849661in}{1.022473in}}%
\pgfpathlineto{\pgfqpoint{0.833333in}{1.055556in}}%
\pgfpathmoveto{\pgfqpoint{1.000000in}{1.055556in}}%
\pgfpathlineto{\pgfqpoint{0.983673in}{1.022473in}}%
\pgfpathlineto{\pgfqpoint{0.947164in}{1.017168in}}%
\pgfpathlineto{\pgfqpoint{0.973582in}{0.991416in}}%
\pgfpathlineto{\pgfqpoint{0.967345in}{0.955055in}}%
\pgfpathlineto{\pgfqpoint{1.000000in}{0.972222in}}%
\pgfpathlineto{\pgfqpoint{1.032655in}{0.955055in}}%
\pgfpathlineto{\pgfqpoint{1.026418in}{0.991416in}}%
\pgfpathlineto{\pgfqpoint{1.052836in}{1.017168in}}%
\pgfpathlineto{\pgfqpoint{1.016327in}{1.022473in}}%
\pgfpathlineto{\pgfqpoint{1.000000in}{1.055556in}}%
\pgfpathlineto{\pgfqpoint{1.000000in}{1.055556in}}%
\pgfusepath{stroke}%
\end{pgfscope}%
}%
\pgfsys@transformshift{9.073315in}{6.003887in}%
\pgfsys@useobject{currentpattern}{}%
\pgfsys@transformshift{1in}{0in}%
\pgfsys@transformshift{-1in}{0in}%
\pgfsys@transformshift{0in}{1in}%
\pgfsys@useobject{currentpattern}{}%
\pgfsys@transformshift{1in}{0in}%
\pgfsys@transformshift{-1in}{0in}%
\pgfsys@transformshift{0in}{1in}%
\end{pgfscope}%
\begin{pgfscope}%
\pgfpathrectangle{\pgfqpoint{0.935815in}{0.637495in}}{\pgfqpoint{9.300000in}{9.060000in}}%
\pgfusepath{clip}%
\pgfsetbuttcap%
\pgfsetmiterjoin%
\definecolor{currentfill}{rgb}{0.121569,0.466667,0.705882}%
\pgfsetfillcolor{currentfill}%
\pgfsetfillopacity{0.990000}%
\pgfsetlinewidth{0.000000pt}%
\definecolor{currentstroke}{rgb}{0.000000,0.000000,0.000000}%
\pgfsetstrokecolor{currentstroke}%
\pgfsetstrokeopacity{0.990000}%
\pgfsetdash{}{0pt}%
\pgfpathmoveto{\pgfqpoint{1.323315in}{2.908608in}}%
\pgfpathlineto{\pgfqpoint{2.098315in}{2.908608in}}%
\pgfpathlineto{\pgfqpoint{2.098315in}{4.982791in}}%
\pgfpathlineto{\pgfqpoint{1.323315in}{4.982791in}}%
\pgfpathclose%
\pgfusepath{fill}%
\end{pgfscope}%
\begin{pgfscope}%
\pgfsetbuttcap%
\pgfsetmiterjoin%
\definecolor{currentfill}{rgb}{0.121569,0.466667,0.705882}%
\pgfsetfillcolor{currentfill}%
\pgfsetfillopacity{0.990000}%
\pgfsetlinewidth{0.000000pt}%
\definecolor{currentstroke}{rgb}{0.000000,0.000000,0.000000}%
\pgfsetstrokecolor{currentstroke}%
\pgfsetstrokeopacity{0.990000}%
\pgfsetdash{}{0pt}%
\pgfpathrectangle{\pgfqpoint{0.935815in}{0.637495in}}{\pgfqpoint{9.300000in}{9.060000in}}%
\pgfusepath{clip}%
\pgfpathmoveto{\pgfqpoint{1.323315in}{2.908608in}}%
\pgfpathlineto{\pgfqpoint{2.098315in}{2.908608in}}%
\pgfpathlineto{\pgfqpoint{2.098315in}{4.982791in}}%
\pgfpathlineto{\pgfqpoint{1.323315in}{4.982791in}}%
\pgfpathclose%
\pgfusepath{clip}%
\pgfsys@defobject{currentpattern}{\pgfqpoint{0in}{0in}}{\pgfqpoint{1in}{1in}}{%
\begin{pgfscope}%
\pgfpathrectangle{\pgfqpoint{0in}{0in}}{\pgfqpoint{1in}{1in}}%
\pgfusepath{clip}%
\pgfpathmoveto{\pgfqpoint{0.000000in}{0.083333in}}%
\pgfpathlineto{\pgfqpoint{1.000000in}{0.083333in}}%
\pgfpathmoveto{\pgfqpoint{0.000000in}{0.250000in}}%
\pgfpathlineto{\pgfqpoint{1.000000in}{0.250000in}}%
\pgfpathmoveto{\pgfqpoint{0.000000in}{0.416667in}}%
\pgfpathlineto{\pgfqpoint{1.000000in}{0.416667in}}%
\pgfpathmoveto{\pgfqpoint{0.000000in}{0.583333in}}%
\pgfpathlineto{\pgfqpoint{1.000000in}{0.583333in}}%
\pgfpathmoveto{\pgfqpoint{0.000000in}{0.750000in}}%
\pgfpathlineto{\pgfqpoint{1.000000in}{0.750000in}}%
\pgfpathmoveto{\pgfqpoint{0.000000in}{0.916667in}}%
\pgfpathlineto{\pgfqpoint{1.000000in}{0.916667in}}%
\pgfpathmoveto{\pgfqpoint{0.083333in}{0.000000in}}%
\pgfpathlineto{\pgfqpoint{0.083333in}{1.000000in}}%
\pgfpathmoveto{\pgfqpoint{0.250000in}{0.000000in}}%
\pgfpathlineto{\pgfqpoint{0.250000in}{1.000000in}}%
\pgfpathmoveto{\pgfqpoint{0.416667in}{0.000000in}}%
\pgfpathlineto{\pgfqpoint{0.416667in}{1.000000in}}%
\pgfpathmoveto{\pgfqpoint{0.583333in}{0.000000in}}%
\pgfpathlineto{\pgfqpoint{0.583333in}{1.000000in}}%
\pgfpathmoveto{\pgfqpoint{0.750000in}{0.000000in}}%
\pgfpathlineto{\pgfqpoint{0.750000in}{1.000000in}}%
\pgfpathmoveto{\pgfqpoint{0.916667in}{0.000000in}}%
\pgfpathlineto{\pgfqpoint{0.916667in}{1.000000in}}%
\pgfusepath{stroke}%
\end{pgfscope}%
}%
\pgfsys@transformshift{1.323315in}{2.908608in}%
\pgfsys@useobject{currentpattern}{}%
\pgfsys@transformshift{1in}{0in}%
\pgfsys@transformshift{-1in}{0in}%
\pgfsys@transformshift{0in}{1in}%
\pgfsys@useobject{currentpattern}{}%
\pgfsys@transformshift{1in}{0in}%
\pgfsys@transformshift{-1in}{0in}%
\pgfsys@transformshift{0in}{1in}%
\pgfsys@useobject{currentpattern}{}%
\pgfsys@transformshift{1in}{0in}%
\pgfsys@transformshift{-1in}{0in}%
\pgfsys@transformshift{0in}{1in}%
\end{pgfscope}%
\begin{pgfscope}%
\pgfpathrectangle{\pgfqpoint{0.935815in}{0.637495in}}{\pgfqpoint{9.300000in}{9.060000in}}%
\pgfusepath{clip}%
\pgfsetbuttcap%
\pgfsetmiterjoin%
\definecolor{currentfill}{rgb}{0.121569,0.466667,0.705882}%
\pgfsetfillcolor{currentfill}%
\pgfsetfillopacity{0.990000}%
\pgfsetlinewidth{0.000000pt}%
\definecolor{currentstroke}{rgb}{0.000000,0.000000,0.000000}%
\pgfsetstrokecolor{currentstroke}%
\pgfsetstrokeopacity{0.990000}%
\pgfsetdash{}{0pt}%
\pgfpathmoveto{\pgfqpoint{2.873315in}{4.265443in}}%
\pgfpathlineto{\pgfqpoint{3.648315in}{4.265443in}}%
\pgfpathlineto{\pgfqpoint{3.648315in}{6.344686in}}%
\pgfpathlineto{\pgfqpoint{2.873315in}{6.344686in}}%
\pgfpathclose%
\pgfusepath{fill}%
\end{pgfscope}%
\begin{pgfscope}%
\pgfsetbuttcap%
\pgfsetmiterjoin%
\definecolor{currentfill}{rgb}{0.121569,0.466667,0.705882}%
\pgfsetfillcolor{currentfill}%
\pgfsetfillopacity{0.990000}%
\pgfsetlinewidth{0.000000pt}%
\definecolor{currentstroke}{rgb}{0.000000,0.000000,0.000000}%
\pgfsetstrokecolor{currentstroke}%
\pgfsetstrokeopacity{0.990000}%
\pgfsetdash{}{0pt}%
\pgfpathrectangle{\pgfqpoint{0.935815in}{0.637495in}}{\pgfqpoint{9.300000in}{9.060000in}}%
\pgfusepath{clip}%
\pgfpathmoveto{\pgfqpoint{2.873315in}{4.265443in}}%
\pgfpathlineto{\pgfqpoint{3.648315in}{4.265443in}}%
\pgfpathlineto{\pgfqpoint{3.648315in}{6.344686in}}%
\pgfpathlineto{\pgfqpoint{2.873315in}{6.344686in}}%
\pgfpathclose%
\pgfusepath{clip}%
\pgfsys@defobject{currentpattern}{\pgfqpoint{0in}{0in}}{\pgfqpoint{1in}{1in}}{%
\begin{pgfscope}%
\pgfpathrectangle{\pgfqpoint{0in}{0in}}{\pgfqpoint{1in}{1in}}%
\pgfusepath{clip}%
\pgfpathmoveto{\pgfqpoint{0.000000in}{0.083333in}}%
\pgfpathlineto{\pgfqpoint{1.000000in}{0.083333in}}%
\pgfpathmoveto{\pgfqpoint{0.000000in}{0.250000in}}%
\pgfpathlineto{\pgfqpoint{1.000000in}{0.250000in}}%
\pgfpathmoveto{\pgfqpoint{0.000000in}{0.416667in}}%
\pgfpathlineto{\pgfqpoint{1.000000in}{0.416667in}}%
\pgfpathmoveto{\pgfqpoint{0.000000in}{0.583333in}}%
\pgfpathlineto{\pgfqpoint{1.000000in}{0.583333in}}%
\pgfpathmoveto{\pgfqpoint{0.000000in}{0.750000in}}%
\pgfpathlineto{\pgfqpoint{1.000000in}{0.750000in}}%
\pgfpathmoveto{\pgfqpoint{0.000000in}{0.916667in}}%
\pgfpathlineto{\pgfqpoint{1.000000in}{0.916667in}}%
\pgfpathmoveto{\pgfqpoint{0.083333in}{0.000000in}}%
\pgfpathlineto{\pgfqpoint{0.083333in}{1.000000in}}%
\pgfpathmoveto{\pgfqpoint{0.250000in}{0.000000in}}%
\pgfpathlineto{\pgfqpoint{0.250000in}{1.000000in}}%
\pgfpathmoveto{\pgfqpoint{0.416667in}{0.000000in}}%
\pgfpathlineto{\pgfqpoint{0.416667in}{1.000000in}}%
\pgfpathmoveto{\pgfqpoint{0.583333in}{0.000000in}}%
\pgfpathlineto{\pgfqpoint{0.583333in}{1.000000in}}%
\pgfpathmoveto{\pgfqpoint{0.750000in}{0.000000in}}%
\pgfpathlineto{\pgfqpoint{0.750000in}{1.000000in}}%
\pgfpathmoveto{\pgfqpoint{0.916667in}{0.000000in}}%
\pgfpathlineto{\pgfqpoint{0.916667in}{1.000000in}}%
\pgfusepath{stroke}%
\end{pgfscope}%
}%
\pgfsys@transformshift{2.873315in}{4.265443in}%
\pgfsys@useobject{currentpattern}{}%
\pgfsys@transformshift{1in}{0in}%
\pgfsys@transformshift{-1in}{0in}%
\pgfsys@transformshift{0in}{1in}%
\pgfsys@useobject{currentpattern}{}%
\pgfsys@transformshift{1in}{0in}%
\pgfsys@transformshift{-1in}{0in}%
\pgfsys@transformshift{0in}{1in}%
\pgfsys@useobject{currentpattern}{}%
\pgfsys@transformshift{1in}{0in}%
\pgfsys@transformshift{-1in}{0in}%
\pgfsys@transformshift{0in}{1in}%
\end{pgfscope}%
\begin{pgfscope}%
\pgfpathrectangle{\pgfqpoint{0.935815in}{0.637495in}}{\pgfqpoint{9.300000in}{9.060000in}}%
\pgfusepath{clip}%
\pgfsetbuttcap%
\pgfsetmiterjoin%
\definecolor{currentfill}{rgb}{0.121569,0.466667,0.705882}%
\pgfsetfillcolor{currentfill}%
\pgfsetfillopacity{0.990000}%
\pgfsetlinewidth{0.000000pt}%
\definecolor{currentstroke}{rgb}{0.000000,0.000000,0.000000}%
\pgfsetstrokecolor{currentstroke}%
\pgfsetstrokeopacity{0.990000}%
\pgfsetdash{}{0pt}%
\pgfpathmoveto{\pgfqpoint{4.423315in}{5.531339in}}%
\pgfpathlineto{\pgfqpoint{5.198315in}{5.531339in}}%
\pgfpathlineto{\pgfqpoint{5.198315in}{7.610583in}}%
\pgfpathlineto{\pgfqpoint{4.423315in}{7.610583in}}%
\pgfpathclose%
\pgfusepath{fill}%
\end{pgfscope}%
\begin{pgfscope}%
\pgfsetbuttcap%
\pgfsetmiterjoin%
\definecolor{currentfill}{rgb}{0.121569,0.466667,0.705882}%
\pgfsetfillcolor{currentfill}%
\pgfsetfillopacity{0.990000}%
\pgfsetlinewidth{0.000000pt}%
\definecolor{currentstroke}{rgb}{0.000000,0.000000,0.000000}%
\pgfsetstrokecolor{currentstroke}%
\pgfsetstrokeopacity{0.990000}%
\pgfsetdash{}{0pt}%
\pgfpathrectangle{\pgfqpoint{0.935815in}{0.637495in}}{\pgfqpoint{9.300000in}{9.060000in}}%
\pgfusepath{clip}%
\pgfpathmoveto{\pgfqpoint{4.423315in}{5.531339in}}%
\pgfpathlineto{\pgfqpoint{5.198315in}{5.531339in}}%
\pgfpathlineto{\pgfqpoint{5.198315in}{7.610583in}}%
\pgfpathlineto{\pgfqpoint{4.423315in}{7.610583in}}%
\pgfpathclose%
\pgfusepath{clip}%
\pgfsys@defobject{currentpattern}{\pgfqpoint{0in}{0in}}{\pgfqpoint{1in}{1in}}{%
\begin{pgfscope}%
\pgfpathrectangle{\pgfqpoint{0in}{0in}}{\pgfqpoint{1in}{1in}}%
\pgfusepath{clip}%
\pgfpathmoveto{\pgfqpoint{0.000000in}{0.083333in}}%
\pgfpathlineto{\pgfqpoint{1.000000in}{0.083333in}}%
\pgfpathmoveto{\pgfqpoint{0.000000in}{0.250000in}}%
\pgfpathlineto{\pgfqpoint{1.000000in}{0.250000in}}%
\pgfpathmoveto{\pgfqpoint{0.000000in}{0.416667in}}%
\pgfpathlineto{\pgfqpoint{1.000000in}{0.416667in}}%
\pgfpathmoveto{\pgfqpoint{0.000000in}{0.583333in}}%
\pgfpathlineto{\pgfqpoint{1.000000in}{0.583333in}}%
\pgfpathmoveto{\pgfqpoint{0.000000in}{0.750000in}}%
\pgfpathlineto{\pgfqpoint{1.000000in}{0.750000in}}%
\pgfpathmoveto{\pgfqpoint{0.000000in}{0.916667in}}%
\pgfpathlineto{\pgfqpoint{1.000000in}{0.916667in}}%
\pgfpathmoveto{\pgfqpoint{0.083333in}{0.000000in}}%
\pgfpathlineto{\pgfqpoint{0.083333in}{1.000000in}}%
\pgfpathmoveto{\pgfqpoint{0.250000in}{0.000000in}}%
\pgfpathlineto{\pgfqpoint{0.250000in}{1.000000in}}%
\pgfpathmoveto{\pgfqpoint{0.416667in}{0.000000in}}%
\pgfpathlineto{\pgfqpoint{0.416667in}{1.000000in}}%
\pgfpathmoveto{\pgfqpoint{0.583333in}{0.000000in}}%
\pgfpathlineto{\pgfqpoint{0.583333in}{1.000000in}}%
\pgfpathmoveto{\pgfqpoint{0.750000in}{0.000000in}}%
\pgfpathlineto{\pgfqpoint{0.750000in}{1.000000in}}%
\pgfpathmoveto{\pgfqpoint{0.916667in}{0.000000in}}%
\pgfpathlineto{\pgfqpoint{0.916667in}{1.000000in}}%
\pgfusepath{stroke}%
\end{pgfscope}%
}%
\pgfsys@transformshift{4.423315in}{5.531339in}%
\pgfsys@useobject{currentpattern}{}%
\pgfsys@transformshift{1in}{0in}%
\pgfsys@transformshift{-1in}{0in}%
\pgfsys@transformshift{0in}{1in}%
\pgfsys@useobject{currentpattern}{}%
\pgfsys@transformshift{1in}{0in}%
\pgfsys@transformshift{-1in}{0in}%
\pgfsys@transformshift{0in}{1in}%
\pgfsys@useobject{currentpattern}{}%
\pgfsys@transformshift{1in}{0in}%
\pgfsys@transformshift{-1in}{0in}%
\pgfsys@transformshift{0in}{1in}%
\end{pgfscope}%
\begin{pgfscope}%
\pgfpathrectangle{\pgfqpoint{0.935815in}{0.637495in}}{\pgfqpoint{9.300000in}{9.060000in}}%
\pgfusepath{clip}%
\pgfsetbuttcap%
\pgfsetmiterjoin%
\definecolor{currentfill}{rgb}{0.121569,0.466667,0.705882}%
\pgfsetfillcolor{currentfill}%
\pgfsetfillopacity{0.990000}%
\pgfsetlinewidth{0.000000pt}%
\definecolor{currentstroke}{rgb}{0.000000,0.000000,0.000000}%
\pgfsetstrokecolor{currentstroke}%
\pgfsetstrokeopacity{0.990000}%
\pgfsetdash{}{0pt}%
\pgfpathmoveto{\pgfqpoint{5.973315in}{6.327614in}}%
\pgfpathlineto{\pgfqpoint{6.748315in}{6.327614in}}%
\pgfpathlineto{\pgfqpoint{6.748315in}{8.406858in}}%
\pgfpathlineto{\pgfqpoint{5.973315in}{8.406858in}}%
\pgfpathclose%
\pgfusepath{fill}%
\end{pgfscope}%
\begin{pgfscope}%
\pgfsetbuttcap%
\pgfsetmiterjoin%
\definecolor{currentfill}{rgb}{0.121569,0.466667,0.705882}%
\pgfsetfillcolor{currentfill}%
\pgfsetfillopacity{0.990000}%
\pgfsetlinewidth{0.000000pt}%
\definecolor{currentstroke}{rgb}{0.000000,0.000000,0.000000}%
\pgfsetstrokecolor{currentstroke}%
\pgfsetstrokeopacity{0.990000}%
\pgfsetdash{}{0pt}%
\pgfpathrectangle{\pgfqpoint{0.935815in}{0.637495in}}{\pgfqpoint{9.300000in}{9.060000in}}%
\pgfusepath{clip}%
\pgfpathmoveto{\pgfqpoint{5.973315in}{6.327614in}}%
\pgfpathlineto{\pgfqpoint{6.748315in}{6.327614in}}%
\pgfpathlineto{\pgfqpoint{6.748315in}{8.406858in}}%
\pgfpathlineto{\pgfqpoint{5.973315in}{8.406858in}}%
\pgfpathclose%
\pgfusepath{clip}%
\pgfsys@defobject{currentpattern}{\pgfqpoint{0in}{0in}}{\pgfqpoint{1in}{1in}}{%
\begin{pgfscope}%
\pgfpathrectangle{\pgfqpoint{0in}{0in}}{\pgfqpoint{1in}{1in}}%
\pgfusepath{clip}%
\pgfpathmoveto{\pgfqpoint{0.000000in}{0.083333in}}%
\pgfpathlineto{\pgfqpoint{1.000000in}{0.083333in}}%
\pgfpathmoveto{\pgfqpoint{0.000000in}{0.250000in}}%
\pgfpathlineto{\pgfqpoint{1.000000in}{0.250000in}}%
\pgfpathmoveto{\pgfqpoint{0.000000in}{0.416667in}}%
\pgfpathlineto{\pgfqpoint{1.000000in}{0.416667in}}%
\pgfpathmoveto{\pgfqpoint{0.000000in}{0.583333in}}%
\pgfpathlineto{\pgfqpoint{1.000000in}{0.583333in}}%
\pgfpathmoveto{\pgfqpoint{0.000000in}{0.750000in}}%
\pgfpathlineto{\pgfqpoint{1.000000in}{0.750000in}}%
\pgfpathmoveto{\pgfqpoint{0.000000in}{0.916667in}}%
\pgfpathlineto{\pgfqpoint{1.000000in}{0.916667in}}%
\pgfpathmoveto{\pgfqpoint{0.083333in}{0.000000in}}%
\pgfpathlineto{\pgfqpoint{0.083333in}{1.000000in}}%
\pgfpathmoveto{\pgfqpoint{0.250000in}{0.000000in}}%
\pgfpathlineto{\pgfqpoint{0.250000in}{1.000000in}}%
\pgfpathmoveto{\pgfqpoint{0.416667in}{0.000000in}}%
\pgfpathlineto{\pgfqpoint{0.416667in}{1.000000in}}%
\pgfpathmoveto{\pgfqpoint{0.583333in}{0.000000in}}%
\pgfpathlineto{\pgfqpoint{0.583333in}{1.000000in}}%
\pgfpathmoveto{\pgfqpoint{0.750000in}{0.000000in}}%
\pgfpathlineto{\pgfqpoint{0.750000in}{1.000000in}}%
\pgfpathmoveto{\pgfqpoint{0.916667in}{0.000000in}}%
\pgfpathlineto{\pgfqpoint{0.916667in}{1.000000in}}%
\pgfusepath{stroke}%
\end{pgfscope}%
}%
\pgfsys@transformshift{5.973315in}{6.327614in}%
\pgfsys@useobject{currentpattern}{}%
\pgfsys@transformshift{1in}{0in}%
\pgfsys@transformshift{-1in}{0in}%
\pgfsys@transformshift{0in}{1in}%
\pgfsys@useobject{currentpattern}{}%
\pgfsys@transformshift{1in}{0in}%
\pgfsys@transformshift{-1in}{0in}%
\pgfsys@transformshift{0in}{1in}%
\pgfsys@useobject{currentpattern}{}%
\pgfsys@transformshift{1in}{0in}%
\pgfsys@transformshift{-1in}{0in}%
\pgfsys@transformshift{0in}{1in}%
\end{pgfscope}%
\begin{pgfscope}%
\pgfpathrectangle{\pgfqpoint{0.935815in}{0.637495in}}{\pgfqpoint{9.300000in}{9.060000in}}%
\pgfusepath{clip}%
\pgfsetbuttcap%
\pgfsetmiterjoin%
\definecolor{currentfill}{rgb}{0.121569,0.466667,0.705882}%
\pgfsetfillcolor{currentfill}%
\pgfsetfillopacity{0.990000}%
\pgfsetlinewidth{0.000000pt}%
\definecolor{currentstroke}{rgb}{0.000000,0.000000,0.000000}%
\pgfsetstrokecolor{currentstroke}%
\pgfsetstrokeopacity{0.990000}%
\pgfsetdash{}{0pt}%
\pgfpathmoveto{\pgfqpoint{7.523315in}{6.633305in}}%
\pgfpathlineto{\pgfqpoint{8.298315in}{6.633305in}}%
\pgfpathlineto{\pgfqpoint{8.298315in}{8.712548in}}%
\pgfpathlineto{\pgfqpoint{7.523315in}{8.712548in}}%
\pgfpathclose%
\pgfusepath{fill}%
\end{pgfscope}%
\begin{pgfscope}%
\pgfsetbuttcap%
\pgfsetmiterjoin%
\definecolor{currentfill}{rgb}{0.121569,0.466667,0.705882}%
\pgfsetfillcolor{currentfill}%
\pgfsetfillopacity{0.990000}%
\pgfsetlinewidth{0.000000pt}%
\definecolor{currentstroke}{rgb}{0.000000,0.000000,0.000000}%
\pgfsetstrokecolor{currentstroke}%
\pgfsetstrokeopacity{0.990000}%
\pgfsetdash{}{0pt}%
\pgfpathrectangle{\pgfqpoint{0.935815in}{0.637495in}}{\pgfqpoint{9.300000in}{9.060000in}}%
\pgfusepath{clip}%
\pgfpathmoveto{\pgfqpoint{7.523315in}{6.633305in}}%
\pgfpathlineto{\pgfqpoint{8.298315in}{6.633305in}}%
\pgfpathlineto{\pgfqpoint{8.298315in}{8.712548in}}%
\pgfpathlineto{\pgfqpoint{7.523315in}{8.712548in}}%
\pgfpathclose%
\pgfusepath{clip}%
\pgfsys@defobject{currentpattern}{\pgfqpoint{0in}{0in}}{\pgfqpoint{1in}{1in}}{%
\begin{pgfscope}%
\pgfpathrectangle{\pgfqpoint{0in}{0in}}{\pgfqpoint{1in}{1in}}%
\pgfusepath{clip}%
\pgfpathmoveto{\pgfqpoint{0.000000in}{0.083333in}}%
\pgfpathlineto{\pgfqpoint{1.000000in}{0.083333in}}%
\pgfpathmoveto{\pgfqpoint{0.000000in}{0.250000in}}%
\pgfpathlineto{\pgfqpoint{1.000000in}{0.250000in}}%
\pgfpathmoveto{\pgfqpoint{0.000000in}{0.416667in}}%
\pgfpathlineto{\pgfqpoint{1.000000in}{0.416667in}}%
\pgfpathmoveto{\pgfqpoint{0.000000in}{0.583333in}}%
\pgfpathlineto{\pgfqpoint{1.000000in}{0.583333in}}%
\pgfpathmoveto{\pgfqpoint{0.000000in}{0.750000in}}%
\pgfpathlineto{\pgfqpoint{1.000000in}{0.750000in}}%
\pgfpathmoveto{\pgfqpoint{0.000000in}{0.916667in}}%
\pgfpathlineto{\pgfqpoint{1.000000in}{0.916667in}}%
\pgfpathmoveto{\pgfqpoint{0.083333in}{0.000000in}}%
\pgfpathlineto{\pgfqpoint{0.083333in}{1.000000in}}%
\pgfpathmoveto{\pgfqpoint{0.250000in}{0.000000in}}%
\pgfpathlineto{\pgfqpoint{0.250000in}{1.000000in}}%
\pgfpathmoveto{\pgfqpoint{0.416667in}{0.000000in}}%
\pgfpathlineto{\pgfqpoint{0.416667in}{1.000000in}}%
\pgfpathmoveto{\pgfqpoint{0.583333in}{0.000000in}}%
\pgfpathlineto{\pgfqpoint{0.583333in}{1.000000in}}%
\pgfpathmoveto{\pgfqpoint{0.750000in}{0.000000in}}%
\pgfpathlineto{\pgfqpoint{0.750000in}{1.000000in}}%
\pgfpathmoveto{\pgfqpoint{0.916667in}{0.000000in}}%
\pgfpathlineto{\pgfqpoint{0.916667in}{1.000000in}}%
\pgfusepath{stroke}%
\end{pgfscope}%
}%
\pgfsys@transformshift{7.523315in}{6.633305in}%
\pgfsys@useobject{currentpattern}{}%
\pgfsys@transformshift{1in}{0in}%
\pgfsys@transformshift{-1in}{0in}%
\pgfsys@transformshift{0in}{1in}%
\pgfsys@useobject{currentpattern}{}%
\pgfsys@transformshift{1in}{0in}%
\pgfsys@transformshift{-1in}{0in}%
\pgfsys@transformshift{0in}{1in}%
\pgfsys@useobject{currentpattern}{}%
\pgfsys@transformshift{1in}{0in}%
\pgfsys@transformshift{-1in}{0in}%
\pgfsys@transformshift{0in}{1in}%
\end{pgfscope}%
\begin{pgfscope}%
\pgfpathrectangle{\pgfqpoint{0.935815in}{0.637495in}}{\pgfqpoint{9.300000in}{9.060000in}}%
\pgfusepath{clip}%
\pgfsetbuttcap%
\pgfsetmiterjoin%
\definecolor{currentfill}{rgb}{0.121569,0.466667,0.705882}%
\pgfsetfillcolor{currentfill}%
\pgfsetfillopacity{0.990000}%
\pgfsetlinewidth{0.000000pt}%
\definecolor{currentstroke}{rgb}{0.000000,0.000000,0.000000}%
\pgfsetstrokecolor{currentstroke}%
\pgfsetstrokeopacity{0.990000}%
\pgfsetdash{}{0pt}%
\pgfpathmoveto{\pgfqpoint{9.073315in}{7.186823in}}%
\pgfpathlineto{\pgfqpoint{9.848315in}{7.186823in}}%
\pgfpathlineto{\pgfqpoint{9.848315in}{9.266067in}}%
\pgfpathlineto{\pgfqpoint{9.073315in}{9.266067in}}%
\pgfpathclose%
\pgfusepath{fill}%
\end{pgfscope}%
\begin{pgfscope}%
\pgfsetbuttcap%
\pgfsetmiterjoin%
\definecolor{currentfill}{rgb}{0.121569,0.466667,0.705882}%
\pgfsetfillcolor{currentfill}%
\pgfsetfillopacity{0.990000}%
\pgfsetlinewidth{0.000000pt}%
\definecolor{currentstroke}{rgb}{0.000000,0.000000,0.000000}%
\pgfsetstrokecolor{currentstroke}%
\pgfsetstrokeopacity{0.990000}%
\pgfsetdash{}{0pt}%
\pgfpathrectangle{\pgfqpoint{0.935815in}{0.637495in}}{\pgfqpoint{9.300000in}{9.060000in}}%
\pgfusepath{clip}%
\pgfpathmoveto{\pgfqpoint{9.073315in}{7.186823in}}%
\pgfpathlineto{\pgfqpoint{9.848315in}{7.186823in}}%
\pgfpathlineto{\pgfqpoint{9.848315in}{9.266067in}}%
\pgfpathlineto{\pgfqpoint{9.073315in}{9.266067in}}%
\pgfpathclose%
\pgfusepath{clip}%
\pgfsys@defobject{currentpattern}{\pgfqpoint{0in}{0in}}{\pgfqpoint{1in}{1in}}{%
\begin{pgfscope}%
\pgfpathrectangle{\pgfqpoint{0in}{0in}}{\pgfqpoint{1in}{1in}}%
\pgfusepath{clip}%
\pgfpathmoveto{\pgfqpoint{0.000000in}{0.083333in}}%
\pgfpathlineto{\pgfqpoint{1.000000in}{0.083333in}}%
\pgfpathmoveto{\pgfqpoint{0.000000in}{0.250000in}}%
\pgfpathlineto{\pgfqpoint{1.000000in}{0.250000in}}%
\pgfpathmoveto{\pgfqpoint{0.000000in}{0.416667in}}%
\pgfpathlineto{\pgfqpoint{1.000000in}{0.416667in}}%
\pgfpathmoveto{\pgfqpoint{0.000000in}{0.583333in}}%
\pgfpathlineto{\pgfqpoint{1.000000in}{0.583333in}}%
\pgfpathmoveto{\pgfqpoint{0.000000in}{0.750000in}}%
\pgfpathlineto{\pgfqpoint{1.000000in}{0.750000in}}%
\pgfpathmoveto{\pgfqpoint{0.000000in}{0.916667in}}%
\pgfpathlineto{\pgfqpoint{1.000000in}{0.916667in}}%
\pgfpathmoveto{\pgfqpoint{0.083333in}{0.000000in}}%
\pgfpathlineto{\pgfqpoint{0.083333in}{1.000000in}}%
\pgfpathmoveto{\pgfqpoint{0.250000in}{0.000000in}}%
\pgfpathlineto{\pgfqpoint{0.250000in}{1.000000in}}%
\pgfpathmoveto{\pgfqpoint{0.416667in}{0.000000in}}%
\pgfpathlineto{\pgfqpoint{0.416667in}{1.000000in}}%
\pgfpathmoveto{\pgfqpoint{0.583333in}{0.000000in}}%
\pgfpathlineto{\pgfqpoint{0.583333in}{1.000000in}}%
\pgfpathmoveto{\pgfqpoint{0.750000in}{0.000000in}}%
\pgfpathlineto{\pgfqpoint{0.750000in}{1.000000in}}%
\pgfpathmoveto{\pgfqpoint{0.916667in}{0.000000in}}%
\pgfpathlineto{\pgfqpoint{0.916667in}{1.000000in}}%
\pgfusepath{stroke}%
\end{pgfscope}%
}%
\pgfsys@transformshift{9.073315in}{7.186823in}%
\pgfsys@useobject{currentpattern}{}%
\pgfsys@transformshift{1in}{0in}%
\pgfsys@transformshift{-1in}{0in}%
\pgfsys@transformshift{0in}{1in}%
\pgfsys@useobject{currentpattern}{}%
\pgfsys@transformshift{1in}{0in}%
\pgfsys@transformshift{-1in}{0in}%
\pgfsys@transformshift{0in}{1in}%
\pgfsys@useobject{currentpattern}{}%
\pgfsys@transformshift{1in}{0in}%
\pgfsys@transformshift{-1in}{0in}%
\pgfsys@transformshift{0in}{1in}%
\end{pgfscope}%
\begin{pgfscope}%
\pgfsetrectcap%
\pgfsetmiterjoin%
\pgfsetlinewidth{1.003750pt}%
\definecolor{currentstroke}{rgb}{1.000000,1.000000,1.000000}%
\pgfsetstrokecolor{currentstroke}%
\pgfsetdash{}{0pt}%
\pgfpathmoveto{\pgfqpoint{0.935815in}{0.637495in}}%
\pgfpathlineto{\pgfqpoint{0.935815in}{9.697495in}}%
\pgfusepath{stroke}%
\end{pgfscope}%
\begin{pgfscope}%
\pgfsetrectcap%
\pgfsetmiterjoin%
\pgfsetlinewidth{1.003750pt}%
\definecolor{currentstroke}{rgb}{1.000000,1.000000,1.000000}%
\pgfsetstrokecolor{currentstroke}%
\pgfsetdash{}{0pt}%
\pgfpathmoveto{\pgfqpoint{10.235815in}{0.637495in}}%
\pgfpathlineto{\pgfqpoint{10.235815in}{9.697495in}}%
\pgfusepath{stroke}%
\end{pgfscope}%
\begin{pgfscope}%
\pgfsetrectcap%
\pgfsetmiterjoin%
\pgfsetlinewidth{1.003750pt}%
\definecolor{currentstroke}{rgb}{1.000000,1.000000,1.000000}%
\pgfsetstrokecolor{currentstroke}%
\pgfsetdash{}{0pt}%
\pgfpathmoveto{\pgfqpoint{0.935815in}{0.637495in}}%
\pgfpathlineto{\pgfqpoint{10.235815in}{0.637495in}}%
\pgfusepath{stroke}%
\end{pgfscope}%
\begin{pgfscope}%
\pgfsetrectcap%
\pgfsetmiterjoin%
\pgfsetlinewidth{1.003750pt}%
\definecolor{currentstroke}{rgb}{1.000000,1.000000,1.000000}%
\pgfsetstrokecolor{currentstroke}%
\pgfsetdash{}{0pt}%
\pgfpathmoveto{\pgfqpoint{0.935815in}{9.697495in}}%
\pgfpathlineto{\pgfqpoint{10.235815in}{9.697495in}}%
\pgfusepath{stroke}%
\end{pgfscope}%
\begin{pgfscope}%
\definecolor{textcolor}{rgb}{0.000000,0.000000,0.000000}%
\pgfsetstrokecolor{textcolor}%
\pgfsetfillcolor{textcolor}%
\pgftext[x=5.585815in,y=9.780828in,,base]{\color{textcolor}\rmfamily\fontsize{24.000000}{28.800000}\selectfont UIUC Electric Capacity}%
\end{pgfscope}%
\begin{pgfscope}%
\pgfsetbuttcap%
\pgfsetmiterjoin%
\definecolor{currentfill}{rgb}{0.269412,0.269412,0.269412}%
\pgfsetfillcolor{currentfill}%
\pgfsetfillopacity{0.500000}%
\pgfsetlinewidth{0.501875pt}%
\definecolor{currentstroke}{rgb}{0.269412,0.269412,0.269412}%
\pgfsetstrokecolor{currentstroke}%
\pgfsetstrokeopacity{0.500000}%
\pgfsetdash{}{0pt}%
\pgfpathmoveto{\pgfqpoint{1.119148in}{7.545181in}}%
\pgfpathlineto{\pgfqpoint{3.338489in}{7.545181in}}%
\pgfpathquadraticcurveto{\pgfqpoint{3.382933in}{7.545181in}}{\pgfqpoint{3.382933in}{7.589626in}}%
\pgfpathlineto{\pgfqpoint{3.382933in}{9.514162in}}%
\pgfpathquadraticcurveto{\pgfqpoint{3.382933in}{9.558606in}}{\pgfqpoint{3.338489in}{9.558606in}}%
\pgfpathlineto{\pgfqpoint{1.119148in}{9.558606in}}%
\pgfpathquadraticcurveto{\pgfqpoint{1.074703in}{9.558606in}}{\pgfqpoint{1.074703in}{9.514162in}}%
\pgfpathlineto{\pgfqpoint{1.074703in}{7.589626in}}%
\pgfpathquadraticcurveto{\pgfqpoint{1.074703in}{7.545181in}}{\pgfqpoint{1.119148in}{7.545181in}}%
\pgfpathclose%
\pgfusepath{stroke,fill}%
\end{pgfscope}%
\begin{pgfscope}%
\pgfsetbuttcap%
\pgfsetmiterjoin%
\definecolor{currentfill}{rgb}{0.898039,0.898039,0.898039}%
\pgfsetfillcolor{currentfill}%
\pgfsetlinewidth{0.501875pt}%
\definecolor{currentstroke}{rgb}{0.800000,0.800000,0.800000}%
\pgfsetstrokecolor{currentstroke}%
\pgfsetdash{}{0pt}%
\pgfpathmoveto{\pgfqpoint{1.091370in}{7.572959in}}%
\pgfpathlineto{\pgfqpoint{3.310711in}{7.572959in}}%
\pgfpathquadraticcurveto{\pgfqpoint{3.355156in}{7.572959in}}{\pgfqpoint{3.355156in}{7.617403in}}%
\pgfpathlineto{\pgfqpoint{3.355156in}{9.541940in}}%
\pgfpathquadraticcurveto{\pgfqpoint{3.355156in}{9.586384in}}{\pgfqpoint{3.310711in}{9.586384in}}%
\pgfpathlineto{\pgfqpoint{1.091370in}{9.586384in}}%
\pgfpathquadraticcurveto{\pgfqpoint{1.046926in}{9.586384in}}{\pgfqpoint{1.046926in}{9.541940in}}%
\pgfpathlineto{\pgfqpoint{1.046926in}{7.617403in}}%
\pgfpathquadraticcurveto{\pgfqpoint{1.046926in}{7.572959in}}{\pgfqpoint{1.091370in}{7.572959in}}%
\pgfpathclose%
\pgfusepath{stroke,fill}%
\end{pgfscope}%
\begin{pgfscope}%
\pgfsetbuttcap%
\pgfsetmiterjoin%
\definecolor{currentfill}{rgb}{0.839216,0.152941,0.156863}%
\pgfsetfillcolor{currentfill}%
\pgfsetfillopacity{0.990000}%
\pgfsetlinewidth{0.000000pt}%
\definecolor{currentstroke}{rgb}{0.000000,0.000000,0.000000}%
\pgfsetstrokecolor{currentstroke}%
\pgfsetstrokeopacity{0.990000}%
\pgfsetdash{}{0pt}%
\pgfpathmoveto{\pgfqpoint{1.135815in}{9.330828in}}%
\pgfpathlineto{\pgfqpoint{1.580259in}{9.330828in}}%
\pgfpathlineto{\pgfqpoint{1.580259in}{9.486384in}}%
\pgfpathlineto{\pgfqpoint{1.135815in}{9.486384in}}%
\pgfpathclose%
\pgfusepath{fill}%
\end{pgfscope}%
\begin{pgfscope}%
\pgfsetbuttcap%
\pgfsetmiterjoin%
\definecolor{currentfill}{rgb}{0.839216,0.152941,0.156863}%
\pgfsetfillcolor{currentfill}%
\pgfsetfillopacity{0.990000}%
\pgfsetlinewidth{0.000000pt}%
\definecolor{currentstroke}{rgb}{0.000000,0.000000,0.000000}%
\pgfsetstrokecolor{currentstroke}%
\pgfsetstrokeopacity{0.990000}%
\pgfsetdash{}{0pt}%
\pgfpathmoveto{\pgfqpoint{1.135815in}{9.330828in}}%
\pgfpathlineto{\pgfqpoint{1.580259in}{9.330828in}}%
\pgfpathlineto{\pgfqpoint{1.580259in}{9.486384in}}%
\pgfpathlineto{\pgfqpoint{1.135815in}{9.486384in}}%
\pgfpathclose%
\pgfusepath{clip}%
\pgfsys@defobject{currentpattern}{\pgfqpoint{0in}{0in}}{\pgfqpoint{1in}{1in}}{%
\begin{pgfscope}%
\pgfpathrectangle{\pgfqpoint{0in}{0in}}{\pgfqpoint{1in}{1in}}%
\pgfusepath{clip}%
\pgfpathmoveto{\pgfqpoint{-0.500000in}{0.500000in}}%
\pgfpathlineto{\pgfqpoint{0.500000in}{1.500000in}}%
\pgfpathmoveto{\pgfqpoint{-0.333333in}{0.333333in}}%
\pgfpathlineto{\pgfqpoint{0.666667in}{1.333333in}}%
\pgfpathmoveto{\pgfqpoint{-0.166667in}{0.166667in}}%
\pgfpathlineto{\pgfqpoint{0.833333in}{1.166667in}}%
\pgfpathmoveto{\pgfqpoint{0.000000in}{0.000000in}}%
\pgfpathlineto{\pgfqpoint{1.000000in}{1.000000in}}%
\pgfpathmoveto{\pgfqpoint{0.166667in}{-0.166667in}}%
\pgfpathlineto{\pgfqpoint{1.166667in}{0.833333in}}%
\pgfpathmoveto{\pgfqpoint{0.333333in}{-0.333333in}}%
\pgfpathlineto{\pgfqpoint{1.333333in}{0.666667in}}%
\pgfpathmoveto{\pgfqpoint{0.500000in}{-0.500000in}}%
\pgfpathlineto{\pgfqpoint{1.500000in}{0.500000in}}%
\pgfpathmoveto{\pgfqpoint{-0.500000in}{0.500000in}}%
\pgfpathlineto{\pgfqpoint{0.500000in}{-0.500000in}}%
\pgfpathmoveto{\pgfqpoint{-0.333333in}{0.666667in}}%
\pgfpathlineto{\pgfqpoint{0.666667in}{-0.333333in}}%
\pgfpathmoveto{\pgfqpoint{-0.166667in}{0.833333in}}%
\pgfpathlineto{\pgfqpoint{0.833333in}{-0.166667in}}%
\pgfpathmoveto{\pgfqpoint{0.000000in}{1.000000in}}%
\pgfpathlineto{\pgfqpoint{1.000000in}{0.000000in}}%
\pgfpathmoveto{\pgfqpoint{0.166667in}{1.166667in}}%
\pgfpathlineto{\pgfqpoint{1.166667in}{0.166667in}}%
\pgfpathmoveto{\pgfqpoint{0.333333in}{1.333333in}}%
\pgfpathlineto{\pgfqpoint{1.333333in}{0.333333in}}%
\pgfpathmoveto{\pgfqpoint{0.500000in}{1.500000in}}%
\pgfpathlineto{\pgfqpoint{1.500000in}{0.500000in}}%
\pgfusepath{stroke}%
\end{pgfscope}%
}%
\pgfsys@transformshift{1.135815in}{9.330828in}%
\pgfsys@useobject{currentpattern}{}%
\pgfsys@transformshift{1in}{0in}%
\pgfsys@transformshift{-1in}{0in}%
\pgfsys@transformshift{0in}{1in}%
\end{pgfscope}%
\begin{pgfscope}%
\definecolor{textcolor}{rgb}{0.000000,0.000000,0.000000}%
\pgfsetstrokecolor{textcolor}%
\pgfsetfillcolor{textcolor}%
\pgftext[x=1.758037in,y=9.330828in,left,base]{\color{textcolor}\rmfamily\fontsize{16.000000}{19.200000}\selectfont ABBOTT\_TB}%
\end{pgfscope}%
\begin{pgfscope}%
\pgfsetbuttcap%
\pgfsetmiterjoin%
\definecolor{currentfill}{rgb}{0.549020,0.337255,0.294118}%
\pgfsetfillcolor{currentfill}%
\pgfsetfillopacity{0.990000}%
\pgfsetlinewidth{0.000000pt}%
\definecolor{currentstroke}{rgb}{0.000000,0.000000,0.000000}%
\pgfsetstrokecolor{currentstroke}%
\pgfsetstrokeopacity{0.990000}%
\pgfsetdash{}{0pt}%
\pgfpathmoveto{\pgfqpoint{1.135815in}{9.006369in}}%
\pgfpathlineto{\pgfqpoint{1.580259in}{9.006369in}}%
\pgfpathlineto{\pgfqpoint{1.580259in}{9.161924in}}%
\pgfpathlineto{\pgfqpoint{1.135815in}{9.161924in}}%
\pgfpathclose%
\pgfusepath{fill}%
\end{pgfscope}%
\begin{pgfscope}%
\pgfsetbuttcap%
\pgfsetmiterjoin%
\definecolor{currentfill}{rgb}{0.549020,0.337255,0.294118}%
\pgfsetfillcolor{currentfill}%
\pgfsetfillopacity{0.990000}%
\pgfsetlinewidth{0.000000pt}%
\definecolor{currentstroke}{rgb}{0.000000,0.000000,0.000000}%
\pgfsetstrokecolor{currentstroke}%
\pgfsetstrokeopacity{0.990000}%
\pgfsetdash{}{0pt}%
\pgfpathmoveto{\pgfqpoint{1.135815in}{9.006369in}}%
\pgfpathlineto{\pgfqpoint{1.580259in}{9.006369in}}%
\pgfpathlineto{\pgfqpoint{1.580259in}{9.161924in}}%
\pgfpathlineto{\pgfqpoint{1.135815in}{9.161924in}}%
\pgfpathclose%
\pgfusepath{clip}%
\pgfsys@defobject{currentpattern}{\pgfqpoint{0in}{0in}}{\pgfqpoint{1in}{1in}}{%
\begin{pgfscope}%
\pgfpathrectangle{\pgfqpoint{0in}{0in}}{\pgfqpoint{1in}{1in}}%
\pgfusepath{clip}%
\pgfpathmoveto{\pgfqpoint{0.000000in}{-0.058333in}}%
\pgfpathcurveto{\pgfqpoint{0.015470in}{-0.058333in}}{\pgfqpoint{0.030309in}{-0.052187in}}{\pgfqpoint{0.041248in}{-0.041248in}}%
\pgfpathcurveto{\pgfqpoint{0.052187in}{-0.030309in}}{\pgfqpoint{0.058333in}{-0.015470in}}{\pgfqpoint{0.058333in}{0.000000in}}%
\pgfpathcurveto{\pgfqpoint{0.058333in}{0.015470in}}{\pgfqpoint{0.052187in}{0.030309in}}{\pgfqpoint{0.041248in}{0.041248in}}%
\pgfpathcurveto{\pgfqpoint{0.030309in}{0.052187in}}{\pgfqpoint{0.015470in}{0.058333in}}{\pgfqpoint{0.000000in}{0.058333in}}%
\pgfpathcurveto{\pgfqpoint{-0.015470in}{0.058333in}}{\pgfqpoint{-0.030309in}{0.052187in}}{\pgfqpoint{-0.041248in}{0.041248in}}%
\pgfpathcurveto{\pgfqpoint{-0.052187in}{0.030309in}}{\pgfqpoint{-0.058333in}{0.015470in}}{\pgfqpoint{-0.058333in}{0.000000in}}%
\pgfpathcurveto{\pgfqpoint{-0.058333in}{-0.015470in}}{\pgfqpoint{-0.052187in}{-0.030309in}}{\pgfqpoint{-0.041248in}{-0.041248in}}%
\pgfpathcurveto{\pgfqpoint{-0.030309in}{-0.052187in}}{\pgfqpoint{-0.015470in}{-0.058333in}}{\pgfqpoint{0.000000in}{-0.058333in}}%
\pgfpathclose%
\pgfpathmoveto{\pgfqpoint{0.000000in}{-0.052500in}}%
\pgfpathcurveto{\pgfqpoint{0.000000in}{-0.052500in}}{\pgfqpoint{-0.013923in}{-0.052500in}}{\pgfqpoint{-0.027278in}{-0.046968in}}%
\pgfpathcurveto{\pgfqpoint{-0.037123in}{-0.037123in}}{\pgfqpoint{-0.046968in}{-0.027278in}}{\pgfqpoint{-0.052500in}{-0.013923in}}%
\pgfpathcurveto{\pgfqpoint{-0.052500in}{0.000000in}}{\pgfqpoint{-0.052500in}{0.013923in}}{\pgfqpoint{-0.046968in}{0.027278in}}%
\pgfpathcurveto{\pgfqpoint{-0.037123in}{0.037123in}}{\pgfqpoint{-0.027278in}{0.046968in}}{\pgfqpoint{-0.013923in}{0.052500in}}%
\pgfpathcurveto{\pgfqpoint{0.000000in}{0.052500in}}{\pgfqpoint{0.013923in}{0.052500in}}{\pgfqpoint{0.027278in}{0.046968in}}%
\pgfpathcurveto{\pgfqpoint{0.037123in}{0.037123in}}{\pgfqpoint{0.046968in}{0.027278in}}{\pgfqpoint{0.052500in}{0.013923in}}%
\pgfpathcurveto{\pgfqpoint{0.052500in}{0.000000in}}{\pgfqpoint{0.052500in}{-0.013923in}}{\pgfqpoint{0.046968in}{-0.027278in}}%
\pgfpathcurveto{\pgfqpoint{0.037123in}{-0.037123in}}{\pgfqpoint{0.027278in}{-0.046968in}}{\pgfqpoint{0.013923in}{-0.052500in}}%
\pgfpathclose%
\pgfpathmoveto{\pgfqpoint{0.166667in}{-0.058333in}}%
\pgfpathcurveto{\pgfqpoint{0.182137in}{-0.058333in}}{\pgfqpoint{0.196975in}{-0.052187in}}{\pgfqpoint{0.207915in}{-0.041248in}}%
\pgfpathcurveto{\pgfqpoint{0.218854in}{-0.030309in}}{\pgfqpoint{0.225000in}{-0.015470in}}{\pgfqpoint{0.225000in}{0.000000in}}%
\pgfpathcurveto{\pgfqpoint{0.225000in}{0.015470in}}{\pgfqpoint{0.218854in}{0.030309in}}{\pgfqpoint{0.207915in}{0.041248in}}%
\pgfpathcurveto{\pgfqpoint{0.196975in}{0.052187in}}{\pgfqpoint{0.182137in}{0.058333in}}{\pgfqpoint{0.166667in}{0.058333in}}%
\pgfpathcurveto{\pgfqpoint{0.151196in}{0.058333in}}{\pgfqpoint{0.136358in}{0.052187in}}{\pgfqpoint{0.125419in}{0.041248in}}%
\pgfpathcurveto{\pgfqpoint{0.114480in}{0.030309in}}{\pgfqpoint{0.108333in}{0.015470in}}{\pgfqpoint{0.108333in}{0.000000in}}%
\pgfpathcurveto{\pgfqpoint{0.108333in}{-0.015470in}}{\pgfqpoint{0.114480in}{-0.030309in}}{\pgfqpoint{0.125419in}{-0.041248in}}%
\pgfpathcurveto{\pgfqpoint{0.136358in}{-0.052187in}}{\pgfqpoint{0.151196in}{-0.058333in}}{\pgfqpoint{0.166667in}{-0.058333in}}%
\pgfpathclose%
\pgfpathmoveto{\pgfqpoint{0.166667in}{-0.052500in}}%
\pgfpathcurveto{\pgfqpoint{0.166667in}{-0.052500in}}{\pgfqpoint{0.152744in}{-0.052500in}}{\pgfqpoint{0.139389in}{-0.046968in}}%
\pgfpathcurveto{\pgfqpoint{0.129544in}{-0.037123in}}{\pgfqpoint{0.119698in}{-0.027278in}}{\pgfqpoint{0.114167in}{-0.013923in}}%
\pgfpathcurveto{\pgfqpoint{0.114167in}{0.000000in}}{\pgfqpoint{0.114167in}{0.013923in}}{\pgfqpoint{0.119698in}{0.027278in}}%
\pgfpathcurveto{\pgfqpoint{0.129544in}{0.037123in}}{\pgfqpoint{0.139389in}{0.046968in}}{\pgfqpoint{0.152744in}{0.052500in}}%
\pgfpathcurveto{\pgfqpoint{0.166667in}{0.052500in}}{\pgfqpoint{0.180590in}{0.052500in}}{\pgfqpoint{0.193945in}{0.046968in}}%
\pgfpathcurveto{\pgfqpoint{0.203790in}{0.037123in}}{\pgfqpoint{0.213635in}{0.027278in}}{\pgfqpoint{0.219167in}{0.013923in}}%
\pgfpathcurveto{\pgfqpoint{0.219167in}{0.000000in}}{\pgfqpoint{0.219167in}{-0.013923in}}{\pgfqpoint{0.213635in}{-0.027278in}}%
\pgfpathcurveto{\pgfqpoint{0.203790in}{-0.037123in}}{\pgfqpoint{0.193945in}{-0.046968in}}{\pgfqpoint{0.180590in}{-0.052500in}}%
\pgfpathclose%
\pgfpathmoveto{\pgfqpoint{0.333333in}{-0.058333in}}%
\pgfpathcurveto{\pgfqpoint{0.348804in}{-0.058333in}}{\pgfqpoint{0.363642in}{-0.052187in}}{\pgfqpoint{0.374581in}{-0.041248in}}%
\pgfpathcurveto{\pgfqpoint{0.385520in}{-0.030309in}}{\pgfqpoint{0.391667in}{-0.015470in}}{\pgfqpoint{0.391667in}{0.000000in}}%
\pgfpathcurveto{\pgfqpoint{0.391667in}{0.015470in}}{\pgfqpoint{0.385520in}{0.030309in}}{\pgfqpoint{0.374581in}{0.041248in}}%
\pgfpathcurveto{\pgfqpoint{0.363642in}{0.052187in}}{\pgfqpoint{0.348804in}{0.058333in}}{\pgfqpoint{0.333333in}{0.058333in}}%
\pgfpathcurveto{\pgfqpoint{0.317863in}{0.058333in}}{\pgfqpoint{0.303025in}{0.052187in}}{\pgfqpoint{0.292085in}{0.041248in}}%
\pgfpathcurveto{\pgfqpoint{0.281146in}{0.030309in}}{\pgfqpoint{0.275000in}{0.015470in}}{\pgfqpoint{0.275000in}{0.000000in}}%
\pgfpathcurveto{\pgfqpoint{0.275000in}{-0.015470in}}{\pgfqpoint{0.281146in}{-0.030309in}}{\pgfqpoint{0.292085in}{-0.041248in}}%
\pgfpathcurveto{\pgfqpoint{0.303025in}{-0.052187in}}{\pgfqpoint{0.317863in}{-0.058333in}}{\pgfqpoint{0.333333in}{-0.058333in}}%
\pgfpathclose%
\pgfpathmoveto{\pgfqpoint{0.333333in}{-0.052500in}}%
\pgfpathcurveto{\pgfqpoint{0.333333in}{-0.052500in}}{\pgfqpoint{0.319410in}{-0.052500in}}{\pgfqpoint{0.306055in}{-0.046968in}}%
\pgfpathcurveto{\pgfqpoint{0.296210in}{-0.037123in}}{\pgfqpoint{0.286365in}{-0.027278in}}{\pgfqpoint{0.280833in}{-0.013923in}}%
\pgfpathcurveto{\pgfqpoint{0.280833in}{0.000000in}}{\pgfqpoint{0.280833in}{0.013923in}}{\pgfqpoint{0.286365in}{0.027278in}}%
\pgfpathcurveto{\pgfqpoint{0.296210in}{0.037123in}}{\pgfqpoint{0.306055in}{0.046968in}}{\pgfqpoint{0.319410in}{0.052500in}}%
\pgfpathcurveto{\pgfqpoint{0.333333in}{0.052500in}}{\pgfqpoint{0.347256in}{0.052500in}}{\pgfqpoint{0.360611in}{0.046968in}}%
\pgfpathcurveto{\pgfqpoint{0.370456in}{0.037123in}}{\pgfqpoint{0.380302in}{0.027278in}}{\pgfqpoint{0.385833in}{0.013923in}}%
\pgfpathcurveto{\pgfqpoint{0.385833in}{0.000000in}}{\pgfqpoint{0.385833in}{-0.013923in}}{\pgfqpoint{0.380302in}{-0.027278in}}%
\pgfpathcurveto{\pgfqpoint{0.370456in}{-0.037123in}}{\pgfqpoint{0.360611in}{-0.046968in}}{\pgfqpoint{0.347256in}{-0.052500in}}%
\pgfpathclose%
\pgfpathmoveto{\pgfqpoint{0.500000in}{-0.058333in}}%
\pgfpathcurveto{\pgfqpoint{0.515470in}{-0.058333in}}{\pgfqpoint{0.530309in}{-0.052187in}}{\pgfqpoint{0.541248in}{-0.041248in}}%
\pgfpathcurveto{\pgfqpoint{0.552187in}{-0.030309in}}{\pgfqpoint{0.558333in}{-0.015470in}}{\pgfqpoint{0.558333in}{0.000000in}}%
\pgfpathcurveto{\pgfqpoint{0.558333in}{0.015470in}}{\pgfqpoint{0.552187in}{0.030309in}}{\pgfqpoint{0.541248in}{0.041248in}}%
\pgfpathcurveto{\pgfqpoint{0.530309in}{0.052187in}}{\pgfqpoint{0.515470in}{0.058333in}}{\pgfqpoint{0.500000in}{0.058333in}}%
\pgfpathcurveto{\pgfqpoint{0.484530in}{0.058333in}}{\pgfqpoint{0.469691in}{0.052187in}}{\pgfqpoint{0.458752in}{0.041248in}}%
\pgfpathcurveto{\pgfqpoint{0.447813in}{0.030309in}}{\pgfqpoint{0.441667in}{0.015470in}}{\pgfqpoint{0.441667in}{0.000000in}}%
\pgfpathcurveto{\pgfqpoint{0.441667in}{-0.015470in}}{\pgfqpoint{0.447813in}{-0.030309in}}{\pgfqpoint{0.458752in}{-0.041248in}}%
\pgfpathcurveto{\pgfqpoint{0.469691in}{-0.052187in}}{\pgfqpoint{0.484530in}{-0.058333in}}{\pgfqpoint{0.500000in}{-0.058333in}}%
\pgfpathclose%
\pgfpathmoveto{\pgfqpoint{0.500000in}{-0.052500in}}%
\pgfpathcurveto{\pgfqpoint{0.500000in}{-0.052500in}}{\pgfqpoint{0.486077in}{-0.052500in}}{\pgfqpoint{0.472722in}{-0.046968in}}%
\pgfpathcurveto{\pgfqpoint{0.462877in}{-0.037123in}}{\pgfqpoint{0.453032in}{-0.027278in}}{\pgfqpoint{0.447500in}{-0.013923in}}%
\pgfpathcurveto{\pgfqpoint{0.447500in}{0.000000in}}{\pgfqpoint{0.447500in}{0.013923in}}{\pgfqpoint{0.453032in}{0.027278in}}%
\pgfpathcurveto{\pgfqpoint{0.462877in}{0.037123in}}{\pgfqpoint{0.472722in}{0.046968in}}{\pgfqpoint{0.486077in}{0.052500in}}%
\pgfpathcurveto{\pgfqpoint{0.500000in}{0.052500in}}{\pgfqpoint{0.513923in}{0.052500in}}{\pgfqpoint{0.527278in}{0.046968in}}%
\pgfpathcurveto{\pgfqpoint{0.537123in}{0.037123in}}{\pgfqpoint{0.546968in}{0.027278in}}{\pgfqpoint{0.552500in}{0.013923in}}%
\pgfpathcurveto{\pgfqpoint{0.552500in}{0.000000in}}{\pgfqpoint{0.552500in}{-0.013923in}}{\pgfqpoint{0.546968in}{-0.027278in}}%
\pgfpathcurveto{\pgfqpoint{0.537123in}{-0.037123in}}{\pgfqpoint{0.527278in}{-0.046968in}}{\pgfqpoint{0.513923in}{-0.052500in}}%
\pgfpathclose%
\pgfpathmoveto{\pgfqpoint{0.666667in}{-0.058333in}}%
\pgfpathcurveto{\pgfqpoint{0.682137in}{-0.058333in}}{\pgfqpoint{0.696975in}{-0.052187in}}{\pgfqpoint{0.707915in}{-0.041248in}}%
\pgfpathcurveto{\pgfqpoint{0.718854in}{-0.030309in}}{\pgfqpoint{0.725000in}{-0.015470in}}{\pgfqpoint{0.725000in}{0.000000in}}%
\pgfpathcurveto{\pgfqpoint{0.725000in}{0.015470in}}{\pgfqpoint{0.718854in}{0.030309in}}{\pgfqpoint{0.707915in}{0.041248in}}%
\pgfpathcurveto{\pgfqpoint{0.696975in}{0.052187in}}{\pgfqpoint{0.682137in}{0.058333in}}{\pgfqpoint{0.666667in}{0.058333in}}%
\pgfpathcurveto{\pgfqpoint{0.651196in}{0.058333in}}{\pgfqpoint{0.636358in}{0.052187in}}{\pgfqpoint{0.625419in}{0.041248in}}%
\pgfpathcurveto{\pgfqpoint{0.614480in}{0.030309in}}{\pgfqpoint{0.608333in}{0.015470in}}{\pgfqpoint{0.608333in}{0.000000in}}%
\pgfpathcurveto{\pgfqpoint{0.608333in}{-0.015470in}}{\pgfqpoint{0.614480in}{-0.030309in}}{\pgfqpoint{0.625419in}{-0.041248in}}%
\pgfpathcurveto{\pgfqpoint{0.636358in}{-0.052187in}}{\pgfqpoint{0.651196in}{-0.058333in}}{\pgfqpoint{0.666667in}{-0.058333in}}%
\pgfpathclose%
\pgfpathmoveto{\pgfqpoint{0.666667in}{-0.052500in}}%
\pgfpathcurveto{\pgfqpoint{0.666667in}{-0.052500in}}{\pgfqpoint{0.652744in}{-0.052500in}}{\pgfqpoint{0.639389in}{-0.046968in}}%
\pgfpathcurveto{\pgfqpoint{0.629544in}{-0.037123in}}{\pgfqpoint{0.619698in}{-0.027278in}}{\pgfqpoint{0.614167in}{-0.013923in}}%
\pgfpathcurveto{\pgfqpoint{0.614167in}{0.000000in}}{\pgfqpoint{0.614167in}{0.013923in}}{\pgfqpoint{0.619698in}{0.027278in}}%
\pgfpathcurveto{\pgfqpoint{0.629544in}{0.037123in}}{\pgfqpoint{0.639389in}{0.046968in}}{\pgfqpoint{0.652744in}{0.052500in}}%
\pgfpathcurveto{\pgfqpoint{0.666667in}{0.052500in}}{\pgfqpoint{0.680590in}{0.052500in}}{\pgfqpoint{0.693945in}{0.046968in}}%
\pgfpathcurveto{\pgfqpoint{0.703790in}{0.037123in}}{\pgfqpoint{0.713635in}{0.027278in}}{\pgfqpoint{0.719167in}{0.013923in}}%
\pgfpathcurveto{\pgfqpoint{0.719167in}{0.000000in}}{\pgfqpoint{0.719167in}{-0.013923in}}{\pgfqpoint{0.713635in}{-0.027278in}}%
\pgfpathcurveto{\pgfqpoint{0.703790in}{-0.037123in}}{\pgfqpoint{0.693945in}{-0.046968in}}{\pgfqpoint{0.680590in}{-0.052500in}}%
\pgfpathclose%
\pgfpathmoveto{\pgfqpoint{0.833333in}{-0.058333in}}%
\pgfpathcurveto{\pgfqpoint{0.848804in}{-0.058333in}}{\pgfqpoint{0.863642in}{-0.052187in}}{\pgfqpoint{0.874581in}{-0.041248in}}%
\pgfpathcurveto{\pgfqpoint{0.885520in}{-0.030309in}}{\pgfqpoint{0.891667in}{-0.015470in}}{\pgfqpoint{0.891667in}{0.000000in}}%
\pgfpathcurveto{\pgfqpoint{0.891667in}{0.015470in}}{\pgfqpoint{0.885520in}{0.030309in}}{\pgfqpoint{0.874581in}{0.041248in}}%
\pgfpathcurveto{\pgfqpoint{0.863642in}{0.052187in}}{\pgfqpoint{0.848804in}{0.058333in}}{\pgfqpoint{0.833333in}{0.058333in}}%
\pgfpathcurveto{\pgfqpoint{0.817863in}{0.058333in}}{\pgfqpoint{0.803025in}{0.052187in}}{\pgfqpoint{0.792085in}{0.041248in}}%
\pgfpathcurveto{\pgfqpoint{0.781146in}{0.030309in}}{\pgfqpoint{0.775000in}{0.015470in}}{\pgfqpoint{0.775000in}{0.000000in}}%
\pgfpathcurveto{\pgfqpoint{0.775000in}{-0.015470in}}{\pgfqpoint{0.781146in}{-0.030309in}}{\pgfqpoint{0.792085in}{-0.041248in}}%
\pgfpathcurveto{\pgfqpoint{0.803025in}{-0.052187in}}{\pgfqpoint{0.817863in}{-0.058333in}}{\pgfqpoint{0.833333in}{-0.058333in}}%
\pgfpathclose%
\pgfpathmoveto{\pgfqpoint{0.833333in}{-0.052500in}}%
\pgfpathcurveto{\pgfqpoint{0.833333in}{-0.052500in}}{\pgfqpoint{0.819410in}{-0.052500in}}{\pgfqpoint{0.806055in}{-0.046968in}}%
\pgfpathcurveto{\pgfqpoint{0.796210in}{-0.037123in}}{\pgfqpoint{0.786365in}{-0.027278in}}{\pgfqpoint{0.780833in}{-0.013923in}}%
\pgfpathcurveto{\pgfqpoint{0.780833in}{0.000000in}}{\pgfqpoint{0.780833in}{0.013923in}}{\pgfqpoint{0.786365in}{0.027278in}}%
\pgfpathcurveto{\pgfqpoint{0.796210in}{0.037123in}}{\pgfqpoint{0.806055in}{0.046968in}}{\pgfqpoint{0.819410in}{0.052500in}}%
\pgfpathcurveto{\pgfqpoint{0.833333in}{0.052500in}}{\pgfqpoint{0.847256in}{0.052500in}}{\pgfqpoint{0.860611in}{0.046968in}}%
\pgfpathcurveto{\pgfqpoint{0.870456in}{0.037123in}}{\pgfqpoint{0.880302in}{0.027278in}}{\pgfqpoint{0.885833in}{0.013923in}}%
\pgfpathcurveto{\pgfqpoint{0.885833in}{0.000000in}}{\pgfqpoint{0.885833in}{-0.013923in}}{\pgfqpoint{0.880302in}{-0.027278in}}%
\pgfpathcurveto{\pgfqpoint{0.870456in}{-0.037123in}}{\pgfqpoint{0.860611in}{-0.046968in}}{\pgfqpoint{0.847256in}{-0.052500in}}%
\pgfpathclose%
\pgfpathmoveto{\pgfqpoint{1.000000in}{-0.058333in}}%
\pgfpathcurveto{\pgfqpoint{1.015470in}{-0.058333in}}{\pgfqpoint{1.030309in}{-0.052187in}}{\pgfqpoint{1.041248in}{-0.041248in}}%
\pgfpathcurveto{\pgfqpoint{1.052187in}{-0.030309in}}{\pgfqpoint{1.058333in}{-0.015470in}}{\pgfqpoint{1.058333in}{0.000000in}}%
\pgfpathcurveto{\pgfqpoint{1.058333in}{0.015470in}}{\pgfqpoint{1.052187in}{0.030309in}}{\pgfqpoint{1.041248in}{0.041248in}}%
\pgfpathcurveto{\pgfqpoint{1.030309in}{0.052187in}}{\pgfqpoint{1.015470in}{0.058333in}}{\pgfqpoint{1.000000in}{0.058333in}}%
\pgfpathcurveto{\pgfqpoint{0.984530in}{0.058333in}}{\pgfqpoint{0.969691in}{0.052187in}}{\pgfqpoint{0.958752in}{0.041248in}}%
\pgfpathcurveto{\pgfqpoint{0.947813in}{0.030309in}}{\pgfqpoint{0.941667in}{0.015470in}}{\pgfqpoint{0.941667in}{0.000000in}}%
\pgfpathcurveto{\pgfqpoint{0.941667in}{-0.015470in}}{\pgfqpoint{0.947813in}{-0.030309in}}{\pgfqpoint{0.958752in}{-0.041248in}}%
\pgfpathcurveto{\pgfqpoint{0.969691in}{-0.052187in}}{\pgfqpoint{0.984530in}{-0.058333in}}{\pgfqpoint{1.000000in}{-0.058333in}}%
\pgfpathclose%
\pgfpathmoveto{\pgfqpoint{1.000000in}{-0.052500in}}%
\pgfpathcurveto{\pgfqpoint{1.000000in}{-0.052500in}}{\pgfqpoint{0.986077in}{-0.052500in}}{\pgfqpoint{0.972722in}{-0.046968in}}%
\pgfpathcurveto{\pgfqpoint{0.962877in}{-0.037123in}}{\pgfqpoint{0.953032in}{-0.027278in}}{\pgfqpoint{0.947500in}{-0.013923in}}%
\pgfpathcurveto{\pgfqpoint{0.947500in}{0.000000in}}{\pgfqpoint{0.947500in}{0.013923in}}{\pgfqpoint{0.953032in}{0.027278in}}%
\pgfpathcurveto{\pgfqpoint{0.962877in}{0.037123in}}{\pgfqpoint{0.972722in}{0.046968in}}{\pgfqpoint{0.986077in}{0.052500in}}%
\pgfpathcurveto{\pgfqpoint{1.000000in}{0.052500in}}{\pgfqpoint{1.013923in}{0.052500in}}{\pgfqpoint{1.027278in}{0.046968in}}%
\pgfpathcurveto{\pgfqpoint{1.037123in}{0.037123in}}{\pgfqpoint{1.046968in}{0.027278in}}{\pgfqpoint{1.052500in}{0.013923in}}%
\pgfpathcurveto{\pgfqpoint{1.052500in}{0.000000in}}{\pgfqpoint{1.052500in}{-0.013923in}}{\pgfqpoint{1.046968in}{-0.027278in}}%
\pgfpathcurveto{\pgfqpoint{1.037123in}{-0.037123in}}{\pgfqpoint{1.027278in}{-0.046968in}}{\pgfqpoint{1.013923in}{-0.052500in}}%
\pgfpathclose%
\pgfpathmoveto{\pgfqpoint{0.083333in}{0.108333in}}%
\pgfpathcurveto{\pgfqpoint{0.098804in}{0.108333in}}{\pgfqpoint{0.113642in}{0.114480in}}{\pgfqpoint{0.124581in}{0.125419in}}%
\pgfpathcurveto{\pgfqpoint{0.135520in}{0.136358in}}{\pgfqpoint{0.141667in}{0.151196in}}{\pgfqpoint{0.141667in}{0.166667in}}%
\pgfpathcurveto{\pgfqpoint{0.141667in}{0.182137in}}{\pgfqpoint{0.135520in}{0.196975in}}{\pgfqpoint{0.124581in}{0.207915in}}%
\pgfpathcurveto{\pgfqpoint{0.113642in}{0.218854in}}{\pgfqpoint{0.098804in}{0.225000in}}{\pgfqpoint{0.083333in}{0.225000in}}%
\pgfpathcurveto{\pgfqpoint{0.067863in}{0.225000in}}{\pgfqpoint{0.053025in}{0.218854in}}{\pgfqpoint{0.042085in}{0.207915in}}%
\pgfpathcurveto{\pgfqpoint{0.031146in}{0.196975in}}{\pgfqpoint{0.025000in}{0.182137in}}{\pgfqpoint{0.025000in}{0.166667in}}%
\pgfpathcurveto{\pgfqpoint{0.025000in}{0.151196in}}{\pgfqpoint{0.031146in}{0.136358in}}{\pgfqpoint{0.042085in}{0.125419in}}%
\pgfpathcurveto{\pgfqpoint{0.053025in}{0.114480in}}{\pgfqpoint{0.067863in}{0.108333in}}{\pgfqpoint{0.083333in}{0.108333in}}%
\pgfpathclose%
\pgfpathmoveto{\pgfqpoint{0.083333in}{0.114167in}}%
\pgfpathcurveto{\pgfqpoint{0.083333in}{0.114167in}}{\pgfqpoint{0.069410in}{0.114167in}}{\pgfqpoint{0.056055in}{0.119698in}}%
\pgfpathcurveto{\pgfqpoint{0.046210in}{0.129544in}}{\pgfqpoint{0.036365in}{0.139389in}}{\pgfqpoint{0.030833in}{0.152744in}}%
\pgfpathcurveto{\pgfqpoint{0.030833in}{0.166667in}}{\pgfqpoint{0.030833in}{0.180590in}}{\pgfqpoint{0.036365in}{0.193945in}}%
\pgfpathcurveto{\pgfqpoint{0.046210in}{0.203790in}}{\pgfqpoint{0.056055in}{0.213635in}}{\pgfqpoint{0.069410in}{0.219167in}}%
\pgfpathcurveto{\pgfqpoint{0.083333in}{0.219167in}}{\pgfqpoint{0.097256in}{0.219167in}}{\pgfqpoint{0.110611in}{0.213635in}}%
\pgfpathcurveto{\pgfqpoint{0.120456in}{0.203790in}}{\pgfqpoint{0.130302in}{0.193945in}}{\pgfqpoint{0.135833in}{0.180590in}}%
\pgfpathcurveto{\pgfqpoint{0.135833in}{0.166667in}}{\pgfqpoint{0.135833in}{0.152744in}}{\pgfqpoint{0.130302in}{0.139389in}}%
\pgfpathcurveto{\pgfqpoint{0.120456in}{0.129544in}}{\pgfqpoint{0.110611in}{0.119698in}}{\pgfqpoint{0.097256in}{0.114167in}}%
\pgfpathclose%
\pgfpathmoveto{\pgfqpoint{0.250000in}{0.108333in}}%
\pgfpathcurveto{\pgfqpoint{0.265470in}{0.108333in}}{\pgfqpoint{0.280309in}{0.114480in}}{\pgfqpoint{0.291248in}{0.125419in}}%
\pgfpathcurveto{\pgfqpoint{0.302187in}{0.136358in}}{\pgfqpoint{0.308333in}{0.151196in}}{\pgfqpoint{0.308333in}{0.166667in}}%
\pgfpathcurveto{\pgfqpoint{0.308333in}{0.182137in}}{\pgfqpoint{0.302187in}{0.196975in}}{\pgfqpoint{0.291248in}{0.207915in}}%
\pgfpathcurveto{\pgfqpoint{0.280309in}{0.218854in}}{\pgfqpoint{0.265470in}{0.225000in}}{\pgfqpoint{0.250000in}{0.225000in}}%
\pgfpathcurveto{\pgfqpoint{0.234530in}{0.225000in}}{\pgfqpoint{0.219691in}{0.218854in}}{\pgfqpoint{0.208752in}{0.207915in}}%
\pgfpathcurveto{\pgfqpoint{0.197813in}{0.196975in}}{\pgfqpoint{0.191667in}{0.182137in}}{\pgfqpoint{0.191667in}{0.166667in}}%
\pgfpathcurveto{\pgfqpoint{0.191667in}{0.151196in}}{\pgfqpoint{0.197813in}{0.136358in}}{\pgfqpoint{0.208752in}{0.125419in}}%
\pgfpathcurveto{\pgfqpoint{0.219691in}{0.114480in}}{\pgfqpoint{0.234530in}{0.108333in}}{\pgfqpoint{0.250000in}{0.108333in}}%
\pgfpathclose%
\pgfpathmoveto{\pgfqpoint{0.250000in}{0.114167in}}%
\pgfpathcurveto{\pgfqpoint{0.250000in}{0.114167in}}{\pgfqpoint{0.236077in}{0.114167in}}{\pgfqpoint{0.222722in}{0.119698in}}%
\pgfpathcurveto{\pgfqpoint{0.212877in}{0.129544in}}{\pgfqpoint{0.203032in}{0.139389in}}{\pgfqpoint{0.197500in}{0.152744in}}%
\pgfpathcurveto{\pgfqpoint{0.197500in}{0.166667in}}{\pgfqpoint{0.197500in}{0.180590in}}{\pgfqpoint{0.203032in}{0.193945in}}%
\pgfpathcurveto{\pgfqpoint{0.212877in}{0.203790in}}{\pgfqpoint{0.222722in}{0.213635in}}{\pgfqpoint{0.236077in}{0.219167in}}%
\pgfpathcurveto{\pgfqpoint{0.250000in}{0.219167in}}{\pgfqpoint{0.263923in}{0.219167in}}{\pgfqpoint{0.277278in}{0.213635in}}%
\pgfpathcurveto{\pgfqpoint{0.287123in}{0.203790in}}{\pgfqpoint{0.296968in}{0.193945in}}{\pgfqpoint{0.302500in}{0.180590in}}%
\pgfpathcurveto{\pgfqpoint{0.302500in}{0.166667in}}{\pgfqpoint{0.302500in}{0.152744in}}{\pgfqpoint{0.296968in}{0.139389in}}%
\pgfpathcurveto{\pgfqpoint{0.287123in}{0.129544in}}{\pgfqpoint{0.277278in}{0.119698in}}{\pgfqpoint{0.263923in}{0.114167in}}%
\pgfpathclose%
\pgfpathmoveto{\pgfqpoint{0.416667in}{0.108333in}}%
\pgfpathcurveto{\pgfqpoint{0.432137in}{0.108333in}}{\pgfqpoint{0.446975in}{0.114480in}}{\pgfqpoint{0.457915in}{0.125419in}}%
\pgfpathcurveto{\pgfqpoint{0.468854in}{0.136358in}}{\pgfqpoint{0.475000in}{0.151196in}}{\pgfqpoint{0.475000in}{0.166667in}}%
\pgfpathcurveto{\pgfqpoint{0.475000in}{0.182137in}}{\pgfqpoint{0.468854in}{0.196975in}}{\pgfqpoint{0.457915in}{0.207915in}}%
\pgfpathcurveto{\pgfqpoint{0.446975in}{0.218854in}}{\pgfqpoint{0.432137in}{0.225000in}}{\pgfqpoint{0.416667in}{0.225000in}}%
\pgfpathcurveto{\pgfqpoint{0.401196in}{0.225000in}}{\pgfqpoint{0.386358in}{0.218854in}}{\pgfqpoint{0.375419in}{0.207915in}}%
\pgfpathcurveto{\pgfqpoint{0.364480in}{0.196975in}}{\pgfqpoint{0.358333in}{0.182137in}}{\pgfqpoint{0.358333in}{0.166667in}}%
\pgfpathcurveto{\pgfqpoint{0.358333in}{0.151196in}}{\pgfqpoint{0.364480in}{0.136358in}}{\pgfqpoint{0.375419in}{0.125419in}}%
\pgfpathcurveto{\pgfqpoint{0.386358in}{0.114480in}}{\pgfqpoint{0.401196in}{0.108333in}}{\pgfqpoint{0.416667in}{0.108333in}}%
\pgfpathclose%
\pgfpathmoveto{\pgfqpoint{0.416667in}{0.114167in}}%
\pgfpathcurveto{\pgfqpoint{0.416667in}{0.114167in}}{\pgfqpoint{0.402744in}{0.114167in}}{\pgfqpoint{0.389389in}{0.119698in}}%
\pgfpathcurveto{\pgfqpoint{0.379544in}{0.129544in}}{\pgfqpoint{0.369698in}{0.139389in}}{\pgfqpoint{0.364167in}{0.152744in}}%
\pgfpathcurveto{\pgfqpoint{0.364167in}{0.166667in}}{\pgfqpoint{0.364167in}{0.180590in}}{\pgfqpoint{0.369698in}{0.193945in}}%
\pgfpathcurveto{\pgfqpoint{0.379544in}{0.203790in}}{\pgfqpoint{0.389389in}{0.213635in}}{\pgfqpoint{0.402744in}{0.219167in}}%
\pgfpathcurveto{\pgfqpoint{0.416667in}{0.219167in}}{\pgfqpoint{0.430590in}{0.219167in}}{\pgfqpoint{0.443945in}{0.213635in}}%
\pgfpathcurveto{\pgfqpoint{0.453790in}{0.203790in}}{\pgfqpoint{0.463635in}{0.193945in}}{\pgfqpoint{0.469167in}{0.180590in}}%
\pgfpathcurveto{\pgfqpoint{0.469167in}{0.166667in}}{\pgfqpoint{0.469167in}{0.152744in}}{\pgfqpoint{0.463635in}{0.139389in}}%
\pgfpathcurveto{\pgfqpoint{0.453790in}{0.129544in}}{\pgfqpoint{0.443945in}{0.119698in}}{\pgfqpoint{0.430590in}{0.114167in}}%
\pgfpathclose%
\pgfpathmoveto{\pgfqpoint{0.583333in}{0.108333in}}%
\pgfpathcurveto{\pgfqpoint{0.598804in}{0.108333in}}{\pgfqpoint{0.613642in}{0.114480in}}{\pgfqpoint{0.624581in}{0.125419in}}%
\pgfpathcurveto{\pgfqpoint{0.635520in}{0.136358in}}{\pgfqpoint{0.641667in}{0.151196in}}{\pgfqpoint{0.641667in}{0.166667in}}%
\pgfpathcurveto{\pgfqpoint{0.641667in}{0.182137in}}{\pgfqpoint{0.635520in}{0.196975in}}{\pgfqpoint{0.624581in}{0.207915in}}%
\pgfpathcurveto{\pgfqpoint{0.613642in}{0.218854in}}{\pgfqpoint{0.598804in}{0.225000in}}{\pgfqpoint{0.583333in}{0.225000in}}%
\pgfpathcurveto{\pgfqpoint{0.567863in}{0.225000in}}{\pgfqpoint{0.553025in}{0.218854in}}{\pgfqpoint{0.542085in}{0.207915in}}%
\pgfpathcurveto{\pgfqpoint{0.531146in}{0.196975in}}{\pgfqpoint{0.525000in}{0.182137in}}{\pgfqpoint{0.525000in}{0.166667in}}%
\pgfpathcurveto{\pgfqpoint{0.525000in}{0.151196in}}{\pgfqpoint{0.531146in}{0.136358in}}{\pgfqpoint{0.542085in}{0.125419in}}%
\pgfpathcurveto{\pgfqpoint{0.553025in}{0.114480in}}{\pgfqpoint{0.567863in}{0.108333in}}{\pgfqpoint{0.583333in}{0.108333in}}%
\pgfpathclose%
\pgfpathmoveto{\pgfqpoint{0.583333in}{0.114167in}}%
\pgfpathcurveto{\pgfqpoint{0.583333in}{0.114167in}}{\pgfqpoint{0.569410in}{0.114167in}}{\pgfqpoint{0.556055in}{0.119698in}}%
\pgfpathcurveto{\pgfqpoint{0.546210in}{0.129544in}}{\pgfqpoint{0.536365in}{0.139389in}}{\pgfqpoint{0.530833in}{0.152744in}}%
\pgfpathcurveto{\pgfqpoint{0.530833in}{0.166667in}}{\pgfqpoint{0.530833in}{0.180590in}}{\pgfqpoint{0.536365in}{0.193945in}}%
\pgfpathcurveto{\pgfqpoint{0.546210in}{0.203790in}}{\pgfqpoint{0.556055in}{0.213635in}}{\pgfqpoint{0.569410in}{0.219167in}}%
\pgfpathcurveto{\pgfqpoint{0.583333in}{0.219167in}}{\pgfqpoint{0.597256in}{0.219167in}}{\pgfqpoint{0.610611in}{0.213635in}}%
\pgfpathcurveto{\pgfqpoint{0.620456in}{0.203790in}}{\pgfqpoint{0.630302in}{0.193945in}}{\pgfqpoint{0.635833in}{0.180590in}}%
\pgfpathcurveto{\pgfqpoint{0.635833in}{0.166667in}}{\pgfqpoint{0.635833in}{0.152744in}}{\pgfqpoint{0.630302in}{0.139389in}}%
\pgfpathcurveto{\pgfqpoint{0.620456in}{0.129544in}}{\pgfqpoint{0.610611in}{0.119698in}}{\pgfqpoint{0.597256in}{0.114167in}}%
\pgfpathclose%
\pgfpathmoveto{\pgfqpoint{0.750000in}{0.108333in}}%
\pgfpathcurveto{\pgfqpoint{0.765470in}{0.108333in}}{\pgfqpoint{0.780309in}{0.114480in}}{\pgfqpoint{0.791248in}{0.125419in}}%
\pgfpathcurveto{\pgfqpoint{0.802187in}{0.136358in}}{\pgfqpoint{0.808333in}{0.151196in}}{\pgfqpoint{0.808333in}{0.166667in}}%
\pgfpathcurveto{\pgfqpoint{0.808333in}{0.182137in}}{\pgfqpoint{0.802187in}{0.196975in}}{\pgfqpoint{0.791248in}{0.207915in}}%
\pgfpathcurveto{\pgfqpoint{0.780309in}{0.218854in}}{\pgfqpoint{0.765470in}{0.225000in}}{\pgfqpoint{0.750000in}{0.225000in}}%
\pgfpathcurveto{\pgfqpoint{0.734530in}{0.225000in}}{\pgfqpoint{0.719691in}{0.218854in}}{\pgfqpoint{0.708752in}{0.207915in}}%
\pgfpathcurveto{\pgfqpoint{0.697813in}{0.196975in}}{\pgfqpoint{0.691667in}{0.182137in}}{\pgfqpoint{0.691667in}{0.166667in}}%
\pgfpathcurveto{\pgfqpoint{0.691667in}{0.151196in}}{\pgfqpoint{0.697813in}{0.136358in}}{\pgfqpoint{0.708752in}{0.125419in}}%
\pgfpathcurveto{\pgfqpoint{0.719691in}{0.114480in}}{\pgfqpoint{0.734530in}{0.108333in}}{\pgfqpoint{0.750000in}{0.108333in}}%
\pgfpathclose%
\pgfpathmoveto{\pgfqpoint{0.750000in}{0.114167in}}%
\pgfpathcurveto{\pgfqpoint{0.750000in}{0.114167in}}{\pgfqpoint{0.736077in}{0.114167in}}{\pgfqpoint{0.722722in}{0.119698in}}%
\pgfpathcurveto{\pgfqpoint{0.712877in}{0.129544in}}{\pgfqpoint{0.703032in}{0.139389in}}{\pgfqpoint{0.697500in}{0.152744in}}%
\pgfpathcurveto{\pgfqpoint{0.697500in}{0.166667in}}{\pgfqpoint{0.697500in}{0.180590in}}{\pgfqpoint{0.703032in}{0.193945in}}%
\pgfpathcurveto{\pgfqpoint{0.712877in}{0.203790in}}{\pgfqpoint{0.722722in}{0.213635in}}{\pgfqpoint{0.736077in}{0.219167in}}%
\pgfpathcurveto{\pgfqpoint{0.750000in}{0.219167in}}{\pgfqpoint{0.763923in}{0.219167in}}{\pgfqpoint{0.777278in}{0.213635in}}%
\pgfpathcurveto{\pgfqpoint{0.787123in}{0.203790in}}{\pgfqpoint{0.796968in}{0.193945in}}{\pgfqpoint{0.802500in}{0.180590in}}%
\pgfpathcurveto{\pgfqpoint{0.802500in}{0.166667in}}{\pgfqpoint{0.802500in}{0.152744in}}{\pgfqpoint{0.796968in}{0.139389in}}%
\pgfpathcurveto{\pgfqpoint{0.787123in}{0.129544in}}{\pgfqpoint{0.777278in}{0.119698in}}{\pgfqpoint{0.763923in}{0.114167in}}%
\pgfpathclose%
\pgfpathmoveto{\pgfqpoint{0.916667in}{0.108333in}}%
\pgfpathcurveto{\pgfqpoint{0.932137in}{0.108333in}}{\pgfqpoint{0.946975in}{0.114480in}}{\pgfqpoint{0.957915in}{0.125419in}}%
\pgfpathcurveto{\pgfqpoint{0.968854in}{0.136358in}}{\pgfqpoint{0.975000in}{0.151196in}}{\pgfqpoint{0.975000in}{0.166667in}}%
\pgfpathcurveto{\pgfqpoint{0.975000in}{0.182137in}}{\pgfqpoint{0.968854in}{0.196975in}}{\pgfqpoint{0.957915in}{0.207915in}}%
\pgfpathcurveto{\pgfqpoint{0.946975in}{0.218854in}}{\pgfqpoint{0.932137in}{0.225000in}}{\pgfqpoint{0.916667in}{0.225000in}}%
\pgfpathcurveto{\pgfqpoint{0.901196in}{0.225000in}}{\pgfqpoint{0.886358in}{0.218854in}}{\pgfqpoint{0.875419in}{0.207915in}}%
\pgfpathcurveto{\pgfqpoint{0.864480in}{0.196975in}}{\pgfqpoint{0.858333in}{0.182137in}}{\pgfqpoint{0.858333in}{0.166667in}}%
\pgfpathcurveto{\pgfqpoint{0.858333in}{0.151196in}}{\pgfqpoint{0.864480in}{0.136358in}}{\pgfqpoint{0.875419in}{0.125419in}}%
\pgfpathcurveto{\pgfqpoint{0.886358in}{0.114480in}}{\pgfqpoint{0.901196in}{0.108333in}}{\pgfqpoint{0.916667in}{0.108333in}}%
\pgfpathclose%
\pgfpathmoveto{\pgfqpoint{0.916667in}{0.114167in}}%
\pgfpathcurveto{\pgfqpoint{0.916667in}{0.114167in}}{\pgfqpoint{0.902744in}{0.114167in}}{\pgfqpoint{0.889389in}{0.119698in}}%
\pgfpathcurveto{\pgfqpoint{0.879544in}{0.129544in}}{\pgfqpoint{0.869698in}{0.139389in}}{\pgfqpoint{0.864167in}{0.152744in}}%
\pgfpathcurveto{\pgfqpoint{0.864167in}{0.166667in}}{\pgfqpoint{0.864167in}{0.180590in}}{\pgfqpoint{0.869698in}{0.193945in}}%
\pgfpathcurveto{\pgfqpoint{0.879544in}{0.203790in}}{\pgfqpoint{0.889389in}{0.213635in}}{\pgfqpoint{0.902744in}{0.219167in}}%
\pgfpathcurveto{\pgfqpoint{0.916667in}{0.219167in}}{\pgfqpoint{0.930590in}{0.219167in}}{\pgfqpoint{0.943945in}{0.213635in}}%
\pgfpathcurveto{\pgfqpoint{0.953790in}{0.203790in}}{\pgfqpoint{0.963635in}{0.193945in}}{\pgfqpoint{0.969167in}{0.180590in}}%
\pgfpathcurveto{\pgfqpoint{0.969167in}{0.166667in}}{\pgfqpoint{0.969167in}{0.152744in}}{\pgfqpoint{0.963635in}{0.139389in}}%
\pgfpathcurveto{\pgfqpoint{0.953790in}{0.129544in}}{\pgfqpoint{0.943945in}{0.119698in}}{\pgfqpoint{0.930590in}{0.114167in}}%
\pgfpathclose%
\pgfpathmoveto{\pgfqpoint{0.000000in}{0.275000in}}%
\pgfpathcurveto{\pgfqpoint{0.015470in}{0.275000in}}{\pgfqpoint{0.030309in}{0.281146in}}{\pgfqpoint{0.041248in}{0.292085in}}%
\pgfpathcurveto{\pgfqpoint{0.052187in}{0.303025in}}{\pgfqpoint{0.058333in}{0.317863in}}{\pgfqpoint{0.058333in}{0.333333in}}%
\pgfpathcurveto{\pgfqpoint{0.058333in}{0.348804in}}{\pgfqpoint{0.052187in}{0.363642in}}{\pgfqpoint{0.041248in}{0.374581in}}%
\pgfpathcurveto{\pgfqpoint{0.030309in}{0.385520in}}{\pgfqpoint{0.015470in}{0.391667in}}{\pgfqpoint{0.000000in}{0.391667in}}%
\pgfpathcurveto{\pgfqpoint{-0.015470in}{0.391667in}}{\pgfqpoint{-0.030309in}{0.385520in}}{\pgfqpoint{-0.041248in}{0.374581in}}%
\pgfpathcurveto{\pgfqpoint{-0.052187in}{0.363642in}}{\pgfqpoint{-0.058333in}{0.348804in}}{\pgfqpoint{-0.058333in}{0.333333in}}%
\pgfpathcurveto{\pgfqpoint{-0.058333in}{0.317863in}}{\pgfqpoint{-0.052187in}{0.303025in}}{\pgfqpoint{-0.041248in}{0.292085in}}%
\pgfpathcurveto{\pgfqpoint{-0.030309in}{0.281146in}}{\pgfqpoint{-0.015470in}{0.275000in}}{\pgfqpoint{0.000000in}{0.275000in}}%
\pgfpathclose%
\pgfpathmoveto{\pgfqpoint{0.000000in}{0.280833in}}%
\pgfpathcurveto{\pgfqpoint{0.000000in}{0.280833in}}{\pgfqpoint{-0.013923in}{0.280833in}}{\pgfqpoint{-0.027278in}{0.286365in}}%
\pgfpathcurveto{\pgfqpoint{-0.037123in}{0.296210in}}{\pgfqpoint{-0.046968in}{0.306055in}}{\pgfqpoint{-0.052500in}{0.319410in}}%
\pgfpathcurveto{\pgfqpoint{-0.052500in}{0.333333in}}{\pgfqpoint{-0.052500in}{0.347256in}}{\pgfqpoint{-0.046968in}{0.360611in}}%
\pgfpathcurveto{\pgfqpoint{-0.037123in}{0.370456in}}{\pgfqpoint{-0.027278in}{0.380302in}}{\pgfqpoint{-0.013923in}{0.385833in}}%
\pgfpathcurveto{\pgfqpoint{0.000000in}{0.385833in}}{\pgfqpoint{0.013923in}{0.385833in}}{\pgfqpoint{0.027278in}{0.380302in}}%
\pgfpathcurveto{\pgfqpoint{0.037123in}{0.370456in}}{\pgfqpoint{0.046968in}{0.360611in}}{\pgfqpoint{0.052500in}{0.347256in}}%
\pgfpathcurveto{\pgfqpoint{0.052500in}{0.333333in}}{\pgfqpoint{0.052500in}{0.319410in}}{\pgfqpoint{0.046968in}{0.306055in}}%
\pgfpathcurveto{\pgfqpoint{0.037123in}{0.296210in}}{\pgfqpoint{0.027278in}{0.286365in}}{\pgfqpoint{0.013923in}{0.280833in}}%
\pgfpathclose%
\pgfpathmoveto{\pgfqpoint{0.166667in}{0.275000in}}%
\pgfpathcurveto{\pgfqpoint{0.182137in}{0.275000in}}{\pgfqpoint{0.196975in}{0.281146in}}{\pgfqpoint{0.207915in}{0.292085in}}%
\pgfpathcurveto{\pgfqpoint{0.218854in}{0.303025in}}{\pgfqpoint{0.225000in}{0.317863in}}{\pgfqpoint{0.225000in}{0.333333in}}%
\pgfpathcurveto{\pgfqpoint{0.225000in}{0.348804in}}{\pgfqpoint{0.218854in}{0.363642in}}{\pgfqpoint{0.207915in}{0.374581in}}%
\pgfpathcurveto{\pgfqpoint{0.196975in}{0.385520in}}{\pgfqpoint{0.182137in}{0.391667in}}{\pgfqpoint{0.166667in}{0.391667in}}%
\pgfpathcurveto{\pgfqpoint{0.151196in}{0.391667in}}{\pgfqpoint{0.136358in}{0.385520in}}{\pgfqpoint{0.125419in}{0.374581in}}%
\pgfpathcurveto{\pgfqpoint{0.114480in}{0.363642in}}{\pgfqpoint{0.108333in}{0.348804in}}{\pgfqpoint{0.108333in}{0.333333in}}%
\pgfpathcurveto{\pgfqpoint{0.108333in}{0.317863in}}{\pgfqpoint{0.114480in}{0.303025in}}{\pgfqpoint{0.125419in}{0.292085in}}%
\pgfpathcurveto{\pgfqpoint{0.136358in}{0.281146in}}{\pgfqpoint{0.151196in}{0.275000in}}{\pgfqpoint{0.166667in}{0.275000in}}%
\pgfpathclose%
\pgfpathmoveto{\pgfqpoint{0.166667in}{0.280833in}}%
\pgfpathcurveto{\pgfqpoint{0.166667in}{0.280833in}}{\pgfqpoint{0.152744in}{0.280833in}}{\pgfqpoint{0.139389in}{0.286365in}}%
\pgfpathcurveto{\pgfqpoint{0.129544in}{0.296210in}}{\pgfqpoint{0.119698in}{0.306055in}}{\pgfqpoint{0.114167in}{0.319410in}}%
\pgfpathcurveto{\pgfqpoint{0.114167in}{0.333333in}}{\pgfqpoint{0.114167in}{0.347256in}}{\pgfqpoint{0.119698in}{0.360611in}}%
\pgfpathcurveto{\pgfqpoint{0.129544in}{0.370456in}}{\pgfqpoint{0.139389in}{0.380302in}}{\pgfqpoint{0.152744in}{0.385833in}}%
\pgfpathcurveto{\pgfqpoint{0.166667in}{0.385833in}}{\pgfqpoint{0.180590in}{0.385833in}}{\pgfqpoint{0.193945in}{0.380302in}}%
\pgfpathcurveto{\pgfqpoint{0.203790in}{0.370456in}}{\pgfqpoint{0.213635in}{0.360611in}}{\pgfqpoint{0.219167in}{0.347256in}}%
\pgfpathcurveto{\pgfqpoint{0.219167in}{0.333333in}}{\pgfqpoint{0.219167in}{0.319410in}}{\pgfqpoint{0.213635in}{0.306055in}}%
\pgfpathcurveto{\pgfqpoint{0.203790in}{0.296210in}}{\pgfqpoint{0.193945in}{0.286365in}}{\pgfqpoint{0.180590in}{0.280833in}}%
\pgfpathclose%
\pgfpathmoveto{\pgfqpoint{0.333333in}{0.275000in}}%
\pgfpathcurveto{\pgfqpoint{0.348804in}{0.275000in}}{\pgfqpoint{0.363642in}{0.281146in}}{\pgfqpoint{0.374581in}{0.292085in}}%
\pgfpathcurveto{\pgfqpoint{0.385520in}{0.303025in}}{\pgfqpoint{0.391667in}{0.317863in}}{\pgfqpoint{0.391667in}{0.333333in}}%
\pgfpathcurveto{\pgfqpoint{0.391667in}{0.348804in}}{\pgfqpoint{0.385520in}{0.363642in}}{\pgfqpoint{0.374581in}{0.374581in}}%
\pgfpathcurveto{\pgfqpoint{0.363642in}{0.385520in}}{\pgfqpoint{0.348804in}{0.391667in}}{\pgfqpoint{0.333333in}{0.391667in}}%
\pgfpathcurveto{\pgfqpoint{0.317863in}{0.391667in}}{\pgfqpoint{0.303025in}{0.385520in}}{\pgfqpoint{0.292085in}{0.374581in}}%
\pgfpathcurveto{\pgfqpoint{0.281146in}{0.363642in}}{\pgfqpoint{0.275000in}{0.348804in}}{\pgfqpoint{0.275000in}{0.333333in}}%
\pgfpathcurveto{\pgfqpoint{0.275000in}{0.317863in}}{\pgfqpoint{0.281146in}{0.303025in}}{\pgfqpoint{0.292085in}{0.292085in}}%
\pgfpathcurveto{\pgfqpoint{0.303025in}{0.281146in}}{\pgfqpoint{0.317863in}{0.275000in}}{\pgfqpoint{0.333333in}{0.275000in}}%
\pgfpathclose%
\pgfpathmoveto{\pgfqpoint{0.333333in}{0.280833in}}%
\pgfpathcurveto{\pgfqpoint{0.333333in}{0.280833in}}{\pgfqpoint{0.319410in}{0.280833in}}{\pgfqpoint{0.306055in}{0.286365in}}%
\pgfpathcurveto{\pgfqpoint{0.296210in}{0.296210in}}{\pgfqpoint{0.286365in}{0.306055in}}{\pgfqpoint{0.280833in}{0.319410in}}%
\pgfpathcurveto{\pgfqpoint{0.280833in}{0.333333in}}{\pgfqpoint{0.280833in}{0.347256in}}{\pgfqpoint{0.286365in}{0.360611in}}%
\pgfpathcurveto{\pgfqpoint{0.296210in}{0.370456in}}{\pgfqpoint{0.306055in}{0.380302in}}{\pgfqpoint{0.319410in}{0.385833in}}%
\pgfpathcurveto{\pgfqpoint{0.333333in}{0.385833in}}{\pgfqpoint{0.347256in}{0.385833in}}{\pgfqpoint{0.360611in}{0.380302in}}%
\pgfpathcurveto{\pgfqpoint{0.370456in}{0.370456in}}{\pgfqpoint{0.380302in}{0.360611in}}{\pgfqpoint{0.385833in}{0.347256in}}%
\pgfpathcurveto{\pgfqpoint{0.385833in}{0.333333in}}{\pgfqpoint{0.385833in}{0.319410in}}{\pgfqpoint{0.380302in}{0.306055in}}%
\pgfpathcurveto{\pgfqpoint{0.370456in}{0.296210in}}{\pgfqpoint{0.360611in}{0.286365in}}{\pgfqpoint{0.347256in}{0.280833in}}%
\pgfpathclose%
\pgfpathmoveto{\pgfqpoint{0.500000in}{0.275000in}}%
\pgfpathcurveto{\pgfqpoint{0.515470in}{0.275000in}}{\pgfqpoint{0.530309in}{0.281146in}}{\pgfqpoint{0.541248in}{0.292085in}}%
\pgfpathcurveto{\pgfqpoint{0.552187in}{0.303025in}}{\pgfqpoint{0.558333in}{0.317863in}}{\pgfqpoint{0.558333in}{0.333333in}}%
\pgfpathcurveto{\pgfqpoint{0.558333in}{0.348804in}}{\pgfqpoint{0.552187in}{0.363642in}}{\pgfqpoint{0.541248in}{0.374581in}}%
\pgfpathcurveto{\pgfqpoint{0.530309in}{0.385520in}}{\pgfqpoint{0.515470in}{0.391667in}}{\pgfqpoint{0.500000in}{0.391667in}}%
\pgfpathcurveto{\pgfqpoint{0.484530in}{0.391667in}}{\pgfqpoint{0.469691in}{0.385520in}}{\pgfqpoint{0.458752in}{0.374581in}}%
\pgfpathcurveto{\pgfqpoint{0.447813in}{0.363642in}}{\pgfqpoint{0.441667in}{0.348804in}}{\pgfqpoint{0.441667in}{0.333333in}}%
\pgfpathcurveto{\pgfqpoint{0.441667in}{0.317863in}}{\pgfqpoint{0.447813in}{0.303025in}}{\pgfqpoint{0.458752in}{0.292085in}}%
\pgfpathcurveto{\pgfqpoint{0.469691in}{0.281146in}}{\pgfqpoint{0.484530in}{0.275000in}}{\pgfqpoint{0.500000in}{0.275000in}}%
\pgfpathclose%
\pgfpathmoveto{\pgfqpoint{0.500000in}{0.280833in}}%
\pgfpathcurveto{\pgfqpoint{0.500000in}{0.280833in}}{\pgfqpoint{0.486077in}{0.280833in}}{\pgfqpoint{0.472722in}{0.286365in}}%
\pgfpathcurveto{\pgfqpoint{0.462877in}{0.296210in}}{\pgfqpoint{0.453032in}{0.306055in}}{\pgfqpoint{0.447500in}{0.319410in}}%
\pgfpathcurveto{\pgfqpoint{0.447500in}{0.333333in}}{\pgfqpoint{0.447500in}{0.347256in}}{\pgfqpoint{0.453032in}{0.360611in}}%
\pgfpathcurveto{\pgfqpoint{0.462877in}{0.370456in}}{\pgfqpoint{0.472722in}{0.380302in}}{\pgfqpoint{0.486077in}{0.385833in}}%
\pgfpathcurveto{\pgfqpoint{0.500000in}{0.385833in}}{\pgfqpoint{0.513923in}{0.385833in}}{\pgfqpoint{0.527278in}{0.380302in}}%
\pgfpathcurveto{\pgfqpoint{0.537123in}{0.370456in}}{\pgfqpoint{0.546968in}{0.360611in}}{\pgfqpoint{0.552500in}{0.347256in}}%
\pgfpathcurveto{\pgfqpoint{0.552500in}{0.333333in}}{\pgfqpoint{0.552500in}{0.319410in}}{\pgfqpoint{0.546968in}{0.306055in}}%
\pgfpathcurveto{\pgfqpoint{0.537123in}{0.296210in}}{\pgfqpoint{0.527278in}{0.286365in}}{\pgfqpoint{0.513923in}{0.280833in}}%
\pgfpathclose%
\pgfpathmoveto{\pgfqpoint{0.666667in}{0.275000in}}%
\pgfpathcurveto{\pgfqpoint{0.682137in}{0.275000in}}{\pgfqpoint{0.696975in}{0.281146in}}{\pgfqpoint{0.707915in}{0.292085in}}%
\pgfpathcurveto{\pgfqpoint{0.718854in}{0.303025in}}{\pgfqpoint{0.725000in}{0.317863in}}{\pgfqpoint{0.725000in}{0.333333in}}%
\pgfpathcurveto{\pgfqpoint{0.725000in}{0.348804in}}{\pgfqpoint{0.718854in}{0.363642in}}{\pgfqpoint{0.707915in}{0.374581in}}%
\pgfpathcurveto{\pgfqpoint{0.696975in}{0.385520in}}{\pgfqpoint{0.682137in}{0.391667in}}{\pgfqpoint{0.666667in}{0.391667in}}%
\pgfpathcurveto{\pgfqpoint{0.651196in}{0.391667in}}{\pgfqpoint{0.636358in}{0.385520in}}{\pgfqpoint{0.625419in}{0.374581in}}%
\pgfpathcurveto{\pgfqpoint{0.614480in}{0.363642in}}{\pgfqpoint{0.608333in}{0.348804in}}{\pgfqpoint{0.608333in}{0.333333in}}%
\pgfpathcurveto{\pgfqpoint{0.608333in}{0.317863in}}{\pgfqpoint{0.614480in}{0.303025in}}{\pgfqpoint{0.625419in}{0.292085in}}%
\pgfpathcurveto{\pgfqpoint{0.636358in}{0.281146in}}{\pgfqpoint{0.651196in}{0.275000in}}{\pgfqpoint{0.666667in}{0.275000in}}%
\pgfpathclose%
\pgfpathmoveto{\pgfqpoint{0.666667in}{0.280833in}}%
\pgfpathcurveto{\pgfqpoint{0.666667in}{0.280833in}}{\pgfqpoint{0.652744in}{0.280833in}}{\pgfqpoint{0.639389in}{0.286365in}}%
\pgfpathcurveto{\pgfqpoint{0.629544in}{0.296210in}}{\pgfqpoint{0.619698in}{0.306055in}}{\pgfqpoint{0.614167in}{0.319410in}}%
\pgfpathcurveto{\pgfqpoint{0.614167in}{0.333333in}}{\pgfqpoint{0.614167in}{0.347256in}}{\pgfqpoint{0.619698in}{0.360611in}}%
\pgfpathcurveto{\pgfqpoint{0.629544in}{0.370456in}}{\pgfqpoint{0.639389in}{0.380302in}}{\pgfqpoint{0.652744in}{0.385833in}}%
\pgfpathcurveto{\pgfqpoint{0.666667in}{0.385833in}}{\pgfqpoint{0.680590in}{0.385833in}}{\pgfqpoint{0.693945in}{0.380302in}}%
\pgfpathcurveto{\pgfqpoint{0.703790in}{0.370456in}}{\pgfqpoint{0.713635in}{0.360611in}}{\pgfqpoint{0.719167in}{0.347256in}}%
\pgfpathcurveto{\pgfqpoint{0.719167in}{0.333333in}}{\pgfqpoint{0.719167in}{0.319410in}}{\pgfqpoint{0.713635in}{0.306055in}}%
\pgfpathcurveto{\pgfqpoint{0.703790in}{0.296210in}}{\pgfqpoint{0.693945in}{0.286365in}}{\pgfqpoint{0.680590in}{0.280833in}}%
\pgfpathclose%
\pgfpathmoveto{\pgfqpoint{0.833333in}{0.275000in}}%
\pgfpathcurveto{\pgfqpoint{0.848804in}{0.275000in}}{\pgfqpoint{0.863642in}{0.281146in}}{\pgfqpoint{0.874581in}{0.292085in}}%
\pgfpathcurveto{\pgfqpoint{0.885520in}{0.303025in}}{\pgfqpoint{0.891667in}{0.317863in}}{\pgfqpoint{0.891667in}{0.333333in}}%
\pgfpathcurveto{\pgfqpoint{0.891667in}{0.348804in}}{\pgfqpoint{0.885520in}{0.363642in}}{\pgfqpoint{0.874581in}{0.374581in}}%
\pgfpathcurveto{\pgfqpoint{0.863642in}{0.385520in}}{\pgfqpoint{0.848804in}{0.391667in}}{\pgfqpoint{0.833333in}{0.391667in}}%
\pgfpathcurveto{\pgfqpoint{0.817863in}{0.391667in}}{\pgfqpoint{0.803025in}{0.385520in}}{\pgfqpoint{0.792085in}{0.374581in}}%
\pgfpathcurveto{\pgfqpoint{0.781146in}{0.363642in}}{\pgfqpoint{0.775000in}{0.348804in}}{\pgfqpoint{0.775000in}{0.333333in}}%
\pgfpathcurveto{\pgfqpoint{0.775000in}{0.317863in}}{\pgfqpoint{0.781146in}{0.303025in}}{\pgfqpoint{0.792085in}{0.292085in}}%
\pgfpathcurveto{\pgfqpoint{0.803025in}{0.281146in}}{\pgfqpoint{0.817863in}{0.275000in}}{\pgfqpoint{0.833333in}{0.275000in}}%
\pgfpathclose%
\pgfpathmoveto{\pgfqpoint{0.833333in}{0.280833in}}%
\pgfpathcurveto{\pgfqpoint{0.833333in}{0.280833in}}{\pgfqpoint{0.819410in}{0.280833in}}{\pgfqpoint{0.806055in}{0.286365in}}%
\pgfpathcurveto{\pgfqpoint{0.796210in}{0.296210in}}{\pgfqpoint{0.786365in}{0.306055in}}{\pgfqpoint{0.780833in}{0.319410in}}%
\pgfpathcurveto{\pgfqpoint{0.780833in}{0.333333in}}{\pgfqpoint{0.780833in}{0.347256in}}{\pgfqpoint{0.786365in}{0.360611in}}%
\pgfpathcurveto{\pgfqpoint{0.796210in}{0.370456in}}{\pgfqpoint{0.806055in}{0.380302in}}{\pgfqpoint{0.819410in}{0.385833in}}%
\pgfpathcurveto{\pgfqpoint{0.833333in}{0.385833in}}{\pgfqpoint{0.847256in}{0.385833in}}{\pgfqpoint{0.860611in}{0.380302in}}%
\pgfpathcurveto{\pgfqpoint{0.870456in}{0.370456in}}{\pgfqpoint{0.880302in}{0.360611in}}{\pgfqpoint{0.885833in}{0.347256in}}%
\pgfpathcurveto{\pgfqpoint{0.885833in}{0.333333in}}{\pgfqpoint{0.885833in}{0.319410in}}{\pgfqpoint{0.880302in}{0.306055in}}%
\pgfpathcurveto{\pgfqpoint{0.870456in}{0.296210in}}{\pgfqpoint{0.860611in}{0.286365in}}{\pgfqpoint{0.847256in}{0.280833in}}%
\pgfpathclose%
\pgfpathmoveto{\pgfqpoint{1.000000in}{0.275000in}}%
\pgfpathcurveto{\pgfqpoint{1.015470in}{0.275000in}}{\pgfqpoint{1.030309in}{0.281146in}}{\pgfqpoint{1.041248in}{0.292085in}}%
\pgfpathcurveto{\pgfqpoint{1.052187in}{0.303025in}}{\pgfqpoint{1.058333in}{0.317863in}}{\pgfqpoint{1.058333in}{0.333333in}}%
\pgfpathcurveto{\pgfqpoint{1.058333in}{0.348804in}}{\pgfqpoint{1.052187in}{0.363642in}}{\pgfqpoint{1.041248in}{0.374581in}}%
\pgfpathcurveto{\pgfqpoint{1.030309in}{0.385520in}}{\pgfqpoint{1.015470in}{0.391667in}}{\pgfqpoint{1.000000in}{0.391667in}}%
\pgfpathcurveto{\pgfqpoint{0.984530in}{0.391667in}}{\pgfqpoint{0.969691in}{0.385520in}}{\pgfqpoint{0.958752in}{0.374581in}}%
\pgfpathcurveto{\pgfqpoint{0.947813in}{0.363642in}}{\pgfqpoint{0.941667in}{0.348804in}}{\pgfqpoint{0.941667in}{0.333333in}}%
\pgfpathcurveto{\pgfqpoint{0.941667in}{0.317863in}}{\pgfqpoint{0.947813in}{0.303025in}}{\pgfqpoint{0.958752in}{0.292085in}}%
\pgfpathcurveto{\pgfqpoint{0.969691in}{0.281146in}}{\pgfqpoint{0.984530in}{0.275000in}}{\pgfqpoint{1.000000in}{0.275000in}}%
\pgfpathclose%
\pgfpathmoveto{\pgfqpoint{1.000000in}{0.280833in}}%
\pgfpathcurveto{\pgfqpoint{1.000000in}{0.280833in}}{\pgfqpoint{0.986077in}{0.280833in}}{\pgfqpoint{0.972722in}{0.286365in}}%
\pgfpathcurveto{\pgfqpoint{0.962877in}{0.296210in}}{\pgfqpoint{0.953032in}{0.306055in}}{\pgfqpoint{0.947500in}{0.319410in}}%
\pgfpathcurveto{\pgfqpoint{0.947500in}{0.333333in}}{\pgfqpoint{0.947500in}{0.347256in}}{\pgfqpoint{0.953032in}{0.360611in}}%
\pgfpathcurveto{\pgfqpoint{0.962877in}{0.370456in}}{\pgfqpoint{0.972722in}{0.380302in}}{\pgfqpoint{0.986077in}{0.385833in}}%
\pgfpathcurveto{\pgfqpoint{1.000000in}{0.385833in}}{\pgfqpoint{1.013923in}{0.385833in}}{\pgfqpoint{1.027278in}{0.380302in}}%
\pgfpathcurveto{\pgfqpoint{1.037123in}{0.370456in}}{\pgfqpoint{1.046968in}{0.360611in}}{\pgfqpoint{1.052500in}{0.347256in}}%
\pgfpathcurveto{\pgfqpoint{1.052500in}{0.333333in}}{\pgfqpoint{1.052500in}{0.319410in}}{\pgfqpoint{1.046968in}{0.306055in}}%
\pgfpathcurveto{\pgfqpoint{1.037123in}{0.296210in}}{\pgfqpoint{1.027278in}{0.286365in}}{\pgfqpoint{1.013923in}{0.280833in}}%
\pgfpathclose%
\pgfpathmoveto{\pgfqpoint{0.083333in}{0.441667in}}%
\pgfpathcurveto{\pgfqpoint{0.098804in}{0.441667in}}{\pgfqpoint{0.113642in}{0.447813in}}{\pgfqpoint{0.124581in}{0.458752in}}%
\pgfpathcurveto{\pgfqpoint{0.135520in}{0.469691in}}{\pgfqpoint{0.141667in}{0.484530in}}{\pgfqpoint{0.141667in}{0.500000in}}%
\pgfpathcurveto{\pgfqpoint{0.141667in}{0.515470in}}{\pgfqpoint{0.135520in}{0.530309in}}{\pgfqpoint{0.124581in}{0.541248in}}%
\pgfpathcurveto{\pgfqpoint{0.113642in}{0.552187in}}{\pgfqpoint{0.098804in}{0.558333in}}{\pgfqpoint{0.083333in}{0.558333in}}%
\pgfpathcurveto{\pgfqpoint{0.067863in}{0.558333in}}{\pgfqpoint{0.053025in}{0.552187in}}{\pgfqpoint{0.042085in}{0.541248in}}%
\pgfpathcurveto{\pgfqpoint{0.031146in}{0.530309in}}{\pgfqpoint{0.025000in}{0.515470in}}{\pgfqpoint{0.025000in}{0.500000in}}%
\pgfpathcurveto{\pgfqpoint{0.025000in}{0.484530in}}{\pgfqpoint{0.031146in}{0.469691in}}{\pgfqpoint{0.042085in}{0.458752in}}%
\pgfpathcurveto{\pgfqpoint{0.053025in}{0.447813in}}{\pgfqpoint{0.067863in}{0.441667in}}{\pgfqpoint{0.083333in}{0.441667in}}%
\pgfpathclose%
\pgfpathmoveto{\pgfqpoint{0.083333in}{0.447500in}}%
\pgfpathcurveto{\pgfqpoint{0.083333in}{0.447500in}}{\pgfqpoint{0.069410in}{0.447500in}}{\pgfqpoint{0.056055in}{0.453032in}}%
\pgfpathcurveto{\pgfqpoint{0.046210in}{0.462877in}}{\pgfqpoint{0.036365in}{0.472722in}}{\pgfqpoint{0.030833in}{0.486077in}}%
\pgfpathcurveto{\pgfqpoint{0.030833in}{0.500000in}}{\pgfqpoint{0.030833in}{0.513923in}}{\pgfqpoint{0.036365in}{0.527278in}}%
\pgfpathcurveto{\pgfqpoint{0.046210in}{0.537123in}}{\pgfqpoint{0.056055in}{0.546968in}}{\pgfqpoint{0.069410in}{0.552500in}}%
\pgfpathcurveto{\pgfqpoint{0.083333in}{0.552500in}}{\pgfqpoint{0.097256in}{0.552500in}}{\pgfqpoint{0.110611in}{0.546968in}}%
\pgfpathcurveto{\pgfqpoint{0.120456in}{0.537123in}}{\pgfqpoint{0.130302in}{0.527278in}}{\pgfqpoint{0.135833in}{0.513923in}}%
\pgfpathcurveto{\pgfqpoint{0.135833in}{0.500000in}}{\pgfqpoint{0.135833in}{0.486077in}}{\pgfqpoint{0.130302in}{0.472722in}}%
\pgfpathcurveto{\pgfqpoint{0.120456in}{0.462877in}}{\pgfqpoint{0.110611in}{0.453032in}}{\pgfqpoint{0.097256in}{0.447500in}}%
\pgfpathclose%
\pgfpathmoveto{\pgfqpoint{0.250000in}{0.441667in}}%
\pgfpathcurveto{\pgfqpoint{0.265470in}{0.441667in}}{\pgfqpoint{0.280309in}{0.447813in}}{\pgfqpoint{0.291248in}{0.458752in}}%
\pgfpathcurveto{\pgfqpoint{0.302187in}{0.469691in}}{\pgfqpoint{0.308333in}{0.484530in}}{\pgfqpoint{0.308333in}{0.500000in}}%
\pgfpathcurveto{\pgfqpoint{0.308333in}{0.515470in}}{\pgfqpoint{0.302187in}{0.530309in}}{\pgfqpoint{0.291248in}{0.541248in}}%
\pgfpathcurveto{\pgfqpoint{0.280309in}{0.552187in}}{\pgfqpoint{0.265470in}{0.558333in}}{\pgfqpoint{0.250000in}{0.558333in}}%
\pgfpathcurveto{\pgfqpoint{0.234530in}{0.558333in}}{\pgfqpoint{0.219691in}{0.552187in}}{\pgfqpoint{0.208752in}{0.541248in}}%
\pgfpathcurveto{\pgfqpoint{0.197813in}{0.530309in}}{\pgfqpoint{0.191667in}{0.515470in}}{\pgfqpoint{0.191667in}{0.500000in}}%
\pgfpathcurveto{\pgfqpoint{0.191667in}{0.484530in}}{\pgfqpoint{0.197813in}{0.469691in}}{\pgfqpoint{0.208752in}{0.458752in}}%
\pgfpathcurveto{\pgfqpoint{0.219691in}{0.447813in}}{\pgfqpoint{0.234530in}{0.441667in}}{\pgfqpoint{0.250000in}{0.441667in}}%
\pgfpathclose%
\pgfpathmoveto{\pgfqpoint{0.250000in}{0.447500in}}%
\pgfpathcurveto{\pgfqpoint{0.250000in}{0.447500in}}{\pgfqpoint{0.236077in}{0.447500in}}{\pgfqpoint{0.222722in}{0.453032in}}%
\pgfpathcurveto{\pgfqpoint{0.212877in}{0.462877in}}{\pgfqpoint{0.203032in}{0.472722in}}{\pgfqpoint{0.197500in}{0.486077in}}%
\pgfpathcurveto{\pgfqpoint{0.197500in}{0.500000in}}{\pgfqpoint{0.197500in}{0.513923in}}{\pgfqpoint{0.203032in}{0.527278in}}%
\pgfpathcurveto{\pgfqpoint{0.212877in}{0.537123in}}{\pgfqpoint{0.222722in}{0.546968in}}{\pgfqpoint{0.236077in}{0.552500in}}%
\pgfpathcurveto{\pgfqpoint{0.250000in}{0.552500in}}{\pgfqpoint{0.263923in}{0.552500in}}{\pgfqpoint{0.277278in}{0.546968in}}%
\pgfpathcurveto{\pgfqpoint{0.287123in}{0.537123in}}{\pgfqpoint{0.296968in}{0.527278in}}{\pgfqpoint{0.302500in}{0.513923in}}%
\pgfpathcurveto{\pgfqpoint{0.302500in}{0.500000in}}{\pgfqpoint{0.302500in}{0.486077in}}{\pgfqpoint{0.296968in}{0.472722in}}%
\pgfpathcurveto{\pgfqpoint{0.287123in}{0.462877in}}{\pgfqpoint{0.277278in}{0.453032in}}{\pgfqpoint{0.263923in}{0.447500in}}%
\pgfpathclose%
\pgfpathmoveto{\pgfqpoint{0.416667in}{0.441667in}}%
\pgfpathcurveto{\pgfqpoint{0.432137in}{0.441667in}}{\pgfqpoint{0.446975in}{0.447813in}}{\pgfqpoint{0.457915in}{0.458752in}}%
\pgfpathcurveto{\pgfqpoint{0.468854in}{0.469691in}}{\pgfqpoint{0.475000in}{0.484530in}}{\pgfqpoint{0.475000in}{0.500000in}}%
\pgfpathcurveto{\pgfqpoint{0.475000in}{0.515470in}}{\pgfqpoint{0.468854in}{0.530309in}}{\pgfqpoint{0.457915in}{0.541248in}}%
\pgfpathcurveto{\pgfqpoint{0.446975in}{0.552187in}}{\pgfqpoint{0.432137in}{0.558333in}}{\pgfqpoint{0.416667in}{0.558333in}}%
\pgfpathcurveto{\pgfqpoint{0.401196in}{0.558333in}}{\pgfqpoint{0.386358in}{0.552187in}}{\pgfqpoint{0.375419in}{0.541248in}}%
\pgfpathcurveto{\pgfqpoint{0.364480in}{0.530309in}}{\pgfqpoint{0.358333in}{0.515470in}}{\pgfqpoint{0.358333in}{0.500000in}}%
\pgfpathcurveto{\pgfqpoint{0.358333in}{0.484530in}}{\pgfqpoint{0.364480in}{0.469691in}}{\pgfqpoint{0.375419in}{0.458752in}}%
\pgfpathcurveto{\pgfqpoint{0.386358in}{0.447813in}}{\pgfqpoint{0.401196in}{0.441667in}}{\pgfqpoint{0.416667in}{0.441667in}}%
\pgfpathclose%
\pgfpathmoveto{\pgfqpoint{0.416667in}{0.447500in}}%
\pgfpathcurveto{\pgfqpoint{0.416667in}{0.447500in}}{\pgfqpoint{0.402744in}{0.447500in}}{\pgfqpoint{0.389389in}{0.453032in}}%
\pgfpathcurveto{\pgfqpoint{0.379544in}{0.462877in}}{\pgfqpoint{0.369698in}{0.472722in}}{\pgfqpoint{0.364167in}{0.486077in}}%
\pgfpathcurveto{\pgfqpoint{0.364167in}{0.500000in}}{\pgfqpoint{0.364167in}{0.513923in}}{\pgfqpoint{0.369698in}{0.527278in}}%
\pgfpathcurveto{\pgfqpoint{0.379544in}{0.537123in}}{\pgfqpoint{0.389389in}{0.546968in}}{\pgfqpoint{0.402744in}{0.552500in}}%
\pgfpathcurveto{\pgfqpoint{0.416667in}{0.552500in}}{\pgfqpoint{0.430590in}{0.552500in}}{\pgfqpoint{0.443945in}{0.546968in}}%
\pgfpathcurveto{\pgfqpoint{0.453790in}{0.537123in}}{\pgfqpoint{0.463635in}{0.527278in}}{\pgfqpoint{0.469167in}{0.513923in}}%
\pgfpathcurveto{\pgfqpoint{0.469167in}{0.500000in}}{\pgfqpoint{0.469167in}{0.486077in}}{\pgfqpoint{0.463635in}{0.472722in}}%
\pgfpathcurveto{\pgfqpoint{0.453790in}{0.462877in}}{\pgfqpoint{0.443945in}{0.453032in}}{\pgfqpoint{0.430590in}{0.447500in}}%
\pgfpathclose%
\pgfpathmoveto{\pgfqpoint{0.583333in}{0.441667in}}%
\pgfpathcurveto{\pgfqpoint{0.598804in}{0.441667in}}{\pgfqpoint{0.613642in}{0.447813in}}{\pgfqpoint{0.624581in}{0.458752in}}%
\pgfpathcurveto{\pgfqpoint{0.635520in}{0.469691in}}{\pgfqpoint{0.641667in}{0.484530in}}{\pgfqpoint{0.641667in}{0.500000in}}%
\pgfpathcurveto{\pgfqpoint{0.641667in}{0.515470in}}{\pgfqpoint{0.635520in}{0.530309in}}{\pgfqpoint{0.624581in}{0.541248in}}%
\pgfpathcurveto{\pgfqpoint{0.613642in}{0.552187in}}{\pgfqpoint{0.598804in}{0.558333in}}{\pgfqpoint{0.583333in}{0.558333in}}%
\pgfpathcurveto{\pgfqpoint{0.567863in}{0.558333in}}{\pgfqpoint{0.553025in}{0.552187in}}{\pgfqpoint{0.542085in}{0.541248in}}%
\pgfpathcurveto{\pgfqpoint{0.531146in}{0.530309in}}{\pgfqpoint{0.525000in}{0.515470in}}{\pgfqpoint{0.525000in}{0.500000in}}%
\pgfpathcurveto{\pgfqpoint{0.525000in}{0.484530in}}{\pgfqpoint{0.531146in}{0.469691in}}{\pgfqpoint{0.542085in}{0.458752in}}%
\pgfpathcurveto{\pgfqpoint{0.553025in}{0.447813in}}{\pgfqpoint{0.567863in}{0.441667in}}{\pgfqpoint{0.583333in}{0.441667in}}%
\pgfpathclose%
\pgfpathmoveto{\pgfqpoint{0.583333in}{0.447500in}}%
\pgfpathcurveto{\pgfqpoint{0.583333in}{0.447500in}}{\pgfqpoint{0.569410in}{0.447500in}}{\pgfqpoint{0.556055in}{0.453032in}}%
\pgfpathcurveto{\pgfqpoint{0.546210in}{0.462877in}}{\pgfqpoint{0.536365in}{0.472722in}}{\pgfqpoint{0.530833in}{0.486077in}}%
\pgfpathcurveto{\pgfqpoint{0.530833in}{0.500000in}}{\pgfqpoint{0.530833in}{0.513923in}}{\pgfqpoint{0.536365in}{0.527278in}}%
\pgfpathcurveto{\pgfqpoint{0.546210in}{0.537123in}}{\pgfqpoint{0.556055in}{0.546968in}}{\pgfqpoint{0.569410in}{0.552500in}}%
\pgfpathcurveto{\pgfqpoint{0.583333in}{0.552500in}}{\pgfqpoint{0.597256in}{0.552500in}}{\pgfqpoint{0.610611in}{0.546968in}}%
\pgfpathcurveto{\pgfqpoint{0.620456in}{0.537123in}}{\pgfqpoint{0.630302in}{0.527278in}}{\pgfqpoint{0.635833in}{0.513923in}}%
\pgfpathcurveto{\pgfqpoint{0.635833in}{0.500000in}}{\pgfqpoint{0.635833in}{0.486077in}}{\pgfqpoint{0.630302in}{0.472722in}}%
\pgfpathcurveto{\pgfqpoint{0.620456in}{0.462877in}}{\pgfqpoint{0.610611in}{0.453032in}}{\pgfqpoint{0.597256in}{0.447500in}}%
\pgfpathclose%
\pgfpathmoveto{\pgfqpoint{0.750000in}{0.441667in}}%
\pgfpathcurveto{\pgfqpoint{0.765470in}{0.441667in}}{\pgfqpoint{0.780309in}{0.447813in}}{\pgfqpoint{0.791248in}{0.458752in}}%
\pgfpathcurveto{\pgfqpoint{0.802187in}{0.469691in}}{\pgfqpoint{0.808333in}{0.484530in}}{\pgfqpoint{0.808333in}{0.500000in}}%
\pgfpathcurveto{\pgfqpoint{0.808333in}{0.515470in}}{\pgfqpoint{0.802187in}{0.530309in}}{\pgfqpoint{0.791248in}{0.541248in}}%
\pgfpathcurveto{\pgfqpoint{0.780309in}{0.552187in}}{\pgfqpoint{0.765470in}{0.558333in}}{\pgfqpoint{0.750000in}{0.558333in}}%
\pgfpathcurveto{\pgfqpoint{0.734530in}{0.558333in}}{\pgfqpoint{0.719691in}{0.552187in}}{\pgfqpoint{0.708752in}{0.541248in}}%
\pgfpathcurveto{\pgfqpoint{0.697813in}{0.530309in}}{\pgfqpoint{0.691667in}{0.515470in}}{\pgfqpoint{0.691667in}{0.500000in}}%
\pgfpathcurveto{\pgfqpoint{0.691667in}{0.484530in}}{\pgfqpoint{0.697813in}{0.469691in}}{\pgfqpoint{0.708752in}{0.458752in}}%
\pgfpathcurveto{\pgfqpoint{0.719691in}{0.447813in}}{\pgfqpoint{0.734530in}{0.441667in}}{\pgfqpoint{0.750000in}{0.441667in}}%
\pgfpathclose%
\pgfpathmoveto{\pgfqpoint{0.750000in}{0.447500in}}%
\pgfpathcurveto{\pgfqpoint{0.750000in}{0.447500in}}{\pgfqpoint{0.736077in}{0.447500in}}{\pgfqpoint{0.722722in}{0.453032in}}%
\pgfpathcurveto{\pgfqpoint{0.712877in}{0.462877in}}{\pgfqpoint{0.703032in}{0.472722in}}{\pgfqpoint{0.697500in}{0.486077in}}%
\pgfpathcurveto{\pgfqpoint{0.697500in}{0.500000in}}{\pgfqpoint{0.697500in}{0.513923in}}{\pgfqpoint{0.703032in}{0.527278in}}%
\pgfpathcurveto{\pgfqpoint{0.712877in}{0.537123in}}{\pgfqpoint{0.722722in}{0.546968in}}{\pgfqpoint{0.736077in}{0.552500in}}%
\pgfpathcurveto{\pgfqpoint{0.750000in}{0.552500in}}{\pgfqpoint{0.763923in}{0.552500in}}{\pgfqpoint{0.777278in}{0.546968in}}%
\pgfpathcurveto{\pgfqpoint{0.787123in}{0.537123in}}{\pgfqpoint{0.796968in}{0.527278in}}{\pgfqpoint{0.802500in}{0.513923in}}%
\pgfpathcurveto{\pgfqpoint{0.802500in}{0.500000in}}{\pgfqpoint{0.802500in}{0.486077in}}{\pgfqpoint{0.796968in}{0.472722in}}%
\pgfpathcurveto{\pgfqpoint{0.787123in}{0.462877in}}{\pgfqpoint{0.777278in}{0.453032in}}{\pgfqpoint{0.763923in}{0.447500in}}%
\pgfpathclose%
\pgfpathmoveto{\pgfqpoint{0.916667in}{0.441667in}}%
\pgfpathcurveto{\pgfqpoint{0.932137in}{0.441667in}}{\pgfqpoint{0.946975in}{0.447813in}}{\pgfqpoint{0.957915in}{0.458752in}}%
\pgfpathcurveto{\pgfqpoint{0.968854in}{0.469691in}}{\pgfqpoint{0.975000in}{0.484530in}}{\pgfqpoint{0.975000in}{0.500000in}}%
\pgfpathcurveto{\pgfqpoint{0.975000in}{0.515470in}}{\pgfqpoint{0.968854in}{0.530309in}}{\pgfqpoint{0.957915in}{0.541248in}}%
\pgfpathcurveto{\pgfqpoint{0.946975in}{0.552187in}}{\pgfqpoint{0.932137in}{0.558333in}}{\pgfqpoint{0.916667in}{0.558333in}}%
\pgfpathcurveto{\pgfqpoint{0.901196in}{0.558333in}}{\pgfqpoint{0.886358in}{0.552187in}}{\pgfqpoint{0.875419in}{0.541248in}}%
\pgfpathcurveto{\pgfqpoint{0.864480in}{0.530309in}}{\pgfqpoint{0.858333in}{0.515470in}}{\pgfqpoint{0.858333in}{0.500000in}}%
\pgfpathcurveto{\pgfqpoint{0.858333in}{0.484530in}}{\pgfqpoint{0.864480in}{0.469691in}}{\pgfqpoint{0.875419in}{0.458752in}}%
\pgfpathcurveto{\pgfqpoint{0.886358in}{0.447813in}}{\pgfqpoint{0.901196in}{0.441667in}}{\pgfqpoint{0.916667in}{0.441667in}}%
\pgfpathclose%
\pgfpathmoveto{\pgfqpoint{0.916667in}{0.447500in}}%
\pgfpathcurveto{\pgfqpoint{0.916667in}{0.447500in}}{\pgfqpoint{0.902744in}{0.447500in}}{\pgfqpoint{0.889389in}{0.453032in}}%
\pgfpathcurveto{\pgfqpoint{0.879544in}{0.462877in}}{\pgfqpoint{0.869698in}{0.472722in}}{\pgfqpoint{0.864167in}{0.486077in}}%
\pgfpathcurveto{\pgfqpoint{0.864167in}{0.500000in}}{\pgfqpoint{0.864167in}{0.513923in}}{\pgfqpoint{0.869698in}{0.527278in}}%
\pgfpathcurveto{\pgfqpoint{0.879544in}{0.537123in}}{\pgfqpoint{0.889389in}{0.546968in}}{\pgfqpoint{0.902744in}{0.552500in}}%
\pgfpathcurveto{\pgfqpoint{0.916667in}{0.552500in}}{\pgfqpoint{0.930590in}{0.552500in}}{\pgfqpoint{0.943945in}{0.546968in}}%
\pgfpathcurveto{\pgfqpoint{0.953790in}{0.537123in}}{\pgfqpoint{0.963635in}{0.527278in}}{\pgfqpoint{0.969167in}{0.513923in}}%
\pgfpathcurveto{\pgfqpoint{0.969167in}{0.500000in}}{\pgfqpoint{0.969167in}{0.486077in}}{\pgfqpoint{0.963635in}{0.472722in}}%
\pgfpathcurveto{\pgfqpoint{0.953790in}{0.462877in}}{\pgfqpoint{0.943945in}{0.453032in}}{\pgfqpoint{0.930590in}{0.447500in}}%
\pgfpathclose%
\pgfpathmoveto{\pgfqpoint{0.000000in}{0.608333in}}%
\pgfpathcurveto{\pgfqpoint{0.015470in}{0.608333in}}{\pgfqpoint{0.030309in}{0.614480in}}{\pgfqpoint{0.041248in}{0.625419in}}%
\pgfpathcurveto{\pgfqpoint{0.052187in}{0.636358in}}{\pgfqpoint{0.058333in}{0.651196in}}{\pgfqpoint{0.058333in}{0.666667in}}%
\pgfpathcurveto{\pgfqpoint{0.058333in}{0.682137in}}{\pgfqpoint{0.052187in}{0.696975in}}{\pgfqpoint{0.041248in}{0.707915in}}%
\pgfpathcurveto{\pgfqpoint{0.030309in}{0.718854in}}{\pgfqpoint{0.015470in}{0.725000in}}{\pgfqpoint{0.000000in}{0.725000in}}%
\pgfpathcurveto{\pgfqpoint{-0.015470in}{0.725000in}}{\pgfqpoint{-0.030309in}{0.718854in}}{\pgfqpoint{-0.041248in}{0.707915in}}%
\pgfpathcurveto{\pgfqpoint{-0.052187in}{0.696975in}}{\pgfqpoint{-0.058333in}{0.682137in}}{\pgfqpoint{-0.058333in}{0.666667in}}%
\pgfpathcurveto{\pgfqpoint{-0.058333in}{0.651196in}}{\pgfqpoint{-0.052187in}{0.636358in}}{\pgfqpoint{-0.041248in}{0.625419in}}%
\pgfpathcurveto{\pgfqpoint{-0.030309in}{0.614480in}}{\pgfqpoint{-0.015470in}{0.608333in}}{\pgfqpoint{0.000000in}{0.608333in}}%
\pgfpathclose%
\pgfpathmoveto{\pgfqpoint{0.000000in}{0.614167in}}%
\pgfpathcurveto{\pgfqpoint{0.000000in}{0.614167in}}{\pgfqpoint{-0.013923in}{0.614167in}}{\pgfqpoint{-0.027278in}{0.619698in}}%
\pgfpathcurveto{\pgfqpoint{-0.037123in}{0.629544in}}{\pgfqpoint{-0.046968in}{0.639389in}}{\pgfqpoint{-0.052500in}{0.652744in}}%
\pgfpathcurveto{\pgfqpoint{-0.052500in}{0.666667in}}{\pgfqpoint{-0.052500in}{0.680590in}}{\pgfqpoint{-0.046968in}{0.693945in}}%
\pgfpathcurveto{\pgfqpoint{-0.037123in}{0.703790in}}{\pgfqpoint{-0.027278in}{0.713635in}}{\pgfqpoint{-0.013923in}{0.719167in}}%
\pgfpathcurveto{\pgfqpoint{0.000000in}{0.719167in}}{\pgfqpoint{0.013923in}{0.719167in}}{\pgfqpoint{0.027278in}{0.713635in}}%
\pgfpathcurveto{\pgfqpoint{0.037123in}{0.703790in}}{\pgfqpoint{0.046968in}{0.693945in}}{\pgfqpoint{0.052500in}{0.680590in}}%
\pgfpathcurveto{\pgfqpoint{0.052500in}{0.666667in}}{\pgfqpoint{0.052500in}{0.652744in}}{\pgfqpoint{0.046968in}{0.639389in}}%
\pgfpathcurveto{\pgfqpoint{0.037123in}{0.629544in}}{\pgfqpoint{0.027278in}{0.619698in}}{\pgfqpoint{0.013923in}{0.614167in}}%
\pgfpathclose%
\pgfpathmoveto{\pgfqpoint{0.166667in}{0.608333in}}%
\pgfpathcurveto{\pgfqpoint{0.182137in}{0.608333in}}{\pgfqpoint{0.196975in}{0.614480in}}{\pgfqpoint{0.207915in}{0.625419in}}%
\pgfpathcurveto{\pgfqpoint{0.218854in}{0.636358in}}{\pgfqpoint{0.225000in}{0.651196in}}{\pgfqpoint{0.225000in}{0.666667in}}%
\pgfpathcurveto{\pgfqpoint{0.225000in}{0.682137in}}{\pgfqpoint{0.218854in}{0.696975in}}{\pgfqpoint{0.207915in}{0.707915in}}%
\pgfpathcurveto{\pgfqpoint{0.196975in}{0.718854in}}{\pgfqpoint{0.182137in}{0.725000in}}{\pgfqpoint{0.166667in}{0.725000in}}%
\pgfpathcurveto{\pgfqpoint{0.151196in}{0.725000in}}{\pgfqpoint{0.136358in}{0.718854in}}{\pgfqpoint{0.125419in}{0.707915in}}%
\pgfpathcurveto{\pgfqpoint{0.114480in}{0.696975in}}{\pgfqpoint{0.108333in}{0.682137in}}{\pgfqpoint{0.108333in}{0.666667in}}%
\pgfpathcurveto{\pgfqpoint{0.108333in}{0.651196in}}{\pgfqpoint{0.114480in}{0.636358in}}{\pgfqpoint{0.125419in}{0.625419in}}%
\pgfpathcurveto{\pgfqpoint{0.136358in}{0.614480in}}{\pgfqpoint{0.151196in}{0.608333in}}{\pgfqpoint{0.166667in}{0.608333in}}%
\pgfpathclose%
\pgfpathmoveto{\pgfqpoint{0.166667in}{0.614167in}}%
\pgfpathcurveto{\pgfqpoint{0.166667in}{0.614167in}}{\pgfqpoint{0.152744in}{0.614167in}}{\pgfqpoint{0.139389in}{0.619698in}}%
\pgfpathcurveto{\pgfqpoint{0.129544in}{0.629544in}}{\pgfqpoint{0.119698in}{0.639389in}}{\pgfqpoint{0.114167in}{0.652744in}}%
\pgfpathcurveto{\pgfqpoint{0.114167in}{0.666667in}}{\pgfqpoint{0.114167in}{0.680590in}}{\pgfqpoint{0.119698in}{0.693945in}}%
\pgfpathcurveto{\pgfqpoint{0.129544in}{0.703790in}}{\pgfqpoint{0.139389in}{0.713635in}}{\pgfqpoint{0.152744in}{0.719167in}}%
\pgfpathcurveto{\pgfqpoint{0.166667in}{0.719167in}}{\pgfqpoint{0.180590in}{0.719167in}}{\pgfqpoint{0.193945in}{0.713635in}}%
\pgfpathcurveto{\pgfqpoint{0.203790in}{0.703790in}}{\pgfqpoint{0.213635in}{0.693945in}}{\pgfqpoint{0.219167in}{0.680590in}}%
\pgfpathcurveto{\pgfqpoint{0.219167in}{0.666667in}}{\pgfqpoint{0.219167in}{0.652744in}}{\pgfqpoint{0.213635in}{0.639389in}}%
\pgfpathcurveto{\pgfqpoint{0.203790in}{0.629544in}}{\pgfqpoint{0.193945in}{0.619698in}}{\pgfqpoint{0.180590in}{0.614167in}}%
\pgfpathclose%
\pgfpathmoveto{\pgfqpoint{0.333333in}{0.608333in}}%
\pgfpathcurveto{\pgfqpoint{0.348804in}{0.608333in}}{\pgfqpoint{0.363642in}{0.614480in}}{\pgfqpoint{0.374581in}{0.625419in}}%
\pgfpathcurveto{\pgfqpoint{0.385520in}{0.636358in}}{\pgfqpoint{0.391667in}{0.651196in}}{\pgfqpoint{0.391667in}{0.666667in}}%
\pgfpathcurveto{\pgfqpoint{0.391667in}{0.682137in}}{\pgfqpoint{0.385520in}{0.696975in}}{\pgfqpoint{0.374581in}{0.707915in}}%
\pgfpathcurveto{\pgfqpoint{0.363642in}{0.718854in}}{\pgfqpoint{0.348804in}{0.725000in}}{\pgfqpoint{0.333333in}{0.725000in}}%
\pgfpathcurveto{\pgfqpoint{0.317863in}{0.725000in}}{\pgfqpoint{0.303025in}{0.718854in}}{\pgfqpoint{0.292085in}{0.707915in}}%
\pgfpathcurveto{\pgfqpoint{0.281146in}{0.696975in}}{\pgfqpoint{0.275000in}{0.682137in}}{\pgfqpoint{0.275000in}{0.666667in}}%
\pgfpathcurveto{\pgfqpoint{0.275000in}{0.651196in}}{\pgfqpoint{0.281146in}{0.636358in}}{\pgfqpoint{0.292085in}{0.625419in}}%
\pgfpathcurveto{\pgfqpoint{0.303025in}{0.614480in}}{\pgfqpoint{0.317863in}{0.608333in}}{\pgfqpoint{0.333333in}{0.608333in}}%
\pgfpathclose%
\pgfpathmoveto{\pgfqpoint{0.333333in}{0.614167in}}%
\pgfpathcurveto{\pgfqpoint{0.333333in}{0.614167in}}{\pgfqpoint{0.319410in}{0.614167in}}{\pgfqpoint{0.306055in}{0.619698in}}%
\pgfpathcurveto{\pgfqpoint{0.296210in}{0.629544in}}{\pgfqpoint{0.286365in}{0.639389in}}{\pgfqpoint{0.280833in}{0.652744in}}%
\pgfpathcurveto{\pgfqpoint{0.280833in}{0.666667in}}{\pgfqpoint{0.280833in}{0.680590in}}{\pgfqpoint{0.286365in}{0.693945in}}%
\pgfpathcurveto{\pgfqpoint{0.296210in}{0.703790in}}{\pgfqpoint{0.306055in}{0.713635in}}{\pgfqpoint{0.319410in}{0.719167in}}%
\pgfpathcurveto{\pgfqpoint{0.333333in}{0.719167in}}{\pgfqpoint{0.347256in}{0.719167in}}{\pgfqpoint{0.360611in}{0.713635in}}%
\pgfpathcurveto{\pgfqpoint{0.370456in}{0.703790in}}{\pgfqpoint{0.380302in}{0.693945in}}{\pgfqpoint{0.385833in}{0.680590in}}%
\pgfpathcurveto{\pgfqpoint{0.385833in}{0.666667in}}{\pgfqpoint{0.385833in}{0.652744in}}{\pgfqpoint{0.380302in}{0.639389in}}%
\pgfpathcurveto{\pgfqpoint{0.370456in}{0.629544in}}{\pgfqpoint{0.360611in}{0.619698in}}{\pgfqpoint{0.347256in}{0.614167in}}%
\pgfpathclose%
\pgfpathmoveto{\pgfqpoint{0.500000in}{0.608333in}}%
\pgfpathcurveto{\pgfqpoint{0.515470in}{0.608333in}}{\pgfqpoint{0.530309in}{0.614480in}}{\pgfqpoint{0.541248in}{0.625419in}}%
\pgfpathcurveto{\pgfqpoint{0.552187in}{0.636358in}}{\pgfqpoint{0.558333in}{0.651196in}}{\pgfqpoint{0.558333in}{0.666667in}}%
\pgfpathcurveto{\pgfqpoint{0.558333in}{0.682137in}}{\pgfqpoint{0.552187in}{0.696975in}}{\pgfqpoint{0.541248in}{0.707915in}}%
\pgfpathcurveto{\pgfqpoint{0.530309in}{0.718854in}}{\pgfqpoint{0.515470in}{0.725000in}}{\pgfqpoint{0.500000in}{0.725000in}}%
\pgfpathcurveto{\pgfqpoint{0.484530in}{0.725000in}}{\pgfqpoint{0.469691in}{0.718854in}}{\pgfqpoint{0.458752in}{0.707915in}}%
\pgfpathcurveto{\pgfqpoint{0.447813in}{0.696975in}}{\pgfqpoint{0.441667in}{0.682137in}}{\pgfqpoint{0.441667in}{0.666667in}}%
\pgfpathcurveto{\pgfqpoint{0.441667in}{0.651196in}}{\pgfqpoint{0.447813in}{0.636358in}}{\pgfqpoint{0.458752in}{0.625419in}}%
\pgfpathcurveto{\pgfqpoint{0.469691in}{0.614480in}}{\pgfqpoint{0.484530in}{0.608333in}}{\pgfqpoint{0.500000in}{0.608333in}}%
\pgfpathclose%
\pgfpathmoveto{\pgfqpoint{0.500000in}{0.614167in}}%
\pgfpathcurveto{\pgfqpoint{0.500000in}{0.614167in}}{\pgfqpoint{0.486077in}{0.614167in}}{\pgfqpoint{0.472722in}{0.619698in}}%
\pgfpathcurveto{\pgfqpoint{0.462877in}{0.629544in}}{\pgfqpoint{0.453032in}{0.639389in}}{\pgfqpoint{0.447500in}{0.652744in}}%
\pgfpathcurveto{\pgfqpoint{0.447500in}{0.666667in}}{\pgfqpoint{0.447500in}{0.680590in}}{\pgfqpoint{0.453032in}{0.693945in}}%
\pgfpathcurveto{\pgfqpoint{0.462877in}{0.703790in}}{\pgfqpoint{0.472722in}{0.713635in}}{\pgfqpoint{0.486077in}{0.719167in}}%
\pgfpathcurveto{\pgfqpoint{0.500000in}{0.719167in}}{\pgfqpoint{0.513923in}{0.719167in}}{\pgfqpoint{0.527278in}{0.713635in}}%
\pgfpathcurveto{\pgfqpoint{0.537123in}{0.703790in}}{\pgfqpoint{0.546968in}{0.693945in}}{\pgfqpoint{0.552500in}{0.680590in}}%
\pgfpathcurveto{\pgfqpoint{0.552500in}{0.666667in}}{\pgfqpoint{0.552500in}{0.652744in}}{\pgfqpoint{0.546968in}{0.639389in}}%
\pgfpathcurveto{\pgfqpoint{0.537123in}{0.629544in}}{\pgfqpoint{0.527278in}{0.619698in}}{\pgfqpoint{0.513923in}{0.614167in}}%
\pgfpathclose%
\pgfpathmoveto{\pgfqpoint{0.666667in}{0.608333in}}%
\pgfpathcurveto{\pgfqpoint{0.682137in}{0.608333in}}{\pgfqpoint{0.696975in}{0.614480in}}{\pgfqpoint{0.707915in}{0.625419in}}%
\pgfpathcurveto{\pgfqpoint{0.718854in}{0.636358in}}{\pgfqpoint{0.725000in}{0.651196in}}{\pgfqpoint{0.725000in}{0.666667in}}%
\pgfpathcurveto{\pgfqpoint{0.725000in}{0.682137in}}{\pgfqpoint{0.718854in}{0.696975in}}{\pgfqpoint{0.707915in}{0.707915in}}%
\pgfpathcurveto{\pgfqpoint{0.696975in}{0.718854in}}{\pgfqpoint{0.682137in}{0.725000in}}{\pgfqpoint{0.666667in}{0.725000in}}%
\pgfpathcurveto{\pgfqpoint{0.651196in}{0.725000in}}{\pgfqpoint{0.636358in}{0.718854in}}{\pgfqpoint{0.625419in}{0.707915in}}%
\pgfpathcurveto{\pgfqpoint{0.614480in}{0.696975in}}{\pgfqpoint{0.608333in}{0.682137in}}{\pgfqpoint{0.608333in}{0.666667in}}%
\pgfpathcurveto{\pgfqpoint{0.608333in}{0.651196in}}{\pgfqpoint{0.614480in}{0.636358in}}{\pgfqpoint{0.625419in}{0.625419in}}%
\pgfpathcurveto{\pgfqpoint{0.636358in}{0.614480in}}{\pgfqpoint{0.651196in}{0.608333in}}{\pgfqpoint{0.666667in}{0.608333in}}%
\pgfpathclose%
\pgfpathmoveto{\pgfqpoint{0.666667in}{0.614167in}}%
\pgfpathcurveto{\pgfqpoint{0.666667in}{0.614167in}}{\pgfqpoint{0.652744in}{0.614167in}}{\pgfqpoint{0.639389in}{0.619698in}}%
\pgfpathcurveto{\pgfqpoint{0.629544in}{0.629544in}}{\pgfqpoint{0.619698in}{0.639389in}}{\pgfqpoint{0.614167in}{0.652744in}}%
\pgfpathcurveto{\pgfqpoint{0.614167in}{0.666667in}}{\pgfqpoint{0.614167in}{0.680590in}}{\pgfqpoint{0.619698in}{0.693945in}}%
\pgfpathcurveto{\pgfqpoint{0.629544in}{0.703790in}}{\pgfqpoint{0.639389in}{0.713635in}}{\pgfqpoint{0.652744in}{0.719167in}}%
\pgfpathcurveto{\pgfqpoint{0.666667in}{0.719167in}}{\pgfqpoint{0.680590in}{0.719167in}}{\pgfqpoint{0.693945in}{0.713635in}}%
\pgfpathcurveto{\pgfqpoint{0.703790in}{0.703790in}}{\pgfqpoint{0.713635in}{0.693945in}}{\pgfqpoint{0.719167in}{0.680590in}}%
\pgfpathcurveto{\pgfqpoint{0.719167in}{0.666667in}}{\pgfqpoint{0.719167in}{0.652744in}}{\pgfqpoint{0.713635in}{0.639389in}}%
\pgfpathcurveto{\pgfqpoint{0.703790in}{0.629544in}}{\pgfqpoint{0.693945in}{0.619698in}}{\pgfqpoint{0.680590in}{0.614167in}}%
\pgfpathclose%
\pgfpathmoveto{\pgfqpoint{0.833333in}{0.608333in}}%
\pgfpathcurveto{\pgfqpoint{0.848804in}{0.608333in}}{\pgfqpoint{0.863642in}{0.614480in}}{\pgfqpoint{0.874581in}{0.625419in}}%
\pgfpathcurveto{\pgfqpoint{0.885520in}{0.636358in}}{\pgfqpoint{0.891667in}{0.651196in}}{\pgfqpoint{0.891667in}{0.666667in}}%
\pgfpathcurveto{\pgfqpoint{0.891667in}{0.682137in}}{\pgfqpoint{0.885520in}{0.696975in}}{\pgfqpoint{0.874581in}{0.707915in}}%
\pgfpathcurveto{\pgfqpoint{0.863642in}{0.718854in}}{\pgfqpoint{0.848804in}{0.725000in}}{\pgfqpoint{0.833333in}{0.725000in}}%
\pgfpathcurveto{\pgfqpoint{0.817863in}{0.725000in}}{\pgfqpoint{0.803025in}{0.718854in}}{\pgfqpoint{0.792085in}{0.707915in}}%
\pgfpathcurveto{\pgfqpoint{0.781146in}{0.696975in}}{\pgfqpoint{0.775000in}{0.682137in}}{\pgfqpoint{0.775000in}{0.666667in}}%
\pgfpathcurveto{\pgfqpoint{0.775000in}{0.651196in}}{\pgfqpoint{0.781146in}{0.636358in}}{\pgfqpoint{0.792085in}{0.625419in}}%
\pgfpathcurveto{\pgfqpoint{0.803025in}{0.614480in}}{\pgfqpoint{0.817863in}{0.608333in}}{\pgfqpoint{0.833333in}{0.608333in}}%
\pgfpathclose%
\pgfpathmoveto{\pgfqpoint{0.833333in}{0.614167in}}%
\pgfpathcurveto{\pgfqpoint{0.833333in}{0.614167in}}{\pgfqpoint{0.819410in}{0.614167in}}{\pgfqpoint{0.806055in}{0.619698in}}%
\pgfpathcurveto{\pgfqpoint{0.796210in}{0.629544in}}{\pgfqpoint{0.786365in}{0.639389in}}{\pgfqpoint{0.780833in}{0.652744in}}%
\pgfpathcurveto{\pgfqpoint{0.780833in}{0.666667in}}{\pgfqpoint{0.780833in}{0.680590in}}{\pgfqpoint{0.786365in}{0.693945in}}%
\pgfpathcurveto{\pgfqpoint{0.796210in}{0.703790in}}{\pgfqpoint{0.806055in}{0.713635in}}{\pgfqpoint{0.819410in}{0.719167in}}%
\pgfpathcurveto{\pgfqpoint{0.833333in}{0.719167in}}{\pgfqpoint{0.847256in}{0.719167in}}{\pgfqpoint{0.860611in}{0.713635in}}%
\pgfpathcurveto{\pgfqpoint{0.870456in}{0.703790in}}{\pgfqpoint{0.880302in}{0.693945in}}{\pgfqpoint{0.885833in}{0.680590in}}%
\pgfpathcurveto{\pgfqpoint{0.885833in}{0.666667in}}{\pgfqpoint{0.885833in}{0.652744in}}{\pgfqpoint{0.880302in}{0.639389in}}%
\pgfpathcurveto{\pgfqpoint{0.870456in}{0.629544in}}{\pgfqpoint{0.860611in}{0.619698in}}{\pgfqpoint{0.847256in}{0.614167in}}%
\pgfpathclose%
\pgfpathmoveto{\pgfqpoint{1.000000in}{0.608333in}}%
\pgfpathcurveto{\pgfqpoint{1.015470in}{0.608333in}}{\pgfqpoint{1.030309in}{0.614480in}}{\pgfqpoint{1.041248in}{0.625419in}}%
\pgfpathcurveto{\pgfqpoint{1.052187in}{0.636358in}}{\pgfqpoint{1.058333in}{0.651196in}}{\pgfqpoint{1.058333in}{0.666667in}}%
\pgfpathcurveto{\pgfqpoint{1.058333in}{0.682137in}}{\pgfqpoint{1.052187in}{0.696975in}}{\pgfqpoint{1.041248in}{0.707915in}}%
\pgfpathcurveto{\pgfqpoint{1.030309in}{0.718854in}}{\pgfqpoint{1.015470in}{0.725000in}}{\pgfqpoint{1.000000in}{0.725000in}}%
\pgfpathcurveto{\pgfqpoint{0.984530in}{0.725000in}}{\pgfqpoint{0.969691in}{0.718854in}}{\pgfqpoint{0.958752in}{0.707915in}}%
\pgfpathcurveto{\pgfqpoint{0.947813in}{0.696975in}}{\pgfqpoint{0.941667in}{0.682137in}}{\pgfqpoint{0.941667in}{0.666667in}}%
\pgfpathcurveto{\pgfqpoint{0.941667in}{0.651196in}}{\pgfqpoint{0.947813in}{0.636358in}}{\pgfqpoint{0.958752in}{0.625419in}}%
\pgfpathcurveto{\pgfqpoint{0.969691in}{0.614480in}}{\pgfqpoint{0.984530in}{0.608333in}}{\pgfqpoint{1.000000in}{0.608333in}}%
\pgfpathclose%
\pgfpathmoveto{\pgfqpoint{1.000000in}{0.614167in}}%
\pgfpathcurveto{\pgfqpoint{1.000000in}{0.614167in}}{\pgfqpoint{0.986077in}{0.614167in}}{\pgfqpoint{0.972722in}{0.619698in}}%
\pgfpathcurveto{\pgfqpoint{0.962877in}{0.629544in}}{\pgfqpoint{0.953032in}{0.639389in}}{\pgfqpoint{0.947500in}{0.652744in}}%
\pgfpathcurveto{\pgfqpoint{0.947500in}{0.666667in}}{\pgfqpoint{0.947500in}{0.680590in}}{\pgfqpoint{0.953032in}{0.693945in}}%
\pgfpathcurveto{\pgfqpoint{0.962877in}{0.703790in}}{\pgfqpoint{0.972722in}{0.713635in}}{\pgfqpoint{0.986077in}{0.719167in}}%
\pgfpathcurveto{\pgfqpoint{1.000000in}{0.719167in}}{\pgfqpoint{1.013923in}{0.719167in}}{\pgfqpoint{1.027278in}{0.713635in}}%
\pgfpathcurveto{\pgfqpoint{1.037123in}{0.703790in}}{\pgfqpoint{1.046968in}{0.693945in}}{\pgfqpoint{1.052500in}{0.680590in}}%
\pgfpathcurveto{\pgfqpoint{1.052500in}{0.666667in}}{\pgfqpoint{1.052500in}{0.652744in}}{\pgfqpoint{1.046968in}{0.639389in}}%
\pgfpathcurveto{\pgfqpoint{1.037123in}{0.629544in}}{\pgfqpoint{1.027278in}{0.619698in}}{\pgfqpoint{1.013923in}{0.614167in}}%
\pgfpathclose%
\pgfpathmoveto{\pgfqpoint{0.083333in}{0.775000in}}%
\pgfpathcurveto{\pgfqpoint{0.098804in}{0.775000in}}{\pgfqpoint{0.113642in}{0.781146in}}{\pgfqpoint{0.124581in}{0.792085in}}%
\pgfpathcurveto{\pgfqpoint{0.135520in}{0.803025in}}{\pgfqpoint{0.141667in}{0.817863in}}{\pgfqpoint{0.141667in}{0.833333in}}%
\pgfpathcurveto{\pgfqpoint{0.141667in}{0.848804in}}{\pgfqpoint{0.135520in}{0.863642in}}{\pgfqpoint{0.124581in}{0.874581in}}%
\pgfpathcurveto{\pgfqpoint{0.113642in}{0.885520in}}{\pgfqpoint{0.098804in}{0.891667in}}{\pgfqpoint{0.083333in}{0.891667in}}%
\pgfpathcurveto{\pgfqpoint{0.067863in}{0.891667in}}{\pgfqpoint{0.053025in}{0.885520in}}{\pgfqpoint{0.042085in}{0.874581in}}%
\pgfpathcurveto{\pgfqpoint{0.031146in}{0.863642in}}{\pgfqpoint{0.025000in}{0.848804in}}{\pgfqpoint{0.025000in}{0.833333in}}%
\pgfpathcurveto{\pgfqpoint{0.025000in}{0.817863in}}{\pgfqpoint{0.031146in}{0.803025in}}{\pgfqpoint{0.042085in}{0.792085in}}%
\pgfpathcurveto{\pgfqpoint{0.053025in}{0.781146in}}{\pgfqpoint{0.067863in}{0.775000in}}{\pgfqpoint{0.083333in}{0.775000in}}%
\pgfpathclose%
\pgfpathmoveto{\pgfqpoint{0.083333in}{0.780833in}}%
\pgfpathcurveto{\pgfqpoint{0.083333in}{0.780833in}}{\pgfqpoint{0.069410in}{0.780833in}}{\pgfqpoint{0.056055in}{0.786365in}}%
\pgfpathcurveto{\pgfqpoint{0.046210in}{0.796210in}}{\pgfqpoint{0.036365in}{0.806055in}}{\pgfqpoint{0.030833in}{0.819410in}}%
\pgfpathcurveto{\pgfqpoint{0.030833in}{0.833333in}}{\pgfqpoint{0.030833in}{0.847256in}}{\pgfqpoint{0.036365in}{0.860611in}}%
\pgfpathcurveto{\pgfqpoint{0.046210in}{0.870456in}}{\pgfqpoint{0.056055in}{0.880302in}}{\pgfqpoint{0.069410in}{0.885833in}}%
\pgfpathcurveto{\pgfqpoint{0.083333in}{0.885833in}}{\pgfqpoint{0.097256in}{0.885833in}}{\pgfqpoint{0.110611in}{0.880302in}}%
\pgfpathcurveto{\pgfqpoint{0.120456in}{0.870456in}}{\pgfqpoint{0.130302in}{0.860611in}}{\pgfqpoint{0.135833in}{0.847256in}}%
\pgfpathcurveto{\pgfqpoint{0.135833in}{0.833333in}}{\pgfqpoint{0.135833in}{0.819410in}}{\pgfqpoint{0.130302in}{0.806055in}}%
\pgfpathcurveto{\pgfqpoint{0.120456in}{0.796210in}}{\pgfqpoint{0.110611in}{0.786365in}}{\pgfqpoint{0.097256in}{0.780833in}}%
\pgfpathclose%
\pgfpathmoveto{\pgfqpoint{0.250000in}{0.775000in}}%
\pgfpathcurveto{\pgfqpoint{0.265470in}{0.775000in}}{\pgfqpoint{0.280309in}{0.781146in}}{\pgfqpoint{0.291248in}{0.792085in}}%
\pgfpathcurveto{\pgfqpoint{0.302187in}{0.803025in}}{\pgfqpoint{0.308333in}{0.817863in}}{\pgfqpoint{0.308333in}{0.833333in}}%
\pgfpathcurveto{\pgfqpoint{0.308333in}{0.848804in}}{\pgfqpoint{0.302187in}{0.863642in}}{\pgfqpoint{0.291248in}{0.874581in}}%
\pgfpathcurveto{\pgfqpoint{0.280309in}{0.885520in}}{\pgfqpoint{0.265470in}{0.891667in}}{\pgfqpoint{0.250000in}{0.891667in}}%
\pgfpathcurveto{\pgfqpoint{0.234530in}{0.891667in}}{\pgfqpoint{0.219691in}{0.885520in}}{\pgfqpoint{0.208752in}{0.874581in}}%
\pgfpathcurveto{\pgfqpoint{0.197813in}{0.863642in}}{\pgfqpoint{0.191667in}{0.848804in}}{\pgfqpoint{0.191667in}{0.833333in}}%
\pgfpathcurveto{\pgfqpoint{0.191667in}{0.817863in}}{\pgfqpoint{0.197813in}{0.803025in}}{\pgfqpoint{0.208752in}{0.792085in}}%
\pgfpathcurveto{\pgfqpoint{0.219691in}{0.781146in}}{\pgfqpoint{0.234530in}{0.775000in}}{\pgfqpoint{0.250000in}{0.775000in}}%
\pgfpathclose%
\pgfpathmoveto{\pgfqpoint{0.250000in}{0.780833in}}%
\pgfpathcurveto{\pgfqpoint{0.250000in}{0.780833in}}{\pgfqpoint{0.236077in}{0.780833in}}{\pgfqpoint{0.222722in}{0.786365in}}%
\pgfpathcurveto{\pgfqpoint{0.212877in}{0.796210in}}{\pgfqpoint{0.203032in}{0.806055in}}{\pgfqpoint{0.197500in}{0.819410in}}%
\pgfpathcurveto{\pgfqpoint{0.197500in}{0.833333in}}{\pgfqpoint{0.197500in}{0.847256in}}{\pgfqpoint{0.203032in}{0.860611in}}%
\pgfpathcurveto{\pgfqpoint{0.212877in}{0.870456in}}{\pgfqpoint{0.222722in}{0.880302in}}{\pgfqpoint{0.236077in}{0.885833in}}%
\pgfpathcurveto{\pgfqpoint{0.250000in}{0.885833in}}{\pgfqpoint{0.263923in}{0.885833in}}{\pgfqpoint{0.277278in}{0.880302in}}%
\pgfpathcurveto{\pgfqpoint{0.287123in}{0.870456in}}{\pgfqpoint{0.296968in}{0.860611in}}{\pgfqpoint{0.302500in}{0.847256in}}%
\pgfpathcurveto{\pgfqpoint{0.302500in}{0.833333in}}{\pgfqpoint{0.302500in}{0.819410in}}{\pgfqpoint{0.296968in}{0.806055in}}%
\pgfpathcurveto{\pgfqpoint{0.287123in}{0.796210in}}{\pgfqpoint{0.277278in}{0.786365in}}{\pgfqpoint{0.263923in}{0.780833in}}%
\pgfpathclose%
\pgfpathmoveto{\pgfqpoint{0.416667in}{0.775000in}}%
\pgfpathcurveto{\pgfqpoint{0.432137in}{0.775000in}}{\pgfqpoint{0.446975in}{0.781146in}}{\pgfqpoint{0.457915in}{0.792085in}}%
\pgfpathcurveto{\pgfqpoint{0.468854in}{0.803025in}}{\pgfqpoint{0.475000in}{0.817863in}}{\pgfqpoint{0.475000in}{0.833333in}}%
\pgfpathcurveto{\pgfqpoint{0.475000in}{0.848804in}}{\pgfqpoint{0.468854in}{0.863642in}}{\pgfqpoint{0.457915in}{0.874581in}}%
\pgfpathcurveto{\pgfqpoint{0.446975in}{0.885520in}}{\pgfqpoint{0.432137in}{0.891667in}}{\pgfqpoint{0.416667in}{0.891667in}}%
\pgfpathcurveto{\pgfqpoint{0.401196in}{0.891667in}}{\pgfqpoint{0.386358in}{0.885520in}}{\pgfqpoint{0.375419in}{0.874581in}}%
\pgfpathcurveto{\pgfqpoint{0.364480in}{0.863642in}}{\pgfqpoint{0.358333in}{0.848804in}}{\pgfqpoint{0.358333in}{0.833333in}}%
\pgfpathcurveto{\pgfqpoint{0.358333in}{0.817863in}}{\pgfqpoint{0.364480in}{0.803025in}}{\pgfqpoint{0.375419in}{0.792085in}}%
\pgfpathcurveto{\pgfqpoint{0.386358in}{0.781146in}}{\pgfqpoint{0.401196in}{0.775000in}}{\pgfqpoint{0.416667in}{0.775000in}}%
\pgfpathclose%
\pgfpathmoveto{\pgfqpoint{0.416667in}{0.780833in}}%
\pgfpathcurveto{\pgfqpoint{0.416667in}{0.780833in}}{\pgfqpoint{0.402744in}{0.780833in}}{\pgfqpoint{0.389389in}{0.786365in}}%
\pgfpathcurveto{\pgfqpoint{0.379544in}{0.796210in}}{\pgfqpoint{0.369698in}{0.806055in}}{\pgfqpoint{0.364167in}{0.819410in}}%
\pgfpathcurveto{\pgfqpoint{0.364167in}{0.833333in}}{\pgfqpoint{0.364167in}{0.847256in}}{\pgfqpoint{0.369698in}{0.860611in}}%
\pgfpathcurveto{\pgfqpoint{0.379544in}{0.870456in}}{\pgfqpoint{0.389389in}{0.880302in}}{\pgfqpoint{0.402744in}{0.885833in}}%
\pgfpathcurveto{\pgfqpoint{0.416667in}{0.885833in}}{\pgfqpoint{0.430590in}{0.885833in}}{\pgfqpoint{0.443945in}{0.880302in}}%
\pgfpathcurveto{\pgfqpoint{0.453790in}{0.870456in}}{\pgfqpoint{0.463635in}{0.860611in}}{\pgfqpoint{0.469167in}{0.847256in}}%
\pgfpathcurveto{\pgfqpoint{0.469167in}{0.833333in}}{\pgfqpoint{0.469167in}{0.819410in}}{\pgfqpoint{0.463635in}{0.806055in}}%
\pgfpathcurveto{\pgfqpoint{0.453790in}{0.796210in}}{\pgfqpoint{0.443945in}{0.786365in}}{\pgfqpoint{0.430590in}{0.780833in}}%
\pgfpathclose%
\pgfpathmoveto{\pgfqpoint{0.583333in}{0.775000in}}%
\pgfpathcurveto{\pgfqpoint{0.598804in}{0.775000in}}{\pgfqpoint{0.613642in}{0.781146in}}{\pgfqpoint{0.624581in}{0.792085in}}%
\pgfpathcurveto{\pgfqpoint{0.635520in}{0.803025in}}{\pgfqpoint{0.641667in}{0.817863in}}{\pgfqpoint{0.641667in}{0.833333in}}%
\pgfpathcurveto{\pgfqpoint{0.641667in}{0.848804in}}{\pgfqpoint{0.635520in}{0.863642in}}{\pgfqpoint{0.624581in}{0.874581in}}%
\pgfpathcurveto{\pgfqpoint{0.613642in}{0.885520in}}{\pgfqpoint{0.598804in}{0.891667in}}{\pgfqpoint{0.583333in}{0.891667in}}%
\pgfpathcurveto{\pgfqpoint{0.567863in}{0.891667in}}{\pgfqpoint{0.553025in}{0.885520in}}{\pgfqpoint{0.542085in}{0.874581in}}%
\pgfpathcurveto{\pgfqpoint{0.531146in}{0.863642in}}{\pgfqpoint{0.525000in}{0.848804in}}{\pgfqpoint{0.525000in}{0.833333in}}%
\pgfpathcurveto{\pgfqpoint{0.525000in}{0.817863in}}{\pgfqpoint{0.531146in}{0.803025in}}{\pgfqpoint{0.542085in}{0.792085in}}%
\pgfpathcurveto{\pgfqpoint{0.553025in}{0.781146in}}{\pgfqpoint{0.567863in}{0.775000in}}{\pgfqpoint{0.583333in}{0.775000in}}%
\pgfpathclose%
\pgfpathmoveto{\pgfqpoint{0.583333in}{0.780833in}}%
\pgfpathcurveto{\pgfqpoint{0.583333in}{0.780833in}}{\pgfqpoint{0.569410in}{0.780833in}}{\pgfqpoint{0.556055in}{0.786365in}}%
\pgfpathcurveto{\pgfqpoint{0.546210in}{0.796210in}}{\pgfqpoint{0.536365in}{0.806055in}}{\pgfqpoint{0.530833in}{0.819410in}}%
\pgfpathcurveto{\pgfqpoint{0.530833in}{0.833333in}}{\pgfqpoint{0.530833in}{0.847256in}}{\pgfqpoint{0.536365in}{0.860611in}}%
\pgfpathcurveto{\pgfqpoint{0.546210in}{0.870456in}}{\pgfqpoint{0.556055in}{0.880302in}}{\pgfqpoint{0.569410in}{0.885833in}}%
\pgfpathcurveto{\pgfqpoint{0.583333in}{0.885833in}}{\pgfqpoint{0.597256in}{0.885833in}}{\pgfqpoint{0.610611in}{0.880302in}}%
\pgfpathcurveto{\pgfqpoint{0.620456in}{0.870456in}}{\pgfqpoint{0.630302in}{0.860611in}}{\pgfqpoint{0.635833in}{0.847256in}}%
\pgfpathcurveto{\pgfqpoint{0.635833in}{0.833333in}}{\pgfqpoint{0.635833in}{0.819410in}}{\pgfqpoint{0.630302in}{0.806055in}}%
\pgfpathcurveto{\pgfqpoint{0.620456in}{0.796210in}}{\pgfqpoint{0.610611in}{0.786365in}}{\pgfqpoint{0.597256in}{0.780833in}}%
\pgfpathclose%
\pgfpathmoveto{\pgfqpoint{0.750000in}{0.775000in}}%
\pgfpathcurveto{\pgfqpoint{0.765470in}{0.775000in}}{\pgfqpoint{0.780309in}{0.781146in}}{\pgfqpoint{0.791248in}{0.792085in}}%
\pgfpathcurveto{\pgfqpoint{0.802187in}{0.803025in}}{\pgfqpoint{0.808333in}{0.817863in}}{\pgfqpoint{0.808333in}{0.833333in}}%
\pgfpathcurveto{\pgfqpoint{0.808333in}{0.848804in}}{\pgfqpoint{0.802187in}{0.863642in}}{\pgfqpoint{0.791248in}{0.874581in}}%
\pgfpathcurveto{\pgfqpoint{0.780309in}{0.885520in}}{\pgfqpoint{0.765470in}{0.891667in}}{\pgfqpoint{0.750000in}{0.891667in}}%
\pgfpathcurveto{\pgfqpoint{0.734530in}{0.891667in}}{\pgfqpoint{0.719691in}{0.885520in}}{\pgfqpoint{0.708752in}{0.874581in}}%
\pgfpathcurveto{\pgfqpoint{0.697813in}{0.863642in}}{\pgfqpoint{0.691667in}{0.848804in}}{\pgfqpoint{0.691667in}{0.833333in}}%
\pgfpathcurveto{\pgfqpoint{0.691667in}{0.817863in}}{\pgfqpoint{0.697813in}{0.803025in}}{\pgfqpoint{0.708752in}{0.792085in}}%
\pgfpathcurveto{\pgfqpoint{0.719691in}{0.781146in}}{\pgfqpoint{0.734530in}{0.775000in}}{\pgfqpoint{0.750000in}{0.775000in}}%
\pgfpathclose%
\pgfpathmoveto{\pgfqpoint{0.750000in}{0.780833in}}%
\pgfpathcurveto{\pgfqpoint{0.750000in}{0.780833in}}{\pgfqpoint{0.736077in}{0.780833in}}{\pgfqpoint{0.722722in}{0.786365in}}%
\pgfpathcurveto{\pgfqpoint{0.712877in}{0.796210in}}{\pgfqpoint{0.703032in}{0.806055in}}{\pgfqpoint{0.697500in}{0.819410in}}%
\pgfpathcurveto{\pgfqpoint{0.697500in}{0.833333in}}{\pgfqpoint{0.697500in}{0.847256in}}{\pgfqpoint{0.703032in}{0.860611in}}%
\pgfpathcurveto{\pgfqpoint{0.712877in}{0.870456in}}{\pgfqpoint{0.722722in}{0.880302in}}{\pgfqpoint{0.736077in}{0.885833in}}%
\pgfpathcurveto{\pgfqpoint{0.750000in}{0.885833in}}{\pgfqpoint{0.763923in}{0.885833in}}{\pgfqpoint{0.777278in}{0.880302in}}%
\pgfpathcurveto{\pgfqpoint{0.787123in}{0.870456in}}{\pgfqpoint{0.796968in}{0.860611in}}{\pgfqpoint{0.802500in}{0.847256in}}%
\pgfpathcurveto{\pgfqpoint{0.802500in}{0.833333in}}{\pgfqpoint{0.802500in}{0.819410in}}{\pgfqpoint{0.796968in}{0.806055in}}%
\pgfpathcurveto{\pgfqpoint{0.787123in}{0.796210in}}{\pgfqpoint{0.777278in}{0.786365in}}{\pgfqpoint{0.763923in}{0.780833in}}%
\pgfpathclose%
\pgfpathmoveto{\pgfqpoint{0.916667in}{0.775000in}}%
\pgfpathcurveto{\pgfqpoint{0.932137in}{0.775000in}}{\pgfqpoint{0.946975in}{0.781146in}}{\pgfqpoint{0.957915in}{0.792085in}}%
\pgfpathcurveto{\pgfqpoint{0.968854in}{0.803025in}}{\pgfqpoint{0.975000in}{0.817863in}}{\pgfqpoint{0.975000in}{0.833333in}}%
\pgfpathcurveto{\pgfqpoint{0.975000in}{0.848804in}}{\pgfqpoint{0.968854in}{0.863642in}}{\pgfqpoint{0.957915in}{0.874581in}}%
\pgfpathcurveto{\pgfqpoint{0.946975in}{0.885520in}}{\pgfqpoint{0.932137in}{0.891667in}}{\pgfqpoint{0.916667in}{0.891667in}}%
\pgfpathcurveto{\pgfqpoint{0.901196in}{0.891667in}}{\pgfqpoint{0.886358in}{0.885520in}}{\pgfqpoint{0.875419in}{0.874581in}}%
\pgfpathcurveto{\pgfqpoint{0.864480in}{0.863642in}}{\pgfqpoint{0.858333in}{0.848804in}}{\pgfqpoint{0.858333in}{0.833333in}}%
\pgfpathcurveto{\pgfqpoint{0.858333in}{0.817863in}}{\pgfqpoint{0.864480in}{0.803025in}}{\pgfqpoint{0.875419in}{0.792085in}}%
\pgfpathcurveto{\pgfqpoint{0.886358in}{0.781146in}}{\pgfqpoint{0.901196in}{0.775000in}}{\pgfqpoint{0.916667in}{0.775000in}}%
\pgfpathclose%
\pgfpathmoveto{\pgfqpoint{0.916667in}{0.780833in}}%
\pgfpathcurveto{\pgfqpoint{0.916667in}{0.780833in}}{\pgfqpoint{0.902744in}{0.780833in}}{\pgfqpoint{0.889389in}{0.786365in}}%
\pgfpathcurveto{\pgfqpoint{0.879544in}{0.796210in}}{\pgfqpoint{0.869698in}{0.806055in}}{\pgfqpoint{0.864167in}{0.819410in}}%
\pgfpathcurveto{\pgfqpoint{0.864167in}{0.833333in}}{\pgfqpoint{0.864167in}{0.847256in}}{\pgfqpoint{0.869698in}{0.860611in}}%
\pgfpathcurveto{\pgfqpoint{0.879544in}{0.870456in}}{\pgfqpoint{0.889389in}{0.880302in}}{\pgfqpoint{0.902744in}{0.885833in}}%
\pgfpathcurveto{\pgfqpoint{0.916667in}{0.885833in}}{\pgfqpoint{0.930590in}{0.885833in}}{\pgfqpoint{0.943945in}{0.880302in}}%
\pgfpathcurveto{\pgfqpoint{0.953790in}{0.870456in}}{\pgfqpoint{0.963635in}{0.860611in}}{\pgfqpoint{0.969167in}{0.847256in}}%
\pgfpathcurveto{\pgfqpoint{0.969167in}{0.833333in}}{\pgfqpoint{0.969167in}{0.819410in}}{\pgfqpoint{0.963635in}{0.806055in}}%
\pgfpathcurveto{\pgfqpoint{0.953790in}{0.796210in}}{\pgfqpoint{0.943945in}{0.786365in}}{\pgfqpoint{0.930590in}{0.780833in}}%
\pgfpathclose%
\pgfpathmoveto{\pgfqpoint{0.000000in}{0.941667in}}%
\pgfpathcurveto{\pgfqpoint{0.015470in}{0.941667in}}{\pgfqpoint{0.030309in}{0.947813in}}{\pgfqpoint{0.041248in}{0.958752in}}%
\pgfpathcurveto{\pgfqpoint{0.052187in}{0.969691in}}{\pgfqpoint{0.058333in}{0.984530in}}{\pgfqpoint{0.058333in}{1.000000in}}%
\pgfpathcurveto{\pgfqpoint{0.058333in}{1.015470in}}{\pgfqpoint{0.052187in}{1.030309in}}{\pgfqpoint{0.041248in}{1.041248in}}%
\pgfpathcurveto{\pgfqpoint{0.030309in}{1.052187in}}{\pgfqpoint{0.015470in}{1.058333in}}{\pgfqpoint{0.000000in}{1.058333in}}%
\pgfpathcurveto{\pgfqpoint{-0.015470in}{1.058333in}}{\pgfqpoint{-0.030309in}{1.052187in}}{\pgfqpoint{-0.041248in}{1.041248in}}%
\pgfpathcurveto{\pgfqpoint{-0.052187in}{1.030309in}}{\pgfqpoint{-0.058333in}{1.015470in}}{\pgfqpoint{-0.058333in}{1.000000in}}%
\pgfpathcurveto{\pgfqpoint{-0.058333in}{0.984530in}}{\pgfqpoint{-0.052187in}{0.969691in}}{\pgfqpoint{-0.041248in}{0.958752in}}%
\pgfpathcurveto{\pgfqpoint{-0.030309in}{0.947813in}}{\pgfqpoint{-0.015470in}{0.941667in}}{\pgfqpoint{0.000000in}{0.941667in}}%
\pgfpathclose%
\pgfpathmoveto{\pgfqpoint{0.000000in}{0.947500in}}%
\pgfpathcurveto{\pgfqpoint{0.000000in}{0.947500in}}{\pgfqpoint{-0.013923in}{0.947500in}}{\pgfqpoint{-0.027278in}{0.953032in}}%
\pgfpathcurveto{\pgfqpoint{-0.037123in}{0.962877in}}{\pgfqpoint{-0.046968in}{0.972722in}}{\pgfqpoint{-0.052500in}{0.986077in}}%
\pgfpathcurveto{\pgfqpoint{-0.052500in}{1.000000in}}{\pgfqpoint{-0.052500in}{1.013923in}}{\pgfqpoint{-0.046968in}{1.027278in}}%
\pgfpathcurveto{\pgfqpoint{-0.037123in}{1.037123in}}{\pgfqpoint{-0.027278in}{1.046968in}}{\pgfqpoint{-0.013923in}{1.052500in}}%
\pgfpathcurveto{\pgfqpoint{0.000000in}{1.052500in}}{\pgfqpoint{0.013923in}{1.052500in}}{\pgfqpoint{0.027278in}{1.046968in}}%
\pgfpathcurveto{\pgfqpoint{0.037123in}{1.037123in}}{\pgfqpoint{0.046968in}{1.027278in}}{\pgfqpoint{0.052500in}{1.013923in}}%
\pgfpathcurveto{\pgfqpoint{0.052500in}{1.000000in}}{\pgfqpoint{0.052500in}{0.986077in}}{\pgfqpoint{0.046968in}{0.972722in}}%
\pgfpathcurveto{\pgfqpoint{0.037123in}{0.962877in}}{\pgfqpoint{0.027278in}{0.953032in}}{\pgfqpoint{0.013923in}{0.947500in}}%
\pgfpathclose%
\pgfpathmoveto{\pgfqpoint{0.166667in}{0.941667in}}%
\pgfpathcurveto{\pgfqpoint{0.182137in}{0.941667in}}{\pgfqpoint{0.196975in}{0.947813in}}{\pgfqpoint{0.207915in}{0.958752in}}%
\pgfpathcurveto{\pgfqpoint{0.218854in}{0.969691in}}{\pgfqpoint{0.225000in}{0.984530in}}{\pgfqpoint{0.225000in}{1.000000in}}%
\pgfpathcurveto{\pgfqpoint{0.225000in}{1.015470in}}{\pgfqpoint{0.218854in}{1.030309in}}{\pgfqpoint{0.207915in}{1.041248in}}%
\pgfpathcurveto{\pgfqpoint{0.196975in}{1.052187in}}{\pgfqpoint{0.182137in}{1.058333in}}{\pgfqpoint{0.166667in}{1.058333in}}%
\pgfpathcurveto{\pgfqpoint{0.151196in}{1.058333in}}{\pgfqpoint{0.136358in}{1.052187in}}{\pgfqpoint{0.125419in}{1.041248in}}%
\pgfpathcurveto{\pgfqpoint{0.114480in}{1.030309in}}{\pgfqpoint{0.108333in}{1.015470in}}{\pgfqpoint{0.108333in}{1.000000in}}%
\pgfpathcurveto{\pgfqpoint{0.108333in}{0.984530in}}{\pgfqpoint{0.114480in}{0.969691in}}{\pgfqpoint{0.125419in}{0.958752in}}%
\pgfpathcurveto{\pgfqpoint{0.136358in}{0.947813in}}{\pgfqpoint{0.151196in}{0.941667in}}{\pgfqpoint{0.166667in}{0.941667in}}%
\pgfpathclose%
\pgfpathmoveto{\pgfqpoint{0.166667in}{0.947500in}}%
\pgfpathcurveto{\pgfqpoint{0.166667in}{0.947500in}}{\pgfqpoint{0.152744in}{0.947500in}}{\pgfqpoint{0.139389in}{0.953032in}}%
\pgfpathcurveto{\pgfqpoint{0.129544in}{0.962877in}}{\pgfqpoint{0.119698in}{0.972722in}}{\pgfqpoint{0.114167in}{0.986077in}}%
\pgfpathcurveto{\pgfqpoint{0.114167in}{1.000000in}}{\pgfqpoint{0.114167in}{1.013923in}}{\pgfqpoint{0.119698in}{1.027278in}}%
\pgfpathcurveto{\pgfqpoint{0.129544in}{1.037123in}}{\pgfqpoint{0.139389in}{1.046968in}}{\pgfqpoint{0.152744in}{1.052500in}}%
\pgfpathcurveto{\pgfqpoint{0.166667in}{1.052500in}}{\pgfqpoint{0.180590in}{1.052500in}}{\pgfqpoint{0.193945in}{1.046968in}}%
\pgfpathcurveto{\pgfqpoint{0.203790in}{1.037123in}}{\pgfqpoint{0.213635in}{1.027278in}}{\pgfqpoint{0.219167in}{1.013923in}}%
\pgfpathcurveto{\pgfqpoint{0.219167in}{1.000000in}}{\pgfqpoint{0.219167in}{0.986077in}}{\pgfqpoint{0.213635in}{0.972722in}}%
\pgfpathcurveto{\pgfqpoint{0.203790in}{0.962877in}}{\pgfqpoint{0.193945in}{0.953032in}}{\pgfqpoint{0.180590in}{0.947500in}}%
\pgfpathclose%
\pgfpathmoveto{\pgfqpoint{0.333333in}{0.941667in}}%
\pgfpathcurveto{\pgfqpoint{0.348804in}{0.941667in}}{\pgfqpoint{0.363642in}{0.947813in}}{\pgfqpoint{0.374581in}{0.958752in}}%
\pgfpathcurveto{\pgfqpoint{0.385520in}{0.969691in}}{\pgfqpoint{0.391667in}{0.984530in}}{\pgfqpoint{0.391667in}{1.000000in}}%
\pgfpathcurveto{\pgfqpoint{0.391667in}{1.015470in}}{\pgfqpoint{0.385520in}{1.030309in}}{\pgfqpoint{0.374581in}{1.041248in}}%
\pgfpathcurveto{\pgfqpoint{0.363642in}{1.052187in}}{\pgfqpoint{0.348804in}{1.058333in}}{\pgfqpoint{0.333333in}{1.058333in}}%
\pgfpathcurveto{\pgfqpoint{0.317863in}{1.058333in}}{\pgfqpoint{0.303025in}{1.052187in}}{\pgfqpoint{0.292085in}{1.041248in}}%
\pgfpathcurveto{\pgfqpoint{0.281146in}{1.030309in}}{\pgfqpoint{0.275000in}{1.015470in}}{\pgfqpoint{0.275000in}{1.000000in}}%
\pgfpathcurveto{\pgfqpoint{0.275000in}{0.984530in}}{\pgfqpoint{0.281146in}{0.969691in}}{\pgfqpoint{0.292085in}{0.958752in}}%
\pgfpathcurveto{\pgfqpoint{0.303025in}{0.947813in}}{\pgfqpoint{0.317863in}{0.941667in}}{\pgfqpoint{0.333333in}{0.941667in}}%
\pgfpathclose%
\pgfpathmoveto{\pgfqpoint{0.333333in}{0.947500in}}%
\pgfpathcurveto{\pgfqpoint{0.333333in}{0.947500in}}{\pgfqpoint{0.319410in}{0.947500in}}{\pgfqpoint{0.306055in}{0.953032in}}%
\pgfpathcurveto{\pgfqpoint{0.296210in}{0.962877in}}{\pgfqpoint{0.286365in}{0.972722in}}{\pgfqpoint{0.280833in}{0.986077in}}%
\pgfpathcurveto{\pgfqpoint{0.280833in}{1.000000in}}{\pgfqpoint{0.280833in}{1.013923in}}{\pgfqpoint{0.286365in}{1.027278in}}%
\pgfpathcurveto{\pgfqpoint{0.296210in}{1.037123in}}{\pgfqpoint{0.306055in}{1.046968in}}{\pgfqpoint{0.319410in}{1.052500in}}%
\pgfpathcurveto{\pgfqpoint{0.333333in}{1.052500in}}{\pgfqpoint{0.347256in}{1.052500in}}{\pgfqpoint{0.360611in}{1.046968in}}%
\pgfpathcurveto{\pgfqpoint{0.370456in}{1.037123in}}{\pgfqpoint{0.380302in}{1.027278in}}{\pgfqpoint{0.385833in}{1.013923in}}%
\pgfpathcurveto{\pgfqpoint{0.385833in}{1.000000in}}{\pgfqpoint{0.385833in}{0.986077in}}{\pgfqpoint{0.380302in}{0.972722in}}%
\pgfpathcurveto{\pgfqpoint{0.370456in}{0.962877in}}{\pgfqpoint{0.360611in}{0.953032in}}{\pgfqpoint{0.347256in}{0.947500in}}%
\pgfpathclose%
\pgfpathmoveto{\pgfqpoint{0.500000in}{0.941667in}}%
\pgfpathcurveto{\pgfqpoint{0.515470in}{0.941667in}}{\pgfqpoint{0.530309in}{0.947813in}}{\pgfqpoint{0.541248in}{0.958752in}}%
\pgfpathcurveto{\pgfqpoint{0.552187in}{0.969691in}}{\pgfqpoint{0.558333in}{0.984530in}}{\pgfqpoint{0.558333in}{1.000000in}}%
\pgfpathcurveto{\pgfqpoint{0.558333in}{1.015470in}}{\pgfqpoint{0.552187in}{1.030309in}}{\pgfqpoint{0.541248in}{1.041248in}}%
\pgfpathcurveto{\pgfqpoint{0.530309in}{1.052187in}}{\pgfqpoint{0.515470in}{1.058333in}}{\pgfqpoint{0.500000in}{1.058333in}}%
\pgfpathcurveto{\pgfqpoint{0.484530in}{1.058333in}}{\pgfqpoint{0.469691in}{1.052187in}}{\pgfqpoint{0.458752in}{1.041248in}}%
\pgfpathcurveto{\pgfqpoint{0.447813in}{1.030309in}}{\pgfqpoint{0.441667in}{1.015470in}}{\pgfqpoint{0.441667in}{1.000000in}}%
\pgfpathcurveto{\pgfqpoint{0.441667in}{0.984530in}}{\pgfqpoint{0.447813in}{0.969691in}}{\pgfqpoint{0.458752in}{0.958752in}}%
\pgfpathcurveto{\pgfqpoint{0.469691in}{0.947813in}}{\pgfqpoint{0.484530in}{0.941667in}}{\pgfqpoint{0.500000in}{0.941667in}}%
\pgfpathclose%
\pgfpathmoveto{\pgfqpoint{0.500000in}{0.947500in}}%
\pgfpathcurveto{\pgfqpoint{0.500000in}{0.947500in}}{\pgfqpoint{0.486077in}{0.947500in}}{\pgfqpoint{0.472722in}{0.953032in}}%
\pgfpathcurveto{\pgfqpoint{0.462877in}{0.962877in}}{\pgfqpoint{0.453032in}{0.972722in}}{\pgfqpoint{0.447500in}{0.986077in}}%
\pgfpathcurveto{\pgfqpoint{0.447500in}{1.000000in}}{\pgfqpoint{0.447500in}{1.013923in}}{\pgfqpoint{0.453032in}{1.027278in}}%
\pgfpathcurveto{\pgfqpoint{0.462877in}{1.037123in}}{\pgfqpoint{0.472722in}{1.046968in}}{\pgfqpoint{0.486077in}{1.052500in}}%
\pgfpathcurveto{\pgfqpoint{0.500000in}{1.052500in}}{\pgfqpoint{0.513923in}{1.052500in}}{\pgfqpoint{0.527278in}{1.046968in}}%
\pgfpathcurveto{\pgfqpoint{0.537123in}{1.037123in}}{\pgfqpoint{0.546968in}{1.027278in}}{\pgfqpoint{0.552500in}{1.013923in}}%
\pgfpathcurveto{\pgfqpoint{0.552500in}{1.000000in}}{\pgfqpoint{0.552500in}{0.986077in}}{\pgfqpoint{0.546968in}{0.972722in}}%
\pgfpathcurveto{\pgfqpoint{0.537123in}{0.962877in}}{\pgfqpoint{0.527278in}{0.953032in}}{\pgfqpoint{0.513923in}{0.947500in}}%
\pgfpathclose%
\pgfpathmoveto{\pgfqpoint{0.666667in}{0.941667in}}%
\pgfpathcurveto{\pgfqpoint{0.682137in}{0.941667in}}{\pgfqpoint{0.696975in}{0.947813in}}{\pgfqpoint{0.707915in}{0.958752in}}%
\pgfpathcurveto{\pgfqpoint{0.718854in}{0.969691in}}{\pgfqpoint{0.725000in}{0.984530in}}{\pgfqpoint{0.725000in}{1.000000in}}%
\pgfpathcurveto{\pgfqpoint{0.725000in}{1.015470in}}{\pgfqpoint{0.718854in}{1.030309in}}{\pgfqpoint{0.707915in}{1.041248in}}%
\pgfpathcurveto{\pgfqpoint{0.696975in}{1.052187in}}{\pgfqpoint{0.682137in}{1.058333in}}{\pgfqpoint{0.666667in}{1.058333in}}%
\pgfpathcurveto{\pgfqpoint{0.651196in}{1.058333in}}{\pgfqpoint{0.636358in}{1.052187in}}{\pgfqpoint{0.625419in}{1.041248in}}%
\pgfpathcurveto{\pgfqpoint{0.614480in}{1.030309in}}{\pgfqpoint{0.608333in}{1.015470in}}{\pgfqpoint{0.608333in}{1.000000in}}%
\pgfpathcurveto{\pgfqpoint{0.608333in}{0.984530in}}{\pgfqpoint{0.614480in}{0.969691in}}{\pgfqpoint{0.625419in}{0.958752in}}%
\pgfpathcurveto{\pgfqpoint{0.636358in}{0.947813in}}{\pgfqpoint{0.651196in}{0.941667in}}{\pgfqpoint{0.666667in}{0.941667in}}%
\pgfpathclose%
\pgfpathmoveto{\pgfqpoint{0.666667in}{0.947500in}}%
\pgfpathcurveto{\pgfqpoint{0.666667in}{0.947500in}}{\pgfqpoint{0.652744in}{0.947500in}}{\pgfqpoint{0.639389in}{0.953032in}}%
\pgfpathcurveto{\pgfqpoint{0.629544in}{0.962877in}}{\pgfqpoint{0.619698in}{0.972722in}}{\pgfqpoint{0.614167in}{0.986077in}}%
\pgfpathcurveto{\pgfqpoint{0.614167in}{1.000000in}}{\pgfqpoint{0.614167in}{1.013923in}}{\pgfqpoint{0.619698in}{1.027278in}}%
\pgfpathcurveto{\pgfqpoint{0.629544in}{1.037123in}}{\pgfqpoint{0.639389in}{1.046968in}}{\pgfqpoint{0.652744in}{1.052500in}}%
\pgfpathcurveto{\pgfqpoint{0.666667in}{1.052500in}}{\pgfqpoint{0.680590in}{1.052500in}}{\pgfqpoint{0.693945in}{1.046968in}}%
\pgfpathcurveto{\pgfqpoint{0.703790in}{1.037123in}}{\pgfqpoint{0.713635in}{1.027278in}}{\pgfqpoint{0.719167in}{1.013923in}}%
\pgfpathcurveto{\pgfqpoint{0.719167in}{1.000000in}}{\pgfqpoint{0.719167in}{0.986077in}}{\pgfqpoint{0.713635in}{0.972722in}}%
\pgfpathcurveto{\pgfqpoint{0.703790in}{0.962877in}}{\pgfqpoint{0.693945in}{0.953032in}}{\pgfqpoint{0.680590in}{0.947500in}}%
\pgfpathclose%
\pgfpathmoveto{\pgfqpoint{0.833333in}{0.941667in}}%
\pgfpathcurveto{\pgfqpoint{0.848804in}{0.941667in}}{\pgfqpoint{0.863642in}{0.947813in}}{\pgfqpoint{0.874581in}{0.958752in}}%
\pgfpathcurveto{\pgfqpoint{0.885520in}{0.969691in}}{\pgfqpoint{0.891667in}{0.984530in}}{\pgfqpoint{0.891667in}{1.000000in}}%
\pgfpathcurveto{\pgfqpoint{0.891667in}{1.015470in}}{\pgfqpoint{0.885520in}{1.030309in}}{\pgfqpoint{0.874581in}{1.041248in}}%
\pgfpathcurveto{\pgfqpoint{0.863642in}{1.052187in}}{\pgfqpoint{0.848804in}{1.058333in}}{\pgfqpoint{0.833333in}{1.058333in}}%
\pgfpathcurveto{\pgfqpoint{0.817863in}{1.058333in}}{\pgfqpoint{0.803025in}{1.052187in}}{\pgfqpoint{0.792085in}{1.041248in}}%
\pgfpathcurveto{\pgfqpoint{0.781146in}{1.030309in}}{\pgfqpoint{0.775000in}{1.015470in}}{\pgfqpoint{0.775000in}{1.000000in}}%
\pgfpathcurveto{\pgfqpoint{0.775000in}{0.984530in}}{\pgfqpoint{0.781146in}{0.969691in}}{\pgfqpoint{0.792085in}{0.958752in}}%
\pgfpathcurveto{\pgfqpoint{0.803025in}{0.947813in}}{\pgfqpoint{0.817863in}{0.941667in}}{\pgfqpoint{0.833333in}{0.941667in}}%
\pgfpathclose%
\pgfpathmoveto{\pgfqpoint{0.833333in}{0.947500in}}%
\pgfpathcurveto{\pgfqpoint{0.833333in}{0.947500in}}{\pgfqpoint{0.819410in}{0.947500in}}{\pgfqpoint{0.806055in}{0.953032in}}%
\pgfpathcurveto{\pgfqpoint{0.796210in}{0.962877in}}{\pgfqpoint{0.786365in}{0.972722in}}{\pgfqpoint{0.780833in}{0.986077in}}%
\pgfpathcurveto{\pgfqpoint{0.780833in}{1.000000in}}{\pgfqpoint{0.780833in}{1.013923in}}{\pgfqpoint{0.786365in}{1.027278in}}%
\pgfpathcurveto{\pgfqpoint{0.796210in}{1.037123in}}{\pgfqpoint{0.806055in}{1.046968in}}{\pgfqpoint{0.819410in}{1.052500in}}%
\pgfpathcurveto{\pgfqpoint{0.833333in}{1.052500in}}{\pgfqpoint{0.847256in}{1.052500in}}{\pgfqpoint{0.860611in}{1.046968in}}%
\pgfpathcurveto{\pgfqpoint{0.870456in}{1.037123in}}{\pgfqpoint{0.880302in}{1.027278in}}{\pgfqpoint{0.885833in}{1.013923in}}%
\pgfpathcurveto{\pgfqpoint{0.885833in}{1.000000in}}{\pgfqpoint{0.885833in}{0.986077in}}{\pgfqpoint{0.880302in}{0.972722in}}%
\pgfpathcurveto{\pgfqpoint{0.870456in}{0.962877in}}{\pgfqpoint{0.860611in}{0.953032in}}{\pgfqpoint{0.847256in}{0.947500in}}%
\pgfpathclose%
\pgfpathmoveto{\pgfqpoint{1.000000in}{0.941667in}}%
\pgfpathcurveto{\pgfqpoint{1.015470in}{0.941667in}}{\pgfqpoint{1.030309in}{0.947813in}}{\pgfqpoint{1.041248in}{0.958752in}}%
\pgfpathcurveto{\pgfqpoint{1.052187in}{0.969691in}}{\pgfqpoint{1.058333in}{0.984530in}}{\pgfqpoint{1.058333in}{1.000000in}}%
\pgfpathcurveto{\pgfqpoint{1.058333in}{1.015470in}}{\pgfqpoint{1.052187in}{1.030309in}}{\pgfqpoint{1.041248in}{1.041248in}}%
\pgfpathcurveto{\pgfqpoint{1.030309in}{1.052187in}}{\pgfqpoint{1.015470in}{1.058333in}}{\pgfqpoint{1.000000in}{1.058333in}}%
\pgfpathcurveto{\pgfqpoint{0.984530in}{1.058333in}}{\pgfqpoint{0.969691in}{1.052187in}}{\pgfqpoint{0.958752in}{1.041248in}}%
\pgfpathcurveto{\pgfqpoint{0.947813in}{1.030309in}}{\pgfqpoint{0.941667in}{1.015470in}}{\pgfqpoint{0.941667in}{1.000000in}}%
\pgfpathcurveto{\pgfqpoint{0.941667in}{0.984530in}}{\pgfqpoint{0.947813in}{0.969691in}}{\pgfqpoint{0.958752in}{0.958752in}}%
\pgfpathcurveto{\pgfqpoint{0.969691in}{0.947813in}}{\pgfqpoint{0.984530in}{0.941667in}}{\pgfqpoint{1.000000in}{0.941667in}}%
\pgfpathclose%
\pgfpathmoveto{\pgfqpoint{1.000000in}{0.947500in}}%
\pgfpathcurveto{\pgfqpoint{1.000000in}{0.947500in}}{\pgfqpoint{0.986077in}{0.947500in}}{\pgfqpoint{0.972722in}{0.953032in}}%
\pgfpathcurveto{\pgfqpoint{0.962877in}{0.962877in}}{\pgfqpoint{0.953032in}{0.972722in}}{\pgfqpoint{0.947500in}{0.986077in}}%
\pgfpathcurveto{\pgfqpoint{0.947500in}{1.000000in}}{\pgfqpoint{0.947500in}{1.013923in}}{\pgfqpoint{0.953032in}{1.027278in}}%
\pgfpathcurveto{\pgfqpoint{0.962877in}{1.037123in}}{\pgfqpoint{0.972722in}{1.046968in}}{\pgfqpoint{0.986077in}{1.052500in}}%
\pgfpathcurveto{\pgfqpoint{1.000000in}{1.052500in}}{\pgfqpoint{1.013923in}{1.052500in}}{\pgfqpoint{1.027278in}{1.046968in}}%
\pgfpathcurveto{\pgfqpoint{1.037123in}{1.037123in}}{\pgfqpoint{1.046968in}{1.027278in}}{\pgfqpoint{1.052500in}{1.013923in}}%
\pgfpathcurveto{\pgfqpoint{1.052500in}{1.000000in}}{\pgfqpoint{1.052500in}{0.986077in}}{\pgfqpoint{1.046968in}{0.972722in}}%
\pgfpathcurveto{\pgfqpoint{1.037123in}{0.962877in}}{\pgfqpoint{1.027278in}{0.953032in}}{\pgfqpoint{1.013923in}{0.947500in}}%
\pgfpathclose%
\pgfusepath{stroke}%
\end{pgfscope}%
}%
\pgfsys@transformshift{1.135815in}{9.006369in}%
\pgfsys@useobject{currentpattern}{}%
\pgfsys@transformshift{1in}{0in}%
\pgfsys@transformshift{-1in}{0in}%
\pgfsys@transformshift{0in}{1in}%
\end{pgfscope}%
\begin{pgfscope}%
\definecolor{textcolor}{rgb}{0.000000,0.000000,0.000000}%
\pgfsetstrokecolor{textcolor}%
\pgfsetfillcolor{textcolor}%
\pgftext[x=1.758037in,y=9.006369in,left,base]{\color{textcolor}\rmfamily\fontsize{16.000000}{19.200000}\selectfont IMP\_ELC}%
\end{pgfscope}%
\begin{pgfscope}%
\pgfsetbuttcap%
\pgfsetmiterjoin%
\definecolor{currentfill}{rgb}{0.411765,0.411765,0.411765}%
\pgfsetfillcolor{currentfill}%
\pgfsetfillopacity{0.990000}%
\pgfsetlinewidth{0.000000pt}%
\definecolor{currentstroke}{rgb}{0.000000,0.000000,0.000000}%
\pgfsetstrokecolor{currentstroke}%
\pgfsetstrokeopacity{0.990000}%
\pgfsetdash{}{0pt}%
\pgfpathmoveto{\pgfqpoint{1.135815in}{8.681909in}}%
\pgfpathlineto{\pgfqpoint{1.580259in}{8.681909in}}%
\pgfpathlineto{\pgfqpoint{1.580259in}{8.837465in}}%
\pgfpathlineto{\pgfqpoint{1.135815in}{8.837465in}}%
\pgfpathclose%
\pgfusepath{fill}%
\end{pgfscope}%
\begin{pgfscope}%
\pgfsetbuttcap%
\pgfsetmiterjoin%
\definecolor{currentfill}{rgb}{0.411765,0.411765,0.411765}%
\pgfsetfillcolor{currentfill}%
\pgfsetfillopacity{0.990000}%
\pgfsetlinewidth{0.000000pt}%
\definecolor{currentstroke}{rgb}{0.000000,0.000000,0.000000}%
\pgfsetstrokecolor{currentstroke}%
\pgfsetstrokeopacity{0.990000}%
\pgfsetdash{}{0pt}%
\pgfpathmoveto{\pgfqpoint{1.135815in}{8.681909in}}%
\pgfpathlineto{\pgfqpoint{1.580259in}{8.681909in}}%
\pgfpathlineto{\pgfqpoint{1.580259in}{8.837465in}}%
\pgfpathlineto{\pgfqpoint{1.135815in}{8.837465in}}%
\pgfpathclose%
\pgfusepath{clip}%
\pgfsys@defobject{currentpattern}{\pgfqpoint{0in}{0in}}{\pgfqpoint{1in}{1in}}{%
\begin{pgfscope}%
\pgfpathrectangle{\pgfqpoint{0in}{0in}}{\pgfqpoint{1in}{1in}}%
\pgfusepath{clip}%
\pgfpathmoveto{\pgfqpoint{-0.500000in}{0.500000in}}%
\pgfpathlineto{\pgfqpoint{0.500000in}{1.500000in}}%
\pgfpathmoveto{\pgfqpoint{-0.333333in}{0.333333in}}%
\pgfpathlineto{\pgfqpoint{0.666667in}{1.333333in}}%
\pgfpathmoveto{\pgfqpoint{-0.166667in}{0.166667in}}%
\pgfpathlineto{\pgfqpoint{0.833333in}{1.166667in}}%
\pgfpathmoveto{\pgfqpoint{0.000000in}{0.000000in}}%
\pgfpathlineto{\pgfqpoint{1.000000in}{1.000000in}}%
\pgfpathmoveto{\pgfqpoint{0.166667in}{-0.166667in}}%
\pgfpathlineto{\pgfqpoint{1.166667in}{0.833333in}}%
\pgfpathmoveto{\pgfqpoint{0.333333in}{-0.333333in}}%
\pgfpathlineto{\pgfqpoint{1.333333in}{0.666667in}}%
\pgfpathmoveto{\pgfqpoint{0.500000in}{-0.500000in}}%
\pgfpathlineto{\pgfqpoint{1.500000in}{0.500000in}}%
\pgfusepath{stroke}%
\end{pgfscope}%
}%
\pgfsys@transformshift{1.135815in}{8.681909in}%
\pgfsys@useobject{currentpattern}{}%
\pgfsys@transformshift{1in}{0in}%
\pgfsys@transformshift{-1in}{0in}%
\pgfsys@transformshift{0in}{1in}%
\end{pgfscope}%
\begin{pgfscope}%
\definecolor{textcolor}{rgb}{0.000000,0.000000,0.000000}%
\pgfsetstrokecolor{textcolor}%
\pgfsetfillcolor{textcolor}%
\pgftext[x=1.758037in,y=8.681909in,left,base]{\color{textcolor}\rmfamily\fontsize{16.000000}{19.200000}\selectfont LI\_BATTERY}%
\end{pgfscope}%
\begin{pgfscope}%
\pgfsetbuttcap%
\pgfsetmiterjoin%
\definecolor{currentfill}{rgb}{0.172549,0.627451,0.172549}%
\pgfsetfillcolor{currentfill}%
\pgfsetfillopacity{0.990000}%
\pgfsetlinewidth{0.000000pt}%
\definecolor{currentstroke}{rgb}{0.000000,0.000000,0.000000}%
\pgfsetstrokecolor{currentstroke}%
\pgfsetstrokeopacity{0.990000}%
\pgfsetdash{}{0pt}%
\pgfpathmoveto{\pgfqpoint{1.135815in}{8.357449in}}%
\pgfpathlineto{\pgfqpoint{1.580259in}{8.357449in}}%
\pgfpathlineto{\pgfqpoint{1.580259in}{8.513005in}}%
\pgfpathlineto{\pgfqpoint{1.135815in}{8.513005in}}%
\pgfpathclose%
\pgfusepath{fill}%
\end{pgfscope}%
\begin{pgfscope}%
\pgfsetbuttcap%
\pgfsetmiterjoin%
\definecolor{currentfill}{rgb}{0.172549,0.627451,0.172549}%
\pgfsetfillcolor{currentfill}%
\pgfsetfillopacity{0.990000}%
\pgfsetlinewidth{0.000000pt}%
\definecolor{currentstroke}{rgb}{0.000000,0.000000,0.000000}%
\pgfsetstrokecolor{currentstroke}%
\pgfsetstrokeopacity{0.990000}%
\pgfsetdash{}{0pt}%
\pgfpathmoveto{\pgfqpoint{1.135815in}{8.357449in}}%
\pgfpathlineto{\pgfqpoint{1.580259in}{8.357449in}}%
\pgfpathlineto{\pgfqpoint{1.580259in}{8.513005in}}%
\pgfpathlineto{\pgfqpoint{1.135815in}{8.513005in}}%
\pgfpathclose%
\pgfusepath{clip}%
\pgfsys@defobject{currentpattern}{\pgfqpoint{0in}{0in}}{\pgfqpoint{1in}{1in}}{%
\begin{pgfscope}%
\pgfpathrectangle{\pgfqpoint{0in}{0in}}{\pgfqpoint{1in}{1in}}%
\pgfusepath{clip}%
\pgfpathmoveto{\pgfqpoint{0.000000in}{-0.016667in}}%
\pgfpathcurveto{\pgfqpoint{0.004420in}{-0.016667in}}{\pgfqpoint{0.008660in}{-0.014911in}}{\pgfqpoint{0.011785in}{-0.011785in}}%
\pgfpathcurveto{\pgfqpoint{0.014911in}{-0.008660in}}{\pgfqpoint{0.016667in}{-0.004420in}}{\pgfqpoint{0.016667in}{0.000000in}}%
\pgfpathcurveto{\pgfqpoint{0.016667in}{0.004420in}}{\pgfqpoint{0.014911in}{0.008660in}}{\pgfqpoint{0.011785in}{0.011785in}}%
\pgfpathcurveto{\pgfqpoint{0.008660in}{0.014911in}}{\pgfqpoint{0.004420in}{0.016667in}}{\pgfqpoint{0.000000in}{0.016667in}}%
\pgfpathcurveto{\pgfqpoint{-0.004420in}{0.016667in}}{\pgfqpoint{-0.008660in}{0.014911in}}{\pgfqpoint{-0.011785in}{0.011785in}}%
\pgfpathcurveto{\pgfqpoint{-0.014911in}{0.008660in}}{\pgfqpoint{-0.016667in}{0.004420in}}{\pgfqpoint{-0.016667in}{0.000000in}}%
\pgfpathcurveto{\pgfqpoint{-0.016667in}{-0.004420in}}{\pgfqpoint{-0.014911in}{-0.008660in}}{\pgfqpoint{-0.011785in}{-0.011785in}}%
\pgfpathcurveto{\pgfqpoint{-0.008660in}{-0.014911in}}{\pgfqpoint{-0.004420in}{-0.016667in}}{\pgfqpoint{0.000000in}{-0.016667in}}%
\pgfpathclose%
\pgfpathmoveto{\pgfqpoint{0.166667in}{-0.016667in}}%
\pgfpathcurveto{\pgfqpoint{0.171087in}{-0.016667in}}{\pgfqpoint{0.175326in}{-0.014911in}}{\pgfqpoint{0.178452in}{-0.011785in}}%
\pgfpathcurveto{\pgfqpoint{0.181577in}{-0.008660in}}{\pgfqpoint{0.183333in}{-0.004420in}}{\pgfqpoint{0.183333in}{0.000000in}}%
\pgfpathcurveto{\pgfqpoint{0.183333in}{0.004420in}}{\pgfqpoint{0.181577in}{0.008660in}}{\pgfqpoint{0.178452in}{0.011785in}}%
\pgfpathcurveto{\pgfqpoint{0.175326in}{0.014911in}}{\pgfqpoint{0.171087in}{0.016667in}}{\pgfqpoint{0.166667in}{0.016667in}}%
\pgfpathcurveto{\pgfqpoint{0.162247in}{0.016667in}}{\pgfqpoint{0.158007in}{0.014911in}}{\pgfqpoint{0.154882in}{0.011785in}}%
\pgfpathcurveto{\pgfqpoint{0.151756in}{0.008660in}}{\pgfqpoint{0.150000in}{0.004420in}}{\pgfqpoint{0.150000in}{0.000000in}}%
\pgfpathcurveto{\pgfqpoint{0.150000in}{-0.004420in}}{\pgfqpoint{0.151756in}{-0.008660in}}{\pgfqpoint{0.154882in}{-0.011785in}}%
\pgfpathcurveto{\pgfqpoint{0.158007in}{-0.014911in}}{\pgfqpoint{0.162247in}{-0.016667in}}{\pgfqpoint{0.166667in}{-0.016667in}}%
\pgfpathclose%
\pgfpathmoveto{\pgfqpoint{0.333333in}{-0.016667in}}%
\pgfpathcurveto{\pgfqpoint{0.337753in}{-0.016667in}}{\pgfqpoint{0.341993in}{-0.014911in}}{\pgfqpoint{0.345118in}{-0.011785in}}%
\pgfpathcurveto{\pgfqpoint{0.348244in}{-0.008660in}}{\pgfqpoint{0.350000in}{-0.004420in}}{\pgfqpoint{0.350000in}{0.000000in}}%
\pgfpathcurveto{\pgfqpoint{0.350000in}{0.004420in}}{\pgfqpoint{0.348244in}{0.008660in}}{\pgfqpoint{0.345118in}{0.011785in}}%
\pgfpathcurveto{\pgfqpoint{0.341993in}{0.014911in}}{\pgfqpoint{0.337753in}{0.016667in}}{\pgfqpoint{0.333333in}{0.016667in}}%
\pgfpathcurveto{\pgfqpoint{0.328913in}{0.016667in}}{\pgfqpoint{0.324674in}{0.014911in}}{\pgfqpoint{0.321548in}{0.011785in}}%
\pgfpathcurveto{\pgfqpoint{0.318423in}{0.008660in}}{\pgfqpoint{0.316667in}{0.004420in}}{\pgfqpoint{0.316667in}{0.000000in}}%
\pgfpathcurveto{\pgfqpoint{0.316667in}{-0.004420in}}{\pgfqpoint{0.318423in}{-0.008660in}}{\pgfqpoint{0.321548in}{-0.011785in}}%
\pgfpathcurveto{\pgfqpoint{0.324674in}{-0.014911in}}{\pgfqpoint{0.328913in}{-0.016667in}}{\pgfqpoint{0.333333in}{-0.016667in}}%
\pgfpathclose%
\pgfpathmoveto{\pgfqpoint{0.500000in}{-0.016667in}}%
\pgfpathcurveto{\pgfqpoint{0.504420in}{-0.016667in}}{\pgfqpoint{0.508660in}{-0.014911in}}{\pgfqpoint{0.511785in}{-0.011785in}}%
\pgfpathcurveto{\pgfqpoint{0.514911in}{-0.008660in}}{\pgfqpoint{0.516667in}{-0.004420in}}{\pgfqpoint{0.516667in}{0.000000in}}%
\pgfpathcurveto{\pgfqpoint{0.516667in}{0.004420in}}{\pgfqpoint{0.514911in}{0.008660in}}{\pgfqpoint{0.511785in}{0.011785in}}%
\pgfpathcurveto{\pgfqpoint{0.508660in}{0.014911in}}{\pgfqpoint{0.504420in}{0.016667in}}{\pgfqpoint{0.500000in}{0.016667in}}%
\pgfpathcurveto{\pgfqpoint{0.495580in}{0.016667in}}{\pgfqpoint{0.491340in}{0.014911in}}{\pgfqpoint{0.488215in}{0.011785in}}%
\pgfpathcurveto{\pgfqpoint{0.485089in}{0.008660in}}{\pgfqpoint{0.483333in}{0.004420in}}{\pgfqpoint{0.483333in}{0.000000in}}%
\pgfpathcurveto{\pgfqpoint{0.483333in}{-0.004420in}}{\pgfqpoint{0.485089in}{-0.008660in}}{\pgfqpoint{0.488215in}{-0.011785in}}%
\pgfpathcurveto{\pgfqpoint{0.491340in}{-0.014911in}}{\pgfqpoint{0.495580in}{-0.016667in}}{\pgfqpoint{0.500000in}{-0.016667in}}%
\pgfpathclose%
\pgfpathmoveto{\pgfqpoint{0.666667in}{-0.016667in}}%
\pgfpathcurveto{\pgfqpoint{0.671087in}{-0.016667in}}{\pgfqpoint{0.675326in}{-0.014911in}}{\pgfqpoint{0.678452in}{-0.011785in}}%
\pgfpathcurveto{\pgfqpoint{0.681577in}{-0.008660in}}{\pgfqpoint{0.683333in}{-0.004420in}}{\pgfqpoint{0.683333in}{0.000000in}}%
\pgfpathcurveto{\pgfqpoint{0.683333in}{0.004420in}}{\pgfqpoint{0.681577in}{0.008660in}}{\pgfqpoint{0.678452in}{0.011785in}}%
\pgfpathcurveto{\pgfqpoint{0.675326in}{0.014911in}}{\pgfqpoint{0.671087in}{0.016667in}}{\pgfqpoint{0.666667in}{0.016667in}}%
\pgfpathcurveto{\pgfqpoint{0.662247in}{0.016667in}}{\pgfqpoint{0.658007in}{0.014911in}}{\pgfqpoint{0.654882in}{0.011785in}}%
\pgfpathcurveto{\pgfqpoint{0.651756in}{0.008660in}}{\pgfqpoint{0.650000in}{0.004420in}}{\pgfqpoint{0.650000in}{0.000000in}}%
\pgfpathcurveto{\pgfqpoint{0.650000in}{-0.004420in}}{\pgfqpoint{0.651756in}{-0.008660in}}{\pgfqpoint{0.654882in}{-0.011785in}}%
\pgfpathcurveto{\pgfqpoint{0.658007in}{-0.014911in}}{\pgfqpoint{0.662247in}{-0.016667in}}{\pgfqpoint{0.666667in}{-0.016667in}}%
\pgfpathclose%
\pgfpathmoveto{\pgfqpoint{0.833333in}{-0.016667in}}%
\pgfpathcurveto{\pgfqpoint{0.837753in}{-0.016667in}}{\pgfqpoint{0.841993in}{-0.014911in}}{\pgfqpoint{0.845118in}{-0.011785in}}%
\pgfpathcurveto{\pgfqpoint{0.848244in}{-0.008660in}}{\pgfqpoint{0.850000in}{-0.004420in}}{\pgfqpoint{0.850000in}{0.000000in}}%
\pgfpathcurveto{\pgfqpoint{0.850000in}{0.004420in}}{\pgfqpoint{0.848244in}{0.008660in}}{\pgfqpoint{0.845118in}{0.011785in}}%
\pgfpathcurveto{\pgfqpoint{0.841993in}{0.014911in}}{\pgfqpoint{0.837753in}{0.016667in}}{\pgfqpoint{0.833333in}{0.016667in}}%
\pgfpathcurveto{\pgfqpoint{0.828913in}{0.016667in}}{\pgfqpoint{0.824674in}{0.014911in}}{\pgfqpoint{0.821548in}{0.011785in}}%
\pgfpathcurveto{\pgfqpoint{0.818423in}{0.008660in}}{\pgfqpoint{0.816667in}{0.004420in}}{\pgfqpoint{0.816667in}{0.000000in}}%
\pgfpathcurveto{\pgfqpoint{0.816667in}{-0.004420in}}{\pgfqpoint{0.818423in}{-0.008660in}}{\pgfqpoint{0.821548in}{-0.011785in}}%
\pgfpathcurveto{\pgfqpoint{0.824674in}{-0.014911in}}{\pgfqpoint{0.828913in}{-0.016667in}}{\pgfqpoint{0.833333in}{-0.016667in}}%
\pgfpathclose%
\pgfpathmoveto{\pgfqpoint{1.000000in}{-0.016667in}}%
\pgfpathcurveto{\pgfqpoint{1.004420in}{-0.016667in}}{\pgfqpoint{1.008660in}{-0.014911in}}{\pgfqpoint{1.011785in}{-0.011785in}}%
\pgfpathcurveto{\pgfqpoint{1.014911in}{-0.008660in}}{\pgfqpoint{1.016667in}{-0.004420in}}{\pgfqpoint{1.016667in}{0.000000in}}%
\pgfpathcurveto{\pgfqpoint{1.016667in}{0.004420in}}{\pgfqpoint{1.014911in}{0.008660in}}{\pgfqpoint{1.011785in}{0.011785in}}%
\pgfpathcurveto{\pgfqpoint{1.008660in}{0.014911in}}{\pgfqpoint{1.004420in}{0.016667in}}{\pgfqpoint{1.000000in}{0.016667in}}%
\pgfpathcurveto{\pgfqpoint{0.995580in}{0.016667in}}{\pgfqpoint{0.991340in}{0.014911in}}{\pgfqpoint{0.988215in}{0.011785in}}%
\pgfpathcurveto{\pgfqpoint{0.985089in}{0.008660in}}{\pgfqpoint{0.983333in}{0.004420in}}{\pgfqpoint{0.983333in}{0.000000in}}%
\pgfpathcurveto{\pgfqpoint{0.983333in}{-0.004420in}}{\pgfqpoint{0.985089in}{-0.008660in}}{\pgfqpoint{0.988215in}{-0.011785in}}%
\pgfpathcurveto{\pgfqpoint{0.991340in}{-0.014911in}}{\pgfqpoint{0.995580in}{-0.016667in}}{\pgfqpoint{1.000000in}{-0.016667in}}%
\pgfpathclose%
\pgfpathmoveto{\pgfqpoint{0.083333in}{0.150000in}}%
\pgfpathcurveto{\pgfqpoint{0.087753in}{0.150000in}}{\pgfqpoint{0.091993in}{0.151756in}}{\pgfqpoint{0.095118in}{0.154882in}}%
\pgfpathcurveto{\pgfqpoint{0.098244in}{0.158007in}}{\pgfqpoint{0.100000in}{0.162247in}}{\pgfqpoint{0.100000in}{0.166667in}}%
\pgfpathcurveto{\pgfqpoint{0.100000in}{0.171087in}}{\pgfqpoint{0.098244in}{0.175326in}}{\pgfqpoint{0.095118in}{0.178452in}}%
\pgfpathcurveto{\pgfqpoint{0.091993in}{0.181577in}}{\pgfqpoint{0.087753in}{0.183333in}}{\pgfqpoint{0.083333in}{0.183333in}}%
\pgfpathcurveto{\pgfqpoint{0.078913in}{0.183333in}}{\pgfqpoint{0.074674in}{0.181577in}}{\pgfqpoint{0.071548in}{0.178452in}}%
\pgfpathcurveto{\pgfqpoint{0.068423in}{0.175326in}}{\pgfqpoint{0.066667in}{0.171087in}}{\pgfqpoint{0.066667in}{0.166667in}}%
\pgfpathcurveto{\pgfqpoint{0.066667in}{0.162247in}}{\pgfqpoint{0.068423in}{0.158007in}}{\pgfqpoint{0.071548in}{0.154882in}}%
\pgfpathcurveto{\pgfqpoint{0.074674in}{0.151756in}}{\pgfqpoint{0.078913in}{0.150000in}}{\pgfqpoint{0.083333in}{0.150000in}}%
\pgfpathclose%
\pgfpathmoveto{\pgfqpoint{0.250000in}{0.150000in}}%
\pgfpathcurveto{\pgfqpoint{0.254420in}{0.150000in}}{\pgfqpoint{0.258660in}{0.151756in}}{\pgfqpoint{0.261785in}{0.154882in}}%
\pgfpathcurveto{\pgfqpoint{0.264911in}{0.158007in}}{\pgfqpoint{0.266667in}{0.162247in}}{\pgfqpoint{0.266667in}{0.166667in}}%
\pgfpathcurveto{\pgfqpoint{0.266667in}{0.171087in}}{\pgfqpoint{0.264911in}{0.175326in}}{\pgfqpoint{0.261785in}{0.178452in}}%
\pgfpathcurveto{\pgfqpoint{0.258660in}{0.181577in}}{\pgfqpoint{0.254420in}{0.183333in}}{\pgfqpoint{0.250000in}{0.183333in}}%
\pgfpathcurveto{\pgfqpoint{0.245580in}{0.183333in}}{\pgfqpoint{0.241340in}{0.181577in}}{\pgfqpoint{0.238215in}{0.178452in}}%
\pgfpathcurveto{\pgfqpoint{0.235089in}{0.175326in}}{\pgfqpoint{0.233333in}{0.171087in}}{\pgfqpoint{0.233333in}{0.166667in}}%
\pgfpathcurveto{\pgfqpoint{0.233333in}{0.162247in}}{\pgfqpoint{0.235089in}{0.158007in}}{\pgfqpoint{0.238215in}{0.154882in}}%
\pgfpathcurveto{\pgfqpoint{0.241340in}{0.151756in}}{\pgfqpoint{0.245580in}{0.150000in}}{\pgfqpoint{0.250000in}{0.150000in}}%
\pgfpathclose%
\pgfpathmoveto{\pgfqpoint{0.416667in}{0.150000in}}%
\pgfpathcurveto{\pgfqpoint{0.421087in}{0.150000in}}{\pgfqpoint{0.425326in}{0.151756in}}{\pgfqpoint{0.428452in}{0.154882in}}%
\pgfpathcurveto{\pgfqpoint{0.431577in}{0.158007in}}{\pgfqpoint{0.433333in}{0.162247in}}{\pgfqpoint{0.433333in}{0.166667in}}%
\pgfpathcurveto{\pgfqpoint{0.433333in}{0.171087in}}{\pgfqpoint{0.431577in}{0.175326in}}{\pgfqpoint{0.428452in}{0.178452in}}%
\pgfpathcurveto{\pgfqpoint{0.425326in}{0.181577in}}{\pgfqpoint{0.421087in}{0.183333in}}{\pgfqpoint{0.416667in}{0.183333in}}%
\pgfpathcurveto{\pgfqpoint{0.412247in}{0.183333in}}{\pgfqpoint{0.408007in}{0.181577in}}{\pgfqpoint{0.404882in}{0.178452in}}%
\pgfpathcurveto{\pgfqpoint{0.401756in}{0.175326in}}{\pgfqpoint{0.400000in}{0.171087in}}{\pgfqpoint{0.400000in}{0.166667in}}%
\pgfpathcurveto{\pgfqpoint{0.400000in}{0.162247in}}{\pgfqpoint{0.401756in}{0.158007in}}{\pgfqpoint{0.404882in}{0.154882in}}%
\pgfpathcurveto{\pgfqpoint{0.408007in}{0.151756in}}{\pgfqpoint{0.412247in}{0.150000in}}{\pgfqpoint{0.416667in}{0.150000in}}%
\pgfpathclose%
\pgfpathmoveto{\pgfqpoint{0.583333in}{0.150000in}}%
\pgfpathcurveto{\pgfqpoint{0.587753in}{0.150000in}}{\pgfqpoint{0.591993in}{0.151756in}}{\pgfqpoint{0.595118in}{0.154882in}}%
\pgfpathcurveto{\pgfqpoint{0.598244in}{0.158007in}}{\pgfqpoint{0.600000in}{0.162247in}}{\pgfqpoint{0.600000in}{0.166667in}}%
\pgfpathcurveto{\pgfqpoint{0.600000in}{0.171087in}}{\pgfqpoint{0.598244in}{0.175326in}}{\pgfqpoint{0.595118in}{0.178452in}}%
\pgfpathcurveto{\pgfqpoint{0.591993in}{0.181577in}}{\pgfqpoint{0.587753in}{0.183333in}}{\pgfqpoint{0.583333in}{0.183333in}}%
\pgfpathcurveto{\pgfqpoint{0.578913in}{0.183333in}}{\pgfqpoint{0.574674in}{0.181577in}}{\pgfqpoint{0.571548in}{0.178452in}}%
\pgfpathcurveto{\pgfqpoint{0.568423in}{0.175326in}}{\pgfqpoint{0.566667in}{0.171087in}}{\pgfqpoint{0.566667in}{0.166667in}}%
\pgfpathcurveto{\pgfqpoint{0.566667in}{0.162247in}}{\pgfqpoint{0.568423in}{0.158007in}}{\pgfqpoint{0.571548in}{0.154882in}}%
\pgfpathcurveto{\pgfqpoint{0.574674in}{0.151756in}}{\pgfqpoint{0.578913in}{0.150000in}}{\pgfqpoint{0.583333in}{0.150000in}}%
\pgfpathclose%
\pgfpathmoveto{\pgfqpoint{0.750000in}{0.150000in}}%
\pgfpathcurveto{\pgfqpoint{0.754420in}{0.150000in}}{\pgfqpoint{0.758660in}{0.151756in}}{\pgfqpoint{0.761785in}{0.154882in}}%
\pgfpathcurveto{\pgfqpoint{0.764911in}{0.158007in}}{\pgfqpoint{0.766667in}{0.162247in}}{\pgfqpoint{0.766667in}{0.166667in}}%
\pgfpathcurveto{\pgfqpoint{0.766667in}{0.171087in}}{\pgfqpoint{0.764911in}{0.175326in}}{\pgfqpoint{0.761785in}{0.178452in}}%
\pgfpathcurveto{\pgfqpoint{0.758660in}{0.181577in}}{\pgfqpoint{0.754420in}{0.183333in}}{\pgfqpoint{0.750000in}{0.183333in}}%
\pgfpathcurveto{\pgfqpoint{0.745580in}{0.183333in}}{\pgfqpoint{0.741340in}{0.181577in}}{\pgfqpoint{0.738215in}{0.178452in}}%
\pgfpathcurveto{\pgfqpoint{0.735089in}{0.175326in}}{\pgfqpoint{0.733333in}{0.171087in}}{\pgfqpoint{0.733333in}{0.166667in}}%
\pgfpathcurveto{\pgfqpoint{0.733333in}{0.162247in}}{\pgfqpoint{0.735089in}{0.158007in}}{\pgfqpoint{0.738215in}{0.154882in}}%
\pgfpathcurveto{\pgfqpoint{0.741340in}{0.151756in}}{\pgfqpoint{0.745580in}{0.150000in}}{\pgfqpoint{0.750000in}{0.150000in}}%
\pgfpathclose%
\pgfpathmoveto{\pgfqpoint{0.916667in}{0.150000in}}%
\pgfpathcurveto{\pgfqpoint{0.921087in}{0.150000in}}{\pgfqpoint{0.925326in}{0.151756in}}{\pgfqpoint{0.928452in}{0.154882in}}%
\pgfpathcurveto{\pgfqpoint{0.931577in}{0.158007in}}{\pgfqpoint{0.933333in}{0.162247in}}{\pgfqpoint{0.933333in}{0.166667in}}%
\pgfpathcurveto{\pgfqpoint{0.933333in}{0.171087in}}{\pgfqpoint{0.931577in}{0.175326in}}{\pgfqpoint{0.928452in}{0.178452in}}%
\pgfpathcurveto{\pgfqpoint{0.925326in}{0.181577in}}{\pgfqpoint{0.921087in}{0.183333in}}{\pgfqpoint{0.916667in}{0.183333in}}%
\pgfpathcurveto{\pgfqpoint{0.912247in}{0.183333in}}{\pgfqpoint{0.908007in}{0.181577in}}{\pgfqpoint{0.904882in}{0.178452in}}%
\pgfpathcurveto{\pgfqpoint{0.901756in}{0.175326in}}{\pgfqpoint{0.900000in}{0.171087in}}{\pgfqpoint{0.900000in}{0.166667in}}%
\pgfpathcurveto{\pgfqpoint{0.900000in}{0.162247in}}{\pgfqpoint{0.901756in}{0.158007in}}{\pgfqpoint{0.904882in}{0.154882in}}%
\pgfpathcurveto{\pgfqpoint{0.908007in}{0.151756in}}{\pgfqpoint{0.912247in}{0.150000in}}{\pgfqpoint{0.916667in}{0.150000in}}%
\pgfpathclose%
\pgfpathmoveto{\pgfqpoint{0.000000in}{0.316667in}}%
\pgfpathcurveto{\pgfqpoint{0.004420in}{0.316667in}}{\pgfqpoint{0.008660in}{0.318423in}}{\pgfqpoint{0.011785in}{0.321548in}}%
\pgfpathcurveto{\pgfqpoint{0.014911in}{0.324674in}}{\pgfqpoint{0.016667in}{0.328913in}}{\pgfqpoint{0.016667in}{0.333333in}}%
\pgfpathcurveto{\pgfqpoint{0.016667in}{0.337753in}}{\pgfqpoint{0.014911in}{0.341993in}}{\pgfqpoint{0.011785in}{0.345118in}}%
\pgfpathcurveto{\pgfqpoint{0.008660in}{0.348244in}}{\pgfqpoint{0.004420in}{0.350000in}}{\pgfqpoint{0.000000in}{0.350000in}}%
\pgfpathcurveto{\pgfqpoint{-0.004420in}{0.350000in}}{\pgfqpoint{-0.008660in}{0.348244in}}{\pgfqpoint{-0.011785in}{0.345118in}}%
\pgfpathcurveto{\pgfqpoint{-0.014911in}{0.341993in}}{\pgfqpoint{-0.016667in}{0.337753in}}{\pgfqpoint{-0.016667in}{0.333333in}}%
\pgfpathcurveto{\pgfqpoint{-0.016667in}{0.328913in}}{\pgfqpoint{-0.014911in}{0.324674in}}{\pgfqpoint{-0.011785in}{0.321548in}}%
\pgfpathcurveto{\pgfqpoint{-0.008660in}{0.318423in}}{\pgfqpoint{-0.004420in}{0.316667in}}{\pgfqpoint{0.000000in}{0.316667in}}%
\pgfpathclose%
\pgfpathmoveto{\pgfqpoint{0.166667in}{0.316667in}}%
\pgfpathcurveto{\pgfqpoint{0.171087in}{0.316667in}}{\pgfqpoint{0.175326in}{0.318423in}}{\pgfqpoint{0.178452in}{0.321548in}}%
\pgfpathcurveto{\pgfqpoint{0.181577in}{0.324674in}}{\pgfqpoint{0.183333in}{0.328913in}}{\pgfqpoint{0.183333in}{0.333333in}}%
\pgfpathcurveto{\pgfqpoint{0.183333in}{0.337753in}}{\pgfqpoint{0.181577in}{0.341993in}}{\pgfqpoint{0.178452in}{0.345118in}}%
\pgfpathcurveto{\pgfqpoint{0.175326in}{0.348244in}}{\pgfqpoint{0.171087in}{0.350000in}}{\pgfqpoint{0.166667in}{0.350000in}}%
\pgfpathcurveto{\pgfqpoint{0.162247in}{0.350000in}}{\pgfqpoint{0.158007in}{0.348244in}}{\pgfqpoint{0.154882in}{0.345118in}}%
\pgfpathcurveto{\pgfqpoint{0.151756in}{0.341993in}}{\pgfqpoint{0.150000in}{0.337753in}}{\pgfqpoint{0.150000in}{0.333333in}}%
\pgfpathcurveto{\pgfqpoint{0.150000in}{0.328913in}}{\pgfqpoint{0.151756in}{0.324674in}}{\pgfqpoint{0.154882in}{0.321548in}}%
\pgfpathcurveto{\pgfqpoint{0.158007in}{0.318423in}}{\pgfqpoint{0.162247in}{0.316667in}}{\pgfqpoint{0.166667in}{0.316667in}}%
\pgfpathclose%
\pgfpathmoveto{\pgfqpoint{0.333333in}{0.316667in}}%
\pgfpathcurveto{\pgfqpoint{0.337753in}{0.316667in}}{\pgfqpoint{0.341993in}{0.318423in}}{\pgfqpoint{0.345118in}{0.321548in}}%
\pgfpathcurveto{\pgfqpoint{0.348244in}{0.324674in}}{\pgfqpoint{0.350000in}{0.328913in}}{\pgfqpoint{0.350000in}{0.333333in}}%
\pgfpathcurveto{\pgfqpoint{0.350000in}{0.337753in}}{\pgfqpoint{0.348244in}{0.341993in}}{\pgfqpoint{0.345118in}{0.345118in}}%
\pgfpathcurveto{\pgfqpoint{0.341993in}{0.348244in}}{\pgfqpoint{0.337753in}{0.350000in}}{\pgfqpoint{0.333333in}{0.350000in}}%
\pgfpathcurveto{\pgfqpoint{0.328913in}{0.350000in}}{\pgfqpoint{0.324674in}{0.348244in}}{\pgfqpoint{0.321548in}{0.345118in}}%
\pgfpathcurveto{\pgfqpoint{0.318423in}{0.341993in}}{\pgfqpoint{0.316667in}{0.337753in}}{\pgfqpoint{0.316667in}{0.333333in}}%
\pgfpathcurveto{\pgfqpoint{0.316667in}{0.328913in}}{\pgfqpoint{0.318423in}{0.324674in}}{\pgfqpoint{0.321548in}{0.321548in}}%
\pgfpathcurveto{\pgfqpoint{0.324674in}{0.318423in}}{\pgfqpoint{0.328913in}{0.316667in}}{\pgfqpoint{0.333333in}{0.316667in}}%
\pgfpathclose%
\pgfpathmoveto{\pgfqpoint{0.500000in}{0.316667in}}%
\pgfpathcurveto{\pgfqpoint{0.504420in}{0.316667in}}{\pgfqpoint{0.508660in}{0.318423in}}{\pgfqpoint{0.511785in}{0.321548in}}%
\pgfpathcurveto{\pgfqpoint{0.514911in}{0.324674in}}{\pgfqpoint{0.516667in}{0.328913in}}{\pgfqpoint{0.516667in}{0.333333in}}%
\pgfpathcurveto{\pgfqpoint{0.516667in}{0.337753in}}{\pgfqpoint{0.514911in}{0.341993in}}{\pgfqpoint{0.511785in}{0.345118in}}%
\pgfpathcurveto{\pgfqpoint{0.508660in}{0.348244in}}{\pgfqpoint{0.504420in}{0.350000in}}{\pgfqpoint{0.500000in}{0.350000in}}%
\pgfpathcurveto{\pgfqpoint{0.495580in}{0.350000in}}{\pgfqpoint{0.491340in}{0.348244in}}{\pgfqpoint{0.488215in}{0.345118in}}%
\pgfpathcurveto{\pgfqpoint{0.485089in}{0.341993in}}{\pgfqpoint{0.483333in}{0.337753in}}{\pgfqpoint{0.483333in}{0.333333in}}%
\pgfpathcurveto{\pgfqpoint{0.483333in}{0.328913in}}{\pgfqpoint{0.485089in}{0.324674in}}{\pgfqpoint{0.488215in}{0.321548in}}%
\pgfpathcurveto{\pgfqpoint{0.491340in}{0.318423in}}{\pgfqpoint{0.495580in}{0.316667in}}{\pgfqpoint{0.500000in}{0.316667in}}%
\pgfpathclose%
\pgfpathmoveto{\pgfqpoint{0.666667in}{0.316667in}}%
\pgfpathcurveto{\pgfqpoint{0.671087in}{0.316667in}}{\pgfqpoint{0.675326in}{0.318423in}}{\pgfqpoint{0.678452in}{0.321548in}}%
\pgfpathcurveto{\pgfqpoint{0.681577in}{0.324674in}}{\pgfqpoint{0.683333in}{0.328913in}}{\pgfqpoint{0.683333in}{0.333333in}}%
\pgfpathcurveto{\pgfqpoint{0.683333in}{0.337753in}}{\pgfqpoint{0.681577in}{0.341993in}}{\pgfqpoint{0.678452in}{0.345118in}}%
\pgfpathcurveto{\pgfqpoint{0.675326in}{0.348244in}}{\pgfqpoint{0.671087in}{0.350000in}}{\pgfqpoint{0.666667in}{0.350000in}}%
\pgfpathcurveto{\pgfqpoint{0.662247in}{0.350000in}}{\pgfqpoint{0.658007in}{0.348244in}}{\pgfqpoint{0.654882in}{0.345118in}}%
\pgfpathcurveto{\pgfqpoint{0.651756in}{0.341993in}}{\pgfqpoint{0.650000in}{0.337753in}}{\pgfqpoint{0.650000in}{0.333333in}}%
\pgfpathcurveto{\pgfqpoint{0.650000in}{0.328913in}}{\pgfqpoint{0.651756in}{0.324674in}}{\pgfqpoint{0.654882in}{0.321548in}}%
\pgfpathcurveto{\pgfqpoint{0.658007in}{0.318423in}}{\pgfqpoint{0.662247in}{0.316667in}}{\pgfqpoint{0.666667in}{0.316667in}}%
\pgfpathclose%
\pgfpathmoveto{\pgfqpoint{0.833333in}{0.316667in}}%
\pgfpathcurveto{\pgfqpoint{0.837753in}{0.316667in}}{\pgfqpoint{0.841993in}{0.318423in}}{\pgfqpoint{0.845118in}{0.321548in}}%
\pgfpathcurveto{\pgfqpoint{0.848244in}{0.324674in}}{\pgfqpoint{0.850000in}{0.328913in}}{\pgfqpoint{0.850000in}{0.333333in}}%
\pgfpathcurveto{\pgfqpoint{0.850000in}{0.337753in}}{\pgfqpoint{0.848244in}{0.341993in}}{\pgfqpoint{0.845118in}{0.345118in}}%
\pgfpathcurveto{\pgfqpoint{0.841993in}{0.348244in}}{\pgfqpoint{0.837753in}{0.350000in}}{\pgfqpoint{0.833333in}{0.350000in}}%
\pgfpathcurveto{\pgfqpoint{0.828913in}{0.350000in}}{\pgfqpoint{0.824674in}{0.348244in}}{\pgfqpoint{0.821548in}{0.345118in}}%
\pgfpathcurveto{\pgfqpoint{0.818423in}{0.341993in}}{\pgfqpoint{0.816667in}{0.337753in}}{\pgfqpoint{0.816667in}{0.333333in}}%
\pgfpathcurveto{\pgfqpoint{0.816667in}{0.328913in}}{\pgfqpoint{0.818423in}{0.324674in}}{\pgfqpoint{0.821548in}{0.321548in}}%
\pgfpathcurveto{\pgfqpoint{0.824674in}{0.318423in}}{\pgfqpoint{0.828913in}{0.316667in}}{\pgfqpoint{0.833333in}{0.316667in}}%
\pgfpathclose%
\pgfpathmoveto{\pgfqpoint{1.000000in}{0.316667in}}%
\pgfpathcurveto{\pgfqpoint{1.004420in}{0.316667in}}{\pgfqpoint{1.008660in}{0.318423in}}{\pgfqpoint{1.011785in}{0.321548in}}%
\pgfpathcurveto{\pgfqpoint{1.014911in}{0.324674in}}{\pgfqpoint{1.016667in}{0.328913in}}{\pgfqpoint{1.016667in}{0.333333in}}%
\pgfpathcurveto{\pgfqpoint{1.016667in}{0.337753in}}{\pgfqpoint{1.014911in}{0.341993in}}{\pgfqpoint{1.011785in}{0.345118in}}%
\pgfpathcurveto{\pgfqpoint{1.008660in}{0.348244in}}{\pgfqpoint{1.004420in}{0.350000in}}{\pgfqpoint{1.000000in}{0.350000in}}%
\pgfpathcurveto{\pgfqpoint{0.995580in}{0.350000in}}{\pgfqpoint{0.991340in}{0.348244in}}{\pgfqpoint{0.988215in}{0.345118in}}%
\pgfpathcurveto{\pgfqpoint{0.985089in}{0.341993in}}{\pgfqpoint{0.983333in}{0.337753in}}{\pgfqpoint{0.983333in}{0.333333in}}%
\pgfpathcurveto{\pgfqpoint{0.983333in}{0.328913in}}{\pgfqpoint{0.985089in}{0.324674in}}{\pgfqpoint{0.988215in}{0.321548in}}%
\pgfpathcurveto{\pgfqpoint{0.991340in}{0.318423in}}{\pgfqpoint{0.995580in}{0.316667in}}{\pgfqpoint{1.000000in}{0.316667in}}%
\pgfpathclose%
\pgfpathmoveto{\pgfqpoint{0.083333in}{0.483333in}}%
\pgfpathcurveto{\pgfqpoint{0.087753in}{0.483333in}}{\pgfqpoint{0.091993in}{0.485089in}}{\pgfqpoint{0.095118in}{0.488215in}}%
\pgfpathcurveto{\pgfqpoint{0.098244in}{0.491340in}}{\pgfqpoint{0.100000in}{0.495580in}}{\pgfqpoint{0.100000in}{0.500000in}}%
\pgfpathcurveto{\pgfqpoint{0.100000in}{0.504420in}}{\pgfqpoint{0.098244in}{0.508660in}}{\pgfqpoint{0.095118in}{0.511785in}}%
\pgfpathcurveto{\pgfqpoint{0.091993in}{0.514911in}}{\pgfqpoint{0.087753in}{0.516667in}}{\pgfqpoint{0.083333in}{0.516667in}}%
\pgfpathcurveto{\pgfqpoint{0.078913in}{0.516667in}}{\pgfqpoint{0.074674in}{0.514911in}}{\pgfqpoint{0.071548in}{0.511785in}}%
\pgfpathcurveto{\pgfqpoint{0.068423in}{0.508660in}}{\pgfqpoint{0.066667in}{0.504420in}}{\pgfqpoint{0.066667in}{0.500000in}}%
\pgfpathcurveto{\pgfqpoint{0.066667in}{0.495580in}}{\pgfqpoint{0.068423in}{0.491340in}}{\pgfqpoint{0.071548in}{0.488215in}}%
\pgfpathcurveto{\pgfqpoint{0.074674in}{0.485089in}}{\pgfqpoint{0.078913in}{0.483333in}}{\pgfqpoint{0.083333in}{0.483333in}}%
\pgfpathclose%
\pgfpathmoveto{\pgfqpoint{0.250000in}{0.483333in}}%
\pgfpathcurveto{\pgfqpoint{0.254420in}{0.483333in}}{\pgfqpoint{0.258660in}{0.485089in}}{\pgfqpoint{0.261785in}{0.488215in}}%
\pgfpathcurveto{\pgfqpoint{0.264911in}{0.491340in}}{\pgfqpoint{0.266667in}{0.495580in}}{\pgfqpoint{0.266667in}{0.500000in}}%
\pgfpathcurveto{\pgfqpoint{0.266667in}{0.504420in}}{\pgfqpoint{0.264911in}{0.508660in}}{\pgfqpoint{0.261785in}{0.511785in}}%
\pgfpathcurveto{\pgfqpoint{0.258660in}{0.514911in}}{\pgfqpoint{0.254420in}{0.516667in}}{\pgfqpoint{0.250000in}{0.516667in}}%
\pgfpathcurveto{\pgfqpoint{0.245580in}{0.516667in}}{\pgfqpoint{0.241340in}{0.514911in}}{\pgfqpoint{0.238215in}{0.511785in}}%
\pgfpathcurveto{\pgfqpoint{0.235089in}{0.508660in}}{\pgfqpoint{0.233333in}{0.504420in}}{\pgfqpoint{0.233333in}{0.500000in}}%
\pgfpathcurveto{\pgfqpoint{0.233333in}{0.495580in}}{\pgfqpoint{0.235089in}{0.491340in}}{\pgfqpoint{0.238215in}{0.488215in}}%
\pgfpathcurveto{\pgfqpoint{0.241340in}{0.485089in}}{\pgfqpoint{0.245580in}{0.483333in}}{\pgfqpoint{0.250000in}{0.483333in}}%
\pgfpathclose%
\pgfpathmoveto{\pgfqpoint{0.416667in}{0.483333in}}%
\pgfpathcurveto{\pgfqpoint{0.421087in}{0.483333in}}{\pgfqpoint{0.425326in}{0.485089in}}{\pgfqpoint{0.428452in}{0.488215in}}%
\pgfpathcurveto{\pgfqpoint{0.431577in}{0.491340in}}{\pgfqpoint{0.433333in}{0.495580in}}{\pgfqpoint{0.433333in}{0.500000in}}%
\pgfpathcurveto{\pgfqpoint{0.433333in}{0.504420in}}{\pgfqpoint{0.431577in}{0.508660in}}{\pgfqpoint{0.428452in}{0.511785in}}%
\pgfpathcurveto{\pgfqpoint{0.425326in}{0.514911in}}{\pgfqpoint{0.421087in}{0.516667in}}{\pgfqpoint{0.416667in}{0.516667in}}%
\pgfpathcurveto{\pgfqpoint{0.412247in}{0.516667in}}{\pgfqpoint{0.408007in}{0.514911in}}{\pgfqpoint{0.404882in}{0.511785in}}%
\pgfpathcurveto{\pgfqpoint{0.401756in}{0.508660in}}{\pgfqpoint{0.400000in}{0.504420in}}{\pgfqpoint{0.400000in}{0.500000in}}%
\pgfpathcurveto{\pgfqpoint{0.400000in}{0.495580in}}{\pgfqpoint{0.401756in}{0.491340in}}{\pgfqpoint{0.404882in}{0.488215in}}%
\pgfpathcurveto{\pgfqpoint{0.408007in}{0.485089in}}{\pgfqpoint{0.412247in}{0.483333in}}{\pgfqpoint{0.416667in}{0.483333in}}%
\pgfpathclose%
\pgfpathmoveto{\pgfqpoint{0.583333in}{0.483333in}}%
\pgfpathcurveto{\pgfqpoint{0.587753in}{0.483333in}}{\pgfqpoint{0.591993in}{0.485089in}}{\pgfqpoint{0.595118in}{0.488215in}}%
\pgfpathcurveto{\pgfqpoint{0.598244in}{0.491340in}}{\pgfqpoint{0.600000in}{0.495580in}}{\pgfqpoint{0.600000in}{0.500000in}}%
\pgfpathcurveto{\pgfqpoint{0.600000in}{0.504420in}}{\pgfqpoint{0.598244in}{0.508660in}}{\pgfqpoint{0.595118in}{0.511785in}}%
\pgfpathcurveto{\pgfqpoint{0.591993in}{0.514911in}}{\pgfqpoint{0.587753in}{0.516667in}}{\pgfqpoint{0.583333in}{0.516667in}}%
\pgfpathcurveto{\pgfqpoint{0.578913in}{0.516667in}}{\pgfqpoint{0.574674in}{0.514911in}}{\pgfqpoint{0.571548in}{0.511785in}}%
\pgfpathcurveto{\pgfqpoint{0.568423in}{0.508660in}}{\pgfqpoint{0.566667in}{0.504420in}}{\pgfqpoint{0.566667in}{0.500000in}}%
\pgfpathcurveto{\pgfqpoint{0.566667in}{0.495580in}}{\pgfqpoint{0.568423in}{0.491340in}}{\pgfqpoint{0.571548in}{0.488215in}}%
\pgfpathcurveto{\pgfqpoint{0.574674in}{0.485089in}}{\pgfqpoint{0.578913in}{0.483333in}}{\pgfqpoint{0.583333in}{0.483333in}}%
\pgfpathclose%
\pgfpathmoveto{\pgfqpoint{0.750000in}{0.483333in}}%
\pgfpathcurveto{\pgfqpoint{0.754420in}{0.483333in}}{\pgfqpoint{0.758660in}{0.485089in}}{\pgfqpoint{0.761785in}{0.488215in}}%
\pgfpathcurveto{\pgfqpoint{0.764911in}{0.491340in}}{\pgfqpoint{0.766667in}{0.495580in}}{\pgfqpoint{0.766667in}{0.500000in}}%
\pgfpathcurveto{\pgfqpoint{0.766667in}{0.504420in}}{\pgfqpoint{0.764911in}{0.508660in}}{\pgfqpoint{0.761785in}{0.511785in}}%
\pgfpathcurveto{\pgfqpoint{0.758660in}{0.514911in}}{\pgfqpoint{0.754420in}{0.516667in}}{\pgfqpoint{0.750000in}{0.516667in}}%
\pgfpathcurveto{\pgfqpoint{0.745580in}{0.516667in}}{\pgfqpoint{0.741340in}{0.514911in}}{\pgfqpoint{0.738215in}{0.511785in}}%
\pgfpathcurveto{\pgfqpoint{0.735089in}{0.508660in}}{\pgfqpoint{0.733333in}{0.504420in}}{\pgfqpoint{0.733333in}{0.500000in}}%
\pgfpathcurveto{\pgfqpoint{0.733333in}{0.495580in}}{\pgfqpoint{0.735089in}{0.491340in}}{\pgfqpoint{0.738215in}{0.488215in}}%
\pgfpathcurveto{\pgfqpoint{0.741340in}{0.485089in}}{\pgfqpoint{0.745580in}{0.483333in}}{\pgfqpoint{0.750000in}{0.483333in}}%
\pgfpathclose%
\pgfpathmoveto{\pgfqpoint{0.916667in}{0.483333in}}%
\pgfpathcurveto{\pgfqpoint{0.921087in}{0.483333in}}{\pgfqpoint{0.925326in}{0.485089in}}{\pgfqpoint{0.928452in}{0.488215in}}%
\pgfpathcurveto{\pgfqpoint{0.931577in}{0.491340in}}{\pgfqpoint{0.933333in}{0.495580in}}{\pgfqpoint{0.933333in}{0.500000in}}%
\pgfpathcurveto{\pgfqpoint{0.933333in}{0.504420in}}{\pgfqpoint{0.931577in}{0.508660in}}{\pgfqpoint{0.928452in}{0.511785in}}%
\pgfpathcurveto{\pgfqpoint{0.925326in}{0.514911in}}{\pgfqpoint{0.921087in}{0.516667in}}{\pgfqpoint{0.916667in}{0.516667in}}%
\pgfpathcurveto{\pgfqpoint{0.912247in}{0.516667in}}{\pgfqpoint{0.908007in}{0.514911in}}{\pgfqpoint{0.904882in}{0.511785in}}%
\pgfpathcurveto{\pgfqpoint{0.901756in}{0.508660in}}{\pgfqpoint{0.900000in}{0.504420in}}{\pgfqpoint{0.900000in}{0.500000in}}%
\pgfpathcurveto{\pgfqpoint{0.900000in}{0.495580in}}{\pgfqpoint{0.901756in}{0.491340in}}{\pgfqpoint{0.904882in}{0.488215in}}%
\pgfpathcurveto{\pgfqpoint{0.908007in}{0.485089in}}{\pgfqpoint{0.912247in}{0.483333in}}{\pgfqpoint{0.916667in}{0.483333in}}%
\pgfpathclose%
\pgfpathmoveto{\pgfqpoint{0.000000in}{0.650000in}}%
\pgfpathcurveto{\pgfqpoint{0.004420in}{0.650000in}}{\pgfqpoint{0.008660in}{0.651756in}}{\pgfqpoint{0.011785in}{0.654882in}}%
\pgfpathcurveto{\pgfqpoint{0.014911in}{0.658007in}}{\pgfqpoint{0.016667in}{0.662247in}}{\pgfqpoint{0.016667in}{0.666667in}}%
\pgfpathcurveto{\pgfqpoint{0.016667in}{0.671087in}}{\pgfqpoint{0.014911in}{0.675326in}}{\pgfqpoint{0.011785in}{0.678452in}}%
\pgfpathcurveto{\pgfqpoint{0.008660in}{0.681577in}}{\pgfqpoint{0.004420in}{0.683333in}}{\pgfqpoint{0.000000in}{0.683333in}}%
\pgfpathcurveto{\pgfqpoint{-0.004420in}{0.683333in}}{\pgfqpoint{-0.008660in}{0.681577in}}{\pgfqpoint{-0.011785in}{0.678452in}}%
\pgfpathcurveto{\pgfqpoint{-0.014911in}{0.675326in}}{\pgfqpoint{-0.016667in}{0.671087in}}{\pgfqpoint{-0.016667in}{0.666667in}}%
\pgfpathcurveto{\pgfqpoint{-0.016667in}{0.662247in}}{\pgfqpoint{-0.014911in}{0.658007in}}{\pgfqpoint{-0.011785in}{0.654882in}}%
\pgfpathcurveto{\pgfqpoint{-0.008660in}{0.651756in}}{\pgfqpoint{-0.004420in}{0.650000in}}{\pgfqpoint{0.000000in}{0.650000in}}%
\pgfpathclose%
\pgfpathmoveto{\pgfqpoint{0.166667in}{0.650000in}}%
\pgfpathcurveto{\pgfqpoint{0.171087in}{0.650000in}}{\pgfqpoint{0.175326in}{0.651756in}}{\pgfqpoint{0.178452in}{0.654882in}}%
\pgfpathcurveto{\pgfqpoint{0.181577in}{0.658007in}}{\pgfqpoint{0.183333in}{0.662247in}}{\pgfqpoint{0.183333in}{0.666667in}}%
\pgfpathcurveto{\pgfqpoint{0.183333in}{0.671087in}}{\pgfqpoint{0.181577in}{0.675326in}}{\pgfqpoint{0.178452in}{0.678452in}}%
\pgfpathcurveto{\pgfqpoint{0.175326in}{0.681577in}}{\pgfqpoint{0.171087in}{0.683333in}}{\pgfqpoint{0.166667in}{0.683333in}}%
\pgfpathcurveto{\pgfqpoint{0.162247in}{0.683333in}}{\pgfqpoint{0.158007in}{0.681577in}}{\pgfqpoint{0.154882in}{0.678452in}}%
\pgfpathcurveto{\pgfqpoint{0.151756in}{0.675326in}}{\pgfqpoint{0.150000in}{0.671087in}}{\pgfqpoint{0.150000in}{0.666667in}}%
\pgfpathcurveto{\pgfqpoint{0.150000in}{0.662247in}}{\pgfqpoint{0.151756in}{0.658007in}}{\pgfqpoint{0.154882in}{0.654882in}}%
\pgfpathcurveto{\pgfqpoint{0.158007in}{0.651756in}}{\pgfqpoint{0.162247in}{0.650000in}}{\pgfqpoint{0.166667in}{0.650000in}}%
\pgfpathclose%
\pgfpathmoveto{\pgfqpoint{0.333333in}{0.650000in}}%
\pgfpathcurveto{\pgfqpoint{0.337753in}{0.650000in}}{\pgfqpoint{0.341993in}{0.651756in}}{\pgfqpoint{0.345118in}{0.654882in}}%
\pgfpathcurveto{\pgfqpoint{0.348244in}{0.658007in}}{\pgfqpoint{0.350000in}{0.662247in}}{\pgfqpoint{0.350000in}{0.666667in}}%
\pgfpathcurveto{\pgfqpoint{0.350000in}{0.671087in}}{\pgfqpoint{0.348244in}{0.675326in}}{\pgfqpoint{0.345118in}{0.678452in}}%
\pgfpathcurveto{\pgfqpoint{0.341993in}{0.681577in}}{\pgfqpoint{0.337753in}{0.683333in}}{\pgfqpoint{0.333333in}{0.683333in}}%
\pgfpathcurveto{\pgfqpoint{0.328913in}{0.683333in}}{\pgfqpoint{0.324674in}{0.681577in}}{\pgfqpoint{0.321548in}{0.678452in}}%
\pgfpathcurveto{\pgfqpoint{0.318423in}{0.675326in}}{\pgfqpoint{0.316667in}{0.671087in}}{\pgfqpoint{0.316667in}{0.666667in}}%
\pgfpathcurveto{\pgfqpoint{0.316667in}{0.662247in}}{\pgfqpoint{0.318423in}{0.658007in}}{\pgfqpoint{0.321548in}{0.654882in}}%
\pgfpathcurveto{\pgfqpoint{0.324674in}{0.651756in}}{\pgfqpoint{0.328913in}{0.650000in}}{\pgfqpoint{0.333333in}{0.650000in}}%
\pgfpathclose%
\pgfpathmoveto{\pgfqpoint{0.500000in}{0.650000in}}%
\pgfpathcurveto{\pgfqpoint{0.504420in}{0.650000in}}{\pgfqpoint{0.508660in}{0.651756in}}{\pgfqpoint{0.511785in}{0.654882in}}%
\pgfpathcurveto{\pgfqpoint{0.514911in}{0.658007in}}{\pgfqpoint{0.516667in}{0.662247in}}{\pgfqpoint{0.516667in}{0.666667in}}%
\pgfpathcurveto{\pgfqpoint{0.516667in}{0.671087in}}{\pgfqpoint{0.514911in}{0.675326in}}{\pgfqpoint{0.511785in}{0.678452in}}%
\pgfpathcurveto{\pgfqpoint{0.508660in}{0.681577in}}{\pgfqpoint{0.504420in}{0.683333in}}{\pgfqpoint{0.500000in}{0.683333in}}%
\pgfpathcurveto{\pgfqpoint{0.495580in}{0.683333in}}{\pgfqpoint{0.491340in}{0.681577in}}{\pgfqpoint{0.488215in}{0.678452in}}%
\pgfpathcurveto{\pgfqpoint{0.485089in}{0.675326in}}{\pgfqpoint{0.483333in}{0.671087in}}{\pgfqpoint{0.483333in}{0.666667in}}%
\pgfpathcurveto{\pgfqpoint{0.483333in}{0.662247in}}{\pgfqpoint{0.485089in}{0.658007in}}{\pgfqpoint{0.488215in}{0.654882in}}%
\pgfpathcurveto{\pgfqpoint{0.491340in}{0.651756in}}{\pgfqpoint{0.495580in}{0.650000in}}{\pgfqpoint{0.500000in}{0.650000in}}%
\pgfpathclose%
\pgfpathmoveto{\pgfqpoint{0.666667in}{0.650000in}}%
\pgfpathcurveto{\pgfqpoint{0.671087in}{0.650000in}}{\pgfqpoint{0.675326in}{0.651756in}}{\pgfqpoint{0.678452in}{0.654882in}}%
\pgfpathcurveto{\pgfqpoint{0.681577in}{0.658007in}}{\pgfqpoint{0.683333in}{0.662247in}}{\pgfqpoint{0.683333in}{0.666667in}}%
\pgfpathcurveto{\pgfqpoint{0.683333in}{0.671087in}}{\pgfqpoint{0.681577in}{0.675326in}}{\pgfqpoint{0.678452in}{0.678452in}}%
\pgfpathcurveto{\pgfqpoint{0.675326in}{0.681577in}}{\pgfqpoint{0.671087in}{0.683333in}}{\pgfqpoint{0.666667in}{0.683333in}}%
\pgfpathcurveto{\pgfqpoint{0.662247in}{0.683333in}}{\pgfqpoint{0.658007in}{0.681577in}}{\pgfqpoint{0.654882in}{0.678452in}}%
\pgfpathcurveto{\pgfqpoint{0.651756in}{0.675326in}}{\pgfqpoint{0.650000in}{0.671087in}}{\pgfqpoint{0.650000in}{0.666667in}}%
\pgfpathcurveto{\pgfqpoint{0.650000in}{0.662247in}}{\pgfqpoint{0.651756in}{0.658007in}}{\pgfqpoint{0.654882in}{0.654882in}}%
\pgfpathcurveto{\pgfqpoint{0.658007in}{0.651756in}}{\pgfqpoint{0.662247in}{0.650000in}}{\pgfqpoint{0.666667in}{0.650000in}}%
\pgfpathclose%
\pgfpathmoveto{\pgfqpoint{0.833333in}{0.650000in}}%
\pgfpathcurveto{\pgfqpoint{0.837753in}{0.650000in}}{\pgfqpoint{0.841993in}{0.651756in}}{\pgfqpoint{0.845118in}{0.654882in}}%
\pgfpathcurveto{\pgfqpoint{0.848244in}{0.658007in}}{\pgfqpoint{0.850000in}{0.662247in}}{\pgfqpoint{0.850000in}{0.666667in}}%
\pgfpathcurveto{\pgfqpoint{0.850000in}{0.671087in}}{\pgfqpoint{0.848244in}{0.675326in}}{\pgfqpoint{0.845118in}{0.678452in}}%
\pgfpathcurveto{\pgfqpoint{0.841993in}{0.681577in}}{\pgfqpoint{0.837753in}{0.683333in}}{\pgfqpoint{0.833333in}{0.683333in}}%
\pgfpathcurveto{\pgfqpoint{0.828913in}{0.683333in}}{\pgfqpoint{0.824674in}{0.681577in}}{\pgfqpoint{0.821548in}{0.678452in}}%
\pgfpathcurveto{\pgfqpoint{0.818423in}{0.675326in}}{\pgfqpoint{0.816667in}{0.671087in}}{\pgfqpoint{0.816667in}{0.666667in}}%
\pgfpathcurveto{\pgfqpoint{0.816667in}{0.662247in}}{\pgfqpoint{0.818423in}{0.658007in}}{\pgfqpoint{0.821548in}{0.654882in}}%
\pgfpathcurveto{\pgfqpoint{0.824674in}{0.651756in}}{\pgfqpoint{0.828913in}{0.650000in}}{\pgfqpoint{0.833333in}{0.650000in}}%
\pgfpathclose%
\pgfpathmoveto{\pgfqpoint{1.000000in}{0.650000in}}%
\pgfpathcurveto{\pgfqpoint{1.004420in}{0.650000in}}{\pgfqpoint{1.008660in}{0.651756in}}{\pgfqpoint{1.011785in}{0.654882in}}%
\pgfpathcurveto{\pgfqpoint{1.014911in}{0.658007in}}{\pgfqpoint{1.016667in}{0.662247in}}{\pgfqpoint{1.016667in}{0.666667in}}%
\pgfpathcurveto{\pgfqpoint{1.016667in}{0.671087in}}{\pgfqpoint{1.014911in}{0.675326in}}{\pgfqpoint{1.011785in}{0.678452in}}%
\pgfpathcurveto{\pgfqpoint{1.008660in}{0.681577in}}{\pgfqpoint{1.004420in}{0.683333in}}{\pgfqpoint{1.000000in}{0.683333in}}%
\pgfpathcurveto{\pgfqpoint{0.995580in}{0.683333in}}{\pgfqpoint{0.991340in}{0.681577in}}{\pgfqpoint{0.988215in}{0.678452in}}%
\pgfpathcurveto{\pgfqpoint{0.985089in}{0.675326in}}{\pgfqpoint{0.983333in}{0.671087in}}{\pgfqpoint{0.983333in}{0.666667in}}%
\pgfpathcurveto{\pgfqpoint{0.983333in}{0.662247in}}{\pgfqpoint{0.985089in}{0.658007in}}{\pgfqpoint{0.988215in}{0.654882in}}%
\pgfpathcurveto{\pgfqpoint{0.991340in}{0.651756in}}{\pgfqpoint{0.995580in}{0.650000in}}{\pgfqpoint{1.000000in}{0.650000in}}%
\pgfpathclose%
\pgfpathmoveto{\pgfqpoint{0.083333in}{0.816667in}}%
\pgfpathcurveto{\pgfqpoint{0.087753in}{0.816667in}}{\pgfqpoint{0.091993in}{0.818423in}}{\pgfqpoint{0.095118in}{0.821548in}}%
\pgfpathcurveto{\pgfqpoint{0.098244in}{0.824674in}}{\pgfqpoint{0.100000in}{0.828913in}}{\pgfqpoint{0.100000in}{0.833333in}}%
\pgfpathcurveto{\pgfqpoint{0.100000in}{0.837753in}}{\pgfqpoint{0.098244in}{0.841993in}}{\pgfqpoint{0.095118in}{0.845118in}}%
\pgfpathcurveto{\pgfqpoint{0.091993in}{0.848244in}}{\pgfqpoint{0.087753in}{0.850000in}}{\pgfqpoint{0.083333in}{0.850000in}}%
\pgfpathcurveto{\pgfqpoint{0.078913in}{0.850000in}}{\pgfqpoint{0.074674in}{0.848244in}}{\pgfqpoint{0.071548in}{0.845118in}}%
\pgfpathcurveto{\pgfqpoint{0.068423in}{0.841993in}}{\pgfqpoint{0.066667in}{0.837753in}}{\pgfqpoint{0.066667in}{0.833333in}}%
\pgfpathcurveto{\pgfqpoint{0.066667in}{0.828913in}}{\pgfqpoint{0.068423in}{0.824674in}}{\pgfqpoint{0.071548in}{0.821548in}}%
\pgfpathcurveto{\pgfqpoint{0.074674in}{0.818423in}}{\pgfqpoint{0.078913in}{0.816667in}}{\pgfqpoint{0.083333in}{0.816667in}}%
\pgfpathclose%
\pgfpathmoveto{\pgfqpoint{0.250000in}{0.816667in}}%
\pgfpathcurveto{\pgfqpoint{0.254420in}{0.816667in}}{\pgfqpoint{0.258660in}{0.818423in}}{\pgfqpoint{0.261785in}{0.821548in}}%
\pgfpathcurveto{\pgfqpoint{0.264911in}{0.824674in}}{\pgfqpoint{0.266667in}{0.828913in}}{\pgfqpoint{0.266667in}{0.833333in}}%
\pgfpathcurveto{\pgfqpoint{0.266667in}{0.837753in}}{\pgfqpoint{0.264911in}{0.841993in}}{\pgfqpoint{0.261785in}{0.845118in}}%
\pgfpathcurveto{\pgfqpoint{0.258660in}{0.848244in}}{\pgfqpoint{0.254420in}{0.850000in}}{\pgfqpoint{0.250000in}{0.850000in}}%
\pgfpathcurveto{\pgfqpoint{0.245580in}{0.850000in}}{\pgfqpoint{0.241340in}{0.848244in}}{\pgfqpoint{0.238215in}{0.845118in}}%
\pgfpathcurveto{\pgfqpoint{0.235089in}{0.841993in}}{\pgfqpoint{0.233333in}{0.837753in}}{\pgfqpoint{0.233333in}{0.833333in}}%
\pgfpathcurveto{\pgfqpoint{0.233333in}{0.828913in}}{\pgfqpoint{0.235089in}{0.824674in}}{\pgfqpoint{0.238215in}{0.821548in}}%
\pgfpathcurveto{\pgfqpoint{0.241340in}{0.818423in}}{\pgfqpoint{0.245580in}{0.816667in}}{\pgfqpoint{0.250000in}{0.816667in}}%
\pgfpathclose%
\pgfpathmoveto{\pgfqpoint{0.416667in}{0.816667in}}%
\pgfpathcurveto{\pgfqpoint{0.421087in}{0.816667in}}{\pgfqpoint{0.425326in}{0.818423in}}{\pgfqpoint{0.428452in}{0.821548in}}%
\pgfpathcurveto{\pgfqpoint{0.431577in}{0.824674in}}{\pgfqpoint{0.433333in}{0.828913in}}{\pgfqpoint{0.433333in}{0.833333in}}%
\pgfpathcurveto{\pgfqpoint{0.433333in}{0.837753in}}{\pgfqpoint{0.431577in}{0.841993in}}{\pgfqpoint{0.428452in}{0.845118in}}%
\pgfpathcurveto{\pgfqpoint{0.425326in}{0.848244in}}{\pgfqpoint{0.421087in}{0.850000in}}{\pgfqpoint{0.416667in}{0.850000in}}%
\pgfpathcurveto{\pgfqpoint{0.412247in}{0.850000in}}{\pgfqpoint{0.408007in}{0.848244in}}{\pgfqpoint{0.404882in}{0.845118in}}%
\pgfpathcurveto{\pgfqpoint{0.401756in}{0.841993in}}{\pgfqpoint{0.400000in}{0.837753in}}{\pgfqpoint{0.400000in}{0.833333in}}%
\pgfpathcurveto{\pgfqpoint{0.400000in}{0.828913in}}{\pgfqpoint{0.401756in}{0.824674in}}{\pgfqpoint{0.404882in}{0.821548in}}%
\pgfpathcurveto{\pgfqpoint{0.408007in}{0.818423in}}{\pgfqpoint{0.412247in}{0.816667in}}{\pgfqpoint{0.416667in}{0.816667in}}%
\pgfpathclose%
\pgfpathmoveto{\pgfqpoint{0.583333in}{0.816667in}}%
\pgfpathcurveto{\pgfqpoint{0.587753in}{0.816667in}}{\pgfqpoint{0.591993in}{0.818423in}}{\pgfqpoint{0.595118in}{0.821548in}}%
\pgfpathcurveto{\pgfqpoint{0.598244in}{0.824674in}}{\pgfqpoint{0.600000in}{0.828913in}}{\pgfqpoint{0.600000in}{0.833333in}}%
\pgfpathcurveto{\pgfqpoint{0.600000in}{0.837753in}}{\pgfqpoint{0.598244in}{0.841993in}}{\pgfqpoint{0.595118in}{0.845118in}}%
\pgfpathcurveto{\pgfqpoint{0.591993in}{0.848244in}}{\pgfqpoint{0.587753in}{0.850000in}}{\pgfqpoint{0.583333in}{0.850000in}}%
\pgfpathcurveto{\pgfqpoint{0.578913in}{0.850000in}}{\pgfqpoint{0.574674in}{0.848244in}}{\pgfqpoint{0.571548in}{0.845118in}}%
\pgfpathcurveto{\pgfqpoint{0.568423in}{0.841993in}}{\pgfqpoint{0.566667in}{0.837753in}}{\pgfqpoint{0.566667in}{0.833333in}}%
\pgfpathcurveto{\pgfqpoint{0.566667in}{0.828913in}}{\pgfqpoint{0.568423in}{0.824674in}}{\pgfqpoint{0.571548in}{0.821548in}}%
\pgfpathcurveto{\pgfqpoint{0.574674in}{0.818423in}}{\pgfqpoint{0.578913in}{0.816667in}}{\pgfqpoint{0.583333in}{0.816667in}}%
\pgfpathclose%
\pgfpathmoveto{\pgfqpoint{0.750000in}{0.816667in}}%
\pgfpathcurveto{\pgfqpoint{0.754420in}{0.816667in}}{\pgfqpoint{0.758660in}{0.818423in}}{\pgfqpoint{0.761785in}{0.821548in}}%
\pgfpathcurveto{\pgfqpoint{0.764911in}{0.824674in}}{\pgfqpoint{0.766667in}{0.828913in}}{\pgfqpoint{0.766667in}{0.833333in}}%
\pgfpathcurveto{\pgfqpoint{0.766667in}{0.837753in}}{\pgfqpoint{0.764911in}{0.841993in}}{\pgfqpoint{0.761785in}{0.845118in}}%
\pgfpathcurveto{\pgfqpoint{0.758660in}{0.848244in}}{\pgfqpoint{0.754420in}{0.850000in}}{\pgfqpoint{0.750000in}{0.850000in}}%
\pgfpathcurveto{\pgfqpoint{0.745580in}{0.850000in}}{\pgfqpoint{0.741340in}{0.848244in}}{\pgfqpoint{0.738215in}{0.845118in}}%
\pgfpathcurveto{\pgfqpoint{0.735089in}{0.841993in}}{\pgfqpoint{0.733333in}{0.837753in}}{\pgfqpoint{0.733333in}{0.833333in}}%
\pgfpathcurveto{\pgfqpoint{0.733333in}{0.828913in}}{\pgfqpoint{0.735089in}{0.824674in}}{\pgfqpoint{0.738215in}{0.821548in}}%
\pgfpathcurveto{\pgfqpoint{0.741340in}{0.818423in}}{\pgfqpoint{0.745580in}{0.816667in}}{\pgfqpoint{0.750000in}{0.816667in}}%
\pgfpathclose%
\pgfpathmoveto{\pgfqpoint{0.916667in}{0.816667in}}%
\pgfpathcurveto{\pgfqpoint{0.921087in}{0.816667in}}{\pgfqpoint{0.925326in}{0.818423in}}{\pgfqpoint{0.928452in}{0.821548in}}%
\pgfpathcurveto{\pgfqpoint{0.931577in}{0.824674in}}{\pgfqpoint{0.933333in}{0.828913in}}{\pgfqpoint{0.933333in}{0.833333in}}%
\pgfpathcurveto{\pgfqpoint{0.933333in}{0.837753in}}{\pgfqpoint{0.931577in}{0.841993in}}{\pgfqpoint{0.928452in}{0.845118in}}%
\pgfpathcurveto{\pgfqpoint{0.925326in}{0.848244in}}{\pgfqpoint{0.921087in}{0.850000in}}{\pgfqpoint{0.916667in}{0.850000in}}%
\pgfpathcurveto{\pgfqpoint{0.912247in}{0.850000in}}{\pgfqpoint{0.908007in}{0.848244in}}{\pgfqpoint{0.904882in}{0.845118in}}%
\pgfpathcurveto{\pgfqpoint{0.901756in}{0.841993in}}{\pgfqpoint{0.900000in}{0.837753in}}{\pgfqpoint{0.900000in}{0.833333in}}%
\pgfpathcurveto{\pgfqpoint{0.900000in}{0.828913in}}{\pgfqpoint{0.901756in}{0.824674in}}{\pgfqpoint{0.904882in}{0.821548in}}%
\pgfpathcurveto{\pgfqpoint{0.908007in}{0.818423in}}{\pgfqpoint{0.912247in}{0.816667in}}{\pgfqpoint{0.916667in}{0.816667in}}%
\pgfpathclose%
\pgfpathmoveto{\pgfqpoint{0.000000in}{0.983333in}}%
\pgfpathcurveto{\pgfqpoint{0.004420in}{0.983333in}}{\pgfqpoint{0.008660in}{0.985089in}}{\pgfqpoint{0.011785in}{0.988215in}}%
\pgfpathcurveto{\pgfqpoint{0.014911in}{0.991340in}}{\pgfqpoint{0.016667in}{0.995580in}}{\pgfqpoint{0.016667in}{1.000000in}}%
\pgfpathcurveto{\pgfqpoint{0.016667in}{1.004420in}}{\pgfqpoint{0.014911in}{1.008660in}}{\pgfqpoint{0.011785in}{1.011785in}}%
\pgfpathcurveto{\pgfqpoint{0.008660in}{1.014911in}}{\pgfqpoint{0.004420in}{1.016667in}}{\pgfqpoint{0.000000in}{1.016667in}}%
\pgfpathcurveto{\pgfqpoint{-0.004420in}{1.016667in}}{\pgfqpoint{-0.008660in}{1.014911in}}{\pgfqpoint{-0.011785in}{1.011785in}}%
\pgfpathcurveto{\pgfqpoint{-0.014911in}{1.008660in}}{\pgfqpoint{-0.016667in}{1.004420in}}{\pgfqpoint{-0.016667in}{1.000000in}}%
\pgfpathcurveto{\pgfqpoint{-0.016667in}{0.995580in}}{\pgfqpoint{-0.014911in}{0.991340in}}{\pgfqpoint{-0.011785in}{0.988215in}}%
\pgfpathcurveto{\pgfqpoint{-0.008660in}{0.985089in}}{\pgfqpoint{-0.004420in}{0.983333in}}{\pgfqpoint{0.000000in}{0.983333in}}%
\pgfpathclose%
\pgfpathmoveto{\pgfqpoint{0.166667in}{0.983333in}}%
\pgfpathcurveto{\pgfqpoint{0.171087in}{0.983333in}}{\pgfqpoint{0.175326in}{0.985089in}}{\pgfqpoint{0.178452in}{0.988215in}}%
\pgfpathcurveto{\pgfqpoint{0.181577in}{0.991340in}}{\pgfqpoint{0.183333in}{0.995580in}}{\pgfqpoint{0.183333in}{1.000000in}}%
\pgfpathcurveto{\pgfqpoint{0.183333in}{1.004420in}}{\pgfqpoint{0.181577in}{1.008660in}}{\pgfqpoint{0.178452in}{1.011785in}}%
\pgfpathcurveto{\pgfqpoint{0.175326in}{1.014911in}}{\pgfqpoint{0.171087in}{1.016667in}}{\pgfqpoint{0.166667in}{1.016667in}}%
\pgfpathcurveto{\pgfqpoint{0.162247in}{1.016667in}}{\pgfqpoint{0.158007in}{1.014911in}}{\pgfqpoint{0.154882in}{1.011785in}}%
\pgfpathcurveto{\pgfqpoint{0.151756in}{1.008660in}}{\pgfqpoint{0.150000in}{1.004420in}}{\pgfqpoint{0.150000in}{1.000000in}}%
\pgfpathcurveto{\pgfqpoint{0.150000in}{0.995580in}}{\pgfqpoint{0.151756in}{0.991340in}}{\pgfqpoint{0.154882in}{0.988215in}}%
\pgfpathcurveto{\pgfqpoint{0.158007in}{0.985089in}}{\pgfqpoint{0.162247in}{0.983333in}}{\pgfqpoint{0.166667in}{0.983333in}}%
\pgfpathclose%
\pgfpathmoveto{\pgfqpoint{0.333333in}{0.983333in}}%
\pgfpathcurveto{\pgfqpoint{0.337753in}{0.983333in}}{\pgfqpoint{0.341993in}{0.985089in}}{\pgfqpoint{0.345118in}{0.988215in}}%
\pgfpathcurveto{\pgfqpoint{0.348244in}{0.991340in}}{\pgfqpoint{0.350000in}{0.995580in}}{\pgfqpoint{0.350000in}{1.000000in}}%
\pgfpathcurveto{\pgfqpoint{0.350000in}{1.004420in}}{\pgfqpoint{0.348244in}{1.008660in}}{\pgfqpoint{0.345118in}{1.011785in}}%
\pgfpathcurveto{\pgfqpoint{0.341993in}{1.014911in}}{\pgfqpoint{0.337753in}{1.016667in}}{\pgfqpoint{0.333333in}{1.016667in}}%
\pgfpathcurveto{\pgfqpoint{0.328913in}{1.016667in}}{\pgfqpoint{0.324674in}{1.014911in}}{\pgfqpoint{0.321548in}{1.011785in}}%
\pgfpathcurveto{\pgfqpoint{0.318423in}{1.008660in}}{\pgfqpoint{0.316667in}{1.004420in}}{\pgfqpoint{0.316667in}{1.000000in}}%
\pgfpathcurveto{\pgfqpoint{0.316667in}{0.995580in}}{\pgfqpoint{0.318423in}{0.991340in}}{\pgfqpoint{0.321548in}{0.988215in}}%
\pgfpathcurveto{\pgfqpoint{0.324674in}{0.985089in}}{\pgfqpoint{0.328913in}{0.983333in}}{\pgfqpoint{0.333333in}{0.983333in}}%
\pgfpathclose%
\pgfpathmoveto{\pgfqpoint{0.500000in}{0.983333in}}%
\pgfpathcurveto{\pgfqpoint{0.504420in}{0.983333in}}{\pgfqpoint{0.508660in}{0.985089in}}{\pgfqpoint{0.511785in}{0.988215in}}%
\pgfpathcurveto{\pgfqpoint{0.514911in}{0.991340in}}{\pgfqpoint{0.516667in}{0.995580in}}{\pgfqpoint{0.516667in}{1.000000in}}%
\pgfpathcurveto{\pgfqpoint{0.516667in}{1.004420in}}{\pgfqpoint{0.514911in}{1.008660in}}{\pgfqpoint{0.511785in}{1.011785in}}%
\pgfpathcurveto{\pgfqpoint{0.508660in}{1.014911in}}{\pgfqpoint{0.504420in}{1.016667in}}{\pgfqpoint{0.500000in}{1.016667in}}%
\pgfpathcurveto{\pgfqpoint{0.495580in}{1.016667in}}{\pgfqpoint{0.491340in}{1.014911in}}{\pgfqpoint{0.488215in}{1.011785in}}%
\pgfpathcurveto{\pgfqpoint{0.485089in}{1.008660in}}{\pgfqpoint{0.483333in}{1.004420in}}{\pgfqpoint{0.483333in}{1.000000in}}%
\pgfpathcurveto{\pgfqpoint{0.483333in}{0.995580in}}{\pgfqpoint{0.485089in}{0.991340in}}{\pgfqpoint{0.488215in}{0.988215in}}%
\pgfpathcurveto{\pgfqpoint{0.491340in}{0.985089in}}{\pgfqpoint{0.495580in}{0.983333in}}{\pgfqpoint{0.500000in}{0.983333in}}%
\pgfpathclose%
\pgfpathmoveto{\pgfqpoint{0.666667in}{0.983333in}}%
\pgfpathcurveto{\pgfqpoint{0.671087in}{0.983333in}}{\pgfqpoint{0.675326in}{0.985089in}}{\pgfqpoint{0.678452in}{0.988215in}}%
\pgfpathcurveto{\pgfqpoint{0.681577in}{0.991340in}}{\pgfqpoint{0.683333in}{0.995580in}}{\pgfqpoint{0.683333in}{1.000000in}}%
\pgfpathcurveto{\pgfqpoint{0.683333in}{1.004420in}}{\pgfqpoint{0.681577in}{1.008660in}}{\pgfqpoint{0.678452in}{1.011785in}}%
\pgfpathcurveto{\pgfqpoint{0.675326in}{1.014911in}}{\pgfqpoint{0.671087in}{1.016667in}}{\pgfqpoint{0.666667in}{1.016667in}}%
\pgfpathcurveto{\pgfqpoint{0.662247in}{1.016667in}}{\pgfqpoint{0.658007in}{1.014911in}}{\pgfqpoint{0.654882in}{1.011785in}}%
\pgfpathcurveto{\pgfqpoint{0.651756in}{1.008660in}}{\pgfqpoint{0.650000in}{1.004420in}}{\pgfqpoint{0.650000in}{1.000000in}}%
\pgfpathcurveto{\pgfqpoint{0.650000in}{0.995580in}}{\pgfqpoint{0.651756in}{0.991340in}}{\pgfqpoint{0.654882in}{0.988215in}}%
\pgfpathcurveto{\pgfqpoint{0.658007in}{0.985089in}}{\pgfqpoint{0.662247in}{0.983333in}}{\pgfqpoint{0.666667in}{0.983333in}}%
\pgfpathclose%
\pgfpathmoveto{\pgfqpoint{0.833333in}{0.983333in}}%
\pgfpathcurveto{\pgfqpoint{0.837753in}{0.983333in}}{\pgfqpoint{0.841993in}{0.985089in}}{\pgfqpoint{0.845118in}{0.988215in}}%
\pgfpathcurveto{\pgfqpoint{0.848244in}{0.991340in}}{\pgfqpoint{0.850000in}{0.995580in}}{\pgfqpoint{0.850000in}{1.000000in}}%
\pgfpathcurveto{\pgfqpoint{0.850000in}{1.004420in}}{\pgfqpoint{0.848244in}{1.008660in}}{\pgfqpoint{0.845118in}{1.011785in}}%
\pgfpathcurveto{\pgfqpoint{0.841993in}{1.014911in}}{\pgfqpoint{0.837753in}{1.016667in}}{\pgfqpoint{0.833333in}{1.016667in}}%
\pgfpathcurveto{\pgfqpoint{0.828913in}{1.016667in}}{\pgfqpoint{0.824674in}{1.014911in}}{\pgfqpoint{0.821548in}{1.011785in}}%
\pgfpathcurveto{\pgfqpoint{0.818423in}{1.008660in}}{\pgfqpoint{0.816667in}{1.004420in}}{\pgfqpoint{0.816667in}{1.000000in}}%
\pgfpathcurveto{\pgfqpoint{0.816667in}{0.995580in}}{\pgfqpoint{0.818423in}{0.991340in}}{\pgfqpoint{0.821548in}{0.988215in}}%
\pgfpathcurveto{\pgfqpoint{0.824674in}{0.985089in}}{\pgfqpoint{0.828913in}{0.983333in}}{\pgfqpoint{0.833333in}{0.983333in}}%
\pgfpathclose%
\pgfpathmoveto{\pgfqpoint{1.000000in}{0.983333in}}%
\pgfpathcurveto{\pgfqpoint{1.004420in}{0.983333in}}{\pgfqpoint{1.008660in}{0.985089in}}{\pgfqpoint{1.011785in}{0.988215in}}%
\pgfpathcurveto{\pgfqpoint{1.014911in}{0.991340in}}{\pgfqpoint{1.016667in}{0.995580in}}{\pgfqpoint{1.016667in}{1.000000in}}%
\pgfpathcurveto{\pgfqpoint{1.016667in}{1.004420in}}{\pgfqpoint{1.014911in}{1.008660in}}{\pgfqpoint{1.011785in}{1.011785in}}%
\pgfpathcurveto{\pgfqpoint{1.008660in}{1.014911in}}{\pgfqpoint{1.004420in}{1.016667in}}{\pgfqpoint{1.000000in}{1.016667in}}%
\pgfpathcurveto{\pgfqpoint{0.995580in}{1.016667in}}{\pgfqpoint{0.991340in}{1.014911in}}{\pgfqpoint{0.988215in}{1.011785in}}%
\pgfpathcurveto{\pgfqpoint{0.985089in}{1.008660in}}{\pgfqpoint{0.983333in}{1.004420in}}{\pgfqpoint{0.983333in}{1.000000in}}%
\pgfpathcurveto{\pgfqpoint{0.983333in}{0.995580in}}{\pgfqpoint{0.985089in}{0.991340in}}{\pgfqpoint{0.988215in}{0.988215in}}%
\pgfpathcurveto{\pgfqpoint{0.991340in}{0.985089in}}{\pgfqpoint{0.995580in}{0.983333in}}{\pgfqpoint{1.000000in}{0.983333in}}%
\pgfpathclose%
\pgfusepath{stroke}%
\end{pgfscope}%
}%
\pgfsys@transformshift{1.135815in}{8.357449in}%
\pgfsys@useobject{currentpattern}{}%
\pgfsys@transformshift{1in}{0in}%
\pgfsys@transformshift{-1in}{0in}%
\pgfsys@transformshift{0in}{1in}%
\end{pgfscope}%
\begin{pgfscope}%
\definecolor{textcolor}{rgb}{0.000000,0.000000,0.000000}%
\pgfsetstrokecolor{textcolor}%
\pgfsetfillcolor{textcolor}%
\pgftext[x=1.758037in,y=8.357449in,left,base]{\color{textcolor}\rmfamily\fontsize{16.000000}{19.200000}\selectfont NUCLEAR\_TB}%
\end{pgfscope}%
\begin{pgfscope}%
\pgfsetbuttcap%
\pgfsetmiterjoin%
\definecolor{currentfill}{rgb}{1.000000,1.000000,0.000000}%
\pgfsetfillcolor{currentfill}%
\pgfsetfillopacity{0.990000}%
\pgfsetlinewidth{0.000000pt}%
\definecolor{currentstroke}{rgb}{0.000000,0.000000,0.000000}%
\pgfsetstrokecolor{currentstroke}%
\pgfsetstrokeopacity{0.990000}%
\pgfsetdash{}{0pt}%
\pgfpathmoveto{\pgfqpoint{1.135815in}{8.032990in}}%
\pgfpathlineto{\pgfqpoint{1.580259in}{8.032990in}}%
\pgfpathlineto{\pgfqpoint{1.580259in}{8.188545in}}%
\pgfpathlineto{\pgfqpoint{1.135815in}{8.188545in}}%
\pgfpathclose%
\pgfusepath{fill}%
\end{pgfscope}%
\begin{pgfscope}%
\pgfsetbuttcap%
\pgfsetmiterjoin%
\definecolor{currentfill}{rgb}{1.000000,1.000000,0.000000}%
\pgfsetfillcolor{currentfill}%
\pgfsetfillopacity{0.990000}%
\pgfsetlinewidth{0.000000pt}%
\definecolor{currentstroke}{rgb}{0.000000,0.000000,0.000000}%
\pgfsetstrokecolor{currentstroke}%
\pgfsetstrokeopacity{0.990000}%
\pgfsetdash{}{0pt}%
\pgfpathmoveto{\pgfqpoint{1.135815in}{8.032990in}}%
\pgfpathlineto{\pgfqpoint{1.580259in}{8.032990in}}%
\pgfpathlineto{\pgfqpoint{1.580259in}{8.188545in}}%
\pgfpathlineto{\pgfqpoint{1.135815in}{8.188545in}}%
\pgfpathclose%
\pgfusepath{clip}%
\pgfsys@defobject{currentpattern}{\pgfqpoint{0in}{0in}}{\pgfqpoint{1in}{1in}}{%
\begin{pgfscope}%
\pgfpathrectangle{\pgfqpoint{0in}{0in}}{\pgfqpoint{1in}{1in}}%
\pgfusepath{clip}%
\pgfpathmoveto{\pgfqpoint{0.000000in}{0.055556in}}%
\pgfpathlineto{\pgfqpoint{-0.016327in}{0.022473in}}%
\pgfpathlineto{\pgfqpoint{-0.052836in}{0.017168in}}%
\pgfpathlineto{\pgfqpoint{-0.026418in}{-0.008584in}}%
\pgfpathlineto{\pgfqpoint{-0.032655in}{-0.044945in}}%
\pgfpathlineto{\pgfqpoint{-0.000000in}{-0.027778in}}%
\pgfpathlineto{\pgfqpoint{0.032655in}{-0.044945in}}%
\pgfpathlineto{\pgfqpoint{0.026418in}{-0.008584in}}%
\pgfpathlineto{\pgfqpoint{0.052836in}{0.017168in}}%
\pgfpathlineto{\pgfqpoint{0.016327in}{0.022473in}}%
\pgfpathlineto{\pgfqpoint{0.000000in}{0.055556in}}%
\pgfpathmoveto{\pgfqpoint{0.166667in}{0.055556in}}%
\pgfpathlineto{\pgfqpoint{0.150339in}{0.022473in}}%
\pgfpathlineto{\pgfqpoint{0.113830in}{0.017168in}}%
\pgfpathlineto{\pgfqpoint{0.140248in}{-0.008584in}}%
\pgfpathlineto{\pgfqpoint{0.134012in}{-0.044945in}}%
\pgfpathlineto{\pgfqpoint{0.166667in}{-0.027778in}}%
\pgfpathlineto{\pgfqpoint{0.199321in}{-0.044945in}}%
\pgfpathlineto{\pgfqpoint{0.193085in}{-0.008584in}}%
\pgfpathlineto{\pgfqpoint{0.219503in}{0.017168in}}%
\pgfpathlineto{\pgfqpoint{0.182994in}{0.022473in}}%
\pgfpathlineto{\pgfqpoint{0.166667in}{0.055556in}}%
\pgfpathmoveto{\pgfqpoint{0.333333in}{0.055556in}}%
\pgfpathlineto{\pgfqpoint{0.317006in}{0.022473in}}%
\pgfpathlineto{\pgfqpoint{0.280497in}{0.017168in}}%
\pgfpathlineto{\pgfqpoint{0.306915in}{-0.008584in}}%
\pgfpathlineto{\pgfqpoint{0.300679in}{-0.044945in}}%
\pgfpathlineto{\pgfqpoint{0.333333in}{-0.027778in}}%
\pgfpathlineto{\pgfqpoint{0.365988in}{-0.044945in}}%
\pgfpathlineto{\pgfqpoint{0.359752in}{-0.008584in}}%
\pgfpathlineto{\pgfqpoint{0.386170in}{0.017168in}}%
\pgfpathlineto{\pgfqpoint{0.349661in}{0.022473in}}%
\pgfpathlineto{\pgfqpoint{0.333333in}{0.055556in}}%
\pgfpathmoveto{\pgfqpoint{0.500000in}{0.055556in}}%
\pgfpathlineto{\pgfqpoint{0.483673in}{0.022473in}}%
\pgfpathlineto{\pgfqpoint{0.447164in}{0.017168in}}%
\pgfpathlineto{\pgfqpoint{0.473582in}{-0.008584in}}%
\pgfpathlineto{\pgfqpoint{0.467345in}{-0.044945in}}%
\pgfpathlineto{\pgfqpoint{0.500000in}{-0.027778in}}%
\pgfpathlineto{\pgfqpoint{0.532655in}{-0.044945in}}%
\pgfpathlineto{\pgfqpoint{0.526418in}{-0.008584in}}%
\pgfpathlineto{\pgfqpoint{0.552836in}{0.017168in}}%
\pgfpathlineto{\pgfqpoint{0.516327in}{0.022473in}}%
\pgfpathlineto{\pgfqpoint{0.500000in}{0.055556in}}%
\pgfpathmoveto{\pgfqpoint{0.666667in}{0.055556in}}%
\pgfpathlineto{\pgfqpoint{0.650339in}{0.022473in}}%
\pgfpathlineto{\pgfqpoint{0.613830in}{0.017168in}}%
\pgfpathlineto{\pgfqpoint{0.640248in}{-0.008584in}}%
\pgfpathlineto{\pgfqpoint{0.634012in}{-0.044945in}}%
\pgfpathlineto{\pgfqpoint{0.666667in}{-0.027778in}}%
\pgfpathlineto{\pgfqpoint{0.699321in}{-0.044945in}}%
\pgfpathlineto{\pgfqpoint{0.693085in}{-0.008584in}}%
\pgfpathlineto{\pgfqpoint{0.719503in}{0.017168in}}%
\pgfpathlineto{\pgfqpoint{0.682994in}{0.022473in}}%
\pgfpathlineto{\pgfqpoint{0.666667in}{0.055556in}}%
\pgfpathmoveto{\pgfqpoint{0.833333in}{0.055556in}}%
\pgfpathlineto{\pgfqpoint{0.817006in}{0.022473in}}%
\pgfpathlineto{\pgfqpoint{0.780497in}{0.017168in}}%
\pgfpathlineto{\pgfqpoint{0.806915in}{-0.008584in}}%
\pgfpathlineto{\pgfqpoint{0.800679in}{-0.044945in}}%
\pgfpathlineto{\pgfqpoint{0.833333in}{-0.027778in}}%
\pgfpathlineto{\pgfqpoint{0.865988in}{-0.044945in}}%
\pgfpathlineto{\pgfqpoint{0.859752in}{-0.008584in}}%
\pgfpathlineto{\pgfqpoint{0.886170in}{0.017168in}}%
\pgfpathlineto{\pgfqpoint{0.849661in}{0.022473in}}%
\pgfpathlineto{\pgfqpoint{0.833333in}{0.055556in}}%
\pgfpathmoveto{\pgfqpoint{1.000000in}{0.055556in}}%
\pgfpathlineto{\pgfqpoint{0.983673in}{0.022473in}}%
\pgfpathlineto{\pgfqpoint{0.947164in}{0.017168in}}%
\pgfpathlineto{\pgfqpoint{0.973582in}{-0.008584in}}%
\pgfpathlineto{\pgfqpoint{0.967345in}{-0.044945in}}%
\pgfpathlineto{\pgfqpoint{1.000000in}{-0.027778in}}%
\pgfpathlineto{\pgfqpoint{1.032655in}{-0.044945in}}%
\pgfpathlineto{\pgfqpoint{1.026418in}{-0.008584in}}%
\pgfpathlineto{\pgfqpoint{1.052836in}{0.017168in}}%
\pgfpathlineto{\pgfqpoint{1.016327in}{0.022473in}}%
\pgfpathlineto{\pgfqpoint{1.000000in}{0.055556in}}%
\pgfpathmoveto{\pgfqpoint{0.083333in}{0.222222in}}%
\pgfpathlineto{\pgfqpoint{0.067006in}{0.189139in}}%
\pgfpathlineto{\pgfqpoint{0.030497in}{0.183834in}}%
\pgfpathlineto{\pgfqpoint{0.056915in}{0.158083in}}%
\pgfpathlineto{\pgfqpoint{0.050679in}{0.121721in}}%
\pgfpathlineto{\pgfqpoint{0.083333in}{0.138889in}}%
\pgfpathlineto{\pgfqpoint{0.115988in}{0.121721in}}%
\pgfpathlineto{\pgfqpoint{0.109752in}{0.158083in}}%
\pgfpathlineto{\pgfqpoint{0.136170in}{0.183834in}}%
\pgfpathlineto{\pgfqpoint{0.099661in}{0.189139in}}%
\pgfpathlineto{\pgfqpoint{0.083333in}{0.222222in}}%
\pgfpathmoveto{\pgfqpoint{0.250000in}{0.222222in}}%
\pgfpathlineto{\pgfqpoint{0.233673in}{0.189139in}}%
\pgfpathlineto{\pgfqpoint{0.197164in}{0.183834in}}%
\pgfpathlineto{\pgfqpoint{0.223582in}{0.158083in}}%
\pgfpathlineto{\pgfqpoint{0.217345in}{0.121721in}}%
\pgfpathlineto{\pgfqpoint{0.250000in}{0.138889in}}%
\pgfpathlineto{\pgfqpoint{0.282655in}{0.121721in}}%
\pgfpathlineto{\pgfqpoint{0.276418in}{0.158083in}}%
\pgfpathlineto{\pgfqpoint{0.302836in}{0.183834in}}%
\pgfpathlineto{\pgfqpoint{0.266327in}{0.189139in}}%
\pgfpathlineto{\pgfqpoint{0.250000in}{0.222222in}}%
\pgfpathmoveto{\pgfqpoint{0.416667in}{0.222222in}}%
\pgfpathlineto{\pgfqpoint{0.400339in}{0.189139in}}%
\pgfpathlineto{\pgfqpoint{0.363830in}{0.183834in}}%
\pgfpathlineto{\pgfqpoint{0.390248in}{0.158083in}}%
\pgfpathlineto{\pgfqpoint{0.384012in}{0.121721in}}%
\pgfpathlineto{\pgfqpoint{0.416667in}{0.138889in}}%
\pgfpathlineto{\pgfqpoint{0.449321in}{0.121721in}}%
\pgfpathlineto{\pgfqpoint{0.443085in}{0.158083in}}%
\pgfpathlineto{\pgfqpoint{0.469503in}{0.183834in}}%
\pgfpathlineto{\pgfqpoint{0.432994in}{0.189139in}}%
\pgfpathlineto{\pgfqpoint{0.416667in}{0.222222in}}%
\pgfpathmoveto{\pgfqpoint{0.583333in}{0.222222in}}%
\pgfpathlineto{\pgfqpoint{0.567006in}{0.189139in}}%
\pgfpathlineto{\pgfqpoint{0.530497in}{0.183834in}}%
\pgfpathlineto{\pgfqpoint{0.556915in}{0.158083in}}%
\pgfpathlineto{\pgfqpoint{0.550679in}{0.121721in}}%
\pgfpathlineto{\pgfqpoint{0.583333in}{0.138889in}}%
\pgfpathlineto{\pgfqpoint{0.615988in}{0.121721in}}%
\pgfpathlineto{\pgfqpoint{0.609752in}{0.158083in}}%
\pgfpathlineto{\pgfqpoint{0.636170in}{0.183834in}}%
\pgfpathlineto{\pgfqpoint{0.599661in}{0.189139in}}%
\pgfpathlineto{\pgfqpoint{0.583333in}{0.222222in}}%
\pgfpathmoveto{\pgfqpoint{0.750000in}{0.222222in}}%
\pgfpathlineto{\pgfqpoint{0.733673in}{0.189139in}}%
\pgfpathlineto{\pgfqpoint{0.697164in}{0.183834in}}%
\pgfpathlineto{\pgfqpoint{0.723582in}{0.158083in}}%
\pgfpathlineto{\pgfqpoint{0.717345in}{0.121721in}}%
\pgfpathlineto{\pgfqpoint{0.750000in}{0.138889in}}%
\pgfpathlineto{\pgfqpoint{0.782655in}{0.121721in}}%
\pgfpathlineto{\pgfqpoint{0.776418in}{0.158083in}}%
\pgfpathlineto{\pgfqpoint{0.802836in}{0.183834in}}%
\pgfpathlineto{\pgfqpoint{0.766327in}{0.189139in}}%
\pgfpathlineto{\pgfqpoint{0.750000in}{0.222222in}}%
\pgfpathmoveto{\pgfqpoint{0.916667in}{0.222222in}}%
\pgfpathlineto{\pgfqpoint{0.900339in}{0.189139in}}%
\pgfpathlineto{\pgfqpoint{0.863830in}{0.183834in}}%
\pgfpathlineto{\pgfqpoint{0.890248in}{0.158083in}}%
\pgfpathlineto{\pgfqpoint{0.884012in}{0.121721in}}%
\pgfpathlineto{\pgfqpoint{0.916667in}{0.138889in}}%
\pgfpathlineto{\pgfqpoint{0.949321in}{0.121721in}}%
\pgfpathlineto{\pgfqpoint{0.943085in}{0.158083in}}%
\pgfpathlineto{\pgfqpoint{0.969503in}{0.183834in}}%
\pgfpathlineto{\pgfqpoint{0.932994in}{0.189139in}}%
\pgfpathlineto{\pgfqpoint{0.916667in}{0.222222in}}%
\pgfpathmoveto{\pgfqpoint{0.000000in}{0.388889in}}%
\pgfpathlineto{\pgfqpoint{-0.016327in}{0.355806in}}%
\pgfpathlineto{\pgfqpoint{-0.052836in}{0.350501in}}%
\pgfpathlineto{\pgfqpoint{-0.026418in}{0.324750in}}%
\pgfpathlineto{\pgfqpoint{-0.032655in}{0.288388in}}%
\pgfpathlineto{\pgfqpoint{-0.000000in}{0.305556in}}%
\pgfpathlineto{\pgfqpoint{0.032655in}{0.288388in}}%
\pgfpathlineto{\pgfqpoint{0.026418in}{0.324750in}}%
\pgfpathlineto{\pgfqpoint{0.052836in}{0.350501in}}%
\pgfpathlineto{\pgfqpoint{0.016327in}{0.355806in}}%
\pgfpathlineto{\pgfqpoint{0.000000in}{0.388889in}}%
\pgfpathmoveto{\pgfqpoint{0.166667in}{0.388889in}}%
\pgfpathlineto{\pgfqpoint{0.150339in}{0.355806in}}%
\pgfpathlineto{\pgfqpoint{0.113830in}{0.350501in}}%
\pgfpathlineto{\pgfqpoint{0.140248in}{0.324750in}}%
\pgfpathlineto{\pgfqpoint{0.134012in}{0.288388in}}%
\pgfpathlineto{\pgfqpoint{0.166667in}{0.305556in}}%
\pgfpathlineto{\pgfqpoint{0.199321in}{0.288388in}}%
\pgfpathlineto{\pgfqpoint{0.193085in}{0.324750in}}%
\pgfpathlineto{\pgfqpoint{0.219503in}{0.350501in}}%
\pgfpathlineto{\pgfqpoint{0.182994in}{0.355806in}}%
\pgfpathlineto{\pgfqpoint{0.166667in}{0.388889in}}%
\pgfpathmoveto{\pgfqpoint{0.333333in}{0.388889in}}%
\pgfpathlineto{\pgfqpoint{0.317006in}{0.355806in}}%
\pgfpathlineto{\pgfqpoint{0.280497in}{0.350501in}}%
\pgfpathlineto{\pgfqpoint{0.306915in}{0.324750in}}%
\pgfpathlineto{\pgfqpoint{0.300679in}{0.288388in}}%
\pgfpathlineto{\pgfqpoint{0.333333in}{0.305556in}}%
\pgfpathlineto{\pgfqpoint{0.365988in}{0.288388in}}%
\pgfpathlineto{\pgfqpoint{0.359752in}{0.324750in}}%
\pgfpathlineto{\pgfqpoint{0.386170in}{0.350501in}}%
\pgfpathlineto{\pgfqpoint{0.349661in}{0.355806in}}%
\pgfpathlineto{\pgfqpoint{0.333333in}{0.388889in}}%
\pgfpathmoveto{\pgfqpoint{0.500000in}{0.388889in}}%
\pgfpathlineto{\pgfqpoint{0.483673in}{0.355806in}}%
\pgfpathlineto{\pgfqpoint{0.447164in}{0.350501in}}%
\pgfpathlineto{\pgfqpoint{0.473582in}{0.324750in}}%
\pgfpathlineto{\pgfqpoint{0.467345in}{0.288388in}}%
\pgfpathlineto{\pgfqpoint{0.500000in}{0.305556in}}%
\pgfpathlineto{\pgfqpoint{0.532655in}{0.288388in}}%
\pgfpathlineto{\pgfqpoint{0.526418in}{0.324750in}}%
\pgfpathlineto{\pgfqpoint{0.552836in}{0.350501in}}%
\pgfpathlineto{\pgfqpoint{0.516327in}{0.355806in}}%
\pgfpathlineto{\pgfqpoint{0.500000in}{0.388889in}}%
\pgfpathmoveto{\pgfqpoint{0.666667in}{0.388889in}}%
\pgfpathlineto{\pgfqpoint{0.650339in}{0.355806in}}%
\pgfpathlineto{\pgfqpoint{0.613830in}{0.350501in}}%
\pgfpathlineto{\pgfqpoint{0.640248in}{0.324750in}}%
\pgfpathlineto{\pgfqpoint{0.634012in}{0.288388in}}%
\pgfpathlineto{\pgfqpoint{0.666667in}{0.305556in}}%
\pgfpathlineto{\pgfqpoint{0.699321in}{0.288388in}}%
\pgfpathlineto{\pgfqpoint{0.693085in}{0.324750in}}%
\pgfpathlineto{\pgfqpoint{0.719503in}{0.350501in}}%
\pgfpathlineto{\pgfqpoint{0.682994in}{0.355806in}}%
\pgfpathlineto{\pgfqpoint{0.666667in}{0.388889in}}%
\pgfpathmoveto{\pgfqpoint{0.833333in}{0.388889in}}%
\pgfpathlineto{\pgfqpoint{0.817006in}{0.355806in}}%
\pgfpathlineto{\pgfqpoint{0.780497in}{0.350501in}}%
\pgfpathlineto{\pgfqpoint{0.806915in}{0.324750in}}%
\pgfpathlineto{\pgfqpoint{0.800679in}{0.288388in}}%
\pgfpathlineto{\pgfqpoint{0.833333in}{0.305556in}}%
\pgfpathlineto{\pgfqpoint{0.865988in}{0.288388in}}%
\pgfpathlineto{\pgfqpoint{0.859752in}{0.324750in}}%
\pgfpathlineto{\pgfqpoint{0.886170in}{0.350501in}}%
\pgfpathlineto{\pgfqpoint{0.849661in}{0.355806in}}%
\pgfpathlineto{\pgfqpoint{0.833333in}{0.388889in}}%
\pgfpathmoveto{\pgfqpoint{1.000000in}{0.388889in}}%
\pgfpathlineto{\pgfqpoint{0.983673in}{0.355806in}}%
\pgfpathlineto{\pgfqpoint{0.947164in}{0.350501in}}%
\pgfpathlineto{\pgfqpoint{0.973582in}{0.324750in}}%
\pgfpathlineto{\pgfqpoint{0.967345in}{0.288388in}}%
\pgfpathlineto{\pgfqpoint{1.000000in}{0.305556in}}%
\pgfpathlineto{\pgfqpoint{1.032655in}{0.288388in}}%
\pgfpathlineto{\pgfqpoint{1.026418in}{0.324750in}}%
\pgfpathlineto{\pgfqpoint{1.052836in}{0.350501in}}%
\pgfpathlineto{\pgfqpoint{1.016327in}{0.355806in}}%
\pgfpathlineto{\pgfqpoint{1.000000in}{0.388889in}}%
\pgfpathmoveto{\pgfqpoint{0.083333in}{0.555556in}}%
\pgfpathlineto{\pgfqpoint{0.067006in}{0.522473in}}%
\pgfpathlineto{\pgfqpoint{0.030497in}{0.517168in}}%
\pgfpathlineto{\pgfqpoint{0.056915in}{0.491416in}}%
\pgfpathlineto{\pgfqpoint{0.050679in}{0.455055in}}%
\pgfpathlineto{\pgfqpoint{0.083333in}{0.472222in}}%
\pgfpathlineto{\pgfqpoint{0.115988in}{0.455055in}}%
\pgfpathlineto{\pgfqpoint{0.109752in}{0.491416in}}%
\pgfpathlineto{\pgfqpoint{0.136170in}{0.517168in}}%
\pgfpathlineto{\pgfqpoint{0.099661in}{0.522473in}}%
\pgfpathlineto{\pgfqpoint{0.083333in}{0.555556in}}%
\pgfpathmoveto{\pgfqpoint{0.250000in}{0.555556in}}%
\pgfpathlineto{\pgfqpoint{0.233673in}{0.522473in}}%
\pgfpathlineto{\pgfqpoint{0.197164in}{0.517168in}}%
\pgfpathlineto{\pgfqpoint{0.223582in}{0.491416in}}%
\pgfpathlineto{\pgfqpoint{0.217345in}{0.455055in}}%
\pgfpathlineto{\pgfqpoint{0.250000in}{0.472222in}}%
\pgfpathlineto{\pgfqpoint{0.282655in}{0.455055in}}%
\pgfpathlineto{\pgfqpoint{0.276418in}{0.491416in}}%
\pgfpathlineto{\pgfqpoint{0.302836in}{0.517168in}}%
\pgfpathlineto{\pgfqpoint{0.266327in}{0.522473in}}%
\pgfpathlineto{\pgfqpoint{0.250000in}{0.555556in}}%
\pgfpathmoveto{\pgfqpoint{0.416667in}{0.555556in}}%
\pgfpathlineto{\pgfqpoint{0.400339in}{0.522473in}}%
\pgfpathlineto{\pgfqpoint{0.363830in}{0.517168in}}%
\pgfpathlineto{\pgfqpoint{0.390248in}{0.491416in}}%
\pgfpathlineto{\pgfqpoint{0.384012in}{0.455055in}}%
\pgfpathlineto{\pgfqpoint{0.416667in}{0.472222in}}%
\pgfpathlineto{\pgfqpoint{0.449321in}{0.455055in}}%
\pgfpathlineto{\pgfqpoint{0.443085in}{0.491416in}}%
\pgfpathlineto{\pgfqpoint{0.469503in}{0.517168in}}%
\pgfpathlineto{\pgfqpoint{0.432994in}{0.522473in}}%
\pgfpathlineto{\pgfqpoint{0.416667in}{0.555556in}}%
\pgfpathmoveto{\pgfqpoint{0.583333in}{0.555556in}}%
\pgfpathlineto{\pgfqpoint{0.567006in}{0.522473in}}%
\pgfpathlineto{\pgfqpoint{0.530497in}{0.517168in}}%
\pgfpathlineto{\pgfqpoint{0.556915in}{0.491416in}}%
\pgfpathlineto{\pgfqpoint{0.550679in}{0.455055in}}%
\pgfpathlineto{\pgfqpoint{0.583333in}{0.472222in}}%
\pgfpathlineto{\pgfqpoint{0.615988in}{0.455055in}}%
\pgfpathlineto{\pgfqpoint{0.609752in}{0.491416in}}%
\pgfpathlineto{\pgfqpoint{0.636170in}{0.517168in}}%
\pgfpathlineto{\pgfqpoint{0.599661in}{0.522473in}}%
\pgfpathlineto{\pgfqpoint{0.583333in}{0.555556in}}%
\pgfpathmoveto{\pgfqpoint{0.750000in}{0.555556in}}%
\pgfpathlineto{\pgfqpoint{0.733673in}{0.522473in}}%
\pgfpathlineto{\pgfqpoint{0.697164in}{0.517168in}}%
\pgfpathlineto{\pgfqpoint{0.723582in}{0.491416in}}%
\pgfpathlineto{\pgfqpoint{0.717345in}{0.455055in}}%
\pgfpathlineto{\pgfqpoint{0.750000in}{0.472222in}}%
\pgfpathlineto{\pgfqpoint{0.782655in}{0.455055in}}%
\pgfpathlineto{\pgfqpoint{0.776418in}{0.491416in}}%
\pgfpathlineto{\pgfqpoint{0.802836in}{0.517168in}}%
\pgfpathlineto{\pgfqpoint{0.766327in}{0.522473in}}%
\pgfpathlineto{\pgfqpoint{0.750000in}{0.555556in}}%
\pgfpathmoveto{\pgfqpoint{0.916667in}{0.555556in}}%
\pgfpathlineto{\pgfqpoint{0.900339in}{0.522473in}}%
\pgfpathlineto{\pgfqpoint{0.863830in}{0.517168in}}%
\pgfpathlineto{\pgfqpoint{0.890248in}{0.491416in}}%
\pgfpathlineto{\pgfqpoint{0.884012in}{0.455055in}}%
\pgfpathlineto{\pgfqpoint{0.916667in}{0.472222in}}%
\pgfpathlineto{\pgfqpoint{0.949321in}{0.455055in}}%
\pgfpathlineto{\pgfqpoint{0.943085in}{0.491416in}}%
\pgfpathlineto{\pgfqpoint{0.969503in}{0.517168in}}%
\pgfpathlineto{\pgfqpoint{0.932994in}{0.522473in}}%
\pgfpathlineto{\pgfqpoint{0.916667in}{0.555556in}}%
\pgfpathmoveto{\pgfqpoint{0.000000in}{0.722222in}}%
\pgfpathlineto{\pgfqpoint{-0.016327in}{0.689139in}}%
\pgfpathlineto{\pgfqpoint{-0.052836in}{0.683834in}}%
\pgfpathlineto{\pgfqpoint{-0.026418in}{0.658083in}}%
\pgfpathlineto{\pgfqpoint{-0.032655in}{0.621721in}}%
\pgfpathlineto{\pgfqpoint{-0.000000in}{0.638889in}}%
\pgfpathlineto{\pgfqpoint{0.032655in}{0.621721in}}%
\pgfpathlineto{\pgfqpoint{0.026418in}{0.658083in}}%
\pgfpathlineto{\pgfqpoint{0.052836in}{0.683834in}}%
\pgfpathlineto{\pgfqpoint{0.016327in}{0.689139in}}%
\pgfpathlineto{\pgfqpoint{0.000000in}{0.722222in}}%
\pgfpathmoveto{\pgfqpoint{0.166667in}{0.722222in}}%
\pgfpathlineto{\pgfqpoint{0.150339in}{0.689139in}}%
\pgfpathlineto{\pgfqpoint{0.113830in}{0.683834in}}%
\pgfpathlineto{\pgfqpoint{0.140248in}{0.658083in}}%
\pgfpathlineto{\pgfqpoint{0.134012in}{0.621721in}}%
\pgfpathlineto{\pgfqpoint{0.166667in}{0.638889in}}%
\pgfpathlineto{\pgfqpoint{0.199321in}{0.621721in}}%
\pgfpathlineto{\pgfqpoint{0.193085in}{0.658083in}}%
\pgfpathlineto{\pgfqpoint{0.219503in}{0.683834in}}%
\pgfpathlineto{\pgfqpoint{0.182994in}{0.689139in}}%
\pgfpathlineto{\pgfqpoint{0.166667in}{0.722222in}}%
\pgfpathmoveto{\pgfqpoint{0.333333in}{0.722222in}}%
\pgfpathlineto{\pgfqpoint{0.317006in}{0.689139in}}%
\pgfpathlineto{\pgfqpoint{0.280497in}{0.683834in}}%
\pgfpathlineto{\pgfqpoint{0.306915in}{0.658083in}}%
\pgfpathlineto{\pgfqpoint{0.300679in}{0.621721in}}%
\pgfpathlineto{\pgfqpoint{0.333333in}{0.638889in}}%
\pgfpathlineto{\pgfqpoint{0.365988in}{0.621721in}}%
\pgfpathlineto{\pgfqpoint{0.359752in}{0.658083in}}%
\pgfpathlineto{\pgfqpoint{0.386170in}{0.683834in}}%
\pgfpathlineto{\pgfqpoint{0.349661in}{0.689139in}}%
\pgfpathlineto{\pgfqpoint{0.333333in}{0.722222in}}%
\pgfpathmoveto{\pgfqpoint{0.500000in}{0.722222in}}%
\pgfpathlineto{\pgfqpoint{0.483673in}{0.689139in}}%
\pgfpathlineto{\pgfqpoint{0.447164in}{0.683834in}}%
\pgfpathlineto{\pgfqpoint{0.473582in}{0.658083in}}%
\pgfpathlineto{\pgfqpoint{0.467345in}{0.621721in}}%
\pgfpathlineto{\pgfqpoint{0.500000in}{0.638889in}}%
\pgfpathlineto{\pgfqpoint{0.532655in}{0.621721in}}%
\pgfpathlineto{\pgfqpoint{0.526418in}{0.658083in}}%
\pgfpathlineto{\pgfqpoint{0.552836in}{0.683834in}}%
\pgfpathlineto{\pgfqpoint{0.516327in}{0.689139in}}%
\pgfpathlineto{\pgfqpoint{0.500000in}{0.722222in}}%
\pgfpathmoveto{\pgfqpoint{0.666667in}{0.722222in}}%
\pgfpathlineto{\pgfqpoint{0.650339in}{0.689139in}}%
\pgfpathlineto{\pgfqpoint{0.613830in}{0.683834in}}%
\pgfpathlineto{\pgfqpoint{0.640248in}{0.658083in}}%
\pgfpathlineto{\pgfqpoint{0.634012in}{0.621721in}}%
\pgfpathlineto{\pgfqpoint{0.666667in}{0.638889in}}%
\pgfpathlineto{\pgfqpoint{0.699321in}{0.621721in}}%
\pgfpathlineto{\pgfqpoint{0.693085in}{0.658083in}}%
\pgfpathlineto{\pgfqpoint{0.719503in}{0.683834in}}%
\pgfpathlineto{\pgfqpoint{0.682994in}{0.689139in}}%
\pgfpathlineto{\pgfqpoint{0.666667in}{0.722222in}}%
\pgfpathmoveto{\pgfqpoint{0.833333in}{0.722222in}}%
\pgfpathlineto{\pgfqpoint{0.817006in}{0.689139in}}%
\pgfpathlineto{\pgfqpoint{0.780497in}{0.683834in}}%
\pgfpathlineto{\pgfqpoint{0.806915in}{0.658083in}}%
\pgfpathlineto{\pgfqpoint{0.800679in}{0.621721in}}%
\pgfpathlineto{\pgfqpoint{0.833333in}{0.638889in}}%
\pgfpathlineto{\pgfqpoint{0.865988in}{0.621721in}}%
\pgfpathlineto{\pgfqpoint{0.859752in}{0.658083in}}%
\pgfpathlineto{\pgfqpoint{0.886170in}{0.683834in}}%
\pgfpathlineto{\pgfqpoint{0.849661in}{0.689139in}}%
\pgfpathlineto{\pgfqpoint{0.833333in}{0.722222in}}%
\pgfpathmoveto{\pgfqpoint{1.000000in}{0.722222in}}%
\pgfpathlineto{\pgfqpoint{0.983673in}{0.689139in}}%
\pgfpathlineto{\pgfqpoint{0.947164in}{0.683834in}}%
\pgfpathlineto{\pgfqpoint{0.973582in}{0.658083in}}%
\pgfpathlineto{\pgfqpoint{0.967345in}{0.621721in}}%
\pgfpathlineto{\pgfqpoint{1.000000in}{0.638889in}}%
\pgfpathlineto{\pgfqpoint{1.032655in}{0.621721in}}%
\pgfpathlineto{\pgfqpoint{1.026418in}{0.658083in}}%
\pgfpathlineto{\pgfqpoint{1.052836in}{0.683834in}}%
\pgfpathlineto{\pgfqpoint{1.016327in}{0.689139in}}%
\pgfpathlineto{\pgfqpoint{1.000000in}{0.722222in}}%
\pgfpathmoveto{\pgfqpoint{0.083333in}{0.888889in}}%
\pgfpathlineto{\pgfqpoint{0.067006in}{0.855806in}}%
\pgfpathlineto{\pgfqpoint{0.030497in}{0.850501in}}%
\pgfpathlineto{\pgfqpoint{0.056915in}{0.824750in}}%
\pgfpathlineto{\pgfqpoint{0.050679in}{0.788388in}}%
\pgfpathlineto{\pgfqpoint{0.083333in}{0.805556in}}%
\pgfpathlineto{\pgfqpoint{0.115988in}{0.788388in}}%
\pgfpathlineto{\pgfqpoint{0.109752in}{0.824750in}}%
\pgfpathlineto{\pgfqpoint{0.136170in}{0.850501in}}%
\pgfpathlineto{\pgfqpoint{0.099661in}{0.855806in}}%
\pgfpathlineto{\pgfqpoint{0.083333in}{0.888889in}}%
\pgfpathmoveto{\pgfqpoint{0.250000in}{0.888889in}}%
\pgfpathlineto{\pgfqpoint{0.233673in}{0.855806in}}%
\pgfpathlineto{\pgfqpoint{0.197164in}{0.850501in}}%
\pgfpathlineto{\pgfqpoint{0.223582in}{0.824750in}}%
\pgfpathlineto{\pgfqpoint{0.217345in}{0.788388in}}%
\pgfpathlineto{\pgfqpoint{0.250000in}{0.805556in}}%
\pgfpathlineto{\pgfqpoint{0.282655in}{0.788388in}}%
\pgfpathlineto{\pgfqpoint{0.276418in}{0.824750in}}%
\pgfpathlineto{\pgfqpoint{0.302836in}{0.850501in}}%
\pgfpathlineto{\pgfqpoint{0.266327in}{0.855806in}}%
\pgfpathlineto{\pgfqpoint{0.250000in}{0.888889in}}%
\pgfpathmoveto{\pgfqpoint{0.416667in}{0.888889in}}%
\pgfpathlineto{\pgfqpoint{0.400339in}{0.855806in}}%
\pgfpathlineto{\pgfqpoint{0.363830in}{0.850501in}}%
\pgfpathlineto{\pgfqpoint{0.390248in}{0.824750in}}%
\pgfpathlineto{\pgfqpoint{0.384012in}{0.788388in}}%
\pgfpathlineto{\pgfqpoint{0.416667in}{0.805556in}}%
\pgfpathlineto{\pgfqpoint{0.449321in}{0.788388in}}%
\pgfpathlineto{\pgfqpoint{0.443085in}{0.824750in}}%
\pgfpathlineto{\pgfqpoint{0.469503in}{0.850501in}}%
\pgfpathlineto{\pgfqpoint{0.432994in}{0.855806in}}%
\pgfpathlineto{\pgfqpoint{0.416667in}{0.888889in}}%
\pgfpathmoveto{\pgfqpoint{0.583333in}{0.888889in}}%
\pgfpathlineto{\pgfqpoint{0.567006in}{0.855806in}}%
\pgfpathlineto{\pgfqpoint{0.530497in}{0.850501in}}%
\pgfpathlineto{\pgfqpoint{0.556915in}{0.824750in}}%
\pgfpathlineto{\pgfqpoint{0.550679in}{0.788388in}}%
\pgfpathlineto{\pgfqpoint{0.583333in}{0.805556in}}%
\pgfpathlineto{\pgfqpoint{0.615988in}{0.788388in}}%
\pgfpathlineto{\pgfqpoint{0.609752in}{0.824750in}}%
\pgfpathlineto{\pgfqpoint{0.636170in}{0.850501in}}%
\pgfpathlineto{\pgfqpoint{0.599661in}{0.855806in}}%
\pgfpathlineto{\pgfqpoint{0.583333in}{0.888889in}}%
\pgfpathmoveto{\pgfqpoint{0.750000in}{0.888889in}}%
\pgfpathlineto{\pgfqpoint{0.733673in}{0.855806in}}%
\pgfpathlineto{\pgfqpoint{0.697164in}{0.850501in}}%
\pgfpathlineto{\pgfqpoint{0.723582in}{0.824750in}}%
\pgfpathlineto{\pgfqpoint{0.717345in}{0.788388in}}%
\pgfpathlineto{\pgfqpoint{0.750000in}{0.805556in}}%
\pgfpathlineto{\pgfqpoint{0.782655in}{0.788388in}}%
\pgfpathlineto{\pgfqpoint{0.776418in}{0.824750in}}%
\pgfpathlineto{\pgfqpoint{0.802836in}{0.850501in}}%
\pgfpathlineto{\pgfqpoint{0.766327in}{0.855806in}}%
\pgfpathlineto{\pgfqpoint{0.750000in}{0.888889in}}%
\pgfpathmoveto{\pgfqpoint{0.916667in}{0.888889in}}%
\pgfpathlineto{\pgfqpoint{0.900339in}{0.855806in}}%
\pgfpathlineto{\pgfqpoint{0.863830in}{0.850501in}}%
\pgfpathlineto{\pgfqpoint{0.890248in}{0.824750in}}%
\pgfpathlineto{\pgfqpoint{0.884012in}{0.788388in}}%
\pgfpathlineto{\pgfqpoint{0.916667in}{0.805556in}}%
\pgfpathlineto{\pgfqpoint{0.949321in}{0.788388in}}%
\pgfpathlineto{\pgfqpoint{0.943085in}{0.824750in}}%
\pgfpathlineto{\pgfqpoint{0.969503in}{0.850501in}}%
\pgfpathlineto{\pgfqpoint{0.932994in}{0.855806in}}%
\pgfpathlineto{\pgfqpoint{0.916667in}{0.888889in}}%
\pgfpathmoveto{\pgfqpoint{0.000000in}{1.055556in}}%
\pgfpathlineto{\pgfqpoint{-0.016327in}{1.022473in}}%
\pgfpathlineto{\pgfqpoint{-0.052836in}{1.017168in}}%
\pgfpathlineto{\pgfqpoint{-0.026418in}{0.991416in}}%
\pgfpathlineto{\pgfqpoint{-0.032655in}{0.955055in}}%
\pgfpathlineto{\pgfqpoint{-0.000000in}{0.972222in}}%
\pgfpathlineto{\pgfqpoint{0.032655in}{0.955055in}}%
\pgfpathlineto{\pgfqpoint{0.026418in}{0.991416in}}%
\pgfpathlineto{\pgfqpoint{0.052836in}{1.017168in}}%
\pgfpathlineto{\pgfqpoint{0.016327in}{1.022473in}}%
\pgfpathlineto{\pgfqpoint{0.000000in}{1.055556in}}%
\pgfpathmoveto{\pgfqpoint{0.166667in}{1.055556in}}%
\pgfpathlineto{\pgfqpoint{0.150339in}{1.022473in}}%
\pgfpathlineto{\pgfqpoint{0.113830in}{1.017168in}}%
\pgfpathlineto{\pgfqpoint{0.140248in}{0.991416in}}%
\pgfpathlineto{\pgfqpoint{0.134012in}{0.955055in}}%
\pgfpathlineto{\pgfqpoint{0.166667in}{0.972222in}}%
\pgfpathlineto{\pgfqpoint{0.199321in}{0.955055in}}%
\pgfpathlineto{\pgfqpoint{0.193085in}{0.991416in}}%
\pgfpathlineto{\pgfqpoint{0.219503in}{1.017168in}}%
\pgfpathlineto{\pgfqpoint{0.182994in}{1.022473in}}%
\pgfpathlineto{\pgfqpoint{0.166667in}{1.055556in}}%
\pgfpathmoveto{\pgfqpoint{0.333333in}{1.055556in}}%
\pgfpathlineto{\pgfqpoint{0.317006in}{1.022473in}}%
\pgfpathlineto{\pgfqpoint{0.280497in}{1.017168in}}%
\pgfpathlineto{\pgfqpoint{0.306915in}{0.991416in}}%
\pgfpathlineto{\pgfqpoint{0.300679in}{0.955055in}}%
\pgfpathlineto{\pgfqpoint{0.333333in}{0.972222in}}%
\pgfpathlineto{\pgfqpoint{0.365988in}{0.955055in}}%
\pgfpathlineto{\pgfqpoint{0.359752in}{0.991416in}}%
\pgfpathlineto{\pgfqpoint{0.386170in}{1.017168in}}%
\pgfpathlineto{\pgfqpoint{0.349661in}{1.022473in}}%
\pgfpathlineto{\pgfqpoint{0.333333in}{1.055556in}}%
\pgfpathmoveto{\pgfqpoint{0.500000in}{1.055556in}}%
\pgfpathlineto{\pgfqpoint{0.483673in}{1.022473in}}%
\pgfpathlineto{\pgfqpoint{0.447164in}{1.017168in}}%
\pgfpathlineto{\pgfqpoint{0.473582in}{0.991416in}}%
\pgfpathlineto{\pgfqpoint{0.467345in}{0.955055in}}%
\pgfpathlineto{\pgfqpoint{0.500000in}{0.972222in}}%
\pgfpathlineto{\pgfqpoint{0.532655in}{0.955055in}}%
\pgfpathlineto{\pgfqpoint{0.526418in}{0.991416in}}%
\pgfpathlineto{\pgfqpoint{0.552836in}{1.017168in}}%
\pgfpathlineto{\pgfqpoint{0.516327in}{1.022473in}}%
\pgfpathlineto{\pgfqpoint{0.500000in}{1.055556in}}%
\pgfpathmoveto{\pgfqpoint{0.666667in}{1.055556in}}%
\pgfpathlineto{\pgfqpoint{0.650339in}{1.022473in}}%
\pgfpathlineto{\pgfqpoint{0.613830in}{1.017168in}}%
\pgfpathlineto{\pgfqpoint{0.640248in}{0.991416in}}%
\pgfpathlineto{\pgfqpoint{0.634012in}{0.955055in}}%
\pgfpathlineto{\pgfqpoint{0.666667in}{0.972222in}}%
\pgfpathlineto{\pgfqpoint{0.699321in}{0.955055in}}%
\pgfpathlineto{\pgfqpoint{0.693085in}{0.991416in}}%
\pgfpathlineto{\pgfqpoint{0.719503in}{1.017168in}}%
\pgfpathlineto{\pgfqpoint{0.682994in}{1.022473in}}%
\pgfpathlineto{\pgfqpoint{0.666667in}{1.055556in}}%
\pgfpathmoveto{\pgfqpoint{0.833333in}{1.055556in}}%
\pgfpathlineto{\pgfqpoint{0.817006in}{1.022473in}}%
\pgfpathlineto{\pgfqpoint{0.780497in}{1.017168in}}%
\pgfpathlineto{\pgfqpoint{0.806915in}{0.991416in}}%
\pgfpathlineto{\pgfqpoint{0.800679in}{0.955055in}}%
\pgfpathlineto{\pgfqpoint{0.833333in}{0.972222in}}%
\pgfpathlineto{\pgfqpoint{0.865988in}{0.955055in}}%
\pgfpathlineto{\pgfqpoint{0.859752in}{0.991416in}}%
\pgfpathlineto{\pgfqpoint{0.886170in}{1.017168in}}%
\pgfpathlineto{\pgfqpoint{0.849661in}{1.022473in}}%
\pgfpathlineto{\pgfqpoint{0.833333in}{1.055556in}}%
\pgfpathmoveto{\pgfqpoint{1.000000in}{1.055556in}}%
\pgfpathlineto{\pgfqpoint{0.983673in}{1.022473in}}%
\pgfpathlineto{\pgfqpoint{0.947164in}{1.017168in}}%
\pgfpathlineto{\pgfqpoint{0.973582in}{0.991416in}}%
\pgfpathlineto{\pgfqpoint{0.967345in}{0.955055in}}%
\pgfpathlineto{\pgfqpoint{1.000000in}{0.972222in}}%
\pgfpathlineto{\pgfqpoint{1.032655in}{0.955055in}}%
\pgfpathlineto{\pgfqpoint{1.026418in}{0.991416in}}%
\pgfpathlineto{\pgfqpoint{1.052836in}{1.017168in}}%
\pgfpathlineto{\pgfqpoint{1.016327in}{1.022473in}}%
\pgfpathlineto{\pgfqpoint{1.000000in}{1.055556in}}%
\pgfpathlineto{\pgfqpoint{1.000000in}{1.055556in}}%
\pgfusepath{stroke}%
\end{pgfscope}%
}%
\pgfsys@transformshift{1.135815in}{8.032990in}%
\pgfsys@useobject{currentpattern}{}%
\pgfsys@transformshift{1in}{0in}%
\pgfsys@transformshift{-1in}{0in}%
\pgfsys@transformshift{0in}{1in}%
\end{pgfscope}%
\begin{pgfscope}%
\definecolor{textcolor}{rgb}{0.000000,0.000000,0.000000}%
\pgfsetstrokecolor{textcolor}%
\pgfsetfillcolor{textcolor}%
\pgftext[x=1.758037in,y=8.032990in,left,base]{\color{textcolor}\rmfamily\fontsize{16.000000}{19.200000}\selectfont SOLAR\_FARM}%
\end{pgfscope}%
\begin{pgfscope}%
\pgfsetbuttcap%
\pgfsetmiterjoin%
\definecolor{currentfill}{rgb}{0.121569,0.466667,0.705882}%
\pgfsetfillcolor{currentfill}%
\pgfsetfillopacity{0.990000}%
\pgfsetlinewidth{0.000000pt}%
\definecolor{currentstroke}{rgb}{0.000000,0.000000,0.000000}%
\pgfsetstrokecolor{currentstroke}%
\pgfsetstrokeopacity{0.990000}%
\pgfsetdash{}{0pt}%
\pgfpathmoveto{\pgfqpoint{1.135815in}{7.708530in}}%
\pgfpathlineto{\pgfqpoint{1.580259in}{7.708530in}}%
\pgfpathlineto{\pgfqpoint{1.580259in}{7.864085in}}%
\pgfpathlineto{\pgfqpoint{1.135815in}{7.864085in}}%
\pgfpathclose%
\pgfusepath{fill}%
\end{pgfscope}%
\begin{pgfscope}%
\pgfsetbuttcap%
\pgfsetmiterjoin%
\definecolor{currentfill}{rgb}{0.121569,0.466667,0.705882}%
\pgfsetfillcolor{currentfill}%
\pgfsetfillopacity{0.990000}%
\pgfsetlinewidth{0.000000pt}%
\definecolor{currentstroke}{rgb}{0.000000,0.000000,0.000000}%
\pgfsetstrokecolor{currentstroke}%
\pgfsetstrokeopacity{0.990000}%
\pgfsetdash{}{0pt}%
\pgfpathmoveto{\pgfqpoint{1.135815in}{7.708530in}}%
\pgfpathlineto{\pgfqpoint{1.580259in}{7.708530in}}%
\pgfpathlineto{\pgfqpoint{1.580259in}{7.864085in}}%
\pgfpathlineto{\pgfqpoint{1.135815in}{7.864085in}}%
\pgfpathclose%
\pgfusepath{clip}%
\pgfsys@defobject{currentpattern}{\pgfqpoint{0in}{0in}}{\pgfqpoint{1in}{1in}}{%
\begin{pgfscope}%
\pgfpathrectangle{\pgfqpoint{0in}{0in}}{\pgfqpoint{1in}{1in}}%
\pgfusepath{clip}%
\pgfpathmoveto{\pgfqpoint{0.000000in}{0.083333in}}%
\pgfpathlineto{\pgfqpoint{1.000000in}{0.083333in}}%
\pgfpathmoveto{\pgfqpoint{0.000000in}{0.250000in}}%
\pgfpathlineto{\pgfqpoint{1.000000in}{0.250000in}}%
\pgfpathmoveto{\pgfqpoint{0.000000in}{0.416667in}}%
\pgfpathlineto{\pgfqpoint{1.000000in}{0.416667in}}%
\pgfpathmoveto{\pgfqpoint{0.000000in}{0.583333in}}%
\pgfpathlineto{\pgfqpoint{1.000000in}{0.583333in}}%
\pgfpathmoveto{\pgfqpoint{0.000000in}{0.750000in}}%
\pgfpathlineto{\pgfqpoint{1.000000in}{0.750000in}}%
\pgfpathmoveto{\pgfqpoint{0.000000in}{0.916667in}}%
\pgfpathlineto{\pgfqpoint{1.000000in}{0.916667in}}%
\pgfpathmoveto{\pgfqpoint{0.083333in}{0.000000in}}%
\pgfpathlineto{\pgfqpoint{0.083333in}{1.000000in}}%
\pgfpathmoveto{\pgfqpoint{0.250000in}{0.000000in}}%
\pgfpathlineto{\pgfqpoint{0.250000in}{1.000000in}}%
\pgfpathmoveto{\pgfqpoint{0.416667in}{0.000000in}}%
\pgfpathlineto{\pgfqpoint{0.416667in}{1.000000in}}%
\pgfpathmoveto{\pgfqpoint{0.583333in}{0.000000in}}%
\pgfpathlineto{\pgfqpoint{0.583333in}{1.000000in}}%
\pgfpathmoveto{\pgfqpoint{0.750000in}{0.000000in}}%
\pgfpathlineto{\pgfqpoint{0.750000in}{1.000000in}}%
\pgfpathmoveto{\pgfqpoint{0.916667in}{0.000000in}}%
\pgfpathlineto{\pgfqpoint{0.916667in}{1.000000in}}%
\pgfusepath{stroke}%
\end{pgfscope}%
}%
\pgfsys@transformshift{1.135815in}{7.708530in}%
\pgfsys@useobject{currentpattern}{}%
\pgfsys@transformshift{1in}{0in}%
\pgfsys@transformshift{-1in}{0in}%
\pgfsys@transformshift{0in}{1in}%
\end{pgfscope}%
\begin{pgfscope}%
\definecolor{textcolor}{rgb}{0.000000,0.000000,0.000000}%
\pgfsetstrokecolor{textcolor}%
\pgfsetfillcolor{textcolor}%
\pgftext[x=1.758037in,y=7.708530in,left,base]{\color{textcolor}\rmfamily\fontsize{16.000000}{19.200000}\selectfont WIND\_FARM}%
\end{pgfscope}%
\end{pgfpicture}%
\makeatother%
\endgroup%
}
    \caption[]{Least cost capacity expansion.}
    \label{fig:uiuc_elc_cap}
  \end{minipage}
  \begin{minipage}{0.48\textwidth}
    \centering
    \resizebox{\columnwidth}{!}{%% Creator: Matplotlib, PGF backend
%%
%% To include the figure in your LaTeX document, write
%%   \input{<filename>.pgf}
%%
%% Make sure the required packages are loaded in your preamble
%%   \usepackage{pgf}
%%
%% Figures using additional raster images can only be included by \input if
%% they are in the same directory as the main LaTeX file. For loading figures
%% from other directories you can use the `import` package
%%   \usepackage{import}
%%
%% and then include the figures with
%%   \import{<path to file>}{<filename>.pgf}
%%
%% Matplotlib used the following preamble
%%
\begingroup%
\makeatletter%
\begin{pgfpicture}%
\pgfpathrectangle{\pgfpointorigin}{\pgfqpoint{10.270538in}{10.120798in}}%
\pgfusepath{use as bounding box, clip}%
\begin{pgfscope}%
\pgfsetbuttcap%
\pgfsetmiterjoin%
\definecolor{currentfill}{rgb}{1.000000,1.000000,1.000000}%
\pgfsetfillcolor{currentfill}%
\pgfsetlinewidth{0.000000pt}%
\definecolor{currentstroke}{rgb}{0.000000,0.000000,0.000000}%
\pgfsetstrokecolor{currentstroke}%
\pgfsetdash{}{0pt}%
\pgfpathmoveto{\pgfqpoint{0.000000in}{0.000000in}}%
\pgfpathlineto{\pgfqpoint{10.270538in}{0.000000in}}%
\pgfpathlineto{\pgfqpoint{10.270538in}{10.120798in}}%
\pgfpathlineto{\pgfqpoint{0.000000in}{10.120798in}}%
\pgfpathclose%
\pgfusepath{fill}%
\end{pgfscope}%
\begin{pgfscope}%
\pgfsetbuttcap%
\pgfsetmiterjoin%
\definecolor{currentfill}{rgb}{0.898039,0.898039,0.898039}%
\pgfsetfillcolor{currentfill}%
\pgfsetlinewidth{0.000000pt}%
\definecolor{currentstroke}{rgb}{0.000000,0.000000,0.000000}%
\pgfsetstrokecolor{currentstroke}%
\pgfsetstrokeopacity{0.000000}%
\pgfsetdash{}{0pt}%
\pgfpathmoveto{\pgfqpoint{0.870538in}{0.637495in}}%
\pgfpathlineto{\pgfqpoint{10.170538in}{0.637495in}}%
\pgfpathlineto{\pgfqpoint{10.170538in}{9.697495in}}%
\pgfpathlineto{\pgfqpoint{0.870538in}{9.697495in}}%
\pgfpathclose%
\pgfusepath{fill}%
\end{pgfscope}%
\begin{pgfscope}%
\pgfpathrectangle{\pgfqpoint{0.870538in}{0.637495in}}{\pgfqpoint{9.300000in}{9.060000in}}%
\pgfusepath{clip}%
\pgfsetrectcap%
\pgfsetroundjoin%
\pgfsetlinewidth{0.803000pt}%
\definecolor{currentstroke}{rgb}{1.000000,1.000000,1.000000}%
\pgfsetstrokecolor{currentstroke}%
\pgfsetdash{}{0pt}%
\pgfpathmoveto{\pgfqpoint{1.645538in}{0.637495in}}%
\pgfpathlineto{\pgfqpoint{1.645538in}{9.697495in}}%
\pgfusepath{stroke}%
\end{pgfscope}%
\begin{pgfscope}%
\pgfsetbuttcap%
\pgfsetroundjoin%
\definecolor{currentfill}{rgb}{0.333333,0.333333,0.333333}%
\pgfsetfillcolor{currentfill}%
\pgfsetlinewidth{0.803000pt}%
\definecolor{currentstroke}{rgb}{0.333333,0.333333,0.333333}%
\pgfsetstrokecolor{currentstroke}%
\pgfsetdash{}{0pt}%
\pgfsys@defobject{currentmarker}{\pgfqpoint{0.000000in}{-0.048611in}}{\pgfqpoint{0.000000in}{0.000000in}}{%
\pgfpathmoveto{\pgfqpoint{0.000000in}{0.000000in}}%
\pgfpathlineto{\pgfqpoint{0.000000in}{-0.048611in}}%
\pgfusepath{stroke,fill}%
}%
\begin{pgfscope}%
\pgfsys@transformshift{1.645538in}{0.637495in}%
\pgfsys@useobject{currentmarker}{}%
\end{pgfscope}%
\end{pgfscope}%
\begin{pgfscope}%
\definecolor{textcolor}{rgb}{0.333333,0.333333,0.333333}%
\pgfsetstrokecolor{textcolor}%
\pgfsetfillcolor{textcolor}%
\pgftext[x=1.705530in, y=0.100000in, left, base,rotate=90.000000]{\color{textcolor}\rmfamily\fontsize{16.000000}{19.200000}\selectfont 2025}%
\end{pgfscope}%
\begin{pgfscope}%
\pgfpathrectangle{\pgfqpoint{0.870538in}{0.637495in}}{\pgfqpoint{9.300000in}{9.060000in}}%
\pgfusepath{clip}%
\pgfsetrectcap%
\pgfsetroundjoin%
\pgfsetlinewidth{0.803000pt}%
\definecolor{currentstroke}{rgb}{1.000000,1.000000,1.000000}%
\pgfsetstrokecolor{currentstroke}%
\pgfsetdash{}{0pt}%
\pgfpathmoveto{\pgfqpoint{3.195538in}{0.637495in}}%
\pgfpathlineto{\pgfqpoint{3.195538in}{9.697495in}}%
\pgfusepath{stroke}%
\end{pgfscope}%
\begin{pgfscope}%
\pgfsetbuttcap%
\pgfsetroundjoin%
\definecolor{currentfill}{rgb}{0.333333,0.333333,0.333333}%
\pgfsetfillcolor{currentfill}%
\pgfsetlinewidth{0.803000pt}%
\definecolor{currentstroke}{rgb}{0.333333,0.333333,0.333333}%
\pgfsetstrokecolor{currentstroke}%
\pgfsetdash{}{0pt}%
\pgfsys@defobject{currentmarker}{\pgfqpoint{0.000000in}{-0.048611in}}{\pgfqpoint{0.000000in}{0.000000in}}{%
\pgfpathmoveto{\pgfqpoint{0.000000in}{0.000000in}}%
\pgfpathlineto{\pgfqpoint{0.000000in}{-0.048611in}}%
\pgfusepath{stroke,fill}%
}%
\begin{pgfscope}%
\pgfsys@transformshift{3.195538in}{0.637495in}%
\pgfsys@useobject{currentmarker}{}%
\end{pgfscope}%
\end{pgfscope}%
\begin{pgfscope}%
\definecolor{textcolor}{rgb}{0.333333,0.333333,0.333333}%
\pgfsetstrokecolor{textcolor}%
\pgfsetfillcolor{textcolor}%
\pgftext[x=3.255530in, y=0.100000in, left, base,rotate=90.000000]{\color{textcolor}\rmfamily\fontsize{16.000000}{19.200000}\selectfont 2030}%
\end{pgfscope}%
\begin{pgfscope}%
\pgfpathrectangle{\pgfqpoint{0.870538in}{0.637495in}}{\pgfqpoint{9.300000in}{9.060000in}}%
\pgfusepath{clip}%
\pgfsetrectcap%
\pgfsetroundjoin%
\pgfsetlinewidth{0.803000pt}%
\definecolor{currentstroke}{rgb}{1.000000,1.000000,1.000000}%
\pgfsetstrokecolor{currentstroke}%
\pgfsetdash{}{0pt}%
\pgfpathmoveto{\pgfqpoint{4.745538in}{0.637495in}}%
\pgfpathlineto{\pgfqpoint{4.745538in}{9.697495in}}%
\pgfusepath{stroke}%
\end{pgfscope}%
\begin{pgfscope}%
\pgfsetbuttcap%
\pgfsetroundjoin%
\definecolor{currentfill}{rgb}{0.333333,0.333333,0.333333}%
\pgfsetfillcolor{currentfill}%
\pgfsetlinewidth{0.803000pt}%
\definecolor{currentstroke}{rgb}{0.333333,0.333333,0.333333}%
\pgfsetstrokecolor{currentstroke}%
\pgfsetdash{}{0pt}%
\pgfsys@defobject{currentmarker}{\pgfqpoint{0.000000in}{-0.048611in}}{\pgfqpoint{0.000000in}{0.000000in}}{%
\pgfpathmoveto{\pgfqpoint{0.000000in}{0.000000in}}%
\pgfpathlineto{\pgfqpoint{0.000000in}{-0.048611in}}%
\pgfusepath{stroke,fill}%
}%
\begin{pgfscope}%
\pgfsys@transformshift{4.745538in}{0.637495in}%
\pgfsys@useobject{currentmarker}{}%
\end{pgfscope}%
\end{pgfscope}%
\begin{pgfscope}%
\definecolor{textcolor}{rgb}{0.333333,0.333333,0.333333}%
\pgfsetstrokecolor{textcolor}%
\pgfsetfillcolor{textcolor}%
\pgftext[x=4.805530in, y=0.100000in, left, base,rotate=90.000000]{\color{textcolor}\rmfamily\fontsize{16.000000}{19.200000}\selectfont 2035}%
\end{pgfscope}%
\begin{pgfscope}%
\pgfpathrectangle{\pgfqpoint{0.870538in}{0.637495in}}{\pgfqpoint{9.300000in}{9.060000in}}%
\pgfusepath{clip}%
\pgfsetrectcap%
\pgfsetroundjoin%
\pgfsetlinewidth{0.803000pt}%
\definecolor{currentstroke}{rgb}{1.000000,1.000000,1.000000}%
\pgfsetstrokecolor{currentstroke}%
\pgfsetdash{}{0pt}%
\pgfpathmoveto{\pgfqpoint{6.295538in}{0.637495in}}%
\pgfpathlineto{\pgfqpoint{6.295538in}{9.697495in}}%
\pgfusepath{stroke}%
\end{pgfscope}%
\begin{pgfscope}%
\pgfsetbuttcap%
\pgfsetroundjoin%
\definecolor{currentfill}{rgb}{0.333333,0.333333,0.333333}%
\pgfsetfillcolor{currentfill}%
\pgfsetlinewidth{0.803000pt}%
\definecolor{currentstroke}{rgb}{0.333333,0.333333,0.333333}%
\pgfsetstrokecolor{currentstroke}%
\pgfsetdash{}{0pt}%
\pgfsys@defobject{currentmarker}{\pgfqpoint{0.000000in}{-0.048611in}}{\pgfqpoint{0.000000in}{0.000000in}}{%
\pgfpathmoveto{\pgfqpoint{0.000000in}{0.000000in}}%
\pgfpathlineto{\pgfqpoint{0.000000in}{-0.048611in}}%
\pgfusepath{stroke,fill}%
}%
\begin{pgfscope}%
\pgfsys@transformshift{6.295538in}{0.637495in}%
\pgfsys@useobject{currentmarker}{}%
\end{pgfscope}%
\end{pgfscope}%
\begin{pgfscope}%
\definecolor{textcolor}{rgb}{0.333333,0.333333,0.333333}%
\pgfsetstrokecolor{textcolor}%
\pgfsetfillcolor{textcolor}%
\pgftext[x=6.355530in, y=0.100000in, left, base,rotate=90.000000]{\color{textcolor}\rmfamily\fontsize{16.000000}{19.200000}\selectfont 2040}%
\end{pgfscope}%
\begin{pgfscope}%
\pgfpathrectangle{\pgfqpoint{0.870538in}{0.637495in}}{\pgfqpoint{9.300000in}{9.060000in}}%
\pgfusepath{clip}%
\pgfsetrectcap%
\pgfsetroundjoin%
\pgfsetlinewidth{0.803000pt}%
\definecolor{currentstroke}{rgb}{1.000000,1.000000,1.000000}%
\pgfsetstrokecolor{currentstroke}%
\pgfsetdash{}{0pt}%
\pgfpathmoveto{\pgfqpoint{7.845538in}{0.637495in}}%
\pgfpathlineto{\pgfqpoint{7.845538in}{9.697495in}}%
\pgfusepath{stroke}%
\end{pgfscope}%
\begin{pgfscope}%
\pgfsetbuttcap%
\pgfsetroundjoin%
\definecolor{currentfill}{rgb}{0.333333,0.333333,0.333333}%
\pgfsetfillcolor{currentfill}%
\pgfsetlinewidth{0.803000pt}%
\definecolor{currentstroke}{rgb}{0.333333,0.333333,0.333333}%
\pgfsetstrokecolor{currentstroke}%
\pgfsetdash{}{0pt}%
\pgfsys@defobject{currentmarker}{\pgfqpoint{0.000000in}{-0.048611in}}{\pgfqpoint{0.000000in}{0.000000in}}{%
\pgfpathmoveto{\pgfqpoint{0.000000in}{0.000000in}}%
\pgfpathlineto{\pgfqpoint{0.000000in}{-0.048611in}}%
\pgfusepath{stroke,fill}%
}%
\begin{pgfscope}%
\pgfsys@transformshift{7.845538in}{0.637495in}%
\pgfsys@useobject{currentmarker}{}%
\end{pgfscope}%
\end{pgfscope}%
\begin{pgfscope}%
\definecolor{textcolor}{rgb}{0.333333,0.333333,0.333333}%
\pgfsetstrokecolor{textcolor}%
\pgfsetfillcolor{textcolor}%
\pgftext[x=7.905530in, y=0.100000in, left, base,rotate=90.000000]{\color{textcolor}\rmfamily\fontsize{16.000000}{19.200000}\selectfont 2045}%
\end{pgfscope}%
\begin{pgfscope}%
\pgfpathrectangle{\pgfqpoint{0.870538in}{0.637495in}}{\pgfqpoint{9.300000in}{9.060000in}}%
\pgfusepath{clip}%
\pgfsetrectcap%
\pgfsetroundjoin%
\pgfsetlinewidth{0.803000pt}%
\definecolor{currentstroke}{rgb}{1.000000,1.000000,1.000000}%
\pgfsetstrokecolor{currentstroke}%
\pgfsetdash{}{0pt}%
\pgfpathmoveto{\pgfqpoint{9.395538in}{0.637495in}}%
\pgfpathlineto{\pgfqpoint{9.395538in}{9.697495in}}%
\pgfusepath{stroke}%
\end{pgfscope}%
\begin{pgfscope}%
\pgfsetbuttcap%
\pgfsetroundjoin%
\definecolor{currentfill}{rgb}{0.333333,0.333333,0.333333}%
\pgfsetfillcolor{currentfill}%
\pgfsetlinewidth{0.803000pt}%
\definecolor{currentstroke}{rgb}{0.333333,0.333333,0.333333}%
\pgfsetstrokecolor{currentstroke}%
\pgfsetdash{}{0pt}%
\pgfsys@defobject{currentmarker}{\pgfqpoint{0.000000in}{-0.048611in}}{\pgfqpoint{0.000000in}{0.000000in}}{%
\pgfpathmoveto{\pgfqpoint{0.000000in}{0.000000in}}%
\pgfpathlineto{\pgfqpoint{0.000000in}{-0.048611in}}%
\pgfusepath{stroke,fill}%
}%
\begin{pgfscope}%
\pgfsys@transformshift{9.395538in}{0.637495in}%
\pgfsys@useobject{currentmarker}{}%
\end{pgfscope}%
\end{pgfscope}%
\begin{pgfscope}%
\definecolor{textcolor}{rgb}{0.333333,0.333333,0.333333}%
\pgfsetstrokecolor{textcolor}%
\pgfsetfillcolor{textcolor}%
\pgftext[x=9.455530in, y=0.100000in, left, base,rotate=90.000000]{\color{textcolor}\rmfamily\fontsize{16.000000}{19.200000}\selectfont 2050}%
\end{pgfscope}%
\begin{pgfscope}%
\pgfpathrectangle{\pgfqpoint{0.870538in}{0.637495in}}{\pgfqpoint{9.300000in}{9.060000in}}%
\pgfusepath{clip}%
\pgfsetrectcap%
\pgfsetroundjoin%
\pgfsetlinewidth{0.803000pt}%
\definecolor{currentstroke}{rgb}{1.000000,1.000000,1.000000}%
\pgfsetstrokecolor{currentstroke}%
\pgfsetdash{}{0pt}%
\pgfpathmoveto{\pgfqpoint{0.870538in}{0.637495in}}%
\pgfpathlineto{\pgfqpoint{10.170538in}{0.637495in}}%
\pgfusepath{stroke}%
\end{pgfscope}%
\begin{pgfscope}%
\pgfsetbuttcap%
\pgfsetroundjoin%
\definecolor{currentfill}{rgb}{0.333333,0.333333,0.333333}%
\pgfsetfillcolor{currentfill}%
\pgfsetlinewidth{0.803000pt}%
\definecolor{currentstroke}{rgb}{0.333333,0.333333,0.333333}%
\pgfsetstrokecolor{currentstroke}%
\pgfsetdash{}{0pt}%
\pgfsys@defobject{currentmarker}{\pgfqpoint{-0.048611in}{0.000000in}}{\pgfqpoint{-0.000000in}{0.000000in}}{%
\pgfpathmoveto{\pgfqpoint{-0.000000in}{0.000000in}}%
\pgfpathlineto{\pgfqpoint{-0.048611in}{0.000000in}}%
\pgfusepath{stroke,fill}%
}%
\begin{pgfscope}%
\pgfsys@transformshift{0.870538in}{0.637495in}%
\pgfsys@useobject{currentmarker}{}%
\end{pgfscope}%
\end{pgfscope}%
\begin{pgfscope}%
\definecolor{textcolor}{rgb}{0.333333,0.333333,0.333333}%
\pgfsetstrokecolor{textcolor}%
\pgfsetfillcolor{textcolor}%
\pgftext[x=0.663247in, y=0.554162in, left, base]{\color{textcolor}\rmfamily\fontsize{16.000000}{19.200000}\selectfont \(\displaystyle {0}\)}%
\end{pgfscope}%
\begin{pgfscope}%
\pgfpathrectangle{\pgfqpoint{0.870538in}{0.637495in}}{\pgfqpoint{9.300000in}{9.060000in}}%
\pgfusepath{clip}%
\pgfsetrectcap%
\pgfsetroundjoin%
\pgfsetlinewidth{0.803000pt}%
\definecolor{currentstroke}{rgb}{1.000000,1.000000,1.000000}%
\pgfsetstrokecolor{currentstroke}%
\pgfsetdash{}{0pt}%
\pgfpathmoveto{\pgfqpoint{0.870538in}{2.009297in}}%
\pgfpathlineto{\pgfqpoint{10.170538in}{2.009297in}}%
\pgfusepath{stroke}%
\end{pgfscope}%
\begin{pgfscope}%
\pgfsetbuttcap%
\pgfsetroundjoin%
\definecolor{currentfill}{rgb}{0.333333,0.333333,0.333333}%
\pgfsetfillcolor{currentfill}%
\pgfsetlinewidth{0.803000pt}%
\definecolor{currentstroke}{rgb}{0.333333,0.333333,0.333333}%
\pgfsetstrokecolor{currentstroke}%
\pgfsetdash{}{0pt}%
\pgfsys@defobject{currentmarker}{\pgfqpoint{-0.048611in}{0.000000in}}{\pgfqpoint{-0.000000in}{0.000000in}}{%
\pgfpathmoveto{\pgfqpoint{-0.000000in}{0.000000in}}%
\pgfpathlineto{\pgfqpoint{-0.048611in}{0.000000in}}%
\pgfusepath{stroke,fill}%
}%
\begin{pgfscope}%
\pgfsys@transformshift{0.870538in}{2.009297in}%
\pgfsys@useobject{currentmarker}{}%
\end{pgfscope}%
\end{pgfscope}%
\begin{pgfscope}%
\definecolor{textcolor}{rgb}{0.333333,0.333333,0.333333}%
\pgfsetstrokecolor{textcolor}%
\pgfsetfillcolor{textcolor}%
\pgftext[x=0.443111in, y=1.925964in, left, base]{\color{textcolor}\rmfamily\fontsize{16.000000}{19.200000}\selectfont \(\displaystyle {100}\)}%
\end{pgfscope}%
\begin{pgfscope}%
\pgfpathrectangle{\pgfqpoint{0.870538in}{0.637495in}}{\pgfqpoint{9.300000in}{9.060000in}}%
\pgfusepath{clip}%
\pgfsetrectcap%
\pgfsetroundjoin%
\pgfsetlinewidth{0.803000pt}%
\definecolor{currentstroke}{rgb}{1.000000,1.000000,1.000000}%
\pgfsetstrokecolor{currentstroke}%
\pgfsetdash{}{0pt}%
\pgfpathmoveto{\pgfqpoint{0.870538in}{3.381099in}}%
\pgfpathlineto{\pgfqpoint{10.170538in}{3.381099in}}%
\pgfusepath{stroke}%
\end{pgfscope}%
\begin{pgfscope}%
\pgfsetbuttcap%
\pgfsetroundjoin%
\definecolor{currentfill}{rgb}{0.333333,0.333333,0.333333}%
\pgfsetfillcolor{currentfill}%
\pgfsetlinewidth{0.803000pt}%
\definecolor{currentstroke}{rgb}{0.333333,0.333333,0.333333}%
\pgfsetstrokecolor{currentstroke}%
\pgfsetdash{}{0pt}%
\pgfsys@defobject{currentmarker}{\pgfqpoint{-0.048611in}{0.000000in}}{\pgfqpoint{-0.000000in}{0.000000in}}{%
\pgfpathmoveto{\pgfqpoint{-0.000000in}{0.000000in}}%
\pgfpathlineto{\pgfqpoint{-0.048611in}{0.000000in}}%
\pgfusepath{stroke,fill}%
}%
\begin{pgfscope}%
\pgfsys@transformshift{0.870538in}{3.381099in}%
\pgfsys@useobject{currentmarker}{}%
\end{pgfscope}%
\end{pgfscope}%
\begin{pgfscope}%
\definecolor{textcolor}{rgb}{0.333333,0.333333,0.333333}%
\pgfsetstrokecolor{textcolor}%
\pgfsetfillcolor{textcolor}%
\pgftext[x=0.443111in, y=3.297766in, left, base]{\color{textcolor}\rmfamily\fontsize{16.000000}{19.200000}\selectfont \(\displaystyle {200}\)}%
\end{pgfscope}%
\begin{pgfscope}%
\pgfpathrectangle{\pgfqpoint{0.870538in}{0.637495in}}{\pgfqpoint{9.300000in}{9.060000in}}%
\pgfusepath{clip}%
\pgfsetrectcap%
\pgfsetroundjoin%
\pgfsetlinewidth{0.803000pt}%
\definecolor{currentstroke}{rgb}{1.000000,1.000000,1.000000}%
\pgfsetstrokecolor{currentstroke}%
\pgfsetdash{}{0pt}%
\pgfpathmoveto{\pgfqpoint{0.870538in}{4.752901in}}%
\pgfpathlineto{\pgfqpoint{10.170538in}{4.752901in}}%
\pgfusepath{stroke}%
\end{pgfscope}%
\begin{pgfscope}%
\pgfsetbuttcap%
\pgfsetroundjoin%
\definecolor{currentfill}{rgb}{0.333333,0.333333,0.333333}%
\pgfsetfillcolor{currentfill}%
\pgfsetlinewidth{0.803000pt}%
\definecolor{currentstroke}{rgb}{0.333333,0.333333,0.333333}%
\pgfsetstrokecolor{currentstroke}%
\pgfsetdash{}{0pt}%
\pgfsys@defobject{currentmarker}{\pgfqpoint{-0.048611in}{0.000000in}}{\pgfqpoint{-0.000000in}{0.000000in}}{%
\pgfpathmoveto{\pgfqpoint{-0.000000in}{0.000000in}}%
\pgfpathlineto{\pgfqpoint{-0.048611in}{0.000000in}}%
\pgfusepath{stroke,fill}%
}%
\begin{pgfscope}%
\pgfsys@transformshift{0.870538in}{4.752901in}%
\pgfsys@useobject{currentmarker}{}%
\end{pgfscope}%
\end{pgfscope}%
\begin{pgfscope}%
\definecolor{textcolor}{rgb}{0.333333,0.333333,0.333333}%
\pgfsetstrokecolor{textcolor}%
\pgfsetfillcolor{textcolor}%
\pgftext[x=0.443111in, y=4.669568in, left, base]{\color{textcolor}\rmfamily\fontsize{16.000000}{19.200000}\selectfont \(\displaystyle {300}\)}%
\end{pgfscope}%
\begin{pgfscope}%
\pgfpathrectangle{\pgfqpoint{0.870538in}{0.637495in}}{\pgfqpoint{9.300000in}{9.060000in}}%
\pgfusepath{clip}%
\pgfsetrectcap%
\pgfsetroundjoin%
\pgfsetlinewidth{0.803000pt}%
\definecolor{currentstroke}{rgb}{1.000000,1.000000,1.000000}%
\pgfsetstrokecolor{currentstroke}%
\pgfsetdash{}{0pt}%
\pgfpathmoveto{\pgfqpoint{0.870538in}{6.124703in}}%
\pgfpathlineto{\pgfqpoint{10.170538in}{6.124703in}}%
\pgfusepath{stroke}%
\end{pgfscope}%
\begin{pgfscope}%
\pgfsetbuttcap%
\pgfsetroundjoin%
\definecolor{currentfill}{rgb}{0.333333,0.333333,0.333333}%
\pgfsetfillcolor{currentfill}%
\pgfsetlinewidth{0.803000pt}%
\definecolor{currentstroke}{rgb}{0.333333,0.333333,0.333333}%
\pgfsetstrokecolor{currentstroke}%
\pgfsetdash{}{0pt}%
\pgfsys@defobject{currentmarker}{\pgfqpoint{-0.048611in}{0.000000in}}{\pgfqpoint{-0.000000in}{0.000000in}}{%
\pgfpathmoveto{\pgfqpoint{-0.000000in}{0.000000in}}%
\pgfpathlineto{\pgfqpoint{-0.048611in}{0.000000in}}%
\pgfusepath{stroke,fill}%
}%
\begin{pgfscope}%
\pgfsys@transformshift{0.870538in}{6.124703in}%
\pgfsys@useobject{currentmarker}{}%
\end{pgfscope}%
\end{pgfscope}%
\begin{pgfscope}%
\definecolor{textcolor}{rgb}{0.333333,0.333333,0.333333}%
\pgfsetstrokecolor{textcolor}%
\pgfsetfillcolor{textcolor}%
\pgftext[x=0.443111in, y=6.041370in, left, base]{\color{textcolor}\rmfamily\fontsize{16.000000}{19.200000}\selectfont \(\displaystyle {400}\)}%
\end{pgfscope}%
\begin{pgfscope}%
\pgfpathrectangle{\pgfqpoint{0.870538in}{0.637495in}}{\pgfqpoint{9.300000in}{9.060000in}}%
\pgfusepath{clip}%
\pgfsetrectcap%
\pgfsetroundjoin%
\pgfsetlinewidth{0.803000pt}%
\definecolor{currentstroke}{rgb}{1.000000,1.000000,1.000000}%
\pgfsetstrokecolor{currentstroke}%
\pgfsetdash{}{0pt}%
\pgfpathmoveto{\pgfqpoint{0.870538in}{7.496505in}}%
\pgfpathlineto{\pgfqpoint{10.170538in}{7.496505in}}%
\pgfusepath{stroke}%
\end{pgfscope}%
\begin{pgfscope}%
\pgfsetbuttcap%
\pgfsetroundjoin%
\definecolor{currentfill}{rgb}{0.333333,0.333333,0.333333}%
\pgfsetfillcolor{currentfill}%
\pgfsetlinewidth{0.803000pt}%
\definecolor{currentstroke}{rgb}{0.333333,0.333333,0.333333}%
\pgfsetstrokecolor{currentstroke}%
\pgfsetdash{}{0pt}%
\pgfsys@defobject{currentmarker}{\pgfqpoint{-0.048611in}{0.000000in}}{\pgfqpoint{-0.000000in}{0.000000in}}{%
\pgfpathmoveto{\pgfqpoint{-0.000000in}{0.000000in}}%
\pgfpathlineto{\pgfqpoint{-0.048611in}{0.000000in}}%
\pgfusepath{stroke,fill}%
}%
\begin{pgfscope}%
\pgfsys@transformshift{0.870538in}{7.496505in}%
\pgfsys@useobject{currentmarker}{}%
\end{pgfscope}%
\end{pgfscope}%
\begin{pgfscope}%
\definecolor{textcolor}{rgb}{0.333333,0.333333,0.333333}%
\pgfsetstrokecolor{textcolor}%
\pgfsetfillcolor{textcolor}%
\pgftext[x=0.443111in, y=7.413172in, left, base]{\color{textcolor}\rmfamily\fontsize{16.000000}{19.200000}\selectfont \(\displaystyle {500}\)}%
\end{pgfscope}%
\begin{pgfscope}%
\pgfpathrectangle{\pgfqpoint{0.870538in}{0.637495in}}{\pgfqpoint{9.300000in}{9.060000in}}%
\pgfusepath{clip}%
\pgfsetrectcap%
\pgfsetroundjoin%
\pgfsetlinewidth{0.803000pt}%
\definecolor{currentstroke}{rgb}{1.000000,1.000000,1.000000}%
\pgfsetstrokecolor{currentstroke}%
\pgfsetdash{}{0pt}%
\pgfpathmoveto{\pgfqpoint{0.870538in}{8.868307in}}%
\pgfpathlineto{\pgfqpoint{10.170538in}{8.868307in}}%
\pgfusepath{stroke}%
\end{pgfscope}%
\begin{pgfscope}%
\pgfsetbuttcap%
\pgfsetroundjoin%
\definecolor{currentfill}{rgb}{0.333333,0.333333,0.333333}%
\pgfsetfillcolor{currentfill}%
\pgfsetlinewidth{0.803000pt}%
\definecolor{currentstroke}{rgb}{0.333333,0.333333,0.333333}%
\pgfsetstrokecolor{currentstroke}%
\pgfsetdash{}{0pt}%
\pgfsys@defobject{currentmarker}{\pgfqpoint{-0.048611in}{0.000000in}}{\pgfqpoint{-0.000000in}{0.000000in}}{%
\pgfpathmoveto{\pgfqpoint{-0.000000in}{0.000000in}}%
\pgfpathlineto{\pgfqpoint{-0.048611in}{0.000000in}}%
\pgfusepath{stroke,fill}%
}%
\begin{pgfscope}%
\pgfsys@transformshift{0.870538in}{8.868307in}%
\pgfsys@useobject{currentmarker}{}%
\end{pgfscope}%
\end{pgfscope}%
\begin{pgfscope}%
\definecolor{textcolor}{rgb}{0.333333,0.333333,0.333333}%
\pgfsetstrokecolor{textcolor}%
\pgfsetfillcolor{textcolor}%
\pgftext[x=0.443111in, y=8.784974in, left, base]{\color{textcolor}\rmfamily\fontsize{16.000000}{19.200000}\selectfont \(\displaystyle {600}\)}%
\end{pgfscope}%
\begin{pgfscope}%
\definecolor{textcolor}{rgb}{0.333333,0.333333,0.333333}%
\pgfsetstrokecolor{textcolor}%
\pgfsetfillcolor{textcolor}%
\pgftext[x=0.387555in,y=5.167495in,,bottom,rotate=90.000000]{\color{textcolor}\rmfamily\fontsize{20.000000}{24.000000}\selectfont Generation [GWh]}%
\end{pgfscope}%
\begin{pgfscope}%
\pgfpathrectangle{\pgfqpoint{0.870538in}{0.637495in}}{\pgfqpoint{9.300000in}{9.060000in}}%
\pgfusepath{clip}%
\pgfsetbuttcap%
\pgfsetmiterjoin%
\definecolor{currentfill}{rgb}{0.839216,0.152941,0.156863}%
\pgfsetfillcolor{currentfill}%
\pgfsetfillopacity{0.990000}%
\pgfsetlinewidth{0.000000pt}%
\definecolor{currentstroke}{rgb}{0.000000,0.000000,0.000000}%
\pgfsetstrokecolor{currentstroke}%
\pgfsetstrokeopacity{0.990000}%
\pgfsetdash{}{0pt}%
\pgfpathmoveto{\pgfqpoint{1.258038in}{0.637495in}}%
\pgfpathlineto{\pgfqpoint{2.033038in}{0.637495in}}%
\pgfpathlineto{\pgfqpoint{2.033038in}{3.191361in}}%
\pgfpathlineto{\pgfqpoint{1.258038in}{3.191361in}}%
\pgfpathclose%
\pgfusepath{fill}%
\end{pgfscope}%
\begin{pgfscope}%
\pgfsetbuttcap%
\pgfsetmiterjoin%
\definecolor{currentfill}{rgb}{0.839216,0.152941,0.156863}%
\pgfsetfillcolor{currentfill}%
\pgfsetfillopacity{0.990000}%
\pgfsetlinewidth{0.000000pt}%
\definecolor{currentstroke}{rgb}{0.000000,0.000000,0.000000}%
\pgfsetstrokecolor{currentstroke}%
\pgfsetstrokeopacity{0.990000}%
\pgfsetdash{}{0pt}%
\pgfpathrectangle{\pgfqpoint{0.870538in}{0.637495in}}{\pgfqpoint{9.300000in}{9.060000in}}%
\pgfusepath{clip}%
\pgfpathmoveto{\pgfqpoint{1.258038in}{0.637495in}}%
\pgfpathlineto{\pgfqpoint{2.033038in}{0.637495in}}%
\pgfpathlineto{\pgfqpoint{2.033038in}{3.191361in}}%
\pgfpathlineto{\pgfqpoint{1.258038in}{3.191361in}}%
\pgfpathclose%
\pgfusepath{clip}%
\pgfsys@defobject{currentpattern}{\pgfqpoint{0in}{0in}}{\pgfqpoint{1in}{1in}}{%
\begin{pgfscope}%
\pgfpathrectangle{\pgfqpoint{0in}{0in}}{\pgfqpoint{1in}{1in}}%
\pgfusepath{clip}%
\pgfpathmoveto{\pgfqpoint{-0.500000in}{0.500000in}}%
\pgfpathlineto{\pgfqpoint{0.500000in}{1.500000in}}%
\pgfpathmoveto{\pgfqpoint{-0.333333in}{0.333333in}}%
\pgfpathlineto{\pgfqpoint{0.666667in}{1.333333in}}%
\pgfpathmoveto{\pgfqpoint{-0.166667in}{0.166667in}}%
\pgfpathlineto{\pgfqpoint{0.833333in}{1.166667in}}%
\pgfpathmoveto{\pgfqpoint{0.000000in}{0.000000in}}%
\pgfpathlineto{\pgfqpoint{1.000000in}{1.000000in}}%
\pgfpathmoveto{\pgfqpoint{0.166667in}{-0.166667in}}%
\pgfpathlineto{\pgfqpoint{1.166667in}{0.833333in}}%
\pgfpathmoveto{\pgfqpoint{0.333333in}{-0.333333in}}%
\pgfpathlineto{\pgfqpoint{1.333333in}{0.666667in}}%
\pgfpathmoveto{\pgfqpoint{0.500000in}{-0.500000in}}%
\pgfpathlineto{\pgfqpoint{1.500000in}{0.500000in}}%
\pgfpathmoveto{\pgfqpoint{-0.500000in}{0.500000in}}%
\pgfpathlineto{\pgfqpoint{0.500000in}{-0.500000in}}%
\pgfpathmoveto{\pgfqpoint{-0.333333in}{0.666667in}}%
\pgfpathlineto{\pgfqpoint{0.666667in}{-0.333333in}}%
\pgfpathmoveto{\pgfqpoint{-0.166667in}{0.833333in}}%
\pgfpathlineto{\pgfqpoint{0.833333in}{-0.166667in}}%
\pgfpathmoveto{\pgfqpoint{0.000000in}{1.000000in}}%
\pgfpathlineto{\pgfqpoint{1.000000in}{0.000000in}}%
\pgfpathmoveto{\pgfqpoint{0.166667in}{1.166667in}}%
\pgfpathlineto{\pgfqpoint{1.166667in}{0.166667in}}%
\pgfpathmoveto{\pgfqpoint{0.333333in}{1.333333in}}%
\pgfpathlineto{\pgfqpoint{1.333333in}{0.333333in}}%
\pgfpathmoveto{\pgfqpoint{0.500000in}{1.500000in}}%
\pgfpathlineto{\pgfqpoint{1.500000in}{0.500000in}}%
\pgfusepath{stroke}%
\end{pgfscope}%
}%
\pgfsys@transformshift{1.258038in}{0.637495in}%
\pgfsys@useobject{currentpattern}{}%
\pgfsys@transformshift{1in}{0in}%
\pgfsys@transformshift{-1in}{0in}%
\pgfsys@transformshift{0in}{1in}%
\pgfsys@useobject{currentpattern}{}%
\pgfsys@transformshift{1in}{0in}%
\pgfsys@transformshift{-1in}{0in}%
\pgfsys@transformshift{0in}{1in}%
\pgfsys@useobject{currentpattern}{}%
\pgfsys@transformshift{1in}{0in}%
\pgfsys@transformshift{-1in}{0in}%
\pgfsys@transformshift{0in}{1in}%
\end{pgfscope}%
\begin{pgfscope}%
\pgfpathrectangle{\pgfqpoint{0.870538in}{0.637495in}}{\pgfqpoint{9.300000in}{9.060000in}}%
\pgfusepath{clip}%
\pgfsetbuttcap%
\pgfsetmiterjoin%
\definecolor{currentfill}{rgb}{0.839216,0.152941,0.156863}%
\pgfsetfillcolor{currentfill}%
\pgfsetfillopacity{0.990000}%
\pgfsetlinewidth{0.000000pt}%
\definecolor{currentstroke}{rgb}{0.000000,0.000000,0.000000}%
\pgfsetstrokecolor{currentstroke}%
\pgfsetstrokeopacity{0.990000}%
\pgfsetdash{}{0pt}%
\pgfpathmoveto{\pgfqpoint{2.808038in}{0.637495in}}%
\pgfpathlineto{\pgfqpoint{3.583038in}{0.637495in}}%
\pgfpathlineto{\pgfqpoint{3.583038in}{2.630003in}}%
\pgfpathlineto{\pgfqpoint{2.808038in}{2.630003in}}%
\pgfpathclose%
\pgfusepath{fill}%
\end{pgfscope}%
\begin{pgfscope}%
\pgfsetbuttcap%
\pgfsetmiterjoin%
\definecolor{currentfill}{rgb}{0.839216,0.152941,0.156863}%
\pgfsetfillcolor{currentfill}%
\pgfsetfillopacity{0.990000}%
\pgfsetlinewidth{0.000000pt}%
\definecolor{currentstroke}{rgb}{0.000000,0.000000,0.000000}%
\pgfsetstrokecolor{currentstroke}%
\pgfsetstrokeopacity{0.990000}%
\pgfsetdash{}{0pt}%
\pgfpathrectangle{\pgfqpoint{0.870538in}{0.637495in}}{\pgfqpoint{9.300000in}{9.060000in}}%
\pgfusepath{clip}%
\pgfpathmoveto{\pgfqpoint{2.808038in}{0.637495in}}%
\pgfpathlineto{\pgfqpoint{3.583038in}{0.637495in}}%
\pgfpathlineto{\pgfqpoint{3.583038in}{2.630003in}}%
\pgfpathlineto{\pgfqpoint{2.808038in}{2.630003in}}%
\pgfpathclose%
\pgfusepath{clip}%
\pgfsys@defobject{currentpattern}{\pgfqpoint{0in}{0in}}{\pgfqpoint{1in}{1in}}{%
\begin{pgfscope}%
\pgfpathrectangle{\pgfqpoint{0in}{0in}}{\pgfqpoint{1in}{1in}}%
\pgfusepath{clip}%
\pgfpathmoveto{\pgfqpoint{-0.500000in}{0.500000in}}%
\pgfpathlineto{\pgfqpoint{0.500000in}{1.500000in}}%
\pgfpathmoveto{\pgfqpoint{-0.333333in}{0.333333in}}%
\pgfpathlineto{\pgfqpoint{0.666667in}{1.333333in}}%
\pgfpathmoveto{\pgfqpoint{-0.166667in}{0.166667in}}%
\pgfpathlineto{\pgfqpoint{0.833333in}{1.166667in}}%
\pgfpathmoveto{\pgfqpoint{0.000000in}{0.000000in}}%
\pgfpathlineto{\pgfqpoint{1.000000in}{1.000000in}}%
\pgfpathmoveto{\pgfqpoint{0.166667in}{-0.166667in}}%
\pgfpathlineto{\pgfqpoint{1.166667in}{0.833333in}}%
\pgfpathmoveto{\pgfqpoint{0.333333in}{-0.333333in}}%
\pgfpathlineto{\pgfqpoint{1.333333in}{0.666667in}}%
\pgfpathmoveto{\pgfqpoint{0.500000in}{-0.500000in}}%
\pgfpathlineto{\pgfqpoint{1.500000in}{0.500000in}}%
\pgfpathmoveto{\pgfqpoint{-0.500000in}{0.500000in}}%
\pgfpathlineto{\pgfqpoint{0.500000in}{-0.500000in}}%
\pgfpathmoveto{\pgfqpoint{-0.333333in}{0.666667in}}%
\pgfpathlineto{\pgfqpoint{0.666667in}{-0.333333in}}%
\pgfpathmoveto{\pgfqpoint{-0.166667in}{0.833333in}}%
\pgfpathlineto{\pgfqpoint{0.833333in}{-0.166667in}}%
\pgfpathmoveto{\pgfqpoint{0.000000in}{1.000000in}}%
\pgfpathlineto{\pgfqpoint{1.000000in}{0.000000in}}%
\pgfpathmoveto{\pgfqpoint{0.166667in}{1.166667in}}%
\pgfpathlineto{\pgfqpoint{1.166667in}{0.166667in}}%
\pgfpathmoveto{\pgfqpoint{0.333333in}{1.333333in}}%
\pgfpathlineto{\pgfqpoint{1.333333in}{0.333333in}}%
\pgfpathmoveto{\pgfqpoint{0.500000in}{1.500000in}}%
\pgfpathlineto{\pgfqpoint{1.500000in}{0.500000in}}%
\pgfusepath{stroke}%
\end{pgfscope}%
}%
\pgfsys@transformshift{2.808038in}{0.637495in}%
\pgfsys@useobject{currentpattern}{}%
\pgfsys@transformshift{1in}{0in}%
\pgfsys@transformshift{-1in}{0in}%
\pgfsys@transformshift{0in}{1in}%
\pgfsys@useobject{currentpattern}{}%
\pgfsys@transformshift{1in}{0in}%
\pgfsys@transformshift{-1in}{0in}%
\pgfsys@transformshift{0in}{1in}%
\end{pgfscope}%
\begin{pgfscope}%
\pgfpathrectangle{\pgfqpoint{0.870538in}{0.637495in}}{\pgfqpoint{9.300000in}{9.060000in}}%
\pgfusepath{clip}%
\pgfsetbuttcap%
\pgfsetmiterjoin%
\definecolor{currentfill}{rgb}{0.839216,0.152941,0.156863}%
\pgfsetfillcolor{currentfill}%
\pgfsetfillopacity{0.990000}%
\pgfsetlinewidth{0.000000pt}%
\definecolor{currentstroke}{rgb}{0.000000,0.000000,0.000000}%
\pgfsetstrokecolor{currentstroke}%
\pgfsetstrokeopacity{0.990000}%
\pgfsetdash{}{0pt}%
\pgfpathmoveto{\pgfqpoint{4.358038in}{0.637495in}}%
\pgfpathlineto{\pgfqpoint{5.133038in}{0.637495in}}%
\pgfpathlineto{\pgfqpoint{5.133038in}{1.677933in}}%
\pgfpathlineto{\pgfqpoint{4.358038in}{1.677933in}}%
\pgfpathclose%
\pgfusepath{fill}%
\end{pgfscope}%
\begin{pgfscope}%
\pgfsetbuttcap%
\pgfsetmiterjoin%
\definecolor{currentfill}{rgb}{0.839216,0.152941,0.156863}%
\pgfsetfillcolor{currentfill}%
\pgfsetfillopacity{0.990000}%
\pgfsetlinewidth{0.000000pt}%
\definecolor{currentstroke}{rgb}{0.000000,0.000000,0.000000}%
\pgfsetstrokecolor{currentstroke}%
\pgfsetstrokeopacity{0.990000}%
\pgfsetdash{}{0pt}%
\pgfpathrectangle{\pgfqpoint{0.870538in}{0.637495in}}{\pgfqpoint{9.300000in}{9.060000in}}%
\pgfusepath{clip}%
\pgfpathmoveto{\pgfqpoint{4.358038in}{0.637495in}}%
\pgfpathlineto{\pgfqpoint{5.133038in}{0.637495in}}%
\pgfpathlineto{\pgfqpoint{5.133038in}{1.677933in}}%
\pgfpathlineto{\pgfqpoint{4.358038in}{1.677933in}}%
\pgfpathclose%
\pgfusepath{clip}%
\pgfsys@defobject{currentpattern}{\pgfqpoint{0in}{0in}}{\pgfqpoint{1in}{1in}}{%
\begin{pgfscope}%
\pgfpathrectangle{\pgfqpoint{0in}{0in}}{\pgfqpoint{1in}{1in}}%
\pgfusepath{clip}%
\pgfpathmoveto{\pgfqpoint{-0.500000in}{0.500000in}}%
\pgfpathlineto{\pgfqpoint{0.500000in}{1.500000in}}%
\pgfpathmoveto{\pgfqpoint{-0.333333in}{0.333333in}}%
\pgfpathlineto{\pgfqpoint{0.666667in}{1.333333in}}%
\pgfpathmoveto{\pgfqpoint{-0.166667in}{0.166667in}}%
\pgfpathlineto{\pgfqpoint{0.833333in}{1.166667in}}%
\pgfpathmoveto{\pgfqpoint{0.000000in}{0.000000in}}%
\pgfpathlineto{\pgfqpoint{1.000000in}{1.000000in}}%
\pgfpathmoveto{\pgfqpoint{0.166667in}{-0.166667in}}%
\pgfpathlineto{\pgfqpoint{1.166667in}{0.833333in}}%
\pgfpathmoveto{\pgfqpoint{0.333333in}{-0.333333in}}%
\pgfpathlineto{\pgfqpoint{1.333333in}{0.666667in}}%
\pgfpathmoveto{\pgfqpoint{0.500000in}{-0.500000in}}%
\pgfpathlineto{\pgfqpoint{1.500000in}{0.500000in}}%
\pgfpathmoveto{\pgfqpoint{-0.500000in}{0.500000in}}%
\pgfpathlineto{\pgfqpoint{0.500000in}{-0.500000in}}%
\pgfpathmoveto{\pgfqpoint{-0.333333in}{0.666667in}}%
\pgfpathlineto{\pgfqpoint{0.666667in}{-0.333333in}}%
\pgfpathmoveto{\pgfqpoint{-0.166667in}{0.833333in}}%
\pgfpathlineto{\pgfqpoint{0.833333in}{-0.166667in}}%
\pgfpathmoveto{\pgfqpoint{0.000000in}{1.000000in}}%
\pgfpathlineto{\pgfqpoint{1.000000in}{0.000000in}}%
\pgfpathmoveto{\pgfqpoint{0.166667in}{1.166667in}}%
\pgfpathlineto{\pgfqpoint{1.166667in}{0.166667in}}%
\pgfpathmoveto{\pgfqpoint{0.333333in}{1.333333in}}%
\pgfpathlineto{\pgfqpoint{1.333333in}{0.333333in}}%
\pgfpathmoveto{\pgfqpoint{0.500000in}{1.500000in}}%
\pgfpathlineto{\pgfqpoint{1.500000in}{0.500000in}}%
\pgfusepath{stroke}%
\end{pgfscope}%
}%
\pgfsys@transformshift{4.358038in}{0.637495in}%
\pgfsys@useobject{currentpattern}{}%
\pgfsys@transformshift{1in}{0in}%
\pgfsys@transformshift{-1in}{0in}%
\pgfsys@transformshift{0in}{1in}%
\pgfsys@useobject{currentpattern}{}%
\pgfsys@transformshift{1in}{0in}%
\pgfsys@transformshift{-1in}{0in}%
\pgfsys@transformshift{0in}{1in}%
\end{pgfscope}%
\begin{pgfscope}%
\pgfpathrectangle{\pgfqpoint{0.870538in}{0.637495in}}{\pgfqpoint{9.300000in}{9.060000in}}%
\pgfusepath{clip}%
\pgfsetbuttcap%
\pgfsetmiterjoin%
\definecolor{currentfill}{rgb}{0.839216,0.152941,0.156863}%
\pgfsetfillcolor{currentfill}%
\pgfsetfillopacity{0.990000}%
\pgfsetlinewidth{0.000000pt}%
\definecolor{currentstroke}{rgb}{0.000000,0.000000,0.000000}%
\pgfsetstrokecolor{currentstroke}%
\pgfsetstrokeopacity{0.990000}%
\pgfsetdash{}{0pt}%
\pgfpathmoveto{\pgfqpoint{5.908038in}{0.637495in}}%
\pgfpathlineto{\pgfqpoint{6.683038in}{0.637495in}}%
\pgfpathlineto{\pgfqpoint{6.683038in}{0.875116in}}%
\pgfpathlineto{\pgfqpoint{5.908038in}{0.875116in}}%
\pgfpathclose%
\pgfusepath{fill}%
\end{pgfscope}%
\begin{pgfscope}%
\pgfsetbuttcap%
\pgfsetmiterjoin%
\definecolor{currentfill}{rgb}{0.839216,0.152941,0.156863}%
\pgfsetfillcolor{currentfill}%
\pgfsetfillopacity{0.990000}%
\pgfsetlinewidth{0.000000pt}%
\definecolor{currentstroke}{rgb}{0.000000,0.000000,0.000000}%
\pgfsetstrokecolor{currentstroke}%
\pgfsetstrokeopacity{0.990000}%
\pgfsetdash{}{0pt}%
\pgfpathrectangle{\pgfqpoint{0.870538in}{0.637495in}}{\pgfqpoint{9.300000in}{9.060000in}}%
\pgfusepath{clip}%
\pgfpathmoveto{\pgfqpoint{5.908038in}{0.637495in}}%
\pgfpathlineto{\pgfqpoint{6.683038in}{0.637495in}}%
\pgfpathlineto{\pgfqpoint{6.683038in}{0.875116in}}%
\pgfpathlineto{\pgfqpoint{5.908038in}{0.875116in}}%
\pgfpathclose%
\pgfusepath{clip}%
\pgfsys@defobject{currentpattern}{\pgfqpoint{0in}{0in}}{\pgfqpoint{1in}{1in}}{%
\begin{pgfscope}%
\pgfpathrectangle{\pgfqpoint{0in}{0in}}{\pgfqpoint{1in}{1in}}%
\pgfusepath{clip}%
\pgfpathmoveto{\pgfqpoint{-0.500000in}{0.500000in}}%
\pgfpathlineto{\pgfqpoint{0.500000in}{1.500000in}}%
\pgfpathmoveto{\pgfqpoint{-0.333333in}{0.333333in}}%
\pgfpathlineto{\pgfqpoint{0.666667in}{1.333333in}}%
\pgfpathmoveto{\pgfqpoint{-0.166667in}{0.166667in}}%
\pgfpathlineto{\pgfqpoint{0.833333in}{1.166667in}}%
\pgfpathmoveto{\pgfqpoint{0.000000in}{0.000000in}}%
\pgfpathlineto{\pgfqpoint{1.000000in}{1.000000in}}%
\pgfpathmoveto{\pgfqpoint{0.166667in}{-0.166667in}}%
\pgfpathlineto{\pgfqpoint{1.166667in}{0.833333in}}%
\pgfpathmoveto{\pgfqpoint{0.333333in}{-0.333333in}}%
\pgfpathlineto{\pgfqpoint{1.333333in}{0.666667in}}%
\pgfpathmoveto{\pgfqpoint{0.500000in}{-0.500000in}}%
\pgfpathlineto{\pgfqpoint{1.500000in}{0.500000in}}%
\pgfpathmoveto{\pgfqpoint{-0.500000in}{0.500000in}}%
\pgfpathlineto{\pgfqpoint{0.500000in}{-0.500000in}}%
\pgfpathmoveto{\pgfqpoint{-0.333333in}{0.666667in}}%
\pgfpathlineto{\pgfqpoint{0.666667in}{-0.333333in}}%
\pgfpathmoveto{\pgfqpoint{-0.166667in}{0.833333in}}%
\pgfpathlineto{\pgfqpoint{0.833333in}{-0.166667in}}%
\pgfpathmoveto{\pgfqpoint{0.000000in}{1.000000in}}%
\pgfpathlineto{\pgfqpoint{1.000000in}{0.000000in}}%
\pgfpathmoveto{\pgfqpoint{0.166667in}{1.166667in}}%
\pgfpathlineto{\pgfqpoint{1.166667in}{0.166667in}}%
\pgfpathmoveto{\pgfqpoint{0.333333in}{1.333333in}}%
\pgfpathlineto{\pgfqpoint{1.333333in}{0.333333in}}%
\pgfpathmoveto{\pgfqpoint{0.500000in}{1.500000in}}%
\pgfpathlineto{\pgfqpoint{1.500000in}{0.500000in}}%
\pgfusepath{stroke}%
\end{pgfscope}%
}%
\pgfsys@transformshift{5.908038in}{0.637495in}%
\pgfsys@useobject{currentpattern}{}%
\pgfsys@transformshift{1in}{0in}%
\pgfsys@transformshift{-1in}{0in}%
\pgfsys@transformshift{0in}{1in}%
\end{pgfscope}%
\begin{pgfscope}%
\pgfpathrectangle{\pgfqpoint{0.870538in}{0.637495in}}{\pgfqpoint{9.300000in}{9.060000in}}%
\pgfusepath{clip}%
\pgfsetbuttcap%
\pgfsetmiterjoin%
\definecolor{currentfill}{rgb}{0.839216,0.152941,0.156863}%
\pgfsetfillcolor{currentfill}%
\pgfsetfillopacity{0.990000}%
\pgfsetlinewidth{0.000000pt}%
\definecolor{currentstroke}{rgb}{0.000000,0.000000,0.000000}%
\pgfsetstrokecolor{currentstroke}%
\pgfsetstrokeopacity{0.990000}%
\pgfsetdash{}{0pt}%
\pgfpathmoveto{\pgfqpoint{7.458038in}{0.637495in}}%
\pgfpathlineto{\pgfqpoint{8.233038in}{0.637495in}}%
\pgfpathlineto{\pgfqpoint{8.233038in}{1.058777in}}%
\pgfpathlineto{\pgfqpoint{7.458038in}{1.058777in}}%
\pgfpathclose%
\pgfusepath{fill}%
\end{pgfscope}%
\begin{pgfscope}%
\pgfsetbuttcap%
\pgfsetmiterjoin%
\definecolor{currentfill}{rgb}{0.839216,0.152941,0.156863}%
\pgfsetfillcolor{currentfill}%
\pgfsetfillopacity{0.990000}%
\pgfsetlinewidth{0.000000pt}%
\definecolor{currentstroke}{rgb}{0.000000,0.000000,0.000000}%
\pgfsetstrokecolor{currentstroke}%
\pgfsetstrokeopacity{0.990000}%
\pgfsetdash{}{0pt}%
\pgfpathrectangle{\pgfqpoint{0.870538in}{0.637495in}}{\pgfqpoint{9.300000in}{9.060000in}}%
\pgfusepath{clip}%
\pgfpathmoveto{\pgfqpoint{7.458038in}{0.637495in}}%
\pgfpathlineto{\pgfqpoint{8.233038in}{0.637495in}}%
\pgfpathlineto{\pgfqpoint{8.233038in}{1.058777in}}%
\pgfpathlineto{\pgfqpoint{7.458038in}{1.058777in}}%
\pgfpathclose%
\pgfusepath{clip}%
\pgfsys@defobject{currentpattern}{\pgfqpoint{0in}{0in}}{\pgfqpoint{1in}{1in}}{%
\begin{pgfscope}%
\pgfpathrectangle{\pgfqpoint{0in}{0in}}{\pgfqpoint{1in}{1in}}%
\pgfusepath{clip}%
\pgfpathmoveto{\pgfqpoint{-0.500000in}{0.500000in}}%
\pgfpathlineto{\pgfqpoint{0.500000in}{1.500000in}}%
\pgfpathmoveto{\pgfqpoint{-0.333333in}{0.333333in}}%
\pgfpathlineto{\pgfqpoint{0.666667in}{1.333333in}}%
\pgfpathmoveto{\pgfqpoint{-0.166667in}{0.166667in}}%
\pgfpathlineto{\pgfqpoint{0.833333in}{1.166667in}}%
\pgfpathmoveto{\pgfqpoint{0.000000in}{0.000000in}}%
\pgfpathlineto{\pgfqpoint{1.000000in}{1.000000in}}%
\pgfpathmoveto{\pgfqpoint{0.166667in}{-0.166667in}}%
\pgfpathlineto{\pgfqpoint{1.166667in}{0.833333in}}%
\pgfpathmoveto{\pgfqpoint{0.333333in}{-0.333333in}}%
\pgfpathlineto{\pgfqpoint{1.333333in}{0.666667in}}%
\pgfpathmoveto{\pgfqpoint{0.500000in}{-0.500000in}}%
\pgfpathlineto{\pgfqpoint{1.500000in}{0.500000in}}%
\pgfpathmoveto{\pgfqpoint{-0.500000in}{0.500000in}}%
\pgfpathlineto{\pgfqpoint{0.500000in}{-0.500000in}}%
\pgfpathmoveto{\pgfqpoint{-0.333333in}{0.666667in}}%
\pgfpathlineto{\pgfqpoint{0.666667in}{-0.333333in}}%
\pgfpathmoveto{\pgfqpoint{-0.166667in}{0.833333in}}%
\pgfpathlineto{\pgfqpoint{0.833333in}{-0.166667in}}%
\pgfpathmoveto{\pgfqpoint{0.000000in}{1.000000in}}%
\pgfpathlineto{\pgfqpoint{1.000000in}{0.000000in}}%
\pgfpathmoveto{\pgfqpoint{0.166667in}{1.166667in}}%
\pgfpathlineto{\pgfqpoint{1.166667in}{0.166667in}}%
\pgfpathmoveto{\pgfqpoint{0.333333in}{1.333333in}}%
\pgfpathlineto{\pgfqpoint{1.333333in}{0.333333in}}%
\pgfpathmoveto{\pgfqpoint{0.500000in}{1.500000in}}%
\pgfpathlineto{\pgfqpoint{1.500000in}{0.500000in}}%
\pgfusepath{stroke}%
\end{pgfscope}%
}%
\pgfsys@transformshift{7.458038in}{0.637495in}%
\pgfsys@useobject{currentpattern}{}%
\pgfsys@transformshift{1in}{0in}%
\pgfsys@transformshift{-1in}{0in}%
\pgfsys@transformshift{0in}{1in}%
\end{pgfscope}%
\begin{pgfscope}%
\pgfpathrectangle{\pgfqpoint{0.870538in}{0.637495in}}{\pgfqpoint{9.300000in}{9.060000in}}%
\pgfusepath{clip}%
\pgfsetbuttcap%
\pgfsetmiterjoin%
\definecolor{currentfill}{rgb}{0.839216,0.152941,0.156863}%
\pgfsetfillcolor{currentfill}%
\pgfsetfillopacity{0.990000}%
\pgfsetlinewidth{0.000000pt}%
\definecolor{currentstroke}{rgb}{0.000000,0.000000,0.000000}%
\pgfsetstrokecolor{currentstroke}%
\pgfsetstrokeopacity{0.990000}%
\pgfsetdash{}{0pt}%
\pgfpathmoveto{\pgfqpoint{9.008038in}{0.637495in}}%
\pgfpathlineto{\pgfqpoint{9.783038in}{0.637495in}}%
\pgfpathlineto{\pgfqpoint{9.783038in}{0.637495in}}%
\pgfpathlineto{\pgfqpoint{9.008038in}{0.637495in}}%
\pgfpathclose%
\pgfusepath{fill}%
\end{pgfscope}%
\begin{pgfscope}%
\pgfsetbuttcap%
\pgfsetmiterjoin%
\definecolor{currentfill}{rgb}{0.839216,0.152941,0.156863}%
\pgfsetfillcolor{currentfill}%
\pgfsetfillopacity{0.990000}%
\pgfsetlinewidth{0.000000pt}%
\definecolor{currentstroke}{rgb}{0.000000,0.000000,0.000000}%
\pgfsetstrokecolor{currentstroke}%
\pgfsetstrokeopacity{0.990000}%
\pgfsetdash{}{0pt}%
\pgfpathrectangle{\pgfqpoint{0.870538in}{0.637495in}}{\pgfqpoint{9.300000in}{9.060000in}}%
\pgfusepath{clip}%
\pgfpathmoveto{\pgfqpoint{9.008038in}{0.637495in}}%
\pgfpathlineto{\pgfqpoint{9.783038in}{0.637495in}}%
\pgfpathlineto{\pgfqpoint{9.783038in}{0.637495in}}%
\pgfpathlineto{\pgfqpoint{9.008038in}{0.637495in}}%
\pgfpathclose%
\pgfusepath{clip}%
\pgfsys@defobject{currentpattern}{\pgfqpoint{0in}{0in}}{\pgfqpoint{1in}{1in}}{%
\begin{pgfscope}%
\pgfpathrectangle{\pgfqpoint{0in}{0in}}{\pgfqpoint{1in}{1in}}%
\pgfusepath{clip}%
\pgfpathmoveto{\pgfqpoint{-0.500000in}{0.500000in}}%
\pgfpathlineto{\pgfqpoint{0.500000in}{1.500000in}}%
\pgfpathmoveto{\pgfqpoint{-0.333333in}{0.333333in}}%
\pgfpathlineto{\pgfqpoint{0.666667in}{1.333333in}}%
\pgfpathmoveto{\pgfqpoint{-0.166667in}{0.166667in}}%
\pgfpathlineto{\pgfqpoint{0.833333in}{1.166667in}}%
\pgfpathmoveto{\pgfqpoint{0.000000in}{0.000000in}}%
\pgfpathlineto{\pgfqpoint{1.000000in}{1.000000in}}%
\pgfpathmoveto{\pgfqpoint{0.166667in}{-0.166667in}}%
\pgfpathlineto{\pgfqpoint{1.166667in}{0.833333in}}%
\pgfpathmoveto{\pgfqpoint{0.333333in}{-0.333333in}}%
\pgfpathlineto{\pgfqpoint{1.333333in}{0.666667in}}%
\pgfpathmoveto{\pgfqpoint{0.500000in}{-0.500000in}}%
\pgfpathlineto{\pgfqpoint{1.500000in}{0.500000in}}%
\pgfpathmoveto{\pgfqpoint{-0.500000in}{0.500000in}}%
\pgfpathlineto{\pgfqpoint{0.500000in}{-0.500000in}}%
\pgfpathmoveto{\pgfqpoint{-0.333333in}{0.666667in}}%
\pgfpathlineto{\pgfqpoint{0.666667in}{-0.333333in}}%
\pgfpathmoveto{\pgfqpoint{-0.166667in}{0.833333in}}%
\pgfpathlineto{\pgfqpoint{0.833333in}{-0.166667in}}%
\pgfpathmoveto{\pgfqpoint{0.000000in}{1.000000in}}%
\pgfpathlineto{\pgfqpoint{1.000000in}{0.000000in}}%
\pgfpathmoveto{\pgfqpoint{0.166667in}{1.166667in}}%
\pgfpathlineto{\pgfqpoint{1.166667in}{0.166667in}}%
\pgfpathmoveto{\pgfqpoint{0.333333in}{1.333333in}}%
\pgfpathlineto{\pgfqpoint{1.333333in}{0.333333in}}%
\pgfpathmoveto{\pgfqpoint{0.500000in}{1.500000in}}%
\pgfpathlineto{\pgfqpoint{1.500000in}{0.500000in}}%
\pgfusepath{stroke}%
\end{pgfscope}%
}%
\pgfsys@transformshift{9.008038in}{0.637495in}%
\end{pgfscope}%
\begin{pgfscope}%
\pgfpathrectangle{\pgfqpoint{0.870538in}{0.637495in}}{\pgfqpoint{9.300000in}{9.060000in}}%
\pgfusepath{clip}%
\pgfsetbuttcap%
\pgfsetmiterjoin%
\definecolor{currentfill}{rgb}{0.549020,0.337255,0.294118}%
\pgfsetfillcolor{currentfill}%
\pgfsetfillopacity{0.990000}%
\pgfsetlinewidth{0.000000pt}%
\definecolor{currentstroke}{rgb}{0.000000,0.000000,0.000000}%
\pgfsetstrokecolor{currentstroke}%
\pgfsetstrokeopacity{0.990000}%
\pgfsetdash{}{0pt}%
\pgfpathmoveto{\pgfqpoint{1.258038in}{3.191361in}}%
\pgfpathlineto{\pgfqpoint{2.033038in}{3.191361in}}%
\pgfpathlineto{\pgfqpoint{2.033038in}{3.502115in}}%
\pgfpathlineto{\pgfqpoint{1.258038in}{3.502115in}}%
\pgfpathclose%
\pgfusepath{fill}%
\end{pgfscope}%
\begin{pgfscope}%
\pgfsetbuttcap%
\pgfsetmiterjoin%
\definecolor{currentfill}{rgb}{0.549020,0.337255,0.294118}%
\pgfsetfillcolor{currentfill}%
\pgfsetfillopacity{0.990000}%
\pgfsetlinewidth{0.000000pt}%
\definecolor{currentstroke}{rgb}{0.000000,0.000000,0.000000}%
\pgfsetstrokecolor{currentstroke}%
\pgfsetstrokeopacity{0.990000}%
\pgfsetdash{}{0pt}%
\pgfpathrectangle{\pgfqpoint{0.870538in}{0.637495in}}{\pgfqpoint{9.300000in}{9.060000in}}%
\pgfusepath{clip}%
\pgfpathmoveto{\pgfqpoint{1.258038in}{3.191361in}}%
\pgfpathlineto{\pgfqpoint{2.033038in}{3.191361in}}%
\pgfpathlineto{\pgfqpoint{2.033038in}{3.502115in}}%
\pgfpathlineto{\pgfqpoint{1.258038in}{3.502115in}}%
\pgfpathclose%
\pgfusepath{clip}%
\pgfsys@defobject{currentpattern}{\pgfqpoint{0in}{0in}}{\pgfqpoint{1in}{1in}}{%
\begin{pgfscope}%
\pgfpathrectangle{\pgfqpoint{0in}{0in}}{\pgfqpoint{1in}{1in}}%
\pgfusepath{clip}%
\pgfpathmoveto{\pgfqpoint{0.000000in}{-0.058333in}}%
\pgfpathcurveto{\pgfqpoint{0.015470in}{-0.058333in}}{\pgfqpoint{0.030309in}{-0.052187in}}{\pgfqpoint{0.041248in}{-0.041248in}}%
\pgfpathcurveto{\pgfqpoint{0.052187in}{-0.030309in}}{\pgfqpoint{0.058333in}{-0.015470in}}{\pgfqpoint{0.058333in}{0.000000in}}%
\pgfpathcurveto{\pgfqpoint{0.058333in}{0.015470in}}{\pgfqpoint{0.052187in}{0.030309in}}{\pgfqpoint{0.041248in}{0.041248in}}%
\pgfpathcurveto{\pgfqpoint{0.030309in}{0.052187in}}{\pgfqpoint{0.015470in}{0.058333in}}{\pgfqpoint{0.000000in}{0.058333in}}%
\pgfpathcurveto{\pgfqpoint{-0.015470in}{0.058333in}}{\pgfqpoint{-0.030309in}{0.052187in}}{\pgfqpoint{-0.041248in}{0.041248in}}%
\pgfpathcurveto{\pgfqpoint{-0.052187in}{0.030309in}}{\pgfqpoint{-0.058333in}{0.015470in}}{\pgfqpoint{-0.058333in}{0.000000in}}%
\pgfpathcurveto{\pgfqpoint{-0.058333in}{-0.015470in}}{\pgfqpoint{-0.052187in}{-0.030309in}}{\pgfqpoint{-0.041248in}{-0.041248in}}%
\pgfpathcurveto{\pgfqpoint{-0.030309in}{-0.052187in}}{\pgfqpoint{-0.015470in}{-0.058333in}}{\pgfqpoint{0.000000in}{-0.058333in}}%
\pgfpathclose%
\pgfpathmoveto{\pgfqpoint{0.000000in}{-0.052500in}}%
\pgfpathcurveto{\pgfqpoint{0.000000in}{-0.052500in}}{\pgfqpoint{-0.013923in}{-0.052500in}}{\pgfqpoint{-0.027278in}{-0.046968in}}%
\pgfpathcurveto{\pgfqpoint{-0.037123in}{-0.037123in}}{\pgfqpoint{-0.046968in}{-0.027278in}}{\pgfqpoint{-0.052500in}{-0.013923in}}%
\pgfpathcurveto{\pgfqpoint{-0.052500in}{0.000000in}}{\pgfqpoint{-0.052500in}{0.013923in}}{\pgfqpoint{-0.046968in}{0.027278in}}%
\pgfpathcurveto{\pgfqpoint{-0.037123in}{0.037123in}}{\pgfqpoint{-0.027278in}{0.046968in}}{\pgfqpoint{-0.013923in}{0.052500in}}%
\pgfpathcurveto{\pgfqpoint{0.000000in}{0.052500in}}{\pgfqpoint{0.013923in}{0.052500in}}{\pgfqpoint{0.027278in}{0.046968in}}%
\pgfpathcurveto{\pgfqpoint{0.037123in}{0.037123in}}{\pgfqpoint{0.046968in}{0.027278in}}{\pgfqpoint{0.052500in}{0.013923in}}%
\pgfpathcurveto{\pgfqpoint{0.052500in}{0.000000in}}{\pgfqpoint{0.052500in}{-0.013923in}}{\pgfqpoint{0.046968in}{-0.027278in}}%
\pgfpathcurveto{\pgfqpoint{0.037123in}{-0.037123in}}{\pgfqpoint{0.027278in}{-0.046968in}}{\pgfqpoint{0.013923in}{-0.052500in}}%
\pgfpathclose%
\pgfpathmoveto{\pgfqpoint{0.166667in}{-0.058333in}}%
\pgfpathcurveto{\pgfqpoint{0.182137in}{-0.058333in}}{\pgfqpoint{0.196975in}{-0.052187in}}{\pgfqpoint{0.207915in}{-0.041248in}}%
\pgfpathcurveto{\pgfqpoint{0.218854in}{-0.030309in}}{\pgfqpoint{0.225000in}{-0.015470in}}{\pgfqpoint{0.225000in}{0.000000in}}%
\pgfpathcurveto{\pgfqpoint{0.225000in}{0.015470in}}{\pgfqpoint{0.218854in}{0.030309in}}{\pgfqpoint{0.207915in}{0.041248in}}%
\pgfpathcurveto{\pgfqpoint{0.196975in}{0.052187in}}{\pgfqpoint{0.182137in}{0.058333in}}{\pgfqpoint{0.166667in}{0.058333in}}%
\pgfpathcurveto{\pgfqpoint{0.151196in}{0.058333in}}{\pgfqpoint{0.136358in}{0.052187in}}{\pgfqpoint{0.125419in}{0.041248in}}%
\pgfpathcurveto{\pgfqpoint{0.114480in}{0.030309in}}{\pgfqpoint{0.108333in}{0.015470in}}{\pgfqpoint{0.108333in}{0.000000in}}%
\pgfpathcurveto{\pgfqpoint{0.108333in}{-0.015470in}}{\pgfqpoint{0.114480in}{-0.030309in}}{\pgfqpoint{0.125419in}{-0.041248in}}%
\pgfpathcurveto{\pgfqpoint{0.136358in}{-0.052187in}}{\pgfqpoint{0.151196in}{-0.058333in}}{\pgfqpoint{0.166667in}{-0.058333in}}%
\pgfpathclose%
\pgfpathmoveto{\pgfqpoint{0.166667in}{-0.052500in}}%
\pgfpathcurveto{\pgfqpoint{0.166667in}{-0.052500in}}{\pgfqpoint{0.152744in}{-0.052500in}}{\pgfqpoint{0.139389in}{-0.046968in}}%
\pgfpathcurveto{\pgfqpoint{0.129544in}{-0.037123in}}{\pgfqpoint{0.119698in}{-0.027278in}}{\pgfqpoint{0.114167in}{-0.013923in}}%
\pgfpathcurveto{\pgfqpoint{0.114167in}{0.000000in}}{\pgfqpoint{0.114167in}{0.013923in}}{\pgfqpoint{0.119698in}{0.027278in}}%
\pgfpathcurveto{\pgfqpoint{0.129544in}{0.037123in}}{\pgfqpoint{0.139389in}{0.046968in}}{\pgfqpoint{0.152744in}{0.052500in}}%
\pgfpathcurveto{\pgfqpoint{0.166667in}{0.052500in}}{\pgfqpoint{0.180590in}{0.052500in}}{\pgfqpoint{0.193945in}{0.046968in}}%
\pgfpathcurveto{\pgfqpoint{0.203790in}{0.037123in}}{\pgfqpoint{0.213635in}{0.027278in}}{\pgfqpoint{0.219167in}{0.013923in}}%
\pgfpathcurveto{\pgfqpoint{0.219167in}{0.000000in}}{\pgfqpoint{0.219167in}{-0.013923in}}{\pgfqpoint{0.213635in}{-0.027278in}}%
\pgfpathcurveto{\pgfqpoint{0.203790in}{-0.037123in}}{\pgfqpoint{0.193945in}{-0.046968in}}{\pgfqpoint{0.180590in}{-0.052500in}}%
\pgfpathclose%
\pgfpathmoveto{\pgfqpoint{0.333333in}{-0.058333in}}%
\pgfpathcurveto{\pgfqpoint{0.348804in}{-0.058333in}}{\pgfqpoint{0.363642in}{-0.052187in}}{\pgfqpoint{0.374581in}{-0.041248in}}%
\pgfpathcurveto{\pgfqpoint{0.385520in}{-0.030309in}}{\pgfqpoint{0.391667in}{-0.015470in}}{\pgfqpoint{0.391667in}{0.000000in}}%
\pgfpathcurveto{\pgfqpoint{0.391667in}{0.015470in}}{\pgfqpoint{0.385520in}{0.030309in}}{\pgfqpoint{0.374581in}{0.041248in}}%
\pgfpathcurveto{\pgfqpoint{0.363642in}{0.052187in}}{\pgfqpoint{0.348804in}{0.058333in}}{\pgfqpoint{0.333333in}{0.058333in}}%
\pgfpathcurveto{\pgfqpoint{0.317863in}{0.058333in}}{\pgfqpoint{0.303025in}{0.052187in}}{\pgfqpoint{0.292085in}{0.041248in}}%
\pgfpathcurveto{\pgfqpoint{0.281146in}{0.030309in}}{\pgfqpoint{0.275000in}{0.015470in}}{\pgfqpoint{0.275000in}{0.000000in}}%
\pgfpathcurveto{\pgfqpoint{0.275000in}{-0.015470in}}{\pgfqpoint{0.281146in}{-0.030309in}}{\pgfqpoint{0.292085in}{-0.041248in}}%
\pgfpathcurveto{\pgfqpoint{0.303025in}{-0.052187in}}{\pgfqpoint{0.317863in}{-0.058333in}}{\pgfqpoint{0.333333in}{-0.058333in}}%
\pgfpathclose%
\pgfpathmoveto{\pgfqpoint{0.333333in}{-0.052500in}}%
\pgfpathcurveto{\pgfqpoint{0.333333in}{-0.052500in}}{\pgfqpoint{0.319410in}{-0.052500in}}{\pgfqpoint{0.306055in}{-0.046968in}}%
\pgfpathcurveto{\pgfqpoint{0.296210in}{-0.037123in}}{\pgfqpoint{0.286365in}{-0.027278in}}{\pgfqpoint{0.280833in}{-0.013923in}}%
\pgfpathcurveto{\pgfqpoint{0.280833in}{0.000000in}}{\pgfqpoint{0.280833in}{0.013923in}}{\pgfqpoint{0.286365in}{0.027278in}}%
\pgfpathcurveto{\pgfqpoint{0.296210in}{0.037123in}}{\pgfqpoint{0.306055in}{0.046968in}}{\pgfqpoint{0.319410in}{0.052500in}}%
\pgfpathcurveto{\pgfqpoint{0.333333in}{0.052500in}}{\pgfqpoint{0.347256in}{0.052500in}}{\pgfqpoint{0.360611in}{0.046968in}}%
\pgfpathcurveto{\pgfqpoint{0.370456in}{0.037123in}}{\pgfqpoint{0.380302in}{0.027278in}}{\pgfqpoint{0.385833in}{0.013923in}}%
\pgfpathcurveto{\pgfqpoint{0.385833in}{0.000000in}}{\pgfqpoint{0.385833in}{-0.013923in}}{\pgfqpoint{0.380302in}{-0.027278in}}%
\pgfpathcurveto{\pgfqpoint{0.370456in}{-0.037123in}}{\pgfqpoint{0.360611in}{-0.046968in}}{\pgfqpoint{0.347256in}{-0.052500in}}%
\pgfpathclose%
\pgfpathmoveto{\pgfqpoint{0.500000in}{-0.058333in}}%
\pgfpathcurveto{\pgfqpoint{0.515470in}{-0.058333in}}{\pgfqpoint{0.530309in}{-0.052187in}}{\pgfqpoint{0.541248in}{-0.041248in}}%
\pgfpathcurveto{\pgfqpoint{0.552187in}{-0.030309in}}{\pgfqpoint{0.558333in}{-0.015470in}}{\pgfqpoint{0.558333in}{0.000000in}}%
\pgfpathcurveto{\pgfqpoint{0.558333in}{0.015470in}}{\pgfqpoint{0.552187in}{0.030309in}}{\pgfqpoint{0.541248in}{0.041248in}}%
\pgfpathcurveto{\pgfqpoint{0.530309in}{0.052187in}}{\pgfqpoint{0.515470in}{0.058333in}}{\pgfqpoint{0.500000in}{0.058333in}}%
\pgfpathcurveto{\pgfqpoint{0.484530in}{0.058333in}}{\pgfqpoint{0.469691in}{0.052187in}}{\pgfqpoint{0.458752in}{0.041248in}}%
\pgfpathcurveto{\pgfqpoint{0.447813in}{0.030309in}}{\pgfqpoint{0.441667in}{0.015470in}}{\pgfqpoint{0.441667in}{0.000000in}}%
\pgfpathcurveto{\pgfqpoint{0.441667in}{-0.015470in}}{\pgfqpoint{0.447813in}{-0.030309in}}{\pgfqpoint{0.458752in}{-0.041248in}}%
\pgfpathcurveto{\pgfqpoint{0.469691in}{-0.052187in}}{\pgfqpoint{0.484530in}{-0.058333in}}{\pgfqpoint{0.500000in}{-0.058333in}}%
\pgfpathclose%
\pgfpathmoveto{\pgfqpoint{0.500000in}{-0.052500in}}%
\pgfpathcurveto{\pgfqpoint{0.500000in}{-0.052500in}}{\pgfqpoint{0.486077in}{-0.052500in}}{\pgfqpoint{0.472722in}{-0.046968in}}%
\pgfpathcurveto{\pgfqpoint{0.462877in}{-0.037123in}}{\pgfqpoint{0.453032in}{-0.027278in}}{\pgfqpoint{0.447500in}{-0.013923in}}%
\pgfpathcurveto{\pgfqpoint{0.447500in}{0.000000in}}{\pgfqpoint{0.447500in}{0.013923in}}{\pgfqpoint{0.453032in}{0.027278in}}%
\pgfpathcurveto{\pgfqpoint{0.462877in}{0.037123in}}{\pgfqpoint{0.472722in}{0.046968in}}{\pgfqpoint{0.486077in}{0.052500in}}%
\pgfpathcurveto{\pgfqpoint{0.500000in}{0.052500in}}{\pgfqpoint{0.513923in}{0.052500in}}{\pgfqpoint{0.527278in}{0.046968in}}%
\pgfpathcurveto{\pgfqpoint{0.537123in}{0.037123in}}{\pgfqpoint{0.546968in}{0.027278in}}{\pgfqpoint{0.552500in}{0.013923in}}%
\pgfpathcurveto{\pgfqpoint{0.552500in}{0.000000in}}{\pgfqpoint{0.552500in}{-0.013923in}}{\pgfqpoint{0.546968in}{-0.027278in}}%
\pgfpathcurveto{\pgfqpoint{0.537123in}{-0.037123in}}{\pgfqpoint{0.527278in}{-0.046968in}}{\pgfqpoint{0.513923in}{-0.052500in}}%
\pgfpathclose%
\pgfpathmoveto{\pgfqpoint{0.666667in}{-0.058333in}}%
\pgfpathcurveto{\pgfqpoint{0.682137in}{-0.058333in}}{\pgfqpoint{0.696975in}{-0.052187in}}{\pgfqpoint{0.707915in}{-0.041248in}}%
\pgfpathcurveto{\pgfqpoint{0.718854in}{-0.030309in}}{\pgfqpoint{0.725000in}{-0.015470in}}{\pgfqpoint{0.725000in}{0.000000in}}%
\pgfpathcurveto{\pgfqpoint{0.725000in}{0.015470in}}{\pgfqpoint{0.718854in}{0.030309in}}{\pgfqpoint{0.707915in}{0.041248in}}%
\pgfpathcurveto{\pgfqpoint{0.696975in}{0.052187in}}{\pgfqpoint{0.682137in}{0.058333in}}{\pgfqpoint{0.666667in}{0.058333in}}%
\pgfpathcurveto{\pgfqpoint{0.651196in}{0.058333in}}{\pgfqpoint{0.636358in}{0.052187in}}{\pgfqpoint{0.625419in}{0.041248in}}%
\pgfpathcurveto{\pgfqpoint{0.614480in}{0.030309in}}{\pgfqpoint{0.608333in}{0.015470in}}{\pgfqpoint{0.608333in}{0.000000in}}%
\pgfpathcurveto{\pgfqpoint{0.608333in}{-0.015470in}}{\pgfqpoint{0.614480in}{-0.030309in}}{\pgfqpoint{0.625419in}{-0.041248in}}%
\pgfpathcurveto{\pgfqpoint{0.636358in}{-0.052187in}}{\pgfqpoint{0.651196in}{-0.058333in}}{\pgfqpoint{0.666667in}{-0.058333in}}%
\pgfpathclose%
\pgfpathmoveto{\pgfqpoint{0.666667in}{-0.052500in}}%
\pgfpathcurveto{\pgfqpoint{0.666667in}{-0.052500in}}{\pgfqpoint{0.652744in}{-0.052500in}}{\pgfqpoint{0.639389in}{-0.046968in}}%
\pgfpathcurveto{\pgfqpoint{0.629544in}{-0.037123in}}{\pgfqpoint{0.619698in}{-0.027278in}}{\pgfqpoint{0.614167in}{-0.013923in}}%
\pgfpathcurveto{\pgfqpoint{0.614167in}{0.000000in}}{\pgfqpoint{0.614167in}{0.013923in}}{\pgfqpoint{0.619698in}{0.027278in}}%
\pgfpathcurveto{\pgfqpoint{0.629544in}{0.037123in}}{\pgfqpoint{0.639389in}{0.046968in}}{\pgfqpoint{0.652744in}{0.052500in}}%
\pgfpathcurveto{\pgfqpoint{0.666667in}{0.052500in}}{\pgfqpoint{0.680590in}{0.052500in}}{\pgfqpoint{0.693945in}{0.046968in}}%
\pgfpathcurveto{\pgfqpoint{0.703790in}{0.037123in}}{\pgfqpoint{0.713635in}{0.027278in}}{\pgfqpoint{0.719167in}{0.013923in}}%
\pgfpathcurveto{\pgfqpoint{0.719167in}{0.000000in}}{\pgfqpoint{0.719167in}{-0.013923in}}{\pgfqpoint{0.713635in}{-0.027278in}}%
\pgfpathcurveto{\pgfqpoint{0.703790in}{-0.037123in}}{\pgfqpoint{0.693945in}{-0.046968in}}{\pgfqpoint{0.680590in}{-0.052500in}}%
\pgfpathclose%
\pgfpathmoveto{\pgfqpoint{0.833333in}{-0.058333in}}%
\pgfpathcurveto{\pgfqpoint{0.848804in}{-0.058333in}}{\pgfqpoint{0.863642in}{-0.052187in}}{\pgfqpoint{0.874581in}{-0.041248in}}%
\pgfpathcurveto{\pgfqpoint{0.885520in}{-0.030309in}}{\pgfqpoint{0.891667in}{-0.015470in}}{\pgfqpoint{0.891667in}{0.000000in}}%
\pgfpathcurveto{\pgfqpoint{0.891667in}{0.015470in}}{\pgfqpoint{0.885520in}{0.030309in}}{\pgfqpoint{0.874581in}{0.041248in}}%
\pgfpathcurveto{\pgfqpoint{0.863642in}{0.052187in}}{\pgfqpoint{0.848804in}{0.058333in}}{\pgfqpoint{0.833333in}{0.058333in}}%
\pgfpathcurveto{\pgfqpoint{0.817863in}{0.058333in}}{\pgfqpoint{0.803025in}{0.052187in}}{\pgfqpoint{0.792085in}{0.041248in}}%
\pgfpathcurveto{\pgfqpoint{0.781146in}{0.030309in}}{\pgfqpoint{0.775000in}{0.015470in}}{\pgfqpoint{0.775000in}{0.000000in}}%
\pgfpathcurveto{\pgfqpoint{0.775000in}{-0.015470in}}{\pgfqpoint{0.781146in}{-0.030309in}}{\pgfqpoint{0.792085in}{-0.041248in}}%
\pgfpathcurveto{\pgfqpoint{0.803025in}{-0.052187in}}{\pgfqpoint{0.817863in}{-0.058333in}}{\pgfqpoint{0.833333in}{-0.058333in}}%
\pgfpathclose%
\pgfpathmoveto{\pgfqpoint{0.833333in}{-0.052500in}}%
\pgfpathcurveto{\pgfqpoint{0.833333in}{-0.052500in}}{\pgfqpoint{0.819410in}{-0.052500in}}{\pgfqpoint{0.806055in}{-0.046968in}}%
\pgfpathcurveto{\pgfqpoint{0.796210in}{-0.037123in}}{\pgfqpoint{0.786365in}{-0.027278in}}{\pgfqpoint{0.780833in}{-0.013923in}}%
\pgfpathcurveto{\pgfqpoint{0.780833in}{0.000000in}}{\pgfqpoint{0.780833in}{0.013923in}}{\pgfqpoint{0.786365in}{0.027278in}}%
\pgfpathcurveto{\pgfqpoint{0.796210in}{0.037123in}}{\pgfqpoint{0.806055in}{0.046968in}}{\pgfqpoint{0.819410in}{0.052500in}}%
\pgfpathcurveto{\pgfqpoint{0.833333in}{0.052500in}}{\pgfqpoint{0.847256in}{0.052500in}}{\pgfqpoint{0.860611in}{0.046968in}}%
\pgfpathcurveto{\pgfqpoint{0.870456in}{0.037123in}}{\pgfqpoint{0.880302in}{0.027278in}}{\pgfqpoint{0.885833in}{0.013923in}}%
\pgfpathcurveto{\pgfqpoint{0.885833in}{0.000000in}}{\pgfqpoint{0.885833in}{-0.013923in}}{\pgfqpoint{0.880302in}{-0.027278in}}%
\pgfpathcurveto{\pgfqpoint{0.870456in}{-0.037123in}}{\pgfqpoint{0.860611in}{-0.046968in}}{\pgfqpoint{0.847256in}{-0.052500in}}%
\pgfpathclose%
\pgfpathmoveto{\pgfqpoint{1.000000in}{-0.058333in}}%
\pgfpathcurveto{\pgfqpoint{1.015470in}{-0.058333in}}{\pgfqpoint{1.030309in}{-0.052187in}}{\pgfqpoint{1.041248in}{-0.041248in}}%
\pgfpathcurveto{\pgfqpoint{1.052187in}{-0.030309in}}{\pgfqpoint{1.058333in}{-0.015470in}}{\pgfqpoint{1.058333in}{0.000000in}}%
\pgfpathcurveto{\pgfqpoint{1.058333in}{0.015470in}}{\pgfqpoint{1.052187in}{0.030309in}}{\pgfqpoint{1.041248in}{0.041248in}}%
\pgfpathcurveto{\pgfqpoint{1.030309in}{0.052187in}}{\pgfqpoint{1.015470in}{0.058333in}}{\pgfqpoint{1.000000in}{0.058333in}}%
\pgfpathcurveto{\pgfqpoint{0.984530in}{0.058333in}}{\pgfqpoint{0.969691in}{0.052187in}}{\pgfqpoint{0.958752in}{0.041248in}}%
\pgfpathcurveto{\pgfqpoint{0.947813in}{0.030309in}}{\pgfqpoint{0.941667in}{0.015470in}}{\pgfqpoint{0.941667in}{0.000000in}}%
\pgfpathcurveto{\pgfqpoint{0.941667in}{-0.015470in}}{\pgfqpoint{0.947813in}{-0.030309in}}{\pgfqpoint{0.958752in}{-0.041248in}}%
\pgfpathcurveto{\pgfqpoint{0.969691in}{-0.052187in}}{\pgfqpoint{0.984530in}{-0.058333in}}{\pgfqpoint{1.000000in}{-0.058333in}}%
\pgfpathclose%
\pgfpathmoveto{\pgfqpoint{1.000000in}{-0.052500in}}%
\pgfpathcurveto{\pgfqpoint{1.000000in}{-0.052500in}}{\pgfqpoint{0.986077in}{-0.052500in}}{\pgfqpoint{0.972722in}{-0.046968in}}%
\pgfpathcurveto{\pgfqpoint{0.962877in}{-0.037123in}}{\pgfqpoint{0.953032in}{-0.027278in}}{\pgfqpoint{0.947500in}{-0.013923in}}%
\pgfpathcurveto{\pgfqpoint{0.947500in}{0.000000in}}{\pgfqpoint{0.947500in}{0.013923in}}{\pgfqpoint{0.953032in}{0.027278in}}%
\pgfpathcurveto{\pgfqpoint{0.962877in}{0.037123in}}{\pgfqpoint{0.972722in}{0.046968in}}{\pgfqpoint{0.986077in}{0.052500in}}%
\pgfpathcurveto{\pgfqpoint{1.000000in}{0.052500in}}{\pgfqpoint{1.013923in}{0.052500in}}{\pgfqpoint{1.027278in}{0.046968in}}%
\pgfpathcurveto{\pgfqpoint{1.037123in}{0.037123in}}{\pgfqpoint{1.046968in}{0.027278in}}{\pgfqpoint{1.052500in}{0.013923in}}%
\pgfpathcurveto{\pgfqpoint{1.052500in}{0.000000in}}{\pgfqpoint{1.052500in}{-0.013923in}}{\pgfqpoint{1.046968in}{-0.027278in}}%
\pgfpathcurveto{\pgfqpoint{1.037123in}{-0.037123in}}{\pgfqpoint{1.027278in}{-0.046968in}}{\pgfqpoint{1.013923in}{-0.052500in}}%
\pgfpathclose%
\pgfpathmoveto{\pgfqpoint{0.083333in}{0.108333in}}%
\pgfpathcurveto{\pgfqpoint{0.098804in}{0.108333in}}{\pgfqpoint{0.113642in}{0.114480in}}{\pgfqpoint{0.124581in}{0.125419in}}%
\pgfpathcurveto{\pgfqpoint{0.135520in}{0.136358in}}{\pgfqpoint{0.141667in}{0.151196in}}{\pgfqpoint{0.141667in}{0.166667in}}%
\pgfpathcurveto{\pgfqpoint{0.141667in}{0.182137in}}{\pgfqpoint{0.135520in}{0.196975in}}{\pgfqpoint{0.124581in}{0.207915in}}%
\pgfpathcurveto{\pgfqpoint{0.113642in}{0.218854in}}{\pgfqpoint{0.098804in}{0.225000in}}{\pgfqpoint{0.083333in}{0.225000in}}%
\pgfpathcurveto{\pgfqpoint{0.067863in}{0.225000in}}{\pgfqpoint{0.053025in}{0.218854in}}{\pgfqpoint{0.042085in}{0.207915in}}%
\pgfpathcurveto{\pgfqpoint{0.031146in}{0.196975in}}{\pgfqpoint{0.025000in}{0.182137in}}{\pgfqpoint{0.025000in}{0.166667in}}%
\pgfpathcurveto{\pgfqpoint{0.025000in}{0.151196in}}{\pgfqpoint{0.031146in}{0.136358in}}{\pgfqpoint{0.042085in}{0.125419in}}%
\pgfpathcurveto{\pgfqpoint{0.053025in}{0.114480in}}{\pgfqpoint{0.067863in}{0.108333in}}{\pgfqpoint{0.083333in}{0.108333in}}%
\pgfpathclose%
\pgfpathmoveto{\pgfqpoint{0.083333in}{0.114167in}}%
\pgfpathcurveto{\pgfqpoint{0.083333in}{0.114167in}}{\pgfqpoint{0.069410in}{0.114167in}}{\pgfqpoint{0.056055in}{0.119698in}}%
\pgfpathcurveto{\pgfqpoint{0.046210in}{0.129544in}}{\pgfqpoint{0.036365in}{0.139389in}}{\pgfqpoint{0.030833in}{0.152744in}}%
\pgfpathcurveto{\pgfqpoint{0.030833in}{0.166667in}}{\pgfqpoint{0.030833in}{0.180590in}}{\pgfqpoint{0.036365in}{0.193945in}}%
\pgfpathcurveto{\pgfqpoint{0.046210in}{0.203790in}}{\pgfqpoint{0.056055in}{0.213635in}}{\pgfqpoint{0.069410in}{0.219167in}}%
\pgfpathcurveto{\pgfqpoint{0.083333in}{0.219167in}}{\pgfqpoint{0.097256in}{0.219167in}}{\pgfqpoint{0.110611in}{0.213635in}}%
\pgfpathcurveto{\pgfqpoint{0.120456in}{0.203790in}}{\pgfqpoint{0.130302in}{0.193945in}}{\pgfqpoint{0.135833in}{0.180590in}}%
\pgfpathcurveto{\pgfqpoint{0.135833in}{0.166667in}}{\pgfqpoint{0.135833in}{0.152744in}}{\pgfqpoint{0.130302in}{0.139389in}}%
\pgfpathcurveto{\pgfqpoint{0.120456in}{0.129544in}}{\pgfqpoint{0.110611in}{0.119698in}}{\pgfqpoint{0.097256in}{0.114167in}}%
\pgfpathclose%
\pgfpathmoveto{\pgfqpoint{0.250000in}{0.108333in}}%
\pgfpathcurveto{\pgfqpoint{0.265470in}{0.108333in}}{\pgfqpoint{0.280309in}{0.114480in}}{\pgfqpoint{0.291248in}{0.125419in}}%
\pgfpathcurveto{\pgfqpoint{0.302187in}{0.136358in}}{\pgfqpoint{0.308333in}{0.151196in}}{\pgfqpoint{0.308333in}{0.166667in}}%
\pgfpathcurveto{\pgfqpoint{0.308333in}{0.182137in}}{\pgfqpoint{0.302187in}{0.196975in}}{\pgfqpoint{0.291248in}{0.207915in}}%
\pgfpathcurveto{\pgfqpoint{0.280309in}{0.218854in}}{\pgfqpoint{0.265470in}{0.225000in}}{\pgfqpoint{0.250000in}{0.225000in}}%
\pgfpathcurveto{\pgfqpoint{0.234530in}{0.225000in}}{\pgfqpoint{0.219691in}{0.218854in}}{\pgfqpoint{0.208752in}{0.207915in}}%
\pgfpathcurveto{\pgfqpoint{0.197813in}{0.196975in}}{\pgfqpoint{0.191667in}{0.182137in}}{\pgfqpoint{0.191667in}{0.166667in}}%
\pgfpathcurveto{\pgfqpoint{0.191667in}{0.151196in}}{\pgfqpoint{0.197813in}{0.136358in}}{\pgfqpoint{0.208752in}{0.125419in}}%
\pgfpathcurveto{\pgfqpoint{0.219691in}{0.114480in}}{\pgfqpoint{0.234530in}{0.108333in}}{\pgfqpoint{0.250000in}{0.108333in}}%
\pgfpathclose%
\pgfpathmoveto{\pgfqpoint{0.250000in}{0.114167in}}%
\pgfpathcurveto{\pgfqpoint{0.250000in}{0.114167in}}{\pgfqpoint{0.236077in}{0.114167in}}{\pgfqpoint{0.222722in}{0.119698in}}%
\pgfpathcurveto{\pgfqpoint{0.212877in}{0.129544in}}{\pgfqpoint{0.203032in}{0.139389in}}{\pgfqpoint{0.197500in}{0.152744in}}%
\pgfpathcurveto{\pgfqpoint{0.197500in}{0.166667in}}{\pgfqpoint{0.197500in}{0.180590in}}{\pgfqpoint{0.203032in}{0.193945in}}%
\pgfpathcurveto{\pgfqpoint{0.212877in}{0.203790in}}{\pgfqpoint{0.222722in}{0.213635in}}{\pgfqpoint{0.236077in}{0.219167in}}%
\pgfpathcurveto{\pgfqpoint{0.250000in}{0.219167in}}{\pgfqpoint{0.263923in}{0.219167in}}{\pgfqpoint{0.277278in}{0.213635in}}%
\pgfpathcurveto{\pgfqpoint{0.287123in}{0.203790in}}{\pgfqpoint{0.296968in}{0.193945in}}{\pgfqpoint{0.302500in}{0.180590in}}%
\pgfpathcurveto{\pgfqpoint{0.302500in}{0.166667in}}{\pgfqpoint{0.302500in}{0.152744in}}{\pgfqpoint{0.296968in}{0.139389in}}%
\pgfpathcurveto{\pgfqpoint{0.287123in}{0.129544in}}{\pgfqpoint{0.277278in}{0.119698in}}{\pgfqpoint{0.263923in}{0.114167in}}%
\pgfpathclose%
\pgfpathmoveto{\pgfqpoint{0.416667in}{0.108333in}}%
\pgfpathcurveto{\pgfqpoint{0.432137in}{0.108333in}}{\pgfqpoint{0.446975in}{0.114480in}}{\pgfqpoint{0.457915in}{0.125419in}}%
\pgfpathcurveto{\pgfqpoint{0.468854in}{0.136358in}}{\pgfqpoint{0.475000in}{0.151196in}}{\pgfqpoint{0.475000in}{0.166667in}}%
\pgfpathcurveto{\pgfqpoint{0.475000in}{0.182137in}}{\pgfqpoint{0.468854in}{0.196975in}}{\pgfqpoint{0.457915in}{0.207915in}}%
\pgfpathcurveto{\pgfqpoint{0.446975in}{0.218854in}}{\pgfqpoint{0.432137in}{0.225000in}}{\pgfqpoint{0.416667in}{0.225000in}}%
\pgfpathcurveto{\pgfqpoint{0.401196in}{0.225000in}}{\pgfqpoint{0.386358in}{0.218854in}}{\pgfqpoint{0.375419in}{0.207915in}}%
\pgfpathcurveto{\pgfqpoint{0.364480in}{0.196975in}}{\pgfqpoint{0.358333in}{0.182137in}}{\pgfqpoint{0.358333in}{0.166667in}}%
\pgfpathcurveto{\pgfqpoint{0.358333in}{0.151196in}}{\pgfqpoint{0.364480in}{0.136358in}}{\pgfqpoint{0.375419in}{0.125419in}}%
\pgfpathcurveto{\pgfqpoint{0.386358in}{0.114480in}}{\pgfqpoint{0.401196in}{0.108333in}}{\pgfqpoint{0.416667in}{0.108333in}}%
\pgfpathclose%
\pgfpathmoveto{\pgfqpoint{0.416667in}{0.114167in}}%
\pgfpathcurveto{\pgfqpoint{0.416667in}{0.114167in}}{\pgfqpoint{0.402744in}{0.114167in}}{\pgfqpoint{0.389389in}{0.119698in}}%
\pgfpathcurveto{\pgfqpoint{0.379544in}{0.129544in}}{\pgfqpoint{0.369698in}{0.139389in}}{\pgfqpoint{0.364167in}{0.152744in}}%
\pgfpathcurveto{\pgfqpoint{0.364167in}{0.166667in}}{\pgfqpoint{0.364167in}{0.180590in}}{\pgfqpoint{0.369698in}{0.193945in}}%
\pgfpathcurveto{\pgfqpoint{0.379544in}{0.203790in}}{\pgfqpoint{0.389389in}{0.213635in}}{\pgfqpoint{0.402744in}{0.219167in}}%
\pgfpathcurveto{\pgfqpoint{0.416667in}{0.219167in}}{\pgfqpoint{0.430590in}{0.219167in}}{\pgfqpoint{0.443945in}{0.213635in}}%
\pgfpathcurveto{\pgfqpoint{0.453790in}{0.203790in}}{\pgfqpoint{0.463635in}{0.193945in}}{\pgfqpoint{0.469167in}{0.180590in}}%
\pgfpathcurveto{\pgfqpoint{0.469167in}{0.166667in}}{\pgfqpoint{0.469167in}{0.152744in}}{\pgfqpoint{0.463635in}{0.139389in}}%
\pgfpathcurveto{\pgfqpoint{0.453790in}{0.129544in}}{\pgfqpoint{0.443945in}{0.119698in}}{\pgfqpoint{0.430590in}{0.114167in}}%
\pgfpathclose%
\pgfpathmoveto{\pgfqpoint{0.583333in}{0.108333in}}%
\pgfpathcurveto{\pgfqpoint{0.598804in}{0.108333in}}{\pgfqpoint{0.613642in}{0.114480in}}{\pgfqpoint{0.624581in}{0.125419in}}%
\pgfpathcurveto{\pgfqpoint{0.635520in}{0.136358in}}{\pgfqpoint{0.641667in}{0.151196in}}{\pgfqpoint{0.641667in}{0.166667in}}%
\pgfpathcurveto{\pgfqpoint{0.641667in}{0.182137in}}{\pgfqpoint{0.635520in}{0.196975in}}{\pgfqpoint{0.624581in}{0.207915in}}%
\pgfpathcurveto{\pgfqpoint{0.613642in}{0.218854in}}{\pgfqpoint{0.598804in}{0.225000in}}{\pgfqpoint{0.583333in}{0.225000in}}%
\pgfpathcurveto{\pgfqpoint{0.567863in}{0.225000in}}{\pgfqpoint{0.553025in}{0.218854in}}{\pgfqpoint{0.542085in}{0.207915in}}%
\pgfpathcurveto{\pgfqpoint{0.531146in}{0.196975in}}{\pgfqpoint{0.525000in}{0.182137in}}{\pgfqpoint{0.525000in}{0.166667in}}%
\pgfpathcurveto{\pgfqpoint{0.525000in}{0.151196in}}{\pgfqpoint{0.531146in}{0.136358in}}{\pgfqpoint{0.542085in}{0.125419in}}%
\pgfpathcurveto{\pgfqpoint{0.553025in}{0.114480in}}{\pgfqpoint{0.567863in}{0.108333in}}{\pgfqpoint{0.583333in}{0.108333in}}%
\pgfpathclose%
\pgfpathmoveto{\pgfqpoint{0.583333in}{0.114167in}}%
\pgfpathcurveto{\pgfqpoint{0.583333in}{0.114167in}}{\pgfqpoint{0.569410in}{0.114167in}}{\pgfqpoint{0.556055in}{0.119698in}}%
\pgfpathcurveto{\pgfqpoint{0.546210in}{0.129544in}}{\pgfqpoint{0.536365in}{0.139389in}}{\pgfqpoint{0.530833in}{0.152744in}}%
\pgfpathcurveto{\pgfqpoint{0.530833in}{0.166667in}}{\pgfqpoint{0.530833in}{0.180590in}}{\pgfqpoint{0.536365in}{0.193945in}}%
\pgfpathcurveto{\pgfqpoint{0.546210in}{0.203790in}}{\pgfqpoint{0.556055in}{0.213635in}}{\pgfqpoint{0.569410in}{0.219167in}}%
\pgfpathcurveto{\pgfqpoint{0.583333in}{0.219167in}}{\pgfqpoint{0.597256in}{0.219167in}}{\pgfqpoint{0.610611in}{0.213635in}}%
\pgfpathcurveto{\pgfqpoint{0.620456in}{0.203790in}}{\pgfqpoint{0.630302in}{0.193945in}}{\pgfqpoint{0.635833in}{0.180590in}}%
\pgfpathcurveto{\pgfqpoint{0.635833in}{0.166667in}}{\pgfqpoint{0.635833in}{0.152744in}}{\pgfqpoint{0.630302in}{0.139389in}}%
\pgfpathcurveto{\pgfqpoint{0.620456in}{0.129544in}}{\pgfqpoint{0.610611in}{0.119698in}}{\pgfqpoint{0.597256in}{0.114167in}}%
\pgfpathclose%
\pgfpathmoveto{\pgfqpoint{0.750000in}{0.108333in}}%
\pgfpathcurveto{\pgfqpoint{0.765470in}{0.108333in}}{\pgfqpoint{0.780309in}{0.114480in}}{\pgfqpoint{0.791248in}{0.125419in}}%
\pgfpathcurveto{\pgfqpoint{0.802187in}{0.136358in}}{\pgfqpoint{0.808333in}{0.151196in}}{\pgfqpoint{0.808333in}{0.166667in}}%
\pgfpathcurveto{\pgfqpoint{0.808333in}{0.182137in}}{\pgfqpoint{0.802187in}{0.196975in}}{\pgfqpoint{0.791248in}{0.207915in}}%
\pgfpathcurveto{\pgfqpoint{0.780309in}{0.218854in}}{\pgfqpoint{0.765470in}{0.225000in}}{\pgfqpoint{0.750000in}{0.225000in}}%
\pgfpathcurveto{\pgfqpoint{0.734530in}{0.225000in}}{\pgfqpoint{0.719691in}{0.218854in}}{\pgfqpoint{0.708752in}{0.207915in}}%
\pgfpathcurveto{\pgfqpoint{0.697813in}{0.196975in}}{\pgfqpoint{0.691667in}{0.182137in}}{\pgfqpoint{0.691667in}{0.166667in}}%
\pgfpathcurveto{\pgfqpoint{0.691667in}{0.151196in}}{\pgfqpoint{0.697813in}{0.136358in}}{\pgfqpoint{0.708752in}{0.125419in}}%
\pgfpathcurveto{\pgfqpoint{0.719691in}{0.114480in}}{\pgfqpoint{0.734530in}{0.108333in}}{\pgfqpoint{0.750000in}{0.108333in}}%
\pgfpathclose%
\pgfpathmoveto{\pgfqpoint{0.750000in}{0.114167in}}%
\pgfpathcurveto{\pgfqpoint{0.750000in}{0.114167in}}{\pgfqpoint{0.736077in}{0.114167in}}{\pgfqpoint{0.722722in}{0.119698in}}%
\pgfpathcurveto{\pgfqpoint{0.712877in}{0.129544in}}{\pgfqpoint{0.703032in}{0.139389in}}{\pgfqpoint{0.697500in}{0.152744in}}%
\pgfpathcurveto{\pgfqpoint{0.697500in}{0.166667in}}{\pgfqpoint{0.697500in}{0.180590in}}{\pgfqpoint{0.703032in}{0.193945in}}%
\pgfpathcurveto{\pgfqpoint{0.712877in}{0.203790in}}{\pgfqpoint{0.722722in}{0.213635in}}{\pgfqpoint{0.736077in}{0.219167in}}%
\pgfpathcurveto{\pgfqpoint{0.750000in}{0.219167in}}{\pgfqpoint{0.763923in}{0.219167in}}{\pgfqpoint{0.777278in}{0.213635in}}%
\pgfpathcurveto{\pgfqpoint{0.787123in}{0.203790in}}{\pgfqpoint{0.796968in}{0.193945in}}{\pgfqpoint{0.802500in}{0.180590in}}%
\pgfpathcurveto{\pgfqpoint{0.802500in}{0.166667in}}{\pgfqpoint{0.802500in}{0.152744in}}{\pgfqpoint{0.796968in}{0.139389in}}%
\pgfpathcurveto{\pgfqpoint{0.787123in}{0.129544in}}{\pgfqpoint{0.777278in}{0.119698in}}{\pgfqpoint{0.763923in}{0.114167in}}%
\pgfpathclose%
\pgfpathmoveto{\pgfqpoint{0.916667in}{0.108333in}}%
\pgfpathcurveto{\pgfqpoint{0.932137in}{0.108333in}}{\pgfqpoint{0.946975in}{0.114480in}}{\pgfqpoint{0.957915in}{0.125419in}}%
\pgfpathcurveto{\pgfqpoint{0.968854in}{0.136358in}}{\pgfqpoint{0.975000in}{0.151196in}}{\pgfqpoint{0.975000in}{0.166667in}}%
\pgfpathcurveto{\pgfqpoint{0.975000in}{0.182137in}}{\pgfqpoint{0.968854in}{0.196975in}}{\pgfqpoint{0.957915in}{0.207915in}}%
\pgfpathcurveto{\pgfqpoint{0.946975in}{0.218854in}}{\pgfqpoint{0.932137in}{0.225000in}}{\pgfqpoint{0.916667in}{0.225000in}}%
\pgfpathcurveto{\pgfqpoint{0.901196in}{0.225000in}}{\pgfqpoint{0.886358in}{0.218854in}}{\pgfqpoint{0.875419in}{0.207915in}}%
\pgfpathcurveto{\pgfqpoint{0.864480in}{0.196975in}}{\pgfqpoint{0.858333in}{0.182137in}}{\pgfqpoint{0.858333in}{0.166667in}}%
\pgfpathcurveto{\pgfqpoint{0.858333in}{0.151196in}}{\pgfqpoint{0.864480in}{0.136358in}}{\pgfqpoint{0.875419in}{0.125419in}}%
\pgfpathcurveto{\pgfqpoint{0.886358in}{0.114480in}}{\pgfqpoint{0.901196in}{0.108333in}}{\pgfqpoint{0.916667in}{0.108333in}}%
\pgfpathclose%
\pgfpathmoveto{\pgfqpoint{0.916667in}{0.114167in}}%
\pgfpathcurveto{\pgfqpoint{0.916667in}{0.114167in}}{\pgfqpoint{0.902744in}{0.114167in}}{\pgfqpoint{0.889389in}{0.119698in}}%
\pgfpathcurveto{\pgfqpoint{0.879544in}{0.129544in}}{\pgfqpoint{0.869698in}{0.139389in}}{\pgfqpoint{0.864167in}{0.152744in}}%
\pgfpathcurveto{\pgfqpoint{0.864167in}{0.166667in}}{\pgfqpoint{0.864167in}{0.180590in}}{\pgfqpoint{0.869698in}{0.193945in}}%
\pgfpathcurveto{\pgfqpoint{0.879544in}{0.203790in}}{\pgfqpoint{0.889389in}{0.213635in}}{\pgfqpoint{0.902744in}{0.219167in}}%
\pgfpathcurveto{\pgfqpoint{0.916667in}{0.219167in}}{\pgfqpoint{0.930590in}{0.219167in}}{\pgfqpoint{0.943945in}{0.213635in}}%
\pgfpathcurveto{\pgfqpoint{0.953790in}{0.203790in}}{\pgfqpoint{0.963635in}{0.193945in}}{\pgfqpoint{0.969167in}{0.180590in}}%
\pgfpathcurveto{\pgfqpoint{0.969167in}{0.166667in}}{\pgfqpoint{0.969167in}{0.152744in}}{\pgfqpoint{0.963635in}{0.139389in}}%
\pgfpathcurveto{\pgfqpoint{0.953790in}{0.129544in}}{\pgfqpoint{0.943945in}{0.119698in}}{\pgfqpoint{0.930590in}{0.114167in}}%
\pgfpathclose%
\pgfpathmoveto{\pgfqpoint{0.000000in}{0.275000in}}%
\pgfpathcurveto{\pgfqpoint{0.015470in}{0.275000in}}{\pgfqpoint{0.030309in}{0.281146in}}{\pgfqpoint{0.041248in}{0.292085in}}%
\pgfpathcurveto{\pgfqpoint{0.052187in}{0.303025in}}{\pgfqpoint{0.058333in}{0.317863in}}{\pgfqpoint{0.058333in}{0.333333in}}%
\pgfpathcurveto{\pgfqpoint{0.058333in}{0.348804in}}{\pgfqpoint{0.052187in}{0.363642in}}{\pgfqpoint{0.041248in}{0.374581in}}%
\pgfpathcurveto{\pgfqpoint{0.030309in}{0.385520in}}{\pgfqpoint{0.015470in}{0.391667in}}{\pgfqpoint{0.000000in}{0.391667in}}%
\pgfpathcurveto{\pgfqpoint{-0.015470in}{0.391667in}}{\pgfqpoint{-0.030309in}{0.385520in}}{\pgfqpoint{-0.041248in}{0.374581in}}%
\pgfpathcurveto{\pgfqpoint{-0.052187in}{0.363642in}}{\pgfqpoint{-0.058333in}{0.348804in}}{\pgfqpoint{-0.058333in}{0.333333in}}%
\pgfpathcurveto{\pgfqpoint{-0.058333in}{0.317863in}}{\pgfqpoint{-0.052187in}{0.303025in}}{\pgfqpoint{-0.041248in}{0.292085in}}%
\pgfpathcurveto{\pgfqpoint{-0.030309in}{0.281146in}}{\pgfqpoint{-0.015470in}{0.275000in}}{\pgfqpoint{0.000000in}{0.275000in}}%
\pgfpathclose%
\pgfpathmoveto{\pgfqpoint{0.000000in}{0.280833in}}%
\pgfpathcurveto{\pgfqpoint{0.000000in}{0.280833in}}{\pgfqpoint{-0.013923in}{0.280833in}}{\pgfqpoint{-0.027278in}{0.286365in}}%
\pgfpathcurveto{\pgfqpoint{-0.037123in}{0.296210in}}{\pgfqpoint{-0.046968in}{0.306055in}}{\pgfqpoint{-0.052500in}{0.319410in}}%
\pgfpathcurveto{\pgfqpoint{-0.052500in}{0.333333in}}{\pgfqpoint{-0.052500in}{0.347256in}}{\pgfqpoint{-0.046968in}{0.360611in}}%
\pgfpathcurveto{\pgfqpoint{-0.037123in}{0.370456in}}{\pgfqpoint{-0.027278in}{0.380302in}}{\pgfqpoint{-0.013923in}{0.385833in}}%
\pgfpathcurveto{\pgfqpoint{0.000000in}{0.385833in}}{\pgfqpoint{0.013923in}{0.385833in}}{\pgfqpoint{0.027278in}{0.380302in}}%
\pgfpathcurveto{\pgfqpoint{0.037123in}{0.370456in}}{\pgfqpoint{0.046968in}{0.360611in}}{\pgfqpoint{0.052500in}{0.347256in}}%
\pgfpathcurveto{\pgfqpoint{0.052500in}{0.333333in}}{\pgfqpoint{0.052500in}{0.319410in}}{\pgfqpoint{0.046968in}{0.306055in}}%
\pgfpathcurveto{\pgfqpoint{0.037123in}{0.296210in}}{\pgfqpoint{0.027278in}{0.286365in}}{\pgfqpoint{0.013923in}{0.280833in}}%
\pgfpathclose%
\pgfpathmoveto{\pgfqpoint{0.166667in}{0.275000in}}%
\pgfpathcurveto{\pgfqpoint{0.182137in}{0.275000in}}{\pgfqpoint{0.196975in}{0.281146in}}{\pgfqpoint{0.207915in}{0.292085in}}%
\pgfpathcurveto{\pgfqpoint{0.218854in}{0.303025in}}{\pgfqpoint{0.225000in}{0.317863in}}{\pgfqpoint{0.225000in}{0.333333in}}%
\pgfpathcurveto{\pgfqpoint{0.225000in}{0.348804in}}{\pgfqpoint{0.218854in}{0.363642in}}{\pgfqpoint{0.207915in}{0.374581in}}%
\pgfpathcurveto{\pgfqpoint{0.196975in}{0.385520in}}{\pgfqpoint{0.182137in}{0.391667in}}{\pgfqpoint{0.166667in}{0.391667in}}%
\pgfpathcurveto{\pgfqpoint{0.151196in}{0.391667in}}{\pgfqpoint{0.136358in}{0.385520in}}{\pgfqpoint{0.125419in}{0.374581in}}%
\pgfpathcurveto{\pgfqpoint{0.114480in}{0.363642in}}{\pgfqpoint{0.108333in}{0.348804in}}{\pgfqpoint{0.108333in}{0.333333in}}%
\pgfpathcurveto{\pgfqpoint{0.108333in}{0.317863in}}{\pgfqpoint{0.114480in}{0.303025in}}{\pgfqpoint{0.125419in}{0.292085in}}%
\pgfpathcurveto{\pgfqpoint{0.136358in}{0.281146in}}{\pgfqpoint{0.151196in}{0.275000in}}{\pgfqpoint{0.166667in}{0.275000in}}%
\pgfpathclose%
\pgfpathmoveto{\pgfqpoint{0.166667in}{0.280833in}}%
\pgfpathcurveto{\pgfqpoint{0.166667in}{0.280833in}}{\pgfqpoint{0.152744in}{0.280833in}}{\pgfqpoint{0.139389in}{0.286365in}}%
\pgfpathcurveto{\pgfqpoint{0.129544in}{0.296210in}}{\pgfqpoint{0.119698in}{0.306055in}}{\pgfqpoint{0.114167in}{0.319410in}}%
\pgfpathcurveto{\pgfqpoint{0.114167in}{0.333333in}}{\pgfqpoint{0.114167in}{0.347256in}}{\pgfqpoint{0.119698in}{0.360611in}}%
\pgfpathcurveto{\pgfqpoint{0.129544in}{0.370456in}}{\pgfqpoint{0.139389in}{0.380302in}}{\pgfqpoint{0.152744in}{0.385833in}}%
\pgfpathcurveto{\pgfqpoint{0.166667in}{0.385833in}}{\pgfqpoint{0.180590in}{0.385833in}}{\pgfqpoint{0.193945in}{0.380302in}}%
\pgfpathcurveto{\pgfqpoint{0.203790in}{0.370456in}}{\pgfqpoint{0.213635in}{0.360611in}}{\pgfqpoint{0.219167in}{0.347256in}}%
\pgfpathcurveto{\pgfqpoint{0.219167in}{0.333333in}}{\pgfqpoint{0.219167in}{0.319410in}}{\pgfqpoint{0.213635in}{0.306055in}}%
\pgfpathcurveto{\pgfqpoint{0.203790in}{0.296210in}}{\pgfqpoint{0.193945in}{0.286365in}}{\pgfqpoint{0.180590in}{0.280833in}}%
\pgfpathclose%
\pgfpathmoveto{\pgfqpoint{0.333333in}{0.275000in}}%
\pgfpathcurveto{\pgfqpoint{0.348804in}{0.275000in}}{\pgfqpoint{0.363642in}{0.281146in}}{\pgfqpoint{0.374581in}{0.292085in}}%
\pgfpathcurveto{\pgfqpoint{0.385520in}{0.303025in}}{\pgfqpoint{0.391667in}{0.317863in}}{\pgfqpoint{0.391667in}{0.333333in}}%
\pgfpathcurveto{\pgfqpoint{0.391667in}{0.348804in}}{\pgfqpoint{0.385520in}{0.363642in}}{\pgfqpoint{0.374581in}{0.374581in}}%
\pgfpathcurveto{\pgfqpoint{0.363642in}{0.385520in}}{\pgfqpoint{0.348804in}{0.391667in}}{\pgfqpoint{0.333333in}{0.391667in}}%
\pgfpathcurveto{\pgfqpoint{0.317863in}{0.391667in}}{\pgfqpoint{0.303025in}{0.385520in}}{\pgfqpoint{0.292085in}{0.374581in}}%
\pgfpathcurveto{\pgfqpoint{0.281146in}{0.363642in}}{\pgfqpoint{0.275000in}{0.348804in}}{\pgfqpoint{0.275000in}{0.333333in}}%
\pgfpathcurveto{\pgfqpoint{0.275000in}{0.317863in}}{\pgfqpoint{0.281146in}{0.303025in}}{\pgfqpoint{0.292085in}{0.292085in}}%
\pgfpathcurveto{\pgfqpoint{0.303025in}{0.281146in}}{\pgfqpoint{0.317863in}{0.275000in}}{\pgfqpoint{0.333333in}{0.275000in}}%
\pgfpathclose%
\pgfpathmoveto{\pgfqpoint{0.333333in}{0.280833in}}%
\pgfpathcurveto{\pgfqpoint{0.333333in}{0.280833in}}{\pgfqpoint{0.319410in}{0.280833in}}{\pgfqpoint{0.306055in}{0.286365in}}%
\pgfpathcurveto{\pgfqpoint{0.296210in}{0.296210in}}{\pgfqpoint{0.286365in}{0.306055in}}{\pgfqpoint{0.280833in}{0.319410in}}%
\pgfpathcurveto{\pgfqpoint{0.280833in}{0.333333in}}{\pgfqpoint{0.280833in}{0.347256in}}{\pgfqpoint{0.286365in}{0.360611in}}%
\pgfpathcurveto{\pgfqpoint{0.296210in}{0.370456in}}{\pgfqpoint{0.306055in}{0.380302in}}{\pgfqpoint{0.319410in}{0.385833in}}%
\pgfpathcurveto{\pgfqpoint{0.333333in}{0.385833in}}{\pgfqpoint{0.347256in}{0.385833in}}{\pgfqpoint{0.360611in}{0.380302in}}%
\pgfpathcurveto{\pgfqpoint{0.370456in}{0.370456in}}{\pgfqpoint{0.380302in}{0.360611in}}{\pgfqpoint{0.385833in}{0.347256in}}%
\pgfpathcurveto{\pgfqpoint{0.385833in}{0.333333in}}{\pgfqpoint{0.385833in}{0.319410in}}{\pgfqpoint{0.380302in}{0.306055in}}%
\pgfpathcurveto{\pgfqpoint{0.370456in}{0.296210in}}{\pgfqpoint{0.360611in}{0.286365in}}{\pgfqpoint{0.347256in}{0.280833in}}%
\pgfpathclose%
\pgfpathmoveto{\pgfqpoint{0.500000in}{0.275000in}}%
\pgfpathcurveto{\pgfqpoint{0.515470in}{0.275000in}}{\pgfqpoint{0.530309in}{0.281146in}}{\pgfqpoint{0.541248in}{0.292085in}}%
\pgfpathcurveto{\pgfqpoint{0.552187in}{0.303025in}}{\pgfqpoint{0.558333in}{0.317863in}}{\pgfqpoint{0.558333in}{0.333333in}}%
\pgfpathcurveto{\pgfqpoint{0.558333in}{0.348804in}}{\pgfqpoint{0.552187in}{0.363642in}}{\pgfqpoint{0.541248in}{0.374581in}}%
\pgfpathcurveto{\pgfqpoint{0.530309in}{0.385520in}}{\pgfqpoint{0.515470in}{0.391667in}}{\pgfqpoint{0.500000in}{0.391667in}}%
\pgfpathcurveto{\pgfqpoint{0.484530in}{0.391667in}}{\pgfqpoint{0.469691in}{0.385520in}}{\pgfqpoint{0.458752in}{0.374581in}}%
\pgfpathcurveto{\pgfqpoint{0.447813in}{0.363642in}}{\pgfqpoint{0.441667in}{0.348804in}}{\pgfqpoint{0.441667in}{0.333333in}}%
\pgfpathcurveto{\pgfqpoint{0.441667in}{0.317863in}}{\pgfqpoint{0.447813in}{0.303025in}}{\pgfqpoint{0.458752in}{0.292085in}}%
\pgfpathcurveto{\pgfqpoint{0.469691in}{0.281146in}}{\pgfqpoint{0.484530in}{0.275000in}}{\pgfqpoint{0.500000in}{0.275000in}}%
\pgfpathclose%
\pgfpathmoveto{\pgfqpoint{0.500000in}{0.280833in}}%
\pgfpathcurveto{\pgfqpoint{0.500000in}{0.280833in}}{\pgfqpoint{0.486077in}{0.280833in}}{\pgfqpoint{0.472722in}{0.286365in}}%
\pgfpathcurveto{\pgfqpoint{0.462877in}{0.296210in}}{\pgfqpoint{0.453032in}{0.306055in}}{\pgfqpoint{0.447500in}{0.319410in}}%
\pgfpathcurveto{\pgfqpoint{0.447500in}{0.333333in}}{\pgfqpoint{0.447500in}{0.347256in}}{\pgfqpoint{0.453032in}{0.360611in}}%
\pgfpathcurveto{\pgfqpoint{0.462877in}{0.370456in}}{\pgfqpoint{0.472722in}{0.380302in}}{\pgfqpoint{0.486077in}{0.385833in}}%
\pgfpathcurveto{\pgfqpoint{0.500000in}{0.385833in}}{\pgfqpoint{0.513923in}{0.385833in}}{\pgfqpoint{0.527278in}{0.380302in}}%
\pgfpathcurveto{\pgfqpoint{0.537123in}{0.370456in}}{\pgfqpoint{0.546968in}{0.360611in}}{\pgfqpoint{0.552500in}{0.347256in}}%
\pgfpathcurveto{\pgfqpoint{0.552500in}{0.333333in}}{\pgfqpoint{0.552500in}{0.319410in}}{\pgfqpoint{0.546968in}{0.306055in}}%
\pgfpathcurveto{\pgfqpoint{0.537123in}{0.296210in}}{\pgfqpoint{0.527278in}{0.286365in}}{\pgfqpoint{0.513923in}{0.280833in}}%
\pgfpathclose%
\pgfpathmoveto{\pgfqpoint{0.666667in}{0.275000in}}%
\pgfpathcurveto{\pgfqpoint{0.682137in}{0.275000in}}{\pgfqpoint{0.696975in}{0.281146in}}{\pgfqpoint{0.707915in}{0.292085in}}%
\pgfpathcurveto{\pgfqpoint{0.718854in}{0.303025in}}{\pgfqpoint{0.725000in}{0.317863in}}{\pgfqpoint{0.725000in}{0.333333in}}%
\pgfpathcurveto{\pgfqpoint{0.725000in}{0.348804in}}{\pgfqpoint{0.718854in}{0.363642in}}{\pgfqpoint{0.707915in}{0.374581in}}%
\pgfpathcurveto{\pgfqpoint{0.696975in}{0.385520in}}{\pgfqpoint{0.682137in}{0.391667in}}{\pgfqpoint{0.666667in}{0.391667in}}%
\pgfpathcurveto{\pgfqpoint{0.651196in}{0.391667in}}{\pgfqpoint{0.636358in}{0.385520in}}{\pgfqpoint{0.625419in}{0.374581in}}%
\pgfpathcurveto{\pgfqpoint{0.614480in}{0.363642in}}{\pgfqpoint{0.608333in}{0.348804in}}{\pgfqpoint{0.608333in}{0.333333in}}%
\pgfpathcurveto{\pgfqpoint{0.608333in}{0.317863in}}{\pgfqpoint{0.614480in}{0.303025in}}{\pgfqpoint{0.625419in}{0.292085in}}%
\pgfpathcurveto{\pgfqpoint{0.636358in}{0.281146in}}{\pgfqpoint{0.651196in}{0.275000in}}{\pgfqpoint{0.666667in}{0.275000in}}%
\pgfpathclose%
\pgfpathmoveto{\pgfqpoint{0.666667in}{0.280833in}}%
\pgfpathcurveto{\pgfqpoint{0.666667in}{0.280833in}}{\pgfqpoint{0.652744in}{0.280833in}}{\pgfqpoint{0.639389in}{0.286365in}}%
\pgfpathcurveto{\pgfqpoint{0.629544in}{0.296210in}}{\pgfqpoint{0.619698in}{0.306055in}}{\pgfqpoint{0.614167in}{0.319410in}}%
\pgfpathcurveto{\pgfqpoint{0.614167in}{0.333333in}}{\pgfqpoint{0.614167in}{0.347256in}}{\pgfqpoint{0.619698in}{0.360611in}}%
\pgfpathcurveto{\pgfqpoint{0.629544in}{0.370456in}}{\pgfqpoint{0.639389in}{0.380302in}}{\pgfqpoint{0.652744in}{0.385833in}}%
\pgfpathcurveto{\pgfqpoint{0.666667in}{0.385833in}}{\pgfqpoint{0.680590in}{0.385833in}}{\pgfqpoint{0.693945in}{0.380302in}}%
\pgfpathcurveto{\pgfqpoint{0.703790in}{0.370456in}}{\pgfqpoint{0.713635in}{0.360611in}}{\pgfqpoint{0.719167in}{0.347256in}}%
\pgfpathcurveto{\pgfqpoint{0.719167in}{0.333333in}}{\pgfqpoint{0.719167in}{0.319410in}}{\pgfqpoint{0.713635in}{0.306055in}}%
\pgfpathcurveto{\pgfqpoint{0.703790in}{0.296210in}}{\pgfqpoint{0.693945in}{0.286365in}}{\pgfqpoint{0.680590in}{0.280833in}}%
\pgfpathclose%
\pgfpathmoveto{\pgfqpoint{0.833333in}{0.275000in}}%
\pgfpathcurveto{\pgfqpoint{0.848804in}{0.275000in}}{\pgfqpoint{0.863642in}{0.281146in}}{\pgfqpoint{0.874581in}{0.292085in}}%
\pgfpathcurveto{\pgfqpoint{0.885520in}{0.303025in}}{\pgfqpoint{0.891667in}{0.317863in}}{\pgfqpoint{0.891667in}{0.333333in}}%
\pgfpathcurveto{\pgfqpoint{0.891667in}{0.348804in}}{\pgfqpoint{0.885520in}{0.363642in}}{\pgfqpoint{0.874581in}{0.374581in}}%
\pgfpathcurveto{\pgfqpoint{0.863642in}{0.385520in}}{\pgfqpoint{0.848804in}{0.391667in}}{\pgfqpoint{0.833333in}{0.391667in}}%
\pgfpathcurveto{\pgfqpoint{0.817863in}{0.391667in}}{\pgfqpoint{0.803025in}{0.385520in}}{\pgfqpoint{0.792085in}{0.374581in}}%
\pgfpathcurveto{\pgfqpoint{0.781146in}{0.363642in}}{\pgfqpoint{0.775000in}{0.348804in}}{\pgfqpoint{0.775000in}{0.333333in}}%
\pgfpathcurveto{\pgfqpoint{0.775000in}{0.317863in}}{\pgfqpoint{0.781146in}{0.303025in}}{\pgfqpoint{0.792085in}{0.292085in}}%
\pgfpathcurveto{\pgfqpoint{0.803025in}{0.281146in}}{\pgfqpoint{0.817863in}{0.275000in}}{\pgfqpoint{0.833333in}{0.275000in}}%
\pgfpathclose%
\pgfpathmoveto{\pgfqpoint{0.833333in}{0.280833in}}%
\pgfpathcurveto{\pgfqpoint{0.833333in}{0.280833in}}{\pgfqpoint{0.819410in}{0.280833in}}{\pgfqpoint{0.806055in}{0.286365in}}%
\pgfpathcurveto{\pgfqpoint{0.796210in}{0.296210in}}{\pgfqpoint{0.786365in}{0.306055in}}{\pgfqpoint{0.780833in}{0.319410in}}%
\pgfpathcurveto{\pgfqpoint{0.780833in}{0.333333in}}{\pgfqpoint{0.780833in}{0.347256in}}{\pgfqpoint{0.786365in}{0.360611in}}%
\pgfpathcurveto{\pgfqpoint{0.796210in}{0.370456in}}{\pgfqpoint{0.806055in}{0.380302in}}{\pgfqpoint{0.819410in}{0.385833in}}%
\pgfpathcurveto{\pgfqpoint{0.833333in}{0.385833in}}{\pgfqpoint{0.847256in}{0.385833in}}{\pgfqpoint{0.860611in}{0.380302in}}%
\pgfpathcurveto{\pgfqpoint{0.870456in}{0.370456in}}{\pgfqpoint{0.880302in}{0.360611in}}{\pgfqpoint{0.885833in}{0.347256in}}%
\pgfpathcurveto{\pgfqpoint{0.885833in}{0.333333in}}{\pgfqpoint{0.885833in}{0.319410in}}{\pgfqpoint{0.880302in}{0.306055in}}%
\pgfpathcurveto{\pgfqpoint{0.870456in}{0.296210in}}{\pgfqpoint{0.860611in}{0.286365in}}{\pgfqpoint{0.847256in}{0.280833in}}%
\pgfpathclose%
\pgfpathmoveto{\pgfqpoint{1.000000in}{0.275000in}}%
\pgfpathcurveto{\pgfqpoint{1.015470in}{0.275000in}}{\pgfqpoint{1.030309in}{0.281146in}}{\pgfqpoint{1.041248in}{0.292085in}}%
\pgfpathcurveto{\pgfqpoint{1.052187in}{0.303025in}}{\pgfqpoint{1.058333in}{0.317863in}}{\pgfqpoint{1.058333in}{0.333333in}}%
\pgfpathcurveto{\pgfqpoint{1.058333in}{0.348804in}}{\pgfqpoint{1.052187in}{0.363642in}}{\pgfqpoint{1.041248in}{0.374581in}}%
\pgfpathcurveto{\pgfqpoint{1.030309in}{0.385520in}}{\pgfqpoint{1.015470in}{0.391667in}}{\pgfqpoint{1.000000in}{0.391667in}}%
\pgfpathcurveto{\pgfqpoint{0.984530in}{0.391667in}}{\pgfqpoint{0.969691in}{0.385520in}}{\pgfqpoint{0.958752in}{0.374581in}}%
\pgfpathcurveto{\pgfqpoint{0.947813in}{0.363642in}}{\pgfqpoint{0.941667in}{0.348804in}}{\pgfqpoint{0.941667in}{0.333333in}}%
\pgfpathcurveto{\pgfqpoint{0.941667in}{0.317863in}}{\pgfqpoint{0.947813in}{0.303025in}}{\pgfqpoint{0.958752in}{0.292085in}}%
\pgfpathcurveto{\pgfqpoint{0.969691in}{0.281146in}}{\pgfqpoint{0.984530in}{0.275000in}}{\pgfqpoint{1.000000in}{0.275000in}}%
\pgfpathclose%
\pgfpathmoveto{\pgfqpoint{1.000000in}{0.280833in}}%
\pgfpathcurveto{\pgfqpoint{1.000000in}{0.280833in}}{\pgfqpoint{0.986077in}{0.280833in}}{\pgfqpoint{0.972722in}{0.286365in}}%
\pgfpathcurveto{\pgfqpoint{0.962877in}{0.296210in}}{\pgfqpoint{0.953032in}{0.306055in}}{\pgfqpoint{0.947500in}{0.319410in}}%
\pgfpathcurveto{\pgfqpoint{0.947500in}{0.333333in}}{\pgfqpoint{0.947500in}{0.347256in}}{\pgfqpoint{0.953032in}{0.360611in}}%
\pgfpathcurveto{\pgfqpoint{0.962877in}{0.370456in}}{\pgfqpoint{0.972722in}{0.380302in}}{\pgfqpoint{0.986077in}{0.385833in}}%
\pgfpathcurveto{\pgfqpoint{1.000000in}{0.385833in}}{\pgfqpoint{1.013923in}{0.385833in}}{\pgfqpoint{1.027278in}{0.380302in}}%
\pgfpathcurveto{\pgfqpoint{1.037123in}{0.370456in}}{\pgfqpoint{1.046968in}{0.360611in}}{\pgfqpoint{1.052500in}{0.347256in}}%
\pgfpathcurveto{\pgfqpoint{1.052500in}{0.333333in}}{\pgfqpoint{1.052500in}{0.319410in}}{\pgfqpoint{1.046968in}{0.306055in}}%
\pgfpathcurveto{\pgfqpoint{1.037123in}{0.296210in}}{\pgfqpoint{1.027278in}{0.286365in}}{\pgfqpoint{1.013923in}{0.280833in}}%
\pgfpathclose%
\pgfpathmoveto{\pgfqpoint{0.083333in}{0.441667in}}%
\pgfpathcurveto{\pgfqpoint{0.098804in}{0.441667in}}{\pgfqpoint{0.113642in}{0.447813in}}{\pgfqpoint{0.124581in}{0.458752in}}%
\pgfpathcurveto{\pgfqpoint{0.135520in}{0.469691in}}{\pgfqpoint{0.141667in}{0.484530in}}{\pgfqpoint{0.141667in}{0.500000in}}%
\pgfpathcurveto{\pgfqpoint{0.141667in}{0.515470in}}{\pgfqpoint{0.135520in}{0.530309in}}{\pgfqpoint{0.124581in}{0.541248in}}%
\pgfpathcurveto{\pgfqpoint{0.113642in}{0.552187in}}{\pgfqpoint{0.098804in}{0.558333in}}{\pgfqpoint{0.083333in}{0.558333in}}%
\pgfpathcurveto{\pgfqpoint{0.067863in}{0.558333in}}{\pgfqpoint{0.053025in}{0.552187in}}{\pgfqpoint{0.042085in}{0.541248in}}%
\pgfpathcurveto{\pgfqpoint{0.031146in}{0.530309in}}{\pgfqpoint{0.025000in}{0.515470in}}{\pgfqpoint{0.025000in}{0.500000in}}%
\pgfpathcurveto{\pgfqpoint{0.025000in}{0.484530in}}{\pgfqpoint{0.031146in}{0.469691in}}{\pgfqpoint{0.042085in}{0.458752in}}%
\pgfpathcurveto{\pgfqpoint{0.053025in}{0.447813in}}{\pgfqpoint{0.067863in}{0.441667in}}{\pgfqpoint{0.083333in}{0.441667in}}%
\pgfpathclose%
\pgfpathmoveto{\pgfqpoint{0.083333in}{0.447500in}}%
\pgfpathcurveto{\pgfqpoint{0.083333in}{0.447500in}}{\pgfqpoint{0.069410in}{0.447500in}}{\pgfqpoint{0.056055in}{0.453032in}}%
\pgfpathcurveto{\pgfqpoint{0.046210in}{0.462877in}}{\pgfqpoint{0.036365in}{0.472722in}}{\pgfqpoint{0.030833in}{0.486077in}}%
\pgfpathcurveto{\pgfqpoint{0.030833in}{0.500000in}}{\pgfqpoint{0.030833in}{0.513923in}}{\pgfqpoint{0.036365in}{0.527278in}}%
\pgfpathcurveto{\pgfqpoint{0.046210in}{0.537123in}}{\pgfqpoint{0.056055in}{0.546968in}}{\pgfqpoint{0.069410in}{0.552500in}}%
\pgfpathcurveto{\pgfqpoint{0.083333in}{0.552500in}}{\pgfqpoint{0.097256in}{0.552500in}}{\pgfqpoint{0.110611in}{0.546968in}}%
\pgfpathcurveto{\pgfqpoint{0.120456in}{0.537123in}}{\pgfqpoint{0.130302in}{0.527278in}}{\pgfqpoint{0.135833in}{0.513923in}}%
\pgfpathcurveto{\pgfqpoint{0.135833in}{0.500000in}}{\pgfqpoint{0.135833in}{0.486077in}}{\pgfqpoint{0.130302in}{0.472722in}}%
\pgfpathcurveto{\pgfqpoint{0.120456in}{0.462877in}}{\pgfqpoint{0.110611in}{0.453032in}}{\pgfqpoint{0.097256in}{0.447500in}}%
\pgfpathclose%
\pgfpathmoveto{\pgfqpoint{0.250000in}{0.441667in}}%
\pgfpathcurveto{\pgfqpoint{0.265470in}{0.441667in}}{\pgfqpoint{0.280309in}{0.447813in}}{\pgfqpoint{0.291248in}{0.458752in}}%
\pgfpathcurveto{\pgfqpoint{0.302187in}{0.469691in}}{\pgfqpoint{0.308333in}{0.484530in}}{\pgfqpoint{0.308333in}{0.500000in}}%
\pgfpathcurveto{\pgfqpoint{0.308333in}{0.515470in}}{\pgfqpoint{0.302187in}{0.530309in}}{\pgfqpoint{0.291248in}{0.541248in}}%
\pgfpathcurveto{\pgfqpoint{0.280309in}{0.552187in}}{\pgfqpoint{0.265470in}{0.558333in}}{\pgfqpoint{0.250000in}{0.558333in}}%
\pgfpathcurveto{\pgfqpoint{0.234530in}{0.558333in}}{\pgfqpoint{0.219691in}{0.552187in}}{\pgfqpoint{0.208752in}{0.541248in}}%
\pgfpathcurveto{\pgfqpoint{0.197813in}{0.530309in}}{\pgfqpoint{0.191667in}{0.515470in}}{\pgfqpoint{0.191667in}{0.500000in}}%
\pgfpathcurveto{\pgfqpoint{0.191667in}{0.484530in}}{\pgfqpoint{0.197813in}{0.469691in}}{\pgfqpoint{0.208752in}{0.458752in}}%
\pgfpathcurveto{\pgfqpoint{0.219691in}{0.447813in}}{\pgfqpoint{0.234530in}{0.441667in}}{\pgfqpoint{0.250000in}{0.441667in}}%
\pgfpathclose%
\pgfpathmoveto{\pgfqpoint{0.250000in}{0.447500in}}%
\pgfpathcurveto{\pgfqpoint{0.250000in}{0.447500in}}{\pgfqpoint{0.236077in}{0.447500in}}{\pgfqpoint{0.222722in}{0.453032in}}%
\pgfpathcurveto{\pgfqpoint{0.212877in}{0.462877in}}{\pgfqpoint{0.203032in}{0.472722in}}{\pgfqpoint{0.197500in}{0.486077in}}%
\pgfpathcurveto{\pgfqpoint{0.197500in}{0.500000in}}{\pgfqpoint{0.197500in}{0.513923in}}{\pgfqpoint{0.203032in}{0.527278in}}%
\pgfpathcurveto{\pgfqpoint{0.212877in}{0.537123in}}{\pgfqpoint{0.222722in}{0.546968in}}{\pgfqpoint{0.236077in}{0.552500in}}%
\pgfpathcurveto{\pgfqpoint{0.250000in}{0.552500in}}{\pgfqpoint{0.263923in}{0.552500in}}{\pgfqpoint{0.277278in}{0.546968in}}%
\pgfpathcurveto{\pgfqpoint{0.287123in}{0.537123in}}{\pgfqpoint{0.296968in}{0.527278in}}{\pgfqpoint{0.302500in}{0.513923in}}%
\pgfpathcurveto{\pgfqpoint{0.302500in}{0.500000in}}{\pgfqpoint{0.302500in}{0.486077in}}{\pgfqpoint{0.296968in}{0.472722in}}%
\pgfpathcurveto{\pgfqpoint{0.287123in}{0.462877in}}{\pgfqpoint{0.277278in}{0.453032in}}{\pgfqpoint{0.263923in}{0.447500in}}%
\pgfpathclose%
\pgfpathmoveto{\pgfqpoint{0.416667in}{0.441667in}}%
\pgfpathcurveto{\pgfqpoint{0.432137in}{0.441667in}}{\pgfqpoint{0.446975in}{0.447813in}}{\pgfqpoint{0.457915in}{0.458752in}}%
\pgfpathcurveto{\pgfqpoint{0.468854in}{0.469691in}}{\pgfqpoint{0.475000in}{0.484530in}}{\pgfqpoint{0.475000in}{0.500000in}}%
\pgfpathcurveto{\pgfqpoint{0.475000in}{0.515470in}}{\pgfqpoint{0.468854in}{0.530309in}}{\pgfqpoint{0.457915in}{0.541248in}}%
\pgfpathcurveto{\pgfqpoint{0.446975in}{0.552187in}}{\pgfqpoint{0.432137in}{0.558333in}}{\pgfqpoint{0.416667in}{0.558333in}}%
\pgfpathcurveto{\pgfqpoint{0.401196in}{0.558333in}}{\pgfqpoint{0.386358in}{0.552187in}}{\pgfqpoint{0.375419in}{0.541248in}}%
\pgfpathcurveto{\pgfqpoint{0.364480in}{0.530309in}}{\pgfqpoint{0.358333in}{0.515470in}}{\pgfqpoint{0.358333in}{0.500000in}}%
\pgfpathcurveto{\pgfqpoint{0.358333in}{0.484530in}}{\pgfqpoint{0.364480in}{0.469691in}}{\pgfqpoint{0.375419in}{0.458752in}}%
\pgfpathcurveto{\pgfqpoint{0.386358in}{0.447813in}}{\pgfqpoint{0.401196in}{0.441667in}}{\pgfqpoint{0.416667in}{0.441667in}}%
\pgfpathclose%
\pgfpathmoveto{\pgfqpoint{0.416667in}{0.447500in}}%
\pgfpathcurveto{\pgfqpoint{0.416667in}{0.447500in}}{\pgfqpoint{0.402744in}{0.447500in}}{\pgfqpoint{0.389389in}{0.453032in}}%
\pgfpathcurveto{\pgfqpoint{0.379544in}{0.462877in}}{\pgfqpoint{0.369698in}{0.472722in}}{\pgfqpoint{0.364167in}{0.486077in}}%
\pgfpathcurveto{\pgfqpoint{0.364167in}{0.500000in}}{\pgfqpoint{0.364167in}{0.513923in}}{\pgfqpoint{0.369698in}{0.527278in}}%
\pgfpathcurveto{\pgfqpoint{0.379544in}{0.537123in}}{\pgfqpoint{0.389389in}{0.546968in}}{\pgfqpoint{0.402744in}{0.552500in}}%
\pgfpathcurveto{\pgfqpoint{0.416667in}{0.552500in}}{\pgfqpoint{0.430590in}{0.552500in}}{\pgfqpoint{0.443945in}{0.546968in}}%
\pgfpathcurveto{\pgfqpoint{0.453790in}{0.537123in}}{\pgfqpoint{0.463635in}{0.527278in}}{\pgfqpoint{0.469167in}{0.513923in}}%
\pgfpathcurveto{\pgfqpoint{0.469167in}{0.500000in}}{\pgfqpoint{0.469167in}{0.486077in}}{\pgfqpoint{0.463635in}{0.472722in}}%
\pgfpathcurveto{\pgfqpoint{0.453790in}{0.462877in}}{\pgfqpoint{0.443945in}{0.453032in}}{\pgfqpoint{0.430590in}{0.447500in}}%
\pgfpathclose%
\pgfpathmoveto{\pgfqpoint{0.583333in}{0.441667in}}%
\pgfpathcurveto{\pgfqpoint{0.598804in}{0.441667in}}{\pgfqpoint{0.613642in}{0.447813in}}{\pgfqpoint{0.624581in}{0.458752in}}%
\pgfpathcurveto{\pgfqpoint{0.635520in}{0.469691in}}{\pgfqpoint{0.641667in}{0.484530in}}{\pgfqpoint{0.641667in}{0.500000in}}%
\pgfpathcurveto{\pgfqpoint{0.641667in}{0.515470in}}{\pgfqpoint{0.635520in}{0.530309in}}{\pgfqpoint{0.624581in}{0.541248in}}%
\pgfpathcurveto{\pgfqpoint{0.613642in}{0.552187in}}{\pgfqpoint{0.598804in}{0.558333in}}{\pgfqpoint{0.583333in}{0.558333in}}%
\pgfpathcurveto{\pgfqpoint{0.567863in}{0.558333in}}{\pgfqpoint{0.553025in}{0.552187in}}{\pgfqpoint{0.542085in}{0.541248in}}%
\pgfpathcurveto{\pgfqpoint{0.531146in}{0.530309in}}{\pgfqpoint{0.525000in}{0.515470in}}{\pgfqpoint{0.525000in}{0.500000in}}%
\pgfpathcurveto{\pgfqpoint{0.525000in}{0.484530in}}{\pgfqpoint{0.531146in}{0.469691in}}{\pgfqpoint{0.542085in}{0.458752in}}%
\pgfpathcurveto{\pgfqpoint{0.553025in}{0.447813in}}{\pgfqpoint{0.567863in}{0.441667in}}{\pgfqpoint{0.583333in}{0.441667in}}%
\pgfpathclose%
\pgfpathmoveto{\pgfqpoint{0.583333in}{0.447500in}}%
\pgfpathcurveto{\pgfqpoint{0.583333in}{0.447500in}}{\pgfqpoint{0.569410in}{0.447500in}}{\pgfqpoint{0.556055in}{0.453032in}}%
\pgfpathcurveto{\pgfqpoint{0.546210in}{0.462877in}}{\pgfqpoint{0.536365in}{0.472722in}}{\pgfqpoint{0.530833in}{0.486077in}}%
\pgfpathcurveto{\pgfqpoint{0.530833in}{0.500000in}}{\pgfqpoint{0.530833in}{0.513923in}}{\pgfqpoint{0.536365in}{0.527278in}}%
\pgfpathcurveto{\pgfqpoint{0.546210in}{0.537123in}}{\pgfqpoint{0.556055in}{0.546968in}}{\pgfqpoint{0.569410in}{0.552500in}}%
\pgfpathcurveto{\pgfqpoint{0.583333in}{0.552500in}}{\pgfqpoint{0.597256in}{0.552500in}}{\pgfqpoint{0.610611in}{0.546968in}}%
\pgfpathcurveto{\pgfqpoint{0.620456in}{0.537123in}}{\pgfqpoint{0.630302in}{0.527278in}}{\pgfqpoint{0.635833in}{0.513923in}}%
\pgfpathcurveto{\pgfqpoint{0.635833in}{0.500000in}}{\pgfqpoint{0.635833in}{0.486077in}}{\pgfqpoint{0.630302in}{0.472722in}}%
\pgfpathcurveto{\pgfqpoint{0.620456in}{0.462877in}}{\pgfqpoint{0.610611in}{0.453032in}}{\pgfqpoint{0.597256in}{0.447500in}}%
\pgfpathclose%
\pgfpathmoveto{\pgfqpoint{0.750000in}{0.441667in}}%
\pgfpathcurveto{\pgfqpoint{0.765470in}{0.441667in}}{\pgfqpoint{0.780309in}{0.447813in}}{\pgfqpoint{0.791248in}{0.458752in}}%
\pgfpathcurveto{\pgfqpoint{0.802187in}{0.469691in}}{\pgfqpoint{0.808333in}{0.484530in}}{\pgfqpoint{0.808333in}{0.500000in}}%
\pgfpathcurveto{\pgfqpoint{0.808333in}{0.515470in}}{\pgfqpoint{0.802187in}{0.530309in}}{\pgfqpoint{0.791248in}{0.541248in}}%
\pgfpathcurveto{\pgfqpoint{0.780309in}{0.552187in}}{\pgfqpoint{0.765470in}{0.558333in}}{\pgfqpoint{0.750000in}{0.558333in}}%
\pgfpathcurveto{\pgfqpoint{0.734530in}{0.558333in}}{\pgfqpoint{0.719691in}{0.552187in}}{\pgfqpoint{0.708752in}{0.541248in}}%
\pgfpathcurveto{\pgfqpoint{0.697813in}{0.530309in}}{\pgfqpoint{0.691667in}{0.515470in}}{\pgfqpoint{0.691667in}{0.500000in}}%
\pgfpathcurveto{\pgfqpoint{0.691667in}{0.484530in}}{\pgfqpoint{0.697813in}{0.469691in}}{\pgfqpoint{0.708752in}{0.458752in}}%
\pgfpathcurveto{\pgfqpoint{0.719691in}{0.447813in}}{\pgfqpoint{0.734530in}{0.441667in}}{\pgfqpoint{0.750000in}{0.441667in}}%
\pgfpathclose%
\pgfpathmoveto{\pgfqpoint{0.750000in}{0.447500in}}%
\pgfpathcurveto{\pgfqpoint{0.750000in}{0.447500in}}{\pgfqpoint{0.736077in}{0.447500in}}{\pgfqpoint{0.722722in}{0.453032in}}%
\pgfpathcurveto{\pgfqpoint{0.712877in}{0.462877in}}{\pgfqpoint{0.703032in}{0.472722in}}{\pgfqpoint{0.697500in}{0.486077in}}%
\pgfpathcurveto{\pgfqpoint{0.697500in}{0.500000in}}{\pgfqpoint{0.697500in}{0.513923in}}{\pgfqpoint{0.703032in}{0.527278in}}%
\pgfpathcurveto{\pgfqpoint{0.712877in}{0.537123in}}{\pgfqpoint{0.722722in}{0.546968in}}{\pgfqpoint{0.736077in}{0.552500in}}%
\pgfpathcurveto{\pgfqpoint{0.750000in}{0.552500in}}{\pgfqpoint{0.763923in}{0.552500in}}{\pgfqpoint{0.777278in}{0.546968in}}%
\pgfpathcurveto{\pgfqpoint{0.787123in}{0.537123in}}{\pgfqpoint{0.796968in}{0.527278in}}{\pgfqpoint{0.802500in}{0.513923in}}%
\pgfpathcurveto{\pgfqpoint{0.802500in}{0.500000in}}{\pgfqpoint{0.802500in}{0.486077in}}{\pgfqpoint{0.796968in}{0.472722in}}%
\pgfpathcurveto{\pgfqpoint{0.787123in}{0.462877in}}{\pgfqpoint{0.777278in}{0.453032in}}{\pgfqpoint{0.763923in}{0.447500in}}%
\pgfpathclose%
\pgfpathmoveto{\pgfqpoint{0.916667in}{0.441667in}}%
\pgfpathcurveto{\pgfqpoint{0.932137in}{0.441667in}}{\pgfqpoint{0.946975in}{0.447813in}}{\pgfqpoint{0.957915in}{0.458752in}}%
\pgfpathcurveto{\pgfqpoint{0.968854in}{0.469691in}}{\pgfqpoint{0.975000in}{0.484530in}}{\pgfqpoint{0.975000in}{0.500000in}}%
\pgfpathcurveto{\pgfqpoint{0.975000in}{0.515470in}}{\pgfqpoint{0.968854in}{0.530309in}}{\pgfqpoint{0.957915in}{0.541248in}}%
\pgfpathcurveto{\pgfqpoint{0.946975in}{0.552187in}}{\pgfqpoint{0.932137in}{0.558333in}}{\pgfqpoint{0.916667in}{0.558333in}}%
\pgfpathcurveto{\pgfqpoint{0.901196in}{0.558333in}}{\pgfqpoint{0.886358in}{0.552187in}}{\pgfqpoint{0.875419in}{0.541248in}}%
\pgfpathcurveto{\pgfqpoint{0.864480in}{0.530309in}}{\pgfqpoint{0.858333in}{0.515470in}}{\pgfqpoint{0.858333in}{0.500000in}}%
\pgfpathcurveto{\pgfqpoint{0.858333in}{0.484530in}}{\pgfqpoint{0.864480in}{0.469691in}}{\pgfqpoint{0.875419in}{0.458752in}}%
\pgfpathcurveto{\pgfqpoint{0.886358in}{0.447813in}}{\pgfqpoint{0.901196in}{0.441667in}}{\pgfqpoint{0.916667in}{0.441667in}}%
\pgfpathclose%
\pgfpathmoveto{\pgfqpoint{0.916667in}{0.447500in}}%
\pgfpathcurveto{\pgfqpoint{0.916667in}{0.447500in}}{\pgfqpoint{0.902744in}{0.447500in}}{\pgfqpoint{0.889389in}{0.453032in}}%
\pgfpathcurveto{\pgfqpoint{0.879544in}{0.462877in}}{\pgfqpoint{0.869698in}{0.472722in}}{\pgfqpoint{0.864167in}{0.486077in}}%
\pgfpathcurveto{\pgfqpoint{0.864167in}{0.500000in}}{\pgfqpoint{0.864167in}{0.513923in}}{\pgfqpoint{0.869698in}{0.527278in}}%
\pgfpathcurveto{\pgfqpoint{0.879544in}{0.537123in}}{\pgfqpoint{0.889389in}{0.546968in}}{\pgfqpoint{0.902744in}{0.552500in}}%
\pgfpathcurveto{\pgfqpoint{0.916667in}{0.552500in}}{\pgfqpoint{0.930590in}{0.552500in}}{\pgfqpoint{0.943945in}{0.546968in}}%
\pgfpathcurveto{\pgfqpoint{0.953790in}{0.537123in}}{\pgfqpoint{0.963635in}{0.527278in}}{\pgfqpoint{0.969167in}{0.513923in}}%
\pgfpathcurveto{\pgfqpoint{0.969167in}{0.500000in}}{\pgfqpoint{0.969167in}{0.486077in}}{\pgfqpoint{0.963635in}{0.472722in}}%
\pgfpathcurveto{\pgfqpoint{0.953790in}{0.462877in}}{\pgfqpoint{0.943945in}{0.453032in}}{\pgfqpoint{0.930590in}{0.447500in}}%
\pgfpathclose%
\pgfpathmoveto{\pgfqpoint{0.000000in}{0.608333in}}%
\pgfpathcurveto{\pgfqpoint{0.015470in}{0.608333in}}{\pgfqpoint{0.030309in}{0.614480in}}{\pgfqpoint{0.041248in}{0.625419in}}%
\pgfpathcurveto{\pgfqpoint{0.052187in}{0.636358in}}{\pgfqpoint{0.058333in}{0.651196in}}{\pgfqpoint{0.058333in}{0.666667in}}%
\pgfpathcurveto{\pgfqpoint{0.058333in}{0.682137in}}{\pgfqpoint{0.052187in}{0.696975in}}{\pgfqpoint{0.041248in}{0.707915in}}%
\pgfpathcurveto{\pgfqpoint{0.030309in}{0.718854in}}{\pgfqpoint{0.015470in}{0.725000in}}{\pgfqpoint{0.000000in}{0.725000in}}%
\pgfpathcurveto{\pgfqpoint{-0.015470in}{0.725000in}}{\pgfqpoint{-0.030309in}{0.718854in}}{\pgfqpoint{-0.041248in}{0.707915in}}%
\pgfpathcurveto{\pgfqpoint{-0.052187in}{0.696975in}}{\pgfqpoint{-0.058333in}{0.682137in}}{\pgfqpoint{-0.058333in}{0.666667in}}%
\pgfpathcurveto{\pgfqpoint{-0.058333in}{0.651196in}}{\pgfqpoint{-0.052187in}{0.636358in}}{\pgfqpoint{-0.041248in}{0.625419in}}%
\pgfpathcurveto{\pgfqpoint{-0.030309in}{0.614480in}}{\pgfqpoint{-0.015470in}{0.608333in}}{\pgfqpoint{0.000000in}{0.608333in}}%
\pgfpathclose%
\pgfpathmoveto{\pgfqpoint{0.000000in}{0.614167in}}%
\pgfpathcurveto{\pgfqpoint{0.000000in}{0.614167in}}{\pgfqpoint{-0.013923in}{0.614167in}}{\pgfqpoint{-0.027278in}{0.619698in}}%
\pgfpathcurveto{\pgfqpoint{-0.037123in}{0.629544in}}{\pgfqpoint{-0.046968in}{0.639389in}}{\pgfqpoint{-0.052500in}{0.652744in}}%
\pgfpathcurveto{\pgfqpoint{-0.052500in}{0.666667in}}{\pgfqpoint{-0.052500in}{0.680590in}}{\pgfqpoint{-0.046968in}{0.693945in}}%
\pgfpathcurveto{\pgfqpoint{-0.037123in}{0.703790in}}{\pgfqpoint{-0.027278in}{0.713635in}}{\pgfqpoint{-0.013923in}{0.719167in}}%
\pgfpathcurveto{\pgfqpoint{0.000000in}{0.719167in}}{\pgfqpoint{0.013923in}{0.719167in}}{\pgfqpoint{0.027278in}{0.713635in}}%
\pgfpathcurveto{\pgfqpoint{0.037123in}{0.703790in}}{\pgfqpoint{0.046968in}{0.693945in}}{\pgfqpoint{0.052500in}{0.680590in}}%
\pgfpathcurveto{\pgfqpoint{0.052500in}{0.666667in}}{\pgfqpoint{0.052500in}{0.652744in}}{\pgfqpoint{0.046968in}{0.639389in}}%
\pgfpathcurveto{\pgfqpoint{0.037123in}{0.629544in}}{\pgfqpoint{0.027278in}{0.619698in}}{\pgfqpoint{0.013923in}{0.614167in}}%
\pgfpathclose%
\pgfpathmoveto{\pgfqpoint{0.166667in}{0.608333in}}%
\pgfpathcurveto{\pgfqpoint{0.182137in}{0.608333in}}{\pgfqpoint{0.196975in}{0.614480in}}{\pgfqpoint{0.207915in}{0.625419in}}%
\pgfpathcurveto{\pgfqpoint{0.218854in}{0.636358in}}{\pgfqpoint{0.225000in}{0.651196in}}{\pgfqpoint{0.225000in}{0.666667in}}%
\pgfpathcurveto{\pgfqpoint{0.225000in}{0.682137in}}{\pgfqpoint{0.218854in}{0.696975in}}{\pgfqpoint{0.207915in}{0.707915in}}%
\pgfpathcurveto{\pgfqpoint{0.196975in}{0.718854in}}{\pgfqpoint{0.182137in}{0.725000in}}{\pgfqpoint{0.166667in}{0.725000in}}%
\pgfpathcurveto{\pgfqpoint{0.151196in}{0.725000in}}{\pgfqpoint{0.136358in}{0.718854in}}{\pgfqpoint{0.125419in}{0.707915in}}%
\pgfpathcurveto{\pgfqpoint{0.114480in}{0.696975in}}{\pgfqpoint{0.108333in}{0.682137in}}{\pgfqpoint{0.108333in}{0.666667in}}%
\pgfpathcurveto{\pgfqpoint{0.108333in}{0.651196in}}{\pgfqpoint{0.114480in}{0.636358in}}{\pgfqpoint{0.125419in}{0.625419in}}%
\pgfpathcurveto{\pgfqpoint{0.136358in}{0.614480in}}{\pgfqpoint{0.151196in}{0.608333in}}{\pgfqpoint{0.166667in}{0.608333in}}%
\pgfpathclose%
\pgfpathmoveto{\pgfqpoint{0.166667in}{0.614167in}}%
\pgfpathcurveto{\pgfqpoint{0.166667in}{0.614167in}}{\pgfqpoint{0.152744in}{0.614167in}}{\pgfqpoint{0.139389in}{0.619698in}}%
\pgfpathcurveto{\pgfqpoint{0.129544in}{0.629544in}}{\pgfqpoint{0.119698in}{0.639389in}}{\pgfqpoint{0.114167in}{0.652744in}}%
\pgfpathcurveto{\pgfqpoint{0.114167in}{0.666667in}}{\pgfqpoint{0.114167in}{0.680590in}}{\pgfqpoint{0.119698in}{0.693945in}}%
\pgfpathcurveto{\pgfqpoint{0.129544in}{0.703790in}}{\pgfqpoint{0.139389in}{0.713635in}}{\pgfqpoint{0.152744in}{0.719167in}}%
\pgfpathcurveto{\pgfqpoint{0.166667in}{0.719167in}}{\pgfqpoint{0.180590in}{0.719167in}}{\pgfqpoint{0.193945in}{0.713635in}}%
\pgfpathcurveto{\pgfqpoint{0.203790in}{0.703790in}}{\pgfqpoint{0.213635in}{0.693945in}}{\pgfqpoint{0.219167in}{0.680590in}}%
\pgfpathcurveto{\pgfqpoint{0.219167in}{0.666667in}}{\pgfqpoint{0.219167in}{0.652744in}}{\pgfqpoint{0.213635in}{0.639389in}}%
\pgfpathcurveto{\pgfqpoint{0.203790in}{0.629544in}}{\pgfqpoint{0.193945in}{0.619698in}}{\pgfqpoint{0.180590in}{0.614167in}}%
\pgfpathclose%
\pgfpathmoveto{\pgfqpoint{0.333333in}{0.608333in}}%
\pgfpathcurveto{\pgfqpoint{0.348804in}{0.608333in}}{\pgfqpoint{0.363642in}{0.614480in}}{\pgfqpoint{0.374581in}{0.625419in}}%
\pgfpathcurveto{\pgfqpoint{0.385520in}{0.636358in}}{\pgfqpoint{0.391667in}{0.651196in}}{\pgfqpoint{0.391667in}{0.666667in}}%
\pgfpathcurveto{\pgfqpoint{0.391667in}{0.682137in}}{\pgfqpoint{0.385520in}{0.696975in}}{\pgfqpoint{0.374581in}{0.707915in}}%
\pgfpathcurveto{\pgfqpoint{0.363642in}{0.718854in}}{\pgfqpoint{0.348804in}{0.725000in}}{\pgfqpoint{0.333333in}{0.725000in}}%
\pgfpathcurveto{\pgfqpoint{0.317863in}{0.725000in}}{\pgfqpoint{0.303025in}{0.718854in}}{\pgfqpoint{0.292085in}{0.707915in}}%
\pgfpathcurveto{\pgfqpoint{0.281146in}{0.696975in}}{\pgfqpoint{0.275000in}{0.682137in}}{\pgfqpoint{0.275000in}{0.666667in}}%
\pgfpathcurveto{\pgfqpoint{0.275000in}{0.651196in}}{\pgfqpoint{0.281146in}{0.636358in}}{\pgfqpoint{0.292085in}{0.625419in}}%
\pgfpathcurveto{\pgfqpoint{0.303025in}{0.614480in}}{\pgfqpoint{0.317863in}{0.608333in}}{\pgfqpoint{0.333333in}{0.608333in}}%
\pgfpathclose%
\pgfpathmoveto{\pgfqpoint{0.333333in}{0.614167in}}%
\pgfpathcurveto{\pgfqpoint{0.333333in}{0.614167in}}{\pgfqpoint{0.319410in}{0.614167in}}{\pgfqpoint{0.306055in}{0.619698in}}%
\pgfpathcurveto{\pgfqpoint{0.296210in}{0.629544in}}{\pgfqpoint{0.286365in}{0.639389in}}{\pgfqpoint{0.280833in}{0.652744in}}%
\pgfpathcurveto{\pgfqpoint{0.280833in}{0.666667in}}{\pgfqpoint{0.280833in}{0.680590in}}{\pgfqpoint{0.286365in}{0.693945in}}%
\pgfpathcurveto{\pgfqpoint{0.296210in}{0.703790in}}{\pgfqpoint{0.306055in}{0.713635in}}{\pgfqpoint{0.319410in}{0.719167in}}%
\pgfpathcurveto{\pgfqpoint{0.333333in}{0.719167in}}{\pgfqpoint{0.347256in}{0.719167in}}{\pgfqpoint{0.360611in}{0.713635in}}%
\pgfpathcurveto{\pgfqpoint{0.370456in}{0.703790in}}{\pgfqpoint{0.380302in}{0.693945in}}{\pgfqpoint{0.385833in}{0.680590in}}%
\pgfpathcurveto{\pgfqpoint{0.385833in}{0.666667in}}{\pgfqpoint{0.385833in}{0.652744in}}{\pgfqpoint{0.380302in}{0.639389in}}%
\pgfpathcurveto{\pgfqpoint{0.370456in}{0.629544in}}{\pgfqpoint{0.360611in}{0.619698in}}{\pgfqpoint{0.347256in}{0.614167in}}%
\pgfpathclose%
\pgfpathmoveto{\pgfqpoint{0.500000in}{0.608333in}}%
\pgfpathcurveto{\pgfqpoint{0.515470in}{0.608333in}}{\pgfqpoint{0.530309in}{0.614480in}}{\pgfqpoint{0.541248in}{0.625419in}}%
\pgfpathcurveto{\pgfqpoint{0.552187in}{0.636358in}}{\pgfqpoint{0.558333in}{0.651196in}}{\pgfqpoint{0.558333in}{0.666667in}}%
\pgfpathcurveto{\pgfqpoint{0.558333in}{0.682137in}}{\pgfqpoint{0.552187in}{0.696975in}}{\pgfqpoint{0.541248in}{0.707915in}}%
\pgfpathcurveto{\pgfqpoint{0.530309in}{0.718854in}}{\pgfqpoint{0.515470in}{0.725000in}}{\pgfqpoint{0.500000in}{0.725000in}}%
\pgfpathcurveto{\pgfqpoint{0.484530in}{0.725000in}}{\pgfqpoint{0.469691in}{0.718854in}}{\pgfqpoint{0.458752in}{0.707915in}}%
\pgfpathcurveto{\pgfqpoint{0.447813in}{0.696975in}}{\pgfqpoint{0.441667in}{0.682137in}}{\pgfqpoint{0.441667in}{0.666667in}}%
\pgfpathcurveto{\pgfqpoint{0.441667in}{0.651196in}}{\pgfqpoint{0.447813in}{0.636358in}}{\pgfqpoint{0.458752in}{0.625419in}}%
\pgfpathcurveto{\pgfqpoint{0.469691in}{0.614480in}}{\pgfqpoint{0.484530in}{0.608333in}}{\pgfqpoint{0.500000in}{0.608333in}}%
\pgfpathclose%
\pgfpathmoveto{\pgfqpoint{0.500000in}{0.614167in}}%
\pgfpathcurveto{\pgfqpoint{0.500000in}{0.614167in}}{\pgfqpoint{0.486077in}{0.614167in}}{\pgfqpoint{0.472722in}{0.619698in}}%
\pgfpathcurveto{\pgfqpoint{0.462877in}{0.629544in}}{\pgfqpoint{0.453032in}{0.639389in}}{\pgfqpoint{0.447500in}{0.652744in}}%
\pgfpathcurveto{\pgfqpoint{0.447500in}{0.666667in}}{\pgfqpoint{0.447500in}{0.680590in}}{\pgfqpoint{0.453032in}{0.693945in}}%
\pgfpathcurveto{\pgfqpoint{0.462877in}{0.703790in}}{\pgfqpoint{0.472722in}{0.713635in}}{\pgfqpoint{0.486077in}{0.719167in}}%
\pgfpathcurveto{\pgfqpoint{0.500000in}{0.719167in}}{\pgfqpoint{0.513923in}{0.719167in}}{\pgfqpoint{0.527278in}{0.713635in}}%
\pgfpathcurveto{\pgfqpoint{0.537123in}{0.703790in}}{\pgfqpoint{0.546968in}{0.693945in}}{\pgfqpoint{0.552500in}{0.680590in}}%
\pgfpathcurveto{\pgfqpoint{0.552500in}{0.666667in}}{\pgfqpoint{0.552500in}{0.652744in}}{\pgfqpoint{0.546968in}{0.639389in}}%
\pgfpathcurveto{\pgfqpoint{0.537123in}{0.629544in}}{\pgfqpoint{0.527278in}{0.619698in}}{\pgfqpoint{0.513923in}{0.614167in}}%
\pgfpathclose%
\pgfpathmoveto{\pgfqpoint{0.666667in}{0.608333in}}%
\pgfpathcurveto{\pgfqpoint{0.682137in}{0.608333in}}{\pgfqpoint{0.696975in}{0.614480in}}{\pgfqpoint{0.707915in}{0.625419in}}%
\pgfpathcurveto{\pgfqpoint{0.718854in}{0.636358in}}{\pgfqpoint{0.725000in}{0.651196in}}{\pgfqpoint{0.725000in}{0.666667in}}%
\pgfpathcurveto{\pgfqpoint{0.725000in}{0.682137in}}{\pgfqpoint{0.718854in}{0.696975in}}{\pgfqpoint{0.707915in}{0.707915in}}%
\pgfpathcurveto{\pgfqpoint{0.696975in}{0.718854in}}{\pgfqpoint{0.682137in}{0.725000in}}{\pgfqpoint{0.666667in}{0.725000in}}%
\pgfpathcurveto{\pgfqpoint{0.651196in}{0.725000in}}{\pgfqpoint{0.636358in}{0.718854in}}{\pgfqpoint{0.625419in}{0.707915in}}%
\pgfpathcurveto{\pgfqpoint{0.614480in}{0.696975in}}{\pgfqpoint{0.608333in}{0.682137in}}{\pgfqpoint{0.608333in}{0.666667in}}%
\pgfpathcurveto{\pgfqpoint{0.608333in}{0.651196in}}{\pgfqpoint{0.614480in}{0.636358in}}{\pgfqpoint{0.625419in}{0.625419in}}%
\pgfpathcurveto{\pgfqpoint{0.636358in}{0.614480in}}{\pgfqpoint{0.651196in}{0.608333in}}{\pgfqpoint{0.666667in}{0.608333in}}%
\pgfpathclose%
\pgfpathmoveto{\pgfqpoint{0.666667in}{0.614167in}}%
\pgfpathcurveto{\pgfqpoint{0.666667in}{0.614167in}}{\pgfqpoint{0.652744in}{0.614167in}}{\pgfqpoint{0.639389in}{0.619698in}}%
\pgfpathcurveto{\pgfqpoint{0.629544in}{0.629544in}}{\pgfqpoint{0.619698in}{0.639389in}}{\pgfqpoint{0.614167in}{0.652744in}}%
\pgfpathcurveto{\pgfqpoint{0.614167in}{0.666667in}}{\pgfqpoint{0.614167in}{0.680590in}}{\pgfqpoint{0.619698in}{0.693945in}}%
\pgfpathcurveto{\pgfqpoint{0.629544in}{0.703790in}}{\pgfqpoint{0.639389in}{0.713635in}}{\pgfqpoint{0.652744in}{0.719167in}}%
\pgfpathcurveto{\pgfqpoint{0.666667in}{0.719167in}}{\pgfqpoint{0.680590in}{0.719167in}}{\pgfqpoint{0.693945in}{0.713635in}}%
\pgfpathcurveto{\pgfqpoint{0.703790in}{0.703790in}}{\pgfqpoint{0.713635in}{0.693945in}}{\pgfqpoint{0.719167in}{0.680590in}}%
\pgfpathcurveto{\pgfqpoint{0.719167in}{0.666667in}}{\pgfqpoint{0.719167in}{0.652744in}}{\pgfqpoint{0.713635in}{0.639389in}}%
\pgfpathcurveto{\pgfqpoint{0.703790in}{0.629544in}}{\pgfqpoint{0.693945in}{0.619698in}}{\pgfqpoint{0.680590in}{0.614167in}}%
\pgfpathclose%
\pgfpathmoveto{\pgfqpoint{0.833333in}{0.608333in}}%
\pgfpathcurveto{\pgfqpoint{0.848804in}{0.608333in}}{\pgfqpoint{0.863642in}{0.614480in}}{\pgfqpoint{0.874581in}{0.625419in}}%
\pgfpathcurveto{\pgfqpoint{0.885520in}{0.636358in}}{\pgfqpoint{0.891667in}{0.651196in}}{\pgfqpoint{0.891667in}{0.666667in}}%
\pgfpathcurveto{\pgfqpoint{0.891667in}{0.682137in}}{\pgfqpoint{0.885520in}{0.696975in}}{\pgfqpoint{0.874581in}{0.707915in}}%
\pgfpathcurveto{\pgfqpoint{0.863642in}{0.718854in}}{\pgfqpoint{0.848804in}{0.725000in}}{\pgfqpoint{0.833333in}{0.725000in}}%
\pgfpathcurveto{\pgfqpoint{0.817863in}{0.725000in}}{\pgfqpoint{0.803025in}{0.718854in}}{\pgfqpoint{0.792085in}{0.707915in}}%
\pgfpathcurveto{\pgfqpoint{0.781146in}{0.696975in}}{\pgfqpoint{0.775000in}{0.682137in}}{\pgfqpoint{0.775000in}{0.666667in}}%
\pgfpathcurveto{\pgfqpoint{0.775000in}{0.651196in}}{\pgfqpoint{0.781146in}{0.636358in}}{\pgfqpoint{0.792085in}{0.625419in}}%
\pgfpathcurveto{\pgfqpoint{0.803025in}{0.614480in}}{\pgfqpoint{0.817863in}{0.608333in}}{\pgfqpoint{0.833333in}{0.608333in}}%
\pgfpathclose%
\pgfpathmoveto{\pgfqpoint{0.833333in}{0.614167in}}%
\pgfpathcurveto{\pgfqpoint{0.833333in}{0.614167in}}{\pgfqpoint{0.819410in}{0.614167in}}{\pgfqpoint{0.806055in}{0.619698in}}%
\pgfpathcurveto{\pgfqpoint{0.796210in}{0.629544in}}{\pgfqpoint{0.786365in}{0.639389in}}{\pgfqpoint{0.780833in}{0.652744in}}%
\pgfpathcurveto{\pgfqpoint{0.780833in}{0.666667in}}{\pgfqpoint{0.780833in}{0.680590in}}{\pgfqpoint{0.786365in}{0.693945in}}%
\pgfpathcurveto{\pgfqpoint{0.796210in}{0.703790in}}{\pgfqpoint{0.806055in}{0.713635in}}{\pgfqpoint{0.819410in}{0.719167in}}%
\pgfpathcurveto{\pgfqpoint{0.833333in}{0.719167in}}{\pgfqpoint{0.847256in}{0.719167in}}{\pgfqpoint{0.860611in}{0.713635in}}%
\pgfpathcurveto{\pgfqpoint{0.870456in}{0.703790in}}{\pgfqpoint{0.880302in}{0.693945in}}{\pgfqpoint{0.885833in}{0.680590in}}%
\pgfpathcurveto{\pgfqpoint{0.885833in}{0.666667in}}{\pgfqpoint{0.885833in}{0.652744in}}{\pgfqpoint{0.880302in}{0.639389in}}%
\pgfpathcurveto{\pgfqpoint{0.870456in}{0.629544in}}{\pgfqpoint{0.860611in}{0.619698in}}{\pgfqpoint{0.847256in}{0.614167in}}%
\pgfpathclose%
\pgfpathmoveto{\pgfqpoint{1.000000in}{0.608333in}}%
\pgfpathcurveto{\pgfqpoint{1.015470in}{0.608333in}}{\pgfqpoint{1.030309in}{0.614480in}}{\pgfqpoint{1.041248in}{0.625419in}}%
\pgfpathcurveto{\pgfqpoint{1.052187in}{0.636358in}}{\pgfqpoint{1.058333in}{0.651196in}}{\pgfqpoint{1.058333in}{0.666667in}}%
\pgfpathcurveto{\pgfqpoint{1.058333in}{0.682137in}}{\pgfqpoint{1.052187in}{0.696975in}}{\pgfqpoint{1.041248in}{0.707915in}}%
\pgfpathcurveto{\pgfqpoint{1.030309in}{0.718854in}}{\pgfqpoint{1.015470in}{0.725000in}}{\pgfqpoint{1.000000in}{0.725000in}}%
\pgfpathcurveto{\pgfqpoint{0.984530in}{0.725000in}}{\pgfqpoint{0.969691in}{0.718854in}}{\pgfqpoint{0.958752in}{0.707915in}}%
\pgfpathcurveto{\pgfqpoint{0.947813in}{0.696975in}}{\pgfqpoint{0.941667in}{0.682137in}}{\pgfqpoint{0.941667in}{0.666667in}}%
\pgfpathcurveto{\pgfqpoint{0.941667in}{0.651196in}}{\pgfqpoint{0.947813in}{0.636358in}}{\pgfqpoint{0.958752in}{0.625419in}}%
\pgfpathcurveto{\pgfqpoint{0.969691in}{0.614480in}}{\pgfqpoint{0.984530in}{0.608333in}}{\pgfqpoint{1.000000in}{0.608333in}}%
\pgfpathclose%
\pgfpathmoveto{\pgfqpoint{1.000000in}{0.614167in}}%
\pgfpathcurveto{\pgfqpoint{1.000000in}{0.614167in}}{\pgfqpoint{0.986077in}{0.614167in}}{\pgfqpoint{0.972722in}{0.619698in}}%
\pgfpathcurveto{\pgfqpoint{0.962877in}{0.629544in}}{\pgfqpoint{0.953032in}{0.639389in}}{\pgfqpoint{0.947500in}{0.652744in}}%
\pgfpathcurveto{\pgfqpoint{0.947500in}{0.666667in}}{\pgfqpoint{0.947500in}{0.680590in}}{\pgfqpoint{0.953032in}{0.693945in}}%
\pgfpathcurveto{\pgfqpoint{0.962877in}{0.703790in}}{\pgfqpoint{0.972722in}{0.713635in}}{\pgfqpoint{0.986077in}{0.719167in}}%
\pgfpathcurveto{\pgfqpoint{1.000000in}{0.719167in}}{\pgfqpoint{1.013923in}{0.719167in}}{\pgfqpoint{1.027278in}{0.713635in}}%
\pgfpathcurveto{\pgfqpoint{1.037123in}{0.703790in}}{\pgfqpoint{1.046968in}{0.693945in}}{\pgfqpoint{1.052500in}{0.680590in}}%
\pgfpathcurveto{\pgfqpoint{1.052500in}{0.666667in}}{\pgfqpoint{1.052500in}{0.652744in}}{\pgfqpoint{1.046968in}{0.639389in}}%
\pgfpathcurveto{\pgfqpoint{1.037123in}{0.629544in}}{\pgfqpoint{1.027278in}{0.619698in}}{\pgfqpoint{1.013923in}{0.614167in}}%
\pgfpathclose%
\pgfpathmoveto{\pgfqpoint{0.083333in}{0.775000in}}%
\pgfpathcurveto{\pgfqpoint{0.098804in}{0.775000in}}{\pgfqpoint{0.113642in}{0.781146in}}{\pgfqpoint{0.124581in}{0.792085in}}%
\pgfpathcurveto{\pgfqpoint{0.135520in}{0.803025in}}{\pgfqpoint{0.141667in}{0.817863in}}{\pgfqpoint{0.141667in}{0.833333in}}%
\pgfpathcurveto{\pgfqpoint{0.141667in}{0.848804in}}{\pgfqpoint{0.135520in}{0.863642in}}{\pgfqpoint{0.124581in}{0.874581in}}%
\pgfpathcurveto{\pgfqpoint{0.113642in}{0.885520in}}{\pgfqpoint{0.098804in}{0.891667in}}{\pgfqpoint{0.083333in}{0.891667in}}%
\pgfpathcurveto{\pgfqpoint{0.067863in}{0.891667in}}{\pgfqpoint{0.053025in}{0.885520in}}{\pgfqpoint{0.042085in}{0.874581in}}%
\pgfpathcurveto{\pgfqpoint{0.031146in}{0.863642in}}{\pgfqpoint{0.025000in}{0.848804in}}{\pgfqpoint{0.025000in}{0.833333in}}%
\pgfpathcurveto{\pgfqpoint{0.025000in}{0.817863in}}{\pgfqpoint{0.031146in}{0.803025in}}{\pgfqpoint{0.042085in}{0.792085in}}%
\pgfpathcurveto{\pgfqpoint{0.053025in}{0.781146in}}{\pgfqpoint{0.067863in}{0.775000in}}{\pgfqpoint{0.083333in}{0.775000in}}%
\pgfpathclose%
\pgfpathmoveto{\pgfqpoint{0.083333in}{0.780833in}}%
\pgfpathcurveto{\pgfqpoint{0.083333in}{0.780833in}}{\pgfqpoint{0.069410in}{0.780833in}}{\pgfqpoint{0.056055in}{0.786365in}}%
\pgfpathcurveto{\pgfqpoint{0.046210in}{0.796210in}}{\pgfqpoint{0.036365in}{0.806055in}}{\pgfqpoint{0.030833in}{0.819410in}}%
\pgfpathcurveto{\pgfqpoint{0.030833in}{0.833333in}}{\pgfqpoint{0.030833in}{0.847256in}}{\pgfqpoint{0.036365in}{0.860611in}}%
\pgfpathcurveto{\pgfqpoint{0.046210in}{0.870456in}}{\pgfqpoint{0.056055in}{0.880302in}}{\pgfqpoint{0.069410in}{0.885833in}}%
\pgfpathcurveto{\pgfqpoint{0.083333in}{0.885833in}}{\pgfqpoint{0.097256in}{0.885833in}}{\pgfqpoint{0.110611in}{0.880302in}}%
\pgfpathcurveto{\pgfqpoint{0.120456in}{0.870456in}}{\pgfqpoint{0.130302in}{0.860611in}}{\pgfqpoint{0.135833in}{0.847256in}}%
\pgfpathcurveto{\pgfqpoint{0.135833in}{0.833333in}}{\pgfqpoint{0.135833in}{0.819410in}}{\pgfqpoint{0.130302in}{0.806055in}}%
\pgfpathcurveto{\pgfqpoint{0.120456in}{0.796210in}}{\pgfqpoint{0.110611in}{0.786365in}}{\pgfqpoint{0.097256in}{0.780833in}}%
\pgfpathclose%
\pgfpathmoveto{\pgfqpoint{0.250000in}{0.775000in}}%
\pgfpathcurveto{\pgfqpoint{0.265470in}{0.775000in}}{\pgfqpoint{0.280309in}{0.781146in}}{\pgfqpoint{0.291248in}{0.792085in}}%
\pgfpathcurveto{\pgfqpoint{0.302187in}{0.803025in}}{\pgfqpoint{0.308333in}{0.817863in}}{\pgfqpoint{0.308333in}{0.833333in}}%
\pgfpathcurveto{\pgfqpoint{0.308333in}{0.848804in}}{\pgfqpoint{0.302187in}{0.863642in}}{\pgfqpoint{0.291248in}{0.874581in}}%
\pgfpathcurveto{\pgfqpoint{0.280309in}{0.885520in}}{\pgfqpoint{0.265470in}{0.891667in}}{\pgfqpoint{0.250000in}{0.891667in}}%
\pgfpathcurveto{\pgfqpoint{0.234530in}{0.891667in}}{\pgfqpoint{0.219691in}{0.885520in}}{\pgfqpoint{0.208752in}{0.874581in}}%
\pgfpathcurveto{\pgfqpoint{0.197813in}{0.863642in}}{\pgfqpoint{0.191667in}{0.848804in}}{\pgfqpoint{0.191667in}{0.833333in}}%
\pgfpathcurveto{\pgfqpoint{0.191667in}{0.817863in}}{\pgfqpoint{0.197813in}{0.803025in}}{\pgfqpoint{0.208752in}{0.792085in}}%
\pgfpathcurveto{\pgfqpoint{0.219691in}{0.781146in}}{\pgfqpoint{0.234530in}{0.775000in}}{\pgfqpoint{0.250000in}{0.775000in}}%
\pgfpathclose%
\pgfpathmoveto{\pgfqpoint{0.250000in}{0.780833in}}%
\pgfpathcurveto{\pgfqpoint{0.250000in}{0.780833in}}{\pgfqpoint{0.236077in}{0.780833in}}{\pgfqpoint{0.222722in}{0.786365in}}%
\pgfpathcurveto{\pgfqpoint{0.212877in}{0.796210in}}{\pgfqpoint{0.203032in}{0.806055in}}{\pgfqpoint{0.197500in}{0.819410in}}%
\pgfpathcurveto{\pgfqpoint{0.197500in}{0.833333in}}{\pgfqpoint{0.197500in}{0.847256in}}{\pgfqpoint{0.203032in}{0.860611in}}%
\pgfpathcurveto{\pgfqpoint{0.212877in}{0.870456in}}{\pgfqpoint{0.222722in}{0.880302in}}{\pgfqpoint{0.236077in}{0.885833in}}%
\pgfpathcurveto{\pgfqpoint{0.250000in}{0.885833in}}{\pgfqpoint{0.263923in}{0.885833in}}{\pgfqpoint{0.277278in}{0.880302in}}%
\pgfpathcurveto{\pgfqpoint{0.287123in}{0.870456in}}{\pgfqpoint{0.296968in}{0.860611in}}{\pgfqpoint{0.302500in}{0.847256in}}%
\pgfpathcurveto{\pgfqpoint{0.302500in}{0.833333in}}{\pgfqpoint{0.302500in}{0.819410in}}{\pgfqpoint{0.296968in}{0.806055in}}%
\pgfpathcurveto{\pgfqpoint{0.287123in}{0.796210in}}{\pgfqpoint{0.277278in}{0.786365in}}{\pgfqpoint{0.263923in}{0.780833in}}%
\pgfpathclose%
\pgfpathmoveto{\pgfqpoint{0.416667in}{0.775000in}}%
\pgfpathcurveto{\pgfqpoint{0.432137in}{0.775000in}}{\pgfqpoint{0.446975in}{0.781146in}}{\pgfqpoint{0.457915in}{0.792085in}}%
\pgfpathcurveto{\pgfqpoint{0.468854in}{0.803025in}}{\pgfqpoint{0.475000in}{0.817863in}}{\pgfqpoint{0.475000in}{0.833333in}}%
\pgfpathcurveto{\pgfqpoint{0.475000in}{0.848804in}}{\pgfqpoint{0.468854in}{0.863642in}}{\pgfqpoint{0.457915in}{0.874581in}}%
\pgfpathcurveto{\pgfqpoint{0.446975in}{0.885520in}}{\pgfqpoint{0.432137in}{0.891667in}}{\pgfqpoint{0.416667in}{0.891667in}}%
\pgfpathcurveto{\pgfqpoint{0.401196in}{0.891667in}}{\pgfqpoint{0.386358in}{0.885520in}}{\pgfqpoint{0.375419in}{0.874581in}}%
\pgfpathcurveto{\pgfqpoint{0.364480in}{0.863642in}}{\pgfqpoint{0.358333in}{0.848804in}}{\pgfqpoint{0.358333in}{0.833333in}}%
\pgfpathcurveto{\pgfqpoint{0.358333in}{0.817863in}}{\pgfqpoint{0.364480in}{0.803025in}}{\pgfqpoint{0.375419in}{0.792085in}}%
\pgfpathcurveto{\pgfqpoint{0.386358in}{0.781146in}}{\pgfqpoint{0.401196in}{0.775000in}}{\pgfqpoint{0.416667in}{0.775000in}}%
\pgfpathclose%
\pgfpathmoveto{\pgfqpoint{0.416667in}{0.780833in}}%
\pgfpathcurveto{\pgfqpoint{0.416667in}{0.780833in}}{\pgfqpoint{0.402744in}{0.780833in}}{\pgfqpoint{0.389389in}{0.786365in}}%
\pgfpathcurveto{\pgfqpoint{0.379544in}{0.796210in}}{\pgfqpoint{0.369698in}{0.806055in}}{\pgfqpoint{0.364167in}{0.819410in}}%
\pgfpathcurveto{\pgfqpoint{0.364167in}{0.833333in}}{\pgfqpoint{0.364167in}{0.847256in}}{\pgfqpoint{0.369698in}{0.860611in}}%
\pgfpathcurveto{\pgfqpoint{0.379544in}{0.870456in}}{\pgfqpoint{0.389389in}{0.880302in}}{\pgfqpoint{0.402744in}{0.885833in}}%
\pgfpathcurveto{\pgfqpoint{0.416667in}{0.885833in}}{\pgfqpoint{0.430590in}{0.885833in}}{\pgfqpoint{0.443945in}{0.880302in}}%
\pgfpathcurveto{\pgfqpoint{0.453790in}{0.870456in}}{\pgfqpoint{0.463635in}{0.860611in}}{\pgfqpoint{0.469167in}{0.847256in}}%
\pgfpathcurveto{\pgfqpoint{0.469167in}{0.833333in}}{\pgfqpoint{0.469167in}{0.819410in}}{\pgfqpoint{0.463635in}{0.806055in}}%
\pgfpathcurveto{\pgfqpoint{0.453790in}{0.796210in}}{\pgfqpoint{0.443945in}{0.786365in}}{\pgfqpoint{0.430590in}{0.780833in}}%
\pgfpathclose%
\pgfpathmoveto{\pgfqpoint{0.583333in}{0.775000in}}%
\pgfpathcurveto{\pgfqpoint{0.598804in}{0.775000in}}{\pgfqpoint{0.613642in}{0.781146in}}{\pgfqpoint{0.624581in}{0.792085in}}%
\pgfpathcurveto{\pgfqpoint{0.635520in}{0.803025in}}{\pgfqpoint{0.641667in}{0.817863in}}{\pgfqpoint{0.641667in}{0.833333in}}%
\pgfpathcurveto{\pgfqpoint{0.641667in}{0.848804in}}{\pgfqpoint{0.635520in}{0.863642in}}{\pgfqpoint{0.624581in}{0.874581in}}%
\pgfpathcurveto{\pgfqpoint{0.613642in}{0.885520in}}{\pgfqpoint{0.598804in}{0.891667in}}{\pgfqpoint{0.583333in}{0.891667in}}%
\pgfpathcurveto{\pgfqpoint{0.567863in}{0.891667in}}{\pgfqpoint{0.553025in}{0.885520in}}{\pgfqpoint{0.542085in}{0.874581in}}%
\pgfpathcurveto{\pgfqpoint{0.531146in}{0.863642in}}{\pgfqpoint{0.525000in}{0.848804in}}{\pgfqpoint{0.525000in}{0.833333in}}%
\pgfpathcurveto{\pgfqpoint{0.525000in}{0.817863in}}{\pgfqpoint{0.531146in}{0.803025in}}{\pgfqpoint{0.542085in}{0.792085in}}%
\pgfpathcurveto{\pgfqpoint{0.553025in}{0.781146in}}{\pgfqpoint{0.567863in}{0.775000in}}{\pgfqpoint{0.583333in}{0.775000in}}%
\pgfpathclose%
\pgfpathmoveto{\pgfqpoint{0.583333in}{0.780833in}}%
\pgfpathcurveto{\pgfqpoint{0.583333in}{0.780833in}}{\pgfqpoint{0.569410in}{0.780833in}}{\pgfqpoint{0.556055in}{0.786365in}}%
\pgfpathcurveto{\pgfqpoint{0.546210in}{0.796210in}}{\pgfqpoint{0.536365in}{0.806055in}}{\pgfqpoint{0.530833in}{0.819410in}}%
\pgfpathcurveto{\pgfqpoint{0.530833in}{0.833333in}}{\pgfqpoint{0.530833in}{0.847256in}}{\pgfqpoint{0.536365in}{0.860611in}}%
\pgfpathcurveto{\pgfqpoint{0.546210in}{0.870456in}}{\pgfqpoint{0.556055in}{0.880302in}}{\pgfqpoint{0.569410in}{0.885833in}}%
\pgfpathcurveto{\pgfqpoint{0.583333in}{0.885833in}}{\pgfqpoint{0.597256in}{0.885833in}}{\pgfqpoint{0.610611in}{0.880302in}}%
\pgfpathcurveto{\pgfqpoint{0.620456in}{0.870456in}}{\pgfqpoint{0.630302in}{0.860611in}}{\pgfqpoint{0.635833in}{0.847256in}}%
\pgfpathcurveto{\pgfqpoint{0.635833in}{0.833333in}}{\pgfqpoint{0.635833in}{0.819410in}}{\pgfqpoint{0.630302in}{0.806055in}}%
\pgfpathcurveto{\pgfqpoint{0.620456in}{0.796210in}}{\pgfqpoint{0.610611in}{0.786365in}}{\pgfqpoint{0.597256in}{0.780833in}}%
\pgfpathclose%
\pgfpathmoveto{\pgfqpoint{0.750000in}{0.775000in}}%
\pgfpathcurveto{\pgfqpoint{0.765470in}{0.775000in}}{\pgfqpoint{0.780309in}{0.781146in}}{\pgfqpoint{0.791248in}{0.792085in}}%
\pgfpathcurveto{\pgfqpoint{0.802187in}{0.803025in}}{\pgfqpoint{0.808333in}{0.817863in}}{\pgfqpoint{0.808333in}{0.833333in}}%
\pgfpathcurveto{\pgfqpoint{0.808333in}{0.848804in}}{\pgfqpoint{0.802187in}{0.863642in}}{\pgfqpoint{0.791248in}{0.874581in}}%
\pgfpathcurveto{\pgfqpoint{0.780309in}{0.885520in}}{\pgfqpoint{0.765470in}{0.891667in}}{\pgfqpoint{0.750000in}{0.891667in}}%
\pgfpathcurveto{\pgfqpoint{0.734530in}{0.891667in}}{\pgfqpoint{0.719691in}{0.885520in}}{\pgfqpoint{0.708752in}{0.874581in}}%
\pgfpathcurveto{\pgfqpoint{0.697813in}{0.863642in}}{\pgfqpoint{0.691667in}{0.848804in}}{\pgfqpoint{0.691667in}{0.833333in}}%
\pgfpathcurveto{\pgfqpoint{0.691667in}{0.817863in}}{\pgfqpoint{0.697813in}{0.803025in}}{\pgfqpoint{0.708752in}{0.792085in}}%
\pgfpathcurveto{\pgfqpoint{0.719691in}{0.781146in}}{\pgfqpoint{0.734530in}{0.775000in}}{\pgfqpoint{0.750000in}{0.775000in}}%
\pgfpathclose%
\pgfpathmoveto{\pgfqpoint{0.750000in}{0.780833in}}%
\pgfpathcurveto{\pgfqpoint{0.750000in}{0.780833in}}{\pgfqpoint{0.736077in}{0.780833in}}{\pgfqpoint{0.722722in}{0.786365in}}%
\pgfpathcurveto{\pgfqpoint{0.712877in}{0.796210in}}{\pgfqpoint{0.703032in}{0.806055in}}{\pgfqpoint{0.697500in}{0.819410in}}%
\pgfpathcurveto{\pgfqpoint{0.697500in}{0.833333in}}{\pgfqpoint{0.697500in}{0.847256in}}{\pgfqpoint{0.703032in}{0.860611in}}%
\pgfpathcurveto{\pgfqpoint{0.712877in}{0.870456in}}{\pgfqpoint{0.722722in}{0.880302in}}{\pgfqpoint{0.736077in}{0.885833in}}%
\pgfpathcurveto{\pgfqpoint{0.750000in}{0.885833in}}{\pgfqpoint{0.763923in}{0.885833in}}{\pgfqpoint{0.777278in}{0.880302in}}%
\pgfpathcurveto{\pgfqpoint{0.787123in}{0.870456in}}{\pgfqpoint{0.796968in}{0.860611in}}{\pgfqpoint{0.802500in}{0.847256in}}%
\pgfpathcurveto{\pgfqpoint{0.802500in}{0.833333in}}{\pgfqpoint{0.802500in}{0.819410in}}{\pgfqpoint{0.796968in}{0.806055in}}%
\pgfpathcurveto{\pgfqpoint{0.787123in}{0.796210in}}{\pgfqpoint{0.777278in}{0.786365in}}{\pgfqpoint{0.763923in}{0.780833in}}%
\pgfpathclose%
\pgfpathmoveto{\pgfqpoint{0.916667in}{0.775000in}}%
\pgfpathcurveto{\pgfqpoint{0.932137in}{0.775000in}}{\pgfqpoint{0.946975in}{0.781146in}}{\pgfqpoint{0.957915in}{0.792085in}}%
\pgfpathcurveto{\pgfqpoint{0.968854in}{0.803025in}}{\pgfqpoint{0.975000in}{0.817863in}}{\pgfqpoint{0.975000in}{0.833333in}}%
\pgfpathcurveto{\pgfqpoint{0.975000in}{0.848804in}}{\pgfqpoint{0.968854in}{0.863642in}}{\pgfqpoint{0.957915in}{0.874581in}}%
\pgfpathcurveto{\pgfqpoint{0.946975in}{0.885520in}}{\pgfqpoint{0.932137in}{0.891667in}}{\pgfqpoint{0.916667in}{0.891667in}}%
\pgfpathcurveto{\pgfqpoint{0.901196in}{0.891667in}}{\pgfqpoint{0.886358in}{0.885520in}}{\pgfqpoint{0.875419in}{0.874581in}}%
\pgfpathcurveto{\pgfqpoint{0.864480in}{0.863642in}}{\pgfqpoint{0.858333in}{0.848804in}}{\pgfqpoint{0.858333in}{0.833333in}}%
\pgfpathcurveto{\pgfqpoint{0.858333in}{0.817863in}}{\pgfqpoint{0.864480in}{0.803025in}}{\pgfqpoint{0.875419in}{0.792085in}}%
\pgfpathcurveto{\pgfqpoint{0.886358in}{0.781146in}}{\pgfqpoint{0.901196in}{0.775000in}}{\pgfqpoint{0.916667in}{0.775000in}}%
\pgfpathclose%
\pgfpathmoveto{\pgfqpoint{0.916667in}{0.780833in}}%
\pgfpathcurveto{\pgfqpoint{0.916667in}{0.780833in}}{\pgfqpoint{0.902744in}{0.780833in}}{\pgfqpoint{0.889389in}{0.786365in}}%
\pgfpathcurveto{\pgfqpoint{0.879544in}{0.796210in}}{\pgfqpoint{0.869698in}{0.806055in}}{\pgfqpoint{0.864167in}{0.819410in}}%
\pgfpathcurveto{\pgfqpoint{0.864167in}{0.833333in}}{\pgfqpoint{0.864167in}{0.847256in}}{\pgfqpoint{0.869698in}{0.860611in}}%
\pgfpathcurveto{\pgfqpoint{0.879544in}{0.870456in}}{\pgfqpoint{0.889389in}{0.880302in}}{\pgfqpoint{0.902744in}{0.885833in}}%
\pgfpathcurveto{\pgfqpoint{0.916667in}{0.885833in}}{\pgfqpoint{0.930590in}{0.885833in}}{\pgfqpoint{0.943945in}{0.880302in}}%
\pgfpathcurveto{\pgfqpoint{0.953790in}{0.870456in}}{\pgfqpoint{0.963635in}{0.860611in}}{\pgfqpoint{0.969167in}{0.847256in}}%
\pgfpathcurveto{\pgfqpoint{0.969167in}{0.833333in}}{\pgfqpoint{0.969167in}{0.819410in}}{\pgfqpoint{0.963635in}{0.806055in}}%
\pgfpathcurveto{\pgfqpoint{0.953790in}{0.796210in}}{\pgfqpoint{0.943945in}{0.786365in}}{\pgfqpoint{0.930590in}{0.780833in}}%
\pgfpathclose%
\pgfpathmoveto{\pgfqpoint{0.000000in}{0.941667in}}%
\pgfpathcurveto{\pgfqpoint{0.015470in}{0.941667in}}{\pgfqpoint{0.030309in}{0.947813in}}{\pgfqpoint{0.041248in}{0.958752in}}%
\pgfpathcurveto{\pgfqpoint{0.052187in}{0.969691in}}{\pgfqpoint{0.058333in}{0.984530in}}{\pgfqpoint{0.058333in}{1.000000in}}%
\pgfpathcurveto{\pgfqpoint{0.058333in}{1.015470in}}{\pgfqpoint{0.052187in}{1.030309in}}{\pgfqpoint{0.041248in}{1.041248in}}%
\pgfpathcurveto{\pgfqpoint{0.030309in}{1.052187in}}{\pgfqpoint{0.015470in}{1.058333in}}{\pgfqpoint{0.000000in}{1.058333in}}%
\pgfpathcurveto{\pgfqpoint{-0.015470in}{1.058333in}}{\pgfqpoint{-0.030309in}{1.052187in}}{\pgfqpoint{-0.041248in}{1.041248in}}%
\pgfpathcurveto{\pgfqpoint{-0.052187in}{1.030309in}}{\pgfqpoint{-0.058333in}{1.015470in}}{\pgfqpoint{-0.058333in}{1.000000in}}%
\pgfpathcurveto{\pgfqpoint{-0.058333in}{0.984530in}}{\pgfqpoint{-0.052187in}{0.969691in}}{\pgfqpoint{-0.041248in}{0.958752in}}%
\pgfpathcurveto{\pgfqpoint{-0.030309in}{0.947813in}}{\pgfqpoint{-0.015470in}{0.941667in}}{\pgfqpoint{0.000000in}{0.941667in}}%
\pgfpathclose%
\pgfpathmoveto{\pgfqpoint{0.000000in}{0.947500in}}%
\pgfpathcurveto{\pgfqpoint{0.000000in}{0.947500in}}{\pgfqpoint{-0.013923in}{0.947500in}}{\pgfqpoint{-0.027278in}{0.953032in}}%
\pgfpathcurveto{\pgfqpoint{-0.037123in}{0.962877in}}{\pgfqpoint{-0.046968in}{0.972722in}}{\pgfqpoint{-0.052500in}{0.986077in}}%
\pgfpathcurveto{\pgfqpoint{-0.052500in}{1.000000in}}{\pgfqpoint{-0.052500in}{1.013923in}}{\pgfqpoint{-0.046968in}{1.027278in}}%
\pgfpathcurveto{\pgfqpoint{-0.037123in}{1.037123in}}{\pgfqpoint{-0.027278in}{1.046968in}}{\pgfqpoint{-0.013923in}{1.052500in}}%
\pgfpathcurveto{\pgfqpoint{0.000000in}{1.052500in}}{\pgfqpoint{0.013923in}{1.052500in}}{\pgfqpoint{0.027278in}{1.046968in}}%
\pgfpathcurveto{\pgfqpoint{0.037123in}{1.037123in}}{\pgfqpoint{0.046968in}{1.027278in}}{\pgfqpoint{0.052500in}{1.013923in}}%
\pgfpathcurveto{\pgfqpoint{0.052500in}{1.000000in}}{\pgfqpoint{0.052500in}{0.986077in}}{\pgfqpoint{0.046968in}{0.972722in}}%
\pgfpathcurveto{\pgfqpoint{0.037123in}{0.962877in}}{\pgfqpoint{0.027278in}{0.953032in}}{\pgfqpoint{0.013923in}{0.947500in}}%
\pgfpathclose%
\pgfpathmoveto{\pgfqpoint{0.166667in}{0.941667in}}%
\pgfpathcurveto{\pgfqpoint{0.182137in}{0.941667in}}{\pgfqpoint{0.196975in}{0.947813in}}{\pgfqpoint{0.207915in}{0.958752in}}%
\pgfpathcurveto{\pgfqpoint{0.218854in}{0.969691in}}{\pgfqpoint{0.225000in}{0.984530in}}{\pgfqpoint{0.225000in}{1.000000in}}%
\pgfpathcurveto{\pgfqpoint{0.225000in}{1.015470in}}{\pgfqpoint{0.218854in}{1.030309in}}{\pgfqpoint{0.207915in}{1.041248in}}%
\pgfpathcurveto{\pgfqpoint{0.196975in}{1.052187in}}{\pgfqpoint{0.182137in}{1.058333in}}{\pgfqpoint{0.166667in}{1.058333in}}%
\pgfpathcurveto{\pgfqpoint{0.151196in}{1.058333in}}{\pgfqpoint{0.136358in}{1.052187in}}{\pgfqpoint{0.125419in}{1.041248in}}%
\pgfpathcurveto{\pgfqpoint{0.114480in}{1.030309in}}{\pgfqpoint{0.108333in}{1.015470in}}{\pgfqpoint{0.108333in}{1.000000in}}%
\pgfpathcurveto{\pgfqpoint{0.108333in}{0.984530in}}{\pgfqpoint{0.114480in}{0.969691in}}{\pgfqpoint{0.125419in}{0.958752in}}%
\pgfpathcurveto{\pgfqpoint{0.136358in}{0.947813in}}{\pgfqpoint{0.151196in}{0.941667in}}{\pgfqpoint{0.166667in}{0.941667in}}%
\pgfpathclose%
\pgfpathmoveto{\pgfqpoint{0.166667in}{0.947500in}}%
\pgfpathcurveto{\pgfqpoint{0.166667in}{0.947500in}}{\pgfqpoint{0.152744in}{0.947500in}}{\pgfqpoint{0.139389in}{0.953032in}}%
\pgfpathcurveto{\pgfqpoint{0.129544in}{0.962877in}}{\pgfqpoint{0.119698in}{0.972722in}}{\pgfqpoint{0.114167in}{0.986077in}}%
\pgfpathcurveto{\pgfqpoint{0.114167in}{1.000000in}}{\pgfqpoint{0.114167in}{1.013923in}}{\pgfqpoint{0.119698in}{1.027278in}}%
\pgfpathcurveto{\pgfqpoint{0.129544in}{1.037123in}}{\pgfqpoint{0.139389in}{1.046968in}}{\pgfqpoint{0.152744in}{1.052500in}}%
\pgfpathcurveto{\pgfqpoint{0.166667in}{1.052500in}}{\pgfqpoint{0.180590in}{1.052500in}}{\pgfqpoint{0.193945in}{1.046968in}}%
\pgfpathcurveto{\pgfqpoint{0.203790in}{1.037123in}}{\pgfqpoint{0.213635in}{1.027278in}}{\pgfqpoint{0.219167in}{1.013923in}}%
\pgfpathcurveto{\pgfqpoint{0.219167in}{1.000000in}}{\pgfqpoint{0.219167in}{0.986077in}}{\pgfqpoint{0.213635in}{0.972722in}}%
\pgfpathcurveto{\pgfqpoint{0.203790in}{0.962877in}}{\pgfqpoint{0.193945in}{0.953032in}}{\pgfqpoint{0.180590in}{0.947500in}}%
\pgfpathclose%
\pgfpathmoveto{\pgfqpoint{0.333333in}{0.941667in}}%
\pgfpathcurveto{\pgfqpoint{0.348804in}{0.941667in}}{\pgfqpoint{0.363642in}{0.947813in}}{\pgfqpoint{0.374581in}{0.958752in}}%
\pgfpathcurveto{\pgfqpoint{0.385520in}{0.969691in}}{\pgfqpoint{0.391667in}{0.984530in}}{\pgfqpoint{0.391667in}{1.000000in}}%
\pgfpathcurveto{\pgfqpoint{0.391667in}{1.015470in}}{\pgfqpoint{0.385520in}{1.030309in}}{\pgfqpoint{0.374581in}{1.041248in}}%
\pgfpathcurveto{\pgfqpoint{0.363642in}{1.052187in}}{\pgfqpoint{0.348804in}{1.058333in}}{\pgfqpoint{0.333333in}{1.058333in}}%
\pgfpathcurveto{\pgfqpoint{0.317863in}{1.058333in}}{\pgfqpoint{0.303025in}{1.052187in}}{\pgfqpoint{0.292085in}{1.041248in}}%
\pgfpathcurveto{\pgfqpoint{0.281146in}{1.030309in}}{\pgfqpoint{0.275000in}{1.015470in}}{\pgfqpoint{0.275000in}{1.000000in}}%
\pgfpathcurveto{\pgfqpoint{0.275000in}{0.984530in}}{\pgfqpoint{0.281146in}{0.969691in}}{\pgfqpoint{0.292085in}{0.958752in}}%
\pgfpathcurveto{\pgfqpoint{0.303025in}{0.947813in}}{\pgfqpoint{0.317863in}{0.941667in}}{\pgfqpoint{0.333333in}{0.941667in}}%
\pgfpathclose%
\pgfpathmoveto{\pgfqpoint{0.333333in}{0.947500in}}%
\pgfpathcurveto{\pgfqpoint{0.333333in}{0.947500in}}{\pgfqpoint{0.319410in}{0.947500in}}{\pgfqpoint{0.306055in}{0.953032in}}%
\pgfpathcurveto{\pgfqpoint{0.296210in}{0.962877in}}{\pgfqpoint{0.286365in}{0.972722in}}{\pgfqpoint{0.280833in}{0.986077in}}%
\pgfpathcurveto{\pgfqpoint{0.280833in}{1.000000in}}{\pgfqpoint{0.280833in}{1.013923in}}{\pgfqpoint{0.286365in}{1.027278in}}%
\pgfpathcurveto{\pgfqpoint{0.296210in}{1.037123in}}{\pgfqpoint{0.306055in}{1.046968in}}{\pgfqpoint{0.319410in}{1.052500in}}%
\pgfpathcurveto{\pgfqpoint{0.333333in}{1.052500in}}{\pgfqpoint{0.347256in}{1.052500in}}{\pgfqpoint{0.360611in}{1.046968in}}%
\pgfpathcurveto{\pgfqpoint{0.370456in}{1.037123in}}{\pgfqpoint{0.380302in}{1.027278in}}{\pgfqpoint{0.385833in}{1.013923in}}%
\pgfpathcurveto{\pgfqpoint{0.385833in}{1.000000in}}{\pgfqpoint{0.385833in}{0.986077in}}{\pgfqpoint{0.380302in}{0.972722in}}%
\pgfpathcurveto{\pgfqpoint{0.370456in}{0.962877in}}{\pgfqpoint{0.360611in}{0.953032in}}{\pgfqpoint{0.347256in}{0.947500in}}%
\pgfpathclose%
\pgfpathmoveto{\pgfqpoint{0.500000in}{0.941667in}}%
\pgfpathcurveto{\pgfqpoint{0.515470in}{0.941667in}}{\pgfqpoint{0.530309in}{0.947813in}}{\pgfqpoint{0.541248in}{0.958752in}}%
\pgfpathcurveto{\pgfqpoint{0.552187in}{0.969691in}}{\pgfqpoint{0.558333in}{0.984530in}}{\pgfqpoint{0.558333in}{1.000000in}}%
\pgfpathcurveto{\pgfqpoint{0.558333in}{1.015470in}}{\pgfqpoint{0.552187in}{1.030309in}}{\pgfqpoint{0.541248in}{1.041248in}}%
\pgfpathcurveto{\pgfqpoint{0.530309in}{1.052187in}}{\pgfqpoint{0.515470in}{1.058333in}}{\pgfqpoint{0.500000in}{1.058333in}}%
\pgfpathcurveto{\pgfqpoint{0.484530in}{1.058333in}}{\pgfqpoint{0.469691in}{1.052187in}}{\pgfqpoint{0.458752in}{1.041248in}}%
\pgfpathcurveto{\pgfqpoint{0.447813in}{1.030309in}}{\pgfqpoint{0.441667in}{1.015470in}}{\pgfqpoint{0.441667in}{1.000000in}}%
\pgfpathcurveto{\pgfqpoint{0.441667in}{0.984530in}}{\pgfqpoint{0.447813in}{0.969691in}}{\pgfqpoint{0.458752in}{0.958752in}}%
\pgfpathcurveto{\pgfqpoint{0.469691in}{0.947813in}}{\pgfqpoint{0.484530in}{0.941667in}}{\pgfqpoint{0.500000in}{0.941667in}}%
\pgfpathclose%
\pgfpathmoveto{\pgfqpoint{0.500000in}{0.947500in}}%
\pgfpathcurveto{\pgfqpoint{0.500000in}{0.947500in}}{\pgfqpoint{0.486077in}{0.947500in}}{\pgfqpoint{0.472722in}{0.953032in}}%
\pgfpathcurveto{\pgfqpoint{0.462877in}{0.962877in}}{\pgfqpoint{0.453032in}{0.972722in}}{\pgfqpoint{0.447500in}{0.986077in}}%
\pgfpathcurveto{\pgfqpoint{0.447500in}{1.000000in}}{\pgfqpoint{0.447500in}{1.013923in}}{\pgfqpoint{0.453032in}{1.027278in}}%
\pgfpathcurveto{\pgfqpoint{0.462877in}{1.037123in}}{\pgfqpoint{0.472722in}{1.046968in}}{\pgfqpoint{0.486077in}{1.052500in}}%
\pgfpathcurveto{\pgfqpoint{0.500000in}{1.052500in}}{\pgfqpoint{0.513923in}{1.052500in}}{\pgfqpoint{0.527278in}{1.046968in}}%
\pgfpathcurveto{\pgfqpoint{0.537123in}{1.037123in}}{\pgfqpoint{0.546968in}{1.027278in}}{\pgfqpoint{0.552500in}{1.013923in}}%
\pgfpathcurveto{\pgfqpoint{0.552500in}{1.000000in}}{\pgfqpoint{0.552500in}{0.986077in}}{\pgfqpoint{0.546968in}{0.972722in}}%
\pgfpathcurveto{\pgfqpoint{0.537123in}{0.962877in}}{\pgfqpoint{0.527278in}{0.953032in}}{\pgfqpoint{0.513923in}{0.947500in}}%
\pgfpathclose%
\pgfpathmoveto{\pgfqpoint{0.666667in}{0.941667in}}%
\pgfpathcurveto{\pgfqpoint{0.682137in}{0.941667in}}{\pgfqpoint{0.696975in}{0.947813in}}{\pgfqpoint{0.707915in}{0.958752in}}%
\pgfpathcurveto{\pgfqpoint{0.718854in}{0.969691in}}{\pgfqpoint{0.725000in}{0.984530in}}{\pgfqpoint{0.725000in}{1.000000in}}%
\pgfpathcurveto{\pgfqpoint{0.725000in}{1.015470in}}{\pgfqpoint{0.718854in}{1.030309in}}{\pgfqpoint{0.707915in}{1.041248in}}%
\pgfpathcurveto{\pgfqpoint{0.696975in}{1.052187in}}{\pgfqpoint{0.682137in}{1.058333in}}{\pgfqpoint{0.666667in}{1.058333in}}%
\pgfpathcurveto{\pgfqpoint{0.651196in}{1.058333in}}{\pgfqpoint{0.636358in}{1.052187in}}{\pgfqpoint{0.625419in}{1.041248in}}%
\pgfpathcurveto{\pgfqpoint{0.614480in}{1.030309in}}{\pgfqpoint{0.608333in}{1.015470in}}{\pgfqpoint{0.608333in}{1.000000in}}%
\pgfpathcurveto{\pgfqpoint{0.608333in}{0.984530in}}{\pgfqpoint{0.614480in}{0.969691in}}{\pgfqpoint{0.625419in}{0.958752in}}%
\pgfpathcurveto{\pgfqpoint{0.636358in}{0.947813in}}{\pgfqpoint{0.651196in}{0.941667in}}{\pgfqpoint{0.666667in}{0.941667in}}%
\pgfpathclose%
\pgfpathmoveto{\pgfqpoint{0.666667in}{0.947500in}}%
\pgfpathcurveto{\pgfqpoint{0.666667in}{0.947500in}}{\pgfqpoint{0.652744in}{0.947500in}}{\pgfqpoint{0.639389in}{0.953032in}}%
\pgfpathcurveto{\pgfqpoint{0.629544in}{0.962877in}}{\pgfqpoint{0.619698in}{0.972722in}}{\pgfqpoint{0.614167in}{0.986077in}}%
\pgfpathcurveto{\pgfqpoint{0.614167in}{1.000000in}}{\pgfqpoint{0.614167in}{1.013923in}}{\pgfqpoint{0.619698in}{1.027278in}}%
\pgfpathcurveto{\pgfqpoint{0.629544in}{1.037123in}}{\pgfqpoint{0.639389in}{1.046968in}}{\pgfqpoint{0.652744in}{1.052500in}}%
\pgfpathcurveto{\pgfqpoint{0.666667in}{1.052500in}}{\pgfqpoint{0.680590in}{1.052500in}}{\pgfqpoint{0.693945in}{1.046968in}}%
\pgfpathcurveto{\pgfqpoint{0.703790in}{1.037123in}}{\pgfqpoint{0.713635in}{1.027278in}}{\pgfqpoint{0.719167in}{1.013923in}}%
\pgfpathcurveto{\pgfqpoint{0.719167in}{1.000000in}}{\pgfqpoint{0.719167in}{0.986077in}}{\pgfqpoint{0.713635in}{0.972722in}}%
\pgfpathcurveto{\pgfqpoint{0.703790in}{0.962877in}}{\pgfqpoint{0.693945in}{0.953032in}}{\pgfqpoint{0.680590in}{0.947500in}}%
\pgfpathclose%
\pgfpathmoveto{\pgfqpoint{0.833333in}{0.941667in}}%
\pgfpathcurveto{\pgfqpoint{0.848804in}{0.941667in}}{\pgfqpoint{0.863642in}{0.947813in}}{\pgfqpoint{0.874581in}{0.958752in}}%
\pgfpathcurveto{\pgfqpoint{0.885520in}{0.969691in}}{\pgfqpoint{0.891667in}{0.984530in}}{\pgfqpoint{0.891667in}{1.000000in}}%
\pgfpathcurveto{\pgfqpoint{0.891667in}{1.015470in}}{\pgfqpoint{0.885520in}{1.030309in}}{\pgfqpoint{0.874581in}{1.041248in}}%
\pgfpathcurveto{\pgfqpoint{0.863642in}{1.052187in}}{\pgfqpoint{0.848804in}{1.058333in}}{\pgfqpoint{0.833333in}{1.058333in}}%
\pgfpathcurveto{\pgfqpoint{0.817863in}{1.058333in}}{\pgfqpoint{0.803025in}{1.052187in}}{\pgfqpoint{0.792085in}{1.041248in}}%
\pgfpathcurveto{\pgfqpoint{0.781146in}{1.030309in}}{\pgfqpoint{0.775000in}{1.015470in}}{\pgfqpoint{0.775000in}{1.000000in}}%
\pgfpathcurveto{\pgfqpoint{0.775000in}{0.984530in}}{\pgfqpoint{0.781146in}{0.969691in}}{\pgfqpoint{0.792085in}{0.958752in}}%
\pgfpathcurveto{\pgfqpoint{0.803025in}{0.947813in}}{\pgfqpoint{0.817863in}{0.941667in}}{\pgfqpoint{0.833333in}{0.941667in}}%
\pgfpathclose%
\pgfpathmoveto{\pgfqpoint{0.833333in}{0.947500in}}%
\pgfpathcurveto{\pgfqpoint{0.833333in}{0.947500in}}{\pgfqpoint{0.819410in}{0.947500in}}{\pgfqpoint{0.806055in}{0.953032in}}%
\pgfpathcurveto{\pgfqpoint{0.796210in}{0.962877in}}{\pgfqpoint{0.786365in}{0.972722in}}{\pgfqpoint{0.780833in}{0.986077in}}%
\pgfpathcurveto{\pgfqpoint{0.780833in}{1.000000in}}{\pgfqpoint{0.780833in}{1.013923in}}{\pgfqpoint{0.786365in}{1.027278in}}%
\pgfpathcurveto{\pgfqpoint{0.796210in}{1.037123in}}{\pgfqpoint{0.806055in}{1.046968in}}{\pgfqpoint{0.819410in}{1.052500in}}%
\pgfpathcurveto{\pgfqpoint{0.833333in}{1.052500in}}{\pgfqpoint{0.847256in}{1.052500in}}{\pgfqpoint{0.860611in}{1.046968in}}%
\pgfpathcurveto{\pgfqpoint{0.870456in}{1.037123in}}{\pgfqpoint{0.880302in}{1.027278in}}{\pgfqpoint{0.885833in}{1.013923in}}%
\pgfpathcurveto{\pgfqpoint{0.885833in}{1.000000in}}{\pgfqpoint{0.885833in}{0.986077in}}{\pgfqpoint{0.880302in}{0.972722in}}%
\pgfpathcurveto{\pgfqpoint{0.870456in}{0.962877in}}{\pgfqpoint{0.860611in}{0.953032in}}{\pgfqpoint{0.847256in}{0.947500in}}%
\pgfpathclose%
\pgfpathmoveto{\pgfqpoint{1.000000in}{0.941667in}}%
\pgfpathcurveto{\pgfqpoint{1.015470in}{0.941667in}}{\pgfqpoint{1.030309in}{0.947813in}}{\pgfqpoint{1.041248in}{0.958752in}}%
\pgfpathcurveto{\pgfqpoint{1.052187in}{0.969691in}}{\pgfqpoint{1.058333in}{0.984530in}}{\pgfqpoint{1.058333in}{1.000000in}}%
\pgfpathcurveto{\pgfqpoint{1.058333in}{1.015470in}}{\pgfqpoint{1.052187in}{1.030309in}}{\pgfqpoint{1.041248in}{1.041248in}}%
\pgfpathcurveto{\pgfqpoint{1.030309in}{1.052187in}}{\pgfqpoint{1.015470in}{1.058333in}}{\pgfqpoint{1.000000in}{1.058333in}}%
\pgfpathcurveto{\pgfqpoint{0.984530in}{1.058333in}}{\pgfqpoint{0.969691in}{1.052187in}}{\pgfqpoint{0.958752in}{1.041248in}}%
\pgfpathcurveto{\pgfqpoint{0.947813in}{1.030309in}}{\pgfqpoint{0.941667in}{1.015470in}}{\pgfqpoint{0.941667in}{1.000000in}}%
\pgfpathcurveto{\pgfqpoint{0.941667in}{0.984530in}}{\pgfqpoint{0.947813in}{0.969691in}}{\pgfqpoint{0.958752in}{0.958752in}}%
\pgfpathcurveto{\pgfqpoint{0.969691in}{0.947813in}}{\pgfqpoint{0.984530in}{0.941667in}}{\pgfqpoint{1.000000in}{0.941667in}}%
\pgfpathclose%
\pgfpathmoveto{\pgfqpoint{1.000000in}{0.947500in}}%
\pgfpathcurveto{\pgfqpoint{1.000000in}{0.947500in}}{\pgfqpoint{0.986077in}{0.947500in}}{\pgfqpoint{0.972722in}{0.953032in}}%
\pgfpathcurveto{\pgfqpoint{0.962877in}{0.962877in}}{\pgfqpoint{0.953032in}{0.972722in}}{\pgfqpoint{0.947500in}{0.986077in}}%
\pgfpathcurveto{\pgfqpoint{0.947500in}{1.000000in}}{\pgfqpoint{0.947500in}{1.013923in}}{\pgfqpoint{0.953032in}{1.027278in}}%
\pgfpathcurveto{\pgfqpoint{0.962877in}{1.037123in}}{\pgfqpoint{0.972722in}{1.046968in}}{\pgfqpoint{0.986077in}{1.052500in}}%
\pgfpathcurveto{\pgfqpoint{1.000000in}{1.052500in}}{\pgfqpoint{1.013923in}{1.052500in}}{\pgfqpoint{1.027278in}{1.046968in}}%
\pgfpathcurveto{\pgfqpoint{1.037123in}{1.037123in}}{\pgfqpoint{1.046968in}{1.027278in}}{\pgfqpoint{1.052500in}{1.013923in}}%
\pgfpathcurveto{\pgfqpoint{1.052500in}{1.000000in}}{\pgfqpoint{1.052500in}{0.986077in}}{\pgfqpoint{1.046968in}{0.972722in}}%
\pgfpathcurveto{\pgfqpoint{1.037123in}{0.962877in}}{\pgfqpoint{1.027278in}{0.953032in}}{\pgfqpoint{1.013923in}{0.947500in}}%
\pgfpathclose%
\pgfusepath{stroke}%
\end{pgfscope}%
}%
\pgfsys@transformshift{1.258038in}{3.191361in}%
\pgfsys@useobject{currentpattern}{}%
\pgfsys@transformshift{1in}{0in}%
\pgfsys@transformshift{-1in}{0in}%
\pgfsys@transformshift{0in}{1in}%
\end{pgfscope}%
\begin{pgfscope}%
\pgfpathrectangle{\pgfqpoint{0.870538in}{0.637495in}}{\pgfqpoint{9.300000in}{9.060000in}}%
\pgfusepath{clip}%
\pgfsetbuttcap%
\pgfsetmiterjoin%
\definecolor{currentfill}{rgb}{0.549020,0.337255,0.294118}%
\pgfsetfillcolor{currentfill}%
\pgfsetfillopacity{0.990000}%
\pgfsetlinewidth{0.000000pt}%
\definecolor{currentstroke}{rgb}{0.000000,0.000000,0.000000}%
\pgfsetstrokecolor{currentstroke}%
\pgfsetstrokeopacity{0.990000}%
\pgfsetdash{}{0pt}%
\pgfpathmoveto{\pgfqpoint{2.808038in}{2.630003in}}%
\pgfpathlineto{\pgfqpoint{3.583038in}{2.630003in}}%
\pgfpathlineto{\pgfqpoint{3.583038in}{3.468565in}}%
\pgfpathlineto{\pgfqpoint{2.808038in}{3.468565in}}%
\pgfpathclose%
\pgfusepath{fill}%
\end{pgfscope}%
\begin{pgfscope}%
\pgfsetbuttcap%
\pgfsetmiterjoin%
\definecolor{currentfill}{rgb}{0.549020,0.337255,0.294118}%
\pgfsetfillcolor{currentfill}%
\pgfsetfillopacity{0.990000}%
\pgfsetlinewidth{0.000000pt}%
\definecolor{currentstroke}{rgb}{0.000000,0.000000,0.000000}%
\pgfsetstrokecolor{currentstroke}%
\pgfsetstrokeopacity{0.990000}%
\pgfsetdash{}{0pt}%
\pgfpathrectangle{\pgfqpoint{0.870538in}{0.637495in}}{\pgfqpoint{9.300000in}{9.060000in}}%
\pgfusepath{clip}%
\pgfpathmoveto{\pgfqpoint{2.808038in}{2.630003in}}%
\pgfpathlineto{\pgfqpoint{3.583038in}{2.630003in}}%
\pgfpathlineto{\pgfqpoint{3.583038in}{3.468565in}}%
\pgfpathlineto{\pgfqpoint{2.808038in}{3.468565in}}%
\pgfpathclose%
\pgfusepath{clip}%
\pgfsys@defobject{currentpattern}{\pgfqpoint{0in}{0in}}{\pgfqpoint{1in}{1in}}{%
\begin{pgfscope}%
\pgfpathrectangle{\pgfqpoint{0in}{0in}}{\pgfqpoint{1in}{1in}}%
\pgfusepath{clip}%
\pgfpathmoveto{\pgfqpoint{0.000000in}{-0.058333in}}%
\pgfpathcurveto{\pgfqpoint{0.015470in}{-0.058333in}}{\pgfqpoint{0.030309in}{-0.052187in}}{\pgfqpoint{0.041248in}{-0.041248in}}%
\pgfpathcurveto{\pgfqpoint{0.052187in}{-0.030309in}}{\pgfqpoint{0.058333in}{-0.015470in}}{\pgfqpoint{0.058333in}{0.000000in}}%
\pgfpathcurveto{\pgfqpoint{0.058333in}{0.015470in}}{\pgfqpoint{0.052187in}{0.030309in}}{\pgfqpoint{0.041248in}{0.041248in}}%
\pgfpathcurveto{\pgfqpoint{0.030309in}{0.052187in}}{\pgfqpoint{0.015470in}{0.058333in}}{\pgfqpoint{0.000000in}{0.058333in}}%
\pgfpathcurveto{\pgfqpoint{-0.015470in}{0.058333in}}{\pgfqpoint{-0.030309in}{0.052187in}}{\pgfqpoint{-0.041248in}{0.041248in}}%
\pgfpathcurveto{\pgfqpoint{-0.052187in}{0.030309in}}{\pgfqpoint{-0.058333in}{0.015470in}}{\pgfqpoint{-0.058333in}{0.000000in}}%
\pgfpathcurveto{\pgfqpoint{-0.058333in}{-0.015470in}}{\pgfqpoint{-0.052187in}{-0.030309in}}{\pgfqpoint{-0.041248in}{-0.041248in}}%
\pgfpathcurveto{\pgfqpoint{-0.030309in}{-0.052187in}}{\pgfqpoint{-0.015470in}{-0.058333in}}{\pgfqpoint{0.000000in}{-0.058333in}}%
\pgfpathclose%
\pgfpathmoveto{\pgfqpoint{0.000000in}{-0.052500in}}%
\pgfpathcurveto{\pgfqpoint{0.000000in}{-0.052500in}}{\pgfqpoint{-0.013923in}{-0.052500in}}{\pgfqpoint{-0.027278in}{-0.046968in}}%
\pgfpathcurveto{\pgfqpoint{-0.037123in}{-0.037123in}}{\pgfqpoint{-0.046968in}{-0.027278in}}{\pgfqpoint{-0.052500in}{-0.013923in}}%
\pgfpathcurveto{\pgfqpoint{-0.052500in}{0.000000in}}{\pgfqpoint{-0.052500in}{0.013923in}}{\pgfqpoint{-0.046968in}{0.027278in}}%
\pgfpathcurveto{\pgfqpoint{-0.037123in}{0.037123in}}{\pgfqpoint{-0.027278in}{0.046968in}}{\pgfqpoint{-0.013923in}{0.052500in}}%
\pgfpathcurveto{\pgfqpoint{0.000000in}{0.052500in}}{\pgfqpoint{0.013923in}{0.052500in}}{\pgfqpoint{0.027278in}{0.046968in}}%
\pgfpathcurveto{\pgfqpoint{0.037123in}{0.037123in}}{\pgfqpoint{0.046968in}{0.027278in}}{\pgfqpoint{0.052500in}{0.013923in}}%
\pgfpathcurveto{\pgfqpoint{0.052500in}{0.000000in}}{\pgfqpoint{0.052500in}{-0.013923in}}{\pgfqpoint{0.046968in}{-0.027278in}}%
\pgfpathcurveto{\pgfqpoint{0.037123in}{-0.037123in}}{\pgfqpoint{0.027278in}{-0.046968in}}{\pgfqpoint{0.013923in}{-0.052500in}}%
\pgfpathclose%
\pgfpathmoveto{\pgfqpoint{0.166667in}{-0.058333in}}%
\pgfpathcurveto{\pgfqpoint{0.182137in}{-0.058333in}}{\pgfqpoint{0.196975in}{-0.052187in}}{\pgfqpoint{0.207915in}{-0.041248in}}%
\pgfpathcurveto{\pgfqpoint{0.218854in}{-0.030309in}}{\pgfqpoint{0.225000in}{-0.015470in}}{\pgfqpoint{0.225000in}{0.000000in}}%
\pgfpathcurveto{\pgfqpoint{0.225000in}{0.015470in}}{\pgfqpoint{0.218854in}{0.030309in}}{\pgfqpoint{0.207915in}{0.041248in}}%
\pgfpathcurveto{\pgfqpoint{0.196975in}{0.052187in}}{\pgfqpoint{0.182137in}{0.058333in}}{\pgfqpoint{0.166667in}{0.058333in}}%
\pgfpathcurveto{\pgfqpoint{0.151196in}{0.058333in}}{\pgfqpoint{0.136358in}{0.052187in}}{\pgfqpoint{0.125419in}{0.041248in}}%
\pgfpathcurveto{\pgfqpoint{0.114480in}{0.030309in}}{\pgfqpoint{0.108333in}{0.015470in}}{\pgfqpoint{0.108333in}{0.000000in}}%
\pgfpathcurveto{\pgfqpoint{0.108333in}{-0.015470in}}{\pgfqpoint{0.114480in}{-0.030309in}}{\pgfqpoint{0.125419in}{-0.041248in}}%
\pgfpathcurveto{\pgfqpoint{0.136358in}{-0.052187in}}{\pgfqpoint{0.151196in}{-0.058333in}}{\pgfqpoint{0.166667in}{-0.058333in}}%
\pgfpathclose%
\pgfpathmoveto{\pgfqpoint{0.166667in}{-0.052500in}}%
\pgfpathcurveto{\pgfqpoint{0.166667in}{-0.052500in}}{\pgfqpoint{0.152744in}{-0.052500in}}{\pgfqpoint{0.139389in}{-0.046968in}}%
\pgfpathcurveto{\pgfqpoint{0.129544in}{-0.037123in}}{\pgfqpoint{0.119698in}{-0.027278in}}{\pgfqpoint{0.114167in}{-0.013923in}}%
\pgfpathcurveto{\pgfqpoint{0.114167in}{0.000000in}}{\pgfqpoint{0.114167in}{0.013923in}}{\pgfqpoint{0.119698in}{0.027278in}}%
\pgfpathcurveto{\pgfqpoint{0.129544in}{0.037123in}}{\pgfqpoint{0.139389in}{0.046968in}}{\pgfqpoint{0.152744in}{0.052500in}}%
\pgfpathcurveto{\pgfqpoint{0.166667in}{0.052500in}}{\pgfqpoint{0.180590in}{0.052500in}}{\pgfqpoint{0.193945in}{0.046968in}}%
\pgfpathcurveto{\pgfqpoint{0.203790in}{0.037123in}}{\pgfqpoint{0.213635in}{0.027278in}}{\pgfqpoint{0.219167in}{0.013923in}}%
\pgfpathcurveto{\pgfqpoint{0.219167in}{0.000000in}}{\pgfqpoint{0.219167in}{-0.013923in}}{\pgfqpoint{0.213635in}{-0.027278in}}%
\pgfpathcurveto{\pgfqpoint{0.203790in}{-0.037123in}}{\pgfqpoint{0.193945in}{-0.046968in}}{\pgfqpoint{0.180590in}{-0.052500in}}%
\pgfpathclose%
\pgfpathmoveto{\pgfqpoint{0.333333in}{-0.058333in}}%
\pgfpathcurveto{\pgfqpoint{0.348804in}{-0.058333in}}{\pgfqpoint{0.363642in}{-0.052187in}}{\pgfqpoint{0.374581in}{-0.041248in}}%
\pgfpathcurveto{\pgfqpoint{0.385520in}{-0.030309in}}{\pgfqpoint{0.391667in}{-0.015470in}}{\pgfqpoint{0.391667in}{0.000000in}}%
\pgfpathcurveto{\pgfqpoint{0.391667in}{0.015470in}}{\pgfqpoint{0.385520in}{0.030309in}}{\pgfqpoint{0.374581in}{0.041248in}}%
\pgfpathcurveto{\pgfqpoint{0.363642in}{0.052187in}}{\pgfqpoint{0.348804in}{0.058333in}}{\pgfqpoint{0.333333in}{0.058333in}}%
\pgfpathcurveto{\pgfqpoint{0.317863in}{0.058333in}}{\pgfqpoint{0.303025in}{0.052187in}}{\pgfqpoint{0.292085in}{0.041248in}}%
\pgfpathcurveto{\pgfqpoint{0.281146in}{0.030309in}}{\pgfqpoint{0.275000in}{0.015470in}}{\pgfqpoint{0.275000in}{0.000000in}}%
\pgfpathcurveto{\pgfqpoint{0.275000in}{-0.015470in}}{\pgfqpoint{0.281146in}{-0.030309in}}{\pgfqpoint{0.292085in}{-0.041248in}}%
\pgfpathcurveto{\pgfqpoint{0.303025in}{-0.052187in}}{\pgfqpoint{0.317863in}{-0.058333in}}{\pgfqpoint{0.333333in}{-0.058333in}}%
\pgfpathclose%
\pgfpathmoveto{\pgfqpoint{0.333333in}{-0.052500in}}%
\pgfpathcurveto{\pgfqpoint{0.333333in}{-0.052500in}}{\pgfqpoint{0.319410in}{-0.052500in}}{\pgfqpoint{0.306055in}{-0.046968in}}%
\pgfpathcurveto{\pgfqpoint{0.296210in}{-0.037123in}}{\pgfqpoint{0.286365in}{-0.027278in}}{\pgfqpoint{0.280833in}{-0.013923in}}%
\pgfpathcurveto{\pgfqpoint{0.280833in}{0.000000in}}{\pgfqpoint{0.280833in}{0.013923in}}{\pgfqpoint{0.286365in}{0.027278in}}%
\pgfpathcurveto{\pgfqpoint{0.296210in}{0.037123in}}{\pgfqpoint{0.306055in}{0.046968in}}{\pgfqpoint{0.319410in}{0.052500in}}%
\pgfpathcurveto{\pgfqpoint{0.333333in}{0.052500in}}{\pgfqpoint{0.347256in}{0.052500in}}{\pgfqpoint{0.360611in}{0.046968in}}%
\pgfpathcurveto{\pgfqpoint{0.370456in}{0.037123in}}{\pgfqpoint{0.380302in}{0.027278in}}{\pgfqpoint{0.385833in}{0.013923in}}%
\pgfpathcurveto{\pgfqpoint{0.385833in}{0.000000in}}{\pgfqpoint{0.385833in}{-0.013923in}}{\pgfqpoint{0.380302in}{-0.027278in}}%
\pgfpathcurveto{\pgfqpoint{0.370456in}{-0.037123in}}{\pgfqpoint{0.360611in}{-0.046968in}}{\pgfqpoint{0.347256in}{-0.052500in}}%
\pgfpathclose%
\pgfpathmoveto{\pgfqpoint{0.500000in}{-0.058333in}}%
\pgfpathcurveto{\pgfqpoint{0.515470in}{-0.058333in}}{\pgfqpoint{0.530309in}{-0.052187in}}{\pgfqpoint{0.541248in}{-0.041248in}}%
\pgfpathcurveto{\pgfqpoint{0.552187in}{-0.030309in}}{\pgfqpoint{0.558333in}{-0.015470in}}{\pgfqpoint{0.558333in}{0.000000in}}%
\pgfpathcurveto{\pgfqpoint{0.558333in}{0.015470in}}{\pgfqpoint{0.552187in}{0.030309in}}{\pgfqpoint{0.541248in}{0.041248in}}%
\pgfpathcurveto{\pgfqpoint{0.530309in}{0.052187in}}{\pgfqpoint{0.515470in}{0.058333in}}{\pgfqpoint{0.500000in}{0.058333in}}%
\pgfpathcurveto{\pgfqpoint{0.484530in}{0.058333in}}{\pgfqpoint{0.469691in}{0.052187in}}{\pgfqpoint{0.458752in}{0.041248in}}%
\pgfpathcurveto{\pgfqpoint{0.447813in}{0.030309in}}{\pgfqpoint{0.441667in}{0.015470in}}{\pgfqpoint{0.441667in}{0.000000in}}%
\pgfpathcurveto{\pgfqpoint{0.441667in}{-0.015470in}}{\pgfqpoint{0.447813in}{-0.030309in}}{\pgfqpoint{0.458752in}{-0.041248in}}%
\pgfpathcurveto{\pgfqpoint{0.469691in}{-0.052187in}}{\pgfqpoint{0.484530in}{-0.058333in}}{\pgfqpoint{0.500000in}{-0.058333in}}%
\pgfpathclose%
\pgfpathmoveto{\pgfqpoint{0.500000in}{-0.052500in}}%
\pgfpathcurveto{\pgfqpoint{0.500000in}{-0.052500in}}{\pgfqpoint{0.486077in}{-0.052500in}}{\pgfqpoint{0.472722in}{-0.046968in}}%
\pgfpathcurveto{\pgfqpoint{0.462877in}{-0.037123in}}{\pgfqpoint{0.453032in}{-0.027278in}}{\pgfqpoint{0.447500in}{-0.013923in}}%
\pgfpathcurveto{\pgfqpoint{0.447500in}{0.000000in}}{\pgfqpoint{0.447500in}{0.013923in}}{\pgfqpoint{0.453032in}{0.027278in}}%
\pgfpathcurveto{\pgfqpoint{0.462877in}{0.037123in}}{\pgfqpoint{0.472722in}{0.046968in}}{\pgfqpoint{0.486077in}{0.052500in}}%
\pgfpathcurveto{\pgfqpoint{0.500000in}{0.052500in}}{\pgfqpoint{0.513923in}{0.052500in}}{\pgfqpoint{0.527278in}{0.046968in}}%
\pgfpathcurveto{\pgfqpoint{0.537123in}{0.037123in}}{\pgfqpoint{0.546968in}{0.027278in}}{\pgfqpoint{0.552500in}{0.013923in}}%
\pgfpathcurveto{\pgfqpoint{0.552500in}{0.000000in}}{\pgfqpoint{0.552500in}{-0.013923in}}{\pgfqpoint{0.546968in}{-0.027278in}}%
\pgfpathcurveto{\pgfqpoint{0.537123in}{-0.037123in}}{\pgfqpoint{0.527278in}{-0.046968in}}{\pgfqpoint{0.513923in}{-0.052500in}}%
\pgfpathclose%
\pgfpathmoveto{\pgfqpoint{0.666667in}{-0.058333in}}%
\pgfpathcurveto{\pgfqpoint{0.682137in}{-0.058333in}}{\pgfqpoint{0.696975in}{-0.052187in}}{\pgfqpoint{0.707915in}{-0.041248in}}%
\pgfpathcurveto{\pgfqpoint{0.718854in}{-0.030309in}}{\pgfqpoint{0.725000in}{-0.015470in}}{\pgfqpoint{0.725000in}{0.000000in}}%
\pgfpathcurveto{\pgfqpoint{0.725000in}{0.015470in}}{\pgfqpoint{0.718854in}{0.030309in}}{\pgfqpoint{0.707915in}{0.041248in}}%
\pgfpathcurveto{\pgfqpoint{0.696975in}{0.052187in}}{\pgfqpoint{0.682137in}{0.058333in}}{\pgfqpoint{0.666667in}{0.058333in}}%
\pgfpathcurveto{\pgfqpoint{0.651196in}{0.058333in}}{\pgfqpoint{0.636358in}{0.052187in}}{\pgfqpoint{0.625419in}{0.041248in}}%
\pgfpathcurveto{\pgfqpoint{0.614480in}{0.030309in}}{\pgfqpoint{0.608333in}{0.015470in}}{\pgfqpoint{0.608333in}{0.000000in}}%
\pgfpathcurveto{\pgfqpoint{0.608333in}{-0.015470in}}{\pgfqpoint{0.614480in}{-0.030309in}}{\pgfqpoint{0.625419in}{-0.041248in}}%
\pgfpathcurveto{\pgfqpoint{0.636358in}{-0.052187in}}{\pgfqpoint{0.651196in}{-0.058333in}}{\pgfqpoint{0.666667in}{-0.058333in}}%
\pgfpathclose%
\pgfpathmoveto{\pgfqpoint{0.666667in}{-0.052500in}}%
\pgfpathcurveto{\pgfqpoint{0.666667in}{-0.052500in}}{\pgfqpoint{0.652744in}{-0.052500in}}{\pgfqpoint{0.639389in}{-0.046968in}}%
\pgfpathcurveto{\pgfqpoint{0.629544in}{-0.037123in}}{\pgfqpoint{0.619698in}{-0.027278in}}{\pgfqpoint{0.614167in}{-0.013923in}}%
\pgfpathcurveto{\pgfqpoint{0.614167in}{0.000000in}}{\pgfqpoint{0.614167in}{0.013923in}}{\pgfqpoint{0.619698in}{0.027278in}}%
\pgfpathcurveto{\pgfqpoint{0.629544in}{0.037123in}}{\pgfqpoint{0.639389in}{0.046968in}}{\pgfqpoint{0.652744in}{0.052500in}}%
\pgfpathcurveto{\pgfqpoint{0.666667in}{0.052500in}}{\pgfqpoint{0.680590in}{0.052500in}}{\pgfqpoint{0.693945in}{0.046968in}}%
\pgfpathcurveto{\pgfqpoint{0.703790in}{0.037123in}}{\pgfqpoint{0.713635in}{0.027278in}}{\pgfqpoint{0.719167in}{0.013923in}}%
\pgfpathcurveto{\pgfqpoint{0.719167in}{0.000000in}}{\pgfqpoint{0.719167in}{-0.013923in}}{\pgfqpoint{0.713635in}{-0.027278in}}%
\pgfpathcurveto{\pgfqpoint{0.703790in}{-0.037123in}}{\pgfqpoint{0.693945in}{-0.046968in}}{\pgfqpoint{0.680590in}{-0.052500in}}%
\pgfpathclose%
\pgfpathmoveto{\pgfqpoint{0.833333in}{-0.058333in}}%
\pgfpathcurveto{\pgfqpoint{0.848804in}{-0.058333in}}{\pgfqpoint{0.863642in}{-0.052187in}}{\pgfqpoint{0.874581in}{-0.041248in}}%
\pgfpathcurveto{\pgfqpoint{0.885520in}{-0.030309in}}{\pgfqpoint{0.891667in}{-0.015470in}}{\pgfqpoint{0.891667in}{0.000000in}}%
\pgfpathcurveto{\pgfqpoint{0.891667in}{0.015470in}}{\pgfqpoint{0.885520in}{0.030309in}}{\pgfqpoint{0.874581in}{0.041248in}}%
\pgfpathcurveto{\pgfqpoint{0.863642in}{0.052187in}}{\pgfqpoint{0.848804in}{0.058333in}}{\pgfqpoint{0.833333in}{0.058333in}}%
\pgfpathcurveto{\pgfqpoint{0.817863in}{0.058333in}}{\pgfqpoint{0.803025in}{0.052187in}}{\pgfqpoint{0.792085in}{0.041248in}}%
\pgfpathcurveto{\pgfqpoint{0.781146in}{0.030309in}}{\pgfqpoint{0.775000in}{0.015470in}}{\pgfqpoint{0.775000in}{0.000000in}}%
\pgfpathcurveto{\pgfqpoint{0.775000in}{-0.015470in}}{\pgfqpoint{0.781146in}{-0.030309in}}{\pgfqpoint{0.792085in}{-0.041248in}}%
\pgfpathcurveto{\pgfqpoint{0.803025in}{-0.052187in}}{\pgfqpoint{0.817863in}{-0.058333in}}{\pgfqpoint{0.833333in}{-0.058333in}}%
\pgfpathclose%
\pgfpathmoveto{\pgfqpoint{0.833333in}{-0.052500in}}%
\pgfpathcurveto{\pgfqpoint{0.833333in}{-0.052500in}}{\pgfqpoint{0.819410in}{-0.052500in}}{\pgfqpoint{0.806055in}{-0.046968in}}%
\pgfpathcurveto{\pgfqpoint{0.796210in}{-0.037123in}}{\pgfqpoint{0.786365in}{-0.027278in}}{\pgfqpoint{0.780833in}{-0.013923in}}%
\pgfpathcurveto{\pgfqpoint{0.780833in}{0.000000in}}{\pgfqpoint{0.780833in}{0.013923in}}{\pgfqpoint{0.786365in}{0.027278in}}%
\pgfpathcurveto{\pgfqpoint{0.796210in}{0.037123in}}{\pgfqpoint{0.806055in}{0.046968in}}{\pgfqpoint{0.819410in}{0.052500in}}%
\pgfpathcurveto{\pgfqpoint{0.833333in}{0.052500in}}{\pgfqpoint{0.847256in}{0.052500in}}{\pgfqpoint{0.860611in}{0.046968in}}%
\pgfpathcurveto{\pgfqpoint{0.870456in}{0.037123in}}{\pgfqpoint{0.880302in}{0.027278in}}{\pgfqpoint{0.885833in}{0.013923in}}%
\pgfpathcurveto{\pgfqpoint{0.885833in}{0.000000in}}{\pgfqpoint{0.885833in}{-0.013923in}}{\pgfqpoint{0.880302in}{-0.027278in}}%
\pgfpathcurveto{\pgfqpoint{0.870456in}{-0.037123in}}{\pgfqpoint{0.860611in}{-0.046968in}}{\pgfqpoint{0.847256in}{-0.052500in}}%
\pgfpathclose%
\pgfpathmoveto{\pgfqpoint{1.000000in}{-0.058333in}}%
\pgfpathcurveto{\pgfqpoint{1.015470in}{-0.058333in}}{\pgfqpoint{1.030309in}{-0.052187in}}{\pgfqpoint{1.041248in}{-0.041248in}}%
\pgfpathcurveto{\pgfqpoint{1.052187in}{-0.030309in}}{\pgfqpoint{1.058333in}{-0.015470in}}{\pgfqpoint{1.058333in}{0.000000in}}%
\pgfpathcurveto{\pgfqpoint{1.058333in}{0.015470in}}{\pgfqpoint{1.052187in}{0.030309in}}{\pgfqpoint{1.041248in}{0.041248in}}%
\pgfpathcurveto{\pgfqpoint{1.030309in}{0.052187in}}{\pgfqpoint{1.015470in}{0.058333in}}{\pgfqpoint{1.000000in}{0.058333in}}%
\pgfpathcurveto{\pgfqpoint{0.984530in}{0.058333in}}{\pgfqpoint{0.969691in}{0.052187in}}{\pgfqpoint{0.958752in}{0.041248in}}%
\pgfpathcurveto{\pgfqpoint{0.947813in}{0.030309in}}{\pgfqpoint{0.941667in}{0.015470in}}{\pgfqpoint{0.941667in}{0.000000in}}%
\pgfpathcurveto{\pgfqpoint{0.941667in}{-0.015470in}}{\pgfqpoint{0.947813in}{-0.030309in}}{\pgfqpoint{0.958752in}{-0.041248in}}%
\pgfpathcurveto{\pgfqpoint{0.969691in}{-0.052187in}}{\pgfqpoint{0.984530in}{-0.058333in}}{\pgfqpoint{1.000000in}{-0.058333in}}%
\pgfpathclose%
\pgfpathmoveto{\pgfqpoint{1.000000in}{-0.052500in}}%
\pgfpathcurveto{\pgfqpoint{1.000000in}{-0.052500in}}{\pgfqpoint{0.986077in}{-0.052500in}}{\pgfqpoint{0.972722in}{-0.046968in}}%
\pgfpathcurveto{\pgfqpoint{0.962877in}{-0.037123in}}{\pgfqpoint{0.953032in}{-0.027278in}}{\pgfqpoint{0.947500in}{-0.013923in}}%
\pgfpathcurveto{\pgfqpoint{0.947500in}{0.000000in}}{\pgfqpoint{0.947500in}{0.013923in}}{\pgfqpoint{0.953032in}{0.027278in}}%
\pgfpathcurveto{\pgfqpoint{0.962877in}{0.037123in}}{\pgfqpoint{0.972722in}{0.046968in}}{\pgfqpoint{0.986077in}{0.052500in}}%
\pgfpathcurveto{\pgfqpoint{1.000000in}{0.052500in}}{\pgfqpoint{1.013923in}{0.052500in}}{\pgfqpoint{1.027278in}{0.046968in}}%
\pgfpathcurveto{\pgfqpoint{1.037123in}{0.037123in}}{\pgfqpoint{1.046968in}{0.027278in}}{\pgfqpoint{1.052500in}{0.013923in}}%
\pgfpathcurveto{\pgfqpoint{1.052500in}{0.000000in}}{\pgfqpoint{1.052500in}{-0.013923in}}{\pgfqpoint{1.046968in}{-0.027278in}}%
\pgfpathcurveto{\pgfqpoint{1.037123in}{-0.037123in}}{\pgfqpoint{1.027278in}{-0.046968in}}{\pgfqpoint{1.013923in}{-0.052500in}}%
\pgfpathclose%
\pgfpathmoveto{\pgfqpoint{0.083333in}{0.108333in}}%
\pgfpathcurveto{\pgfqpoint{0.098804in}{0.108333in}}{\pgfqpoint{0.113642in}{0.114480in}}{\pgfqpoint{0.124581in}{0.125419in}}%
\pgfpathcurveto{\pgfqpoint{0.135520in}{0.136358in}}{\pgfqpoint{0.141667in}{0.151196in}}{\pgfqpoint{0.141667in}{0.166667in}}%
\pgfpathcurveto{\pgfqpoint{0.141667in}{0.182137in}}{\pgfqpoint{0.135520in}{0.196975in}}{\pgfqpoint{0.124581in}{0.207915in}}%
\pgfpathcurveto{\pgfqpoint{0.113642in}{0.218854in}}{\pgfqpoint{0.098804in}{0.225000in}}{\pgfqpoint{0.083333in}{0.225000in}}%
\pgfpathcurveto{\pgfqpoint{0.067863in}{0.225000in}}{\pgfqpoint{0.053025in}{0.218854in}}{\pgfqpoint{0.042085in}{0.207915in}}%
\pgfpathcurveto{\pgfqpoint{0.031146in}{0.196975in}}{\pgfqpoint{0.025000in}{0.182137in}}{\pgfqpoint{0.025000in}{0.166667in}}%
\pgfpathcurveto{\pgfqpoint{0.025000in}{0.151196in}}{\pgfqpoint{0.031146in}{0.136358in}}{\pgfqpoint{0.042085in}{0.125419in}}%
\pgfpathcurveto{\pgfqpoint{0.053025in}{0.114480in}}{\pgfqpoint{0.067863in}{0.108333in}}{\pgfqpoint{0.083333in}{0.108333in}}%
\pgfpathclose%
\pgfpathmoveto{\pgfqpoint{0.083333in}{0.114167in}}%
\pgfpathcurveto{\pgfqpoint{0.083333in}{0.114167in}}{\pgfqpoint{0.069410in}{0.114167in}}{\pgfqpoint{0.056055in}{0.119698in}}%
\pgfpathcurveto{\pgfqpoint{0.046210in}{0.129544in}}{\pgfqpoint{0.036365in}{0.139389in}}{\pgfqpoint{0.030833in}{0.152744in}}%
\pgfpathcurveto{\pgfqpoint{0.030833in}{0.166667in}}{\pgfqpoint{0.030833in}{0.180590in}}{\pgfqpoint{0.036365in}{0.193945in}}%
\pgfpathcurveto{\pgfqpoint{0.046210in}{0.203790in}}{\pgfqpoint{0.056055in}{0.213635in}}{\pgfqpoint{0.069410in}{0.219167in}}%
\pgfpathcurveto{\pgfqpoint{0.083333in}{0.219167in}}{\pgfqpoint{0.097256in}{0.219167in}}{\pgfqpoint{0.110611in}{0.213635in}}%
\pgfpathcurveto{\pgfqpoint{0.120456in}{0.203790in}}{\pgfqpoint{0.130302in}{0.193945in}}{\pgfqpoint{0.135833in}{0.180590in}}%
\pgfpathcurveto{\pgfqpoint{0.135833in}{0.166667in}}{\pgfqpoint{0.135833in}{0.152744in}}{\pgfqpoint{0.130302in}{0.139389in}}%
\pgfpathcurveto{\pgfqpoint{0.120456in}{0.129544in}}{\pgfqpoint{0.110611in}{0.119698in}}{\pgfqpoint{0.097256in}{0.114167in}}%
\pgfpathclose%
\pgfpathmoveto{\pgfqpoint{0.250000in}{0.108333in}}%
\pgfpathcurveto{\pgfqpoint{0.265470in}{0.108333in}}{\pgfqpoint{0.280309in}{0.114480in}}{\pgfqpoint{0.291248in}{0.125419in}}%
\pgfpathcurveto{\pgfqpoint{0.302187in}{0.136358in}}{\pgfqpoint{0.308333in}{0.151196in}}{\pgfqpoint{0.308333in}{0.166667in}}%
\pgfpathcurveto{\pgfqpoint{0.308333in}{0.182137in}}{\pgfqpoint{0.302187in}{0.196975in}}{\pgfqpoint{0.291248in}{0.207915in}}%
\pgfpathcurveto{\pgfqpoint{0.280309in}{0.218854in}}{\pgfqpoint{0.265470in}{0.225000in}}{\pgfqpoint{0.250000in}{0.225000in}}%
\pgfpathcurveto{\pgfqpoint{0.234530in}{0.225000in}}{\pgfqpoint{0.219691in}{0.218854in}}{\pgfqpoint{0.208752in}{0.207915in}}%
\pgfpathcurveto{\pgfqpoint{0.197813in}{0.196975in}}{\pgfqpoint{0.191667in}{0.182137in}}{\pgfqpoint{0.191667in}{0.166667in}}%
\pgfpathcurveto{\pgfqpoint{0.191667in}{0.151196in}}{\pgfqpoint{0.197813in}{0.136358in}}{\pgfqpoint{0.208752in}{0.125419in}}%
\pgfpathcurveto{\pgfqpoint{0.219691in}{0.114480in}}{\pgfqpoint{0.234530in}{0.108333in}}{\pgfqpoint{0.250000in}{0.108333in}}%
\pgfpathclose%
\pgfpathmoveto{\pgfqpoint{0.250000in}{0.114167in}}%
\pgfpathcurveto{\pgfqpoint{0.250000in}{0.114167in}}{\pgfqpoint{0.236077in}{0.114167in}}{\pgfqpoint{0.222722in}{0.119698in}}%
\pgfpathcurveto{\pgfqpoint{0.212877in}{0.129544in}}{\pgfqpoint{0.203032in}{0.139389in}}{\pgfqpoint{0.197500in}{0.152744in}}%
\pgfpathcurveto{\pgfqpoint{0.197500in}{0.166667in}}{\pgfqpoint{0.197500in}{0.180590in}}{\pgfqpoint{0.203032in}{0.193945in}}%
\pgfpathcurveto{\pgfqpoint{0.212877in}{0.203790in}}{\pgfqpoint{0.222722in}{0.213635in}}{\pgfqpoint{0.236077in}{0.219167in}}%
\pgfpathcurveto{\pgfqpoint{0.250000in}{0.219167in}}{\pgfqpoint{0.263923in}{0.219167in}}{\pgfqpoint{0.277278in}{0.213635in}}%
\pgfpathcurveto{\pgfqpoint{0.287123in}{0.203790in}}{\pgfqpoint{0.296968in}{0.193945in}}{\pgfqpoint{0.302500in}{0.180590in}}%
\pgfpathcurveto{\pgfqpoint{0.302500in}{0.166667in}}{\pgfqpoint{0.302500in}{0.152744in}}{\pgfqpoint{0.296968in}{0.139389in}}%
\pgfpathcurveto{\pgfqpoint{0.287123in}{0.129544in}}{\pgfqpoint{0.277278in}{0.119698in}}{\pgfqpoint{0.263923in}{0.114167in}}%
\pgfpathclose%
\pgfpathmoveto{\pgfqpoint{0.416667in}{0.108333in}}%
\pgfpathcurveto{\pgfqpoint{0.432137in}{0.108333in}}{\pgfqpoint{0.446975in}{0.114480in}}{\pgfqpoint{0.457915in}{0.125419in}}%
\pgfpathcurveto{\pgfqpoint{0.468854in}{0.136358in}}{\pgfqpoint{0.475000in}{0.151196in}}{\pgfqpoint{0.475000in}{0.166667in}}%
\pgfpathcurveto{\pgfqpoint{0.475000in}{0.182137in}}{\pgfqpoint{0.468854in}{0.196975in}}{\pgfqpoint{0.457915in}{0.207915in}}%
\pgfpathcurveto{\pgfqpoint{0.446975in}{0.218854in}}{\pgfqpoint{0.432137in}{0.225000in}}{\pgfqpoint{0.416667in}{0.225000in}}%
\pgfpathcurveto{\pgfqpoint{0.401196in}{0.225000in}}{\pgfqpoint{0.386358in}{0.218854in}}{\pgfqpoint{0.375419in}{0.207915in}}%
\pgfpathcurveto{\pgfqpoint{0.364480in}{0.196975in}}{\pgfqpoint{0.358333in}{0.182137in}}{\pgfqpoint{0.358333in}{0.166667in}}%
\pgfpathcurveto{\pgfqpoint{0.358333in}{0.151196in}}{\pgfqpoint{0.364480in}{0.136358in}}{\pgfqpoint{0.375419in}{0.125419in}}%
\pgfpathcurveto{\pgfqpoint{0.386358in}{0.114480in}}{\pgfqpoint{0.401196in}{0.108333in}}{\pgfqpoint{0.416667in}{0.108333in}}%
\pgfpathclose%
\pgfpathmoveto{\pgfqpoint{0.416667in}{0.114167in}}%
\pgfpathcurveto{\pgfqpoint{0.416667in}{0.114167in}}{\pgfqpoint{0.402744in}{0.114167in}}{\pgfqpoint{0.389389in}{0.119698in}}%
\pgfpathcurveto{\pgfqpoint{0.379544in}{0.129544in}}{\pgfqpoint{0.369698in}{0.139389in}}{\pgfqpoint{0.364167in}{0.152744in}}%
\pgfpathcurveto{\pgfqpoint{0.364167in}{0.166667in}}{\pgfqpoint{0.364167in}{0.180590in}}{\pgfqpoint{0.369698in}{0.193945in}}%
\pgfpathcurveto{\pgfqpoint{0.379544in}{0.203790in}}{\pgfqpoint{0.389389in}{0.213635in}}{\pgfqpoint{0.402744in}{0.219167in}}%
\pgfpathcurveto{\pgfqpoint{0.416667in}{0.219167in}}{\pgfqpoint{0.430590in}{0.219167in}}{\pgfqpoint{0.443945in}{0.213635in}}%
\pgfpathcurveto{\pgfqpoint{0.453790in}{0.203790in}}{\pgfqpoint{0.463635in}{0.193945in}}{\pgfqpoint{0.469167in}{0.180590in}}%
\pgfpathcurveto{\pgfqpoint{0.469167in}{0.166667in}}{\pgfqpoint{0.469167in}{0.152744in}}{\pgfqpoint{0.463635in}{0.139389in}}%
\pgfpathcurveto{\pgfqpoint{0.453790in}{0.129544in}}{\pgfqpoint{0.443945in}{0.119698in}}{\pgfqpoint{0.430590in}{0.114167in}}%
\pgfpathclose%
\pgfpathmoveto{\pgfqpoint{0.583333in}{0.108333in}}%
\pgfpathcurveto{\pgfqpoint{0.598804in}{0.108333in}}{\pgfqpoint{0.613642in}{0.114480in}}{\pgfqpoint{0.624581in}{0.125419in}}%
\pgfpathcurveto{\pgfqpoint{0.635520in}{0.136358in}}{\pgfqpoint{0.641667in}{0.151196in}}{\pgfqpoint{0.641667in}{0.166667in}}%
\pgfpathcurveto{\pgfqpoint{0.641667in}{0.182137in}}{\pgfqpoint{0.635520in}{0.196975in}}{\pgfqpoint{0.624581in}{0.207915in}}%
\pgfpathcurveto{\pgfqpoint{0.613642in}{0.218854in}}{\pgfqpoint{0.598804in}{0.225000in}}{\pgfqpoint{0.583333in}{0.225000in}}%
\pgfpathcurveto{\pgfqpoint{0.567863in}{0.225000in}}{\pgfqpoint{0.553025in}{0.218854in}}{\pgfqpoint{0.542085in}{0.207915in}}%
\pgfpathcurveto{\pgfqpoint{0.531146in}{0.196975in}}{\pgfqpoint{0.525000in}{0.182137in}}{\pgfqpoint{0.525000in}{0.166667in}}%
\pgfpathcurveto{\pgfqpoint{0.525000in}{0.151196in}}{\pgfqpoint{0.531146in}{0.136358in}}{\pgfqpoint{0.542085in}{0.125419in}}%
\pgfpathcurveto{\pgfqpoint{0.553025in}{0.114480in}}{\pgfqpoint{0.567863in}{0.108333in}}{\pgfqpoint{0.583333in}{0.108333in}}%
\pgfpathclose%
\pgfpathmoveto{\pgfqpoint{0.583333in}{0.114167in}}%
\pgfpathcurveto{\pgfqpoint{0.583333in}{0.114167in}}{\pgfqpoint{0.569410in}{0.114167in}}{\pgfqpoint{0.556055in}{0.119698in}}%
\pgfpathcurveto{\pgfqpoint{0.546210in}{0.129544in}}{\pgfqpoint{0.536365in}{0.139389in}}{\pgfqpoint{0.530833in}{0.152744in}}%
\pgfpathcurveto{\pgfqpoint{0.530833in}{0.166667in}}{\pgfqpoint{0.530833in}{0.180590in}}{\pgfqpoint{0.536365in}{0.193945in}}%
\pgfpathcurveto{\pgfqpoint{0.546210in}{0.203790in}}{\pgfqpoint{0.556055in}{0.213635in}}{\pgfqpoint{0.569410in}{0.219167in}}%
\pgfpathcurveto{\pgfqpoint{0.583333in}{0.219167in}}{\pgfqpoint{0.597256in}{0.219167in}}{\pgfqpoint{0.610611in}{0.213635in}}%
\pgfpathcurveto{\pgfqpoint{0.620456in}{0.203790in}}{\pgfqpoint{0.630302in}{0.193945in}}{\pgfqpoint{0.635833in}{0.180590in}}%
\pgfpathcurveto{\pgfqpoint{0.635833in}{0.166667in}}{\pgfqpoint{0.635833in}{0.152744in}}{\pgfqpoint{0.630302in}{0.139389in}}%
\pgfpathcurveto{\pgfqpoint{0.620456in}{0.129544in}}{\pgfqpoint{0.610611in}{0.119698in}}{\pgfqpoint{0.597256in}{0.114167in}}%
\pgfpathclose%
\pgfpathmoveto{\pgfqpoint{0.750000in}{0.108333in}}%
\pgfpathcurveto{\pgfqpoint{0.765470in}{0.108333in}}{\pgfqpoint{0.780309in}{0.114480in}}{\pgfqpoint{0.791248in}{0.125419in}}%
\pgfpathcurveto{\pgfqpoint{0.802187in}{0.136358in}}{\pgfqpoint{0.808333in}{0.151196in}}{\pgfqpoint{0.808333in}{0.166667in}}%
\pgfpathcurveto{\pgfqpoint{0.808333in}{0.182137in}}{\pgfqpoint{0.802187in}{0.196975in}}{\pgfqpoint{0.791248in}{0.207915in}}%
\pgfpathcurveto{\pgfqpoint{0.780309in}{0.218854in}}{\pgfqpoint{0.765470in}{0.225000in}}{\pgfqpoint{0.750000in}{0.225000in}}%
\pgfpathcurveto{\pgfqpoint{0.734530in}{0.225000in}}{\pgfqpoint{0.719691in}{0.218854in}}{\pgfqpoint{0.708752in}{0.207915in}}%
\pgfpathcurveto{\pgfqpoint{0.697813in}{0.196975in}}{\pgfqpoint{0.691667in}{0.182137in}}{\pgfqpoint{0.691667in}{0.166667in}}%
\pgfpathcurveto{\pgfqpoint{0.691667in}{0.151196in}}{\pgfqpoint{0.697813in}{0.136358in}}{\pgfqpoint{0.708752in}{0.125419in}}%
\pgfpathcurveto{\pgfqpoint{0.719691in}{0.114480in}}{\pgfqpoint{0.734530in}{0.108333in}}{\pgfqpoint{0.750000in}{0.108333in}}%
\pgfpathclose%
\pgfpathmoveto{\pgfqpoint{0.750000in}{0.114167in}}%
\pgfpathcurveto{\pgfqpoint{0.750000in}{0.114167in}}{\pgfqpoint{0.736077in}{0.114167in}}{\pgfqpoint{0.722722in}{0.119698in}}%
\pgfpathcurveto{\pgfqpoint{0.712877in}{0.129544in}}{\pgfqpoint{0.703032in}{0.139389in}}{\pgfqpoint{0.697500in}{0.152744in}}%
\pgfpathcurveto{\pgfqpoint{0.697500in}{0.166667in}}{\pgfqpoint{0.697500in}{0.180590in}}{\pgfqpoint{0.703032in}{0.193945in}}%
\pgfpathcurveto{\pgfqpoint{0.712877in}{0.203790in}}{\pgfqpoint{0.722722in}{0.213635in}}{\pgfqpoint{0.736077in}{0.219167in}}%
\pgfpathcurveto{\pgfqpoint{0.750000in}{0.219167in}}{\pgfqpoint{0.763923in}{0.219167in}}{\pgfqpoint{0.777278in}{0.213635in}}%
\pgfpathcurveto{\pgfqpoint{0.787123in}{0.203790in}}{\pgfqpoint{0.796968in}{0.193945in}}{\pgfqpoint{0.802500in}{0.180590in}}%
\pgfpathcurveto{\pgfqpoint{0.802500in}{0.166667in}}{\pgfqpoint{0.802500in}{0.152744in}}{\pgfqpoint{0.796968in}{0.139389in}}%
\pgfpathcurveto{\pgfqpoint{0.787123in}{0.129544in}}{\pgfqpoint{0.777278in}{0.119698in}}{\pgfqpoint{0.763923in}{0.114167in}}%
\pgfpathclose%
\pgfpathmoveto{\pgfqpoint{0.916667in}{0.108333in}}%
\pgfpathcurveto{\pgfqpoint{0.932137in}{0.108333in}}{\pgfqpoint{0.946975in}{0.114480in}}{\pgfqpoint{0.957915in}{0.125419in}}%
\pgfpathcurveto{\pgfqpoint{0.968854in}{0.136358in}}{\pgfqpoint{0.975000in}{0.151196in}}{\pgfqpoint{0.975000in}{0.166667in}}%
\pgfpathcurveto{\pgfqpoint{0.975000in}{0.182137in}}{\pgfqpoint{0.968854in}{0.196975in}}{\pgfqpoint{0.957915in}{0.207915in}}%
\pgfpathcurveto{\pgfqpoint{0.946975in}{0.218854in}}{\pgfqpoint{0.932137in}{0.225000in}}{\pgfqpoint{0.916667in}{0.225000in}}%
\pgfpathcurveto{\pgfqpoint{0.901196in}{0.225000in}}{\pgfqpoint{0.886358in}{0.218854in}}{\pgfqpoint{0.875419in}{0.207915in}}%
\pgfpathcurveto{\pgfqpoint{0.864480in}{0.196975in}}{\pgfqpoint{0.858333in}{0.182137in}}{\pgfqpoint{0.858333in}{0.166667in}}%
\pgfpathcurveto{\pgfqpoint{0.858333in}{0.151196in}}{\pgfqpoint{0.864480in}{0.136358in}}{\pgfqpoint{0.875419in}{0.125419in}}%
\pgfpathcurveto{\pgfqpoint{0.886358in}{0.114480in}}{\pgfqpoint{0.901196in}{0.108333in}}{\pgfqpoint{0.916667in}{0.108333in}}%
\pgfpathclose%
\pgfpathmoveto{\pgfqpoint{0.916667in}{0.114167in}}%
\pgfpathcurveto{\pgfqpoint{0.916667in}{0.114167in}}{\pgfqpoint{0.902744in}{0.114167in}}{\pgfqpoint{0.889389in}{0.119698in}}%
\pgfpathcurveto{\pgfqpoint{0.879544in}{0.129544in}}{\pgfqpoint{0.869698in}{0.139389in}}{\pgfqpoint{0.864167in}{0.152744in}}%
\pgfpathcurveto{\pgfqpoint{0.864167in}{0.166667in}}{\pgfqpoint{0.864167in}{0.180590in}}{\pgfqpoint{0.869698in}{0.193945in}}%
\pgfpathcurveto{\pgfqpoint{0.879544in}{0.203790in}}{\pgfqpoint{0.889389in}{0.213635in}}{\pgfqpoint{0.902744in}{0.219167in}}%
\pgfpathcurveto{\pgfqpoint{0.916667in}{0.219167in}}{\pgfqpoint{0.930590in}{0.219167in}}{\pgfqpoint{0.943945in}{0.213635in}}%
\pgfpathcurveto{\pgfqpoint{0.953790in}{0.203790in}}{\pgfqpoint{0.963635in}{0.193945in}}{\pgfqpoint{0.969167in}{0.180590in}}%
\pgfpathcurveto{\pgfqpoint{0.969167in}{0.166667in}}{\pgfqpoint{0.969167in}{0.152744in}}{\pgfqpoint{0.963635in}{0.139389in}}%
\pgfpathcurveto{\pgfqpoint{0.953790in}{0.129544in}}{\pgfqpoint{0.943945in}{0.119698in}}{\pgfqpoint{0.930590in}{0.114167in}}%
\pgfpathclose%
\pgfpathmoveto{\pgfqpoint{0.000000in}{0.275000in}}%
\pgfpathcurveto{\pgfqpoint{0.015470in}{0.275000in}}{\pgfqpoint{0.030309in}{0.281146in}}{\pgfqpoint{0.041248in}{0.292085in}}%
\pgfpathcurveto{\pgfqpoint{0.052187in}{0.303025in}}{\pgfqpoint{0.058333in}{0.317863in}}{\pgfqpoint{0.058333in}{0.333333in}}%
\pgfpathcurveto{\pgfqpoint{0.058333in}{0.348804in}}{\pgfqpoint{0.052187in}{0.363642in}}{\pgfqpoint{0.041248in}{0.374581in}}%
\pgfpathcurveto{\pgfqpoint{0.030309in}{0.385520in}}{\pgfqpoint{0.015470in}{0.391667in}}{\pgfqpoint{0.000000in}{0.391667in}}%
\pgfpathcurveto{\pgfqpoint{-0.015470in}{0.391667in}}{\pgfqpoint{-0.030309in}{0.385520in}}{\pgfqpoint{-0.041248in}{0.374581in}}%
\pgfpathcurveto{\pgfqpoint{-0.052187in}{0.363642in}}{\pgfqpoint{-0.058333in}{0.348804in}}{\pgfqpoint{-0.058333in}{0.333333in}}%
\pgfpathcurveto{\pgfqpoint{-0.058333in}{0.317863in}}{\pgfqpoint{-0.052187in}{0.303025in}}{\pgfqpoint{-0.041248in}{0.292085in}}%
\pgfpathcurveto{\pgfqpoint{-0.030309in}{0.281146in}}{\pgfqpoint{-0.015470in}{0.275000in}}{\pgfqpoint{0.000000in}{0.275000in}}%
\pgfpathclose%
\pgfpathmoveto{\pgfqpoint{0.000000in}{0.280833in}}%
\pgfpathcurveto{\pgfqpoint{0.000000in}{0.280833in}}{\pgfqpoint{-0.013923in}{0.280833in}}{\pgfqpoint{-0.027278in}{0.286365in}}%
\pgfpathcurveto{\pgfqpoint{-0.037123in}{0.296210in}}{\pgfqpoint{-0.046968in}{0.306055in}}{\pgfqpoint{-0.052500in}{0.319410in}}%
\pgfpathcurveto{\pgfqpoint{-0.052500in}{0.333333in}}{\pgfqpoint{-0.052500in}{0.347256in}}{\pgfqpoint{-0.046968in}{0.360611in}}%
\pgfpathcurveto{\pgfqpoint{-0.037123in}{0.370456in}}{\pgfqpoint{-0.027278in}{0.380302in}}{\pgfqpoint{-0.013923in}{0.385833in}}%
\pgfpathcurveto{\pgfqpoint{0.000000in}{0.385833in}}{\pgfqpoint{0.013923in}{0.385833in}}{\pgfqpoint{0.027278in}{0.380302in}}%
\pgfpathcurveto{\pgfqpoint{0.037123in}{0.370456in}}{\pgfqpoint{0.046968in}{0.360611in}}{\pgfqpoint{0.052500in}{0.347256in}}%
\pgfpathcurveto{\pgfqpoint{0.052500in}{0.333333in}}{\pgfqpoint{0.052500in}{0.319410in}}{\pgfqpoint{0.046968in}{0.306055in}}%
\pgfpathcurveto{\pgfqpoint{0.037123in}{0.296210in}}{\pgfqpoint{0.027278in}{0.286365in}}{\pgfqpoint{0.013923in}{0.280833in}}%
\pgfpathclose%
\pgfpathmoveto{\pgfqpoint{0.166667in}{0.275000in}}%
\pgfpathcurveto{\pgfqpoint{0.182137in}{0.275000in}}{\pgfqpoint{0.196975in}{0.281146in}}{\pgfqpoint{0.207915in}{0.292085in}}%
\pgfpathcurveto{\pgfqpoint{0.218854in}{0.303025in}}{\pgfqpoint{0.225000in}{0.317863in}}{\pgfqpoint{0.225000in}{0.333333in}}%
\pgfpathcurveto{\pgfqpoint{0.225000in}{0.348804in}}{\pgfqpoint{0.218854in}{0.363642in}}{\pgfqpoint{0.207915in}{0.374581in}}%
\pgfpathcurveto{\pgfqpoint{0.196975in}{0.385520in}}{\pgfqpoint{0.182137in}{0.391667in}}{\pgfqpoint{0.166667in}{0.391667in}}%
\pgfpathcurveto{\pgfqpoint{0.151196in}{0.391667in}}{\pgfqpoint{0.136358in}{0.385520in}}{\pgfqpoint{0.125419in}{0.374581in}}%
\pgfpathcurveto{\pgfqpoint{0.114480in}{0.363642in}}{\pgfqpoint{0.108333in}{0.348804in}}{\pgfqpoint{0.108333in}{0.333333in}}%
\pgfpathcurveto{\pgfqpoint{0.108333in}{0.317863in}}{\pgfqpoint{0.114480in}{0.303025in}}{\pgfqpoint{0.125419in}{0.292085in}}%
\pgfpathcurveto{\pgfqpoint{0.136358in}{0.281146in}}{\pgfqpoint{0.151196in}{0.275000in}}{\pgfqpoint{0.166667in}{0.275000in}}%
\pgfpathclose%
\pgfpathmoveto{\pgfqpoint{0.166667in}{0.280833in}}%
\pgfpathcurveto{\pgfqpoint{0.166667in}{0.280833in}}{\pgfqpoint{0.152744in}{0.280833in}}{\pgfqpoint{0.139389in}{0.286365in}}%
\pgfpathcurveto{\pgfqpoint{0.129544in}{0.296210in}}{\pgfqpoint{0.119698in}{0.306055in}}{\pgfqpoint{0.114167in}{0.319410in}}%
\pgfpathcurveto{\pgfqpoint{0.114167in}{0.333333in}}{\pgfqpoint{0.114167in}{0.347256in}}{\pgfqpoint{0.119698in}{0.360611in}}%
\pgfpathcurveto{\pgfqpoint{0.129544in}{0.370456in}}{\pgfqpoint{0.139389in}{0.380302in}}{\pgfqpoint{0.152744in}{0.385833in}}%
\pgfpathcurveto{\pgfqpoint{0.166667in}{0.385833in}}{\pgfqpoint{0.180590in}{0.385833in}}{\pgfqpoint{0.193945in}{0.380302in}}%
\pgfpathcurveto{\pgfqpoint{0.203790in}{0.370456in}}{\pgfqpoint{0.213635in}{0.360611in}}{\pgfqpoint{0.219167in}{0.347256in}}%
\pgfpathcurveto{\pgfqpoint{0.219167in}{0.333333in}}{\pgfqpoint{0.219167in}{0.319410in}}{\pgfqpoint{0.213635in}{0.306055in}}%
\pgfpathcurveto{\pgfqpoint{0.203790in}{0.296210in}}{\pgfqpoint{0.193945in}{0.286365in}}{\pgfqpoint{0.180590in}{0.280833in}}%
\pgfpathclose%
\pgfpathmoveto{\pgfqpoint{0.333333in}{0.275000in}}%
\pgfpathcurveto{\pgfqpoint{0.348804in}{0.275000in}}{\pgfqpoint{0.363642in}{0.281146in}}{\pgfqpoint{0.374581in}{0.292085in}}%
\pgfpathcurveto{\pgfqpoint{0.385520in}{0.303025in}}{\pgfqpoint{0.391667in}{0.317863in}}{\pgfqpoint{0.391667in}{0.333333in}}%
\pgfpathcurveto{\pgfqpoint{0.391667in}{0.348804in}}{\pgfqpoint{0.385520in}{0.363642in}}{\pgfqpoint{0.374581in}{0.374581in}}%
\pgfpathcurveto{\pgfqpoint{0.363642in}{0.385520in}}{\pgfqpoint{0.348804in}{0.391667in}}{\pgfqpoint{0.333333in}{0.391667in}}%
\pgfpathcurveto{\pgfqpoint{0.317863in}{0.391667in}}{\pgfqpoint{0.303025in}{0.385520in}}{\pgfqpoint{0.292085in}{0.374581in}}%
\pgfpathcurveto{\pgfqpoint{0.281146in}{0.363642in}}{\pgfqpoint{0.275000in}{0.348804in}}{\pgfqpoint{0.275000in}{0.333333in}}%
\pgfpathcurveto{\pgfqpoint{0.275000in}{0.317863in}}{\pgfqpoint{0.281146in}{0.303025in}}{\pgfqpoint{0.292085in}{0.292085in}}%
\pgfpathcurveto{\pgfqpoint{0.303025in}{0.281146in}}{\pgfqpoint{0.317863in}{0.275000in}}{\pgfqpoint{0.333333in}{0.275000in}}%
\pgfpathclose%
\pgfpathmoveto{\pgfqpoint{0.333333in}{0.280833in}}%
\pgfpathcurveto{\pgfqpoint{0.333333in}{0.280833in}}{\pgfqpoint{0.319410in}{0.280833in}}{\pgfqpoint{0.306055in}{0.286365in}}%
\pgfpathcurveto{\pgfqpoint{0.296210in}{0.296210in}}{\pgfqpoint{0.286365in}{0.306055in}}{\pgfqpoint{0.280833in}{0.319410in}}%
\pgfpathcurveto{\pgfqpoint{0.280833in}{0.333333in}}{\pgfqpoint{0.280833in}{0.347256in}}{\pgfqpoint{0.286365in}{0.360611in}}%
\pgfpathcurveto{\pgfqpoint{0.296210in}{0.370456in}}{\pgfqpoint{0.306055in}{0.380302in}}{\pgfqpoint{0.319410in}{0.385833in}}%
\pgfpathcurveto{\pgfqpoint{0.333333in}{0.385833in}}{\pgfqpoint{0.347256in}{0.385833in}}{\pgfqpoint{0.360611in}{0.380302in}}%
\pgfpathcurveto{\pgfqpoint{0.370456in}{0.370456in}}{\pgfqpoint{0.380302in}{0.360611in}}{\pgfqpoint{0.385833in}{0.347256in}}%
\pgfpathcurveto{\pgfqpoint{0.385833in}{0.333333in}}{\pgfqpoint{0.385833in}{0.319410in}}{\pgfqpoint{0.380302in}{0.306055in}}%
\pgfpathcurveto{\pgfqpoint{0.370456in}{0.296210in}}{\pgfqpoint{0.360611in}{0.286365in}}{\pgfqpoint{0.347256in}{0.280833in}}%
\pgfpathclose%
\pgfpathmoveto{\pgfqpoint{0.500000in}{0.275000in}}%
\pgfpathcurveto{\pgfqpoint{0.515470in}{0.275000in}}{\pgfqpoint{0.530309in}{0.281146in}}{\pgfqpoint{0.541248in}{0.292085in}}%
\pgfpathcurveto{\pgfqpoint{0.552187in}{0.303025in}}{\pgfqpoint{0.558333in}{0.317863in}}{\pgfqpoint{0.558333in}{0.333333in}}%
\pgfpathcurveto{\pgfqpoint{0.558333in}{0.348804in}}{\pgfqpoint{0.552187in}{0.363642in}}{\pgfqpoint{0.541248in}{0.374581in}}%
\pgfpathcurveto{\pgfqpoint{0.530309in}{0.385520in}}{\pgfqpoint{0.515470in}{0.391667in}}{\pgfqpoint{0.500000in}{0.391667in}}%
\pgfpathcurveto{\pgfqpoint{0.484530in}{0.391667in}}{\pgfqpoint{0.469691in}{0.385520in}}{\pgfqpoint{0.458752in}{0.374581in}}%
\pgfpathcurveto{\pgfqpoint{0.447813in}{0.363642in}}{\pgfqpoint{0.441667in}{0.348804in}}{\pgfqpoint{0.441667in}{0.333333in}}%
\pgfpathcurveto{\pgfqpoint{0.441667in}{0.317863in}}{\pgfqpoint{0.447813in}{0.303025in}}{\pgfqpoint{0.458752in}{0.292085in}}%
\pgfpathcurveto{\pgfqpoint{0.469691in}{0.281146in}}{\pgfqpoint{0.484530in}{0.275000in}}{\pgfqpoint{0.500000in}{0.275000in}}%
\pgfpathclose%
\pgfpathmoveto{\pgfqpoint{0.500000in}{0.280833in}}%
\pgfpathcurveto{\pgfqpoint{0.500000in}{0.280833in}}{\pgfqpoint{0.486077in}{0.280833in}}{\pgfqpoint{0.472722in}{0.286365in}}%
\pgfpathcurveto{\pgfqpoint{0.462877in}{0.296210in}}{\pgfqpoint{0.453032in}{0.306055in}}{\pgfqpoint{0.447500in}{0.319410in}}%
\pgfpathcurveto{\pgfqpoint{0.447500in}{0.333333in}}{\pgfqpoint{0.447500in}{0.347256in}}{\pgfqpoint{0.453032in}{0.360611in}}%
\pgfpathcurveto{\pgfqpoint{0.462877in}{0.370456in}}{\pgfqpoint{0.472722in}{0.380302in}}{\pgfqpoint{0.486077in}{0.385833in}}%
\pgfpathcurveto{\pgfqpoint{0.500000in}{0.385833in}}{\pgfqpoint{0.513923in}{0.385833in}}{\pgfqpoint{0.527278in}{0.380302in}}%
\pgfpathcurveto{\pgfqpoint{0.537123in}{0.370456in}}{\pgfqpoint{0.546968in}{0.360611in}}{\pgfqpoint{0.552500in}{0.347256in}}%
\pgfpathcurveto{\pgfqpoint{0.552500in}{0.333333in}}{\pgfqpoint{0.552500in}{0.319410in}}{\pgfqpoint{0.546968in}{0.306055in}}%
\pgfpathcurveto{\pgfqpoint{0.537123in}{0.296210in}}{\pgfqpoint{0.527278in}{0.286365in}}{\pgfqpoint{0.513923in}{0.280833in}}%
\pgfpathclose%
\pgfpathmoveto{\pgfqpoint{0.666667in}{0.275000in}}%
\pgfpathcurveto{\pgfqpoint{0.682137in}{0.275000in}}{\pgfqpoint{0.696975in}{0.281146in}}{\pgfqpoint{0.707915in}{0.292085in}}%
\pgfpathcurveto{\pgfqpoint{0.718854in}{0.303025in}}{\pgfqpoint{0.725000in}{0.317863in}}{\pgfqpoint{0.725000in}{0.333333in}}%
\pgfpathcurveto{\pgfqpoint{0.725000in}{0.348804in}}{\pgfqpoint{0.718854in}{0.363642in}}{\pgfqpoint{0.707915in}{0.374581in}}%
\pgfpathcurveto{\pgfqpoint{0.696975in}{0.385520in}}{\pgfqpoint{0.682137in}{0.391667in}}{\pgfqpoint{0.666667in}{0.391667in}}%
\pgfpathcurveto{\pgfqpoint{0.651196in}{0.391667in}}{\pgfqpoint{0.636358in}{0.385520in}}{\pgfqpoint{0.625419in}{0.374581in}}%
\pgfpathcurveto{\pgfqpoint{0.614480in}{0.363642in}}{\pgfqpoint{0.608333in}{0.348804in}}{\pgfqpoint{0.608333in}{0.333333in}}%
\pgfpathcurveto{\pgfqpoint{0.608333in}{0.317863in}}{\pgfqpoint{0.614480in}{0.303025in}}{\pgfqpoint{0.625419in}{0.292085in}}%
\pgfpathcurveto{\pgfqpoint{0.636358in}{0.281146in}}{\pgfqpoint{0.651196in}{0.275000in}}{\pgfqpoint{0.666667in}{0.275000in}}%
\pgfpathclose%
\pgfpathmoveto{\pgfqpoint{0.666667in}{0.280833in}}%
\pgfpathcurveto{\pgfqpoint{0.666667in}{0.280833in}}{\pgfqpoint{0.652744in}{0.280833in}}{\pgfqpoint{0.639389in}{0.286365in}}%
\pgfpathcurveto{\pgfqpoint{0.629544in}{0.296210in}}{\pgfqpoint{0.619698in}{0.306055in}}{\pgfqpoint{0.614167in}{0.319410in}}%
\pgfpathcurveto{\pgfqpoint{0.614167in}{0.333333in}}{\pgfqpoint{0.614167in}{0.347256in}}{\pgfqpoint{0.619698in}{0.360611in}}%
\pgfpathcurveto{\pgfqpoint{0.629544in}{0.370456in}}{\pgfqpoint{0.639389in}{0.380302in}}{\pgfqpoint{0.652744in}{0.385833in}}%
\pgfpathcurveto{\pgfqpoint{0.666667in}{0.385833in}}{\pgfqpoint{0.680590in}{0.385833in}}{\pgfqpoint{0.693945in}{0.380302in}}%
\pgfpathcurveto{\pgfqpoint{0.703790in}{0.370456in}}{\pgfqpoint{0.713635in}{0.360611in}}{\pgfqpoint{0.719167in}{0.347256in}}%
\pgfpathcurveto{\pgfqpoint{0.719167in}{0.333333in}}{\pgfqpoint{0.719167in}{0.319410in}}{\pgfqpoint{0.713635in}{0.306055in}}%
\pgfpathcurveto{\pgfqpoint{0.703790in}{0.296210in}}{\pgfqpoint{0.693945in}{0.286365in}}{\pgfqpoint{0.680590in}{0.280833in}}%
\pgfpathclose%
\pgfpathmoveto{\pgfqpoint{0.833333in}{0.275000in}}%
\pgfpathcurveto{\pgfqpoint{0.848804in}{0.275000in}}{\pgfqpoint{0.863642in}{0.281146in}}{\pgfqpoint{0.874581in}{0.292085in}}%
\pgfpathcurveto{\pgfqpoint{0.885520in}{0.303025in}}{\pgfqpoint{0.891667in}{0.317863in}}{\pgfqpoint{0.891667in}{0.333333in}}%
\pgfpathcurveto{\pgfqpoint{0.891667in}{0.348804in}}{\pgfqpoint{0.885520in}{0.363642in}}{\pgfqpoint{0.874581in}{0.374581in}}%
\pgfpathcurveto{\pgfqpoint{0.863642in}{0.385520in}}{\pgfqpoint{0.848804in}{0.391667in}}{\pgfqpoint{0.833333in}{0.391667in}}%
\pgfpathcurveto{\pgfqpoint{0.817863in}{0.391667in}}{\pgfqpoint{0.803025in}{0.385520in}}{\pgfqpoint{0.792085in}{0.374581in}}%
\pgfpathcurveto{\pgfqpoint{0.781146in}{0.363642in}}{\pgfqpoint{0.775000in}{0.348804in}}{\pgfqpoint{0.775000in}{0.333333in}}%
\pgfpathcurveto{\pgfqpoint{0.775000in}{0.317863in}}{\pgfqpoint{0.781146in}{0.303025in}}{\pgfqpoint{0.792085in}{0.292085in}}%
\pgfpathcurveto{\pgfqpoint{0.803025in}{0.281146in}}{\pgfqpoint{0.817863in}{0.275000in}}{\pgfqpoint{0.833333in}{0.275000in}}%
\pgfpathclose%
\pgfpathmoveto{\pgfqpoint{0.833333in}{0.280833in}}%
\pgfpathcurveto{\pgfqpoint{0.833333in}{0.280833in}}{\pgfqpoint{0.819410in}{0.280833in}}{\pgfqpoint{0.806055in}{0.286365in}}%
\pgfpathcurveto{\pgfqpoint{0.796210in}{0.296210in}}{\pgfqpoint{0.786365in}{0.306055in}}{\pgfqpoint{0.780833in}{0.319410in}}%
\pgfpathcurveto{\pgfqpoint{0.780833in}{0.333333in}}{\pgfqpoint{0.780833in}{0.347256in}}{\pgfqpoint{0.786365in}{0.360611in}}%
\pgfpathcurveto{\pgfqpoint{0.796210in}{0.370456in}}{\pgfqpoint{0.806055in}{0.380302in}}{\pgfqpoint{0.819410in}{0.385833in}}%
\pgfpathcurveto{\pgfqpoint{0.833333in}{0.385833in}}{\pgfqpoint{0.847256in}{0.385833in}}{\pgfqpoint{0.860611in}{0.380302in}}%
\pgfpathcurveto{\pgfqpoint{0.870456in}{0.370456in}}{\pgfqpoint{0.880302in}{0.360611in}}{\pgfqpoint{0.885833in}{0.347256in}}%
\pgfpathcurveto{\pgfqpoint{0.885833in}{0.333333in}}{\pgfqpoint{0.885833in}{0.319410in}}{\pgfqpoint{0.880302in}{0.306055in}}%
\pgfpathcurveto{\pgfqpoint{0.870456in}{0.296210in}}{\pgfqpoint{0.860611in}{0.286365in}}{\pgfqpoint{0.847256in}{0.280833in}}%
\pgfpathclose%
\pgfpathmoveto{\pgfqpoint{1.000000in}{0.275000in}}%
\pgfpathcurveto{\pgfqpoint{1.015470in}{0.275000in}}{\pgfqpoint{1.030309in}{0.281146in}}{\pgfqpoint{1.041248in}{0.292085in}}%
\pgfpathcurveto{\pgfqpoint{1.052187in}{0.303025in}}{\pgfqpoint{1.058333in}{0.317863in}}{\pgfqpoint{1.058333in}{0.333333in}}%
\pgfpathcurveto{\pgfqpoint{1.058333in}{0.348804in}}{\pgfqpoint{1.052187in}{0.363642in}}{\pgfqpoint{1.041248in}{0.374581in}}%
\pgfpathcurveto{\pgfqpoint{1.030309in}{0.385520in}}{\pgfqpoint{1.015470in}{0.391667in}}{\pgfqpoint{1.000000in}{0.391667in}}%
\pgfpathcurveto{\pgfqpoint{0.984530in}{0.391667in}}{\pgfqpoint{0.969691in}{0.385520in}}{\pgfqpoint{0.958752in}{0.374581in}}%
\pgfpathcurveto{\pgfqpoint{0.947813in}{0.363642in}}{\pgfqpoint{0.941667in}{0.348804in}}{\pgfqpoint{0.941667in}{0.333333in}}%
\pgfpathcurveto{\pgfqpoint{0.941667in}{0.317863in}}{\pgfqpoint{0.947813in}{0.303025in}}{\pgfqpoint{0.958752in}{0.292085in}}%
\pgfpathcurveto{\pgfqpoint{0.969691in}{0.281146in}}{\pgfqpoint{0.984530in}{0.275000in}}{\pgfqpoint{1.000000in}{0.275000in}}%
\pgfpathclose%
\pgfpathmoveto{\pgfqpoint{1.000000in}{0.280833in}}%
\pgfpathcurveto{\pgfqpoint{1.000000in}{0.280833in}}{\pgfqpoint{0.986077in}{0.280833in}}{\pgfqpoint{0.972722in}{0.286365in}}%
\pgfpathcurveto{\pgfqpoint{0.962877in}{0.296210in}}{\pgfqpoint{0.953032in}{0.306055in}}{\pgfqpoint{0.947500in}{0.319410in}}%
\pgfpathcurveto{\pgfqpoint{0.947500in}{0.333333in}}{\pgfqpoint{0.947500in}{0.347256in}}{\pgfqpoint{0.953032in}{0.360611in}}%
\pgfpathcurveto{\pgfqpoint{0.962877in}{0.370456in}}{\pgfqpoint{0.972722in}{0.380302in}}{\pgfqpoint{0.986077in}{0.385833in}}%
\pgfpathcurveto{\pgfqpoint{1.000000in}{0.385833in}}{\pgfqpoint{1.013923in}{0.385833in}}{\pgfqpoint{1.027278in}{0.380302in}}%
\pgfpathcurveto{\pgfqpoint{1.037123in}{0.370456in}}{\pgfqpoint{1.046968in}{0.360611in}}{\pgfqpoint{1.052500in}{0.347256in}}%
\pgfpathcurveto{\pgfqpoint{1.052500in}{0.333333in}}{\pgfqpoint{1.052500in}{0.319410in}}{\pgfqpoint{1.046968in}{0.306055in}}%
\pgfpathcurveto{\pgfqpoint{1.037123in}{0.296210in}}{\pgfqpoint{1.027278in}{0.286365in}}{\pgfqpoint{1.013923in}{0.280833in}}%
\pgfpathclose%
\pgfpathmoveto{\pgfqpoint{0.083333in}{0.441667in}}%
\pgfpathcurveto{\pgfqpoint{0.098804in}{0.441667in}}{\pgfqpoint{0.113642in}{0.447813in}}{\pgfqpoint{0.124581in}{0.458752in}}%
\pgfpathcurveto{\pgfqpoint{0.135520in}{0.469691in}}{\pgfqpoint{0.141667in}{0.484530in}}{\pgfqpoint{0.141667in}{0.500000in}}%
\pgfpathcurveto{\pgfqpoint{0.141667in}{0.515470in}}{\pgfqpoint{0.135520in}{0.530309in}}{\pgfqpoint{0.124581in}{0.541248in}}%
\pgfpathcurveto{\pgfqpoint{0.113642in}{0.552187in}}{\pgfqpoint{0.098804in}{0.558333in}}{\pgfqpoint{0.083333in}{0.558333in}}%
\pgfpathcurveto{\pgfqpoint{0.067863in}{0.558333in}}{\pgfqpoint{0.053025in}{0.552187in}}{\pgfqpoint{0.042085in}{0.541248in}}%
\pgfpathcurveto{\pgfqpoint{0.031146in}{0.530309in}}{\pgfqpoint{0.025000in}{0.515470in}}{\pgfqpoint{0.025000in}{0.500000in}}%
\pgfpathcurveto{\pgfqpoint{0.025000in}{0.484530in}}{\pgfqpoint{0.031146in}{0.469691in}}{\pgfqpoint{0.042085in}{0.458752in}}%
\pgfpathcurveto{\pgfqpoint{0.053025in}{0.447813in}}{\pgfqpoint{0.067863in}{0.441667in}}{\pgfqpoint{0.083333in}{0.441667in}}%
\pgfpathclose%
\pgfpathmoveto{\pgfqpoint{0.083333in}{0.447500in}}%
\pgfpathcurveto{\pgfqpoint{0.083333in}{0.447500in}}{\pgfqpoint{0.069410in}{0.447500in}}{\pgfqpoint{0.056055in}{0.453032in}}%
\pgfpathcurveto{\pgfqpoint{0.046210in}{0.462877in}}{\pgfqpoint{0.036365in}{0.472722in}}{\pgfqpoint{0.030833in}{0.486077in}}%
\pgfpathcurveto{\pgfqpoint{0.030833in}{0.500000in}}{\pgfqpoint{0.030833in}{0.513923in}}{\pgfqpoint{0.036365in}{0.527278in}}%
\pgfpathcurveto{\pgfqpoint{0.046210in}{0.537123in}}{\pgfqpoint{0.056055in}{0.546968in}}{\pgfqpoint{0.069410in}{0.552500in}}%
\pgfpathcurveto{\pgfqpoint{0.083333in}{0.552500in}}{\pgfqpoint{0.097256in}{0.552500in}}{\pgfqpoint{0.110611in}{0.546968in}}%
\pgfpathcurveto{\pgfqpoint{0.120456in}{0.537123in}}{\pgfqpoint{0.130302in}{0.527278in}}{\pgfqpoint{0.135833in}{0.513923in}}%
\pgfpathcurveto{\pgfqpoint{0.135833in}{0.500000in}}{\pgfqpoint{0.135833in}{0.486077in}}{\pgfqpoint{0.130302in}{0.472722in}}%
\pgfpathcurveto{\pgfqpoint{0.120456in}{0.462877in}}{\pgfqpoint{0.110611in}{0.453032in}}{\pgfqpoint{0.097256in}{0.447500in}}%
\pgfpathclose%
\pgfpathmoveto{\pgfqpoint{0.250000in}{0.441667in}}%
\pgfpathcurveto{\pgfqpoint{0.265470in}{0.441667in}}{\pgfqpoint{0.280309in}{0.447813in}}{\pgfqpoint{0.291248in}{0.458752in}}%
\pgfpathcurveto{\pgfqpoint{0.302187in}{0.469691in}}{\pgfqpoint{0.308333in}{0.484530in}}{\pgfqpoint{0.308333in}{0.500000in}}%
\pgfpathcurveto{\pgfqpoint{0.308333in}{0.515470in}}{\pgfqpoint{0.302187in}{0.530309in}}{\pgfqpoint{0.291248in}{0.541248in}}%
\pgfpathcurveto{\pgfqpoint{0.280309in}{0.552187in}}{\pgfqpoint{0.265470in}{0.558333in}}{\pgfqpoint{0.250000in}{0.558333in}}%
\pgfpathcurveto{\pgfqpoint{0.234530in}{0.558333in}}{\pgfqpoint{0.219691in}{0.552187in}}{\pgfqpoint{0.208752in}{0.541248in}}%
\pgfpathcurveto{\pgfqpoint{0.197813in}{0.530309in}}{\pgfqpoint{0.191667in}{0.515470in}}{\pgfqpoint{0.191667in}{0.500000in}}%
\pgfpathcurveto{\pgfqpoint{0.191667in}{0.484530in}}{\pgfqpoint{0.197813in}{0.469691in}}{\pgfqpoint{0.208752in}{0.458752in}}%
\pgfpathcurveto{\pgfqpoint{0.219691in}{0.447813in}}{\pgfqpoint{0.234530in}{0.441667in}}{\pgfqpoint{0.250000in}{0.441667in}}%
\pgfpathclose%
\pgfpathmoveto{\pgfqpoint{0.250000in}{0.447500in}}%
\pgfpathcurveto{\pgfqpoint{0.250000in}{0.447500in}}{\pgfqpoint{0.236077in}{0.447500in}}{\pgfqpoint{0.222722in}{0.453032in}}%
\pgfpathcurveto{\pgfqpoint{0.212877in}{0.462877in}}{\pgfqpoint{0.203032in}{0.472722in}}{\pgfqpoint{0.197500in}{0.486077in}}%
\pgfpathcurveto{\pgfqpoint{0.197500in}{0.500000in}}{\pgfqpoint{0.197500in}{0.513923in}}{\pgfqpoint{0.203032in}{0.527278in}}%
\pgfpathcurveto{\pgfqpoint{0.212877in}{0.537123in}}{\pgfqpoint{0.222722in}{0.546968in}}{\pgfqpoint{0.236077in}{0.552500in}}%
\pgfpathcurveto{\pgfqpoint{0.250000in}{0.552500in}}{\pgfqpoint{0.263923in}{0.552500in}}{\pgfqpoint{0.277278in}{0.546968in}}%
\pgfpathcurveto{\pgfqpoint{0.287123in}{0.537123in}}{\pgfqpoint{0.296968in}{0.527278in}}{\pgfqpoint{0.302500in}{0.513923in}}%
\pgfpathcurveto{\pgfqpoint{0.302500in}{0.500000in}}{\pgfqpoint{0.302500in}{0.486077in}}{\pgfqpoint{0.296968in}{0.472722in}}%
\pgfpathcurveto{\pgfqpoint{0.287123in}{0.462877in}}{\pgfqpoint{0.277278in}{0.453032in}}{\pgfqpoint{0.263923in}{0.447500in}}%
\pgfpathclose%
\pgfpathmoveto{\pgfqpoint{0.416667in}{0.441667in}}%
\pgfpathcurveto{\pgfqpoint{0.432137in}{0.441667in}}{\pgfqpoint{0.446975in}{0.447813in}}{\pgfqpoint{0.457915in}{0.458752in}}%
\pgfpathcurveto{\pgfqpoint{0.468854in}{0.469691in}}{\pgfqpoint{0.475000in}{0.484530in}}{\pgfqpoint{0.475000in}{0.500000in}}%
\pgfpathcurveto{\pgfqpoint{0.475000in}{0.515470in}}{\pgfqpoint{0.468854in}{0.530309in}}{\pgfqpoint{0.457915in}{0.541248in}}%
\pgfpathcurveto{\pgfqpoint{0.446975in}{0.552187in}}{\pgfqpoint{0.432137in}{0.558333in}}{\pgfqpoint{0.416667in}{0.558333in}}%
\pgfpathcurveto{\pgfqpoint{0.401196in}{0.558333in}}{\pgfqpoint{0.386358in}{0.552187in}}{\pgfqpoint{0.375419in}{0.541248in}}%
\pgfpathcurveto{\pgfqpoint{0.364480in}{0.530309in}}{\pgfqpoint{0.358333in}{0.515470in}}{\pgfqpoint{0.358333in}{0.500000in}}%
\pgfpathcurveto{\pgfqpoint{0.358333in}{0.484530in}}{\pgfqpoint{0.364480in}{0.469691in}}{\pgfqpoint{0.375419in}{0.458752in}}%
\pgfpathcurveto{\pgfqpoint{0.386358in}{0.447813in}}{\pgfqpoint{0.401196in}{0.441667in}}{\pgfqpoint{0.416667in}{0.441667in}}%
\pgfpathclose%
\pgfpathmoveto{\pgfqpoint{0.416667in}{0.447500in}}%
\pgfpathcurveto{\pgfqpoint{0.416667in}{0.447500in}}{\pgfqpoint{0.402744in}{0.447500in}}{\pgfqpoint{0.389389in}{0.453032in}}%
\pgfpathcurveto{\pgfqpoint{0.379544in}{0.462877in}}{\pgfqpoint{0.369698in}{0.472722in}}{\pgfqpoint{0.364167in}{0.486077in}}%
\pgfpathcurveto{\pgfqpoint{0.364167in}{0.500000in}}{\pgfqpoint{0.364167in}{0.513923in}}{\pgfqpoint{0.369698in}{0.527278in}}%
\pgfpathcurveto{\pgfqpoint{0.379544in}{0.537123in}}{\pgfqpoint{0.389389in}{0.546968in}}{\pgfqpoint{0.402744in}{0.552500in}}%
\pgfpathcurveto{\pgfqpoint{0.416667in}{0.552500in}}{\pgfqpoint{0.430590in}{0.552500in}}{\pgfqpoint{0.443945in}{0.546968in}}%
\pgfpathcurveto{\pgfqpoint{0.453790in}{0.537123in}}{\pgfqpoint{0.463635in}{0.527278in}}{\pgfqpoint{0.469167in}{0.513923in}}%
\pgfpathcurveto{\pgfqpoint{0.469167in}{0.500000in}}{\pgfqpoint{0.469167in}{0.486077in}}{\pgfqpoint{0.463635in}{0.472722in}}%
\pgfpathcurveto{\pgfqpoint{0.453790in}{0.462877in}}{\pgfqpoint{0.443945in}{0.453032in}}{\pgfqpoint{0.430590in}{0.447500in}}%
\pgfpathclose%
\pgfpathmoveto{\pgfqpoint{0.583333in}{0.441667in}}%
\pgfpathcurveto{\pgfqpoint{0.598804in}{0.441667in}}{\pgfqpoint{0.613642in}{0.447813in}}{\pgfqpoint{0.624581in}{0.458752in}}%
\pgfpathcurveto{\pgfqpoint{0.635520in}{0.469691in}}{\pgfqpoint{0.641667in}{0.484530in}}{\pgfqpoint{0.641667in}{0.500000in}}%
\pgfpathcurveto{\pgfqpoint{0.641667in}{0.515470in}}{\pgfqpoint{0.635520in}{0.530309in}}{\pgfqpoint{0.624581in}{0.541248in}}%
\pgfpathcurveto{\pgfqpoint{0.613642in}{0.552187in}}{\pgfqpoint{0.598804in}{0.558333in}}{\pgfqpoint{0.583333in}{0.558333in}}%
\pgfpathcurveto{\pgfqpoint{0.567863in}{0.558333in}}{\pgfqpoint{0.553025in}{0.552187in}}{\pgfqpoint{0.542085in}{0.541248in}}%
\pgfpathcurveto{\pgfqpoint{0.531146in}{0.530309in}}{\pgfqpoint{0.525000in}{0.515470in}}{\pgfqpoint{0.525000in}{0.500000in}}%
\pgfpathcurveto{\pgfqpoint{0.525000in}{0.484530in}}{\pgfqpoint{0.531146in}{0.469691in}}{\pgfqpoint{0.542085in}{0.458752in}}%
\pgfpathcurveto{\pgfqpoint{0.553025in}{0.447813in}}{\pgfqpoint{0.567863in}{0.441667in}}{\pgfqpoint{0.583333in}{0.441667in}}%
\pgfpathclose%
\pgfpathmoveto{\pgfqpoint{0.583333in}{0.447500in}}%
\pgfpathcurveto{\pgfqpoint{0.583333in}{0.447500in}}{\pgfqpoint{0.569410in}{0.447500in}}{\pgfqpoint{0.556055in}{0.453032in}}%
\pgfpathcurveto{\pgfqpoint{0.546210in}{0.462877in}}{\pgfqpoint{0.536365in}{0.472722in}}{\pgfqpoint{0.530833in}{0.486077in}}%
\pgfpathcurveto{\pgfqpoint{0.530833in}{0.500000in}}{\pgfqpoint{0.530833in}{0.513923in}}{\pgfqpoint{0.536365in}{0.527278in}}%
\pgfpathcurveto{\pgfqpoint{0.546210in}{0.537123in}}{\pgfqpoint{0.556055in}{0.546968in}}{\pgfqpoint{0.569410in}{0.552500in}}%
\pgfpathcurveto{\pgfqpoint{0.583333in}{0.552500in}}{\pgfqpoint{0.597256in}{0.552500in}}{\pgfqpoint{0.610611in}{0.546968in}}%
\pgfpathcurveto{\pgfqpoint{0.620456in}{0.537123in}}{\pgfqpoint{0.630302in}{0.527278in}}{\pgfqpoint{0.635833in}{0.513923in}}%
\pgfpathcurveto{\pgfqpoint{0.635833in}{0.500000in}}{\pgfqpoint{0.635833in}{0.486077in}}{\pgfqpoint{0.630302in}{0.472722in}}%
\pgfpathcurveto{\pgfqpoint{0.620456in}{0.462877in}}{\pgfqpoint{0.610611in}{0.453032in}}{\pgfqpoint{0.597256in}{0.447500in}}%
\pgfpathclose%
\pgfpathmoveto{\pgfqpoint{0.750000in}{0.441667in}}%
\pgfpathcurveto{\pgfqpoint{0.765470in}{0.441667in}}{\pgfqpoint{0.780309in}{0.447813in}}{\pgfqpoint{0.791248in}{0.458752in}}%
\pgfpathcurveto{\pgfqpoint{0.802187in}{0.469691in}}{\pgfqpoint{0.808333in}{0.484530in}}{\pgfqpoint{0.808333in}{0.500000in}}%
\pgfpathcurveto{\pgfqpoint{0.808333in}{0.515470in}}{\pgfqpoint{0.802187in}{0.530309in}}{\pgfqpoint{0.791248in}{0.541248in}}%
\pgfpathcurveto{\pgfqpoint{0.780309in}{0.552187in}}{\pgfqpoint{0.765470in}{0.558333in}}{\pgfqpoint{0.750000in}{0.558333in}}%
\pgfpathcurveto{\pgfqpoint{0.734530in}{0.558333in}}{\pgfqpoint{0.719691in}{0.552187in}}{\pgfqpoint{0.708752in}{0.541248in}}%
\pgfpathcurveto{\pgfqpoint{0.697813in}{0.530309in}}{\pgfqpoint{0.691667in}{0.515470in}}{\pgfqpoint{0.691667in}{0.500000in}}%
\pgfpathcurveto{\pgfqpoint{0.691667in}{0.484530in}}{\pgfqpoint{0.697813in}{0.469691in}}{\pgfqpoint{0.708752in}{0.458752in}}%
\pgfpathcurveto{\pgfqpoint{0.719691in}{0.447813in}}{\pgfqpoint{0.734530in}{0.441667in}}{\pgfqpoint{0.750000in}{0.441667in}}%
\pgfpathclose%
\pgfpathmoveto{\pgfqpoint{0.750000in}{0.447500in}}%
\pgfpathcurveto{\pgfqpoint{0.750000in}{0.447500in}}{\pgfqpoint{0.736077in}{0.447500in}}{\pgfqpoint{0.722722in}{0.453032in}}%
\pgfpathcurveto{\pgfqpoint{0.712877in}{0.462877in}}{\pgfqpoint{0.703032in}{0.472722in}}{\pgfqpoint{0.697500in}{0.486077in}}%
\pgfpathcurveto{\pgfqpoint{0.697500in}{0.500000in}}{\pgfqpoint{0.697500in}{0.513923in}}{\pgfqpoint{0.703032in}{0.527278in}}%
\pgfpathcurveto{\pgfqpoint{0.712877in}{0.537123in}}{\pgfqpoint{0.722722in}{0.546968in}}{\pgfqpoint{0.736077in}{0.552500in}}%
\pgfpathcurveto{\pgfqpoint{0.750000in}{0.552500in}}{\pgfqpoint{0.763923in}{0.552500in}}{\pgfqpoint{0.777278in}{0.546968in}}%
\pgfpathcurveto{\pgfqpoint{0.787123in}{0.537123in}}{\pgfqpoint{0.796968in}{0.527278in}}{\pgfqpoint{0.802500in}{0.513923in}}%
\pgfpathcurveto{\pgfqpoint{0.802500in}{0.500000in}}{\pgfqpoint{0.802500in}{0.486077in}}{\pgfqpoint{0.796968in}{0.472722in}}%
\pgfpathcurveto{\pgfqpoint{0.787123in}{0.462877in}}{\pgfqpoint{0.777278in}{0.453032in}}{\pgfqpoint{0.763923in}{0.447500in}}%
\pgfpathclose%
\pgfpathmoveto{\pgfqpoint{0.916667in}{0.441667in}}%
\pgfpathcurveto{\pgfqpoint{0.932137in}{0.441667in}}{\pgfqpoint{0.946975in}{0.447813in}}{\pgfqpoint{0.957915in}{0.458752in}}%
\pgfpathcurveto{\pgfqpoint{0.968854in}{0.469691in}}{\pgfqpoint{0.975000in}{0.484530in}}{\pgfqpoint{0.975000in}{0.500000in}}%
\pgfpathcurveto{\pgfqpoint{0.975000in}{0.515470in}}{\pgfqpoint{0.968854in}{0.530309in}}{\pgfqpoint{0.957915in}{0.541248in}}%
\pgfpathcurveto{\pgfqpoint{0.946975in}{0.552187in}}{\pgfqpoint{0.932137in}{0.558333in}}{\pgfqpoint{0.916667in}{0.558333in}}%
\pgfpathcurveto{\pgfqpoint{0.901196in}{0.558333in}}{\pgfqpoint{0.886358in}{0.552187in}}{\pgfqpoint{0.875419in}{0.541248in}}%
\pgfpathcurveto{\pgfqpoint{0.864480in}{0.530309in}}{\pgfqpoint{0.858333in}{0.515470in}}{\pgfqpoint{0.858333in}{0.500000in}}%
\pgfpathcurveto{\pgfqpoint{0.858333in}{0.484530in}}{\pgfqpoint{0.864480in}{0.469691in}}{\pgfqpoint{0.875419in}{0.458752in}}%
\pgfpathcurveto{\pgfqpoint{0.886358in}{0.447813in}}{\pgfqpoint{0.901196in}{0.441667in}}{\pgfqpoint{0.916667in}{0.441667in}}%
\pgfpathclose%
\pgfpathmoveto{\pgfqpoint{0.916667in}{0.447500in}}%
\pgfpathcurveto{\pgfqpoint{0.916667in}{0.447500in}}{\pgfqpoint{0.902744in}{0.447500in}}{\pgfqpoint{0.889389in}{0.453032in}}%
\pgfpathcurveto{\pgfqpoint{0.879544in}{0.462877in}}{\pgfqpoint{0.869698in}{0.472722in}}{\pgfqpoint{0.864167in}{0.486077in}}%
\pgfpathcurveto{\pgfqpoint{0.864167in}{0.500000in}}{\pgfqpoint{0.864167in}{0.513923in}}{\pgfqpoint{0.869698in}{0.527278in}}%
\pgfpathcurveto{\pgfqpoint{0.879544in}{0.537123in}}{\pgfqpoint{0.889389in}{0.546968in}}{\pgfqpoint{0.902744in}{0.552500in}}%
\pgfpathcurveto{\pgfqpoint{0.916667in}{0.552500in}}{\pgfqpoint{0.930590in}{0.552500in}}{\pgfqpoint{0.943945in}{0.546968in}}%
\pgfpathcurveto{\pgfqpoint{0.953790in}{0.537123in}}{\pgfqpoint{0.963635in}{0.527278in}}{\pgfqpoint{0.969167in}{0.513923in}}%
\pgfpathcurveto{\pgfqpoint{0.969167in}{0.500000in}}{\pgfqpoint{0.969167in}{0.486077in}}{\pgfqpoint{0.963635in}{0.472722in}}%
\pgfpathcurveto{\pgfqpoint{0.953790in}{0.462877in}}{\pgfqpoint{0.943945in}{0.453032in}}{\pgfqpoint{0.930590in}{0.447500in}}%
\pgfpathclose%
\pgfpathmoveto{\pgfqpoint{0.000000in}{0.608333in}}%
\pgfpathcurveto{\pgfqpoint{0.015470in}{0.608333in}}{\pgfqpoint{0.030309in}{0.614480in}}{\pgfqpoint{0.041248in}{0.625419in}}%
\pgfpathcurveto{\pgfqpoint{0.052187in}{0.636358in}}{\pgfqpoint{0.058333in}{0.651196in}}{\pgfqpoint{0.058333in}{0.666667in}}%
\pgfpathcurveto{\pgfqpoint{0.058333in}{0.682137in}}{\pgfqpoint{0.052187in}{0.696975in}}{\pgfqpoint{0.041248in}{0.707915in}}%
\pgfpathcurveto{\pgfqpoint{0.030309in}{0.718854in}}{\pgfqpoint{0.015470in}{0.725000in}}{\pgfqpoint{0.000000in}{0.725000in}}%
\pgfpathcurveto{\pgfqpoint{-0.015470in}{0.725000in}}{\pgfqpoint{-0.030309in}{0.718854in}}{\pgfqpoint{-0.041248in}{0.707915in}}%
\pgfpathcurveto{\pgfqpoint{-0.052187in}{0.696975in}}{\pgfqpoint{-0.058333in}{0.682137in}}{\pgfqpoint{-0.058333in}{0.666667in}}%
\pgfpathcurveto{\pgfqpoint{-0.058333in}{0.651196in}}{\pgfqpoint{-0.052187in}{0.636358in}}{\pgfqpoint{-0.041248in}{0.625419in}}%
\pgfpathcurveto{\pgfqpoint{-0.030309in}{0.614480in}}{\pgfqpoint{-0.015470in}{0.608333in}}{\pgfqpoint{0.000000in}{0.608333in}}%
\pgfpathclose%
\pgfpathmoveto{\pgfqpoint{0.000000in}{0.614167in}}%
\pgfpathcurveto{\pgfqpoint{0.000000in}{0.614167in}}{\pgfqpoint{-0.013923in}{0.614167in}}{\pgfqpoint{-0.027278in}{0.619698in}}%
\pgfpathcurveto{\pgfqpoint{-0.037123in}{0.629544in}}{\pgfqpoint{-0.046968in}{0.639389in}}{\pgfqpoint{-0.052500in}{0.652744in}}%
\pgfpathcurveto{\pgfqpoint{-0.052500in}{0.666667in}}{\pgfqpoint{-0.052500in}{0.680590in}}{\pgfqpoint{-0.046968in}{0.693945in}}%
\pgfpathcurveto{\pgfqpoint{-0.037123in}{0.703790in}}{\pgfqpoint{-0.027278in}{0.713635in}}{\pgfqpoint{-0.013923in}{0.719167in}}%
\pgfpathcurveto{\pgfqpoint{0.000000in}{0.719167in}}{\pgfqpoint{0.013923in}{0.719167in}}{\pgfqpoint{0.027278in}{0.713635in}}%
\pgfpathcurveto{\pgfqpoint{0.037123in}{0.703790in}}{\pgfqpoint{0.046968in}{0.693945in}}{\pgfqpoint{0.052500in}{0.680590in}}%
\pgfpathcurveto{\pgfqpoint{0.052500in}{0.666667in}}{\pgfqpoint{0.052500in}{0.652744in}}{\pgfqpoint{0.046968in}{0.639389in}}%
\pgfpathcurveto{\pgfqpoint{0.037123in}{0.629544in}}{\pgfqpoint{0.027278in}{0.619698in}}{\pgfqpoint{0.013923in}{0.614167in}}%
\pgfpathclose%
\pgfpathmoveto{\pgfqpoint{0.166667in}{0.608333in}}%
\pgfpathcurveto{\pgfqpoint{0.182137in}{0.608333in}}{\pgfqpoint{0.196975in}{0.614480in}}{\pgfqpoint{0.207915in}{0.625419in}}%
\pgfpathcurveto{\pgfqpoint{0.218854in}{0.636358in}}{\pgfqpoint{0.225000in}{0.651196in}}{\pgfqpoint{0.225000in}{0.666667in}}%
\pgfpathcurveto{\pgfqpoint{0.225000in}{0.682137in}}{\pgfqpoint{0.218854in}{0.696975in}}{\pgfqpoint{0.207915in}{0.707915in}}%
\pgfpathcurveto{\pgfqpoint{0.196975in}{0.718854in}}{\pgfqpoint{0.182137in}{0.725000in}}{\pgfqpoint{0.166667in}{0.725000in}}%
\pgfpathcurveto{\pgfqpoint{0.151196in}{0.725000in}}{\pgfqpoint{0.136358in}{0.718854in}}{\pgfqpoint{0.125419in}{0.707915in}}%
\pgfpathcurveto{\pgfqpoint{0.114480in}{0.696975in}}{\pgfqpoint{0.108333in}{0.682137in}}{\pgfqpoint{0.108333in}{0.666667in}}%
\pgfpathcurveto{\pgfqpoint{0.108333in}{0.651196in}}{\pgfqpoint{0.114480in}{0.636358in}}{\pgfqpoint{0.125419in}{0.625419in}}%
\pgfpathcurveto{\pgfqpoint{0.136358in}{0.614480in}}{\pgfqpoint{0.151196in}{0.608333in}}{\pgfqpoint{0.166667in}{0.608333in}}%
\pgfpathclose%
\pgfpathmoveto{\pgfqpoint{0.166667in}{0.614167in}}%
\pgfpathcurveto{\pgfqpoint{0.166667in}{0.614167in}}{\pgfqpoint{0.152744in}{0.614167in}}{\pgfqpoint{0.139389in}{0.619698in}}%
\pgfpathcurveto{\pgfqpoint{0.129544in}{0.629544in}}{\pgfqpoint{0.119698in}{0.639389in}}{\pgfqpoint{0.114167in}{0.652744in}}%
\pgfpathcurveto{\pgfqpoint{0.114167in}{0.666667in}}{\pgfqpoint{0.114167in}{0.680590in}}{\pgfqpoint{0.119698in}{0.693945in}}%
\pgfpathcurveto{\pgfqpoint{0.129544in}{0.703790in}}{\pgfqpoint{0.139389in}{0.713635in}}{\pgfqpoint{0.152744in}{0.719167in}}%
\pgfpathcurveto{\pgfqpoint{0.166667in}{0.719167in}}{\pgfqpoint{0.180590in}{0.719167in}}{\pgfqpoint{0.193945in}{0.713635in}}%
\pgfpathcurveto{\pgfqpoint{0.203790in}{0.703790in}}{\pgfqpoint{0.213635in}{0.693945in}}{\pgfqpoint{0.219167in}{0.680590in}}%
\pgfpathcurveto{\pgfqpoint{0.219167in}{0.666667in}}{\pgfqpoint{0.219167in}{0.652744in}}{\pgfqpoint{0.213635in}{0.639389in}}%
\pgfpathcurveto{\pgfqpoint{0.203790in}{0.629544in}}{\pgfqpoint{0.193945in}{0.619698in}}{\pgfqpoint{0.180590in}{0.614167in}}%
\pgfpathclose%
\pgfpathmoveto{\pgfqpoint{0.333333in}{0.608333in}}%
\pgfpathcurveto{\pgfqpoint{0.348804in}{0.608333in}}{\pgfqpoint{0.363642in}{0.614480in}}{\pgfqpoint{0.374581in}{0.625419in}}%
\pgfpathcurveto{\pgfqpoint{0.385520in}{0.636358in}}{\pgfqpoint{0.391667in}{0.651196in}}{\pgfqpoint{0.391667in}{0.666667in}}%
\pgfpathcurveto{\pgfqpoint{0.391667in}{0.682137in}}{\pgfqpoint{0.385520in}{0.696975in}}{\pgfqpoint{0.374581in}{0.707915in}}%
\pgfpathcurveto{\pgfqpoint{0.363642in}{0.718854in}}{\pgfqpoint{0.348804in}{0.725000in}}{\pgfqpoint{0.333333in}{0.725000in}}%
\pgfpathcurveto{\pgfqpoint{0.317863in}{0.725000in}}{\pgfqpoint{0.303025in}{0.718854in}}{\pgfqpoint{0.292085in}{0.707915in}}%
\pgfpathcurveto{\pgfqpoint{0.281146in}{0.696975in}}{\pgfqpoint{0.275000in}{0.682137in}}{\pgfqpoint{0.275000in}{0.666667in}}%
\pgfpathcurveto{\pgfqpoint{0.275000in}{0.651196in}}{\pgfqpoint{0.281146in}{0.636358in}}{\pgfqpoint{0.292085in}{0.625419in}}%
\pgfpathcurveto{\pgfqpoint{0.303025in}{0.614480in}}{\pgfqpoint{0.317863in}{0.608333in}}{\pgfqpoint{0.333333in}{0.608333in}}%
\pgfpathclose%
\pgfpathmoveto{\pgfqpoint{0.333333in}{0.614167in}}%
\pgfpathcurveto{\pgfqpoint{0.333333in}{0.614167in}}{\pgfqpoint{0.319410in}{0.614167in}}{\pgfqpoint{0.306055in}{0.619698in}}%
\pgfpathcurveto{\pgfqpoint{0.296210in}{0.629544in}}{\pgfqpoint{0.286365in}{0.639389in}}{\pgfqpoint{0.280833in}{0.652744in}}%
\pgfpathcurveto{\pgfqpoint{0.280833in}{0.666667in}}{\pgfqpoint{0.280833in}{0.680590in}}{\pgfqpoint{0.286365in}{0.693945in}}%
\pgfpathcurveto{\pgfqpoint{0.296210in}{0.703790in}}{\pgfqpoint{0.306055in}{0.713635in}}{\pgfqpoint{0.319410in}{0.719167in}}%
\pgfpathcurveto{\pgfqpoint{0.333333in}{0.719167in}}{\pgfqpoint{0.347256in}{0.719167in}}{\pgfqpoint{0.360611in}{0.713635in}}%
\pgfpathcurveto{\pgfqpoint{0.370456in}{0.703790in}}{\pgfqpoint{0.380302in}{0.693945in}}{\pgfqpoint{0.385833in}{0.680590in}}%
\pgfpathcurveto{\pgfqpoint{0.385833in}{0.666667in}}{\pgfqpoint{0.385833in}{0.652744in}}{\pgfqpoint{0.380302in}{0.639389in}}%
\pgfpathcurveto{\pgfqpoint{0.370456in}{0.629544in}}{\pgfqpoint{0.360611in}{0.619698in}}{\pgfqpoint{0.347256in}{0.614167in}}%
\pgfpathclose%
\pgfpathmoveto{\pgfqpoint{0.500000in}{0.608333in}}%
\pgfpathcurveto{\pgfqpoint{0.515470in}{0.608333in}}{\pgfqpoint{0.530309in}{0.614480in}}{\pgfqpoint{0.541248in}{0.625419in}}%
\pgfpathcurveto{\pgfqpoint{0.552187in}{0.636358in}}{\pgfqpoint{0.558333in}{0.651196in}}{\pgfqpoint{0.558333in}{0.666667in}}%
\pgfpathcurveto{\pgfqpoint{0.558333in}{0.682137in}}{\pgfqpoint{0.552187in}{0.696975in}}{\pgfqpoint{0.541248in}{0.707915in}}%
\pgfpathcurveto{\pgfqpoint{0.530309in}{0.718854in}}{\pgfqpoint{0.515470in}{0.725000in}}{\pgfqpoint{0.500000in}{0.725000in}}%
\pgfpathcurveto{\pgfqpoint{0.484530in}{0.725000in}}{\pgfqpoint{0.469691in}{0.718854in}}{\pgfqpoint{0.458752in}{0.707915in}}%
\pgfpathcurveto{\pgfqpoint{0.447813in}{0.696975in}}{\pgfqpoint{0.441667in}{0.682137in}}{\pgfqpoint{0.441667in}{0.666667in}}%
\pgfpathcurveto{\pgfqpoint{0.441667in}{0.651196in}}{\pgfqpoint{0.447813in}{0.636358in}}{\pgfqpoint{0.458752in}{0.625419in}}%
\pgfpathcurveto{\pgfqpoint{0.469691in}{0.614480in}}{\pgfqpoint{0.484530in}{0.608333in}}{\pgfqpoint{0.500000in}{0.608333in}}%
\pgfpathclose%
\pgfpathmoveto{\pgfqpoint{0.500000in}{0.614167in}}%
\pgfpathcurveto{\pgfqpoint{0.500000in}{0.614167in}}{\pgfqpoint{0.486077in}{0.614167in}}{\pgfqpoint{0.472722in}{0.619698in}}%
\pgfpathcurveto{\pgfqpoint{0.462877in}{0.629544in}}{\pgfqpoint{0.453032in}{0.639389in}}{\pgfqpoint{0.447500in}{0.652744in}}%
\pgfpathcurveto{\pgfqpoint{0.447500in}{0.666667in}}{\pgfqpoint{0.447500in}{0.680590in}}{\pgfqpoint{0.453032in}{0.693945in}}%
\pgfpathcurveto{\pgfqpoint{0.462877in}{0.703790in}}{\pgfqpoint{0.472722in}{0.713635in}}{\pgfqpoint{0.486077in}{0.719167in}}%
\pgfpathcurveto{\pgfqpoint{0.500000in}{0.719167in}}{\pgfqpoint{0.513923in}{0.719167in}}{\pgfqpoint{0.527278in}{0.713635in}}%
\pgfpathcurveto{\pgfqpoint{0.537123in}{0.703790in}}{\pgfqpoint{0.546968in}{0.693945in}}{\pgfqpoint{0.552500in}{0.680590in}}%
\pgfpathcurveto{\pgfqpoint{0.552500in}{0.666667in}}{\pgfqpoint{0.552500in}{0.652744in}}{\pgfqpoint{0.546968in}{0.639389in}}%
\pgfpathcurveto{\pgfqpoint{0.537123in}{0.629544in}}{\pgfqpoint{0.527278in}{0.619698in}}{\pgfqpoint{0.513923in}{0.614167in}}%
\pgfpathclose%
\pgfpathmoveto{\pgfqpoint{0.666667in}{0.608333in}}%
\pgfpathcurveto{\pgfqpoint{0.682137in}{0.608333in}}{\pgfqpoint{0.696975in}{0.614480in}}{\pgfqpoint{0.707915in}{0.625419in}}%
\pgfpathcurveto{\pgfqpoint{0.718854in}{0.636358in}}{\pgfqpoint{0.725000in}{0.651196in}}{\pgfqpoint{0.725000in}{0.666667in}}%
\pgfpathcurveto{\pgfqpoint{0.725000in}{0.682137in}}{\pgfqpoint{0.718854in}{0.696975in}}{\pgfqpoint{0.707915in}{0.707915in}}%
\pgfpathcurveto{\pgfqpoint{0.696975in}{0.718854in}}{\pgfqpoint{0.682137in}{0.725000in}}{\pgfqpoint{0.666667in}{0.725000in}}%
\pgfpathcurveto{\pgfqpoint{0.651196in}{0.725000in}}{\pgfqpoint{0.636358in}{0.718854in}}{\pgfqpoint{0.625419in}{0.707915in}}%
\pgfpathcurveto{\pgfqpoint{0.614480in}{0.696975in}}{\pgfqpoint{0.608333in}{0.682137in}}{\pgfqpoint{0.608333in}{0.666667in}}%
\pgfpathcurveto{\pgfqpoint{0.608333in}{0.651196in}}{\pgfqpoint{0.614480in}{0.636358in}}{\pgfqpoint{0.625419in}{0.625419in}}%
\pgfpathcurveto{\pgfqpoint{0.636358in}{0.614480in}}{\pgfqpoint{0.651196in}{0.608333in}}{\pgfqpoint{0.666667in}{0.608333in}}%
\pgfpathclose%
\pgfpathmoveto{\pgfqpoint{0.666667in}{0.614167in}}%
\pgfpathcurveto{\pgfqpoint{0.666667in}{0.614167in}}{\pgfqpoint{0.652744in}{0.614167in}}{\pgfqpoint{0.639389in}{0.619698in}}%
\pgfpathcurveto{\pgfqpoint{0.629544in}{0.629544in}}{\pgfqpoint{0.619698in}{0.639389in}}{\pgfqpoint{0.614167in}{0.652744in}}%
\pgfpathcurveto{\pgfqpoint{0.614167in}{0.666667in}}{\pgfqpoint{0.614167in}{0.680590in}}{\pgfqpoint{0.619698in}{0.693945in}}%
\pgfpathcurveto{\pgfqpoint{0.629544in}{0.703790in}}{\pgfqpoint{0.639389in}{0.713635in}}{\pgfqpoint{0.652744in}{0.719167in}}%
\pgfpathcurveto{\pgfqpoint{0.666667in}{0.719167in}}{\pgfqpoint{0.680590in}{0.719167in}}{\pgfqpoint{0.693945in}{0.713635in}}%
\pgfpathcurveto{\pgfqpoint{0.703790in}{0.703790in}}{\pgfqpoint{0.713635in}{0.693945in}}{\pgfqpoint{0.719167in}{0.680590in}}%
\pgfpathcurveto{\pgfqpoint{0.719167in}{0.666667in}}{\pgfqpoint{0.719167in}{0.652744in}}{\pgfqpoint{0.713635in}{0.639389in}}%
\pgfpathcurveto{\pgfqpoint{0.703790in}{0.629544in}}{\pgfqpoint{0.693945in}{0.619698in}}{\pgfqpoint{0.680590in}{0.614167in}}%
\pgfpathclose%
\pgfpathmoveto{\pgfqpoint{0.833333in}{0.608333in}}%
\pgfpathcurveto{\pgfqpoint{0.848804in}{0.608333in}}{\pgfqpoint{0.863642in}{0.614480in}}{\pgfqpoint{0.874581in}{0.625419in}}%
\pgfpathcurveto{\pgfqpoint{0.885520in}{0.636358in}}{\pgfqpoint{0.891667in}{0.651196in}}{\pgfqpoint{0.891667in}{0.666667in}}%
\pgfpathcurveto{\pgfqpoint{0.891667in}{0.682137in}}{\pgfqpoint{0.885520in}{0.696975in}}{\pgfqpoint{0.874581in}{0.707915in}}%
\pgfpathcurveto{\pgfqpoint{0.863642in}{0.718854in}}{\pgfqpoint{0.848804in}{0.725000in}}{\pgfqpoint{0.833333in}{0.725000in}}%
\pgfpathcurveto{\pgfqpoint{0.817863in}{0.725000in}}{\pgfqpoint{0.803025in}{0.718854in}}{\pgfqpoint{0.792085in}{0.707915in}}%
\pgfpathcurveto{\pgfqpoint{0.781146in}{0.696975in}}{\pgfqpoint{0.775000in}{0.682137in}}{\pgfqpoint{0.775000in}{0.666667in}}%
\pgfpathcurveto{\pgfqpoint{0.775000in}{0.651196in}}{\pgfqpoint{0.781146in}{0.636358in}}{\pgfqpoint{0.792085in}{0.625419in}}%
\pgfpathcurveto{\pgfqpoint{0.803025in}{0.614480in}}{\pgfqpoint{0.817863in}{0.608333in}}{\pgfqpoint{0.833333in}{0.608333in}}%
\pgfpathclose%
\pgfpathmoveto{\pgfqpoint{0.833333in}{0.614167in}}%
\pgfpathcurveto{\pgfqpoint{0.833333in}{0.614167in}}{\pgfqpoint{0.819410in}{0.614167in}}{\pgfqpoint{0.806055in}{0.619698in}}%
\pgfpathcurveto{\pgfqpoint{0.796210in}{0.629544in}}{\pgfqpoint{0.786365in}{0.639389in}}{\pgfqpoint{0.780833in}{0.652744in}}%
\pgfpathcurveto{\pgfqpoint{0.780833in}{0.666667in}}{\pgfqpoint{0.780833in}{0.680590in}}{\pgfqpoint{0.786365in}{0.693945in}}%
\pgfpathcurveto{\pgfqpoint{0.796210in}{0.703790in}}{\pgfqpoint{0.806055in}{0.713635in}}{\pgfqpoint{0.819410in}{0.719167in}}%
\pgfpathcurveto{\pgfqpoint{0.833333in}{0.719167in}}{\pgfqpoint{0.847256in}{0.719167in}}{\pgfqpoint{0.860611in}{0.713635in}}%
\pgfpathcurveto{\pgfqpoint{0.870456in}{0.703790in}}{\pgfqpoint{0.880302in}{0.693945in}}{\pgfqpoint{0.885833in}{0.680590in}}%
\pgfpathcurveto{\pgfqpoint{0.885833in}{0.666667in}}{\pgfqpoint{0.885833in}{0.652744in}}{\pgfqpoint{0.880302in}{0.639389in}}%
\pgfpathcurveto{\pgfqpoint{0.870456in}{0.629544in}}{\pgfqpoint{0.860611in}{0.619698in}}{\pgfqpoint{0.847256in}{0.614167in}}%
\pgfpathclose%
\pgfpathmoveto{\pgfqpoint{1.000000in}{0.608333in}}%
\pgfpathcurveto{\pgfqpoint{1.015470in}{0.608333in}}{\pgfqpoint{1.030309in}{0.614480in}}{\pgfqpoint{1.041248in}{0.625419in}}%
\pgfpathcurveto{\pgfqpoint{1.052187in}{0.636358in}}{\pgfqpoint{1.058333in}{0.651196in}}{\pgfqpoint{1.058333in}{0.666667in}}%
\pgfpathcurveto{\pgfqpoint{1.058333in}{0.682137in}}{\pgfqpoint{1.052187in}{0.696975in}}{\pgfqpoint{1.041248in}{0.707915in}}%
\pgfpathcurveto{\pgfqpoint{1.030309in}{0.718854in}}{\pgfqpoint{1.015470in}{0.725000in}}{\pgfqpoint{1.000000in}{0.725000in}}%
\pgfpathcurveto{\pgfqpoint{0.984530in}{0.725000in}}{\pgfqpoint{0.969691in}{0.718854in}}{\pgfqpoint{0.958752in}{0.707915in}}%
\pgfpathcurveto{\pgfqpoint{0.947813in}{0.696975in}}{\pgfqpoint{0.941667in}{0.682137in}}{\pgfqpoint{0.941667in}{0.666667in}}%
\pgfpathcurveto{\pgfqpoint{0.941667in}{0.651196in}}{\pgfqpoint{0.947813in}{0.636358in}}{\pgfqpoint{0.958752in}{0.625419in}}%
\pgfpathcurveto{\pgfqpoint{0.969691in}{0.614480in}}{\pgfqpoint{0.984530in}{0.608333in}}{\pgfqpoint{1.000000in}{0.608333in}}%
\pgfpathclose%
\pgfpathmoveto{\pgfqpoint{1.000000in}{0.614167in}}%
\pgfpathcurveto{\pgfqpoint{1.000000in}{0.614167in}}{\pgfqpoint{0.986077in}{0.614167in}}{\pgfqpoint{0.972722in}{0.619698in}}%
\pgfpathcurveto{\pgfqpoint{0.962877in}{0.629544in}}{\pgfqpoint{0.953032in}{0.639389in}}{\pgfqpoint{0.947500in}{0.652744in}}%
\pgfpathcurveto{\pgfqpoint{0.947500in}{0.666667in}}{\pgfqpoint{0.947500in}{0.680590in}}{\pgfqpoint{0.953032in}{0.693945in}}%
\pgfpathcurveto{\pgfqpoint{0.962877in}{0.703790in}}{\pgfqpoint{0.972722in}{0.713635in}}{\pgfqpoint{0.986077in}{0.719167in}}%
\pgfpathcurveto{\pgfqpoint{1.000000in}{0.719167in}}{\pgfqpoint{1.013923in}{0.719167in}}{\pgfqpoint{1.027278in}{0.713635in}}%
\pgfpathcurveto{\pgfqpoint{1.037123in}{0.703790in}}{\pgfqpoint{1.046968in}{0.693945in}}{\pgfqpoint{1.052500in}{0.680590in}}%
\pgfpathcurveto{\pgfqpoint{1.052500in}{0.666667in}}{\pgfqpoint{1.052500in}{0.652744in}}{\pgfqpoint{1.046968in}{0.639389in}}%
\pgfpathcurveto{\pgfqpoint{1.037123in}{0.629544in}}{\pgfqpoint{1.027278in}{0.619698in}}{\pgfqpoint{1.013923in}{0.614167in}}%
\pgfpathclose%
\pgfpathmoveto{\pgfqpoint{0.083333in}{0.775000in}}%
\pgfpathcurveto{\pgfqpoint{0.098804in}{0.775000in}}{\pgfqpoint{0.113642in}{0.781146in}}{\pgfqpoint{0.124581in}{0.792085in}}%
\pgfpathcurveto{\pgfqpoint{0.135520in}{0.803025in}}{\pgfqpoint{0.141667in}{0.817863in}}{\pgfqpoint{0.141667in}{0.833333in}}%
\pgfpathcurveto{\pgfqpoint{0.141667in}{0.848804in}}{\pgfqpoint{0.135520in}{0.863642in}}{\pgfqpoint{0.124581in}{0.874581in}}%
\pgfpathcurveto{\pgfqpoint{0.113642in}{0.885520in}}{\pgfqpoint{0.098804in}{0.891667in}}{\pgfqpoint{0.083333in}{0.891667in}}%
\pgfpathcurveto{\pgfqpoint{0.067863in}{0.891667in}}{\pgfqpoint{0.053025in}{0.885520in}}{\pgfqpoint{0.042085in}{0.874581in}}%
\pgfpathcurveto{\pgfqpoint{0.031146in}{0.863642in}}{\pgfqpoint{0.025000in}{0.848804in}}{\pgfqpoint{0.025000in}{0.833333in}}%
\pgfpathcurveto{\pgfqpoint{0.025000in}{0.817863in}}{\pgfqpoint{0.031146in}{0.803025in}}{\pgfqpoint{0.042085in}{0.792085in}}%
\pgfpathcurveto{\pgfqpoint{0.053025in}{0.781146in}}{\pgfqpoint{0.067863in}{0.775000in}}{\pgfqpoint{0.083333in}{0.775000in}}%
\pgfpathclose%
\pgfpathmoveto{\pgfqpoint{0.083333in}{0.780833in}}%
\pgfpathcurveto{\pgfqpoint{0.083333in}{0.780833in}}{\pgfqpoint{0.069410in}{0.780833in}}{\pgfqpoint{0.056055in}{0.786365in}}%
\pgfpathcurveto{\pgfqpoint{0.046210in}{0.796210in}}{\pgfqpoint{0.036365in}{0.806055in}}{\pgfqpoint{0.030833in}{0.819410in}}%
\pgfpathcurveto{\pgfqpoint{0.030833in}{0.833333in}}{\pgfqpoint{0.030833in}{0.847256in}}{\pgfqpoint{0.036365in}{0.860611in}}%
\pgfpathcurveto{\pgfqpoint{0.046210in}{0.870456in}}{\pgfqpoint{0.056055in}{0.880302in}}{\pgfqpoint{0.069410in}{0.885833in}}%
\pgfpathcurveto{\pgfqpoint{0.083333in}{0.885833in}}{\pgfqpoint{0.097256in}{0.885833in}}{\pgfqpoint{0.110611in}{0.880302in}}%
\pgfpathcurveto{\pgfqpoint{0.120456in}{0.870456in}}{\pgfqpoint{0.130302in}{0.860611in}}{\pgfqpoint{0.135833in}{0.847256in}}%
\pgfpathcurveto{\pgfqpoint{0.135833in}{0.833333in}}{\pgfqpoint{0.135833in}{0.819410in}}{\pgfqpoint{0.130302in}{0.806055in}}%
\pgfpathcurveto{\pgfqpoint{0.120456in}{0.796210in}}{\pgfqpoint{0.110611in}{0.786365in}}{\pgfqpoint{0.097256in}{0.780833in}}%
\pgfpathclose%
\pgfpathmoveto{\pgfqpoint{0.250000in}{0.775000in}}%
\pgfpathcurveto{\pgfqpoint{0.265470in}{0.775000in}}{\pgfqpoint{0.280309in}{0.781146in}}{\pgfqpoint{0.291248in}{0.792085in}}%
\pgfpathcurveto{\pgfqpoint{0.302187in}{0.803025in}}{\pgfqpoint{0.308333in}{0.817863in}}{\pgfqpoint{0.308333in}{0.833333in}}%
\pgfpathcurveto{\pgfqpoint{0.308333in}{0.848804in}}{\pgfqpoint{0.302187in}{0.863642in}}{\pgfqpoint{0.291248in}{0.874581in}}%
\pgfpathcurveto{\pgfqpoint{0.280309in}{0.885520in}}{\pgfqpoint{0.265470in}{0.891667in}}{\pgfqpoint{0.250000in}{0.891667in}}%
\pgfpathcurveto{\pgfqpoint{0.234530in}{0.891667in}}{\pgfqpoint{0.219691in}{0.885520in}}{\pgfqpoint{0.208752in}{0.874581in}}%
\pgfpathcurveto{\pgfqpoint{0.197813in}{0.863642in}}{\pgfqpoint{0.191667in}{0.848804in}}{\pgfqpoint{0.191667in}{0.833333in}}%
\pgfpathcurveto{\pgfqpoint{0.191667in}{0.817863in}}{\pgfqpoint{0.197813in}{0.803025in}}{\pgfqpoint{0.208752in}{0.792085in}}%
\pgfpathcurveto{\pgfqpoint{0.219691in}{0.781146in}}{\pgfqpoint{0.234530in}{0.775000in}}{\pgfqpoint{0.250000in}{0.775000in}}%
\pgfpathclose%
\pgfpathmoveto{\pgfqpoint{0.250000in}{0.780833in}}%
\pgfpathcurveto{\pgfqpoint{0.250000in}{0.780833in}}{\pgfqpoint{0.236077in}{0.780833in}}{\pgfqpoint{0.222722in}{0.786365in}}%
\pgfpathcurveto{\pgfqpoint{0.212877in}{0.796210in}}{\pgfqpoint{0.203032in}{0.806055in}}{\pgfqpoint{0.197500in}{0.819410in}}%
\pgfpathcurveto{\pgfqpoint{0.197500in}{0.833333in}}{\pgfqpoint{0.197500in}{0.847256in}}{\pgfqpoint{0.203032in}{0.860611in}}%
\pgfpathcurveto{\pgfqpoint{0.212877in}{0.870456in}}{\pgfqpoint{0.222722in}{0.880302in}}{\pgfqpoint{0.236077in}{0.885833in}}%
\pgfpathcurveto{\pgfqpoint{0.250000in}{0.885833in}}{\pgfqpoint{0.263923in}{0.885833in}}{\pgfqpoint{0.277278in}{0.880302in}}%
\pgfpathcurveto{\pgfqpoint{0.287123in}{0.870456in}}{\pgfqpoint{0.296968in}{0.860611in}}{\pgfqpoint{0.302500in}{0.847256in}}%
\pgfpathcurveto{\pgfqpoint{0.302500in}{0.833333in}}{\pgfqpoint{0.302500in}{0.819410in}}{\pgfqpoint{0.296968in}{0.806055in}}%
\pgfpathcurveto{\pgfqpoint{0.287123in}{0.796210in}}{\pgfqpoint{0.277278in}{0.786365in}}{\pgfqpoint{0.263923in}{0.780833in}}%
\pgfpathclose%
\pgfpathmoveto{\pgfqpoint{0.416667in}{0.775000in}}%
\pgfpathcurveto{\pgfqpoint{0.432137in}{0.775000in}}{\pgfqpoint{0.446975in}{0.781146in}}{\pgfqpoint{0.457915in}{0.792085in}}%
\pgfpathcurveto{\pgfqpoint{0.468854in}{0.803025in}}{\pgfqpoint{0.475000in}{0.817863in}}{\pgfqpoint{0.475000in}{0.833333in}}%
\pgfpathcurveto{\pgfqpoint{0.475000in}{0.848804in}}{\pgfqpoint{0.468854in}{0.863642in}}{\pgfqpoint{0.457915in}{0.874581in}}%
\pgfpathcurveto{\pgfqpoint{0.446975in}{0.885520in}}{\pgfqpoint{0.432137in}{0.891667in}}{\pgfqpoint{0.416667in}{0.891667in}}%
\pgfpathcurveto{\pgfqpoint{0.401196in}{0.891667in}}{\pgfqpoint{0.386358in}{0.885520in}}{\pgfqpoint{0.375419in}{0.874581in}}%
\pgfpathcurveto{\pgfqpoint{0.364480in}{0.863642in}}{\pgfqpoint{0.358333in}{0.848804in}}{\pgfqpoint{0.358333in}{0.833333in}}%
\pgfpathcurveto{\pgfqpoint{0.358333in}{0.817863in}}{\pgfqpoint{0.364480in}{0.803025in}}{\pgfqpoint{0.375419in}{0.792085in}}%
\pgfpathcurveto{\pgfqpoint{0.386358in}{0.781146in}}{\pgfqpoint{0.401196in}{0.775000in}}{\pgfqpoint{0.416667in}{0.775000in}}%
\pgfpathclose%
\pgfpathmoveto{\pgfqpoint{0.416667in}{0.780833in}}%
\pgfpathcurveto{\pgfqpoint{0.416667in}{0.780833in}}{\pgfqpoint{0.402744in}{0.780833in}}{\pgfqpoint{0.389389in}{0.786365in}}%
\pgfpathcurveto{\pgfqpoint{0.379544in}{0.796210in}}{\pgfqpoint{0.369698in}{0.806055in}}{\pgfqpoint{0.364167in}{0.819410in}}%
\pgfpathcurveto{\pgfqpoint{0.364167in}{0.833333in}}{\pgfqpoint{0.364167in}{0.847256in}}{\pgfqpoint{0.369698in}{0.860611in}}%
\pgfpathcurveto{\pgfqpoint{0.379544in}{0.870456in}}{\pgfqpoint{0.389389in}{0.880302in}}{\pgfqpoint{0.402744in}{0.885833in}}%
\pgfpathcurveto{\pgfqpoint{0.416667in}{0.885833in}}{\pgfqpoint{0.430590in}{0.885833in}}{\pgfqpoint{0.443945in}{0.880302in}}%
\pgfpathcurveto{\pgfqpoint{0.453790in}{0.870456in}}{\pgfqpoint{0.463635in}{0.860611in}}{\pgfqpoint{0.469167in}{0.847256in}}%
\pgfpathcurveto{\pgfqpoint{0.469167in}{0.833333in}}{\pgfqpoint{0.469167in}{0.819410in}}{\pgfqpoint{0.463635in}{0.806055in}}%
\pgfpathcurveto{\pgfqpoint{0.453790in}{0.796210in}}{\pgfqpoint{0.443945in}{0.786365in}}{\pgfqpoint{0.430590in}{0.780833in}}%
\pgfpathclose%
\pgfpathmoveto{\pgfqpoint{0.583333in}{0.775000in}}%
\pgfpathcurveto{\pgfqpoint{0.598804in}{0.775000in}}{\pgfqpoint{0.613642in}{0.781146in}}{\pgfqpoint{0.624581in}{0.792085in}}%
\pgfpathcurveto{\pgfqpoint{0.635520in}{0.803025in}}{\pgfqpoint{0.641667in}{0.817863in}}{\pgfqpoint{0.641667in}{0.833333in}}%
\pgfpathcurveto{\pgfqpoint{0.641667in}{0.848804in}}{\pgfqpoint{0.635520in}{0.863642in}}{\pgfqpoint{0.624581in}{0.874581in}}%
\pgfpathcurveto{\pgfqpoint{0.613642in}{0.885520in}}{\pgfqpoint{0.598804in}{0.891667in}}{\pgfqpoint{0.583333in}{0.891667in}}%
\pgfpathcurveto{\pgfqpoint{0.567863in}{0.891667in}}{\pgfqpoint{0.553025in}{0.885520in}}{\pgfqpoint{0.542085in}{0.874581in}}%
\pgfpathcurveto{\pgfqpoint{0.531146in}{0.863642in}}{\pgfqpoint{0.525000in}{0.848804in}}{\pgfqpoint{0.525000in}{0.833333in}}%
\pgfpathcurveto{\pgfqpoint{0.525000in}{0.817863in}}{\pgfqpoint{0.531146in}{0.803025in}}{\pgfqpoint{0.542085in}{0.792085in}}%
\pgfpathcurveto{\pgfqpoint{0.553025in}{0.781146in}}{\pgfqpoint{0.567863in}{0.775000in}}{\pgfqpoint{0.583333in}{0.775000in}}%
\pgfpathclose%
\pgfpathmoveto{\pgfqpoint{0.583333in}{0.780833in}}%
\pgfpathcurveto{\pgfqpoint{0.583333in}{0.780833in}}{\pgfqpoint{0.569410in}{0.780833in}}{\pgfqpoint{0.556055in}{0.786365in}}%
\pgfpathcurveto{\pgfqpoint{0.546210in}{0.796210in}}{\pgfqpoint{0.536365in}{0.806055in}}{\pgfqpoint{0.530833in}{0.819410in}}%
\pgfpathcurveto{\pgfqpoint{0.530833in}{0.833333in}}{\pgfqpoint{0.530833in}{0.847256in}}{\pgfqpoint{0.536365in}{0.860611in}}%
\pgfpathcurveto{\pgfqpoint{0.546210in}{0.870456in}}{\pgfqpoint{0.556055in}{0.880302in}}{\pgfqpoint{0.569410in}{0.885833in}}%
\pgfpathcurveto{\pgfqpoint{0.583333in}{0.885833in}}{\pgfqpoint{0.597256in}{0.885833in}}{\pgfqpoint{0.610611in}{0.880302in}}%
\pgfpathcurveto{\pgfqpoint{0.620456in}{0.870456in}}{\pgfqpoint{0.630302in}{0.860611in}}{\pgfqpoint{0.635833in}{0.847256in}}%
\pgfpathcurveto{\pgfqpoint{0.635833in}{0.833333in}}{\pgfqpoint{0.635833in}{0.819410in}}{\pgfqpoint{0.630302in}{0.806055in}}%
\pgfpathcurveto{\pgfqpoint{0.620456in}{0.796210in}}{\pgfqpoint{0.610611in}{0.786365in}}{\pgfqpoint{0.597256in}{0.780833in}}%
\pgfpathclose%
\pgfpathmoveto{\pgfqpoint{0.750000in}{0.775000in}}%
\pgfpathcurveto{\pgfqpoint{0.765470in}{0.775000in}}{\pgfqpoint{0.780309in}{0.781146in}}{\pgfqpoint{0.791248in}{0.792085in}}%
\pgfpathcurveto{\pgfqpoint{0.802187in}{0.803025in}}{\pgfqpoint{0.808333in}{0.817863in}}{\pgfqpoint{0.808333in}{0.833333in}}%
\pgfpathcurveto{\pgfqpoint{0.808333in}{0.848804in}}{\pgfqpoint{0.802187in}{0.863642in}}{\pgfqpoint{0.791248in}{0.874581in}}%
\pgfpathcurveto{\pgfqpoint{0.780309in}{0.885520in}}{\pgfqpoint{0.765470in}{0.891667in}}{\pgfqpoint{0.750000in}{0.891667in}}%
\pgfpathcurveto{\pgfqpoint{0.734530in}{0.891667in}}{\pgfqpoint{0.719691in}{0.885520in}}{\pgfqpoint{0.708752in}{0.874581in}}%
\pgfpathcurveto{\pgfqpoint{0.697813in}{0.863642in}}{\pgfqpoint{0.691667in}{0.848804in}}{\pgfqpoint{0.691667in}{0.833333in}}%
\pgfpathcurveto{\pgfqpoint{0.691667in}{0.817863in}}{\pgfqpoint{0.697813in}{0.803025in}}{\pgfqpoint{0.708752in}{0.792085in}}%
\pgfpathcurveto{\pgfqpoint{0.719691in}{0.781146in}}{\pgfqpoint{0.734530in}{0.775000in}}{\pgfqpoint{0.750000in}{0.775000in}}%
\pgfpathclose%
\pgfpathmoveto{\pgfqpoint{0.750000in}{0.780833in}}%
\pgfpathcurveto{\pgfqpoint{0.750000in}{0.780833in}}{\pgfqpoint{0.736077in}{0.780833in}}{\pgfqpoint{0.722722in}{0.786365in}}%
\pgfpathcurveto{\pgfqpoint{0.712877in}{0.796210in}}{\pgfqpoint{0.703032in}{0.806055in}}{\pgfqpoint{0.697500in}{0.819410in}}%
\pgfpathcurveto{\pgfqpoint{0.697500in}{0.833333in}}{\pgfqpoint{0.697500in}{0.847256in}}{\pgfqpoint{0.703032in}{0.860611in}}%
\pgfpathcurveto{\pgfqpoint{0.712877in}{0.870456in}}{\pgfqpoint{0.722722in}{0.880302in}}{\pgfqpoint{0.736077in}{0.885833in}}%
\pgfpathcurveto{\pgfqpoint{0.750000in}{0.885833in}}{\pgfqpoint{0.763923in}{0.885833in}}{\pgfqpoint{0.777278in}{0.880302in}}%
\pgfpathcurveto{\pgfqpoint{0.787123in}{0.870456in}}{\pgfqpoint{0.796968in}{0.860611in}}{\pgfqpoint{0.802500in}{0.847256in}}%
\pgfpathcurveto{\pgfqpoint{0.802500in}{0.833333in}}{\pgfqpoint{0.802500in}{0.819410in}}{\pgfqpoint{0.796968in}{0.806055in}}%
\pgfpathcurveto{\pgfqpoint{0.787123in}{0.796210in}}{\pgfqpoint{0.777278in}{0.786365in}}{\pgfqpoint{0.763923in}{0.780833in}}%
\pgfpathclose%
\pgfpathmoveto{\pgfqpoint{0.916667in}{0.775000in}}%
\pgfpathcurveto{\pgfqpoint{0.932137in}{0.775000in}}{\pgfqpoint{0.946975in}{0.781146in}}{\pgfqpoint{0.957915in}{0.792085in}}%
\pgfpathcurveto{\pgfqpoint{0.968854in}{0.803025in}}{\pgfqpoint{0.975000in}{0.817863in}}{\pgfqpoint{0.975000in}{0.833333in}}%
\pgfpathcurveto{\pgfqpoint{0.975000in}{0.848804in}}{\pgfqpoint{0.968854in}{0.863642in}}{\pgfqpoint{0.957915in}{0.874581in}}%
\pgfpathcurveto{\pgfqpoint{0.946975in}{0.885520in}}{\pgfqpoint{0.932137in}{0.891667in}}{\pgfqpoint{0.916667in}{0.891667in}}%
\pgfpathcurveto{\pgfqpoint{0.901196in}{0.891667in}}{\pgfqpoint{0.886358in}{0.885520in}}{\pgfqpoint{0.875419in}{0.874581in}}%
\pgfpathcurveto{\pgfqpoint{0.864480in}{0.863642in}}{\pgfqpoint{0.858333in}{0.848804in}}{\pgfqpoint{0.858333in}{0.833333in}}%
\pgfpathcurveto{\pgfqpoint{0.858333in}{0.817863in}}{\pgfqpoint{0.864480in}{0.803025in}}{\pgfqpoint{0.875419in}{0.792085in}}%
\pgfpathcurveto{\pgfqpoint{0.886358in}{0.781146in}}{\pgfqpoint{0.901196in}{0.775000in}}{\pgfqpoint{0.916667in}{0.775000in}}%
\pgfpathclose%
\pgfpathmoveto{\pgfqpoint{0.916667in}{0.780833in}}%
\pgfpathcurveto{\pgfqpoint{0.916667in}{0.780833in}}{\pgfqpoint{0.902744in}{0.780833in}}{\pgfqpoint{0.889389in}{0.786365in}}%
\pgfpathcurveto{\pgfqpoint{0.879544in}{0.796210in}}{\pgfqpoint{0.869698in}{0.806055in}}{\pgfqpoint{0.864167in}{0.819410in}}%
\pgfpathcurveto{\pgfqpoint{0.864167in}{0.833333in}}{\pgfqpoint{0.864167in}{0.847256in}}{\pgfqpoint{0.869698in}{0.860611in}}%
\pgfpathcurveto{\pgfqpoint{0.879544in}{0.870456in}}{\pgfqpoint{0.889389in}{0.880302in}}{\pgfqpoint{0.902744in}{0.885833in}}%
\pgfpathcurveto{\pgfqpoint{0.916667in}{0.885833in}}{\pgfqpoint{0.930590in}{0.885833in}}{\pgfqpoint{0.943945in}{0.880302in}}%
\pgfpathcurveto{\pgfqpoint{0.953790in}{0.870456in}}{\pgfqpoint{0.963635in}{0.860611in}}{\pgfqpoint{0.969167in}{0.847256in}}%
\pgfpathcurveto{\pgfqpoint{0.969167in}{0.833333in}}{\pgfqpoint{0.969167in}{0.819410in}}{\pgfqpoint{0.963635in}{0.806055in}}%
\pgfpathcurveto{\pgfqpoint{0.953790in}{0.796210in}}{\pgfqpoint{0.943945in}{0.786365in}}{\pgfqpoint{0.930590in}{0.780833in}}%
\pgfpathclose%
\pgfpathmoveto{\pgfqpoint{0.000000in}{0.941667in}}%
\pgfpathcurveto{\pgfqpoint{0.015470in}{0.941667in}}{\pgfqpoint{0.030309in}{0.947813in}}{\pgfqpoint{0.041248in}{0.958752in}}%
\pgfpathcurveto{\pgfqpoint{0.052187in}{0.969691in}}{\pgfqpoint{0.058333in}{0.984530in}}{\pgfqpoint{0.058333in}{1.000000in}}%
\pgfpathcurveto{\pgfqpoint{0.058333in}{1.015470in}}{\pgfqpoint{0.052187in}{1.030309in}}{\pgfqpoint{0.041248in}{1.041248in}}%
\pgfpathcurveto{\pgfqpoint{0.030309in}{1.052187in}}{\pgfqpoint{0.015470in}{1.058333in}}{\pgfqpoint{0.000000in}{1.058333in}}%
\pgfpathcurveto{\pgfqpoint{-0.015470in}{1.058333in}}{\pgfqpoint{-0.030309in}{1.052187in}}{\pgfqpoint{-0.041248in}{1.041248in}}%
\pgfpathcurveto{\pgfqpoint{-0.052187in}{1.030309in}}{\pgfqpoint{-0.058333in}{1.015470in}}{\pgfqpoint{-0.058333in}{1.000000in}}%
\pgfpathcurveto{\pgfqpoint{-0.058333in}{0.984530in}}{\pgfqpoint{-0.052187in}{0.969691in}}{\pgfqpoint{-0.041248in}{0.958752in}}%
\pgfpathcurveto{\pgfqpoint{-0.030309in}{0.947813in}}{\pgfqpoint{-0.015470in}{0.941667in}}{\pgfqpoint{0.000000in}{0.941667in}}%
\pgfpathclose%
\pgfpathmoveto{\pgfqpoint{0.000000in}{0.947500in}}%
\pgfpathcurveto{\pgfqpoint{0.000000in}{0.947500in}}{\pgfqpoint{-0.013923in}{0.947500in}}{\pgfqpoint{-0.027278in}{0.953032in}}%
\pgfpathcurveto{\pgfqpoint{-0.037123in}{0.962877in}}{\pgfqpoint{-0.046968in}{0.972722in}}{\pgfqpoint{-0.052500in}{0.986077in}}%
\pgfpathcurveto{\pgfqpoint{-0.052500in}{1.000000in}}{\pgfqpoint{-0.052500in}{1.013923in}}{\pgfqpoint{-0.046968in}{1.027278in}}%
\pgfpathcurveto{\pgfqpoint{-0.037123in}{1.037123in}}{\pgfqpoint{-0.027278in}{1.046968in}}{\pgfqpoint{-0.013923in}{1.052500in}}%
\pgfpathcurveto{\pgfqpoint{0.000000in}{1.052500in}}{\pgfqpoint{0.013923in}{1.052500in}}{\pgfqpoint{0.027278in}{1.046968in}}%
\pgfpathcurveto{\pgfqpoint{0.037123in}{1.037123in}}{\pgfqpoint{0.046968in}{1.027278in}}{\pgfqpoint{0.052500in}{1.013923in}}%
\pgfpathcurveto{\pgfqpoint{0.052500in}{1.000000in}}{\pgfqpoint{0.052500in}{0.986077in}}{\pgfqpoint{0.046968in}{0.972722in}}%
\pgfpathcurveto{\pgfqpoint{0.037123in}{0.962877in}}{\pgfqpoint{0.027278in}{0.953032in}}{\pgfqpoint{0.013923in}{0.947500in}}%
\pgfpathclose%
\pgfpathmoveto{\pgfqpoint{0.166667in}{0.941667in}}%
\pgfpathcurveto{\pgfqpoint{0.182137in}{0.941667in}}{\pgfqpoint{0.196975in}{0.947813in}}{\pgfqpoint{0.207915in}{0.958752in}}%
\pgfpathcurveto{\pgfqpoint{0.218854in}{0.969691in}}{\pgfqpoint{0.225000in}{0.984530in}}{\pgfqpoint{0.225000in}{1.000000in}}%
\pgfpathcurveto{\pgfqpoint{0.225000in}{1.015470in}}{\pgfqpoint{0.218854in}{1.030309in}}{\pgfqpoint{0.207915in}{1.041248in}}%
\pgfpathcurveto{\pgfqpoint{0.196975in}{1.052187in}}{\pgfqpoint{0.182137in}{1.058333in}}{\pgfqpoint{0.166667in}{1.058333in}}%
\pgfpathcurveto{\pgfqpoint{0.151196in}{1.058333in}}{\pgfqpoint{0.136358in}{1.052187in}}{\pgfqpoint{0.125419in}{1.041248in}}%
\pgfpathcurveto{\pgfqpoint{0.114480in}{1.030309in}}{\pgfqpoint{0.108333in}{1.015470in}}{\pgfqpoint{0.108333in}{1.000000in}}%
\pgfpathcurveto{\pgfqpoint{0.108333in}{0.984530in}}{\pgfqpoint{0.114480in}{0.969691in}}{\pgfqpoint{0.125419in}{0.958752in}}%
\pgfpathcurveto{\pgfqpoint{0.136358in}{0.947813in}}{\pgfqpoint{0.151196in}{0.941667in}}{\pgfqpoint{0.166667in}{0.941667in}}%
\pgfpathclose%
\pgfpathmoveto{\pgfqpoint{0.166667in}{0.947500in}}%
\pgfpathcurveto{\pgfqpoint{0.166667in}{0.947500in}}{\pgfqpoint{0.152744in}{0.947500in}}{\pgfqpoint{0.139389in}{0.953032in}}%
\pgfpathcurveto{\pgfqpoint{0.129544in}{0.962877in}}{\pgfqpoint{0.119698in}{0.972722in}}{\pgfqpoint{0.114167in}{0.986077in}}%
\pgfpathcurveto{\pgfqpoint{0.114167in}{1.000000in}}{\pgfqpoint{0.114167in}{1.013923in}}{\pgfqpoint{0.119698in}{1.027278in}}%
\pgfpathcurveto{\pgfqpoint{0.129544in}{1.037123in}}{\pgfqpoint{0.139389in}{1.046968in}}{\pgfqpoint{0.152744in}{1.052500in}}%
\pgfpathcurveto{\pgfqpoint{0.166667in}{1.052500in}}{\pgfqpoint{0.180590in}{1.052500in}}{\pgfqpoint{0.193945in}{1.046968in}}%
\pgfpathcurveto{\pgfqpoint{0.203790in}{1.037123in}}{\pgfqpoint{0.213635in}{1.027278in}}{\pgfqpoint{0.219167in}{1.013923in}}%
\pgfpathcurveto{\pgfqpoint{0.219167in}{1.000000in}}{\pgfqpoint{0.219167in}{0.986077in}}{\pgfqpoint{0.213635in}{0.972722in}}%
\pgfpathcurveto{\pgfqpoint{0.203790in}{0.962877in}}{\pgfqpoint{0.193945in}{0.953032in}}{\pgfqpoint{0.180590in}{0.947500in}}%
\pgfpathclose%
\pgfpathmoveto{\pgfqpoint{0.333333in}{0.941667in}}%
\pgfpathcurveto{\pgfqpoint{0.348804in}{0.941667in}}{\pgfqpoint{0.363642in}{0.947813in}}{\pgfqpoint{0.374581in}{0.958752in}}%
\pgfpathcurveto{\pgfqpoint{0.385520in}{0.969691in}}{\pgfqpoint{0.391667in}{0.984530in}}{\pgfqpoint{0.391667in}{1.000000in}}%
\pgfpathcurveto{\pgfqpoint{0.391667in}{1.015470in}}{\pgfqpoint{0.385520in}{1.030309in}}{\pgfqpoint{0.374581in}{1.041248in}}%
\pgfpathcurveto{\pgfqpoint{0.363642in}{1.052187in}}{\pgfqpoint{0.348804in}{1.058333in}}{\pgfqpoint{0.333333in}{1.058333in}}%
\pgfpathcurveto{\pgfqpoint{0.317863in}{1.058333in}}{\pgfqpoint{0.303025in}{1.052187in}}{\pgfqpoint{0.292085in}{1.041248in}}%
\pgfpathcurveto{\pgfqpoint{0.281146in}{1.030309in}}{\pgfqpoint{0.275000in}{1.015470in}}{\pgfqpoint{0.275000in}{1.000000in}}%
\pgfpathcurveto{\pgfqpoint{0.275000in}{0.984530in}}{\pgfqpoint{0.281146in}{0.969691in}}{\pgfqpoint{0.292085in}{0.958752in}}%
\pgfpathcurveto{\pgfqpoint{0.303025in}{0.947813in}}{\pgfqpoint{0.317863in}{0.941667in}}{\pgfqpoint{0.333333in}{0.941667in}}%
\pgfpathclose%
\pgfpathmoveto{\pgfqpoint{0.333333in}{0.947500in}}%
\pgfpathcurveto{\pgfqpoint{0.333333in}{0.947500in}}{\pgfqpoint{0.319410in}{0.947500in}}{\pgfqpoint{0.306055in}{0.953032in}}%
\pgfpathcurveto{\pgfqpoint{0.296210in}{0.962877in}}{\pgfqpoint{0.286365in}{0.972722in}}{\pgfqpoint{0.280833in}{0.986077in}}%
\pgfpathcurveto{\pgfqpoint{0.280833in}{1.000000in}}{\pgfqpoint{0.280833in}{1.013923in}}{\pgfqpoint{0.286365in}{1.027278in}}%
\pgfpathcurveto{\pgfqpoint{0.296210in}{1.037123in}}{\pgfqpoint{0.306055in}{1.046968in}}{\pgfqpoint{0.319410in}{1.052500in}}%
\pgfpathcurveto{\pgfqpoint{0.333333in}{1.052500in}}{\pgfqpoint{0.347256in}{1.052500in}}{\pgfqpoint{0.360611in}{1.046968in}}%
\pgfpathcurveto{\pgfqpoint{0.370456in}{1.037123in}}{\pgfqpoint{0.380302in}{1.027278in}}{\pgfqpoint{0.385833in}{1.013923in}}%
\pgfpathcurveto{\pgfqpoint{0.385833in}{1.000000in}}{\pgfqpoint{0.385833in}{0.986077in}}{\pgfqpoint{0.380302in}{0.972722in}}%
\pgfpathcurveto{\pgfqpoint{0.370456in}{0.962877in}}{\pgfqpoint{0.360611in}{0.953032in}}{\pgfqpoint{0.347256in}{0.947500in}}%
\pgfpathclose%
\pgfpathmoveto{\pgfqpoint{0.500000in}{0.941667in}}%
\pgfpathcurveto{\pgfqpoint{0.515470in}{0.941667in}}{\pgfqpoint{0.530309in}{0.947813in}}{\pgfqpoint{0.541248in}{0.958752in}}%
\pgfpathcurveto{\pgfqpoint{0.552187in}{0.969691in}}{\pgfqpoint{0.558333in}{0.984530in}}{\pgfqpoint{0.558333in}{1.000000in}}%
\pgfpathcurveto{\pgfqpoint{0.558333in}{1.015470in}}{\pgfqpoint{0.552187in}{1.030309in}}{\pgfqpoint{0.541248in}{1.041248in}}%
\pgfpathcurveto{\pgfqpoint{0.530309in}{1.052187in}}{\pgfqpoint{0.515470in}{1.058333in}}{\pgfqpoint{0.500000in}{1.058333in}}%
\pgfpathcurveto{\pgfqpoint{0.484530in}{1.058333in}}{\pgfqpoint{0.469691in}{1.052187in}}{\pgfqpoint{0.458752in}{1.041248in}}%
\pgfpathcurveto{\pgfqpoint{0.447813in}{1.030309in}}{\pgfqpoint{0.441667in}{1.015470in}}{\pgfqpoint{0.441667in}{1.000000in}}%
\pgfpathcurveto{\pgfqpoint{0.441667in}{0.984530in}}{\pgfqpoint{0.447813in}{0.969691in}}{\pgfqpoint{0.458752in}{0.958752in}}%
\pgfpathcurveto{\pgfqpoint{0.469691in}{0.947813in}}{\pgfqpoint{0.484530in}{0.941667in}}{\pgfqpoint{0.500000in}{0.941667in}}%
\pgfpathclose%
\pgfpathmoveto{\pgfqpoint{0.500000in}{0.947500in}}%
\pgfpathcurveto{\pgfqpoint{0.500000in}{0.947500in}}{\pgfqpoint{0.486077in}{0.947500in}}{\pgfqpoint{0.472722in}{0.953032in}}%
\pgfpathcurveto{\pgfqpoint{0.462877in}{0.962877in}}{\pgfqpoint{0.453032in}{0.972722in}}{\pgfqpoint{0.447500in}{0.986077in}}%
\pgfpathcurveto{\pgfqpoint{0.447500in}{1.000000in}}{\pgfqpoint{0.447500in}{1.013923in}}{\pgfqpoint{0.453032in}{1.027278in}}%
\pgfpathcurveto{\pgfqpoint{0.462877in}{1.037123in}}{\pgfqpoint{0.472722in}{1.046968in}}{\pgfqpoint{0.486077in}{1.052500in}}%
\pgfpathcurveto{\pgfqpoint{0.500000in}{1.052500in}}{\pgfqpoint{0.513923in}{1.052500in}}{\pgfqpoint{0.527278in}{1.046968in}}%
\pgfpathcurveto{\pgfqpoint{0.537123in}{1.037123in}}{\pgfqpoint{0.546968in}{1.027278in}}{\pgfqpoint{0.552500in}{1.013923in}}%
\pgfpathcurveto{\pgfqpoint{0.552500in}{1.000000in}}{\pgfqpoint{0.552500in}{0.986077in}}{\pgfqpoint{0.546968in}{0.972722in}}%
\pgfpathcurveto{\pgfqpoint{0.537123in}{0.962877in}}{\pgfqpoint{0.527278in}{0.953032in}}{\pgfqpoint{0.513923in}{0.947500in}}%
\pgfpathclose%
\pgfpathmoveto{\pgfqpoint{0.666667in}{0.941667in}}%
\pgfpathcurveto{\pgfqpoint{0.682137in}{0.941667in}}{\pgfqpoint{0.696975in}{0.947813in}}{\pgfqpoint{0.707915in}{0.958752in}}%
\pgfpathcurveto{\pgfqpoint{0.718854in}{0.969691in}}{\pgfqpoint{0.725000in}{0.984530in}}{\pgfqpoint{0.725000in}{1.000000in}}%
\pgfpathcurveto{\pgfqpoint{0.725000in}{1.015470in}}{\pgfqpoint{0.718854in}{1.030309in}}{\pgfqpoint{0.707915in}{1.041248in}}%
\pgfpathcurveto{\pgfqpoint{0.696975in}{1.052187in}}{\pgfqpoint{0.682137in}{1.058333in}}{\pgfqpoint{0.666667in}{1.058333in}}%
\pgfpathcurveto{\pgfqpoint{0.651196in}{1.058333in}}{\pgfqpoint{0.636358in}{1.052187in}}{\pgfqpoint{0.625419in}{1.041248in}}%
\pgfpathcurveto{\pgfqpoint{0.614480in}{1.030309in}}{\pgfqpoint{0.608333in}{1.015470in}}{\pgfqpoint{0.608333in}{1.000000in}}%
\pgfpathcurveto{\pgfqpoint{0.608333in}{0.984530in}}{\pgfqpoint{0.614480in}{0.969691in}}{\pgfqpoint{0.625419in}{0.958752in}}%
\pgfpathcurveto{\pgfqpoint{0.636358in}{0.947813in}}{\pgfqpoint{0.651196in}{0.941667in}}{\pgfqpoint{0.666667in}{0.941667in}}%
\pgfpathclose%
\pgfpathmoveto{\pgfqpoint{0.666667in}{0.947500in}}%
\pgfpathcurveto{\pgfqpoint{0.666667in}{0.947500in}}{\pgfqpoint{0.652744in}{0.947500in}}{\pgfqpoint{0.639389in}{0.953032in}}%
\pgfpathcurveto{\pgfqpoint{0.629544in}{0.962877in}}{\pgfqpoint{0.619698in}{0.972722in}}{\pgfqpoint{0.614167in}{0.986077in}}%
\pgfpathcurveto{\pgfqpoint{0.614167in}{1.000000in}}{\pgfqpoint{0.614167in}{1.013923in}}{\pgfqpoint{0.619698in}{1.027278in}}%
\pgfpathcurveto{\pgfqpoint{0.629544in}{1.037123in}}{\pgfqpoint{0.639389in}{1.046968in}}{\pgfqpoint{0.652744in}{1.052500in}}%
\pgfpathcurveto{\pgfqpoint{0.666667in}{1.052500in}}{\pgfqpoint{0.680590in}{1.052500in}}{\pgfqpoint{0.693945in}{1.046968in}}%
\pgfpathcurveto{\pgfqpoint{0.703790in}{1.037123in}}{\pgfqpoint{0.713635in}{1.027278in}}{\pgfqpoint{0.719167in}{1.013923in}}%
\pgfpathcurveto{\pgfqpoint{0.719167in}{1.000000in}}{\pgfqpoint{0.719167in}{0.986077in}}{\pgfqpoint{0.713635in}{0.972722in}}%
\pgfpathcurveto{\pgfqpoint{0.703790in}{0.962877in}}{\pgfqpoint{0.693945in}{0.953032in}}{\pgfqpoint{0.680590in}{0.947500in}}%
\pgfpathclose%
\pgfpathmoveto{\pgfqpoint{0.833333in}{0.941667in}}%
\pgfpathcurveto{\pgfqpoint{0.848804in}{0.941667in}}{\pgfqpoint{0.863642in}{0.947813in}}{\pgfqpoint{0.874581in}{0.958752in}}%
\pgfpathcurveto{\pgfqpoint{0.885520in}{0.969691in}}{\pgfqpoint{0.891667in}{0.984530in}}{\pgfqpoint{0.891667in}{1.000000in}}%
\pgfpathcurveto{\pgfqpoint{0.891667in}{1.015470in}}{\pgfqpoint{0.885520in}{1.030309in}}{\pgfqpoint{0.874581in}{1.041248in}}%
\pgfpathcurveto{\pgfqpoint{0.863642in}{1.052187in}}{\pgfqpoint{0.848804in}{1.058333in}}{\pgfqpoint{0.833333in}{1.058333in}}%
\pgfpathcurveto{\pgfqpoint{0.817863in}{1.058333in}}{\pgfqpoint{0.803025in}{1.052187in}}{\pgfqpoint{0.792085in}{1.041248in}}%
\pgfpathcurveto{\pgfqpoint{0.781146in}{1.030309in}}{\pgfqpoint{0.775000in}{1.015470in}}{\pgfqpoint{0.775000in}{1.000000in}}%
\pgfpathcurveto{\pgfqpoint{0.775000in}{0.984530in}}{\pgfqpoint{0.781146in}{0.969691in}}{\pgfqpoint{0.792085in}{0.958752in}}%
\pgfpathcurveto{\pgfqpoint{0.803025in}{0.947813in}}{\pgfqpoint{0.817863in}{0.941667in}}{\pgfqpoint{0.833333in}{0.941667in}}%
\pgfpathclose%
\pgfpathmoveto{\pgfqpoint{0.833333in}{0.947500in}}%
\pgfpathcurveto{\pgfqpoint{0.833333in}{0.947500in}}{\pgfqpoint{0.819410in}{0.947500in}}{\pgfqpoint{0.806055in}{0.953032in}}%
\pgfpathcurveto{\pgfqpoint{0.796210in}{0.962877in}}{\pgfqpoint{0.786365in}{0.972722in}}{\pgfqpoint{0.780833in}{0.986077in}}%
\pgfpathcurveto{\pgfqpoint{0.780833in}{1.000000in}}{\pgfqpoint{0.780833in}{1.013923in}}{\pgfqpoint{0.786365in}{1.027278in}}%
\pgfpathcurveto{\pgfqpoint{0.796210in}{1.037123in}}{\pgfqpoint{0.806055in}{1.046968in}}{\pgfqpoint{0.819410in}{1.052500in}}%
\pgfpathcurveto{\pgfqpoint{0.833333in}{1.052500in}}{\pgfqpoint{0.847256in}{1.052500in}}{\pgfqpoint{0.860611in}{1.046968in}}%
\pgfpathcurveto{\pgfqpoint{0.870456in}{1.037123in}}{\pgfqpoint{0.880302in}{1.027278in}}{\pgfqpoint{0.885833in}{1.013923in}}%
\pgfpathcurveto{\pgfqpoint{0.885833in}{1.000000in}}{\pgfqpoint{0.885833in}{0.986077in}}{\pgfqpoint{0.880302in}{0.972722in}}%
\pgfpathcurveto{\pgfqpoint{0.870456in}{0.962877in}}{\pgfqpoint{0.860611in}{0.953032in}}{\pgfqpoint{0.847256in}{0.947500in}}%
\pgfpathclose%
\pgfpathmoveto{\pgfqpoint{1.000000in}{0.941667in}}%
\pgfpathcurveto{\pgfqpoint{1.015470in}{0.941667in}}{\pgfqpoint{1.030309in}{0.947813in}}{\pgfqpoint{1.041248in}{0.958752in}}%
\pgfpathcurveto{\pgfqpoint{1.052187in}{0.969691in}}{\pgfqpoint{1.058333in}{0.984530in}}{\pgfqpoint{1.058333in}{1.000000in}}%
\pgfpathcurveto{\pgfqpoint{1.058333in}{1.015470in}}{\pgfqpoint{1.052187in}{1.030309in}}{\pgfqpoint{1.041248in}{1.041248in}}%
\pgfpathcurveto{\pgfqpoint{1.030309in}{1.052187in}}{\pgfqpoint{1.015470in}{1.058333in}}{\pgfqpoint{1.000000in}{1.058333in}}%
\pgfpathcurveto{\pgfqpoint{0.984530in}{1.058333in}}{\pgfqpoint{0.969691in}{1.052187in}}{\pgfqpoint{0.958752in}{1.041248in}}%
\pgfpathcurveto{\pgfqpoint{0.947813in}{1.030309in}}{\pgfqpoint{0.941667in}{1.015470in}}{\pgfqpoint{0.941667in}{1.000000in}}%
\pgfpathcurveto{\pgfqpoint{0.941667in}{0.984530in}}{\pgfqpoint{0.947813in}{0.969691in}}{\pgfqpoint{0.958752in}{0.958752in}}%
\pgfpathcurveto{\pgfqpoint{0.969691in}{0.947813in}}{\pgfqpoint{0.984530in}{0.941667in}}{\pgfqpoint{1.000000in}{0.941667in}}%
\pgfpathclose%
\pgfpathmoveto{\pgfqpoint{1.000000in}{0.947500in}}%
\pgfpathcurveto{\pgfqpoint{1.000000in}{0.947500in}}{\pgfqpoint{0.986077in}{0.947500in}}{\pgfqpoint{0.972722in}{0.953032in}}%
\pgfpathcurveto{\pgfqpoint{0.962877in}{0.962877in}}{\pgfqpoint{0.953032in}{0.972722in}}{\pgfqpoint{0.947500in}{0.986077in}}%
\pgfpathcurveto{\pgfqpoint{0.947500in}{1.000000in}}{\pgfqpoint{0.947500in}{1.013923in}}{\pgfqpoint{0.953032in}{1.027278in}}%
\pgfpathcurveto{\pgfqpoint{0.962877in}{1.037123in}}{\pgfqpoint{0.972722in}{1.046968in}}{\pgfqpoint{0.986077in}{1.052500in}}%
\pgfpathcurveto{\pgfqpoint{1.000000in}{1.052500in}}{\pgfqpoint{1.013923in}{1.052500in}}{\pgfqpoint{1.027278in}{1.046968in}}%
\pgfpathcurveto{\pgfqpoint{1.037123in}{1.037123in}}{\pgfqpoint{1.046968in}{1.027278in}}{\pgfqpoint{1.052500in}{1.013923in}}%
\pgfpathcurveto{\pgfqpoint{1.052500in}{1.000000in}}{\pgfqpoint{1.052500in}{0.986077in}}{\pgfqpoint{1.046968in}{0.972722in}}%
\pgfpathcurveto{\pgfqpoint{1.037123in}{0.962877in}}{\pgfqpoint{1.027278in}{0.953032in}}{\pgfqpoint{1.013923in}{0.947500in}}%
\pgfpathclose%
\pgfusepath{stroke}%
\end{pgfscope}%
}%
\pgfsys@transformshift{2.808038in}{2.630003in}%
\pgfsys@useobject{currentpattern}{}%
\pgfsys@transformshift{1in}{0in}%
\pgfsys@transformshift{-1in}{0in}%
\pgfsys@transformshift{0in}{1in}%
\end{pgfscope}%
\begin{pgfscope}%
\pgfpathrectangle{\pgfqpoint{0.870538in}{0.637495in}}{\pgfqpoint{9.300000in}{9.060000in}}%
\pgfusepath{clip}%
\pgfsetbuttcap%
\pgfsetmiterjoin%
\definecolor{currentfill}{rgb}{0.549020,0.337255,0.294118}%
\pgfsetfillcolor{currentfill}%
\pgfsetfillopacity{0.990000}%
\pgfsetlinewidth{0.000000pt}%
\definecolor{currentstroke}{rgb}{0.000000,0.000000,0.000000}%
\pgfsetstrokecolor{currentstroke}%
\pgfsetstrokeopacity{0.990000}%
\pgfsetdash{}{0pt}%
\pgfpathmoveto{\pgfqpoint{4.358038in}{1.677933in}}%
\pgfpathlineto{\pgfqpoint{5.133038in}{1.677933in}}%
\pgfpathlineto{\pgfqpoint{5.133038in}{3.454505in}}%
\pgfpathlineto{\pgfqpoint{4.358038in}{3.454505in}}%
\pgfpathclose%
\pgfusepath{fill}%
\end{pgfscope}%
\begin{pgfscope}%
\pgfsetbuttcap%
\pgfsetmiterjoin%
\definecolor{currentfill}{rgb}{0.549020,0.337255,0.294118}%
\pgfsetfillcolor{currentfill}%
\pgfsetfillopacity{0.990000}%
\pgfsetlinewidth{0.000000pt}%
\definecolor{currentstroke}{rgb}{0.000000,0.000000,0.000000}%
\pgfsetstrokecolor{currentstroke}%
\pgfsetstrokeopacity{0.990000}%
\pgfsetdash{}{0pt}%
\pgfpathrectangle{\pgfqpoint{0.870538in}{0.637495in}}{\pgfqpoint{9.300000in}{9.060000in}}%
\pgfusepath{clip}%
\pgfpathmoveto{\pgfqpoint{4.358038in}{1.677933in}}%
\pgfpathlineto{\pgfqpoint{5.133038in}{1.677933in}}%
\pgfpathlineto{\pgfqpoint{5.133038in}{3.454505in}}%
\pgfpathlineto{\pgfqpoint{4.358038in}{3.454505in}}%
\pgfpathclose%
\pgfusepath{clip}%
\pgfsys@defobject{currentpattern}{\pgfqpoint{0in}{0in}}{\pgfqpoint{1in}{1in}}{%
\begin{pgfscope}%
\pgfpathrectangle{\pgfqpoint{0in}{0in}}{\pgfqpoint{1in}{1in}}%
\pgfusepath{clip}%
\pgfpathmoveto{\pgfqpoint{0.000000in}{-0.058333in}}%
\pgfpathcurveto{\pgfqpoint{0.015470in}{-0.058333in}}{\pgfqpoint{0.030309in}{-0.052187in}}{\pgfqpoint{0.041248in}{-0.041248in}}%
\pgfpathcurveto{\pgfqpoint{0.052187in}{-0.030309in}}{\pgfqpoint{0.058333in}{-0.015470in}}{\pgfqpoint{0.058333in}{0.000000in}}%
\pgfpathcurveto{\pgfqpoint{0.058333in}{0.015470in}}{\pgfqpoint{0.052187in}{0.030309in}}{\pgfqpoint{0.041248in}{0.041248in}}%
\pgfpathcurveto{\pgfqpoint{0.030309in}{0.052187in}}{\pgfqpoint{0.015470in}{0.058333in}}{\pgfqpoint{0.000000in}{0.058333in}}%
\pgfpathcurveto{\pgfqpoint{-0.015470in}{0.058333in}}{\pgfqpoint{-0.030309in}{0.052187in}}{\pgfqpoint{-0.041248in}{0.041248in}}%
\pgfpathcurveto{\pgfqpoint{-0.052187in}{0.030309in}}{\pgfqpoint{-0.058333in}{0.015470in}}{\pgfqpoint{-0.058333in}{0.000000in}}%
\pgfpathcurveto{\pgfqpoint{-0.058333in}{-0.015470in}}{\pgfqpoint{-0.052187in}{-0.030309in}}{\pgfqpoint{-0.041248in}{-0.041248in}}%
\pgfpathcurveto{\pgfqpoint{-0.030309in}{-0.052187in}}{\pgfqpoint{-0.015470in}{-0.058333in}}{\pgfqpoint{0.000000in}{-0.058333in}}%
\pgfpathclose%
\pgfpathmoveto{\pgfqpoint{0.000000in}{-0.052500in}}%
\pgfpathcurveto{\pgfqpoint{0.000000in}{-0.052500in}}{\pgfqpoint{-0.013923in}{-0.052500in}}{\pgfqpoint{-0.027278in}{-0.046968in}}%
\pgfpathcurveto{\pgfqpoint{-0.037123in}{-0.037123in}}{\pgfqpoint{-0.046968in}{-0.027278in}}{\pgfqpoint{-0.052500in}{-0.013923in}}%
\pgfpathcurveto{\pgfqpoint{-0.052500in}{0.000000in}}{\pgfqpoint{-0.052500in}{0.013923in}}{\pgfqpoint{-0.046968in}{0.027278in}}%
\pgfpathcurveto{\pgfqpoint{-0.037123in}{0.037123in}}{\pgfqpoint{-0.027278in}{0.046968in}}{\pgfqpoint{-0.013923in}{0.052500in}}%
\pgfpathcurveto{\pgfqpoint{0.000000in}{0.052500in}}{\pgfqpoint{0.013923in}{0.052500in}}{\pgfqpoint{0.027278in}{0.046968in}}%
\pgfpathcurveto{\pgfqpoint{0.037123in}{0.037123in}}{\pgfqpoint{0.046968in}{0.027278in}}{\pgfqpoint{0.052500in}{0.013923in}}%
\pgfpathcurveto{\pgfqpoint{0.052500in}{0.000000in}}{\pgfqpoint{0.052500in}{-0.013923in}}{\pgfqpoint{0.046968in}{-0.027278in}}%
\pgfpathcurveto{\pgfqpoint{0.037123in}{-0.037123in}}{\pgfqpoint{0.027278in}{-0.046968in}}{\pgfqpoint{0.013923in}{-0.052500in}}%
\pgfpathclose%
\pgfpathmoveto{\pgfqpoint{0.166667in}{-0.058333in}}%
\pgfpathcurveto{\pgfqpoint{0.182137in}{-0.058333in}}{\pgfqpoint{0.196975in}{-0.052187in}}{\pgfqpoint{0.207915in}{-0.041248in}}%
\pgfpathcurveto{\pgfqpoint{0.218854in}{-0.030309in}}{\pgfqpoint{0.225000in}{-0.015470in}}{\pgfqpoint{0.225000in}{0.000000in}}%
\pgfpathcurveto{\pgfqpoint{0.225000in}{0.015470in}}{\pgfqpoint{0.218854in}{0.030309in}}{\pgfqpoint{0.207915in}{0.041248in}}%
\pgfpathcurveto{\pgfqpoint{0.196975in}{0.052187in}}{\pgfqpoint{0.182137in}{0.058333in}}{\pgfqpoint{0.166667in}{0.058333in}}%
\pgfpathcurveto{\pgfqpoint{0.151196in}{0.058333in}}{\pgfqpoint{0.136358in}{0.052187in}}{\pgfqpoint{0.125419in}{0.041248in}}%
\pgfpathcurveto{\pgfqpoint{0.114480in}{0.030309in}}{\pgfqpoint{0.108333in}{0.015470in}}{\pgfqpoint{0.108333in}{0.000000in}}%
\pgfpathcurveto{\pgfqpoint{0.108333in}{-0.015470in}}{\pgfqpoint{0.114480in}{-0.030309in}}{\pgfqpoint{0.125419in}{-0.041248in}}%
\pgfpathcurveto{\pgfqpoint{0.136358in}{-0.052187in}}{\pgfqpoint{0.151196in}{-0.058333in}}{\pgfqpoint{0.166667in}{-0.058333in}}%
\pgfpathclose%
\pgfpathmoveto{\pgfqpoint{0.166667in}{-0.052500in}}%
\pgfpathcurveto{\pgfqpoint{0.166667in}{-0.052500in}}{\pgfqpoint{0.152744in}{-0.052500in}}{\pgfqpoint{0.139389in}{-0.046968in}}%
\pgfpathcurveto{\pgfqpoint{0.129544in}{-0.037123in}}{\pgfqpoint{0.119698in}{-0.027278in}}{\pgfqpoint{0.114167in}{-0.013923in}}%
\pgfpathcurveto{\pgfqpoint{0.114167in}{0.000000in}}{\pgfqpoint{0.114167in}{0.013923in}}{\pgfqpoint{0.119698in}{0.027278in}}%
\pgfpathcurveto{\pgfqpoint{0.129544in}{0.037123in}}{\pgfqpoint{0.139389in}{0.046968in}}{\pgfqpoint{0.152744in}{0.052500in}}%
\pgfpathcurveto{\pgfqpoint{0.166667in}{0.052500in}}{\pgfqpoint{0.180590in}{0.052500in}}{\pgfqpoint{0.193945in}{0.046968in}}%
\pgfpathcurveto{\pgfqpoint{0.203790in}{0.037123in}}{\pgfqpoint{0.213635in}{0.027278in}}{\pgfqpoint{0.219167in}{0.013923in}}%
\pgfpathcurveto{\pgfqpoint{0.219167in}{0.000000in}}{\pgfqpoint{0.219167in}{-0.013923in}}{\pgfqpoint{0.213635in}{-0.027278in}}%
\pgfpathcurveto{\pgfqpoint{0.203790in}{-0.037123in}}{\pgfqpoint{0.193945in}{-0.046968in}}{\pgfqpoint{0.180590in}{-0.052500in}}%
\pgfpathclose%
\pgfpathmoveto{\pgfqpoint{0.333333in}{-0.058333in}}%
\pgfpathcurveto{\pgfqpoint{0.348804in}{-0.058333in}}{\pgfqpoint{0.363642in}{-0.052187in}}{\pgfqpoint{0.374581in}{-0.041248in}}%
\pgfpathcurveto{\pgfqpoint{0.385520in}{-0.030309in}}{\pgfqpoint{0.391667in}{-0.015470in}}{\pgfqpoint{0.391667in}{0.000000in}}%
\pgfpathcurveto{\pgfqpoint{0.391667in}{0.015470in}}{\pgfqpoint{0.385520in}{0.030309in}}{\pgfqpoint{0.374581in}{0.041248in}}%
\pgfpathcurveto{\pgfqpoint{0.363642in}{0.052187in}}{\pgfqpoint{0.348804in}{0.058333in}}{\pgfqpoint{0.333333in}{0.058333in}}%
\pgfpathcurveto{\pgfqpoint{0.317863in}{0.058333in}}{\pgfqpoint{0.303025in}{0.052187in}}{\pgfqpoint{0.292085in}{0.041248in}}%
\pgfpathcurveto{\pgfqpoint{0.281146in}{0.030309in}}{\pgfqpoint{0.275000in}{0.015470in}}{\pgfqpoint{0.275000in}{0.000000in}}%
\pgfpathcurveto{\pgfqpoint{0.275000in}{-0.015470in}}{\pgfqpoint{0.281146in}{-0.030309in}}{\pgfqpoint{0.292085in}{-0.041248in}}%
\pgfpathcurveto{\pgfqpoint{0.303025in}{-0.052187in}}{\pgfqpoint{0.317863in}{-0.058333in}}{\pgfqpoint{0.333333in}{-0.058333in}}%
\pgfpathclose%
\pgfpathmoveto{\pgfqpoint{0.333333in}{-0.052500in}}%
\pgfpathcurveto{\pgfqpoint{0.333333in}{-0.052500in}}{\pgfqpoint{0.319410in}{-0.052500in}}{\pgfqpoint{0.306055in}{-0.046968in}}%
\pgfpathcurveto{\pgfqpoint{0.296210in}{-0.037123in}}{\pgfqpoint{0.286365in}{-0.027278in}}{\pgfqpoint{0.280833in}{-0.013923in}}%
\pgfpathcurveto{\pgfqpoint{0.280833in}{0.000000in}}{\pgfqpoint{0.280833in}{0.013923in}}{\pgfqpoint{0.286365in}{0.027278in}}%
\pgfpathcurveto{\pgfqpoint{0.296210in}{0.037123in}}{\pgfqpoint{0.306055in}{0.046968in}}{\pgfqpoint{0.319410in}{0.052500in}}%
\pgfpathcurveto{\pgfqpoint{0.333333in}{0.052500in}}{\pgfqpoint{0.347256in}{0.052500in}}{\pgfqpoint{0.360611in}{0.046968in}}%
\pgfpathcurveto{\pgfqpoint{0.370456in}{0.037123in}}{\pgfqpoint{0.380302in}{0.027278in}}{\pgfqpoint{0.385833in}{0.013923in}}%
\pgfpathcurveto{\pgfqpoint{0.385833in}{0.000000in}}{\pgfqpoint{0.385833in}{-0.013923in}}{\pgfqpoint{0.380302in}{-0.027278in}}%
\pgfpathcurveto{\pgfqpoint{0.370456in}{-0.037123in}}{\pgfqpoint{0.360611in}{-0.046968in}}{\pgfqpoint{0.347256in}{-0.052500in}}%
\pgfpathclose%
\pgfpathmoveto{\pgfqpoint{0.500000in}{-0.058333in}}%
\pgfpathcurveto{\pgfqpoint{0.515470in}{-0.058333in}}{\pgfqpoint{0.530309in}{-0.052187in}}{\pgfqpoint{0.541248in}{-0.041248in}}%
\pgfpathcurveto{\pgfqpoint{0.552187in}{-0.030309in}}{\pgfqpoint{0.558333in}{-0.015470in}}{\pgfqpoint{0.558333in}{0.000000in}}%
\pgfpathcurveto{\pgfqpoint{0.558333in}{0.015470in}}{\pgfqpoint{0.552187in}{0.030309in}}{\pgfqpoint{0.541248in}{0.041248in}}%
\pgfpathcurveto{\pgfqpoint{0.530309in}{0.052187in}}{\pgfqpoint{0.515470in}{0.058333in}}{\pgfqpoint{0.500000in}{0.058333in}}%
\pgfpathcurveto{\pgfqpoint{0.484530in}{0.058333in}}{\pgfqpoint{0.469691in}{0.052187in}}{\pgfqpoint{0.458752in}{0.041248in}}%
\pgfpathcurveto{\pgfqpoint{0.447813in}{0.030309in}}{\pgfqpoint{0.441667in}{0.015470in}}{\pgfqpoint{0.441667in}{0.000000in}}%
\pgfpathcurveto{\pgfqpoint{0.441667in}{-0.015470in}}{\pgfqpoint{0.447813in}{-0.030309in}}{\pgfqpoint{0.458752in}{-0.041248in}}%
\pgfpathcurveto{\pgfqpoint{0.469691in}{-0.052187in}}{\pgfqpoint{0.484530in}{-0.058333in}}{\pgfqpoint{0.500000in}{-0.058333in}}%
\pgfpathclose%
\pgfpathmoveto{\pgfqpoint{0.500000in}{-0.052500in}}%
\pgfpathcurveto{\pgfqpoint{0.500000in}{-0.052500in}}{\pgfqpoint{0.486077in}{-0.052500in}}{\pgfqpoint{0.472722in}{-0.046968in}}%
\pgfpathcurveto{\pgfqpoint{0.462877in}{-0.037123in}}{\pgfqpoint{0.453032in}{-0.027278in}}{\pgfqpoint{0.447500in}{-0.013923in}}%
\pgfpathcurveto{\pgfqpoint{0.447500in}{0.000000in}}{\pgfqpoint{0.447500in}{0.013923in}}{\pgfqpoint{0.453032in}{0.027278in}}%
\pgfpathcurveto{\pgfqpoint{0.462877in}{0.037123in}}{\pgfqpoint{0.472722in}{0.046968in}}{\pgfqpoint{0.486077in}{0.052500in}}%
\pgfpathcurveto{\pgfqpoint{0.500000in}{0.052500in}}{\pgfqpoint{0.513923in}{0.052500in}}{\pgfqpoint{0.527278in}{0.046968in}}%
\pgfpathcurveto{\pgfqpoint{0.537123in}{0.037123in}}{\pgfqpoint{0.546968in}{0.027278in}}{\pgfqpoint{0.552500in}{0.013923in}}%
\pgfpathcurveto{\pgfqpoint{0.552500in}{0.000000in}}{\pgfqpoint{0.552500in}{-0.013923in}}{\pgfqpoint{0.546968in}{-0.027278in}}%
\pgfpathcurveto{\pgfqpoint{0.537123in}{-0.037123in}}{\pgfqpoint{0.527278in}{-0.046968in}}{\pgfqpoint{0.513923in}{-0.052500in}}%
\pgfpathclose%
\pgfpathmoveto{\pgfqpoint{0.666667in}{-0.058333in}}%
\pgfpathcurveto{\pgfqpoint{0.682137in}{-0.058333in}}{\pgfqpoint{0.696975in}{-0.052187in}}{\pgfqpoint{0.707915in}{-0.041248in}}%
\pgfpathcurveto{\pgfqpoint{0.718854in}{-0.030309in}}{\pgfqpoint{0.725000in}{-0.015470in}}{\pgfqpoint{0.725000in}{0.000000in}}%
\pgfpathcurveto{\pgfqpoint{0.725000in}{0.015470in}}{\pgfqpoint{0.718854in}{0.030309in}}{\pgfqpoint{0.707915in}{0.041248in}}%
\pgfpathcurveto{\pgfqpoint{0.696975in}{0.052187in}}{\pgfqpoint{0.682137in}{0.058333in}}{\pgfqpoint{0.666667in}{0.058333in}}%
\pgfpathcurveto{\pgfqpoint{0.651196in}{0.058333in}}{\pgfqpoint{0.636358in}{0.052187in}}{\pgfqpoint{0.625419in}{0.041248in}}%
\pgfpathcurveto{\pgfqpoint{0.614480in}{0.030309in}}{\pgfqpoint{0.608333in}{0.015470in}}{\pgfqpoint{0.608333in}{0.000000in}}%
\pgfpathcurveto{\pgfqpoint{0.608333in}{-0.015470in}}{\pgfqpoint{0.614480in}{-0.030309in}}{\pgfqpoint{0.625419in}{-0.041248in}}%
\pgfpathcurveto{\pgfqpoint{0.636358in}{-0.052187in}}{\pgfqpoint{0.651196in}{-0.058333in}}{\pgfqpoint{0.666667in}{-0.058333in}}%
\pgfpathclose%
\pgfpathmoveto{\pgfqpoint{0.666667in}{-0.052500in}}%
\pgfpathcurveto{\pgfqpoint{0.666667in}{-0.052500in}}{\pgfqpoint{0.652744in}{-0.052500in}}{\pgfqpoint{0.639389in}{-0.046968in}}%
\pgfpathcurveto{\pgfqpoint{0.629544in}{-0.037123in}}{\pgfqpoint{0.619698in}{-0.027278in}}{\pgfqpoint{0.614167in}{-0.013923in}}%
\pgfpathcurveto{\pgfqpoint{0.614167in}{0.000000in}}{\pgfqpoint{0.614167in}{0.013923in}}{\pgfqpoint{0.619698in}{0.027278in}}%
\pgfpathcurveto{\pgfqpoint{0.629544in}{0.037123in}}{\pgfqpoint{0.639389in}{0.046968in}}{\pgfqpoint{0.652744in}{0.052500in}}%
\pgfpathcurveto{\pgfqpoint{0.666667in}{0.052500in}}{\pgfqpoint{0.680590in}{0.052500in}}{\pgfqpoint{0.693945in}{0.046968in}}%
\pgfpathcurveto{\pgfqpoint{0.703790in}{0.037123in}}{\pgfqpoint{0.713635in}{0.027278in}}{\pgfqpoint{0.719167in}{0.013923in}}%
\pgfpathcurveto{\pgfqpoint{0.719167in}{0.000000in}}{\pgfqpoint{0.719167in}{-0.013923in}}{\pgfqpoint{0.713635in}{-0.027278in}}%
\pgfpathcurveto{\pgfqpoint{0.703790in}{-0.037123in}}{\pgfqpoint{0.693945in}{-0.046968in}}{\pgfqpoint{0.680590in}{-0.052500in}}%
\pgfpathclose%
\pgfpathmoveto{\pgfqpoint{0.833333in}{-0.058333in}}%
\pgfpathcurveto{\pgfqpoint{0.848804in}{-0.058333in}}{\pgfqpoint{0.863642in}{-0.052187in}}{\pgfqpoint{0.874581in}{-0.041248in}}%
\pgfpathcurveto{\pgfqpoint{0.885520in}{-0.030309in}}{\pgfqpoint{0.891667in}{-0.015470in}}{\pgfqpoint{0.891667in}{0.000000in}}%
\pgfpathcurveto{\pgfqpoint{0.891667in}{0.015470in}}{\pgfqpoint{0.885520in}{0.030309in}}{\pgfqpoint{0.874581in}{0.041248in}}%
\pgfpathcurveto{\pgfqpoint{0.863642in}{0.052187in}}{\pgfqpoint{0.848804in}{0.058333in}}{\pgfqpoint{0.833333in}{0.058333in}}%
\pgfpathcurveto{\pgfqpoint{0.817863in}{0.058333in}}{\pgfqpoint{0.803025in}{0.052187in}}{\pgfqpoint{0.792085in}{0.041248in}}%
\pgfpathcurveto{\pgfqpoint{0.781146in}{0.030309in}}{\pgfqpoint{0.775000in}{0.015470in}}{\pgfqpoint{0.775000in}{0.000000in}}%
\pgfpathcurveto{\pgfqpoint{0.775000in}{-0.015470in}}{\pgfqpoint{0.781146in}{-0.030309in}}{\pgfqpoint{0.792085in}{-0.041248in}}%
\pgfpathcurveto{\pgfqpoint{0.803025in}{-0.052187in}}{\pgfqpoint{0.817863in}{-0.058333in}}{\pgfqpoint{0.833333in}{-0.058333in}}%
\pgfpathclose%
\pgfpathmoveto{\pgfqpoint{0.833333in}{-0.052500in}}%
\pgfpathcurveto{\pgfqpoint{0.833333in}{-0.052500in}}{\pgfqpoint{0.819410in}{-0.052500in}}{\pgfqpoint{0.806055in}{-0.046968in}}%
\pgfpathcurveto{\pgfqpoint{0.796210in}{-0.037123in}}{\pgfqpoint{0.786365in}{-0.027278in}}{\pgfqpoint{0.780833in}{-0.013923in}}%
\pgfpathcurveto{\pgfqpoint{0.780833in}{0.000000in}}{\pgfqpoint{0.780833in}{0.013923in}}{\pgfqpoint{0.786365in}{0.027278in}}%
\pgfpathcurveto{\pgfqpoint{0.796210in}{0.037123in}}{\pgfqpoint{0.806055in}{0.046968in}}{\pgfqpoint{0.819410in}{0.052500in}}%
\pgfpathcurveto{\pgfqpoint{0.833333in}{0.052500in}}{\pgfqpoint{0.847256in}{0.052500in}}{\pgfqpoint{0.860611in}{0.046968in}}%
\pgfpathcurveto{\pgfqpoint{0.870456in}{0.037123in}}{\pgfqpoint{0.880302in}{0.027278in}}{\pgfqpoint{0.885833in}{0.013923in}}%
\pgfpathcurveto{\pgfqpoint{0.885833in}{0.000000in}}{\pgfqpoint{0.885833in}{-0.013923in}}{\pgfqpoint{0.880302in}{-0.027278in}}%
\pgfpathcurveto{\pgfqpoint{0.870456in}{-0.037123in}}{\pgfqpoint{0.860611in}{-0.046968in}}{\pgfqpoint{0.847256in}{-0.052500in}}%
\pgfpathclose%
\pgfpathmoveto{\pgfqpoint{1.000000in}{-0.058333in}}%
\pgfpathcurveto{\pgfqpoint{1.015470in}{-0.058333in}}{\pgfqpoint{1.030309in}{-0.052187in}}{\pgfqpoint{1.041248in}{-0.041248in}}%
\pgfpathcurveto{\pgfqpoint{1.052187in}{-0.030309in}}{\pgfqpoint{1.058333in}{-0.015470in}}{\pgfqpoint{1.058333in}{0.000000in}}%
\pgfpathcurveto{\pgfqpoint{1.058333in}{0.015470in}}{\pgfqpoint{1.052187in}{0.030309in}}{\pgfqpoint{1.041248in}{0.041248in}}%
\pgfpathcurveto{\pgfqpoint{1.030309in}{0.052187in}}{\pgfqpoint{1.015470in}{0.058333in}}{\pgfqpoint{1.000000in}{0.058333in}}%
\pgfpathcurveto{\pgfqpoint{0.984530in}{0.058333in}}{\pgfqpoint{0.969691in}{0.052187in}}{\pgfqpoint{0.958752in}{0.041248in}}%
\pgfpathcurveto{\pgfqpoint{0.947813in}{0.030309in}}{\pgfqpoint{0.941667in}{0.015470in}}{\pgfqpoint{0.941667in}{0.000000in}}%
\pgfpathcurveto{\pgfqpoint{0.941667in}{-0.015470in}}{\pgfqpoint{0.947813in}{-0.030309in}}{\pgfqpoint{0.958752in}{-0.041248in}}%
\pgfpathcurveto{\pgfqpoint{0.969691in}{-0.052187in}}{\pgfqpoint{0.984530in}{-0.058333in}}{\pgfqpoint{1.000000in}{-0.058333in}}%
\pgfpathclose%
\pgfpathmoveto{\pgfqpoint{1.000000in}{-0.052500in}}%
\pgfpathcurveto{\pgfqpoint{1.000000in}{-0.052500in}}{\pgfqpoint{0.986077in}{-0.052500in}}{\pgfqpoint{0.972722in}{-0.046968in}}%
\pgfpathcurveto{\pgfqpoint{0.962877in}{-0.037123in}}{\pgfqpoint{0.953032in}{-0.027278in}}{\pgfqpoint{0.947500in}{-0.013923in}}%
\pgfpathcurveto{\pgfqpoint{0.947500in}{0.000000in}}{\pgfqpoint{0.947500in}{0.013923in}}{\pgfqpoint{0.953032in}{0.027278in}}%
\pgfpathcurveto{\pgfqpoint{0.962877in}{0.037123in}}{\pgfqpoint{0.972722in}{0.046968in}}{\pgfqpoint{0.986077in}{0.052500in}}%
\pgfpathcurveto{\pgfqpoint{1.000000in}{0.052500in}}{\pgfqpoint{1.013923in}{0.052500in}}{\pgfqpoint{1.027278in}{0.046968in}}%
\pgfpathcurveto{\pgfqpoint{1.037123in}{0.037123in}}{\pgfqpoint{1.046968in}{0.027278in}}{\pgfqpoint{1.052500in}{0.013923in}}%
\pgfpathcurveto{\pgfqpoint{1.052500in}{0.000000in}}{\pgfqpoint{1.052500in}{-0.013923in}}{\pgfqpoint{1.046968in}{-0.027278in}}%
\pgfpathcurveto{\pgfqpoint{1.037123in}{-0.037123in}}{\pgfqpoint{1.027278in}{-0.046968in}}{\pgfqpoint{1.013923in}{-0.052500in}}%
\pgfpathclose%
\pgfpathmoveto{\pgfqpoint{0.083333in}{0.108333in}}%
\pgfpathcurveto{\pgfqpoint{0.098804in}{0.108333in}}{\pgfqpoint{0.113642in}{0.114480in}}{\pgfqpoint{0.124581in}{0.125419in}}%
\pgfpathcurveto{\pgfqpoint{0.135520in}{0.136358in}}{\pgfqpoint{0.141667in}{0.151196in}}{\pgfqpoint{0.141667in}{0.166667in}}%
\pgfpathcurveto{\pgfqpoint{0.141667in}{0.182137in}}{\pgfqpoint{0.135520in}{0.196975in}}{\pgfqpoint{0.124581in}{0.207915in}}%
\pgfpathcurveto{\pgfqpoint{0.113642in}{0.218854in}}{\pgfqpoint{0.098804in}{0.225000in}}{\pgfqpoint{0.083333in}{0.225000in}}%
\pgfpathcurveto{\pgfqpoint{0.067863in}{0.225000in}}{\pgfqpoint{0.053025in}{0.218854in}}{\pgfqpoint{0.042085in}{0.207915in}}%
\pgfpathcurveto{\pgfqpoint{0.031146in}{0.196975in}}{\pgfqpoint{0.025000in}{0.182137in}}{\pgfqpoint{0.025000in}{0.166667in}}%
\pgfpathcurveto{\pgfqpoint{0.025000in}{0.151196in}}{\pgfqpoint{0.031146in}{0.136358in}}{\pgfqpoint{0.042085in}{0.125419in}}%
\pgfpathcurveto{\pgfqpoint{0.053025in}{0.114480in}}{\pgfqpoint{0.067863in}{0.108333in}}{\pgfqpoint{0.083333in}{0.108333in}}%
\pgfpathclose%
\pgfpathmoveto{\pgfqpoint{0.083333in}{0.114167in}}%
\pgfpathcurveto{\pgfqpoint{0.083333in}{0.114167in}}{\pgfqpoint{0.069410in}{0.114167in}}{\pgfqpoint{0.056055in}{0.119698in}}%
\pgfpathcurveto{\pgfqpoint{0.046210in}{0.129544in}}{\pgfqpoint{0.036365in}{0.139389in}}{\pgfqpoint{0.030833in}{0.152744in}}%
\pgfpathcurveto{\pgfqpoint{0.030833in}{0.166667in}}{\pgfqpoint{0.030833in}{0.180590in}}{\pgfqpoint{0.036365in}{0.193945in}}%
\pgfpathcurveto{\pgfqpoint{0.046210in}{0.203790in}}{\pgfqpoint{0.056055in}{0.213635in}}{\pgfqpoint{0.069410in}{0.219167in}}%
\pgfpathcurveto{\pgfqpoint{0.083333in}{0.219167in}}{\pgfqpoint{0.097256in}{0.219167in}}{\pgfqpoint{0.110611in}{0.213635in}}%
\pgfpathcurveto{\pgfqpoint{0.120456in}{0.203790in}}{\pgfqpoint{0.130302in}{0.193945in}}{\pgfqpoint{0.135833in}{0.180590in}}%
\pgfpathcurveto{\pgfqpoint{0.135833in}{0.166667in}}{\pgfqpoint{0.135833in}{0.152744in}}{\pgfqpoint{0.130302in}{0.139389in}}%
\pgfpathcurveto{\pgfqpoint{0.120456in}{0.129544in}}{\pgfqpoint{0.110611in}{0.119698in}}{\pgfqpoint{0.097256in}{0.114167in}}%
\pgfpathclose%
\pgfpathmoveto{\pgfqpoint{0.250000in}{0.108333in}}%
\pgfpathcurveto{\pgfqpoint{0.265470in}{0.108333in}}{\pgfqpoint{0.280309in}{0.114480in}}{\pgfqpoint{0.291248in}{0.125419in}}%
\pgfpathcurveto{\pgfqpoint{0.302187in}{0.136358in}}{\pgfqpoint{0.308333in}{0.151196in}}{\pgfqpoint{0.308333in}{0.166667in}}%
\pgfpathcurveto{\pgfqpoint{0.308333in}{0.182137in}}{\pgfqpoint{0.302187in}{0.196975in}}{\pgfqpoint{0.291248in}{0.207915in}}%
\pgfpathcurveto{\pgfqpoint{0.280309in}{0.218854in}}{\pgfqpoint{0.265470in}{0.225000in}}{\pgfqpoint{0.250000in}{0.225000in}}%
\pgfpathcurveto{\pgfqpoint{0.234530in}{0.225000in}}{\pgfqpoint{0.219691in}{0.218854in}}{\pgfqpoint{0.208752in}{0.207915in}}%
\pgfpathcurveto{\pgfqpoint{0.197813in}{0.196975in}}{\pgfqpoint{0.191667in}{0.182137in}}{\pgfqpoint{0.191667in}{0.166667in}}%
\pgfpathcurveto{\pgfqpoint{0.191667in}{0.151196in}}{\pgfqpoint{0.197813in}{0.136358in}}{\pgfqpoint{0.208752in}{0.125419in}}%
\pgfpathcurveto{\pgfqpoint{0.219691in}{0.114480in}}{\pgfqpoint{0.234530in}{0.108333in}}{\pgfqpoint{0.250000in}{0.108333in}}%
\pgfpathclose%
\pgfpathmoveto{\pgfqpoint{0.250000in}{0.114167in}}%
\pgfpathcurveto{\pgfqpoint{0.250000in}{0.114167in}}{\pgfqpoint{0.236077in}{0.114167in}}{\pgfqpoint{0.222722in}{0.119698in}}%
\pgfpathcurveto{\pgfqpoint{0.212877in}{0.129544in}}{\pgfqpoint{0.203032in}{0.139389in}}{\pgfqpoint{0.197500in}{0.152744in}}%
\pgfpathcurveto{\pgfqpoint{0.197500in}{0.166667in}}{\pgfqpoint{0.197500in}{0.180590in}}{\pgfqpoint{0.203032in}{0.193945in}}%
\pgfpathcurveto{\pgfqpoint{0.212877in}{0.203790in}}{\pgfqpoint{0.222722in}{0.213635in}}{\pgfqpoint{0.236077in}{0.219167in}}%
\pgfpathcurveto{\pgfqpoint{0.250000in}{0.219167in}}{\pgfqpoint{0.263923in}{0.219167in}}{\pgfqpoint{0.277278in}{0.213635in}}%
\pgfpathcurveto{\pgfqpoint{0.287123in}{0.203790in}}{\pgfqpoint{0.296968in}{0.193945in}}{\pgfqpoint{0.302500in}{0.180590in}}%
\pgfpathcurveto{\pgfqpoint{0.302500in}{0.166667in}}{\pgfqpoint{0.302500in}{0.152744in}}{\pgfqpoint{0.296968in}{0.139389in}}%
\pgfpathcurveto{\pgfqpoint{0.287123in}{0.129544in}}{\pgfqpoint{0.277278in}{0.119698in}}{\pgfqpoint{0.263923in}{0.114167in}}%
\pgfpathclose%
\pgfpathmoveto{\pgfqpoint{0.416667in}{0.108333in}}%
\pgfpathcurveto{\pgfqpoint{0.432137in}{0.108333in}}{\pgfqpoint{0.446975in}{0.114480in}}{\pgfqpoint{0.457915in}{0.125419in}}%
\pgfpathcurveto{\pgfqpoint{0.468854in}{0.136358in}}{\pgfqpoint{0.475000in}{0.151196in}}{\pgfqpoint{0.475000in}{0.166667in}}%
\pgfpathcurveto{\pgfqpoint{0.475000in}{0.182137in}}{\pgfqpoint{0.468854in}{0.196975in}}{\pgfqpoint{0.457915in}{0.207915in}}%
\pgfpathcurveto{\pgfqpoint{0.446975in}{0.218854in}}{\pgfqpoint{0.432137in}{0.225000in}}{\pgfqpoint{0.416667in}{0.225000in}}%
\pgfpathcurveto{\pgfqpoint{0.401196in}{0.225000in}}{\pgfqpoint{0.386358in}{0.218854in}}{\pgfqpoint{0.375419in}{0.207915in}}%
\pgfpathcurveto{\pgfqpoint{0.364480in}{0.196975in}}{\pgfqpoint{0.358333in}{0.182137in}}{\pgfqpoint{0.358333in}{0.166667in}}%
\pgfpathcurveto{\pgfqpoint{0.358333in}{0.151196in}}{\pgfqpoint{0.364480in}{0.136358in}}{\pgfqpoint{0.375419in}{0.125419in}}%
\pgfpathcurveto{\pgfqpoint{0.386358in}{0.114480in}}{\pgfqpoint{0.401196in}{0.108333in}}{\pgfqpoint{0.416667in}{0.108333in}}%
\pgfpathclose%
\pgfpathmoveto{\pgfqpoint{0.416667in}{0.114167in}}%
\pgfpathcurveto{\pgfqpoint{0.416667in}{0.114167in}}{\pgfqpoint{0.402744in}{0.114167in}}{\pgfqpoint{0.389389in}{0.119698in}}%
\pgfpathcurveto{\pgfqpoint{0.379544in}{0.129544in}}{\pgfqpoint{0.369698in}{0.139389in}}{\pgfqpoint{0.364167in}{0.152744in}}%
\pgfpathcurveto{\pgfqpoint{0.364167in}{0.166667in}}{\pgfqpoint{0.364167in}{0.180590in}}{\pgfqpoint{0.369698in}{0.193945in}}%
\pgfpathcurveto{\pgfqpoint{0.379544in}{0.203790in}}{\pgfqpoint{0.389389in}{0.213635in}}{\pgfqpoint{0.402744in}{0.219167in}}%
\pgfpathcurveto{\pgfqpoint{0.416667in}{0.219167in}}{\pgfqpoint{0.430590in}{0.219167in}}{\pgfqpoint{0.443945in}{0.213635in}}%
\pgfpathcurveto{\pgfqpoint{0.453790in}{0.203790in}}{\pgfqpoint{0.463635in}{0.193945in}}{\pgfqpoint{0.469167in}{0.180590in}}%
\pgfpathcurveto{\pgfqpoint{0.469167in}{0.166667in}}{\pgfqpoint{0.469167in}{0.152744in}}{\pgfqpoint{0.463635in}{0.139389in}}%
\pgfpathcurveto{\pgfqpoint{0.453790in}{0.129544in}}{\pgfqpoint{0.443945in}{0.119698in}}{\pgfqpoint{0.430590in}{0.114167in}}%
\pgfpathclose%
\pgfpathmoveto{\pgfqpoint{0.583333in}{0.108333in}}%
\pgfpathcurveto{\pgfqpoint{0.598804in}{0.108333in}}{\pgfqpoint{0.613642in}{0.114480in}}{\pgfqpoint{0.624581in}{0.125419in}}%
\pgfpathcurveto{\pgfqpoint{0.635520in}{0.136358in}}{\pgfqpoint{0.641667in}{0.151196in}}{\pgfqpoint{0.641667in}{0.166667in}}%
\pgfpathcurveto{\pgfqpoint{0.641667in}{0.182137in}}{\pgfqpoint{0.635520in}{0.196975in}}{\pgfqpoint{0.624581in}{0.207915in}}%
\pgfpathcurveto{\pgfqpoint{0.613642in}{0.218854in}}{\pgfqpoint{0.598804in}{0.225000in}}{\pgfqpoint{0.583333in}{0.225000in}}%
\pgfpathcurveto{\pgfqpoint{0.567863in}{0.225000in}}{\pgfqpoint{0.553025in}{0.218854in}}{\pgfqpoint{0.542085in}{0.207915in}}%
\pgfpathcurveto{\pgfqpoint{0.531146in}{0.196975in}}{\pgfqpoint{0.525000in}{0.182137in}}{\pgfqpoint{0.525000in}{0.166667in}}%
\pgfpathcurveto{\pgfqpoint{0.525000in}{0.151196in}}{\pgfqpoint{0.531146in}{0.136358in}}{\pgfqpoint{0.542085in}{0.125419in}}%
\pgfpathcurveto{\pgfqpoint{0.553025in}{0.114480in}}{\pgfqpoint{0.567863in}{0.108333in}}{\pgfqpoint{0.583333in}{0.108333in}}%
\pgfpathclose%
\pgfpathmoveto{\pgfqpoint{0.583333in}{0.114167in}}%
\pgfpathcurveto{\pgfqpoint{0.583333in}{0.114167in}}{\pgfqpoint{0.569410in}{0.114167in}}{\pgfqpoint{0.556055in}{0.119698in}}%
\pgfpathcurveto{\pgfqpoint{0.546210in}{0.129544in}}{\pgfqpoint{0.536365in}{0.139389in}}{\pgfqpoint{0.530833in}{0.152744in}}%
\pgfpathcurveto{\pgfqpoint{0.530833in}{0.166667in}}{\pgfqpoint{0.530833in}{0.180590in}}{\pgfqpoint{0.536365in}{0.193945in}}%
\pgfpathcurveto{\pgfqpoint{0.546210in}{0.203790in}}{\pgfqpoint{0.556055in}{0.213635in}}{\pgfqpoint{0.569410in}{0.219167in}}%
\pgfpathcurveto{\pgfqpoint{0.583333in}{0.219167in}}{\pgfqpoint{0.597256in}{0.219167in}}{\pgfqpoint{0.610611in}{0.213635in}}%
\pgfpathcurveto{\pgfqpoint{0.620456in}{0.203790in}}{\pgfqpoint{0.630302in}{0.193945in}}{\pgfqpoint{0.635833in}{0.180590in}}%
\pgfpathcurveto{\pgfqpoint{0.635833in}{0.166667in}}{\pgfqpoint{0.635833in}{0.152744in}}{\pgfqpoint{0.630302in}{0.139389in}}%
\pgfpathcurveto{\pgfqpoint{0.620456in}{0.129544in}}{\pgfqpoint{0.610611in}{0.119698in}}{\pgfqpoint{0.597256in}{0.114167in}}%
\pgfpathclose%
\pgfpathmoveto{\pgfqpoint{0.750000in}{0.108333in}}%
\pgfpathcurveto{\pgfqpoint{0.765470in}{0.108333in}}{\pgfqpoint{0.780309in}{0.114480in}}{\pgfqpoint{0.791248in}{0.125419in}}%
\pgfpathcurveto{\pgfqpoint{0.802187in}{0.136358in}}{\pgfqpoint{0.808333in}{0.151196in}}{\pgfqpoint{0.808333in}{0.166667in}}%
\pgfpathcurveto{\pgfqpoint{0.808333in}{0.182137in}}{\pgfqpoint{0.802187in}{0.196975in}}{\pgfqpoint{0.791248in}{0.207915in}}%
\pgfpathcurveto{\pgfqpoint{0.780309in}{0.218854in}}{\pgfqpoint{0.765470in}{0.225000in}}{\pgfqpoint{0.750000in}{0.225000in}}%
\pgfpathcurveto{\pgfqpoint{0.734530in}{0.225000in}}{\pgfqpoint{0.719691in}{0.218854in}}{\pgfqpoint{0.708752in}{0.207915in}}%
\pgfpathcurveto{\pgfqpoint{0.697813in}{0.196975in}}{\pgfqpoint{0.691667in}{0.182137in}}{\pgfqpoint{0.691667in}{0.166667in}}%
\pgfpathcurveto{\pgfqpoint{0.691667in}{0.151196in}}{\pgfqpoint{0.697813in}{0.136358in}}{\pgfqpoint{0.708752in}{0.125419in}}%
\pgfpathcurveto{\pgfqpoint{0.719691in}{0.114480in}}{\pgfqpoint{0.734530in}{0.108333in}}{\pgfqpoint{0.750000in}{0.108333in}}%
\pgfpathclose%
\pgfpathmoveto{\pgfqpoint{0.750000in}{0.114167in}}%
\pgfpathcurveto{\pgfqpoint{0.750000in}{0.114167in}}{\pgfqpoint{0.736077in}{0.114167in}}{\pgfqpoint{0.722722in}{0.119698in}}%
\pgfpathcurveto{\pgfqpoint{0.712877in}{0.129544in}}{\pgfqpoint{0.703032in}{0.139389in}}{\pgfqpoint{0.697500in}{0.152744in}}%
\pgfpathcurveto{\pgfqpoint{0.697500in}{0.166667in}}{\pgfqpoint{0.697500in}{0.180590in}}{\pgfqpoint{0.703032in}{0.193945in}}%
\pgfpathcurveto{\pgfqpoint{0.712877in}{0.203790in}}{\pgfqpoint{0.722722in}{0.213635in}}{\pgfqpoint{0.736077in}{0.219167in}}%
\pgfpathcurveto{\pgfqpoint{0.750000in}{0.219167in}}{\pgfqpoint{0.763923in}{0.219167in}}{\pgfqpoint{0.777278in}{0.213635in}}%
\pgfpathcurveto{\pgfqpoint{0.787123in}{0.203790in}}{\pgfqpoint{0.796968in}{0.193945in}}{\pgfqpoint{0.802500in}{0.180590in}}%
\pgfpathcurveto{\pgfqpoint{0.802500in}{0.166667in}}{\pgfqpoint{0.802500in}{0.152744in}}{\pgfqpoint{0.796968in}{0.139389in}}%
\pgfpathcurveto{\pgfqpoint{0.787123in}{0.129544in}}{\pgfqpoint{0.777278in}{0.119698in}}{\pgfqpoint{0.763923in}{0.114167in}}%
\pgfpathclose%
\pgfpathmoveto{\pgfqpoint{0.916667in}{0.108333in}}%
\pgfpathcurveto{\pgfqpoint{0.932137in}{0.108333in}}{\pgfqpoint{0.946975in}{0.114480in}}{\pgfqpoint{0.957915in}{0.125419in}}%
\pgfpathcurveto{\pgfqpoint{0.968854in}{0.136358in}}{\pgfqpoint{0.975000in}{0.151196in}}{\pgfqpoint{0.975000in}{0.166667in}}%
\pgfpathcurveto{\pgfqpoint{0.975000in}{0.182137in}}{\pgfqpoint{0.968854in}{0.196975in}}{\pgfqpoint{0.957915in}{0.207915in}}%
\pgfpathcurveto{\pgfqpoint{0.946975in}{0.218854in}}{\pgfqpoint{0.932137in}{0.225000in}}{\pgfqpoint{0.916667in}{0.225000in}}%
\pgfpathcurveto{\pgfqpoint{0.901196in}{0.225000in}}{\pgfqpoint{0.886358in}{0.218854in}}{\pgfqpoint{0.875419in}{0.207915in}}%
\pgfpathcurveto{\pgfqpoint{0.864480in}{0.196975in}}{\pgfqpoint{0.858333in}{0.182137in}}{\pgfqpoint{0.858333in}{0.166667in}}%
\pgfpathcurveto{\pgfqpoint{0.858333in}{0.151196in}}{\pgfqpoint{0.864480in}{0.136358in}}{\pgfqpoint{0.875419in}{0.125419in}}%
\pgfpathcurveto{\pgfqpoint{0.886358in}{0.114480in}}{\pgfqpoint{0.901196in}{0.108333in}}{\pgfqpoint{0.916667in}{0.108333in}}%
\pgfpathclose%
\pgfpathmoveto{\pgfqpoint{0.916667in}{0.114167in}}%
\pgfpathcurveto{\pgfqpoint{0.916667in}{0.114167in}}{\pgfqpoint{0.902744in}{0.114167in}}{\pgfqpoint{0.889389in}{0.119698in}}%
\pgfpathcurveto{\pgfqpoint{0.879544in}{0.129544in}}{\pgfqpoint{0.869698in}{0.139389in}}{\pgfqpoint{0.864167in}{0.152744in}}%
\pgfpathcurveto{\pgfqpoint{0.864167in}{0.166667in}}{\pgfqpoint{0.864167in}{0.180590in}}{\pgfqpoint{0.869698in}{0.193945in}}%
\pgfpathcurveto{\pgfqpoint{0.879544in}{0.203790in}}{\pgfqpoint{0.889389in}{0.213635in}}{\pgfqpoint{0.902744in}{0.219167in}}%
\pgfpathcurveto{\pgfqpoint{0.916667in}{0.219167in}}{\pgfqpoint{0.930590in}{0.219167in}}{\pgfqpoint{0.943945in}{0.213635in}}%
\pgfpathcurveto{\pgfqpoint{0.953790in}{0.203790in}}{\pgfqpoint{0.963635in}{0.193945in}}{\pgfqpoint{0.969167in}{0.180590in}}%
\pgfpathcurveto{\pgfqpoint{0.969167in}{0.166667in}}{\pgfqpoint{0.969167in}{0.152744in}}{\pgfqpoint{0.963635in}{0.139389in}}%
\pgfpathcurveto{\pgfqpoint{0.953790in}{0.129544in}}{\pgfqpoint{0.943945in}{0.119698in}}{\pgfqpoint{0.930590in}{0.114167in}}%
\pgfpathclose%
\pgfpathmoveto{\pgfqpoint{0.000000in}{0.275000in}}%
\pgfpathcurveto{\pgfqpoint{0.015470in}{0.275000in}}{\pgfqpoint{0.030309in}{0.281146in}}{\pgfqpoint{0.041248in}{0.292085in}}%
\pgfpathcurveto{\pgfqpoint{0.052187in}{0.303025in}}{\pgfqpoint{0.058333in}{0.317863in}}{\pgfqpoint{0.058333in}{0.333333in}}%
\pgfpathcurveto{\pgfqpoint{0.058333in}{0.348804in}}{\pgfqpoint{0.052187in}{0.363642in}}{\pgfqpoint{0.041248in}{0.374581in}}%
\pgfpathcurveto{\pgfqpoint{0.030309in}{0.385520in}}{\pgfqpoint{0.015470in}{0.391667in}}{\pgfqpoint{0.000000in}{0.391667in}}%
\pgfpathcurveto{\pgfqpoint{-0.015470in}{0.391667in}}{\pgfqpoint{-0.030309in}{0.385520in}}{\pgfqpoint{-0.041248in}{0.374581in}}%
\pgfpathcurveto{\pgfqpoint{-0.052187in}{0.363642in}}{\pgfqpoint{-0.058333in}{0.348804in}}{\pgfqpoint{-0.058333in}{0.333333in}}%
\pgfpathcurveto{\pgfqpoint{-0.058333in}{0.317863in}}{\pgfqpoint{-0.052187in}{0.303025in}}{\pgfqpoint{-0.041248in}{0.292085in}}%
\pgfpathcurveto{\pgfqpoint{-0.030309in}{0.281146in}}{\pgfqpoint{-0.015470in}{0.275000in}}{\pgfqpoint{0.000000in}{0.275000in}}%
\pgfpathclose%
\pgfpathmoveto{\pgfqpoint{0.000000in}{0.280833in}}%
\pgfpathcurveto{\pgfqpoint{0.000000in}{0.280833in}}{\pgfqpoint{-0.013923in}{0.280833in}}{\pgfqpoint{-0.027278in}{0.286365in}}%
\pgfpathcurveto{\pgfqpoint{-0.037123in}{0.296210in}}{\pgfqpoint{-0.046968in}{0.306055in}}{\pgfqpoint{-0.052500in}{0.319410in}}%
\pgfpathcurveto{\pgfqpoint{-0.052500in}{0.333333in}}{\pgfqpoint{-0.052500in}{0.347256in}}{\pgfqpoint{-0.046968in}{0.360611in}}%
\pgfpathcurveto{\pgfqpoint{-0.037123in}{0.370456in}}{\pgfqpoint{-0.027278in}{0.380302in}}{\pgfqpoint{-0.013923in}{0.385833in}}%
\pgfpathcurveto{\pgfqpoint{0.000000in}{0.385833in}}{\pgfqpoint{0.013923in}{0.385833in}}{\pgfqpoint{0.027278in}{0.380302in}}%
\pgfpathcurveto{\pgfqpoint{0.037123in}{0.370456in}}{\pgfqpoint{0.046968in}{0.360611in}}{\pgfqpoint{0.052500in}{0.347256in}}%
\pgfpathcurveto{\pgfqpoint{0.052500in}{0.333333in}}{\pgfqpoint{0.052500in}{0.319410in}}{\pgfqpoint{0.046968in}{0.306055in}}%
\pgfpathcurveto{\pgfqpoint{0.037123in}{0.296210in}}{\pgfqpoint{0.027278in}{0.286365in}}{\pgfqpoint{0.013923in}{0.280833in}}%
\pgfpathclose%
\pgfpathmoveto{\pgfqpoint{0.166667in}{0.275000in}}%
\pgfpathcurveto{\pgfqpoint{0.182137in}{0.275000in}}{\pgfqpoint{0.196975in}{0.281146in}}{\pgfqpoint{0.207915in}{0.292085in}}%
\pgfpathcurveto{\pgfqpoint{0.218854in}{0.303025in}}{\pgfqpoint{0.225000in}{0.317863in}}{\pgfqpoint{0.225000in}{0.333333in}}%
\pgfpathcurveto{\pgfqpoint{0.225000in}{0.348804in}}{\pgfqpoint{0.218854in}{0.363642in}}{\pgfqpoint{0.207915in}{0.374581in}}%
\pgfpathcurveto{\pgfqpoint{0.196975in}{0.385520in}}{\pgfqpoint{0.182137in}{0.391667in}}{\pgfqpoint{0.166667in}{0.391667in}}%
\pgfpathcurveto{\pgfqpoint{0.151196in}{0.391667in}}{\pgfqpoint{0.136358in}{0.385520in}}{\pgfqpoint{0.125419in}{0.374581in}}%
\pgfpathcurveto{\pgfqpoint{0.114480in}{0.363642in}}{\pgfqpoint{0.108333in}{0.348804in}}{\pgfqpoint{0.108333in}{0.333333in}}%
\pgfpathcurveto{\pgfqpoint{0.108333in}{0.317863in}}{\pgfqpoint{0.114480in}{0.303025in}}{\pgfqpoint{0.125419in}{0.292085in}}%
\pgfpathcurveto{\pgfqpoint{0.136358in}{0.281146in}}{\pgfqpoint{0.151196in}{0.275000in}}{\pgfqpoint{0.166667in}{0.275000in}}%
\pgfpathclose%
\pgfpathmoveto{\pgfqpoint{0.166667in}{0.280833in}}%
\pgfpathcurveto{\pgfqpoint{0.166667in}{0.280833in}}{\pgfqpoint{0.152744in}{0.280833in}}{\pgfqpoint{0.139389in}{0.286365in}}%
\pgfpathcurveto{\pgfqpoint{0.129544in}{0.296210in}}{\pgfqpoint{0.119698in}{0.306055in}}{\pgfqpoint{0.114167in}{0.319410in}}%
\pgfpathcurveto{\pgfqpoint{0.114167in}{0.333333in}}{\pgfqpoint{0.114167in}{0.347256in}}{\pgfqpoint{0.119698in}{0.360611in}}%
\pgfpathcurveto{\pgfqpoint{0.129544in}{0.370456in}}{\pgfqpoint{0.139389in}{0.380302in}}{\pgfqpoint{0.152744in}{0.385833in}}%
\pgfpathcurveto{\pgfqpoint{0.166667in}{0.385833in}}{\pgfqpoint{0.180590in}{0.385833in}}{\pgfqpoint{0.193945in}{0.380302in}}%
\pgfpathcurveto{\pgfqpoint{0.203790in}{0.370456in}}{\pgfqpoint{0.213635in}{0.360611in}}{\pgfqpoint{0.219167in}{0.347256in}}%
\pgfpathcurveto{\pgfqpoint{0.219167in}{0.333333in}}{\pgfqpoint{0.219167in}{0.319410in}}{\pgfqpoint{0.213635in}{0.306055in}}%
\pgfpathcurveto{\pgfqpoint{0.203790in}{0.296210in}}{\pgfqpoint{0.193945in}{0.286365in}}{\pgfqpoint{0.180590in}{0.280833in}}%
\pgfpathclose%
\pgfpathmoveto{\pgfqpoint{0.333333in}{0.275000in}}%
\pgfpathcurveto{\pgfqpoint{0.348804in}{0.275000in}}{\pgfqpoint{0.363642in}{0.281146in}}{\pgfqpoint{0.374581in}{0.292085in}}%
\pgfpathcurveto{\pgfqpoint{0.385520in}{0.303025in}}{\pgfqpoint{0.391667in}{0.317863in}}{\pgfqpoint{0.391667in}{0.333333in}}%
\pgfpathcurveto{\pgfqpoint{0.391667in}{0.348804in}}{\pgfqpoint{0.385520in}{0.363642in}}{\pgfqpoint{0.374581in}{0.374581in}}%
\pgfpathcurveto{\pgfqpoint{0.363642in}{0.385520in}}{\pgfqpoint{0.348804in}{0.391667in}}{\pgfqpoint{0.333333in}{0.391667in}}%
\pgfpathcurveto{\pgfqpoint{0.317863in}{0.391667in}}{\pgfqpoint{0.303025in}{0.385520in}}{\pgfqpoint{0.292085in}{0.374581in}}%
\pgfpathcurveto{\pgfqpoint{0.281146in}{0.363642in}}{\pgfqpoint{0.275000in}{0.348804in}}{\pgfqpoint{0.275000in}{0.333333in}}%
\pgfpathcurveto{\pgfqpoint{0.275000in}{0.317863in}}{\pgfqpoint{0.281146in}{0.303025in}}{\pgfqpoint{0.292085in}{0.292085in}}%
\pgfpathcurveto{\pgfqpoint{0.303025in}{0.281146in}}{\pgfqpoint{0.317863in}{0.275000in}}{\pgfqpoint{0.333333in}{0.275000in}}%
\pgfpathclose%
\pgfpathmoveto{\pgfqpoint{0.333333in}{0.280833in}}%
\pgfpathcurveto{\pgfqpoint{0.333333in}{0.280833in}}{\pgfqpoint{0.319410in}{0.280833in}}{\pgfqpoint{0.306055in}{0.286365in}}%
\pgfpathcurveto{\pgfqpoint{0.296210in}{0.296210in}}{\pgfqpoint{0.286365in}{0.306055in}}{\pgfqpoint{0.280833in}{0.319410in}}%
\pgfpathcurveto{\pgfqpoint{0.280833in}{0.333333in}}{\pgfqpoint{0.280833in}{0.347256in}}{\pgfqpoint{0.286365in}{0.360611in}}%
\pgfpathcurveto{\pgfqpoint{0.296210in}{0.370456in}}{\pgfqpoint{0.306055in}{0.380302in}}{\pgfqpoint{0.319410in}{0.385833in}}%
\pgfpathcurveto{\pgfqpoint{0.333333in}{0.385833in}}{\pgfqpoint{0.347256in}{0.385833in}}{\pgfqpoint{0.360611in}{0.380302in}}%
\pgfpathcurveto{\pgfqpoint{0.370456in}{0.370456in}}{\pgfqpoint{0.380302in}{0.360611in}}{\pgfqpoint{0.385833in}{0.347256in}}%
\pgfpathcurveto{\pgfqpoint{0.385833in}{0.333333in}}{\pgfqpoint{0.385833in}{0.319410in}}{\pgfqpoint{0.380302in}{0.306055in}}%
\pgfpathcurveto{\pgfqpoint{0.370456in}{0.296210in}}{\pgfqpoint{0.360611in}{0.286365in}}{\pgfqpoint{0.347256in}{0.280833in}}%
\pgfpathclose%
\pgfpathmoveto{\pgfqpoint{0.500000in}{0.275000in}}%
\pgfpathcurveto{\pgfqpoint{0.515470in}{0.275000in}}{\pgfqpoint{0.530309in}{0.281146in}}{\pgfqpoint{0.541248in}{0.292085in}}%
\pgfpathcurveto{\pgfqpoint{0.552187in}{0.303025in}}{\pgfqpoint{0.558333in}{0.317863in}}{\pgfqpoint{0.558333in}{0.333333in}}%
\pgfpathcurveto{\pgfqpoint{0.558333in}{0.348804in}}{\pgfqpoint{0.552187in}{0.363642in}}{\pgfqpoint{0.541248in}{0.374581in}}%
\pgfpathcurveto{\pgfqpoint{0.530309in}{0.385520in}}{\pgfqpoint{0.515470in}{0.391667in}}{\pgfqpoint{0.500000in}{0.391667in}}%
\pgfpathcurveto{\pgfqpoint{0.484530in}{0.391667in}}{\pgfqpoint{0.469691in}{0.385520in}}{\pgfqpoint{0.458752in}{0.374581in}}%
\pgfpathcurveto{\pgfqpoint{0.447813in}{0.363642in}}{\pgfqpoint{0.441667in}{0.348804in}}{\pgfqpoint{0.441667in}{0.333333in}}%
\pgfpathcurveto{\pgfqpoint{0.441667in}{0.317863in}}{\pgfqpoint{0.447813in}{0.303025in}}{\pgfqpoint{0.458752in}{0.292085in}}%
\pgfpathcurveto{\pgfqpoint{0.469691in}{0.281146in}}{\pgfqpoint{0.484530in}{0.275000in}}{\pgfqpoint{0.500000in}{0.275000in}}%
\pgfpathclose%
\pgfpathmoveto{\pgfqpoint{0.500000in}{0.280833in}}%
\pgfpathcurveto{\pgfqpoint{0.500000in}{0.280833in}}{\pgfqpoint{0.486077in}{0.280833in}}{\pgfqpoint{0.472722in}{0.286365in}}%
\pgfpathcurveto{\pgfqpoint{0.462877in}{0.296210in}}{\pgfqpoint{0.453032in}{0.306055in}}{\pgfqpoint{0.447500in}{0.319410in}}%
\pgfpathcurveto{\pgfqpoint{0.447500in}{0.333333in}}{\pgfqpoint{0.447500in}{0.347256in}}{\pgfqpoint{0.453032in}{0.360611in}}%
\pgfpathcurveto{\pgfqpoint{0.462877in}{0.370456in}}{\pgfqpoint{0.472722in}{0.380302in}}{\pgfqpoint{0.486077in}{0.385833in}}%
\pgfpathcurveto{\pgfqpoint{0.500000in}{0.385833in}}{\pgfqpoint{0.513923in}{0.385833in}}{\pgfqpoint{0.527278in}{0.380302in}}%
\pgfpathcurveto{\pgfqpoint{0.537123in}{0.370456in}}{\pgfqpoint{0.546968in}{0.360611in}}{\pgfqpoint{0.552500in}{0.347256in}}%
\pgfpathcurveto{\pgfqpoint{0.552500in}{0.333333in}}{\pgfqpoint{0.552500in}{0.319410in}}{\pgfqpoint{0.546968in}{0.306055in}}%
\pgfpathcurveto{\pgfqpoint{0.537123in}{0.296210in}}{\pgfqpoint{0.527278in}{0.286365in}}{\pgfqpoint{0.513923in}{0.280833in}}%
\pgfpathclose%
\pgfpathmoveto{\pgfqpoint{0.666667in}{0.275000in}}%
\pgfpathcurveto{\pgfqpoint{0.682137in}{0.275000in}}{\pgfqpoint{0.696975in}{0.281146in}}{\pgfqpoint{0.707915in}{0.292085in}}%
\pgfpathcurveto{\pgfqpoint{0.718854in}{0.303025in}}{\pgfqpoint{0.725000in}{0.317863in}}{\pgfqpoint{0.725000in}{0.333333in}}%
\pgfpathcurveto{\pgfqpoint{0.725000in}{0.348804in}}{\pgfqpoint{0.718854in}{0.363642in}}{\pgfqpoint{0.707915in}{0.374581in}}%
\pgfpathcurveto{\pgfqpoint{0.696975in}{0.385520in}}{\pgfqpoint{0.682137in}{0.391667in}}{\pgfqpoint{0.666667in}{0.391667in}}%
\pgfpathcurveto{\pgfqpoint{0.651196in}{0.391667in}}{\pgfqpoint{0.636358in}{0.385520in}}{\pgfqpoint{0.625419in}{0.374581in}}%
\pgfpathcurveto{\pgfqpoint{0.614480in}{0.363642in}}{\pgfqpoint{0.608333in}{0.348804in}}{\pgfqpoint{0.608333in}{0.333333in}}%
\pgfpathcurveto{\pgfqpoint{0.608333in}{0.317863in}}{\pgfqpoint{0.614480in}{0.303025in}}{\pgfqpoint{0.625419in}{0.292085in}}%
\pgfpathcurveto{\pgfqpoint{0.636358in}{0.281146in}}{\pgfqpoint{0.651196in}{0.275000in}}{\pgfqpoint{0.666667in}{0.275000in}}%
\pgfpathclose%
\pgfpathmoveto{\pgfqpoint{0.666667in}{0.280833in}}%
\pgfpathcurveto{\pgfqpoint{0.666667in}{0.280833in}}{\pgfqpoint{0.652744in}{0.280833in}}{\pgfqpoint{0.639389in}{0.286365in}}%
\pgfpathcurveto{\pgfqpoint{0.629544in}{0.296210in}}{\pgfqpoint{0.619698in}{0.306055in}}{\pgfqpoint{0.614167in}{0.319410in}}%
\pgfpathcurveto{\pgfqpoint{0.614167in}{0.333333in}}{\pgfqpoint{0.614167in}{0.347256in}}{\pgfqpoint{0.619698in}{0.360611in}}%
\pgfpathcurveto{\pgfqpoint{0.629544in}{0.370456in}}{\pgfqpoint{0.639389in}{0.380302in}}{\pgfqpoint{0.652744in}{0.385833in}}%
\pgfpathcurveto{\pgfqpoint{0.666667in}{0.385833in}}{\pgfqpoint{0.680590in}{0.385833in}}{\pgfqpoint{0.693945in}{0.380302in}}%
\pgfpathcurveto{\pgfqpoint{0.703790in}{0.370456in}}{\pgfqpoint{0.713635in}{0.360611in}}{\pgfqpoint{0.719167in}{0.347256in}}%
\pgfpathcurveto{\pgfqpoint{0.719167in}{0.333333in}}{\pgfqpoint{0.719167in}{0.319410in}}{\pgfqpoint{0.713635in}{0.306055in}}%
\pgfpathcurveto{\pgfqpoint{0.703790in}{0.296210in}}{\pgfqpoint{0.693945in}{0.286365in}}{\pgfqpoint{0.680590in}{0.280833in}}%
\pgfpathclose%
\pgfpathmoveto{\pgfqpoint{0.833333in}{0.275000in}}%
\pgfpathcurveto{\pgfqpoint{0.848804in}{0.275000in}}{\pgfqpoint{0.863642in}{0.281146in}}{\pgfqpoint{0.874581in}{0.292085in}}%
\pgfpathcurveto{\pgfqpoint{0.885520in}{0.303025in}}{\pgfqpoint{0.891667in}{0.317863in}}{\pgfqpoint{0.891667in}{0.333333in}}%
\pgfpathcurveto{\pgfqpoint{0.891667in}{0.348804in}}{\pgfqpoint{0.885520in}{0.363642in}}{\pgfqpoint{0.874581in}{0.374581in}}%
\pgfpathcurveto{\pgfqpoint{0.863642in}{0.385520in}}{\pgfqpoint{0.848804in}{0.391667in}}{\pgfqpoint{0.833333in}{0.391667in}}%
\pgfpathcurveto{\pgfqpoint{0.817863in}{0.391667in}}{\pgfqpoint{0.803025in}{0.385520in}}{\pgfqpoint{0.792085in}{0.374581in}}%
\pgfpathcurveto{\pgfqpoint{0.781146in}{0.363642in}}{\pgfqpoint{0.775000in}{0.348804in}}{\pgfqpoint{0.775000in}{0.333333in}}%
\pgfpathcurveto{\pgfqpoint{0.775000in}{0.317863in}}{\pgfqpoint{0.781146in}{0.303025in}}{\pgfqpoint{0.792085in}{0.292085in}}%
\pgfpathcurveto{\pgfqpoint{0.803025in}{0.281146in}}{\pgfqpoint{0.817863in}{0.275000in}}{\pgfqpoint{0.833333in}{0.275000in}}%
\pgfpathclose%
\pgfpathmoveto{\pgfqpoint{0.833333in}{0.280833in}}%
\pgfpathcurveto{\pgfqpoint{0.833333in}{0.280833in}}{\pgfqpoint{0.819410in}{0.280833in}}{\pgfqpoint{0.806055in}{0.286365in}}%
\pgfpathcurveto{\pgfqpoint{0.796210in}{0.296210in}}{\pgfqpoint{0.786365in}{0.306055in}}{\pgfqpoint{0.780833in}{0.319410in}}%
\pgfpathcurveto{\pgfqpoint{0.780833in}{0.333333in}}{\pgfqpoint{0.780833in}{0.347256in}}{\pgfqpoint{0.786365in}{0.360611in}}%
\pgfpathcurveto{\pgfqpoint{0.796210in}{0.370456in}}{\pgfqpoint{0.806055in}{0.380302in}}{\pgfqpoint{0.819410in}{0.385833in}}%
\pgfpathcurveto{\pgfqpoint{0.833333in}{0.385833in}}{\pgfqpoint{0.847256in}{0.385833in}}{\pgfqpoint{0.860611in}{0.380302in}}%
\pgfpathcurveto{\pgfqpoint{0.870456in}{0.370456in}}{\pgfqpoint{0.880302in}{0.360611in}}{\pgfqpoint{0.885833in}{0.347256in}}%
\pgfpathcurveto{\pgfqpoint{0.885833in}{0.333333in}}{\pgfqpoint{0.885833in}{0.319410in}}{\pgfqpoint{0.880302in}{0.306055in}}%
\pgfpathcurveto{\pgfqpoint{0.870456in}{0.296210in}}{\pgfqpoint{0.860611in}{0.286365in}}{\pgfqpoint{0.847256in}{0.280833in}}%
\pgfpathclose%
\pgfpathmoveto{\pgfqpoint{1.000000in}{0.275000in}}%
\pgfpathcurveto{\pgfqpoint{1.015470in}{0.275000in}}{\pgfqpoint{1.030309in}{0.281146in}}{\pgfqpoint{1.041248in}{0.292085in}}%
\pgfpathcurveto{\pgfqpoint{1.052187in}{0.303025in}}{\pgfqpoint{1.058333in}{0.317863in}}{\pgfqpoint{1.058333in}{0.333333in}}%
\pgfpathcurveto{\pgfqpoint{1.058333in}{0.348804in}}{\pgfqpoint{1.052187in}{0.363642in}}{\pgfqpoint{1.041248in}{0.374581in}}%
\pgfpathcurveto{\pgfqpoint{1.030309in}{0.385520in}}{\pgfqpoint{1.015470in}{0.391667in}}{\pgfqpoint{1.000000in}{0.391667in}}%
\pgfpathcurveto{\pgfqpoint{0.984530in}{0.391667in}}{\pgfqpoint{0.969691in}{0.385520in}}{\pgfqpoint{0.958752in}{0.374581in}}%
\pgfpathcurveto{\pgfqpoint{0.947813in}{0.363642in}}{\pgfqpoint{0.941667in}{0.348804in}}{\pgfqpoint{0.941667in}{0.333333in}}%
\pgfpathcurveto{\pgfqpoint{0.941667in}{0.317863in}}{\pgfqpoint{0.947813in}{0.303025in}}{\pgfqpoint{0.958752in}{0.292085in}}%
\pgfpathcurveto{\pgfqpoint{0.969691in}{0.281146in}}{\pgfqpoint{0.984530in}{0.275000in}}{\pgfqpoint{1.000000in}{0.275000in}}%
\pgfpathclose%
\pgfpathmoveto{\pgfqpoint{1.000000in}{0.280833in}}%
\pgfpathcurveto{\pgfqpoint{1.000000in}{0.280833in}}{\pgfqpoint{0.986077in}{0.280833in}}{\pgfqpoint{0.972722in}{0.286365in}}%
\pgfpathcurveto{\pgfqpoint{0.962877in}{0.296210in}}{\pgfqpoint{0.953032in}{0.306055in}}{\pgfqpoint{0.947500in}{0.319410in}}%
\pgfpathcurveto{\pgfqpoint{0.947500in}{0.333333in}}{\pgfqpoint{0.947500in}{0.347256in}}{\pgfqpoint{0.953032in}{0.360611in}}%
\pgfpathcurveto{\pgfqpoint{0.962877in}{0.370456in}}{\pgfqpoint{0.972722in}{0.380302in}}{\pgfqpoint{0.986077in}{0.385833in}}%
\pgfpathcurveto{\pgfqpoint{1.000000in}{0.385833in}}{\pgfqpoint{1.013923in}{0.385833in}}{\pgfqpoint{1.027278in}{0.380302in}}%
\pgfpathcurveto{\pgfqpoint{1.037123in}{0.370456in}}{\pgfqpoint{1.046968in}{0.360611in}}{\pgfqpoint{1.052500in}{0.347256in}}%
\pgfpathcurveto{\pgfqpoint{1.052500in}{0.333333in}}{\pgfqpoint{1.052500in}{0.319410in}}{\pgfqpoint{1.046968in}{0.306055in}}%
\pgfpathcurveto{\pgfqpoint{1.037123in}{0.296210in}}{\pgfqpoint{1.027278in}{0.286365in}}{\pgfqpoint{1.013923in}{0.280833in}}%
\pgfpathclose%
\pgfpathmoveto{\pgfqpoint{0.083333in}{0.441667in}}%
\pgfpathcurveto{\pgfqpoint{0.098804in}{0.441667in}}{\pgfqpoint{0.113642in}{0.447813in}}{\pgfqpoint{0.124581in}{0.458752in}}%
\pgfpathcurveto{\pgfqpoint{0.135520in}{0.469691in}}{\pgfqpoint{0.141667in}{0.484530in}}{\pgfqpoint{0.141667in}{0.500000in}}%
\pgfpathcurveto{\pgfqpoint{0.141667in}{0.515470in}}{\pgfqpoint{0.135520in}{0.530309in}}{\pgfqpoint{0.124581in}{0.541248in}}%
\pgfpathcurveto{\pgfqpoint{0.113642in}{0.552187in}}{\pgfqpoint{0.098804in}{0.558333in}}{\pgfqpoint{0.083333in}{0.558333in}}%
\pgfpathcurveto{\pgfqpoint{0.067863in}{0.558333in}}{\pgfqpoint{0.053025in}{0.552187in}}{\pgfqpoint{0.042085in}{0.541248in}}%
\pgfpathcurveto{\pgfqpoint{0.031146in}{0.530309in}}{\pgfqpoint{0.025000in}{0.515470in}}{\pgfqpoint{0.025000in}{0.500000in}}%
\pgfpathcurveto{\pgfqpoint{0.025000in}{0.484530in}}{\pgfqpoint{0.031146in}{0.469691in}}{\pgfqpoint{0.042085in}{0.458752in}}%
\pgfpathcurveto{\pgfqpoint{0.053025in}{0.447813in}}{\pgfqpoint{0.067863in}{0.441667in}}{\pgfqpoint{0.083333in}{0.441667in}}%
\pgfpathclose%
\pgfpathmoveto{\pgfqpoint{0.083333in}{0.447500in}}%
\pgfpathcurveto{\pgfqpoint{0.083333in}{0.447500in}}{\pgfqpoint{0.069410in}{0.447500in}}{\pgfqpoint{0.056055in}{0.453032in}}%
\pgfpathcurveto{\pgfqpoint{0.046210in}{0.462877in}}{\pgfqpoint{0.036365in}{0.472722in}}{\pgfqpoint{0.030833in}{0.486077in}}%
\pgfpathcurveto{\pgfqpoint{0.030833in}{0.500000in}}{\pgfqpoint{0.030833in}{0.513923in}}{\pgfqpoint{0.036365in}{0.527278in}}%
\pgfpathcurveto{\pgfqpoint{0.046210in}{0.537123in}}{\pgfqpoint{0.056055in}{0.546968in}}{\pgfqpoint{0.069410in}{0.552500in}}%
\pgfpathcurveto{\pgfqpoint{0.083333in}{0.552500in}}{\pgfqpoint{0.097256in}{0.552500in}}{\pgfqpoint{0.110611in}{0.546968in}}%
\pgfpathcurveto{\pgfqpoint{0.120456in}{0.537123in}}{\pgfqpoint{0.130302in}{0.527278in}}{\pgfqpoint{0.135833in}{0.513923in}}%
\pgfpathcurveto{\pgfqpoint{0.135833in}{0.500000in}}{\pgfqpoint{0.135833in}{0.486077in}}{\pgfqpoint{0.130302in}{0.472722in}}%
\pgfpathcurveto{\pgfqpoint{0.120456in}{0.462877in}}{\pgfqpoint{0.110611in}{0.453032in}}{\pgfqpoint{0.097256in}{0.447500in}}%
\pgfpathclose%
\pgfpathmoveto{\pgfqpoint{0.250000in}{0.441667in}}%
\pgfpathcurveto{\pgfqpoint{0.265470in}{0.441667in}}{\pgfqpoint{0.280309in}{0.447813in}}{\pgfqpoint{0.291248in}{0.458752in}}%
\pgfpathcurveto{\pgfqpoint{0.302187in}{0.469691in}}{\pgfqpoint{0.308333in}{0.484530in}}{\pgfqpoint{0.308333in}{0.500000in}}%
\pgfpathcurveto{\pgfqpoint{0.308333in}{0.515470in}}{\pgfqpoint{0.302187in}{0.530309in}}{\pgfqpoint{0.291248in}{0.541248in}}%
\pgfpathcurveto{\pgfqpoint{0.280309in}{0.552187in}}{\pgfqpoint{0.265470in}{0.558333in}}{\pgfqpoint{0.250000in}{0.558333in}}%
\pgfpathcurveto{\pgfqpoint{0.234530in}{0.558333in}}{\pgfqpoint{0.219691in}{0.552187in}}{\pgfqpoint{0.208752in}{0.541248in}}%
\pgfpathcurveto{\pgfqpoint{0.197813in}{0.530309in}}{\pgfqpoint{0.191667in}{0.515470in}}{\pgfqpoint{0.191667in}{0.500000in}}%
\pgfpathcurveto{\pgfqpoint{0.191667in}{0.484530in}}{\pgfqpoint{0.197813in}{0.469691in}}{\pgfqpoint{0.208752in}{0.458752in}}%
\pgfpathcurveto{\pgfqpoint{0.219691in}{0.447813in}}{\pgfqpoint{0.234530in}{0.441667in}}{\pgfqpoint{0.250000in}{0.441667in}}%
\pgfpathclose%
\pgfpathmoveto{\pgfqpoint{0.250000in}{0.447500in}}%
\pgfpathcurveto{\pgfqpoint{0.250000in}{0.447500in}}{\pgfqpoint{0.236077in}{0.447500in}}{\pgfqpoint{0.222722in}{0.453032in}}%
\pgfpathcurveto{\pgfqpoint{0.212877in}{0.462877in}}{\pgfqpoint{0.203032in}{0.472722in}}{\pgfqpoint{0.197500in}{0.486077in}}%
\pgfpathcurveto{\pgfqpoint{0.197500in}{0.500000in}}{\pgfqpoint{0.197500in}{0.513923in}}{\pgfqpoint{0.203032in}{0.527278in}}%
\pgfpathcurveto{\pgfqpoint{0.212877in}{0.537123in}}{\pgfqpoint{0.222722in}{0.546968in}}{\pgfqpoint{0.236077in}{0.552500in}}%
\pgfpathcurveto{\pgfqpoint{0.250000in}{0.552500in}}{\pgfqpoint{0.263923in}{0.552500in}}{\pgfqpoint{0.277278in}{0.546968in}}%
\pgfpathcurveto{\pgfqpoint{0.287123in}{0.537123in}}{\pgfqpoint{0.296968in}{0.527278in}}{\pgfqpoint{0.302500in}{0.513923in}}%
\pgfpathcurveto{\pgfqpoint{0.302500in}{0.500000in}}{\pgfqpoint{0.302500in}{0.486077in}}{\pgfqpoint{0.296968in}{0.472722in}}%
\pgfpathcurveto{\pgfqpoint{0.287123in}{0.462877in}}{\pgfqpoint{0.277278in}{0.453032in}}{\pgfqpoint{0.263923in}{0.447500in}}%
\pgfpathclose%
\pgfpathmoveto{\pgfqpoint{0.416667in}{0.441667in}}%
\pgfpathcurveto{\pgfqpoint{0.432137in}{0.441667in}}{\pgfqpoint{0.446975in}{0.447813in}}{\pgfqpoint{0.457915in}{0.458752in}}%
\pgfpathcurveto{\pgfqpoint{0.468854in}{0.469691in}}{\pgfqpoint{0.475000in}{0.484530in}}{\pgfqpoint{0.475000in}{0.500000in}}%
\pgfpathcurveto{\pgfqpoint{0.475000in}{0.515470in}}{\pgfqpoint{0.468854in}{0.530309in}}{\pgfqpoint{0.457915in}{0.541248in}}%
\pgfpathcurveto{\pgfqpoint{0.446975in}{0.552187in}}{\pgfqpoint{0.432137in}{0.558333in}}{\pgfqpoint{0.416667in}{0.558333in}}%
\pgfpathcurveto{\pgfqpoint{0.401196in}{0.558333in}}{\pgfqpoint{0.386358in}{0.552187in}}{\pgfqpoint{0.375419in}{0.541248in}}%
\pgfpathcurveto{\pgfqpoint{0.364480in}{0.530309in}}{\pgfqpoint{0.358333in}{0.515470in}}{\pgfqpoint{0.358333in}{0.500000in}}%
\pgfpathcurveto{\pgfqpoint{0.358333in}{0.484530in}}{\pgfqpoint{0.364480in}{0.469691in}}{\pgfqpoint{0.375419in}{0.458752in}}%
\pgfpathcurveto{\pgfqpoint{0.386358in}{0.447813in}}{\pgfqpoint{0.401196in}{0.441667in}}{\pgfqpoint{0.416667in}{0.441667in}}%
\pgfpathclose%
\pgfpathmoveto{\pgfqpoint{0.416667in}{0.447500in}}%
\pgfpathcurveto{\pgfqpoint{0.416667in}{0.447500in}}{\pgfqpoint{0.402744in}{0.447500in}}{\pgfqpoint{0.389389in}{0.453032in}}%
\pgfpathcurveto{\pgfqpoint{0.379544in}{0.462877in}}{\pgfqpoint{0.369698in}{0.472722in}}{\pgfqpoint{0.364167in}{0.486077in}}%
\pgfpathcurveto{\pgfqpoint{0.364167in}{0.500000in}}{\pgfqpoint{0.364167in}{0.513923in}}{\pgfqpoint{0.369698in}{0.527278in}}%
\pgfpathcurveto{\pgfqpoint{0.379544in}{0.537123in}}{\pgfqpoint{0.389389in}{0.546968in}}{\pgfqpoint{0.402744in}{0.552500in}}%
\pgfpathcurveto{\pgfqpoint{0.416667in}{0.552500in}}{\pgfqpoint{0.430590in}{0.552500in}}{\pgfqpoint{0.443945in}{0.546968in}}%
\pgfpathcurveto{\pgfqpoint{0.453790in}{0.537123in}}{\pgfqpoint{0.463635in}{0.527278in}}{\pgfqpoint{0.469167in}{0.513923in}}%
\pgfpathcurveto{\pgfqpoint{0.469167in}{0.500000in}}{\pgfqpoint{0.469167in}{0.486077in}}{\pgfqpoint{0.463635in}{0.472722in}}%
\pgfpathcurveto{\pgfqpoint{0.453790in}{0.462877in}}{\pgfqpoint{0.443945in}{0.453032in}}{\pgfqpoint{0.430590in}{0.447500in}}%
\pgfpathclose%
\pgfpathmoveto{\pgfqpoint{0.583333in}{0.441667in}}%
\pgfpathcurveto{\pgfqpoint{0.598804in}{0.441667in}}{\pgfqpoint{0.613642in}{0.447813in}}{\pgfqpoint{0.624581in}{0.458752in}}%
\pgfpathcurveto{\pgfqpoint{0.635520in}{0.469691in}}{\pgfqpoint{0.641667in}{0.484530in}}{\pgfqpoint{0.641667in}{0.500000in}}%
\pgfpathcurveto{\pgfqpoint{0.641667in}{0.515470in}}{\pgfqpoint{0.635520in}{0.530309in}}{\pgfqpoint{0.624581in}{0.541248in}}%
\pgfpathcurveto{\pgfqpoint{0.613642in}{0.552187in}}{\pgfqpoint{0.598804in}{0.558333in}}{\pgfqpoint{0.583333in}{0.558333in}}%
\pgfpathcurveto{\pgfqpoint{0.567863in}{0.558333in}}{\pgfqpoint{0.553025in}{0.552187in}}{\pgfqpoint{0.542085in}{0.541248in}}%
\pgfpathcurveto{\pgfqpoint{0.531146in}{0.530309in}}{\pgfqpoint{0.525000in}{0.515470in}}{\pgfqpoint{0.525000in}{0.500000in}}%
\pgfpathcurveto{\pgfqpoint{0.525000in}{0.484530in}}{\pgfqpoint{0.531146in}{0.469691in}}{\pgfqpoint{0.542085in}{0.458752in}}%
\pgfpathcurveto{\pgfqpoint{0.553025in}{0.447813in}}{\pgfqpoint{0.567863in}{0.441667in}}{\pgfqpoint{0.583333in}{0.441667in}}%
\pgfpathclose%
\pgfpathmoveto{\pgfqpoint{0.583333in}{0.447500in}}%
\pgfpathcurveto{\pgfqpoint{0.583333in}{0.447500in}}{\pgfqpoint{0.569410in}{0.447500in}}{\pgfqpoint{0.556055in}{0.453032in}}%
\pgfpathcurveto{\pgfqpoint{0.546210in}{0.462877in}}{\pgfqpoint{0.536365in}{0.472722in}}{\pgfqpoint{0.530833in}{0.486077in}}%
\pgfpathcurveto{\pgfqpoint{0.530833in}{0.500000in}}{\pgfqpoint{0.530833in}{0.513923in}}{\pgfqpoint{0.536365in}{0.527278in}}%
\pgfpathcurveto{\pgfqpoint{0.546210in}{0.537123in}}{\pgfqpoint{0.556055in}{0.546968in}}{\pgfqpoint{0.569410in}{0.552500in}}%
\pgfpathcurveto{\pgfqpoint{0.583333in}{0.552500in}}{\pgfqpoint{0.597256in}{0.552500in}}{\pgfqpoint{0.610611in}{0.546968in}}%
\pgfpathcurveto{\pgfqpoint{0.620456in}{0.537123in}}{\pgfqpoint{0.630302in}{0.527278in}}{\pgfqpoint{0.635833in}{0.513923in}}%
\pgfpathcurveto{\pgfqpoint{0.635833in}{0.500000in}}{\pgfqpoint{0.635833in}{0.486077in}}{\pgfqpoint{0.630302in}{0.472722in}}%
\pgfpathcurveto{\pgfqpoint{0.620456in}{0.462877in}}{\pgfqpoint{0.610611in}{0.453032in}}{\pgfqpoint{0.597256in}{0.447500in}}%
\pgfpathclose%
\pgfpathmoveto{\pgfqpoint{0.750000in}{0.441667in}}%
\pgfpathcurveto{\pgfqpoint{0.765470in}{0.441667in}}{\pgfqpoint{0.780309in}{0.447813in}}{\pgfqpoint{0.791248in}{0.458752in}}%
\pgfpathcurveto{\pgfqpoint{0.802187in}{0.469691in}}{\pgfqpoint{0.808333in}{0.484530in}}{\pgfqpoint{0.808333in}{0.500000in}}%
\pgfpathcurveto{\pgfqpoint{0.808333in}{0.515470in}}{\pgfqpoint{0.802187in}{0.530309in}}{\pgfqpoint{0.791248in}{0.541248in}}%
\pgfpathcurveto{\pgfqpoint{0.780309in}{0.552187in}}{\pgfqpoint{0.765470in}{0.558333in}}{\pgfqpoint{0.750000in}{0.558333in}}%
\pgfpathcurveto{\pgfqpoint{0.734530in}{0.558333in}}{\pgfqpoint{0.719691in}{0.552187in}}{\pgfqpoint{0.708752in}{0.541248in}}%
\pgfpathcurveto{\pgfqpoint{0.697813in}{0.530309in}}{\pgfqpoint{0.691667in}{0.515470in}}{\pgfqpoint{0.691667in}{0.500000in}}%
\pgfpathcurveto{\pgfqpoint{0.691667in}{0.484530in}}{\pgfqpoint{0.697813in}{0.469691in}}{\pgfqpoint{0.708752in}{0.458752in}}%
\pgfpathcurveto{\pgfqpoint{0.719691in}{0.447813in}}{\pgfqpoint{0.734530in}{0.441667in}}{\pgfqpoint{0.750000in}{0.441667in}}%
\pgfpathclose%
\pgfpathmoveto{\pgfqpoint{0.750000in}{0.447500in}}%
\pgfpathcurveto{\pgfqpoint{0.750000in}{0.447500in}}{\pgfqpoint{0.736077in}{0.447500in}}{\pgfqpoint{0.722722in}{0.453032in}}%
\pgfpathcurveto{\pgfqpoint{0.712877in}{0.462877in}}{\pgfqpoint{0.703032in}{0.472722in}}{\pgfqpoint{0.697500in}{0.486077in}}%
\pgfpathcurveto{\pgfqpoint{0.697500in}{0.500000in}}{\pgfqpoint{0.697500in}{0.513923in}}{\pgfqpoint{0.703032in}{0.527278in}}%
\pgfpathcurveto{\pgfqpoint{0.712877in}{0.537123in}}{\pgfqpoint{0.722722in}{0.546968in}}{\pgfqpoint{0.736077in}{0.552500in}}%
\pgfpathcurveto{\pgfqpoint{0.750000in}{0.552500in}}{\pgfqpoint{0.763923in}{0.552500in}}{\pgfqpoint{0.777278in}{0.546968in}}%
\pgfpathcurveto{\pgfqpoint{0.787123in}{0.537123in}}{\pgfqpoint{0.796968in}{0.527278in}}{\pgfqpoint{0.802500in}{0.513923in}}%
\pgfpathcurveto{\pgfqpoint{0.802500in}{0.500000in}}{\pgfqpoint{0.802500in}{0.486077in}}{\pgfqpoint{0.796968in}{0.472722in}}%
\pgfpathcurveto{\pgfqpoint{0.787123in}{0.462877in}}{\pgfqpoint{0.777278in}{0.453032in}}{\pgfqpoint{0.763923in}{0.447500in}}%
\pgfpathclose%
\pgfpathmoveto{\pgfqpoint{0.916667in}{0.441667in}}%
\pgfpathcurveto{\pgfqpoint{0.932137in}{0.441667in}}{\pgfqpoint{0.946975in}{0.447813in}}{\pgfqpoint{0.957915in}{0.458752in}}%
\pgfpathcurveto{\pgfqpoint{0.968854in}{0.469691in}}{\pgfqpoint{0.975000in}{0.484530in}}{\pgfqpoint{0.975000in}{0.500000in}}%
\pgfpathcurveto{\pgfqpoint{0.975000in}{0.515470in}}{\pgfqpoint{0.968854in}{0.530309in}}{\pgfqpoint{0.957915in}{0.541248in}}%
\pgfpathcurveto{\pgfqpoint{0.946975in}{0.552187in}}{\pgfqpoint{0.932137in}{0.558333in}}{\pgfqpoint{0.916667in}{0.558333in}}%
\pgfpathcurveto{\pgfqpoint{0.901196in}{0.558333in}}{\pgfqpoint{0.886358in}{0.552187in}}{\pgfqpoint{0.875419in}{0.541248in}}%
\pgfpathcurveto{\pgfqpoint{0.864480in}{0.530309in}}{\pgfqpoint{0.858333in}{0.515470in}}{\pgfqpoint{0.858333in}{0.500000in}}%
\pgfpathcurveto{\pgfqpoint{0.858333in}{0.484530in}}{\pgfqpoint{0.864480in}{0.469691in}}{\pgfqpoint{0.875419in}{0.458752in}}%
\pgfpathcurveto{\pgfqpoint{0.886358in}{0.447813in}}{\pgfqpoint{0.901196in}{0.441667in}}{\pgfqpoint{0.916667in}{0.441667in}}%
\pgfpathclose%
\pgfpathmoveto{\pgfqpoint{0.916667in}{0.447500in}}%
\pgfpathcurveto{\pgfqpoint{0.916667in}{0.447500in}}{\pgfqpoint{0.902744in}{0.447500in}}{\pgfqpoint{0.889389in}{0.453032in}}%
\pgfpathcurveto{\pgfqpoint{0.879544in}{0.462877in}}{\pgfqpoint{0.869698in}{0.472722in}}{\pgfqpoint{0.864167in}{0.486077in}}%
\pgfpathcurveto{\pgfqpoint{0.864167in}{0.500000in}}{\pgfqpoint{0.864167in}{0.513923in}}{\pgfqpoint{0.869698in}{0.527278in}}%
\pgfpathcurveto{\pgfqpoint{0.879544in}{0.537123in}}{\pgfqpoint{0.889389in}{0.546968in}}{\pgfqpoint{0.902744in}{0.552500in}}%
\pgfpathcurveto{\pgfqpoint{0.916667in}{0.552500in}}{\pgfqpoint{0.930590in}{0.552500in}}{\pgfqpoint{0.943945in}{0.546968in}}%
\pgfpathcurveto{\pgfqpoint{0.953790in}{0.537123in}}{\pgfqpoint{0.963635in}{0.527278in}}{\pgfqpoint{0.969167in}{0.513923in}}%
\pgfpathcurveto{\pgfqpoint{0.969167in}{0.500000in}}{\pgfqpoint{0.969167in}{0.486077in}}{\pgfqpoint{0.963635in}{0.472722in}}%
\pgfpathcurveto{\pgfqpoint{0.953790in}{0.462877in}}{\pgfqpoint{0.943945in}{0.453032in}}{\pgfqpoint{0.930590in}{0.447500in}}%
\pgfpathclose%
\pgfpathmoveto{\pgfqpoint{0.000000in}{0.608333in}}%
\pgfpathcurveto{\pgfqpoint{0.015470in}{0.608333in}}{\pgfqpoint{0.030309in}{0.614480in}}{\pgfqpoint{0.041248in}{0.625419in}}%
\pgfpathcurveto{\pgfqpoint{0.052187in}{0.636358in}}{\pgfqpoint{0.058333in}{0.651196in}}{\pgfqpoint{0.058333in}{0.666667in}}%
\pgfpathcurveto{\pgfqpoint{0.058333in}{0.682137in}}{\pgfqpoint{0.052187in}{0.696975in}}{\pgfqpoint{0.041248in}{0.707915in}}%
\pgfpathcurveto{\pgfqpoint{0.030309in}{0.718854in}}{\pgfqpoint{0.015470in}{0.725000in}}{\pgfqpoint{0.000000in}{0.725000in}}%
\pgfpathcurveto{\pgfqpoint{-0.015470in}{0.725000in}}{\pgfqpoint{-0.030309in}{0.718854in}}{\pgfqpoint{-0.041248in}{0.707915in}}%
\pgfpathcurveto{\pgfqpoint{-0.052187in}{0.696975in}}{\pgfqpoint{-0.058333in}{0.682137in}}{\pgfqpoint{-0.058333in}{0.666667in}}%
\pgfpathcurveto{\pgfqpoint{-0.058333in}{0.651196in}}{\pgfqpoint{-0.052187in}{0.636358in}}{\pgfqpoint{-0.041248in}{0.625419in}}%
\pgfpathcurveto{\pgfqpoint{-0.030309in}{0.614480in}}{\pgfqpoint{-0.015470in}{0.608333in}}{\pgfqpoint{0.000000in}{0.608333in}}%
\pgfpathclose%
\pgfpathmoveto{\pgfqpoint{0.000000in}{0.614167in}}%
\pgfpathcurveto{\pgfqpoint{0.000000in}{0.614167in}}{\pgfqpoint{-0.013923in}{0.614167in}}{\pgfqpoint{-0.027278in}{0.619698in}}%
\pgfpathcurveto{\pgfqpoint{-0.037123in}{0.629544in}}{\pgfqpoint{-0.046968in}{0.639389in}}{\pgfqpoint{-0.052500in}{0.652744in}}%
\pgfpathcurveto{\pgfqpoint{-0.052500in}{0.666667in}}{\pgfqpoint{-0.052500in}{0.680590in}}{\pgfqpoint{-0.046968in}{0.693945in}}%
\pgfpathcurveto{\pgfqpoint{-0.037123in}{0.703790in}}{\pgfqpoint{-0.027278in}{0.713635in}}{\pgfqpoint{-0.013923in}{0.719167in}}%
\pgfpathcurveto{\pgfqpoint{0.000000in}{0.719167in}}{\pgfqpoint{0.013923in}{0.719167in}}{\pgfqpoint{0.027278in}{0.713635in}}%
\pgfpathcurveto{\pgfqpoint{0.037123in}{0.703790in}}{\pgfqpoint{0.046968in}{0.693945in}}{\pgfqpoint{0.052500in}{0.680590in}}%
\pgfpathcurveto{\pgfqpoint{0.052500in}{0.666667in}}{\pgfqpoint{0.052500in}{0.652744in}}{\pgfqpoint{0.046968in}{0.639389in}}%
\pgfpathcurveto{\pgfqpoint{0.037123in}{0.629544in}}{\pgfqpoint{0.027278in}{0.619698in}}{\pgfqpoint{0.013923in}{0.614167in}}%
\pgfpathclose%
\pgfpathmoveto{\pgfqpoint{0.166667in}{0.608333in}}%
\pgfpathcurveto{\pgfqpoint{0.182137in}{0.608333in}}{\pgfqpoint{0.196975in}{0.614480in}}{\pgfqpoint{0.207915in}{0.625419in}}%
\pgfpathcurveto{\pgfqpoint{0.218854in}{0.636358in}}{\pgfqpoint{0.225000in}{0.651196in}}{\pgfqpoint{0.225000in}{0.666667in}}%
\pgfpathcurveto{\pgfqpoint{0.225000in}{0.682137in}}{\pgfqpoint{0.218854in}{0.696975in}}{\pgfqpoint{0.207915in}{0.707915in}}%
\pgfpathcurveto{\pgfqpoint{0.196975in}{0.718854in}}{\pgfqpoint{0.182137in}{0.725000in}}{\pgfqpoint{0.166667in}{0.725000in}}%
\pgfpathcurveto{\pgfqpoint{0.151196in}{0.725000in}}{\pgfqpoint{0.136358in}{0.718854in}}{\pgfqpoint{0.125419in}{0.707915in}}%
\pgfpathcurveto{\pgfqpoint{0.114480in}{0.696975in}}{\pgfqpoint{0.108333in}{0.682137in}}{\pgfqpoint{0.108333in}{0.666667in}}%
\pgfpathcurveto{\pgfqpoint{0.108333in}{0.651196in}}{\pgfqpoint{0.114480in}{0.636358in}}{\pgfqpoint{0.125419in}{0.625419in}}%
\pgfpathcurveto{\pgfqpoint{0.136358in}{0.614480in}}{\pgfqpoint{0.151196in}{0.608333in}}{\pgfqpoint{0.166667in}{0.608333in}}%
\pgfpathclose%
\pgfpathmoveto{\pgfqpoint{0.166667in}{0.614167in}}%
\pgfpathcurveto{\pgfqpoint{0.166667in}{0.614167in}}{\pgfqpoint{0.152744in}{0.614167in}}{\pgfqpoint{0.139389in}{0.619698in}}%
\pgfpathcurveto{\pgfqpoint{0.129544in}{0.629544in}}{\pgfqpoint{0.119698in}{0.639389in}}{\pgfqpoint{0.114167in}{0.652744in}}%
\pgfpathcurveto{\pgfqpoint{0.114167in}{0.666667in}}{\pgfqpoint{0.114167in}{0.680590in}}{\pgfqpoint{0.119698in}{0.693945in}}%
\pgfpathcurveto{\pgfqpoint{0.129544in}{0.703790in}}{\pgfqpoint{0.139389in}{0.713635in}}{\pgfqpoint{0.152744in}{0.719167in}}%
\pgfpathcurveto{\pgfqpoint{0.166667in}{0.719167in}}{\pgfqpoint{0.180590in}{0.719167in}}{\pgfqpoint{0.193945in}{0.713635in}}%
\pgfpathcurveto{\pgfqpoint{0.203790in}{0.703790in}}{\pgfqpoint{0.213635in}{0.693945in}}{\pgfqpoint{0.219167in}{0.680590in}}%
\pgfpathcurveto{\pgfqpoint{0.219167in}{0.666667in}}{\pgfqpoint{0.219167in}{0.652744in}}{\pgfqpoint{0.213635in}{0.639389in}}%
\pgfpathcurveto{\pgfqpoint{0.203790in}{0.629544in}}{\pgfqpoint{0.193945in}{0.619698in}}{\pgfqpoint{0.180590in}{0.614167in}}%
\pgfpathclose%
\pgfpathmoveto{\pgfqpoint{0.333333in}{0.608333in}}%
\pgfpathcurveto{\pgfqpoint{0.348804in}{0.608333in}}{\pgfqpoint{0.363642in}{0.614480in}}{\pgfqpoint{0.374581in}{0.625419in}}%
\pgfpathcurveto{\pgfqpoint{0.385520in}{0.636358in}}{\pgfqpoint{0.391667in}{0.651196in}}{\pgfqpoint{0.391667in}{0.666667in}}%
\pgfpathcurveto{\pgfqpoint{0.391667in}{0.682137in}}{\pgfqpoint{0.385520in}{0.696975in}}{\pgfqpoint{0.374581in}{0.707915in}}%
\pgfpathcurveto{\pgfqpoint{0.363642in}{0.718854in}}{\pgfqpoint{0.348804in}{0.725000in}}{\pgfqpoint{0.333333in}{0.725000in}}%
\pgfpathcurveto{\pgfqpoint{0.317863in}{0.725000in}}{\pgfqpoint{0.303025in}{0.718854in}}{\pgfqpoint{0.292085in}{0.707915in}}%
\pgfpathcurveto{\pgfqpoint{0.281146in}{0.696975in}}{\pgfqpoint{0.275000in}{0.682137in}}{\pgfqpoint{0.275000in}{0.666667in}}%
\pgfpathcurveto{\pgfqpoint{0.275000in}{0.651196in}}{\pgfqpoint{0.281146in}{0.636358in}}{\pgfqpoint{0.292085in}{0.625419in}}%
\pgfpathcurveto{\pgfqpoint{0.303025in}{0.614480in}}{\pgfqpoint{0.317863in}{0.608333in}}{\pgfqpoint{0.333333in}{0.608333in}}%
\pgfpathclose%
\pgfpathmoveto{\pgfqpoint{0.333333in}{0.614167in}}%
\pgfpathcurveto{\pgfqpoint{0.333333in}{0.614167in}}{\pgfqpoint{0.319410in}{0.614167in}}{\pgfqpoint{0.306055in}{0.619698in}}%
\pgfpathcurveto{\pgfqpoint{0.296210in}{0.629544in}}{\pgfqpoint{0.286365in}{0.639389in}}{\pgfqpoint{0.280833in}{0.652744in}}%
\pgfpathcurveto{\pgfqpoint{0.280833in}{0.666667in}}{\pgfqpoint{0.280833in}{0.680590in}}{\pgfqpoint{0.286365in}{0.693945in}}%
\pgfpathcurveto{\pgfqpoint{0.296210in}{0.703790in}}{\pgfqpoint{0.306055in}{0.713635in}}{\pgfqpoint{0.319410in}{0.719167in}}%
\pgfpathcurveto{\pgfqpoint{0.333333in}{0.719167in}}{\pgfqpoint{0.347256in}{0.719167in}}{\pgfqpoint{0.360611in}{0.713635in}}%
\pgfpathcurveto{\pgfqpoint{0.370456in}{0.703790in}}{\pgfqpoint{0.380302in}{0.693945in}}{\pgfqpoint{0.385833in}{0.680590in}}%
\pgfpathcurveto{\pgfqpoint{0.385833in}{0.666667in}}{\pgfqpoint{0.385833in}{0.652744in}}{\pgfqpoint{0.380302in}{0.639389in}}%
\pgfpathcurveto{\pgfqpoint{0.370456in}{0.629544in}}{\pgfqpoint{0.360611in}{0.619698in}}{\pgfqpoint{0.347256in}{0.614167in}}%
\pgfpathclose%
\pgfpathmoveto{\pgfqpoint{0.500000in}{0.608333in}}%
\pgfpathcurveto{\pgfqpoint{0.515470in}{0.608333in}}{\pgfqpoint{0.530309in}{0.614480in}}{\pgfqpoint{0.541248in}{0.625419in}}%
\pgfpathcurveto{\pgfqpoint{0.552187in}{0.636358in}}{\pgfqpoint{0.558333in}{0.651196in}}{\pgfqpoint{0.558333in}{0.666667in}}%
\pgfpathcurveto{\pgfqpoint{0.558333in}{0.682137in}}{\pgfqpoint{0.552187in}{0.696975in}}{\pgfqpoint{0.541248in}{0.707915in}}%
\pgfpathcurveto{\pgfqpoint{0.530309in}{0.718854in}}{\pgfqpoint{0.515470in}{0.725000in}}{\pgfqpoint{0.500000in}{0.725000in}}%
\pgfpathcurveto{\pgfqpoint{0.484530in}{0.725000in}}{\pgfqpoint{0.469691in}{0.718854in}}{\pgfqpoint{0.458752in}{0.707915in}}%
\pgfpathcurveto{\pgfqpoint{0.447813in}{0.696975in}}{\pgfqpoint{0.441667in}{0.682137in}}{\pgfqpoint{0.441667in}{0.666667in}}%
\pgfpathcurveto{\pgfqpoint{0.441667in}{0.651196in}}{\pgfqpoint{0.447813in}{0.636358in}}{\pgfqpoint{0.458752in}{0.625419in}}%
\pgfpathcurveto{\pgfqpoint{0.469691in}{0.614480in}}{\pgfqpoint{0.484530in}{0.608333in}}{\pgfqpoint{0.500000in}{0.608333in}}%
\pgfpathclose%
\pgfpathmoveto{\pgfqpoint{0.500000in}{0.614167in}}%
\pgfpathcurveto{\pgfqpoint{0.500000in}{0.614167in}}{\pgfqpoint{0.486077in}{0.614167in}}{\pgfqpoint{0.472722in}{0.619698in}}%
\pgfpathcurveto{\pgfqpoint{0.462877in}{0.629544in}}{\pgfqpoint{0.453032in}{0.639389in}}{\pgfqpoint{0.447500in}{0.652744in}}%
\pgfpathcurveto{\pgfqpoint{0.447500in}{0.666667in}}{\pgfqpoint{0.447500in}{0.680590in}}{\pgfqpoint{0.453032in}{0.693945in}}%
\pgfpathcurveto{\pgfqpoint{0.462877in}{0.703790in}}{\pgfqpoint{0.472722in}{0.713635in}}{\pgfqpoint{0.486077in}{0.719167in}}%
\pgfpathcurveto{\pgfqpoint{0.500000in}{0.719167in}}{\pgfqpoint{0.513923in}{0.719167in}}{\pgfqpoint{0.527278in}{0.713635in}}%
\pgfpathcurveto{\pgfqpoint{0.537123in}{0.703790in}}{\pgfqpoint{0.546968in}{0.693945in}}{\pgfqpoint{0.552500in}{0.680590in}}%
\pgfpathcurveto{\pgfqpoint{0.552500in}{0.666667in}}{\pgfqpoint{0.552500in}{0.652744in}}{\pgfqpoint{0.546968in}{0.639389in}}%
\pgfpathcurveto{\pgfqpoint{0.537123in}{0.629544in}}{\pgfqpoint{0.527278in}{0.619698in}}{\pgfqpoint{0.513923in}{0.614167in}}%
\pgfpathclose%
\pgfpathmoveto{\pgfqpoint{0.666667in}{0.608333in}}%
\pgfpathcurveto{\pgfqpoint{0.682137in}{0.608333in}}{\pgfqpoint{0.696975in}{0.614480in}}{\pgfqpoint{0.707915in}{0.625419in}}%
\pgfpathcurveto{\pgfqpoint{0.718854in}{0.636358in}}{\pgfqpoint{0.725000in}{0.651196in}}{\pgfqpoint{0.725000in}{0.666667in}}%
\pgfpathcurveto{\pgfqpoint{0.725000in}{0.682137in}}{\pgfqpoint{0.718854in}{0.696975in}}{\pgfqpoint{0.707915in}{0.707915in}}%
\pgfpathcurveto{\pgfqpoint{0.696975in}{0.718854in}}{\pgfqpoint{0.682137in}{0.725000in}}{\pgfqpoint{0.666667in}{0.725000in}}%
\pgfpathcurveto{\pgfqpoint{0.651196in}{0.725000in}}{\pgfqpoint{0.636358in}{0.718854in}}{\pgfqpoint{0.625419in}{0.707915in}}%
\pgfpathcurveto{\pgfqpoint{0.614480in}{0.696975in}}{\pgfqpoint{0.608333in}{0.682137in}}{\pgfqpoint{0.608333in}{0.666667in}}%
\pgfpathcurveto{\pgfqpoint{0.608333in}{0.651196in}}{\pgfqpoint{0.614480in}{0.636358in}}{\pgfqpoint{0.625419in}{0.625419in}}%
\pgfpathcurveto{\pgfqpoint{0.636358in}{0.614480in}}{\pgfqpoint{0.651196in}{0.608333in}}{\pgfqpoint{0.666667in}{0.608333in}}%
\pgfpathclose%
\pgfpathmoveto{\pgfqpoint{0.666667in}{0.614167in}}%
\pgfpathcurveto{\pgfqpoint{0.666667in}{0.614167in}}{\pgfqpoint{0.652744in}{0.614167in}}{\pgfqpoint{0.639389in}{0.619698in}}%
\pgfpathcurveto{\pgfqpoint{0.629544in}{0.629544in}}{\pgfqpoint{0.619698in}{0.639389in}}{\pgfqpoint{0.614167in}{0.652744in}}%
\pgfpathcurveto{\pgfqpoint{0.614167in}{0.666667in}}{\pgfqpoint{0.614167in}{0.680590in}}{\pgfqpoint{0.619698in}{0.693945in}}%
\pgfpathcurveto{\pgfqpoint{0.629544in}{0.703790in}}{\pgfqpoint{0.639389in}{0.713635in}}{\pgfqpoint{0.652744in}{0.719167in}}%
\pgfpathcurveto{\pgfqpoint{0.666667in}{0.719167in}}{\pgfqpoint{0.680590in}{0.719167in}}{\pgfqpoint{0.693945in}{0.713635in}}%
\pgfpathcurveto{\pgfqpoint{0.703790in}{0.703790in}}{\pgfqpoint{0.713635in}{0.693945in}}{\pgfqpoint{0.719167in}{0.680590in}}%
\pgfpathcurveto{\pgfqpoint{0.719167in}{0.666667in}}{\pgfqpoint{0.719167in}{0.652744in}}{\pgfqpoint{0.713635in}{0.639389in}}%
\pgfpathcurveto{\pgfqpoint{0.703790in}{0.629544in}}{\pgfqpoint{0.693945in}{0.619698in}}{\pgfqpoint{0.680590in}{0.614167in}}%
\pgfpathclose%
\pgfpathmoveto{\pgfqpoint{0.833333in}{0.608333in}}%
\pgfpathcurveto{\pgfqpoint{0.848804in}{0.608333in}}{\pgfqpoint{0.863642in}{0.614480in}}{\pgfqpoint{0.874581in}{0.625419in}}%
\pgfpathcurveto{\pgfqpoint{0.885520in}{0.636358in}}{\pgfqpoint{0.891667in}{0.651196in}}{\pgfqpoint{0.891667in}{0.666667in}}%
\pgfpathcurveto{\pgfqpoint{0.891667in}{0.682137in}}{\pgfqpoint{0.885520in}{0.696975in}}{\pgfqpoint{0.874581in}{0.707915in}}%
\pgfpathcurveto{\pgfqpoint{0.863642in}{0.718854in}}{\pgfqpoint{0.848804in}{0.725000in}}{\pgfqpoint{0.833333in}{0.725000in}}%
\pgfpathcurveto{\pgfqpoint{0.817863in}{0.725000in}}{\pgfqpoint{0.803025in}{0.718854in}}{\pgfqpoint{0.792085in}{0.707915in}}%
\pgfpathcurveto{\pgfqpoint{0.781146in}{0.696975in}}{\pgfqpoint{0.775000in}{0.682137in}}{\pgfqpoint{0.775000in}{0.666667in}}%
\pgfpathcurveto{\pgfqpoint{0.775000in}{0.651196in}}{\pgfqpoint{0.781146in}{0.636358in}}{\pgfqpoint{0.792085in}{0.625419in}}%
\pgfpathcurveto{\pgfqpoint{0.803025in}{0.614480in}}{\pgfqpoint{0.817863in}{0.608333in}}{\pgfqpoint{0.833333in}{0.608333in}}%
\pgfpathclose%
\pgfpathmoveto{\pgfqpoint{0.833333in}{0.614167in}}%
\pgfpathcurveto{\pgfqpoint{0.833333in}{0.614167in}}{\pgfqpoint{0.819410in}{0.614167in}}{\pgfqpoint{0.806055in}{0.619698in}}%
\pgfpathcurveto{\pgfqpoint{0.796210in}{0.629544in}}{\pgfqpoint{0.786365in}{0.639389in}}{\pgfqpoint{0.780833in}{0.652744in}}%
\pgfpathcurveto{\pgfqpoint{0.780833in}{0.666667in}}{\pgfqpoint{0.780833in}{0.680590in}}{\pgfqpoint{0.786365in}{0.693945in}}%
\pgfpathcurveto{\pgfqpoint{0.796210in}{0.703790in}}{\pgfqpoint{0.806055in}{0.713635in}}{\pgfqpoint{0.819410in}{0.719167in}}%
\pgfpathcurveto{\pgfqpoint{0.833333in}{0.719167in}}{\pgfqpoint{0.847256in}{0.719167in}}{\pgfqpoint{0.860611in}{0.713635in}}%
\pgfpathcurveto{\pgfqpoint{0.870456in}{0.703790in}}{\pgfqpoint{0.880302in}{0.693945in}}{\pgfqpoint{0.885833in}{0.680590in}}%
\pgfpathcurveto{\pgfqpoint{0.885833in}{0.666667in}}{\pgfqpoint{0.885833in}{0.652744in}}{\pgfqpoint{0.880302in}{0.639389in}}%
\pgfpathcurveto{\pgfqpoint{0.870456in}{0.629544in}}{\pgfqpoint{0.860611in}{0.619698in}}{\pgfqpoint{0.847256in}{0.614167in}}%
\pgfpathclose%
\pgfpathmoveto{\pgfqpoint{1.000000in}{0.608333in}}%
\pgfpathcurveto{\pgfqpoint{1.015470in}{0.608333in}}{\pgfqpoint{1.030309in}{0.614480in}}{\pgfqpoint{1.041248in}{0.625419in}}%
\pgfpathcurveto{\pgfqpoint{1.052187in}{0.636358in}}{\pgfqpoint{1.058333in}{0.651196in}}{\pgfqpoint{1.058333in}{0.666667in}}%
\pgfpathcurveto{\pgfqpoint{1.058333in}{0.682137in}}{\pgfqpoint{1.052187in}{0.696975in}}{\pgfqpoint{1.041248in}{0.707915in}}%
\pgfpathcurveto{\pgfqpoint{1.030309in}{0.718854in}}{\pgfqpoint{1.015470in}{0.725000in}}{\pgfqpoint{1.000000in}{0.725000in}}%
\pgfpathcurveto{\pgfqpoint{0.984530in}{0.725000in}}{\pgfqpoint{0.969691in}{0.718854in}}{\pgfqpoint{0.958752in}{0.707915in}}%
\pgfpathcurveto{\pgfqpoint{0.947813in}{0.696975in}}{\pgfqpoint{0.941667in}{0.682137in}}{\pgfqpoint{0.941667in}{0.666667in}}%
\pgfpathcurveto{\pgfqpoint{0.941667in}{0.651196in}}{\pgfqpoint{0.947813in}{0.636358in}}{\pgfqpoint{0.958752in}{0.625419in}}%
\pgfpathcurveto{\pgfqpoint{0.969691in}{0.614480in}}{\pgfqpoint{0.984530in}{0.608333in}}{\pgfqpoint{1.000000in}{0.608333in}}%
\pgfpathclose%
\pgfpathmoveto{\pgfqpoint{1.000000in}{0.614167in}}%
\pgfpathcurveto{\pgfqpoint{1.000000in}{0.614167in}}{\pgfqpoint{0.986077in}{0.614167in}}{\pgfqpoint{0.972722in}{0.619698in}}%
\pgfpathcurveto{\pgfqpoint{0.962877in}{0.629544in}}{\pgfqpoint{0.953032in}{0.639389in}}{\pgfqpoint{0.947500in}{0.652744in}}%
\pgfpathcurveto{\pgfqpoint{0.947500in}{0.666667in}}{\pgfqpoint{0.947500in}{0.680590in}}{\pgfqpoint{0.953032in}{0.693945in}}%
\pgfpathcurveto{\pgfqpoint{0.962877in}{0.703790in}}{\pgfqpoint{0.972722in}{0.713635in}}{\pgfqpoint{0.986077in}{0.719167in}}%
\pgfpathcurveto{\pgfqpoint{1.000000in}{0.719167in}}{\pgfqpoint{1.013923in}{0.719167in}}{\pgfqpoint{1.027278in}{0.713635in}}%
\pgfpathcurveto{\pgfqpoint{1.037123in}{0.703790in}}{\pgfqpoint{1.046968in}{0.693945in}}{\pgfqpoint{1.052500in}{0.680590in}}%
\pgfpathcurveto{\pgfqpoint{1.052500in}{0.666667in}}{\pgfqpoint{1.052500in}{0.652744in}}{\pgfqpoint{1.046968in}{0.639389in}}%
\pgfpathcurveto{\pgfqpoint{1.037123in}{0.629544in}}{\pgfqpoint{1.027278in}{0.619698in}}{\pgfqpoint{1.013923in}{0.614167in}}%
\pgfpathclose%
\pgfpathmoveto{\pgfqpoint{0.083333in}{0.775000in}}%
\pgfpathcurveto{\pgfqpoint{0.098804in}{0.775000in}}{\pgfqpoint{0.113642in}{0.781146in}}{\pgfqpoint{0.124581in}{0.792085in}}%
\pgfpathcurveto{\pgfqpoint{0.135520in}{0.803025in}}{\pgfqpoint{0.141667in}{0.817863in}}{\pgfqpoint{0.141667in}{0.833333in}}%
\pgfpathcurveto{\pgfqpoint{0.141667in}{0.848804in}}{\pgfqpoint{0.135520in}{0.863642in}}{\pgfqpoint{0.124581in}{0.874581in}}%
\pgfpathcurveto{\pgfqpoint{0.113642in}{0.885520in}}{\pgfqpoint{0.098804in}{0.891667in}}{\pgfqpoint{0.083333in}{0.891667in}}%
\pgfpathcurveto{\pgfqpoint{0.067863in}{0.891667in}}{\pgfqpoint{0.053025in}{0.885520in}}{\pgfqpoint{0.042085in}{0.874581in}}%
\pgfpathcurveto{\pgfqpoint{0.031146in}{0.863642in}}{\pgfqpoint{0.025000in}{0.848804in}}{\pgfqpoint{0.025000in}{0.833333in}}%
\pgfpathcurveto{\pgfqpoint{0.025000in}{0.817863in}}{\pgfqpoint{0.031146in}{0.803025in}}{\pgfqpoint{0.042085in}{0.792085in}}%
\pgfpathcurveto{\pgfqpoint{0.053025in}{0.781146in}}{\pgfqpoint{0.067863in}{0.775000in}}{\pgfqpoint{0.083333in}{0.775000in}}%
\pgfpathclose%
\pgfpathmoveto{\pgfqpoint{0.083333in}{0.780833in}}%
\pgfpathcurveto{\pgfqpoint{0.083333in}{0.780833in}}{\pgfqpoint{0.069410in}{0.780833in}}{\pgfqpoint{0.056055in}{0.786365in}}%
\pgfpathcurveto{\pgfqpoint{0.046210in}{0.796210in}}{\pgfqpoint{0.036365in}{0.806055in}}{\pgfqpoint{0.030833in}{0.819410in}}%
\pgfpathcurveto{\pgfqpoint{0.030833in}{0.833333in}}{\pgfqpoint{0.030833in}{0.847256in}}{\pgfqpoint{0.036365in}{0.860611in}}%
\pgfpathcurveto{\pgfqpoint{0.046210in}{0.870456in}}{\pgfqpoint{0.056055in}{0.880302in}}{\pgfqpoint{0.069410in}{0.885833in}}%
\pgfpathcurveto{\pgfqpoint{0.083333in}{0.885833in}}{\pgfqpoint{0.097256in}{0.885833in}}{\pgfqpoint{0.110611in}{0.880302in}}%
\pgfpathcurveto{\pgfqpoint{0.120456in}{0.870456in}}{\pgfqpoint{0.130302in}{0.860611in}}{\pgfqpoint{0.135833in}{0.847256in}}%
\pgfpathcurveto{\pgfqpoint{0.135833in}{0.833333in}}{\pgfqpoint{0.135833in}{0.819410in}}{\pgfqpoint{0.130302in}{0.806055in}}%
\pgfpathcurveto{\pgfqpoint{0.120456in}{0.796210in}}{\pgfqpoint{0.110611in}{0.786365in}}{\pgfqpoint{0.097256in}{0.780833in}}%
\pgfpathclose%
\pgfpathmoveto{\pgfqpoint{0.250000in}{0.775000in}}%
\pgfpathcurveto{\pgfqpoint{0.265470in}{0.775000in}}{\pgfqpoint{0.280309in}{0.781146in}}{\pgfqpoint{0.291248in}{0.792085in}}%
\pgfpathcurveto{\pgfqpoint{0.302187in}{0.803025in}}{\pgfqpoint{0.308333in}{0.817863in}}{\pgfqpoint{0.308333in}{0.833333in}}%
\pgfpathcurveto{\pgfqpoint{0.308333in}{0.848804in}}{\pgfqpoint{0.302187in}{0.863642in}}{\pgfqpoint{0.291248in}{0.874581in}}%
\pgfpathcurveto{\pgfqpoint{0.280309in}{0.885520in}}{\pgfqpoint{0.265470in}{0.891667in}}{\pgfqpoint{0.250000in}{0.891667in}}%
\pgfpathcurveto{\pgfqpoint{0.234530in}{0.891667in}}{\pgfqpoint{0.219691in}{0.885520in}}{\pgfqpoint{0.208752in}{0.874581in}}%
\pgfpathcurveto{\pgfqpoint{0.197813in}{0.863642in}}{\pgfqpoint{0.191667in}{0.848804in}}{\pgfqpoint{0.191667in}{0.833333in}}%
\pgfpathcurveto{\pgfqpoint{0.191667in}{0.817863in}}{\pgfqpoint{0.197813in}{0.803025in}}{\pgfqpoint{0.208752in}{0.792085in}}%
\pgfpathcurveto{\pgfqpoint{0.219691in}{0.781146in}}{\pgfqpoint{0.234530in}{0.775000in}}{\pgfqpoint{0.250000in}{0.775000in}}%
\pgfpathclose%
\pgfpathmoveto{\pgfqpoint{0.250000in}{0.780833in}}%
\pgfpathcurveto{\pgfqpoint{0.250000in}{0.780833in}}{\pgfqpoint{0.236077in}{0.780833in}}{\pgfqpoint{0.222722in}{0.786365in}}%
\pgfpathcurveto{\pgfqpoint{0.212877in}{0.796210in}}{\pgfqpoint{0.203032in}{0.806055in}}{\pgfqpoint{0.197500in}{0.819410in}}%
\pgfpathcurveto{\pgfqpoint{0.197500in}{0.833333in}}{\pgfqpoint{0.197500in}{0.847256in}}{\pgfqpoint{0.203032in}{0.860611in}}%
\pgfpathcurveto{\pgfqpoint{0.212877in}{0.870456in}}{\pgfqpoint{0.222722in}{0.880302in}}{\pgfqpoint{0.236077in}{0.885833in}}%
\pgfpathcurveto{\pgfqpoint{0.250000in}{0.885833in}}{\pgfqpoint{0.263923in}{0.885833in}}{\pgfqpoint{0.277278in}{0.880302in}}%
\pgfpathcurveto{\pgfqpoint{0.287123in}{0.870456in}}{\pgfqpoint{0.296968in}{0.860611in}}{\pgfqpoint{0.302500in}{0.847256in}}%
\pgfpathcurveto{\pgfqpoint{0.302500in}{0.833333in}}{\pgfqpoint{0.302500in}{0.819410in}}{\pgfqpoint{0.296968in}{0.806055in}}%
\pgfpathcurveto{\pgfqpoint{0.287123in}{0.796210in}}{\pgfqpoint{0.277278in}{0.786365in}}{\pgfqpoint{0.263923in}{0.780833in}}%
\pgfpathclose%
\pgfpathmoveto{\pgfqpoint{0.416667in}{0.775000in}}%
\pgfpathcurveto{\pgfqpoint{0.432137in}{0.775000in}}{\pgfqpoint{0.446975in}{0.781146in}}{\pgfqpoint{0.457915in}{0.792085in}}%
\pgfpathcurveto{\pgfqpoint{0.468854in}{0.803025in}}{\pgfqpoint{0.475000in}{0.817863in}}{\pgfqpoint{0.475000in}{0.833333in}}%
\pgfpathcurveto{\pgfqpoint{0.475000in}{0.848804in}}{\pgfqpoint{0.468854in}{0.863642in}}{\pgfqpoint{0.457915in}{0.874581in}}%
\pgfpathcurveto{\pgfqpoint{0.446975in}{0.885520in}}{\pgfqpoint{0.432137in}{0.891667in}}{\pgfqpoint{0.416667in}{0.891667in}}%
\pgfpathcurveto{\pgfqpoint{0.401196in}{0.891667in}}{\pgfqpoint{0.386358in}{0.885520in}}{\pgfqpoint{0.375419in}{0.874581in}}%
\pgfpathcurveto{\pgfqpoint{0.364480in}{0.863642in}}{\pgfqpoint{0.358333in}{0.848804in}}{\pgfqpoint{0.358333in}{0.833333in}}%
\pgfpathcurveto{\pgfqpoint{0.358333in}{0.817863in}}{\pgfqpoint{0.364480in}{0.803025in}}{\pgfqpoint{0.375419in}{0.792085in}}%
\pgfpathcurveto{\pgfqpoint{0.386358in}{0.781146in}}{\pgfqpoint{0.401196in}{0.775000in}}{\pgfqpoint{0.416667in}{0.775000in}}%
\pgfpathclose%
\pgfpathmoveto{\pgfqpoint{0.416667in}{0.780833in}}%
\pgfpathcurveto{\pgfqpoint{0.416667in}{0.780833in}}{\pgfqpoint{0.402744in}{0.780833in}}{\pgfqpoint{0.389389in}{0.786365in}}%
\pgfpathcurveto{\pgfqpoint{0.379544in}{0.796210in}}{\pgfqpoint{0.369698in}{0.806055in}}{\pgfqpoint{0.364167in}{0.819410in}}%
\pgfpathcurveto{\pgfqpoint{0.364167in}{0.833333in}}{\pgfqpoint{0.364167in}{0.847256in}}{\pgfqpoint{0.369698in}{0.860611in}}%
\pgfpathcurveto{\pgfqpoint{0.379544in}{0.870456in}}{\pgfqpoint{0.389389in}{0.880302in}}{\pgfqpoint{0.402744in}{0.885833in}}%
\pgfpathcurveto{\pgfqpoint{0.416667in}{0.885833in}}{\pgfqpoint{0.430590in}{0.885833in}}{\pgfqpoint{0.443945in}{0.880302in}}%
\pgfpathcurveto{\pgfqpoint{0.453790in}{0.870456in}}{\pgfqpoint{0.463635in}{0.860611in}}{\pgfqpoint{0.469167in}{0.847256in}}%
\pgfpathcurveto{\pgfqpoint{0.469167in}{0.833333in}}{\pgfqpoint{0.469167in}{0.819410in}}{\pgfqpoint{0.463635in}{0.806055in}}%
\pgfpathcurveto{\pgfqpoint{0.453790in}{0.796210in}}{\pgfqpoint{0.443945in}{0.786365in}}{\pgfqpoint{0.430590in}{0.780833in}}%
\pgfpathclose%
\pgfpathmoveto{\pgfqpoint{0.583333in}{0.775000in}}%
\pgfpathcurveto{\pgfqpoint{0.598804in}{0.775000in}}{\pgfqpoint{0.613642in}{0.781146in}}{\pgfqpoint{0.624581in}{0.792085in}}%
\pgfpathcurveto{\pgfqpoint{0.635520in}{0.803025in}}{\pgfqpoint{0.641667in}{0.817863in}}{\pgfqpoint{0.641667in}{0.833333in}}%
\pgfpathcurveto{\pgfqpoint{0.641667in}{0.848804in}}{\pgfqpoint{0.635520in}{0.863642in}}{\pgfqpoint{0.624581in}{0.874581in}}%
\pgfpathcurveto{\pgfqpoint{0.613642in}{0.885520in}}{\pgfqpoint{0.598804in}{0.891667in}}{\pgfqpoint{0.583333in}{0.891667in}}%
\pgfpathcurveto{\pgfqpoint{0.567863in}{0.891667in}}{\pgfqpoint{0.553025in}{0.885520in}}{\pgfqpoint{0.542085in}{0.874581in}}%
\pgfpathcurveto{\pgfqpoint{0.531146in}{0.863642in}}{\pgfqpoint{0.525000in}{0.848804in}}{\pgfqpoint{0.525000in}{0.833333in}}%
\pgfpathcurveto{\pgfqpoint{0.525000in}{0.817863in}}{\pgfqpoint{0.531146in}{0.803025in}}{\pgfqpoint{0.542085in}{0.792085in}}%
\pgfpathcurveto{\pgfqpoint{0.553025in}{0.781146in}}{\pgfqpoint{0.567863in}{0.775000in}}{\pgfqpoint{0.583333in}{0.775000in}}%
\pgfpathclose%
\pgfpathmoveto{\pgfqpoint{0.583333in}{0.780833in}}%
\pgfpathcurveto{\pgfqpoint{0.583333in}{0.780833in}}{\pgfqpoint{0.569410in}{0.780833in}}{\pgfqpoint{0.556055in}{0.786365in}}%
\pgfpathcurveto{\pgfqpoint{0.546210in}{0.796210in}}{\pgfqpoint{0.536365in}{0.806055in}}{\pgfqpoint{0.530833in}{0.819410in}}%
\pgfpathcurveto{\pgfqpoint{0.530833in}{0.833333in}}{\pgfqpoint{0.530833in}{0.847256in}}{\pgfqpoint{0.536365in}{0.860611in}}%
\pgfpathcurveto{\pgfqpoint{0.546210in}{0.870456in}}{\pgfqpoint{0.556055in}{0.880302in}}{\pgfqpoint{0.569410in}{0.885833in}}%
\pgfpathcurveto{\pgfqpoint{0.583333in}{0.885833in}}{\pgfqpoint{0.597256in}{0.885833in}}{\pgfqpoint{0.610611in}{0.880302in}}%
\pgfpathcurveto{\pgfqpoint{0.620456in}{0.870456in}}{\pgfqpoint{0.630302in}{0.860611in}}{\pgfqpoint{0.635833in}{0.847256in}}%
\pgfpathcurveto{\pgfqpoint{0.635833in}{0.833333in}}{\pgfqpoint{0.635833in}{0.819410in}}{\pgfqpoint{0.630302in}{0.806055in}}%
\pgfpathcurveto{\pgfqpoint{0.620456in}{0.796210in}}{\pgfqpoint{0.610611in}{0.786365in}}{\pgfqpoint{0.597256in}{0.780833in}}%
\pgfpathclose%
\pgfpathmoveto{\pgfqpoint{0.750000in}{0.775000in}}%
\pgfpathcurveto{\pgfqpoint{0.765470in}{0.775000in}}{\pgfqpoint{0.780309in}{0.781146in}}{\pgfqpoint{0.791248in}{0.792085in}}%
\pgfpathcurveto{\pgfqpoint{0.802187in}{0.803025in}}{\pgfqpoint{0.808333in}{0.817863in}}{\pgfqpoint{0.808333in}{0.833333in}}%
\pgfpathcurveto{\pgfqpoint{0.808333in}{0.848804in}}{\pgfqpoint{0.802187in}{0.863642in}}{\pgfqpoint{0.791248in}{0.874581in}}%
\pgfpathcurveto{\pgfqpoint{0.780309in}{0.885520in}}{\pgfqpoint{0.765470in}{0.891667in}}{\pgfqpoint{0.750000in}{0.891667in}}%
\pgfpathcurveto{\pgfqpoint{0.734530in}{0.891667in}}{\pgfqpoint{0.719691in}{0.885520in}}{\pgfqpoint{0.708752in}{0.874581in}}%
\pgfpathcurveto{\pgfqpoint{0.697813in}{0.863642in}}{\pgfqpoint{0.691667in}{0.848804in}}{\pgfqpoint{0.691667in}{0.833333in}}%
\pgfpathcurveto{\pgfqpoint{0.691667in}{0.817863in}}{\pgfqpoint{0.697813in}{0.803025in}}{\pgfqpoint{0.708752in}{0.792085in}}%
\pgfpathcurveto{\pgfqpoint{0.719691in}{0.781146in}}{\pgfqpoint{0.734530in}{0.775000in}}{\pgfqpoint{0.750000in}{0.775000in}}%
\pgfpathclose%
\pgfpathmoveto{\pgfqpoint{0.750000in}{0.780833in}}%
\pgfpathcurveto{\pgfqpoint{0.750000in}{0.780833in}}{\pgfqpoint{0.736077in}{0.780833in}}{\pgfqpoint{0.722722in}{0.786365in}}%
\pgfpathcurveto{\pgfqpoint{0.712877in}{0.796210in}}{\pgfqpoint{0.703032in}{0.806055in}}{\pgfqpoint{0.697500in}{0.819410in}}%
\pgfpathcurveto{\pgfqpoint{0.697500in}{0.833333in}}{\pgfqpoint{0.697500in}{0.847256in}}{\pgfqpoint{0.703032in}{0.860611in}}%
\pgfpathcurveto{\pgfqpoint{0.712877in}{0.870456in}}{\pgfqpoint{0.722722in}{0.880302in}}{\pgfqpoint{0.736077in}{0.885833in}}%
\pgfpathcurveto{\pgfqpoint{0.750000in}{0.885833in}}{\pgfqpoint{0.763923in}{0.885833in}}{\pgfqpoint{0.777278in}{0.880302in}}%
\pgfpathcurveto{\pgfqpoint{0.787123in}{0.870456in}}{\pgfqpoint{0.796968in}{0.860611in}}{\pgfqpoint{0.802500in}{0.847256in}}%
\pgfpathcurveto{\pgfqpoint{0.802500in}{0.833333in}}{\pgfqpoint{0.802500in}{0.819410in}}{\pgfqpoint{0.796968in}{0.806055in}}%
\pgfpathcurveto{\pgfqpoint{0.787123in}{0.796210in}}{\pgfqpoint{0.777278in}{0.786365in}}{\pgfqpoint{0.763923in}{0.780833in}}%
\pgfpathclose%
\pgfpathmoveto{\pgfqpoint{0.916667in}{0.775000in}}%
\pgfpathcurveto{\pgfqpoint{0.932137in}{0.775000in}}{\pgfqpoint{0.946975in}{0.781146in}}{\pgfqpoint{0.957915in}{0.792085in}}%
\pgfpathcurveto{\pgfqpoint{0.968854in}{0.803025in}}{\pgfqpoint{0.975000in}{0.817863in}}{\pgfqpoint{0.975000in}{0.833333in}}%
\pgfpathcurveto{\pgfqpoint{0.975000in}{0.848804in}}{\pgfqpoint{0.968854in}{0.863642in}}{\pgfqpoint{0.957915in}{0.874581in}}%
\pgfpathcurveto{\pgfqpoint{0.946975in}{0.885520in}}{\pgfqpoint{0.932137in}{0.891667in}}{\pgfqpoint{0.916667in}{0.891667in}}%
\pgfpathcurveto{\pgfqpoint{0.901196in}{0.891667in}}{\pgfqpoint{0.886358in}{0.885520in}}{\pgfqpoint{0.875419in}{0.874581in}}%
\pgfpathcurveto{\pgfqpoint{0.864480in}{0.863642in}}{\pgfqpoint{0.858333in}{0.848804in}}{\pgfqpoint{0.858333in}{0.833333in}}%
\pgfpathcurveto{\pgfqpoint{0.858333in}{0.817863in}}{\pgfqpoint{0.864480in}{0.803025in}}{\pgfqpoint{0.875419in}{0.792085in}}%
\pgfpathcurveto{\pgfqpoint{0.886358in}{0.781146in}}{\pgfqpoint{0.901196in}{0.775000in}}{\pgfqpoint{0.916667in}{0.775000in}}%
\pgfpathclose%
\pgfpathmoveto{\pgfqpoint{0.916667in}{0.780833in}}%
\pgfpathcurveto{\pgfqpoint{0.916667in}{0.780833in}}{\pgfqpoint{0.902744in}{0.780833in}}{\pgfqpoint{0.889389in}{0.786365in}}%
\pgfpathcurveto{\pgfqpoint{0.879544in}{0.796210in}}{\pgfqpoint{0.869698in}{0.806055in}}{\pgfqpoint{0.864167in}{0.819410in}}%
\pgfpathcurveto{\pgfqpoint{0.864167in}{0.833333in}}{\pgfqpoint{0.864167in}{0.847256in}}{\pgfqpoint{0.869698in}{0.860611in}}%
\pgfpathcurveto{\pgfqpoint{0.879544in}{0.870456in}}{\pgfqpoint{0.889389in}{0.880302in}}{\pgfqpoint{0.902744in}{0.885833in}}%
\pgfpathcurveto{\pgfqpoint{0.916667in}{0.885833in}}{\pgfqpoint{0.930590in}{0.885833in}}{\pgfqpoint{0.943945in}{0.880302in}}%
\pgfpathcurveto{\pgfqpoint{0.953790in}{0.870456in}}{\pgfqpoint{0.963635in}{0.860611in}}{\pgfqpoint{0.969167in}{0.847256in}}%
\pgfpathcurveto{\pgfqpoint{0.969167in}{0.833333in}}{\pgfqpoint{0.969167in}{0.819410in}}{\pgfqpoint{0.963635in}{0.806055in}}%
\pgfpathcurveto{\pgfqpoint{0.953790in}{0.796210in}}{\pgfqpoint{0.943945in}{0.786365in}}{\pgfqpoint{0.930590in}{0.780833in}}%
\pgfpathclose%
\pgfpathmoveto{\pgfqpoint{0.000000in}{0.941667in}}%
\pgfpathcurveto{\pgfqpoint{0.015470in}{0.941667in}}{\pgfqpoint{0.030309in}{0.947813in}}{\pgfqpoint{0.041248in}{0.958752in}}%
\pgfpathcurveto{\pgfqpoint{0.052187in}{0.969691in}}{\pgfqpoint{0.058333in}{0.984530in}}{\pgfqpoint{0.058333in}{1.000000in}}%
\pgfpathcurveto{\pgfqpoint{0.058333in}{1.015470in}}{\pgfqpoint{0.052187in}{1.030309in}}{\pgfqpoint{0.041248in}{1.041248in}}%
\pgfpathcurveto{\pgfqpoint{0.030309in}{1.052187in}}{\pgfqpoint{0.015470in}{1.058333in}}{\pgfqpoint{0.000000in}{1.058333in}}%
\pgfpathcurveto{\pgfqpoint{-0.015470in}{1.058333in}}{\pgfqpoint{-0.030309in}{1.052187in}}{\pgfqpoint{-0.041248in}{1.041248in}}%
\pgfpathcurveto{\pgfqpoint{-0.052187in}{1.030309in}}{\pgfqpoint{-0.058333in}{1.015470in}}{\pgfqpoint{-0.058333in}{1.000000in}}%
\pgfpathcurveto{\pgfqpoint{-0.058333in}{0.984530in}}{\pgfqpoint{-0.052187in}{0.969691in}}{\pgfqpoint{-0.041248in}{0.958752in}}%
\pgfpathcurveto{\pgfqpoint{-0.030309in}{0.947813in}}{\pgfqpoint{-0.015470in}{0.941667in}}{\pgfqpoint{0.000000in}{0.941667in}}%
\pgfpathclose%
\pgfpathmoveto{\pgfqpoint{0.000000in}{0.947500in}}%
\pgfpathcurveto{\pgfqpoint{0.000000in}{0.947500in}}{\pgfqpoint{-0.013923in}{0.947500in}}{\pgfqpoint{-0.027278in}{0.953032in}}%
\pgfpathcurveto{\pgfqpoint{-0.037123in}{0.962877in}}{\pgfqpoint{-0.046968in}{0.972722in}}{\pgfqpoint{-0.052500in}{0.986077in}}%
\pgfpathcurveto{\pgfqpoint{-0.052500in}{1.000000in}}{\pgfqpoint{-0.052500in}{1.013923in}}{\pgfqpoint{-0.046968in}{1.027278in}}%
\pgfpathcurveto{\pgfqpoint{-0.037123in}{1.037123in}}{\pgfqpoint{-0.027278in}{1.046968in}}{\pgfqpoint{-0.013923in}{1.052500in}}%
\pgfpathcurveto{\pgfqpoint{0.000000in}{1.052500in}}{\pgfqpoint{0.013923in}{1.052500in}}{\pgfqpoint{0.027278in}{1.046968in}}%
\pgfpathcurveto{\pgfqpoint{0.037123in}{1.037123in}}{\pgfqpoint{0.046968in}{1.027278in}}{\pgfqpoint{0.052500in}{1.013923in}}%
\pgfpathcurveto{\pgfqpoint{0.052500in}{1.000000in}}{\pgfqpoint{0.052500in}{0.986077in}}{\pgfqpoint{0.046968in}{0.972722in}}%
\pgfpathcurveto{\pgfqpoint{0.037123in}{0.962877in}}{\pgfqpoint{0.027278in}{0.953032in}}{\pgfqpoint{0.013923in}{0.947500in}}%
\pgfpathclose%
\pgfpathmoveto{\pgfqpoint{0.166667in}{0.941667in}}%
\pgfpathcurveto{\pgfqpoint{0.182137in}{0.941667in}}{\pgfqpoint{0.196975in}{0.947813in}}{\pgfqpoint{0.207915in}{0.958752in}}%
\pgfpathcurveto{\pgfqpoint{0.218854in}{0.969691in}}{\pgfqpoint{0.225000in}{0.984530in}}{\pgfqpoint{0.225000in}{1.000000in}}%
\pgfpathcurveto{\pgfqpoint{0.225000in}{1.015470in}}{\pgfqpoint{0.218854in}{1.030309in}}{\pgfqpoint{0.207915in}{1.041248in}}%
\pgfpathcurveto{\pgfqpoint{0.196975in}{1.052187in}}{\pgfqpoint{0.182137in}{1.058333in}}{\pgfqpoint{0.166667in}{1.058333in}}%
\pgfpathcurveto{\pgfqpoint{0.151196in}{1.058333in}}{\pgfqpoint{0.136358in}{1.052187in}}{\pgfqpoint{0.125419in}{1.041248in}}%
\pgfpathcurveto{\pgfqpoint{0.114480in}{1.030309in}}{\pgfqpoint{0.108333in}{1.015470in}}{\pgfqpoint{0.108333in}{1.000000in}}%
\pgfpathcurveto{\pgfqpoint{0.108333in}{0.984530in}}{\pgfqpoint{0.114480in}{0.969691in}}{\pgfqpoint{0.125419in}{0.958752in}}%
\pgfpathcurveto{\pgfqpoint{0.136358in}{0.947813in}}{\pgfqpoint{0.151196in}{0.941667in}}{\pgfqpoint{0.166667in}{0.941667in}}%
\pgfpathclose%
\pgfpathmoveto{\pgfqpoint{0.166667in}{0.947500in}}%
\pgfpathcurveto{\pgfqpoint{0.166667in}{0.947500in}}{\pgfqpoint{0.152744in}{0.947500in}}{\pgfqpoint{0.139389in}{0.953032in}}%
\pgfpathcurveto{\pgfqpoint{0.129544in}{0.962877in}}{\pgfqpoint{0.119698in}{0.972722in}}{\pgfqpoint{0.114167in}{0.986077in}}%
\pgfpathcurveto{\pgfqpoint{0.114167in}{1.000000in}}{\pgfqpoint{0.114167in}{1.013923in}}{\pgfqpoint{0.119698in}{1.027278in}}%
\pgfpathcurveto{\pgfqpoint{0.129544in}{1.037123in}}{\pgfqpoint{0.139389in}{1.046968in}}{\pgfqpoint{0.152744in}{1.052500in}}%
\pgfpathcurveto{\pgfqpoint{0.166667in}{1.052500in}}{\pgfqpoint{0.180590in}{1.052500in}}{\pgfqpoint{0.193945in}{1.046968in}}%
\pgfpathcurveto{\pgfqpoint{0.203790in}{1.037123in}}{\pgfqpoint{0.213635in}{1.027278in}}{\pgfqpoint{0.219167in}{1.013923in}}%
\pgfpathcurveto{\pgfqpoint{0.219167in}{1.000000in}}{\pgfqpoint{0.219167in}{0.986077in}}{\pgfqpoint{0.213635in}{0.972722in}}%
\pgfpathcurveto{\pgfqpoint{0.203790in}{0.962877in}}{\pgfqpoint{0.193945in}{0.953032in}}{\pgfqpoint{0.180590in}{0.947500in}}%
\pgfpathclose%
\pgfpathmoveto{\pgfqpoint{0.333333in}{0.941667in}}%
\pgfpathcurveto{\pgfqpoint{0.348804in}{0.941667in}}{\pgfqpoint{0.363642in}{0.947813in}}{\pgfqpoint{0.374581in}{0.958752in}}%
\pgfpathcurveto{\pgfqpoint{0.385520in}{0.969691in}}{\pgfqpoint{0.391667in}{0.984530in}}{\pgfqpoint{0.391667in}{1.000000in}}%
\pgfpathcurveto{\pgfqpoint{0.391667in}{1.015470in}}{\pgfqpoint{0.385520in}{1.030309in}}{\pgfqpoint{0.374581in}{1.041248in}}%
\pgfpathcurveto{\pgfqpoint{0.363642in}{1.052187in}}{\pgfqpoint{0.348804in}{1.058333in}}{\pgfqpoint{0.333333in}{1.058333in}}%
\pgfpathcurveto{\pgfqpoint{0.317863in}{1.058333in}}{\pgfqpoint{0.303025in}{1.052187in}}{\pgfqpoint{0.292085in}{1.041248in}}%
\pgfpathcurveto{\pgfqpoint{0.281146in}{1.030309in}}{\pgfqpoint{0.275000in}{1.015470in}}{\pgfqpoint{0.275000in}{1.000000in}}%
\pgfpathcurveto{\pgfqpoint{0.275000in}{0.984530in}}{\pgfqpoint{0.281146in}{0.969691in}}{\pgfqpoint{0.292085in}{0.958752in}}%
\pgfpathcurveto{\pgfqpoint{0.303025in}{0.947813in}}{\pgfqpoint{0.317863in}{0.941667in}}{\pgfqpoint{0.333333in}{0.941667in}}%
\pgfpathclose%
\pgfpathmoveto{\pgfqpoint{0.333333in}{0.947500in}}%
\pgfpathcurveto{\pgfqpoint{0.333333in}{0.947500in}}{\pgfqpoint{0.319410in}{0.947500in}}{\pgfqpoint{0.306055in}{0.953032in}}%
\pgfpathcurveto{\pgfqpoint{0.296210in}{0.962877in}}{\pgfqpoint{0.286365in}{0.972722in}}{\pgfqpoint{0.280833in}{0.986077in}}%
\pgfpathcurveto{\pgfqpoint{0.280833in}{1.000000in}}{\pgfqpoint{0.280833in}{1.013923in}}{\pgfqpoint{0.286365in}{1.027278in}}%
\pgfpathcurveto{\pgfqpoint{0.296210in}{1.037123in}}{\pgfqpoint{0.306055in}{1.046968in}}{\pgfqpoint{0.319410in}{1.052500in}}%
\pgfpathcurveto{\pgfqpoint{0.333333in}{1.052500in}}{\pgfqpoint{0.347256in}{1.052500in}}{\pgfqpoint{0.360611in}{1.046968in}}%
\pgfpathcurveto{\pgfqpoint{0.370456in}{1.037123in}}{\pgfqpoint{0.380302in}{1.027278in}}{\pgfqpoint{0.385833in}{1.013923in}}%
\pgfpathcurveto{\pgfqpoint{0.385833in}{1.000000in}}{\pgfqpoint{0.385833in}{0.986077in}}{\pgfqpoint{0.380302in}{0.972722in}}%
\pgfpathcurveto{\pgfqpoint{0.370456in}{0.962877in}}{\pgfqpoint{0.360611in}{0.953032in}}{\pgfqpoint{0.347256in}{0.947500in}}%
\pgfpathclose%
\pgfpathmoveto{\pgfqpoint{0.500000in}{0.941667in}}%
\pgfpathcurveto{\pgfqpoint{0.515470in}{0.941667in}}{\pgfqpoint{0.530309in}{0.947813in}}{\pgfqpoint{0.541248in}{0.958752in}}%
\pgfpathcurveto{\pgfqpoint{0.552187in}{0.969691in}}{\pgfqpoint{0.558333in}{0.984530in}}{\pgfqpoint{0.558333in}{1.000000in}}%
\pgfpathcurveto{\pgfqpoint{0.558333in}{1.015470in}}{\pgfqpoint{0.552187in}{1.030309in}}{\pgfqpoint{0.541248in}{1.041248in}}%
\pgfpathcurveto{\pgfqpoint{0.530309in}{1.052187in}}{\pgfqpoint{0.515470in}{1.058333in}}{\pgfqpoint{0.500000in}{1.058333in}}%
\pgfpathcurveto{\pgfqpoint{0.484530in}{1.058333in}}{\pgfqpoint{0.469691in}{1.052187in}}{\pgfqpoint{0.458752in}{1.041248in}}%
\pgfpathcurveto{\pgfqpoint{0.447813in}{1.030309in}}{\pgfqpoint{0.441667in}{1.015470in}}{\pgfqpoint{0.441667in}{1.000000in}}%
\pgfpathcurveto{\pgfqpoint{0.441667in}{0.984530in}}{\pgfqpoint{0.447813in}{0.969691in}}{\pgfqpoint{0.458752in}{0.958752in}}%
\pgfpathcurveto{\pgfqpoint{0.469691in}{0.947813in}}{\pgfqpoint{0.484530in}{0.941667in}}{\pgfqpoint{0.500000in}{0.941667in}}%
\pgfpathclose%
\pgfpathmoveto{\pgfqpoint{0.500000in}{0.947500in}}%
\pgfpathcurveto{\pgfqpoint{0.500000in}{0.947500in}}{\pgfqpoint{0.486077in}{0.947500in}}{\pgfqpoint{0.472722in}{0.953032in}}%
\pgfpathcurveto{\pgfqpoint{0.462877in}{0.962877in}}{\pgfqpoint{0.453032in}{0.972722in}}{\pgfqpoint{0.447500in}{0.986077in}}%
\pgfpathcurveto{\pgfqpoint{0.447500in}{1.000000in}}{\pgfqpoint{0.447500in}{1.013923in}}{\pgfqpoint{0.453032in}{1.027278in}}%
\pgfpathcurveto{\pgfqpoint{0.462877in}{1.037123in}}{\pgfqpoint{0.472722in}{1.046968in}}{\pgfqpoint{0.486077in}{1.052500in}}%
\pgfpathcurveto{\pgfqpoint{0.500000in}{1.052500in}}{\pgfqpoint{0.513923in}{1.052500in}}{\pgfqpoint{0.527278in}{1.046968in}}%
\pgfpathcurveto{\pgfqpoint{0.537123in}{1.037123in}}{\pgfqpoint{0.546968in}{1.027278in}}{\pgfqpoint{0.552500in}{1.013923in}}%
\pgfpathcurveto{\pgfqpoint{0.552500in}{1.000000in}}{\pgfqpoint{0.552500in}{0.986077in}}{\pgfqpoint{0.546968in}{0.972722in}}%
\pgfpathcurveto{\pgfqpoint{0.537123in}{0.962877in}}{\pgfqpoint{0.527278in}{0.953032in}}{\pgfqpoint{0.513923in}{0.947500in}}%
\pgfpathclose%
\pgfpathmoveto{\pgfqpoint{0.666667in}{0.941667in}}%
\pgfpathcurveto{\pgfqpoint{0.682137in}{0.941667in}}{\pgfqpoint{0.696975in}{0.947813in}}{\pgfqpoint{0.707915in}{0.958752in}}%
\pgfpathcurveto{\pgfqpoint{0.718854in}{0.969691in}}{\pgfqpoint{0.725000in}{0.984530in}}{\pgfqpoint{0.725000in}{1.000000in}}%
\pgfpathcurveto{\pgfqpoint{0.725000in}{1.015470in}}{\pgfqpoint{0.718854in}{1.030309in}}{\pgfqpoint{0.707915in}{1.041248in}}%
\pgfpathcurveto{\pgfqpoint{0.696975in}{1.052187in}}{\pgfqpoint{0.682137in}{1.058333in}}{\pgfqpoint{0.666667in}{1.058333in}}%
\pgfpathcurveto{\pgfqpoint{0.651196in}{1.058333in}}{\pgfqpoint{0.636358in}{1.052187in}}{\pgfqpoint{0.625419in}{1.041248in}}%
\pgfpathcurveto{\pgfqpoint{0.614480in}{1.030309in}}{\pgfqpoint{0.608333in}{1.015470in}}{\pgfqpoint{0.608333in}{1.000000in}}%
\pgfpathcurveto{\pgfqpoint{0.608333in}{0.984530in}}{\pgfqpoint{0.614480in}{0.969691in}}{\pgfqpoint{0.625419in}{0.958752in}}%
\pgfpathcurveto{\pgfqpoint{0.636358in}{0.947813in}}{\pgfqpoint{0.651196in}{0.941667in}}{\pgfqpoint{0.666667in}{0.941667in}}%
\pgfpathclose%
\pgfpathmoveto{\pgfqpoint{0.666667in}{0.947500in}}%
\pgfpathcurveto{\pgfqpoint{0.666667in}{0.947500in}}{\pgfqpoint{0.652744in}{0.947500in}}{\pgfqpoint{0.639389in}{0.953032in}}%
\pgfpathcurveto{\pgfqpoint{0.629544in}{0.962877in}}{\pgfqpoint{0.619698in}{0.972722in}}{\pgfqpoint{0.614167in}{0.986077in}}%
\pgfpathcurveto{\pgfqpoint{0.614167in}{1.000000in}}{\pgfqpoint{0.614167in}{1.013923in}}{\pgfqpoint{0.619698in}{1.027278in}}%
\pgfpathcurveto{\pgfqpoint{0.629544in}{1.037123in}}{\pgfqpoint{0.639389in}{1.046968in}}{\pgfqpoint{0.652744in}{1.052500in}}%
\pgfpathcurveto{\pgfqpoint{0.666667in}{1.052500in}}{\pgfqpoint{0.680590in}{1.052500in}}{\pgfqpoint{0.693945in}{1.046968in}}%
\pgfpathcurveto{\pgfqpoint{0.703790in}{1.037123in}}{\pgfqpoint{0.713635in}{1.027278in}}{\pgfqpoint{0.719167in}{1.013923in}}%
\pgfpathcurveto{\pgfqpoint{0.719167in}{1.000000in}}{\pgfqpoint{0.719167in}{0.986077in}}{\pgfqpoint{0.713635in}{0.972722in}}%
\pgfpathcurveto{\pgfqpoint{0.703790in}{0.962877in}}{\pgfqpoint{0.693945in}{0.953032in}}{\pgfqpoint{0.680590in}{0.947500in}}%
\pgfpathclose%
\pgfpathmoveto{\pgfqpoint{0.833333in}{0.941667in}}%
\pgfpathcurveto{\pgfqpoint{0.848804in}{0.941667in}}{\pgfqpoint{0.863642in}{0.947813in}}{\pgfqpoint{0.874581in}{0.958752in}}%
\pgfpathcurveto{\pgfqpoint{0.885520in}{0.969691in}}{\pgfqpoint{0.891667in}{0.984530in}}{\pgfqpoint{0.891667in}{1.000000in}}%
\pgfpathcurveto{\pgfqpoint{0.891667in}{1.015470in}}{\pgfqpoint{0.885520in}{1.030309in}}{\pgfqpoint{0.874581in}{1.041248in}}%
\pgfpathcurveto{\pgfqpoint{0.863642in}{1.052187in}}{\pgfqpoint{0.848804in}{1.058333in}}{\pgfqpoint{0.833333in}{1.058333in}}%
\pgfpathcurveto{\pgfqpoint{0.817863in}{1.058333in}}{\pgfqpoint{0.803025in}{1.052187in}}{\pgfqpoint{0.792085in}{1.041248in}}%
\pgfpathcurveto{\pgfqpoint{0.781146in}{1.030309in}}{\pgfqpoint{0.775000in}{1.015470in}}{\pgfqpoint{0.775000in}{1.000000in}}%
\pgfpathcurveto{\pgfqpoint{0.775000in}{0.984530in}}{\pgfqpoint{0.781146in}{0.969691in}}{\pgfqpoint{0.792085in}{0.958752in}}%
\pgfpathcurveto{\pgfqpoint{0.803025in}{0.947813in}}{\pgfqpoint{0.817863in}{0.941667in}}{\pgfqpoint{0.833333in}{0.941667in}}%
\pgfpathclose%
\pgfpathmoveto{\pgfqpoint{0.833333in}{0.947500in}}%
\pgfpathcurveto{\pgfqpoint{0.833333in}{0.947500in}}{\pgfqpoint{0.819410in}{0.947500in}}{\pgfqpoint{0.806055in}{0.953032in}}%
\pgfpathcurveto{\pgfqpoint{0.796210in}{0.962877in}}{\pgfqpoint{0.786365in}{0.972722in}}{\pgfqpoint{0.780833in}{0.986077in}}%
\pgfpathcurveto{\pgfqpoint{0.780833in}{1.000000in}}{\pgfqpoint{0.780833in}{1.013923in}}{\pgfqpoint{0.786365in}{1.027278in}}%
\pgfpathcurveto{\pgfqpoint{0.796210in}{1.037123in}}{\pgfqpoint{0.806055in}{1.046968in}}{\pgfqpoint{0.819410in}{1.052500in}}%
\pgfpathcurveto{\pgfqpoint{0.833333in}{1.052500in}}{\pgfqpoint{0.847256in}{1.052500in}}{\pgfqpoint{0.860611in}{1.046968in}}%
\pgfpathcurveto{\pgfqpoint{0.870456in}{1.037123in}}{\pgfqpoint{0.880302in}{1.027278in}}{\pgfqpoint{0.885833in}{1.013923in}}%
\pgfpathcurveto{\pgfqpoint{0.885833in}{1.000000in}}{\pgfqpoint{0.885833in}{0.986077in}}{\pgfqpoint{0.880302in}{0.972722in}}%
\pgfpathcurveto{\pgfqpoint{0.870456in}{0.962877in}}{\pgfqpoint{0.860611in}{0.953032in}}{\pgfqpoint{0.847256in}{0.947500in}}%
\pgfpathclose%
\pgfpathmoveto{\pgfqpoint{1.000000in}{0.941667in}}%
\pgfpathcurveto{\pgfqpoint{1.015470in}{0.941667in}}{\pgfqpoint{1.030309in}{0.947813in}}{\pgfqpoint{1.041248in}{0.958752in}}%
\pgfpathcurveto{\pgfqpoint{1.052187in}{0.969691in}}{\pgfqpoint{1.058333in}{0.984530in}}{\pgfqpoint{1.058333in}{1.000000in}}%
\pgfpathcurveto{\pgfqpoint{1.058333in}{1.015470in}}{\pgfqpoint{1.052187in}{1.030309in}}{\pgfqpoint{1.041248in}{1.041248in}}%
\pgfpathcurveto{\pgfqpoint{1.030309in}{1.052187in}}{\pgfqpoint{1.015470in}{1.058333in}}{\pgfqpoint{1.000000in}{1.058333in}}%
\pgfpathcurveto{\pgfqpoint{0.984530in}{1.058333in}}{\pgfqpoint{0.969691in}{1.052187in}}{\pgfqpoint{0.958752in}{1.041248in}}%
\pgfpathcurveto{\pgfqpoint{0.947813in}{1.030309in}}{\pgfqpoint{0.941667in}{1.015470in}}{\pgfqpoint{0.941667in}{1.000000in}}%
\pgfpathcurveto{\pgfqpoint{0.941667in}{0.984530in}}{\pgfqpoint{0.947813in}{0.969691in}}{\pgfqpoint{0.958752in}{0.958752in}}%
\pgfpathcurveto{\pgfqpoint{0.969691in}{0.947813in}}{\pgfqpoint{0.984530in}{0.941667in}}{\pgfqpoint{1.000000in}{0.941667in}}%
\pgfpathclose%
\pgfpathmoveto{\pgfqpoint{1.000000in}{0.947500in}}%
\pgfpathcurveto{\pgfqpoint{1.000000in}{0.947500in}}{\pgfqpoint{0.986077in}{0.947500in}}{\pgfqpoint{0.972722in}{0.953032in}}%
\pgfpathcurveto{\pgfqpoint{0.962877in}{0.962877in}}{\pgfqpoint{0.953032in}{0.972722in}}{\pgfqpoint{0.947500in}{0.986077in}}%
\pgfpathcurveto{\pgfqpoint{0.947500in}{1.000000in}}{\pgfqpoint{0.947500in}{1.013923in}}{\pgfqpoint{0.953032in}{1.027278in}}%
\pgfpathcurveto{\pgfqpoint{0.962877in}{1.037123in}}{\pgfqpoint{0.972722in}{1.046968in}}{\pgfqpoint{0.986077in}{1.052500in}}%
\pgfpathcurveto{\pgfqpoint{1.000000in}{1.052500in}}{\pgfqpoint{1.013923in}{1.052500in}}{\pgfqpoint{1.027278in}{1.046968in}}%
\pgfpathcurveto{\pgfqpoint{1.037123in}{1.037123in}}{\pgfqpoint{1.046968in}{1.027278in}}{\pgfqpoint{1.052500in}{1.013923in}}%
\pgfpathcurveto{\pgfqpoint{1.052500in}{1.000000in}}{\pgfqpoint{1.052500in}{0.986077in}}{\pgfqpoint{1.046968in}{0.972722in}}%
\pgfpathcurveto{\pgfqpoint{1.037123in}{0.962877in}}{\pgfqpoint{1.027278in}{0.953032in}}{\pgfqpoint{1.013923in}{0.947500in}}%
\pgfpathclose%
\pgfusepath{stroke}%
\end{pgfscope}%
}%
\pgfsys@transformshift{4.358038in}{1.677933in}%
\pgfsys@useobject{currentpattern}{}%
\pgfsys@transformshift{1in}{0in}%
\pgfsys@transformshift{-1in}{0in}%
\pgfsys@transformshift{0in}{1in}%
\pgfsys@useobject{currentpattern}{}%
\pgfsys@transformshift{1in}{0in}%
\pgfsys@transformshift{-1in}{0in}%
\pgfsys@transformshift{0in}{1in}%
\end{pgfscope}%
\begin{pgfscope}%
\pgfpathrectangle{\pgfqpoint{0.870538in}{0.637495in}}{\pgfqpoint{9.300000in}{9.060000in}}%
\pgfusepath{clip}%
\pgfsetbuttcap%
\pgfsetmiterjoin%
\definecolor{currentfill}{rgb}{0.549020,0.337255,0.294118}%
\pgfsetfillcolor{currentfill}%
\pgfsetfillopacity{0.990000}%
\pgfsetlinewidth{0.000000pt}%
\definecolor{currentstroke}{rgb}{0.000000,0.000000,0.000000}%
\pgfsetstrokecolor{currentstroke}%
\pgfsetstrokeopacity{0.990000}%
\pgfsetdash{}{0pt}%
\pgfpathmoveto{\pgfqpoint{5.908038in}{0.875116in}}%
\pgfpathlineto{\pgfqpoint{6.683038in}{0.875116in}}%
\pgfpathlineto{\pgfqpoint{6.683038in}{3.636133in}}%
\pgfpathlineto{\pgfqpoint{5.908038in}{3.636133in}}%
\pgfpathclose%
\pgfusepath{fill}%
\end{pgfscope}%
\begin{pgfscope}%
\pgfsetbuttcap%
\pgfsetmiterjoin%
\definecolor{currentfill}{rgb}{0.549020,0.337255,0.294118}%
\pgfsetfillcolor{currentfill}%
\pgfsetfillopacity{0.990000}%
\pgfsetlinewidth{0.000000pt}%
\definecolor{currentstroke}{rgb}{0.000000,0.000000,0.000000}%
\pgfsetstrokecolor{currentstroke}%
\pgfsetstrokeopacity{0.990000}%
\pgfsetdash{}{0pt}%
\pgfpathrectangle{\pgfqpoint{0.870538in}{0.637495in}}{\pgfqpoint{9.300000in}{9.060000in}}%
\pgfusepath{clip}%
\pgfpathmoveto{\pgfqpoint{5.908038in}{0.875116in}}%
\pgfpathlineto{\pgfqpoint{6.683038in}{0.875116in}}%
\pgfpathlineto{\pgfqpoint{6.683038in}{3.636133in}}%
\pgfpathlineto{\pgfqpoint{5.908038in}{3.636133in}}%
\pgfpathclose%
\pgfusepath{clip}%
\pgfsys@defobject{currentpattern}{\pgfqpoint{0in}{0in}}{\pgfqpoint{1in}{1in}}{%
\begin{pgfscope}%
\pgfpathrectangle{\pgfqpoint{0in}{0in}}{\pgfqpoint{1in}{1in}}%
\pgfusepath{clip}%
\pgfpathmoveto{\pgfqpoint{0.000000in}{-0.058333in}}%
\pgfpathcurveto{\pgfqpoint{0.015470in}{-0.058333in}}{\pgfqpoint{0.030309in}{-0.052187in}}{\pgfqpoint{0.041248in}{-0.041248in}}%
\pgfpathcurveto{\pgfqpoint{0.052187in}{-0.030309in}}{\pgfqpoint{0.058333in}{-0.015470in}}{\pgfqpoint{0.058333in}{0.000000in}}%
\pgfpathcurveto{\pgfqpoint{0.058333in}{0.015470in}}{\pgfqpoint{0.052187in}{0.030309in}}{\pgfqpoint{0.041248in}{0.041248in}}%
\pgfpathcurveto{\pgfqpoint{0.030309in}{0.052187in}}{\pgfqpoint{0.015470in}{0.058333in}}{\pgfqpoint{0.000000in}{0.058333in}}%
\pgfpathcurveto{\pgfqpoint{-0.015470in}{0.058333in}}{\pgfqpoint{-0.030309in}{0.052187in}}{\pgfqpoint{-0.041248in}{0.041248in}}%
\pgfpathcurveto{\pgfqpoint{-0.052187in}{0.030309in}}{\pgfqpoint{-0.058333in}{0.015470in}}{\pgfqpoint{-0.058333in}{0.000000in}}%
\pgfpathcurveto{\pgfqpoint{-0.058333in}{-0.015470in}}{\pgfqpoint{-0.052187in}{-0.030309in}}{\pgfqpoint{-0.041248in}{-0.041248in}}%
\pgfpathcurveto{\pgfqpoint{-0.030309in}{-0.052187in}}{\pgfqpoint{-0.015470in}{-0.058333in}}{\pgfqpoint{0.000000in}{-0.058333in}}%
\pgfpathclose%
\pgfpathmoveto{\pgfqpoint{0.000000in}{-0.052500in}}%
\pgfpathcurveto{\pgfqpoint{0.000000in}{-0.052500in}}{\pgfqpoint{-0.013923in}{-0.052500in}}{\pgfqpoint{-0.027278in}{-0.046968in}}%
\pgfpathcurveto{\pgfqpoint{-0.037123in}{-0.037123in}}{\pgfqpoint{-0.046968in}{-0.027278in}}{\pgfqpoint{-0.052500in}{-0.013923in}}%
\pgfpathcurveto{\pgfqpoint{-0.052500in}{0.000000in}}{\pgfqpoint{-0.052500in}{0.013923in}}{\pgfqpoint{-0.046968in}{0.027278in}}%
\pgfpathcurveto{\pgfqpoint{-0.037123in}{0.037123in}}{\pgfqpoint{-0.027278in}{0.046968in}}{\pgfqpoint{-0.013923in}{0.052500in}}%
\pgfpathcurveto{\pgfqpoint{0.000000in}{0.052500in}}{\pgfqpoint{0.013923in}{0.052500in}}{\pgfqpoint{0.027278in}{0.046968in}}%
\pgfpathcurveto{\pgfqpoint{0.037123in}{0.037123in}}{\pgfqpoint{0.046968in}{0.027278in}}{\pgfqpoint{0.052500in}{0.013923in}}%
\pgfpathcurveto{\pgfqpoint{0.052500in}{0.000000in}}{\pgfqpoint{0.052500in}{-0.013923in}}{\pgfqpoint{0.046968in}{-0.027278in}}%
\pgfpathcurveto{\pgfqpoint{0.037123in}{-0.037123in}}{\pgfqpoint{0.027278in}{-0.046968in}}{\pgfqpoint{0.013923in}{-0.052500in}}%
\pgfpathclose%
\pgfpathmoveto{\pgfqpoint{0.166667in}{-0.058333in}}%
\pgfpathcurveto{\pgfqpoint{0.182137in}{-0.058333in}}{\pgfqpoint{0.196975in}{-0.052187in}}{\pgfqpoint{0.207915in}{-0.041248in}}%
\pgfpathcurveto{\pgfqpoint{0.218854in}{-0.030309in}}{\pgfqpoint{0.225000in}{-0.015470in}}{\pgfqpoint{0.225000in}{0.000000in}}%
\pgfpathcurveto{\pgfqpoint{0.225000in}{0.015470in}}{\pgfqpoint{0.218854in}{0.030309in}}{\pgfqpoint{0.207915in}{0.041248in}}%
\pgfpathcurveto{\pgfqpoint{0.196975in}{0.052187in}}{\pgfqpoint{0.182137in}{0.058333in}}{\pgfqpoint{0.166667in}{0.058333in}}%
\pgfpathcurveto{\pgfqpoint{0.151196in}{0.058333in}}{\pgfqpoint{0.136358in}{0.052187in}}{\pgfqpoint{0.125419in}{0.041248in}}%
\pgfpathcurveto{\pgfqpoint{0.114480in}{0.030309in}}{\pgfqpoint{0.108333in}{0.015470in}}{\pgfqpoint{0.108333in}{0.000000in}}%
\pgfpathcurveto{\pgfqpoint{0.108333in}{-0.015470in}}{\pgfqpoint{0.114480in}{-0.030309in}}{\pgfqpoint{0.125419in}{-0.041248in}}%
\pgfpathcurveto{\pgfqpoint{0.136358in}{-0.052187in}}{\pgfqpoint{0.151196in}{-0.058333in}}{\pgfqpoint{0.166667in}{-0.058333in}}%
\pgfpathclose%
\pgfpathmoveto{\pgfqpoint{0.166667in}{-0.052500in}}%
\pgfpathcurveto{\pgfqpoint{0.166667in}{-0.052500in}}{\pgfqpoint{0.152744in}{-0.052500in}}{\pgfqpoint{0.139389in}{-0.046968in}}%
\pgfpathcurveto{\pgfqpoint{0.129544in}{-0.037123in}}{\pgfqpoint{0.119698in}{-0.027278in}}{\pgfqpoint{0.114167in}{-0.013923in}}%
\pgfpathcurveto{\pgfqpoint{0.114167in}{0.000000in}}{\pgfqpoint{0.114167in}{0.013923in}}{\pgfqpoint{0.119698in}{0.027278in}}%
\pgfpathcurveto{\pgfqpoint{0.129544in}{0.037123in}}{\pgfqpoint{0.139389in}{0.046968in}}{\pgfqpoint{0.152744in}{0.052500in}}%
\pgfpathcurveto{\pgfqpoint{0.166667in}{0.052500in}}{\pgfqpoint{0.180590in}{0.052500in}}{\pgfqpoint{0.193945in}{0.046968in}}%
\pgfpathcurveto{\pgfqpoint{0.203790in}{0.037123in}}{\pgfqpoint{0.213635in}{0.027278in}}{\pgfqpoint{0.219167in}{0.013923in}}%
\pgfpathcurveto{\pgfqpoint{0.219167in}{0.000000in}}{\pgfqpoint{0.219167in}{-0.013923in}}{\pgfqpoint{0.213635in}{-0.027278in}}%
\pgfpathcurveto{\pgfqpoint{0.203790in}{-0.037123in}}{\pgfqpoint{0.193945in}{-0.046968in}}{\pgfqpoint{0.180590in}{-0.052500in}}%
\pgfpathclose%
\pgfpathmoveto{\pgfqpoint{0.333333in}{-0.058333in}}%
\pgfpathcurveto{\pgfqpoint{0.348804in}{-0.058333in}}{\pgfqpoint{0.363642in}{-0.052187in}}{\pgfqpoint{0.374581in}{-0.041248in}}%
\pgfpathcurveto{\pgfqpoint{0.385520in}{-0.030309in}}{\pgfqpoint{0.391667in}{-0.015470in}}{\pgfqpoint{0.391667in}{0.000000in}}%
\pgfpathcurveto{\pgfqpoint{0.391667in}{0.015470in}}{\pgfqpoint{0.385520in}{0.030309in}}{\pgfqpoint{0.374581in}{0.041248in}}%
\pgfpathcurveto{\pgfqpoint{0.363642in}{0.052187in}}{\pgfqpoint{0.348804in}{0.058333in}}{\pgfqpoint{0.333333in}{0.058333in}}%
\pgfpathcurveto{\pgfqpoint{0.317863in}{0.058333in}}{\pgfqpoint{0.303025in}{0.052187in}}{\pgfqpoint{0.292085in}{0.041248in}}%
\pgfpathcurveto{\pgfqpoint{0.281146in}{0.030309in}}{\pgfqpoint{0.275000in}{0.015470in}}{\pgfqpoint{0.275000in}{0.000000in}}%
\pgfpathcurveto{\pgfqpoint{0.275000in}{-0.015470in}}{\pgfqpoint{0.281146in}{-0.030309in}}{\pgfqpoint{0.292085in}{-0.041248in}}%
\pgfpathcurveto{\pgfqpoint{0.303025in}{-0.052187in}}{\pgfqpoint{0.317863in}{-0.058333in}}{\pgfqpoint{0.333333in}{-0.058333in}}%
\pgfpathclose%
\pgfpathmoveto{\pgfqpoint{0.333333in}{-0.052500in}}%
\pgfpathcurveto{\pgfqpoint{0.333333in}{-0.052500in}}{\pgfqpoint{0.319410in}{-0.052500in}}{\pgfqpoint{0.306055in}{-0.046968in}}%
\pgfpathcurveto{\pgfqpoint{0.296210in}{-0.037123in}}{\pgfqpoint{0.286365in}{-0.027278in}}{\pgfqpoint{0.280833in}{-0.013923in}}%
\pgfpathcurveto{\pgfqpoint{0.280833in}{0.000000in}}{\pgfqpoint{0.280833in}{0.013923in}}{\pgfqpoint{0.286365in}{0.027278in}}%
\pgfpathcurveto{\pgfqpoint{0.296210in}{0.037123in}}{\pgfqpoint{0.306055in}{0.046968in}}{\pgfqpoint{0.319410in}{0.052500in}}%
\pgfpathcurveto{\pgfqpoint{0.333333in}{0.052500in}}{\pgfqpoint{0.347256in}{0.052500in}}{\pgfqpoint{0.360611in}{0.046968in}}%
\pgfpathcurveto{\pgfqpoint{0.370456in}{0.037123in}}{\pgfqpoint{0.380302in}{0.027278in}}{\pgfqpoint{0.385833in}{0.013923in}}%
\pgfpathcurveto{\pgfqpoint{0.385833in}{0.000000in}}{\pgfqpoint{0.385833in}{-0.013923in}}{\pgfqpoint{0.380302in}{-0.027278in}}%
\pgfpathcurveto{\pgfqpoint{0.370456in}{-0.037123in}}{\pgfqpoint{0.360611in}{-0.046968in}}{\pgfqpoint{0.347256in}{-0.052500in}}%
\pgfpathclose%
\pgfpathmoveto{\pgfqpoint{0.500000in}{-0.058333in}}%
\pgfpathcurveto{\pgfqpoint{0.515470in}{-0.058333in}}{\pgfqpoint{0.530309in}{-0.052187in}}{\pgfqpoint{0.541248in}{-0.041248in}}%
\pgfpathcurveto{\pgfqpoint{0.552187in}{-0.030309in}}{\pgfqpoint{0.558333in}{-0.015470in}}{\pgfqpoint{0.558333in}{0.000000in}}%
\pgfpathcurveto{\pgfqpoint{0.558333in}{0.015470in}}{\pgfqpoint{0.552187in}{0.030309in}}{\pgfqpoint{0.541248in}{0.041248in}}%
\pgfpathcurveto{\pgfqpoint{0.530309in}{0.052187in}}{\pgfqpoint{0.515470in}{0.058333in}}{\pgfqpoint{0.500000in}{0.058333in}}%
\pgfpathcurveto{\pgfqpoint{0.484530in}{0.058333in}}{\pgfqpoint{0.469691in}{0.052187in}}{\pgfqpoint{0.458752in}{0.041248in}}%
\pgfpathcurveto{\pgfqpoint{0.447813in}{0.030309in}}{\pgfqpoint{0.441667in}{0.015470in}}{\pgfqpoint{0.441667in}{0.000000in}}%
\pgfpathcurveto{\pgfqpoint{0.441667in}{-0.015470in}}{\pgfqpoint{0.447813in}{-0.030309in}}{\pgfqpoint{0.458752in}{-0.041248in}}%
\pgfpathcurveto{\pgfqpoint{0.469691in}{-0.052187in}}{\pgfqpoint{0.484530in}{-0.058333in}}{\pgfqpoint{0.500000in}{-0.058333in}}%
\pgfpathclose%
\pgfpathmoveto{\pgfqpoint{0.500000in}{-0.052500in}}%
\pgfpathcurveto{\pgfqpoint{0.500000in}{-0.052500in}}{\pgfqpoint{0.486077in}{-0.052500in}}{\pgfqpoint{0.472722in}{-0.046968in}}%
\pgfpathcurveto{\pgfqpoint{0.462877in}{-0.037123in}}{\pgfqpoint{0.453032in}{-0.027278in}}{\pgfqpoint{0.447500in}{-0.013923in}}%
\pgfpathcurveto{\pgfqpoint{0.447500in}{0.000000in}}{\pgfqpoint{0.447500in}{0.013923in}}{\pgfqpoint{0.453032in}{0.027278in}}%
\pgfpathcurveto{\pgfqpoint{0.462877in}{0.037123in}}{\pgfqpoint{0.472722in}{0.046968in}}{\pgfqpoint{0.486077in}{0.052500in}}%
\pgfpathcurveto{\pgfqpoint{0.500000in}{0.052500in}}{\pgfqpoint{0.513923in}{0.052500in}}{\pgfqpoint{0.527278in}{0.046968in}}%
\pgfpathcurveto{\pgfqpoint{0.537123in}{0.037123in}}{\pgfqpoint{0.546968in}{0.027278in}}{\pgfqpoint{0.552500in}{0.013923in}}%
\pgfpathcurveto{\pgfqpoint{0.552500in}{0.000000in}}{\pgfqpoint{0.552500in}{-0.013923in}}{\pgfqpoint{0.546968in}{-0.027278in}}%
\pgfpathcurveto{\pgfqpoint{0.537123in}{-0.037123in}}{\pgfqpoint{0.527278in}{-0.046968in}}{\pgfqpoint{0.513923in}{-0.052500in}}%
\pgfpathclose%
\pgfpathmoveto{\pgfqpoint{0.666667in}{-0.058333in}}%
\pgfpathcurveto{\pgfqpoint{0.682137in}{-0.058333in}}{\pgfqpoint{0.696975in}{-0.052187in}}{\pgfqpoint{0.707915in}{-0.041248in}}%
\pgfpathcurveto{\pgfqpoint{0.718854in}{-0.030309in}}{\pgfqpoint{0.725000in}{-0.015470in}}{\pgfqpoint{0.725000in}{0.000000in}}%
\pgfpathcurveto{\pgfqpoint{0.725000in}{0.015470in}}{\pgfqpoint{0.718854in}{0.030309in}}{\pgfqpoint{0.707915in}{0.041248in}}%
\pgfpathcurveto{\pgfqpoint{0.696975in}{0.052187in}}{\pgfqpoint{0.682137in}{0.058333in}}{\pgfqpoint{0.666667in}{0.058333in}}%
\pgfpathcurveto{\pgfqpoint{0.651196in}{0.058333in}}{\pgfqpoint{0.636358in}{0.052187in}}{\pgfqpoint{0.625419in}{0.041248in}}%
\pgfpathcurveto{\pgfqpoint{0.614480in}{0.030309in}}{\pgfqpoint{0.608333in}{0.015470in}}{\pgfqpoint{0.608333in}{0.000000in}}%
\pgfpathcurveto{\pgfqpoint{0.608333in}{-0.015470in}}{\pgfqpoint{0.614480in}{-0.030309in}}{\pgfqpoint{0.625419in}{-0.041248in}}%
\pgfpathcurveto{\pgfqpoint{0.636358in}{-0.052187in}}{\pgfqpoint{0.651196in}{-0.058333in}}{\pgfqpoint{0.666667in}{-0.058333in}}%
\pgfpathclose%
\pgfpathmoveto{\pgfqpoint{0.666667in}{-0.052500in}}%
\pgfpathcurveto{\pgfqpoint{0.666667in}{-0.052500in}}{\pgfqpoint{0.652744in}{-0.052500in}}{\pgfqpoint{0.639389in}{-0.046968in}}%
\pgfpathcurveto{\pgfqpoint{0.629544in}{-0.037123in}}{\pgfqpoint{0.619698in}{-0.027278in}}{\pgfqpoint{0.614167in}{-0.013923in}}%
\pgfpathcurveto{\pgfqpoint{0.614167in}{0.000000in}}{\pgfqpoint{0.614167in}{0.013923in}}{\pgfqpoint{0.619698in}{0.027278in}}%
\pgfpathcurveto{\pgfqpoint{0.629544in}{0.037123in}}{\pgfqpoint{0.639389in}{0.046968in}}{\pgfqpoint{0.652744in}{0.052500in}}%
\pgfpathcurveto{\pgfqpoint{0.666667in}{0.052500in}}{\pgfqpoint{0.680590in}{0.052500in}}{\pgfqpoint{0.693945in}{0.046968in}}%
\pgfpathcurveto{\pgfqpoint{0.703790in}{0.037123in}}{\pgfqpoint{0.713635in}{0.027278in}}{\pgfqpoint{0.719167in}{0.013923in}}%
\pgfpathcurveto{\pgfqpoint{0.719167in}{0.000000in}}{\pgfqpoint{0.719167in}{-0.013923in}}{\pgfqpoint{0.713635in}{-0.027278in}}%
\pgfpathcurveto{\pgfqpoint{0.703790in}{-0.037123in}}{\pgfqpoint{0.693945in}{-0.046968in}}{\pgfqpoint{0.680590in}{-0.052500in}}%
\pgfpathclose%
\pgfpathmoveto{\pgfqpoint{0.833333in}{-0.058333in}}%
\pgfpathcurveto{\pgfqpoint{0.848804in}{-0.058333in}}{\pgfqpoint{0.863642in}{-0.052187in}}{\pgfqpoint{0.874581in}{-0.041248in}}%
\pgfpathcurveto{\pgfqpoint{0.885520in}{-0.030309in}}{\pgfqpoint{0.891667in}{-0.015470in}}{\pgfqpoint{0.891667in}{0.000000in}}%
\pgfpathcurveto{\pgfqpoint{0.891667in}{0.015470in}}{\pgfqpoint{0.885520in}{0.030309in}}{\pgfqpoint{0.874581in}{0.041248in}}%
\pgfpathcurveto{\pgfqpoint{0.863642in}{0.052187in}}{\pgfqpoint{0.848804in}{0.058333in}}{\pgfqpoint{0.833333in}{0.058333in}}%
\pgfpathcurveto{\pgfqpoint{0.817863in}{0.058333in}}{\pgfqpoint{0.803025in}{0.052187in}}{\pgfqpoint{0.792085in}{0.041248in}}%
\pgfpathcurveto{\pgfqpoint{0.781146in}{0.030309in}}{\pgfqpoint{0.775000in}{0.015470in}}{\pgfqpoint{0.775000in}{0.000000in}}%
\pgfpathcurveto{\pgfqpoint{0.775000in}{-0.015470in}}{\pgfqpoint{0.781146in}{-0.030309in}}{\pgfqpoint{0.792085in}{-0.041248in}}%
\pgfpathcurveto{\pgfqpoint{0.803025in}{-0.052187in}}{\pgfqpoint{0.817863in}{-0.058333in}}{\pgfqpoint{0.833333in}{-0.058333in}}%
\pgfpathclose%
\pgfpathmoveto{\pgfqpoint{0.833333in}{-0.052500in}}%
\pgfpathcurveto{\pgfqpoint{0.833333in}{-0.052500in}}{\pgfqpoint{0.819410in}{-0.052500in}}{\pgfqpoint{0.806055in}{-0.046968in}}%
\pgfpathcurveto{\pgfqpoint{0.796210in}{-0.037123in}}{\pgfqpoint{0.786365in}{-0.027278in}}{\pgfqpoint{0.780833in}{-0.013923in}}%
\pgfpathcurveto{\pgfqpoint{0.780833in}{0.000000in}}{\pgfqpoint{0.780833in}{0.013923in}}{\pgfqpoint{0.786365in}{0.027278in}}%
\pgfpathcurveto{\pgfqpoint{0.796210in}{0.037123in}}{\pgfqpoint{0.806055in}{0.046968in}}{\pgfqpoint{0.819410in}{0.052500in}}%
\pgfpathcurveto{\pgfqpoint{0.833333in}{0.052500in}}{\pgfqpoint{0.847256in}{0.052500in}}{\pgfqpoint{0.860611in}{0.046968in}}%
\pgfpathcurveto{\pgfqpoint{0.870456in}{0.037123in}}{\pgfqpoint{0.880302in}{0.027278in}}{\pgfqpoint{0.885833in}{0.013923in}}%
\pgfpathcurveto{\pgfqpoint{0.885833in}{0.000000in}}{\pgfqpoint{0.885833in}{-0.013923in}}{\pgfqpoint{0.880302in}{-0.027278in}}%
\pgfpathcurveto{\pgfqpoint{0.870456in}{-0.037123in}}{\pgfqpoint{0.860611in}{-0.046968in}}{\pgfqpoint{0.847256in}{-0.052500in}}%
\pgfpathclose%
\pgfpathmoveto{\pgfqpoint{1.000000in}{-0.058333in}}%
\pgfpathcurveto{\pgfqpoint{1.015470in}{-0.058333in}}{\pgfqpoint{1.030309in}{-0.052187in}}{\pgfqpoint{1.041248in}{-0.041248in}}%
\pgfpathcurveto{\pgfqpoint{1.052187in}{-0.030309in}}{\pgfqpoint{1.058333in}{-0.015470in}}{\pgfqpoint{1.058333in}{0.000000in}}%
\pgfpathcurveto{\pgfqpoint{1.058333in}{0.015470in}}{\pgfqpoint{1.052187in}{0.030309in}}{\pgfqpoint{1.041248in}{0.041248in}}%
\pgfpathcurveto{\pgfqpoint{1.030309in}{0.052187in}}{\pgfqpoint{1.015470in}{0.058333in}}{\pgfqpoint{1.000000in}{0.058333in}}%
\pgfpathcurveto{\pgfqpoint{0.984530in}{0.058333in}}{\pgfqpoint{0.969691in}{0.052187in}}{\pgfqpoint{0.958752in}{0.041248in}}%
\pgfpathcurveto{\pgfqpoint{0.947813in}{0.030309in}}{\pgfqpoint{0.941667in}{0.015470in}}{\pgfqpoint{0.941667in}{0.000000in}}%
\pgfpathcurveto{\pgfqpoint{0.941667in}{-0.015470in}}{\pgfqpoint{0.947813in}{-0.030309in}}{\pgfqpoint{0.958752in}{-0.041248in}}%
\pgfpathcurveto{\pgfqpoint{0.969691in}{-0.052187in}}{\pgfqpoint{0.984530in}{-0.058333in}}{\pgfqpoint{1.000000in}{-0.058333in}}%
\pgfpathclose%
\pgfpathmoveto{\pgfqpoint{1.000000in}{-0.052500in}}%
\pgfpathcurveto{\pgfqpoint{1.000000in}{-0.052500in}}{\pgfqpoint{0.986077in}{-0.052500in}}{\pgfqpoint{0.972722in}{-0.046968in}}%
\pgfpathcurveto{\pgfqpoint{0.962877in}{-0.037123in}}{\pgfqpoint{0.953032in}{-0.027278in}}{\pgfqpoint{0.947500in}{-0.013923in}}%
\pgfpathcurveto{\pgfqpoint{0.947500in}{0.000000in}}{\pgfqpoint{0.947500in}{0.013923in}}{\pgfqpoint{0.953032in}{0.027278in}}%
\pgfpathcurveto{\pgfqpoint{0.962877in}{0.037123in}}{\pgfqpoint{0.972722in}{0.046968in}}{\pgfqpoint{0.986077in}{0.052500in}}%
\pgfpathcurveto{\pgfqpoint{1.000000in}{0.052500in}}{\pgfqpoint{1.013923in}{0.052500in}}{\pgfqpoint{1.027278in}{0.046968in}}%
\pgfpathcurveto{\pgfqpoint{1.037123in}{0.037123in}}{\pgfqpoint{1.046968in}{0.027278in}}{\pgfqpoint{1.052500in}{0.013923in}}%
\pgfpathcurveto{\pgfqpoint{1.052500in}{0.000000in}}{\pgfqpoint{1.052500in}{-0.013923in}}{\pgfqpoint{1.046968in}{-0.027278in}}%
\pgfpathcurveto{\pgfqpoint{1.037123in}{-0.037123in}}{\pgfqpoint{1.027278in}{-0.046968in}}{\pgfqpoint{1.013923in}{-0.052500in}}%
\pgfpathclose%
\pgfpathmoveto{\pgfqpoint{0.083333in}{0.108333in}}%
\pgfpathcurveto{\pgfqpoint{0.098804in}{0.108333in}}{\pgfqpoint{0.113642in}{0.114480in}}{\pgfqpoint{0.124581in}{0.125419in}}%
\pgfpathcurveto{\pgfqpoint{0.135520in}{0.136358in}}{\pgfqpoint{0.141667in}{0.151196in}}{\pgfqpoint{0.141667in}{0.166667in}}%
\pgfpathcurveto{\pgfqpoint{0.141667in}{0.182137in}}{\pgfqpoint{0.135520in}{0.196975in}}{\pgfqpoint{0.124581in}{0.207915in}}%
\pgfpathcurveto{\pgfqpoint{0.113642in}{0.218854in}}{\pgfqpoint{0.098804in}{0.225000in}}{\pgfqpoint{0.083333in}{0.225000in}}%
\pgfpathcurveto{\pgfqpoint{0.067863in}{0.225000in}}{\pgfqpoint{0.053025in}{0.218854in}}{\pgfqpoint{0.042085in}{0.207915in}}%
\pgfpathcurveto{\pgfqpoint{0.031146in}{0.196975in}}{\pgfqpoint{0.025000in}{0.182137in}}{\pgfqpoint{0.025000in}{0.166667in}}%
\pgfpathcurveto{\pgfqpoint{0.025000in}{0.151196in}}{\pgfqpoint{0.031146in}{0.136358in}}{\pgfqpoint{0.042085in}{0.125419in}}%
\pgfpathcurveto{\pgfqpoint{0.053025in}{0.114480in}}{\pgfqpoint{0.067863in}{0.108333in}}{\pgfqpoint{0.083333in}{0.108333in}}%
\pgfpathclose%
\pgfpathmoveto{\pgfqpoint{0.083333in}{0.114167in}}%
\pgfpathcurveto{\pgfqpoint{0.083333in}{0.114167in}}{\pgfqpoint{0.069410in}{0.114167in}}{\pgfqpoint{0.056055in}{0.119698in}}%
\pgfpathcurveto{\pgfqpoint{0.046210in}{0.129544in}}{\pgfqpoint{0.036365in}{0.139389in}}{\pgfqpoint{0.030833in}{0.152744in}}%
\pgfpathcurveto{\pgfqpoint{0.030833in}{0.166667in}}{\pgfqpoint{0.030833in}{0.180590in}}{\pgfqpoint{0.036365in}{0.193945in}}%
\pgfpathcurveto{\pgfqpoint{0.046210in}{0.203790in}}{\pgfqpoint{0.056055in}{0.213635in}}{\pgfqpoint{0.069410in}{0.219167in}}%
\pgfpathcurveto{\pgfqpoint{0.083333in}{0.219167in}}{\pgfqpoint{0.097256in}{0.219167in}}{\pgfqpoint{0.110611in}{0.213635in}}%
\pgfpathcurveto{\pgfqpoint{0.120456in}{0.203790in}}{\pgfqpoint{0.130302in}{0.193945in}}{\pgfqpoint{0.135833in}{0.180590in}}%
\pgfpathcurveto{\pgfqpoint{0.135833in}{0.166667in}}{\pgfqpoint{0.135833in}{0.152744in}}{\pgfqpoint{0.130302in}{0.139389in}}%
\pgfpathcurveto{\pgfqpoint{0.120456in}{0.129544in}}{\pgfqpoint{0.110611in}{0.119698in}}{\pgfqpoint{0.097256in}{0.114167in}}%
\pgfpathclose%
\pgfpathmoveto{\pgfqpoint{0.250000in}{0.108333in}}%
\pgfpathcurveto{\pgfqpoint{0.265470in}{0.108333in}}{\pgfqpoint{0.280309in}{0.114480in}}{\pgfqpoint{0.291248in}{0.125419in}}%
\pgfpathcurveto{\pgfqpoint{0.302187in}{0.136358in}}{\pgfqpoint{0.308333in}{0.151196in}}{\pgfqpoint{0.308333in}{0.166667in}}%
\pgfpathcurveto{\pgfqpoint{0.308333in}{0.182137in}}{\pgfqpoint{0.302187in}{0.196975in}}{\pgfqpoint{0.291248in}{0.207915in}}%
\pgfpathcurveto{\pgfqpoint{0.280309in}{0.218854in}}{\pgfqpoint{0.265470in}{0.225000in}}{\pgfqpoint{0.250000in}{0.225000in}}%
\pgfpathcurveto{\pgfqpoint{0.234530in}{0.225000in}}{\pgfqpoint{0.219691in}{0.218854in}}{\pgfqpoint{0.208752in}{0.207915in}}%
\pgfpathcurveto{\pgfqpoint{0.197813in}{0.196975in}}{\pgfqpoint{0.191667in}{0.182137in}}{\pgfqpoint{0.191667in}{0.166667in}}%
\pgfpathcurveto{\pgfqpoint{0.191667in}{0.151196in}}{\pgfqpoint{0.197813in}{0.136358in}}{\pgfqpoint{0.208752in}{0.125419in}}%
\pgfpathcurveto{\pgfqpoint{0.219691in}{0.114480in}}{\pgfqpoint{0.234530in}{0.108333in}}{\pgfqpoint{0.250000in}{0.108333in}}%
\pgfpathclose%
\pgfpathmoveto{\pgfqpoint{0.250000in}{0.114167in}}%
\pgfpathcurveto{\pgfqpoint{0.250000in}{0.114167in}}{\pgfqpoint{0.236077in}{0.114167in}}{\pgfqpoint{0.222722in}{0.119698in}}%
\pgfpathcurveto{\pgfqpoint{0.212877in}{0.129544in}}{\pgfqpoint{0.203032in}{0.139389in}}{\pgfqpoint{0.197500in}{0.152744in}}%
\pgfpathcurveto{\pgfqpoint{0.197500in}{0.166667in}}{\pgfqpoint{0.197500in}{0.180590in}}{\pgfqpoint{0.203032in}{0.193945in}}%
\pgfpathcurveto{\pgfqpoint{0.212877in}{0.203790in}}{\pgfqpoint{0.222722in}{0.213635in}}{\pgfqpoint{0.236077in}{0.219167in}}%
\pgfpathcurveto{\pgfqpoint{0.250000in}{0.219167in}}{\pgfqpoint{0.263923in}{0.219167in}}{\pgfqpoint{0.277278in}{0.213635in}}%
\pgfpathcurveto{\pgfqpoint{0.287123in}{0.203790in}}{\pgfqpoint{0.296968in}{0.193945in}}{\pgfqpoint{0.302500in}{0.180590in}}%
\pgfpathcurveto{\pgfqpoint{0.302500in}{0.166667in}}{\pgfqpoint{0.302500in}{0.152744in}}{\pgfqpoint{0.296968in}{0.139389in}}%
\pgfpathcurveto{\pgfqpoint{0.287123in}{0.129544in}}{\pgfqpoint{0.277278in}{0.119698in}}{\pgfqpoint{0.263923in}{0.114167in}}%
\pgfpathclose%
\pgfpathmoveto{\pgfqpoint{0.416667in}{0.108333in}}%
\pgfpathcurveto{\pgfqpoint{0.432137in}{0.108333in}}{\pgfqpoint{0.446975in}{0.114480in}}{\pgfqpoint{0.457915in}{0.125419in}}%
\pgfpathcurveto{\pgfqpoint{0.468854in}{0.136358in}}{\pgfqpoint{0.475000in}{0.151196in}}{\pgfqpoint{0.475000in}{0.166667in}}%
\pgfpathcurveto{\pgfqpoint{0.475000in}{0.182137in}}{\pgfqpoint{0.468854in}{0.196975in}}{\pgfqpoint{0.457915in}{0.207915in}}%
\pgfpathcurveto{\pgfqpoint{0.446975in}{0.218854in}}{\pgfqpoint{0.432137in}{0.225000in}}{\pgfqpoint{0.416667in}{0.225000in}}%
\pgfpathcurveto{\pgfqpoint{0.401196in}{0.225000in}}{\pgfqpoint{0.386358in}{0.218854in}}{\pgfqpoint{0.375419in}{0.207915in}}%
\pgfpathcurveto{\pgfqpoint{0.364480in}{0.196975in}}{\pgfqpoint{0.358333in}{0.182137in}}{\pgfqpoint{0.358333in}{0.166667in}}%
\pgfpathcurveto{\pgfqpoint{0.358333in}{0.151196in}}{\pgfqpoint{0.364480in}{0.136358in}}{\pgfqpoint{0.375419in}{0.125419in}}%
\pgfpathcurveto{\pgfqpoint{0.386358in}{0.114480in}}{\pgfqpoint{0.401196in}{0.108333in}}{\pgfqpoint{0.416667in}{0.108333in}}%
\pgfpathclose%
\pgfpathmoveto{\pgfqpoint{0.416667in}{0.114167in}}%
\pgfpathcurveto{\pgfqpoint{0.416667in}{0.114167in}}{\pgfqpoint{0.402744in}{0.114167in}}{\pgfqpoint{0.389389in}{0.119698in}}%
\pgfpathcurveto{\pgfqpoint{0.379544in}{0.129544in}}{\pgfqpoint{0.369698in}{0.139389in}}{\pgfqpoint{0.364167in}{0.152744in}}%
\pgfpathcurveto{\pgfqpoint{0.364167in}{0.166667in}}{\pgfqpoint{0.364167in}{0.180590in}}{\pgfqpoint{0.369698in}{0.193945in}}%
\pgfpathcurveto{\pgfqpoint{0.379544in}{0.203790in}}{\pgfqpoint{0.389389in}{0.213635in}}{\pgfqpoint{0.402744in}{0.219167in}}%
\pgfpathcurveto{\pgfqpoint{0.416667in}{0.219167in}}{\pgfqpoint{0.430590in}{0.219167in}}{\pgfqpoint{0.443945in}{0.213635in}}%
\pgfpathcurveto{\pgfqpoint{0.453790in}{0.203790in}}{\pgfqpoint{0.463635in}{0.193945in}}{\pgfqpoint{0.469167in}{0.180590in}}%
\pgfpathcurveto{\pgfqpoint{0.469167in}{0.166667in}}{\pgfqpoint{0.469167in}{0.152744in}}{\pgfqpoint{0.463635in}{0.139389in}}%
\pgfpathcurveto{\pgfqpoint{0.453790in}{0.129544in}}{\pgfqpoint{0.443945in}{0.119698in}}{\pgfqpoint{0.430590in}{0.114167in}}%
\pgfpathclose%
\pgfpathmoveto{\pgfqpoint{0.583333in}{0.108333in}}%
\pgfpathcurveto{\pgfqpoint{0.598804in}{0.108333in}}{\pgfqpoint{0.613642in}{0.114480in}}{\pgfqpoint{0.624581in}{0.125419in}}%
\pgfpathcurveto{\pgfqpoint{0.635520in}{0.136358in}}{\pgfqpoint{0.641667in}{0.151196in}}{\pgfqpoint{0.641667in}{0.166667in}}%
\pgfpathcurveto{\pgfqpoint{0.641667in}{0.182137in}}{\pgfqpoint{0.635520in}{0.196975in}}{\pgfqpoint{0.624581in}{0.207915in}}%
\pgfpathcurveto{\pgfqpoint{0.613642in}{0.218854in}}{\pgfqpoint{0.598804in}{0.225000in}}{\pgfqpoint{0.583333in}{0.225000in}}%
\pgfpathcurveto{\pgfqpoint{0.567863in}{0.225000in}}{\pgfqpoint{0.553025in}{0.218854in}}{\pgfqpoint{0.542085in}{0.207915in}}%
\pgfpathcurveto{\pgfqpoint{0.531146in}{0.196975in}}{\pgfqpoint{0.525000in}{0.182137in}}{\pgfqpoint{0.525000in}{0.166667in}}%
\pgfpathcurveto{\pgfqpoint{0.525000in}{0.151196in}}{\pgfqpoint{0.531146in}{0.136358in}}{\pgfqpoint{0.542085in}{0.125419in}}%
\pgfpathcurveto{\pgfqpoint{0.553025in}{0.114480in}}{\pgfqpoint{0.567863in}{0.108333in}}{\pgfqpoint{0.583333in}{0.108333in}}%
\pgfpathclose%
\pgfpathmoveto{\pgfqpoint{0.583333in}{0.114167in}}%
\pgfpathcurveto{\pgfqpoint{0.583333in}{0.114167in}}{\pgfqpoint{0.569410in}{0.114167in}}{\pgfqpoint{0.556055in}{0.119698in}}%
\pgfpathcurveto{\pgfqpoint{0.546210in}{0.129544in}}{\pgfqpoint{0.536365in}{0.139389in}}{\pgfqpoint{0.530833in}{0.152744in}}%
\pgfpathcurveto{\pgfqpoint{0.530833in}{0.166667in}}{\pgfqpoint{0.530833in}{0.180590in}}{\pgfqpoint{0.536365in}{0.193945in}}%
\pgfpathcurveto{\pgfqpoint{0.546210in}{0.203790in}}{\pgfqpoint{0.556055in}{0.213635in}}{\pgfqpoint{0.569410in}{0.219167in}}%
\pgfpathcurveto{\pgfqpoint{0.583333in}{0.219167in}}{\pgfqpoint{0.597256in}{0.219167in}}{\pgfqpoint{0.610611in}{0.213635in}}%
\pgfpathcurveto{\pgfqpoint{0.620456in}{0.203790in}}{\pgfqpoint{0.630302in}{0.193945in}}{\pgfqpoint{0.635833in}{0.180590in}}%
\pgfpathcurveto{\pgfqpoint{0.635833in}{0.166667in}}{\pgfqpoint{0.635833in}{0.152744in}}{\pgfqpoint{0.630302in}{0.139389in}}%
\pgfpathcurveto{\pgfqpoint{0.620456in}{0.129544in}}{\pgfqpoint{0.610611in}{0.119698in}}{\pgfqpoint{0.597256in}{0.114167in}}%
\pgfpathclose%
\pgfpathmoveto{\pgfqpoint{0.750000in}{0.108333in}}%
\pgfpathcurveto{\pgfqpoint{0.765470in}{0.108333in}}{\pgfqpoint{0.780309in}{0.114480in}}{\pgfqpoint{0.791248in}{0.125419in}}%
\pgfpathcurveto{\pgfqpoint{0.802187in}{0.136358in}}{\pgfqpoint{0.808333in}{0.151196in}}{\pgfqpoint{0.808333in}{0.166667in}}%
\pgfpathcurveto{\pgfqpoint{0.808333in}{0.182137in}}{\pgfqpoint{0.802187in}{0.196975in}}{\pgfqpoint{0.791248in}{0.207915in}}%
\pgfpathcurveto{\pgfqpoint{0.780309in}{0.218854in}}{\pgfqpoint{0.765470in}{0.225000in}}{\pgfqpoint{0.750000in}{0.225000in}}%
\pgfpathcurveto{\pgfqpoint{0.734530in}{0.225000in}}{\pgfqpoint{0.719691in}{0.218854in}}{\pgfqpoint{0.708752in}{0.207915in}}%
\pgfpathcurveto{\pgfqpoint{0.697813in}{0.196975in}}{\pgfqpoint{0.691667in}{0.182137in}}{\pgfqpoint{0.691667in}{0.166667in}}%
\pgfpathcurveto{\pgfqpoint{0.691667in}{0.151196in}}{\pgfqpoint{0.697813in}{0.136358in}}{\pgfqpoint{0.708752in}{0.125419in}}%
\pgfpathcurveto{\pgfqpoint{0.719691in}{0.114480in}}{\pgfqpoint{0.734530in}{0.108333in}}{\pgfqpoint{0.750000in}{0.108333in}}%
\pgfpathclose%
\pgfpathmoveto{\pgfqpoint{0.750000in}{0.114167in}}%
\pgfpathcurveto{\pgfqpoint{0.750000in}{0.114167in}}{\pgfqpoint{0.736077in}{0.114167in}}{\pgfqpoint{0.722722in}{0.119698in}}%
\pgfpathcurveto{\pgfqpoint{0.712877in}{0.129544in}}{\pgfqpoint{0.703032in}{0.139389in}}{\pgfqpoint{0.697500in}{0.152744in}}%
\pgfpathcurveto{\pgfqpoint{0.697500in}{0.166667in}}{\pgfqpoint{0.697500in}{0.180590in}}{\pgfqpoint{0.703032in}{0.193945in}}%
\pgfpathcurveto{\pgfqpoint{0.712877in}{0.203790in}}{\pgfqpoint{0.722722in}{0.213635in}}{\pgfqpoint{0.736077in}{0.219167in}}%
\pgfpathcurveto{\pgfqpoint{0.750000in}{0.219167in}}{\pgfqpoint{0.763923in}{0.219167in}}{\pgfqpoint{0.777278in}{0.213635in}}%
\pgfpathcurveto{\pgfqpoint{0.787123in}{0.203790in}}{\pgfqpoint{0.796968in}{0.193945in}}{\pgfqpoint{0.802500in}{0.180590in}}%
\pgfpathcurveto{\pgfqpoint{0.802500in}{0.166667in}}{\pgfqpoint{0.802500in}{0.152744in}}{\pgfqpoint{0.796968in}{0.139389in}}%
\pgfpathcurveto{\pgfqpoint{0.787123in}{0.129544in}}{\pgfqpoint{0.777278in}{0.119698in}}{\pgfqpoint{0.763923in}{0.114167in}}%
\pgfpathclose%
\pgfpathmoveto{\pgfqpoint{0.916667in}{0.108333in}}%
\pgfpathcurveto{\pgfqpoint{0.932137in}{0.108333in}}{\pgfqpoint{0.946975in}{0.114480in}}{\pgfqpoint{0.957915in}{0.125419in}}%
\pgfpathcurveto{\pgfqpoint{0.968854in}{0.136358in}}{\pgfqpoint{0.975000in}{0.151196in}}{\pgfqpoint{0.975000in}{0.166667in}}%
\pgfpathcurveto{\pgfqpoint{0.975000in}{0.182137in}}{\pgfqpoint{0.968854in}{0.196975in}}{\pgfqpoint{0.957915in}{0.207915in}}%
\pgfpathcurveto{\pgfqpoint{0.946975in}{0.218854in}}{\pgfqpoint{0.932137in}{0.225000in}}{\pgfqpoint{0.916667in}{0.225000in}}%
\pgfpathcurveto{\pgfqpoint{0.901196in}{0.225000in}}{\pgfqpoint{0.886358in}{0.218854in}}{\pgfqpoint{0.875419in}{0.207915in}}%
\pgfpathcurveto{\pgfqpoint{0.864480in}{0.196975in}}{\pgfqpoint{0.858333in}{0.182137in}}{\pgfqpoint{0.858333in}{0.166667in}}%
\pgfpathcurveto{\pgfqpoint{0.858333in}{0.151196in}}{\pgfqpoint{0.864480in}{0.136358in}}{\pgfqpoint{0.875419in}{0.125419in}}%
\pgfpathcurveto{\pgfqpoint{0.886358in}{0.114480in}}{\pgfqpoint{0.901196in}{0.108333in}}{\pgfqpoint{0.916667in}{0.108333in}}%
\pgfpathclose%
\pgfpathmoveto{\pgfqpoint{0.916667in}{0.114167in}}%
\pgfpathcurveto{\pgfqpoint{0.916667in}{0.114167in}}{\pgfqpoint{0.902744in}{0.114167in}}{\pgfqpoint{0.889389in}{0.119698in}}%
\pgfpathcurveto{\pgfqpoint{0.879544in}{0.129544in}}{\pgfqpoint{0.869698in}{0.139389in}}{\pgfqpoint{0.864167in}{0.152744in}}%
\pgfpathcurveto{\pgfqpoint{0.864167in}{0.166667in}}{\pgfqpoint{0.864167in}{0.180590in}}{\pgfqpoint{0.869698in}{0.193945in}}%
\pgfpathcurveto{\pgfqpoint{0.879544in}{0.203790in}}{\pgfqpoint{0.889389in}{0.213635in}}{\pgfqpoint{0.902744in}{0.219167in}}%
\pgfpathcurveto{\pgfqpoint{0.916667in}{0.219167in}}{\pgfqpoint{0.930590in}{0.219167in}}{\pgfqpoint{0.943945in}{0.213635in}}%
\pgfpathcurveto{\pgfqpoint{0.953790in}{0.203790in}}{\pgfqpoint{0.963635in}{0.193945in}}{\pgfqpoint{0.969167in}{0.180590in}}%
\pgfpathcurveto{\pgfqpoint{0.969167in}{0.166667in}}{\pgfqpoint{0.969167in}{0.152744in}}{\pgfqpoint{0.963635in}{0.139389in}}%
\pgfpathcurveto{\pgfqpoint{0.953790in}{0.129544in}}{\pgfqpoint{0.943945in}{0.119698in}}{\pgfqpoint{0.930590in}{0.114167in}}%
\pgfpathclose%
\pgfpathmoveto{\pgfqpoint{0.000000in}{0.275000in}}%
\pgfpathcurveto{\pgfqpoint{0.015470in}{0.275000in}}{\pgfqpoint{0.030309in}{0.281146in}}{\pgfqpoint{0.041248in}{0.292085in}}%
\pgfpathcurveto{\pgfqpoint{0.052187in}{0.303025in}}{\pgfqpoint{0.058333in}{0.317863in}}{\pgfqpoint{0.058333in}{0.333333in}}%
\pgfpathcurveto{\pgfqpoint{0.058333in}{0.348804in}}{\pgfqpoint{0.052187in}{0.363642in}}{\pgfqpoint{0.041248in}{0.374581in}}%
\pgfpathcurveto{\pgfqpoint{0.030309in}{0.385520in}}{\pgfqpoint{0.015470in}{0.391667in}}{\pgfqpoint{0.000000in}{0.391667in}}%
\pgfpathcurveto{\pgfqpoint{-0.015470in}{0.391667in}}{\pgfqpoint{-0.030309in}{0.385520in}}{\pgfqpoint{-0.041248in}{0.374581in}}%
\pgfpathcurveto{\pgfqpoint{-0.052187in}{0.363642in}}{\pgfqpoint{-0.058333in}{0.348804in}}{\pgfqpoint{-0.058333in}{0.333333in}}%
\pgfpathcurveto{\pgfqpoint{-0.058333in}{0.317863in}}{\pgfqpoint{-0.052187in}{0.303025in}}{\pgfqpoint{-0.041248in}{0.292085in}}%
\pgfpathcurveto{\pgfqpoint{-0.030309in}{0.281146in}}{\pgfqpoint{-0.015470in}{0.275000in}}{\pgfqpoint{0.000000in}{0.275000in}}%
\pgfpathclose%
\pgfpathmoveto{\pgfqpoint{0.000000in}{0.280833in}}%
\pgfpathcurveto{\pgfqpoint{0.000000in}{0.280833in}}{\pgfqpoint{-0.013923in}{0.280833in}}{\pgfqpoint{-0.027278in}{0.286365in}}%
\pgfpathcurveto{\pgfqpoint{-0.037123in}{0.296210in}}{\pgfqpoint{-0.046968in}{0.306055in}}{\pgfqpoint{-0.052500in}{0.319410in}}%
\pgfpathcurveto{\pgfqpoint{-0.052500in}{0.333333in}}{\pgfqpoint{-0.052500in}{0.347256in}}{\pgfqpoint{-0.046968in}{0.360611in}}%
\pgfpathcurveto{\pgfqpoint{-0.037123in}{0.370456in}}{\pgfqpoint{-0.027278in}{0.380302in}}{\pgfqpoint{-0.013923in}{0.385833in}}%
\pgfpathcurveto{\pgfqpoint{0.000000in}{0.385833in}}{\pgfqpoint{0.013923in}{0.385833in}}{\pgfqpoint{0.027278in}{0.380302in}}%
\pgfpathcurveto{\pgfqpoint{0.037123in}{0.370456in}}{\pgfqpoint{0.046968in}{0.360611in}}{\pgfqpoint{0.052500in}{0.347256in}}%
\pgfpathcurveto{\pgfqpoint{0.052500in}{0.333333in}}{\pgfqpoint{0.052500in}{0.319410in}}{\pgfqpoint{0.046968in}{0.306055in}}%
\pgfpathcurveto{\pgfqpoint{0.037123in}{0.296210in}}{\pgfqpoint{0.027278in}{0.286365in}}{\pgfqpoint{0.013923in}{0.280833in}}%
\pgfpathclose%
\pgfpathmoveto{\pgfqpoint{0.166667in}{0.275000in}}%
\pgfpathcurveto{\pgfqpoint{0.182137in}{0.275000in}}{\pgfqpoint{0.196975in}{0.281146in}}{\pgfqpoint{0.207915in}{0.292085in}}%
\pgfpathcurveto{\pgfqpoint{0.218854in}{0.303025in}}{\pgfqpoint{0.225000in}{0.317863in}}{\pgfqpoint{0.225000in}{0.333333in}}%
\pgfpathcurveto{\pgfqpoint{0.225000in}{0.348804in}}{\pgfqpoint{0.218854in}{0.363642in}}{\pgfqpoint{0.207915in}{0.374581in}}%
\pgfpathcurveto{\pgfqpoint{0.196975in}{0.385520in}}{\pgfqpoint{0.182137in}{0.391667in}}{\pgfqpoint{0.166667in}{0.391667in}}%
\pgfpathcurveto{\pgfqpoint{0.151196in}{0.391667in}}{\pgfqpoint{0.136358in}{0.385520in}}{\pgfqpoint{0.125419in}{0.374581in}}%
\pgfpathcurveto{\pgfqpoint{0.114480in}{0.363642in}}{\pgfqpoint{0.108333in}{0.348804in}}{\pgfqpoint{0.108333in}{0.333333in}}%
\pgfpathcurveto{\pgfqpoint{0.108333in}{0.317863in}}{\pgfqpoint{0.114480in}{0.303025in}}{\pgfqpoint{0.125419in}{0.292085in}}%
\pgfpathcurveto{\pgfqpoint{0.136358in}{0.281146in}}{\pgfqpoint{0.151196in}{0.275000in}}{\pgfqpoint{0.166667in}{0.275000in}}%
\pgfpathclose%
\pgfpathmoveto{\pgfqpoint{0.166667in}{0.280833in}}%
\pgfpathcurveto{\pgfqpoint{0.166667in}{0.280833in}}{\pgfqpoint{0.152744in}{0.280833in}}{\pgfqpoint{0.139389in}{0.286365in}}%
\pgfpathcurveto{\pgfqpoint{0.129544in}{0.296210in}}{\pgfqpoint{0.119698in}{0.306055in}}{\pgfqpoint{0.114167in}{0.319410in}}%
\pgfpathcurveto{\pgfqpoint{0.114167in}{0.333333in}}{\pgfqpoint{0.114167in}{0.347256in}}{\pgfqpoint{0.119698in}{0.360611in}}%
\pgfpathcurveto{\pgfqpoint{0.129544in}{0.370456in}}{\pgfqpoint{0.139389in}{0.380302in}}{\pgfqpoint{0.152744in}{0.385833in}}%
\pgfpathcurveto{\pgfqpoint{0.166667in}{0.385833in}}{\pgfqpoint{0.180590in}{0.385833in}}{\pgfqpoint{0.193945in}{0.380302in}}%
\pgfpathcurveto{\pgfqpoint{0.203790in}{0.370456in}}{\pgfqpoint{0.213635in}{0.360611in}}{\pgfqpoint{0.219167in}{0.347256in}}%
\pgfpathcurveto{\pgfqpoint{0.219167in}{0.333333in}}{\pgfqpoint{0.219167in}{0.319410in}}{\pgfqpoint{0.213635in}{0.306055in}}%
\pgfpathcurveto{\pgfqpoint{0.203790in}{0.296210in}}{\pgfqpoint{0.193945in}{0.286365in}}{\pgfqpoint{0.180590in}{0.280833in}}%
\pgfpathclose%
\pgfpathmoveto{\pgfqpoint{0.333333in}{0.275000in}}%
\pgfpathcurveto{\pgfqpoint{0.348804in}{0.275000in}}{\pgfqpoint{0.363642in}{0.281146in}}{\pgfqpoint{0.374581in}{0.292085in}}%
\pgfpathcurveto{\pgfqpoint{0.385520in}{0.303025in}}{\pgfqpoint{0.391667in}{0.317863in}}{\pgfqpoint{0.391667in}{0.333333in}}%
\pgfpathcurveto{\pgfqpoint{0.391667in}{0.348804in}}{\pgfqpoint{0.385520in}{0.363642in}}{\pgfqpoint{0.374581in}{0.374581in}}%
\pgfpathcurveto{\pgfqpoint{0.363642in}{0.385520in}}{\pgfqpoint{0.348804in}{0.391667in}}{\pgfqpoint{0.333333in}{0.391667in}}%
\pgfpathcurveto{\pgfqpoint{0.317863in}{0.391667in}}{\pgfqpoint{0.303025in}{0.385520in}}{\pgfqpoint{0.292085in}{0.374581in}}%
\pgfpathcurveto{\pgfqpoint{0.281146in}{0.363642in}}{\pgfqpoint{0.275000in}{0.348804in}}{\pgfqpoint{0.275000in}{0.333333in}}%
\pgfpathcurveto{\pgfqpoint{0.275000in}{0.317863in}}{\pgfqpoint{0.281146in}{0.303025in}}{\pgfqpoint{0.292085in}{0.292085in}}%
\pgfpathcurveto{\pgfqpoint{0.303025in}{0.281146in}}{\pgfqpoint{0.317863in}{0.275000in}}{\pgfqpoint{0.333333in}{0.275000in}}%
\pgfpathclose%
\pgfpathmoveto{\pgfqpoint{0.333333in}{0.280833in}}%
\pgfpathcurveto{\pgfqpoint{0.333333in}{0.280833in}}{\pgfqpoint{0.319410in}{0.280833in}}{\pgfqpoint{0.306055in}{0.286365in}}%
\pgfpathcurveto{\pgfqpoint{0.296210in}{0.296210in}}{\pgfqpoint{0.286365in}{0.306055in}}{\pgfqpoint{0.280833in}{0.319410in}}%
\pgfpathcurveto{\pgfqpoint{0.280833in}{0.333333in}}{\pgfqpoint{0.280833in}{0.347256in}}{\pgfqpoint{0.286365in}{0.360611in}}%
\pgfpathcurveto{\pgfqpoint{0.296210in}{0.370456in}}{\pgfqpoint{0.306055in}{0.380302in}}{\pgfqpoint{0.319410in}{0.385833in}}%
\pgfpathcurveto{\pgfqpoint{0.333333in}{0.385833in}}{\pgfqpoint{0.347256in}{0.385833in}}{\pgfqpoint{0.360611in}{0.380302in}}%
\pgfpathcurveto{\pgfqpoint{0.370456in}{0.370456in}}{\pgfqpoint{0.380302in}{0.360611in}}{\pgfqpoint{0.385833in}{0.347256in}}%
\pgfpathcurveto{\pgfqpoint{0.385833in}{0.333333in}}{\pgfqpoint{0.385833in}{0.319410in}}{\pgfqpoint{0.380302in}{0.306055in}}%
\pgfpathcurveto{\pgfqpoint{0.370456in}{0.296210in}}{\pgfqpoint{0.360611in}{0.286365in}}{\pgfqpoint{0.347256in}{0.280833in}}%
\pgfpathclose%
\pgfpathmoveto{\pgfqpoint{0.500000in}{0.275000in}}%
\pgfpathcurveto{\pgfqpoint{0.515470in}{0.275000in}}{\pgfqpoint{0.530309in}{0.281146in}}{\pgfqpoint{0.541248in}{0.292085in}}%
\pgfpathcurveto{\pgfqpoint{0.552187in}{0.303025in}}{\pgfqpoint{0.558333in}{0.317863in}}{\pgfqpoint{0.558333in}{0.333333in}}%
\pgfpathcurveto{\pgfqpoint{0.558333in}{0.348804in}}{\pgfqpoint{0.552187in}{0.363642in}}{\pgfqpoint{0.541248in}{0.374581in}}%
\pgfpathcurveto{\pgfqpoint{0.530309in}{0.385520in}}{\pgfqpoint{0.515470in}{0.391667in}}{\pgfqpoint{0.500000in}{0.391667in}}%
\pgfpathcurveto{\pgfqpoint{0.484530in}{0.391667in}}{\pgfqpoint{0.469691in}{0.385520in}}{\pgfqpoint{0.458752in}{0.374581in}}%
\pgfpathcurveto{\pgfqpoint{0.447813in}{0.363642in}}{\pgfqpoint{0.441667in}{0.348804in}}{\pgfqpoint{0.441667in}{0.333333in}}%
\pgfpathcurveto{\pgfqpoint{0.441667in}{0.317863in}}{\pgfqpoint{0.447813in}{0.303025in}}{\pgfqpoint{0.458752in}{0.292085in}}%
\pgfpathcurveto{\pgfqpoint{0.469691in}{0.281146in}}{\pgfqpoint{0.484530in}{0.275000in}}{\pgfqpoint{0.500000in}{0.275000in}}%
\pgfpathclose%
\pgfpathmoveto{\pgfqpoint{0.500000in}{0.280833in}}%
\pgfpathcurveto{\pgfqpoint{0.500000in}{0.280833in}}{\pgfqpoint{0.486077in}{0.280833in}}{\pgfqpoint{0.472722in}{0.286365in}}%
\pgfpathcurveto{\pgfqpoint{0.462877in}{0.296210in}}{\pgfqpoint{0.453032in}{0.306055in}}{\pgfqpoint{0.447500in}{0.319410in}}%
\pgfpathcurveto{\pgfqpoint{0.447500in}{0.333333in}}{\pgfqpoint{0.447500in}{0.347256in}}{\pgfqpoint{0.453032in}{0.360611in}}%
\pgfpathcurveto{\pgfqpoint{0.462877in}{0.370456in}}{\pgfqpoint{0.472722in}{0.380302in}}{\pgfqpoint{0.486077in}{0.385833in}}%
\pgfpathcurveto{\pgfqpoint{0.500000in}{0.385833in}}{\pgfqpoint{0.513923in}{0.385833in}}{\pgfqpoint{0.527278in}{0.380302in}}%
\pgfpathcurveto{\pgfqpoint{0.537123in}{0.370456in}}{\pgfqpoint{0.546968in}{0.360611in}}{\pgfqpoint{0.552500in}{0.347256in}}%
\pgfpathcurveto{\pgfqpoint{0.552500in}{0.333333in}}{\pgfqpoint{0.552500in}{0.319410in}}{\pgfqpoint{0.546968in}{0.306055in}}%
\pgfpathcurveto{\pgfqpoint{0.537123in}{0.296210in}}{\pgfqpoint{0.527278in}{0.286365in}}{\pgfqpoint{0.513923in}{0.280833in}}%
\pgfpathclose%
\pgfpathmoveto{\pgfqpoint{0.666667in}{0.275000in}}%
\pgfpathcurveto{\pgfqpoint{0.682137in}{0.275000in}}{\pgfqpoint{0.696975in}{0.281146in}}{\pgfqpoint{0.707915in}{0.292085in}}%
\pgfpathcurveto{\pgfqpoint{0.718854in}{0.303025in}}{\pgfqpoint{0.725000in}{0.317863in}}{\pgfqpoint{0.725000in}{0.333333in}}%
\pgfpathcurveto{\pgfqpoint{0.725000in}{0.348804in}}{\pgfqpoint{0.718854in}{0.363642in}}{\pgfqpoint{0.707915in}{0.374581in}}%
\pgfpathcurveto{\pgfqpoint{0.696975in}{0.385520in}}{\pgfqpoint{0.682137in}{0.391667in}}{\pgfqpoint{0.666667in}{0.391667in}}%
\pgfpathcurveto{\pgfqpoint{0.651196in}{0.391667in}}{\pgfqpoint{0.636358in}{0.385520in}}{\pgfqpoint{0.625419in}{0.374581in}}%
\pgfpathcurveto{\pgfqpoint{0.614480in}{0.363642in}}{\pgfqpoint{0.608333in}{0.348804in}}{\pgfqpoint{0.608333in}{0.333333in}}%
\pgfpathcurveto{\pgfqpoint{0.608333in}{0.317863in}}{\pgfqpoint{0.614480in}{0.303025in}}{\pgfqpoint{0.625419in}{0.292085in}}%
\pgfpathcurveto{\pgfqpoint{0.636358in}{0.281146in}}{\pgfqpoint{0.651196in}{0.275000in}}{\pgfqpoint{0.666667in}{0.275000in}}%
\pgfpathclose%
\pgfpathmoveto{\pgfqpoint{0.666667in}{0.280833in}}%
\pgfpathcurveto{\pgfqpoint{0.666667in}{0.280833in}}{\pgfqpoint{0.652744in}{0.280833in}}{\pgfqpoint{0.639389in}{0.286365in}}%
\pgfpathcurveto{\pgfqpoint{0.629544in}{0.296210in}}{\pgfqpoint{0.619698in}{0.306055in}}{\pgfqpoint{0.614167in}{0.319410in}}%
\pgfpathcurveto{\pgfqpoint{0.614167in}{0.333333in}}{\pgfqpoint{0.614167in}{0.347256in}}{\pgfqpoint{0.619698in}{0.360611in}}%
\pgfpathcurveto{\pgfqpoint{0.629544in}{0.370456in}}{\pgfqpoint{0.639389in}{0.380302in}}{\pgfqpoint{0.652744in}{0.385833in}}%
\pgfpathcurveto{\pgfqpoint{0.666667in}{0.385833in}}{\pgfqpoint{0.680590in}{0.385833in}}{\pgfqpoint{0.693945in}{0.380302in}}%
\pgfpathcurveto{\pgfqpoint{0.703790in}{0.370456in}}{\pgfqpoint{0.713635in}{0.360611in}}{\pgfqpoint{0.719167in}{0.347256in}}%
\pgfpathcurveto{\pgfqpoint{0.719167in}{0.333333in}}{\pgfqpoint{0.719167in}{0.319410in}}{\pgfqpoint{0.713635in}{0.306055in}}%
\pgfpathcurveto{\pgfqpoint{0.703790in}{0.296210in}}{\pgfqpoint{0.693945in}{0.286365in}}{\pgfqpoint{0.680590in}{0.280833in}}%
\pgfpathclose%
\pgfpathmoveto{\pgfqpoint{0.833333in}{0.275000in}}%
\pgfpathcurveto{\pgfqpoint{0.848804in}{0.275000in}}{\pgfqpoint{0.863642in}{0.281146in}}{\pgfqpoint{0.874581in}{0.292085in}}%
\pgfpathcurveto{\pgfqpoint{0.885520in}{0.303025in}}{\pgfqpoint{0.891667in}{0.317863in}}{\pgfqpoint{0.891667in}{0.333333in}}%
\pgfpathcurveto{\pgfqpoint{0.891667in}{0.348804in}}{\pgfqpoint{0.885520in}{0.363642in}}{\pgfqpoint{0.874581in}{0.374581in}}%
\pgfpathcurveto{\pgfqpoint{0.863642in}{0.385520in}}{\pgfqpoint{0.848804in}{0.391667in}}{\pgfqpoint{0.833333in}{0.391667in}}%
\pgfpathcurveto{\pgfqpoint{0.817863in}{0.391667in}}{\pgfqpoint{0.803025in}{0.385520in}}{\pgfqpoint{0.792085in}{0.374581in}}%
\pgfpathcurveto{\pgfqpoint{0.781146in}{0.363642in}}{\pgfqpoint{0.775000in}{0.348804in}}{\pgfqpoint{0.775000in}{0.333333in}}%
\pgfpathcurveto{\pgfqpoint{0.775000in}{0.317863in}}{\pgfqpoint{0.781146in}{0.303025in}}{\pgfqpoint{0.792085in}{0.292085in}}%
\pgfpathcurveto{\pgfqpoint{0.803025in}{0.281146in}}{\pgfqpoint{0.817863in}{0.275000in}}{\pgfqpoint{0.833333in}{0.275000in}}%
\pgfpathclose%
\pgfpathmoveto{\pgfqpoint{0.833333in}{0.280833in}}%
\pgfpathcurveto{\pgfqpoint{0.833333in}{0.280833in}}{\pgfqpoint{0.819410in}{0.280833in}}{\pgfqpoint{0.806055in}{0.286365in}}%
\pgfpathcurveto{\pgfqpoint{0.796210in}{0.296210in}}{\pgfqpoint{0.786365in}{0.306055in}}{\pgfqpoint{0.780833in}{0.319410in}}%
\pgfpathcurveto{\pgfqpoint{0.780833in}{0.333333in}}{\pgfqpoint{0.780833in}{0.347256in}}{\pgfqpoint{0.786365in}{0.360611in}}%
\pgfpathcurveto{\pgfqpoint{0.796210in}{0.370456in}}{\pgfqpoint{0.806055in}{0.380302in}}{\pgfqpoint{0.819410in}{0.385833in}}%
\pgfpathcurveto{\pgfqpoint{0.833333in}{0.385833in}}{\pgfqpoint{0.847256in}{0.385833in}}{\pgfqpoint{0.860611in}{0.380302in}}%
\pgfpathcurveto{\pgfqpoint{0.870456in}{0.370456in}}{\pgfqpoint{0.880302in}{0.360611in}}{\pgfqpoint{0.885833in}{0.347256in}}%
\pgfpathcurveto{\pgfqpoint{0.885833in}{0.333333in}}{\pgfqpoint{0.885833in}{0.319410in}}{\pgfqpoint{0.880302in}{0.306055in}}%
\pgfpathcurveto{\pgfqpoint{0.870456in}{0.296210in}}{\pgfqpoint{0.860611in}{0.286365in}}{\pgfqpoint{0.847256in}{0.280833in}}%
\pgfpathclose%
\pgfpathmoveto{\pgfqpoint{1.000000in}{0.275000in}}%
\pgfpathcurveto{\pgfqpoint{1.015470in}{0.275000in}}{\pgfqpoint{1.030309in}{0.281146in}}{\pgfqpoint{1.041248in}{0.292085in}}%
\pgfpathcurveto{\pgfqpoint{1.052187in}{0.303025in}}{\pgfqpoint{1.058333in}{0.317863in}}{\pgfqpoint{1.058333in}{0.333333in}}%
\pgfpathcurveto{\pgfqpoint{1.058333in}{0.348804in}}{\pgfqpoint{1.052187in}{0.363642in}}{\pgfqpoint{1.041248in}{0.374581in}}%
\pgfpathcurveto{\pgfqpoint{1.030309in}{0.385520in}}{\pgfqpoint{1.015470in}{0.391667in}}{\pgfqpoint{1.000000in}{0.391667in}}%
\pgfpathcurveto{\pgfqpoint{0.984530in}{0.391667in}}{\pgfqpoint{0.969691in}{0.385520in}}{\pgfqpoint{0.958752in}{0.374581in}}%
\pgfpathcurveto{\pgfqpoint{0.947813in}{0.363642in}}{\pgfqpoint{0.941667in}{0.348804in}}{\pgfqpoint{0.941667in}{0.333333in}}%
\pgfpathcurveto{\pgfqpoint{0.941667in}{0.317863in}}{\pgfqpoint{0.947813in}{0.303025in}}{\pgfqpoint{0.958752in}{0.292085in}}%
\pgfpathcurveto{\pgfqpoint{0.969691in}{0.281146in}}{\pgfqpoint{0.984530in}{0.275000in}}{\pgfqpoint{1.000000in}{0.275000in}}%
\pgfpathclose%
\pgfpathmoveto{\pgfqpoint{1.000000in}{0.280833in}}%
\pgfpathcurveto{\pgfqpoint{1.000000in}{0.280833in}}{\pgfqpoint{0.986077in}{0.280833in}}{\pgfqpoint{0.972722in}{0.286365in}}%
\pgfpathcurveto{\pgfqpoint{0.962877in}{0.296210in}}{\pgfqpoint{0.953032in}{0.306055in}}{\pgfqpoint{0.947500in}{0.319410in}}%
\pgfpathcurveto{\pgfqpoint{0.947500in}{0.333333in}}{\pgfqpoint{0.947500in}{0.347256in}}{\pgfqpoint{0.953032in}{0.360611in}}%
\pgfpathcurveto{\pgfqpoint{0.962877in}{0.370456in}}{\pgfqpoint{0.972722in}{0.380302in}}{\pgfqpoint{0.986077in}{0.385833in}}%
\pgfpathcurveto{\pgfqpoint{1.000000in}{0.385833in}}{\pgfqpoint{1.013923in}{0.385833in}}{\pgfqpoint{1.027278in}{0.380302in}}%
\pgfpathcurveto{\pgfqpoint{1.037123in}{0.370456in}}{\pgfqpoint{1.046968in}{0.360611in}}{\pgfqpoint{1.052500in}{0.347256in}}%
\pgfpathcurveto{\pgfqpoint{1.052500in}{0.333333in}}{\pgfqpoint{1.052500in}{0.319410in}}{\pgfqpoint{1.046968in}{0.306055in}}%
\pgfpathcurveto{\pgfqpoint{1.037123in}{0.296210in}}{\pgfqpoint{1.027278in}{0.286365in}}{\pgfqpoint{1.013923in}{0.280833in}}%
\pgfpathclose%
\pgfpathmoveto{\pgfqpoint{0.083333in}{0.441667in}}%
\pgfpathcurveto{\pgfqpoint{0.098804in}{0.441667in}}{\pgfqpoint{0.113642in}{0.447813in}}{\pgfqpoint{0.124581in}{0.458752in}}%
\pgfpathcurveto{\pgfqpoint{0.135520in}{0.469691in}}{\pgfqpoint{0.141667in}{0.484530in}}{\pgfqpoint{0.141667in}{0.500000in}}%
\pgfpathcurveto{\pgfqpoint{0.141667in}{0.515470in}}{\pgfqpoint{0.135520in}{0.530309in}}{\pgfqpoint{0.124581in}{0.541248in}}%
\pgfpathcurveto{\pgfqpoint{0.113642in}{0.552187in}}{\pgfqpoint{0.098804in}{0.558333in}}{\pgfqpoint{0.083333in}{0.558333in}}%
\pgfpathcurveto{\pgfqpoint{0.067863in}{0.558333in}}{\pgfqpoint{0.053025in}{0.552187in}}{\pgfqpoint{0.042085in}{0.541248in}}%
\pgfpathcurveto{\pgfqpoint{0.031146in}{0.530309in}}{\pgfqpoint{0.025000in}{0.515470in}}{\pgfqpoint{0.025000in}{0.500000in}}%
\pgfpathcurveto{\pgfqpoint{0.025000in}{0.484530in}}{\pgfqpoint{0.031146in}{0.469691in}}{\pgfqpoint{0.042085in}{0.458752in}}%
\pgfpathcurveto{\pgfqpoint{0.053025in}{0.447813in}}{\pgfqpoint{0.067863in}{0.441667in}}{\pgfqpoint{0.083333in}{0.441667in}}%
\pgfpathclose%
\pgfpathmoveto{\pgfqpoint{0.083333in}{0.447500in}}%
\pgfpathcurveto{\pgfqpoint{0.083333in}{0.447500in}}{\pgfqpoint{0.069410in}{0.447500in}}{\pgfqpoint{0.056055in}{0.453032in}}%
\pgfpathcurveto{\pgfqpoint{0.046210in}{0.462877in}}{\pgfqpoint{0.036365in}{0.472722in}}{\pgfqpoint{0.030833in}{0.486077in}}%
\pgfpathcurveto{\pgfqpoint{0.030833in}{0.500000in}}{\pgfqpoint{0.030833in}{0.513923in}}{\pgfqpoint{0.036365in}{0.527278in}}%
\pgfpathcurveto{\pgfqpoint{0.046210in}{0.537123in}}{\pgfqpoint{0.056055in}{0.546968in}}{\pgfqpoint{0.069410in}{0.552500in}}%
\pgfpathcurveto{\pgfqpoint{0.083333in}{0.552500in}}{\pgfqpoint{0.097256in}{0.552500in}}{\pgfqpoint{0.110611in}{0.546968in}}%
\pgfpathcurveto{\pgfqpoint{0.120456in}{0.537123in}}{\pgfqpoint{0.130302in}{0.527278in}}{\pgfqpoint{0.135833in}{0.513923in}}%
\pgfpathcurveto{\pgfqpoint{0.135833in}{0.500000in}}{\pgfqpoint{0.135833in}{0.486077in}}{\pgfqpoint{0.130302in}{0.472722in}}%
\pgfpathcurveto{\pgfqpoint{0.120456in}{0.462877in}}{\pgfqpoint{0.110611in}{0.453032in}}{\pgfqpoint{0.097256in}{0.447500in}}%
\pgfpathclose%
\pgfpathmoveto{\pgfqpoint{0.250000in}{0.441667in}}%
\pgfpathcurveto{\pgfqpoint{0.265470in}{0.441667in}}{\pgfqpoint{0.280309in}{0.447813in}}{\pgfqpoint{0.291248in}{0.458752in}}%
\pgfpathcurveto{\pgfqpoint{0.302187in}{0.469691in}}{\pgfqpoint{0.308333in}{0.484530in}}{\pgfqpoint{0.308333in}{0.500000in}}%
\pgfpathcurveto{\pgfqpoint{0.308333in}{0.515470in}}{\pgfqpoint{0.302187in}{0.530309in}}{\pgfqpoint{0.291248in}{0.541248in}}%
\pgfpathcurveto{\pgfqpoint{0.280309in}{0.552187in}}{\pgfqpoint{0.265470in}{0.558333in}}{\pgfqpoint{0.250000in}{0.558333in}}%
\pgfpathcurveto{\pgfqpoint{0.234530in}{0.558333in}}{\pgfqpoint{0.219691in}{0.552187in}}{\pgfqpoint{0.208752in}{0.541248in}}%
\pgfpathcurveto{\pgfqpoint{0.197813in}{0.530309in}}{\pgfqpoint{0.191667in}{0.515470in}}{\pgfqpoint{0.191667in}{0.500000in}}%
\pgfpathcurveto{\pgfqpoint{0.191667in}{0.484530in}}{\pgfqpoint{0.197813in}{0.469691in}}{\pgfqpoint{0.208752in}{0.458752in}}%
\pgfpathcurveto{\pgfqpoint{0.219691in}{0.447813in}}{\pgfqpoint{0.234530in}{0.441667in}}{\pgfqpoint{0.250000in}{0.441667in}}%
\pgfpathclose%
\pgfpathmoveto{\pgfqpoint{0.250000in}{0.447500in}}%
\pgfpathcurveto{\pgfqpoint{0.250000in}{0.447500in}}{\pgfqpoint{0.236077in}{0.447500in}}{\pgfqpoint{0.222722in}{0.453032in}}%
\pgfpathcurveto{\pgfqpoint{0.212877in}{0.462877in}}{\pgfqpoint{0.203032in}{0.472722in}}{\pgfqpoint{0.197500in}{0.486077in}}%
\pgfpathcurveto{\pgfqpoint{0.197500in}{0.500000in}}{\pgfqpoint{0.197500in}{0.513923in}}{\pgfqpoint{0.203032in}{0.527278in}}%
\pgfpathcurveto{\pgfqpoint{0.212877in}{0.537123in}}{\pgfqpoint{0.222722in}{0.546968in}}{\pgfqpoint{0.236077in}{0.552500in}}%
\pgfpathcurveto{\pgfqpoint{0.250000in}{0.552500in}}{\pgfqpoint{0.263923in}{0.552500in}}{\pgfqpoint{0.277278in}{0.546968in}}%
\pgfpathcurveto{\pgfqpoint{0.287123in}{0.537123in}}{\pgfqpoint{0.296968in}{0.527278in}}{\pgfqpoint{0.302500in}{0.513923in}}%
\pgfpathcurveto{\pgfqpoint{0.302500in}{0.500000in}}{\pgfqpoint{0.302500in}{0.486077in}}{\pgfqpoint{0.296968in}{0.472722in}}%
\pgfpathcurveto{\pgfqpoint{0.287123in}{0.462877in}}{\pgfqpoint{0.277278in}{0.453032in}}{\pgfqpoint{0.263923in}{0.447500in}}%
\pgfpathclose%
\pgfpathmoveto{\pgfqpoint{0.416667in}{0.441667in}}%
\pgfpathcurveto{\pgfqpoint{0.432137in}{0.441667in}}{\pgfqpoint{0.446975in}{0.447813in}}{\pgfqpoint{0.457915in}{0.458752in}}%
\pgfpathcurveto{\pgfqpoint{0.468854in}{0.469691in}}{\pgfqpoint{0.475000in}{0.484530in}}{\pgfqpoint{0.475000in}{0.500000in}}%
\pgfpathcurveto{\pgfqpoint{0.475000in}{0.515470in}}{\pgfqpoint{0.468854in}{0.530309in}}{\pgfqpoint{0.457915in}{0.541248in}}%
\pgfpathcurveto{\pgfqpoint{0.446975in}{0.552187in}}{\pgfqpoint{0.432137in}{0.558333in}}{\pgfqpoint{0.416667in}{0.558333in}}%
\pgfpathcurveto{\pgfqpoint{0.401196in}{0.558333in}}{\pgfqpoint{0.386358in}{0.552187in}}{\pgfqpoint{0.375419in}{0.541248in}}%
\pgfpathcurveto{\pgfqpoint{0.364480in}{0.530309in}}{\pgfqpoint{0.358333in}{0.515470in}}{\pgfqpoint{0.358333in}{0.500000in}}%
\pgfpathcurveto{\pgfqpoint{0.358333in}{0.484530in}}{\pgfqpoint{0.364480in}{0.469691in}}{\pgfqpoint{0.375419in}{0.458752in}}%
\pgfpathcurveto{\pgfqpoint{0.386358in}{0.447813in}}{\pgfqpoint{0.401196in}{0.441667in}}{\pgfqpoint{0.416667in}{0.441667in}}%
\pgfpathclose%
\pgfpathmoveto{\pgfqpoint{0.416667in}{0.447500in}}%
\pgfpathcurveto{\pgfqpoint{0.416667in}{0.447500in}}{\pgfqpoint{0.402744in}{0.447500in}}{\pgfqpoint{0.389389in}{0.453032in}}%
\pgfpathcurveto{\pgfqpoint{0.379544in}{0.462877in}}{\pgfqpoint{0.369698in}{0.472722in}}{\pgfqpoint{0.364167in}{0.486077in}}%
\pgfpathcurveto{\pgfqpoint{0.364167in}{0.500000in}}{\pgfqpoint{0.364167in}{0.513923in}}{\pgfqpoint{0.369698in}{0.527278in}}%
\pgfpathcurveto{\pgfqpoint{0.379544in}{0.537123in}}{\pgfqpoint{0.389389in}{0.546968in}}{\pgfqpoint{0.402744in}{0.552500in}}%
\pgfpathcurveto{\pgfqpoint{0.416667in}{0.552500in}}{\pgfqpoint{0.430590in}{0.552500in}}{\pgfqpoint{0.443945in}{0.546968in}}%
\pgfpathcurveto{\pgfqpoint{0.453790in}{0.537123in}}{\pgfqpoint{0.463635in}{0.527278in}}{\pgfqpoint{0.469167in}{0.513923in}}%
\pgfpathcurveto{\pgfqpoint{0.469167in}{0.500000in}}{\pgfqpoint{0.469167in}{0.486077in}}{\pgfqpoint{0.463635in}{0.472722in}}%
\pgfpathcurveto{\pgfqpoint{0.453790in}{0.462877in}}{\pgfqpoint{0.443945in}{0.453032in}}{\pgfqpoint{0.430590in}{0.447500in}}%
\pgfpathclose%
\pgfpathmoveto{\pgfqpoint{0.583333in}{0.441667in}}%
\pgfpathcurveto{\pgfqpoint{0.598804in}{0.441667in}}{\pgfqpoint{0.613642in}{0.447813in}}{\pgfqpoint{0.624581in}{0.458752in}}%
\pgfpathcurveto{\pgfqpoint{0.635520in}{0.469691in}}{\pgfqpoint{0.641667in}{0.484530in}}{\pgfqpoint{0.641667in}{0.500000in}}%
\pgfpathcurveto{\pgfqpoint{0.641667in}{0.515470in}}{\pgfqpoint{0.635520in}{0.530309in}}{\pgfqpoint{0.624581in}{0.541248in}}%
\pgfpathcurveto{\pgfqpoint{0.613642in}{0.552187in}}{\pgfqpoint{0.598804in}{0.558333in}}{\pgfqpoint{0.583333in}{0.558333in}}%
\pgfpathcurveto{\pgfqpoint{0.567863in}{0.558333in}}{\pgfqpoint{0.553025in}{0.552187in}}{\pgfqpoint{0.542085in}{0.541248in}}%
\pgfpathcurveto{\pgfqpoint{0.531146in}{0.530309in}}{\pgfqpoint{0.525000in}{0.515470in}}{\pgfqpoint{0.525000in}{0.500000in}}%
\pgfpathcurveto{\pgfqpoint{0.525000in}{0.484530in}}{\pgfqpoint{0.531146in}{0.469691in}}{\pgfqpoint{0.542085in}{0.458752in}}%
\pgfpathcurveto{\pgfqpoint{0.553025in}{0.447813in}}{\pgfqpoint{0.567863in}{0.441667in}}{\pgfqpoint{0.583333in}{0.441667in}}%
\pgfpathclose%
\pgfpathmoveto{\pgfqpoint{0.583333in}{0.447500in}}%
\pgfpathcurveto{\pgfqpoint{0.583333in}{0.447500in}}{\pgfqpoint{0.569410in}{0.447500in}}{\pgfqpoint{0.556055in}{0.453032in}}%
\pgfpathcurveto{\pgfqpoint{0.546210in}{0.462877in}}{\pgfqpoint{0.536365in}{0.472722in}}{\pgfqpoint{0.530833in}{0.486077in}}%
\pgfpathcurveto{\pgfqpoint{0.530833in}{0.500000in}}{\pgfqpoint{0.530833in}{0.513923in}}{\pgfqpoint{0.536365in}{0.527278in}}%
\pgfpathcurveto{\pgfqpoint{0.546210in}{0.537123in}}{\pgfqpoint{0.556055in}{0.546968in}}{\pgfqpoint{0.569410in}{0.552500in}}%
\pgfpathcurveto{\pgfqpoint{0.583333in}{0.552500in}}{\pgfqpoint{0.597256in}{0.552500in}}{\pgfqpoint{0.610611in}{0.546968in}}%
\pgfpathcurveto{\pgfqpoint{0.620456in}{0.537123in}}{\pgfqpoint{0.630302in}{0.527278in}}{\pgfqpoint{0.635833in}{0.513923in}}%
\pgfpathcurveto{\pgfqpoint{0.635833in}{0.500000in}}{\pgfqpoint{0.635833in}{0.486077in}}{\pgfqpoint{0.630302in}{0.472722in}}%
\pgfpathcurveto{\pgfqpoint{0.620456in}{0.462877in}}{\pgfqpoint{0.610611in}{0.453032in}}{\pgfqpoint{0.597256in}{0.447500in}}%
\pgfpathclose%
\pgfpathmoveto{\pgfqpoint{0.750000in}{0.441667in}}%
\pgfpathcurveto{\pgfqpoint{0.765470in}{0.441667in}}{\pgfqpoint{0.780309in}{0.447813in}}{\pgfqpoint{0.791248in}{0.458752in}}%
\pgfpathcurveto{\pgfqpoint{0.802187in}{0.469691in}}{\pgfqpoint{0.808333in}{0.484530in}}{\pgfqpoint{0.808333in}{0.500000in}}%
\pgfpathcurveto{\pgfqpoint{0.808333in}{0.515470in}}{\pgfqpoint{0.802187in}{0.530309in}}{\pgfqpoint{0.791248in}{0.541248in}}%
\pgfpathcurveto{\pgfqpoint{0.780309in}{0.552187in}}{\pgfqpoint{0.765470in}{0.558333in}}{\pgfqpoint{0.750000in}{0.558333in}}%
\pgfpathcurveto{\pgfqpoint{0.734530in}{0.558333in}}{\pgfqpoint{0.719691in}{0.552187in}}{\pgfqpoint{0.708752in}{0.541248in}}%
\pgfpathcurveto{\pgfqpoint{0.697813in}{0.530309in}}{\pgfqpoint{0.691667in}{0.515470in}}{\pgfqpoint{0.691667in}{0.500000in}}%
\pgfpathcurveto{\pgfqpoint{0.691667in}{0.484530in}}{\pgfqpoint{0.697813in}{0.469691in}}{\pgfqpoint{0.708752in}{0.458752in}}%
\pgfpathcurveto{\pgfqpoint{0.719691in}{0.447813in}}{\pgfqpoint{0.734530in}{0.441667in}}{\pgfqpoint{0.750000in}{0.441667in}}%
\pgfpathclose%
\pgfpathmoveto{\pgfqpoint{0.750000in}{0.447500in}}%
\pgfpathcurveto{\pgfqpoint{0.750000in}{0.447500in}}{\pgfqpoint{0.736077in}{0.447500in}}{\pgfqpoint{0.722722in}{0.453032in}}%
\pgfpathcurveto{\pgfqpoint{0.712877in}{0.462877in}}{\pgfqpoint{0.703032in}{0.472722in}}{\pgfqpoint{0.697500in}{0.486077in}}%
\pgfpathcurveto{\pgfqpoint{0.697500in}{0.500000in}}{\pgfqpoint{0.697500in}{0.513923in}}{\pgfqpoint{0.703032in}{0.527278in}}%
\pgfpathcurveto{\pgfqpoint{0.712877in}{0.537123in}}{\pgfqpoint{0.722722in}{0.546968in}}{\pgfqpoint{0.736077in}{0.552500in}}%
\pgfpathcurveto{\pgfqpoint{0.750000in}{0.552500in}}{\pgfqpoint{0.763923in}{0.552500in}}{\pgfqpoint{0.777278in}{0.546968in}}%
\pgfpathcurveto{\pgfqpoint{0.787123in}{0.537123in}}{\pgfqpoint{0.796968in}{0.527278in}}{\pgfqpoint{0.802500in}{0.513923in}}%
\pgfpathcurveto{\pgfqpoint{0.802500in}{0.500000in}}{\pgfqpoint{0.802500in}{0.486077in}}{\pgfqpoint{0.796968in}{0.472722in}}%
\pgfpathcurveto{\pgfqpoint{0.787123in}{0.462877in}}{\pgfqpoint{0.777278in}{0.453032in}}{\pgfqpoint{0.763923in}{0.447500in}}%
\pgfpathclose%
\pgfpathmoveto{\pgfqpoint{0.916667in}{0.441667in}}%
\pgfpathcurveto{\pgfqpoint{0.932137in}{0.441667in}}{\pgfqpoint{0.946975in}{0.447813in}}{\pgfqpoint{0.957915in}{0.458752in}}%
\pgfpathcurveto{\pgfqpoint{0.968854in}{0.469691in}}{\pgfqpoint{0.975000in}{0.484530in}}{\pgfqpoint{0.975000in}{0.500000in}}%
\pgfpathcurveto{\pgfqpoint{0.975000in}{0.515470in}}{\pgfqpoint{0.968854in}{0.530309in}}{\pgfqpoint{0.957915in}{0.541248in}}%
\pgfpathcurveto{\pgfqpoint{0.946975in}{0.552187in}}{\pgfqpoint{0.932137in}{0.558333in}}{\pgfqpoint{0.916667in}{0.558333in}}%
\pgfpathcurveto{\pgfqpoint{0.901196in}{0.558333in}}{\pgfqpoint{0.886358in}{0.552187in}}{\pgfqpoint{0.875419in}{0.541248in}}%
\pgfpathcurveto{\pgfqpoint{0.864480in}{0.530309in}}{\pgfqpoint{0.858333in}{0.515470in}}{\pgfqpoint{0.858333in}{0.500000in}}%
\pgfpathcurveto{\pgfqpoint{0.858333in}{0.484530in}}{\pgfqpoint{0.864480in}{0.469691in}}{\pgfqpoint{0.875419in}{0.458752in}}%
\pgfpathcurveto{\pgfqpoint{0.886358in}{0.447813in}}{\pgfqpoint{0.901196in}{0.441667in}}{\pgfqpoint{0.916667in}{0.441667in}}%
\pgfpathclose%
\pgfpathmoveto{\pgfqpoint{0.916667in}{0.447500in}}%
\pgfpathcurveto{\pgfqpoint{0.916667in}{0.447500in}}{\pgfqpoint{0.902744in}{0.447500in}}{\pgfqpoint{0.889389in}{0.453032in}}%
\pgfpathcurveto{\pgfqpoint{0.879544in}{0.462877in}}{\pgfqpoint{0.869698in}{0.472722in}}{\pgfqpoint{0.864167in}{0.486077in}}%
\pgfpathcurveto{\pgfqpoint{0.864167in}{0.500000in}}{\pgfqpoint{0.864167in}{0.513923in}}{\pgfqpoint{0.869698in}{0.527278in}}%
\pgfpathcurveto{\pgfqpoint{0.879544in}{0.537123in}}{\pgfqpoint{0.889389in}{0.546968in}}{\pgfqpoint{0.902744in}{0.552500in}}%
\pgfpathcurveto{\pgfqpoint{0.916667in}{0.552500in}}{\pgfqpoint{0.930590in}{0.552500in}}{\pgfqpoint{0.943945in}{0.546968in}}%
\pgfpathcurveto{\pgfqpoint{0.953790in}{0.537123in}}{\pgfqpoint{0.963635in}{0.527278in}}{\pgfqpoint{0.969167in}{0.513923in}}%
\pgfpathcurveto{\pgfqpoint{0.969167in}{0.500000in}}{\pgfqpoint{0.969167in}{0.486077in}}{\pgfqpoint{0.963635in}{0.472722in}}%
\pgfpathcurveto{\pgfqpoint{0.953790in}{0.462877in}}{\pgfqpoint{0.943945in}{0.453032in}}{\pgfqpoint{0.930590in}{0.447500in}}%
\pgfpathclose%
\pgfpathmoveto{\pgfqpoint{0.000000in}{0.608333in}}%
\pgfpathcurveto{\pgfqpoint{0.015470in}{0.608333in}}{\pgfqpoint{0.030309in}{0.614480in}}{\pgfqpoint{0.041248in}{0.625419in}}%
\pgfpathcurveto{\pgfqpoint{0.052187in}{0.636358in}}{\pgfqpoint{0.058333in}{0.651196in}}{\pgfqpoint{0.058333in}{0.666667in}}%
\pgfpathcurveto{\pgfqpoint{0.058333in}{0.682137in}}{\pgfqpoint{0.052187in}{0.696975in}}{\pgfqpoint{0.041248in}{0.707915in}}%
\pgfpathcurveto{\pgfqpoint{0.030309in}{0.718854in}}{\pgfqpoint{0.015470in}{0.725000in}}{\pgfqpoint{0.000000in}{0.725000in}}%
\pgfpathcurveto{\pgfqpoint{-0.015470in}{0.725000in}}{\pgfqpoint{-0.030309in}{0.718854in}}{\pgfqpoint{-0.041248in}{0.707915in}}%
\pgfpathcurveto{\pgfqpoint{-0.052187in}{0.696975in}}{\pgfqpoint{-0.058333in}{0.682137in}}{\pgfqpoint{-0.058333in}{0.666667in}}%
\pgfpathcurveto{\pgfqpoint{-0.058333in}{0.651196in}}{\pgfqpoint{-0.052187in}{0.636358in}}{\pgfqpoint{-0.041248in}{0.625419in}}%
\pgfpathcurveto{\pgfqpoint{-0.030309in}{0.614480in}}{\pgfqpoint{-0.015470in}{0.608333in}}{\pgfqpoint{0.000000in}{0.608333in}}%
\pgfpathclose%
\pgfpathmoveto{\pgfqpoint{0.000000in}{0.614167in}}%
\pgfpathcurveto{\pgfqpoint{0.000000in}{0.614167in}}{\pgfqpoint{-0.013923in}{0.614167in}}{\pgfqpoint{-0.027278in}{0.619698in}}%
\pgfpathcurveto{\pgfqpoint{-0.037123in}{0.629544in}}{\pgfqpoint{-0.046968in}{0.639389in}}{\pgfqpoint{-0.052500in}{0.652744in}}%
\pgfpathcurveto{\pgfqpoint{-0.052500in}{0.666667in}}{\pgfqpoint{-0.052500in}{0.680590in}}{\pgfqpoint{-0.046968in}{0.693945in}}%
\pgfpathcurveto{\pgfqpoint{-0.037123in}{0.703790in}}{\pgfqpoint{-0.027278in}{0.713635in}}{\pgfqpoint{-0.013923in}{0.719167in}}%
\pgfpathcurveto{\pgfqpoint{0.000000in}{0.719167in}}{\pgfqpoint{0.013923in}{0.719167in}}{\pgfqpoint{0.027278in}{0.713635in}}%
\pgfpathcurveto{\pgfqpoint{0.037123in}{0.703790in}}{\pgfqpoint{0.046968in}{0.693945in}}{\pgfqpoint{0.052500in}{0.680590in}}%
\pgfpathcurveto{\pgfqpoint{0.052500in}{0.666667in}}{\pgfqpoint{0.052500in}{0.652744in}}{\pgfqpoint{0.046968in}{0.639389in}}%
\pgfpathcurveto{\pgfqpoint{0.037123in}{0.629544in}}{\pgfqpoint{0.027278in}{0.619698in}}{\pgfqpoint{0.013923in}{0.614167in}}%
\pgfpathclose%
\pgfpathmoveto{\pgfqpoint{0.166667in}{0.608333in}}%
\pgfpathcurveto{\pgfqpoint{0.182137in}{0.608333in}}{\pgfqpoint{0.196975in}{0.614480in}}{\pgfqpoint{0.207915in}{0.625419in}}%
\pgfpathcurveto{\pgfqpoint{0.218854in}{0.636358in}}{\pgfqpoint{0.225000in}{0.651196in}}{\pgfqpoint{0.225000in}{0.666667in}}%
\pgfpathcurveto{\pgfqpoint{0.225000in}{0.682137in}}{\pgfqpoint{0.218854in}{0.696975in}}{\pgfqpoint{0.207915in}{0.707915in}}%
\pgfpathcurveto{\pgfqpoint{0.196975in}{0.718854in}}{\pgfqpoint{0.182137in}{0.725000in}}{\pgfqpoint{0.166667in}{0.725000in}}%
\pgfpathcurveto{\pgfqpoint{0.151196in}{0.725000in}}{\pgfqpoint{0.136358in}{0.718854in}}{\pgfqpoint{0.125419in}{0.707915in}}%
\pgfpathcurveto{\pgfqpoint{0.114480in}{0.696975in}}{\pgfqpoint{0.108333in}{0.682137in}}{\pgfqpoint{0.108333in}{0.666667in}}%
\pgfpathcurveto{\pgfqpoint{0.108333in}{0.651196in}}{\pgfqpoint{0.114480in}{0.636358in}}{\pgfqpoint{0.125419in}{0.625419in}}%
\pgfpathcurveto{\pgfqpoint{0.136358in}{0.614480in}}{\pgfqpoint{0.151196in}{0.608333in}}{\pgfqpoint{0.166667in}{0.608333in}}%
\pgfpathclose%
\pgfpathmoveto{\pgfqpoint{0.166667in}{0.614167in}}%
\pgfpathcurveto{\pgfqpoint{0.166667in}{0.614167in}}{\pgfqpoint{0.152744in}{0.614167in}}{\pgfqpoint{0.139389in}{0.619698in}}%
\pgfpathcurveto{\pgfqpoint{0.129544in}{0.629544in}}{\pgfqpoint{0.119698in}{0.639389in}}{\pgfqpoint{0.114167in}{0.652744in}}%
\pgfpathcurveto{\pgfqpoint{0.114167in}{0.666667in}}{\pgfqpoint{0.114167in}{0.680590in}}{\pgfqpoint{0.119698in}{0.693945in}}%
\pgfpathcurveto{\pgfqpoint{0.129544in}{0.703790in}}{\pgfqpoint{0.139389in}{0.713635in}}{\pgfqpoint{0.152744in}{0.719167in}}%
\pgfpathcurveto{\pgfqpoint{0.166667in}{0.719167in}}{\pgfqpoint{0.180590in}{0.719167in}}{\pgfqpoint{0.193945in}{0.713635in}}%
\pgfpathcurveto{\pgfqpoint{0.203790in}{0.703790in}}{\pgfqpoint{0.213635in}{0.693945in}}{\pgfqpoint{0.219167in}{0.680590in}}%
\pgfpathcurveto{\pgfqpoint{0.219167in}{0.666667in}}{\pgfqpoint{0.219167in}{0.652744in}}{\pgfqpoint{0.213635in}{0.639389in}}%
\pgfpathcurveto{\pgfqpoint{0.203790in}{0.629544in}}{\pgfqpoint{0.193945in}{0.619698in}}{\pgfqpoint{0.180590in}{0.614167in}}%
\pgfpathclose%
\pgfpathmoveto{\pgfqpoint{0.333333in}{0.608333in}}%
\pgfpathcurveto{\pgfqpoint{0.348804in}{0.608333in}}{\pgfqpoint{0.363642in}{0.614480in}}{\pgfqpoint{0.374581in}{0.625419in}}%
\pgfpathcurveto{\pgfqpoint{0.385520in}{0.636358in}}{\pgfqpoint{0.391667in}{0.651196in}}{\pgfqpoint{0.391667in}{0.666667in}}%
\pgfpathcurveto{\pgfqpoint{0.391667in}{0.682137in}}{\pgfqpoint{0.385520in}{0.696975in}}{\pgfqpoint{0.374581in}{0.707915in}}%
\pgfpathcurveto{\pgfqpoint{0.363642in}{0.718854in}}{\pgfqpoint{0.348804in}{0.725000in}}{\pgfqpoint{0.333333in}{0.725000in}}%
\pgfpathcurveto{\pgfqpoint{0.317863in}{0.725000in}}{\pgfqpoint{0.303025in}{0.718854in}}{\pgfqpoint{0.292085in}{0.707915in}}%
\pgfpathcurveto{\pgfqpoint{0.281146in}{0.696975in}}{\pgfqpoint{0.275000in}{0.682137in}}{\pgfqpoint{0.275000in}{0.666667in}}%
\pgfpathcurveto{\pgfqpoint{0.275000in}{0.651196in}}{\pgfqpoint{0.281146in}{0.636358in}}{\pgfqpoint{0.292085in}{0.625419in}}%
\pgfpathcurveto{\pgfqpoint{0.303025in}{0.614480in}}{\pgfqpoint{0.317863in}{0.608333in}}{\pgfqpoint{0.333333in}{0.608333in}}%
\pgfpathclose%
\pgfpathmoveto{\pgfqpoint{0.333333in}{0.614167in}}%
\pgfpathcurveto{\pgfqpoint{0.333333in}{0.614167in}}{\pgfqpoint{0.319410in}{0.614167in}}{\pgfqpoint{0.306055in}{0.619698in}}%
\pgfpathcurveto{\pgfqpoint{0.296210in}{0.629544in}}{\pgfqpoint{0.286365in}{0.639389in}}{\pgfqpoint{0.280833in}{0.652744in}}%
\pgfpathcurveto{\pgfqpoint{0.280833in}{0.666667in}}{\pgfqpoint{0.280833in}{0.680590in}}{\pgfqpoint{0.286365in}{0.693945in}}%
\pgfpathcurveto{\pgfqpoint{0.296210in}{0.703790in}}{\pgfqpoint{0.306055in}{0.713635in}}{\pgfqpoint{0.319410in}{0.719167in}}%
\pgfpathcurveto{\pgfqpoint{0.333333in}{0.719167in}}{\pgfqpoint{0.347256in}{0.719167in}}{\pgfqpoint{0.360611in}{0.713635in}}%
\pgfpathcurveto{\pgfqpoint{0.370456in}{0.703790in}}{\pgfqpoint{0.380302in}{0.693945in}}{\pgfqpoint{0.385833in}{0.680590in}}%
\pgfpathcurveto{\pgfqpoint{0.385833in}{0.666667in}}{\pgfqpoint{0.385833in}{0.652744in}}{\pgfqpoint{0.380302in}{0.639389in}}%
\pgfpathcurveto{\pgfqpoint{0.370456in}{0.629544in}}{\pgfqpoint{0.360611in}{0.619698in}}{\pgfqpoint{0.347256in}{0.614167in}}%
\pgfpathclose%
\pgfpathmoveto{\pgfqpoint{0.500000in}{0.608333in}}%
\pgfpathcurveto{\pgfqpoint{0.515470in}{0.608333in}}{\pgfqpoint{0.530309in}{0.614480in}}{\pgfqpoint{0.541248in}{0.625419in}}%
\pgfpathcurveto{\pgfqpoint{0.552187in}{0.636358in}}{\pgfqpoint{0.558333in}{0.651196in}}{\pgfqpoint{0.558333in}{0.666667in}}%
\pgfpathcurveto{\pgfqpoint{0.558333in}{0.682137in}}{\pgfqpoint{0.552187in}{0.696975in}}{\pgfqpoint{0.541248in}{0.707915in}}%
\pgfpathcurveto{\pgfqpoint{0.530309in}{0.718854in}}{\pgfqpoint{0.515470in}{0.725000in}}{\pgfqpoint{0.500000in}{0.725000in}}%
\pgfpathcurveto{\pgfqpoint{0.484530in}{0.725000in}}{\pgfqpoint{0.469691in}{0.718854in}}{\pgfqpoint{0.458752in}{0.707915in}}%
\pgfpathcurveto{\pgfqpoint{0.447813in}{0.696975in}}{\pgfqpoint{0.441667in}{0.682137in}}{\pgfqpoint{0.441667in}{0.666667in}}%
\pgfpathcurveto{\pgfqpoint{0.441667in}{0.651196in}}{\pgfqpoint{0.447813in}{0.636358in}}{\pgfqpoint{0.458752in}{0.625419in}}%
\pgfpathcurveto{\pgfqpoint{0.469691in}{0.614480in}}{\pgfqpoint{0.484530in}{0.608333in}}{\pgfqpoint{0.500000in}{0.608333in}}%
\pgfpathclose%
\pgfpathmoveto{\pgfqpoint{0.500000in}{0.614167in}}%
\pgfpathcurveto{\pgfqpoint{0.500000in}{0.614167in}}{\pgfqpoint{0.486077in}{0.614167in}}{\pgfqpoint{0.472722in}{0.619698in}}%
\pgfpathcurveto{\pgfqpoint{0.462877in}{0.629544in}}{\pgfqpoint{0.453032in}{0.639389in}}{\pgfqpoint{0.447500in}{0.652744in}}%
\pgfpathcurveto{\pgfqpoint{0.447500in}{0.666667in}}{\pgfqpoint{0.447500in}{0.680590in}}{\pgfqpoint{0.453032in}{0.693945in}}%
\pgfpathcurveto{\pgfqpoint{0.462877in}{0.703790in}}{\pgfqpoint{0.472722in}{0.713635in}}{\pgfqpoint{0.486077in}{0.719167in}}%
\pgfpathcurveto{\pgfqpoint{0.500000in}{0.719167in}}{\pgfqpoint{0.513923in}{0.719167in}}{\pgfqpoint{0.527278in}{0.713635in}}%
\pgfpathcurveto{\pgfqpoint{0.537123in}{0.703790in}}{\pgfqpoint{0.546968in}{0.693945in}}{\pgfqpoint{0.552500in}{0.680590in}}%
\pgfpathcurveto{\pgfqpoint{0.552500in}{0.666667in}}{\pgfqpoint{0.552500in}{0.652744in}}{\pgfqpoint{0.546968in}{0.639389in}}%
\pgfpathcurveto{\pgfqpoint{0.537123in}{0.629544in}}{\pgfqpoint{0.527278in}{0.619698in}}{\pgfqpoint{0.513923in}{0.614167in}}%
\pgfpathclose%
\pgfpathmoveto{\pgfqpoint{0.666667in}{0.608333in}}%
\pgfpathcurveto{\pgfqpoint{0.682137in}{0.608333in}}{\pgfqpoint{0.696975in}{0.614480in}}{\pgfqpoint{0.707915in}{0.625419in}}%
\pgfpathcurveto{\pgfqpoint{0.718854in}{0.636358in}}{\pgfqpoint{0.725000in}{0.651196in}}{\pgfqpoint{0.725000in}{0.666667in}}%
\pgfpathcurveto{\pgfqpoint{0.725000in}{0.682137in}}{\pgfqpoint{0.718854in}{0.696975in}}{\pgfqpoint{0.707915in}{0.707915in}}%
\pgfpathcurveto{\pgfqpoint{0.696975in}{0.718854in}}{\pgfqpoint{0.682137in}{0.725000in}}{\pgfqpoint{0.666667in}{0.725000in}}%
\pgfpathcurveto{\pgfqpoint{0.651196in}{0.725000in}}{\pgfqpoint{0.636358in}{0.718854in}}{\pgfqpoint{0.625419in}{0.707915in}}%
\pgfpathcurveto{\pgfqpoint{0.614480in}{0.696975in}}{\pgfqpoint{0.608333in}{0.682137in}}{\pgfqpoint{0.608333in}{0.666667in}}%
\pgfpathcurveto{\pgfqpoint{0.608333in}{0.651196in}}{\pgfqpoint{0.614480in}{0.636358in}}{\pgfqpoint{0.625419in}{0.625419in}}%
\pgfpathcurveto{\pgfqpoint{0.636358in}{0.614480in}}{\pgfqpoint{0.651196in}{0.608333in}}{\pgfqpoint{0.666667in}{0.608333in}}%
\pgfpathclose%
\pgfpathmoveto{\pgfqpoint{0.666667in}{0.614167in}}%
\pgfpathcurveto{\pgfqpoint{0.666667in}{0.614167in}}{\pgfqpoint{0.652744in}{0.614167in}}{\pgfqpoint{0.639389in}{0.619698in}}%
\pgfpathcurveto{\pgfqpoint{0.629544in}{0.629544in}}{\pgfqpoint{0.619698in}{0.639389in}}{\pgfqpoint{0.614167in}{0.652744in}}%
\pgfpathcurveto{\pgfqpoint{0.614167in}{0.666667in}}{\pgfqpoint{0.614167in}{0.680590in}}{\pgfqpoint{0.619698in}{0.693945in}}%
\pgfpathcurveto{\pgfqpoint{0.629544in}{0.703790in}}{\pgfqpoint{0.639389in}{0.713635in}}{\pgfqpoint{0.652744in}{0.719167in}}%
\pgfpathcurveto{\pgfqpoint{0.666667in}{0.719167in}}{\pgfqpoint{0.680590in}{0.719167in}}{\pgfqpoint{0.693945in}{0.713635in}}%
\pgfpathcurveto{\pgfqpoint{0.703790in}{0.703790in}}{\pgfqpoint{0.713635in}{0.693945in}}{\pgfqpoint{0.719167in}{0.680590in}}%
\pgfpathcurveto{\pgfqpoint{0.719167in}{0.666667in}}{\pgfqpoint{0.719167in}{0.652744in}}{\pgfqpoint{0.713635in}{0.639389in}}%
\pgfpathcurveto{\pgfqpoint{0.703790in}{0.629544in}}{\pgfqpoint{0.693945in}{0.619698in}}{\pgfqpoint{0.680590in}{0.614167in}}%
\pgfpathclose%
\pgfpathmoveto{\pgfqpoint{0.833333in}{0.608333in}}%
\pgfpathcurveto{\pgfqpoint{0.848804in}{0.608333in}}{\pgfqpoint{0.863642in}{0.614480in}}{\pgfqpoint{0.874581in}{0.625419in}}%
\pgfpathcurveto{\pgfqpoint{0.885520in}{0.636358in}}{\pgfqpoint{0.891667in}{0.651196in}}{\pgfqpoint{0.891667in}{0.666667in}}%
\pgfpathcurveto{\pgfqpoint{0.891667in}{0.682137in}}{\pgfqpoint{0.885520in}{0.696975in}}{\pgfqpoint{0.874581in}{0.707915in}}%
\pgfpathcurveto{\pgfqpoint{0.863642in}{0.718854in}}{\pgfqpoint{0.848804in}{0.725000in}}{\pgfqpoint{0.833333in}{0.725000in}}%
\pgfpathcurveto{\pgfqpoint{0.817863in}{0.725000in}}{\pgfqpoint{0.803025in}{0.718854in}}{\pgfqpoint{0.792085in}{0.707915in}}%
\pgfpathcurveto{\pgfqpoint{0.781146in}{0.696975in}}{\pgfqpoint{0.775000in}{0.682137in}}{\pgfqpoint{0.775000in}{0.666667in}}%
\pgfpathcurveto{\pgfqpoint{0.775000in}{0.651196in}}{\pgfqpoint{0.781146in}{0.636358in}}{\pgfqpoint{0.792085in}{0.625419in}}%
\pgfpathcurveto{\pgfqpoint{0.803025in}{0.614480in}}{\pgfqpoint{0.817863in}{0.608333in}}{\pgfqpoint{0.833333in}{0.608333in}}%
\pgfpathclose%
\pgfpathmoveto{\pgfqpoint{0.833333in}{0.614167in}}%
\pgfpathcurveto{\pgfqpoint{0.833333in}{0.614167in}}{\pgfqpoint{0.819410in}{0.614167in}}{\pgfqpoint{0.806055in}{0.619698in}}%
\pgfpathcurveto{\pgfqpoint{0.796210in}{0.629544in}}{\pgfqpoint{0.786365in}{0.639389in}}{\pgfqpoint{0.780833in}{0.652744in}}%
\pgfpathcurveto{\pgfqpoint{0.780833in}{0.666667in}}{\pgfqpoint{0.780833in}{0.680590in}}{\pgfqpoint{0.786365in}{0.693945in}}%
\pgfpathcurveto{\pgfqpoint{0.796210in}{0.703790in}}{\pgfqpoint{0.806055in}{0.713635in}}{\pgfqpoint{0.819410in}{0.719167in}}%
\pgfpathcurveto{\pgfqpoint{0.833333in}{0.719167in}}{\pgfqpoint{0.847256in}{0.719167in}}{\pgfqpoint{0.860611in}{0.713635in}}%
\pgfpathcurveto{\pgfqpoint{0.870456in}{0.703790in}}{\pgfqpoint{0.880302in}{0.693945in}}{\pgfqpoint{0.885833in}{0.680590in}}%
\pgfpathcurveto{\pgfqpoint{0.885833in}{0.666667in}}{\pgfqpoint{0.885833in}{0.652744in}}{\pgfqpoint{0.880302in}{0.639389in}}%
\pgfpathcurveto{\pgfqpoint{0.870456in}{0.629544in}}{\pgfqpoint{0.860611in}{0.619698in}}{\pgfqpoint{0.847256in}{0.614167in}}%
\pgfpathclose%
\pgfpathmoveto{\pgfqpoint{1.000000in}{0.608333in}}%
\pgfpathcurveto{\pgfqpoint{1.015470in}{0.608333in}}{\pgfqpoint{1.030309in}{0.614480in}}{\pgfqpoint{1.041248in}{0.625419in}}%
\pgfpathcurveto{\pgfqpoint{1.052187in}{0.636358in}}{\pgfqpoint{1.058333in}{0.651196in}}{\pgfqpoint{1.058333in}{0.666667in}}%
\pgfpathcurveto{\pgfqpoint{1.058333in}{0.682137in}}{\pgfqpoint{1.052187in}{0.696975in}}{\pgfqpoint{1.041248in}{0.707915in}}%
\pgfpathcurveto{\pgfqpoint{1.030309in}{0.718854in}}{\pgfqpoint{1.015470in}{0.725000in}}{\pgfqpoint{1.000000in}{0.725000in}}%
\pgfpathcurveto{\pgfqpoint{0.984530in}{0.725000in}}{\pgfqpoint{0.969691in}{0.718854in}}{\pgfqpoint{0.958752in}{0.707915in}}%
\pgfpathcurveto{\pgfqpoint{0.947813in}{0.696975in}}{\pgfqpoint{0.941667in}{0.682137in}}{\pgfqpoint{0.941667in}{0.666667in}}%
\pgfpathcurveto{\pgfqpoint{0.941667in}{0.651196in}}{\pgfqpoint{0.947813in}{0.636358in}}{\pgfqpoint{0.958752in}{0.625419in}}%
\pgfpathcurveto{\pgfqpoint{0.969691in}{0.614480in}}{\pgfqpoint{0.984530in}{0.608333in}}{\pgfqpoint{1.000000in}{0.608333in}}%
\pgfpathclose%
\pgfpathmoveto{\pgfqpoint{1.000000in}{0.614167in}}%
\pgfpathcurveto{\pgfqpoint{1.000000in}{0.614167in}}{\pgfqpoint{0.986077in}{0.614167in}}{\pgfqpoint{0.972722in}{0.619698in}}%
\pgfpathcurveto{\pgfqpoint{0.962877in}{0.629544in}}{\pgfqpoint{0.953032in}{0.639389in}}{\pgfqpoint{0.947500in}{0.652744in}}%
\pgfpathcurveto{\pgfqpoint{0.947500in}{0.666667in}}{\pgfqpoint{0.947500in}{0.680590in}}{\pgfqpoint{0.953032in}{0.693945in}}%
\pgfpathcurveto{\pgfqpoint{0.962877in}{0.703790in}}{\pgfqpoint{0.972722in}{0.713635in}}{\pgfqpoint{0.986077in}{0.719167in}}%
\pgfpathcurveto{\pgfqpoint{1.000000in}{0.719167in}}{\pgfqpoint{1.013923in}{0.719167in}}{\pgfqpoint{1.027278in}{0.713635in}}%
\pgfpathcurveto{\pgfqpoint{1.037123in}{0.703790in}}{\pgfqpoint{1.046968in}{0.693945in}}{\pgfqpoint{1.052500in}{0.680590in}}%
\pgfpathcurveto{\pgfqpoint{1.052500in}{0.666667in}}{\pgfqpoint{1.052500in}{0.652744in}}{\pgfqpoint{1.046968in}{0.639389in}}%
\pgfpathcurveto{\pgfqpoint{1.037123in}{0.629544in}}{\pgfqpoint{1.027278in}{0.619698in}}{\pgfqpoint{1.013923in}{0.614167in}}%
\pgfpathclose%
\pgfpathmoveto{\pgfqpoint{0.083333in}{0.775000in}}%
\pgfpathcurveto{\pgfqpoint{0.098804in}{0.775000in}}{\pgfqpoint{0.113642in}{0.781146in}}{\pgfqpoint{0.124581in}{0.792085in}}%
\pgfpathcurveto{\pgfqpoint{0.135520in}{0.803025in}}{\pgfqpoint{0.141667in}{0.817863in}}{\pgfqpoint{0.141667in}{0.833333in}}%
\pgfpathcurveto{\pgfqpoint{0.141667in}{0.848804in}}{\pgfqpoint{0.135520in}{0.863642in}}{\pgfqpoint{0.124581in}{0.874581in}}%
\pgfpathcurveto{\pgfqpoint{0.113642in}{0.885520in}}{\pgfqpoint{0.098804in}{0.891667in}}{\pgfqpoint{0.083333in}{0.891667in}}%
\pgfpathcurveto{\pgfqpoint{0.067863in}{0.891667in}}{\pgfqpoint{0.053025in}{0.885520in}}{\pgfqpoint{0.042085in}{0.874581in}}%
\pgfpathcurveto{\pgfqpoint{0.031146in}{0.863642in}}{\pgfqpoint{0.025000in}{0.848804in}}{\pgfqpoint{0.025000in}{0.833333in}}%
\pgfpathcurveto{\pgfqpoint{0.025000in}{0.817863in}}{\pgfqpoint{0.031146in}{0.803025in}}{\pgfqpoint{0.042085in}{0.792085in}}%
\pgfpathcurveto{\pgfqpoint{0.053025in}{0.781146in}}{\pgfqpoint{0.067863in}{0.775000in}}{\pgfqpoint{0.083333in}{0.775000in}}%
\pgfpathclose%
\pgfpathmoveto{\pgfqpoint{0.083333in}{0.780833in}}%
\pgfpathcurveto{\pgfqpoint{0.083333in}{0.780833in}}{\pgfqpoint{0.069410in}{0.780833in}}{\pgfqpoint{0.056055in}{0.786365in}}%
\pgfpathcurveto{\pgfqpoint{0.046210in}{0.796210in}}{\pgfqpoint{0.036365in}{0.806055in}}{\pgfqpoint{0.030833in}{0.819410in}}%
\pgfpathcurveto{\pgfqpoint{0.030833in}{0.833333in}}{\pgfqpoint{0.030833in}{0.847256in}}{\pgfqpoint{0.036365in}{0.860611in}}%
\pgfpathcurveto{\pgfqpoint{0.046210in}{0.870456in}}{\pgfqpoint{0.056055in}{0.880302in}}{\pgfqpoint{0.069410in}{0.885833in}}%
\pgfpathcurveto{\pgfqpoint{0.083333in}{0.885833in}}{\pgfqpoint{0.097256in}{0.885833in}}{\pgfqpoint{0.110611in}{0.880302in}}%
\pgfpathcurveto{\pgfqpoint{0.120456in}{0.870456in}}{\pgfqpoint{0.130302in}{0.860611in}}{\pgfqpoint{0.135833in}{0.847256in}}%
\pgfpathcurveto{\pgfqpoint{0.135833in}{0.833333in}}{\pgfqpoint{0.135833in}{0.819410in}}{\pgfqpoint{0.130302in}{0.806055in}}%
\pgfpathcurveto{\pgfqpoint{0.120456in}{0.796210in}}{\pgfqpoint{0.110611in}{0.786365in}}{\pgfqpoint{0.097256in}{0.780833in}}%
\pgfpathclose%
\pgfpathmoveto{\pgfqpoint{0.250000in}{0.775000in}}%
\pgfpathcurveto{\pgfqpoint{0.265470in}{0.775000in}}{\pgfqpoint{0.280309in}{0.781146in}}{\pgfqpoint{0.291248in}{0.792085in}}%
\pgfpathcurveto{\pgfqpoint{0.302187in}{0.803025in}}{\pgfqpoint{0.308333in}{0.817863in}}{\pgfqpoint{0.308333in}{0.833333in}}%
\pgfpathcurveto{\pgfqpoint{0.308333in}{0.848804in}}{\pgfqpoint{0.302187in}{0.863642in}}{\pgfqpoint{0.291248in}{0.874581in}}%
\pgfpathcurveto{\pgfqpoint{0.280309in}{0.885520in}}{\pgfqpoint{0.265470in}{0.891667in}}{\pgfqpoint{0.250000in}{0.891667in}}%
\pgfpathcurveto{\pgfqpoint{0.234530in}{0.891667in}}{\pgfqpoint{0.219691in}{0.885520in}}{\pgfqpoint{0.208752in}{0.874581in}}%
\pgfpathcurveto{\pgfqpoint{0.197813in}{0.863642in}}{\pgfqpoint{0.191667in}{0.848804in}}{\pgfqpoint{0.191667in}{0.833333in}}%
\pgfpathcurveto{\pgfqpoint{0.191667in}{0.817863in}}{\pgfqpoint{0.197813in}{0.803025in}}{\pgfqpoint{0.208752in}{0.792085in}}%
\pgfpathcurveto{\pgfqpoint{0.219691in}{0.781146in}}{\pgfqpoint{0.234530in}{0.775000in}}{\pgfqpoint{0.250000in}{0.775000in}}%
\pgfpathclose%
\pgfpathmoveto{\pgfqpoint{0.250000in}{0.780833in}}%
\pgfpathcurveto{\pgfqpoint{0.250000in}{0.780833in}}{\pgfqpoint{0.236077in}{0.780833in}}{\pgfqpoint{0.222722in}{0.786365in}}%
\pgfpathcurveto{\pgfqpoint{0.212877in}{0.796210in}}{\pgfqpoint{0.203032in}{0.806055in}}{\pgfqpoint{0.197500in}{0.819410in}}%
\pgfpathcurveto{\pgfqpoint{0.197500in}{0.833333in}}{\pgfqpoint{0.197500in}{0.847256in}}{\pgfqpoint{0.203032in}{0.860611in}}%
\pgfpathcurveto{\pgfqpoint{0.212877in}{0.870456in}}{\pgfqpoint{0.222722in}{0.880302in}}{\pgfqpoint{0.236077in}{0.885833in}}%
\pgfpathcurveto{\pgfqpoint{0.250000in}{0.885833in}}{\pgfqpoint{0.263923in}{0.885833in}}{\pgfqpoint{0.277278in}{0.880302in}}%
\pgfpathcurveto{\pgfqpoint{0.287123in}{0.870456in}}{\pgfqpoint{0.296968in}{0.860611in}}{\pgfqpoint{0.302500in}{0.847256in}}%
\pgfpathcurveto{\pgfqpoint{0.302500in}{0.833333in}}{\pgfqpoint{0.302500in}{0.819410in}}{\pgfqpoint{0.296968in}{0.806055in}}%
\pgfpathcurveto{\pgfqpoint{0.287123in}{0.796210in}}{\pgfqpoint{0.277278in}{0.786365in}}{\pgfqpoint{0.263923in}{0.780833in}}%
\pgfpathclose%
\pgfpathmoveto{\pgfqpoint{0.416667in}{0.775000in}}%
\pgfpathcurveto{\pgfqpoint{0.432137in}{0.775000in}}{\pgfqpoint{0.446975in}{0.781146in}}{\pgfqpoint{0.457915in}{0.792085in}}%
\pgfpathcurveto{\pgfqpoint{0.468854in}{0.803025in}}{\pgfqpoint{0.475000in}{0.817863in}}{\pgfqpoint{0.475000in}{0.833333in}}%
\pgfpathcurveto{\pgfqpoint{0.475000in}{0.848804in}}{\pgfqpoint{0.468854in}{0.863642in}}{\pgfqpoint{0.457915in}{0.874581in}}%
\pgfpathcurveto{\pgfqpoint{0.446975in}{0.885520in}}{\pgfqpoint{0.432137in}{0.891667in}}{\pgfqpoint{0.416667in}{0.891667in}}%
\pgfpathcurveto{\pgfqpoint{0.401196in}{0.891667in}}{\pgfqpoint{0.386358in}{0.885520in}}{\pgfqpoint{0.375419in}{0.874581in}}%
\pgfpathcurveto{\pgfqpoint{0.364480in}{0.863642in}}{\pgfqpoint{0.358333in}{0.848804in}}{\pgfqpoint{0.358333in}{0.833333in}}%
\pgfpathcurveto{\pgfqpoint{0.358333in}{0.817863in}}{\pgfqpoint{0.364480in}{0.803025in}}{\pgfqpoint{0.375419in}{0.792085in}}%
\pgfpathcurveto{\pgfqpoint{0.386358in}{0.781146in}}{\pgfqpoint{0.401196in}{0.775000in}}{\pgfqpoint{0.416667in}{0.775000in}}%
\pgfpathclose%
\pgfpathmoveto{\pgfqpoint{0.416667in}{0.780833in}}%
\pgfpathcurveto{\pgfqpoint{0.416667in}{0.780833in}}{\pgfqpoint{0.402744in}{0.780833in}}{\pgfqpoint{0.389389in}{0.786365in}}%
\pgfpathcurveto{\pgfqpoint{0.379544in}{0.796210in}}{\pgfqpoint{0.369698in}{0.806055in}}{\pgfqpoint{0.364167in}{0.819410in}}%
\pgfpathcurveto{\pgfqpoint{0.364167in}{0.833333in}}{\pgfqpoint{0.364167in}{0.847256in}}{\pgfqpoint{0.369698in}{0.860611in}}%
\pgfpathcurveto{\pgfqpoint{0.379544in}{0.870456in}}{\pgfqpoint{0.389389in}{0.880302in}}{\pgfqpoint{0.402744in}{0.885833in}}%
\pgfpathcurveto{\pgfqpoint{0.416667in}{0.885833in}}{\pgfqpoint{0.430590in}{0.885833in}}{\pgfqpoint{0.443945in}{0.880302in}}%
\pgfpathcurveto{\pgfqpoint{0.453790in}{0.870456in}}{\pgfqpoint{0.463635in}{0.860611in}}{\pgfqpoint{0.469167in}{0.847256in}}%
\pgfpathcurveto{\pgfqpoint{0.469167in}{0.833333in}}{\pgfqpoint{0.469167in}{0.819410in}}{\pgfqpoint{0.463635in}{0.806055in}}%
\pgfpathcurveto{\pgfqpoint{0.453790in}{0.796210in}}{\pgfqpoint{0.443945in}{0.786365in}}{\pgfqpoint{0.430590in}{0.780833in}}%
\pgfpathclose%
\pgfpathmoveto{\pgfqpoint{0.583333in}{0.775000in}}%
\pgfpathcurveto{\pgfqpoint{0.598804in}{0.775000in}}{\pgfqpoint{0.613642in}{0.781146in}}{\pgfqpoint{0.624581in}{0.792085in}}%
\pgfpathcurveto{\pgfqpoint{0.635520in}{0.803025in}}{\pgfqpoint{0.641667in}{0.817863in}}{\pgfqpoint{0.641667in}{0.833333in}}%
\pgfpathcurveto{\pgfqpoint{0.641667in}{0.848804in}}{\pgfqpoint{0.635520in}{0.863642in}}{\pgfqpoint{0.624581in}{0.874581in}}%
\pgfpathcurveto{\pgfqpoint{0.613642in}{0.885520in}}{\pgfqpoint{0.598804in}{0.891667in}}{\pgfqpoint{0.583333in}{0.891667in}}%
\pgfpathcurveto{\pgfqpoint{0.567863in}{0.891667in}}{\pgfqpoint{0.553025in}{0.885520in}}{\pgfqpoint{0.542085in}{0.874581in}}%
\pgfpathcurveto{\pgfqpoint{0.531146in}{0.863642in}}{\pgfqpoint{0.525000in}{0.848804in}}{\pgfqpoint{0.525000in}{0.833333in}}%
\pgfpathcurveto{\pgfqpoint{0.525000in}{0.817863in}}{\pgfqpoint{0.531146in}{0.803025in}}{\pgfqpoint{0.542085in}{0.792085in}}%
\pgfpathcurveto{\pgfqpoint{0.553025in}{0.781146in}}{\pgfqpoint{0.567863in}{0.775000in}}{\pgfqpoint{0.583333in}{0.775000in}}%
\pgfpathclose%
\pgfpathmoveto{\pgfqpoint{0.583333in}{0.780833in}}%
\pgfpathcurveto{\pgfqpoint{0.583333in}{0.780833in}}{\pgfqpoint{0.569410in}{0.780833in}}{\pgfqpoint{0.556055in}{0.786365in}}%
\pgfpathcurveto{\pgfqpoint{0.546210in}{0.796210in}}{\pgfqpoint{0.536365in}{0.806055in}}{\pgfqpoint{0.530833in}{0.819410in}}%
\pgfpathcurveto{\pgfqpoint{0.530833in}{0.833333in}}{\pgfqpoint{0.530833in}{0.847256in}}{\pgfqpoint{0.536365in}{0.860611in}}%
\pgfpathcurveto{\pgfqpoint{0.546210in}{0.870456in}}{\pgfqpoint{0.556055in}{0.880302in}}{\pgfqpoint{0.569410in}{0.885833in}}%
\pgfpathcurveto{\pgfqpoint{0.583333in}{0.885833in}}{\pgfqpoint{0.597256in}{0.885833in}}{\pgfqpoint{0.610611in}{0.880302in}}%
\pgfpathcurveto{\pgfqpoint{0.620456in}{0.870456in}}{\pgfqpoint{0.630302in}{0.860611in}}{\pgfqpoint{0.635833in}{0.847256in}}%
\pgfpathcurveto{\pgfqpoint{0.635833in}{0.833333in}}{\pgfqpoint{0.635833in}{0.819410in}}{\pgfqpoint{0.630302in}{0.806055in}}%
\pgfpathcurveto{\pgfqpoint{0.620456in}{0.796210in}}{\pgfqpoint{0.610611in}{0.786365in}}{\pgfqpoint{0.597256in}{0.780833in}}%
\pgfpathclose%
\pgfpathmoveto{\pgfqpoint{0.750000in}{0.775000in}}%
\pgfpathcurveto{\pgfqpoint{0.765470in}{0.775000in}}{\pgfqpoint{0.780309in}{0.781146in}}{\pgfqpoint{0.791248in}{0.792085in}}%
\pgfpathcurveto{\pgfqpoint{0.802187in}{0.803025in}}{\pgfqpoint{0.808333in}{0.817863in}}{\pgfqpoint{0.808333in}{0.833333in}}%
\pgfpathcurveto{\pgfqpoint{0.808333in}{0.848804in}}{\pgfqpoint{0.802187in}{0.863642in}}{\pgfqpoint{0.791248in}{0.874581in}}%
\pgfpathcurveto{\pgfqpoint{0.780309in}{0.885520in}}{\pgfqpoint{0.765470in}{0.891667in}}{\pgfqpoint{0.750000in}{0.891667in}}%
\pgfpathcurveto{\pgfqpoint{0.734530in}{0.891667in}}{\pgfqpoint{0.719691in}{0.885520in}}{\pgfqpoint{0.708752in}{0.874581in}}%
\pgfpathcurveto{\pgfqpoint{0.697813in}{0.863642in}}{\pgfqpoint{0.691667in}{0.848804in}}{\pgfqpoint{0.691667in}{0.833333in}}%
\pgfpathcurveto{\pgfqpoint{0.691667in}{0.817863in}}{\pgfqpoint{0.697813in}{0.803025in}}{\pgfqpoint{0.708752in}{0.792085in}}%
\pgfpathcurveto{\pgfqpoint{0.719691in}{0.781146in}}{\pgfqpoint{0.734530in}{0.775000in}}{\pgfqpoint{0.750000in}{0.775000in}}%
\pgfpathclose%
\pgfpathmoveto{\pgfqpoint{0.750000in}{0.780833in}}%
\pgfpathcurveto{\pgfqpoint{0.750000in}{0.780833in}}{\pgfqpoint{0.736077in}{0.780833in}}{\pgfqpoint{0.722722in}{0.786365in}}%
\pgfpathcurveto{\pgfqpoint{0.712877in}{0.796210in}}{\pgfqpoint{0.703032in}{0.806055in}}{\pgfqpoint{0.697500in}{0.819410in}}%
\pgfpathcurveto{\pgfqpoint{0.697500in}{0.833333in}}{\pgfqpoint{0.697500in}{0.847256in}}{\pgfqpoint{0.703032in}{0.860611in}}%
\pgfpathcurveto{\pgfqpoint{0.712877in}{0.870456in}}{\pgfqpoint{0.722722in}{0.880302in}}{\pgfqpoint{0.736077in}{0.885833in}}%
\pgfpathcurveto{\pgfqpoint{0.750000in}{0.885833in}}{\pgfqpoint{0.763923in}{0.885833in}}{\pgfqpoint{0.777278in}{0.880302in}}%
\pgfpathcurveto{\pgfqpoint{0.787123in}{0.870456in}}{\pgfqpoint{0.796968in}{0.860611in}}{\pgfqpoint{0.802500in}{0.847256in}}%
\pgfpathcurveto{\pgfqpoint{0.802500in}{0.833333in}}{\pgfqpoint{0.802500in}{0.819410in}}{\pgfqpoint{0.796968in}{0.806055in}}%
\pgfpathcurveto{\pgfqpoint{0.787123in}{0.796210in}}{\pgfqpoint{0.777278in}{0.786365in}}{\pgfqpoint{0.763923in}{0.780833in}}%
\pgfpathclose%
\pgfpathmoveto{\pgfqpoint{0.916667in}{0.775000in}}%
\pgfpathcurveto{\pgfqpoint{0.932137in}{0.775000in}}{\pgfqpoint{0.946975in}{0.781146in}}{\pgfqpoint{0.957915in}{0.792085in}}%
\pgfpathcurveto{\pgfqpoint{0.968854in}{0.803025in}}{\pgfqpoint{0.975000in}{0.817863in}}{\pgfqpoint{0.975000in}{0.833333in}}%
\pgfpathcurveto{\pgfqpoint{0.975000in}{0.848804in}}{\pgfqpoint{0.968854in}{0.863642in}}{\pgfqpoint{0.957915in}{0.874581in}}%
\pgfpathcurveto{\pgfqpoint{0.946975in}{0.885520in}}{\pgfqpoint{0.932137in}{0.891667in}}{\pgfqpoint{0.916667in}{0.891667in}}%
\pgfpathcurveto{\pgfqpoint{0.901196in}{0.891667in}}{\pgfqpoint{0.886358in}{0.885520in}}{\pgfqpoint{0.875419in}{0.874581in}}%
\pgfpathcurveto{\pgfqpoint{0.864480in}{0.863642in}}{\pgfqpoint{0.858333in}{0.848804in}}{\pgfqpoint{0.858333in}{0.833333in}}%
\pgfpathcurveto{\pgfqpoint{0.858333in}{0.817863in}}{\pgfqpoint{0.864480in}{0.803025in}}{\pgfqpoint{0.875419in}{0.792085in}}%
\pgfpathcurveto{\pgfqpoint{0.886358in}{0.781146in}}{\pgfqpoint{0.901196in}{0.775000in}}{\pgfqpoint{0.916667in}{0.775000in}}%
\pgfpathclose%
\pgfpathmoveto{\pgfqpoint{0.916667in}{0.780833in}}%
\pgfpathcurveto{\pgfqpoint{0.916667in}{0.780833in}}{\pgfqpoint{0.902744in}{0.780833in}}{\pgfqpoint{0.889389in}{0.786365in}}%
\pgfpathcurveto{\pgfqpoint{0.879544in}{0.796210in}}{\pgfqpoint{0.869698in}{0.806055in}}{\pgfqpoint{0.864167in}{0.819410in}}%
\pgfpathcurveto{\pgfqpoint{0.864167in}{0.833333in}}{\pgfqpoint{0.864167in}{0.847256in}}{\pgfqpoint{0.869698in}{0.860611in}}%
\pgfpathcurveto{\pgfqpoint{0.879544in}{0.870456in}}{\pgfqpoint{0.889389in}{0.880302in}}{\pgfqpoint{0.902744in}{0.885833in}}%
\pgfpathcurveto{\pgfqpoint{0.916667in}{0.885833in}}{\pgfqpoint{0.930590in}{0.885833in}}{\pgfqpoint{0.943945in}{0.880302in}}%
\pgfpathcurveto{\pgfqpoint{0.953790in}{0.870456in}}{\pgfqpoint{0.963635in}{0.860611in}}{\pgfqpoint{0.969167in}{0.847256in}}%
\pgfpathcurveto{\pgfqpoint{0.969167in}{0.833333in}}{\pgfqpoint{0.969167in}{0.819410in}}{\pgfqpoint{0.963635in}{0.806055in}}%
\pgfpathcurveto{\pgfqpoint{0.953790in}{0.796210in}}{\pgfqpoint{0.943945in}{0.786365in}}{\pgfqpoint{0.930590in}{0.780833in}}%
\pgfpathclose%
\pgfpathmoveto{\pgfqpoint{0.000000in}{0.941667in}}%
\pgfpathcurveto{\pgfqpoint{0.015470in}{0.941667in}}{\pgfqpoint{0.030309in}{0.947813in}}{\pgfqpoint{0.041248in}{0.958752in}}%
\pgfpathcurveto{\pgfqpoint{0.052187in}{0.969691in}}{\pgfqpoint{0.058333in}{0.984530in}}{\pgfqpoint{0.058333in}{1.000000in}}%
\pgfpathcurveto{\pgfqpoint{0.058333in}{1.015470in}}{\pgfqpoint{0.052187in}{1.030309in}}{\pgfqpoint{0.041248in}{1.041248in}}%
\pgfpathcurveto{\pgfqpoint{0.030309in}{1.052187in}}{\pgfqpoint{0.015470in}{1.058333in}}{\pgfqpoint{0.000000in}{1.058333in}}%
\pgfpathcurveto{\pgfqpoint{-0.015470in}{1.058333in}}{\pgfqpoint{-0.030309in}{1.052187in}}{\pgfqpoint{-0.041248in}{1.041248in}}%
\pgfpathcurveto{\pgfqpoint{-0.052187in}{1.030309in}}{\pgfqpoint{-0.058333in}{1.015470in}}{\pgfqpoint{-0.058333in}{1.000000in}}%
\pgfpathcurveto{\pgfqpoint{-0.058333in}{0.984530in}}{\pgfqpoint{-0.052187in}{0.969691in}}{\pgfqpoint{-0.041248in}{0.958752in}}%
\pgfpathcurveto{\pgfqpoint{-0.030309in}{0.947813in}}{\pgfqpoint{-0.015470in}{0.941667in}}{\pgfqpoint{0.000000in}{0.941667in}}%
\pgfpathclose%
\pgfpathmoveto{\pgfqpoint{0.000000in}{0.947500in}}%
\pgfpathcurveto{\pgfqpoint{0.000000in}{0.947500in}}{\pgfqpoint{-0.013923in}{0.947500in}}{\pgfqpoint{-0.027278in}{0.953032in}}%
\pgfpathcurveto{\pgfqpoint{-0.037123in}{0.962877in}}{\pgfqpoint{-0.046968in}{0.972722in}}{\pgfqpoint{-0.052500in}{0.986077in}}%
\pgfpathcurveto{\pgfqpoint{-0.052500in}{1.000000in}}{\pgfqpoint{-0.052500in}{1.013923in}}{\pgfqpoint{-0.046968in}{1.027278in}}%
\pgfpathcurveto{\pgfqpoint{-0.037123in}{1.037123in}}{\pgfqpoint{-0.027278in}{1.046968in}}{\pgfqpoint{-0.013923in}{1.052500in}}%
\pgfpathcurveto{\pgfqpoint{0.000000in}{1.052500in}}{\pgfqpoint{0.013923in}{1.052500in}}{\pgfqpoint{0.027278in}{1.046968in}}%
\pgfpathcurveto{\pgfqpoint{0.037123in}{1.037123in}}{\pgfqpoint{0.046968in}{1.027278in}}{\pgfqpoint{0.052500in}{1.013923in}}%
\pgfpathcurveto{\pgfqpoint{0.052500in}{1.000000in}}{\pgfqpoint{0.052500in}{0.986077in}}{\pgfqpoint{0.046968in}{0.972722in}}%
\pgfpathcurveto{\pgfqpoint{0.037123in}{0.962877in}}{\pgfqpoint{0.027278in}{0.953032in}}{\pgfqpoint{0.013923in}{0.947500in}}%
\pgfpathclose%
\pgfpathmoveto{\pgfqpoint{0.166667in}{0.941667in}}%
\pgfpathcurveto{\pgfqpoint{0.182137in}{0.941667in}}{\pgfqpoint{0.196975in}{0.947813in}}{\pgfqpoint{0.207915in}{0.958752in}}%
\pgfpathcurveto{\pgfqpoint{0.218854in}{0.969691in}}{\pgfqpoint{0.225000in}{0.984530in}}{\pgfqpoint{0.225000in}{1.000000in}}%
\pgfpathcurveto{\pgfqpoint{0.225000in}{1.015470in}}{\pgfqpoint{0.218854in}{1.030309in}}{\pgfqpoint{0.207915in}{1.041248in}}%
\pgfpathcurveto{\pgfqpoint{0.196975in}{1.052187in}}{\pgfqpoint{0.182137in}{1.058333in}}{\pgfqpoint{0.166667in}{1.058333in}}%
\pgfpathcurveto{\pgfqpoint{0.151196in}{1.058333in}}{\pgfqpoint{0.136358in}{1.052187in}}{\pgfqpoint{0.125419in}{1.041248in}}%
\pgfpathcurveto{\pgfqpoint{0.114480in}{1.030309in}}{\pgfqpoint{0.108333in}{1.015470in}}{\pgfqpoint{0.108333in}{1.000000in}}%
\pgfpathcurveto{\pgfqpoint{0.108333in}{0.984530in}}{\pgfqpoint{0.114480in}{0.969691in}}{\pgfqpoint{0.125419in}{0.958752in}}%
\pgfpathcurveto{\pgfqpoint{0.136358in}{0.947813in}}{\pgfqpoint{0.151196in}{0.941667in}}{\pgfqpoint{0.166667in}{0.941667in}}%
\pgfpathclose%
\pgfpathmoveto{\pgfqpoint{0.166667in}{0.947500in}}%
\pgfpathcurveto{\pgfqpoint{0.166667in}{0.947500in}}{\pgfqpoint{0.152744in}{0.947500in}}{\pgfqpoint{0.139389in}{0.953032in}}%
\pgfpathcurveto{\pgfqpoint{0.129544in}{0.962877in}}{\pgfqpoint{0.119698in}{0.972722in}}{\pgfqpoint{0.114167in}{0.986077in}}%
\pgfpathcurveto{\pgfqpoint{0.114167in}{1.000000in}}{\pgfqpoint{0.114167in}{1.013923in}}{\pgfqpoint{0.119698in}{1.027278in}}%
\pgfpathcurveto{\pgfqpoint{0.129544in}{1.037123in}}{\pgfqpoint{0.139389in}{1.046968in}}{\pgfqpoint{0.152744in}{1.052500in}}%
\pgfpathcurveto{\pgfqpoint{0.166667in}{1.052500in}}{\pgfqpoint{0.180590in}{1.052500in}}{\pgfqpoint{0.193945in}{1.046968in}}%
\pgfpathcurveto{\pgfqpoint{0.203790in}{1.037123in}}{\pgfqpoint{0.213635in}{1.027278in}}{\pgfqpoint{0.219167in}{1.013923in}}%
\pgfpathcurveto{\pgfqpoint{0.219167in}{1.000000in}}{\pgfqpoint{0.219167in}{0.986077in}}{\pgfqpoint{0.213635in}{0.972722in}}%
\pgfpathcurveto{\pgfqpoint{0.203790in}{0.962877in}}{\pgfqpoint{0.193945in}{0.953032in}}{\pgfqpoint{0.180590in}{0.947500in}}%
\pgfpathclose%
\pgfpathmoveto{\pgfqpoint{0.333333in}{0.941667in}}%
\pgfpathcurveto{\pgfqpoint{0.348804in}{0.941667in}}{\pgfqpoint{0.363642in}{0.947813in}}{\pgfqpoint{0.374581in}{0.958752in}}%
\pgfpathcurveto{\pgfqpoint{0.385520in}{0.969691in}}{\pgfqpoint{0.391667in}{0.984530in}}{\pgfqpoint{0.391667in}{1.000000in}}%
\pgfpathcurveto{\pgfqpoint{0.391667in}{1.015470in}}{\pgfqpoint{0.385520in}{1.030309in}}{\pgfqpoint{0.374581in}{1.041248in}}%
\pgfpathcurveto{\pgfqpoint{0.363642in}{1.052187in}}{\pgfqpoint{0.348804in}{1.058333in}}{\pgfqpoint{0.333333in}{1.058333in}}%
\pgfpathcurveto{\pgfqpoint{0.317863in}{1.058333in}}{\pgfqpoint{0.303025in}{1.052187in}}{\pgfqpoint{0.292085in}{1.041248in}}%
\pgfpathcurveto{\pgfqpoint{0.281146in}{1.030309in}}{\pgfqpoint{0.275000in}{1.015470in}}{\pgfqpoint{0.275000in}{1.000000in}}%
\pgfpathcurveto{\pgfqpoint{0.275000in}{0.984530in}}{\pgfqpoint{0.281146in}{0.969691in}}{\pgfqpoint{0.292085in}{0.958752in}}%
\pgfpathcurveto{\pgfqpoint{0.303025in}{0.947813in}}{\pgfqpoint{0.317863in}{0.941667in}}{\pgfqpoint{0.333333in}{0.941667in}}%
\pgfpathclose%
\pgfpathmoveto{\pgfqpoint{0.333333in}{0.947500in}}%
\pgfpathcurveto{\pgfqpoint{0.333333in}{0.947500in}}{\pgfqpoint{0.319410in}{0.947500in}}{\pgfqpoint{0.306055in}{0.953032in}}%
\pgfpathcurveto{\pgfqpoint{0.296210in}{0.962877in}}{\pgfqpoint{0.286365in}{0.972722in}}{\pgfqpoint{0.280833in}{0.986077in}}%
\pgfpathcurveto{\pgfqpoint{0.280833in}{1.000000in}}{\pgfqpoint{0.280833in}{1.013923in}}{\pgfqpoint{0.286365in}{1.027278in}}%
\pgfpathcurveto{\pgfqpoint{0.296210in}{1.037123in}}{\pgfqpoint{0.306055in}{1.046968in}}{\pgfqpoint{0.319410in}{1.052500in}}%
\pgfpathcurveto{\pgfqpoint{0.333333in}{1.052500in}}{\pgfqpoint{0.347256in}{1.052500in}}{\pgfqpoint{0.360611in}{1.046968in}}%
\pgfpathcurveto{\pgfqpoint{0.370456in}{1.037123in}}{\pgfqpoint{0.380302in}{1.027278in}}{\pgfqpoint{0.385833in}{1.013923in}}%
\pgfpathcurveto{\pgfqpoint{0.385833in}{1.000000in}}{\pgfqpoint{0.385833in}{0.986077in}}{\pgfqpoint{0.380302in}{0.972722in}}%
\pgfpathcurveto{\pgfqpoint{0.370456in}{0.962877in}}{\pgfqpoint{0.360611in}{0.953032in}}{\pgfqpoint{0.347256in}{0.947500in}}%
\pgfpathclose%
\pgfpathmoveto{\pgfqpoint{0.500000in}{0.941667in}}%
\pgfpathcurveto{\pgfqpoint{0.515470in}{0.941667in}}{\pgfqpoint{0.530309in}{0.947813in}}{\pgfqpoint{0.541248in}{0.958752in}}%
\pgfpathcurveto{\pgfqpoint{0.552187in}{0.969691in}}{\pgfqpoint{0.558333in}{0.984530in}}{\pgfqpoint{0.558333in}{1.000000in}}%
\pgfpathcurveto{\pgfqpoint{0.558333in}{1.015470in}}{\pgfqpoint{0.552187in}{1.030309in}}{\pgfqpoint{0.541248in}{1.041248in}}%
\pgfpathcurveto{\pgfqpoint{0.530309in}{1.052187in}}{\pgfqpoint{0.515470in}{1.058333in}}{\pgfqpoint{0.500000in}{1.058333in}}%
\pgfpathcurveto{\pgfqpoint{0.484530in}{1.058333in}}{\pgfqpoint{0.469691in}{1.052187in}}{\pgfqpoint{0.458752in}{1.041248in}}%
\pgfpathcurveto{\pgfqpoint{0.447813in}{1.030309in}}{\pgfqpoint{0.441667in}{1.015470in}}{\pgfqpoint{0.441667in}{1.000000in}}%
\pgfpathcurveto{\pgfqpoint{0.441667in}{0.984530in}}{\pgfqpoint{0.447813in}{0.969691in}}{\pgfqpoint{0.458752in}{0.958752in}}%
\pgfpathcurveto{\pgfqpoint{0.469691in}{0.947813in}}{\pgfqpoint{0.484530in}{0.941667in}}{\pgfqpoint{0.500000in}{0.941667in}}%
\pgfpathclose%
\pgfpathmoveto{\pgfqpoint{0.500000in}{0.947500in}}%
\pgfpathcurveto{\pgfqpoint{0.500000in}{0.947500in}}{\pgfqpoint{0.486077in}{0.947500in}}{\pgfqpoint{0.472722in}{0.953032in}}%
\pgfpathcurveto{\pgfqpoint{0.462877in}{0.962877in}}{\pgfqpoint{0.453032in}{0.972722in}}{\pgfqpoint{0.447500in}{0.986077in}}%
\pgfpathcurveto{\pgfqpoint{0.447500in}{1.000000in}}{\pgfqpoint{0.447500in}{1.013923in}}{\pgfqpoint{0.453032in}{1.027278in}}%
\pgfpathcurveto{\pgfqpoint{0.462877in}{1.037123in}}{\pgfqpoint{0.472722in}{1.046968in}}{\pgfqpoint{0.486077in}{1.052500in}}%
\pgfpathcurveto{\pgfqpoint{0.500000in}{1.052500in}}{\pgfqpoint{0.513923in}{1.052500in}}{\pgfqpoint{0.527278in}{1.046968in}}%
\pgfpathcurveto{\pgfqpoint{0.537123in}{1.037123in}}{\pgfqpoint{0.546968in}{1.027278in}}{\pgfqpoint{0.552500in}{1.013923in}}%
\pgfpathcurveto{\pgfqpoint{0.552500in}{1.000000in}}{\pgfqpoint{0.552500in}{0.986077in}}{\pgfqpoint{0.546968in}{0.972722in}}%
\pgfpathcurveto{\pgfqpoint{0.537123in}{0.962877in}}{\pgfqpoint{0.527278in}{0.953032in}}{\pgfqpoint{0.513923in}{0.947500in}}%
\pgfpathclose%
\pgfpathmoveto{\pgfqpoint{0.666667in}{0.941667in}}%
\pgfpathcurveto{\pgfqpoint{0.682137in}{0.941667in}}{\pgfqpoint{0.696975in}{0.947813in}}{\pgfqpoint{0.707915in}{0.958752in}}%
\pgfpathcurveto{\pgfqpoint{0.718854in}{0.969691in}}{\pgfqpoint{0.725000in}{0.984530in}}{\pgfqpoint{0.725000in}{1.000000in}}%
\pgfpathcurveto{\pgfqpoint{0.725000in}{1.015470in}}{\pgfqpoint{0.718854in}{1.030309in}}{\pgfqpoint{0.707915in}{1.041248in}}%
\pgfpathcurveto{\pgfqpoint{0.696975in}{1.052187in}}{\pgfqpoint{0.682137in}{1.058333in}}{\pgfqpoint{0.666667in}{1.058333in}}%
\pgfpathcurveto{\pgfqpoint{0.651196in}{1.058333in}}{\pgfqpoint{0.636358in}{1.052187in}}{\pgfqpoint{0.625419in}{1.041248in}}%
\pgfpathcurveto{\pgfqpoint{0.614480in}{1.030309in}}{\pgfqpoint{0.608333in}{1.015470in}}{\pgfqpoint{0.608333in}{1.000000in}}%
\pgfpathcurveto{\pgfqpoint{0.608333in}{0.984530in}}{\pgfqpoint{0.614480in}{0.969691in}}{\pgfqpoint{0.625419in}{0.958752in}}%
\pgfpathcurveto{\pgfqpoint{0.636358in}{0.947813in}}{\pgfqpoint{0.651196in}{0.941667in}}{\pgfqpoint{0.666667in}{0.941667in}}%
\pgfpathclose%
\pgfpathmoveto{\pgfqpoint{0.666667in}{0.947500in}}%
\pgfpathcurveto{\pgfqpoint{0.666667in}{0.947500in}}{\pgfqpoint{0.652744in}{0.947500in}}{\pgfqpoint{0.639389in}{0.953032in}}%
\pgfpathcurveto{\pgfqpoint{0.629544in}{0.962877in}}{\pgfqpoint{0.619698in}{0.972722in}}{\pgfqpoint{0.614167in}{0.986077in}}%
\pgfpathcurveto{\pgfqpoint{0.614167in}{1.000000in}}{\pgfqpoint{0.614167in}{1.013923in}}{\pgfqpoint{0.619698in}{1.027278in}}%
\pgfpathcurveto{\pgfqpoint{0.629544in}{1.037123in}}{\pgfqpoint{0.639389in}{1.046968in}}{\pgfqpoint{0.652744in}{1.052500in}}%
\pgfpathcurveto{\pgfqpoint{0.666667in}{1.052500in}}{\pgfqpoint{0.680590in}{1.052500in}}{\pgfqpoint{0.693945in}{1.046968in}}%
\pgfpathcurveto{\pgfqpoint{0.703790in}{1.037123in}}{\pgfqpoint{0.713635in}{1.027278in}}{\pgfqpoint{0.719167in}{1.013923in}}%
\pgfpathcurveto{\pgfqpoint{0.719167in}{1.000000in}}{\pgfqpoint{0.719167in}{0.986077in}}{\pgfqpoint{0.713635in}{0.972722in}}%
\pgfpathcurveto{\pgfqpoint{0.703790in}{0.962877in}}{\pgfqpoint{0.693945in}{0.953032in}}{\pgfqpoint{0.680590in}{0.947500in}}%
\pgfpathclose%
\pgfpathmoveto{\pgfqpoint{0.833333in}{0.941667in}}%
\pgfpathcurveto{\pgfqpoint{0.848804in}{0.941667in}}{\pgfqpoint{0.863642in}{0.947813in}}{\pgfqpoint{0.874581in}{0.958752in}}%
\pgfpathcurveto{\pgfqpoint{0.885520in}{0.969691in}}{\pgfqpoint{0.891667in}{0.984530in}}{\pgfqpoint{0.891667in}{1.000000in}}%
\pgfpathcurveto{\pgfqpoint{0.891667in}{1.015470in}}{\pgfqpoint{0.885520in}{1.030309in}}{\pgfqpoint{0.874581in}{1.041248in}}%
\pgfpathcurveto{\pgfqpoint{0.863642in}{1.052187in}}{\pgfqpoint{0.848804in}{1.058333in}}{\pgfqpoint{0.833333in}{1.058333in}}%
\pgfpathcurveto{\pgfqpoint{0.817863in}{1.058333in}}{\pgfqpoint{0.803025in}{1.052187in}}{\pgfqpoint{0.792085in}{1.041248in}}%
\pgfpathcurveto{\pgfqpoint{0.781146in}{1.030309in}}{\pgfqpoint{0.775000in}{1.015470in}}{\pgfqpoint{0.775000in}{1.000000in}}%
\pgfpathcurveto{\pgfqpoint{0.775000in}{0.984530in}}{\pgfqpoint{0.781146in}{0.969691in}}{\pgfqpoint{0.792085in}{0.958752in}}%
\pgfpathcurveto{\pgfqpoint{0.803025in}{0.947813in}}{\pgfqpoint{0.817863in}{0.941667in}}{\pgfqpoint{0.833333in}{0.941667in}}%
\pgfpathclose%
\pgfpathmoveto{\pgfqpoint{0.833333in}{0.947500in}}%
\pgfpathcurveto{\pgfqpoint{0.833333in}{0.947500in}}{\pgfqpoint{0.819410in}{0.947500in}}{\pgfqpoint{0.806055in}{0.953032in}}%
\pgfpathcurveto{\pgfqpoint{0.796210in}{0.962877in}}{\pgfqpoint{0.786365in}{0.972722in}}{\pgfqpoint{0.780833in}{0.986077in}}%
\pgfpathcurveto{\pgfqpoint{0.780833in}{1.000000in}}{\pgfqpoint{0.780833in}{1.013923in}}{\pgfqpoint{0.786365in}{1.027278in}}%
\pgfpathcurveto{\pgfqpoint{0.796210in}{1.037123in}}{\pgfqpoint{0.806055in}{1.046968in}}{\pgfqpoint{0.819410in}{1.052500in}}%
\pgfpathcurveto{\pgfqpoint{0.833333in}{1.052500in}}{\pgfqpoint{0.847256in}{1.052500in}}{\pgfqpoint{0.860611in}{1.046968in}}%
\pgfpathcurveto{\pgfqpoint{0.870456in}{1.037123in}}{\pgfqpoint{0.880302in}{1.027278in}}{\pgfqpoint{0.885833in}{1.013923in}}%
\pgfpathcurveto{\pgfqpoint{0.885833in}{1.000000in}}{\pgfqpoint{0.885833in}{0.986077in}}{\pgfqpoint{0.880302in}{0.972722in}}%
\pgfpathcurveto{\pgfqpoint{0.870456in}{0.962877in}}{\pgfqpoint{0.860611in}{0.953032in}}{\pgfqpoint{0.847256in}{0.947500in}}%
\pgfpathclose%
\pgfpathmoveto{\pgfqpoint{1.000000in}{0.941667in}}%
\pgfpathcurveto{\pgfqpoint{1.015470in}{0.941667in}}{\pgfqpoint{1.030309in}{0.947813in}}{\pgfqpoint{1.041248in}{0.958752in}}%
\pgfpathcurveto{\pgfqpoint{1.052187in}{0.969691in}}{\pgfqpoint{1.058333in}{0.984530in}}{\pgfqpoint{1.058333in}{1.000000in}}%
\pgfpathcurveto{\pgfqpoint{1.058333in}{1.015470in}}{\pgfqpoint{1.052187in}{1.030309in}}{\pgfqpoint{1.041248in}{1.041248in}}%
\pgfpathcurveto{\pgfqpoint{1.030309in}{1.052187in}}{\pgfqpoint{1.015470in}{1.058333in}}{\pgfqpoint{1.000000in}{1.058333in}}%
\pgfpathcurveto{\pgfqpoint{0.984530in}{1.058333in}}{\pgfqpoint{0.969691in}{1.052187in}}{\pgfqpoint{0.958752in}{1.041248in}}%
\pgfpathcurveto{\pgfqpoint{0.947813in}{1.030309in}}{\pgfqpoint{0.941667in}{1.015470in}}{\pgfqpoint{0.941667in}{1.000000in}}%
\pgfpathcurveto{\pgfqpoint{0.941667in}{0.984530in}}{\pgfqpoint{0.947813in}{0.969691in}}{\pgfqpoint{0.958752in}{0.958752in}}%
\pgfpathcurveto{\pgfqpoint{0.969691in}{0.947813in}}{\pgfqpoint{0.984530in}{0.941667in}}{\pgfqpoint{1.000000in}{0.941667in}}%
\pgfpathclose%
\pgfpathmoveto{\pgfqpoint{1.000000in}{0.947500in}}%
\pgfpathcurveto{\pgfqpoint{1.000000in}{0.947500in}}{\pgfqpoint{0.986077in}{0.947500in}}{\pgfqpoint{0.972722in}{0.953032in}}%
\pgfpathcurveto{\pgfqpoint{0.962877in}{0.962877in}}{\pgfqpoint{0.953032in}{0.972722in}}{\pgfqpoint{0.947500in}{0.986077in}}%
\pgfpathcurveto{\pgfqpoint{0.947500in}{1.000000in}}{\pgfqpoint{0.947500in}{1.013923in}}{\pgfqpoint{0.953032in}{1.027278in}}%
\pgfpathcurveto{\pgfqpoint{0.962877in}{1.037123in}}{\pgfqpoint{0.972722in}{1.046968in}}{\pgfqpoint{0.986077in}{1.052500in}}%
\pgfpathcurveto{\pgfqpoint{1.000000in}{1.052500in}}{\pgfqpoint{1.013923in}{1.052500in}}{\pgfqpoint{1.027278in}{1.046968in}}%
\pgfpathcurveto{\pgfqpoint{1.037123in}{1.037123in}}{\pgfqpoint{1.046968in}{1.027278in}}{\pgfqpoint{1.052500in}{1.013923in}}%
\pgfpathcurveto{\pgfqpoint{1.052500in}{1.000000in}}{\pgfqpoint{1.052500in}{0.986077in}}{\pgfqpoint{1.046968in}{0.972722in}}%
\pgfpathcurveto{\pgfqpoint{1.037123in}{0.962877in}}{\pgfqpoint{1.027278in}{0.953032in}}{\pgfqpoint{1.013923in}{0.947500in}}%
\pgfpathclose%
\pgfusepath{stroke}%
\end{pgfscope}%
}%
\pgfsys@transformshift{5.908038in}{0.875116in}%
\pgfsys@useobject{currentpattern}{}%
\pgfsys@transformshift{1in}{0in}%
\pgfsys@transformshift{-1in}{0in}%
\pgfsys@transformshift{0in}{1in}%
\pgfsys@useobject{currentpattern}{}%
\pgfsys@transformshift{1in}{0in}%
\pgfsys@transformshift{-1in}{0in}%
\pgfsys@transformshift{0in}{1in}%
\pgfsys@useobject{currentpattern}{}%
\pgfsys@transformshift{1in}{0in}%
\pgfsys@transformshift{-1in}{0in}%
\pgfsys@transformshift{0in}{1in}%
\end{pgfscope}%
\begin{pgfscope}%
\pgfpathrectangle{\pgfqpoint{0.870538in}{0.637495in}}{\pgfqpoint{9.300000in}{9.060000in}}%
\pgfusepath{clip}%
\pgfsetbuttcap%
\pgfsetmiterjoin%
\definecolor{currentfill}{rgb}{0.549020,0.337255,0.294118}%
\pgfsetfillcolor{currentfill}%
\pgfsetfillopacity{0.990000}%
\pgfsetlinewidth{0.000000pt}%
\definecolor{currentstroke}{rgb}{0.000000,0.000000,0.000000}%
\pgfsetstrokecolor{currentstroke}%
\pgfsetstrokeopacity{0.990000}%
\pgfsetdash{}{0pt}%
\pgfpathmoveto{\pgfqpoint{7.458038in}{1.058777in}}%
\pgfpathlineto{\pgfqpoint{8.233038in}{1.058777in}}%
\pgfpathlineto{\pgfqpoint{8.233038in}{3.877341in}}%
\pgfpathlineto{\pgfqpoint{7.458038in}{3.877341in}}%
\pgfpathclose%
\pgfusepath{fill}%
\end{pgfscope}%
\begin{pgfscope}%
\pgfsetbuttcap%
\pgfsetmiterjoin%
\definecolor{currentfill}{rgb}{0.549020,0.337255,0.294118}%
\pgfsetfillcolor{currentfill}%
\pgfsetfillopacity{0.990000}%
\pgfsetlinewidth{0.000000pt}%
\definecolor{currentstroke}{rgb}{0.000000,0.000000,0.000000}%
\pgfsetstrokecolor{currentstroke}%
\pgfsetstrokeopacity{0.990000}%
\pgfsetdash{}{0pt}%
\pgfpathrectangle{\pgfqpoint{0.870538in}{0.637495in}}{\pgfqpoint{9.300000in}{9.060000in}}%
\pgfusepath{clip}%
\pgfpathmoveto{\pgfqpoint{7.458038in}{1.058777in}}%
\pgfpathlineto{\pgfqpoint{8.233038in}{1.058777in}}%
\pgfpathlineto{\pgfqpoint{8.233038in}{3.877341in}}%
\pgfpathlineto{\pgfqpoint{7.458038in}{3.877341in}}%
\pgfpathclose%
\pgfusepath{clip}%
\pgfsys@defobject{currentpattern}{\pgfqpoint{0in}{0in}}{\pgfqpoint{1in}{1in}}{%
\begin{pgfscope}%
\pgfpathrectangle{\pgfqpoint{0in}{0in}}{\pgfqpoint{1in}{1in}}%
\pgfusepath{clip}%
\pgfpathmoveto{\pgfqpoint{0.000000in}{-0.058333in}}%
\pgfpathcurveto{\pgfqpoint{0.015470in}{-0.058333in}}{\pgfqpoint{0.030309in}{-0.052187in}}{\pgfqpoint{0.041248in}{-0.041248in}}%
\pgfpathcurveto{\pgfqpoint{0.052187in}{-0.030309in}}{\pgfqpoint{0.058333in}{-0.015470in}}{\pgfqpoint{0.058333in}{0.000000in}}%
\pgfpathcurveto{\pgfqpoint{0.058333in}{0.015470in}}{\pgfqpoint{0.052187in}{0.030309in}}{\pgfqpoint{0.041248in}{0.041248in}}%
\pgfpathcurveto{\pgfqpoint{0.030309in}{0.052187in}}{\pgfqpoint{0.015470in}{0.058333in}}{\pgfqpoint{0.000000in}{0.058333in}}%
\pgfpathcurveto{\pgfqpoint{-0.015470in}{0.058333in}}{\pgfqpoint{-0.030309in}{0.052187in}}{\pgfqpoint{-0.041248in}{0.041248in}}%
\pgfpathcurveto{\pgfqpoint{-0.052187in}{0.030309in}}{\pgfqpoint{-0.058333in}{0.015470in}}{\pgfqpoint{-0.058333in}{0.000000in}}%
\pgfpathcurveto{\pgfqpoint{-0.058333in}{-0.015470in}}{\pgfqpoint{-0.052187in}{-0.030309in}}{\pgfqpoint{-0.041248in}{-0.041248in}}%
\pgfpathcurveto{\pgfqpoint{-0.030309in}{-0.052187in}}{\pgfqpoint{-0.015470in}{-0.058333in}}{\pgfqpoint{0.000000in}{-0.058333in}}%
\pgfpathclose%
\pgfpathmoveto{\pgfqpoint{0.000000in}{-0.052500in}}%
\pgfpathcurveto{\pgfqpoint{0.000000in}{-0.052500in}}{\pgfqpoint{-0.013923in}{-0.052500in}}{\pgfqpoint{-0.027278in}{-0.046968in}}%
\pgfpathcurveto{\pgfqpoint{-0.037123in}{-0.037123in}}{\pgfqpoint{-0.046968in}{-0.027278in}}{\pgfqpoint{-0.052500in}{-0.013923in}}%
\pgfpathcurveto{\pgfqpoint{-0.052500in}{0.000000in}}{\pgfqpoint{-0.052500in}{0.013923in}}{\pgfqpoint{-0.046968in}{0.027278in}}%
\pgfpathcurveto{\pgfqpoint{-0.037123in}{0.037123in}}{\pgfqpoint{-0.027278in}{0.046968in}}{\pgfqpoint{-0.013923in}{0.052500in}}%
\pgfpathcurveto{\pgfqpoint{0.000000in}{0.052500in}}{\pgfqpoint{0.013923in}{0.052500in}}{\pgfqpoint{0.027278in}{0.046968in}}%
\pgfpathcurveto{\pgfqpoint{0.037123in}{0.037123in}}{\pgfqpoint{0.046968in}{0.027278in}}{\pgfqpoint{0.052500in}{0.013923in}}%
\pgfpathcurveto{\pgfqpoint{0.052500in}{0.000000in}}{\pgfqpoint{0.052500in}{-0.013923in}}{\pgfqpoint{0.046968in}{-0.027278in}}%
\pgfpathcurveto{\pgfqpoint{0.037123in}{-0.037123in}}{\pgfqpoint{0.027278in}{-0.046968in}}{\pgfqpoint{0.013923in}{-0.052500in}}%
\pgfpathclose%
\pgfpathmoveto{\pgfqpoint{0.166667in}{-0.058333in}}%
\pgfpathcurveto{\pgfqpoint{0.182137in}{-0.058333in}}{\pgfqpoint{0.196975in}{-0.052187in}}{\pgfqpoint{0.207915in}{-0.041248in}}%
\pgfpathcurveto{\pgfqpoint{0.218854in}{-0.030309in}}{\pgfqpoint{0.225000in}{-0.015470in}}{\pgfqpoint{0.225000in}{0.000000in}}%
\pgfpathcurveto{\pgfqpoint{0.225000in}{0.015470in}}{\pgfqpoint{0.218854in}{0.030309in}}{\pgfqpoint{0.207915in}{0.041248in}}%
\pgfpathcurveto{\pgfqpoint{0.196975in}{0.052187in}}{\pgfqpoint{0.182137in}{0.058333in}}{\pgfqpoint{0.166667in}{0.058333in}}%
\pgfpathcurveto{\pgfqpoint{0.151196in}{0.058333in}}{\pgfqpoint{0.136358in}{0.052187in}}{\pgfqpoint{0.125419in}{0.041248in}}%
\pgfpathcurveto{\pgfqpoint{0.114480in}{0.030309in}}{\pgfqpoint{0.108333in}{0.015470in}}{\pgfqpoint{0.108333in}{0.000000in}}%
\pgfpathcurveto{\pgfqpoint{0.108333in}{-0.015470in}}{\pgfqpoint{0.114480in}{-0.030309in}}{\pgfqpoint{0.125419in}{-0.041248in}}%
\pgfpathcurveto{\pgfqpoint{0.136358in}{-0.052187in}}{\pgfqpoint{0.151196in}{-0.058333in}}{\pgfqpoint{0.166667in}{-0.058333in}}%
\pgfpathclose%
\pgfpathmoveto{\pgfqpoint{0.166667in}{-0.052500in}}%
\pgfpathcurveto{\pgfqpoint{0.166667in}{-0.052500in}}{\pgfqpoint{0.152744in}{-0.052500in}}{\pgfqpoint{0.139389in}{-0.046968in}}%
\pgfpathcurveto{\pgfqpoint{0.129544in}{-0.037123in}}{\pgfqpoint{0.119698in}{-0.027278in}}{\pgfqpoint{0.114167in}{-0.013923in}}%
\pgfpathcurveto{\pgfqpoint{0.114167in}{0.000000in}}{\pgfqpoint{0.114167in}{0.013923in}}{\pgfqpoint{0.119698in}{0.027278in}}%
\pgfpathcurveto{\pgfqpoint{0.129544in}{0.037123in}}{\pgfqpoint{0.139389in}{0.046968in}}{\pgfqpoint{0.152744in}{0.052500in}}%
\pgfpathcurveto{\pgfqpoint{0.166667in}{0.052500in}}{\pgfqpoint{0.180590in}{0.052500in}}{\pgfqpoint{0.193945in}{0.046968in}}%
\pgfpathcurveto{\pgfqpoint{0.203790in}{0.037123in}}{\pgfqpoint{0.213635in}{0.027278in}}{\pgfqpoint{0.219167in}{0.013923in}}%
\pgfpathcurveto{\pgfqpoint{0.219167in}{0.000000in}}{\pgfqpoint{0.219167in}{-0.013923in}}{\pgfqpoint{0.213635in}{-0.027278in}}%
\pgfpathcurveto{\pgfqpoint{0.203790in}{-0.037123in}}{\pgfqpoint{0.193945in}{-0.046968in}}{\pgfqpoint{0.180590in}{-0.052500in}}%
\pgfpathclose%
\pgfpathmoveto{\pgfqpoint{0.333333in}{-0.058333in}}%
\pgfpathcurveto{\pgfqpoint{0.348804in}{-0.058333in}}{\pgfqpoint{0.363642in}{-0.052187in}}{\pgfqpoint{0.374581in}{-0.041248in}}%
\pgfpathcurveto{\pgfqpoint{0.385520in}{-0.030309in}}{\pgfqpoint{0.391667in}{-0.015470in}}{\pgfqpoint{0.391667in}{0.000000in}}%
\pgfpathcurveto{\pgfqpoint{0.391667in}{0.015470in}}{\pgfqpoint{0.385520in}{0.030309in}}{\pgfqpoint{0.374581in}{0.041248in}}%
\pgfpathcurveto{\pgfqpoint{0.363642in}{0.052187in}}{\pgfqpoint{0.348804in}{0.058333in}}{\pgfqpoint{0.333333in}{0.058333in}}%
\pgfpathcurveto{\pgfqpoint{0.317863in}{0.058333in}}{\pgfqpoint{0.303025in}{0.052187in}}{\pgfqpoint{0.292085in}{0.041248in}}%
\pgfpathcurveto{\pgfqpoint{0.281146in}{0.030309in}}{\pgfqpoint{0.275000in}{0.015470in}}{\pgfqpoint{0.275000in}{0.000000in}}%
\pgfpathcurveto{\pgfqpoint{0.275000in}{-0.015470in}}{\pgfqpoint{0.281146in}{-0.030309in}}{\pgfqpoint{0.292085in}{-0.041248in}}%
\pgfpathcurveto{\pgfqpoint{0.303025in}{-0.052187in}}{\pgfqpoint{0.317863in}{-0.058333in}}{\pgfqpoint{0.333333in}{-0.058333in}}%
\pgfpathclose%
\pgfpathmoveto{\pgfqpoint{0.333333in}{-0.052500in}}%
\pgfpathcurveto{\pgfqpoint{0.333333in}{-0.052500in}}{\pgfqpoint{0.319410in}{-0.052500in}}{\pgfqpoint{0.306055in}{-0.046968in}}%
\pgfpathcurveto{\pgfqpoint{0.296210in}{-0.037123in}}{\pgfqpoint{0.286365in}{-0.027278in}}{\pgfqpoint{0.280833in}{-0.013923in}}%
\pgfpathcurveto{\pgfqpoint{0.280833in}{0.000000in}}{\pgfqpoint{0.280833in}{0.013923in}}{\pgfqpoint{0.286365in}{0.027278in}}%
\pgfpathcurveto{\pgfqpoint{0.296210in}{0.037123in}}{\pgfqpoint{0.306055in}{0.046968in}}{\pgfqpoint{0.319410in}{0.052500in}}%
\pgfpathcurveto{\pgfqpoint{0.333333in}{0.052500in}}{\pgfqpoint{0.347256in}{0.052500in}}{\pgfqpoint{0.360611in}{0.046968in}}%
\pgfpathcurveto{\pgfqpoint{0.370456in}{0.037123in}}{\pgfqpoint{0.380302in}{0.027278in}}{\pgfqpoint{0.385833in}{0.013923in}}%
\pgfpathcurveto{\pgfqpoint{0.385833in}{0.000000in}}{\pgfqpoint{0.385833in}{-0.013923in}}{\pgfqpoint{0.380302in}{-0.027278in}}%
\pgfpathcurveto{\pgfqpoint{0.370456in}{-0.037123in}}{\pgfqpoint{0.360611in}{-0.046968in}}{\pgfqpoint{0.347256in}{-0.052500in}}%
\pgfpathclose%
\pgfpathmoveto{\pgfqpoint{0.500000in}{-0.058333in}}%
\pgfpathcurveto{\pgfqpoint{0.515470in}{-0.058333in}}{\pgfqpoint{0.530309in}{-0.052187in}}{\pgfqpoint{0.541248in}{-0.041248in}}%
\pgfpathcurveto{\pgfqpoint{0.552187in}{-0.030309in}}{\pgfqpoint{0.558333in}{-0.015470in}}{\pgfqpoint{0.558333in}{0.000000in}}%
\pgfpathcurveto{\pgfqpoint{0.558333in}{0.015470in}}{\pgfqpoint{0.552187in}{0.030309in}}{\pgfqpoint{0.541248in}{0.041248in}}%
\pgfpathcurveto{\pgfqpoint{0.530309in}{0.052187in}}{\pgfqpoint{0.515470in}{0.058333in}}{\pgfqpoint{0.500000in}{0.058333in}}%
\pgfpathcurveto{\pgfqpoint{0.484530in}{0.058333in}}{\pgfqpoint{0.469691in}{0.052187in}}{\pgfqpoint{0.458752in}{0.041248in}}%
\pgfpathcurveto{\pgfqpoint{0.447813in}{0.030309in}}{\pgfqpoint{0.441667in}{0.015470in}}{\pgfqpoint{0.441667in}{0.000000in}}%
\pgfpathcurveto{\pgfqpoint{0.441667in}{-0.015470in}}{\pgfqpoint{0.447813in}{-0.030309in}}{\pgfqpoint{0.458752in}{-0.041248in}}%
\pgfpathcurveto{\pgfqpoint{0.469691in}{-0.052187in}}{\pgfqpoint{0.484530in}{-0.058333in}}{\pgfqpoint{0.500000in}{-0.058333in}}%
\pgfpathclose%
\pgfpathmoveto{\pgfqpoint{0.500000in}{-0.052500in}}%
\pgfpathcurveto{\pgfqpoint{0.500000in}{-0.052500in}}{\pgfqpoint{0.486077in}{-0.052500in}}{\pgfqpoint{0.472722in}{-0.046968in}}%
\pgfpathcurveto{\pgfqpoint{0.462877in}{-0.037123in}}{\pgfqpoint{0.453032in}{-0.027278in}}{\pgfqpoint{0.447500in}{-0.013923in}}%
\pgfpathcurveto{\pgfqpoint{0.447500in}{0.000000in}}{\pgfqpoint{0.447500in}{0.013923in}}{\pgfqpoint{0.453032in}{0.027278in}}%
\pgfpathcurveto{\pgfqpoint{0.462877in}{0.037123in}}{\pgfqpoint{0.472722in}{0.046968in}}{\pgfqpoint{0.486077in}{0.052500in}}%
\pgfpathcurveto{\pgfqpoint{0.500000in}{0.052500in}}{\pgfqpoint{0.513923in}{0.052500in}}{\pgfqpoint{0.527278in}{0.046968in}}%
\pgfpathcurveto{\pgfqpoint{0.537123in}{0.037123in}}{\pgfqpoint{0.546968in}{0.027278in}}{\pgfqpoint{0.552500in}{0.013923in}}%
\pgfpathcurveto{\pgfqpoint{0.552500in}{0.000000in}}{\pgfqpoint{0.552500in}{-0.013923in}}{\pgfqpoint{0.546968in}{-0.027278in}}%
\pgfpathcurveto{\pgfqpoint{0.537123in}{-0.037123in}}{\pgfqpoint{0.527278in}{-0.046968in}}{\pgfqpoint{0.513923in}{-0.052500in}}%
\pgfpathclose%
\pgfpathmoveto{\pgfqpoint{0.666667in}{-0.058333in}}%
\pgfpathcurveto{\pgfqpoint{0.682137in}{-0.058333in}}{\pgfqpoint{0.696975in}{-0.052187in}}{\pgfqpoint{0.707915in}{-0.041248in}}%
\pgfpathcurveto{\pgfqpoint{0.718854in}{-0.030309in}}{\pgfqpoint{0.725000in}{-0.015470in}}{\pgfqpoint{0.725000in}{0.000000in}}%
\pgfpathcurveto{\pgfqpoint{0.725000in}{0.015470in}}{\pgfqpoint{0.718854in}{0.030309in}}{\pgfqpoint{0.707915in}{0.041248in}}%
\pgfpathcurveto{\pgfqpoint{0.696975in}{0.052187in}}{\pgfqpoint{0.682137in}{0.058333in}}{\pgfqpoint{0.666667in}{0.058333in}}%
\pgfpathcurveto{\pgfqpoint{0.651196in}{0.058333in}}{\pgfqpoint{0.636358in}{0.052187in}}{\pgfqpoint{0.625419in}{0.041248in}}%
\pgfpathcurveto{\pgfqpoint{0.614480in}{0.030309in}}{\pgfqpoint{0.608333in}{0.015470in}}{\pgfqpoint{0.608333in}{0.000000in}}%
\pgfpathcurveto{\pgfqpoint{0.608333in}{-0.015470in}}{\pgfqpoint{0.614480in}{-0.030309in}}{\pgfqpoint{0.625419in}{-0.041248in}}%
\pgfpathcurveto{\pgfqpoint{0.636358in}{-0.052187in}}{\pgfqpoint{0.651196in}{-0.058333in}}{\pgfqpoint{0.666667in}{-0.058333in}}%
\pgfpathclose%
\pgfpathmoveto{\pgfqpoint{0.666667in}{-0.052500in}}%
\pgfpathcurveto{\pgfqpoint{0.666667in}{-0.052500in}}{\pgfqpoint{0.652744in}{-0.052500in}}{\pgfqpoint{0.639389in}{-0.046968in}}%
\pgfpathcurveto{\pgfqpoint{0.629544in}{-0.037123in}}{\pgfqpoint{0.619698in}{-0.027278in}}{\pgfqpoint{0.614167in}{-0.013923in}}%
\pgfpathcurveto{\pgfqpoint{0.614167in}{0.000000in}}{\pgfqpoint{0.614167in}{0.013923in}}{\pgfqpoint{0.619698in}{0.027278in}}%
\pgfpathcurveto{\pgfqpoint{0.629544in}{0.037123in}}{\pgfqpoint{0.639389in}{0.046968in}}{\pgfqpoint{0.652744in}{0.052500in}}%
\pgfpathcurveto{\pgfqpoint{0.666667in}{0.052500in}}{\pgfqpoint{0.680590in}{0.052500in}}{\pgfqpoint{0.693945in}{0.046968in}}%
\pgfpathcurveto{\pgfqpoint{0.703790in}{0.037123in}}{\pgfqpoint{0.713635in}{0.027278in}}{\pgfqpoint{0.719167in}{0.013923in}}%
\pgfpathcurveto{\pgfqpoint{0.719167in}{0.000000in}}{\pgfqpoint{0.719167in}{-0.013923in}}{\pgfqpoint{0.713635in}{-0.027278in}}%
\pgfpathcurveto{\pgfqpoint{0.703790in}{-0.037123in}}{\pgfqpoint{0.693945in}{-0.046968in}}{\pgfqpoint{0.680590in}{-0.052500in}}%
\pgfpathclose%
\pgfpathmoveto{\pgfqpoint{0.833333in}{-0.058333in}}%
\pgfpathcurveto{\pgfqpoint{0.848804in}{-0.058333in}}{\pgfqpoint{0.863642in}{-0.052187in}}{\pgfqpoint{0.874581in}{-0.041248in}}%
\pgfpathcurveto{\pgfqpoint{0.885520in}{-0.030309in}}{\pgfqpoint{0.891667in}{-0.015470in}}{\pgfqpoint{0.891667in}{0.000000in}}%
\pgfpathcurveto{\pgfqpoint{0.891667in}{0.015470in}}{\pgfqpoint{0.885520in}{0.030309in}}{\pgfqpoint{0.874581in}{0.041248in}}%
\pgfpathcurveto{\pgfqpoint{0.863642in}{0.052187in}}{\pgfqpoint{0.848804in}{0.058333in}}{\pgfqpoint{0.833333in}{0.058333in}}%
\pgfpathcurveto{\pgfqpoint{0.817863in}{0.058333in}}{\pgfqpoint{0.803025in}{0.052187in}}{\pgfqpoint{0.792085in}{0.041248in}}%
\pgfpathcurveto{\pgfqpoint{0.781146in}{0.030309in}}{\pgfqpoint{0.775000in}{0.015470in}}{\pgfqpoint{0.775000in}{0.000000in}}%
\pgfpathcurveto{\pgfqpoint{0.775000in}{-0.015470in}}{\pgfqpoint{0.781146in}{-0.030309in}}{\pgfqpoint{0.792085in}{-0.041248in}}%
\pgfpathcurveto{\pgfqpoint{0.803025in}{-0.052187in}}{\pgfqpoint{0.817863in}{-0.058333in}}{\pgfqpoint{0.833333in}{-0.058333in}}%
\pgfpathclose%
\pgfpathmoveto{\pgfqpoint{0.833333in}{-0.052500in}}%
\pgfpathcurveto{\pgfqpoint{0.833333in}{-0.052500in}}{\pgfqpoint{0.819410in}{-0.052500in}}{\pgfqpoint{0.806055in}{-0.046968in}}%
\pgfpathcurveto{\pgfqpoint{0.796210in}{-0.037123in}}{\pgfqpoint{0.786365in}{-0.027278in}}{\pgfqpoint{0.780833in}{-0.013923in}}%
\pgfpathcurveto{\pgfqpoint{0.780833in}{0.000000in}}{\pgfqpoint{0.780833in}{0.013923in}}{\pgfqpoint{0.786365in}{0.027278in}}%
\pgfpathcurveto{\pgfqpoint{0.796210in}{0.037123in}}{\pgfqpoint{0.806055in}{0.046968in}}{\pgfqpoint{0.819410in}{0.052500in}}%
\pgfpathcurveto{\pgfqpoint{0.833333in}{0.052500in}}{\pgfqpoint{0.847256in}{0.052500in}}{\pgfqpoint{0.860611in}{0.046968in}}%
\pgfpathcurveto{\pgfqpoint{0.870456in}{0.037123in}}{\pgfqpoint{0.880302in}{0.027278in}}{\pgfqpoint{0.885833in}{0.013923in}}%
\pgfpathcurveto{\pgfqpoint{0.885833in}{0.000000in}}{\pgfqpoint{0.885833in}{-0.013923in}}{\pgfqpoint{0.880302in}{-0.027278in}}%
\pgfpathcurveto{\pgfqpoint{0.870456in}{-0.037123in}}{\pgfqpoint{0.860611in}{-0.046968in}}{\pgfqpoint{0.847256in}{-0.052500in}}%
\pgfpathclose%
\pgfpathmoveto{\pgfqpoint{1.000000in}{-0.058333in}}%
\pgfpathcurveto{\pgfqpoint{1.015470in}{-0.058333in}}{\pgfqpoint{1.030309in}{-0.052187in}}{\pgfqpoint{1.041248in}{-0.041248in}}%
\pgfpathcurveto{\pgfqpoint{1.052187in}{-0.030309in}}{\pgfqpoint{1.058333in}{-0.015470in}}{\pgfqpoint{1.058333in}{0.000000in}}%
\pgfpathcurveto{\pgfqpoint{1.058333in}{0.015470in}}{\pgfqpoint{1.052187in}{0.030309in}}{\pgfqpoint{1.041248in}{0.041248in}}%
\pgfpathcurveto{\pgfqpoint{1.030309in}{0.052187in}}{\pgfqpoint{1.015470in}{0.058333in}}{\pgfqpoint{1.000000in}{0.058333in}}%
\pgfpathcurveto{\pgfqpoint{0.984530in}{0.058333in}}{\pgfqpoint{0.969691in}{0.052187in}}{\pgfqpoint{0.958752in}{0.041248in}}%
\pgfpathcurveto{\pgfqpoint{0.947813in}{0.030309in}}{\pgfqpoint{0.941667in}{0.015470in}}{\pgfqpoint{0.941667in}{0.000000in}}%
\pgfpathcurveto{\pgfqpoint{0.941667in}{-0.015470in}}{\pgfqpoint{0.947813in}{-0.030309in}}{\pgfqpoint{0.958752in}{-0.041248in}}%
\pgfpathcurveto{\pgfqpoint{0.969691in}{-0.052187in}}{\pgfqpoint{0.984530in}{-0.058333in}}{\pgfqpoint{1.000000in}{-0.058333in}}%
\pgfpathclose%
\pgfpathmoveto{\pgfqpoint{1.000000in}{-0.052500in}}%
\pgfpathcurveto{\pgfqpoint{1.000000in}{-0.052500in}}{\pgfqpoint{0.986077in}{-0.052500in}}{\pgfqpoint{0.972722in}{-0.046968in}}%
\pgfpathcurveto{\pgfqpoint{0.962877in}{-0.037123in}}{\pgfqpoint{0.953032in}{-0.027278in}}{\pgfqpoint{0.947500in}{-0.013923in}}%
\pgfpathcurveto{\pgfqpoint{0.947500in}{0.000000in}}{\pgfqpoint{0.947500in}{0.013923in}}{\pgfqpoint{0.953032in}{0.027278in}}%
\pgfpathcurveto{\pgfqpoint{0.962877in}{0.037123in}}{\pgfqpoint{0.972722in}{0.046968in}}{\pgfqpoint{0.986077in}{0.052500in}}%
\pgfpathcurveto{\pgfqpoint{1.000000in}{0.052500in}}{\pgfqpoint{1.013923in}{0.052500in}}{\pgfqpoint{1.027278in}{0.046968in}}%
\pgfpathcurveto{\pgfqpoint{1.037123in}{0.037123in}}{\pgfqpoint{1.046968in}{0.027278in}}{\pgfqpoint{1.052500in}{0.013923in}}%
\pgfpathcurveto{\pgfqpoint{1.052500in}{0.000000in}}{\pgfqpoint{1.052500in}{-0.013923in}}{\pgfqpoint{1.046968in}{-0.027278in}}%
\pgfpathcurveto{\pgfqpoint{1.037123in}{-0.037123in}}{\pgfqpoint{1.027278in}{-0.046968in}}{\pgfqpoint{1.013923in}{-0.052500in}}%
\pgfpathclose%
\pgfpathmoveto{\pgfqpoint{0.083333in}{0.108333in}}%
\pgfpathcurveto{\pgfqpoint{0.098804in}{0.108333in}}{\pgfqpoint{0.113642in}{0.114480in}}{\pgfqpoint{0.124581in}{0.125419in}}%
\pgfpathcurveto{\pgfqpoint{0.135520in}{0.136358in}}{\pgfqpoint{0.141667in}{0.151196in}}{\pgfqpoint{0.141667in}{0.166667in}}%
\pgfpathcurveto{\pgfqpoint{0.141667in}{0.182137in}}{\pgfqpoint{0.135520in}{0.196975in}}{\pgfqpoint{0.124581in}{0.207915in}}%
\pgfpathcurveto{\pgfqpoint{0.113642in}{0.218854in}}{\pgfqpoint{0.098804in}{0.225000in}}{\pgfqpoint{0.083333in}{0.225000in}}%
\pgfpathcurveto{\pgfqpoint{0.067863in}{0.225000in}}{\pgfqpoint{0.053025in}{0.218854in}}{\pgfqpoint{0.042085in}{0.207915in}}%
\pgfpathcurveto{\pgfqpoint{0.031146in}{0.196975in}}{\pgfqpoint{0.025000in}{0.182137in}}{\pgfqpoint{0.025000in}{0.166667in}}%
\pgfpathcurveto{\pgfqpoint{0.025000in}{0.151196in}}{\pgfqpoint{0.031146in}{0.136358in}}{\pgfqpoint{0.042085in}{0.125419in}}%
\pgfpathcurveto{\pgfqpoint{0.053025in}{0.114480in}}{\pgfqpoint{0.067863in}{0.108333in}}{\pgfqpoint{0.083333in}{0.108333in}}%
\pgfpathclose%
\pgfpathmoveto{\pgfqpoint{0.083333in}{0.114167in}}%
\pgfpathcurveto{\pgfqpoint{0.083333in}{0.114167in}}{\pgfqpoint{0.069410in}{0.114167in}}{\pgfqpoint{0.056055in}{0.119698in}}%
\pgfpathcurveto{\pgfqpoint{0.046210in}{0.129544in}}{\pgfqpoint{0.036365in}{0.139389in}}{\pgfqpoint{0.030833in}{0.152744in}}%
\pgfpathcurveto{\pgfqpoint{0.030833in}{0.166667in}}{\pgfqpoint{0.030833in}{0.180590in}}{\pgfqpoint{0.036365in}{0.193945in}}%
\pgfpathcurveto{\pgfqpoint{0.046210in}{0.203790in}}{\pgfqpoint{0.056055in}{0.213635in}}{\pgfqpoint{0.069410in}{0.219167in}}%
\pgfpathcurveto{\pgfqpoint{0.083333in}{0.219167in}}{\pgfqpoint{0.097256in}{0.219167in}}{\pgfqpoint{0.110611in}{0.213635in}}%
\pgfpathcurveto{\pgfqpoint{0.120456in}{0.203790in}}{\pgfqpoint{0.130302in}{0.193945in}}{\pgfqpoint{0.135833in}{0.180590in}}%
\pgfpathcurveto{\pgfqpoint{0.135833in}{0.166667in}}{\pgfqpoint{0.135833in}{0.152744in}}{\pgfqpoint{0.130302in}{0.139389in}}%
\pgfpathcurveto{\pgfqpoint{0.120456in}{0.129544in}}{\pgfqpoint{0.110611in}{0.119698in}}{\pgfqpoint{0.097256in}{0.114167in}}%
\pgfpathclose%
\pgfpathmoveto{\pgfqpoint{0.250000in}{0.108333in}}%
\pgfpathcurveto{\pgfqpoint{0.265470in}{0.108333in}}{\pgfqpoint{0.280309in}{0.114480in}}{\pgfqpoint{0.291248in}{0.125419in}}%
\pgfpathcurveto{\pgfqpoint{0.302187in}{0.136358in}}{\pgfqpoint{0.308333in}{0.151196in}}{\pgfqpoint{0.308333in}{0.166667in}}%
\pgfpathcurveto{\pgfqpoint{0.308333in}{0.182137in}}{\pgfqpoint{0.302187in}{0.196975in}}{\pgfqpoint{0.291248in}{0.207915in}}%
\pgfpathcurveto{\pgfqpoint{0.280309in}{0.218854in}}{\pgfqpoint{0.265470in}{0.225000in}}{\pgfqpoint{0.250000in}{0.225000in}}%
\pgfpathcurveto{\pgfqpoint{0.234530in}{0.225000in}}{\pgfqpoint{0.219691in}{0.218854in}}{\pgfqpoint{0.208752in}{0.207915in}}%
\pgfpathcurveto{\pgfqpoint{0.197813in}{0.196975in}}{\pgfqpoint{0.191667in}{0.182137in}}{\pgfqpoint{0.191667in}{0.166667in}}%
\pgfpathcurveto{\pgfqpoint{0.191667in}{0.151196in}}{\pgfqpoint{0.197813in}{0.136358in}}{\pgfqpoint{0.208752in}{0.125419in}}%
\pgfpathcurveto{\pgfqpoint{0.219691in}{0.114480in}}{\pgfqpoint{0.234530in}{0.108333in}}{\pgfqpoint{0.250000in}{0.108333in}}%
\pgfpathclose%
\pgfpathmoveto{\pgfqpoint{0.250000in}{0.114167in}}%
\pgfpathcurveto{\pgfqpoint{0.250000in}{0.114167in}}{\pgfqpoint{0.236077in}{0.114167in}}{\pgfqpoint{0.222722in}{0.119698in}}%
\pgfpathcurveto{\pgfqpoint{0.212877in}{0.129544in}}{\pgfqpoint{0.203032in}{0.139389in}}{\pgfqpoint{0.197500in}{0.152744in}}%
\pgfpathcurveto{\pgfqpoint{0.197500in}{0.166667in}}{\pgfqpoint{0.197500in}{0.180590in}}{\pgfqpoint{0.203032in}{0.193945in}}%
\pgfpathcurveto{\pgfqpoint{0.212877in}{0.203790in}}{\pgfqpoint{0.222722in}{0.213635in}}{\pgfqpoint{0.236077in}{0.219167in}}%
\pgfpathcurveto{\pgfqpoint{0.250000in}{0.219167in}}{\pgfqpoint{0.263923in}{0.219167in}}{\pgfqpoint{0.277278in}{0.213635in}}%
\pgfpathcurveto{\pgfqpoint{0.287123in}{0.203790in}}{\pgfqpoint{0.296968in}{0.193945in}}{\pgfqpoint{0.302500in}{0.180590in}}%
\pgfpathcurveto{\pgfqpoint{0.302500in}{0.166667in}}{\pgfqpoint{0.302500in}{0.152744in}}{\pgfqpoint{0.296968in}{0.139389in}}%
\pgfpathcurveto{\pgfqpoint{0.287123in}{0.129544in}}{\pgfqpoint{0.277278in}{0.119698in}}{\pgfqpoint{0.263923in}{0.114167in}}%
\pgfpathclose%
\pgfpathmoveto{\pgfqpoint{0.416667in}{0.108333in}}%
\pgfpathcurveto{\pgfqpoint{0.432137in}{0.108333in}}{\pgfqpoint{0.446975in}{0.114480in}}{\pgfqpoint{0.457915in}{0.125419in}}%
\pgfpathcurveto{\pgfqpoint{0.468854in}{0.136358in}}{\pgfqpoint{0.475000in}{0.151196in}}{\pgfqpoint{0.475000in}{0.166667in}}%
\pgfpathcurveto{\pgfqpoint{0.475000in}{0.182137in}}{\pgfqpoint{0.468854in}{0.196975in}}{\pgfqpoint{0.457915in}{0.207915in}}%
\pgfpathcurveto{\pgfqpoint{0.446975in}{0.218854in}}{\pgfqpoint{0.432137in}{0.225000in}}{\pgfqpoint{0.416667in}{0.225000in}}%
\pgfpathcurveto{\pgfqpoint{0.401196in}{0.225000in}}{\pgfqpoint{0.386358in}{0.218854in}}{\pgfqpoint{0.375419in}{0.207915in}}%
\pgfpathcurveto{\pgfqpoint{0.364480in}{0.196975in}}{\pgfqpoint{0.358333in}{0.182137in}}{\pgfqpoint{0.358333in}{0.166667in}}%
\pgfpathcurveto{\pgfqpoint{0.358333in}{0.151196in}}{\pgfqpoint{0.364480in}{0.136358in}}{\pgfqpoint{0.375419in}{0.125419in}}%
\pgfpathcurveto{\pgfqpoint{0.386358in}{0.114480in}}{\pgfqpoint{0.401196in}{0.108333in}}{\pgfqpoint{0.416667in}{0.108333in}}%
\pgfpathclose%
\pgfpathmoveto{\pgfqpoint{0.416667in}{0.114167in}}%
\pgfpathcurveto{\pgfqpoint{0.416667in}{0.114167in}}{\pgfqpoint{0.402744in}{0.114167in}}{\pgfqpoint{0.389389in}{0.119698in}}%
\pgfpathcurveto{\pgfqpoint{0.379544in}{0.129544in}}{\pgfqpoint{0.369698in}{0.139389in}}{\pgfqpoint{0.364167in}{0.152744in}}%
\pgfpathcurveto{\pgfqpoint{0.364167in}{0.166667in}}{\pgfqpoint{0.364167in}{0.180590in}}{\pgfqpoint{0.369698in}{0.193945in}}%
\pgfpathcurveto{\pgfqpoint{0.379544in}{0.203790in}}{\pgfqpoint{0.389389in}{0.213635in}}{\pgfqpoint{0.402744in}{0.219167in}}%
\pgfpathcurveto{\pgfqpoint{0.416667in}{0.219167in}}{\pgfqpoint{0.430590in}{0.219167in}}{\pgfqpoint{0.443945in}{0.213635in}}%
\pgfpathcurveto{\pgfqpoint{0.453790in}{0.203790in}}{\pgfqpoint{0.463635in}{0.193945in}}{\pgfqpoint{0.469167in}{0.180590in}}%
\pgfpathcurveto{\pgfqpoint{0.469167in}{0.166667in}}{\pgfqpoint{0.469167in}{0.152744in}}{\pgfqpoint{0.463635in}{0.139389in}}%
\pgfpathcurveto{\pgfqpoint{0.453790in}{0.129544in}}{\pgfqpoint{0.443945in}{0.119698in}}{\pgfqpoint{0.430590in}{0.114167in}}%
\pgfpathclose%
\pgfpathmoveto{\pgfqpoint{0.583333in}{0.108333in}}%
\pgfpathcurveto{\pgfqpoint{0.598804in}{0.108333in}}{\pgfqpoint{0.613642in}{0.114480in}}{\pgfqpoint{0.624581in}{0.125419in}}%
\pgfpathcurveto{\pgfqpoint{0.635520in}{0.136358in}}{\pgfqpoint{0.641667in}{0.151196in}}{\pgfqpoint{0.641667in}{0.166667in}}%
\pgfpathcurveto{\pgfqpoint{0.641667in}{0.182137in}}{\pgfqpoint{0.635520in}{0.196975in}}{\pgfqpoint{0.624581in}{0.207915in}}%
\pgfpathcurveto{\pgfqpoint{0.613642in}{0.218854in}}{\pgfqpoint{0.598804in}{0.225000in}}{\pgfqpoint{0.583333in}{0.225000in}}%
\pgfpathcurveto{\pgfqpoint{0.567863in}{0.225000in}}{\pgfqpoint{0.553025in}{0.218854in}}{\pgfqpoint{0.542085in}{0.207915in}}%
\pgfpathcurveto{\pgfqpoint{0.531146in}{0.196975in}}{\pgfqpoint{0.525000in}{0.182137in}}{\pgfqpoint{0.525000in}{0.166667in}}%
\pgfpathcurveto{\pgfqpoint{0.525000in}{0.151196in}}{\pgfqpoint{0.531146in}{0.136358in}}{\pgfqpoint{0.542085in}{0.125419in}}%
\pgfpathcurveto{\pgfqpoint{0.553025in}{0.114480in}}{\pgfqpoint{0.567863in}{0.108333in}}{\pgfqpoint{0.583333in}{0.108333in}}%
\pgfpathclose%
\pgfpathmoveto{\pgfqpoint{0.583333in}{0.114167in}}%
\pgfpathcurveto{\pgfqpoint{0.583333in}{0.114167in}}{\pgfqpoint{0.569410in}{0.114167in}}{\pgfqpoint{0.556055in}{0.119698in}}%
\pgfpathcurveto{\pgfqpoint{0.546210in}{0.129544in}}{\pgfqpoint{0.536365in}{0.139389in}}{\pgfqpoint{0.530833in}{0.152744in}}%
\pgfpathcurveto{\pgfqpoint{0.530833in}{0.166667in}}{\pgfqpoint{0.530833in}{0.180590in}}{\pgfqpoint{0.536365in}{0.193945in}}%
\pgfpathcurveto{\pgfqpoint{0.546210in}{0.203790in}}{\pgfqpoint{0.556055in}{0.213635in}}{\pgfqpoint{0.569410in}{0.219167in}}%
\pgfpathcurveto{\pgfqpoint{0.583333in}{0.219167in}}{\pgfqpoint{0.597256in}{0.219167in}}{\pgfqpoint{0.610611in}{0.213635in}}%
\pgfpathcurveto{\pgfqpoint{0.620456in}{0.203790in}}{\pgfqpoint{0.630302in}{0.193945in}}{\pgfqpoint{0.635833in}{0.180590in}}%
\pgfpathcurveto{\pgfqpoint{0.635833in}{0.166667in}}{\pgfqpoint{0.635833in}{0.152744in}}{\pgfqpoint{0.630302in}{0.139389in}}%
\pgfpathcurveto{\pgfqpoint{0.620456in}{0.129544in}}{\pgfqpoint{0.610611in}{0.119698in}}{\pgfqpoint{0.597256in}{0.114167in}}%
\pgfpathclose%
\pgfpathmoveto{\pgfqpoint{0.750000in}{0.108333in}}%
\pgfpathcurveto{\pgfqpoint{0.765470in}{0.108333in}}{\pgfqpoint{0.780309in}{0.114480in}}{\pgfqpoint{0.791248in}{0.125419in}}%
\pgfpathcurveto{\pgfqpoint{0.802187in}{0.136358in}}{\pgfqpoint{0.808333in}{0.151196in}}{\pgfqpoint{0.808333in}{0.166667in}}%
\pgfpathcurveto{\pgfqpoint{0.808333in}{0.182137in}}{\pgfqpoint{0.802187in}{0.196975in}}{\pgfqpoint{0.791248in}{0.207915in}}%
\pgfpathcurveto{\pgfqpoint{0.780309in}{0.218854in}}{\pgfqpoint{0.765470in}{0.225000in}}{\pgfqpoint{0.750000in}{0.225000in}}%
\pgfpathcurveto{\pgfqpoint{0.734530in}{0.225000in}}{\pgfqpoint{0.719691in}{0.218854in}}{\pgfqpoint{0.708752in}{0.207915in}}%
\pgfpathcurveto{\pgfqpoint{0.697813in}{0.196975in}}{\pgfqpoint{0.691667in}{0.182137in}}{\pgfqpoint{0.691667in}{0.166667in}}%
\pgfpathcurveto{\pgfqpoint{0.691667in}{0.151196in}}{\pgfqpoint{0.697813in}{0.136358in}}{\pgfqpoint{0.708752in}{0.125419in}}%
\pgfpathcurveto{\pgfqpoint{0.719691in}{0.114480in}}{\pgfqpoint{0.734530in}{0.108333in}}{\pgfqpoint{0.750000in}{0.108333in}}%
\pgfpathclose%
\pgfpathmoveto{\pgfqpoint{0.750000in}{0.114167in}}%
\pgfpathcurveto{\pgfqpoint{0.750000in}{0.114167in}}{\pgfqpoint{0.736077in}{0.114167in}}{\pgfqpoint{0.722722in}{0.119698in}}%
\pgfpathcurveto{\pgfqpoint{0.712877in}{0.129544in}}{\pgfqpoint{0.703032in}{0.139389in}}{\pgfqpoint{0.697500in}{0.152744in}}%
\pgfpathcurveto{\pgfqpoint{0.697500in}{0.166667in}}{\pgfqpoint{0.697500in}{0.180590in}}{\pgfqpoint{0.703032in}{0.193945in}}%
\pgfpathcurveto{\pgfqpoint{0.712877in}{0.203790in}}{\pgfqpoint{0.722722in}{0.213635in}}{\pgfqpoint{0.736077in}{0.219167in}}%
\pgfpathcurveto{\pgfqpoint{0.750000in}{0.219167in}}{\pgfqpoint{0.763923in}{0.219167in}}{\pgfqpoint{0.777278in}{0.213635in}}%
\pgfpathcurveto{\pgfqpoint{0.787123in}{0.203790in}}{\pgfqpoint{0.796968in}{0.193945in}}{\pgfqpoint{0.802500in}{0.180590in}}%
\pgfpathcurveto{\pgfqpoint{0.802500in}{0.166667in}}{\pgfqpoint{0.802500in}{0.152744in}}{\pgfqpoint{0.796968in}{0.139389in}}%
\pgfpathcurveto{\pgfqpoint{0.787123in}{0.129544in}}{\pgfqpoint{0.777278in}{0.119698in}}{\pgfqpoint{0.763923in}{0.114167in}}%
\pgfpathclose%
\pgfpathmoveto{\pgfqpoint{0.916667in}{0.108333in}}%
\pgfpathcurveto{\pgfqpoint{0.932137in}{0.108333in}}{\pgfqpoint{0.946975in}{0.114480in}}{\pgfqpoint{0.957915in}{0.125419in}}%
\pgfpathcurveto{\pgfqpoint{0.968854in}{0.136358in}}{\pgfqpoint{0.975000in}{0.151196in}}{\pgfqpoint{0.975000in}{0.166667in}}%
\pgfpathcurveto{\pgfqpoint{0.975000in}{0.182137in}}{\pgfqpoint{0.968854in}{0.196975in}}{\pgfqpoint{0.957915in}{0.207915in}}%
\pgfpathcurveto{\pgfqpoint{0.946975in}{0.218854in}}{\pgfqpoint{0.932137in}{0.225000in}}{\pgfqpoint{0.916667in}{0.225000in}}%
\pgfpathcurveto{\pgfqpoint{0.901196in}{0.225000in}}{\pgfqpoint{0.886358in}{0.218854in}}{\pgfqpoint{0.875419in}{0.207915in}}%
\pgfpathcurveto{\pgfqpoint{0.864480in}{0.196975in}}{\pgfqpoint{0.858333in}{0.182137in}}{\pgfqpoint{0.858333in}{0.166667in}}%
\pgfpathcurveto{\pgfqpoint{0.858333in}{0.151196in}}{\pgfqpoint{0.864480in}{0.136358in}}{\pgfqpoint{0.875419in}{0.125419in}}%
\pgfpathcurveto{\pgfqpoint{0.886358in}{0.114480in}}{\pgfqpoint{0.901196in}{0.108333in}}{\pgfqpoint{0.916667in}{0.108333in}}%
\pgfpathclose%
\pgfpathmoveto{\pgfqpoint{0.916667in}{0.114167in}}%
\pgfpathcurveto{\pgfqpoint{0.916667in}{0.114167in}}{\pgfqpoint{0.902744in}{0.114167in}}{\pgfqpoint{0.889389in}{0.119698in}}%
\pgfpathcurveto{\pgfqpoint{0.879544in}{0.129544in}}{\pgfqpoint{0.869698in}{0.139389in}}{\pgfqpoint{0.864167in}{0.152744in}}%
\pgfpathcurveto{\pgfqpoint{0.864167in}{0.166667in}}{\pgfqpoint{0.864167in}{0.180590in}}{\pgfqpoint{0.869698in}{0.193945in}}%
\pgfpathcurveto{\pgfqpoint{0.879544in}{0.203790in}}{\pgfqpoint{0.889389in}{0.213635in}}{\pgfqpoint{0.902744in}{0.219167in}}%
\pgfpathcurveto{\pgfqpoint{0.916667in}{0.219167in}}{\pgfqpoint{0.930590in}{0.219167in}}{\pgfqpoint{0.943945in}{0.213635in}}%
\pgfpathcurveto{\pgfqpoint{0.953790in}{0.203790in}}{\pgfqpoint{0.963635in}{0.193945in}}{\pgfqpoint{0.969167in}{0.180590in}}%
\pgfpathcurveto{\pgfqpoint{0.969167in}{0.166667in}}{\pgfqpoint{0.969167in}{0.152744in}}{\pgfqpoint{0.963635in}{0.139389in}}%
\pgfpathcurveto{\pgfqpoint{0.953790in}{0.129544in}}{\pgfqpoint{0.943945in}{0.119698in}}{\pgfqpoint{0.930590in}{0.114167in}}%
\pgfpathclose%
\pgfpathmoveto{\pgfqpoint{0.000000in}{0.275000in}}%
\pgfpathcurveto{\pgfqpoint{0.015470in}{0.275000in}}{\pgfqpoint{0.030309in}{0.281146in}}{\pgfqpoint{0.041248in}{0.292085in}}%
\pgfpathcurveto{\pgfqpoint{0.052187in}{0.303025in}}{\pgfqpoint{0.058333in}{0.317863in}}{\pgfqpoint{0.058333in}{0.333333in}}%
\pgfpathcurveto{\pgfqpoint{0.058333in}{0.348804in}}{\pgfqpoint{0.052187in}{0.363642in}}{\pgfqpoint{0.041248in}{0.374581in}}%
\pgfpathcurveto{\pgfqpoint{0.030309in}{0.385520in}}{\pgfqpoint{0.015470in}{0.391667in}}{\pgfqpoint{0.000000in}{0.391667in}}%
\pgfpathcurveto{\pgfqpoint{-0.015470in}{0.391667in}}{\pgfqpoint{-0.030309in}{0.385520in}}{\pgfqpoint{-0.041248in}{0.374581in}}%
\pgfpathcurveto{\pgfqpoint{-0.052187in}{0.363642in}}{\pgfqpoint{-0.058333in}{0.348804in}}{\pgfqpoint{-0.058333in}{0.333333in}}%
\pgfpathcurveto{\pgfqpoint{-0.058333in}{0.317863in}}{\pgfqpoint{-0.052187in}{0.303025in}}{\pgfqpoint{-0.041248in}{0.292085in}}%
\pgfpathcurveto{\pgfqpoint{-0.030309in}{0.281146in}}{\pgfqpoint{-0.015470in}{0.275000in}}{\pgfqpoint{0.000000in}{0.275000in}}%
\pgfpathclose%
\pgfpathmoveto{\pgfqpoint{0.000000in}{0.280833in}}%
\pgfpathcurveto{\pgfqpoint{0.000000in}{0.280833in}}{\pgfqpoint{-0.013923in}{0.280833in}}{\pgfqpoint{-0.027278in}{0.286365in}}%
\pgfpathcurveto{\pgfqpoint{-0.037123in}{0.296210in}}{\pgfqpoint{-0.046968in}{0.306055in}}{\pgfqpoint{-0.052500in}{0.319410in}}%
\pgfpathcurveto{\pgfqpoint{-0.052500in}{0.333333in}}{\pgfqpoint{-0.052500in}{0.347256in}}{\pgfqpoint{-0.046968in}{0.360611in}}%
\pgfpathcurveto{\pgfqpoint{-0.037123in}{0.370456in}}{\pgfqpoint{-0.027278in}{0.380302in}}{\pgfqpoint{-0.013923in}{0.385833in}}%
\pgfpathcurveto{\pgfqpoint{0.000000in}{0.385833in}}{\pgfqpoint{0.013923in}{0.385833in}}{\pgfqpoint{0.027278in}{0.380302in}}%
\pgfpathcurveto{\pgfqpoint{0.037123in}{0.370456in}}{\pgfqpoint{0.046968in}{0.360611in}}{\pgfqpoint{0.052500in}{0.347256in}}%
\pgfpathcurveto{\pgfqpoint{0.052500in}{0.333333in}}{\pgfqpoint{0.052500in}{0.319410in}}{\pgfqpoint{0.046968in}{0.306055in}}%
\pgfpathcurveto{\pgfqpoint{0.037123in}{0.296210in}}{\pgfqpoint{0.027278in}{0.286365in}}{\pgfqpoint{0.013923in}{0.280833in}}%
\pgfpathclose%
\pgfpathmoveto{\pgfqpoint{0.166667in}{0.275000in}}%
\pgfpathcurveto{\pgfqpoint{0.182137in}{0.275000in}}{\pgfqpoint{0.196975in}{0.281146in}}{\pgfqpoint{0.207915in}{0.292085in}}%
\pgfpathcurveto{\pgfqpoint{0.218854in}{0.303025in}}{\pgfqpoint{0.225000in}{0.317863in}}{\pgfqpoint{0.225000in}{0.333333in}}%
\pgfpathcurveto{\pgfqpoint{0.225000in}{0.348804in}}{\pgfqpoint{0.218854in}{0.363642in}}{\pgfqpoint{0.207915in}{0.374581in}}%
\pgfpathcurveto{\pgfqpoint{0.196975in}{0.385520in}}{\pgfqpoint{0.182137in}{0.391667in}}{\pgfqpoint{0.166667in}{0.391667in}}%
\pgfpathcurveto{\pgfqpoint{0.151196in}{0.391667in}}{\pgfqpoint{0.136358in}{0.385520in}}{\pgfqpoint{0.125419in}{0.374581in}}%
\pgfpathcurveto{\pgfqpoint{0.114480in}{0.363642in}}{\pgfqpoint{0.108333in}{0.348804in}}{\pgfqpoint{0.108333in}{0.333333in}}%
\pgfpathcurveto{\pgfqpoint{0.108333in}{0.317863in}}{\pgfqpoint{0.114480in}{0.303025in}}{\pgfqpoint{0.125419in}{0.292085in}}%
\pgfpathcurveto{\pgfqpoint{0.136358in}{0.281146in}}{\pgfqpoint{0.151196in}{0.275000in}}{\pgfqpoint{0.166667in}{0.275000in}}%
\pgfpathclose%
\pgfpathmoveto{\pgfqpoint{0.166667in}{0.280833in}}%
\pgfpathcurveto{\pgfqpoint{0.166667in}{0.280833in}}{\pgfqpoint{0.152744in}{0.280833in}}{\pgfqpoint{0.139389in}{0.286365in}}%
\pgfpathcurveto{\pgfqpoint{0.129544in}{0.296210in}}{\pgfqpoint{0.119698in}{0.306055in}}{\pgfqpoint{0.114167in}{0.319410in}}%
\pgfpathcurveto{\pgfqpoint{0.114167in}{0.333333in}}{\pgfqpoint{0.114167in}{0.347256in}}{\pgfqpoint{0.119698in}{0.360611in}}%
\pgfpathcurveto{\pgfqpoint{0.129544in}{0.370456in}}{\pgfqpoint{0.139389in}{0.380302in}}{\pgfqpoint{0.152744in}{0.385833in}}%
\pgfpathcurveto{\pgfqpoint{0.166667in}{0.385833in}}{\pgfqpoint{0.180590in}{0.385833in}}{\pgfqpoint{0.193945in}{0.380302in}}%
\pgfpathcurveto{\pgfqpoint{0.203790in}{0.370456in}}{\pgfqpoint{0.213635in}{0.360611in}}{\pgfqpoint{0.219167in}{0.347256in}}%
\pgfpathcurveto{\pgfqpoint{0.219167in}{0.333333in}}{\pgfqpoint{0.219167in}{0.319410in}}{\pgfqpoint{0.213635in}{0.306055in}}%
\pgfpathcurveto{\pgfqpoint{0.203790in}{0.296210in}}{\pgfqpoint{0.193945in}{0.286365in}}{\pgfqpoint{0.180590in}{0.280833in}}%
\pgfpathclose%
\pgfpathmoveto{\pgfqpoint{0.333333in}{0.275000in}}%
\pgfpathcurveto{\pgfqpoint{0.348804in}{0.275000in}}{\pgfqpoint{0.363642in}{0.281146in}}{\pgfqpoint{0.374581in}{0.292085in}}%
\pgfpathcurveto{\pgfqpoint{0.385520in}{0.303025in}}{\pgfqpoint{0.391667in}{0.317863in}}{\pgfqpoint{0.391667in}{0.333333in}}%
\pgfpathcurveto{\pgfqpoint{0.391667in}{0.348804in}}{\pgfqpoint{0.385520in}{0.363642in}}{\pgfqpoint{0.374581in}{0.374581in}}%
\pgfpathcurveto{\pgfqpoint{0.363642in}{0.385520in}}{\pgfqpoint{0.348804in}{0.391667in}}{\pgfqpoint{0.333333in}{0.391667in}}%
\pgfpathcurveto{\pgfqpoint{0.317863in}{0.391667in}}{\pgfqpoint{0.303025in}{0.385520in}}{\pgfqpoint{0.292085in}{0.374581in}}%
\pgfpathcurveto{\pgfqpoint{0.281146in}{0.363642in}}{\pgfqpoint{0.275000in}{0.348804in}}{\pgfqpoint{0.275000in}{0.333333in}}%
\pgfpathcurveto{\pgfqpoint{0.275000in}{0.317863in}}{\pgfqpoint{0.281146in}{0.303025in}}{\pgfqpoint{0.292085in}{0.292085in}}%
\pgfpathcurveto{\pgfqpoint{0.303025in}{0.281146in}}{\pgfqpoint{0.317863in}{0.275000in}}{\pgfqpoint{0.333333in}{0.275000in}}%
\pgfpathclose%
\pgfpathmoveto{\pgfqpoint{0.333333in}{0.280833in}}%
\pgfpathcurveto{\pgfqpoint{0.333333in}{0.280833in}}{\pgfqpoint{0.319410in}{0.280833in}}{\pgfqpoint{0.306055in}{0.286365in}}%
\pgfpathcurveto{\pgfqpoint{0.296210in}{0.296210in}}{\pgfqpoint{0.286365in}{0.306055in}}{\pgfqpoint{0.280833in}{0.319410in}}%
\pgfpathcurveto{\pgfqpoint{0.280833in}{0.333333in}}{\pgfqpoint{0.280833in}{0.347256in}}{\pgfqpoint{0.286365in}{0.360611in}}%
\pgfpathcurveto{\pgfqpoint{0.296210in}{0.370456in}}{\pgfqpoint{0.306055in}{0.380302in}}{\pgfqpoint{0.319410in}{0.385833in}}%
\pgfpathcurveto{\pgfqpoint{0.333333in}{0.385833in}}{\pgfqpoint{0.347256in}{0.385833in}}{\pgfqpoint{0.360611in}{0.380302in}}%
\pgfpathcurveto{\pgfqpoint{0.370456in}{0.370456in}}{\pgfqpoint{0.380302in}{0.360611in}}{\pgfqpoint{0.385833in}{0.347256in}}%
\pgfpathcurveto{\pgfqpoint{0.385833in}{0.333333in}}{\pgfqpoint{0.385833in}{0.319410in}}{\pgfqpoint{0.380302in}{0.306055in}}%
\pgfpathcurveto{\pgfqpoint{0.370456in}{0.296210in}}{\pgfqpoint{0.360611in}{0.286365in}}{\pgfqpoint{0.347256in}{0.280833in}}%
\pgfpathclose%
\pgfpathmoveto{\pgfqpoint{0.500000in}{0.275000in}}%
\pgfpathcurveto{\pgfqpoint{0.515470in}{0.275000in}}{\pgfqpoint{0.530309in}{0.281146in}}{\pgfqpoint{0.541248in}{0.292085in}}%
\pgfpathcurveto{\pgfqpoint{0.552187in}{0.303025in}}{\pgfqpoint{0.558333in}{0.317863in}}{\pgfqpoint{0.558333in}{0.333333in}}%
\pgfpathcurveto{\pgfqpoint{0.558333in}{0.348804in}}{\pgfqpoint{0.552187in}{0.363642in}}{\pgfqpoint{0.541248in}{0.374581in}}%
\pgfpathcurveto{\pgfqpoint{0.530309in}{0.385520in}}{\pgfqpoint{0.515470in}{0.391667in}}{\pgfqpoint{0.500000in}{0.391667in}}%
\pgfpathcurveto{\pgfqpoint{0.484530in}{0.391667in}}{\pgfqpoint{0.469691in}{0.385520in}}{\pgfqpoint{0.458752in}{0.374581in}}%
\pgfpathcurveto{\pgfqpoint{0.447813in}{0.363642in}}{\pgfqpoint{0.441667in}{0.348804in}}{\pgfqpoint{0.441667in}{0.333333in}}%
\pgfpathcurveto{\pgfqpoint{0.441667in}{0.317863in}}{\pgfqpoint{0.447813in}{0.303025in}}{\pgfqpoint{0.458752in}{0.292085in}}%
\pgfpathcurveto{\pgfqpoint{0.469691in}{0.281146in}}{\pgfqpoint{0.484530in}{0.275000in}}{\pgfqpoint{0.500000in}{0.275000in}}%
\pgfpathclose%
\pgfpathmoveto{\pgfqpoint{0.500000in}{0.280833in}}%
\pgfpathcurveto{\pgfqpoint{0.500000in}{0.280833in}}{\pgfqpoint{0.486077in}{0.280833in}}{\pgfqpoint{0.472722in}{0.286365in}}%
\pgfpathcurveto{\pgfqpoint{0.462877in}{0.296210in}}{\pgfqpoint{0.453032in}{0.306055in}}{\pgfqpoint{0.447500in}{0.319410in}}%
\pgfpathcurveto{\pgfqpoint{0.447500in}{0.333333in}}{\pgfqpoint{0.447500in}{0.347256in}}{\pgfqpoint{0.453032in}{0.360611in}}%
\pgfpathcurveto{\pgfqpoint{0.462877in}{0.370456in}}{\pgfqpoint{0.472722in}{0.380302in}}{\pgfqpoint{0.486077in}{0.385833in}}%
\pgfpathcurveto{\pgfqpoint{0.500000in}{0.385833in}}{\pgfqpoint{0.513923in}{0.385833in}}{\pgfqpoint{0.527278in}{0.380302in}}%
\pgfpathcurveto{\pgfqpoint{0.537123in}{0.370456in}}{\pgfqpoint{0.546968in}{0.360611in}}{\pgfqpoint{0.552500in}{0.347256in}}%
\pgfpathcurveto{\pgfqpoint{0.552500in}{0.333333in}}{\pgfqpoint{0.552500in}{0.319410in}}{\pgfqpoint{0.546968in}{0.306055in}}%
\pgfpathcurveto{\pgfqpoint{0.537123in}{0.296210in}}{\pgfqpoint{0.527278in}{0.286365in}}{\pgfqpoint{0.513923in}{0.280833in}}%
\pgfpathclose%
\pgfpathmoveto{\pgfqpoint{0.666667in}{0.275000in}}%
\pgfpathcurveto{\pgfqpoint{0.682137in}{0.275000in}}{\pgfqpoint{0.696975in}{0.281146in}}{\pgfqpoint{0.707915in}{0.292085in}}%
\pgfpathcurveto{\pgfqpoint{0.718854in}{0.303025in}}{\pgfqpoint{0.725000in}{0.317863in}}{\pgfqpoint{0.725000in}{0.333333in}}%
\pgfpathcurveto{\pgfqpoint{0.725000in}{0.348804in}}{\pgfqpoint{0.718854in}{0.363642in}}{\pgfqpoint{0.707915in}{0.374581in}}%
\pgfpathcurveto{\pgfqpoint{0.696975in}{0.385520in}}{\pgfqpoint{0.682137in}{0.391667in}}{\pgfqpoint{0.666667in}{0.391667in}}%
\pgfpathcurveto{\pgfqpoint{0.651196in}{0.391667in}}{\pgfqpoint{0.636358in}{0.385520in}}{\pgfqpoint{0.625419in}{0.374581in}}%
\pgfpathcurveto{\pgfqpoint{0.614480in}{0.363642in}}{\pgfqpoint{0.608333in}{0.348804in}}{\pgfqpoint{0.608333in}{0.333333in}}%
\pgfpathcurveto{\pgfqpoint{0.608333in}{0.317863in}}{\pgfqpoint{0.614480in}{0.303025in}}{\pgfqpoint{0.625419in}{0.292085in}}%
\pgfpathcurveto{\pgfqpoint{0.636358in}{0.281146in}}{\pgfqpoint{0.651196in}{0.275000in}}{\pgfqpoint{0.666667in}{0.275000in}}%
\pgfpathclose%
\pgfpathmoveto{\pgfqpoint{0.666667in}{0.280833in}}%
\pgfpathcurveto{\pgfqpoint{0.666667in}{0.280833in}}{\pgfqpoint{0.652744in}{0.280833in}}{\pgfqpoint{0.639389in}{0.286365in}}%
\pgfpathcurveto{\pgfqpoint{0.629544in}{0.296210in}}{\pgfqpoint{0.619698in}{0.306055in}}{\pgfqpoint{0.614167in}{0.319410in}}%
\pgfpathcurveto{\pgfqpoint{0.614167in}{0.333333in}}{\pgfqpoint{0.614167in}{0.347256in}}{\pgfqpoint{0.619698in}{0.360611in}}%
\pgfpathcurveto{\pgfqpoint{0.629544in}{0.370456in}}{\pgfqpoint{0.639389in}{0.380302in}}{\pgfqpoint{0.652744in}{0.385833in}}%
\pgfpathcurveto{\pgfqpoint{0.666667in}{0.385833in}}{\pgfqpoint{0.680590in}{0.385833in}}{\pgfqpoint{0.693945in}{0.380302in}}%
\pgfpathcurveto{\pgfqpoint{0.703790in}{0.370456in}}{\pgfqpoint{0.713635in}{0.360611in}}{\pgfqpoint{0.719167in}{0.347256in}}%
\pgfpathcurveto{\pgfqpoint{0.719167in}{0.333333in}}{\pgfqpoint{0.719167in}{0.319410in}}{\pgfqpoint{0.713635in}{0.306055in}}%
\pgfpathcurveto{\pgfqpoint{0.703790in}{0.296210in}}{\pgfqpoint{0.693945in}{0.286365in}}{\pgfqpoint{0.680590in}{0.280833in}}%
\pgfpathclose%
\pgfpathmoveto{\pgfqpoint{0.833333in}{0.275000in}}%
\pgfpathcurveto{\pgfqpoint{0.848804in}{0.275000in}}{\pgfqpoint{0.863642in}{0.281146in}}{\pgfqpoint{0.874581in}{0.292085in}}%
\pgfpathcurveto{\pgfqpoint{0.885520in}{0.303025in}}{\pgfqpoint{0.891667in}{0.317863in}}{\pgfqpoint{0.891667in}{0.333333in}}%
\pgfpathcurveto{\pgfqpoint{0.891667in}{0.348804in}}{\pgfqpoint{0.885520in}{0.363642in}}{\pgfqpoint{0.874581in}{0.374581in}}%
\pgfpathcurveto{\pgfqpoint{0.863642in}{0.385520in}}{\pgfqpoint{0.848804in}{0.391667in}}{\pgfqpoint{0.833333in}{0.391667in}}%
\pgfpathcurveto{\pgfqpoint{0.817863in}{0.391667in}}{\pgfqpoint{0.803025in}{0.385520in}}{\pgfqpoint{0.792085in}{0.374581in}}%
\pgfpathcurveto{\pgfqpoint{0.781146in}{0.363642in}}{\pgfqpoint{0.775000in}{0.348804in}}{\pgfqpoint{0.775000in}{0.333333in}}%
\pgfpathcurveto{\pgfqpoint{0.775000in}{0.317863in}}{\pgfqpoint{0.781146in}{0.303025in}}{\pgfqpoint{0.792085in}{0.292085in}}%
\pgfpathcurveto{\pgfqpoint{0.803025in}{0.281146in}}{\pgfqpoint{0.817863in}{0.275000in}}{\pgfqpoint{0.833333in}{0.275000in}}%
\pgfpathclose%
\pgfpathmoveto{\pgfqpoint{0.833333in}{0.280833in}}%
\pgfpathcurveto{\pgfqpoint{0.833333in}{0.280833in}}{\pgfqpoint{0.819410in}{0.280833in}}{\pgfqpoint{0.806055in}{0.286365in}}%
\pgfpathcurveto{\pgfqpoint{0.796210in}{0.296210in}}{\pgfqpoint{0.786365in}{0.306055in}}{\pgfqpoint{0.780833in}{0.319410in}}%
\pgfpathcurveto{\pgfqpoint{0.780833in}{0.333333in}}{\pgfqpoint{0.780833in}{0.347256in}}{\pgfqpoint{0.786365in}{0.360611in}}%
\pgfpathcurveto{\pgfqpoint{0.796210in}{0.370456in}}{\pgfqpoint{0.806055in}{0.380302in}}{\pgfqpoint{0.819410in}{0.385833in}}%
\pgfpathcurveto{\pgfqpoint{0.833333in}{0.385833in}}{\pgfqpoint{0.847256in}{0.385833in}}{\pgfqpoint{0.860611in}{0.380302in}}%
\pgfpathcurveto{\pgfqpoint{0.870456in}{0.370456in}}{\pgfqpoint{0.880302in}{0.360611in}}{\pgfqpoint{0.885833in}{0.347256in}}%
\pgfpathcurveto{\pgfqpoint{0.885833in}{0.333333in}}{\pgfqpoint{0.885833in}{0.319410in}}{\pgfqpoint{0.880302in}{0.306055in}}%
\pgfpathcurveto{\pgfqpoint{0.870456in}{0.296210in}}{\pgfqpoint{0.860611in}{0.286365in}}{\pgfqpoint{0.847256in}{0.280833in}}%
\pgfpathclose%
\pgfpathmoveto{\pgfqpoint{1.000000in}{0.275000in}}%
\pgfpathcurveto{\pgfqpoint{1.015470in}{0.275000in}}{\pgfqpoint{1.030309in}{0.281146in}}{\pgfqpoint{1.041248in}{0.292085in}}%
\pgfpathcurveto{\pgfqpoint{1.052187in}{0.303025in}}{\pgfqpoint{1.058333in}{0.317863in}}{\pgfqpoint{1.058333in}{0.333333in}}%
\pgfpathcurveto{\pgfqpoint{1.058333in}{0.348804in}}{\pgfqpoint{1.052187in}{0.363642in}}{\pgfqpoint{1.041248in}{0.374581in}}%
\pgfpathcurveto{\pgfqpoint{1.030309in}{0.385520in}}{\pgfqpoint{1.015470in}{0.391667in}}{\pgfqpoint{1.000000in}{0.391667in}}%
\pgfpathcurveto{\pgfqpoint{0.984530in}{0.391667in}}{\pgfqpoint{0.969691in}{0.385520in}}{\pgfqpoint{0.958752in}{0.374581in}}%
\pgfpathcurveto{\pgfqpoint{0.947813in}{0.363642in}}{\pgfqpoint{0.941667in}{0.348804in}}{\pgfqpoint{0.941667in}{0.333333in}}%
\pgfpathcurveto{\pgfqpoint{0.941667in}{0.317863in}}{\pgfqpoint{0.947813in}{0.303025in}}{\pgfqpoint{0.958752in}{0.292085in}}%
\pgfpathcurveto{\pgfqpoint{0.969691in}{0.281146in}}{\pgfqpoint{0.984530in}{0.275000in}}{\pgfqpoint{1.000000in}{0.275000in}}%
\pgfpathclose%
\pgfpathmoveto{\pgfqpoint{1.000000in}{0.280833in}}%
\pgfpathcurveto{\pgfqpoint{1.000000in}{0.280833in}}{\pgfqpoint{0.986077in}{0.280833in}}{\pgfqpoint{0.972722in}{0.286365in}}%
\pgfpathcurveto{\pgfqpoint{0.962877in}{0.296210in}}{\pgfqpoint{0.953032in}{0.306055in}}{\pgfqpoint{0.947500in}{0.319410in}}%
\pgfpathcurveto{\pgfqpoint{0.947500in}{0.333333in}}{\pgfqpoint{0.947500in}{0.347256in}}{\pgfqpoint{0.953032in}{0.360611in}}%
\pgfpathcurveto{\pgfqpoint{0.962877in}{0.370456in}}{\pgfqpoint{0.972722in}{0.380302in}}{\pgfqpoint{0.986077in}{0.385833in}}%
\pgfpathcurveto{\pgfqpoint{1.000000in}{0.385833in}}{\pgfqpoint{1.013923in}{0.385833in}}{\pgfqpoint{1.027278in}{0.380302in}}%
\pgfpathcurveto{\pgfqpoint{1.037123in}{0.370456in}}{\pgfqpoint{1.046968in}{0.360611in}}{\pgfqpoint{1.052500in}{0.347256in}}%
\pgfpathcurveto{\pgfqpoint{1.052500in}{0.333333in}}{\pgfqpoint{1.052500in}{0.319410in}}{\pgfqpoint{1.046968in}{0.306055in}}%
\pgfpathcurveto{\pgfqpoint{1.037123in}{0.296210in}}{\pgfqpoint{1.027278in}{0.286365in}}{\pgfqpoint{1.013923in}{0.280833in}}%
\pgfpathclose%
\pgfpathmoveto{\pgfqpoint{0.083333in}{0.441667in}}%
\pgfpathcurveto{\pgfqpoint{0.098804in}{0.441667in}}{\pgfqpoint{0.113642in}{0.447813in}}{\pgfqpoint{0.124581in}{0.458752in}}%
\pgfpathcurveto{\pgfqpoint{0.135520in}{0.469691in}}{\pgfqpoint{0.141667in}{0.484530in}}{\pgfqpoint{0.141667in}{0.500000in}}%
\pgfpathcurveto{\pgfqpoint{0.141667in}{0.515470in}}{\pgfqpoint{0.135520in}{0.530309in}}{\pgfqpoint{0.124581in}{0.541248in}}%
\pgfpathcurveto{\pgfqpoint{0.113642in}{0.552187in}}{\pgfqpoint{0.098804in}{0.558333in}}{\pgfqpoint{0.083333in}{0.558333in}}%
\pgfpathcurveto{\pgfqpoint{0.067863in}{0.558333in}}{\pgfqpoint{0.053025in}{0.552187in}}{\pgfqpoint{0.042085in}{0.541248in}}%
\pgfpathcurveto{\pgfqpoint{0.031146in}{0.530309in}}{\pgfqpoint{0.025000in}{0.515470in}}{\pgfqpoint{0.025000in}{0.500000in}}%
\pgfpathcurveto{\pgfqpoint{0.025000in}{0.484530in}}{\pgfqpoint{0.031146in}{0.469691in}}{\pgfqpoint{0.042085in}{0.458752in}}%
\pgfpathcurveto{\pgfqpoint{0.053025in}{0.447813in}}{\pgfqpoint{0.067863in}{0.441667in}}{\pgfqpoint{0.083333in}{0.441667in}}%
\pgfpathclose%
\pgfpathmoveto{\pgfqpoint{0.083333in}{0.447500in}}%
\pgfpathcurveto{\pgfqpoint{0.083333in}{0.447500in}}{\pgfqpoint{0.069410in}{0.447500in}}{\pgfqpoint{0.056055in}{0.453032in}}%
\pgfpathcurveto{\pgfqpoint{0.046210in}{0.462877in}}{\pgfqpoint{0.036365in}{0.472722in}}{\pgfqpoint{0.030833in}{0.486077in}}%
\pgfpathcurveto{\pgfqpoint{0.030833in}{0.500000in}}{\pgfqpoint{0.030833in}{0.513923in}}{\pgfqpoint{0.036365in}{0.527278in}}%
\pgfpathcurveto{\pgfqpoint{0.046210in}{0.537123in}}{\pgfqpoint{0.056055in}{0.546968in}}{\pgfqpoint{0.069410in}{0.552500in}}%
\pgfpathcurveto{\pgfqpoint{0.083333in}{0.552500in}}{\pgfqpoint{0.097256in}{0.552500in}}{\pgfqpoint{0.110611in}{0.546968in}}%
\pgfpathcurveto{\pgfqpoint{0.120456in}{0.537123in}}{\pgfqpoint{0.130302in}{0.527278in}}{\pgfqpoint{0.135833in}{0.513923in}}%
\pgfpathcurveto{\pgfqpoint{0.135833in}{0.500000in}}{\pgfqpoint{0.135833in}{0.486077in}}{\pgfqpoint{0.130302in}{0.472722in}}%
\pgfpathcurveto{\pgfqpoint{0.120456in}{0.462877in}}{\pgfqpoint{0.110611in}{0.453032in}}{\pgfqpoint{0.097256in}{0.447500in}}%
\pgfpathclose%
\pgfpathmoveto{\pgfqpoint{0.250000in}{0.441667in}}%
\pgfpathcurveto{\pgfqpoint{0.265470in}{0.441667in}}{\pgfqpoint{0.280309in}{0.447813in}}{\pgfqpoint{0.291248in}{0.458752in}}%
\pgfpathcurveto{\pgfqpoint{0.302187in}{0.469691in}}{\pgfqpoint{0.308333in}{0.484530in}}{\pgfqpoint{0.308333in}{0.500000in}}%
\pgfpathcurveto{\pgfqpoint{0.308333in}{0.515470in}}{\pgfqpoint{0.302187in}{0.530309in}}{\pgfqpoint{0.291248in}{0.541248in}}%
\pgfpathcurveto{\pgfqpoint{0.280309in}{0.552187in}}{\pgfqpoint{0.265470in}{0.558333in}}{\pgfqpoint{0.250000in}{0.558333in}}%
\pgfpathcurveto{\pgfqpoint{0.234530in}{0.558333in}}{\pgfqpoint{0.219691in}{0.552187in}}{\pgfqpoint{0.208752in}{0.541248in}}%
\pgfpathcurveto{\pgfqpoint{0.197813in}{0.530309in}}{\pgfqpoint{0.191667in}{0.515470in}}{\pgfqpoint{0.191667in}{0.500000in}}%
\pgfpathcurveto{\pgfqpoint{0.191667in}{0.484530in}}{\pgfqpoint{0.197813in}{0.469691in}}{\pgfqpoint{0.208752in}{0.458752in}}%
\pgfpathcurveto{\pgfqpoint{0.219691in}{0.447813in}}{\pgfqpoint{0.234530in}{0.441667in}}{\pgfqpoint{0.250000in}{0.441667in}}%
\pgfpathclose%
\pgfpathmoveto{\pgfqpoint{0.250000in}{0.447500in}}%
\pgfpathcurveto{\pgfqpoint{0.250000in}{0.447500in}}{\pgfqpoint{0.236077in}{0.447500in}}{\pgfqpoint{0.222722in}{0.453032in}}%
\pgfpathcurveto{\pgfqpoint{0.212877in}{0.462877in}}{\pgfqpoint{0.203032in}{0.472722in}}{\pgfqpoint{0.197500in}{0.486077in}}%
\pgfpathcurveto{\pgfqpoint{0.197500in}{0.500000in}}{\pgfqpoint{0.197500in}{0.513923in}}{\pgfqpoint{0.203032in}{0.527278in}}%
\pgfpathcurveto{\pgfqpoint{0.212877in}{0.537123in}}{\pgfqpoint{0.222722in}{0.546968in}}{\pgfqpoint{0.236077in}{0.552500in}}%
\pgfpathcurveto{\pgfqpoint{0.250000in}{0.552500in}}{\pgfqpoint{0.263923in}{0.552500in}}{\pgfqpoint{0.277278in}{0.546968in}}%
\pgfpathcurveto{\pgfqpoint{0.287123in}{0.537123in}}{\pgfqpoint{0.296968in}{0.527278in}}{\pgfqpoint{0.302500in}{0.513923in}}%
\pgfpathcurveto{\pgfqpoint{0.302500in}{0.500000in}}{\pgfqpoint{0.302500in}{0.486077in}}{\pgfqpoint{0.296968in}{0.472722in}}%
\pgfpathcurveto{\pgfqpoint{0.287123in}{0.462877in}}{\pgfqpoint{0.277278in}{0.453032in}}{\pgfqpoint{0.263923in}{0.447500in}}%
\pgfpathclose%
\pgfpathmoveto{\pgfqpoint{0.416667in}{0.441667in}}%
\pgfpathcurveto{\pgfqpoint{0.432137in}{0.441667in}}{\pgfqpoint{0.446975in}{0.447813in}}{\pgfqpoint{0.457915in}{0.458752in}}%
\pgfpathcurveto{\pgfqpoint{0.468854in}{0.469691in}}{\pgfqpoint{0.475000in}{0.484530in}}{\pgfqpoint{0.475000in}{0.500000in}}%
\pgfpathcurveto{\pgfqpoint{0.475000in}{0.515470in}}{\pgfqpoint{0.468854in}{0.530309in}}{\pgfqpoint{0.457915in}{0.541248in}}%
\pgfpathcurveto{\pgfqpoint{0.446975in}{0.552187in}}{\pgfqpoint{0.432137in}{0.558333in}}{\pgfqpoint{0.416667in}{0.558333in}}%
\pgfpathcurveto{\pgfqpoint{0.401196in}{0.558333in}}{\pgfqpoint{0.386358in}{0.552187in}}{\pgfqpoint{0.375419in}{0.541248in}}%
\pgfpathcurveto{\pgfqpoint{0.364480in}{0.530309in}}{\pgfqpoint{0.358333in}{0.515470in}}{\pgfqpoint{0.358333in}{0.500000in}}%
\pgfpathcurveto{\pgfqpoint{0.358333in}{0.484530in}}{\pgfqpoint{0.364480in}{0.469691in}}{\pgfqpoint{0.375419in}{0.458752in}}%
\pgfpathcurveto{\pgfqpoint{0.386358in}{0.447813in}}{\pgfqpoint{0.401196in}{0.441667in}}{\pgfqpoint{0.416667in}{0.441667in}}%
\pgfpathclose%
\pgfpathmoveto{\pgfqpoint{0.416667in}{0.447500in}}%
\pgfpathcurveto{\pgfqpoint{0.416667in}{0.447500in}}{\pgfqpoint{0.402744in}{0.447500in}}{\pgfqpoint{0.389389in}{0.453032in}}%
\pgfpathcurveto{\pgfqpoint{0.379544in}{0.462877in}}{\pgfqpoint{0.369698in}{0.472722in}}{\pgfqpoint{0.364167in}{0.486077in}}%
\pgfpathcurveto{\pgfqpoint{0.364167in}{0.500000in}}{\pgfqpoint{0.364167in}{0.513923in}}{\pgfqpoint{0.369698in}{0.527278in}}%
\pgfpathcurveto{\pgfqpoint{0.379544in}{0.537123in}}{\pgfqpoint{0.389389in}{0.546968in}}{\pgfqpoint{0.402744in}{0.552500in}}%
\pgfpathcurveto{\pgfqpoint{0.416667in}{0.552500in}}{\pgfqpoint{0.430590in}{0.552500in}}{\pgfqpoint{0.443945in}{0.546968in}}%
\pgfpathcurveto{\pgfqpoint{0.453790in}{0.537123in}}{\pgfqpoint{0.463635in}{0.527278in}}{\pgfqpoint{0.469167in}{0.513923in}}%
\pgfpathcurveto{\pgfqpoint{0.469167in}{0.500000in}}{\pgfqpoint{0.469167in}{0.486077in}}{\pgfqpoint{0.463635in}{0.472722in}}%
\pgfpathcurveto{\pgfqpoint{0.453790in}{0.462877in}}{\pgfqpoint{0.443945in}{0.453032in}}{\pgfqpoint{0.430590in}{0.447500in}}%
\pgfpathclose%
\pgfpathmoveto{\pgfqpoint{0.583333in}{0.441667in}}%
\pgfpathcurveto{\pgfqpoint{0.598804in}{0.441667in}}{\pgfqpoint{0.613642in}{0.447813in}}{\pgfqpoint{0.624581in}{0.458752in}}%
\pgfpathcurveto{\pgfqpoint{0.635520in}{0.469691in}}{\pgfqpoint{0.641667in}{0.484530in}}{\pgfqpoint{0.641667in}{0.500000in}}%
\pgfpathcurveto{\pgfqpoint{0.641667in}{0.515470in}}{\pgfqpoint{0.635520in}{0.530309in}}{\pgfqpoint{0.624581in}{0.541248in}}%
\pgfpathcurveto{\pgfqpoint{0.613642in}{0.552187in}}{\pgfqpoint{0.598804in}{0.558333in}}{\pgfqpoint{0.583333in}{0.558333in}}%
\pgfpathcurveto{\pgfqpoint{0.567863in}{0.558333in}}{\pgfqpoint{0.553025in}{0.552187in}}{\pgfqpoint{0.542085in}{0.541248in}}%
\pgfpathcurveto{\pgfqpoint{0.531146in}{0.530309in}}{\pgfqpoint{0.525000in}{0.515470in}}{\pgfqpoint{0.525000in}{0.500000in}}%
\pgfpathcurveto{\pgfqpoint{0.525000in}{0.484530in}}{\pgfqpoint{0.531146in}{0.469691in}}{\pgfqpoint{0.542085in}{0.458752in}}%
\pgfpathcurveto{\pgfqpoint{0.553025in}{0.447813in}}{\pgfqpoint{0.567863in}{0.441667in}}{\pgfqpoint{0.583333in}{0.441667in}}%
\pgfpathclose%
\pgfpathmoveto{\pgfqpoint{0.583333in}{0.447500in}}%
\pgfpathcurveto{\pgfqpoint{0.583333in}{0.447500in}}{\pgfqpoint{0.569410in}{0.447500in}}{\pgfqpoint{0.556055in}{0.453032in}}%
\pgfpathcurveto{\pgfqpoint{0.546210in}{0.462877in}}{\pgfqpoint{0.536365in}{0.472722in}}{\pgfqpoint{0.530833in}{0.486077in}}%
\pgfpathcurveto{\pgfqpoint{0.530833in}{0.500000in}}{\pgfqpoint{0.530833in}{0.513923in}}{\pgfqpoint{0.536365in}{0.527278in}}%
\pgfpathcurveto{\pgfqpoint{0.546210in}{0.537123in}}{\pgfqpoint{0.556055in}{0.546968in}}{\pgfqpoint{0.569410in}{0.552500in}}%
\pgfpathcurveto{\pgfqpoint{0.583333in}{0.552500in}}{\pgfqpoint{0.597256in}{0.552500in}}{\pgfqpoint{0.610611in}{0.546968in}}%
\pgfpathcurveto{\pgfqpoint{0.620456in}{0.537123in}}{\pgfqpoint{0.630302in}{0.527278in}}{\pgfqpoint{0.635833in}{0.513923in}}%
\pgfpathcurveto{\pgfqpoint{0.635833in}{0.500000in}}{\pgfqpoint{0.635833in}{0.486077in}}{\pgfqpoint{0.630302in}{0.472722in}}%
\pgfpathcurveto{\pgfqpoint{0.620456in}{0.462877in}}{\pgfqpoint{0.610611in}{0.453032in}}{\pgfqpoint{0.597256in}{0.447500in}}%
\pgfpathclose%
\pgfpathmoveto{\pgfqpoint{0.750000in}{0.441667in}}%
\pgfpathcurveto{\pgfqpoint{0.765470in}{0.441667in}}{\pgfqpoint{0.780309in}{0.447813in}}{\pgfqpoint{0.791248in}{0.458752in}}%
\pgfpathcurveto{\pgfqpoint{0.802187in}{0.469691in}}{\pgfqpoint{0.808333in}{0.484530in}}{\pgfqpoint{0.808333in}{0.500000in}}%
\pgfpathcurveto{\pgfqpoint{0.808333in}{0.515470in}}{\pgfqpoint{0.802187in}{0.530309in}}{\pgfqpoint{0.791248in}{0.541248in}}%
\pgfpathcurveto{\pgfqpoint{0.780309in}{0.552187in}}{\pgfqpoint{0.765470in}{0.558333in}}{\pgfqpoint{0.750000in}{0.558333in}}%
\pgfpathcurveto{\pgfqpoint{0.734530in}{0.558333in}}{\pgfqpoint{0.719691in}{0.552187in}}{\pgfqpoint{0.708752in}{0.541248in}}%
\pgfpathcurveto{\pgfqpoint{0.697813in}{0.530309in}}{\pgfqpoint{0.691667in}{0.515470in}}{\pgfqpoint{0.691667in}{0.500000in}}%
\pgfpathcurveto{\pgfqpoint{0.691667in}{0.484530in}}{\pgfqpoint{0.697813in}{0.469691in}}{\pgfqpoint{0.708752in}{0.458752in}}%
\pgfpathcurveto{\pgfqpoint{0.719691in}{0.447813in}}{\pgfqpoint{0.734530in}{0.441667in}}{\pgfqpoint{0.750000in}{0.441667in}}%
\pgfpathclose%
\pgfpathmoveto{\pgfqpoint{0.750000in}{0.447500in}}%
\pgfpathcurveto{\pgfqpoint{0.750000in}{0.447500in}}{\pgfqpoint{0.736077in}{0.447500in}}{\pgfqpoint{0.722722in}{0.453032in}}%
\pgfpathcurveto{\pgfqpoint{0.712877in}{0.462877in}}{\pgfqpoint{0.703032in}{0.472722in}}{\pgfqpoint{0.697500in}{0.486077in}}%
\pgfpathcurveto{\pgfqpoint{0.697500in}{0.500000in}}{\pgfqpoint{0.697500in}{0.513923in}}{\pgfqpoint{0.703032in}{0.527278in}}%
\pgfpathcurveto{\pgfqpoint{0.712877in}{0.537123in}}{\pgfqpoint{0.722722in}{0.546968in}}{\pgfqpoint{0.736077in}{0.552500in}}%
\pgfpathcurveto{\pgfqpoint{0.750000in}{0.552500in}}{\pgfqpoint{0.763923in}{0.552500in}}{\pgfqpoint{0.777278in}{0.546968in}}%
\pgfpathcurveto{\pgfqpoint{0.787123in}{0.537123in}}{\pgfqpoint{0.796968in}{0.527278in}}{\pgfqpoint{0.802500in}{0.513923in}}%
\pgfpathcurveto{\pgfqpoint{0.802500in}{0.500000in}}{\pgfqpoint{0.802500in}{0.486077in}}{\pgfqpoint{0.796968in}{0.472722in}}%
\pgfpathcurveto{\pgfqpoint{0.787123in}{0.462877in}}{\pgfqpoint{0.777278in}{0.453032in}}{\pgfqpoint{0.763923in}{0.447500in}}%
\pgfpathclose%
\pgfpathmoveto{\pgfqpoint{0.916667in}{0.441667in}}%
\pgfpathcurveto{\pgfqpoint{0.932137in}{0.441667in}}{\pgfqpoint{0.946975in}{0.447813in}}{\pgfqpoint{0.957915in}{0.458752in}}%
\pgfpathcurveto{\pgfqpoint{0.968854in}{0.469691in}}{\pgfqpoint{0.975000in}{0.484530in}}{\pgfqpoint{0.975000in}{0.500000in}}%
\pgfpathcurveto{\pgfqpoint{0.975000in}{0.515470in}}{\pgfqpoint{0.968854in}{0.530309in}}{\pgfqpoint{0.957915in}{0.541248in}}%
\pgfpathcurveto{\pgfqpoint{0.946975in}{0.552187in}}{\pgfqpoint{0.932137in}{0.558333in}}{\pgfqpoint{0.916667in}{0.558333in}}%
\pgfpathcurveto{\pgfqpoint{0.901196in}{0.558333in}}{\pgfqpoint{0.886358in}{0.552187in}}{\pgfqpoint{0.875419in}{0.541248in}}%
\pgfpathcurveto{\pgfqpoint{0.864480in}{0.530309in}}{\pgfqpoint{0.858333in}{0.515470in}}{\pgfqpoint{0.858333in}{0.500000in}}%
\pgfpathcurveto{\pgfqpoint{0.858333in}{0.484530in}}{\pgfqpoint{0.864480in}{0.469691in}}{\pgfqpoint{0.875419in}{0.458752in}}%
\pgfpathcurveto{\pgfqpoint{0.886358in}{0.447813in}}{\pgfqpoint{0.901196in}{0.441667in}}{\pgfqpoint{0.916667in}{0.441667in}}%
\pgfpathclose%
\pgfpathmoveto{\pgfqpoint{0.916667in}{0.447500in}}%
\pgfpathcurveto{\pgfqpoint{0.916667in}{0.447500in}}{\pgfqpoint{0.902744in}{0.447500in}}{\pgfqpoint{0.889389in}{0.453032in}}%
\pgfpathcurveto{\pgfqpoint{0.879544in}{0.462877in}}{\pgfqpoint{0.869698in}{0.472722in}}{\pgfqpoint{0.864167in}{0.486077in}}%
\pgfpathcurveto{\pgfqpoint{0.864167in}{0.500000in}}{\pgfqpoint{0.864167in}{0.513923in}}{\pgfqpoint{0.869698in}{0.527278in}}%
\pgfpathcurveto{\pgfqpoint{0.879544in}{0.537123in}}{\pgfqpoint{0.889389in}{0.546968in}}{\pgfqpoint{0.902744in}{0.552500in}}%
\pgfpathcurveto{\pgfqpoint{0.916667in}{0.552500in}}{\pgfqpoint{0.930590in}{0.552500in}}{\pgfqpoint{0.943945in}{0.546968in}}%
\pgfpathcurveto{\pgfqpoint{0.953790in}{0.537123in}}{\pgfqpoint{0.963635in}{0.527278in}}{\pgfqpoint{0.969167in}{0.513923in}}%
\pgfpathcurveto{\pgfqpoint{0.969167in}{0.500000in}}{\pgfqpoint{0.969167in}{0.486077in}}{\pgfqpoint{0.963635in}{0.472722in}}%
\pgfpathcurveto{\pgfqpoint{0.953790in}{0.462877in}}{\pgfqpoint{0.943945in}{0.453032in}}{\pgfqpoint{0.930590in}{0.447500in}}%
\pgfpathclose%
\pgfpathmoveto{\pgfqpoint{0.000000in}{0.608333in}}%
\pgfpathcurveto{\pgfqpoint{0.015470in}{0.608333in}}{\pgfqpoint{0.030309in}{0.614480in}}{\pgfqpoint{0.041248in}{0.625419in}}%
\pgfpathcurveto{\pgfqpoint{0.052187in}{0.636358in}}{\pgfqpoint{0.058333in}{0.651196in}}{\pgfqpoint{0.058333in}{0.666667in}}%
\pgfpathcurveto{\pgfqpoint{0.058333in}{0.682137in}}{\pgfqpoint{0.052187in}{0.696975in}}{\pgfqpoint{0.041248in}{0.707915in}}%
\pgfpathcurveto{\pgfqpoint{0.030309in}{0.718854in}}{\pgfqpoint{0.015470in}{0.725000in}}{\pgfqpoint{0.000000in}{0.725000in}}%
\pgfpathcurveto{\pgfqpoint{-0.015470in}{0.725000in}}{\pgfqpoint{-0.030309in}{0.718854in}}{\pgfqpoint{-0.041248in}{0.707915in}}%
\pgfpathcurveto{\pgfqpoint{-0.052187in}{0.696975in}}{\pgfqpoint{-0.058333in}{0.682137in}}{\pgfqpoint{-0.058333in}{0.666667in}}%
\pgfpathcurveto{\pgfqpoint{-0.058333in}{0.651196in}}{\pgfqpoint{-0.052187in}{0.636358in}}{\pgfqpoint{-0.041248in}{0.625419in}}%
\pgfpathcurveto{\pgfqpoint{-0.030309in}{0.614480in}}{\pgfqpoint{-0.015470in}{0.608333in}}{\pgfqpoint{0.000000in}{0.608333in}}%
\pgfpathclose%
\pgfpathmoveto{\pgfqpoint{0.000000in}{0.614167in}}%
\pgfpathcurveto{\pgfqpoint{0.000000in}{0.614167in}}{\pgfqpoint{-0.013923in}{0.614167in}}{\pgfqpoint{-0.027278in}{0.619698in}}%
\pgfpathcurveto{\pgfqpoint{-0.037123in}{0.629544in}}{\pgfqpoint{-0.046968in}{0.639389in}}{\pgfqpoint{-0.052500in}{0.652744in}}%
\pgfpathcurveto{\pgfqpoint{-0.052500in}{0.666667in}}{\pgfqpoint{-0.052500in}{0.680590in}}{\pgfqpoint{-0.046968in}{0.693945in}}%
\pgfpathcurveto{\pgfqpoint{-0.037123in}{0.703790in}}{\pgfqpoint{-0.027278in}{0.713635in}}{\pgfqpoint{-0.013923in}{0.719167in}}%
\pgfpathcurveto{\pgfqpoint{0.000000in}{0.719167in}}{\pgfqpoint{0.013923in}{0.719167in}}{\pgfqpoint{0.027278in}{0.713635in}}%
\pgfpathcurveto{\pgfqpoint{0.037123in}{0.703790in}}{\pgfqpoint{0.046968in}{0.693945in}}{\pgfqpoint{0.052500in}{0.680590in}}%
\pgfpathcurveto{\pgfqpoint{0.052500in}{0.666667in}}{\pgfqpoint{0.052500in}{0.652744in}}{\pgfqpoint{0.046968in}{0.639389in}}%
\pgfpathcurveto{\pgfqpoint{0.037123in}{0.629544in}}{\pgfqpoint{0.027278in}{0.619698in}}{\pgfqpoint{0.013923in}{0.614167in}}%
\pgfpathclose%
\pgfpathmoveto{\pgfqpoint{0.166667in}{0.608333in}}%
\pgfpathcurveto{\pgfqpoint{0.182137in}{0.608333in}}{\pgfqpoint{0.196975in}{0.614480in}}{\pgfqpoint{0.207915in}{0.625419in}}%
\pgfpathcurveto{\pgfqpoint{0.218854in}{0.636358in}}{\pgfqpoint{0.225000in}{0.651196in}}{\pgfqpoint{0.225000in}{0.666667in}}%
\pgfpathcurveto{\pgfqpoint{0.225000in}{0.682137in}}{\pgfqpoint{0.218854in}{0.696975in}}{\pgfqpoint{0.207915in}{0.707915in}}%
\pgfpathcurveto{\pgfqpoint{0.196975in}{0.718854in}}{\pgfqpoint{0.182137in}{0.725000in}}{\pgfqpoint{0.166667in}{0.725000in}}%
\pgfpathcurveto{\pgfqpoint{0.151196in}{0.725000in}}{\pgfqpoint{0.136358in}{0.718854in}}{\pgfqpoint{0.125419in}{0.707915in}}%
\pgfpathcurveto{\pgfqpoint{0.114480in}{0.696975in}}{\pgfqpoint{0.108333in}{0.682137in}}{\pgfqpoint{0.108333in}{0.666667in}}%
\pgfpathcurveto{\pgfqpoint{0.108333in}{0.651196in}}{\pgfqpoint{0.114480in}{0.636358in}}{\pgfqpoint{0.125419in}{0.625419in}}%
\pgfpathcurveto{\pgfqpoint{0.136358in}{0.614480in}}{\pgfqpoint{0.151196in}{0.608333in}}{\pgfqpoint{0.166667in}{0.608333in}}%
\pgfpathclose%
\pgfpathmoveto{\pgfqpoint{0.166667in}{0.614167in}}%
\pgfpathcurveto{\pgfqpoint{0.166667in}{0.614167in}}{\pgfqpoint{0.152744in}{0.614167in}}{\pgfqpoint{0.139389in}{0.619698in}}%
\pgfpathcurveto{\pgfqpoint{0.129544in}{0.629544in}}{\pgfqpoint{0.119698in}{0.639389in}}{\pgfqpoint{0.114167in}{0.652744in}}%
\pgfpathcurveto{\pgfqpoint{0.114167in}{0.666667in}}{\pgfqpoint{0.114167in}{0.680590in}}{\pgfqpoint{0.119698in}{0.693945in}}%
\pgfpathcurveto{\pgfqpoint{0.129544in}{0.703790in}}{\pgfqpoint{0.139389in}{0.713635in}}{\pgfqpoint{0.152744in}{0.719167in}}%
\pgfpathcurveto{\pgfqpoint{0.166667in}{0.719167in}}{\pgfqpoint{0.180590in}{0.719167in}}{\pgfqpoint{0.193945in}{0.713635in}}%
\pgfpathcurveto{\pgfqpoint{0.203790in}{0.703790in}}{\pgfqpoint{0.213635in}{0.693945in}}{\pgfqpoint{0.219167in}{0.680590in}}%
\pgfpathcurveto{\pgfqpoint{0.219167in}{0.666667in}}{\pgfqpoint{0.219167in}{0.652744in}}{\pgfqpoint{0.213635in}{0.639389in}}%
\pgfpathcurveto{\pgfqpoint{0.203790in}{0.629544in}}{\pgfqpoint{0.193945in}{0.619698in}}{\pgfqpoint{0.180590in}{0.614167in}}%
\pgfpathclose%
\pgfpathmoveto{\pgfqpoint{0.333333in}{0.608333in}}%
\pgfpathcurveto{\pgfqpoint{0.348804in}{0.608333in}}{\pgfqpoint{0.363642in}{0.614480in}}{\pgfqpoint{0.374581in}{0.625419in}}%
\pgfpathcurveto{\pgfqpoint{0.385520in}{0.636358in}}{\pgfqpoint{0.391667in}{0.651196in}}{\pgfqpoint{0.391667in}{0.666667in}}%
\pgfpathcurveto{\pgfqpoint{0.391667in}{0.682137in}}{\pgfqpoint{0.385520in}{0.696975in}}{\pgfqpoint{0.374581in}{0.707915in}}%
\pgfpathcurveto{\pgfqpoint{0.363642in}{0.718854in}}{\pgfqpoint{0.348804in}{0.725000in}}{\pgfqpoint{0.333333in}{0.725000in}}%
\pgfpathcurveto{\pgfqpoint{0.317863in}{0.725000in}}{\pgfqpoint{0.303025in}{0.718854in}}{\pgfqpoint{0.292085in}{0.707915in}}%
\pgfpathcurveto{\pgfqpoint{0.281146in}{0.696975in}}{\pgfqpoint{0.275000in}{0.682137in}}{\pgfqpoint{0.275000in}{0.666667in}}%
\pgfpathcurveto{\pgfqpoint{0.275000in}{0.651196in}}{\pgfqpoint{0.281146in}{0.636358in}}{\pgfqpoint{0.292085in}{0.625419in}}%
\pgfpathcurveto{\pgfqpoint{0.303025in}{0.614480in}}{\pgfqpoint{0.317863in}{0.608333in}}{\pgfqpoint{0.333333in}{0.608333in}}%
\pgfpathclose%
\pgfpathmoveto{\pgfqpoint{0.333333in}{0.614167in}}%
\pgfpathcurveto{\pgfqpoint{0.333333in}{0.614167in}}{\pgfqpoint{0.319410in}{0.614167in}}{\pgfqpoint{0.306055in}{0.619698in}}%
\pgfpathcurveto{\pgfqpoint{0.296210in}{0.629544in}}{\pgfqpoint{0.286365in}{0.639389in}}{\pgfqpoint{0.280833in}{0.652744in}}%
\pgfpathcurveto{\pgfqpoint{0.280833in}{0.666667in}}{\pgfqpoint{0.280833in}{0.680590in}}{\pgfqpoint{0.286365in}{0.693945in}}%
\pgfpathcurveto{\pgfqpoint{0.296210in}{0.703790in}}{\pgfqpoint{0.306055in}{0.713635in}}{\pgfqpoint{0.319410in}{0.719167in}}%
\pgfpathcurveto{\pgfqpoint{0.333333in}{0.719167in}}{\pgfqpoint{0.347256in}{0.719167in}}{\pgfqpoint{0.360611in}{0.713635in}}%
\pgfpathcurveto{\pgfqpoint{0.370456in}{0.703790in}}{\pgfqpoint{0.380302in}{0.693945in}}{\pgfqpoint{0.385833in}{0.680590in}}%
\pgfpathcurveto{\pgfqpoint{0.385833in}{0.666667in}}{\pgfqpoint{0.385833in}{0.652744in}}{\pgfqpoint{0.380302in}{0.639389in}}%
\pgfpathcurveto{\pgfqpoint{0.370456in}{0.629544in}}{\pgfqpoint{0.360611in}{0.619698in}}{\pgfqpoint{0.347256in}{0.614167in}}%
\pgfpathclose%
\pgfpathmoveto{\pgfqpoint{0.500000in}{0.608333in}}%
\pgfpathcurveto{\pgfqpoint{0.515470in}{0.608333in}}{\pgfqpoint{0.530309in}{0.614480in}}{\pgfqpoint{0.541248in}{0.625419in}}%
\pgfpathcurveto{\pgfqpoint{0.552187in}{0.636358in}}{\pgfqpoint{0.558333in}{0.651196in}}{\pgfqpoint{0.558333in}{0.666667in}}%
\pgfpathcurveto{\pgfqpoint{0.558333in}{0.682137in}}{\pgfqpoint{0.552187in}{0.696975in}}{\pgfqpoint{0.541248in}{0.707915in}}%
\pgfpathcurveto{\pgfqpoint{0.530309in}{0.718854in}}{\pgfqpoint{0.515470in}{0.725000in}}{\pgfqpoint{0.500000in}{0.725000in}}%
\pgfpathcurveto{\pgfqpoint{0.484530in}{0.725000in}}{\pgfqpoint{0.469691in}{0.718854in}}{\pgfqpoint{0.458752in}{0.707915in}}%
\pgfpathcurveto{\pgfqpoint{0.447813in}{0.696975in}}{\pgfqpoint{0.441667in}{0.682137in}}{\pgfqpoint{0.441667in}{0.666667in}}%
\pgfpathcurveto{\pgfqpoint{0.441667in}{0.651196in}}{\pgfqpoint{0.447813in}{0.636358in}}{\pgfqpoint{0.458752in}{0.625419in}}%
\pgfpathcurveto{\pgfqpoint{0.469691in}{0.614480in}}{\pgfqpoint{0.484530in}{0.608333in}}{\pgfqpoint{0.500000in}{0.608333in}}%
\pgfpathclose%
\pgfpathmoveto{\pgfqpoint{0.500000in}{0.614167in}}%
\pgfpathcurveto{\pgfqpoint{0.500000in}{0.614167in}}{\pgfqpoint{0.486077in}{0.614167in}}{\pgfqpoint{0.472722in}{0.619698in}}%
\pgfpathcurveto{\pgfqpoint{0.462877in}{0.629544in}}{\pgfqpoint{0.453032in}{0.639389in}}{\pgfqpoint{0.447500in}{0.652744in}}%
\pgfpathcurveto{\pgfqpoint{0.447500in}{0.666667in}}{\pgfqpoint{0.447500in}{0.680590in}}{\pgfqpoint{0.453032in}{0.693945in}}%
\pgfpathcurveto{\pgfqpoint{0.462877in}{0.703790in}}{\pgfqpoint{0.472722in}{0.713635in}}{\pgfqpoint{0.486077in}{0.719167in}}%
\pgfpathcurveto{\pgfqpoint{0.500000in}{0.719167in}}{\pgfqpoint{0.513923in}{0.719167in}}{\pgfqpoint{0.527278in}{0.713635in}}%
\pgfpathcurveto{\pgfqpoint{0.537123in}{0.703790in}}{\pgfqpoint{0.546968in}{0.693945in}}{\pgfqpoint{0.552500in}{0.680590in}}%
\pgfpathcurveto{\pgfqpoint{0.552500in}{0.666667in}}{\pgfqpoint{0.552500in}{0.652744in}}{\pgfqpoint{0.546968in}{0.639389in}}%
\pgfpathcurveto{\pgfqpoint{0.537123in}{0.629544in}}{\pgfqpoint{0.527278in}{0.619698in}}{\pgfqpoint{0.513923in}{0.614167in}}%
\pgfpathclose%
\pgfpathmoveto{\pgfqpoint{0.666667in}{0.608333in}}%
\pgfpathcurveto{\pgfqpoint{0.682137in}{0.608333in}}{\pgfqpoint{0.696975in}{0.614480in}}{\pgfqpoint{0.707915in}{0.625419in}}%
\pgfpathcurveto{\pgfqpoint{0.718854in}{0.636358in}}{\pgfqpoint{0.725000in}{0.651196in}}{\pgfqpoint{0.725000in}{0.666667in}}%
\pgfpathcurveto{\pgfqpoint{0.725000in}{0.682137in}}{\pgfqpoint{0.718854in}{0.696975in}}{\pgfqpoint{0.707915in}{0.707915in}}%
\pgfpathcurveto{\pgfqpoint{0.696975in}{0.718854in}}{\pgfqpoint{0.682137in}{0.725000in}}{\pgfqpoint{0.666667in}{0.725000in}}%
\pgfpathcurveto{\pgfqpoint{0.651196in}{0.725000in}}{\pgfqpoint{0.636358in}{0.718854in}}{\pgfqpoint{0.625419in}{0.707915in}}%
\pgfpathcurveto{\pgfqpoint{0.614480in}{0.696975in}}{\pgfqpoint{0.608333in}{0.682137in}}{\pgfqpoint{0.608333in}{0.666667in}}%
\pgfpathcurveto{\pgfqpoint{0.608333in}{0.651196in}}{\pgfqpoint{0.614480in}{0.636358in}}{\pgfqpoint{0.625419in}{0.625419in}}%
\pgfpathcurveto{\pgfqpoint{0.636358in}{0.614480in}}{\pgfqpoint{0.651196in}{0.608333in}}{\pgfqpoint{0.666667in}{0.608333in}}%
\pgfpathclose%
\pgfpathmoveto{\pgfqpoint{0.666667in}{0.614167in}}%
\pgfpathcurveto{\pgfqpoint{0.666667in}{0.614167in}}{\pgfqpoint{0.652744in}{0.614167in}}{\pgfqpoint{0.639389in}{0.619698in}}%
\pgfpathcurveto{\pgfqpoint{0.629544in}{0.629544in}}{\pgfqpoint{0.619698in}{0.639389in}}{\pgfqpoint{0.614167in}{0.652744in}}%
\pgfpathcurveto{\pgfqpoint{0.614167in}{0.666667in}}{\pgfqpoint{0.614167in}{0.680590in}}{\pgfqpoint{0.619698in}{0.693945in}}%
\pgfpathcurveto{\pgfqpoint{0.629544in}{0.703790in}}{\pgfqpoint{0.639389in}{0.713635in}}{\pgfqpoint{0.652744in}{0.719167in}}%
\pgfpathcurveto{\pgfqpoint{0.666667in}{0.719167in}}{\pgfqpoint{0.680590in}{0.719167in}}{\pgfqpoint{0.693945in}{0.713635in}}%
\pgfpathcurveto{\pgfqpoint{0.703790in}{0.703790in}}{\pgfqpoint{0.713635in}{0.693945in}}{\pgfqpoint{0.719167in}{0.680590in}}%
\pgfpathcurveto{\pgfqpoint{0.719167in}{0.666667in}}{\pgfqpoint{0.719167in}{0.652744in}}{\pgfqpoint{0.713635in}{0.639389in}}%
\pgfpathcurveto{\pgfqpoint{0.703790in}{0.629544in}}{\pgfqpoint{0.693945in}{0.619698in}}{\pgfqpoint{0.680590in}{0.614167in}}%
\pgfpathclose%
\pgfpathmoveto{\pgfqpoint{0.833333in}{0.608333in}}%
\pgfpathcurveto{\pgfqpoint{0.848804in}{0.608333in}}{\pgfqpoint{0.863642in}{0.614480in}}{\pgfqpoint{0.874581in}{0.625419in}}%
\pgfpathcurveto{\pgfqpoint{0.885520in}{0.636358in}}{\pgfqpoint{0.891667in}{0.651196in}}{\pgfqpoint{0.891667in}{0.666667in}}%
\pgfpathcurveto{\pgfqpoint{0.891667in}{0.682137in}}{\pgfqpoint{0.885520in}{0.696975in}}{\pgfqpoint{0.874581in}{0.707915in}}%
\pgfpathcurveto{\pgfqpoint{0.863642in}{0.718854in}}{\pgfqpoint{0.848804in}{0.725000in}}{\pgfqpoint{0.833333in}{0.725000in}}%
\pgfpathcurveto{\pgfqpoint{0.817863in}{0.725000in}}{\pgfqpoint{0.803025in}{0.718854in}}{\pgfqpoint{0.792085in}{0.707915in}}%
\pgfpathcurveto{\pgfqpoint{0.781146in}{0.696975in}}{\pgfqpoint{0.775000in}{0.682137in}}{\pgfqpoint{0.775000in}{0.666667in}}%
\pgfpathcurveto{\pgfqpoint{0.775000in}{0.651196in}}{\pgfqpoint{0.781146in}{0.636358in}}{\pgfqpoint{0.792085in}{0.625419in}}%
\pgfpathcurveto{\pgfqpoint{0.803025in}{0.614480in}}{\pgfqpoint{0.817863in}{0.608333in}}{\pgfqpoint{0.833333in}{0.608333in}}%
\pgfpathclose%
\pgfpathmoveto{\pgfqpoint{0.833333in}{0.614167in}}%
\pgfpathcurveto{\pgfqpoint{0.833333in}{0.614167in}}{\pgfqpoint{0.819410in}{0.614167in}}{\pgfqpoint{0.806055in}{0.619698in}}%
\pgfpathcurveto{\pgfqpoint{0.796210in}{0.629544in}}{\pgfqpoint{0.786365in}{0.639389in}}{\pgfqpoint{0.780833in}{0.652744in}}%
\pgfpathcurveto{\pgfqpoint{0.780833in}{0.666667in}}{\pgfqpoint{0.780833in}{0.680590in}}{\pgfqpoint{0.786365in}{0.693945in}}%
\pgfpathcurveto{\pgfqpoint{0.796210in}{0.703790in}}{\pgfqpoint{0.806055in}{0.713635in}}{\pgfqpoint{0.819410in}{0.719167in}}%
\pgfpathcurveto{\pgfqpoint{0.833333in}{0.719167in}}{\pgfqpoint{0.847256in}{0.719167in}}{\pgfqpoint{0.860611in}{0.713635in}}%
\pgfpathcurveto{\pgfqpoint{0.870456in}{0.703790in}}{\pgfqpoint{0.880302in}{0.693945in}}{\pgfqpoint{0.885833in}{0.680590in}}%
\pgfpathcurveto{\pgfqpoint{0.885833in}{0.666667in}}{\pgfqpoint{0.885833in}{0.652744in}}{\pgfqpoint{0.880302in}{0.639389in}}%
\pgfpathcurveto{\pgfqpoint{0.870456in}{0.629544in}}{\pgfqpoint{0.860611in}{0.619698in}}{\pgfqpoint{0.847256in}{0.614167in}}%
\pgfpathclose%
\pgfpathmoveto{\pgfqpoint{1.000000in}{0.608333in}}%
\pgfpathcurveto{\pgfqpoint{1.015470in}{0.608333in}}{\pgfqpoint{1.030309in}{0.614480in}}{\pgfqpoint{1.041248in}{0.625419in}}%
\pgfpathcurveto{\pgfqpoint{1.052187in}{0.636358in}}{\pgfqpoint{1.058333in}{0.651196in}}{\pgfqpoint{1.058333in}{0.666667in}}%
\pgfpathcurveto{\pgfqpoint{1.058333in}{0.682137in}}{\pgfqpoint{1.052187in}{0.696975in}}{\pgfqpoint{1.041248in}{0.707915in}}%
\pgfpathcurveto{\pgfqpoint{1.030309in}{0.718854in}}{\pgfqpoint{1.015470in}{0.725000in}}{\pgfqpoint{1.000000in}{0.725000in}}%
\pgfpathcurveto{\pgfqpoint{0.984530in}{0.725000in}}{\pgfqpoint{0.969691in}{0.718854in}}{\pgfqpoint{0.958752in}{0.707915in}}%
\pgfpathcurveto{\pgfqpoint{0.947813in}{0.696975in}}{\pgfqpoint{0.941667in}{0.682137in}}{\pgfqpoint{0.941667in}{0.666667in}}%
\pgfpathcurveto{\pgfqpoint{0.941667in}{0.651196in}}{\pgfqpoint{0.947813in}{0.636358in}}{\pgfqpoint{0.958752in}{0.625419in}}%
\pgfpathcurveto{\pgfqpoint{0.969691in}{0.614480in}}{\pgfqpoint{0.984530in}{0.608333in}}{\pgfqpoint{1.000000in}{0.608333in}}%
\pgfpathclose%
\pgfpathmoveto{\pgfqpoint{1.000000in}{0.614167in}}%
\pgfpathcurveto{\pgfqpoint{1.000000in}{0.614167in}}{\pgfqpoint{0.986077in}{0.614167in}}{\pgfqpoint{0.972722in}{0.619698in}}%
\pgfpathcurveto{\pgfqpoint{0.962877in}{0.629544in}}{\pgfqpoint{0.953032in}{0.639389in}}{\pgfqpoint{0.947500in}{0.652744in}}%
\pgfpathcurveto{\pgfqpoint{0.947500in}{0.666667in}}{\pgfqpoint{0.947500in}{0.680590in}}{\pgfqpoint{0.953032in}{0.693945in}}%
\pgfpathcurveto{\pgfqpoint{0.962877in}{0.703790in}}{\pgfqpoint{0.972722in}{0.713635in}}{\pgfqpoint{0.986077in}{0.719167in}}%
\pgfpathcurveto{\pgfqpoint{1.000000in}{0.719167in}}{\pgfqpoint{1.013923in}{0.719167in}}{\pgfqpoint{1.027278in}{0.713635in}}%
\pgfpathcurveto{\pgfqpoint{1.037123in}{0.703790in}}{\pgfqpoint{1.046968in}{0.693945in}}{\pgfqpoint{1.052500in}{0.680590in}}%
\pgfpathcurveto{\pgfqpoint{1.052500in}{0.666667in}}{\pgfqpoint{1.052500in}{0.652744in}}{\pgfqpoint{1.046968in}{0.639389in}}%
\pgfpathcurveto{\pgfqpoint{1.037123in}{0.629544in}}{\pgfqpoint{1.027278in}{0.619698in}}{\pgfqpoint{1.013923in}{0.614167in}}%
\pgfpathclose%
\pgfpathmoveto{\pgfqpoint{0.083333in}{0.775000in}}%
\pgfpathcurveto{\pgfqpoint{0.098804in}{0.775000in}}{\pgfqpoint{0.113642in}{0.781146in}}{\pgfqpoint{0.124581in}{0.792085in}}%
\pgfpathcurveto{\pgfqpoint{0.135520in}{0.803025in}}{\pgfqpoint{0.141667in}{0.817863in}}{\pgfqpoint{0.141667in}{0.833333in}}%
\pgfpathcurveto{\pgfqpoint{0.141667in}{0.848804in}}{\pgfqpoint{0.135520in}{0.863642in}}{\pgfqpoint{0.124581in}{0.874581in}}%
\pgfpathcurveto{\pgfqpoint{0.113642in}{0.885520in}}{\pgfqpoint{0.098804in}{0.891667in}}{\pgfqpoint{0.083333in}{0.891667in}}%
\pgfpathcurveto{\pgfqpoint{0.067863in}{0.891667in}}{\pgfqpoint{0.053025in}{0.885520in}}{\pgfqpoint{0.042085in}{0.874581in}}%
\pgfpathcurveto{\pgfqpoint{0.031146in}{0.863642in}}{\pgfqpoint{0.025000in}{0.848804in}}{\pgfqpoint{0.025000in}{0.833333in}}%
\pgfpathcurveto{\pgfqpoint{0.025000in}{0.817863in}}{\pgfqpoint{0.031146in}{0.803025in}}{\pgfqpoint{0.042085in}{0.792085in}}%
\pgfpathcurveto{\pgfqpoint{0.053025in}{0.781146in}}{\pgfqpoint{0.067863in}{0.775000in}}{\pgfqpoint{0.083333in}{0.775000in}}%
\pgfpathclose%
\pgfpathmoveto{\pgfqpoint{0.083333in}{0.780833in}}%
\pgfpathcurveto{\pgfqpoint{0.083333in}{0.780833in}}{\pgfqpoint{0.069410in}{0.780833in}}{\pgfqpoint{0.056055in}{0.786365in}}%
\pgfpathcurveto{\pgfqpoint{0.046210in}{0.796210in}}{\pgfqpoint{0.036365in}{0.806055in}}{\pgfqpoint{0.030833in}{0.819410in}}%
\pgfpathcurveto{\pgfqpoint{0.030833in}{0.833333in}}{\pgfqpoint{0.030833in}{0.847256in}}{\pgfqpoint{0.036365in}{0.860611in}}%
\pgfpathcurveto{\pgfqpoint{0.046210in}{0.870456in}}{\pgfqpoint{0.056055in}{0.880302in}}{\pgfqpoint{0.069410in}{0.885833in}}%
\pgfpathcurveto{\pgfqpoint{0.083333in}{0.885833in}}{\pgfqpoint{0.097256in}{0.885833in}}{\pgfqpoint{0.110611in}{0.880302in}}%
\pgfpathcurveto{\pgfqpoint{0.120456in}{0.870456in}}{\pgfqpoint{0.130302in}{0.860611in}}{\pgfqpoint{0.135833in}{0.847256in}}%
\pgfpathcurveto{\pgfqpoint{0.135833in}{0.833333in}}{\pgfqpoint{0.135833in}{0.819410in}}{\pgfqpoint{0.130302in}{0.806055in}}%
\pgfpathcurveto{\pgfqpoint{0.120456in}{0.796210in}}{\pgfqpoint{0.110611in}{0.786365in}}{\pgfqpoint{0.097256in}{0.780833in}}%
\pgfpathclose%
\pgfpathmoveto{\pgfqpoint{0.250000in}{0.775000in}}%
\pgfpathcurveto{\pgfqpoint{0.265470in}{0.775000in}}{\pgfqpoint{0.280309in}{0.781146in}}{\pgfqpoint{0.291248in}{0.792085in}}%
\pgfpathcurveto{\pgfqpoint{0.302187in}{0.803025in}}{\pgfqpoint{0.308333in}{0.817863in}}{\pgfqpoint{0.308333in}{0.833333in}}%
\pgfpathcurveto{\pgfqpoint{0.308333in}{0.848804in}}{\pgfqpoint{0.302187in}{0.863642in}}{\pgfqpoint{0.291248in}{0.874581in}}%
\pgfpathcurveto{\pgfqpoint{0.280309in}{0.885520in}}{\pgfqpoint{0.265470in}{0.891667in}}{\pgfqpoint{0.250000in}{0.891667in}}%
\pgfpathcurveto{\pgfqpoint{0.234530in}{0.891667in}}{\pgfqpoint{0.219691in}{0.885520in}}{\pgfqpoint{0.208752in}{0.874581in}}%
\pgfpathcurveto{\pgfqpoint{0.197813in}{0.863642in}}{\pgfqpoint{0.191667in}{0.848804in}}{\pgfqpoint{0.191667in}{0.833333in}}%
\pgfpathcurveto{\pgfqpoint{0.191667in}{0.817863in}}{\pgfqpoint{0.197813in}{0.803025in}}{\pgfqpoint{0.208752in}{0.792085in}}%
\pgfpathcurveto{\pgfqpoint{0.219691in}{0.781146in}}{\pgfqpoint{0.234530in}{0.775000in}}{\pgfqpoint{0.250000in}{0.775000in}}%
\pgfpathclose%
\pgfpathmoveto{\pgfqpoint{0.250000in}{0.780833in}}%
\pgfpathcurveto{\pgfqpoint{0.250000in}{0.780833in}}{\pgfqpoint{0.236077in}{0.780833in}}{\pgfqpoint{0.222722in}{0.786365in}}%
\pgfpathcurveto{\pgfqpoint{0.212877in}{0.796210in}}{\pgfqpoint{0.203032in}{0.806055in}}{\pgfqpoint{0.197500in}{0.819410in}}%
\pgfpathcurveto{\pgfqpoint{0.197500in}{0.833333in}}{\pgfqpoint{0.197500in}{0.847256in}}{\pgfqpoint{0.203032in}{0.860611in}}%
\pgfpathcurveto{\pgfqpoint{0.212877in}{0.870456in}}{\pgfqpoint{0.222722in}{0.880302in}}{\pgfqpoint{0.236077in}{0.885833in}}%
\pgfpathcurveto{\pgfqpoint{0.250000in}{0.885833in}}{\pgfqpoint{0.263923in}{0.885833in}}{\pgfqpoint{0.277278in}{0.880302in}}%
\pgfpathcurveto{\pgfqpoint{0.287123in}{0.870456in}}{\pgfqpoint{0.296968in}{0.860611in}}{\pgfqpoint{0.302500in}{0.847256in}}%
\pgfpathcurveto{\pgfqpoint{0.302500in}{0.833333in}}{\pgfqpoint{0.302500in}{0.819410in}}{\pgfqpoint{0.296968in}{0.806055in}}%
\pgfpathcurveto{\pgfqpoint{0.287123in}{0.796210in}}{\pgfqpoint{0.277278in}{0.786365in}}{\pgfqpoint{0.263923in}{0.780833in}}%
\pgfpathclose%
\pgfpathmoveto{\pgfqpoint{0.416667in}{0.775000in}}%
\pgfpathcurveto{\pgfqpoint{0.432137in}{0.775000in}}{\pgfqpoint{0.446975in}{0.781146in}}{\pgfqpoint{0.457915in}{0.792085in}}%
\pgfpathcurveto{\pgfqpoint{0.468854in}{0.803025in}}{\pgfqpoint{0.475000in}{0.817863in}}{\pgfqpoint{0.475000in}{0.833333in}}%
\pgfpathcurveto{\pgfqpoint{0.475000in}{0.848804in}}{\pgfqpoint{0.468854in}{0.863642in}}{\pgfqpoint{0.457915in}{0.874581in}}%
\pgfpathcurveto{\pgfqpoint{0.446975in}{0.885520in}}{\pgfqpoint{0.432137in}{0.891667in}}{\pgfqpoint{0.416667in}{0.891667in}}%
\pgfpathcurveto{\pgfqpoint{0.401196in}{0.891667in}}{\pgfqpoint{0.386358in}{0.885520in}}{\pgfqpoint{0.375419in}{0.874581in}}%
\pgfpathcurveto{\pgfqpoint{0.364480in}{0.863642in}}{\pgfqpoint{0.358333in}{0.848804in}}{\pgfqpoint{0.358333in}{0.833333in}}%
\pgfpathcurveto{\pgfqpoint{0.358333in}{0.817863in}}{\pgfqpoint{0.364480in}{0.803025in}}{\pgfqpoint{0.375419in}{0.792085in}}%
\pgfpathcurveto{\pgfqpoint{0.386358in}{0.781146in}}{\pgfqpoint{0.401196in}{0.775000in}}{\pgfqpoint{0.416667in}{0.775000in}}%
\pgfpathclose%
\pgfpathmoveto{\pgfqpoint{0.416667in}{0.780833in}}%
\pgfpathcurveto{\pgfqpoint{0.416667in}{0.780833in}}{\pgfqpoint{0.402744in}{0.780833in}}{\pgfqpoint{0.389389in}{0.786365in}}%
\pgfpathcurveto{\pgfqpoint{0.379544in}{0.796210in}}{\pgfqpoint{0.369698in}{0.806055in}}{\pgfqpoint{0.364167in}{0.819410in}}%
\pgfpathcurveto{\pgfqpoint{0.364167in}{0.833333in}}{\pgfqpoint{0.364167in}{0.847256in}}{\pgfqpoint{0.369698in}{0.860611in}}%
\pgfpathcurveto{\pgfqpoint{0.379544in}{0.870456in}}{\pgfqpoint{0.389389in}{0.880302in}}{\pgfqpoint{0.402744in}{0.885833in}}%
\pgfpathcurveto{\pgfqpoint{0.416667in}{0.885833in}}{\pgfqpoint{0.430590in}{0.885833in}}{\pgfqpoint{0.443945in}{0.880302in}}%
\pgfpathcurveto{\pgfqpoint{0.453790in}{0.870456in}}{\pgfqpoint{0.463635in}{0.860611in}}{\pgfqpoint{0.469167in}{0.847256in}}%
\pgfpathcurveto{\pgfqpoint{0.469167in}{0.833333in}}{\pgfqpoint{0.469167in}{0.819410in}}{\pgfqpoint{0.463635in}{0.806055in}}%
\pgfpathcurveto{\pgfqpoint{0.453790in}{0.796210in}}{\pgfqpoint{0.443945in}{0.786365in}}{\pgfqpoint{0.430590in}{0.780833in}}%
\pgfpathclose%
\pgfpathmoveto{\pgfqpoint{0.583333in}{0.775000in}}%
\pgfpathcurveto{\pgfqpoint{0.598804in}{0.775000in}}{\pgfqpoint{0.613642in}{0.781146in}}{\pgfqpoint{0.624581in}{0.792085in}}%
\pgfpathcurveto{\pgfqpoint{0.635520in}{0.803025in}}{\pgfqpoint{0.641667in}{0.817863in}}{\pgfqpoint{0.641667in}{0.833333in}}%
\pgfpathcurveto{\pgfqpoint{0.641667in}{0.848804in}}{\pgfqpoint{0.635520in}{0.863642in}}{\pgfqpoint{0.624581in}{0.874581in}}%
\pgfpathcurveto{\pgfqpoint{0.613642in}{0.885520in}}{\pgfqpoint{0.598804in}{0.891667in}}{\pgfqpoint{0.583333in}{0.891667in}}%
\pgfpathcurveto{\pgfqpoint{0.567863in}{0.891667in}}{\pgfqpoint{0.553025in}{0.885520in}}{\pgfqpoint{0.542085in}{0.874581in}}%
\pgfpathcurveto{\pgfqpoint{0.531146in}{0.863642in}}{\pgfqpoint{0.525000in}{0.848804in}}{\pgfqpoint{0.525000in}{0.833333in}}%
\pgfpathcurveto{\pgfqpoint{0.525000in}{0.817863in}}{\pgfqpoint{0.531146in}{0.803025in}}{\pgfqpoint{0.542085in}{0.792085in}}%
\pgfpathcurveto{\pgfqpoint{0.553025in}{0.781146in}}{\pgfqpoint{0.567863in}{0.775000in}}{\pgfqpoint{0.583333in}{0.775000in}}%
\pgfpathclose%
\pgfpathmoveto{\pgfqpoint{0.583333in}{0.780833in}}%
\pgfpathcurveto{\pgfqpoint{0.583333in}{0.780833in}}{\pgfqpoint{0.569410in}{0.780833in}}{\pgfqpoint{0.556055in}{0.786365in}}%
\pgfpathcurveto{\pgfqpoint{0.546210in}{0.796210in}}{\pgfqpoint{0.536365in}{0.806055in}}{\pgfqpoint{0.530833in}{0.819410in}}%
\pgfpathcurveto{\pgfqpoint{0.530833in}{0.833333in}}{\pgfqpoint{0.530833in}{0.847256in}}{\pgfqpoint{0.536365in}{0.860611in}}%
\pgfpathcurveto{\pgfqpoint{0.546210in}{0.870456in}}{\pgfqpoint{0.556055in}{0.880302in}}{\pgfqpoint{0.569410in}{0.885833in}}%
\pgfpathcurveto{\pgfqpoint{0.583333in}{0.885833in}}{\pgfqpoint{0.597256in}{0.885833in}}{\pgfqpoint{0.610611in}{0.880302in}}%
\pgfpathcurveto{\pgfqpoint{0.620456in}{0.870456in}}{\pgfqpoint{0.630302in}{0.860611in}}{\pgfqpoint{0.635833in}{0.847256in}}%
\pgfpathcurveto{\pgfqpoint{0.635833in}{0.833333in}}{\pgfqpoint{0.635833in}{0.819410in}}{\pgfqpoint{0.630302in}{0.806055in}}%
\pgfpathcurveto{\pgfqpoint{0.620456in}{0.796210in}}{\pgfqpoint{0.610611in}{0.786365in}}{\pgfqpoint{0.597256in}{0.780833in}}%
\pgfpathclose%
\pgfpathmoveto{\pgfqpoint{0.750000in}{0.775000in}}%
\pgfpathcurveto{\pgfqpoint{0.765470in}{0.775000in}}{\pgfqpoint{0.780309in}{0.781146in}}{\pgfqpoint{0.791248in}{0.792085in}}%
\pgfpathcurveto{\pgfqpoint{0.802187in}{0.803025in}}{\pgfqpoint{0.808333in}{0.817863in}}{\pgfqpoint{0.808333in}{0.833333in}}%
\pgfpathcurveto{\pgfqpoint{0.808333in}{0.848804in}}{\pgfqpoint{0.802187in}{0.863642in}}{\pgfqpoint{0.791248in}{0.874581in}}%
\pgfpathcurveto{\pgfqpoint{0.780309in}{0.885520in}}{\pgfqpoint{0.765470in}{0.891667in}}{\pgfqpoint{0.750000in}{0.891667in}}%
\pgfpathcurveto{\pgfqpoint{0.734530in}{0.891667in}}{\pgfqpoint{0.719691in}{0.885520in}}{\pgfqpoint{0.708752in}{0.874581in}}%
\pgfpathcurveto{\pgfqpoint{0.697813in}{0.863642in}}{\pgfqpoint{0.691667in}{0.848804in}}{\pgfqpoint{0.691667in}{0.833333in}}%
\pgfpathcurveto{\pgfqpoint{0.691667in}{0.817863in}}{\pgfqpoint{0.697813in}{0.803025in}}{\pgfqpoint{0.708752in}{0.792085in}}%
\pgfpathcurveto{\pgfqpoint{0.719691in}{0.781146in}}{\pgfqpoint{0.734530in}{0.775000in}}{\pgfqpoint{0.750000in}{0.775000in}}%
\pgfpathclose%
\pgfpathmoveto{\pgfqpoint{0.750000in}{0.780833in}}%
\pgfpathcurveto{\pgfqpoint{0.750000in}{0.780833in}}{\pgfqpoint{0.736077in}{0.780833in}}{\pgfqpoint{0.722722in}{0.786365in}}%
\pgfpathcurveto{\pgfqpoint{0.712877in}{0.796210in}}{\pgfqpoint{0.703032in}{0.806055in}}{\pgfqpoint{0.697500in}{0.819410in}}%
\pgfpathcurveto{\pgfqpoint{0.697500in}{0.833333in}}{\pgfqpoint{0.697500in}{0.847256in}}{\pgfqpoint{0.703032in}{0.860611in}}%
\pgfpathcurveto{\pgfqpoint{0.712877in}{0.870456in}}{\pgfqpoint{0.722722in}{0.880302in}}{\pgfqpoint{0.736077in}{0.885833in}}%
\pgfpathcurveto{\pgfqpoint{0.750000in}{0.885833in}}{\pgfqpoint{0.763923in}{0.885833in}}{\pgfqpoint{0.777278in}{0.880302in}}%
\pgfpathcurveto{\pgfqpoint{0.787123in}{0.870456in}}{\pgfqpoint{0.796968in}{0.860611in}}{\pgfqpoint{0.802500in}{0.847256in}}%
\pgfpathcurveto{\pgfqpoint{0.802500in}{0.833333in}}{\pgfqpoint{0.802500in}{0.819410in}}{\pgfqpoint{0.796968in}{0.806055in}}%
\pgfpathcurveto{\pgfqpoint{0.787123in}{0.796210in}}{\pgfqpoint{0.777278in}{0.786365in}}{\pgfqpoint{0.763923in}{0.780833in}}%
\pgfpathclose%
\pgfpathmoveto{\pgfqpoint{0.916667in}{0.775000in}}%
\pgfpathcurveto{\pgfqpoint{0.932137in}{0.775000in}}{\pgfqpoint{0.946975in}{0.781146in}}{\pgfqpoint{0.957915in}{0.792085in}}%
\pgfpathcurveto{\pgfqpoint{0.968854in}{0.803025in}}{\pgfqpoint{0.975000in}{0.817863in}}{\pgfqpoint{0.975000in}{0.833333in}}%
\pgfpathcurveto{\pgfqpoint{0.975000in}{0.848804in}}{\pgfqpoint{0.968854in}{0.863642in}}{\pgfqpoint{0.957915in}{0.874581in}}%
\pgfpathcurveto{\pgfqpoint{0.946975in}{0.885520in}}{\pgfqpoint{0.932137in}{0.891667in}}{\pgfqpoint{0.916667in}{0.891667in}}%
\pgfpathcurveto{\pgfqpoint{0.901196in}{0.891667in}}{\pgfqpoint{0.886358in}{0.885520in}}{\pgfqpoint{0.875419in}{0.874581in}}%
\pgfpathcurveto{\pgfqpoint{0.864480in}{0.863642in}}{\pgfqpoint{0.858333in}{0.848804in}}{\pgfqpoint{0.858333in}{0.833333in}}%
\pgfpathcurveto{\pgfqpoint{0.858333in}{0.817863in}}{\pgfqpoint{0.864480in}{0.803025in}}{\pgfqpoint{0.875419in}{0.792085in}}%
\pgfpathcurveto{\pgfqpoint{0.886358in}{0.781146in}}{\pgfqpoint{0.901196in}{0.775000in}}{\pgfqpoint{0.916667in}{0.775000in}}%
\pgfpathclose%
\pgfpathmoveto{\pgfqpoint{0.916667in}{0.780833in}}%
\pgfpathcurveto{\pgfqpoint{0.916667in}{0.780833in}}{\pgfqpoint{0.902744in}{0.780833in}}{\pgfqpoint{0.889389in}{0.786365in}}%
\pgfpathcurveto{\pgfqpoint{0.879544in}{0.796210in}}{\pgfqpoint{0.869698in}{0.806055in}}{\pgfqpoint{0.864167in}{0.819410in}}%
\pgfpathcurveto{\pgfqpoint{0.864167in}{0.833333in}}{\pgfqpoint{0.864167in}{0.847256in}}{\pgfqpoint{0.869698in}{0.860611in}}%
\pgfpathcurveto{\pgfqpoint{0.879544in}{0.870456in}}{\pgfqpoint{0.889389in}{0.880302in}}{\pgfqpoint{0.902744in}{0.885833in}}%
\pgfpathcurveto{\pgfqpoint{0.916667in}{0.885833in}}{\pgfqpoint{0.930590in}{0.885833in}}{\pgfqpoint{0.943945in}{0.880302in}}%
\pgfpathcurveto{\pgfqpoint{0.953790in}{0.870456in}}{\pgfqpoint{0.963635in}{0.860611in}}{\pgfqpoint{0.969167in}{0.847256in}}%
\pgfpathcurveto{\pgfqpoint{0.969167in}{0.833333in}}{\pgfqpoint{0.969167in}{0.819410in}}{\pgfqpoint{0.963635in}{0.806055in}}%
\pgfpathcurveto{\pgfqpoint{0.953790in}{0.796210in}}{\pgfqpoint{0.943945in}{0.786365in}}{\pgfqpoint{0.930590in}{0.780833in}}%
\pgfpathclose%
\pgfpathmoveto{\pgfqpoint{0.000000in}{0.941667in}}%
\pgfpathcurveto{\pgfqpoint{0.015470in}{0.941667in}}{\pgfqpoint{0.030309in}{0.947813in}}{\pgfqpoint{0.041248in}{0.958752in}}%
\pgfpathcurveto{\pgfqpoint{0.052187in}{0.969691in}}{\pgfqpoint{0.058333in}{0.984530in}}{\pgfqpoint{0.058333in}{1.000000in}}%
\pgfpathcurveto{\pgfqpoint{0.058333in}{1.015470in}}{\pgfqpoint{0.052187in}{1.030309in}}{\pgfqpoint{0.041248in}{1.041248in}}%
\pgfpathcurveto{\pgfqpoint{0.030309in}{1.052187in}}{\pgfqpoint{0.015470in}{1.058333in}}{\pgfqpoint{0.000000in}{1.058333in}}%
\pgfpathcurveto{\pgfqpoint{-0.015470in}{1.058333in}}{\pgfqpoint{-0.030309in}{1.052187in}}{\pgfqpoint{-0.041248in}{1.041248in}}%
\pgfpathcurveto{\pgfqpoint{-0.052187in}{1.030309in}}{\pgfqpoint{-0.058333in}{1.015470in}}{\pgfqpoint{-0.058333in}{1.000000in}}%
\pgfpathcurveto{\pgfqpoint{-0.058333in}{0.984530in}}{\pgfqpoint{-0.052187in}{0.969691in}}{\pgfqpoint{-0.041248in}{0.958752in}}%
\pgfpathcurveto{\pgfqpoint{-0.030309in}{0.947813in}}{\pgfqpoint{-0.015470in}{0.941667in}}{\pgfqpoint{0.000000in}{0.941667in}}%
\pgfpathclose%
\pgfpathmoveto{\pgfqpoint{0.000000in}{0.947500in}}%
\pgfpathcurveto{\pgfqpoint{0.000000in}{0.947500in}}{\pgfqpoint{-0.013923in}{0.947500in}}{\pgfqpoint{-0.027278in}{0.953032in}}%
\pgfpathcurveto{\pgfqpoint{-0.037123in}{0.962877in}}{\pgfqpoint{-0.046968in}{0.972722in}}{\pgfqpoint{-0.052500in}{0.986077in}}%
\pgfpathcurveto{\pgfqpoint{-0.052500in}{1.000000in}}{\pgfqpoint{-0.052500in}{1.013923in}}{\pgfqpoint{-0.046968in}{1.027278in}}%
\pgfpathcurveto{\pgfqpoint{-0.037123in}{1.037123in}}{\pgfqpoint{-0.027278in}{1.046968in}}{\pgfqpoint{-0.013923in}{1.052500in}}%
\pgfpathcurveto{\pgfqpoint{0.000000in}{1.052500in}}{\pgfqpoint{0.013923in}{1.052500in}}{\pgfqpoint{0.027278in}{1.046968in}}%
\pgfpathcurveto{\pgfqpoint{0.037123in}{1.037123in}}{\pgfqpoint{0.046968in}{1.027278in}}{\pgfqpoint{0.052500in}{1.013923in}}%
\pgfpathcurveto{\pgfqpoint{0.052500in}{1.000000in}}{\pgfqpoint{0.052500in}{0.986077in}}{\pgfqpoint{0.046968in}{0.972722in}}%
\pgfpathcurveto{\pgfqpoint{0.037123in}{0.962877in}}{\pgfqpoint{0.027278in}{0.953032in}}{\pgfqpoint{0.013923in}{0.947500in}}%
\pgfpathclose%
\pgfpathmoveto{\pgfqpoint{0.166667in}{0.941667in}}%
\pgfpathcurveto{\pgfqpoint{0.182137in}{0.941667in}}{\pgfqpoint{0.196975in}{0.947813in}}{\pgfqpoint{0.207915in}{0.958752in}}%
\pgfpathcurveto{\pgfqpoint{0.218854in}{0.969691in}}{\pgfqpoint{0.225000in}{0.984530in}}{\pgfqpoint{0.225000in}{1.000000in}}%
\pgfpathcurveto{\pgfqpoint{0.225000in}{1.015470in}}{\pgfqpoint{0.218854in}{1.030309in}}{\pgfqpoint{0.207915in}{1.041248in}}%
\pgfpathcurveto{\pgfqpoint{0.196975in}{1.052187in}}{\pgfqpoint{0.182137in}{1.058333in}}{\pgfqpoint{0.166667in}{1.058333in}}%
\pgfpathcurveto{\pgfqpoint{0.151196in}{1.058333in}}{\pgfqpoint{0.136358in}{1.052187in}}{\pgfqpoint{0.125419in}{1.041248in}}%
\pgfpathcurveto{\pgfqpoint{0.114480in}{1.030309in}}{\pgfqpoint{0.108333in}{1.015470in}}{\pgfqpoint{0.108333in}{1.000000in}}%
\pgfpathcurveto{\pgfqpoint{0.108333in}{0.984530in}}{\pgfqpoint{0.114480in}{0.969691in}}{\pgfqpoint{0.125419in}{0.958752in}}%
\pgfpathcurveto{\pgfqpoint{0.136358in}{0.947813in}}{\pgfqpoint{0.151196in}{0.941667in}}{\pgfqpoint{0.166667in}{0.941667in}}%
\pgfpathclose%
\pgfpathmoveto{\pgfqpoint{0.166667in}{0.947500in}}%
\pgfpathcurveto{\pgfqpoint{0.166667in}{0.947500in}}{\pgfqpoint{0.152744in}{0.947500in}}{\pgfqpoint{0.139389in}{0.953032in}}%
\pgfpathcurveto{\pgfqpoint{0.129544in}{0.962877in}}{\pgfqpoint{0.119698in}{0.972722in}}{\pgfqpoint{0.114167in}{0.986077in}}%
\pgfpathcurveto{\pgfqpoint{0.114167in}{1.000000in}}{\pgfqpoint{0.114167in}{1.013923in}}{\pgfqpoint{0.119698in}{1.027278in}}%
\pgfpathcurveto{\pgfqpoint{0.129544in}{1.037123in}}{\pgfqpoint{0.139389in}{1.046968in}}{\pgfqpoint{0.152744in}{1.052500in}}%
\pgfpathcurveto{\pgfqpoint{0.166667in}{1.052500in}}{\pgfqpoint{0.180590in}{1.052500in}}{\pgfqpoint{0.193945in}{1.046968in}}%
\pgfpathcurveto{\pgfqpoint{0.203790in}{1.037123in}}{\pgfqpoint{0.213635in}{1.027278in}}{\pgfqpoint{0.219167in}{1.013923in}}%
\pgfpathcurveto{\pgfqpoint{0.219167in}{1.000000in}}{\pgfqpoint{0.219167in}{0.986077in}}{\pgfqpoint{0.213635in}{0.972722in}}%
\pgfpathcurveto{\pgfqpoint{0.203790in}{0.962877in}}{\pgfqpoint{0.193945in}{0.953032in}}{\pgfqpoint{0.180590in}{0.947500in}}%
\pgfpathclose%
\pgfpathmoveto{\pgfqpoint{0.333333in}{0.941667in}}%
\pgfpathcurveto{\pgfqpoint{0.348804in}{0.941667in}}{\pgfqpoint{0.363642in}{0.947813in}}{\pgfqpoint{0.374581in}{0.958752in}}%
\pgfpathcurveto{\pgfqpoint{0.385520in}{0.969691in}}{\pgfqpoint{0.391667in}{0.984530in}}{\pgfqpoint{0.391667in}{1.000000in}}%
\pgfpathcurveto{\pgfqpoint{0.391667in}{1.015470in}}{\pgfqpoint{0.385520in}{1.030309in}}{\pgfqpoint{0.374581in}{1.041248in}}%
\pgfpathcurveto{\pgfqpoint{0.363642in}{1.052187in}}{\pgfqpoint{0.348804in}{1.058333in}}{\pgfqpoint{0.333333in}{1.058333in}}%
\pgfpathcurveto{\pgfqpoint{0.317863in}{1.058333in}}{\pgfqpoint{0.303025in}{1.052187in}}{\pgfqpoint{0.292085in}{1.041248in}}%
\pgfpathcurveto{\pgfqpoint{0.281146in}{1.030309in}}{\pgfqpoint{0.275000in}{1.015470in}}{\pgfqpoint{0.275000in}{1.000000in}}%
\pgfpathcurveto{\pgfqpoint{0.275000in}{0.984530in}}{\pgfqpoint{0.281146in}{0.969691in}}{\pgfqpoint{0.292085in}{0.958752in}}%
\pgfpathcurveto{\pgfqpoint{0.303025in}{0.947813in}}{\pgfqpoint{0.317863in}{0.941667in}}{\pgfqpoint{0.333333in}{0.941667in}}%
\pgfpathclose%
\pgfpathmoveto{\pgfqpoint{0.333333in}{0.947500in}}%
\pgfpathcurveto{\pgfqpoint{0.333333in}{0.947500in}}{\pgfqpoint{0.319410in}{0.947500in}}{\pgfqpoint{0.306055in}{0.953032in}}%
\pgfpathcurveto{\pgfqpoint{0.296210in}{0.962877in}}{\pgfqpoint{0.286365in}{0.972722in}}{\pgfqpoint{0.280833in}{0.986077in}}%
\pgfpathcurveto{\pgfqpoint{0.280833in}{1.000000in}}{\pgfqpoint{0.280833in}{1.013923in}}{\pgfqpoint{0.286365in}{1.027278in}}%
\pgfpathcurveto{\pgfqpoint{0.296210in}{1.037123in}}{\pgfqpoint{0.306055in}{1.046968in}}{\pgfqpoint{0.319410in}{1.052500in}}%
\pgfpathcurveto{\pgfqpoint{0.333333in}{1.052500in}}{\pgfqpoint{0.347256in}{1.052500in}}{\pgfqpoint{0.360611in}{1.046968in}}%
\pgfpathcurveto{\pgfqpoint{0.370456in}{1.037123in}}{\pgfqpoint{0.380302in}{1.027278in}}{\pgfqpoint{0.385833in}{1.013923in}}%
\pgfpathcurveto{\pgfqpoint{0.385833in}{1.000000in}}{\pgfqpoint{0.385833in}{0.986077in}}{\pgfqpoint{0.380302in}{0.972722in}}%
\pgfpathcurveto{\pgfqpoint{0.370456in}{0.962877in}}{\pgfqpoint{0.360611in}{0.953032in}}{\pgfqpoint{0.347256in}{0.947500in}}%
\pgfpathclose%
\pgfpathmoveto{\pgfqpoint{0.500000in}{0.941667in}}%
\pgfpathcurveto{\pgfqpoint{0.515470in}{0.941667in}}{\pgfqpoint{0.530309in}{0.947813in}}{\pgfqpoint{0.541248in}{0.958752in}}%
\pgfpathcurveto{\pgfqpoint{0.552187in}{0.969691in}}{\pgfqpoint{0.558333in}{0.984530in}}{\pgfqpoint{0.558333in}{1.000000in}}%
\pgfpathcurveto{\pgfqpoint{0.558333in}{1.015470in}}{\pgfqpoint{0.552187in}{1.030309in}}{\pgfqpoint{0.541248in}{1.041248in}}%
\pgfpathcurveto{\pgfqpoint{0.530309in}{1.052187in}}{\pgfqpoint{0.515470in}{1.058333in}}{\pgfqpoint{0.500000in}{1.058333in}}%
\pgfpathcurveto{\pgfqpoint{0.484530in}{1.058333in}}{\pgfqpoint{0.469691in}{1.052187in}}{\pgfqpoint{0.458752in}{1.041248in}}%
\pgfpathcurveto{\pgfqpoint{0.447813in}{1.030309in}}{\pgfqpoint{0.441667in}{1.015470in}}{\pgfqpoint{0.441667in}{1.000000in}}%
\pgfpathcurveto{\pgfqpoint{0.441667in}{0.984530in}}{\pgfqpoint{0.447813in}{0.969691in}}{\pgfqpoint{0.458752in}{0.958752in}}%
\pgfpathcurveto{\pgfqpoint{0.469691in}{0.947813in}}{\pgfqpoint{0.484530in}{0.941667in}}{\pgfqpoint{0.500000in}{0.941667in}}%
\pgfpathclose%
\pgfpathmoveto{\pgfqpoint{0.500000in}{0.947500in}}%
\pgfpathcurveto{\pgfqpoint{0.500000in}{0.947500in}}{\pgfqpoint{0.486077in}{0.947500in}}{\pgfqpoint{0.472722in}{0.953032in}}%
\pgfpathcurveto{\pgfqpoint{0.462877in}{0.962877in}}{\pgfqpoint{0.453032in}{0.972722in}}{\pgfqpoint{0.447500in}{0.986077in}}%
\pgfpathcurveto{\pgfqpoint{0.447500in}{1.000000in}}{\pgfqpoint{0.447500in}{1.013923in}}{\pgfqpoint{0.453032in}{1.027278in}}%
\pgfpathcurveto{\pgfqpoint{0.462877in}{1.037123in}}{\pgfqpoint{0.472722in}{1.046968in}}{\pgfqpoint{0.486077in}{1.052500in}}%
\pgfpathcurveto{\pgfqpoint{0.500000in}{1.052500in}}{\pgfqpoint{0.513923in}{1.052500in}}{\pgfqpoint{0.527278in}{1.046968in}}%
\pgfpathcurveto{\pgfqpoint{0.537123in}{1.037123in}}{\pgfqpoint{0.546968in}{1.027278in}}{\pgfqpoint{0.552500in}{1.013923in}}%
\pgfpathcurveto{\pgfqpoint{0.552500in}{1.000000in}}{\pgfqpoint{0.552500in}{0.986077in}}{\pgfqpoint{0.546968in}{0.972722in}}%
\pgfpathcurveto{\pgfqpoint{0.537123in}{0.962877in}}{\pgfqpoint{0.527278in}{0.953032in}}{\pgfqpoint{0.513923in}{0.947500in}}%
\pgfpathclose%
\pgfpathmoveto{\pgfqpoint{0.666667in}{0.941667in}}%
\pgfpathcurveto{\pgfqpoint{0.682137in}{0.941667in}}{\pgfqpoint{0.696975in}{0.947813in}}{\pgfqpoint{0.707915in}{0.958752in}}%
\pgfpathcurveto{\pgfqpoint{0.718854in}{0.969691in}}{\pgfqpoint{0.725000in}{0.984530in}}{\pgfqpoint{0.725000in}{1.000000in}}%
\pgfpathcurveto{\pgfqpoint{0.725000in}{1.015470in}}{\pgfqpoint{0.718854in}{1.030309in}}{\pgfqpoint{0.707915in}{1.041248in}}%
\pgfpathcurveto{\pgfqpoint{0.696975in}{1.052187in}}{\pgfqpoint{0.682137in}{1.058333in}}{\pgfqpoint{0.666667in}{1.058333in}}%
\pgfpathcurveto{\pgfqpoint{0.651196in}{1.058333in}}{\pgfqpoint{0.636358in}{1.052187in}}{\pgfqpoint{0.625419in}{1.041248in}}%
\pgfpathcurveto{\pgfqpoint{0.614480in}{1.030309in}}{\pgfqpoint{0.608333in}{1.015470in}}{\pgfqpoint{0.608333in}{1.000000in}}%
\pgfpathcurveto{\pgfqpoint{0.608333in}{0.984530in}}{\pgfqpoint{0.614480in}{0.969691in}}{\pgfqpoint{0.625419in}{0.958752in}}%
\pgfpathcurveto{\pgfqpoint{0.636358in}{0.947813in}}{\pgfqpoint{0.651196in}{0.941667in}}{\pgfqpoint{0.666667in}{0.941667in}}%
\pgfpathclose%
\pgfpathmoveto{\pgfqpoint{0.666667in}{0.947500in}}%
\pgfpathcurveto{\pgfqpoint{0.666667in}{0.947500in}}{\pgfqpoint{0.652744in}{0.947500in}}{\pgfqpoint{0.639389in}{0.953032in}}%
\pgfpathcurveto{\pgfqpoint{0.629544in}{0.962877in}}{\pgfqpoint{0.619698in}{0.972722in}}{\pgfqpoint{0.614167in}{0.986077in}}%
\pgfpathcurveto{\pgfqpoint{0.614167in}{1.000000in}}{\pgfqpoint{0.614167in}{1.013923in}}{\pgfqpoint{0.619698in}{1.027278in}}%
\pgfpathcurveto{\pgfqpoint{0.629544in}{1.037123in}}{\pgfqpoint{0.639389in}{1.046968in}}{\pgfqpoint{0.652744in}{1.052500in}}%
\pgfpathcurveto{\pgfqpoint{0.666667in}{1.052500in}}{\pgfqpoint{0.680590in}{1.052500in}}{\pgfqpoint{0.693945in}{1.046968in}}%
\pgfpathcurveto{\pgfqpoint{0.703790in}{1.037123in}}{\pgfqpoint{0.713635in}{1.027278in}}{\pgfqpoint{0.719167in}{1.013923in}}%
\pgfpathcurveto{\pgfqpoint{0.719167in}{1.000000in}}{\pgfqpoint{0.719167in}{0.986077in}}{\pgfqpoint{0.713635in}{0.972722in}}%
\pgfpathcurveto{\pgfqpoint{0.703790in}{0.962877in}}{\pgfqpoint{0.693945in}{0.953032in}}{\pgfqpoint{0.680590in}{0.947500in}}%
\pgfpathclose%
\pgfpathmoveto{\pgfqpoint{0.833333in}{0.941667in}}%
\pgfpathcurveto{\pgfqpoint{0.848804in}{0.941667in}}{\pgfqpoint{0.863642in}{0.947813in}}{\pgfqpoint{0.874581in}{0.958752in}}%
\pgfpathcurveto{\pgfqpoint{0.885520in}{0.969691in}}{\pgfqpoint{0.891667in}{0.984530in}}{\pgfqpoint{0.891667in}{1.000000in}}%
\pgfpathcurveto{\pgfqpoint{0.891667in}{1.015470in}}{\pgfqpoint{0.885520in}{1.030309in}}{\pgfqpoint{0.874581in}{1.041248in}}%
\pgfpathcurveto{\pgfqpoint{0.863642in}{1.052187in}}{\pgfqpoint{0.848804in}{1.058333in}}{\pgfqpoint{0.833333in}{1.058333in}}%
\pgfpathcurveto{\pgfqpoint{0.817863in}{1.058333in}}{\pgfqpoint{0.803025in}{1.052187in}}{\pgfqpoint{0.792085in}{1.041248in}}%
\pgfpathcurveto{\pgfqpoint{0.781146in}{1.030309in}}{\pgfqpoint{0.775000in}{1.015470in}}{\pgfqpoint{0.775000in}{1.000000in}}%
\pgfpathcurveto{\pgfqpoint{0.775000in}{0.984530in}}{\pgfqpoint{0.781146in}{0.969691in}}{\pgfqpoint{0.792085in}{0.958752in}}%
\pgfpathcurveto{\pgfqpoint{0.803025in}{0.947813in}}{\pgfqpoint{0.817863in}{0.941667in}}{\pgfqpoint{0.833333in}{0.941667in}}%
\pgfpathclose%
\pgfpathmoveto{\pgfqpoint{0.833333in}{0.947500in}}%
\pgfpathcurveto{\pgfqpoint{0.833333in}{0.947500in}}{\pgfqpoint{0.819410in}{0.947500in}}{\pgfqpoint{0.806055in}{0.953032in}}%
\pgfpathcurveto{\pgfqpoint{0.796210in}{0.962877in}}{\pgfqpoint{0.786365in}{0.972722in}}{\pgfqpoint{0.780833in}{0.986077in}}%
\pgfpathcurveto{\pgfqpoint{0.780833in}{1.000000in}}{\pgfqpoint{0.780833in}{1.013923in}}{\pgfqpoint{0.786365in}{1.027278in}}%
\pgfpathcurveto{\pgfqpoint{0.796210in}{1.037123in}}{\pgfqpoint{0.806055in}{1.046968in}}{\pgfqpoint{0.819410in}{1.052500in}}%
\pgfpathcurveto{\pgfqpoint{0.833333in}{1.052500in}}{\pgfqpoint{0.847256in}{1.052500in}}{\pgfqpoint{0.860611in}{1.046968in}}%
\pgfpathcurveto{\pgfqpoint{0.870456in}{1.037123in}}{\pgfqpoint{0.880302in}{1.027278in}}{\pgfqpoint{0.885833in}{1.013923in}}%
\pgfpathcurveto{\pgfqpoint{0.885833in}{1.000000in}}{\pgfqpoint{0.885833in}{0.986077in}}{\pgfqpoint{0.880302in}{0.972722in}}%
\pgfpathcurveto{\pgfqpoint{0.870456in}{0.962877in}}{\pgfqpoint{0.860611in}{0.953032in}}{\pgfqpoint{0.847256in}{0.947500in}}%
\pgfpathclose%
\pgfpathmoveto{\pgfqpoint{1.000000in}{0.941667in}}%
\pgfpathcurveto{\pgfqpoint{1.015470in}{0.941667in}}{\pgfqpoint{1.030309in}{0.947813in}}{\pgfqpoint{1.041248in}{0.958752in}}%
\pgfpathcurveto{\pgfqpoint{1.052187in}{0.969691in}}{\pgfqpoint{1.058333in}{0.984530in}}{\pgfqpoint{1.058333in}{1.000000in}}%
\pgfpathcurveto{\pgfqpoint{1.058333in}{1.015470in}}{\pgfqpoint{1.052187in}{1.030309in}}{\pgfqpoint{1.041248in}{1.041248in}}%
\pgfpathcurveto{\pgfqpoint{1.030309in}{1.052187in}}{\pgfqpoint{1.015470in}{1.058333in}}{\pgfqpoint{1.000000in}{1.058333in}}%
\pgfpathcurveto{\pgfqpoint{0.984530in}{1.058333in}}{\pgfqpoint{0.969691in}{1.052187in}}{\pgfqpoint{0.958752in}{1.041248in}}%
\pgfpathcurveto{\pgfqpoint{0.947813in}{1.030309in}}{\pgfqpoint{0.941667in}{1.015470in}}{\pgfqpoint{0.941667in}{1.000000in}}%
\pgfpathcurveto{\pgfqpoint{0.941667in}{0.984530in}}{\pgfqpoint{0.947813in}{0.969691in}}{\pgfqpoint{0.958752in}{0.958752in}}%
\pgfpathcurveto{\pgfqpoint{0.969691in}{0.947813in}}{\pgfqpoint{0.984530in}{0.941667in}}{\pgfqpoint{1.000000in}{0.941667in}}%
\pgfpathclose%
\pgfpathmoveto{\pgfqpoint{1.000000in}{0.947500in}}%
\pgfpathcurveto{\pgfqpoint{1.000000in}{0.947500in}}{\pgfqpoint{0.986077in}{0.947500in}}{\pgfqpoint{0.972722in}{0.953032in}}%
\pgfpathcurveto{\pgfqpoint{0.962877in}{0.962877in}}{\pgfqpoint{0.953032in}{0.972722in}}{\pgfqpoint{0.947500in}{0.986077in}}%
\pgfpathcurveto{\pgfqpoint{0.947500in}{1.000000in}}{\pgfqpoint{0.947500in}{1.013923in}}{\pgfqpoint{0.953032in}{1.027278in}}%
\pgfpathcurveto{\pgfqpoint{0.962877in}{1.037123in}}{\pgfqpoint{0.972722in}{1.046968in}}{\pgfqpoint{0.986077in}{1.052500in}}%
\pgfpathcurveto{\pgfqpoint{1.000000in}{1.052500in}}{\pgfqpoint{1.013923in}{1.052500in}}{\pgfqpoint{1.027278in}{1.046968in}}%
\pgfpathcurveto{\pgfqpoint{1.037123in}{1.037123in}}{\pgfqpoint{1.046968in}{1.027278in}}{\pgfqpoint{1.052500in}{1.013923in}}%
\pgfpathcurveto{\pgfqpoint{1.052500in}{1.000000in}}{\pgfqpoint{1.052500in}{0.986077in}}{\pgfqpoint{1.046968in}{0.972722in}}%
\pgfpathcurveto{\pgfqpoint{1.037123in}{0.962877in}}{\pgfqpoint{1.027278in}{0.953032in}}{\pgfqpoint{1.013923in}{0.947500in}}%
\pgfpathclose%
\pgfusepath{stroke}%
\end{pgfscope}%
}%
\pgfsys@transformshift{7.458038in}{1.058777in}%
\pgfsys@useobject{currentpattern}{}%
\pgfsys@transformshift{1in}{0in}%
\pgfsys@transformshift{-1in}{0in}%
\pgfsys@transformshift{0in}{1in}%
\pgfsys@useobject{currentpattern}{}%
\pgfsys@transformshift{1in}{0in}%
\pgfsys@transformshift{-1in}{0in}%
\pgfsys@transformshift{0in}{1in}%
\pgfsys@useobject{currentpattern}{}%
\pgfsys@transformshift{1in}{0in}%
\pgfsys@transformshift{-1in}{0in}%
\pgfsys@transformshift{0in}{1in}%
\end{pgfscope}%
\begin{pgfscope}%
\pgfpathrectangle{\pgfqpoint{0.870538in}{0.637495in}}{\pgfqpoint{9.300000in}{9.060000in}}%
\pgfusepath{clip}%
\pgfsetbuttcap%
\pgfsetmiterjoin%
\definecolor{currentfill}{rgb}{0.549020,0.337255,0.294118}%
\pgfsetfillcolor{currentfill}%
\pgfsetfillopacity{0.990000}%
\pgfsetlinewidth{0.000000pt}%
\definecolor{currentstroke}{rgb}{0.000000,0.000000,0.000000}%
\pgfsetstrokecolor{currentstroke}%
\pgfsetstrokeopacity{0.990000}%
\pgfsetdash{}{0pt}%
\pgfpathmoveto{\pgfqpoint{9.008038in}{0.637495in}}%
\pgfpathlineto{\pgfqpoint{9.783038in}{0.637495in}}%
\pgfpathlineto{\pgfqpoint{9.783038in}{3.751700in}}%
\pgfpathlineto{\pgfqpoint{9.008038in}{3.751700in}}%
\pgfpathclose%
\pgfusepath{fill}%
\end{pgfscope}%
\begin{pgfscope}%
\pgfsetbuttcap%
\pgfsetmiterjoin%
\definecolor{currentfill}{rgb}{0.549020,0.337255,0.294118}%
\pgfsetfillcolor{currentfill}%
\pgfsetfillopacity{0.990000}%
\pgfsetlinewidth{0.000000pt}%
\definecolor{currentstroke}{rgb}{0.000000,0.000000,0.000000}%
\pgfsetstrokecolor{currentstroke}%
\pgfsetstrokeopacity{0.990000}%
\pgfsetdash{}{0pt}%
\pgfpathrectangle{\pgfqpoint{0.870538in}{0.637495in}}{\pgfqpoint{9.300000in}{9.060000in}}%
\pgfusepath{clip}%
\pgfpathmoveto{\pgfqpoint{9.008038in}{0.637495in}}%
\pgfpathlineto{\pgfqpoint{9.783038in}{0.637495in}}%
\pgfpathlineto{\pgfqpoint{9.783038in}{3.751700in}}%
\pgfpathlineto{\pgfqpoint{9.008038in}{3.751700in}}%
\pgfpathclose%
\pgfusepath{clip}%
\pgfsys@defobject{currentpattern}{\pgfqpoint{0in}{0in}}{\pgfqpoint{1in}{1in}}{%
\begin{pgfscope}%
\pgfpathrectangle{\pgfqpoint{0in}{0in}}{\pgfqpoint{1in}{1in}}%
\pgfusepath{clip}%
\pgfpathmoveto{\pgfqpoint{0.000000in}{-0.058333in}}%
\pgfpathcurveto{\pgfqpoint{0.015470in}{-0.058333in}}{\pgfqpoint{0.030309in}{-0.052187in}}{\pgfqpoint{0.041248in}{-0.041248in}}%
\pgfpathcurveto{\pgfqpoint{0.052187in}{-0.030309in}}{\pgfqpoint{0.058333in}{-0.015470in}}{\pgfqpoint{0.058333in}{0.000000in}}%
\pgfpathcurveto{\pgfqpoint{0.058333in}{0.015470in}}{\pgfqpoint{0.052187in}{0.030309in}}{\pgfqpoint{0.041248in}{0.041248in}}%
\pgfpathcurveto{\pgfqpoint{0.030309in}{0.052187in}}{\pgfqpoint{0.015470in}{0.058333in}}{\pgfqpoint{0.000000in}{0.058333in}}%
\pgfpathcurveto{\pgfqpoint{-0.015470in}{0.058333in}}{\pgfqpoint{-0.030309in}{0.052187in}}{\pgfqpoint{-0.041248in}{0.041248in}}%
\pgfpathcurveto{\pgfqpoint{-0.052187in}{0.030309in}}{\pgfqpoint{-0.058333in}{0.015470in}}{\pgfqpoint{-0.058333in}{0.000000in}}%
\pgfpathcurveto{\pgfqpoint{-0.058333in}{-0.015470in}}{\pgfqpoint{-0.052187in}{-0.030309in}}{\pgfqpoint{-0.041248in}{-0.041248in}}%
\pgfpathcurveto{\pgfqpoint{-0.030309in}{-0.052187in}}{\pgfqpoint{-0.015470in}{-0.058333in}}{\pgfqpoint{0.000000in}{-0.058333in}}%
\pgfpathclose%
\pgfpathmoveto{\pgfqpoint{0.000000in}{-0.052500in}}%
\pgfpathcurveto{\pgfqpoint{0.000000in}{-0.052500in}}{\pgfqpoint{-0.013923in}{-0.052500in}}{\pgfqpoint{-0.027278in}{-0.046968in}}%
\pgfpathcurveto{\pgfqpoint{-0.037123in}{-0.037123in}}{\pgfqpoint{-0.046968in}{-0.027278in}}{\pgfqpoint{-0.052500in}{-0.013923in}}%
\pgfpathcurveto{\pgfqpoint{-0.052500in}{0.000000in}}{\pgfqpoint{-0.052500in}{0.013923in}}{\pgfqpoint{-0.046968in}{0.027278in}}%
\pgfpathcurveto{\pgfqpoint{-0.037123in}{0.037123in}}{\pgfqpoint{-0.027278in}{0.046968in}}{\pgfqpoint{-0.013923in}{0.052500in}}%
\pgfpathcurveto{\pgfqpoint{0.000000in}{0.052500in}}{\pgfqpoint{0.013923in}{0.052500in}}{\pgfqpoint{0.027278in}{0.046968in}}%
\pgfpathcurveto{\pgfqpoint{0.037123in}{0.037123in}}{\pgfqpoint{0.046968in}{0.027278in}}{\pgfqpoint{0.052500in}{0.013923in}}%
\pgfpathcurveto{\pgfqpoint{0.052500in}{0.000000in}}{\pgfqpoint{0.052500in}{-0.013923in}}{\pgfqpoint{0.046968in}{-0.027278in}}%
\pgfpathcurveto{\pgfqpoint{0.037123in}{-0.037123in}}{\pgfqpoint{0.027278in}{-0.046968in}}{\pgfqpoint{0.013923in}{-0.052500in}}%
\pgfpathclose%
\pgfpathmoveto{\pgfqpoint{0.166667in}{-0.058333in}}%
\pgfpathcurveto{\pgfqpoint{0.182137in}{-0.058333in}}{\pgfqpoint{0.196975in}{-0.052187in}}{\pgfqpoint{0.207915in}{-0.041248in}}%
\pgfpathcurveto{\pgfqpoint{0.218854in}{-0.030309in}}{\pgfqpoint{0.225000in}{-0.015470in}}{\pgfqpoint{0.225000in}{0.000000in}}%
\pgfpathcurveto{\pgfqpoint{0.225000in}{0.015470in}}{\pgfqpoint{0.218854in}{0.030309in}}{\pgfqpoint{0.207915in}{0.041248in}}%
\pgfpathcurveto{\pgfqpoint{0.196975in}{0.052187in}}{\pgfqpoint{0.182137in}{0.058333in}}{\pgfqpoint{0.166667in}{0.058333in}}%
\pgfpathcurveto{\pgfqpoint{0.151196in}{0.058333in}}{\pgfqpoint{0.136358in}{0.052187in}}{\pgfqpoint{0.125419in}{0.041248in}}%
\pgfpathcurveto{\pgfqpoint{0.114480in}{0.030309in}}{\pgfqpoint{0.108333in}{0.015470in}}{\pgfqpoint{0.108333in}{0.000000in}}%
\pgfpathcurveto{\pgfqpoint{0.108333in}{-0.015470in}}{\pgfqpoint{0.114480in}{-0.030309in}}{\pgfqpoint{0.125419in}{-0.041248in}}%
\pgfpathcurveto{\pgfqpoint{0.136358in}{-0.052187in}}{\pgfqpoint{0.151196in}{-0.058333in}}{\pgfqpoint{0.166667in}{-0.058333in}}%
\pgfpathclose%
\pgfpathmoveto{\pgfqpoint{0.166667in}{-0.052500in}}%
\pgfpathcurveto{\pgfqpoint{0.166667in}{-0.052500in}}{\pgfqpoint{0.152744in}{-0.052500in}}{\pgfqpoint{0.139389in}{-0.046968in}}%
\pgfpathcurveto{\pgfqpoint{0.129544in}{-0.037123in}}{\pgfqpoint{0.119698in}{-0.027278in}}{\pgfqpoint{0.114167in}{-0.013923in}}%
\pgfpathcurveto{\pgfqpoint{0.114167in}{0.000000in}}{\pgfqpoint{0.114167in}{0.013923in}}{\pgfqpoint{0.119698in}{0.027278in}}%
\pgfpathcurveto{\pgfqpoint{0.129544in}{0.037123in}}{\pgfqpoint{0.139389in}{0.046968in}}{\pgfqpoint{0.152744in}{0.052500in}}%
\pgfpathcurveto{\pgfqpoint{0.166667in}{0.052500in}}{\pgfqpoint{0.180590in}{0.052500in}}{\pgfqpoint{0.193945in}{0.046968in}}%
\pgfpathcurveto{\pgfqpoint{0.203790in}{0.037123in}}{\pgfqpoint{0.213635in}{0.027278in}}{\pgfqpoint{0.219167in}{0.013923in}}%
\pgfpathcurveto{\pgfqpoint{0.219167in}{0.000000in}}{\pgfqpoint{0.219167in}{-0.013923in}}{\pgfqpoint{0.213635in}{-0.027278in}}%
\pgfpathcurveto{\pgfqpoint{0.203790in}{-0.037123in}}{\pgfqpoint{0.193945in}{-0.046968in}}{\pgfqpoint{0.180590in}{-0.052500in}}%
\pgfpathclose%
\pgfpathmoveto{\pgfqpoint{0.333333in}{-0.058333in}}%
\pgfpathcurveto{\pgfqpoint{0.348804in}{-0.058333in}}{\pgfqpoint{0.363642in}{-0.052187in}}{\pgfqpoint{0.374581in}{-0.041248in}}%
\pgfpathcurveto{\pgfqpoint{0.385520in}{-0.030309in}}{\pgfqpoint{0.391667in}{-0.015470in}}{\pgfqpoint{0.391667in}{0.000000in}}%
\pgfpathcurveto{\pgfqpoint{0.391667in}{0.015470in}}{\pgfqpoint{0.385520in}{0.030309in}}{\pgfqpoint{0.374581in}{0.041248in}}%
\pgfpathcurveto{\pgfqpoint{0.363642in}{0.052187in}}{\pgfqpoint{0.348804in}{0.058333in}}{\pgfqpoint{0.333333in}{0.058333in}}%
\pgfpathcurveto{\pgfqpoint{0.317863in}{0.058333in}}{\pgfqpoint{0.303025in}{0.052187in}}{\pgfqpoint{0.292085in}{0.041248in}}%
\pgfpathcurveto{\pgfqpoint{0.281146in}{0.030309in}}{\pgfqpoint{0.275000in}{0.015470in}}{\pgfqpoint{0.275000in}{0.000000in}}%
\pgfpathcurveto{\pgfqpoint{0.275000in}{-0.015470in}}{\pgfqpoint{0.281146in}{-0.030309in}}{\pgfqpoint{0.292085in}{-0.041248in}}%
\pgfpathcurveto{\pgfqpoint{0.303025in}{-0.052187in}}{\pgfqpoint{0.317863in}{-0.058333in}}{\pgfqpoint{0.333333in}{-0.058333in}}%
\pgfpathclose%
\pgfpathmoveto{\pgfqpoint{0.333333in}{-0.052500in}}%
\pgfpathcurveto{\pgfqpoint{0.333333in}{-0.052500in}}{\pgfqpoint{0.319410in}{-0.052500in}}{\pgfqpoint{0.306055in}{-0.046968in}}%
\pgfpathcurveto{\pgfqpoint{0.296210in}{-0.037123in}}{\pgfqpoint{0.286365in}{-0.027278in}}{\pgfqpoint{0.280833in}{-0.013923in}}%
\pgfpathcurveto{\pgfqpoint{0.280833in}{0.000000in}}{\pgfqpoint{0.280833in}{0.013923in}}{\pgfqpoint{0.286365in}{0.027278in}}%
\pgfpathcurveto{\pgfqpoint{0.296210in}{0.037123in}}{\pgfqpoint{0.306055in}{0.046968in}}{\pgfqpoint{0.319410in}{0.052500in}}%
\pgfpathcurveto{\pgfqpoint{0.333333in}{0.052500in}}{\pgfqpoint{0.347256in}{0.052500in}}{\pgfqpoint{0.360611in}{0.046968in}}%
\pgfpathcurveto{\pgfqpoint{0.370456in}{0.037123in}}{\pgfqpoint{0.380302in}{0.027278in}}{\pgfqpoint{0.385833in}{0.013923in}}%
\pgfpathcurveto{\pgfqpoint{0.385833in}{0.000000in}}{\pgfqpoint{0.385833in}{-0.013923in}}{\pgfqpoint{0.380302in}{-0.027278in}}%
\pgfpathcurveto{\pgfqpoint{0.370456in}{-0.037123in}}{\pgfqpoint{0.360611in}{-0.046968in}}{\pgfqpoint{0.347256in}{-0.052500in}}%
\pgfpathclose%
\pgfpathmoveto{\pgfqpoint{0.500000in}{-0.058333in}}%
\pgfpathcurveto{\pgfqpoint{0.515470in}{-0.058333in}}{\pgfqpoint{0.530309in}{-0.052187in}}{\pgfqpoint{0.541248in}{-0.041248in}}%
\pgfpathcurveto{\pgfqpoint{0.552187in}{-0.030309in}}{\pgfqpoint{0.558333in}{-0.015470in}}{\pgfqpoint{0.558333in}{0.000000in}}%
\pgfpathcurveto{\pgfqpoint{0.558333in}{0.015470in}}{\pgfqpoint{0.552187in}{0.030309in}}{\pgfqpoint{0.541248in}{0.041248in}}%
\pgfpathcurveto{\pgfqpoint{0.530309in}{0.052187in}}{\pgfqpoint{0.515470in}{0.058333in}}{\pgfqpoint{0.500000in}{0.058333in}}%
\pgfpathcurveto{\pgfqpoint{0.484530in}{0.058333in}}{\pgfqpoint{0.469691in}{0.052187in}}{\pgfqpoint{0.458752in}{0.041248in}}%
\pgfpathcurveto{\pgfqpoint{0.447813in}{0.030309in}}{\pgfqpoint{0.441667in}{0.015470in}}{\pgfqpoint{0.441667in}{0.000000in}}%
\pgfpathcurveto{\pgfqpoint{0.441667in}{-0.015470in}}{\pgfqpoint{0.447813in}{-0.030309in}}{\pgfqpoint{0.458752in}{-0.041248in}}%
\pgfpathcurveto{\pgfqpoint{0.469691in}{-0.052187in}}{\pgfqpoint{0.484530in}{-0.058333in}}{\pgfqpoint{0.500000in}{-0.058333in}}%
\pgfpathclose%
\pgfpathmoveto{\pgfqpoint{0.500000in}{-0.052500in}}%
\pgfpathcurveto{\pgfqpoint{0.500000in}{-0.052500in}}{\pgfqpoint{0.486077in}{-0.052500in}}{\pgfqpoint{0.472722in}{-0.046968in}}%
\pgfpathcurveto{\pgfqpoint{0.462877in}{-0.037123in}}{\pgfqpoint{0.453032in}{-0.027278in}}{\pgfqpoint{0.447500in}{-0.013923in}}%
\pgfpathcurveto{\pgfqpoint{0.447500in}{0.000000in}}{\pgfqpoint{0.447500in}{0.013923in}}{\pgfqpoint{0.453032in}{0.027278in}}%
\pgfpathcurveto{\pgfqpoint{0.462877in}{0.037123in}}{\pgfqpoint{0.472722in}{0.046968in}}{\pgfqpoint{0.486077in}{0.052500in}}%
\pgfpathcurveto{\pgfqpoint{0.500000in}{0.052500in}}{\pgfqpoint{0.513923in}{0.052500in}}{\pgfqpoint{0.527278in}{0.046968in}}%
\pgfpathcurveto{\pgfqpoint{0.537123in}{0.037123in}}{\pgfqpoint{0.546968in}{0.027278in}}{\pgfqpoint{0.552500in}{0.013923in}}%
\pgfpathcurveto{\pgfqpoint{0.552500in}{0.000000in}}{\pgfqpoint{0.552500in}{-0.013923in}}{\pgfqpoint{0.546968in}{-0.027278in}}%
\pgfpathcurveto{\pgfqpoint{0.537123in}{-0.037123in}}{\pgfqpoint{0.527278in}{-0.046968in}}{\pgfqpoint{0.513923in}{-0.052500in}}%
\pgfpathclose%
\pgfpathmoveto{\pgfqpoint{0.666667in}{-0.058333in}}%
\pgfpathcurveto{\pgfqpoint{0.682137in}{-0.058333in}}{\pgfqpoint{0.696975in}{-0.052187in}}{\pgfqpoint{0.707915in}{-0.041248in}}%
\pgfpathcurveto{\pgfqpoint{0.718854in}{-0.030309in}}{\pgfqpoint{0.725000in}{-0.015470in}}{\pgfqpoint{0.725000in}{0.000000in}}%
\pgfpathcurveto{\pgfqpoint{0.725000in}{0.015470in}}{\pgfqpoint{0.718854in}{0.030309in}}{\pgfqpoint{0.707915in}{0.041248in}}%
\pgfpathcurveto{\pgfqpoint{0.696975in}{0.052187in}}{\pgfqpoint{0.682137in}{0.058333in}}{\pgfqpoint{0.666667in}{0.058333in}}%
\pgfpathcurveto{\pgfqpoint{0.651196in}{0.058333in}}{\pgfqpoint{0.636358in}{0.052187in}}{\pgfqpoint{0.625419in}{0.041248in}}%
\pgfpathcurveto{\pgfqpoint{0.614480in}{0.030309in}}{\pgfqpoint{0.608333in}{0.015470in}}{\pgfqpoint{0.608333in}{0.000000in}}%
\pgfpathcurveto{\pgfqpoint{0.608333in}{-0.015470in}}{\pgfqpoint{0.614480in}{-0.030309in}}{\pgfqpoint{0.625419in}{-0.041248in}}%
\pgfpathcurveto{\pgfqpoint{0.636358in}{-0.052187in}}{\pgfqpoint{0.651196in}{-0.058333in}}{\pgfqpoint{0.666667in}{-0.058333in}}%
\pgfpathclose%
\pgfpathmoveto{\pgfqpoint{0.666667in}{-0.052500in}}%
\pgfpathcurveto{\pgfqpoint{0.666667in}{-0.052500in}}{\pgfqpoint{0.652744in}{-0.052500in}}{\pgfqpoint{0.639389in}{-0.046968in}}%
\pgfpathcurveto{\pgfqpoint{0.629544in}{-0.037123in}}{\pgfqpoint{0.619698in}{-0.027278in}}{\pgfqpoint{0.614167in}{-0.013923in}}%
\pgfpathcurveto{\pgfqpoint{0.614167in}{0.000000in}}{\pgfqpoint{0.614167in}{0.013923in}}{\pgfqpoint{0.619698in}{0.027278in}}%
\pgfpathcurveto{\pgfqpoint{0.629544in}{0.037123in}}{\pgfqpoint{0.639389in}{0.046968in}}{\pgfqpoint{0.652744in}{0.052500in}}%
\pgfpathcurveto{\pgfqpoint{0.666667in}{0.052500in}}{\pgfqpoint{0.680590in}{0.052500in}}{\pgfqpoint{0.693945in}{0.046968in}}%
\pgfpathcurveto{\pgfqpoint{0.703790in}{0.037123in}}{\pgfqpoint{0.713635in}{0.027278in}}{\pgfqpoint{0.719167in}{0.013923in}}%
\pgfpathcurveto{\pgfqpoint{0.719167in}{0.000000in}}{\pgfqpoint{0.719167in}{-0.013923in}}{\pgfqpoint{0.713635in}{-0.027278in}}%
\pgfpathcurveto{\pgfqpoint{0.703790in}{-0.037123in}}{\pgfqpoint{0.693945in}{-0.046968in}}{\pgfqpoint{0.680590in}{-0.052500in}}%
\pgfpathclose%
\pgfpathmoveto{\pgfqpoint{0.833333in}{-0.058333in}}%
\pgfpathcurveto{\pgfqpoint{0.848804in}{-0.058333in}}{\pgfqpoint{0.863642in}{-0.052187in}}{\pgfqpoint{0.874581in}{-0.041248in}}%
\pgfpathcurveto{\pgfqpoint{0.885520in}{-0.030309in}}{\pgfqpoint{0.891667in}{-0.015470in}}{\pgfqpoint{0.891667in}{0.000000in}}%
\pgfpathcurveto{\pgfqpoint{0.891667in}{0.015470in}}{\pgfqpoint{0.885520in}{0.030309in}}{\pgfqpoint{0.874581in}{0.041248in}}%
\pgfpathcurveto{\pgfqpoint{0.863642in}{0.052187in}}{\pgfqpoint{0.848804in}{0.058333in}}{\pgfqpoint{0.833333in}{0.058333in}}%
\pgfpathcurveto{\pgfqpoint{0.817863in}{0.058333in}}{\pgfqpoint{0.803025in}{0.052187in}}{\pgfqpoint{0.792085in}{0.041248in}}%
\pgfpathcurveto{\pgfqpoint{0.781146in}{0.030309in}}{\pgfqpoint{0.775000in}{0.015470in}}{\pgfqpoint{0.775000in}{0.000000in}}%
\pgfpathcurveto{\pgfqpoint{0.775000in}{-0.015470in}}{\pgfqpoint{0.781146in}{-0.030309in}}{\pgfqpoint{0.792085in}{-0.041248in}}%
\pgfpathcurveto{\pgfqpoint{0.803025in}{-0.052187in}}{\pgfqpoint{0.817863in}{-0.058333in}}{\pgfqpoint{0.833333in}{-0.058333in}}%
\pgfpathclose%
\pgfpathmoveto{\pgfqpoint{0.833333in}{-0.052500in}}%
\pgfpathcurveto{\pgfqpoint{0.833333in}{-0.052500in}}{\pgfqpoint{0.819410in}{-0.052500in}}{\pgfqpoint{0.806055in}{-0.046968in}}%
\pgfpathcurveto{\pgfqpoint{0.796210in}{-0.037123in}}{\pgfqpoint{0.786365in}{-0.027278in}}{\pgfqpoint{0.780833in}{-0.013923in}}%
\pgfpathcurveto{\pgfqpoint{0.780833in}{0.000000in}}{\pgfqpoint{0.780833in}{0.013923in}}{\pgfqpoint{0.786365in}{0.027278in}}%
\pgfpathcurveto{\pgfqpoint{0.796210in}{0.037123in}}{\pgfqpoint{0.806055in}{0.046968in}}{\pgfqpoint{0.819410in}{0.052500in}}%
\pgfpathcurveto{\pgfqpoint{0.833333in}{0.052500in}}{\pgfqpoint{0.847256in}{0.052500in}}{\pgfqpoint{0.860611in}{0.046968in}}%
\pgfpathcurveto{\pgfqpoint{0.870456in}{0.037123in}}{\pgfqpoint{0.880302in}{0.027278in}}{\pgfqpoint{0.885833in}{0.013923in}}%
\pgfpathcurveto{\pgfqpoint{0.885833in}{0.000000in}}{\pgfqpoint{0.885833in}{-0.013923in}}{\pgfqpoint{0.880302in}{-0.027278in}}%
\pgfpathcurveto{\pgfqpoint{0.870456in}{-0.037123in}}{\pgfqpoint{0.860611in}{-0.046968in}}{\pgfqpoint{0.847256in}{-0.052500in}}%
\pgfpathclose%
\pgfpathmoveto{\pgfqpoint{1.000000in}{-0.058333in}}%
\pgfpathcurveto{\pgfqpoint{1.015470in}{-0.058333in}}{\pgfqpoint{1.030309in}{-0.052187in}}{\pgfqpoint{1.041248in}{-0.041248in}}%
\pgfpathcurveto{\pgfqpoint{1.052187in}{-0.030309in}}{\pgfqpoint{1.058333in}{-0.015470in}}{\pgfqpoint{1.058333in}{0.000000in}}%
\pgfpathcurveto{\pgfqpoint{1.058333in}{0.015470in}}{\pgfqpoint{1.052187in}{0.030309in}}{\pgfqpoint{1.041248in}{0.041248in}}%
\pgfpathcurveto{\pgfqpoint{1.030309in}{0.052187in}}{\pgfqpoint{1.015470in}{0.058333in}}{\pgfqpoint{1.000000in}{0.058333in}}%
\pgfpathcurveto{\pgfqpoint{0.984530in}{0.058333in}}{\pgfqpoint{0.969691in}{0.052187in}}{\pgfqpoint{0.958752in}{0.041248in}}%
\pgfpathcurveto{\pgfqpoint{0.947813in}{0.030309in}}{\pgfqpoint{0.941667in}{0.015470in}}{\pgfqpoint{0.941667in}{0.000000in}}%
\pgfpathcurveto{\pgfqpoint{0.941667in}{-0.015470in}}{\pgfqpoint{0.947813in}{-0.030309in}}{\pgfqpoint{0.958752in}{-0.041248in}}%
\pgfpathcurveto{\pgfqpoint{0.969691in}{-0.052187in}}{\pgfqpoint{0.984530in}{-0.058333in}}{\pgfqpoint{1.000000in}{-0.058333in}}%
\pgfpathclose%
\pgfpathmoveto{\pgfqpoint{1.000000in}{-0.052500in}}%
\pgfpathcurveto{\pgfqpoint{1.000000in}{-0.052500in}}{\pgfqpoint{0.986077in}{-0.052500in}}{\pgfqpoint{0.972722in}{-0.046968in}}%
\pgfpathcurveto{\pgfqpoint{0.962877in}{-0.037123in}}{\pgfqpoint{0.953032in}{-0.027278in}}{\pgfqpoint{0.947500in}{-0.013923in}}%
\pgfpathcurveto{\pgfqpoint{0.947500in}{0.000000in}}{\pgfqpoint{0.947500in}{0.013923in}}{\pgfqpoint{0.953032in}{0.027278in}}%
\pgfpathcurveto{\pgfqpoint{0.962877in}{0.037123in}}{\pgfqpoint{0.972722in}{0.046968in}}{\pgfqpoint{0.986077in}{0.052500in}}%
\pgfpathcurveto{\pgfqpoint{1.000000in}{0.052500in}}{\pgfqpoint{1.013923in}{0.052500in}}{\pgfqpoint{1.027278in}{0.046968in}}%
\pgfpathcurveto{\pgfqpoint{1.037123in}{0.037123in}}{\pgfqpoint{1.046968in}{0.027278in}}{\pgfqpoint{1.052500in}{0.013923in}}%
\pgfpathcurveto{\pgfqpoint{1.052500in}{0.000000in}}{\pgfqpoint{1.052500in}{-0.013923in}}{\pgfqpoint{1.046968in}{-0.027278in}}%
\pgfpathcurveto{\pgfqpoint{1.037123in}{-0.037123in}}{\pgfqpoint{1.027278in}{-0.046968in}}{\pgfqpoint{1.013923in}{-0.052500in}}%
\pgfpathclose%
\pgfpathmoveto{\pgfqpoint{0.083333in}{0.108333in}}%
\pgfpathcurveto{\pgfqpoint{0.098804in}{0.108333in}}{\pgfqpoint{0.113642in}{0.114480in}}{\pgfqpoint{0.124581in}{0.125419in}}%
\pgfpathcurveto{\pgfqpoint{0.135520in}{0.136358in}}{\pgfqpoint{0.141667in}{0.151196in}}{\pgfqpoint{0.141667in}{0.166667in}}%
\pgfpathcurveto{\pgfqpoint{0.141667in}{0.182137in}}{\pgfqpoint{0.135520in}{0.196975in}}{\pgfqpoint{0.124581in}{0.207915in}}%
\pgfpathcurveto{\pgfqpoint{0.113642in}{0.218854in}}{\pgfqpoint{0.098804in}{0.225000in}}{\pgfqpoint{0.083333in}{0.225000in}}%
\pgfpathcurveto{\pgfqpoint{0.067863in}{0.225000in}}{\pgfqpoint{0.053025in}{0.218854in}}{\pgfqpoint{0.042085in}{0.207915in}}%
\pgfpathcurveto{\pgfqpoint{0.031146in}{0.196975in}}{\pgfqpoint{0.025000in}{0.182137in}}{\pgfqpoint{0.025000in}{0.166667in}}%
\pgfpathcurveto{\pgfqpoint{0.025000in}{0.151196in}}{\pgfqpoint{0.031146in}{0.136358in}}{\pgfqpoint{0.042085in}{0.125419in}}%
\pgfpathcurveto{\pgfqpoint{0.053025in}{0.114480in}}{\pgfqpoint{0.067863in}{0.108333in}}{\pgfqpoint{0.083333in}{0.108333in}}%
\pgfpathclose%
\pgfpathmoveto{\pgfqpoint{0.083333in}{0.114167in}}%
\pgfpathcurveto{\pgfqpoint{0.083333in}{0.114167in}}{\pgfqpoint{0.069410in}{0.114167in}}{\pgfqpoint{0.056055in}{0.119698in}}%
\pgfpathcurveto{\pgfqpoint{0.046210in}{0.129544in}}{\pgfqpoint{0.036365in}{0.139389in}}{\pgfqpoint{0.030833in}{0.152744in}}%
\pgfpathcurveto{\pgfqpoint{0.030833in}{0.166667in}}{\pgfqpoint{0.030833in}{0.180590in}}{\pgfqpoint{0.036365in}{0.193945in}}%
\pgfpathcurveto{\pgfqpoint{0.046210in}{0.203790in}}{\pgfqpoint{0.056055in}{0.213635in}}{\pgfqpoint{0.069410in}{0.219167in}}%
\pgfpathcurveto{\pgfqpoint{0.083333in}{0.219167in}}{\pgfqpoint{0.097256in}{0.219167in}}{\pgfqpoint{0.110611in}{0.213635in}}%
\pgfpathcurveto{\pgfqpoint{0.120456in}{0.203790in}}{\pgfqpoint{0.130302in}{0.193945in}}{\pgfqpoint{0.135833in}{0.180590in}}%
\pgfpathcurveto{\pgfqpoint{0.135833in}{0.166667in}}{\pgfqpoint{0.135833in}{0.152744in}}{\pgfqpoint{0.130302in}{0.139389in}}%
\pgfpathcurveto{\pgfqpoint{0.120456in}{0.129544in}}{\pgfqpoint{0.110611in}{0.119698in}}{\pgfqpoint{0.097256in}{0.114167in}}%
\pgfpathclose%
\pgfpathmoveto{\pgfqpoint{0.250000in}{0.108333in}}%
\pgfpathcurveto{\pgfqpoint{0.265470in}{0.108333in}}{\pgfqpoint{0.280309in}{0.114480in}}{\pgfqpoint{0.291248in}{0.125419in}}%
\pgfpathcurveto{\pgfqpoint{0.302187in}{0.136358in}}{\pgfqpoint{0.308333in}{0.151196in}}{\pgfqpoint{0.308333in}{0.166667in}}%
\pgfpathcurveto{\pgfqpoint{0.308333in}{0.182137in}}{\pgfqpoint{0.302187in}{0.196975in}}{\pgfqpoint{0.291248in}{0.207915in}}%
\pgfpathcurveto{\pgfqpoint{0.280309in}{0.218854in}}{\pgfqpoint{0.265470in}{0.225000in}}{\pgfqpoint{0.250000in}{0.225000in}}%
\pgfpathcurveto{\pgfqpoint{0.234530in}{0.225000in}}{\pgfqpoint{0.219691in}{0.218854in}}{\pgfqpoint{0.208752in}{0.207915in}}%
\pgfpathcurveto{\pgfqpoint{0.197813in}{0.196975in}}{\pgfqpoint{0.191667in}{0.182137in}}{\pgfqpoint{0.191667in}{0.166667in}}%
\pgfpathcurveto{\pgfqpoint{0.191667in}{0.151196in}}{\pgfqpoint{0.197813in}{0.136358in}}{\pgfqpoint{0.208752in}{0.125419in}}%
\pgfpathcurveto{\pgfqpoint{0.219691in}{0.114480in}}{\pgfqpoint{0.234530in}{0.108333in}}{\pgfqpoint{0.250000in}{0.108333in}}%
\pgfpathclose%
\pgfpathmoveto{\pgfqpoint{0.250000in}{0.114167in}}%
\pgfpathcurveto{\pgfqpoint{0.250000in}{0.114167in}}{\pgfqpoint{0.236077in}{0.114167in}}{\pgfqpoint{0.222722in}{0.119698in}}%
\pgfpathcurveto{\pgfqpoint{0.212877in}{0.129544in}}{\pgfqpoint{0.203032in}{0.139389in}}{\pgfqpoint{0.197500in}{0.152744in}}%
\pgfpathcurveto{\pgfqpoint{0.197500in}{0.166667in}}{\pgfqpoint{0.197500in}{0.180590in}}{\pgfqpoint{0.203032in}{0.193945in}}%
\pgfpathcurveto{\pgfqpoint{0.212877in}{0.203790in}}{\pgfqpoint{0.222722in}{0.213635in}}{\pgfqpoint{0.236077in}{0.219167in}}%
\pgfpathcurveto{\pgfqpoint{0.250000in}{0.219167in}}{\pgfqpoint{0.263923in}{0.219167in}}{\pgfqpoint{0.277278in}{0.213635in}}%
\pgfpathcurveto{\pgfqpoint{0.287123in}{0.203790in}}{\pgfqpoint{0.296968in}{0.193945in}}{\pgfqpoint{0.302500in}{0.180590in}}%
\pgfpathcurveto{\pgfqpoint{0.302500in}{0.166667in}}{\pgfqpoint{0.302500in}{0.152744in}}{\pgfqpoint{0.296968in}{0.139389in}}%
\pgfpathcurveto{\pgfqpoint{0.287123in}{0.129544in}}{\pgfqpoint{0.277278in}{0.119698in}}{\pgfqpoint{0.263923in}{0.114167in}}%
\pgfpathclose%
\pgfpathmoveto{\pgfqpoint{0.416667in}{0.108333in}}%
\pgfpathcurveto{\pgfqpoint{0.432137in}{0.108333in}}{\pgfqpoint{0.446975in}{0.114480in}}{\pgfqpoint{0.457915in}{0.125419in}}%
\pgfpathcurveto{\pgfqpoint{0.468854in}{0.136358in}}{\pgfqpoint{0.475000in}{0.151196in}}{\pgfqpoint{0.475000in}{0.166667in}}%
\pgfpathcurveto{\pgfqpoint{0.475000in}{0.182137in}}{\pgfqpoint{0.468854in}{0.196975in}}{\pgfqpoint{0.457915in}{0.207915in}}%
\pgfpathcurveto{\pgfqpoint{0.446975in}{0.218854in}}{\pgfqpoint{0.432137in}{0.225000in}}{\pgfqpoint{0.416667in}{0.225000in}}%
\pgfpathcurveto{\pgfqpoint{0.401196in}{0.225000in}}{\pgfqpoint{0.386358in}{0.218854in}}{\pgfqpoint{0.375419in}{0.207915in}}%
\pgfpathcurveto{\pgfqpoint{0.364480in}{0.196975in}}{\pgfqpoint{0.358333in}{0.182137in}}{\pgfqpoint{0.358333in}{0.166667in}}%
\pgfpathcurveto{\pgfqpoint{0.358333in}{0.151196in}}{\pgfqpoint{0.364480in}{0.136358in}}{\pgfqpoint{0.375419in}{0.125419in}}%
\pgfpathcurveto{\pgfqpoint{0.386358in}{0.114480in}}{\pgfqpoint{0.401196in}{0.108333in}}{\pgfqpoint{0.416667in}{0.108333in}}%
\pgfpathclose%
\pgfpathmoveto{\pgfqpoint{0.416667in}{0.114167in}}%
\pgfpathcurveto{\pgfqpoint{0.416667in}{0.114167in}}{\pgfqpoint{0.402744in}{0.114167in}}{\pgfqpoint{0.389389in}{0.119698in}}%
\pgfpathcurveto{\pgfqpoint{0.379544in}{0.129544in}}{\pgfqpoint{0.369698in}{0.139389in}}{\pgfqpoint{0.364167in}{0.152744in}}%
\pgfpathcurveto{\pgfqpoint{0.364167in}{0.166667in}}{\pgfqpoint{0.364167in}{0.180590in}}{\pgfqpoint{0.369698in}{0.193945in}}%
\pgfpathcurveto{\pgfqpoint{0.379544in}{0.203790in}}{\pgfqpoint{0.389389in}{0.213635in}}{\pgfqpoint{0.402744in}{0.219167in}}%
\pgfpathcurveto{\pgfqpoint{0.416667in}{0.219167in}}{\pgfqpoint{0.430590in}{0.219167in}}{\pgfqpoint{0.443945in}{0.213635in}}%
\pgfpathcurveto{\pgfqpoint{0.453790in}{0.203790in}}{\pgfqpoint{0.463635in}{0.193945in}}{\pgfqpoint{0.469167in}{0.180590in}}%
\pgfpathcurveto{\pgfqpoint{0.469167in}{0.166667in}}{\pgfqpoint{0.469167in}{0.152744in}}{\pgfqpoint{0.463635in}{0.139389in}}%
\pgfpathcurveto{\pgfqpoint{0.453790in}{0.129544in}}{\pgfqpoint{0.443945in}{0.119698in}}{\pgfqpoint{0.430590in}{0.114167in}}%
\pgfpathclose%
\pgfpathmoveto{\pgfqpoint{0.583333in}{0.108333in}}%
\pgfpathcurveto{\pgfqpoint{0.598804in}{0.108333in}}{\pgfqpoint{0.613642in}{0.114480in}}{\pgfqpoint{0.624581in}{0.125419in}}%
\pgfpathcurveto{\pgfqpoint{0.635520in}{0.136358in}}{\pgfqpoint{0.641667in}{0.151196in}}{\pgfqpoint{0.641667in}{0.166667in}}%
\pgfpathcurveto{\pgfqpoint{0.641667in}{0.182137in}}{\pgfqpoint{0.635520in}{0.196975in}}{\pgfqpoint{0.624581in}{0.207915in}}%
\pgfpathcurveto{\pgfqpoint{0.613642in}{0.218854in}}{\pgfqpoint{0.598804in}{0.225000in}}{\pgfqpoint{0.583333in}{0.225000in}}%
\pgfpathcurveto{\pgfqpoint{0.567863in}{0.225000in}}{\pgfqpoint{0.553025in}{0.218854in}}{\pgfqpoint{0.542085in}{0.207915in}}%
\pgfpathcurveto{\pgfqpoint{0.531146in}{0.196975in}}{\pgfqpoint{0.525000in}{0.182137in}}{\pgfqpoint{0.525000in}{0.166667in}}%
\pgfpathcurveto{\pgfqpoint{0.525000in}{0.151196in}}{\pgfqpoint{0.531146in}{0.136358in}}{\pgfqpoint{0.542085in}{0.125419in}}%
\pgfpathcurveto{\pgfqpoint{0.553025in}{0.114480in}}{\pgfqpoint{0.567863in}{0.108333in}}{\pgfqpoint{0.583333in}{0.108333in}}%
\pgfpathclose%
\pgfpathmoveto{\pgfqpoint{0.583333in}{0.114167in}}%
\pgfpathcurveto{\pgfqpoint{0.583333in}{0.114167in}}{\pgfqpoint{0.569410in}{0.114167in}}{\pgfqpoint{0.556055in}{0.119698in}}%
\pgfpathcurveto{\pgfqpoint{0.546210in}{0.129544in}}{\pgfqpoint{0.536365in}{0.139389in}}{\pgfqpoint{0.530833in}{0.152744in}}%
\pgfpathcurveto{\pgfqpoint{0.530833in}{0.166667in}}{\pgfqpoint{0.530833in}{0.180590in}}{\pgfqpoint{0.536365in}{0.193945in}}%
\pgfpathcurveto{\pgfqpoint{0.546210in}{0.203790in}}{\pgfqpoint{0.556055in}{0.213635in}}{\pgfqpoint{0.569410in}{0.219167in}}%
\pgfpathcurveto{\pgfqpoint{0.583333in}{0.219167in}}{\pgfqpoint{0.597256in}{0.219167in}}{\pgfqpoint{0.610611in}{0.213635in}}%
\pgfpathcurveto{\pgfqpoint{0.620456in}{0.203790in}}{\pgfqpoint{0.630302in}{0.193945in}}{\pgfqpoint{0.635833in}{0.180590in}}%
\pgfpathcurveto{\pgfqpoint{0.635833in}{0.166667in}}{\pgfqpoint{0.635833in}{0.152744in}}{\pgfqpoint{0.630302in}{0.139389in}}%
\pgfpathcurveto{\pgfqpoint{0.620456in}{0.129544in}}{\pgfqpoint{0.610611in}{0.119698in}}{\pgfqpoint{0.597256in}{0.114167in}}%
\pgfpathclose%
\pgfpathmoveto{\pgfqpoint{0.750000in}{0.108333in}}%
\pgfpathcurveto{\pgfqpoint{0.765470in}{0.108333in}}{\pgfqpoint{0.780309in}{0.114480in}}{\pgfqpoint{0.791248in}{0.125419in}}%
\pgfpathcurveto{\pgfqpoint{0.802187in}{0.136358in}}{\pgfqpoint{0.808333in}{0.151196in}}{\pgfqpoint{0.808333in}{0.166667in}}%
\pgfpathcurveto{\pgfqpoint{0.808333in}{0.182137in}}{\pgfqpoint{0.802187in}{0.196975in}}{\pgfqpoint{0.791248in}{0.207915in}}%
\pgfpathcurveto{\pgfqpoint{0.780309in}{0.218854in}}{\pgfqpoint{0.765470in}{0.225000in}}{\pgfqpoint{0.750000in}{0.225000in}}%
\pgfpathcurveto{\pgfqpoint{0.734530in}{0.225000in}}{\pgfqpoint{0.719691in}{0.218854in}}{\pgfqpoint{0.708752in}{0.207915in}}%
\pgfpathcurveto{\pgfqpoint{0.697813in}{0.196975in}}{\pgfqpoint{0.691667in}{0.182137in}}{\pgfqpoint{0.691667in}{0.166667in}}%
\pgfpathcurveto{\pgfqpoint{0.691667in}{0.151196in}}{\pgfqpoint{0.697813in}{0.136358in}}{\pgfqpoint{0.708752in}{0.125419in}}%
\pgfpathcurveto{\pgfqpoint{0.719691in}{0.114480in}}{\pgfqpoint{0.734530in}{0.108333in}}{\pgfqpoint{0.750000in}{0.108333in}}%
\pgfpathclose%
\pgfpathmoveto{\pgfqpoint{0.750000in}{0.114167in}}%
\pgfpathcurveto{\pgfqpoint{0.750000in}{0.114167in}}{\pgfqpoint{0.736077in}{0.114167in}}{\pgfqpoint{0.722722in}{0.119698in}}%
\pgfpathcurveto{\pgfqpoint{0.712877in}{0.129544in}}{\pgfqpoint{0.703032in}{0.139389in}}{\pgfqpoint{0.697500in}{0.152744in}}%
\pgfpathcurveto{\pgfqpoint{0.697500in}{0.166667in}}{\pgfqpoint{0.697500in}{0.180590in}}{\pgfqpoint{0.703032in}{0.193945in}}%
\pgfpathcurveto{\pgfqpoint{0.712877in}{0.203790in}}{\pgfqpoint{0.722722in}{0.213635in}}{\pgfqpoint{0.736077in}{0.219167in}}%
\pgfpathcurveto{\pgfqpoint{0.750000in}{0.219167in}}{\pgfqpoint{0.763923in}{0.219167in}}{\pgfqpoint{0.777278in}{0.213635in}}%
\pgfpathcurveto{\pgfqpoint{0.787123in}{0.203790in}}{\pgfqpoint{0.796968in}{0.193945in}}{\pgfqpoint{0.802500in}{0.180590in}}%
\pgfpathcurveto{\pgfqpoint{0.802500in}{0.166667in}}{\pgfqpoint{0.802500in}{0.152744in}}{\pgfqpoint{0.796968in}{0.139389in}}%
\pgfpathcurveto{\pgfqpoint{0.787123in}{0.129544in}}{\pgfqpoint{0.777278in}{0.119698in}}{\pgfqpoint{0.763923in}{0.114167in}}%
\pgfpathclose%
\pgfpathmoveto{\pgfqpoint{0.916667in}{0.108333in}}%
\pgfpathcurveto{\pgfqpoint{0.932137in}{0.108333in}}{\pgfqpoint{0.946975in}{0.114480in}}{\pgfqpoint{0.957915in}{0.125419in}}%
\pgfpathcurveto{\pgfqpoint{0.968854in}{0.136358in}}{\pgfqpoint{0.975000in}{0.151196in}}{\pgfqpoint{0.975000in}{0.166667in}}%
\pgfpathcurveto{\pgfqpoint{0.975000in}{0.182137in}}{\pgfqpoint{0.968854in}{0.196975in}}{\pgfqpoint{0.957915in}{0.207915in}}%
\pgfpathcurveto{\pgfqpoint{0.946975in}{0.218854in}}{\pgfqpoint{0.932137in}{0.225000in}}{\pgfqpoint{0.916667in}{0.225000in}}%
\pgfpathcurveto{\pgfqpoint{0.901196in}{0.225000in}}{\pgfqpoint{0.886358in}{0.218854in}}{\pgfqpoint{0.875419in}{0.207915in}}%
\pgfpathcurveto{\pgfqpoint{0.864480in}{0.196975in}}{\pgfqpoint{0.858333in}{0.182137in}}{\pgfqpoint{0.858333in}{0.166667in}}%
\pgfpathcurveto{\pgfqpoint{0.858333in}{0.151196in}}{\pgfqpoint{0.864480in}{0.136358in}}{\pgfqpoint{0.875419in}{0.125419in}}%
\pgfpathcurveto{\pgfqpoint{0.886358in}{0.114480in}}{\pgfqpoint{0.901196in}{0.108333in}}{\pgfqpoint{0.916667in}{0.108333in}}%
\pgfpathclose%
\pgfpathmoveto{\pgfqpoint{0.916667in}{0.114167in}}%
\pgfpathcurveto{\pgfqpoint{0.916667in}{0.114167in}}{\pgfqpoint{0.902744in}{0.114167in}}{\pgfqpoint{0.889389in}{0.119698in}}%
\pgfpathcurveto{\pgfqpoint{0.879544in}{0.129544in}}{\pgfqpoint{0.869698in}{0.139389in}}{\pgfqpoint{0.864167in}{0.152744in}}%
\pgfpathcurveto{\pgfqpoint{0.864167in}{0.166667in}}{\pgfqpoint{0.864167in}{0.180590in}}{\pgfqpoint{0.869698in}{0.193945in}}%
\pgfpathcurveto{\pgfqpoint{0.879544in}{0.203790in}}{\pgfqpoint{0.889389in}{0.213635in}}{\pgfqpoint{0.902744in}{0.219167in}}%
\pgfpathcurveto{\pgfqpoint{0.916667in}{0.219167in}}{\pgfqpoint{0.930590in}{0.219167in}}{\pgfqpoint{0.943945in}{0.213635in}}%
\pgfpathcurveto{\pgfqpoint{0.953790in}{0.203790in}}{\pgfqpoint{0.963635in}{0.193945in}}{\pgfqpoint{0.969167in}{0.180590in}}%
\pgfpathcurveto{\pgfqpoint{0.969167in}{0.166667in}}{\pgfqpoint{0.969167in}{0.152744in}}{\pgfqpoint{0.963635in}{0.139389in}}%
\pgfpathcurveto{\pgfqpoint{0.953790in}{0.129544in}}{\pgfqpoint{0.943945in}{0.119698in}}{\pgfqpoint{0.930590in}{0.114167in}}%
\pgfpathclose%
\pgfpathmoveto{\pgfqpoint{0.000000in}{0.275000in}}%
\pgfpathcurveto{\pgfqpoint{0.015470in}{0.275000in}}{\pgfqpoint{0.030309in}{0.281146in}}{\pgfqpoint{0.041248in}{0.292085in}}%
\pgfpathcurveto{\pgfqpoint{0.052187in}{0.303025in}}{\pgfqpoint{0.058333in}{0.317863in}}{\pgfqpoint{0.058333in}{0.333333in}}%
\pgfpathcurveto{\pgfqpoint{0.058333in}{0.348804in}}{\pgfqpoint{0.052187in}{0.363642in}}{\pgfqpoint{0.041248in}{0.374581in}}%
\pgfpathcurveto{\pgfqpoint{0.030309in}{0.385520in}}{\pgfqpoint{0.015470in}{0.391667in}}{\pgfqpoint{0.000000in}{0.391667in}}%
\pgfpathcurveto{\pgfqpoint{-0.015470in}{0.391667in}}{\pgfqpoint{-0.030309in}{0.385520in}}{\pgfqpoint{-0.041248in}{0.374581in}}%
\pgfpathcurveto{\pgfqpoint{-0.052187in}{0.363642in}}{\pgfqpoint{-0.058333in}{0.348804in}}{\pgfqpoint{-0.058333in}{0.333333in}}%
\pgfpathcurveto{\pgfqpoint{-0.058333in}{0.317863in}}{\pgfqpoint{-0.052187in}{0.303025in}}{\pgfqpoint{-0.041248in}{0.292085in}}%
\pgfpathcurveto{\pgfqpoint{-0.030309in}{0.281146in}}{\pgfqpoint{-0.015470in}{0.275000in}}{\pgfqpoint{0.000000in}{0.275000in}}%
\pgfpathclose%
\pgfpathmoveto{\pgfqpoint{0.000000in}{0.280833in}}%
\pgfpathcurveto{\pgfqpoint{0.000000in}{0.280833in}}{\pgfqpoint{-0.013923in}{0.280833in}}{\pgfqpoint{-0.027278in}{0.286365in}}%
\pgfpathcurveto{\pgfqpoint{-0.037123in}{0.296210in}}{\pgfqpoint{-0.046968in}{0.306055in}}{\pgfqpoint{-0.052500in}{0.319410in}}%
\pgfpathcurveto{\pgfqpoint{-0.052500in}{0.333333in}}{\pgfqpoint{-0.052500in}{0.347256in}}{\pgfqpoint{-0.046968in}{0.360611in}}%
\pgfpathcurveto{\pgfqpoint{-0.037123in}{0.370456in}}{\pgfqpoint{-0.027278in}{0.380302in}}{\pgfqpoint{-0.013923in}{0.385833in}}%
\pgfpathcurveto{\pgfqpoint{0.000000in}{0.385833in}}{\pgfqpoint{0.013923in}{0.385833in}}{\pgfqpoint{0.027278in}{0.380302in}}%
\pgfpathcurveto{\pgfqpoint{0.037123in}{0.370456in}}{\pgfqpoint{0.046968in}{0.360611in}}{\pgfqpoint{0.052500in}{0.347256in}}%
\pgfpathcurveto{\pgfqpoint{0.052500in}{0.333333in}}{\pgfqpoint{0.052500in}{0.319410in}}{\pgfqpoint{0.046968in}{0.306055in}}%
\pgfpathcurveto{\pgfqpoint{0.037123in}{0.296210in}}{\pgfqpoint{0.027278in}{0.286365in}}{\pgfqpoint{0.013923in}{0.280833in}}%
\pgfpathclose%
\pgfpathmoveto{\pgfqpoint{0.166667in}{0.275000in}}%
\pgfpathcurveto{\pgfqpoint{0.182137in}{0.275000in}}{\pgfqpoint{0.196975in}{0.281146in}}{\pgfqpoint{0.207915in}{0.292085in}}%
\pgfpathcurveto{\pgfqpoint{0.218854in}{0.303025in}}{\pgfqpoint{0.225000in}{0.317863in}}{\pgfqpoint{0.225000in}{0.333333in}}%
\pgfpathcurveto{\pgfqpoint{0.225000in}{0.348804in}}{\pgfqpoint{0.218854in}{0.363642in}}{\pgfqpoint{0.207915in}{0.374581in}}%
\pgfpathcurveto{\pgfqpoint{0.196975in}{0.385520in}}{\pgfqpoint{0.182137in}{0.391667in}}{\pgfqpoint{0.166667in}{0.391667in}}%
\pgfpathcurveto{\pgfqpoint{0.151196in}{0.391667in}}{\pgfqpoint{0.136358in}{0.385520in}}{\pgfqpoint{0.125419in}{0.374581in}}%
\pgfpathcurveto{\pgfqpoint{0.114480in}{0.363642in}}{\pgfqpoint{0.108333in}{0.348804in}}{\pgfqpoint{0.108333in}{0.333333in}}%
\pgfpathcurveto{\pgfqpoint{0.108333in}{0.317863in}}{\pgfqpoint{0.114480in}{0.303025in}}{\pgfqpoint{0.125419in}{0.292085in}}%
\pgfpathcurveto{\pgfqpoint{0.136358in}{0.281146in}}{\pgfqpoint{0.151196in}{0.275000in}}{\pgfqpoint{0.166667in}{0.275000in}}%
\pgfpathclose%
\pgfpathmoveto{\pgfqpoint{0.166667in}{0.280833in}}%
\pgfpathcurveto{\pgfqpoint{0.166667in}{0.280833in}}{\pgfqpoint{0.152744in}{0.280833in}}{\pgfqpoint{0.139389in}{0.286365in}}%
\pgfpathcurveto{\pgfqpoint{0.129544in}{0.296210in}}{\pgfqpoint{0.119698in}{0.306055in}}{\pgfqpoint{0.114167in}{0.319410in}}%
\pgfpathcurveto{\pgfqpoint{0.114167in}{0.333333in}}{\pgfqpoint{0.114167in}{0.347256in}}{\pgfqpoint{0.119698in}{0.360611in}}%
\pgfpathcurveto{\pgfqpoint{0.129544in}{0.370456in}}{\pgfqpoint{0.139389in}{0.380302in}}{\pgfqpoint{0.152744in}{0.385833in}}%
\pgfpathcurveto{\pgfqpoint{0.166667in}{0.385833in}}{\pgfqpoint{0.180590in}{0.385833in}}{\pgfqpoint{0.193945in}{0.380302in}}%
\pgfpathcurveto{\pgfqpoint{0.203790in}{0.370456in}}{\pgfqpoint{0.213635in}{0.360611in}}{\pgfqpoint{0.219167in}{0.347256in}}%
\pgfpathcurveto{\pgfqpoint{0.219167in}{0.333333in}}{\pgfqpoint{0.219167in}{0.319410in}}{\pgfqpoint{0.213635in}{0.306055in}}%
\pgfpathcurveto{\pgfqpoint{0.203790in}{0.296210in}}{\pgfqpoint{0.193945in}{0.286365in}}{\pgfqpoint{0.180590in}{0.280833in}}%
\pgfpathclose%
\pgfpathmoveto{\pgfqpoint{0.333333in}{0.275000in}}%
\pgfpathcurveto{\pgfqpoint{0.348804in}{0.275000in}}{\pgfqpoint{0.363642in}{0.281146in}}{\pgfqpoint{0.374581in}{0.292085in}}%
\pgfpathcurveto{\pgfqpoint{0.385520in}{0.303025in}}{\pgfqpoint{0.391667in}{0.317863in}}{\pgfqpoint{0.391667in}{0.333333in}}%
\pgfpathcurveto{\pgfqpoint{0.391667in}{0.348804in}}{\pgfqpoint{0.385520in}{0.363642in}}{\pgfqpoint{0.374581in}{0.374581in}}%
\pgfpathcurveto{\pgfqpoint{0.363642in}{0.385520in}}{\pgfqpoint{0.348804in}{0.391667in}}{\pgfqpoint{0.333333in}{0.391667in}}%
\pgfpathcurveto{\pgfqpoint{0.317863in}{0.391667in}}{\pgfqpoint{0.303025in}{0.385520in}}{\pgfqpoint{0.292085in}{0.374581in}}%
\pgfpathcurveto{\pgfqpoint{0.281146in}{0.363642in}}{\pgfqpoint{0.275000in}{0.348804in}}{\pgfqpoint{0.275000in}{0.333333in}}%
\pgfpathcurveto{\pgfqpoint{0.275000in}{0.317863in}}{\pgfqpoint{0.281146in}{0.303025in}}{\pgfqpoint{0.292085in}{0.292085in}}%
\pgfpathcurveto{\pgfqpoint{0.303025in}{0.281146in}}{\pgfqpoint{0.317863in}{0.275000in}}{\pgfqpoint{0.333333in}{0.275000in}}%
\pgfpathclose%
\pgfpathmoveto{\pgfqpoint{0.333333in}{0.280833in}}%
\pgfpathcurveto{\pgfqpoint{0.333333in}{0.280833in}}{\pgfqpoint{0.319410in}{0.280833in}}{\pgfqpoint{0.306055in}{0.286365in}}%
\pgfpathcurveto{\pgfqpoint{0.296210in}{0.296210in}}{\pgfqpoint{0.286365in}{0.306055in}}{\pgfqpoint{0.280833in}{0.319410in}}%
\pgfpathcurveto{\pgfqpoint{0.280833in}{0.333333in}}{\pgfqpoint{0.280833in}{0.347256in}}{\pgfqpoint{0.286365in}{0.360611in}}%
\pgfpathcurveto{\pgfqpoint{0.296210in}{0.370456in}}{\pgfqpoint{0.306055in}{0.380302in}}{\pgfqpoint{0.319410in}{0.385833in}}%
\pgfpathcurveto{\pgfqpoint{0.333333in}{0.385833in}}{\pgfqpoint{0.347256in}{0.385833in}}{\pgfqpoint{0.360611in}{0.380302in}}%
\pgfpathcurveto{\pgfqpoint{0.370456in}{0.370456in}}{\pgfqpoint{0.380302in}{0.360611in}}{\pgfqpoint{0.385833in}{0.347256in}}%
\pgfpathcurveto{\pgfqpoint{0.385833in}{0.333333in}}{\pgfqpoint{0.385833in}{0.319410in}}{\pgfqpoint{0.380302in}{0.306055in}}%
\pgfpathcurveto{\pgfqpoint{0.370456in}{0.296210in}}{\pgfqpoint{0.360611in}{0.286365in}}{\pgfqpoint{0.347256in}{0.280833in}}%
\pgfpathclose%
\pgfpathmoveto{\pgfqpoint{0.500000in}{0.275000in}}%
\pgfpathcurveto{\pgfqpoint{0.515470in}{0.275000in}}{\pgfqpoint{0.530309in}{0.281146in}}{\pgfqpoint{0.541248in}{0.292085in}}%
\pgfpathcurveto{\pgfqpoint{0.552187in}{0.303025in}}{\pgfqpoint{0.558333in}{0.317863in}}{\pgfqpoint{0.558333in}{0.333333in}}%
\pgfpathcurveto{\pgfqpoint{0.558333in}{0.348804in}}{\pgfqpoint{0.552187in}{0.363642in}}{\pgfqpoint{0.541248in}{0.374581in}}%
\pgfpathcurveto{\pgfqpoint{0.530309in}{0.385520in}}{\pgfqpoint{0.515470in}{0.391667in}}{\pgfqpoint{0.500000in}{0.391667in}}%
\pgfpathcurveto{\pgfqpoint{0.484530in}{0.391667in}}{\pgfqpoint{0.469691in}{0.385520in}}{\pgfqpoint{0.458752in}{0.374581in}}%
\pgfpathcurveto{\pgfqpoint{0.447813in}{0.363642in}}{\pgfqpoint{0.441667in}{0.348804in}}{\pgfqpoint{0.441667in}{0.333333in}}%
\pgfpathcurveto{\pgfqpoint{0.441667in}{0.317863in}}{\pgfqpoint{0.447813in}{0.303025in}}{\pgfqpoint{0.458752in}{0.292085in}}%
\pgfpathcurveto{\pgfqpoint{0.469691in}{0.281146in}}{\pgfqpoint{0.484530in}{0.275000in}}{\pgfqpoint{0.500000in}{0.275000in}}%
\pgfpathclose%
\pgfpathmoveto{\pgfqpoint{0.500000in}{0.280833in}}%
\pgfpathcurveto{\pgfqpoint{0.500000in}{0.280833in}}{\pgfqpoint{0.486077in}{0.280833in}}{\pgfqpoint{0.472722in}{0.286365in}}%
\pgfpathcurveto{\pgfqpoint{0.462877in}{0.296210in}}{\pgfqpoint{0.453032in}{0.306055in}}{\pgfqpoint{0.447500in}{0.319410in}}%
\pgfpathcurveto{\pgfqpoint{0.447500in}{0.333333in}}{\pgfqpoint{0.447500in}{0.347256in}}{\pgfqpoint{0.453032in}{0.360611in}}%
\pgfpathcurveto{\pgfqpoint{0.462877in}{0.370456in}}{\pgfqpoint{0.472722in}{0.380302in}}{\pgfqpoint{0.486077in}{0.385833in}}%
\pgfpathcurveto{\pgfqpoint{0.500000in}{0.385833in}}{\pgfqpoint{0.513923in}{0.385833in}}{\pgfqpoint{0.527278in}{0.380302in}}%
\pgfpathcurveto{\pgfqpoint{0.537123in}{0.370456in}}{\pgfqpoint{0.546968in}{0.360611in}}{\pgfqpoint{0.552500in}{0.347256in}}%
\pgfpathcurveto{\pgfqpoint{0.552500in}{0.333333in}}{\pgfqpoint{0.552500in}{0.319410in}}{\pgfqpoint{0.546968in}{0.306055in}}%
\pgfpathcurveto{\pgfqpoint{0.537123in}{0.296210in}}{\pgfqpoint{0.527278in}{0.286365in}}{\pgfqpoint{0.513923in}{0.280833in}}%
\pgfpathclose%
\pgfpathmoveto{\pgfqpoint{0.666667in}{0.275000in}}%
\pgfpathcurveto{\pgfqpoint{0.682137in}{0.275000in}}{\pgfqpoint{0.696975in}{0.281146in}}{\pgfqpoint{0.707915in}{0.292085in}}%
\pgfpathcurveto{\pgfqpoint{0.718854in}{0.303025in}}{\pgfqpoint{0.725000in}{0.317863in}}{\pgfqpoint{0.725000in}{0.333333in}}%
\pgfpathcurveto{\pgfqpoint{0.725000in}{0.348804in}}{\pgfqpoint{0.718854in}{0.363642in}}{\pgfqpoint{0.707915in}{0.374581in}}%
\pgfpathcurveto{\pgfqpoint{0.696975in}{0.385520in}}{\pgfqpoint{0.682137in}{0.391667in}}{\pgfqpoint{0.666667in}{0.391667in}}%
\pgfpathcurveto{\pgfqpoint{0.651196in}{0.391667in}}{\pgfqpoint{0.636358in}{0.385520in}}{\pgfqpoint{0.625419in}{0.374581in}}%
\pgfpathcurveto{\pgfqpoint{0.614480in}{0.363642in}}{\pgfqpoint{0.608333in}{0.348804in}}{\pgfqpoint{0.608333in}{0.333333in}}%
\pgfpathcurveto{\pgfqpoint{0.608333in}{0.317863in}}{\pgfqpoint{0.614480in}{0.303025in}}{\pgfqpoint{0.625419in}{0.292085in}}%
\pgfpathcurveto{\pgfqpoint{0.636358in}{0.281146in}}{\pgfqpoint{0.651196in}{0.275000in}}{\pgfqpoint{0.666667in}{0.275000in}}%
\pgfpathclose%
\pgfpathmoveto{\pgfqpoint{0.666667in}{0.280833in}}%
\pgfpathcurveto{\pgfqpoint{0.666667in}{0.280833in}}{\pgfqpoint{0.652744in}{0.280833in}}{\pgfqpoint{0.639389in}{0.286365in}}%
\pgfpathcurveto{\pgfqpoint{0.629544in}{0.296210in}}{\pgfqpoint{0.619698in}{0.306055in}}{\pgfqpoint{0.614167in}{0.319410in}}%
\pgfpathcurveto{\pgfqpoint{0.614167in}{0.333333in}}{\pgfqpoint{0.614167in}{0.347256in}}{\pgfqpoint{0.619698in}{0.360611in}}%
\pgfpathcurveto{\pgfqpoint{0.629544in}{0.370456in}}{\pgfqpoint{0.639389in}{0.380302in}}{\pgfqpoint{0.652744in}{0.385833in}}%
\pgfpathcurveto{\pgfqpoint{0.666667in}{0.385833in}}{\pgfqpoint{0.680590in}{0.385833in}}{\pgfqpoint{0.693945in}{0.380302in}}%
\pgfpathcurveto{\pgfqpoint{0.703790in}{0.370456in}}{\pgfqpoint{0.713635in}{0.360611in}}{\pgfqpoint{0.719167in}{0.347256in}}%
\pgfpathcurveto{\pgfqpoint{0.719167in}{0.333333in}}{\pgfqpoint{0.719167in}{0.319410in}}{\pgfqpoint{0.713635in}{0.306055in}}%
\pgfpathcurveto{\pgfqpoint{0.703790in}{0.296210in}}{\pgfqpoint{0.693945in}{0.286365in}}{\pgfqpoint{0.680590in}{0.280833in}}%
\pgfpathclose%
\pgfpathmoveto{\pgfqpoint{0.833333in}{0.275000in}}%
\pgfpathcurveto{\pgfqpoint{0.848804in}{0.275000in}}{\pgfqpoint{0.863642in}{0.281146in}}{\pgfqpoint{0.874581in}{0.292085in}}%
\pgfpathcurveto{\pgfqpoint{0.885520in}{0.303025in}}{\pgfqpoint{0.891667in}{0.317863in}}{\pgfqpoint{0.891667in}{0.333333in}}%
\pgfpathcurveto{\pgfqpoint{0.891667in}{0.348804in}}{\pgfqpoint{0.885520in}{0.363642in}}{\pgfqpoint{0.874581in}{0.374581in}}%
\pgfpathcurveto{\pgfqpoint{0.863642in}{0.385520in}}{\pgfqpoint{0.848804in}{0.391667in}}{\pgfqpoint{0.833333in}{0.391667in}}%
\pgfpathcurveto{\pgfqpoint{0.817863in}{0.391667in}}{\pgfqpoint{0.803025in}{0.385520in}}{\pgfqpoint{0.792085in}{0.374581in}}%
\pgfpathcurveto{\pgfqpoint{0.781146in}{0.363642in}}{\pgfqpoint{0.775000in}{0.348804in}}{\pgfqpoint{0.775000in}{0.333333in}}%
\pgfpathcurveto{\pgfqpoint{0.775000in}{0.317863in}}{\pgfqpoint{0.781146in}{0.303025in}}{\pgfqpoint{0.792085in}{0.292085in}}%
\pgfpathcurveto{\pgfqpoint{0.803025in}{0.281146in}}{\pgfqpoint{0.817863in}{0.275000in}}{\pgfqpoint{0.833333in}{0.275000in}}%
\pgfpathclose%
\pgfpathmoveto{\pgfqpoint{0.833333in}{0.280833in}}%
\pgfpathcurveto{\pgfqpoint{0.833333in}{0.280833in}}{\pgfqpoint{0.819410in}{0.280833in}}{\pgfqpoint{0.806055in}{0.286365in}}%
\pgfpathcurveto{\pgfqpoint{0.796210in}{0.296210in}}{\pgfqpoint{0.786365in}{0.306055in}}{\pgfqpoint{0.780833in}{0.319410in}}%
\pgfpathcurveto{\pgfqpoint{0.780833in}{0.333333in}}{\pgfqpoint{0.780833in}{0.347256in}}{\pgfqpoint{0.786365in}{0.360611in}}%
\pgfpathcurveto{\pgfqpoint{0.796210in}{0.370456in}}{\pgfqpoint{0.806055in}{0.380302in}}{\pgfqpoint{0.819410in}{0.385833in}}%
\pgfpathcurveto{\pgfqpoint{0.833333in}{0.385833in}}{\pgfqpoint{0.847256in}{0.385833in}}{\pgfqpoint{0.860611in}{0.380302in}}%
\pgfpathcurveto{\pgfqpoint{0.870456in}{0.370456in}}{\pgfqpoint{0.880302in}{0.360611in}}{\pgfqpoint{0.885833in}{0.347256in}}%
\pgfpathcurveto{\pgfqpoint{0.885833in}{0.333333in}}{\pgfqpoint{0.885833in}{0.319410in}}{\pgfqpoint{0.880302in}{0.306055in}}%
\pgfpathcurveto{\pgfqpoint{0.870456in}{0.296210in}}{\pgfqpoint{0.860611in}{0.286365in}}{\pgfqpoint{0.847256in}{0.280833in}}%
\pgfpathclose%
\pgfpathmoveto{\pgfqpoint{1.000000in}{0.275000in}}%
\pgfpathcurveto{\pgfqpoint{1.015470in}{0.275000in}}{\pgfqpoint{1.030309in}{0.281146in}}{\pgfqpoint{1.041248in}{0.292085in}}%
\pgfpathcurveto{\pgfqpoint{1.052187in}{0.303025in}}{\pgfqpoint{1.058333in}{0.317863in}}{\pgfqpoint{1.058333in}{0.333333in}}%
\pgfpathcurveto{\pgfqpoint{1.058333in}{0.348804in}}{\pgfqpoint{1.052187in}{0.363642in}}{\pgfqpoint{1.041248in}{0.374581in}}%
\pgfpathcurveto{\pgfqpoint{1.030309in}{0.385520in}}{\pgfqpoint{1.015470in}{0.391667in}}{\pgfqpoint{1.000000in}{0.391667in}}%
\pgfpathcurveto{\pgfqpoint{0.984530in}{0.391667in}}{\pgfqpoint{0.969691in}{0.385520in}}{\pgfqpoint{0.958752in}{0.374581in}}%
\pgfpathcurveto{\pgfqpoint{0.947813in}{0.363642in}}{\pgfqpoint{0.941667in}{0.348804in}}{\pgfqpoint{0.941667in}{0.333333in}}%
\pgfpathcurveto{\pgfqpoint{0.941667in}{0.317863in}}{\pgfqpoint{0.947813in}{0.303025in}}{\pgfqpoint{0.958752in}{0.292085in}}%
\pgfpathcurveto{\pgfqpoint{0.969691in}{0.281146in}}{\pgfqpoint{0.984530in}{0.275000in}}{\pgfqpoint{1.000000in}{0.275000in}}%
\pgfpathclose%
\pgfpathmoveto{\pgfqpoint{1.000000in}{0.280833in}}%
\pgfpathcurveto{\pgfqpoint{1.000000in}{0.280833in}}{\pgfqpoint{0.986077in}{0.280833in}}{\pgfqpoint{0.972722in}{0.286365in}}%
\pgfpathcurveto{\pgfqpoint{0.962877in}{0.296210in}}{\pgfqpoint{0.953032in}{0.306055in}}{\pgfqpoint{0.947500in}{0.319410in}}%
\pgfpathcurveto{\pgfqpoint{0.947500in}{0.333333in}}{\pgfqpoint{0.947500in}{0.347256in}}{\pgfqpoint{0.953032in}{0.360611in}}%
\pgfpathcurveto{\pgfqpoint{0.962877in}{0.370456in}}{\pgfqpoint{0.972722in}{0.380302in}}{\pgfqpoint{0.986077in}{0.385833in}}%
\pgfpathcurveto{\pgfqpoint{1.000000in}{0.385833in}}{\pgfqpoint{1.013923in}{0.385833in}}{\pgfqpoint{1.027278in}{0.380302in}}%
\pgfpathcurveto{\pgfqpoint{1.037123in}{0.370456in}}{\pgfqpoint{1.046968in}{0.360611in}}{\pgfqpoint{1.052500in}{0.347256in}}%
\pgfpathcurveto{\pgfqpoint{1.052500in}{0.333333in}}{\pgfqpoint{1.052500in}{0.319410in}}{\pgfqpoint{1.046968in}{0.306055in}}%
\pgfpathcurveto{\pgfqpoint{1.037123in}{0.296210in}}{\pgfqpoint{1.027278in}{0.286365in}}{\pgfqpoint{1.013923in}{0.280833in}}%
\pgfpathclose%
\pgfpathmoveto{\pgfqpoint{0.083333in}{0.441667in}}%
\pgfpathcurveto{\pgfqpoint{0.098804in}{0.441667in}}{\pgfqpoint{0.113642in}{0.447813in}}{\pgfqpoint{0.124581in}{0.458752in}}%
\pgfpathcurveto{\pgfqpoint{0.135520in}{0.469691in}}{\pgfqpoint{0.141667in}{0.484530in}}{\pgfqpoint{0.141667in}{0.500000in}}%
\pgfpathcurveto{\pgfqpoint{0.141667in}{0.515470in}}{\pgfqpoint{0.135520in}{0.530309in}}{\pgfqpoint{0.124581in}{0.541248in}}%
\pgfpathcurveto{\pgfqpoint{0.113642in}{0.552187in}}{\pgfqpoint{0.098804in}{0.558333in}}{\pgfqpoint{0.083333in}{0.558333in}}%
\pgfpathcurveto{\pgfqpoint{0.067863in}{0.558333in}}{\pgfqpoint{0.053025in}{0.552187in}}{\pgfqpoint{0.042085in}{0.541248in}}%
\pgfpathcurveto{\pgfqpoint{0.031146in}{0.530309in}}{\pgfqpoint{0.025000in}{0.515470in}}{\pgfqpoint{0.025000in}{0.500000in}}%
\pgfpathcurveto{\pgfqpoint{0.025000in}{0.484530in}}{\pgfqpoint{0.031146in}{0.469691in}}{\pgfqpoint{0.042085in}{0.458752in}}%
\pgfpathcurveto{\pgfqpoint{0.053025in}{0.447813in}}{\pgfqpoint{0.067863in}{0.441667in}}{\pgfqpoint{0.083333in}{0.441667in}}%
\pgfpathclose%
\pgfpathmoveto{\pgfqpoint{0.083333in}{0.447500in}}%
\pgfpathcurveto{\pgfqpoint{0.083333in}{0.447500in}}{\pgfqpoint{0.069410in}{0.447500in}}{\pgfqpoint{0.056055in}{0.453032in}}%
\pgfpathcurveto{\pgfqpoint{0.046210in}{0.462877in}}{\pgfqpoint{0.036365in}{0.472722in}}{\pgfqpoint{0.030833in}{0.486077in}}%
\pgfpathcurveto{\pgfqpoint{0.030833in}{0.500000in}}{\pgfqpoint{0.030833in}{0.513923in}}{\pgfqpoint{0.036365in}{0.527278in}}%
\pgfpathcurveto{\pgfqpoint{0.046210in}{0.537123in}}{\pgfqpoint{0.056055in}{0.546968in}}{\pgfqpoint{0.069410in}{0.552500in}}%
\pgfpathcurveto{\pgfqpoint{0.083333in}{0.552500in}}{\pgfqpoint{0.097256in}{0.552500in}}{\pgfqpoint{0.110611in}{0.546968in}}%
\pgfpathcurveto{\pgfqpoint{0.120456in}{0.537123in}}{\pgfqpoint{0.130302in}{0.527278in}}{\pgfqpoint{0.135833in}{0.513923in}}%
\pgfpathcurveto{\pgfqpoint{0.135833in}{0.500000in}}{\pgfqpoint{0.135833in}{0.486077in}}{\pgfqpoint{0.130302in}{0.472722in}}%
\pgfpathcurveto{\pgfqpoint{0.120456in}{0.462877in}}{\pgfqpoint{0.110611in}{0.453032in}}{\pgfqpoint{0.097256in}{0.447500in}}%
\pgfpathclose%
\pgfpathmoveto{\pgfqpoint{0.250000in}{0.441667in}}%
\pgfpathcurveto{\pgfqpoint{0.265470in}{0.441667in}}{\pgfqpoint{0.280309in}{0.447813in}}{\pgfqpoint{0.291248in}{0.458752in}}%
\pgfpathcurveto{\pgfqpoint{0.302187in}{0.469691in}}{\pgfqpoint{0.308333in}{0.484530in}}{\pgfqpoint{0.308333in}{0.500000in}}%
\pgfpathcurveto{\pgfqpoint{0.308333in}{0.515470in}}{\pgfqpoint{0.302187in}{0.530309in}}{\pgfqpoint{0.291248in}{0.541248in}}%
\pgfpathcurveto{\pgfqpoint{0.280309in}{0.552187in}}{\pgfqpoint{0.265470in}{0.558333in}}{\pgfqpoint{0.250000in}{0.558333in}}%
\pgfpathcurveto{\pgfqpoint{0.234530in}{0.558333in}}{\pgfqpoint{0.219691in}{0.552187in}}{\pgfqpoint{0.208752in}{0.541248in}}%
\pgfpathcurveto{\pgfqpoint{0.197813in}{0.530309in}}{\pgfqpoint{0.191667in}{0.515470in}}{\pgfqpoint{0.191667in}{0.500000in}}%
\pgfpathcurveto{\pgfqpoint{0.191667in}{0.484530in}}{\pgfqpoint{0.197813in}{0.469691in}}{\pgfqpoint{0.208752in}{0.458752in}}%
\pgfpathcurveto{\pgfqpoint{0.219691in}{0.447813in}}{\pgfqpoint{0.234530in}{0.441667in}}{\pgfqpoint{0.250000in}{0.441667in}}%
\pgfpathclose%
\pgfpathmoveto{\pgfqpoint{0.250000in}{0.447500in}}%
\pgfpathcurveto{\pgfqpoint{0.250000in}{0.447500in}}{\pgfqpoint{0.236077in}{0.447500in}}{\pgfqpoint{0.222722in}{0.453032in}}%
\pgfpathcurveto{\pgfqpoint{0.212877in}{0.462877in}}{\pgfqpoint{0.203032in}{0.472722in}}{\pgfqpoint{0.197500in}{0.486077in}}%
\pgfpathcurveto{\pgfqpoint{0.197500in}{0.500000in}}{\pgfqpoint{0.197500in}{0.513923in}}{\pgfqpoint{0.203032in}{0.527278in}}%
\pgfpathcurveto{\pgfqpoint{0.212877in}{0.537123in}}{\pgfqpoint{0.222722in}{0.546968in}}{\pgfqpoint{0.236077in}{0.552500in}}%
\pgfpathcurveto{\pgfqpoint{0.250000in}{0.552500in}}{\pgfqpoint{0.263923in}{0.552500in}}{\pgfqpoint{0.277278in}{0.546968in}}%
\pgfpathcurveto{\pgfqpoint{0.287123in}{0.537123in}}{\pgfqpoint{0.296968in}{0.527278in}}{\pgfqpoint{0.302500in}{0.513923in}}%
\pgfpathcurveto{\pgfqpoint{0.302500in}{0.500000in}}{\pgfqpoint{0.302500in}{0.486077in}}{\pgfqpoint{0.296968in}{0.472722in}}%
\pgfpathcurveto{\pgfqpoint{0.287123in}{0.462877in}}{\pgfqpoint{0.277278in}{0.453032in}}{\pgfqpoint{0.263923in}{0.447500in}}%
\pgfpathclose%
\pgfpathmoveto{\pgfqpoint{0.416667in}{0.441667in}}%
\pgfpathcurveto{\pgfqpoint{0.432137in}{0.441667in}}{\pgfqpoint{0.446975in}{0.447813in}}{\pgfqpoint{0.457915in}{0.458752in}}%
\pgfpathcurveto{\pgfqpoint{0.468854in}{0.469691in}}{\pgfqpoint{0.475000in}{0.484530in}}{\pgfqpoint{0.475000in}{0.500000in}}%
\pgfpathcurveto{\pgfqpoint{0.475000in}{0.515470in}}{\pgfqpoint{0.468854in}{0.530309in}}{\pgfqpoint{0.457915in}{0.541248in}}%
\pgfpathcurveto{\pgfqpoint{0.446975in}{0.552187in}}{\pgfqpoint{0.432137in}{0.558333in}}{\pgfqpoint{0.416667in}{0.558333in}}%
\pgfpathcurveto{\pgfqpoint{0.401196in}{0.558333in}}{\pgfqpoint{0.386358in}{0.552187in}}{\pgfqpoint{0.375419in}{0.541248in}}%
\pgfpathcurveto{\pgfqpoint{0.364480in}{0.530309in}}{\pgfqpoint{0.358333in}{0.515470in}}{\pgfqpoint{0.358333in}{0.500000in}}%
\pgfpathcurveto{\pgfqpoint{0.358333in}{0.484530in}}{\pgfqpoint{0.364480in}{0.469691in}}{\pgfqpoint{0.375419in}{0.458752in}}%
\pgfpathcurveto{\pgfqpoint{0.386358in}{0.447813in}}{\pgfqpoint{0.401196in}{0.441667in}}{\pgfqpoint{0.416667in}{0.441667in}}%
\pgfpathclose%
\pgfpathmoveto{\pgfqpoint{0.416667in}{0.447500in}}%
\pgfpathcurveto{\pgfqpoint{0.416667in}{0.447500in}}{\pgfqpoint{0.402744in}{0.447500in}}{\pgfqpoint{0.389389in}{0.453032in}}%
\pgfpathcurveto{\pgfqpoint{0.379544in}{0.462877in}}{\pgfqpoint{0.369698in}{0.472722in}}{\pgfqpoint{0.364167in}{0.486077in}}%
\pgfpathcurveto{\pgfqpoint{0.364167in}{0.500000in}}{\pgfqpoint{0.364167in}{0.513923in}}{\pgfqpoint{0.369698in}{0.527278in}}%
\pgfpathcurveto{\pgfqpoint{0.379544in}{0.537123in}}{\pgfqpoint{0.389389in}{0.546968in}}{\pgfqpoint{0.402744in}{0.552500in}}%
\pgfpathcurveto{\pgfqpoint{0.416667in}{0.552500in}}{\pgfqpoint{0.430590in}{0.552500in}}{\pgfqpoint{0.443945in}{0.546968in}}%
\pgfpathcurveto{\pgfqpoint{0.453790in}{0.537123in}}{\pgfqpoint{0.463635in}{0.527278in}}{\pgfqpoint{0.469167in}{0.513923in}}%
\pgfpathcurveto{\pgfqpoint{0.469167in}{0.500000in}}{\pgfqpoint{0.469167in}{0.486077in}}{\pgfqpoint{0.463635in}{0.472722in}}%
\pgfpathcurveto{\pgfqpoint{0.453790in}{0.462877in}}{\pgfqpoint{0.443945in}{0.453032in}}{\pgfqpoint{0.430590in}{0.447500in}}%
\pgfpathclose%
\pgfpathmoveto{\pgfqpoint{0.583333in}{0.441667in}}%
\pgfpathcurveto{\pgfqpoint{0.598804in}{0.441667in}}{\pgfqpoint{0.613642in}{0.447813in}}{\pgfqpoint{0.624581in}{0.458752in}}%
\pgfpathcurveto{\pgfqpoint{0.635520in}{0.469691in}}{\pgfqpoint{0.641667in}{0.484530in}}{\pgfqpoint{0.641667in}{0.500000in}}%
\pgfpathcurveto{\pgfqpoint{0.641667in}{0.515470in}}{\pgfqpoint{0.635520in}{0.530309in}}{\pgfqpoint{0.624581in}{0.541248in}}%
\pgfpathcurveto{\pgfqpoint{0.613642in}{0.552187in}}{\pgfqpoint{0.598804in}{0.558333in}}{\pgfqpoint{0.583333in}{0.558333in}}%
\pgfpathcurveto{\pgfqpoint{0.567863in}{0.558333in}}{\pgfqpoint{0.553025in}{0.552187in}}{\pgfqpoint{0.542085in}{0.541248in}}%
\pgfpathcurveto{\pgfqpoint{0.531146in}{0.530309in}}{\pgfqpoint{0.525000in}{0.515470in}}{\pgfqpoint{0.525000in}{0.500000in}}%
\pgfpathcurveto{\pgfqpoint{0.525000in}{0.484530in}}{\pgfqpoint{0.531146in}{0.469691in}}{\pgfqpoint{0.542085in}{0.458752in}}%
\pgfpathcurveto{\pgfqpoint{0.553025in}{0.447813in}}{\pgfqpoint{0.567863in}{0.441667in}}{\pgfqpoint{0.583333in}{0.441667in}}%
\pgfpathclose%
\pgfpathmoveto{\pgfqpoint{0.583333in}{0.447500in}}%
\pgfpathcurveto{\pgfqpoint{0.583333in}{0.447500in}}{\pgfqpoint{0.569410in}{0.447500in}}{\pgfqpoint{0.556055in}{0.453032in}}%
\pgfpathcurveto{\pgfqpoint{0.546210in}{0.462877in}}{\pgfqpoint{0.536365in}{0.472722in}}{\pgfqpoint{0.530833in}{0.486077in}}%
\pgfpathcurveto{\pgfqpoint{0.530833in}{0.500000in}}{\pgfqpoint{0.530833in}{0.513923in}}{\pgfqpoint{0.536365in}{0.527278in}}%
\pgfpathcurveto{\pgfqpoint{0.546210in}{0.537123in}}{\pgfqpoint{0.556055in}{0.546968in}}{\pgfqpoint{0.569410in}{0.552500in}}%
\pgfpathcurveto{\pgfqpoint{0.583333in}{0.552500in}}{\pgfqpoint{0.597256in}{0.552500in}}{\pgfqpoint{0.610611in}{0.546968in}}%
\pgfpathcurveto{\pgfqpoint{0.620456in}{0.537123in}}{\pgfqpoint{0.630302in}{0.527278in}}{\pgfqpoint{0.635833in}{0.513923in}}%
\pgfpathcurveto{\pgfqpoint{0.635833in}{0.500000in}}{\pgfqpoint{0.635833in}{0.486077in}}{\pgfqpoint{0.630302in}{0.472722in}}%
\pgfpathcurveto{\pgfqpoint{0.620456in}{0.462877in}}{\pgfqpoint{0.610611in}{0.453032in}}{\pgfqpoint{0.597256in}{0.447500in}}%
\pgfpathclose%
\pgfpathmoveto{\pgfqpoint{0.750000in}{0.441667in}}%
\pgfpathcurveto{\pgfqpoint{0.765470in}{0.441667in}}{\pgfqpoint{0.780309in}{0.447813in}}{\pgfqpoint{0.791248in}{0.458752in}}%
\pgfpathcurveto{\pgfqpoint{0.802187in}{0.469691in}}{\pgfqpoint{0.808333in}{0.484530in}}{\pgfqpoint{0.808333in}{0.500000in}}%
\pgfpathcurveto{\pgfqpoint{0.808333in}{0.515470in}}{\pgfqpoint{0.802187in}{0.530309in}}{\pgfqpoint{0.791248in}{0.541248in}}%
\pgfpathcurveto{\pgfqpoint{0.780309in}{0.552187in}}{\pgfqpoint{0.765470in}{0.558333in}}{\pgfqpoint{0.750000in}{0.558333in}}%
\pgfpathcurveto{\pgfqpoint{0.734530in}{0.558333in}}{\pgfqpoint{0.719691in}{0.552187in}}{\pgfqpoint{0.708752in}{0.541248in}}%
\pgfpathcurveto{\pgfqpoint{0.697813in}{0.530309in}}{\pgfqpoint{0.691667in}{0.515470in}}{\pgfqpoint{0.691667in}{0.500000in}}%
\pgfpathcurveto{\pgfqpoint{0.691667in}{0.484530in}}{\pgfqpoint{0.697813in}{0.469691in}}{\pgfqpoint{0.708752in}{0.458752in}}%
\pgfpathcurveto{\pgfqpoint{0.719691in}{0.447813in}}{\pgfqpoint{0.734530in}{0.441667in}}{\pgfqpoint{0.750000in}{0.441667in}}%
\pgfpathclose%
\pgfpathmoveto{\pgfqpoint{0.750000in}{0.447500in}}%
\pgfpathcurveto{\pgfqpoint{0.750000in}{0.447500in}}{\pgfqpoint{0.736077in}{0.447500in}}{\pgfqpoint{0.722722in}{0.453032in}}%
\pgfpathcurveto{\pgfqpoint{0.712877in}{0.462877in}}{\pgfqpoint{0.703032in}{0.472722in}}{\pgfqpoint{0.697500in}{0.486077in}}%
\pgfpathcurveto{\pgfqpoint{0.697500in}{0.500000in}}{\pgfqpoint{0.697500in}{0.513923in}}{\pgfqpoint{0.703032in}{0.527278in}}%
\pgfpathcurveto{\pgfqpoint{0.712877in}{0.537123in}}{\pgfqpoint{0.722722in}{0.546968in}}{\pgfqpoint{0.736077in}{0.552500in}}%
\pgfpathcurveto{\pgfqpoint{0.750000in}{0.552500in}}{\pgfqpoint{0.763923in}{0.552500in}}{\pgfqpoint{0.777278in}{0.546968in}}%
\pgfpathcurveto{\pgfqpoint{0.787123in}{0.537123in}}{\pgfqpoint{0.796968in}{0.527278in}}{\pgfqpoint{0.802500in}{0.513923in}}%
\pgfpathcurveto{\pgfqpoint{0.802500in}{0.500000in}}{\pgfqpoint{0.802500in}{0.486077in}}{\pgfqpoint{0.796968in}{0.472722in}}%
\pgfpathcurveto{\pgfqpoint{0.787123in}{0.462877in}}{\pgfqpoint{0.777278in}{0.453032in}}{\pgfqpoint{0.763923in}{0.447500in}}%
\pgfpathclose%
\pgfpathmoveto{\pgfqpoint{0.916667in}{0.441667in}}%
\pgfpathcurveto{\pgfqpoint{0.932137in}{0.441667in}}{\pgfqpoint{0.946975in}{0.447813in}}{\pgfqpoint{0.957915in}{0.458752in}}%
\pgfpathcurveto{\pgfqpoint{0.968854in}{0.469691in}}{\pgfqpoint{0.975000in}{0.484530in}}{\pgfqpoint{0.975000in}{0.500000in}}%
\pgfpathcurveto{\pgfqpoint{0.975000in}{0.515470in}}{\pgfqpoint{0.968854in}{0.530309in}}{\pgfqpoint{0.957915in}{0.541248in}}%
\pgfpathcurveto{\pgfqpoint{0.946975in}{0.552187in}}{\pgfqpoint{0.932137in}{0.558333in}}{\pgfqpoint{0.916667in}{0.558333in}}%
\pgfpathcurveto{\pgfqpoint{0.901196in}{0.558333in}}{\pgfqpoint{0.886358in}{0.552187in}}{\pgfqpoint{0.875419in}{0.541248in}}%
\pgfpathcurveto{\pgfqpoint{0.864480in}{0.530309in}}{\pgfqpoint{0.858333in}{0.515470in}}{\pgfqpoint{0.858333in}{0.500000in}}%
\pgfpathcurveto{\pgfqpoint{0.858333in}{0.484530in}}{\pgfqpoint{0.864480in}{0.469691in}}{\pgfqpoint{0.875419in}{0.458752in}}%
\pgfpathcurveto{\pgfqpoint{0.886358in}{0.447813in}}{\pgfqpoint{0.901196in}{0.441667in}}{\pgfqpoint{0.916667in}{0.441667in}}%
\pgfpathclose%
\pgfpathmoveto{\pgfqpoint{0.916667in}{0.447500in}}%
\pgfpathcurveto{\pgfqpoint{0.916667in}{0.447500in}}{\pgfqpoint{0.902744in}{0.447500in}}{\pgfqpoint{0.889389in}{0.453032in}}%
\pgfpathcurveto{\pgfqpoint{0.879544in}{0.462877in}}{\pgfqpoint{0.869698in}{0.472722in}}{\pgfqpoint{0.864167in}{0.486077in}}%
\pgfpathcurveto{\pgfqpoint{0.864167in}{0.500000in}}{\pgfqpoint{0.864167in}{0.513923in}}{\pgfqpoint{0.869698in}{0.527278in}}%
\pgfpathcurveto{\pgfqpoint{0.879544in}{0.537123in}}{\pgfqpoint{0.889389in}{0.546968in}}{\pgfqpoint{0.902744in}{0.552500in}}%
\pgfpathcurveto{\pgfqpoint{0.916667in}{0.552500in}}{\pgfqpoint{0.930590in}{0.552500in}}{\pgfqpoint{0.943945in}{0.546968in}}%
\pgfpathcurveto{\pgfqpoint{0.953790in}{0.537123in}}{\pgfqpoint{0.963635in}{0.527278in}}{\pgfqpoint{0.969167in}{0.513923in}}%
\pgfpathcurveto{\pgfqpoint{0.969167in}{0.500000in}}{\pgfqpoint{0.969167in}{0.486077in}}{\pgfqpoint{0.963635in}{0.472722in}}%
\pgfpathcurveto{\pgfqpoint{0.953790in}{0.462877in}}{\pgfqpoint{0.943945in}{0.453032in}}{\pgfqpoint{0.930590in}{0.447500in}}%
\pgfpathclose%
\pgfpathmoveto{\pgfqpoint{0.000000in}{0.608333in}}%
\pgfpathcurveto{\pgfqpoint{0.015470in}{0.608333in}}{\pgfqpoint{0.030309in}{0.614480in}}{\pgfqpoint{0.041248in}{0.625419in}}%
\pgfpathcurveto{\pgfqpoint{0.052187in}{0.636358in}}{\pgfqpoint{0.058333in}{0.651196in}}{\pgfqpoint{0.058333in}{0.666667in}}%
\pgfpathcurveto{\pgfqpoint{0.058333in}{0.682137in}}{\pgfqpoint{0.052187in}{0.696975in}}{\pgfqpoint{0.041248in}{0.707915in}}%
\pgfpathcurveto{\pgfqpoint{0.030309in}{0.718854in}}{\pgfqpoint{0.015470in}{0.725000in}}{\pgfqpoint{0.000000in}{0.725000in}}%
\pgfpathcurveto{\pgfqpoint{-0.015470in}{0.725000in}}{\pgfqpoint{-0.030309in}{0.718854in}}{\pgfqpoint{-0.041248in}{0.707915in}}%
\pgfpathcurveto{\pgfqpoint{-0.052187in}{0.696975in}}{\pgfqpoint{-0.058333in}{0.682137in}}{\pgfqpoint{-0.058333in}{0.666667in}}%
\pgfpathcurveto{\pgfqpoint{-0.058333in}{0.651196in}}{\pgfqpoint{-0.052187in}{0.636358in}}{\pgfqpoint{-0.041248in}{0.625419in}}%
\pgfpathcurveto{\pgfqpoint{-0.030309in}{0.614480in}}{\pgfqpoint{-0.015470in}{0.608333in}}{\pgfqpoint{0.000000in}{0.608333in}}%
\pgfpathclose%
\pgfpathmoveto{\pgfqpoint{0.000000in}{0.614167in}}%
\pgfpathcurveto{\pgfqpoint{0.000000in}{0.614167in}}{\pgfqpoint{-0.013923in}{0.614167in}}{\pgfqpoint{-0.027278in}{0.619698in}}%
\pgfpathcurveto{\pgfqpoint{-0.037123in}{0.629544in}}{\pgfqpoint{-0.046968in}{0.639389in}}{\pgfqpoint{-0.052500in}{0.652744in}}%
\pgfpathcurveto{\pgfqpoint{-0.052500in}{0.666667in}}{\pgfqpoint{-0.052500in}{0.680590in}}{\pgfqpoint{-0.046968in}{0.693945in}}%
\pgfpathcurveto{\pgfqpoint{-0.037123in}{0.703790in}}{\pgfqpoint{-0.027278in}{0.713635in}}{\pgfqpoint{-0.013923in}{0.719167in}}%
\pgfpathcurveto{\pgfqpoint{0.000000in}{0.719167in}}{\pgfqpoint{0.013923in}{0.719167in}}{\pgfqpoint{0.027278in}{0.713635in}}%
\pgfpathcurveto{\pgfqpoint{0.037123in}{0.703790in}}{\pgfqpoint{0.046968in}{0.693945in}}{\pgfqpoint{0.052500in}{0.680590in}}%
\pgfpathcurveto{\pgfqpoint{0.052500in}{0.666667in}}{\pgfqpoint{0.052500in}{0.652744in}}{\pgfqpoint{0.046968in}{0.639389in}}%
\pgfpathcurveto{\pgfqpoint{0.037123in}{0.629544in}}{\pgfqpoint{0.027278in}{0.619698in}}{\pgfqpoint{0.013923in}{0.614167in}}%
\pgfpathclose%
\pgfpathmoveto{\pgfqpoint{0.166667in}{0.608333in}}%
\pgfpathcurveto{\pgfqpoint{0.182137in}{0.608333in}}{\pgfqpoint{0.196975in}{0.614480in}}{\pgfqpoint{0.207915in}{0.625419in}}%
\pgfpathcurveto{\pgfqpoint{0.218854in}{0.636358in}}{\pgfqpoint{0.225000in}{0.651196in}}{\pgfqpoint{0.225000in}{0.666667in}}%
\pgfpathcurveto{\pgfqpoint{0.225000in}{0.682137in}}{\pgfqpoint{0.218854in}{0.696975in}}{\pgfqpoint{0.207915in}{0.707915in}}%
\pgfpathcurveto{\pgfqpoint{0.196975in}{0.718854in}}{\pgfqpoint{0.182137in}{0.725000in}}{\pgfqpoint{0.166667in}{0.725000in}}%
\pgfpathcurveto{\pgfqpoint{0.151196in}{0.725000in}}{\pgfqpoint{0.136358in}{0.718854in}}{\pgfqpoint{0.125419in}{0.707915in}}%
\pgfpathcurveto{\pgfqpoint{0.114480in}{0.696975in}}{\pgfqpoint{0.108333in}{0.682137in}}{\pgfqpoint{0.108333in}{0.666667in}}%
\pgfpathcurveto{\pgfqpoint{0.108333in}{0.651196in}}{\pgfqpoint{0.114480in}{0.636358in}}{\pgfqpoint{0.125419in}{0.625419in}}%
\pgfpathcurveto{\pgfqpoint{0.136358in}{0.614480in}}{\pgfqpoint{0.151196in}{0.608333in}}{\pgfqpoint{0.166667in}{0.608333in}}%
\pgfpathclose%
\pgfpathmoveto{\pgfqpoint{0.166667in}{0.614167in}}%
\pgfpathcurveto{\pgfqpoint{0.166667in}{0.614167in}}{\pgfqpoint{0.152744in}{0.614167in}}{\pgfqpoint{0.139389in}{0.619698in}}%
\pgfpathcurveto{\pgfqpoint{0.129544in}{0.629544in}}{\pgfqpoint{0.119698in}{0.639389in}}{\pgfqpoint{0.114167in}{0.652744in}}%
\pgfpathcurveto{\pgfqpoint{0.114167in}{0.666667in}}{\pgfqpoint{0.114167in}{0.680590in}}{\pgfqpoint{0.119698in}{0.693945in}}%
\pgfpathcurveto{\pgfqpoint{0.129544in}{0.703790in}}{\pgfqpoint{0.139389in}{0.713635in}}{\pgfqpoint{0.152744in}{0.719167in}}%
\pgfpathcurveto{\pgfqpoint{0.166667in}{0.719167in}}{\pgfqpoint{0.180590in}{0.719167in}}{\pgfqpoint{0.193945in}{0.713635in}}%
\pgfpathcurveto{\pgfqpoint{0.203790in}{0.703790in}}{\pgfqpoint{0.213635in}{0.693945in}}{\pgfqpoint{0.219167in}{0.680590in}}%
\pgfpathcurveto{\pgfqpoint{0.219167in}{0.666667in}}{\pgfqpoint{0.219167in}{0.652744in}}{\pgfqpoint{0.213635in}{0.639389in}}%
\pgfpathcurveto{\pgfqpoint{0.203790in}{0.629544in}}{\pgfqpoint{0.193945in}{0.619698in}}{\pgfqpoint{0.180590in}{0.614167in}}%
\pgfpathclose%
\pgfpathmoveto{\pgfqpoint{0.333333in}{0.608333in}}%
\pgfpathcurveto{\pgfqpoint{0.348804in}{0.608333in}}{\pgfqpoint{0.363642in}{0.614480in}}{\pgfqpoint{0.374581in}{0.625419in}}%
\pgfpathcurveto{\pgfqpoint{0.385520in}{0.636358in}}{\pgfqpoint{0.391667in}{0.651196in}}{\pgfqpoint{0.391667in}{0.666667in}}%
\pgfpathcurveto{\pgfqpoint{0.391667in}{0.682137in}}{\pgfqpoint{0.385520in}{0.696975in}}{\pgfqpoint{0.374581in}{0.707915in}}%
\pgfpathcurveto{\pgfqpoint{0.363642in}{0.718854in}}{\pgfqpoint{0.348804in}{0.725000in}}{\pgfqpoint{0.333333in}{0.725000in}}%
\pgfpathcurveto{\pgfqpoint{0.317863in}{0.725000in}}{\pgfqpoint{0.303025in}{0.718854in}}{\pgfqpoint{0.292085in}{0.707915in}}%
\pgfpathcurveto{\pgfqpoint{0.281146in}{0.696975in}}{\pgfqpoint{0.275000in}{0.682137in}}{\pgfqpoint{0.275000in}{0.666667in}}%
\pgfpathcurveto{\pgfqpoint{0.275000in}{0.651196in}}{\pgfqpoint{0.281146in}{0.636358in}}{\pgfqpoint{0.292085in}{0.625419in}}%
\pgfpathcurveto{\pgfqpoint{0.303025in}{0.614480in}}{\pgfqpoint{0.317863in}{0.608333in}}{\pgfqpoint{0.333333in}{0.608333in}}%
\pgfpathclose%
\pgfpathmoveto{\pgfqpoint{0.333333in}{0.614167in}}%
\pgfpathcurveto{\pgfqpoint{0.333333in}{0.614167in}}{\pgfqpoint{0.319410in}{0.614167in}}{\pgfqpoint{0.306055in}{0.619698in}}%
\pgfpathcurveto{\pgfqpoint{0.296210in}{0.629544in}}{\pgfqpoint{0.286365in}{0.639389in}}{\pgfqpoint{0.280833in}{0.652744in}}%
\pgfpathcurveto{\pgfqpoint{0.280833in}{0.666667in}}{\pgfqpoint{0.280833in}{0.680590in}}{\pgfqpoint{0.286365in}{0.693945in}}%
\pgfpathcurveto{\pgfqpoint{0.296210in}{0.703790in}}{\pgfqpoint{0.306055in}{0.713635in}}{\pgfqpoint{0.319410in}{0.719167in}}%
\pgfpathcurveto{\pgfqpoint{0.333333in}{0.719167in}}{\pgfqpoint{0.347256in}{0.719167in}}{\pgfqpoint{0.360611in}{0.713635in}}%
\pgfpathcurveto{\pgfqpoint{0.370456in}{0.703790in}}{\pgfqpoint{0.380302in}{0.693945in}}{\pgfqpoint{0.385833in}{0.680590in}}%
\pgfpathcurveto{\pgfqpoint{0.385833in}{0.666667in}}{\pgfqpoint{0.385833in}{0.652744in}}{\pgfqpoint{0.380302in}{0.639389in}}%
\pgfpathcurveto{\pgfqpoint{0.370456in}{0.629544in}}{\pgfqpoint{0.360611in}{0.619698in}}{\pgfqpoint{0.347256in}{0.614167in}}%
\pgfpathclose%
\pgfpathmoveto{\pgfqpoint{0.500000in}{0.608333in}}%
\pgfpathcurveto{\pgfqpoint{0.515470in}{0.608333in}}{\pgfqpoint{0.530309in}{0.614480in}}{\pgfqpoint{0.541248in}{0.625419in}}%
\pgfpathcurveto{\pgfqpoint{0.552187in}{0.636358in}}{\pgfqpoint{0.558333in}{0.651196in}}{\pgfqpoint{0.558333in}{0.666667in}}%
\pgfpathcurveto{\pgfqpoint{0.558333in}{0.682137in}}{\pgfqpoint{0.552187in}{0.696975in}}{\pgfqpoint{0.541248in}{0.707915in}}%
\pgfpathcurveto{\pgfqpoint{0.530309in}{0.718854in}}{\pgfqpoint{0.515470in}{0.725000in}}{\pgfqpoint{0.500000in}{0.725000in}}%
\pgfpathcurveto{\pgfqpoint{0.484530in}{0.725000in}}{\pgfqpoint{0.469691in}{0.718854in}}{\pgfqpoint{0.458752in}{0.707915in}}%
\pgfpathcurveto{\pgfqpoint{0.447813in}{0.696975in}}{\pgfqpoint{0.441667in}{0.682137in}}{\pgfqpoint{0.441667in}{0.666667in}}%
\pgfpathcurveto{\pgfqpoint{0.441667in}{0.651196in}}{\pgfqpoint{0.447813in}{0.636358in}}{\pgfqpoint{0.458752in}{0.625419in}}%
\pgfpathcurveto{\pgfqpoint{0.469691in}{0.614480in}}{\pgfqpoint{0.484530in}{0.608333in}}{\pgfqpoint{0.500000in}{0.608333in}}%
\pgfpathclose%
\pgfpathmoveto{\pgfqpoint{0.500000in}{0.614167in}}%
\pgfpathcurveto{\pgfqpoint{0.500000in}{0.614167in}}{\pgfqpoint{0.486077in}{0.614167in}}{\pgfqpoint{0.472722in}{0.619698in}}%
\pgfpathcurveto{\pgfqpoint{0.462877in}{0.629544in}}{\pgfqpoint{0.453032in}{0.639389in}}{\pgfqpoint{0.447500in}{0.652744in}}%
\pgfpathcurveto{\pgfqpoint{0.447500in}{0.666667in}}{\pgfqpoint{0.447500in}{0.680590in}}{\pgfqpoint{0.453032in}{0.693945in}}%
\pgfpathcurveto{\pgfqpoint{0.462877in}{0.703790in}}{\pgfqpoint{0.472722in}{0.713635in}}{\pgfqpoint{0.486077in}{0.719167in}}%
\pgfpathcurveto{\pgfqpoint{0.500000in}{0.719167in}}{\pgfqpoint{0.513923in}{0.719167in}}{\pgfqpoint{0.527278in}{0.713635in}}%
\pgfpathcurveto{\pgfqpoint{0.537123in}{0.703790in}}{\pgfqpoint{0.546968in}{0.693945in}}{\pgfqpoint{0.552500in}{0.680590in}}%
\pgfpathcurveto{\pgfqpoint{0.552500in}{0.666667in}}{\pgfqpoint{0.552500in}{0.652744in}}{\pgfqpoint{0.546968in}{0.639389in}}%
\pgfpathcurveto{\pgfqpoint{0.537123in}{0.629544in}}{\pgfqpoint{0.527278in}{0.619698in}}{\pgfqpoint{0.513923in}{0.614167in}}%
\pgfpathclose%
\pgfpathmoveto{\pgfqpoint{0.666667in}{0.608333in}}%
\pgfpathcurveto{\pgfqpoint{0.682137in}{0.608333in}}{\pgfqpoint{0.696975in}{0.614480in}}{\pgfqpoint{0.707915in}{0.625419in}}%
\pgfpathcurveto{\pgfqpoint{0.718854in}{0.636358in}}{\pgfqpoint{0.725000in}{0.651196in}}{\pgfqpoint{0.725000in}{0.666667in}}%
\pgfpathcurveto{\pgfqpoint{0.725000in}{0.682137in}}{\pgfqpoint{0.718854in}{0.696975in}}{\pgfqpoint{0.707915in}{0.707915in}}%
\pgfpathcurveto{\pgfqpoint{0.696975in}{0.718854in}}{\pgfqpoint{0.682137in}{0.725000in}}{\pgfqpoint{0.666667in}{0.725000in}}%
\pgfpathcurveto{\pgfqpoint{0.651196in}{0.725000in}}{\pgfqpoint{0.636358in}{0.718854in}}{\pgfqpoint{0.625419in}{0.707915in}}%
\pgfpathcurveto{\pgfqpoint{0.614480in}{0.696975in}}{\pgfqpoint{0.608333in}{0.682137in}}{\pgfqpoint{0.608333in}{0.666667in}}%
\pgfpathcurveto{\pgfqpoint{0.608333in}{0.651196in}}{\pgfqpoint{0.614480in}{0.636358in}}{\pgfqpoint{0.625419in}{0.625419in}}%
\pgfpathcurveto{\pgfqpoint{0.636358in}{0.614480in}}{\pgfqpoint{0.651196in}{0.608333in}}{\pgfqpoint{0.666667in}{0.608333in}}%
\pgfpathclose%
\pgfpathmoveto{\pgfqpoint{0.666667in}{0.614167in}}%
\pgfpathcurveto{\pgfqpoint{0.666667in}{0.614167in}}{\pgfqpoint{0.652744in}{0.614167in}}{\pgfqpoint{0.639389in}{0.619698in}}%
\pgfpathcurveto{\pgfqpoint{0.629544in}{0.629544in}}{\pgfqpoint{0.619698in}{0.639389in}}{\pgfqpoint{0.614167in}{0.652744in}}%
\pgfpathcurveto{\pgfqpoint{0.614167in}{0.666667in}}{\pgfqpoint{0.614167in}{0.680590in}}{\pgfqpoint{0.619698in}{0.693945in}}%
\pgfpathcurveto{\pgfqpoint{0.629544in}{0.703790in}}{\pgfqpoint{0.639389in}{0.713635in}}{\pgfqpoint{0.652744in}{0.719167in}}%
\pgfpathcurveto{\pgfqpoint{0.666667in}{0.719167in}}{\pgfqpoint{0.680590in}{0.719167in}}{\pgfqpoint{0.693945in}{0.713635in}}%
\pgfpathcurveto{\pgfqpoint{0.703790in}{0.703790in}}{\pgfqpoint{0.713635in}{0.693945in}}{\pgfqpoint{0.719167in}{0.680590in}}%
\pgfpathcurveto{\pgfqpoint{0.719167in}{0.666667in}}{\pgfqpoint{0.719167in}{0.652744in}}{\pgfqpoint{0.713635in}{0.639389in}}%
\pgfpathcurveto{\pgfqpoint{0.703790in}{0.629544in}}{\pgfqpoint{0.693945in}{0.619698in}}{\pgfqpoint{0.680590in}{0.614167in}}%
\pgfpathclose%
\pgfpathmoveto{\pgfqpoint{0.833333in}{0.608333in}}%
\pgfpathcurveto{\pgfqpoint{0.848804in}{0.608333in}}{\pgfqpoint{0.863642in}{0.614480in}}{\pgfqpoint{0.874581in}{0.625419in}}%
\pgfpathcurveto{\pgfqpoint{0.885520in}{0.636358in}}{\pgfqpoint{0.891667in}{0.651196in}}{\pgfqpoint{0.891667in}{0.666667in}}%
\pgfpathcurveto{\pgfqpoint{0.891667in}{0.682137in}}{\pgfqpoint{0.885520in}{0.696975in}}{\pgfqpoint{0.874581in}{0.707915in}}%
\pgfpathcurveto{\pgfqpoint{0.863642in}{0.718854in}}{\pgfqpoint{0.848804in}{0.725000in}}{\pgfqpoint{0.833333in}{0.725000in}}%
\pgfpathcurveto{\pgfqpoint{0.817863in}{0.725000in}}{\pgfqpoint{0.803025in}{0.718854in}}{\pgfqpoint{0.792085in}{0.707915in}}%
\pgfpathcurveto{\pgfqpoint{0.781146in}{0.696975in}}{\pgfqpoint{0.775000in}{0.682137in}}{\pgfqpoint{0.775000in}{0.666667in}}%
\pgfpathcurveto{\pgfqpoint{0.775000in}{0.651196in}}{\pgfqpoint{0.781146in}{0.636358in}}{\pgfqpoint{0.792085in}{0.625419in}}%
\pgfpathcurveto{\pgfqpoint{0.803025in}{0.614480in}}{\pgfqpoint{0.817863in}{0.608333in}}{\pgfqpoint{0.833333in}{0.608333in}}%
\pgfpathclose%
\pgfpathmoveto{\pgfqpoint{0.833333in}{0.614167in}}%
\pgfpathcurveto{\pgfqpoint{0.833333in}{0.614167in}}{\pgfqpoint{0.819410in}{0.614167in}}{\pgfqpoint{0.806055in}{0.619698in}}%
\pgfpathcurveto{\pgfqpoint{0.796210in}{0.629544in}}{\pgfqpoint{0.786365in}{0.639389in}}{\pgfqpoint{0.780833in}{0.652744in}}%
\pgfpathcurveto{\pgfqpoint{0.780833in}{0.666667in}}{\pgfqpoint{0.780833in}{0.680590in}}{\pgfqpoint{0.786365in}{0.693945in}}%
\pgfpathcurveto{\pgfqpoint{0.796210in}{0.703790in}}{\pgfqpoint{0.806055in}{0.713635in}}{\pgfqpoint{0.819410in}{0.719167in}}%
\pgfpathcurveto{\pgfqpoint{0.833333in}{0.719167in}}{\pgfqpoint{0.847256in}{0.719167in}}{\pgfqpoint{0.860611in}{0.713635in}}%
\pgfpathcurveto{\pgfqpoint{0.870456in}{0.703790in}}{\pgfqpoint{0.880302in}{0.693945in}}{\pgfqpoint{0.885833in}{0.680590in}}%
\pgfpathcurveto{\pgfqpoint{0.885833in}{0.666667in}}{\pgfqpoint{0.885833in}{0.652744in}}{\pgfqpoint{0.880302in}{0.639389in}}%
\pgfpathcurveto{\pgfqpoint{0.870456in}{0.629544in}}{\pgfqpoint{0.860611in}{0.619698in}}{\pgfqpoint{0.847256in}{0.614167in}}%
\pgfpathclose%
\pgfpathmoveto{\pgfqpoint{1.000000in}{0.608333in}}%
\pgfpathcurveto{\pgfqpoint{1.015470in}{0.608333in}}{\pgfqpoint{1.030309in}{0.614480in}}{\pgfqpoint{1.041248in}{0.625419in}}%
\pgfpathcurveto{\pgfqpoint{1.052187in}{0.636358in}}{\pgfqpoint{1.058333in}{0.651196in}}{\pgfqpoint{1.058333in}{0.666667in}}%
\pgfpathcurveto{\pgfqpoint{1.058333in}{0.682137in}}{\pgfqpoint{1.052187in}{0.696975in}}{\pgfqpoint{1.041248in}{0.707915in}}%
\pgfpathcurveto{\pgfqpoint{1.030309in}{0.718854in}}{\pgfqpoint{1.015470in}{0.725000in}}{\pgfqpoint{1.000000in}{0.725000in}}%
\pgfpathcurveto{\pgfqpoint{0.984530in}{0.725000in}}{\pgfqpoint{0.969691in}{0.718854in}}{\pgfqpoint{0.958752in}{0.707915in}}%
\pgfpathcurveto{\pgfqpoint{0.947813in}{0.696975in}}{\pgfqpoint{0.941667in}{0.682137in}}{\pgfqpoint{0.941667in}{0.666667in}}%
\pgfpathcurveto{\pgfqpoint{0.941667in}{0.651196in}}{\pgfqpoint{0.947813in}{0.636358in}}{\pgfqpoint{0.958752in}{0.625419in}}%
\pgfpathcurveto{\pgfqpoint{0.969691in}{0.614480in}}{\pgfqpoint{0.984530in}{0.608333in}}{\pgfqpoint{1.000000in}{0.608333in}}%
\pgfpathclose%
\pgfpathmoveto{\pgfqpoint{1.000000in}{0.614167in}}%
\pgfpathcurveto{\pgfqpoint{1.000000in}{0.614167in}}{\pgfqpoint{0.986077in}{0.614167in}}{\pgfqpoint{0.972722in}{0.619698in}}%
\pgfpathcurveto{\pgfqpoint{0.962877in}{0.629544in}}{\pgfqpoint{0.953032in}{0.639389in}}{\pgfqpoint{0.947500in}{0.652744in}}%
\pgfpathcurveto{\pgfqpoint{0.947500in}{0.666667in}}{\pgfqpoint{0.947500in}{0.680590in}}{\pgfqpoint{0.953032in}{0.693945in}}%
\pgfpathcurveto{\pgfqpoint{0.962877in}{0.703790in}}{\pgfqpoint{0.972722in}{0.713635in}}{\pgfqpoint{0.986077in}{0.719167in}}%
\pgfpathcurveto{\pgfqpoint{1.000000in}{0.719167in}}{\pgfqpoint{1.013923in}{0.719167in}}{\pgfqpoint{1.027278in}{0.713635in}}%
\pgfpathcurveto{\pgfqpoint{1.037123in}{0.703790in}}{\pgfqpoint{1.046968in}{0.693945in}}{\pgfqpoint{1.052500in}{0.680590in}}%
\pgfpathcurveto{\pgfqpoint{1.052500in}{0.666667in}}{\pgfqpoint{1.052500in}{0.652744in}}{\pgfqpoint{1.046968in}{0.639389in}}%
\pgfpathcurveto{\pgfqpoint{1.037123in}{0.629544in}}{\pgfqpoint{1.027278in}{0.619698in}}{\pgfqpoint{1.013923in}{0.614167in}}%
\pgfpathclose%
\pgfpathmoveto{\pgfqpoint{0.083333in}{0.775000in}}%
\pgfpathcurveto{\pgfqpoint{0.098804in}{0.775000in}}{\pgfqpoint{0.113642in}{0.781146in}}{\pgfqpoint{0.124581in}{0.792085in}}%
\pgfpathcurveto{\pgfqpoint{0.135520in}{0.803025in}}{\pgfqpoint{0.141667in}{0.817863in}}{\pgfqpoint{0.141667in}{0.833333in}}%
\pgfpathcurveto{\pgfqpoint{0.141667in}{0.848804in}}{\pgfqpoint{0.135520in}{0.863642in}}{\pgfqpoint{0.124581in}{0.874581in}}%
\pgfpathcurveto{\pgfqpoint{0.113642in}{0.885520in}}{\pgfqpoint{0.098804in}{0.891667in}}{\pgfqpoint{0.083333in}{0.891667in}}%
\pgfpathcurveto{\pgfqpoint{0.067863in}{0.891667in}}{\pgfqpoint{0.053025in}{0.885520in}}{\pgfqpoint{0.042085in}{0.874581in}}%
\pgfpathcurveto{\pgfqpoint{0.031146in}{0.863642in}}{\pgfqpoint{0.025000in}{0.848804in}}{\pgfqpoint{0.025000in}{0.833333in}}%
\pgfpathcurveto{\pgfqpoint{0.025000in}{0.817863in}}{\pgfqpoint{0.031146in}{0.803025in}}{\pgfqpoint{0.042085in}{0.792085in}}%
\pgfpathcurveto{\pgfqpoint{0.053025in}{0.781146in}}{\pgfqpoint{0.067863in}{0.775000in}}{\pgfqpoint{0.083333in}{0.775000in}}%
\pgfpathclose%
\pgfpathmoveto{\pgfqpoint{0.083333in}{0.780833in}}%
\pgfpathcurveto{\pgfqpoint{0.083333in}{0.780833in}}{\pgfqpoint{0.069410in}{0.780833in}}{\pgfqpoint{0.056055in}{0.786365in}}%
\pgfpathcurveto{\pgfqpoint{0.046210in}{0.796210in}}{\pgfqpoint{0.036365in}{0.806055in}}{\pgfqpoint{0.030833in}{0.819410in}}%
\pgfpathcurveto{\pgfqpoint{0.030833in}{0.833333in}}{\pgfqpoint{0.030833in}{0.847256in}}{\pgfqpoint{0.036365in}{0.860611in}}%
\pgfpathcurveto{\pgfqpoint{0.046210in}{0.870456in}}{\pgfqpoint{0.056055in}{0.880302in}}{\pgfqpoint{0.069410in}{0.885833in}}%
\pgfpathcurveto{\pgfqpoint{0.083333in}{0.885833in}}{\pgfqpoint{0.097256in}{0.885833in}}{\pgfqpoint{0.110611in}{0.880302in}}%
\pgfpathcurveto{\pgfqpoint{0.120456in}{0.870456in}}{\pgfqpoint{0.130302in}{0.860611in}}{\pgfqpoint{0.135833in}{0.847256in}}%
\pgfpathcurveto{\pgfqpoint{0.135833in}{0.833333in}}{\pgfqpoint{0.135833in}{0.819410in}}{\pgfqpoint{0.130302in}{0.806055in}}%
\pgfpathcurveto{\pgfqpoint{0.120456in}{0.796210in}}{\pgfqpoint{0.110611in}{0.786365in}}{\pgfqpoint{0.097256in}{0.780833in}}%
\pgfpathclose%
\pgfpathmoveto{\pgfqpoint{0.250000in}{0.775000in}}%
\pgfpathcurveto{\pgfqpoint{0.265470in}{0.775000in}}{\pgfqpoint{0.280309in}{0.781146in}}{\pgfqpoint{0.291248in}{0.792085in}}%
\pgfpathcurveto{\pgfqpoint{0.302187in}{0.803025in}}{\pgfqpoint{0.308333in}{0.817863in}}{\pgfqpoint{0.308333in}{0.833333in}}%
\pgfpathcurveto{\pgfqpoint{0.308333in}{0.848804in}}{\pgfqpoint{0.302187in}{0.863642in}}{\pgfqpoint{0.291248in}{0.874581in}}%
\pgfpathcurveto{\pgfqpoint{0.280309in}{0.885520in}}{\pgfqpoint{0.265470in}{0.891667in}}{\pgfqpoint{0.250000in}{0.891667in}}%
\pgfpathcurveto{\pgfqpoint{0.234530in}{0.891667in}}{\pgfqpoint{0.219691in}{0.885520in}}{\pgfqpoint{0.208752in}{0.874581in}}%
\pgfpathcurveto{\pgfqpoint{0.197813in}{0.863642in}}{\pgfqpoint{0.191667in}{0.848804in}}{\pgfqpoint{0.191667in}{0.833333in}}%
\pgfpathcurveto{\pgfqpoint{0.191667in}{0.817863in}}{\pgfqpoint{0.197813in}{0.803025in}}{\pgfqpoint{0.208752in}{0.792085in}}%
\pgfpathcurveto{\pgfqpoint{0.219691in}{0.781146in}}{\pgfqpoint{0.234530in}{0.775000in}}{\pgfqpoint{0.250000in}{0.775000in}}%
\pgfpathclose%
\pgfpathmoveto{\pgfqpoint{0.250000in}{0.780833in}}%
\pgfpathcurveto{\pgfqpoint{0.250000in}{0.780833in}}{\pgfqpoint{0.236077in}{0.780833in}}{\pgfqpoint{0.222722in}{0.786365in}}%
\pgfpathcurveto{\pgfqpoint{0.212877in}{0.796210in}}{\pgfqpoint{0.203032in}{0.806055in}}{\pgfqpoint{0.197500in}{0.819410in}}%
\pgfpathcurveto{\pgfqpoint{0.197500in}{0.833333in}}{\pgfqpoint{0.197500in}{0.847256in}}{\pgfqpoint{0.203032in}{0.860611in}}%
\pgfpathcurveto{\pgfqpoint{0.212877in}{0.870456in}}{\pgfqpoint{0.222722in}{0.880302in}}{\pgfqpoint{0.236077in}{0.885833in}}%
\pgfpathcurveto{\pgfqpoint{0.250000in}{0.885833in}}{\pgfqpoint{0.263923in}{0.885833in}}{\pgfqpoint{0.277278in}{0.880302in}}%
\pgfpathcurveto{\pgfqpoint{0.287123in}{0.870456in}}{\pgfqpoint{0.296968in}{0.860611in}}{\pgfqpoint{0.302500in}{0.847256in}}%
\pgfpathcurveto{\pgfqpoint{0.302500in}{0.833333in}}{\pgfqpoint{0.302500in}{0.819410in}}{\pgfqpoint{0.296968in}{0.806055in}}%
\pgfpathcurveto{\pgfqpoint{0.287123in}{0.796210in}}{\pgfqpoint{0.277278in}{0.786365in}}{\pgfqpoint{0.263923in}{0.780833in}}%
\pgfpathclose%
\pgfpathmoveto{\pgfqpoint{0.416667in}{0.775000in}}%
\pgfpathcurveto{\pgfqpoint{0.432137in}{0.775000in}}{\pgfqpoint{0.446975in}{0.781146in}}{\pgfqpoint{0.457915in}{0.792085in}}%
\pgfpathcurveto{\pgfqpoint{0.468854in}{0.803025in}}{\pgfqpoint{0.475000in}{0.817863in}}{\pgfqpoint{0.475000in}{0.833333in}}%
\pgfpathcurveto{\pgfqpoint{0.475000in}{0.848804in}}{\pgfqpoint{0.468854in}{0.863642in}}{\pgfqpoint{0.457915in}{0.874581in}}%
\pgfpathcurveto{\pgfqpoint{0.446975in}{0.885520in}}{\pgfqpoint{0.432137in}{0.891667in}}{\pgfqpoint{0.416667in}{0.891667in}}%
\pgfpathcurveto{\pgfqpoint{0.401196in}{0.891667in}}{\pgfqpoint{0.386358in}{0.885520in}}{\pgfqpoint{0.375419in}{0.874581in}}%
\pgfpathcurveto{\pgfqpoint{0.364480in}{0.863642in}}{\pgfqpoint{0.358333in}{0.848804in}}{\pgfqpoint{0.358333in}{0.833333in}}%
\pgfpathcurveto{\pgfqpoint{0.358333in}{0.817863in}}{\pgfqpoint{0.364480in}{0.803025in}}{\pgfqpoint{0.375419in}{0.792085in}}%
\pgfpathcurveto{\pgfqpoint{0.386358in}{0.781146in}}{\pgfqpoint{0.401196in}{0.775000in}}{\pgfqpoint{0.416667in}{0.775000in}}%
\pgfpathclose%
\pgfpathmoveto{\pgfqpoint{0.416667in}{0.780833in}}%
\pgfpathcurveto{\pgfqpoint{0.416667in}{0.780833in}}{\pgfqpoint{0.402744in}{0.780833in}}{\pgfqpoint{0.389389in}{0.786365in}}%
\pgfpathcurveto{\pgfqpoint{0.379544in}{0.796210in}}{\pgfqpoint{0.369698in}{0.806055in}}{\pgfqpoint{0.364167in}{0.819410in}}%
\pgfpathcurveto{\pgfqpoint{0.364167in}{0.833333in}}{\pgfqpoint{0.364167in}{0.847256in}}{\pgfqpoint{0.369698in}{0.860611in}}%
\pgfpathcurveto{\pgfqpoint{0.379544in}{0.870456in}}{\pgfqpoint{0.389389in}{0.880302in}}{\pgfqpoint{0.402744in}{0.885833in}}%
\pgfpathcurveto{\pgfqpoint{0.416667in}{0.885833in}}{\pgfqpoint{0.430590in}{0.885833in}}{\pgfqpoint{0.443945in}{0.880302in}}%
\pgfpathcurveto{\pgfqpoint{0.453790in}{0.870456in}}{\pgfqpoint{0.463635in}{0.860611in}}{\pgfqpoint{0.469167in}{0.847256in}}%
\pgfpathcurveto{\pgfqpoint{0.469167in}{0.833333in}}{\pgfqpoint{0.469167in}{0.819410in}}{\pgfqpoint{0.463635in}{0.806055in}}%
\pgfpathcurveto{\pgfqpoint{0.453790in}{0.796210in}}{\pgfqpoint{0.443945in}{0.786365in}}{\pgfqpoint{0.430590in}{0.780833in}}%
\pgfpathclose%
\pgfpathmoveto{\pgfqpoint{0.583333in}{0.775000in}}%
\pgfpathcurveto{\pgfqpoint{0.598804in}{0.775000in}}{\pgfqpoint{0.613642in}{0.781146in}}{\pgfqpoint{0.624581in}{0.792085in}}%
\pgfpathcurveto{\pgfqpoint{0.635520in}{0.803025in}}{\pgfqpoint{0.641667in}{0.817863in}}{\pgfqpoint{0.641667in}{0.833333in}}%
\pgfpathcurveto{\pgfqpoint{0.641667in}{0.848804in}}{\pgfqpoint{0.635520in}{0.863642in}}{\pgfqpoint{0.624581in}{0.874581in}}%
\pgfpathcurveto{\pgfqpoint{0.613642in}{0.885520in}}{\pgfqpoint{0.598804in}{0.891667in}}{\pgfqpoint{0.583333in}{0.891667in}}%
\pgfpathcurveto{\pgfqpoint{0.567863in}{0.891667in}}{\pgfqpoint{0.553025in}{0.885520in}}{\pgfqpoint{0.542085in}{0.874581in}}%
\pgfpathcurveto{\pgfqpoint{0.531146in}{0.863642in}}{\pgfqpoint{0.525000in}{0.848804in}}{\pgfqpoint{0.525000in}{0.833333in}}%
\pgfpathcurveto{\pgfqpoint{0.525000in}{0.817863in}}{\pgfqpoint{0.531146in}{0.803025in}}{\pgfqpoint{0.542085in}{0.792085in}}%
\pgfpathcurveto{\pgfqpoint{0.553025in}{0.781146in}}{\pgfqpoint{0.567863in}{0.775000in}}{\pgfqpoint{0.583333in}{0.775000in}}%
\pgfpathclose%
\pgfpathmoveto{\pgfqpoint{0.583333in}{0.780833in}}%
\pgfpathcurveto{\pgfqpoint{0.583333in}{0.780833in}}{\pgfqpoint{0.569410in}{0.780833in}}{\pgfqpoint{0.556055in}{0.786365in}}%
\pgfpathcurveto{\pgfqpoint{0.546210in}{0.796210in}}{\pgfqpoint{0.536365in}{0.806055in}}{\pgfqpoint{0.530833in}{0.819410in}}%
\pgfpathcurveto{\pgfqpoint{0.530833in}{0.833333in}}{\pgfqpoint{0.530833in}{0.847256in}}{\pgfqpoint{0.536365in}{0.860611in}}%
\pgfpathcurveto{\pgfqpoint{0.546210in}{0.870456in}}{\pgfqpoint{0.556055in}{0.880302in}}{\pgfqpoint{0.569410in}{0.885833in}}%
\pgfpathcurveto{\pgfqpoint{0.583333in}{0.885833in}}{\pgfqpoint{0.597256in}{0.885833in}}{\pgfqpoint{0.610611in}{0.880302in}}%
\pgfpathcurveto{\pgfqpoint{0.620456in}{0.870456in}}{\pgfqpoint{0.630302in}{0.860611in}}{\pgfqpoint{0.635833in}{0.847256in}}%
\pgfpathcurveto{\pgfqpoint{0.635833in}{0.833333in}}{\pgfqpoint{0.635833in}{0.819410in}}{\pgfqpoint{0.630302in}{0.806055in}}%
\pgfpathcurveto{\pgfqpoint{0.620456in}{0.796210in}}{\pgfqpoint{0.610611in}{0.786365in}}{\pgfqpoint{0.597256in}{0.780833in}}%
\pgfpathclose%
\pgfpathmoveto{\pgfqpoint{0.750000in}{0.775000in}}%
\pgfpathcurveto{\pgfqpoint{0.765470in}{0.775000in}}{\pgfqpoint{0.780309in}{0.781146in}}{\pgfqpoint{0.791248in}{0.792085in}}%
\pgfpathcurveto{\pgfqpoint{0.802187in}{0.803025in}}{\pgfqpoint{0.808333in}{0.817863in}}{\pgfqpoint{0.808333in}{0.833333in}}%
\pgfpathcurveto{\pgfqpoint{0.808333in}{0.848804in}}{\pgfqpoint{0.802187in}{0.863642in}}{\pgfqpoint{0.791248in}{0.874581in}}%
\pgfpathcurveto{\pgfqpoint{0.780309in}{0.885520in}}{\pgfqpoint{0.765470in}{0.891667in}}{\pgfqpoint{0.750000in}{0.891667in}}%
\pgfpathcurveto{\pgfqpoint{0.734530in}{0.891667in}}{\pgfqpoint{0.719691in}{0.885520in}}{\pgfqpoint{0.708752in}{0.874581in}}%
\pgfpathcurveto{\pgfqpoint{0.697813in}{0.863642in}}{\pgfqpoint{0.691667in}{0.848804in}}{\pgfqpoint{0.691667in}{0.833333in}}%
\pgfpathcurveto{\pgfqpoint{0.691667in}{0.817863in}}{\pgfqpoint{0.697813in}{0.803025in}}{\pgfqpoint{0.708752in}{0.792085in}}%
\pgfpathcurveto{\pgfqpoint{0.719691in}{0.781146in}}{\pgfqpoint{0.734530in}{0.775000in}}{\pgfqpoint{0.750000in}{0.775000in}}%
\pgfpathclose%
\pgfpathmoveto{\pgfqpoint{0.750000in}{0.780833in}}%
\pgfpathcurveto{\pgfqpoint{0.750000in}{0.780833in}}{\pgfqpoint{0.736077in}{0.780833in}}{\pgfqpoint{0.722722in}{0.786365in}}%
\pgfpathcurveto{\pgfqpoint{0.712877in}{0.796210in}}{\pgfqpoint{0.703032in}{0.806055in}}{\pgfqpoint{0.697500in}{0.819410in}}%
\pgfpathcurveto{\pgfqpoint{0.697500in}{0.833333in}}{\pgfqpoint{0.697500in}{0.847256in}}{\pgfqpoint{0.703032in}{0.860611in}}%
\pgfpathcurveto{\pgfqpoint{0.712877in}{0.870456in}}{\pgfqpoint{0.722722in}{0.880302in}}{\pgfqpoint{0.736077in}{0.885833in}}%
\pgfpathcurveto{\pgfqpoint{0.750000in}{0.885833in}}{\pgfqpoint{0.763923in}{0.885833in}}{\pgfqpoint{0.777278in}{0.880302in}}%
\pgfpathcurveto{\pgfqpoint{0.787123in}{0.870456in}}{\pgfqpoint{0.796968in}{0.860611in}}{\pgfqpoint{0.802500in}{0.847256in}}%
\pgfpathcurveto{\pgfqpoint{0.802500in}{0.833333in}}{\pgfqpoint{0.802500in}{0.819410in}}{\pgfqpoint{0.796968in}{0.806055in}}%
\pgfpathcurveto{\pgfqpoint{0.787123in}{0.796210in}}{\pgfqpoint{0.777278in}{0.786365in}}{\pgfqpoint{0.763923in}{0.780833in}}%
\pgfpathclose%
\pgfpathmoveto{\pgfqpoint{0.916667in}{0.775000in}}%
\pgfpathcurveto{\pgfqpoint{0.932137in}{0.775000in}}{\pgfqpoint{0.946975in}{0.781146in}}{\pgfqpoint{0.957915in}{0.792085in}}%
\pgfpathcurveto{\pgfqpoint{0.968854in}{0.803025in}}{\pgfqpoint{0.975000in}{0.817863in}}{\pgfqpoint{0.975000in}{0.833333in}}%
\pgfpathcurveto{\pgfqpoint{0.975000in}{0.848804in}}{\pgfqpoint{0.968854in}{0.863642in}}{\pgfqpoint{0.957915in}{0.874581in}}%
\pgfpathcurveto{\pgfqpoint{0.946975in}{0.885520in}}{\pgfqpoint{0.932137in}{0.891667in}}{\pgfqpoint{0.916667in}{0.891667in}}%
\pgfpathcurveto{\pgfqpoint{0.901196in}{0.891667in}}{\pgfqpoint{0.886358in}{0.885520in}}{\pgfqpoint{0.875419in}{0.874581in}}%
\pgfpathcurveto{\pgfqpoint{0.864480in}{0.863642in}}{\pgfqpoint{0.858333in}{0.848804in}}{\pgfqpoint{0.858333in}{0.833333in}}%
\pgfpathcurveto{\pgfqpoint{0.858333in}{0.817863in}}{\pgfqpoint{0.864480in}{0.803025in}}{\pgfqpoint{0.875419in}{0.792085in}}%
\pgfpathcurveto{\pgfqpoint{0.886358in}{0.781146in}}{\pgfqpoint{0.901196in}{0.775000in}}{\pgfqpoint{0.916667in}{0.775000in}}%
\pgfpathclose%
\pgfpathmoveto{\pgfqpoint{0.916667in}{0.780833in}}%
\pgfpathcurveto{\pgfqpoint{0.916667in}{0.780833in}}{\pgfqpoint{0.902744in}{0.780833in}}{\pgfqpoint{0.889389in}{0.786365in}}%
\pgfpathcurveto{\pgfqpoint{0.879544in}{0.796210in}}{\pgfqpoint{0.869698in}{0.806055in}}{\pgfqpoint{0.864167in}{0.819410in}}%
\pgfpathcurveto{\pgfqpoint{0.864167in}{0.833333in}}{\pgfqpoint{0.864167in}{0.847256in}}{\pgfqpoint{0.869698in}{0.860611in}}%
\pgfpathcurveto{\pgfqpoint{0.879544in}{0.870456in}}{\pgfqpoint{0.889389in}{0.880302in}}{\pgfqpoint{0.902744in}{0.885833in}}%
\pgfpathcurveto{\pgfqpoint{0.916667in}{0.885833in}}{\pgfqpoint{0.930590in}{0.885833in}}{\pgfqpoint{0.943945in}{0.880302in}}%
\pgfpathcurveto{\pgfqpoint{0.953790in}{0.870456in}}{\pgfqpoint{0.963635in}{0.860611in}}{\pgfqpoint{0.969167in}{0.847256in}}%
\pgfpathcurveto{\pgfqpoint{0.969167in}{0.833333in}}{\pgfqpoint{0.969167in}{0.819410in}}{\pgfqpoint{0.963635in}{0.806055in}}%
\pgfpathcurveto{\pgfqpoint{0.953790in}{0.796210in}}{\pgfqpoint{0.943945in}{0.786365in}}{\pgfqpoint{0.930590in}{0.780833in}}%
\pgfpathclose%
\pgfpathmoveto{\pgfqpoint{0.000000in}{0.941667in}}%
\pgfpathcurveto{\pgfqpoint{0.015470in}{0.941667in}}{\pgfqpoint{0.030309in}{0.947813in}}{\pgfqpoint{0.041248in}{0.958752in}}%
\pgfpathcurveto{\pgfqpoint{0.052187in}{0.969691in}}{\pgfqpoint{0.058333in}{0.984530in}}{\pgfqpoint{0.058333in}{1.000000in}}%
\pgfpathcurveto{\pgfqpoint{0.058333in}{1.015470in}}{\pgfqpoint{0.052187in}{1.030309in}}{\pgfqpoint{0.041248in}{1.041248in}}%
\pgfpathcurveto{\pgfqpoint{0.030309in}{1.052187in}}{\pgfqpoint{0.015470in}{1.058333in}}{\pgfqpoint{0.000000in}{1.058333in}}%
\pgfpathcurveto{\pgfqpoint{-0.015470in}{1.058333in}}{\pgfqpoint{-0.030309in}{1.052187in}}{\pgfqpoint{-0.041248in}{1.041248in}}%
\pgfpathcurveto{\pgfqpoint{-0.052187in}{1.030309in}}{\pgfqpoint{-0.058333in}{1.015470in}}{\pgfqpoint{-0.058333in}{1.000000in}}%
\pgfpathcurveto{\pgfqpoint{-0.058333in}{0.984530in}}{\pgfqpoint{-0.052187in}{0.969691in}}{\pgfqpoint{-0.041248in}{0.958752in}}%
\pgfpathcurveto{\pgfqpoint{-0.030309in}{0.947813in}}{\pgfqpoint{-0.015470in}{0.941667in}}{\pgfqpoint{0.000000in}{0.941667in}}%
\pgfpathclose%
\pgfpathmoveto{\pgfqpoint{0.000000in}{0.947500in}}%
\pgfpathcurveto{\pgfqpoint{0.000000in}{0.947500in}}{\pgfqpoint{-0.013923in}{0.947500in}}{\pgfqpoint{-0.027278in}{0.953032in}}%
\pgfpathcurveto{\pgfqpoint{-0.037123in}{0.962877in}}{\pgfqpoint{-0.046968in}{0.972722in}}{\pgfqpoint{-0.052500in}{0.986077in}}%
\pgfpathcurveto{\pgfqpoint{-0.052500in}{1.000000in}}{\pgfqpoint{-0.052500in}{1.013923in}}{\pgfqpoint{-0.046968in}{1.027278in}}%
\pgfpathcurveto{\pgfqpoint{-0.037123in}{1.037123in}}{\pgfqpoint{-0.027278in}{1.046968in}}{\pgfqpoint{-0.013923in}{1.052500in}}%
\pgfpathcurveto{\pgfqpoint{0.000000in}{1.052500in}}{\pgfqpoint{0.013923in}{1.052500in}}{\pgfqpoint{0.027278in}{1.046968in}}%
\pgfpathcurveto{\pgfqpoint{0.037123in}{1.037123in}}{\pgfqpoint{0.046968in}{1.027278in}}{\pgfqpoint{0.052500in}{1.013923in}}%
\pgfpathcurveto{\pgfqpoint{0.052500in}{1.000000in}}{\pgfqpoint{0.052500in}{0.986077in}}{\pgfqpoint{0.046968in}{0.972722in}}%
\pgfpathcurveto{\pgfqpoint{0.037123in}{0.962877in}}{\pgfqpoint{0.027278in}{0.953032in}}{\pgfqpoint{0.013923in}{0.947500in}}%
\pgfpathclose%
\pgfpathmoveto{\pgfqpoint{0.166667in}{0.941667in}}%
\pgfpathcurveto{\pgfqpoint{0.182137in}{0.941667in}}{\pgfqpoint{0.196975in}{0.947813in}}{\pgfqpoint{0.207915in}{0.958752in}}%
\pgfpathcurveto{\pgfqpoint{0.218854in}{0.969691in}}{\pgfqpoint{0.225000in}{0.984530in}}{\pgfqpoint{0.225000in}{1.000000in}}%
\pgfpathcurveto{\pgfqpoint{0.225000in}{1.015470in}}{\pgfqpoint{0.218854in}{1.030309in}}{\pgfqpoint{0.207915in}{1.041248in}}%
\pgfpathcurveto{\pgfqpoint{0.196975in}{1.052187in}}{\pgfqpoint{0.182137in}{1.058333in}}{\pgfqpoint{0.166667in}{1.058333in}}%
\pgfpathcurveto{\pgfqpoint{0.151196in}{1.058333in}}{\pgfqpoint{0.136358in}{1.052187in}}{\pgfqpoint{0.125419in}{1.041248in}}%
\pgfpathcurveto{\pgfqpoint{0.114480in}{1.030309in}}{\pgfqpoint{0.108333in}{1.015470in}}{\pgfqpoint{0.108333in}{1.000000in}}%
\pgfpathcurveto{\pgfqpoint{0.108333in}{0.984530in}}{\pgfqpoint{0.114480in}{0.969691in}}{\pgfqpoint{0.125419in}{0.958752in}}%
\pgfpathcurveto{\pgfqpoint{0.136358in}{0.947813in}}{\pgfqpoint{0.151196in}{0.941667in}}{\pgfqpoint{0.166667in}{0.941667in}}%
\pgfpathclose%
\pgfpathmoveto{\pgfqpoint{0.166667in}{0.947500in}}%
\pgfpathcurveto{\pgfqpoint{0.166667in}{0.947500in}}{\pgfqpoint{0.152744in}{0.947500in}}{\pgfqpoint{0.139389in}{0.953032in}}%
\pgfpathcurveto{\pgfqpoint{0.129544in}{0.962877in}}{\pgfqpoint{0.119698in}{0.972722in}}{\pgfqpoint{0.114167in}{0.986077in}}%
\pgfpathcurveto{\pgfqpoint{0.114167in}{1.000000in}}{\pgfqpoint{0.114167in}{1.013923in}}{\pgfqpoint{0.119698in}{1.027278in}}%
\pgfpathcurveto{\pgfqpoint{0.129544in}{1.037123in}}{\pgfqpoint{0.139389in}{1.046968in}}{\pgfqpoint{0.152744in}{1.052500in}}%
\pgfpathcurveto{\pgfqpoint{0.166667in}{1.052500in}}{\pgfqpoint{0.180590in}{1.052500in}}{\pgfqpoint{0.193945in}{1.046968in}}%
\pgfpathcurveto{\pgfqpoint{0.203790in}{1.037123in}}{\pgfqpoint{0.213635in}{1.027278in}}{\pgfqpoint{0.219167in}{1.013923in}}%
\pgfpathcurveto{\pgfqpoint{0.219167in}{1.000000in}}{\pgfqpoint{0.219167in}{0.986077in}}{\pgfqpoint{0.213635in}{0.972722in}}%
\pgfpathcurveto{\pgfqpoint{0.203790in}{0.962877in}}{\pgfqpoint{0.193945in}{0.953032in}}{\pgfqpoint{0.180590in}{0.947500in}}%
\pgfpathclose%
\pgfpathmoveto{\pgfqpoint{0.333333in}{0.941667in}}%
\pgfpathcurveto{\pgfqpoint{0.348804in}{0.941667in}}{\pgfqpoint{0.363642in}{0.947813in}}{\pgfqpoint{0.374581in}{0.958752in}}%
\pgfpathcurveto{\pgfqpoint{0.385520in}{0.969691in}}{\pgfqpoint{0.391667in}{0.984530in}}{\pgfqpoint{0.391667in}{1.000000in}}%
\pgfpathcurveto{\pgfqpoint{0.391667in}{1.015470in}}{\pgfqpoint{0.385520in}{1.030309in}}{\pgfqpoint{0.374581in}{1.041248in}}%
\pgfpathcurveto{\pgfqpoint{0.363642in}{1.052187in}}{\pgfqpoint{0.348804in}{1.058333in}}{\pgfqpoint{0.333333in}{1.058333in}}%
\pgfpathcurveto{\pgfqpoint{0.317863in}{1.058333in}}{\pgfqpoint{0.303025in}{1.052187in}}{\pgfqpoint{0.292085in}{1.041248in}}%
\pgfpathcurveto{\pgfqpoint{0.281146in}{1.030309in}}{\pgfqpoint{0.275000in}{1.015470in}}{\pgfqpoint{0.275000in}{1.000000in}}%
\pgfpathcurveto{\pgfqpoint{0.275000in}{0.984530in}}{\pgfqpoint{0.281146in}{0.969691in}}{\pgfqpoint{0.292085in}{0.958752in}}%
\pgfpathcurveto{\pgfqpoint{0.303025in}{0.947813in}}{\pgfqpoint{0.317863in}{0.941667in}}{\pgfqpoint{0.333333in}{0.941667in}}%
\pgfpathclose%
\pgfpathmoveto{\pgfqpoint{0.333333in}{0.947500in}}%
\pgfpathcurveto{\pgfqpoint{0.333333in}{0.947500in}}{\pgfqpoint{0.319410in}{0.947500in}}{\pgfqpoint{0.306055in}{0.953032in}}%
\pgfpathcurveto{\pgfqpoint{0.296210in}{0.962877in}}{\pgfqpoint{0.286365in}{0.972722in}}{\pgfqpoint{0.280833in}{0.986077in}}%
\pgfpathcurveto{\pgfqpoint{0.280833in}{1.000000in}}{\pgfqpoint{0.280833in}{1.013923in}}{\pgfqpoint{0.286365in}{1.027278in}}%
\pgfpathcurveto{\pgfqpoint{0.296210in}{1.037123in}}{\pgfqpoint{0.306055in}{1.046968in}}{\pgfqpoint{0.319410in}{1.052500in}}%
\pgfpathcurveto{\pgfqpoint{0.333333in}{1.052500in}}{\pgfqpoint{0.347256in}{1.052500in}}{\pgfqpoint{0.360611in}{1.046968in}}%
\pgfpathcurveto{\pgfqpoint{0.370456in}{1.037123in}}{\pgfqpoint{0.380302in}{1.027278in}}{\pgfqpoint{0.385833in}{1.013923in}}%
\pgfpathcurveto{\pgfqpoint{0.385833in}{1.000000in}}{\pgfqpoint{0.385833in}{0.986077in}}{\pgfqpoint{0.380302in}{0.972722in}}%
\pgfpathcurveto{\pgfqpoint{0.370456in}{0.962877in}}{\pgfqpoint{0.360611in}{0.953032in}}{\pgfqpoint{0.347256in}{0.947500in}}%
\pgfpathclose%
\pgfpathmoveto{\pgfqpoint{0.500000in}{0.941667in}}%
\pgfpathcurveto{\pgfqpoint{0.515470in}{0.941667in}}{\pgfqpoint{0.530309in}{0.947813in}}{\pgfqpoint{0.541248in}{0.958752in}}%
\pgfpathcurveto{\pgfqpoint{0.552187in}{0.969691in}}{\pgfqpoint{0.558333in}{0.984530in}}{\pgfqpoint{0.558333in}{1.000000in}}%
\pgfpathcurveto{\pgfqpoint{0.558333in}{1.015470in}}{\pgfqpoint{0.552187in}{1.030309in}}{\pgfqpoint{0.541248in}{1.041248in}}%
\pgfpathcurveto{\pgfqpoint{0.530309in}{1.052187in}}{\pgfqpoint{0.515470in}{1.058333in}}{\pgfqpoint{0.500000in}{1.058333in}}%
\pgfpathcurveto{\pgfqpoint{0.484530in}{1.058333in}}{\pgfqpoint{0.469691in}{1.052187in}}{\pgfqpoint{0.458752in}{1.041248in}}%
\pgfpathcurveto{\pgfqpoint{0.447813in}{1.030309in}}{\pgfqpoint{0.441667in}{1.015470in}}{\pgfqpoint{0.441667in}{1.000000in}}%
\pgfpathcurveto{\pgfqpoint{0.441667in}{0.984530in}}{\pgfqpoint{0.447813in}{0.969691in}}{\pgfqpoint{0.458752in}{0.958752in}}%
\pgfpathcurveto{\pgfqpoint{0.469691in}{0.947813in}}{\pgfqpoint{0.484530in}{0.941667in}}{\pgfqpoint{0.500000in}{0.941667in}}%
\pgfpathclose%
\pgfpathmoveto{\pgfqpoint{0.500000in}{0.947500in}}%
\pgfpathcurveto{\pgfqpoint{0.500000in}{0.947500in}}{\pgfqpoint{0.486077in}{0.947500in}}{\pgfqpoint{0.472722in}{0.953032in}}%
\pgfpathcurveto{\pgfqpoint{0.462877in}{0.962877in}}{\pgfqpoint{0.453032in}{0.972722in}}{\pgfqpoint{0.447500in}{0.986077in}}%
\pgfpathcurveto{\pgfqpoint{0.447500in}{1.000000in}}{\pgfqpoint{0.447500in}{1.013923in}}{\pgfqpoint{0.453032in}{1.027278in}}%
\pgfpathcurveto{\pgfqpoint{0.462877in}{1.037123in}}{\pgfqpoint{0.472722in}{1.046968in}}{\pgfqpoint{0.486077in}{1.052500in}}%
\pgfpathcurveto{\pgfqpoint{0.500000in}{1.052500in}}{\pgfqpoint{0.513923in}{1.052500in}}{\pgfqpoint{0.527278in}{1.046968in}}%
\pgfpathcurveto{\pgfqpoint{0.537123in}{1.037123in}}{\pgfqpoint{0.546968in}{1.027278in}}{\pgfqpoint{0.552500in}{1.013923in}}%
\pgfpathcurveto{\pgfqpoint{0.552500in}{1.000000in}}{\pgfqpoint{0.552500in}{0.986077in}}{\pgfqpoint{0.546968in}{0.972722in}}%
\pgfpathcurveto{\pgfqpoint{0.537123in}{0.962877in}}{\pgfqpoint{0.527278in}{0.953032in}}{\pgfqpoint{0.513923in}{0.947500in}}%
\pgfpathclose%
\pgfpathmoveto{\pgfqpoint{0.666667in}{0.941667in}}%
\pgfpathcurveto{\pgfqpoint{0.682137in}{0.941667in}}{\pgfqpoint{0.696975in}{0.947813in}}{\pgfqpoint{0.707915in}{0.958752in}}%
\pgfpathcurveto{\pgfqpoint{0.718854in}{0.969691in}}{\pgfqpoint{0.725000in}{0.984530in}}{\pgfqpoint{0.725000in}{1.000000in}}%
\pgfpathcurveto{\pgfqpoint{0.725000in}{1.015470in}}{\pgfqpoint{0.718854in}{1.030309in}}{\pgfqpoint{0.707915in}{1.041248in}}%
\pgfpathcurveto{\pgfqpoint{0.696975in}{1.052187in}}{\pgfqpoint{0.682137in}{1.058333in}}{\pgfqpoint{0.666667in}{1.058333in}}%
\pgfpathcurveto{\pgfqpoint{0.651196in}{1.058333in}}{\pgfqpoint{0.636358in}{1.052187in}}{\pgfqpoint{0.625419in}{1.041248in}}%
\pgfpathcurveto{\pgfqpoint{0.614480in}{1.030309in}}{\pgfqpoint{0.608333in}{1.015470in}}{\pgfqpoint{0.608333in}{1.000000in}}%
\pgfpathcurveto{\pgfqpoint{0.608333in}{0.984530in}}{\pgfqpoint{0.614480in}{0.969691in}}{\pgfqpoint{0.625419in}{0.958752in}}%
\pgfpathcurveto{\pgfqpoint{0.636358in}{0.947813in}}{\pgfqpoint{0.651196in}{0.941667in}}{\pgfqpoint{0.666667in}{0.941667in}}%
\pgfpathclose%
\pgfpathmoveto{\pgfqpoint{0.666667in}{0.947500in}}%
\pgfpathcurveto{\pgfqpoint{0.666667in}{0.947500in}}{\pgfqpoint{0.652744in}{0.947500in}}{\pgfqpoint{0.639389in}{0.953032in}}%
\pgfpathcurveto{\pgfqpoint{0.629544in}{0.962877in}}{\pgfqpoint{0.619698in}{0.972722in}}{\pgfqpoint{0.614167in}{0.986077in}}%
\pgfpathcurveto{\pgfqpoint{0.614167in}{1.000000in}}{\pgfqpoint{0.614167in}{1.013923in}}{\pgfqpoint{0.619698in}{1.027278in}}%
\pgfpathcurveto{\pgfqpoint{0.629544in}{1.037123in}}{\pgfqpoint{0.639389in}{1.046968in}}{\pgfqpoint{0.652744in}{1.052500in}}%
\pgfpathcurveto{\pgfqpoint{0.666667in}{1.052500in}}{\pgfqpoint{0.680590in}{1.052500in}}{\pgfqpoint{0.693945in}{1.046968in}}%
\pgfpathcurveto{\pgfqpoint{0.703790in}{1.037123in}}{\pgfqpoint{0.713635in}{1.027278in}}{\pgfqpoint{0.719167in}{1.013923in}}%
\pgfpathcurveto{\pgfqpoint{0.719167in}{1.000000in}}{\pgfqpoint{0.719167in}{0.986077in}}{\pgfqpoint{0.713635in}{0.972722in}}%
\pgfpathcurveto{\pgfqpoint{0.703790in}{0.962877in}}{\pgfqpoint{0.693945in}{0.953032in}}{\pgfqpoint{0.680590in}{0.947500in}}%
\pgfpathclose%
\pgfpathmoveto{\pgfqpoint{0.833333in}{0.941667in}}%
\pgfpathcurveto{\pgfqpoint{0.848804in}{0.941667in}}{\pgfqpoint{0.863642in}{0.947813in}}{\pgfqpoint{0.874581in}{0.958752in}}%
\pgfpathcurveto{\pgfqpoint{0.885520in}{0.969691in}}{\pgfqpoint{0.891667in}{0.984530in}}{\pgfqpoint{0.891667in}{1.000000in}}%
\pgfpathcurveto{\pgfqpoint{0.891667in}{1.015470in}}{\pgfqpoint{0.885520in}{1.030309in}}{\pgfqpoint{0.874581in}{1.041248in}}%
\pgfpathcurveto{\pgfqpoint{0.863642in}{1.052187in}}{\pgfqpoint{0.848804in}{1.058333in}}{\pgfqpoint{0.833333in}{1.058333in}}%
\pgfpathcurveto{\pgfqpoint{0.817863in}{1.058333in}}{\pgfqpoint{0.803025in}{1.052187in}}{\pgfqpoint{0.792085in}{1.041248in}}%
\pgfpathcurveto{\pgfqpoint{0.781146in}{1.030309in}}{\pgfqpoint{0.775000in}{1.015470in}}{\pgfqpoint{0.775000in}{1.000000in}}%
\pgfpathcurveto{\pgfqpoint{0.775000in}{0.984530in}}{\pgfqpoint{0.781146in}{0.969691in}}{\pgfqpoint{0.792085in}{0.958752in}}%
\pgfpathcurveto{\pgfqpoint{0.803025in}{0.947813in}}{\pgfqpoint{0.817863in}{0.941667in}}{\pgfqpoint{0.833333in}{0.941667in}}%
\pgfpathclose%
\pgfpathmoveto{\pgfqpoint{0.833333in}{0.947500in}}%
\pgfpathcurveto{\pgfqpoint{0.833333in}{0.947500in}}{\pgfqpoint{0.819410in}{0.947500in}}{\pgfqpoint{0.806055in}{0.953032in}}%
\pgfpathcurveto{\pgfqpoint{0.796210in}{0.962877in}}{\pgfqpoint{0.786365in}{0.972722in}}{\pgfqpoint{0.780833in}{0.986077in}}%
\pgfpathcurveto{\pgfqpoint{0.780833in}{1.000000in}}{\pgfqpoint{0.780833in}{1.013923in}}{\pgfqpoint{0.786365in}{1.027278in}}%
\pgfpathcurveto{\pgfqpoint{0.796210in}{1.037123in}}{\pgfqpoint{0.806055in}{1.046968in}}{\pgfqpoint{0.819410in}{1.052500in}}%
\pgfpathcurveto{\pgfqpoint{0.833333in}{1.052500in}}{\pgfqpoint{0.847256in}{1.052500in}}{\pgfqpoint{0.860611in}{1.046968in}}%
\pgfpathcurveto{\pgfqpoint{0.870456in}{1.037123in}}{\pgfqpoint{0.880302in}{1.027278in}}{\pgfqpoint{0.885833in}{1.013923in}}%
\pgfpathcurveto{\pgfqpoint{0.885833in}{1.000000in}}{\pgfqpoint{0.885833in}{0.986077in}}{\pgfqpoint{0.880302in}{0.972722in}}%
\pgfpathcurveto{\pgfqpoint{0.870456in}{0.962877in}}{\pgfqpoint{0.860611in}{0.953032in}}{\pgfqpoint{0.847256in}{0.947500in}}%
\pgfpathclose%
\pgfpathmoveto{\pgfqpoint{1.000000in}{0.941667in}}%
\pgfpathcurveto{\pgfqpoint{1.015470in}{0.941667in}}{\pgfqpoint{1.030309in}{0.947813in}}{\pgfqpoint{1.041248in}{0.958752in}}%
\pgfpathcurveto{\pgfqpoint{1.052187in}{0.969691in}}{\pgfqpoint{1.058333in}{0.984530in}}{\pgfqpoint{1.058333in}{1.000000in}}%
\pgfpathcurveto{\pgfqpoint{1.058333in}{1.015470in}}{\pgfqpoint{1.052187in}{1.030309in}}{\pgfqpoint{1.041248in}{1.041248in}}%
\pgfpathcurveto{\pgfqpoint{1.030309in}{1.052187in}}{\pgfqpoint{1.015470in}{1.058333in}}{\pgfqpoint{1.000000in}{1.058333in}}%
\pgfpathcurveto{\pgfqpoint{0.984530in}{1.058333in}}{\pgfqpoint{0.969691in}{1.052187in}}{\pgfqpoint{0.958752in}{1.041248in}}%
\pgfpathcurveto{\pgfqpoint{0.947813in}{1.030309in}}{\pgfqpoint{0.941667in}{1.015470in}}{\pgfqpoint{0.941667in}{1.000000in}}%
\pgfpathcurveto{\pgfqpoint{0.941667in}{0.984530in}}{\pgfqpoint{0.947813in}{0.969691in}}{\pgfqpoint{0.958752in}{0.958752in}}%
\pgfpathcurveto{\pgfqpoint{0.969691in}{0.947813in}}{\pgfqpoint{0.984530in}{0.941667in}}{\pgfqpoint{1.000000in}{0.941667in}}%
\pgfpathclose%
\pgfpathmoveto{\pgfqpoint{1.000000in}{0.947500in}}%
\pgfpathcurveto{\pgfqpoint{1.000000in}{0.947500in}}{\pgfqpoint{0.986077in}{0.947500in}}{\pgfqpoint{0.972722in}{0.953032in}}%
\pgfpathcurveto{\pgfqpoint{0.962877in}{0.962877in}}{\pgfqpoint{0.953032in}{0.972722in}}{\pgfqpoint{0.947500in}{0.986077in}}%
\pgfpathcurveto{\pgfqpoint{0.947500in}{1.000000in}}{\pgfqpoint{0.947500in}{1.013923in}}{\pgfqpoint{0.953032in}{1.027278in}}%
\pgfpathcurveto{\pgfqpoint{0.962877in}{1.037123in}}{\pgfqpoint{0.972722in}{1.046968in}}{\pgfqpoint{0.986077in}{1.052500in}}%
\pgfpathcurveto{\pgfqpoint{1.000000in}{1.052500in}}{\pgfqpoint{1.013923in}{1.052500in}}{\pgfqpoint{1.027278in}{1.046968in}}%
\pgfpathcurveto{\pgfqpoint{1.037123in}{1.037123in}}{\pgfqpoint{1.046968in}{1.027278in}}{\pgfqpoint{1.052500in}{1.013923in}}%
\pgfpathcurveto{\pgfqpoint{1.052500in}{1.000000in}}{\pgfqpoint{1.052500in}{0.986077in}}{\pgfqpoint{1.046968in}{0.972722in}}%
\pgfpathcurveto{\pgfqpoint{1.037123in}{0.962877in}}{\pgfqpoint{1.027278in}{0.953032in}}{\pgfqpoint{1.013923in}{0.947500in}}%
\pgfpathclose%
\pgfusepath{stroke}%
\end{pgfscope}%
}%
\pgfsys@transformshift{9.008038in}{0.637495in}%
\pgfsys@useobject{currentpattern}{}%
\pgfsys@transformshift{1in}{0in}%
\pgfsys@transformshift{-1in}{0in}%
\pgfsys@transformshift{0in}{1in}%
\pgfsys@useobject{currentpattern}{}%
\pgfsys@transformshift{1in}{0in}%
\pgfsys@transformshift{-1in}{0in}%
\pgfsys@transformshift{0in}{1in}%
\pgfsys@useobject{currentpattern}{}%
\pgfsys@transformshift{1in}{0in}%
\pgfsys@transformshift{-1in}{0in}%
\pgfsys@transformshift{0in}{1in}%
\pgfsys@useobject{currentpattern}{}%
\pgfsys@transformshift{1in}{0in}%
\pgfsys@transformshift{-1in}{0in}%
\pgfsys@transformshift{0in}{1in}%
\end{pgfscope}%
\begin{pgfscope}%
\pgfpathrectangle{\pgfqpoint{0.870538in}{0.637495in}}{\pgfqpoint{9.300000in}{9.060000in}}%
\pgfusepath{clip}%
\pgfsetbuttcap%
\pgfsetmiterjoin%
\definecolor{currentfill}{rgb}{0.411765,0.411765,0.411765}%
\pgfsetfillcolor{currentfill}%
\pgfsetfillopacity{0.990000}%
\pgfsetlinewidth{0.000000pt}%
\definecolor{currentstroke}{rgb}{0.000000,0.000000,0.000000}%
\pgfsetstrokecolor{currentstroke}%
\pgfsetstrokeopacity{0.990000}%
\pgfsetdash{}{0pt}%
\pgfpathmoveto{\pgfqpoint{1.258038in}{0.637495in}}%
\pgfpathlineto{\pgfqpoint{2.033038in}{0.637495in}}%
\pgfpathlineto{\pgfqpoint{2.033038in}{0.637495in}}%
\pgfpathlineto{\pgfqpoint{1.258038in}{0.637495in}}%
\pgfpathclose%
\pgfusepath{fill}%
\end{pgfscope}%
\begin{pgfscope}%
\pgfsetbuttcap%
\pgfsetmiterjoin%
\definecolor{currentfill}{rgb}{0.411765,0.411765,0.411765}%
\pgfsetfillcolor{currentfill}%
\pgfsetfillopacity{0.990000}%
\pgfsetlinewidth{0.000000pt}%
\definecolor{currentstroke}{rgb}{0.000000,0.000000,0.000000}%
\pgfsetstrokecolor{currentstroke}%
\pgfsetstrokeopacity{0.990000}%
\pgfsetdash{}{0pt}%
\pgfpathrectangle{\pgfqpoint{0.870538in}{0.637495in}}{\pgfqpoint{9.300000in}{9.060000in}}%
\pgfusepath{clip}%
\pgfpathmoveto{\pgfqpoint{1.258038in}{0.637495in}}%
\pgfpathlineto{\pgfqpoint{2.033038in}{0.637495in}}%
\pgfpathlineto{\pgfqpoint{2.033038in}{0.637495in}}%
\pgfpathlineto{\pgfqpoint{1.258038in}{0.637495in}}%
\pgfpathclose%
\pgfusepath{clip}%
\pgfsys@defobject{currentpattern}{\pgfqpoint{0in}{0in}}{\pgfqpoint{1in}{1in}}{%
\begin{pgfscope}%
\pgfpathrectangle{\pgfqpoint{0in}{0in}}{\pgfqpoint{1in}{1in}}%
\pgfusepath{clip}%
\pgfpathmoveto{\pgfqpoint{-0.500000in}{0.500000in}}%
\pgfpathlineto{\pgfqpoint{0.500000in}{1.500000in}}%
\pgfpathmoveto{\pgfqpoint{-0.333333in}{0.333333in}}%
\pgfpathlineto{\pgfqpoint{0.666667in}{1.333333in}}%
\pgfpathmoveto{\pgfqpoint{-0.166667in}{0.166667in}}%
\pgfpathlineto{\pgfqpoint{0.833333in}{1.166667in}}%
\pgfpathmoveto{\pgfqpoint{0.000000in}{0.000000in}}%
\pgfpathlineto{\pgfqpoint{1.000000in}{1.000000in}}%
\pgfpathmoveto{\pgfqpoint{0.166667in}{-0.166667in}}%
\pgfpathlineto{\pgfqpoint{1.166667in}{0.833333in}}%
\pgfpathmoveto{\pgfqpoint{0.333333in}{-0.333333in}}%
\pgfpathlineto{\pgfqpoint{1.333333in}{0.666667in}}%
\pgfpathmoveto{\pgfqpoint{0.500000in}{-0.500000in}}%
\pgfpathlineto{\pgfqpoint{1.500000in}{0.500000in}}%
\pgfusepath{stroke}%
\end{pgfscope}%
}%
\pgfsys@transformshift{1.258038in}{0.637495in}%
\end{pgfscope}%
\begin{pgfscope}%
\pgfpathrectangle{\pgfqpoint{0.870538in}{0.637495in}}{\pgfqpoint{9.300000in}{9.060000in}}%
\pgfusepath{clip}%
\pgfsetbuttcap%
\pgfsetmiterjoin%
\definecolor{currentfill}{rgb}{0.411765,0.411765,0.411765}%
\pgfsetfillcolor{currentfill}%
\pgfsetfillopacity{0.990000}%
\pgfsetlinewidth{0.000000pt}%
\definecolor{currentstroke}{rgb}{0.000000,0.000000,0.000000}%
\pgfsetstrokecolor{currentstroke}%
\pgfsetstrokeopacity{0.990000}%
\pgfsetdash{}{0pt}%
\pgfpathmoveto{\pgfqpoint{2.808038in}{0.637495in}}%
\pgfpathlineto{\pgfqpoint{3.583038in}{0.637495in}}%
\pgfpathlineto{\pgfqpoint{3.583038in}{0.637495in}}%
\pgfpathlineto{\pgfqpoint{2.808038in}{0.637495in}}%
\pgfpathclose%
\pgfusepath{fill}%
\end{pgfscope}%
\begin{pgfscope}%
\pgfsetbuttcap%
\pgfsetmiterjoin%
\definecolor{currentfill}{rgb}{0.411765,0.411765,0.411765}%
\pgfsetfillcolor{currentfill}%
\pgfsetfillopacity{0.990000}%
\pgfsetlinewidth{0.000000pt}%
\definecolor{currentstroke}{rgb}{0.000000,0.000000,0.000000}%
\pgfsetstrokecolor{currentstroke}%
\pgfsetstrokeopacity{0.990000}%
\pgfsetdash{}{0pt}%
\pgfpathrectangle{\pgfqpoint{0.870538in}{0.637495in}}{\pgfqpoint{9.300000in}{9.060000in}}%
\pgfusepath{clip}%
\pgfpathmoveto{\pgfqpoint{2.808038in}{0.637495in}}%
\pgfpathlineto{\pgfqpoint{3.583038in}{0.637495in}}%
\pgfpathlineto{\pgfqpoint{3.583038in}{0.637495in}}%
\pgfpathlineto{\pgfqpoint{2.808038in}{0.637495in}}%
\pgfpathclose%
\pgfusepath{clip}%
\pgfsys@defobject{currentpattern}{\pgfqpoint{0in}{0in}}{\pgfqpoint{1in}{1in}}{%
\begin{pgfscope}%
\pgfpathrectangle{\pgfqpoint{0in}{0in}}{\pgfqpoint{1in}{1in}}%
\pgfusepath{clip}%
\pgfpathmoveto{\pgfqpoint{-0.500000in}{0.500000in}}%
\pgfpathlineto{\pgfqpoint{0.500000in}{1.500000in}}%
\pgfpathmoveto{\pgfqpoint{-0.333333in}{0.333333in}}%
\pgfpathlineto{\pgfqpoint{0.666667in}{1.333333in}}%
\pgfpathmoveto{\pgfqpoint{-0.166667in}{0.166667in}}%
\pgfpathlineto{\pgfqpoint{0.833333in}{1.166667in}}%
\pgfpathmoveto{\pgfqpoint{0.000000in}{0.000000in}}%
\pgfpathlineto{\pgfqpoint{1.000000in}{1.000000in}}%
\pgfpathmoveto{\pgfqpoint{0.166667in}{-0.166667in}}%
\pgfpathlineto{\pgfqpoint{1.166667in}{0.833333in}}%
\pgfpathmoveto{\pgfqpoint{0.333333in}{-0.333333in}}%
\pgfpathlineto{\pgfqpoint{1.333333in}{0.666667in}}%
\pgfpathmoveto{\pgfqpoint{0.500000in}{-0.500000in}}%
\pgfpathlineto{\pgfqpoint{1.500000in}{0.500000in}}%
\pgfusepath{stroke}%
\end{pgfscope}%
}%
\pgfsys@transformshift{2.808038in}{0.637495in}%
\end{pgfscope}%
\begin{pgfscope}%
\pgfpathrectangle{\pgfqpoint{0.870538in}{0.637495in}}{\pgfqpoint{9.300000in}{9.060000in}}%
\pgfusepath{clip}%
\pgfsetbuttcap%
\pgfsetmiterjoin%
\definecolor{currentfill}{rgb}{0.411765,0.411765,0.411765}%
\pgfsetfillcolor{currentfill}%
\pgfsetfillopacity{0.990000}%
\pgfsetlinewidth{0.000000pt}%
\definecolor{currentstroke}{rgb}{0.000000,0.000000,0.000000}%
\pgfsetstrokecolor{currentstroke}%
\pgfsetstrokeopacity{0.990000}%
\pgfsetdash{}{0pt}%
\pgfpathmoveto{\pgfqpoint{4.358038in}{0.637495in}}%
\pgfpathlineto{\pgfqpoint{5.133038in}{0.637495in}}%
\pgfpathlineto{\pgfqpoint{5.133038in}{0.637495in}}%
\pgfpathlineto{\pgfqpoint{4.358038in}{0.637495in}}%
\pgfpathclose%
\pgfusepath{fill}%
\end{pgfscope}%
\begin{pgfscope}%
\pgfsetbuttcap%
\pgfsetmiterjoin%
\definecolor{currentfill}{rgb}{0.411765,0.411765,0.411765}%
\pgfsetfillcolor{currentfill}%
\pgfsetfillopacity{0.990000}%
\pgfsetlinewidth{0.000000pt}%
\definecolor{currentstroke}{rgb}{0.000000,0.000000,0.000000}%
\pgfsetstrokecolor{currentstroke}%
\pgfsetstrokeopacity{0.990000}%
\pgfsetdash{}{0pt}%
\pgfpathrectangle{\pgfqpoint{0.870538in}{0.637495in}}{\pgfqpoint{9.300000in}{9.060000in}}%
\pgfusepath{clip}%
\pgfpathmoveto{\pgfqpoint{4.358038in}{0.637495in}}%
\pgfpathlineto{\pgfqpoint{5.133038in}{0.637495in}}%
\pgfpathlineto{\pgfqpoint{5.133038in}{0.637495in}}%
\pgfpathlineto{\pgfqpoint{4.358038in}{0.637495in}}%
\pgfpathclose%
\pgfusepath{clip}%
\pgfsys@defobject{currentpattern}{\pgfqpoint{0in}{0in}}{\pgfqpoint{1in}{1in}}{%
\begin{pgfscope}%
\pgfpathrectangle{\pgfqpoint{0in}{0in}}{\pgfqpoint{1in}{1in}}%
\pgfusepath{clip}%
\pgfpathmoveto{\pgfqpoint{-0.500000in}{0.500000in}}%
\pgfpathlineto{\pgfqpoint{0.500000in}{1.500000in}}%
\pgfpathmoveto{\pgfqpoint{-0.333333in}{0.333333in}}%
\pgfpathlineto{\pgfqpoint{0.666667in}{1.333333in}}%
\pgfpathmoveto{\pgfqpoint{-0.166667in}{0.166667in}}%
\pgfpathlineto{\pgfqpoint{0.833333in}{1.166667in}}%
\pgfpathmoveto{\pgfqpoint{0.000000in}{0.000000in}}%
\pgfpathlineto{\pgfqpoint{1.000000in}{1.000000in}}%
\pgfpathmoveto{\pgfqpoint{0.166667in}{-0.166667in}}%
\pgfpathlineto{\pgfqpoint{1.166667in}{0.833333in}}%
\pgfpathmoveto{\pgfqpoint{0.333333in}{-0.333333in}}%
\pgfpathlineto{\pgfqpoint{1.333333in}{0.666667in}}%
\pgfpathmoveto{\pgfqpoint{0.500000in}{-0.500000in}}%
\pgfpathlineto{\pgfqpoint{1.500000in}{0.500000in}}%
\pgfusepath{stroke}%
\end{pgfscope}%
}%
\pgfsys@transformshift{4.358038in}{0.637495in}%
\end{pgfscope}%
\begin{pgfscope}%
\pgfpathrectangle{\pgfqpoint{0.870538in}{0.637495in}}{\pgfqpoint{9.300000in}{9.060000in}}%
\pgfusepath{clip}%
\pgfsetbuttcap%
\pgfsetmiterjoin%
\definecolor{currentfill}{rgb}{0.411765,0.411765,0.411765}%
\pgfsetfillcolor{currentfill}%
\pgfsetfillopacity{0.990000}%
\pgfsetlinewidth{0.000000pt}%
\definecolor{currentstroke}{rgb}{0.000000,0.000000,0.000000}%
\pgfsetstrokecolor{currentstroke}%
\pgfsetstrokeopacity{0.990000}%
\pgfsetdash{}{0pt}%
\pgfpathmoveto{\pgfqpoint{5.908038in}{0.637495in}}%
\pgfpathlineto{\pgfqpoint{6.683038in}{0.637495in}}%
\pgfpathlineto{\pgfqpoint{6.683038in}{0.637495in}}%
\pgfpathlineto{\pgfqpoint{5.908038in}{0.637495in}}%
\pgfpathclose%
\pgfusepath{fill}%
\end{pgfscope}%
\begin{pgfscope}%
\pgfsetbuttcap%
\pgfsetmiterjoin%
\definecolor{currentfill}{rgb}{0.411765,0.411765,0.411765}%
\pgfsetfillcolor{currentfill}%
\pgfsetfillopacity{0.990000}%
\pgfsetlinewidth{0.000000pt}%
\definecolor{currentstroke}{rgb}{0.000000,0.000000,0.000000}%
\pgfsetstrokecolor{currentstroke}%
\pgfsetstrokeopacity{0.990000}%
\pgfsetdash{}{0pt}%
\pgfpathrectangle{\pgfqpoint{0.870538in}{0.637495in}}{\pgfqpoint{9.300000in}{9.060000in}}%
\pgfusepath{clip}%
\pgfpathmoveto{\pgfqpoint{5.908038in}{0.637495in}}%
\pgfpathlineto{\pgfqpoint{6.683038in}{0.637495in}}%
\pgfpathlineto{\pgfqpoint{6.683038in}{0.637495in}}%
\pgfpathlineto{\pgfqpoint{5.908038in}{0.637495in}}%
\pgfpathclose%
\pgfusepath{clip}%
\pgfsys@defobject{currentpattern}{\pgfqpoint{0in}{0in}}{\pgfqpoint{1in}{1in}}{%
\begin{pgfscope}%
\pgfpathrectangle{\pgfqpoint{0in}{0in}}{\pgfqpoint{1in}{1in}}%
\pgfusepath{clip}%
\pgfpathmoveto{\pgfqpoint{-0.500000in}{0.500000in}}%
\pgfpathlineto{\pgfqpoint{0.500000in}{1.500000in}}%
\pgfpathmoveto{\pgfqpoint{-0.333333in}{0.333333in}}%
\pgfpathlineto{\pgfqpoint{0.666667in}{1.333333in}}%
\pgfpathmoveto{\pgfqpoint{-0.166667in}{0.166667in}}%
\pgfpathlineto{\pgfqpoint{0.833333in}{1.166667in}}%
\pgfpathmoveto{\pgfqpoint{0.000000in}{0.000000in}}%
\pgfpathlineto{\pgfqpoint{1.000000in}{1.000000in}}%
\pgfpathmoveto{\pgfqpoint{0.166667in}{-0.166667in}}%
\pgfpathlineto{\pgfqpoint{1.166667in}{0.833333in}}%
\pgfpathmoveto{\pgfqpoint{0.333333in}{-0.333333in}}%
\pgfpathlineto{\pgfqpoint{1.333333in}{0.666667in}}%
\pgfpathmoveto{\pgfqpoint{0.500000in}{-0.500000in}}%
\pgfpathlineto{\pgfqpoint{1.500000in}{0.500000in}}%
\pgfusepath{stroke}%
\end{pgfscope}%
}%
\pgfsys@transformshift{5.908038in}{0.637495in}%
\end{pgfscope}%
\begin{pgfscope}%
\pgfpathrectangle{\pgfqpoint{0.870538in}{0.637495in}}{\pgfqpoint{9.300000in}{9.060000in}}%
\pgfusepath{clip}%
\pgfsetbuttcap%
\pgfsetmiterjoin%
\definecolor{currentfill}{rgb}{0.411765,0.411765,0.411765}%
\pgfsetfillcolor{currentfill}%
\pgfsetfillopacity{0.990000}%
\pgfsetlinewidth{0.000000pt}%
\definecolor{currentstroke}{rgb}{0.000000,0.000000,0.000000}%
\pgfsetstrokecolor{currentstroke}%
\pgfsetstrokeopacity{0.990000}%
\pgfsetdash{}{0pt}%
\pgfpathmoveto{\pgfqpoint{7.458038in}{0.637495in}}%
\pgfpathlineto{\pgfqpoint{8.233038in}{0.637495in}}%
\pgfpathlineto{\pgfqpoint{8.233038in}{0.637495in}}%
\pgfpathlineto{\pgfqpoint{7.458038in}{0.637495in}}%
\pgfpathclose%
\pgfusepath{fill}%
\end{pgfscope}%
\begin{pgfscope}%
\pgfsetbuttcap%
\pgfsetmiterjoin%
\definecolor{currentfill}{rgb}{0.411765,0.411765,0.411765}%
\pgfsetfillcolor{currentfill}%
\pgfsetfillopacity{0.990000}%
\pgfsetlinewidth{0.000000pt}%
\definecolor{currentstroke}{rgb}{0.000000,0.000000,0.000000}%
\pgfsetstrokecolor{currentstroke}%
\pgfsetstrokeopacity{0.990000}%
\pgfsetdash{}{0pt}%
\pgfpathrectangle{\pgfqpoint{0.870538in}{0.637495in}}{\pgfqpoint{9.300000in}{9.060000in}}%
\pgfusepath{clip}%
\pgfpathmoveto{\pgfqpoint{7.458038in}{0.637495in}}%
\pgfpathlineto{\pgfqpoint{8.233038in}{0.637495in}}%
\pgfpathlineto{\pgfqpoint{8.233038in}{0.637495in}}%
\pgfpathlineto{\pgfqpoint{7.458038in}{0.637495in}}%
\pgfpathclose%
\pgfusepath{clip}%
\pgfsys@defobject{currentpattern}{\pgfqpoint{0in}{0in}}{\pgfqpoint{1in}{1in}}{%
\begin{pgfscope}%
\pgfpathrectangle{\pgfqpoint{0in}{0in}}{\pgfqpoint{1in}{1in}}%
\pgfusepath{clip}%
\pgfpathmoveto{\pgfqpoint{-0.500000in}{0.500000in}}%
\pgfpathlineto{\pgfqpoint{0.500000in}{1.500000in}}%
\pgfpathmoveto{\pgfqpoint{-0.333333in}{0.333333in}}%
\pgfpathlineto{\pgfqpoint{0.666667in}{1.333333in}}%
\pgfpathmoveto{\pgfqpoint{-0.166667in}{0.166667in}}%
\pgfpathlineto{\pgfqpoint{0.833333in}{1.166667in}}%
\pgfpathmoveto{\pgfqpoint{0.000000in}{0.000000in}}%
\pgfpathlineto{\pgfqpoint{1.000000in}{1.000000in}}%
\pgfpathmoveto{\pgfqpoint{0.166667in}{-0.166667in}}%
\pgfpathlineto{\pgfqpoint{1.166667in}{0.833333in}}%
\pgfpathmoveto{\pgfqpoint{0.333333in}{-0.333333in}}%
\pgfpathlineto{\pgfqpoint{1.333333in}{0.666667in}}%
\pgfpathmoveto{\pgfqpoint{0.500000in}{-0.500000in}}%
\pgfpathlineto{\pgfqpoint{1.500000in}{0.500000in}}%
\pgfusepath{stroke}%
\end{pgfscope}%
}%
\pgfsys@transformshift{7.458038in}{0.637495in}%
\end{pgfscope}%
\begin{pgfscope}%
\pgfpathrectangle{\pgfqpoint{0.870538in}{0.637495in}}{\pgfqpoint{9.300000in}{9.060000in}}%
\pgfusepath{clip}%
\pgfsetbuttcap%
\pgfsetmiterjoin%
\definecolor{currentfill}{rgb}{0.411765,0.411765,0.411765}%
\pgfsetfillcolor{currentfill}%
\pgfsetfillopacity{0.990000}%
\pgfsetlinewidth{0.000000pt}%
\definecolor{currentstroke}{rgb}{0.000000,0.000000,0.000000}%
\pgfsetstrokecolor{currentstroke}%
\pgfsetstrokeopacity{0.990000}%
\pgfsetdash{}{0pt}%
\pgfpathmoveto{\pgfqpoint{9.008038in}{0.637495in}}%
\pgfpathlineto{\pgfqpoint{9.783038in}{0.637495in}}%
\pgfpathlineto{\pgfqpoint{9.783038in}{0.637495in}}%
\pgfpathlineto{\pgfqpoint{9.008038in}{0.637495in}}%
\pgfpathclose%
\pgfusepath{fill}%
\end{pgfscope}%
\begin{pgfscope}%
\pgfsetbuttcap%
\pgfsetmiterjoin%
\definecolor{currentfill}{rgb}{0.411765,0.411765,0.411765}%
\pgfsetfillcolor{currentfill}%
\pgfsetfillopacity{0.990000}%
\pgfsetlinewidth{0.000000pt}%
\definecolor{currentstroke}{rgb}{0.000000,0.000000,0.000000}%
\pgfsetstrokecolor{currentstroke}%
\pgfsetstrokeopacity{0.990000}%
\pgfsetdash{}{0pt}%
\pgfpathrectangle{\pgfqpoint{0.870538in}{0.637495in}}{\pgfqpoint{9.300000in}{9.060000in}}%
\pgfusepath{clip}%
\pgfpathmoveto{\pgfqpoint{9.008038in}{0.637495in}}%
\pgfpathlineto{\pgfqpoint{9.783038in}{0.637495in}}%
\pgfpathlineto{\pgfqpoint{9.783038in}{0.637495in}}%
\pgfpathlineto{\pgfqpoint{9.008038in}{0.637495in}}%
\pgfpathclose%
\pgfusepath{clip}%
\pgfsys@defobject{currentpattern}{\pgfqpoint{0in}{0in}}{\pgfqpoint{1in}{1in}}{%
\begin{pgfscope}%
\pgfpathrectangle{\pgfqpoint{0in}{0in}}{\pgfqpoint{1in}{1in}}%
\pgfusepath{clip}%
\pgfpathmoveto{\pgfqpoint{-0.500000in}{0.500000in}}%
\pgfpathlineto{\pgfqpoint{0.500000in}{1.500000in}}%
\pgfpathmoveto{\pgfqpoint{-0.333333in}{0.333333in}}%
\pgfpathlineto{\pgfqpoint{0.666667in}{1.333333in}}%
\pgfpathmoveto{\pgfqpoint{-0.166667in}{0.166667in}}%
\pgfpathlineto{\pgfqpoint{0.833333in}{1.166667in}}%
\pgfpathmoveto{\pgfqpoint{0.000000in}{0.000000in}}%
\pgfpathlineto{\pgfqpoint{1.000000in}{1.000000in}}%
\pgfpathmoveto{\pgfqpoint{0.166667in}{-0.166667in}}%
\pgfpathlineto{\pgfqpoint{1.166667in}{0.833333in}}%
\pgfpathmoveto{\pgfqpoint{0.333333in}{-0.333333in}}%
\pgfpathlineto{\pgfqpoint{1.333333in}{0.666667in}}%
\pgfpathmoveto{\pgfqpoint{0.500000in}{-0.500000in}}%
\pgfpathlineto{\pgfqpoint{1.500000in}{0.500000in}}%
\pgfusepath{stroke}%
\end{pgfscope}%
}%
\pgfsys@transformshift{9.008038in}{0.637495in}%
\end{pgfscope}%
\begin{pgfscope}%
\pgfpathrectangle{\pgfqpoint{0.870538in}{0.637495in}}{\pgfqpoint{9.300000in}{9.060000in}}%
\pgfusepath{clip}%
\pgfsetbuttcap%
\pgfsetmiterjoin%
\definecolor{currentfill}{rgb}{0.172549,0.627451,0.172549}%
\pgfsetfillcolor{currentfill}%
\pgfsetfillopacity{0.990000}%
\pgfsetlinewidth{0.000000pt}%
\definecolor{currentstroke}{rgb}{0.000000,0.000000,0.000000}%
\pgfsetstrokecolor{currentstroke}%
\pgfsetstrokeopacity{0.990000}%
\pgfsetdash{}{0pt}%
\pgfpathmoveto{\pgfqpoint{1.258038in}{0.637495in}}%
\pgfpathlineto{\pgfqpoint{2.033038in}{0.637495in}}%
\pgfpathlineto{\pgfqpoint{2.033038in}{0.637495in}}%
\pgfpathlineto{\pgfqpoint{1.258038in}{0.637495in}}%
\pgfpathclose%
\pgfusepath{fill}%
\end{pgfscope}%
\begin{pgfscope}%
\pgfsetbuttcap%
\pgfsetmiterjoin%
\definecolor{currentfill}{rgb}{0.172549,0.627451,0.172549}%
\pgfsetfillcolor{currentfill}%
\pgfsetfillopacity{0.990000}%
\pgfsetlinewidth{0.000000pt}%
\definecolor{currentstroke}{rgb}{0.000000,0.000000,0.000000}%
\pgfsetstrokecolor{currentstroke}%
\pgfsetstrokeopacity{0.990000}%
\pgfsetdash{}{0pt}%
\pgfpathrectangle{\pgfqpoint{0.870538in}{0.637495in}}{\pgfqpoint{9.300000in}{9.060000in}}%
\pgfusepath{clip}%
\pgfpathmoveto{\pgfqpoint{1.258038in}{0.637495in}}%
\pgfpathlineto{\pgfqpoint{2.033038in}{0.637495in}}%
\pgfpathlineto{\pgfqpoint{2.033038in}{0.637495in}}%
\pgfpathlineto{\pgfqpoint{1.258038in}{0.637495in}}%
\pgfpathclose%
\pgfusepath{clip}%
\pgfsys@defobject{currentpattern}{\pgfqpoint{0in}{0in}}{\pgfqpoint{1in}{1in}}{%
\begin{pgfscope}%
\pgfpathrectangle{\pgfqpoint{0in}{0in}}{\pgfqpoint{1in}{1in}}%
\pgfusepath{clip}%
\pgfpathmoveto{\pgfqpoint{0.000000in}{-0.016667in}}%
\pgfpathcurveto{\pgfqpoint{0.004420in}{-0.016667in}}{\pgfqpoint{0.008660in}{-0.014911in}}{\pgfqpoint{0.011785in}{-0.011785in}}%
\pgfpathcurveto{\pgfqpoint{0.014911in}{-0.008660in}}{\pgfqpoint{0.016667in}{-0.004420in}}{\pgfqpoint{0.016667in}{0.000000in}}%
\pgfpathcurveto{\pgfqpoint{0.016667in}{0.004420in}}{\pgfqpoint{0.014911in}{0.008660in}}{\pgfqpoint{0.011785in}{0.011785in}}%
\pgfpathcurveto{\pgfqpoint{0.008660in}{0.014911in}}{\pgfqpoint{0.004420in}{0.016667in}}{\pgfqpoint{0.000000in}{0.016667in}}%
\pgfpathcurveto{\pgfqpoint{-0.004420in}{0.016667in}}{\pgfqpoint{-0.008660in}{0.014911in}}{\pgfqpoint{-0.011785in}{0.011785in}}%
\pgfpathcurveto{\pgfqpoint{-0.014911in}{0.008660in}}{\pgfqpoint{-0.016667in}{0.004420in}}{\pgfqpoint{-0.016667in}{0.000000in}}%
\pgfpathcurveto{\pgfqpoint{-0.016667in}{-0.004420in}}{\pgfqpoint{-0.014911in}{-0.008660in}}{\pgfqpoint{-0.011785in}{-0.011785in}}%
\pgfpathcurveto{\pgfqpoint{-0.008660in}{-0.014911in}}{\pgfqpoint{-0.004420in}{-0.016667in}}{\pgfqpoint{0.000000in}{-0.016667in}}%
\pgfpathclose%
\pgfpathmoveto{\pgfqpoint{0.166667in}{-0.016667in}}%
\pgfpathcurveto{\pgfqpoint{0.171087in}{-0.016667in}}{\pgfqpoint{0.175326in}{-0.014911in}}{\pgfqpoint{0.178452in}{-0.011785in}}%
\pgfpathcurveto{\pgfqpoint{0.181577in}{-0.008660in}}{\pgfqpoint{0.183333in}{-0.004420in}}{\pgfqpoint{0.183333in}{0.000000in}}%
\pgfpathcurveto{\pgfqpoint{0.183333in}{0.004420in}}{\pgfqpoint{0.181577in}{0.008660in}}{\pgfqpoint{0.178452in}{0.011785in}}%
\pgfpathcurveto{\pgfqpoint{0.175326in}{0.014911in}}{\pgfqpoint{0.171087in}{0.016667in}}{\pgfqpoint{0.166667in}{0.016667in}}%
\pgfpathcurveto{\pgfqpoint{0.162247in}{0.016667in}}{\pgfqpoint{0.158007in}{0.014911in}}{\pgfqpoint{0.154882in}{0.011785in}}%
\pgfpathcurveto{\pgfqpoint{0.151756in}{0.008660in}}{\pgfqpoint{0.150000in}{0.004420in}}{\pgfqpoint{0.150000in}{0.000000in}}%
\pgfpathcurveto{\pgfqpoint{0.150000in}{-0.004420in}}{\pgfqpoint{0.151756in}{-0.008660in}}{\pgfqpoint{0.154882in}{-0.011785in}}%
\pgfpathcurveto{\pgfqpoint{0.158007in}{-0.014911in}}{\pgfqpoint{0.162247in}{-0.016667in}}{\pgfqpoint{0.166667in}{-0.016667in}}%
\pgfpathclose%
\pgfpathmoveto{\pgfqpoint{0.333333in}{-0.016667in}}%
\pgfpathcurveto{\pgfqpoint{0.337753in}{-0.016667in}}{\pgfqpoint{0.341993in}{-0.014911in}}{\pgfqpoint{0.345118in}{-0.011785in}}%
\pgfpathcurveto{\pgfqpoint{0.348244in}{-0.008660in}}{\pgfqpoint{0.350000in}{-0.004420in}}{\pgfqpoint{0.350000in}{0.000000in}}%
\pgfpathcurveto{\pgfqpoint{0.350000in}{0.004420in}}{\pgfqpoint{0.348244in}{0.008660in}}{\pgfqpoint{0.345118in}{0.011785in}}%
\pgfpathcurveto{\pgfqpoint{0.341993in}{0.014911in}}{\pgfqpoint{0.337753in}{0.016667in}}{\pgfqpoint{0.333333in}{0.016667in}}%
\pgfpathcurveto{\pgfqpoint{0.328913in}{0.016667in}}{\pgfqpoint{0.324674in}{0.014911in}}{\pgfqpoint{0.321548in}{0.011785in}}%
\pgfpathcurveto{\pgfqpoint{0.318423in}{0.008660in}}{\pgfqpoint{0.316667in}{0.004420in}}{\pgfqpoint{0.316667in}{0.000000in}}%
\pgfpathcurveto{\pgfqpoint{0.316667in}{-0.004420in}}{\pgfqpoint{0.318423in}{-0.008660in}}{\pgfqpoint{0.321548in}{-0.011785in}}%
\pgfpathcurveto{\pgfqpoint{0.324674in}{-0.014911in}}{\pgfqpoint{0.328913in}{-0.016667in}}{\pgfqpoint{0.333333in}{-0.016667in}}%
\pgfpathclose%
\pgfpathmoveto{\pgfqpoint{0.500000in}{-0.016667in}}%
\pgfpathcurveto{\pgfqpoint{0.504420in}{-0.016667in}}{\pgfqpoint{0.508660in}{-0.014911in}}{\pgfqpoint{0.511785in}{-0.011785in}}%
\pgfpathcurveto{\pgfqpoint{0.514911in}{-0.008660in}}{\pgfqpoint{0.516667in}{-0.004420in}}{\pgfqpoint{0.516667in}{0.000000in}}%
\pgfpathcurveto{\pgfqpoint{0.516667in}{0.004420in}}{\pgfqpoint{0.514911in}{0.008660in}}{\pgfqpoint{0.511785in}{0.011785in}}%
\pgfpathcurveto{\pgfqpoint{0.508660in}{0.014911in}}{\pgfqpoint{0.504420in}{0.016667in}}{\pgfqpoint{0.500000in}{0.016667in}}%
\pgfpathcurveto{\pgfqpoint{0.495580in}{0.016667in}}{\pgfqpoint{0.491340in}{0.014911in}}{\pgfqpoint{0.488215in}{0.011785in}}%
\pgfpathcurveto{\pgfqpoint{0.485089in}{0.008660in}}{\pgfqpoint{0.483333in}{0.004420in}}{\pgfqpoint{0.483333in}{0.000000in}}%
\pgfpathcurveto{\pgfqpoint{0.483333in}{-0.004420in}}{\pgfqpoint{0.485089in}{-0.008660in}}{\pgfqpoint{0.488215in}{-0.011785in}}%
\pgfpathcurveto{\pgfqpoint{0.491340in}{-0.014911in}}{\pgfqpoint{0.495580in}{-0.016667in}}{\pgfqpoint{0.500000in}{-0.016667in}}%
\pgfpathclose%
\pgfpathmoveto{\pgfqpoint{0.666667in}{-0.016667in}}%
\pgfpathcurveto{\pgfqpoint{0.671087in}{-0.016667in}}{\pgfqpoint{0.675326in}{-0.014911in}}{\pgfqpoint{0.678452in}{-0.011785in}}%
\pgfpathcurveto{\pgfqpoint{0.681577in}{-0.008660in}}{\pgfqpoint{0.683333in}{-0.004420in}}{\pgfqpoint{0.683333in}{0.000000in}}%
\pgfpathcurveto{\pgfqpoint{0.683333in}{0.004420in}}{\pgfqpoint{0.681577in}{0.008660in}}{\pgfqpoint{0.678452in}{0.011785in}}%
\pgfpathcurveto{\pgfqpoint{0.675326in}{0.014911in}}{\pgfqpoint{0.671087in}{0.016667in}}{\pgfqpoint{0.666667in}{0.016667in}}%
\pgfpathcurveto{\pgfqpoint{0.662247in}{0.016667in}}{\pgfqpoint{0.658007in}{0.014911in}}{\pgfqpoint{0.654882in}{0.011785in}}%
\pgfpathcurveto{\pgfqpoint{0.651756in}{0.008660in}}{\pgfqpoint{0.650000in}{0.004420in}}{\pgfqpoint{0.650000in}{0.000000in}}%
\pgfpathcurveto{\pgfqpoint{0.650000in}{-0.004420in}}{\pgfqpoint{0.651756in}{-0.008660in}}{\pgfqpoint{0.654882in}{-0.011785in}}%
\pgfpathcurveto{\pgfqpoint{0.658007in}{-0.014911in}}{\pgfqpoint{0.662247in}{-0.016667in}}{\pgfqpoint{0.666667in}{-0.016667in}}%
\pgfpathclose%
\pgfpathmoveto{\pgfqpoint{0.833333in}{-0.016667in}}%
\pgfpathcurveto{\pgfqpoint{0.837753in}{-0.016667in}}{\pgfqpoint{0.841993in}{-0.014911in}}{\pgfqpoint{0.845118in}{-0.011785in}}%
\pgfpathcurveto{\pgfqpoint{0.848244in}{-0.008660in}}{\pgfqpoint{0.850000in}{-0.004420in}}{\pgfqpoint{0.850000in}{0.000000in}}%
\pgfpathcurveto{\pgfqpoint{0.850000in}{0.004420in}}{\pgfqpoint{0.848244in}{0.008660in}}{\pgfqpoint{0.845118in}{0.011785in}}%
\pgfpathcurveto{\pgfqpoint{0.841993in}{0.014911in}}{\pgfqpoint{0.837753in}{0.016667in}}{\pgfqpoint{0.833333in}{0.016667in}}%
\pgfpathcurveto{\pgfqpoint{0.828913in}{0.016667in}}{\pgfqpoint{0.824674in}{0.014911in}}{\pgfqpoint{0.821548in}{0.011785in}}%
\pgfpathcurveto{\pgfqpoint{0.818423in}{0.008660in}}{\pgfqpoint{0.816667in}{0.004420in}}{\pgfqpoint{0.816667in}{0.000000in}}%
\pgfpathcurveto{\pgfqpoint{0.816667in}{-0.004420in}}{\pgfqpoint{0.818423in}{-0.008660in}}{\pgfqpoint{0.821548in}{-0.011785in}}%
\pgfpathcurveto{\pgfqpoint{0.824674in}{-0.014911in}}{\pgfqpoint{0.828913in}{-0.016667in}}{\pgfqpoint{0.833333in}{-0.016667in}}%
\pgfpathclose%
\pgfpathmoveto{\pgfqpoint{1.000000in}{-0.016667in}}%
\pgfpathcurveto{\pgfqpoint{1.004420in}{-0.016667in}}{\pgfqpoint{1.008660in}{-0.014911in}}{\pgfqpoint{1.011785in}{-0.011785in}}%
\pgfpathcurveto{\pgfqpoint{1.014911in}{-0.008660in}}{\pgfqpoint{1.016667in}{-0.004420in}}{\pgfqpoint{1.016667in}{0.000000in}}%
\pgfpathcurveto{\pgfqpoint{1.016667in}{0.004420in}}{\pgfqpoint{1.014911in}{0.008660in}}{\pgfqpoint{1.011785in}{0.011785in}}%
\pgfpathcurveto{\pgfqpoint{1.008660in}{0.014911in}}{\pgfqpoint{1.004420in}{0.016667in}}{\pgfqpoint{1.000000in}{0.016667in}}%
\pgfpathcurveto{\pgfqpoint{0.995580in}{0.016667in}}{\pgfqpoint{0.991340in}{0.014911in}}{\pgfqpoint{0.988215in}{0.011785in}}%
\pgfpathcurveto{\pgfqpoint{0.985089in}{0.008660in}}{\pgfqpoint{0.983333in}{0.004420in}}{\pgfqpoint{0.983333in}{0.000000in}}%
\pgfpathcurveto{\pgfqpoint{0.983333in}{-0.004420in}}{\pgfqpoint{0.985089in}{-0.008660in}}{\pgfqpoint{0.988215in}{-0.011785in}}%
\pgfpathcurveto{\pgfqpoint{0.991340in}{-0.014911in}}{\pgfqpoint{0.995580in}{-0.016667in}}{\pgfqpoint{1.000000in}{-0.016667in}}%
\pgfpathclose%
\pgfpathmoveto{\pgfqpoint{0.083333in}{0.150000in}}%
\pgfpathcurveto{\pgfqpoint{0.087753in}{0.150000in}}{\pgfqpoint{0.091993in}{0.151756in}}{\pgfqpoint{0.095118in}{0.154882in}}%
\pgfpathcurveto{\pgfqpoint{0.098244in}{0.158007in}}{\pgfqpoint{0.100000in}{0.162247in}}{\pgfqpoint{0.100000in}{0.166667in}}%
\pgfpathcurveto{\pgfqpoint{0.100000in}{0.171087in}}{\pgfqpoint{0.098244in}{0.175326in}}{\pgfqpoint{0.095118in}{0.178452in}}%
\pgfpathcurveto{\pgfqpoint{0.091993in}{0.181577in}}{\pgfqpoint{0.087753in}{0.183333in}}{\pgfqpoint{0.083333in}{0.183333in}}%
\pgfpathcurveto{\pgfqpoint{0.078913in}{0.183333in}}{\pgfqpoint{0.074674in}{0.181577in}}{\pgfqpoint{0.071548in}{0.178452in}}%
\pgfpathcurveto{\pgfqpoint{0.068423in}{0.175326in}}{\pgfqpoint{0.066667in}{0.171087in}}{\pgfqpoint{0.066667in}{0.166667in}}%
\pgfpathcurveto{\pgfqpoint{0.066667in}{0.162247in}}{\pgfqpoint{0.068423in}{0.158007in}}{\pgfqpoint{0.071548in}{0.154882in}}%
\pgfpathcurveto{\pgfqpoint{0.074674in}{0.151756in}}{\pgfqpoint{0.078913in}{0.150000in}}{\pgfqpoint{0.083333in}{0.150000in}}%
\pgfpathclose%
\pgfpathmoveto{\pgfqpoint{0.250000in}{0.150000in}}%
\pgfpathcurveto{\pgfqpoint{0.254420in}{0.150000in}}{\pgfqpoint{0.258660in}{0.151756in}}{\pgfqpoint{0.261785in}{0.154882in}}%
\pgfpathcurveto{\pgfqpoint{0.264911in}{0.158007in}}{\pgfqpoint{0.266667in}{0.162247in}}{\pgfqpoint{0.266667in}{0.166667in}}%
\pgfpathcurveto{\pgfqpoint{0.266667in}{0.171087in}}{\pgfqpoint{0.264911in}{0.175326in}}{\pgfqpoint{0.261785in}{0.178452in}}%
\pgfpathcurveto{\pgfqpoint{0.258660in}{0.181577in}}{\pgfqpoint{0.254420in}{0.183333in}}{\pgfqpoint{0.250000in}{0.183333in}}%
\pgfpathcurveto{\pgfqpoint{0.245580in}{0.183333in}}{\pgfqpoint{0.241340in}{0.181577in}}{\pgfqpoint{0.238215in}{0.178452in}}%
\pgfpathcurveto{\pgfqpoint{0.235089in}{0.175326in}}{\pgfqpoint{0.233333in}{0.171087in}}{\pgfqpoint{0.233333in}{0.166667in}}%
\pgfpathcurveto{\pgfqpoint{0.233333in}{0.162247in}}{\pgfqpoint{0.235089in}{0.158007in}}{\pgfqpoint{0.238215in}{0.154882in}}%
\pgfpathcurveto{\pgfqpoint{0.241340in}{0.151756in}}{\pgfqpoint{0.245580in}{0.150000in}}{\pgfqpoint{0.250000in}{0.150000in}}%
\pgfpathclose%
\pgfpathmoveto{\pgfqpoint{0.416667in}{0.150000in}}%
\pgfpathcurveto{\pgfqpoint{0.421087in}{0.150000in}}{\pgfqpoint{0.425326in}{0.151756in}}{\pgfqpoint{0.428452in}{0.154882in}}%
\pgfpathcurveto{\pgfqpoint{0.431577in}{0.158007in}}{\pgfqpoint{0.433333in}{0.162247in}}{\pgfqpoint{0.433333in}{0.166667in}}%
\pgfpathcurveto{\pgfqpoint{0.433333in}{0.171087in}}{\pgfqpoint{0.431577in}{0.175326in}}{\pgfqpoint{0.428452in}{0.178452in}}%
\pgfpathcurveto{\pgfqpoint{0.425326in}{0.181577in}}{\pgfqpoint{0.421087in}{0.183333in}}{\pgfqpoint{0.416667in}{0.183333in}}%
\pgfpathcurveto{\pgfqpoint{0.412247in}{0.183333in}}{\pgfqpoint{0.408007in}{0.181577in}}{\pgfqpoint{0.404882in}{0.178452in}}%
\pgfpathcurveto{\pgfqpoint{0.401756in}{0.175326in}}{\pgfqpoint{0.400000in}{0.171087in}}{\pgfqpoint{0.400000in}{0.166667in}}%
\pgfpathcurveto{\pgfqpoint{0.400000in}{0.162247in}}{\pgfqpoint{0.401756in}{0.158007in}}{\pgfqpoint{0.404882in}{0.154882in}}%
\pgfpathcurveto{\pgfqpoint{0.408007in}{0.151756in}}{\pgfqpoint{0.412247in}{0.150000in}}{\pgfqpoint{0.416667in}{0.150000in}}%
\pgfpathclose%
\pgfpathmoveto{\pgfqpoint{0.583333in}{0.150000in}}%
\pgfpathcurveto{\pgfqpoint{0.587753in}{0.150000in}}{\pgfqpoint{0.591993in}{0.151756in}}{\pgfqpoint{0.595118in}{0.154882in}}%
\pgfpathcurveto{\pgfqpoint{0.598244in}{0.158007in}}{\pgfqpoint{0.600000in}{0.162247in}}{\pgfqpoint{0.600000in}{0.166667in}}%
\pgfpathcurveto{\pgfqpoint{0.600000in}{0.171087in}}{\pgfqpoint{0.598244in}{0.175326in}}{\pgfqpoint{0.595118in}{0.178452in}}%
\pgfpathcurveto{\pgfqpoint{0.591993in}{0.181577in}}{\pgfqpoint{0.587753in}{0.183333in}}{\pgfqpoint{0.583333in}{0.183333in}}%
\pgfpathcurveto{\pgfqpoint{0.578913in}{0.183333in}}{\pgfqpoint{0.574674in}{0.181577in}}{\pgfqpoint{0.571548in}{0.178452in}}%
\pgfpathcurveto{\pgfqpoint{0.568423in}{0.175326in}}{\pgfqpoint{0.566667in}{0.171087in}}{\pgfqpoint{0.566667in}{0.166667in}}%
\pgfpathcurveto{\pgfqpoint{0.566667in}{0.162247in}}{\pgfqpoint{0.568423in}{0.158007in}}{\pgfqpoint{0.571548in}{0.154882in}}%
\pgfpathcurveto{\pgfqpoint{0.574674in}{0.151756in}}{\pgfqpoint{0.578913in}{0.150000in}}{\pgfqpoint{0.583333in}{0.150000in}}%
\pgfpathclose%
\pgfpathmoveto{\pgfqpoint{0.750000in}{0.150000in}}%
\pgfpathcurveto{\pgfqpoint{0.754420in}{0.150000in}}{\pgfqpoint{0.758660in}{0.151756in}}{\pgfqpoint{0.761785in}{0.154882in}}%
\pgfpathcurveto{\pgfqpoint{0.764911in}{0.158007in}}{\pgfqpoint{0.766667in}{0.162247in}}{\pgfqpoint{0.766667in}{0.166667in}}%
\pgfpathcurveto{\pgfqpoint{0.766667in}{0.171087in}}{\pgfqpoint{0.764911in}{0.175326in}}{\pgfqpoint{0.761785in}{0.178452in}}%
\pgfpathcurveto{\pgfqpoint{0.758660in}{0.181577in}}{\pgfqpoint{0.754420in}{0.183333in}}{\pgfqpoint{0.750000in}{0.183333in}}%
\pgfpathcurveto{\pgfqpoint{0.745580in}{0.183333in}}{\pgfqpoint{0.741340in}{0.181577in}}{\pgfqpoint{0.738215in}{0.178452in}}%
\pgfpathcurveto{\pgfqpoint{0.735089in}{0.175326in}}{\pgfqpoint{0.733333in}{0.171087in}}{\pgfqpoint{0.733333in}{0.166667in}}%
\pgfpathcurveto{\pgfqpoint{0.733333in}{0.162247in}}{\pgfqpoint{0.735089in}{0.158007in}}{\pgfqpoint{0.738215in}{0.154882in}}%
\pgfpathcurveto{\pgfqpoint{0.741340in}{0.151756in}}{\pgfqpoint{0.745580in}{0.150000in}}{\pgfqpoint{0.750000in}{0.150000in}}%
\pgfpathclose%
\pgfpathmoveto{\pgfqpoint{0.916667in}{0.150000in}}%
\pgfpathcurveto{\pgfqpoint{0.921087in}{0.150000in}}{\pgfqpoint{0.925326in}{0.151756in}}{\pgfqpoint{0.928452in}{0.154882in}}%
\pgfpathcurveto{\pgfqpoint{0.931577in}{0.158007in}}{\pgfqpoint{0.933333in}{0.162247in}}{\pgfqpoint{0.933333in}{0.166667in}}%
\pgfpathcurveto{\pgfqpoint{0.933333in}{0.171087in}}{\pgfqpoint{0.931577in}{0.175326in}}{\pgfqpoint{0.928452in}{0.178452in}}%
\pgfpathcurveto{\pgfqpoint{0.925326in}{0.181577in}}{\pgfqpoint{0.921087in}{0.183333in}}{\pgfqpoint{0.916667in}{0.183333in}}%
\pgfpathcurveto{\pgfqpoint{0.912247in}{0.183333in}}{\pgfqpoint{0.908007in}{0.181577in}}{\pgfqpoint{0.904882in}{0.178452in}}%
\pgfpathcurveto{\pgfqpoint{0.901756in}{0.175326in}}{\pgfqpoint{0.900000in}{0.171087in}}{\pgfqpoint{0.900000in}{0.166667in}}%
\pgfpathcurveto{\pgfqpoint{0.900000in}{0.162247in}}{\pgfqpoint{0.901756in}{0.158007in}}{\pgfqpoint{0.904882in}{0.154882in}}%
\pgfpathcurveto{\pgfqpoint{0.908007in}{0.151756in}}{\pgfqpoint{0.912247in}{0.150000in}}{\pgfqpoint{0.916667in}{0.150000in}}%
\pgfpathclose%
\pgfpathmoveto{\pgfqpoint{0.000000in}{0.316667in}}%
\pgfpathcurveto{\pgfqpoint{0.004420in}{0.316667in}}{\pgfqpoint{0.008660in}{0.318423in}}{\pgfqpoint{0.011785in}{0.321548in}}%
\pgfpathcurveto{\pgfqpoint{0.014911in}{0.324674in}}{\pgfqpoint{0.016667in}{0.328913in}}{\pgfqpoint{0.016667in}{0.333333in}}%
\pgfpathcurveto{\pgfqpoint{0.016667in}{0.337753in}}{\pgfqpoint{0.014911in}{0.341993in}}{\pgfqpoint{0.011785in}{0.345118in}}%
\pgfpathcurveto{\pgfqpoint{0.008660in}{0.348244in}}{\pgfqpoint{0.004420in}{0.350000in}}{\pgfqpoint{0.000000in}{0.350000in}}%
\pgfpathcurveto{\pgfqpoint{-0.004420in}{0.350000in}}{\pgfqpoint{-0.008660in}{0.348244in}}{\pgfqpoint{-0.011785in}{0.345118in}}%
\pgfpathcurveto{\pgfqpoint{-0.014911in}{0.341993in}}{\pgfqpoint{-0.016667in}{0.337753in}}{\pgfqpoint{-0.016667in}{0.333333in}}%
\pgfpathcurveto{\pgfqpoint{-0.016667in}{0.328913in}}{\pgfqpoint{-0.014911in}{0.324674in}}{\pgfqpoint{-0.011785in}{0.321548in}}%
\pgfpathcurveto{\pgfqpoint{-0.008660in}{0.318423in}}{\pgfqpoint{-0.004420in}{0.316667in}}{\pgfqpoint{0.000000in}{0.316667in}}%
\pgfpathclose%
\pgfpathmoveto{\pgfqpoint{0.166667in}{0.316667in}}%
\pgfpathcurveto{\pgfqpoint{0.171087in}{0.316667in}}{\pgfqpoint{0.175326in}{0.318423in}}{\pgfqpoint{0.178452in}{0.321548in}}%
\pgfpathcurveto{\pgfqpoint{0.181577in}{0.324674in}}{\pgfqpoint{0.183333in}{0.328913in}}{\pgfqpoint{0.183333in}{0.333333in}}%
\pgfpathcurveto{\pgfqpoint{0.183333in}{0.337753in}}{\pgfqpoint{0.181577in}{0.341993in}}{\pgfqpoint{0.178452in}{0.345118in}}%
\pgfpathcurveto{\pgfqpoint{0.175326in}{0.348244in}}{\pgfqpoint{0.171087in}{0.350000in}}{\pgfqpoint{0.166667in}{0.350000in}}%
\pgfpathcurveto{\pgfqpoint{0.162247in}{0.350000in}}{\pgfqpoint{0.158007in}{0.348244in}}{\pgfqpoint{0.154882in}{0.345118in}}%
\pgfpathcurveto{\pgfqpoint{0.151756in}{0.341993in}}{\pgfqpoint{0.150000in}{0.337753in}}{\pgfqpoint{0.150000in}{0.333333in}}%
\pgfpathcurveto{\pgfqpoint{0.150000in}{0.328913in}}{\pgfqpoint{0.151756in}{0.324674in}}{\pgfqpoint{0.154882in}{0.321548in}}%
\pgfpathcurveto{\pgfqpoint{0.158007in}{0.318423in}}{\pgfqpoint{0.162247in}{0.316667in}}{\pgfqpoint{0.166667in}{0.316667in}}%
\pgfpathclose%
\pgfpathmoveto{\pgfqpoint{0.333333in}{0.316667in}}%
\pgfpathcurveto{\pgfqpoint{0.337753in}{0.316667in}}{\pgfqpoint{0.341993in}{0.318423in}}{\pgfqpoint{0.345118in}{0.321548in}}%
\pgfpathcurveto{\pgfqpoint{0.348244in}{0.324674in}}{\pgfqpoint{0.350000in}{0.328913in}}{\pgfqpoint{0.350000in}{0.333333in}}%
\pgfpathcurveto{\pgfqpoint{0.350000in}{0.337753in}}{\pgfqpoint{0.348244in}{0.341993in}}{\pgfqpoint{0.345118in}{0.345118in}}%
\pgfpathcurveto{\pgfqpoint{0.341993in}{0.348244in}}{\pgfqpoint{0.337753in}{0.350000in}}{\pgfqpoint{0.333333in}{0.350000in}}%
\pgfpathcurveto{\pgfqpoint{0.328913in}{0.350000in}}{\pgfqpoint{0.324674in}{0.348244in}}{\pgfqpoint{0.321548in}{0.345118in}}%
\pgfpathcurveto{\pgfqpoint{0.318423in}{0.341993in}}{\pgfqpoint{0.316667in}{0.337753in}}{\pgfqpoint{0.316667in}{0.333333in}}%
\pgfpathcurveto{\pgfqpoint{0.316667in}{0.328913in}}{\pgfqpoint{0.318423in}{0.324674in}}{\pgfqpoint{0.321548in}{0.321548in}}%
\pgfpathcurveto{\pgfqpoint{0.324674in}{0.318423in}}{\pgfqpoint{0.328913in}{0.316667in}}{\pgfqpoint{0.333333in}{0.316667in}}%
\pgfpathclose%
\pgfpathmoveto{\pgfqpoint{0.500000in}{0.316667in}}%
\pgfpathcurveto{\pgfqpoint{0.504420in}{0.316667in}}{\pgfqpoint{0.508660in}{0.318423in}}{\pgfqpoint{0.511785in}{0.321548in}}%
\pgfpathcurveto{\pgfqpoint{0.514911in}{0.324674in}}{\pgfqpoint{0.516667in}{0.328913in}}{\pgfqpoint{0.516667in}{0.333333in}}%
\pgfpathcurveto{\pgfqpoint{0.516667in}{0.337753in}}{\pgfqpoint{0.514911in}{0.341993in}}{\pgfqpoint{0.511785in}{0.345118in}}%
\pgfpathcurveto{\pgfqpoint{0.508660in}{0.348244in}}{\pgfqpoint{0.504420in}{0.350000in}}{\pgfqpoint{0.500000in}{0.350000in}}%
\pgfpathcurveto{\pgfqpoint{0.495580in}{0.350000in}}{\pgfqpoint{0.491340in}{0.348244in}}{\pgfqpoint{0.488215in}{0.345118in}}%
\pgfpathcurveto{\pgfqpoint{0.485089in}{0.341993in}}{\pgfqpoint{0.483333in}{0.337753in}}{\pgfqpoint{0.483333in}{0.333333in}}%
\pgfpathcurveto{\pgfqpoint{0.483333in}{0.328913in}}{\pgfqpoint{0.485089in}{0.324674in}}{\pgfqpoint{0.488215in}{0.321548in}}%
\pgfpathcurveto{\pgfqpoint{0.491340in}{0.318423in}}{\pgfqpoint{0.495580in}{0.316667in}}{\pgfqpoint{0.500000in}{0.316667in}}%
\pgfpathclose%
\pgfpathmoveto{\pgfqpoint{0.666667in}{0.316667in}}%
\pgfpathcurveto{\pgfqpoint{0.671087in}{0.316667in}}{\pgfqpoint{0.675326in}{0.318423in}}{\pgfqpoint{0.678452in}{0.321548in}}%
\pgfpathcurveto{\pgfqpoint{0.681577in}{0.324674in}}{\pgfqpoint{0.683333in}{0.328913in}}{\pgfqpoint{0.683333in}{0.333333in}}%
\pgfpathcurveto{\pgfqpoint{0.683333in}{0.337753in}}{\pgfqpoint{0.681577in}{0.341993in}}{\pgfqpoint{0.678452in}{0.345118in}}%
\pgfpathcurveto{\pgfqpoint{0.675326in}{0.348244in}}{\pgfqpoint{0.671087in}{0.350000in}}{\pgfqpoint{0.666667in}{0.350000in}}%
\pgfpathcurveto{\pgfqpoint{0.662247in}{0.350000in}}{\pgfqpoint{0.658007in}{0.348244in}}{\pgfqpoint{0.654882in}{0.345118in}}%
\pgfpathcurveto{\pgfqpoint{0.651756in}{0.341993in}}{\pgfqpoint{0.650000in}{0.337753in}}{\pgfqpoint{0.650000in}{0.333333in}}%
\pgfpathcurveto{\pgfqpoint{0.650000in}{0.328913in}}{\pgfqpoint{0.651756in}{0.324674in}}{\pgfqpoint{0.654882in}{0.321548in}}%
\pgfpathcurveto{\pgfqpoint{0.658007in}{0.318423in}}{\pgfqpoint{0.662247in}{0.316667in}}{\pgfqpoint{0.666667in}{0.316667in}}%
\pgfpathclose%
\pgfpathmoveto{\pgfqpoint{0.833333in}{0.316667in}}%
\pgfpathcurveto{\pgfqpoint{0.837753in}{0.316667in}}{\pgfqpoint{0.841993in}{0.318423in}}{\pgfqpoint{0.845118in}{0.321548in}}%
\pgfpathcurveto{\pgfqpoint{0.848244in}{0.324674in}}{\pgfqpoint{0.850000in}{0.328913in}}{\pgfqpoint{0.850000in}{0.333333in}}%
\pgfpathcurveto{\pgfqpoint{0.850000in}{0.337753in}}{\pgfqpoint{0.848244in}{0.341993in}}{\pgfqpoint{0.845118in}{0.345118in}}%
\pgfpathcurveto{\pgfqpoint{0.841993in}{0.348244in}}{\pgfqpoint{0.837753in}{0.350000in}}{\pgfqpoint{0.833333in}{0.350000in}}%
\pgfpathcurveto{\pgfqpoint{0.828913in}{0.350000in}}{\pgfqpoint{0.824674in}{0.348244in}}{\pgfqpoint{0.821548in}{0.345118in}}%
\pgfpathcurveto{\pgfqpoint{0.818423in}{0.341993in}}{\pgfqpoint{0.816667in}{0.337753in}}{\pgfqpoint{0.816667in}{0.333333in}}%
\pgfpathcurveto{\pgfqpoint{0.816667in}{0.328913in}}{\pgfqpoint{0.818423in}{0.324674in}}{\pgfqpoint{0.821548in}{0.321548in}}%
\pgfpathcurveto{\pgfqpoint{0.824674in}{0.318423in}}{\pgfqpoint{0.828913in}{0.316667in}}{\pgfqpoint{0.833333in}{0.316667in}}%
\pgfpathclose%
\pgfpathmoveto{\pgfqpoint{1.000000in}{0.316667in}}%
\pgfpathcurveto{\pgfqpoint{1.004420in}{0.316667in}}{\pgfqpoint{1.008660in}{0.318423in}}{\pgfqpoint{1.011785in}{0.321548in}}%
\pgfpathcurveto{\pgfqpoint{1.014911in}{0.324674in}}{\pgfqpoint{1.016667in}{0.328913in}}{\pgfqpoint{1.016667in}{0.333333in}}%
\pgfpathcurveto{\pgfqpoint{1.016667in}{0.337753in}}{\pgfqpoint{1.014911in}{0.341993in}}{\pgfqpoint{1.011785in}{0.345118in}}%
\pgfpathcurveto{\pgfqpoint{1.008660in}{0.348244in}}{\pgfqpoint{1.004420in}{0.350000in}}{\pgfqpoint{1.000000in}{0.350000in}}%
\pgfpathcurveto{\pgfqpoint{0.995580in}{0.350000in}}{\pgfqpoint{0.991340in}{0.348244in}}{\pgfqpoint{0.988215in}{0.345118in}}%
\pgfpathcurveto{\pgfqpoint{0.985089in}{0.341993in}}{\pgfqpoint{0.983333in}{0.337753in}}{\pgfqpoint{0.983333in}{0.333333in}}%
\pgfpathcurveto{\pgfqpoint{0.983333in}{0.328913in}}{\pgfqpoint{0.985089in}{0.324674in}}{\pgfqpoint{0.988215in}{0.321548in}}%
\pgfpathcurveto{\pgfqpoint{0.991340in}{0.318423in}}{\pgfqpoint{0.995580in}{0.316667in}}{\pgfqpoint{1.000000in}{0.316667in}}%
\pgfpathclose%
\pgfpathmoveto{\pgfqpoint{0.083333in}{0.483333in}}%
\pgfpathcurveto{\pgfqpoint{0.087753in}{0.483333in}}{\pgfqpoint{0.091993in}{0.485089in}}{\pgfqpoint{0.095118in}{0.488215in}}%
\pgfpathcurveto{\pgfqpoint{0.098244in}{0.491340in}}{\pgfqpoint{0.100000in}{0.495580in}}{\pgfqpoint{0.100000in}{0.500000in}}%
\pgfpathcurveto{\pgfqpoint{0.100000in}{0.504420in}}{\pgfqpoint{0.098244in}{0.508660in}}{\pgfqpoint{0.095118in}{0.511785in}}%
\pgfpathcurveto{\pgfqpoint{0.091993in}{0.514911in}}{\pgfqpoint{0.087753in}{0.516667in}}{\pgfqpoint{0.083333in}{0.516667in}}%
\pgfpathcurveto{\pgfqpoint{0.078913in}{0.516667in}}{\pgfqpoint{0.074674in}{0.514911in}}{\pgfqpoint{0.071548in}{0.511785in}}%
\pgfpathcurveto{\pgfqpoint{0.068423in}{0.508660in}}{\pgfqpoint{0.066667in}{0.504420in}}{\pgfqpoint{0.066667in}{0.500000in}}%
\pgfpathcurveto{\pgfqpoint{0.066667in}{0.495580in}}{\pgfqpoint{0.068423in}{0.491340in}}{\pgfqpoint{0.071548in}{0.488215in}}%
\pgfpathcurveto{\pgfqpoint{0.074674in}{0.485089in}}{\pgfqpoint{0.078913in}{0.483333in}}{\pgfqpoint{0.083333in}{0.483333in}}%
\pgfpathclose%
\pgfpathmoveto{\pgfqpoint{0.250000in}{0.483333in}}%
\pgfpathcurveto{\pgfqpoint{0.254420in}{0.483333in}}{\pgfqpoint{0.258660in}{0.485089in}}{\pgfqpoint{0.261785in}{0.488215in}}%
\pgfpathcurveto{\pgfqpoint{0.264911in}{0.491340in}}{\pgfqpoint{0.266667in}{0.495580in}}{\pgfqpoint{0.266667in}{0.500000in}}%
\pgfpathcurveto{\pgfqpoint{0.266667in}{0.504420in}}{\pgfqpoint{0.264911in}{0.508660in}}{\pgfqpoint{0.261785in}{0.511785in}}%
\pgfpathcurveto{\pgfqpoint{0.258660in}{0.514911in}}{\pgfqpoint{0.254420in}{0.516667in}}{\pgfqpoint{0.250000in}{0.516667in}}%
\pgfpathcurveto{\pgfqpoint{0.245580in}{0.516667in}}{\pgfqpoint{0.241340in}{0.514911in}}{\pgfqpoint{0.238215in}{0.511785in}}%
\pgfpathcurveto{\pgfqpoint{0.235089in}{0.508660in}}{\pgfqpoint{0.233333in}{0.504420in}}{\pgfqpoint{0.233333in}{0.500000in}}%
\pgfpathcurveto{\pgfqpoint{0.233333in}{0.495580in}}{\pgfqpoint{0.235089in}{0.491340in}}{\pgfqpoint{0.238215in}{0.488215in}}%
\pgfpathcurveto{\pgfqpoint{0.241340in}{0.485089in}}{\pgfqpoint{0.245580in}{0.483333in}}{\pgfqpoint{0.250000in}{0.483333in}}%
\pgfpathclose%
\pgfpathmoveto{\pgfqpoint{0.416667in}{0.483333in}}%
\pgfpathcurveto{\pgfqpoint{0.421087in}{0.483333in}}{\pgfqpoint{0.425326in}{0.485089in}}{\pgfqpoint{0.428452in}{0.488215in}}%
\pgfpathcurveto{\pgfqpoint{0.431577in}{0.491340in}}{\pgfqpoint{0.433333in}{0.495580in}}{\pgfqpoint{0.433333in}{0.500000in}}%
\pgfpathcurveto{\pgfqpoint{0.433333in}{0.504420in}}{\pgfqpoint{0.431577in}{0.508660in}}{\pgfqpoint{0.428452in}{0.511785in}}%
\pgfpathcurveto{\pgfqpoint{0.425326in}{0.514911in}}{\pgfqpoint{0.421087in}{0.516667in}}{\pgfqpoint{0.416667in}{0.516667in}}%
\pgfpathcurveto{\pgfqpoint{0.412247in}{0.516667in}}{\pgfqpoint{0.408007in}{0.514911in}}{\pgfqpoint{0.404882in}{0.511785in}}%
\pgfpathcurveto{\pgfqpoint{0.401756in}{0.508660in}}{\pgfqpoint{0.400000in}{0.504420in}}{\pgfqpoint{0.400000in}{0.500000in}}%
\pgfpathcurveto{\pgfqpoint{0.400000in}{0.495580in}}{\pgfqpoint{0.401756in}{0.491340in}}{\pgfqpoint{0.404882in}{0.488215in}}%
\pgfpathcurveto{\pgfqpoint{0.408007in}{0.485089in}}{\pgfqpoint{0.412247in}{0.483333in}}{\pgfqpoint{0.416667in}{0.483333in}}%
\pgfpathclose%
\pgfpathmoveto{\pgfqpoint{0.583333in}{0.483333in}}%
\pgfpathcurveto{\pgfqpoint{0.587753in}{0.483333in}}{\pgfqpoint{0.591993in}{0.485089in}}{\pgfqpoint{0.595118in}{0.488215in}}%
\pgfpathcurveto{\pgfqpoint{0.598244in}{0.491340in}}{\pgfqpoint{0.600000in}{0.495580in}}{\pgfqpoint{0.600000in}{0.500000in}}%
\pgfpathcurveto{\pgfqpoint{0.600000in}{0.504420in}}{\pgfqpoint{0.598244in}{0.508660in}}{\pgfqpoint{0.595118in}{0.511785in}}%
\pgfpathcurveto{\pgfqpoint{0.591993in}{0.514911in}}{\pgfqpoint{0.587753in}{0.516667in}}{\pgfqpoint{0.583333in}{0.516667in}}%
\pgfpathcurveto{\pgfqpoint{0.578913in}{0.516667in}}{\pgfqpoint{0.574674in}{0.514911in}}{\pgfqpoint{0.571548in}{0.511785in}}%
\pgfpathcurveto{\pgfqpoint{0.568423in}{0.508660in}}{\pgfqpoint{0.566667in}{0.504420in}}{\pgfqpoint{0.566667in}{0.500000in}}%
\pgfpathcurveto{\pgfqpoint{0.566667in}{0.495580in}}{\pgfqpoint{0.568423in}{0.491340in}}{\pgfqpoint{0.571548in}{0.488215in}}%
\pgfpathcurveto{\pgfqpoint{0.574674in}{0.485089in}}{\pgfqpoint{0.578913in}{0.483333in}}{\pgfqpoint{0.583333in}{0.483333in}}%
\pgfpathclose%
\pgfpathmoveto{\pgfqpoint{0.750000in}{0.483333in}}%
\pgfpathcurveto{\pgfqpoint{0.754420in}{0.483333in}}{\pgfqpoint{0.758660in}{0.485089in}}{\pgfqpoint{0.761785in}{0.488215in}}%
\pgfpathcurveto{\pgfqpoint{0.764911in}{0.491340in}}{\pgfqpoint{0.766667in}{0.495580in}}{\pgfqpoint{0.766667in}{0.500000in}}%
\pgfpathcurveto{\pgfqpoint{0.766667in}{0.504420in}}{\pgfqpoint{0.764911in}{0.508660in}}{\pgfqpoint{0.761785in}{0.511785in}}%
\pgfpathcurveto{\pgfqpoint{0.758660in}{0.514911in}}{\pgfqpoint{0.754420in}{0.516667in}}{\pgfqpoint{0.750000in}{0.516667in}}%
\pgfpathcurveto{\pgfqpoint{0.745580in}{0.516667in}}{\pgfqpoint{0.741340in}{0.514911in}}{\pgfqpoint{0.738215in}{0.511785in}}%
\pgfpathcurveto{\pgfqpoint{0.735089in}{0.508660in}}{\pgfqpoint{0.733333in}{0.504420in}}{\pgfqpoint{0.733333in}{0.500000in}}%
\pgfpathcurveto{\pgfqpoint{0.733333in}{0.495580in}}{\pgfqpoint{0.735089in}{0.491340in}}{\pgfqpoint{0.738215in}{0.488215in}}%
\pgfpathcurveto{\pgfqpoint{0.741340in}{0.485089in}}{\pgfqpoint{0.745580in}{0.483333in}}{\pgfqpoint{0.750000in}{0.483333in}}%
\pgfpathclose%
\pgfpathmoveto{\pgfqpoint{0.916667in}{0.483333in}}%
\pgfpathcurveto{\pgfqpoint{0.921087in}{0.483333in}}{\pgfqpoint{0.925326in}{0.485089in}}{\pgfqpoint{0.928452in}{0.488215in}}%
\pgfpathcurveto{\pgfqpoint{0.931577in}{0.491340in}}{\pgfqpoint{0.933333in}{0.495580in}}{\pgfqpoint{0.933333in}{0.500000in}}%
\pgfpathcurveto{\pgfqpoint{0.933333in}{0.504420in}}{\pgfqpoint{0.931577in}{0.508660in}}{\pgfqpoint{0.928452in}{0.511785in}}%
\pgfpathcurveto{\pgfqpoint{0.925326in}{0.514911in}}{\pgfqpoint{0.921087in}{0.516667in}}{\pgfqpoint{0.916667in}{0.516667in}}%
\pgfpathcurveto{\pgfqpoint{0.912247in}{0.516667in}}{\pgfqpoint{0.908007in}{0.514911in}}{\pgfqpoint{0.904882in}{0.511785in}}%
\pgfpathcurveto{\pgfqpoint{0.901756in}{0.508660in}}{\pgfqpoint{0.900000in}{0.504420in}}{\pgfqpoint{0.900000in}{0.500000in}}%
\pgfpathcurveto{\pgfqpoint{0.900000in}{0.495580in}}{\pgfqpoint{0.901756in}{0.491340in}}{\pgfqpoint{0.904882in}{0.488215in}}%
\pgfpathcurveto{\pgfqpoint{0.908007in}{0.485089in}}{\pgfqpoint{0.912247in}{0.483333in}}{\pgfqpoint{0.916667in}{0.483333in}}%
\pgfpathclose%
\pgfpathmoveto{\pgfqpoint{0.000000in}{0.650000in}}%
\pgfpathcurveto{\pgfqpoint{0.004420in}{0.650000in}}{\pgfqpoint{0.008660in}{0.651756in}}{\pgfqpoint{0.011785in}{0.654882in}}%
\pgfpathcurveto{\pgfqpoint{0.014911in}{0.658007in}}{\pgfqpoint{0.016667in}{0.662247in}}{\pgfqpoint{0.016667in}{0.666667in}}%
\pgfpathcurveto{\pgfqpoint{0.016667in}{0.671087in}}{\pgfqpoint{0.014911in}{0.675326in}}{\pgfqpoint{0.011785in}{0.678452in}}%
\pgfpathcurveto{\pgfqpoint{0.008660in}{0.681577in}}{\pgfqpoint{0.004420in}{0.683333in}}{\pgfqpoint{0.000000in}{0.683333in}}%
\pgfpathcurveto{\pgfqpoint{-0.004420in}{0.683333in}}{\pgfqpoint{-0.008660in}{0.681577in}}{\pgfqpoint{-0.011785in}{0.678452in}}%
\pgfpathcurveto{\pgfqpoint{-0.014911in}{0.675326in}}{\pgfqpoint{-0.016667in}{0.671087in}}{\pgfqpoint{-0.016667in}{0.666667in}}%
\pgfpathcurveto{\pgfqpoint{-0.016667in}{0.662247in}}{\pgfqpoint{-0.014911in}{0.658007in}}{\pgfqpoint{-0.011785in}{0.654882in}}%
\pgfpathcurveto{\pgfqpoint{-0.008660in}{0.651756in}}{\pgfqpoint{-0.004420in}{0.650000in}}{\pgfqpoint{0.000000in}{0.650000in}}%
\pgfpathclose%
\pgfpathmoveto{\pgfqpoint{0.166667in}{0.650000in}}%
\pgfpathcurveto{\pgfqpoint{0.171087in}{0.650000in}}{\pgfqpoint{0.175326in}{0.651756in}}{\pgfqpoint{0.178452in}{0.654882in}}%
\pgfpathcurveto{\pgfqpoint{0.181577in}{0.658007in}}{\pgfqpoint{0.183333in}{0.662247in}}{\pgfqpoint{0.183333in}{0.666667in}}%
\pgfpathcurveto{\pgfqpoint{0.183333in}{0.671087in}}{\pgfqpoint{0.181577in}{0.675326in}}{\pgfqpoint{0.178452in}{0.678452in}}%
\pgfpathcurveto{\pgfqpoint{0.175326in}{0.681577in}}{\pgfqpoint{0.171087in}{0.683333in}}{\pgfqpoint{0.166667in}{0.683333in}}%
\pgfpathcurveto{\pgfqpoint{0.162247in}{0.683333in}}{\pgfqpoint{0.158007in}{0.681577in}}{\pgfqpoint{0.154882in}{0.678452in}}%
\pgfpathcurveto{\pgfqpoint{0.151756in}{0.675326in}}{\pgfqpoint{0.150000in}{0.671087in}}{\pgfqpoint{0.150000in}{0.666667in}}%
\pgfpathcurveto{\pgfqpoint{0.150000in}{0.662247in}}{\pgfqpoint{0.151756in}{0.658007in}}{\pgfqpoint{0.154882in}{0.654882in}}%
\pgfpathcurveto{\pgfqpoint{0.158007in}{0.651756in}}{\pgfqpoint{0.162247in}{0.650000in}}{\pgfqpoint{0.166667in}{0.650000in}}%
\pgfpathclose%
\pgfpathmoveto{\pgfqpoint{0.333333in}{0.650000in}}%
\pgfpathcurveto{\pgfqpoint{0.337753in}{0.650000in}}{\pgfqpoint{0.341993in}{0.651756in}}{\pgfqpoint{0.345118in}{0.654882in}}%
\pgfpathcurveto{\pgfqpoint{0.348244in}{0.658007in}}{\pgfqpoint{0.350000in}{0.662247in}}{\pgfqpoint{0.350000in}{0.666667in}}%
\pgfpathcurveto{\pgfqpoint{0.350000in}{0.671087in}}{\pgfqpoint{0.348244in}{0.675326in}}{\pgfqpoint{0.345118in}{0.678452in}}%
\pgfpathcurveto{\pgfqpoint{0.341993in}{0.681577in}}{\pgfqpoint{0.337753in}{0.683333in}}{\pgfqpoint{0.333333in}{0.683333in}}%
\pgfpathcurveto{\pgfqpoint{0.328913in}{0.683333in}}{\pgfqpoint{0.324674in}{0.681577in}}{\pgfqpoint{0.321548in}{0.678452in}}%
\pgfpathcurveto{\pgfqpoint{0.318423in}{0.675326in}}{\pgfqpoint{0.316667in}{0.671087in}}{\pgfqpoint{0.316667in}{0.666667in}}%
\pgfpathcurveto{\pgfqpoint{0.316667in}{0.662247in}}{\pgfqpoint{0.318423in}{0.658007in}}{\pgfqpoint{0.321548in}{0.654882in}}%
\pgfpathcurveto{\pgfqpoint{0.324674in}{0.651756in}}{\pgfqpoint{0.328913in}{0.650000in}}{\pgfqpoint{0.333333in}{0.650000in}}%
\pgfpathclose%
\pgfpathmoveto{\pgfqpoint{0.500000in}{0.650000in}}%
\pgfpathcurveto{\pgfqpoint{0.504420in}{0.650000in}}{\pgfqpoint{0.508660in}{0.651756in}}{\pgfqpoint{0.511785in}{0.654882in}}%
\pgfpathcurveto{\pgfqpoint{0.514911in}{0.658007in}}{\pgfqpoint{0.516667in}{0.662247in}}{\pgfqpoint{0.516667in}{0.666667in}}%
\pgfpathcurveto{\pgfqpoint{0.516667in}{0.671087in}}{\pgfqpoint{0.514911in}{0.675326in}}{\pgfqpoint{0.511785in}{0.678452in}}%
\pgfpathcurveto{\pgfqpoint{0.508660in}{0.681577in}}{\pgfqpoint{0.504420in}{0.683333in}}{\pgfqpoint{0.500000in}{0.683333in}}%
\pgfpathcurveto{\pgfqpoint{0.495580in}{0.683333in}}{\pgfqpoint{0.491340in}{0.681577in}}{\pgfqpoint{0.488215in}{0.678452in}}%
\pgfpathcurveto{\pgfqpoint{0.485089in}{0.675326in}}{\pgfqpoint{0.483333in}{0.671087in}}{\pgfqpoint{0.483333in}{0.666667in}}%
\pgfpathcurveto{\pgfqpoint{0.483333in}{0.662247in}}{\pgfqpoint{0.485089in}{0.658007in}}{\pgfqpoint{0.488215in}{0.654882in}}%
\pgfpathcurveto{\pgfqpoint{0.491340in}{0.651756in}}{\pgfqpoint{0.495580in}{0.650000in}}{\pgfqpoint{0.500000in}{0.650000in}}%
\pgfpathclose%
\pgfpathmoveto{\pgfqpoint{0.666667in}{0.650000in}}%
\pgfpathcurveto{\pgfqpoint{0.671087in}{0.650000in}}{\pgfqpoint{0.675326in}{0.651756in}}{\pgfqpoint{0.678452in}{0.654882in}}%
\pgfpathcurveto{\pgfqpoint{0.681577in}{0.658007in}}{\pgfqpoint{0.683333in}{0.662247in}}{\pgfqpoint{0.683333in}{0.666667in}}%
\pgfpathcurveto{\pgfqpoint{0.683333in}{0.671087in}}{\pgfqpoint{0.681577in}{0.675326in}}{\pgfqpoint{0.678452in}{0.678452in}}%
\pgfpathcurveto{\pgfqpoint{0.675326in}{0.681577in}}{\pgfqpoint{0.671087in}{0.683333in}}{\pgfqpoint{0.666667in}{0.683333in}}%
\pgfpathcurveto{\pgfqpoint{0.662247in}{0.683333in}}{\pgfqpoint{0.658007in}{0.681577in}}{\pgfqpoint{0.654882in}{0.678452in}}%
\pgfpathcurveto{\pgfqpoint{0.651756in}{0.675326in}}{\pgfqpoint{0.650000in}{0.671087in}}{\pgfqpoint{0.650000in}{0.666667in}}%
\pgfpathcurveto{\pgfqpoint{0.650000in}{0.662247in}}{\pgfqpoint{0.651756in}{0.658007in}}{\pgfqpoint{0.654882in}{0.654882in}}%
\pgfpathcurveto{\pgfqpoint{0.658007in}{0.651756in}}{\pgfqpoint{0.662247in}{0.650000in}}{\pgfqpoint{0.666667in}{0.650000in}}%
\pgfpathclose%
\pgfpathmoveto{\pgfqpoint{0.833333in}{0.650000in}}%
\pgfpathcurveto{\pgfqpoint{0.837753in}{0.650000in}}{\pgfqpoint{0.841993in}{0.651756in}}{\pgfqpoint{0.845118in}{0.654882in}}%
\pgfpathcurveto{\pgfqpoint{0.848244in}{0.658007in}}{\pgfqpoint{0.850000in}{0.662247in}}{\pgfqpoint{0.850000in}{0.666667in}}%
\pgfpathcurveto{\pgfqpoint{0.850000in}{0.671087in}}{\pgfqpoint{0.848244in}{0.675326in}}{\pgfqpoint{0.845118in}{0.678452in}}%
\pgfpathcurveto{\pgfqpoint{0.841993in}{0.681577in}}{\pgfqpoint{0.837753in}{0.683333in}}{\pgfqpoint{0.833333in}{0.683333in}}%
\pgfpathcurveto{\pgfqpoint{0.828913in}{0.683333in}}{\pgfqpoint{0.824674in}{0.681577in}}{\pgfqpoint{0.821548in}{0.678452in}}%
\pgfpathcurveto{\pgfqpoint{0.818423in}{0.675326in}}{\pgfqpoint{0.816667in}{0.671087in}}{\pgfqpoint{0.816667in}{0.666667in}}%
\pgfpathcurveto{\pgfqpoint{0.816667in}{0.662247in}}{\pgfqpoint{0.818423in}{0.658007in}}{\pgfqpoint{0.821548in}{0.654882in}}%
\pgfpathcurveto{\pgfqpoint{0.824674in}{0.651756in}}{\pgfqpoint{0.828913in}{0.650000in}}{\pgfqpoint{0.833333in}{0.650000in}}%
\pgfpathclose%
\pgfpathmoveto{\pgfqpoint{1.000000in}{0.650000in}}%
\pgfpathcurveto{\pgfqpoint{1.004420in}{0.650000in}}{\pgfqpoint{1.008660in}{0.651756in}}{\pgfqpoint{1.011785in}{0.654882in}}%
\pgfpathcurveto{\pgfqpoint{1.014911in}{0.658007in}}{\pgfqpoint{1.016667in}{0.662247in}}{\pgfqpoint{1.016667in}{0.666667in}}%
\pgfpathcurveto{\pgfqpoint{1.016667in}{0.671087in}}{\pgfqpoint{1.014911in}{0.675326in}}{\pgfqpoint{1.011785in}{0.678452in}}%
\pgfpathcurveto{\pgfqpoint{1.008660in}{0.681577in}}{\pgfqpoint{1.004420in}{0.683333in}}{\pgfqpoint{1.000000in}{0.683333in}}%
\pgfpathcurveto{\pgfqpoint{0.995580in}{0.683333in}}{\pgfqpoint{0.991340in}{0.681577in}}{\pgfqpoint{0.988215in}{0.678452in}}%
\pgfpathcurveto{\pgfqpoint{0.985089in}{0.675326in}}{\pgfqpoint{0.983333in}{0.671087in}}{\pgfqpoint{0.983333in}{0.666667in}}%
\pgfpathcurveto{\pgfqpoint{0.983333in}{0.662247in}}{\pgfqpoint{0.985089in}{0.658007in}}{\pgfqpoint{0.988215in}{0.654882in}}%
\pgfpathcurveto{\pgfqpoint{0.991340in}{0.651756in}}{\pgfqpoint{0.995580in}{0.650000in}}{\pgfqpoint{1.000000in}{0.650000in}}%
\pgfpathclose%
\pgfpathmoveto{\pgfqpoint{0.083333in}{0.816667in}}%
\pgfpathcurveto{\pgfqpoint{0.087753in}{0.816667in}}{\pgfqpoint{0.091993in}{0.818423in}}{\pgfqpoint{0.095118in}{0.821548in}}%
\pgfpathcurveto{\pgfqpoint{0.098244in}{0.824674in}}{\pgfqpoint{0.100000in}{0.828913in}}{\pgfqpoint{0.100000in}{0.833333in}}%
\pgfpathcurveto{\pgfqpoint{0.100000in}{0.837753in}}{\pgfqpoint{0.098244in}{0.841993in}}{\pgfqpoint{0.095118in}{0.845118in}}%
\pgfpathcurveto{\pgfqpoint{0.091993in}{0.848244in}}{\pgfqpoint{0.087753in}{0.850000in}}{\pgfqpoint{0.083333in}{0.850000in}}%
\pgfpathcurveto{\pgfqpoint{0.078913in}{0.850000in}}{\pgfqpoint{0.074674in}{0.848244in}}{\pgfqpoint{0.071548in}{0.845118in}}%
\pgfpathcurveto{\pgfqpoint{0.068423in}{0.841993in}}{\pgfqpoint{0.066667in}{0.837753in}}{\pgfqpoint{0.066667in}{0.833333in}}%
\pgfpathcurveto{\pgfqpoint{0.066667in}{0.828913in}}{\pgfqpoint{0.068423in}{0.824674in}}{\pgfqpoint{0.071548in}{0.821548in}}%
\pgfpathcurveto{\pgfqpoint{0.074674in}{0.818423in}}{\pgfqpoint{0.078913in}{0.816667in}}{\pgfqpoint{0.083333in}{0.816667in}}%
\pgfpathclose%
\pgfpathmoveto{\pgfqpoint{0.250000in}{0.816667in}}%
\pgfpathcurveto{\pgfqpoint{0.254420in}{0.816667in}}{\pgfqpoint{0.258660in}{0.818423in}}{\pgfqpoint{0.261785in}{0.821548in}}%
\pgfpathcurveto{\pgfqpoint{0.264911in}{0.824674in}}{\pgfqpoint{0.266667in}{0.828913in}}{\pgfqpoint{0.266667in}{0.833333in}}%
\pgfpathcurveto{\pgfqpoint{0.266667in}{0.837753in}}{\pgfqpoint{0.264911in}{0.841993in}}{\pgfqpoint{0.261785in}{0.845118in}}%
\pgfpathcurveto{\pgfqpoint{0.258660in}{0.848244in}}{\pgfqpoint{0.254420in}{0.850000in}}{\pgfqpoint{0.250000in}{0.850000in}}%
\pgfpathcurveto{\pgfqpoint{0.245580in}{0.850000in}}{\pgfqpoint{0.241340in}{0.848244in}}{\pgfqpoint{0.238215in}{0.845118in}}%
\pgfpathcurveto{\pgfqpoint{0.235089in}{0.841993in}}{\pgfqpoint{0.233333in}{0.837753in}}{\pgfqpoint{0.233333in}{0.833333in}}%
\pgfpathcurveto{\pgfqpoint{0.233333in}{0.828913in}}{\pgfqpoint{0.235089in}{0.824674in}}{\pgfqpoint{0.238215in}{0.821548in}}%
\pgfpathcurveto{\pgfqpoint{0.241340in}{0.818423in}}{\pgfqpoint{0.245580in}{0.816667in}}{\pgfqpoint{0.250000in}{0.816667in}}%
\pgfpathclose%
\pgfpathmoveto{\pgfqpoint{0.416667in}{0.816667in}}%
\pgfpathcurveto{\pgfqpoint{0.421087in}{0.816667in}}{\pgfqpoint{0.425326in}{0.818423in}}{\pgfqpoint{0.428452in}{0.821548in}}%
\pgfpathcurveto{\pgfqpoint{0.431577in}{0.824674in}}{\pgfqpoint{0.433333in}{0.828913in}}{\pgfqpoint{0.433333in}{0.833333in}}%
\pgfpathcurveto{\pgfqpoint{0.433333in}{0.837753in}}{\pgfqpoint{0.431577in}{0.841993in}}{\pgfqpoint{0.428452in}{0.845118in}}%
\pgfpathcurveto{\pgfqpoint{0.425326in}{0.848244in}}{\pgfqpoint{0.421087in}{0.850000in}}{\pgfqpoint{0.416667in}{0.850000in}}%
\pgfpathcurveto{\pgfqpoint{0.412247in}{0.850000in}}{\pgfqpoint{0.408007in}{0.848244in}}{\pgfqpoint{0.404882in}{0.845118in}}%
\pgfpathcurveto{\pgfqpoint{0.401756in}{0.841993in}}{\pgfqpoint{0.400000in}{0.837753in}}{\pgfqpoint{0.400000in}{0.833333in}}%
\pgfpathcurveto{\pgfqpoint{0.400000in}{0.828913in}}{\pgfqpoint{0.401756in}{0.824674in}}{\pgfqpoint{0.404882in}{0.821548in}}%
\pgfpathcurveto{\pgfqpoint{0.408007in}{0.818423in}}{\pgfqpoint{0.412247in}{0.816667in}}{\pgfqpoint{0.416667in}{0.816667in}}%
\pgfpathclose%
\pgfpathmoveto{\pgfqpoint{0.583333in}{0.816667in}}%
\pgfpathcurveto{\pgfqpoint{0.587753in}{0.816667in}}{\pgfqpoint{0.591993in}{0.818423in}}{\pgfqpoint{0.595118in}{0.821548in}}%
\pgfpathcurveto{\pgfqpoint{0.598244in}{0.824674in}}{\pgfqpoint{0.600000in}{0.828913in}}{\pgfqpoint{0.600000in}{0.833333in}}%
\pgfpathcurveto{\pgfqpoint{0.600000in}{0.837753in}}{\pgfqpoint{0.598244in}{0.841993in}}{\pgfqpoint{0.595118in}{0.845118in}}%
\pgfpathcurveto{\pgfqpoint{0.591993in}{0.848244in}}{\pgfqpoint{0.587753in}{0.850000in}}{\pgfqpoint{0.583333in}{0.850000in}}%
\pgfpathcurveto{\pgfqpoint{0.578913in}{0.850000in}}{\pgfqpoint{0.574674in}{0.848244in}}{\pgfqpoint{0.571548in}{0.845118in}}%
\pgfpathcurveto{\pgfqpoint{0.568423in}{0.841993in}}{\pgfqpoint{0.566667in}{0.837753in}}{\pgfqpoint{0.566667in}{0.833333in}}%
\pgfpathcurveto{\pgfqpoint{0.566667in}{0.828913in}}{\pgfqpoint{0.568423in}{0.824674in}}{\pgfqpoint{0.571548in}{0.821548in}}%
\pgfpathcurveto{\pgfqpoint{0.574674in}{0.818423in}}{\pgfqpoint{0.578913in}{0.816667in}}{\pgfqpoint{0.583333in}{0.816667in}}%
\pgfpathclose%
\pgfpathmoveto{\pgfqpoint{0.750000in}{0.816667in}}%
\pgfpathcurveto{\pgfqpoint{0.754420in}{0.816667in}}{\pgfqpoint{0.758660in}{0.818423in}}{\pgfqpoint{0.761785in}{0.821548in}}%
\pgfpathcurveto{\pgfqpoint{0.764911in}{0.824674in}}{\pgfqpoint{0.766667in}{0.828913in}}{\pgfqpoint{0.766667in}{0.833333in}}%
\pgfpathcurveto{\pgfqpoint{0.766667in}{0.837753in}}{\pgfqpoint{0.764911in}{0.841993in}}{\pgfqpoint{0.761785in}{0.845118in}}%
\pgfpathcurveto{\pgfqpoint{0.758660in}{0.848244in}}{\pgfqpoint{0.754420in}{0.850000in}}{\pgfqpoint{0.750000in}{0.850000in}}%
\pgfpathcurveto{\pgfqpoint{0.745580in}{0.850000in}}{\pgfqpoint{0.741340in}{0.848244in}}{\pgfqpoint{0.738215in}{0.845118in}}%
\pgfpathcurveto{\pgfqpoint{0.735089in}{0.841993in}}{\pgfqpoint{0.733333in}{0.837753in}}{\pgfqpoint{0.733333in}{0.833333in}}%
\pgfpathcurveto{\pgfqpoint{0.733333in}{0.828913in}}{\pgfqpoint{0.735089in}{0.824674in}}{\pgfqpoint{0.738215in}{0.821548in}}%
\pgfpathcurveto{\pgfqpoint{0.741340in}{0.818423in}}{\pgfqpoint{0.745580in}{0.816667in}}{\pgfqpoint{0.750000in}{0.816667in}}%
\pgfpathclose%
\pgfpathmoveto{\pgfqpoint{0.916667in}{0.816667in}}%
\pgfpathcurveto{\pgfqpoint{0.921087in}{0.816667in}}{\pgfqpoint{0.925326in}{0.818423in}}{\pgfqpoint{0.928452in}{0.821548in}}%
\pgfpathcurveto{\pgfqpoint{0.931577in}{0.824674in}}{\pgfqpoint{0.933333in}{0.828913in}}{\pgfqpoint{0.933333in}{0.833333in}}%
\pgfpathcurveto{\pgfqpoint{0.933333in}{0.837753in}}{\pgfqpoint{0.931577in}{0.841993in}}{\pgfqpoint{0.928452in}{0.845118in}}%
\pgfpathcurveto{\pgfqpoint{0.925326in}{0.848244in}}{\pgfqpoint{0.921087in}{0.850000in}}{\pgfqpoint{0.916667in}{0.850000in}}%
\pgfpathcurveto{\pgfqpoint{0.912247in}{0.850000in}}{\pgfqpoint{0.908007in}{0.848244in}}{\pgfqpoint{0.904882in}{0.845118in}}%
\pgfpathcurveto{\pgfqpoint{0.901756in}{0.841993in}}{\pgfqpoint{0.900000in}{0.837753in}}{\pgfqpoint{0.900000in}{0.833333in}}%
\pgfpathcurveto{\pgfqpoint{0.900000in}{0.828913in}}{\pgfqpoint{0.901756in}{0.824674in}}{\pgfqpoint{0.904882in}{0.821548in}}%
\pgfpathcurveto{\pgfqpoint{0.908007in}{0.818423in}}{\pgfqpoint{0.912247in}{0.816667in}}{\pgfqpoint{0.916667in}{0.816667in}}%
\pgfpathclose%
\pgfpathmoveto{\pgfqpoint{0.000000in}{0.983333in}}%
\pgfpathcurveto{\pgfqpoint{0.004420in}{0.983333in}}{\pgfqpoint{0.008660in}{0.985089in}}{\pgfqpoint{0.011785in}{0.988215in}}%
\pgfpathcurveto{\pgfqpoint{0.014911in}{0.991340in}}{\pgfqpoint{0.016667in}{0.995580in}}{\pgfqpoint{0.016667in}{1.000000in}}%
\pgfpathcurveto{\pgfqpoint{0.016667in}{1.004420in}}{\pgfqpoint{0.014911in}{1.008660in}}{\pgfqpoint{0.011785in}{1.011785in}}%
\pgfpathcurveto{\pgfqpoint{0.008660in}{1.014911in}}{\pgfqpoint{0.004420in}{1.016667in}}{\pgfqpoint{0.000000in}{1.016667in}}%
\pgfpathcurveto{\pgfqpoint{-0.004420in}{1.016667in}}{\pgfqpoint{-0.008660in}{1.014911in}}{\pgfqpoint{-0.011785in}{1.011785in}}%
\pgfpathcurveto{\pgfqpoint{-0.014911in}{1.008660in}}{\pgfqpoint{-0.016667in}{1.004420in}}{\pgfqpoint{-0.016667in}{1.000000in}}%
\pgfpathcurveto{\pgfqpoint{-0.016667in}{0.995580in}}{\pgfqpoint{-0.014911in}{0.991340in}}{\pgfqpoint{-0.011785in}{0.988215in}}%
\pgfpathcurveto{\pgfqpoint{-0.008660in}{0.985089in}}{\pgfqpoint{-0.004420in}{0.983333in}}{\pgfqpoint{0.000000in}{0.983333in}}%
\pgfpathclose%
\pgfpathmoveto{\pgfqpoint{0.166667in}{0.983333in}}%
\pgfpathcurveto{\pgfqpoint{0.171087in}{0.983333in}}{\pgfqpoint{0.175326in}{0.985089in}}{\pgfqpoint{0.178452in}{0.988215in}}%
\pgfpathcurveto{\pgfqpoint{0.181577in}{0.991340in}}{\pgfqpoint{0.183333in}{0.995580in}}{\pgfqpoint{0.183333in}{1.000000in}}%
\pgfpathcurveto{\pgfqpoint{0.183333in}{1.004420in}}{\pgfqpoint{0.181577in}{1.008660in}}{\pgfqpoint{0.178452in}{1.011785in}}%
\pgfpathcurveto{\pgfqpoint{0.175326in}{1.014911in}}{\pgfqpoint{0.171087in}{1.016667in}}{\pgfqpoint{0.166667in}{1.016667in}}%
\pgfpathcurveto{\pgfqpoint{0.162247in}{1.016667in}}{\pgfqpoint{0.158007in}{1.014911in}}{\pgfqpoint{0.154882in}{1.011785in}}%
\pgfpathcurveto{\pgfqpoint{0.151756in}{1.008660in}}{\pgfqpoint{0.150000in}{1.004420in}}{\pgfqpoint{0.150000in}{1.000000in}}%
\pgfpathcurveto{\pgfqpoint{0.150000in}{0.995580in}}{\pgfqpoint{0.151756in}{0.991340in}}{\pgfqpoint{0.154882in}{0.988215in}}%
\pgfpathcurveto{\pgfqpoint{0.158007in}{0.985089in}}{\pgfqpoint{0.162247in}{0.983333in}}{\pgfqpoint{0.166667in}{0.983333in}}%
\pgfpathclose%
\pgfpathmoveto{\pgfqpoint{0.333333in}{0.983333in}}%
\pgfpathcurveto{\pgfqpoint{0.337753in}{0.983333in}}{\pgfqpoint{0.341993in}{0.985089in}}{\pgfqpoint{0.345118in}{0.988215in}}%
\pgfpathcurveto{\pgfqpoint{0.348244in}{0.991340in}}{\pgfqpoint{0.350000in}{0.995580in}}{\pgfqpoint{0.350000in}{1.000000in}}%
\pgfpathcurveto{\pgfqpoint{0.350000in}{1.004420in}}{\pgfqpoint{0.348244in}{1.008660in}}{\pgfqpoint{0.345118in}{1.011785in}}%
\pgfpathcurveto{\pgfqpoint{0.341993in}{1.014911in}}{\pgfqpoint{0.337753in}{1.016667in}}{\pgfqpoint{0.333333in}{1.016667in}}%
\pgfpathcurveto{\pgfqpoint{0.328913in}{1.016667in}}{\pgfqpoint{0.324674in}{1.014911in}}{\pgfqpoint{0.321548in}{1.011785in}}%
\pgfpathcurveto{\pgfqpoint{0.318423in}{1.008660in}}{\pgfqpoint{0.316667in}{1.004420in}}{\pgfqpoint{0.316667in}{1.000000in}}%
\pgfpathcurveto{\pgfqpoint{0.316667in}{0.995580in}}{\pgfqpoint{0.318423in}{0.991340in}}{\pgfqpoint{0.321548in}{0.988215in}}%
\pgfpathcurveto{\pgfqpoint{0.324674in}{0.985089in}}{\pgfqpoint{0.328913in}{0.983333in}}{\pgfqpoint{0.333333in}{0.983333in}}%
\pgfpathclose%
\pgfpathmoveto{\pgfqpoint{0.500000in}{0.983333in}}%
\pgfpathcurveto{\pgfqpoint{0.504420in}{0.983333in}}{\pgfqpoint{0.508660in}{0.985089in}}{\pgfqpoint{0.511785in}{0.988215in}}%
\pgfpathcurveto{\pgfqpoint{0.514911in}{0.991340in}}{\pgfqpoint{0.516667in}{0.995580in}}{\pgfqpoint{0.516667in}{1.000000in}}%
\pgfpathcurveto{\pgfqpoint{0.516667in}{1.004420in}}{\pgfqpoint{0.514911in}{1.008660in}}{\pgfqpoint{0.511785in}{1.011785in}}%
\pgfpathcurveto{\pgfqpoint{0.508660in}{1.014911in}}{\pgfqpoint{0.504420in}{1.016667in}}{\pgfqpoint{0.500000in}{1.016667in}}%
\pgfpathcurveto{\pgfqpoint{0.495580in}{1.016667in}}{\pgfqpoint{0.491340in}{1.014911in}}{\pgfqpoint{0.488215in}{1.011785in}}%
\pgfpathcurveto{\pgfqpoint{0.485089in}{1.008660in}}{\pgfqpoint{0.483333in}{1.004420in}}{\pgfqpoint{0.483333in}{1.000000in}}%
\pgfpathcurveto{\pgfqpoint{0.483333in}{0.995580in}}{\pgfqpoint{0.485089in}{0.991340in}}{\pgfqpoint{0.488215in}{0.988215in}}%
\pgfpathcurveto{\pgfqpoint{0.491340in}{0.985089in}}{\pgfqpoint{0.495580in}{0.983333in}}{\pgfqpoint{0.500000in}{0.983333in}}%
\pgfpathclose%
\pgfpathmoveto{\pgfqpoint{0.666667in}{0.983333in}}%
\pgfpathcurveto{\pgfqpoint{0.671087in}{0.983333in}}{\pgfqpoint{0.675326in}{0.985089in}}{\pgfqpoint{0.678452in}{0.988215in}}%
\pgfpathcurveto{\pgfqpoint{0.681577in}{0.991340in}}{\pgfqpoint{0.683333in}{0.995580in}}{\pgfqpoint{0.683333in}{1.000000in}}%
\pgfpathcurveto{\pgfqpoint{0.683333in}{1.004420in}}{\pgfqpoint{0.681577in}{1.008660in}}{\pgfqpoint{0.678452in}{1.011785in}}%
\pgfpathcurveto{\pgfqpoint{0.675326in}{1.014911in}}{\pgfqpoint{0.671087in}{1.016667in}}{\pgfqpoint{0.666667in}{1.016667in}}%
\pgfpathcurveto{\pgfqpoint{0.662247in}{1.016667in}}{\pgfqpoint{0.658007in}{1.014911in}}{\pgfqpoint{0.654882in}{1.011785in}}%
\pgfpathcurveto{\pgfqpoint{0.651756in}{1.008660in}}{\pgfqpoint{0.650000in}{1.004420in}}{\pgfqpoint{0.650000in}{1.000000in}}%
\pgfpathcurveto{\pgfqpoint{0.650000in}{0.995580in}}{\pgfqpoint{0.651756in}{0.991340in}}{\pgfqpoint{0.654882in}{0.988215in}}%
\pgfpathcurveto{\pgfqpoint{0.658007in}{0.985089in}}{\pgfqpoint{0.662247in}{0.983333in}}{\pgfqpoint{0.666667in}{0.983333in}}%
\pgfpathclose%
\pgfpathmoveto{\pgfqpoint{0.833333in}{0.983333in}}%
\pgfpathcurveto{\pgfqpoint{0.837753in}{0.983333in}}{\pgfqpoint{0.841993in}{0.985089in}}{\pgfqpoint{0.845118in}{0.988215in}}%
\pgfpathcurveto{\pgfqpoint{0.848244in}{0.991340in}}{\pgfqpoint{0.850000in}{0.995580in}}{\pgfqpoint{0.850000in}{1.000000in}}%
\pgfpathcurveto{\pgfqpoint{0.850000in}{1.004420in}}{\pgfqpoint{0.848244in}{1.008660in}}{\pgfqpoint{0.845118in}{1.011785in}}%
\pgfpathcurveto{\pgfqpoint{0.841993in}{1.014911in}}{\pgfqpoint{0.837753in}{1.016667in}}{\pgfqpoint{0.833333in}{1.016667in}}%
\pgfpathcurveto{\pgfqpoint{0.828913in}{1.016667in}}{\pgfqpoint{0.824674in}{1.014911in}}{\pgfqpoint{0.821548in}{1.011785in}}%
\pgfpathcurveto{\pgfqpoint{0.818423in}{1.008660in}}{\pgfqpoint{0.816667in}{1.004420in}}{\pgfqpoint{0.816667in}{1.000000in}}%
\pgfpathcurveto{\pgfqpoint{0.816667in}{0.995580in}}{\pgfqpoint{0.818423in}{0.991340in}}{\pgfqpoint{0.821548in}{0.988215in}}%
\pgfpathcurveto{\pgfqpoint{0.824674in}{0.985089in}}{\pgfqpoint{0.828913in}{0.983333in}}{\pgfqpoint{0.833333in}{0.983333in}}%
\pgfpathclose%
\pgfpathmoveto{\pgfqpoint{1.000000in}{0.983333in}}%
\pgfpathcurveto{\pgfqpoint{1.004420in}{0.983333in}}{\pgfqpoint{1.008660in}{0.985089in}}{\pgfqpoint{1.011785in}{0.988215in}}%
\pgfpathcurveto{\pgfqpoint{1.014911in}{0.991340in}}{\pgfqpoint{1.016667in}{0.995580in}}{\pgfqpoint{1.016667in}{1.000000in}}%
\pgfpathcurveto{\pgfqpoint{1.016667in}{1.004420in}}{\pgfqpoint{1.014911in}{1.008660in}}{\pgfqpoint{1.011785in}{1.011785in}}%
\pgfpathcurveto{\pgfqpoint{1.008660in}{1.014911in}}{\pgfqpoint{1.004420in}{1.016667in}}{\pgfqpoint{1.000000in}{1.016667in}}%
\pgfpathcurveto{\pgfqpoint{0.995580in}{1.016667in}}{\pgfqpoint{0.991340in}{1.014911in}}{\pgfqpoint{0.988215in}{1.011785in}}%
\pgfpathcurveto{\pgfqpoint{0.985089in}{1.008660in}}{\pgfqpoint{0.983333in}{1.004420in}}{\pgfqpoint{0.983333in}{1.000000in}}%
\pgfpathcurveto{\pgfqpoint{0.983333in}{0.995580in}}{\pgfqpoint{0.985089in}{0.991340in}}{\pgfqpoint{0.988215in}{0.988215in}}%
\pgfpathcurveto{\pgfqpoint{0.991340in}{0.985089in}}{\pgfqpoint{0.995580in}{0.983333in}}{\pgfqpoint{1.000000in}{0.983333in}}%
\pgfpathclose%
\pgfusepath{stroke}%
\end{pgfscope}%
}%
\pgfsys@transformshift{1.258038in}{0.637495in}%
\end{pgfscope}%
\begin{pgfscope}%
\pgfpathrectangle{\pgfqpoint{0.870538in}{0.637495in}}{\pgfqpoint{9.300000in}{9.060000in}}%
\pgfusepath{clip}%
\pgfsetbuttcap%
\pgfsetmiterjoin%
\definecolor{currentfill}{rgb}{0.172549,0.627451,0.172549}%
\pgfsetfillcolor{currentfill}%
\pgfsetfillopacity{0.990000}%
\pgfsetlinewidth{0.000000pt}%
\definecolor{currentstroke}{rgb}{0.000000,0.000000,0.000000}%
\pgfsetstrokecolor{currentstroke}%
\pgfsetstrokeopacity{0.990000}%
\pgfsetdash{}{0pt}%
\pgfpathmoveto{\pgfqpoint{2.808038in}{0.637495in}}%
\pgfpathlineto{\pgfqpoint{3.583038in}{0.637495in}}%
\pgfpathlineto{\pgfqpoint{3.583038in}{0.637495in}}%
\pgfpathlineto{\pgfqpoint{2.808038in}{0.637495in}}%
\pgfpathclose%
\pgfusepath{fill}%
\end{pgfscope}%
\begin{pgfscope}%
\pgfsetbuttcap%
\pgfsetmiterjoin%
\definecolor{currentfill}{rgb}{0.172549,0.627451,0.172549}%
\pgfsetfillcolor{currentfill}%
\pgfsetfillopacity{0.990000}%
\pgfsetlinewidth{0.000000pt}%
\definecolor{currentstroke}{rgb}{0.000000,0.000000,0.000000}%
\pgfsetstrokecolor{currentstroke}%
\pgfsetstrokeopacity{0.990000}%
\pgfsetdash{}{0pt}%
\pgfpathrectangle{\pgfqpoint{0.870538in}{0.637495in}}{\pgfqpoint{9.300000in}{9.060000in}}%
\pgfusepath{clip}%
\pgfpathmoveto{\pgfqpoint{2.808038in}{0.637495in}}%
\pgfpathlineto{\pgfqpoint{3.583038in}{0.637495in}}%
\pgfpathlineto{\pgfqpoint{3.583038in}{0.637495in}}%
\pgfpathlineto{\pgfqpoint{2.808038in}{0.637495in}}%
\pgfpathclose%
\pgfusepath{clip}%
\pgfsys@defobject{currentpattern}{\pgfqpoint{0in}{0in}}{\pgfqpoint{1in}{1in}}{%
\begin{pgfscope}%
\pgfpathrectangle{\pgfqpoint{0in}{0in}}{\pgfqpoint{1in}{1in}}%
\pgfusepath{clip}%
\pgfpathmoveto{\pgfqpoint{0.000000in}{-0.016667in}}%
\pgfpathcurveto{\pgfqpoint{0.004420in}{-0.016667in}}{\pgfqpoint{0.008660in}{-0.014911in}}{\pgfqpoint{0.011785in}{-0.011785in}}%
\pgfpathcurveto{\pgfqpoint{0.014911in}{-0.008660in}}{\pgfqpoint{0.016667in}{-0.004420in}}{\pgfqpoint{0.016667in}{0.000000in}}%
\pgfpathcurveto{\pgfqpoint{0.016667in}{0.004420in}}{\pgfqpoint{0.014911in}{0.008660in}}{\pgfqpoint{0.011785in}{0.011785in}}%
\pgfpathcurveto{\pgfqpoint{0.008660in}{0.014911in}}{\pgfqpoint{0.004420in}{0.016667in}}{\pgfqpoint{0.000000in}{0.016667in}}%
\pgfpathcurveto{\pgfqpoint{-0.004420in}{0.016667in}}{\pgfqpoint{-0.008660in}{0.014911in}}{\pgfqpoint{-0.011785in}{0.011785in}}%
\pgfpathcurveto{\pgfqpoint{-0.014911in}{0.008660in}}{\pgfqpoint{-0.016667in}{0.004420in}}{\pgfqpoint{-0.016667in}{0.000000in}}%
\pgfpathcurveto{\pgfqpoint{-0.016667in}{-0.004420in}}{\pgfqpoint{-0.014911in}{-0.008660in}}{\pgfqpoint{-0.011785in}{-0.011785in}}%
\pgfpathcurveto{\pgfqpoint{-0.008660in}{-0.014911in}}{\pgfqpoint{-0.004420in}{-0.016667in}}{\pgfqpoint{0.000000in}{-0.016667in}}%
\pgfpathclose%
\pgfpathmoveto{\pgfqpoint{0.166667in}{-0.016667in}}%
\pgfpathcurveto{\pgfqpoint{0.171087in}{-0.016667in}}{\pgfqpoint{0.175326in}{-0.014911in}}{\pgfqpoint{0.178452in}{-0.011785in}}%
\pgfpathcurveto{\pgfqpoint{0.181577in}{-0.008660in}}{\pgfqpoint{0.183333in}{-0.004420in}}{\pgfqpoint{0.183333in}{0.000000in}}%
\pgfpathcurveto{\pgfqpoint{0.183333in}{0.004420in}}{\pgfqpoint{0.181577in}{0.008660in}}{\pgfqpoint{0.178452in}{0.011785in}}%
\pgfpathcurveto{\pgfqpoint{0.175326in}{0.014911in}}{\pgfqpoint{0.171087in}{0.016667in}}{\pgfqpoint{0.166667in}{0.016667in}}%
\pgfpathcurveto{\pgfqpoint{0.162247in}{0.016667in}}{\pgfqpoint{0.158007in}{0.014911in}}{\pgfqpoint{0.154882in}{0.011785in}}%
\pgfpathcurveto{\pgfqpoint{0.151756in}{0.008660in}}{\pgfqpoint{0.150000in}{0.004420in}}{\pgfqpoint{0.150000in}{0.000000in}}%
\pgfpathcurveto{\pgfqpoint{0.150000in}{-0.004420in}}{\pgfqpoint{0.151756in}{-0.008660in}}{\pgfqpoint{0.154882in}{-0.011785in}}%
\pgfpathcurveto{\pgfqpoint{0.158007in}{-0.014911in}}{\pgfqpoint{0.162247in}{-0.016667in}}{\pgfqpoint{0.166667in}{-0.016667in}}%
\pgfpathclose%
\pgfpathmoveto{\pgfqpoint{0.333333in}{-0.016667in}}%
\pgfpathcurveto{\pgfqpoint{0.337753in}{-0.016667in}}{\pgfqpoint{0.341993in}{-0.014911in}}{\pgfqpoint{0.345118in}{-0.011785in}}%
\pgfpathcurveto{\pgfqpoint{0.348244in}{-0.008660in}}{\pgfqpoint{0.350000in}{-0.004420in}}{\pgfqpoint{0.350000in}{0.000000in}}%
\pgfpathcurveto{\pgfqpoint{0.350000in}{0.004420in}}{\pgfqpoint{0.348244in}{0.008660in}}{\pgfqpoint{0.345118in}{0.011785in}}%
\pgfpathcurveto{\pgfqpoint{0.341993in}{0.014911in}}{\pgfqpoint{0.337753in}{0.016667in}}{\pgfqpoint{0.333333in}{0.016667in}}%
\pgfpathcurveto{\pgfqpoint{0.328913in}{0.016667in}}{\pgfqpoint{0.324674in}{0.014911in}}{\pgfqpoint{0.321548in}{0.011785in}}%
\pgfpathcurveto{\pgfqpoint{0.318423in}{0.008660in}}{\pgfqpoint{0.316667in}{0.004420in}}{\pgfqpoint{0.316667in}{0.000000in}}%
\pgfpathcurveto{\pgfqpoint{0.316667in}{-0.004420in}}{\pgfqpoint{0.318423in}{-0.008660in}}{\pgfqpoint{0.321548in}{-0.011785in}}%
\pgfpathcurveto{\pgfqpoint{0.324674in}{-0.014911in}}{\pgfqpoint{0.328913in}{-0.016667in}}{\pgfqpoint{0.333333in}{-0.016667in}}%
\pgfpathclose%
\pgfpathmoveto{\pgfqpoint{0.500000in}{-0.016667in}}%
\pgfpathcurveto{\pgfqpoint{0.504420in}{-0.016667in}}{\pgfqpoint{0.508660in}{-0.014911in}}{\pgfqpoint{0.511785in}{-0.011785in}}%
\pgfpathcurveto{\pgfqpoint{0.514911in}{-0.008660in}}{\pgfqpoint{0.516667in}{-0.004420in}}{\pgfqpoint{0.516667in}{0.000000in}}%
\pgfpathcurveto{\pgfqpoint{0.516667in}{0.004420in}}{\pgfqpoint{0.514911in}{0.008660in}}{\pgfqpoint{0.511785in}{0.011785in}}%
\pgfpathcurveto{\pgfqpoint{0.508660in}{0.014911in}}{\pgfqpoint{0.504420in}{0.016667in}}{\pgfqpoint{0.500000in}{0.016667in}}%
\pgfpathcurveto{\pgfqpoint{0.495580in}{0.016667in}}{\pgfqpoint{0.491340in}{0.014911in}}{\pgfqpoint{0.488215in}{0.011785in}}%
\pgfpathcurveto{\pgfqpoint{0.485089in}{0.008660in}}{\pgfqpoint{0.483333in}{0.004420in}}{\pgfqpoint{0.483333in}{0.000000in}}%
\pgfpathcurveto{\pgfqpoint{0.483333in}{-0.004420in}}{\pgfqpoint{0.485089in}{-0.008660in}}{\pgfqpoint{0.488215in}{-0.011785in}}%
\pgfpathcurveto{\pgfqpoint{0.491340in}{-0.014911in}}{\pgfqpoint{0.495580in}{-0.016667in}}{\pgfqpoint{0.500000in}{-0.016667in}}%
\pgfpathclose%
\pgfpathmoveto{\pgfqpoint{0.666667in}{-0.016667in}}%
\pgfpathcurveto{\pgfqpoint{0.671087in}{-0.016667in}}{\pgfqpoint{0.675326in}{-0.014911in}}{\pgfqpoint{0.678452in}{-0.011785in}}%
\pgfpathcurveto{\pgfqpoint{0.681577in}{-0.008660in}}{\pgfqpoint{0.683333in}{-0.004420in}}{\pgfqpoint{0.683333in}{0.000000in}}%
\pgfpathcurveto{\pgfqpoint{0.683333in}{0.004420in}}{\pgfqpoint{0.681577in}{0.008660in}}{\pgfqpoint{0.678452in}{0.011785in}}%
\pgfpathcurveto{\pgfqpoint{0.675326in}{0.014911in}}{\pgfqpoint{0.671087in}{0.016667in}}{\pgfqpoint{0.666667in}{0.016667in}}%
\pgfpathcurveto{\pgfqpoint{0.662247in}{0.016667in}}{\pgfqpoint{0.658007in}{0.014911in}}{\pgfqpoint{0.654882in}{0.011785in}}%
\pgfpathcurveto{\pgfqpoint{0.651756in}{0.008660in}}{\pgfqpoint{0.650000in}{0.004420in}}{\pgfqpoint{0.650000in}{0.000000in}}%
\pgfpathcurveto{\pgfqpoint{0.650000in}{-0.004420in}}{\pgfqpoint{0.651756in}{-0.008660in}}{\pgfqpoint{0.654882in}{-0.011785in}}%
\pgfpathcurveto{\pgfqpoint{0.658007in}{-0.014911in}}{\pgfqpoint{0.662247in}{-0.016667in}}{\pgfqpoint{0.666667in}{-0.016667in}}%
\pgfpathclose%
\pgfpathmoveto{\pgfqpoint{0.833333in}{-0.016667in}}%
\pgfpathcurveto{\pgfqpoint{0.837753in}{-0.016667in}}{\pgfqpoint{0.841993in}{-0.014911in}}{\pgfqpoint{0.845118in}{-0.011785in}}%
\pgfpathcurveto{\pgfqpoint{0.848244in}{-0.008660in}}{\pgfqpoint{0.850000in}{-0.004420in}}{\pgfqpoint{0.850000in}{0.000000in}}%
\pgfpathcurveto{\pgfqpoint{0.850000in}{0.004420in}}{\pgfqpoint{0.848244in}{0.008660in}}{\pgfqpoint{0.845118in}{0.011785in}}%
\pgfpathcurveto{\pgfqpoint{0.841993in}{0.014911in}}{\pgfqpoint{0.837753in}{0.016667in}}{\pgfqpoint{0.833333in}{0.016667in}}%
\pgfpathcurveto{\pgfqpoint{0.828913in}{0.016667in}}{\pgfqpoint{0.824674in}{0.014911in}}{\pgfqpoint{0.821548in}{0.011785in}}%
\pgfpathcurveto{\pgfqpoint{0.818423in}{0.008660in}}{\pgfqpoint{0.816667in}{0.004420in}}{\pgfqpoint{0.816667in}{0.000000in}}%
\pgfpathcurveto{\pgfqpoint{0.816667in}{-0.004420in}}{\pgfqpoint{0.818423in}{-0.008660in}}{\pgfqpoint{0.821548in}{-0.011785in}}%
\pgfpathcurveto{\pgfqpoint{0.824674in}{-0.014911in}}{\pgfqpoint{0.828913in}{-0.016667in}}{\pgfqpoint{0.833333in}{-0.016667in}}%
\pgfpathclose%
\pgfpathmoveto{\pgfqpoint{1.000000in}{-0.016667in}}%
\pgfpathcurveto{\pgfqpoint{1.004420in}{-0.016667in}}{\pgfqpoint{1.008660in}{-0.014911in}}{\pgfqpoint{1.011785in}{-0.011785in}}%
\pgfpathcurveto{\pgfqpoint{1.014911in}{-0.008660in}}{\pgfqpoint{1.016667in}{-0.004420in}}{\pgfqpoint{1.016667in}{0.000000in}}%
\pgfpathcurveto{\pgfqpoint{1.016667in}{0.004420in}}{\pgfqpoint{1.014911in}{0.008660in}}{\pgfqpoint{1.011785in}{0.011785in}}%
\pgfpathcurveto{\pgfqpoint{1.008660in}{0.014911in}}{\pgfqpoint{1.004420in}{0.016667in}}{\pgfqpoint{1.000000in}{0.016667in}}%
\pgfpathcurveto{\pgfqpoint{0.995580in}{0.016667in}}{\pgfqpoint{0.991340in}{0.014911in}}{\pgfqpoint{0.988215in}{0.011785in}}%
\pgfpathcurveto{\pgfqpoint{0.985089in}{0.008660in}}{\pgfqpoint{0.983333in}{0.004420in}}{\pgfqpoint{0.983333in}{0.000000in}}%
\pgfpathcurveto{\pgfqpoint{0.983333in}{-0.004420in}}{\pgfqpoint{0.985089in}{-0.008660in}}{\pgfqpoint{0.988215in}{-0.011785in}}%
\pgfpathcurveto{\pgfqpoint{0.991340in}{-0.014911in}}{\pgfqpoint{0.995580in}{-0.016667in}}{\pgfqpoint{1.000000in}{-0.016667in}}%
\pgfpathclose%
\pgfpathmoveto{\pgfqpoint{0.083333in}{0.150000in}}%
\pgfpathcurveto{\pgfqpoint{0.087753in}{0.150000in}}{\pgfqpoint{0.091993in}{0.151756in}}{\pgfqpoint{0.095118in}{0.154882in}}%
\pgfpathcurveto{\pgfqpoint{0.098244in}{0.158007in}}{\pgfqpoint{0.100000in}{0.162247in}}{\pgfqpoint{0.100000in}{0.166667in}}%
\pgfpathcurveto{\pgfqpoint{0.100000in}{0.171087in}}{\pgfqpoint{0.098244in}{0.175326in}}{\pgfqpoint{0.095118in}{0.178452in}}%
\pgfpathcurveto{\pgfqpoint{0.091993in}{0.181577in}}{\pgfqpoint{0.087753in}{0.183333in}}{\pgfqpoint{0.083333in}{0.183333in}}%
\pgfpathcurveto{\pgfqpoint{0.078913in}{0.183333in}}{\pgfqpoint{0.074674in}{0.181577in}}{\pgfqpoint{0.071548in}{0.178452in}}%
\pgfpathcurveto{\pgfqpoint{0.068423in}{0.175326in}}{\pgfqpoint{0.066667in}{0.171087in}}{\pgfqpoint{0.066667in}{0.166667in}}%
\pgfpathcurveto{\pgfqpoint{0.066667in}{0.162247in}}{\pgfqpoint{0.068423in}{0.158007in}}{\pgfqpoint{0.071548in}{0.154882in}}%
\pgfpathcurveto{\pgfqpoint{0.074674in}{0.151756in}}{\pgfqpoint{0.078913in}{0.150000in}}{\pgfqpoint{0.083333in}{0.150000in}}%
\pgfpathclose%
\pgfpathmoveto{\pgfqpoint{0.250000in}{0.150000in}}%
\pgfpathcurveto{\pgfqpoint{0.254420in}{0.150000in}}{\pgfqpoint{0.258660in}{0.151756in}}{\pgfqpoint{0.261785in}{0.154882in}}%
\pgfpathcurveto{\pgfqpoint{0.264911in}{0.158007in}}{\pgfqpoint{0.266667in}{0.162247in}}{\pgfqpoint{0.266667in}{0.166667in}}%
\pgfpathcurveto{\pgfqpoint{0.266667in}{0.171087in}}{\pgfqpoint{0.264911in}{0.175326in}}{\pgfqpoint{0.261785in}{0.178452in}}%
\pgfpathcurveto{\pgfqpoint{0.258660in}{0.181577in}}{\pgfqpoint{0.254420in}{0.183333in}}{\pgfqpoint{0.250000in}{0.183333in}}%
\pgfpathcurveto{\pgfqpoint{0.245580in}{0.183333in}}{\pgfqpoint{0.241340in}{0.181577in}}{\pgfqpoint{0.238215in}{0.178452in}}%
\pgfpathcurveto{\pgfqpoint{0.235089in}{0.175326in}}{\pgfqpoint{0.233333in}{0.171087in}}{\pgfqpoint{0.233333in}{0.166667in}}%
\pgfpathcurveto{\pgfqpoint{0.233333in}{0.162247in}}{\pgfqpoint{0.235089in}{0.158007in}}{\pgfqpoint{0.238215in}{0.154882in}}%
\pgfpathcurveto{\pgfqpoint{0.241340in}{0.151756in}}{\pgfqpoint{0.245580in}{0.150000in}}{\pgfqpoint{0.250000in}{0.150000in}}%
\pgfpathclose%
\pgfpathmoveto{\pgfqpoint{0.416667in}{0.150000in}}%
\pgfpathcurveto{\pgfqpoint{0.421087in}{0.150000in}}{\pgfqpoint{0.425326in}{0.151756in}}{\pgfqpoint{0.428452in}{0.154882in}}%
\pgfpathcurveto{\pgfqpoint{0.431577in}{0.158007in}}{\pgfqpoint{0.433333in}{0.162247in}}{\pgfqpoint{0.433333in}{0.166667in}}%
\pgfpathcurveto{\pgfqpoint{0.433333in}{0.171087in}}{\pgfqpoint{0.431577in}{0.175326in}}{\pgfqpoint{0.428452in}{0.178452in}}%
\pgfpathcurveto{\pgfqpoint{0.425326in}{0.181577in}}{\pgfqpoint{0.421087in}{0.183333in}}{\pgfqpoint{0.416667in}{0.183333in}}%
\pgfpathcurveto{\pgfqpoint{0.412247in}{0.183333in}}{\pgfqpoint{0.408007in}{0.181577in}}{\pgfqpoint{0.404882in}{0.178452in}}%
\pgfpathcurveto{\pgfqpoint{0.401756in}{0.175326in}}{\pgfqpoint{0.400000in}{0.171087in}}{\pgfqpoint{0.400000in}{0.166667in}}%
\pgfpathcurveto{\pgfqpoint{0.400000in}{0.162247in}}{\pgfqpoint{0.401756in}{0.158007in}}{\pgfqpoint{0.404882in}{0.154882in}}%
\pgfpathcurveto{\pgfqpoint{0.408007in}{0.151756in}}{\pgfqpoint{0.412247in}{0.150000in}}{\pgfqpoint{0.416667in}{0.150000in}}%
\pgfpathclose%
\pgfpathmoveto{\pgfqpoint{0.583333in}{0.150000in}}%
\pgfpathcurveto{\pgfqpoint{0.587753in}{0.150000in}}{\pgfqpoint{0.591993in}{0.151756in}}{\pgfqpoint{0.595118in}{0.154882in}}%
\pgfpathcurveto{\pgfqpoint{0.598244in}{0.158007in}}{\pgfqpoint{0.600000in}{0.162247in}}{\pgfqpoint{0.600000in}{0.166667in}}%
\pgfpathcurveto{\pgfqpoint{0.600000in}{0.171087in}}{\pgfqpoint{0.598244in}{0.175326in}}{\pgfqpoint{0.595118in}{0.178452in}}%
\pgfpathcurveto{\pgfqpoint{0.591993in}{0.181577in}}{\pgfqpoint{0.587753in}{0.183333in}}{\pgfqpoint{0.583333in}{0.183333in}}%
\pgfpathcurveto{\pgfqpoint{0.578913in}{0.183333in}}{\pgfqpoint{0.574674in}{0.181577in}}{\pgfqpoint{0.571548in}{0.178452in}}%
\pgfpathcurveto{\pgfqpoint{0.568423in}{0.175326in}}{\pgfqpoint{0.566667in}{0.171087in}}{\pgfqpoint{0.566667in}{0.166667in}}%
\pgfpathcurveto{\pgfqpoint{0.566667in}{0.162247in}}{\pgfqpoint{0.568423in}{0.158007in}}{\pgfqpoint{0.571548in}{0.154882in}}%
\pgfpathcurveto{\pgfqpoint{0.574674in}{0.151756in}}{\pgfqpoint{0.578913in}{0.150000in}}{\pgfqpoint{0.583333in}{0.150000in}}%
\pgfpathclose%
\pgfpathmoveto{\pgfqpoint{0.750000in}{0.150000in}}%
\pgfpathcurveto{\pgfqpoint{0.754420in}{0.150000in}}{\pgfqpoint{0.758660in}{0.151756in}}{\pgfqpoint{0.761785in}{0.154882in}}%
\pgfpathcurveto{\pgfqpoint{0.764911in}{0.158007in}}{\pgfqpoint{0.766667in}{0.162247in}}{\pgfqpoint{0.766667in}{0.166667in}}%
\pgfpathcurveto{\pgfqpoint{0.766667in}{0.171087in}}{\pgfqpoint{0.764911in}{0.175326in}}{\pgfqpoint{0.761785in}{0.178452in}}%
\pgfpathcurveto{\pgfqpoint{0.758660in}{0.181577in}}{\pgfqpoint{0.754420in}{0.183333in}}{\pgfqpoint{0.750000in}{0.183333in}}%
\pgfpathcurveto{\pgfqpoint{0.745580in}{0.183333in}}{\pgfqpoint{0.741340in}{0.181577in}}{\pgfqpoint{0.738215in}{0.178452in}}%
\pgfpathcurveto{\pgfqpoint{0.735089in}{0.175326in}}{\pgfqpoint{0.733333in}{0.171087in}}{\pgfqpoint{0.733333in}{0.166667in}}%
\pgfpathcurveto{\pgfqpoint{0.733333in}{0.162247in}}{\pgfqpoint{0.735089in}{0.158007in}}{\pgfqpoint{0.738215in}{0.154882in}}%
\pgfpathcurveto{\pgfqpoint{0.741340in}{0.151756in}}{\pgfqpoint{0.745580in}{0.150000in}}{\pgfqpoint{0.750000in}{0.150000in}}%
\pgfpathclose%
\pgfpathmoveto{\pgfqpoint{0.916667in}{0.150000in}}%
\pgfpathcurveto{\pgfqpoint{0.921087in}{0.150000in}}{\pgfqpoint{0.925326in}{0.151756in}}{\pgfqpoint{0.928452in}{0.154882in}}%
\pgfpathcurveto{\pgfqpoint{0.931577in}{0.158007in}}{\pgfqpoint{0.933333in}{0.162247in}}{\pgfqpoint{0.933333in}{0.166667in}}%
\pgfpathcurveto{\pgfqpoint{0.933333in}{0.171087in}}{\pgfqpoint{0.931577in}{0.175326in}}{\pgfqpoint{0.928452in}{0.178452in}}%
\pgfpathcurveto{\pgfqpoint{0.925326in}{0.181577in}}{\pgfqpoint{0.921087in}{0.183333in}}{\pgfqpoint{0.916667in}{0.183333in}}%
\pgfpathcurveto{\pgfqpoint{0.912247in}{0.183333in}}{\pgfqpoint{0.908007in}{0.181577in}}{\pgfqpoint{0.904882in}{0.178452in}}%
\pgfpathcurveto{\pgfqpoint{0.901756in}{0.175326in}}{\pgfqpoint{0.900000in}{0.171087in}}{\pgfqpoint{0.900000in}{0.166667in}}%
\pgfpathcurveto{\pgfqpoint{0.900000in}{0.162247in}}{\pgfqpoint{0.901756in}{0.158007in}}{\pgfqpoint{0.904882in}{0.154882in}}%
\pgfpathcurveto{\pgfqpoint{0.908007in}{0.151756in}}{\pgfqpoint{0.912247in}{0.150000in}}{\pgfqpoint{0.916667in}{0.150000in}}%
\pgfpathclose%
\pgfpathmoveto{\pgfqpoint{0.000000in}{0.316667in}}%
\pgfpathcurveto{\pgfqpoint{0.004420in}{0.316667in}}{\pgfqpoint{0.008660in}{0.318423in}}{\pgfqpoint{0.011785in}{0.321548in}}%
\pgfpathcurveto{\pgfqpoint{0.014911in}{0.324674in}}{\pgfqpoint{0.016667in}{0.328913in}}{\pgfqpoint{0.016667in}{0.333333in}}%
\pgfpathcurveto{\pgfqpoint{0.016667in}{0.337753in}}{\pgfqpoint{0.014911in}{0.341993in}}{\pgfqpoint{0.011785in}{0.345118in}}%
\pgfpathcurveto{\pgfqpoint{0.008660in}{0.348244in}}{\pgfqpoint{0.004420in}{0.350000in}}{\pgfqpoint{0.000000in}{0.350000in}}%
\pgfpathcurveto{\pgfqpoint{-0.004420in}{0.350000in}}{\pgfqpoint{-0.008660in}{0.348244in}}{\pgfqpoint{-0.011785in}{0.345118in}}%
\pgfpathcurveto{\pgfqpoint{-0.014911in}{0.341993in}}{\pgfqpoint{-0.016667in}{0.337753in}}{\pgfqpoint{-0.016667in}{0.333333in}}%
\pgfpathcurveto{\pgfqpoint{-0.016667in}{0.328913in}}{\pgfqpoint{-0.014911in}{0.324674in}}{\pgfqpoint{-0.011785in}{0.321548in}}%
\pgfpathcurveto{\pgfqpoint{-0.008660in}{0.318423in}}{\pgfqpoint{-0.004420in}{0.316667in}}{\pgfqpoint{0.000000in}{0.316667in}}%
\pgfpathclose%
\pgfpathmoveto{\pgfqpoint{0.166667in}{0.316667in}}%
\pgfpathcurveto{\pgfqpoint{0.171087in}{0.316667in}}{\pgfqpoint{0.175326in}{0.318423in}}{\pgfqpoint{0.178452in}{0.321548in}}%
\pgfpathcurveto{\pgfqpoint{0.181577in}{0.324674in}}{\pgfqpoint{0.183333in}{0.328913in}}{\pgfqpoint{0.183333in}{0.333333in}}%
\pgfpathcurveto{\pgfqpoint{0.183333in}{0.337753in}}{\pgfqpoint{0.181577in}{0.341993in}}{\pgfqpoint{0.178452in}{0.345118in}}%
\pgfpathcurveto{\pgfqpoint{0.175326in}{0.348244in}}{\pgfqpoint{0.171087in}{0.350000in}}{\pgfqpoint{0.166667in}{0.350000in}}%
\pgfpathcurveto{\pgfqpoint{0.162247in}{0.350000in}}{\pgfqpoint{0.158007in}{0.348244in}}{\pgfqpoint{0.154882in}{0.345118in}}%
\pgfpathcurveto{\pgfqpoint{0.151756in}{0.341993in}}{\pgfqpoint{0.150000in}{0.337753in}}{\pgfqpoint{0.150000in}{0.333333in}}%
\pgfpathcurveto{\pgfqpoint{0.150000in}{0.328913in}}{\pgfqpoint{0.151756in}{0.324674in}}{\pgfqpoint{0.154882in}{0.321548in}}%
\pgfpathcurveto{\pgfqpoint{0.158007in}{0.318423in}}{\pgfqpoint{0.162247in}{0.316667in}}{\pgfqpoint{0.166667in}{0.316667in}}%
\pgfpathclose%
\pgfpathmoveto{\pgfqpoint{0.333333in}{0.316667in}}%
\pgfpathcurveto{\pgfqpoint{0.337753in}{0.316667in}}{\pgfqpoint{0.341993in}{0.318423in}}{\pgfqpoint{0.345118in}{0.321548in}}%
\pgfpathcurveto{\pgfqpoint{0.348244in}{0.324674in}}{\pgfqpoint{0.350000in}{0.328913in}}{\pgfqpoint{0.350000in}{0.333333in}}%
\pgfpathcurveto{\pgfqpoint{0.350000in}{0.337753in}}{\pgfqpoint{0.348244in}{0.341993in}}{\pgfqpoint{0.345118in}{0.345118in}}%
\pgfpathcurveto{\pgfqpoint{0.341993in}{0.348244in}}{\pgfqpoint{0.337753in}{0.350000in}}{\pgfqpoint{0.333333in}{0.350000in}}%
\pgfpathcurveto{\pgfqpoint{0.328913in}{0.350000in}}{\pgfqpoint{0.324674in}{0.348244in}}{\pgfqpoint{0.321548in}{0.345118in}}%
\pgfpathcurveto{\pgfqpoint{0.318423in}{0.341993in}}{\pgfqpoint{0.316667in}{0.337753in}}{\pgfqpoint{0.316667in}{0.333333in}}%
\pgfpathcurveto{\pgfqpoint{0.316667in}{0.328913in}}{\pgfqpoint{0.318423in}{0.324674in}}{\pgfqpoint{0.321548in}{0.321548in}}%
\pgfpathcurveto{\pgfqpoint{0.324674in}{0.318423in}}{\pgfqpoint{0.328913in}{0.316667in}}{\pgfqpoint{0.333333in}{0.316667in}}%
\pgfpathclose%
\pgfpathmoveto{\pgfqpoint{0.500000in}{0.316667in}}%
\pgfpathcurveto{\pgfqpoint{0.504420in}{0.316667in}}{\pgfqpoint{0.508660in}{0.318423in}}{\pgfqpoint{0.511785in}{0.321548in}}%
\pgfpathcurveto{\pgfqpoint{0.514911in}{0.324674in}}{\pgfqpoint{0.516667in}{0.328913in}}{\pgfqpoint{0.516667in}{0.333333in}}%
\pgfpathcurveto{\pgfqpoint{0.516667in}{0.337753in}}{\pgfqpoint{0.514911in}{0.341993in}}{\pgfqpoint{0.511785in}{0.345118in}}%
\pgfpathcurveto{\pgfqpoint{0.508660in}{0.348244in}}{\pgfqpoint{0.504420in}{0.350000in}}{\pgfqpoint{0.500000in}{0.350000in}}%
\pgfpathcurveto{\pgfqpoint{0.495580in}{0.350000in}}{\pgfqpoint{0.491340in}{0.348244in}}{\pgfqpoint{0.488215in}{0.345118in}}%
\pgfpathcurveto{\pgfqpoint{0.485089in}{0.341993in}}{\pgfqpoint{0.483333in}{0.337753in}}{\pgfqpoint{0.483333in}{0.333333in}}%
\pgfpathcurveto{\pgfqpoint{0.483333in}{0.328913in}}{\pgfqpoint{0.485089in}{0.324674in}}{\pgfqpoint{0.488215in}{0.321548in}}%
\pgfpathcurveto{\pgfqpoint{0.491340in}{0.318423in}}{\pgfqpoint{0.495580in}{0.316667in}}{\pgfqpoint{0.500000in}{0.316667in}}%
\pgfpathclose%
\pgfpathmoveto{\pgfqpoint{0.666667in}{0.316667in}}%
\pgfpathcurveto{\pgfqpoint{0.671087in}{0.316667in}}{\pgfqpoint{0.675326in}{0.318423in}}{\pgfqpoint{0.678452in}{0.321548in}}%
\pgfpathcurveto{\pgfqpoint{0.681577in}{0.324674in}}{\pgfqpoint{0.683333in}{0.328913in}}{\pgfqpoint{0.683333in}{0.333333in}}%
\pgfpathcurveto{\pgfqpoint{0.683333in}{0.337753in}}{\pgfqpoint{0.681577in}{0.341993in}}{\pgfqpoint{0.678452in}{0.345118in}}%
\pgfpathcurveto{\pgfqpoint{0.675326in}{0.348244in}}{\pgfqpoint{0.671087in}{0.350000in}}{\pgfqpoint{0.666667in}{0.350000in}}%
\pgfpathcurveto{\pgfqpoint{0.662247in}{0.350000in}}{\pgfqpoint{0.658007in}{0.348244in}}{\pgfqpoint{0.654882in}{0.345118in}}%
\pgfpathcurveto{\pgfqpoint{0.651756in}{0.341993in}}{\pgfqpoint{0.650000in}{0.337753in}}{\pgfqpoint{0.650000in}{0.333333in}}%
\pgfpathcurveto{\pgfqpoint{0.650000in}{0.328913in}}{\pgfqpoint{0.651756in}{0.324674in}}{\pgfqpoint{0.654882in}{0.321548in}}%
\pgfpathcurveto{\pgfqpoint{0.658007in}{0.318423in}}{\pgfqpoint{0.662247in}{0.316667in}}{\pgfqpoint{0.666667in}{0.316667in}}%
\pgfpathclose%
\pgfpathmoveto{\pgfqpoint{0.833333in}{0.316667in}}%
\pgfpathcurveto{\pgfqpoint{0.837753in}{0.316667in}}{\pgfqpoint{0.841993in}{0.318423in}}{\pgfqpoint{0.845118in}{0.321548in}}%
\pgfpathcurveto{\pgfqpoint{0.848244in}{0.324674in}}{\pgfqpoint{0.850000in}{0.328913in}}{\pgfqpoint{0.850000in}{0.333333in}}%
\pgfpathcurveto{\pgfqpoint{0.850000in}{0.337753in}}{\pgfqpoint{0.848244in}{0.341993in}}{\pgfqpoint{0.845118in}{0.345118in}}%
\pgfpathcurveto{\pgfqpoint{0.841993in}{0.348244in}}{\pgfqpoint{0.837753in}{0.350000in}}{\pgfqpoint{0.833333in}{0.350000in}}%
\pgfpathcurveto{\pgfqpoint{0.828913in}{0.350000in}}{\pgfqpoint{0.824674in}{0.348244in}}{\pgfqpoint{0.821548in}{0.345118in}}%
\pgfpathcurveto{\pgfqpoint{0.818423in}{0.341993in}}{\pgfqpoint{0.816667in}{0.337753in}}{\pgfqpoint{0.816667in}{0.333333in}}%
\pgfpathcurveto{\pgfqpoint{0.816667in}{0.328913in}}{\pgfqpoint{0.818423in}{0.324674in}}{\pgfqpoint{0.821548in}{0.321548in}}%
\pgfpathcurveto{\pgfqpoint{0.824674in}{0.318423in}}{\pgfqpoint{0.828913in}{0.316667in}}{\pgfqpoint{0.833333in}{0.316667in}}%
\pgfpathclose%
\pgfpathmoveto{\pgfqpoint{1.000000in}{0.316667in}}%
\pgfpathcurveto{\pgfqpoint{1.004420in}{0.316667in}}{\pgfqpoint{1.008660in}{0.318423in}}{\pgfqpoint{1.011785in}{0.321548in}}%
\pgfpathcurveto{\pgfqpoint{1.014911in}{0.324674in}}{\pgfqpoint{1.016667in}{0.328913in}}{\pgfqpoint{1.016667in}{0.333333in}}%
\pgfpathcurveto{\pgfqpoint{1.016667in}{0.337753in}}{\pgfqpoint{1.014911in}{0.341993in}}{\pgfqpoint{1.011785in}{0.345118in}}%
\pgfpathcurveto{\pgfqpoint{1.008660in}{0.348244in}}{\pgfqpoint{1.004420in}{0.350000in}}{\pgfqpoint{1.000000in}{0.350000in}}%
\pgfpathcurveto{\pgfqpoint{0.995580in}{0.350000in}}{\pgfqpoint{0.991340in}{0.348244in}}{\pgfqpoint{0.988215in}{0.345118in}}%
\pgfpathcurveto{\pgfqpoint{0.985089in}{0.341993in}}{\pgfqpoint{0.983333in}{0.337753in}}{\pgfqpoint{0.983333in}{0.333333in}}%
\pgfpathcurveto{\pgfqpoint{0.983333in}{0.328913in}}{\pgfqpoint{0.985089in}{0.324674in}}{\pgfqpoint{0.988215in}{0.321548in}}%
\pgfpathcurveto{\pgfqpoint{0.991340in}{0.318423in}}{\pgfqpoint{0.995580in}{0.316667in}}{\pgfqpoint{1.000000in}{0.316667in}}%
\pgfpathclose%
\pgfpathmoveto{\pgfqpoint{0.083333in}{0.483333in}}%
\pgfpathcurveto{\pgfqpoint{0.087753in}{0.483333in}}{\pgfqpoint{0.091993in}{0.485089in}}{\pgfqpoint{0.095118in}{0.488215in}}%
\pgfpathcurveto{\pgfqpoint{0.098244in}{0.491340in}}{\pgfqpoint{0.100000in}{0.495580in}}{\pgfqpoint{0.100000in}{0.500000in}}%
\pgfpathcurveto{\pgfqpoint{0.100000in}{0.504420in}}{\pgfqpoint{0.098244in}{0.508660in}}{\pgfqpoint{0.095118in}{0.511785in}}%
\pgfpathcurveto{\pgfqpoint{0.091993in}{0.514911in}}{\pgfqpoint{0.087753in}{0.516667in}}{\pgfqpoint{0.083333in}{0.516667in}}%
\pgfpathcurveto{\pgfqpoint{0.078913in}{0.516667in}}{\pgfqpoint{0.074674in}{0.514911in}}{\pgfqpoint{0.071548in}{0.511785in}}%
\pgfpathcurveto{\pgfqpoint{0.068423in}{0.508660in}}{\pgfqpoint{0.066667in}{0.504420in}}{\pgfqpoint{0.066667in}{0.500000in}}%
\pgfpathcurveto{\pgfqpoint{0.066667in}{0.495580in}}{\pgfqpoint{0.068423in}{0.491340in}}{\pgfqpoint{0.071548in}{0.488215in}}%
\pgfpathcurveto{\pgfqpoint{0.074674in}{0.485089in}}{\pgfqpoint{0.078913in}{0.483333in}}{\pgfqpoint{0.083333in}{0.483333in}}%
\pgfpathclose%
\pgfpathmoveto{\pgfqpoint{0.250000in}{0.483333in}}%
\pgfpathcurveto{\pgfqpoint{0.254420in}{0.483333in}}{\pgfqpoint{0.258660in}{0.485089in}}{\pgfqpoint{0.261785in}{0.488215in}}%
\pgfpathcurveto{\pgfqpoint{0.264911in}{0.491340in}}{\pgfqpoint{0.266667in}{0.495580in}}{\pgfqpoint{0.266667in}{0.500000in}}%
\pgfpathcurveto{\pgfqpoint{0.266667in}{0.504420in}}{\pgfqpoint{0.264911in}{0.508660in}}{\pgfqpoint{0.261785in}{0.511785in}}%
\pgfpathcurveto{\pgfqpoint{0.258660in}{0.514911in}}{\pgfqpoint{0.254420in}{0.516667in}}{\pgfqpoint{0.250000in}{0.516667in}}%
\pgfpathcurveto{\pgfqpoint{0.245580in}{0.516667in}}{\pgfqpoint{0.241340in}{0.514911in}}{\pgfqpoint{0.238215in}{0.511785in}}%
\pgfpathcurveto{\pgfqpoint{0.235089in}{0.508660in}}{\pgfqpoint{0.233333in}{0.504420in}}{\pgfqpoint{0.233333in}{0.500000in}}%
\pgfpathcurveto{\pgfqpoint{0.233333in}{0.495580in}}{\pgfqpoint{0.235089in}{0.491340in}}{\pgfqpoint{0.238215in}{0.488215in}}%
\pgfpathcurveto{\pgfqpoint{0.241340in}{0.485089in}}{\pgfqpoint{0.245580in}{0.483333in}}{\pgfqpoint{0.250000in}{0.483333in}}%
\pgfpathclose%
\pgfpathmoveto{\pgfqpoint{0.416667in}{0.483333in}}%
\pgfpathcurveto{\pgfqpoint{0.421087in}{0.483333in}}{\pgfqpoint{0.425326in}{0.485089in}}{\pgfqpoint{0.428452in}{0.488215in}}%
\pgfpathcurveto{\pgfqpoint{0.431577in}{0.491340in}}{\pgfqpoint{0.433333in}{0.495580in}}{\pgfqpoint{0.433333in}{0.500000in}}%
\pgfpathcurveto{\pgfqpoint{0.433333in}{0.504420in}}{\pgfqpoint{0.431577in}{0.508660in}}{\pgfqpoint{0.428452in}{0.511785in}}%
\pgfpathcurveto{\pgfqpoint{0.425326in}{0.514911in}}{\pgfqpoint{0.421087in}{0.516667in}}{\pgfqpoint{0.416667in}{0.516667in}}%
\pgfpathcurveto{\pgfqpoint{0.412247in}{0.516667in}}{\pgfqpoint{0.408007in}{0.514911in}}{\pgfqpoint{0.404882in}{0.511785in}}%
\pgfpathcurveto{\pgfqpoint{0.401756in}{0.508660in}}{\pgfqpoint{0.400000in}{0.504420in}}{\pgfqpoint{0.400000in}{0.500000in}}%
\pgfpathcurveto{\pgfqpoint{0.400000in}{0.495580in}}{\pgfqpoint{0.401756in}{0.491340in}}{\pgfqpoint{0.404882in}{0.488215in}}%
\pgfpathcurveto{\pgfqpoint{0.408007in}{0.485089in}}{\pgfqpoint{0.412247in}{0.483333in}}{\pgfqpoint{0.416667in}{0.483333in}}%
\pgfpathclose%
\pgfpathmoveto{\pgfqpoint{0.583333in}{0.483333in}}%
\pgfpathcurveto{\pgfqpoint{0.587753in}{0.483333in}}{\pgfqpoint{0.591993in}{0.485089in}}{\pgfqpoint{0.595118in}{0.488215in}}%
\pgfpathcurveto{\pgfqpoint{0.598244in}{0.491340in}}{\pgfqpoint{0.600000in}{0.495580in}}{\pgfqpoint{0.600000in}{0.500000in}}%
\pgfpathcurveto{\pgfqpoint{0.600000in}{0.504420in}}{\pgfqpoint{0.598244in}{0.508660in}}{\pgfqpoint{0.595118in}{0.511785in}}%
\pgfpathcurveto{\pgfqpoint{0.591993in}{0.514911in}}{\pgfqpoint{0.587753in}{0.516667in}}{\pgfqpoint{0.583333in}{0.516667in}}%
\pgfpathcurveto{\pgfqpoint{0.578913in}{0.516667in}}{\pgfqpoint{0.574674in}{0.514911in}}{\pgfqpoint{0.571548in}{0.511785in}}%
\pgfpathcurveto{\pgfqpoint{0.568423in}{0.508660in}}{\pgfqpoint{0.566667in}{0.504420in}}{\pgfqpoint{0.566667in}{0.500000in}}%
\pgfpathcurveto{\pgfqpoint{0.566667in}{0.495580in}}{\pgfqpoint{0.568423in}{0.491340in}}{\pgfqpoint{0.571548in}{0.488215in}}%
\pgfpathcurveto{\pgfqpoint{0.574674in}{0.485089in}}{\pgfqpoint{0.578913in}{0.483333in}}{\pgfqpoint{0.583333in}{0.483333in}}%
\pgfpathclose%
\pgfpathmoveto{\pgfqpoint{0.750000in}{0.483333in}}%
\pgfpathcurveto{\pgfqpoint{0.754420in}{0.483333in}}{\pgfqpoint{0.758660in}{0.485089in}}{\pgfqpoint{0.761785in}{0.488215in}}%
\pgfpathcurveto{\pgfqpoint{0.764911in}{0.491340in}}{\pgfqpoint{0.766667in}{0.495580in}}{\pgfqpoint{0.766667in}{0.500000in}}%
\pgfpathcurveto{\pgfqpoint{0.766667in}{0.504420in}}{\pgfqpoint{0.764911in}{0.508660in}}{\pgfqpoint{0.761785in}{0.511785in}}%
\pgfpathcurveto{\pgfqpoint{0.758660in}{0.514911in}}{\pgfqpoint{0.754420in}{0.516667in}}{\pgfqpoint{0.750000in}{0.516667in}}%
\pgfpathcurveto{\pgfqpoint{0.745580in}{0.516667in}}{\pgfqpoint{0.741340in}{0.514911in}}{\pgfqpoint{0.738215in}{0.511785in}}%
\pgfpathcurveto{\pgfqpoint{0.735089in}{0.508660in}}{\pgfqpoint{0.733333in}{0.504420in}}{\pgfqpoint{0.733333in}{0.500000in}}%
\pgfpathcurveto{\pgfqpoint{0.733333in}{0.495580in}}{\pgfqpoint{0.735089in}{0.491340in}}{\pgfqpoint{0.738215in}{0.488215in}}%
\pgfpathcurveto{\pgfqpoint{0.741340in}{0.485089in}}{\pgfqpoint{0.745580in}{0.483333in}}{\pgfqpoint{0.750000in}{0.483333in}}%
\pgfpathclose%
\pgfpathmoveto{\pgfqpoint{0.916667in}{0.483333in}}%
\pgfpathcurveto{\pgfqpoint{0.921087in}{0.483333in}}{\pgfqpoint{0.925326in}{0.485089in}}{\pgfqpoint{0.928452in}{0.488215in}}%
\pgfpathcurveto{\pgfqpoint{0.931577in}{0.491340in}}{\pgfqpoint{0.933333in}{0.495580in}}{\pgfqpoint{0.933333in}{0.500000in}}%
\pgfpathcurveto{\pgfqpoint{0.933333in}{0.504420in}}{\pgfqpoint{0.931577in}{0.508660in}}{\pgfqpoint{0.928452in}{0.511785in}}%
\pgfpathcurveto{\pgfqpoint{0.925326in}{0.514911in}}{\pgfqpoint{0.921087in}{0.516667in}}{\pgfqpoint{0.916667in}{0.516667in}}%
\pgfpathcurveto{\pgfqpoint{0.912247in}{0.516667in}}{\pgfqpoint{0.908007in}{0.514911in}}{\pgfqpoint{0.904882in}{0.511785in}}%
\pgfpathcurveto{\pgfqpoint{0.901756in}{0.508660in}}{\pgfqpoint{0.900000in}{0.504420in}}{\pgfqpoint{0.900000in}{0.500000in}}%
\pgfpathcurveto{\pgfqpoint{0.900000in}{0.495580in}}{\pgfqpoint{0.901756in}{0.491340in}}{\pgfqpoint{0.904882in}{0.488215in}}%
\pgfpathcurveto{\pgfqpoint{0.908007in}{0.485089in}}{\pgfqpoint{0.912247in}{0.483333in}}{\pgfqpoint{0.916667in}{0.483333in}}%
\pgfpathclose%
\pgfpathmoveto{\pgfqpoint{0.000000in}{0.650000in}}%
\pgfpathcurveto{\pgfqpoint{0.004420in}{0.650000in}}{\pgfqpoint{0.008660in}{0.651756in}}{\pgfqpoint{0.011785in}{0.654882in}}%
\pgfpathcurveto{\pgfqpoint{0.014911in}{0.658007in}}{\pgfqpoint{0.016667in}{0.662247in}}{\pgfqpoint{0.016667in}{0.666667in}}%
\pgfpathcurveto{\pgfqpoint{0.016667in}{0.671087in}}{\pgfqpoint{0.014911in}{0.675326in}}{\pgfqpoint{0.011785in}{0.678452in}}%
\pgfpathcurveto{\pgfqpoint{0.008660in}{0.681577in}}{\pgfqpoint{0.004420in}{0.683333in}}{\pgfqpoint{0.000000in}{0.683333in}}%
\pgfpathcurveto{\pgfqpoint{-0.004420in}{0.683333in}}{\pgfqpoint{-0.008660in}{0.681577in}}{\pgfqpoint{-0.011785in}{0.678452in}}%
\pgfpathcurveto{\pgfqpoint{-0.014911in}{0.675326in}}{\pgfqpoint{-0.016667in}{0.671087in}}{\pgfqpoint{-0.016667in}{0.666667in}}%
\pgfpathcurveto{\pgfqpoint{-0.016667in}{0.662247in}}{\pgfqpoint{-0.014911in}{0.658007in}}{\pgfqpoint{-0.011785in}{0.654882in}}%
\pgfpathcurveto{\pgfqpoint{-0.008660in}{0.651756in}}{\pgfqpoint{-0.004420in}{0.650000in}}{\pgfqpoint{0.000000in}{0.650000in}}%
\pgfpathclose%
\pgfpathmoveto{\pgfqpoint{0.166667in}{0.650000in}}%
\pgfpathcurveto{\pgfqpoint{0.171087in}{0.650000in}}{\pgfqpoint{0.175326in}{0.651756in}}{\pgfqpoint{0.178452in}{0.654882in}}%
\pgfpathcurveto{\pgfqpoint{0.181577in}{0.658007in}}{\pgfqpoint{0.183333in}{0.662247in}}{\pgfqpoint{0.183333in}{0.666667in}}%
\pgfpathcurveto{\pgfqpoint{0.183333in}{0.671087in}}{\pgfqpoint{0.181577in}{0.675326in}}{\pgfqpoint{0.178452in}{0.678452in}}%
\pgfpathcurveto{\pgfqpoint{0.175326in}{0.681577in}}{\pgfqpoint{0.171087in}{0.683333in}}{\pgfqpoint{0.166667in}{0.683333in}}%
\pgfpathcurveto{\pgfqpoint{0.162247in}{0.683333in}}{\pgfqpoint{0.158007in}{0.681577in}}{\pgfqpoint{0.154882in}{0.678452in}}%
\pgfpathcurveto{\pgfqpoint{0.151756in}{0.675326in}}{\pgfqpoint{0.150000in}{0.671087in}}{\pgfqpoint{0.150000in}{0.666667in}}%
\pgfpathcurveto{\pgfqpoint{0.150000in}{0.662247in}}{\pgfqpoint{0.151756in}{0.658007in}}{\pgfqpoint{0.154882in}{0.654882in}}%
\pgfpathcurveto{\pgfqpoint{0.158007in}{0.651756in}}{\pgfqpoint{0.162247in}{0.650000in}}{\pgfqpoint{0.166667in}{0.650000in}}%
\pgfpathclose%
\pgfpathmoveto{\pgfqpoint{0.333333in}{0.650000in}}%
\pgfpathcurveto{\pgfqpoint{0.337753in}{0.650000in}}{\pgfqpoint{0.341993in}{0.651756in}}{\pgfqpoint{0.345118in}{0.654882in}}%
\pgfpathcurveto{\pgfqpoint{0.348244in}{0.658007in}}{\pgfqpoint{0.350000in}{0.662247in}}{\pgfqpoint{0.350000in}{0.666667in}}%
\pgfpathcurveto{\pgfqpoint{0.350000in}{0.671087in}}{\pgfqpoint{0.348244in}{0.675326in}}{\pgfqpoint{0.345118in}{0.678452in}}%
\pgfpathcurveto{\pgfqpoint{0.341993in}{0.681577in}}{\pgfqpoint{0.337753in}{0.683333in}}{\pgfqpoint{0.333333in}{0.683333in}}%
\pgfpathcurveto{\pgfqpoint{0.328913in}{0.683333in}}{\pgfqpoint{0.324674in}{0.681577in}}{\pgfqpoint{0.321548in}{0.678452in}}%
\pgfpathcurveto{\pgfqpoint{0.318423in}{0.675326in}}{\pgfqpoint{0.316667in}{0.671087in}}{\pgfqpoint{0.316667in}{0.666667in}}%
\pgfpathcurveto{\pgfqpoint{0.316667in}{0.662247in}}{\pgfqpoint{0.318423in}{0.658007in}}{\pgfqpoint{0.321548in}{0.654882in}}%
\pgfpathcurveto{\pgfqpoint{0.324674in}{0.651756in}}{\pgfqpoint{0.328913in}{0.650000in}}{\pgfqpoint{0.333333in}{0.650000in}}%
\pgfpathclose%
\pgfpathmoveto{\pgfqpoint{0.500000in}{0.650000in}}%
\pgfpathcurveto{\pgfqpoint{0.504420in}{0.650000in}}{\pgfqpoint{0.508660in}{0.651756in}}{\pgfqpoint{0.511785in}{0.654882in}}%
\pgfpathcurveto{\pgfqpoint{0.514911in}{0.658007in}}{\pgfqpoint{0.516667in}{0.662247in}}{\pgfqpoint{0.516667in}{0.666667in}}%
\pgfpathcurveto{\pgfqpoint{0.516667in}{0.671087in}}{\pgfqpoint{0.514911in}{0.675326in}}{\pgfqpoint{0.511785in}{0.678452in}}%
\pgfpathcurveto{\pgfqpoint{0.508660in}{0.681577in}}{\pgfqpoint{0.504420in}{0.683333in}}{\pgfqpoint{0.500000in}{0.683333in}}%
\pgfpathcurveto{\pgfqpoint{0.495580in}{0.683333in}}{\pgfqpoint{0.491340in}{0.681577in}}{\pgfqpoint{0.488215in}{0.678452in}}%
\pgfpathcurveto{\pgfqpoint{0.485089in}{0.675326in}}{\pgfqpoint{0.483333in}{0.671087in}}{\pgfqpoint{0.483333in}{0.666667in}}%
\pgfpathcurveto{\pgfqpoint{0.483333in}{0.662247in}}{\pgfqpoint{0.485089in}{0.658007in}}{\pgfqpoint{0.488215in}{0.654882in}}%
\pgfpathcurveto{\pgfqpoint{0.491340in}{0.651756in}}{\pgfqpoint{0.495580in}{0.650000in}}{\pgfqpoint{0.500000in}{0.650000in}}%
\pgfpathclose%
\pgfpathmoveto{\pgfqpoint{0.666667in}{0.650000in}}%
\pgfpathcurveto{\pgfqpoint{0.671087in}{0.650000in}}{\pgfqpoint{0.675326in}{0.651756in}}{\pgfqpoint{0.678452in}{0.654882in}}%
\pgfpathcurveto{\pgfqpoint{0.681577in}{0.658007in}}{\pgfqpoint{0.683333in}{0.662247in}}{\pgfqpoint{0.683333in}{0.666667in}}%
\pgfpathcurveto{\pgfqpoint{0.683333in}{0.671087in}}{\pgfqpoint{0.681577in}{0.675326in}}{\pgfqpoint{0.678452in}{0.678452in}}%
\pgfpathcurveto{\pgfqpoint{0.675326in}{0.681577in}}{\pgfqpoint{0.671087in}{0.683333in}}{\pgfqpoint{0.666667in}{0.683333in}}%
\pgfpathcurveto{\pgfqpoint{0.662247in}{0.683333in}}{\pgfqpoint{0.658007in}{0.681577in}}{\pgfqpoint{0.654882in}{0.678452in}}%
\pgfpathcurveto{\pgfqpoint{0.651756in}{0.675326in}}{\pgfqpoint{0.650000in}{0.671087in}}{\pgfqpoint{0.650000in}{0.666667in}}%
\pgfpathcurveto{\pgfqpoint{0.650000in}{0.662247in}}{\pgfqpoint{0.651756in}{0.658007in}}{\pgfqpoint{0.654882in}{0.654882in}}%
\pgfpathcurveto{\pgfqpoint{0.658007in}{0.651756in}}{\pgfqpoint{0.662247in}{0.650000in}}{\pgfqpoint{0.666667in}{0.650000in}}%
\pgfpathclose%
\pgfpathmoveto{\pgfqpoint{0.833333in}{0.650000in}}%
\pgfpathcurveto{\pgfqpoint{0.837753in}{0.650000in}}{\pgfqpoint{0.841993in}{0.651756in}}{\pgfqpoint{0.845118in}{0.654882in}}%
\pgfpathcurveto{\pgfqpoint{0.848244in}{0.658007in}}{\pgfqpoint{0.850000in}{0.662247in}}{\pgfqpoint{0.850000in}{0.666667in}}%
\pgfpathcurveto{\pgfqpoint{0.850000in}{0.671087in}}{\pgfqpoint{0.848244in}{0.675326in}}{\pgfqpoint{0.845118in}{0.678452in}}%
\pgfpathcurveto{\pgfqpoint{0.841993in}{0.681577in}}{\pgfqpoint{0.837753in}{0.683333in}}{\pgfqpoint{0.833333in}{0.683333in}}%
\pgfpathcurveto{\pgfqpoint{0.828913in}{0.683333in}}{\pgfqpoint{0.824674in}{0.681577in}}{\pgfqpoint{0.821548in}{0.678452in}}%
\pgfpathcurveto{\pgfqpoint{0.818423in}{0.675326in}}{\pgfqpoint{0.816667in}{0.671087in}}{\pgfqpoint{0.816667in}{0.666667in}}%
\pgfpathcurveto{\pgfqpoint{0.816667in}{0.662247in}}{\pgfqpoint{0.818423in}{0.658007in}}{\pgfqpoint{0.821548in}{0.654882in}}%
\pgfpathcurveto{\pgfqpoint{0.824674in}{0.651756in}}{\pgfqpoint{0.828913in}{0.650000in}}{\pgfqpoint{0.833333in}{0.650000in}}%
\pgfpathclose%
\pgfpathmoveto{\pgfqpoint{1.000000in}{0.650000in}}%
\pgfpathcurveto{\pgfqpoint{1.004420in}{0.650000in}}{\pgfqpoint{1.008660in}{0.651756in}}{\pgfqpoint{1.011785in}{0.654882in}}%
\pgfpathcurveto{\pgfqpoint{1.014911in}{0.658007in}}{\pgfqpoint{1.016667in}{0.662247in}}{\pgfqpoint{1.016667in}{0.666667in}}%
\pgfpathcurveto{\pgfqpoint{1.016667in}{0.671087in}}{\pgfqpoint{1.014911in}{0.675326in}}{\pgfqpoint{1.011785in}{0.678452in}}%
\pgfpathcurveto{\pgfqpoint{1.008660in}{0.681577in}}{\pgfqpoint{1.004420in}{0.683333in}}{\pgfqpoint{1.000000in}{0.683333in}}%
\pgfpathcurveto{\pgfqpoint{0.995580in}{0.683333in}}{\pgfqpoint{0.991340in}{0.681577in}}{\pgfqpoint{0.988215in}{0.678452in}}%
\pgfpathcurveto{\pgfqpoint{0.985089in}{0.675326in}}{\pgfqpoint{0.983333in}{0.671087in}}{\pgfqpoint{0.983333in}{0.666667in}}%
\pgfpathcurveto{\pgfqpoint{0.983333in}{0.662247in}}{\pgfqpoint{0.985089in}{0.658007in}}{\pgfqpoint{0.988215in}{0.654882in}}%
\pgfpathcurveto{\pgfqpoint{0.991340in}{0.651756in}}{\pgfqpoint{0.995580in}{0.650000in}}{\pgfqpoint{1.000000in}{0.650000in}}%
\pgfpathclose%
\pgfpathmoveto{\pgfqpoint{0.083333in}{0.816667in}}%
\pgfpathcurveto{\pgfqpoint{0.087753in}{0.816667in}}{\pgfqpoint{0.091993in}{0.818423in}}{\pgfqpoint{0.095118in}{0.821548in}}%
\pgfpathcurveto{\pgfqpoint{0.098244in}{0.824674in}}{\pgfqpoint{0.100000in}{0.828913in}}{\pgfqpoint{0.100000in}{0.833333in}}%
\pgfpathcurveto{\pgfqpoint{0.100000in}{0.837753in}}{\pgfqpoint{0.098244in}{0.841993in}}{\pgfqpoint{0.095118in}{0.845118in}}%
\pgfpathcurveto{\pgfqpoint{0.091993in}{0.848244in}}{\pgfqpoint{0.087753in}{0.850000in}}{\pgfqpoint{0.083333in}{0.850000in}}%
\pgfpathcurveto{\pgfqpoint{0.078913in}{0.850000in}}{\pgfqpoint{0.074674in}{0.848244in}}{\pgfqpoint{0.071548in}{0.845118in}}%
\pgfpathcurveto{\pgfqpoint{0.068423in}{0.841993in}}{\pgfqpoint{0.066667in}{0.837753in}}{\pgfqpoint{0.066667in}{0.833333in}}%
\pgfpathcurveto{\pgfqpoint{0.066667in}{0.828913in}}{\pgfqpoint{0.068423in}{0.824674in}}{\pgfqpoint{0.071548in}{0.821548in}}%
\pgfpathcurveto{\pgfqpoint{0.074674in}{0.818423in}}{\pgfqpoint{0.078913in}{0.816667in}}{\pgfqpoint{0.083333in}{0.816667in}}%
\pgfpathclose%
\pgfpathmoveto{\pgfqpoint{0.250000in}{0.816667in}}%
\pgfpathcurveto{\pgfqpoint{0.254420in}{0.816667in}}{\pgfqpoint{0.258660in}{0.818423in}}{\pgfqpoint{0.261785in}{0.821548in}}%
\pgfpathcurveto{\pgfqpoint{0.264911in}{0.824674in}}{\pgfqpoint{0.266667in}{0.828913in}}{\pgfqpoint{0.266667in}{0.833333in}}%
\pgfpathcurveto{\pgfqpoint{0.266667in}{0.837753in}}{\pgfqpoint{0.264911in}{0.841993in}}{\pgfqpoint{0.261785in}{0.845118in}}%
\pgfpathcurveto{\pgfqpoint{0.258660in}{0.848244in}}{\pgfqpoint{0.254420in}{0.850000in}}{\pgfqpoint{0.250000in}{0.850000in}}%
\pgfpathcurveto{\pgfqpoint{0.245580in}{0.850000in}}{\pgfqpoint{0.241340in}{0.848244in}}{\pgfqpoint{0.238215in}{0.845118in}}%
\pgfpathcurveto{\pgfqpoint{0.235089in}{0.841993in}}{\pgfqpoint{0.233333in}{0.837753in}}{\pgfqpoint{0.233333in}{0.833333in}}%
\pgfpathcurveto{\pgfqpoint{0.233333in}{0.828913in}}{\pgfqpoint{0.235089in}{0.824674in}}{\pgfqpoint{0.238215in}{0.821548in}}%
\pgfpathcurveto{\pgfqpoint{0.241340in}{0.818423in}}{\pgfqpoint{0.245580in}{0.816667in}}{\pgfqpoint{0.250000in}{0.816667in}}%
\pgfpathclose%
\pgfpathmoveto{\pgfqpoint{0.416667in}{0.816667in}}%
\pgfpathcurveto{\pgfqpoint{0.421087in}{0.816667in}}{\pgfqpoint{0.425326in}{0.818423in}}{\pgfqpoint{0.428452in}{0.821548in}}%
\pgfpathcurveto{\pgfqpoint{0.431577in}{0.824674in}}{\pgfqpoint{0.433333in}{0.828913in}}{\pgfqpoint{0.433333in}{0.833333in}}%
\pgfpathcurveto{\pgfqpoint{0.433333in}{0.837753in}}{\pgfqpoint{0.431577in}{0.841993in}}{\pgfqpoint{0.428452in}{0.845118in}}%
\pgfpathcurveto{\pgfqpoint{0.425326in}{0.848244in}}{\pgfqpoint{0.421087in}{0.850000in}}{\pgfqpoint{0.416667in}{0.850000in}}%
\pgfpathcurveto{\pgfqpoint{0.412247in}{0.850000in}}{\pgfqpoint{0.408007in}{0.848244in}}{\pgfqpoint{0.404882in}{0.845118in}}%
\pgfpathcurveto{\pgfqpoint{0.401756in}{0.841993in}}{\pgfqpoint{0.400000in}{0.837753in}}{\pgfqpoint{0.400000in}{0.833333in}}%
\pgfpathcurveto{\pgfqpoint{0.400000in}{0.828913in}}{\pgfqpoint{0.401756in}{0.824674in}}{\pgfqpoint{0.404882in}{0.821548in}}%
\pgfpathcurveto{\pgfqpoint{0.408007in}{0.818423in}}{\pgfqpoint{0.412247in}{0.816667in}}{\pgfqpoint{0.416667in}{0.816667in}}%
\pgfpathclose%
\pgfpathmoveto{\pgfqpoint{0.583333in}{0.816667in}}%
\pgfpathcurveto{\pgfqpoint{0.587753in}{0.816667in}}{\pgfqpoint{0.591993in}{0.818423in}}{\pgfqpoint{0.595118in}{0.821548in}}%
\pgfpathcurveto{\pgfqpoint{0.598244in}{0.824674in}}{\pgfqpoint{0.600000in}{0.828913in}}{\pgfqpoint{0.600000in}{0.833333in}}%
\pgfpathcurveto{\pgfqpoint{0.600000in}{0.837753in}}{\pgfqpoint{0.598244in}{0.841993in}}{\pgfqpoint{0.595118in}{0.845118in}}%
\pgfpathcurveto{\pgfqpoint{0.591993in}{0.848244in}}{\pgfqpoint{0.587753in}{0.850000in}}{\pgfqpoint{0.583333in}{0.850000in}}%
\pgfpathcurveto{\pgfqpoint{0.578913in}{0.850000in}}{\pgfqpoint{0.574674in}{0.848244in}}{\pgfqpoint{0.571548in}{0.845118in}}%
\pgfpathcurveto{\pgfqpoint{0.568423in}{0.841993in}}{\pgfqpoint{0.566667in}{0.837753in}}{\pgfqpoint{0.566667in}{0.833333in}}%
\pgfpathcurveto{\pgfqpoint{0.566667in}{0.828913in}}{\pgfqpoint{0.568423in}{0.824674in}}{\pgfqpoint{0.571548in}{0.821548in}}%
\pgfpathcurveto{\pgfqpoint{0.574674in}{0.818423in}}{\pgfqpoint{0.578913in}{0.816667in}}{\pgfqpoint{0.583333in}{0.816667in}}%
\pgfpathclose%
\pgfpathmoveto{\pgfqpoint{0.750000in}{0.816667in}}%
\pgfpathcurveto{\pgfqpoint{0.754420in}{0.816667in}}{\pgfqpoint{0.758660in}{0.818423in}}{\pgfqpoint{0.761785in}{0.821548in}}%
\pgfpathcurveto{\pgfqpoint{0.764911in}{0.824674in}}{\pgfqpoint{0.766667in}{0.828913in}}{\pgfqpoint{0.766667in}{0.833333in}}%
\pgfpathcurveto{\pgfqpoint{0.766667in}{0.837753in}}{\pgfqpoint{0.764911in}{0.841993in}}{\pgfqpoint{0.761785in}{0.845118in}}%
\pgfpathcurveto{\pgfqpoint{0.758660in}{0.848244in}}{\pgfqpoint{0.754420in}{0.850000in}}{\pgfqpoint{0.750000in}{0.850000in}}%
\pgfpathcurveto{\pgfqpoint{0.745580in}{0.850000in}}{\pgfqpoint{0.741340in}{0.848244in}}{\pgfqpoint{0.738215in}{0.845118in}}%
\pgfpathcurveto{\pgfqpoint{0.735089in}{0.841993in}}{\pgfqpoint{0.733333in}{0.837753in}}{\pgfqpoint{0.733333in}{0.833333in}}%
\pgfpathcurveto{\pgfqpoint{0.733333in}{0.828913in}}{\pgfqpoint{0.735089in}{0.824674in}}{\pgfqpoint{0.738215in}{0.821548in}}%
\pgfpathcurveto{\pgfqpoint{0.741340in}{0.818423in}}{\pgfqpoint{0.745580in}{0.816667in}}{\pgfqpoint{0.750000in}{0.816667in}}%
\pgfpathclose%
\pgfpathmoveto{\pgfqpoint{0.916667in}{0.816667in}}%
\pgfpathcurveto{\pgfqpoint{0.921087in}{0.816667in}}{\pgfqpoint{0.925326in}{0.818423in}}{\pgfqpoint{0.928452in}{0.821548in}}%
\pgfpathcurveto{\pgfqpoint{0.931577in}{0.824674in}}{\pgfqpoint{0.933333in}{0.828913in}}{\pgfqpoint{0.933333in}{0.833333in}}%
\pgfpathcurveto{\pgfqpoint{0.933333in}{0.837753in}}{\pgfqpoint{0.931577in}{0.841993in}}{\pgfqpoint{0.928452in}{0.845118in}}%
\pgfpathcurveto{\pgfqpoint{0.925326in}{0.848244in}}{\pgfqpoint{0.921087in}{0.850000in}}{\pgfqpoint{0.916667in}{0.850000in}}%
\pgfpathcurveto{\pgfqpoint{0.912247in}{0.850000in}}{\pgfqpoint{0.908007in}{0.848244in}}{\pgfqpoint{0.904882in}{0.845118in}}%
\pgfpathcurveto{\pgfqpoint{0.901756in}{0.841993in}}{\pgfqpoint{0.900000in}{0.837753in}}{\pgfqpoint{0.900000in}{0.833333in}}%
\pgfpathcurveto{\pgfqpoint{0.900000in}{0.828913in}}{\pgfqpoint{0.901756in}{0.824674in}}{\pgfqpoint{0.904882in}{0.821548in}}%
\pgfpathcurveto{\pgfqpoint{0.908007in}{0.818423in}}{\pgfqpoint{0.912247in}{0.816667in}}{\pgfqpoint{0.916667in}{0.816667in}}%
\pgfpathclose%
\pgfpathmoveto{\pgfqpoint{0.000000in}{0.983333in}}%
\pgfpathcurveto{\pgfqpoint{0.004420in}{0.983333in}}{\pgfqpoint{0.008660in}{0.985089in}}{\pgfqpoint{0.011785in}{0.988215in}}%
\pgfpathcurveto{\pgfqpoint{0.014911in}{0.991340in}}{\pgfqpoint{0.016667in}{0.995580in}}{\pgfqpoint{0.016667in}{1.000000in}}%
\pgfpathcurveto{\pgfqpoint{0.016667in}{1.004420in}}{\pgfqpoint{0.014911in}{1.008660in}}{\pgfqpoint{0.011785in}{1.011785in}}%
\pgfpathcurveto{\pgfqpoint{0.008660in}{1.014911in}}{\pgfqpoint{0.004420in}{1.016667in}}{\pgfqpoint{0.000000in}{1.016667in}}%
\pgfpathcurveto{\pgfqpoint{-0.004420in}{1.016667in}}{\pgfqpoint{-0.008660in}{1.014911in}}{\pgfqpoint{-0.011785in}{1.011785in}}%
\pgfpathcurveto{\pgfqpoint{-0.014911in}{1.008660in}}{\pgfqpoint{-0.016667in}{1.004420in}}{\pgfqpoint{-0.016667in}{1.000000in}}%
\pgfpathcurveto{\pgfqpoint{-0.016667in}{0.995580in}}{\pgfqpoint{-0.014911in}{0.991340in}}{\pgfqpoint{-0.011785in}{0.988215in}}%
\pgfpathcurveto{\pgfqpoint{-0.008660in}{0.985089in}}{\pgfqpoint{-0.004420in}{0.983333in}}{\pgfqpoint{0.000000in}{0.983333in}}%
\pgfpathclose%
\pgfpathmoveto{\pgfqpoint{0.166667in}{0.983333in}}%
\pgfpathcurveto{\pgfqpoint{0.171087in}{0.983333in}}{\pgfqpoint{0.175326in}{0.985089in}}{\pgfqpoint{0.178452in}{0.988215in}}%
\pgfpathcurveto{\pgfqpoint{0.181577in}{0.991340in}}{\pgfqpoint{0.183333in}{0.995580in}}{\pgfqpoint{0.183333in}{1.000000in}}%
\pgfpathcurveto{\pgfqpoint{0.183333in}{1.004420in}}{\pgfqpoint{0.181577in}{1.008660in}}{\pgfqpoint{0.178452in}{1.011785in}}%
\pgfpathcurveto{\pgfqpoint{0.175326in}{1.014911in}}{\pgfqpoint{0.171087in}{1.016667in}}{\pgfqpoint{0.166667in}{1.016667in}}%
\pgfpathcurveto{\pgfqpoint{0.162247in}{1.016667in}}{\pgfqpoint{0.158007in}{1.014911in}}{\pgfqpoint{0.154882in}{1.011785in}}%
\pgfpathcurveto{\pgfqpoint{0.151756in}{1.008660in}}{\pgfqpoint{0.150000in}{1.004420in}}{\pgfqpoint{0.150000in}{1.000000in}}%
\pgfpathcurveto{\pgfqpoint{0.150000in}{0.995580in}}{\pgfqpoint{0.151756in}{0.991340in}}{\pgfqpoint{0.154882in}{0.988215in}}%
\pgfpathcurveto{\pgfqpoint{0.158007in}{0.985089in}}{\pgfqpoint{0.162247in}{0.983333in}}{\pgfqpoint{0.166667in}{0.983333in}}%
\pgfpathclose%
\pgfpathmoveto{\pgfqpoint{0.333333in}{0.983333in}}%
\pgfpathcurveto{\pgfqpoint{0.337753in}{0.983333in}}{\pgfqpoint{0.341993in}{0.985089in}}{\pgfqpoint{0.345118in}{0.988215in}}%
\pgfpathcurveto{\pgfqpoint{0.348244in}{0.991340in}}{\pgfqpoint{0.350000in}{0.995580in}}{\pgfqpoint{0.350000in}{1.000000in}}%
\pgfpathcurveto{\pgfqpoint{0.350000in}{1.004420in}}{\pgfqpoint{0.348244in}{1.008660in}}{\pgfqpoint{0.345118in}{1.011785in}}%
\pgfpathcurveto{\pgfqpoint{0.341993in}{1.014911in}}{\pgfqpoint{0.337753in}{1.016667in}}{\pgfqpoint{0.333333in}{1.016667in}}%
\pgfpathcurveto{\pgfqpoint{0.328913in}{1.016667in}}{\pgfqpoint{0.324674in}{1.014911in}}{\pgfqpoint{0.321548in}{1.011785in}}%
\pgfpathcurveto{\pgfqpoint{0.318423in}{1.008660in}}{\pgfqpoint{0.316667in}{1.004420in}}{\pgfqpoint{0.316667in}{1.000000in}}%
\pgfpathcurveto{\pgfqpoint{0.316667in}{0.995580in}}{\pgfqpoint{0.318423in}{0.991340in}}{\pgfqpoint{0.321548in}{0.988215in}}%
\pgfpathcurveto{\pgfqpoint{0.324674in}{0.985089in}}{\pgfqpoint{0.328913in}{0.983333in}}{\pgfqpoint{0.333333in}{0.983333in}}%
\pgfpathclose%
\pgfpathmoveto{\pgfqpoint{0.500000in}{0.983333in}}%
\pgfpathcurveto{\pgfqpoint{0.504420in}{0.983333in}}{\pgfqpoint{0.508660in}{0.985089in}}{\pgfqpoint{0.511785in}{0.988215in}}%
\pgfpathcurveto{\pgfqpoint{0.514911in}{0.991340in}}{\pgfqpoint{0.516667in}{0.995580in}}{\pgfqpoint{0.516667in}{1.000000in}}%
\pgfpathcurveto{\pgfqpoint{0.516667in}{1.004420in}}{\pgfqpoint{0.514911in}{1.008660in}}{\pgfqpoint{0.511785in}{1.011785in}}%
\pgfpathcurveto{\pgfqpoint{0.508660in}{1.014911in}}{\pgfqpoint{0.504420in}{1.016667in}}{\pgfqpoint{0.500000in}{1.016667in}}%
\pgfpathcurveto{\pgfqpoint{0.495580in}{1.016667in}}{\pgfqpoint{0.491340in}{1.014911in}}{\pgfqpoint{0.488215in}{1.011785in}}%
\pgfpathcurveto{\pgfqpoint{0.485089in}{1.008660in}}{\pgfqpoint{0.483333in}{1.004420in}}{\pgfqpoint{0.483333in}{1.000000in}}%
\pgfpathcurveto{\pgfqpoint{0.483333in}{0.995580in}}{\pgfqpoint{0.485089in}{0.991340in}}{\pgfqpoint{0.488215in}{0.988215in}}%
\pgfpathcurveto{\pgfqpoint{0.491340in}{0.985089in}}{\pgfqpoint{0.495580in}{0.983333in}}{\pgfqpoint{0.500000in}{0.983333in}}%
\pgfpathclose%
\pgfpathmoveto{\pgfqpoint{0.666667in}{0.983333in}}%
\pgfpathcurveto{\pgfqpoint{0.671087in}{0.983333in}}{\pgfqpoint{0.675326in}{0.985089in}}{\pgfqpoint{0.678452in}{0.988215in}}%
\pgfpathcurveto{\pgfqpoint{0.681577in}{0.991340in}}{\pgfqpoint{0.683333in}{0.995580in}}{\pgfqpoint{0.683333in}{1.000000in}}%
\pgfpathcurveto{\pgfqpoint{0.683333in}{1.004420in}}{\pgfqpoint{0.681577in}{1.008660in}}{\pgfqpoint{0.678452in}{1.011785in}}%
\pgfpathcurveto{\pgfqpoint{0.675326in}{1.014911in}}{\pgfqpoint{0.671087in}{1.016667in}}{\pgfqpoint{0.666667in}{1.016667in}}%
\pgfpathcurveto{\pgfqpoint{0.662247in}{1.016667in}}{\pgfqpoint{0.658007in}{1.014911in}}{\pgfqpoint{0.654882in}{1.011785in}}%
\pgfpathcurveto{\pgfqpoint{0.651756in}{1.008660in}}{\pgfqpoint{0.650000in}{1.004420in}}{\pgfqpoint{0.650000in}{1.000000in}}%
\pgfpathcurveto{\pgfqpoint{0.650000in}{0.995580in}}{\pgfqpoint{0.651756in}{0.991340in}}{\pgfqpoint{0.654882in}{0.988215in}}%
\pgfpathcurveto{\pgfqpoint{0.658007in}{0.985089in}}{\pgfqpoint{0.662247in}{0.983333in}}{\pgfqpoint{0.666667in}{0.983333in}}%
\pgfpathclose%
\pgfpathmoveto{\pgfqpoint{0.833333in}{0.983333in}}%
\pgfpathcurveto{\pgfqpoint{0.837753in}{0.983333in}}{\pgfqpoint{0.841993in}{0.985089in}}{\pgfqpoint{0.845118in}{0.988215in}}%
\pgfpathcurveto{\pgfqpoint{0.848244in}{0.991340in}}{\pgfqpoint{0.850000in}{0.995580in}}{\pgfqpoint{0.850000in}{1.000000in}}%
\pgfpathcurveto{\pgfqpoint{0.850000in}{1.004420in}}{\pgfqpoint{0.848244in}{1.008660in}}{\pgfqpoint{0.845118in}{1.011785in}}%
\pgfpathcurveto{\pgfqpoint{0.841993in}{1.014911in}}{\pgfqpoint{0.837753in}{1.016667in}}{\pgfqpoint{0.833333in}{1.016667in}}%
\pgfpathcurveto{\pgfqpoint{0.828913in}{1.016667in}}{\pgfqpoint{0.824674in}{1.014911in}}{\pgfqpoint{0.821548in}{1.011785in}}%
\pgfpathcurveto{\pgfqpoint{0.818423in}{1.008660in}}{\pgfqpoint{0.816667in}{1.004420in}}{\pgfqpoint{0.816667in}{1.000000in}}%
\pgfpathcurveto{\pgfqpoint{0.816667in}{0.995580in}}{\pgfqpoint{0.818423in}{0.991340in}}{\pgfqpoint{0.821548in}{0.988215in}}%
\pgfpathcurveto{\pgfqpoint{0.824674in}{0.985089in}}{\pgfqpoint{0.828913in}{0.983333in}}{\pgfqpoint{0.833333in}{0.983333in}}%
\pgfpathclose%
\pgfpathmoveto{\pgfqpoint{1.000000in}{0.983333in}}%
\pgfpathcurveto{\pgfqpoint{1.004420in}{0.983333in}}{\pgfqpoint{1.008660in}{0.985089in}}{\pgfqpoint{1.011785in}{0.988215in}}%
\pgfpathcurveto{\pgfqpoint{1.014911in}{0.991340in}}{\pgfqpoint{1.016667in}{0.995580in}}{\pgfqpoint{1.016667in}{1.000000in}}%
\pgfpathcurveto{\pgfqpoint{1.016667in}{1.004420in}}{\pgfqpoint{1.014911in}{1.008660in}}{\pgfqpoint{1.011785in}{1.011785in}}%
\pgfpathcurveto{\pgfqpoint{1.008660in}{1.014911in}}{\pgfqpoint{1.004420in}{1.016667in}}{\pgfqpoint{1.000000in}{1.016667in}}%
\pgfpathcurveto{\pgfqpoint{0.995580in}{1.016667in}}{\pgfqpoint{0.991340in}{1.014911in}}{\pgfqpoint{0.988215in}{1.011785in}}%
\pgfpathcurveto{\pgfqpoint{0.985089in}{1.008660in}}{\pgfqpoint{0.983333in}{1.004420in}}{\pgfqpoint{0.983333in}{1.000000in}}%
\pgfpathcurveto{\pgfqpoint{0.983333in}{0.995580in}}{\pgfqpoint{0.985089in}{0.991340in}}{\pgfqpoint{0.988215in}{0.988215in}}%
\pgfpathcurveto{\pgfqpoint{0.991340in}{0.985089in}}{\pgfqpoint{0.995580in}{0.983333in}}{\pgfqpoint{1.000000in}{0.983333in}}%
\pgfpathclose%
\pgfusepath{stroke}%
\end{pgfscope}%
}%
\pgfsys@transformshift{2.808038in}{0.637495in}%
\end{pgfscope}%
\begin{pgfscope}%
\pgfpathrectangle{\pgfqpoint{0.870538in}{0.637495in}}{\pgfqpoint{9.300000in}{9.060000in}}%
\pgfusepath{clip}%
\pgfsetbuttcap%
\pgfsetmiterjoin%
\definecolor{currentfill}{rgb}{0.172549,0.627451,0.172549}%
\pgfsetfillcolor{currentfill}%
\pgfsetfillopacity{0.990000}%
\pgfsetlinewidth{0.000000pt}%
\definecolor{currentstroke}{rgb}{0.000000,0.000000,0.000000}%
\pgfsetstrokecolor{currentstroke}%
\pgfsetstrokeopacity{0.990000}%
\pgfsetdash{}{0pt}%
\pgfpathmoveto{\pgfqpoint{4.358038in}{0.637495in}}%
\pgfpathlineto{\pgfqpoint{5.133038in}{0.637495in}}%
\pgfpathlineto{\pgfqpoint{5.133038in}{0.637495in}}%
\pgfpathlineto{\pgfqpoint{4.358038in}{0.637495in}}%
\pgfpathclose%
\pgfusepath{fill}%
\end{pgfscope}%
\begin{pgfscope}%
\pgfsetbuttcap%
\pgfsetmiterjoin%
\definecolor{currentfill}{rgb}{0.172549,0.627451,0.172549}%
\pgfsetfillcolor{currentfill}%
\pgfsetfillopacity{0.990000}%
\pgfsetlinewidth{0.000000pt}%
\definecolor{currentstroke}{rgb}{0.000000,0.000000,0.000000}%
\pgfsetstrokecolor{currentstroke}%
\pgfsetstrokeopacity{0.990000}%
\pgfsetdash{}{0pt}%
\pgfpathrectangle{\pgfqpoint{0.870538in}{0.637495in}}{\pgfqpoint{9.300000in}{9.060000in}}%
\pgfusepath{clip}%
\pgfpathmoveto{\pgfqpoint{4.358038in}{0.637495in}}%
\pgfpathlineto{\pgfqpoint{5.133038in}{0.637495in}}%
\pgfpathlineto{\pgfqpoint{5.133038in}{0.637495in}}%
\pgfpathlineto{\pgfqpoint{4.358038in}{0.637495in}}%
\pgfpathclose%
\pgfusepath{clip}%
\pgfsys@defobject{currentpattern}{\pgfqpoint{0in}{0in}}{\pgfqpoint{1in}{1in}}{%
\begin{pgfscope}%
\pgfpathrectangle{\pgfqpoint{0in}{0in}}{\pgfqpoint{1in}{1in}}%
\pgfusepath{clip}%
\pgfpathmoveto{\pgfqpoint{0.000000in}{-0.016667in}}%
\pgfpathcurveto{\pgfqpoint{0.004420in}{-0.016667in}}{\pgfqpoint{0.008660in}{-0.014911in}}{\pgfqpoint{0.011785in}{-0.011785in}}%
\pgfpathcurveto{\pgfqpoint{0.014911in}{-0.008660in}}{\pgfqpoint{0.016667in}{-0.004420in}}{\pgfqpoint{0.016667in}{0.000000in}}%
\pgfpathcurveto{\pgfqpoint{0.016667in}{0.004420in}}{\pgfqpoint{0.014911in}{0.008660in}}{\pgfqpoint{0.011785in}{0.011785in}}%
\pgfpathcurveto{\pgfqpoint{0.008660in}{0.014911in}}{\pgfqpoint{0.004420in}{0.016667in}}{\pgfqpoint{0.000000in}{0.016667in}}%
\pgfpathcurveto{\pgfqpoint{-0.004420in}{0.016667in}}{\pgfqpoint{-0.008660in}{0.014911in}}{\pgfqpoint{-0.011785in}{0.011785in}}%
\pgfpathcurveto{\pgfqpoint{-0.014911in}{0.008660in}}{\pgfqpoint{-0.016667in}{0.004420in}}{\pgfqpoint{-0.016667in}{0.000000in}}%
\pgfpathcurveto{\pgfqpoint{-0.016667in}{-0.004420in}}{\pgfqpoint{-0.014911in}{-0.008660in}}{\pgfqpoint{-0.011785in}{-0.011785in}}%
\pgfpathcurveto{\pgfqpoint{-0.008660in}{-0.014911in}}{\pgfqpoint{-0.004420in}{-0.016667in}}{\pgfqpoint{0.000000in}{-0.016667in}}%
\pgfpathclose%
\pgfpathmoveto{\pgfqpoint{0.166667in}{-0.016667in}}%
\pgfpathcurveto{\pgfqpoint{0.171087in}{-0.016667in}}{\pgfqpoint{0.175326in}{-0.014911in}}{\pgfqpoint{0.178452in}{-0.011785in}}%
\pgfpathcurveto{\pgfqpoint{0.181577in}{-0.008660in}}{\pgfqpoint{0.183333in}{-0.004420in}}{\pgfqpoint{0.183333in}{0.000000in}}%
\pgfpathcurveto{\pgfqpoint{0.183333in}{0.004420in}}{\pgfqpoint{0.181577in}{0.008660in}}{\pgfqpoint{0.178452in}{0.011785in}}%
\pgfpathcurveto{\pgfqpoint{0.175326in}{0.014911in}}{\pgfqpoint{0.171087in}{0.016667in}}{\pgfqpoint{0.166667in}{0.016667in}}%
\pgfpathcurveto{\pgfqpoint{0.162247in}{0.016667in}}{\pgfqpoint{0.158007in}{0.014911in}}{\pgfqpoint{0.154882in}{0.011785in}}%
\pgfpathcurveto{\pgfqpoint{0.151756in}{0.008660in}}{\pgfqpoint{0.150000in}{0.004420in}}{\pgfqpoint{0.150000in}{0.000000in}}%
\pgfpathcurveto{\pgfqpoint{0.150000in}{-0.004420in}}{\pgfqpoint{0.151756in}{-0.008660in}}{\pgfqpoint{0.154882in}{-0.011785in}}%
\pgfpathcurveto{\pgfqpoint{0.158007in}{-0.014911in}}{\pgfqpoint{0.162247in}{-0.016667in}}{\pgfqpoint{0.166667in}{-0.016667in}}%
\pgfpathclose%
\pgfpathmoveto{\pgfqpoint{0.333333in}{-0.016667in}}%
\pgfpathcurveto{\pgfqpoint{0.337753in}{-0.016667in}}{\pgfqpoint{0.341993in}{-0.014911in}}{\pgfqpoint{0.345118in}{-0.011785in}}%
\pgfpathcurveto{\pgfqpoint{0.348244in}{-0.008660in}}{\pgfqpoint{0.350000in}{-0.004420in}}{\pgfqpoint{0.350000in}{0.000000in}}%
\pgfpathcurveto{\pgfqpoint{0.350000in}{0.004420in}}{\pgfqpoint{0.348244in}{0.008660in}}{\pgfqpoint{0.345118in}{0.011785in}}%
\pgfpathcurveto{\pgfqpoint{0.341993in}{0.014911in}}{\pgfqpoint{0.337753in}{0.016667in}}{\pgfqpoint{0.333333in}{0.016667in}}%
\pgfpathcurveto{\pgfqpoint{0.328913in}{0.016667in}}{\pgfqpoint{0.324674in}{0.014911in}}{\pgfqpoint{0.321548in}{0.011785in}}%
\pgfpathcurveto{\pgfqpoint{0.318423in}{0.008660in}}{\pgfqpoint{0.316667in}{0.004420in}}{\pgfqpoint{0.316667in}{0.000000in}}%
\pgfpathcurveto{\pgfqpoint{0.316667in}{-0.004420in}}{\pgfqpoint{0.318423in}{-0.008660in}}{\pgfqpoint{0.321548in}{-0.011785in}}%
\pgfpathcurveto{\pgfqpoint{0.324674in}{-0.014911in}}{\pgfqpoint{0.328913in}{-0.016667in}}{\pgfqpoint{0.333333in}{-0.016667in}}%
\pgfpathclose%
\pgfpathmoveto{\pgfqpoint{0.500000in}{-0.016667in}}%
\pgfpathcurveto{\pgfqpoint{0.504420in}{-0.016667in}}{\pgfqpoint{0.508660in}{-0.014911in}}{\pgfqpoint{0.511785in}{-0.011785in}}%
\pgfpathcurveto{\pgfqpoint{0.514911in}{-0.008660in}}{\pgfqpoint{0.516667in}{-0.004420in}}{\pgfqpoint{0.516667in}{0.000000in}}%
\pgfpathcurveto{\pgfqpoint{0.516667in}{0.004420in}}{\pgfqpoint{0.514911in}{0.008660in}}{\pgfqpoint{0.511785in}{0.011785in}}%
\pgfpathcurveto{\pgfqpoint{0.508660in}{0.014911in}}{\pgfqpoint{0.504420in}{0.016667in}}{\pgfqpoint{0.500000in}{0.016667in}}%
\pgfpathcurveto{\pgfqpoint{0.495580in}{0.016667in}}{\pgfqpoint{0.491340in}{0.014911in}}{\pgfqpoint{0.488215in}{0.011785in}}%
\pgfpathcurveto{\pgfqpoint{0.485089in}{0.008660in}}{\pgfqpoint{0.483333in}{0.004420in}}{\pgfqpoint{0.483333in}{0.000000in}}%
\pgfpathcurveto{\pgfqpoint{0.483333in}{-0.004420in}}{\pgfqpoint{0.485089in}{-0.008660in}}{\pgfqpoint{0.488215in}{-0.011785in}}%
\pgfpathcurveto{\pgfqpoint{0.491340in}{-0.014911in}}{\pgfqpoint{0.495580in}{-0.016667in}}{\pgfqpoint{0.500000in}{-0.016667in}}%
\pgfpathclose%
\pgfpathmoveto{\pgfqpoint{0.666667in}{-0.016667in}}%
\pgfpathcurveto{\pgfqpoint{0.671087in}{-0.016667in}}{\pgfqpoint{0.675326in}{-0.014911in}}{\pgfqpoint{0.678452in}{-0.011785in}}%
\pgfpathcurveto{\pgfqpoint{0.681577in}{-0.008660in}}{\pgfqpoint{0.683333in}{-0.004420in}}{\pgfqpoint{0.683333in}{0.000000in}}%
\pgfpathcurveto{\pgfqpoint{0.683333in}{0.004420in}}{\pgfqpoint{0.681577in}{0.008660in}}{\pgfqpoint{0.678452in}{0.011785in}}%
\pgfpathcurveto{\pgfqpoint{0.675326in}{0.014911in}}{\pgfqpoint{0.671087in}{0.016667in}}{\pgfqpoint{0.666667in}{0.016667in}}%
\pgfpathcurveto{\pgfqpoint{0.662247in}{0.016667in}}{\pgfqpoint{0.658007in}{0.014911in}}{\pgfqpoint{0.654882in}{0.011785in}}%
\pgfpathcurveto{\pgfqpoint{0.651756in}{0.008660in}}{\pgfqpoint{0.650000in}{0.004420in}}{\pgfqpoint{0.650000in}{0.000000in}}%
\pgfpathcurveto{\pgfqpoint{0.650000in}{-0.004420in}}{\pgfqpoint{0.651756in}{-0.008660in}}{\pgfqpoint{0.654882in}{-0.011785in}}%
\pgfpathcurveto{\pgfqpoint{0.658007in}{-0.014911in}}{\pgfqpoint{0.662247in}{-0.016667in}}{\pgfqpoint{0.666667in}{-0.016667in}}%
\pgfpathclose%
\pgfpathmoveto{\pgfqpoint{0.833333in}{-0.016667in}}%
\pgfpathcurveto{\pgfqpoint{0.837753in}{-0.016667in}}{\pgfqpoint{0.841993in}{-0.014911in}}{\pgfqpoint{0.845118in}{-0.011785in}}%
\pgfpathcurveto{\pgfqpoint{0.848244in}{-0.008660in}}{\pgfqpoint{0.850000in}{-0.004420in}}{\pgfqpoint{0.850000in}{0.000000in}}%
\pgfpathcurveto{\pgfqpoint{0.850000in}{0.004420in}}{\pgfqpoint{0.848244in}{0.008660in}}{\pgfqpoint{0.845118in}{0.011785in}}%
\pgfpathcurveto{\pgfqpoint{0.841993in}{0.014911in}}{\pgfqpoint{0.837753in}{0.016667in}}{\pgfqpoint{0.833333in}{0.016667in}}%
\pgfpathcurveto{\pgfqpoint{0.828913in}{0.016667in}}{\pgfqpoint{0.824674in}{0.014911in}}{\pgfqpoint{0.821548in}{0.011785in}}%
\pgfpathcurveto{\pgfqpoint{0.818423in}{0.008660in}}{\pgfqpoint{0.816667in}{0.004420in}}{\pgfqpoint{0.816667in}{0.000000in}}%
\pgfpathcurveto{\pgfqpoint{0.816667in}{-0.004420in}}{\pgfqpoint{0.818423in}{-0.008660in}}{\pgfqpoint{0.821548in}{-0.011785in}}%
\pgfpathcurveto{\pgfqpoint{0.824674in}{-0.014911in}}{\pgfqpoint{0.828913in}{-0.016667in}}{\pgfqpoint{0.833333in}{-0.016667in}}%
\pgfpathclose%
\pgfpathmoveto{\pgfqpoint{1.000000in}{-0.016667in}}%
\pgfpathcurveto{\pgfqpoint{1.004420in}{-0.016667in}}{\pgfqpoint{1.008660in}{-0.014911in}}{\pgfqpoint{1.011785in}{-0.011785in}}%
\pgfpathcurveto{\pgfqpoint{1.014911in}{-0.008660in}}{\pgfqpoint{1.016667in}{-0.004420in}}{\pgfqpoint{1.016667in}{0.000000in}}%
\pgfpathcurveto{\pgfqpoint{1.016667in}{0.004420in}}{\pgfqpoint{1.014911in}{0.008660in}}{\pgfqpoint{1.011785in}{0.011785in}}%
\pgfpathcurveto{\pgfqpoint{1.008660in}{0.014911in}}{\pgfqpoint{1.004420in}{0.016667in}}{\pgfqpoint{1.000000in}{0.016667in}}%
\pgfpathcurveto{\pgfqpoint{0.995580in}{0.016667in}}{\pgfqpoint{0.991340in}{0.014911in}}{\pgfqpoint{0.988215in}{0.011785in}}%
\pgfpathcurveto{\pgfqpoint{0.985089in}{0.008660in}}{\pgfqpoint{0.983333in}{0.004420in}}{\pgfqpoint{0.983333in}{0.000000in}}%
\pgfpathcurveto{\pgfqpoint{0.983333in}{-0.004420in}}{\pgfqpoint{0.985089in}{-0.008660in}}{\pgfqpoint{0.988215in}{-0.011785in}}%
\pgfpathcurveto{\pgfqpoint{0.991340in}{-0.014911in}}{\pgfqpoint{0.995580in}{-0.016667in}}{\pgfqpoint{1.000000in}{-0.016667in}}%
\pgfpathclose%
\pgfpathmoveto{\pgfqpoint{0.083333in}{0.150000in}}%
\pgfpathcurveto{\pgfqpoint{0.087753in}{0.150000in}}{\pgfqpoint{0.091993in}{0.151756in}}{\pgfqpoint{0.095118in}{0.154882in}}%
\pgfpathcurveto{\pgfqpoint{0.098244in}{0.158007in}}{\pgfqpoint{0.100000in}{0.162247in}}{\pgfqpoint{0.100000in}{0.166667in}}%
\pgfpathcurveto{\pgfqpoint{0.100000in}{0.171087in}}{\pgfqpoint{0.098244in}{0.175326in}}{\pgfqpoint{0.095118in}{0.178452in}}%
\pgfpathcurveto{\pgfqpoint{0.091993in}{0.181577in}}{\pgfqpoint{0.087753in}{0.183333in}}{\pgfqpoint{0.083333in}{0.183333in}}%
\pgfpathcurveto{\pgfqpoint{0.078913in}{0.183333in}}{\pgfqpoint{0.074674in}{0.181577in}}{\pgfqpoint{0.071548in}{0.178452in}}%
\pgfpathcurveto{\pgfqpoint{0.068423in}{0.175326in}}{\pgfqpoint{0.066667in}{0.171087in}}{\pgfqpoint{0.066667in}{0.166667in}}%
\pgfpathcurveto{\pgfqpoint{0.066667in}{0.162247in}}{\pgfqpoint{0.068423in}{0.158007in}}{\pgfqpoint{0.071548in}{0.154882in}}%
\pgfpathcurveto{\pgfqpoint{0.074674in}{0.151756in}}{\pgfqpoint{0.078913in}{0.150000in}}{\pgfqpoint{0.083333in}{0.150000in}}%
\pgfpathclose%
\pgfpathmoveto{\pgfqpoint{0.250000in}{0.150000in}}%
\pgfpathcurveto{\pgfqpoint{0.254420in}{0.150000in}}{\pgfqpoint{0.258660in}{0.151756in}}{\pgfqpoint{0.261785in}{0.154882in}}%
\pgfpathcurveto{\pgfqpoint{0.264911in}{0.158007in}}{\pgfqpoint{0.266667in}{0.162247in}}{\pgfqpoint{0.266667in}{0.166667in}}%
\pgfpathcurveto{\pgfqpoint{0.266667in}{0.171087in}}{\pgfqpoint{0.264911in}{0.175326in}}{\pgfqpoint{0.261785in}{0.178452in}}%
\pgfpathcurveto{\pgfqpoint{0.258660in}{0.181577in}}{\pgfqpoint{0.254420in}{0.183333in}}{\pgfqpoint{0.250000in}{0.183333in}}%
\pgfpathcurveto{\pgfqpoint{0.245580in}{0.183333in}}{\pgfqpoint{0.241340in}{0.181577in}}{\pgfqpoint{0.238215in}{0.178452in}}%
\pgfpathcurveto{\pgfqpoint{0.235089in}{0.175326in}}{\pgfqpoint{0.233333in}{0.171087in}}{\pgfqpoint{0.233333in}{0.166667in}}%
\pgfpathcurveto{\pgfqpoint{0.233333in}{0.162247in}}{\pgfqpoint{0.235089in}{0.158007in}}{\pgfqpoint{0.238215in}{0.154882in}}%
\pgfpathcurveto{\pgfqpoint{0.241340in}{0.151756in}}{\pgfqpoint{0.245580in}{0.150000in}}{\pgfqpoint{0.250000in}{0.150000in}}%
\pgfpathclose%
\pgfpathmoveto{\pgfqpoint{0.416667in}{0.150000in}}%
\pgfpathcurveto{\pgfqpoint{0.421087in}{0.150000in}}{\pgfqpoint{0.425326in}{0.151756in}}{\pgfqpoint{0.428452in}{0.154882in}}%
\pgfpathcurveto{\pgfqpoint{0.431577in}{0.158007in}}{\pgfqpoint{0.433333in}{0.162247in}}{\pgfqpoint{0.433333in}{0.166667in}}%
\pgfpathcurveto{\pgfqpoint{0.433333in}{0.171087in}}{\pgfqpoint{0.431577in}{0.175326in}}{\pgfqpoint{0.428452in}{0.178452in}}%
\pgfpathcurveto{\pgfqpoint{0.425326in}{0.181577in}}{\pgfqpoint{0.421087in}{0.183333in}}{\pgfqpoint{0.416667in}{0.183333in}}%
\pgfpathcurveto{\pgfqpoint{0.412247in}{0.183333in}}{\pgfqpoint{0.408007in}{0.181577in}}{\pgfqpoint{0.404882in}{0.178452in}}%
\pgfpathcurveto{\pgfqpoint{0.401756in}{0.175326in}}{\pgfqpoint{0.400000in}{0.171087in}}{\pgfqpoint{0.400000in}{0.166667in}}%
\pgfpathcurveto{\pgfqpoint{0.400000in}{0.162247in}}{\pgfqpoint{0.401756in}{0.158007in}}{\pgfqpoint{0.404882in}{0.154882in}}%
\pgfpathcurveto{\pgfqpoint{0.408007in}{0.151756in}}{\pgfqpoint{0.412247in}{0.150000in}}{\pgfqpoint{0.416667in}{0.150000in}}%
\pgfpathclose%
\pgfpathmoveto{\pgfqpoint{0.583333in}{0.150000in}}%
\pgfpathcurveto{\pgfqpoint{0.587753in}{0.150000in}}{\pgfqpoint{0.591993in}{0.151756in}}{\pgfqpoint{0.595118in}{0.154882in}}%
\pgfpathcurveto{\pgfqpoint{0.598244in}{0.158007in}}{\pgfqpoint{0.600000in}{0.162247in}}{\pgfqpoint{0.600000in}{0.166667in}}%
\pgfpathcurveto{\pgfqpoint{0.600000in}{0.171087in}}{\pgfqpoint{0.598244in}{0.175326in}}{\pgfqpoint{0.595118in}{0.178452in}}%
\pgfpathcurveto{\pgfqpoint{0.591993in}{0.181577in}}{\pgfqpoint{0.587753in}{0.183333in}}{\pgfqpoint{0.583333in}{0.183333in}}%
\pgfpathcurveto{\pgfqpoint{0.578913in}{0.183333in}}{\pgfqpoint{0.574674in}{0.181577in}}{\pgfqpoint{0.571548in}{0.178452in}}%
\pgfpathcurveto{\pgfqpoint{0.568423in}{0.175326in}}{\pgfqpoint{0.566667in}{0.171087in}}{\pgfqpoint{0.566667in}{0.166667in}}%
\pgfpathcurveto{\pgfqpoint{0.566667in}{0.162247in}}{\pgfqpoint{0.568423in}{0.158007in}}{\pgfqpoint{0.571548in}{0.154882in}}%
\pgfpathcurveto{\pgfqpoint{0.574674in}{0.151756in}}{\pgfqpoint{0.578913in}{0.150000in}}{\pgfqpoint{0.583333in}{0.150000in}}%
\pgfpathclose%
\pgfpathmoveto{\pgfqpoint{0.750000in}{0.150000in}}%
\pgfpathcurveto{\pgfqpoint{0.754420in}{0.150000in}}{\pgfqpoint{0.758660in}{0.151756in}}{\pgfqpoint{0.761785in}{0.154882in}}%
\pgfpathcurveto{\pgfqpoint{0.764911in}{0.158007in}}{\pgfqpoint{0.766667in}{0.162247in}}{\pgfqpoint{0.766667in}{0.166667in}}%
\pgfpathcurveto{\pgfqpoint{0.766667in}{0.171087in}}{\pgfqpoint{0.764911in}{0.175326in}}{\pgfqpoint{0.761785in}{0.178452in}}%
\pgfpathcurveto{\pgfqpoint{0.758660in}{0.181577in}}{\pgfqpoint{0.754420in}{0.183333in}}{\pgfqpoint{0.750000in}{0.183333in}}%
\pgfpathcurveto{\pgfqpoint{0.745580in}{0.183333in}}{\pgfqpoint{0.741340in}{0.181577in}}{\pgfqpoint{0.738215in}{0.178452in}}%
\pgfpathcurveto{\pgfqpoint{0.735089in}{0.175326in}}{\pgfqpoint{0.733333in}{0.171087in}}{\pgfqpoint{0.733333in}{0.166667in}}%
\pgfpathcurveto{\pgfqpoint{0.733333in}{0.162247in}}{\pgfqpoint{0.735089in}{0.158007in}}{\pgfqpoint{0.738215in}{0.154882in}}%
\pgfpathcurveto{\pgfqpoint{0.741340in}{0.151756in}}{\pgfqpoint{0.745580in}{0.150000in}}{\pgfqpoint{0.750000in}{0.150000in}}%
\pgfpathclose%
\pgfpathmoveto{\pgfqpoint{0.916667in}{0.150000in}}%
\pgfpathcurveto{\pgfqpoint{0.921087in}{0.150000in}}{\pgfqpoint{0.925326in}{0.151756in}}{\pgfqpoint{0.928452in}{0.154882in}}%
\pgfpathcurveto{\pgfqpoint{0.931577in}{0.158007in}}{\pgfqpoint{0.933333in}{0.162247in}}{\pgfqpoint{0.933333in}{0.166667in}}%
\pgfpathcurveto{\pgfqpoint{0.933333in}{0.171087in}}{\pgfqpoint{0.931577in}{0.175326in}}{\pgfqpoint{0.928452in}{0.178452in}}%
\pgfpathcurveto{\pgfqpoint{0.925326in}{0.181577in}}{\pgfqpoint{0.921087in}{0.183333in}}{\pgfqpoint{0.916667in}{0.183333in}}%
\pgfpathcurveto{\pgfqpoint{0.912247in}{0.183333in}}{\pgfqpoint{0.908007in}{0.181577in}}{\pgfqpoint{0.904882in}{0.178452in}}%
\pgfpathcurveto{\pgfqpoint{0.901756in}{0.175326in}}{\pgfqpoint{0.900000in}{0.171087in}}{\pgfqpoint{0.900000in}{0.166667in}}%
\pgfpathcurveto{\pgfqpoint{0.900000in}{0.162247in}}{\pgfqpoint{0.901756in}{0.158007in}}{\pgfqpoint{0.904882in}{0.154882in}}%
\pgfpathcurveto{\pgfqpoint{0.908007in}{0.151756in}}{\pgfqpoint{0.912247in}{0.150000in}}{\pgfqpoint{0.916667in}{0.150000in}}%
\pgfpathclose%
\pgfpathmoveto{\pgfqpoint{0.000000in}{0.316667in}}%
\pgfpathcurveto{\pgfqpoint{0.004420in}{0.316667in}}{\pgfqpoint{0.008660in}{0.318423in}}{\pgfqpoint{0.011785in}{0.321548in}}%
\pgfpathcurveto{\pgfqpoint{0.014911in}{0.324674in}}{\pgfqpoint{0.016667in}{0.328913in}}{\pgfqpoint{0.016667in}{0.333333in}}%
\pgfpathcurveto{\pgfqpoint{0.016667in}{0.337753in}}{\pgfqpoint{0.014911in}{0.341993in}}{\pgfqpoint{0.011785in}{0.345118in}}%
\pgfpathcurveto{\pgfqpoint{0.008660in}{0.348244in}}{\pgfqpoint{0.004420in}{0.350000in}}{\pgfqpoint{0.000000in}{0.350000in}}%
\pgfpathcurveto{\pgfqpoint{-0.004420in}{0.350000in}}{\pgfqpoint{-0.008660in}{0.348244in}}{\pgfqpoint{-0.011785in}{0.345118in}}%
\pgfpathcurveto{\pgfqpoint{-0.014911in}{0.341993in}}{\pgfqpoint{-0.016667in}{0.337753in}}{\pgfqpoint{-0.016667in}{0.333333in}}%
\pgfpathcurveto{\pgfqpoint{-0.016667in}{0.328913in}}{\pgfqpoint{-0.014911in}{0.324674in}}{\pgfqpoint{-0.011785in}{0.321548in}}%
\pgfpathcurveto{\pgfqpoint{-0.008660in}{0.318423in}}{\pgfqpoint{-0.004420in}{0.316667in}}{\pgfqpoint{0.000000in}{0.316667in}}%
\pgfpathclose%
\pgfpathmoveto{\pgfqpoint{0.166667in}{0.316667in}}%
\pgfpathcurveto{\pgfqpoint{0.171087in}{0.316667in}}{\pgfqpoint{0.175326in}{0.318423in}}{\pgfqpoint{0.178452in}{0.321548in}}%
\pgfpathcurveto{\pgfqpoint{0.181577in}{0.324674in}}{\pgfqpoint{0.183333in}{0.328913in}}{\pgfqpoint{0.183333in}{0.333333in}}%
\pgfpathcurveto{\pgfqpoint{0.183333in}{0.337753in}}{\pgfqpoint{0.181577in}{0.341993in}}{\pgfqpoint{0.178452in}{0.345118in}}%
\pgfpathcurveto{\pgfqpoint{0.175326in}{0.348244in}}{\pgfqpoint{0.171087in}{0.350000in}}{\pgfqpoint{0.166667in}{0.350000in}}%
\pgfpathcurveto{\pgfqpoint{0.162247in}{0.350000in}}{\pgfqpoint{0.158007in}{0.348244in}}{\pgfqpoint{0.154882in}{0.345118in}}%
\pgfpathcurveto{\pgfqpoint{0.151756in}{0.341993in}}{\pgfqpoint{0.150000in}{0.337753in}}{\pgfqpoint{0.150000in}{0.333333in}}%
\pgfpathcurveto{\pgfqpoint{0.150000in}{0.328913in}}{\pgfqpoint{0.151756in}{0.324674in}}{\pgfqpoint{0.154882in}{0.321548in}}%
\pgfpathcurveto{\pgfqpoint{0.158007in}{0.318423in}}{\pgfqpoint{0.162247in}{0.316667in}}{\pgfqpoint{0.166667in}{0.316667in}}%
\pgfpathclose%
\pgfpathmoveto{\pgfqpoint{0.333333in}{0.316667in}}%
\pgfpathcurveto{\pgfqpoint{0.337753in}{0.316667in}}{\pgfqpoint{0.341993in}{0.318423in}}{\pgfqpoint{0.345118in}{0.321548in}}%
\pgfpathcurveto{\pgfqpoint{0.348244in}{0.324674in}}{\pgfqpoint{0.350000in}{0.328913in}}{\pgfqpoint{0.350000in}{0.333333in}}%
\pgfpathcurveto{\pgfqpoint{0.350000in}{0.337753in}}{\pgfqpoint{0.348244in}{0.341993in}}{\pgfqpoint{0.345118in}{0.345118in}}%
\pgfpathcurveto{\pgfqpoint{0.341993in}{0.348244in}}{\pgfqpoint{0.337753in}{0.350000in}}{\pgfqpoint{0.333333in}{0.350000in}}%
\pgfpathcurveto{\pgfqpoint{0.328913in}{0.350000in}}{\pgfqpoint{0.324674in}{0.348244in}}{\pgfqpoint{0.321548in}{0.345118in}}%
\pgfpathcurveto{\pgfqpoint{0.318423in}{0.341993in}}{\pgfqpoint{0.316667in}{0.337753in}}{\pgfqpoint{0.316667in}{0.333333in}}%
\pgfpathcurveto{\pgfqpoint{0.316667in}{0.328913in}}{\pgfqpoint{0.318423in}{0.324674in}}{\pgfqpoint{0.321548in}{0.321548in}}%
\pgfpathcurveto{\pgfqpoint{0.324674in}{0.318423in}}{\pgfqpoint{0.328913in}{0.316667in}}{\pgfqpoint{0.333333in}{0.316667in}}%
\pgfpathclose%
\pgfpathmoveto{\pgfqpoint{0.500000in}{0.316667in}}%
\pgfpathcurveto{\pgfqpoint{0.504420in}{0.316667in}}{\pgfqpoint{0.508660in}{0.318423in}}{\pgfqpoint{0.511785in}{0.321548in}}%
\pgfpathcurveto{\pgfqpoint{0.514911in}{0.324674in}}{\pgfqpoint{0.516667in}{0.328913in}}{\pgfqpoint{0.516667in}{0.333333in}}%
\pgfpathcurveto{\pgfqpoint{0.516667in}{0.337753in}}{\pgfqpoint{0.514911in}{0.341993in}}{\pgfqpoint{0.511785in}{0.345118in}}%
\pgfpathcurveto{\pgfqpoint{0.508660in}{0.348244in}}{\pgfqpoint{0.504420in}{0.350000in}}{\pgfqpoint{0.500000in}{0.350000in}}%
\pgfpathcurveto{\pgfqpoint{0.495580in}{0.350000in}}{\pgfqpoint{0.491340in}{0.348244in}}{\pgfqpoint{0.488215in}{0.345118in}}%
\pgfpathcurveto{\pgfqpoint{0.485089in}{0.341993in}}{\pgfqpoint{0.483333in}{0.337753in}}{\pgfqpoint{0.483333in}{0.333333in}}%
\pgfpathcurveto{\pgfqpoint{0.483333in}{0.328913in}}{\pgfqpoint{0.485089in}{0.324674in}}{\pgfqpoint{0.488215in}{0.321548in}}%
\pgfpathcurveto{\pgfqpoint{0.491340in}{0.318423in}}{\pgfqpoint{0.495580in}{0.316667in}}{\pgfqpoint{0.500000in}{0.316667in}}%
\pgfpathclose%
\pgfpathmoveto{\pgfqpoint{0.666667in}{0.316667in}}%
\pgfpathcurveto{\pgfqpoint{0.671087in}{0.316667in}}{\pgfqpoint{0.675326in}{0.318423in}}{\pgfqpoint{0.678452in}{0.321548in}}%
\pgfpathcurveto{\pgfqpoint{0.681577in}{0.324674in}}{\pgfqpoint{0.683333in}{0.328913in}}{\pgfqpoint{0.683333in}{0.333333in}}%
\pgfpathcurveto{\pgfqpoint{0.683333in}{0.337753in}}{\pgfqpoint{0.681577in}{0.341993in}}{\pgfqpoint{0.678452in}{0.345118in}}%
\pgfpathcurveto{\pgfqpoint{0.675326in}{0.348244in}}{\pgfqpoint{0.671087in}{0.350000in}}{\pgfqpoint{0.666667in}{0.350000in}}%
\pgfpathcurveto{\pgfqpoint{0.662247in}{0.350000in}}{\pgfqpoint{0.658007in}{0.348244in}}{\pgfqpoint{0.654882in}{0.345118in}}%
\pgfpathcurveto{\pgfqpoint{0.651756in}{0.341993in}}{\pgfqpoint{0.650000in}{0.337753in}}{\pgfqpoint{0.650000in}{0.333333in}}%
\pgfpathcurveto{\pgfqpoint{0.650000in}{0.328913in}}{\pgfqpoint{0.651756in}{0.324674in}}{\pgfqpoint{0.654882in}{0.321548in}}%
\pgfpathcurveto{\pgfqpoint{0.658007in}{0.318423in}}{\pgfqpoint{0.662247in}{0.316667in}}{\pgfqpoint{0.666667in}{0.316667in}}%
\pgfpathclose%
\pgfpathmoveto{\pgfqpoint{0.833333in}{0.316667in}}%
\pgfpathcurveto{\pgfqpoint{0.837753in}{0.316667in}}{\pgfqpoint{0.841993in}{0.318423in}}{\pgfqpoint{0.845118in}{0.321548in}}%
\pgfpathcurveto{\pgfqpoint{0.848244in}{0.324674in}}{\pgfqpoint{0.850000in}{0.328913in}}{\pgfqpoint{0.850000in}{0.333333in}}%
\pgfpathcurveto{\pgfqpoint{0.850000in}{0.337753in}}{\pgfqpoint{0.848244in}{0.341993in}}{\pgfqpoint{0.845118in}{0.345118in}}%
\pgfpathcurveto{\pgfqpoint{0.841993in}{0.348244in}}{\pgfqpoint{0.837753in}{0.350000in}}{\pgfqpoint{0.833333in}{0.350000in}}%
\pgfpathcurveto{\pgfqpoint{0.828913in}{0.350000in}}{\pgfqpoint{0.824674in}{0.348244in}}{\pgfqpoint{0.821548in}{0.345118in}}%
\pgfpathcurveto{\pgfqpoint{0.818423in}{0.341993in}}{\pgfqpoint{0.816667in}{0.337753in}}{\pgfqpoint{0.816667in}{0.333333in}}%
\pgfpathcurveto{\pgfqpoint{0.816667in}{0.328913in}}{\pgfqpoint{0.818423in}{0.324674in}}{\pgfqpoint{0.821548in}{0.321548in}}%
\pgfpathcurveto{\pgfqpoint{0.824674in}{0.318423in}}{\pgfqpoint{0.828913in}{0.316667in}}{\pgfqpoint{0.833333in}{0.316667in}}%
\pgfpathclose%
\pgfpathmoveto{\pgfqpoint{1.000000in}{0.316667in}}%
\pgfpathcurveto{\pgfqpoint{1.004420in}{0.316667in}}{\pgfqpoint{1.008660in}{0.318423in}}{\pgfqpoint{1.011785in}{0.321548in}}%
\pgfpathcurveto{\pgfqpoint{1.014911in}{0.324674in}}{\pgfqpoint{1.016667in}{0.328913in}}{\pgfqpoint{1.016667in}{0.333333in}}%
\pgfpathcurveto{\pgfqpoint{1.016667in}{0.337753in}}{\pgfqpoint{1.014911in}{0.341993in}}{\pgfqpoint{1.011785in}{0.345118in}}%
\pgfpathcurveto{\pgfqpoint{1.008660in}{0.348244in}}{\pgfqpoint{1.004420in}{0.350000in}}{\pgfqpoint{1.000000in}{0.350000in}}%
\pgfpathcurveto{\pgfqpoint{0.995580in}{0.350000in}}{\pgfqpoint{0.991340in}{0.348244in}}{\pgfqpoint{0.988215in}{0.345118in}}%
\pgfpathcurveto{\pgfqpoint{0.985089in}{0.341993in}}{\pgfqpoint{0.983333in}{0.337753in}}{\pgfqpoint{0.983333in}{0.333333in}}%
\pgfpathcurveto{\pgfqpoint{0.983333in}{0.328913in}}{\pgfqpoint{0.985089in}{0.324674in}}{\pgfqpoint{0.988215in}{0.321548in}}%
\pgfpathcurveto{\pgfqpoint{0.991340in}{0.318423in}}{\pgfqpoint{0.995580in}{0.316667in}}{\pgfqpoint{1.000000in}{0.316667in}}%
\pgfpathclose%
\pgfpathmoveto{\pgfqpoint{0.083333in}{0.483333in}}%
\pgfpathcurveto{\pgfqpoint{0.087753in}{0.483333in}}{\pgfqpoint{0.091993in}{0.485089in}}{\pgfqpoint{0.095118in}{0.488215in}}%
\pgfpathcurveto{\pgfqpoint{0.098244in}{0.491340in}}{\pgfqpoint{0.100000in}{0.495580in}}{\pgfqpoint{0.100000in}{0.500000in}}%
\pgfpathcurveto{\pgfqpoint{0.100000in}{0.504420in}}{\pgfqpoint{0.098244in}{0.508660in}}{\pgfqpoint{0.095118in}{0.511785in}}%
\pgfpathcurveto{\pgfqpoint{0.091993in}{0.514911in}}{\pgfqpoint{0.087753in}{0.516667in}}{\pgfqpoint{0.083333in}{0.516667in}}%
\pgfpathcurveto{\pgfqpoint{0.078913in}{0.516667in}}{\pgfqpoint{0.074674in}{0.514911in}}{\pgfqpoint{0.071548in}{0.511785in}}%
\pgfpathcurveto{\pgfqpoint{0.068423in}{0.508660in}}{\pgfqpoint{0.066667in}{0.504420in}}{\pgfqpoint{0.066667in}{0.500000in}}%
\pgfpathcurveto{\pgfqpoint{0.066667in}{0.495580in}}{\pgfqpoint{0.068423in}{0.491340in}}{\pgfqpoint{0.071548in}{0.488215in}}%
\pgfpathcurveto{\pgfqpoint{0.074674in}{0.485089in}}{\pgfqpoint{0.078913in}{0.483333in}}{\pgfqpoint{0.083333in}{0.483333in}}%
\pgfpathclose%
\pgfpathmoveto{\pgfqpoint{0.250000in}{0.483333in}}%
\pgfpathcurveto{\pgfqpoint{0.254420in}{0.483333in}}{\pgfqpoint{0.258660in}{0.485089in}}{\pgfqpoint{0.261785in}{0.488215in}}%
\pgfpathcurveto{\pgfqpoint{0.264911in}{0.491340in}}{\pgfqpoint{0.266667in}{0.495580in}}{\pgfqpoint{0.266667in}{0.500000in}}%
\pgfpathcurveto{\pgfqpoint{0.266667in}{0.504420in}}{\pgfqpoint{0.264911in}{0.508660in}}{\pgfqpoint{0.261785in}{0.511785in}}%
\pgfpathcurveto{\pgfqpoint{0.258660in}{0.514911in}}{\pgfqpoint{0.254420in}{0.516667in}}{\pgfqpoint{0.250000in}{0.516667in}}%
\pgfpathcurveto{\pgfqpoint{0.245580in}{0.516667in}}{\pgfqpoint{0.241340in}{0.514911in}}{\pgfqpoint{0.238215in}{0.511785in}}%
\pgfpathcurveto{\pgfqpoint{0.235089in}{0.508660in}}{\pgfqpoint{0.233333in}{0.504420in}}{\pgfqpoint{0.233333in}{0.500000in}}%
\pgfpathcurveto{\pgfqpoint{0.233333in}{0.495580in}}{\pgfqpoint{0.235089in}{0.491340in}}{\pgfqpoint{0.238215in}{0.488215in}}%
\pgfpathcurveto{\pgfqpoint{0.241340in}{0.485089in}}{\pgfqpoint{0.245580in}{0.483333in}}{\pgfqpoint{0.250000in}{0.483333in}}%
\pgfpathclose%
\pgfpathmoveto{\pgfqpoint{0.416667in}{0.483333in}}%
\pgfpathcurveto{\pgfqpoint{0.421087in}{0.483333in}}{\pgfqpoint{0.425326in}{0.485089in}}{\pgfqpoint{0.428452in}{0.488215in}}%
\pgfpathcurveto{\pgfqpoint{0.431577in}{0.491340in}}{\pgfqpoint{0.433333in}{0.495580in}}{\pgfqpoint{0.433333in}{0.500000in}}%
\pgfpathcurveto{\pgfqpoint{0.433333in}{0.504420in}}{\pgfqpoint{0.431577in}{0.508660in}}{\pgfqpoint{0.428452in}{0.511785in}}%
\pgfpathcurveto{\pgfqpoint{0.425326in}{0.514911in}}{\pgfqpoint{0.421087in}{0.516667in}}{\pgfqpoint{0.416667in}{0.516667in}}%
\pgfpathcurveto{\pgfqpoint{0.412247in}{0.516667in}}{\pgfqpoint{0.408007in}{0.514911in}}{\pgfqpoint{0.404882in}{0.511785in}}%
\pgfpathcurveto{\pgfqpoint{0.401756in}{0.508660in}}{\pgfqpoint{0.400000in}{0.504420in}}{\pgfqpoint{0.400000in}{0.500000in}}%
\pgfpathcurveto{\pgfqpoint{0.400000in}{0.495580in}}{\pgfqpoint{0.401756in}{0.491340in}}{\pgfqpoint{0.404882in}{0.488215in}}%
\pgfpathcurveto{\pgfqpoint{0.408007in}{0.485089in}}{\pgfqpoint{0.412247in}{0.483333in}}{\pgfqpoint{0.416667in}{0.483333in}}%
\pgfpathclose%
\pgfpathmoveto{\pgfqpoint{0.583333in}{0.483333in}}%
\pgfpathcurveto{\pgfqpoint{0.587753in}{0.483333in}}{\pgfqpoint{0.591993in}{0.485089in}}{\pgfqpoint{0.595118in}{0.488215in}}%
\pgfpathcurveto{\pgfqpoint{0.598244in}{0.491340in}}{\pgfqpoint{0.600000in}{0.495580in}}{\pgfqpoint{0.600000in}{0.500000in}}%
\pgfpathcurveto{\pgfqpoint{0.600000in}{0.504420in}}{\pgfqpoint{0.598244in}{0.508660in}}{\pgfqpoint{0.595118in}{0.511785in}}%
\pgfpathcurveto{\pgfqpoint{0.591993in}{0.514911in}}{\pgfqpoint{0.587753in}{0.516667in}}{\pgfqpoint{0.583333in}{0.516667in}}%
\pgfpathcurveto{\pgfqpoint{0.578913in}{0.516667in}}{\pgfqpoint{0.574674in}{0.514911in}}{\pgfqpoint{0.571548in}{0.511785in}}%
\pgfpathcurveto{\pgfqpoint{0.568423in}{0.508660in}}{\pgfqpoint{0.566667in}{0.504420in}}{\pgfqpoint{0.566667in}{0.500000in}}%
\pgfpathcurveto{\pgfqpoint{0.566667in}{0.495580in}}{\pgfqpoint{0.568423in}{0.491340in}}{\pgfqpoint{0.571548in}{0.488215in}}%
\pgfpathcurveto{\pgfqpoint{0.574674in}{0.485089in}}{\pgfqpoint{0.578913in}{0.483333in}}{\pgfqpoint{0.583333in}{0.483333in}}%
\pgfpathclose%
\pgfpathmoveto{\pgfqpoint{0.750000in}{0.483333in}}%
\pgfpathcurveto{\pgfqpoint{0.754420in}{0.483333in}}{\pgfqpoint{0.758660in}{0.485089in}}{\pgfqpoint{0.761785in}{0.488215in}}%
\pgfpathcurveto{\pgfqpoint{0.764911in}{0.491340in}}{\pgfqpoint{0.766667in}{0.495580in}}{\pgfqpoint{0.766667in}{0.500000in}}%
\pgfpathcurveto{\pgfqpoint{0.766667in}{0.504420in}}{\pgfqpoint{0.764911in}{0.508660in}}{\pgfqpoint{0.761785in}{0.511785in}}%
\pgfpathcurveto{\pgfqpoint{0.758660in}{0.514911in}}{\pgfqpoint{0.754420in}{0.516667in}}{\pgfqpoint{0.750000in}{0.516667in}}%
\pgfpathcurveto{\pgfqpoint{0.745580in}{0.516667in}}{\pgfqpoint{0.741340in}{0.514911in}}{\pgfqpoint{0.738215in}{0.511785in}}%
\pgfpathcurveto{\pgfqpoint{0.735089in}{0.508660in}}{\pgfqpoint{0.733333in}{0.504420in}}{\pgfqpoint{0.733333in}{0.500000in}}%
\pgfpathcurveto{\pgfqpoint{0.733333in}{0.495580in}}{\pgfqpoint{0.735089in}{0.491340in}}{\pgfqpoint{0.738215in}{0.488215in}}%
\pgfpathcurveto{\pgfqpoint{0.741340in}{0.485089in}}{\pgfqpoint{0.745580in}{0.483333in}}{\pgfqpoint{0.750000in}{0.483333in}}%
\pgfpathclose%
\pgfpathmoveto{\pgfqpoint{0.916667in}{0.483333in}}%
\pgfpathcurveto{\pgfqpoint{0.921087in}{0.483333in}}{\pgfqpoint{0.925326in}{0.485089in}}{\pgfqpoint{0.928452in}{0.488215in}}%
\pgfpathcurveto{\pgfqpoint{0.931577in}{0.491340in}}{\pgfqpoint{0.933333in}{0.495580in}}{\pgfqpoint{0.933333in}{0.500000in}}%
\pgfpathcurveto{\pgfqpoint{0.933333in}{0.504420in}}{\pgfqpoint{0.931577in}{0.508660in}}{\pgfqpoint{0.928452in}{0.511785in}}%
\pgfpathcurveto{\pgfqpoint{0.925326in}{0.514911in}}{\pgfqpoint{0.921087in}{0.516667in}}{\pgfqpoint{0.916667in}{0.516667in}}%
\pgfpathcurveto{\pgfqpoint{0.912247in}{0.516667in}}{\pgfqpoint{0.908007in}{0.514911in}}{\pgfqpoint{0.904882in}{0.511785in}}%
\pgfpathcurveto{\pgfqpoint{0.901756in}{0.508660in}}{\pgfqpoint{0.900000in}{0.504420in}}{\pgfqpoint{0.900000in}{0.500000in}}%
\pgfpathcurveto{\pgfqpoint{0.900000in}{0.495580in}}{\pgfqpoint{0.901756in}{0.491340in}}{\pgfqpoint{0.904882in}{0.488215in}}%
\pgfpathcurveto{\pgfqpoint{0.908007in}{0.485089in}}{\pgfqpoint{0.912247in}{0.483333in}}{\pgfqpoint{0.916667in}{0.483333in}}%
\pgfpathclose%
\pgfpathmoveto{\pgfqpoint{0.000000in}{0.650000in}}%
\pgfpathcurveto{\pgfqpoint{0.004420in}{0.650000in}}{\pgfqpoint{0.008660in}{0.651756in}}{\pgfqpoint{0.011785in}{0.654882in}}%
\pgfpathcurveto{\pgfqpoint{0.014911in}{0.658007in}}{\pgfqpoint{0.016667in}{0.662247in}}{\pgfqpoint{0.016667in}{0.666667in}}%
\pgfpathcurveto{\pgfqpoint{0.016667in}{0.671087in}}{\pgfqpoint{0.014911in}{0.675326in}}{\pgfqpoint{0.011785in}{0.678452in}}%
\pgfpathcurveto{\pgfqpoint{0.008660in}{0.681577in}}{\pgfqpoint{0.004420in}{0.683333in}}{\pgfqpoint{0.000000in}{0.683333in}}%
\pgfpathcurveto{\pgfqpoint{-0.004420in}{0.683333in}}{\pgfqpoint{-0.008660in}{0.681577in}}{\pgfqpoint{-0.011785in}{0.678452in}}%
\pgfpathcurveto{\pgfqpoint{-0.014911in}{0.675326in}}{\pgfqpoint{-0.016667in}{0.671087in}}{\pgfqpoint{-0.016667in}{0.666667in}}%
\pgfpathcurveto{\pgfqpoint{-0.016667in}{0.662247in}}{\pgfqpoint{-0.014911in}{0.658007in}}{\pgfqpoint{-0.011785in}{0.654882in}}%
\pgfpathcurveto{\pgfqpoint{-0.008660in}{0.651756in}}{\pgfqpoint{-0.004420in}{0.650000in}}{\pgfqpoint{0.000000in}{0.650000in}}%
\pgfpathclose%
\pgfpathmoveto{\pgfqpoint{0.166667in}{0.650000in}}%
\pgfpathcurveto{\pgfqpoint{0.171087in}{0.650000in}}{\pgfqpoint{0.175326in}{0.651756in}}{\pgfqpoint{0.178452in}{0.654882in}}%
\pgfpathcurveto{\pgfqpoint{0.181577in}{0.658007in}}{\pgfqpoint{0.183333in}{0.662247in}}{\pgfqpoint{0.183333in}{0.666667in}}%
\pgfpathcurveto{\pgfqpoint{0.183333in}{0.671087in}}{\pgfqpoint{0.181577in}{0.675326in}}{\pgfqpoint{0.178452in}{0.678452in}}%
\pgfpathcurveto{\pgfqpoint{0.175326in}{0.681577in}}{\pgfqpoint{0.171087in}{0.683333in}}{\pgfqpoint{0.166667in}{0.683333in}}%
\pgfpathcurveto{\pgfqpoint{0.162247in}{0.683333in}}{\pgfqpoint{0.158007in}{0.681577in}}{\pgfqpoint{0.154882in}{0.678452in}}%
\pgfpathcurveto{\pgfqpoint{0.151756in}{0.675326in}}{\pgfqpoint{0.150000in}{0.671087in}}{\pgfqpoint{0.150000in}{0.666667in}}%
\pgfpathcurveto{\pgfqpoint{0.150000in}{0.662247in}}{\pgfqpoint{0.151756in}{0.658007in}}{\pgfqpoint{0.154882in}{0.654882in}}%
\pgfpathcurveto{\pgfqpoint{0.158007in}{0.651756in}}{\pgfqpoint{0.162247in}{0.650000in}}{\pgfqpoint{0.166667in}{0.650000in}}%
\pgfpathclose%
\pgfpathmoveto{\pgfqpoint{0.333333in}{0.650000in}}%
\pgfpathcurveto{\pgfqpoint{0.337753in}{0.650000in}}{\pgfqpoint{0.341993in}{0.651756in}}{\pgfqpoint{0.345118in}{0.654882in}}%
\pgfpathcurveto{\pgfqpoint{0.348244in}{0.658007in}}{\pgfqpoint{0.350000in}{0.662247in}}{\pgfqpoint{0.350000in}{0.666667in}}%
\pgfpathcurveto{\pgfqpoint{0.350000in}{0.671087in}}{\pgfqpoint{0.348244in}{0.675326in}}{\pgfqpoint{0.345118in}{0.678452in}}%
\pgfpathcurveto{\pgfqpoint{0.341993in}{0.681577in}}{\pgfqpoint{0.337753in}{0.683333in}}{\pgfqpoint{0.333333in}{0.683333in}}%
\pgfpathcurveto{\pgfqpoint{0.328913in}{0.683333in}}{\pgfqpoint{0.324674in}{0.681577in}}{\pgfqpoint{0.321548in}{0.678452in}}%
\pgfpathcurveto{\pgfqpoint{0.318423in}{0.675326in}}{\pgfqpoint{0.316667in}{0.671087in}}{\pgfqpoint{0.316667in}{0.666667in}}%
\pgfpathcurveto{\pgfqpoint{0.316667in}{0.662247in}}{\pgfqpoint{0.318423in}{0.658007in}}{\pgfqpoint{0.321548in}{0.654882in}}%
\pgfpathcurveto{\pgfqpoint{0.324674in}{0.651756in}}{\pgfqpoint{0.328913in}{0.650000in}}{\pgfqpoint{0.333333in}{0.650000in}}%
\pgfpathclose%
\pgfpathmoveto{\pgfqpoint{0.500000in}{0.650000in}}%
\pgfpathcurveto{\pgfqpoint{0.504420in}{0.650000in}}{\pgfqpoint{0.508660in}{0.651756in}}{\pgfqpoint{0.511785in}{0.654882in}}%
\pgfpathcurveto{\pgfqpoint{0.514911in}{0.658007in}}{\pgfqpoint{0.516667in}{0.662247in}}{\pgfqpoint{0.516667in}{0.666667in}}%
\pgfpathcurveto{\pgfqpoint{0.516667in}{0.671087in}}{\pgfqpoint{0.514911in}{0.675326in}}{\pgfqpoint{0.511785in}{0.678452in}}%
\pgfpathcurveto{\pgfqpoint{0.508660in}{0.681577in}}{\pgfqpoint{0.504420in}{0.683333in}}{\pgfqpoint{0.500000in}{0.683333in}}%
\pgfpathcurveto{\pgfqpoint{0.495580in}{0.683333in}}{\pgfqpoint{0.491340in}{0.681577in}}{\pgfqpoint{0.488215in}{0.678452in}}%
\pgfpathcurveto{\pgfqpoint{0.485089in}{0.675326in}}{\pgfqpoint{0.483333in}{0.671087in}}{\pgfqpoint{0.483333in}{0.666667in}}%
\pgfpathcurveto{\pgfqpoint{0.483333in}{0.662247in}}{\pgfqpoint{0.485089in}{0.658007in}}{\pgfqpoint{0.488215in}{0.654882in}}%
\pgfpathcurveto{\pgfqpoint{0.491340in}{0.651756in}}{\pgfqpoint{0.495580in}{0.650000in}}{\pgfqpoint{0.500000in}{0.650000in}}%
\pgfpathclose%
\pgfpathmoveto{\pgfqpoint{0.666667in}{0.650000in}}%
\pgfpathcurveto{\pgfqpoint{0.671087in}{0.650000in}}{\pgfqpoint{0.675326in}{0.651756in}}{\pgfqpoint{0.678452in}{0.654882in}}%
\pgfpathcurveto{\pgfqpoint{0.681577in}{0.658007in}}{\pgfqpoint{0.683333in}{0.662247in}}{\pgfqpoint{0.683333in}{0.666667in}}%
\pgfpathcurveto{\pgfqpoint{0.683333in}{0.671087in}}{\pgfqpoint{0.681577in}{0.675326in}}{\pgfqpoint{0.678452in}{0.678452in}}%
\pgfpathcurveto{\pgfqpoint{0.675326in}{0.681577in}}{\pgfqpoint{0.671087in}{0.683333in}}{\pgfqpoint{0.666667in}{0.683333in}}%
\pgfpathcurveto{\pgfqpoint{0.662247in}{0.683333in}}{\pgfqpoint{0.658007in}{0.681577in}}{\pgfqpoint{0.654882in}{0.678452in}}%
\pgfpathcurveto{\pgfqpoint{0.651756in}{0.675326in}}{\pgfqpoint{0.650000in}{0.671087in}}{\pgfqpoint{0.650000in}{0.666667in}}%
\pgfpathcurveto{\pgfqpoint{0.650000in}{0.662247in}}{\pgfqpoint{0.651756in}{0.658007in}}{\pgfqpoint{0.654882in}{0.654882in}}%
\pgfpathcurveto{\pgfqpoint{0.658007in}{0.651756in}}{\pgfqpoint{0.662247in}{0.650000in}}{\pgfqpoint{0.666667in}{0.650000in}}%
\pgfpathclose%
\pgfpathmoveto{\pgfqpoint{0.833333in}{0.650000in}}%
\pgfpathcurveto{\pgfqpoint{0.837753in}{0.650000in}}{\pgfqpoint{0.841993in}{0.651756in}}{\pgfqpoint{0.845118in}{0.654882in}}%
\pgfpathcurveto{\pgfqpoint{0.848244in}{0.658007in}}{\pgfqpoint{0.850000in}{0.662247in}}{\pgfqpoint{0.850000in}{0.666667in}}%
\pgfpathcurveto{\pgfqpoint{0.850000in}{0.671087in}}{\pgfqpoint{0.848244in}{0.675326in}}{\pgfqpoint{0.845118in}{0.678452in}}%
\pgfpathcurveto{\pgfqpoint{0.841993in}{0.681577in}}{\pgfqpoint{0.837753in}{0.683333in}}{\pgfqpoint{0.833333in}{0.683333in}}%
\pgfpathcurveto{\pgfqpoint{0.828913in}{0.683333in}}{\pgfqpoint{0.824674in}{0.681577in}}{\pgfqpoint{0.821548in}{0.678452in}}%
\pgfpathcurveto{\pgfqpoint{0.818423in}{0.675326in}}{\pgfqpoint{0.816667in}{0.671087in}}{\pgfqpoint{0.816667in}{0.666667in}}%
\pgfpathcurveto{\pgfqpoint{0.816667in}{0.662247in}}{\pgfqpoint{0.818423in}{0.658007in}}{\pgfqpoint{0.821548in}{0.654882in}}%
\pgfpathcurveto{\pgfqpoint{0.824674in}{0.651756in}}{\pgfqpoint{0.828913in}{0.650000in}}{\pgfqpoint{0.833333in}{0.650000in}}%
\pgfpathclose%
\pgfpathmoveto{\pgfqpoint{1.000000in}{0.650000in}}%
\pgfpathcurveto{\pgfqpoint{1.004420in}{0.650000in}}{\pgfqpoint{1.008660in}{0.651756in}}{\pgfqpoint{1.011785in}{0.654882in}}%
\pgfpathcurveto{\pgfqpoint{1.014911in}{0.658007in}}{\pgfqpoint{1.016667in}{0.662247in}}{\pgfqpoint{1.016667in}{0.666667in}}%
\pgfpathcurveto{\pgfqpoint{1.016667in}{0.671087in}}{\pgfqpoint{1.014911in}{0.675326in}}{\pgfqpoint{1.011785in}{0.678452in}}%
\pgfpathcurveto{\pgfqpoint{1.008660in}{0.681577in}}{\pgfqpoint{1.004420in}{0.683333in}}{\pgfqpoint{1.000000in}{0.683333in}}%
\pgfpathcurveto{\pgfqpoint{0.995580in}{0.683333in}}{\pgfqpoint{0.991340in}{0.681577in}}{\pgfqpoint{0.988215in}{0.678452in}}%
\pgfpathcurveto{\pgfqpoint{0.985089in}{0.675326in}}{\pgfqpoint{0.983333in}{0.671087in}}{\pgfqpoint{0.983333in}{0.666667in}}%
\pgfpathcurveto{\pgfqpoint{0.983333in}{0.662247in}}{\pgfqpoint{0.985089in}{0.658007in}}{\pgfqpoint{0.988215in}{0.654882in}}%
\pgfpathcurveto{\pgfqpoint{0.991340in}{0.651756in}}{\pgfqpoint{0.995580in}{0.650000in}}{\pgfqpoint{1.000000in}{0.650000in}}%
\pgfpathclose%
\pgfpathmoveto{\pgfqpoint{0.083333in}{0.816667in}}%
\pgfpathcurveto{\pgfqpoint{0.087753in}{0.816667in}}{\pgfqpoint{0.091993in}{0.818423in}}{\pgfqpoint{0.095118in}{0.821548in}}%
\pgfpathcurveto{\pgfqpoint{0.098244in}{0.824674in}}{\pgfqpoint{0.100000in}{0.828913in}}{\pgfqpoint{0.100000in}{0.833333in}}%
\pgfpathcurveto{\pgfqpoint{0.100000in}{0.837753in}}{\pgfqpoint{0.098244in}{0.841993in}}{\pgfqpoint{0.095118in}{0.845118in}}%
\pgfpathcurveto{\pgfqpoint{0.091993in}{0.848244in}}{\pgfqpoint{0.087753in}{0.850000in}}{\pgfqpoint{0.083333in}{0.850000in}}%
\pgfpathcurveto{\pgfqpoint{0.078913in}{0.850000in}}{\pgfqpoint{0.074674in}{0.848244in}}{\pgfqpoint{0.071548in}{0.845118in}}%
\pgfpathcurveto{\pgfqpoint{0.068423in}{0.841993in}}{\pgfqpoint{0.066667in}{0.837753in}}{\pgfqpoint{0.066667in}{0.833333in}}%
\pgfpathcurveto{\pgfqpoint{0.066667in}{0.828913in}}{\pgfqpoint{0.068423in}{0.824674in}}{\pgfqpoint{0.071548in}{0.821548in}}%
\pgfpathcurveto{\pgfqpoint{0.074674in}{0.818423in}}{\pgfqpoint{0.078913in}{0.816667in}}{\pgfqpoint{0.083333in}{0.816667in}}%
\pgfpathclose%
\pgfpathmoveto{\pgfqpoint{0.250000in}{0.816667in}}%
\pgfpathcurveto{\pgfqpoint{0.254420in}{0.816667in}}{\pgfqpoint{0.258660in}{0.818423in}}{\pgfqpoint{0.261785in}{0.821548in}}%
\pgfpathcurveto{\pgfqpoint{0.264911in}{0.824674in}}{\pgfqpoint{0.266667in}{0.828913in}}{\pgfqpoint{0.266667in}{0.833333in}}%
\pgfpathcurveto{\pgfqpoint{0.266667in}{0.837753in}}{\pgfqpoint{0.264911in}{0.841993in}}{\pgfqpoint{0.261785in}{0.845118in}}%
\pgfpathcurveto{\pgfqpoint{0.258660in}{0.848244in}}{\pgfqpoint{0.254420in}{0.850000in}}{\pgfqpoint{0.250000in}{0.850000in}}%
\pgfpathcurveto{\pgfqpoint{0.245580in}{0.850000in}}{\pgfqpoint{0.241340in}{0.848244in}}{\pgfqpoint{0.238215in}{0.845118in}}%
\pgfpathcurveto{\pgfqpoint{0.235089in}{0.841993in}}{\pgfqpoint{0.233333in}{0.837753in}}{\pgfqpoint{0.233333in}{0.833333in}}%
\pgfpathcurveto{\pgfqpoint{0.233333in}{0.828913in}}{\pgfqpoint{0.235089in}{0.824674in}}{\pgfqpoint{0.238215in}{0.821548in}}%
\pgfpathcurveto{\pgfqpoint{0.241340in}{0.818423in}}{\pgfqpoint{0.245580in}{0.816667in}}{\pgfqpoint{0.250000in}{0.816667in}}%
\pgfpathclose%
\pgfpathmoveto{\pgfqpoint{0.416667in}{0.816667in}}%
\pgfpathcurveto{\pgfqpoint{0.421087in}{0.816667in}}{\pgfqpoint{0.425326in}{0.818423in}}{\pgfqpoint{0.428452in}{0.821548in}}%
\pgfpathcurveto{\pgfqpoint{0.431577in}{0.824674in}}{\pgfqpoint{0.433333in}{0.828913in}}{\pgfqpoint{0.433333in}{0.833333in}}%
\pgfpathcurveto{\pgfqpoint{0.433333in}{0.837753in}}{\pgfqpoint{0.431577in}{0.841993in}}{\pgfqpoint{0.428452in}{0.845118in}}%
\pgfpathcurveto{\pgfqpoint{0.425326in}{0.848244in}}{\pgfqpoint{0.421087in}{0.850000in}}{\pgfqpoint{0.416667in}{0.850000in}}%
\pgfpathcurveto{\pgfqpoint{0.412247in}{0.850000in}}{\pgfqpoint{0.408007in}{0.848244in}}{\pgfqpoint{0.404882in}{0.845118in}}%
\pgfpathcurveto{\pgfqpoint{0.401756in}{0.841993in}}{\pgfqpoint{0.400000in}{0.837753in}}{\pgfqpoint{0.400000in}{0.833333in}}%
\pgfpathcurveto{\pgfqpoint{0.400000in}{0.828913in}}{\pgfqpoint{0.401756in}{0.824674in}}{\pgfqpoint{0.404882in}{0.821548in}}%
\pgfpathcurveto{\pgfqpoint{0.408007in}{0.818423in}}{\pgfqpoint{0.412247in}{0.816667in}}{\pgfqpoint{0.416667in}{0.816667in}}%
\pgfpathclose%
\pgfpathmoveto{\pgfqpoint{0.583333in}{0.816667in}}%
\pgfpathcurveto{\pgfqpoint{0.587753in}{0.816667in}}{\pgfqpoint{0.591993in}{0.818423in}}{\pgfqpoint{0.595118in}{0.821548in}}%
\pgfpathcurveto{\pgfqpoint{0.598244in}{0.824674in}}{\pgfqpoint{0.600000in}{0.828913in}}{\pgfqpoint{0.600000in}{0.833333in}}%
\pgfpathcurveto{\pgfqpoint{0.600000in}{0.837753in}}{\pgfqpoint{0.598244in}{0.841993in}}{\pgfqpoint{0.595118in}{0.845118in}}%
\pgfpathcurveto{\pgfqpoint{0.591993in}{0.848244in}}{\pgfqpoint{0.587753in}{0.850000in}}{\pgfqpoint{0.583333in}{0.850000in}}%
\pgfpathcurveto{\pgfqpoint{0.578913in}{0.850000in}}{\pgfqpoint{0.574674in}{0.848244in}}{\pgfqpoint{0.571548in}{0.845118in}}%
\pgfpathcurveto{\pgfqpoint{0.568423in}{0.841993in}}{\pgfqpoint{0.566667in}{0.837753in}}{\pgfqpoint{0.566667in}{0.833333in}}%
\pgfpathcurveto{\pgfqpoint{0.566667in}{0.828913in}}{\pgfqpoint{0.568423in}{0.824674in}}{\pgfqpoint{0.571548in}{0.821548in}}%
\pgfpathcurveto{\pgfqpoint{0.574674in}{0.818423in}}{\pgfqpoint{0.578913in}{0.816667in}}{\pgfqpoint{0.583333in}{0.816667in}}%
\pgfpathclose%
\pgfpathmoveto{\pgfqpoint{0.750000in}{0.816667in}}%
\pgfpathcurveto{\pgfqpoint{0.754420in}{0.816667in}}{\pgfqpoint{0.758660in}{0.818423in}}{\pgfqpoint{0.761785in}{0.821548in}}%
\pgfpathcurveto{\pgfqpoint{0.764911in}{0.824674in}}{\pgfqpoint{0.766667in}{0.828913in}}{\pgfqpoint{0.766667in}{0.833333in}}%
\pgfpathcurveto{\pgfqpoint{0.766667in}{0.837753in}}{\pgfqpoint{0.764911in}{0.841993in}}{\pgfqpoint{0.761785in}{0.845118in}}%
\pgfpathcurveto{\pgfqpoint{0.758660in}{0.848244in}}{\pgfqpoint{0.754420in}{0.850000in}}{\pgfqpoint{0.750000in}{0.850000in}}%
\pgfpathcurveto{\pgfqpoint{0.745580in}{0.850000in}}{\pgfqpoint{0.741340in}{0.848244in}}{\pgfqpoint{0.738215in}{0.845118in}}%
\pgfpathcurveto{\pgfqpoint{0.735089in}{0.841993in}}{\pgfqpoint{0.733333in}{0.837753in}}{\pgfqpoint{0.733333in}{0.833333in}}%
\pgfpathcurveto{\pgfqpoint{0.733333in}{0.828913in}}{\pgfqpoint{0.735089in}{0.824674in}}{\pgfqpoint{0.738215in}{0.821548in}}%
\pgfpathcurveto{\pgfqpoint{0.741340in}{0.818423in}}{\pgfqpoint{0.745580in}{0.816667in}}{\pgfqpoint{0.750000in}{0.816667in}}%
\pgfpathclose%
\pgfpathmoveto{\pgfqpoint{0.916667in}{0.816667in}}%
\pgfpathcurveto{\pgfqpoint{0.921087in}{0.816667in}}{\pgfqpoint{0.925326in}{0.818423in}}{\pgfqpoint{0.928452in}{0.821548in}}%
\pgfpathcurveto{\pgfqpoint{0.931577in}{0.824674in}}{\pgfqpoint{0.933333in}{0.828913in}}{\pgfqpoint{0.933333in}{0.833333in}}%
\pgfpathcurveto{\pgfqpoint{0.933333in}{0.837753in}}{\pgfqpoint{0.931577in}{0.841993in}}{\pgfqpoint{0.928452in}{0.845118in}}%
\pgfpathcurveto{\pgfqpoint{0.925326in}{0.848244in}}{\pgfqpoint{0.921087in}{0.850000in}}{\pgfqpoint{0.916667in}{0.850000in}}%
\pgfpathcurveto{\pgfqpoint{0.912247in}{0.850000in}}{\pgfqpoint{0.908007in}{0.848244in}}{\pgfqpoint{0.904882in}{0.845118in}}%
\pgfpathcurveto{\pgfqpoint{0.901756in}{0.841993in}}{\pgfqpoint{0.900000in}{0.837753in}}{\pgfqpoint{0.900000in}{0.833333in}}%
\pgfpathcurveto{\pgfqpoint{0.900000in}{0.828913in}}{\pgfqpoint{0.901756in}{0.824674in}}{\pgfqpoint{0.904882in}{0.821548in}}%
\pgfpathcurveto{\pgfqpoint{0.908007in}{0.818423in}}{\pgfqpoint{0.912247in}{0.816667in}}{\pgfqpoint{0.916667in}{0.816667in}}%
\pgfpathclose%
\pgfpathmoveto{\pgfqpoint{0.000000in}{0.983333in}}%
\pgfpathcurveto{\pgfqpoint{0.004420in}{0.983333in}}{\pgfqpoint{0.008660in}{0.985089in}}{\pgfqpoint{0.011785in}{0.988215in}}%
\pgfpathcurveto{\pgfqpoint{0.014911in}{0.991340in}}{\pgfqpoint{0.016667in}{0.995580in}}{\pgfqpoint{0.016667in}{1.000000in}}%
\pgfpathcurveto{\pgfqpoint{0.016667in}{1.004420in}}{\pgfqpoint{0.014911in}{1.008660in}}{\pgfqpoint{0.011785in}{1.011785in}}%
\pgfpathcurveto{\pgfqpoint{0.008660in}{1.014911in}}{\pgfqpoint{0.004420in}{1.016667in}}{\pgfqpoint{0.000000in}{1.016667in}}%
\pgfpathcurveto{\pgfqpoint{-0.004420in}{1.016667in}}{\pgfqpoint{-0.008660in}{1.014911in}}{\pgfqpoint{-0.011785in}{1.011785in}}%
\pgfpathcurveto{\pgfqpoint{-0.014911in}{1.008660in}}{\pgfqpoint{-0.016667in}{1.004420in}}{\pgfqpoint{-0.016667in}{1.000000in}}%
\pgfpathcurveto{\pgfqpoint{-0.016667in}{0.995580in}}{\pgfqpoint{-0.014911in}{0.991340in}}{\pgfqpoint{-0.011785in}{0.988215in}}%
\pgfpathcurveto{\pgfqpoint{-0.008660in}{0.985089in}}{\pgfqpoint{-0.004420in}{0.983333in}}{\pgfqpoint{0.000000in}{0.983333in}}%
\pgfpathclose%
\pgfpathmoveto{\pgfqpoint{0.166667in}{0.983333in}}%
\pgfpathcurveto{\pgfqpoint{0.171087in}{0.983333in}}{\pgfqpoint{0.175326in}{0.985089in}}{\pgfqpoint{0.178452in}{0.988215in}}%
\pgfpathcurveto{\pgfqpoint{0.181577in}{0.991340in}}{\pgfqpoint{0.183333in}{0.995580in}}{\pgfqpoint{0.183333in}{1.000000in}}%
\pgfpathcurveto{\pgfqpoint{0.183333in}{1.004420in}}{\pgfqpoint{0.181577in}{1.008660in}}{\pgfqpoint{0.178452in}{1.011785in}}%
\pgfpathcurveto{\pgfqpoint{0.175326in}{1.014911in}}{\pgfqpoint{0.171087in}{1.016667in}}{\pgfqpoint{0.166667in}{1.016667in}}%
\pgfpathcurveto{\pgfqpoint{0.162247in}{1.016667in}}{\pgfqpoint{0.158007in}{1.014911in}}{\pgfqpoint{0.154882in}{1.011785in}}%
\pgfpathcurveto{\pgfqpoint{0.151756in}{1.008660in}}{\pgfqpoint{0.150000in}{1.004420in}}{\pgfqpoint{0.150000in}{1.000000in}}%
\pgfpathcurveto{\pgfqpoint{0.150000in}{0.995580in}}{\pgfqpoint{0.151756in}{0.991340in}}{\pgfqpoint{0.154882in}{0.988215in}}%
\pgfpathcurveto{\pgfqpoint{0.158007in}{0.985089in}}{\pgfqpoint{0.162247in}{0.983333in}}{\pgfqpoint{0.166667in}{0.983333in}}%
\pgfpathclose%
\pgfpathmoveto{\pgfqpoint{0.333333in}{0.983333in}}%
\pgfpathcurveto{\pgfqpoint{0.337753in}{0.983333in}}{\pgfqpoint{0.341993in}{0.985089in}}{\pgfqpoint{0.345118in}{0.988215in}}%
\pgfpathcurveto{\pgfqpoint{0.348244in}{0.991340in}}{\pgfqpoint{0.350000in}{0.995580in}}{\pgfqpoint{0.350000in}{1.000000in}}%
\pgfpathcurveto{\pgfqpoint{0.350000in}{1.004420in}}{\pgfqpoint{0.348244in}{1.008660in}}{\pgfqpoint{0.345118in}{1.011785in}}%
\pgfpathcurveto{\pgfqpoint{0.341993in}{1.014911in}}{\pgfqpoint{0.337753in}{1.016667in}}{\pgfqpoint{0.333333in}{1.016667in}}%
\pgfpathcurveto{\pgfqpoint{0.328913in}{1.016667in}}{\pgfqpoint{0.324674in}{1.014911in}}{\pgfqpoint{0.321548in}{1.011785in}}%
\pgfpathcurveto{\pgfqpoint{0.318423in}{1.008660in}}{\pgfqpoint{0.316667in}{1.004420in}}{\pgfqpoint{0.316667in}{1.000000in}}%
\pgfpathcurveto{\pgfqpoint{0.316667in}{0.995580in}}{\pgfqpoint{0.318423in}{0.991340in}}{\pgfqpoint{0.321548in}{0.988215in}}%
\pgfpathcurveto{\pgfqpoint{0.324674in}{0.985089in}}{\pgfqpoint{0.328913in}{0.983333in}}{\pgfqpoint{0.333333in}{0.983333in}}%
\pgfpathclose%
\pgfpathmoveto{\pgfqpoint{0.500000in}{0.983333in}}%
\pgfpathcurveto{\pgfqpoint{0.504420in}{0.983333in}}{\pgfqpoint{0.508660in}{0.985089in}}{\pgfqpoint{0.511785in}{0.988215in}}%
\pgfpathcurveto{\pgfqpoint{0.514911in}{0.991340in}}{\pgfqpoint{0.516667in}{0.995580in}}{\pgfqpoint{0.516667in}{1.000000in}}%
\pgfpathcurveto{\pgfqpoint{0.516667in}{1.004420in}}{\pgfqpoint{0.514911in}{1.008660in}}{\pgfqpoint{0.511785in}{1.011785in}}%
\pgfpathcurveto{\pgfqpoint{0.508660in}{1.014911in}}{\pgfqpoint{0.504420in}{1.016667in}}{\pgfqpoint{0.500000in}{1.016667in}}%
\pgfpathcurveto{\pgfqpoint{0.495580in}{1.016667in}}{\pgfqpoint{0.491340in}{1.014911in}}{\pgfqpoint{0.488215in}{1.011785in}}%
\pgfpathcurveto{\pgfqpoint{0.485089in}{1.008660in}}{\pgfqpoint{0.483333in}{1.004420in}}{\pgfqpoint{0.483333in}{1.000000in}}%
\pgfpathcurveto{\pgfqpoint{0.483333in}{0.995580in}}{\pgfqpoint{0.485089in}{0.991340in}}{\pgfqpoint{0.488215in}{0.988215in}}%
\pgfpathcurveto{\pgfqpoint{0.491340in}{0.985089in}}{\pgfqpoint{0.495580in}{0.983333in}}{\pgfqpoint{0.500000in}{0.983333in}}%
\pgfpathclose%
\pgfpathmoveto{\pgfqpoint{0.666667in}{0.983333in}}%
\pgfpathcurveto{\pgfqpoint{0.671087in}{0.983333in}}{\pgfqpoint{0.675326in}{0.985089in}}{\pgfqpoint{0.678452in}{0.988215in}}%
\pgfpathcurveto{\pgfqpoint{0.681577in}{0.991340in}}{\pgfqpoint{0.683333in}{0.995580in}}{\pgfqpoint{0.683333in}{1.000000in}}%
\pgfpathcurveto{\pgfqpoint{0.683333in}{1.004420in}}{\pgfqpoint{0.681577in}{1.008660in}}{\pgfqpoint{0.678452in}{1.011785in}}%
\pgfpathcurveto{\pgfqpoint{0.675326in}{1.014911in}}{\pgfqpoint{0.671087in}{1.016667in}}{\pgfqpoint{0.666667in}{1.016667in}}%
\pgfpathcurveto{\pgfqpoint{0.662247in}{1.016667in}}{\pgfqpoint{0.658007in}{1.014911in}}{\pgfqpoint{0.654882in}{1.011785in}}%
\pgfpathcurveto{\pgfqpoint{0.651756in}{1.008660in}}{\pgfqpoint{0.650000in}{1.004420in}}{\pgfqpoint{0.650000in}{1.000000in}}%
\pgfpathcurveto{\pgfqpoint{0.650000in}{0.995580in}}{\pgfqpoint{0.651756in}{0.991340in}}{\pgfqpoint{0.654882in}{0.988215in}}%
\pgfpathcurveto{\pgfqpoint{0.658007in}{0.985089in}}{\pgfqpoint{0.662247in}{0.983333in}}{\pgfqpoint{0.666667in}{0.983333in}}%
\pgfpathclose%
\pgfpathmoveto{\pgfqpoint{0.833333in}{0.983333in}}%
\pgfpathcurveto{\pgfqpoint{0.837753in}{0.983333in}}{\pgfqpoint{0.841993in}{0.985089in}}{\pgfqpoint{0.845118in}{0.988215in}}%
\pgfpathcurveto{\pgfqpoint{0.848244in}{0.991340in}}{\pgfqpoint{0.850000in}{0.995580in}}{\pgfqpoint{0.850000in}{1.000000in}}%
\pgfpathcurveto{\pgfqpoint{0.850000in}{1.004420in}}{\pgfqpoint{0.848244in}{1.008660in}}{\pgfqpoint{0.845118in}{1.011785in}}%
\pgfpathcurveto{\pgfqpoint{0.841993in}{1.014911in}}{\pgfqpoint{0.837753in}{1.016667in}}{\pgfqpoint{0.833333in}{1.016667in}}%
\pgfpathcurveto{\pgfqpoint{0.828913in}{1.016667in}}{\pgfqpoint{0.824674in}{1.014911in}}{\pgfqpoint{0.821548in}{1.011785in}}%
\pgfpathcurveto{\pgfqpoint{0.818423in}{1.008660in}}{\pgfqpoint{0.816667in}{1.004420in}}{\pgfqpoint{0.816667in}{1.000000in}}%
\pgfpathcurveto{\pgfqpoint{0.816667in}{0.995580in}}{\pgfqpoint{0.818423in}{0.991340in}}{\pgfqpoint{0.821548in}{0.988215in}}%
\pgfpathcurveto{\pgfqpoint{0.824674in}{0.985089in}}{\pgfqpoint{0.828913in}{0.983333in}}{\pgfqpoint{0.833333in}{0.983333in}}%
\pgfpathclose%
\pgfpathmoveto{\pgfqpoint{1.000000in}{0.983333in}}%
\pgfpathcurveto{\pgfqpoint{1.004420in}{0.983333in}}{\pgfqpoint{1.008660in}{0.985089in}}{\pgfqpoint{1.011785in}{0.988215in}}%
\pgfpathcurveto{\pgfqpoint{1.014911in}{0.991340in}}{\pgfqpoint{1.016667in}{0.995580in}}{\pgfqpoint{1.016667in}{1.000000in}}%
\pgfpathcurveto{\pgfqpoint{1.016667in}{1.004420in}}{\pgfqpoint{1.014911in}{1.008660in}}{\pgfqpoint{1.011785in}{1.011785in}}%
\pgfpathcurveto{\pgfqpoint{1.008660in}{1.014911in}}{\pgfqpoint{1.004420in}{1.016667in}}{\pgfqpoint{1.000000in}{1.016667in}}%
\pgfpathcurveto{\pgfqpoint{0.995580in}{1.016667in}}{\pgfqpoint{0.991340in}{1.014911in}}{\pgfqpoint{0.988215in}{1.011785in}}%
\pgfpathcurveto{\pgfqpoint{0.985089in}{1.008660in}}{\pgfqpoint{0.983333in}{1.004420in}}{\pgfqpoint{0.983333in}{1.000000in}}%
\pgfpathcurveto{\pgfqpoint{0.983333in}{0.995580in}}{\pgfqpoint{0.985089in}{0.991340in}}{\pgfqpoint{0.988215in}{0.988215in}}%
\pgfpathcurveto{\pgfqpoint{0.991340in}{0.985089in}}{\pgfqpoint{0.995580in}{0.983333in}}{\pgfqpoint{1.000000in}{0.983333in}}%
\pgfpathclose%
\pgfusepath{stroke}%
\end{pgfscope}%
}%
\pgfsys@transformshift{4.358038in}{0.637495in}%
\end{pgfscope}%
\begin{pgfscope}%
\pgfpathrectangle{\pgfqpoint{0.870538in}{0.637495in}}{\pgfqpoint{9.300000in}{9.060000in}}%
\pgfusepath{clip}%
\pgfsetbuttcap%
\pgfsetmiterjoin%
\definecolor{currentfill}{rgb}{0.172549,0.627451,0.172549}%
\pgfsetfillcolor{currentfill}%
\pgfsetfillopacity{0.990000}%
\pgfsetlinewidth{0.000000pt}%
\definecolor{currentstroke}{rgb}{0.000000,0.000000,0.000000}%
\pgfsetstrokecolor{currentstroke}%
\pgfsetstrokeopacity{0.990000}%
\pgfsetdash{}{0pt}%
\pgfpathmoveto{\pgfqpoint{5.908038in}{3.636133in}}%
\pgfpathlineto{\pgfqpoint{6.683038in}{3.636133in}}%
\pgfpathlineto{\pgfqpoint{6.683038in}{3.636205in}}%
\pgfpathlineto{\pgfqpoint{5.908038in}{3.636205in}}%
\pgfpathclose%
\pgfusepath{fill}%
\end{pgfscope}%
\begin{pgfscope}%
\pgfsetbuttcap%
\pgfsetmiterjoin%
\definecolor{currentfill}{rgb}{0.172549,0.627451,0.172549}%
\pgfsetfillcolor{currentfill}%
\pgfsetfillopacity{0.990000}%
\pgfsetlinewidth{0.000000pt}%
\definecolor{currentstroke}{rgb}{0.000000,0.000000,0.000000}%
\pgfsetstrokecolor{currentstroke}%
\pgfsetstrokeopacity{0.990000}%
\pgfsetdash{}{0pt}%
\pgfpathrectangle{\pgfqpoint{0.870538in}{0.637495in}}{\pgfqpoint{9.300000in}{9.060000in}}%
\pgfusepath{clip}%
\pgfpathmoveto{\pgfqpoint{5.908038in}{3.636133in}}%
\pgfpathlineto{\pgfqpoint{6.683038in}{3.636133in}}%
\pgfpathlineto{\pgfqpoint{6.683038in}{3.636205in}}%
\pgfpathlineto{\pgfqpoint{5.908038in}{3.636205in}}%
\pgfpathclose%
\pgfusepath{clip}%
\pgfsys@defobject{currentpattern}{\pgfqpoint{0in}{0in}}{\pgfqpoint{1in}{1in}}{%
\begin{pgfscope}%
\pgfpathrectangle{\pgfqpoint{0in}{0in}}{\pgfqpoint{1in}{1in}}%
\pgfusepath{clip}%
\pgfpathmoveto{\pgfqpoint{0.000000in}{-0.016667in}}%
\pgfpathcurveto{\pgfqpoint{0.004420in}{-0.016667in}}{\pgfqpoint{0.008660in}{-0.014911in}}{\pgfqpoint{0.011785in}{-0.011785in}}%
\pgfpathcurveto{\pgfqpoint{0.014911in}{-0.008660in}}{\pgfqpoint{0.016667in}{-0.004420in}}{\pgfqpoint{0.016667in}{0.000000in}}%
\pgfpathcurveto{\pgfqpoint{0.016667in}{0.004420in}}{\pgfqpoint{0.014911in}{0.008660in}}{\pgfqpoint{0.011785in}{0.011785in}}%
\pgfpathcurveto{\pgfqpoint{0.008660in}{0.014911in}}{\pgfqpoint{0.004420in}{0.016667in}}{\pgfqpoint{0.000000in}{0.016667in}}%
\pgfpathcurveto{\pgfqpoint{-0.004420in}{0.016667in}}{\pgfqpoint{-0.008660in}{0.014911in}}{\pgfqpoint{-0.011785in}{0.011785in}}%
\pgfpathcurveto{\pgfqpoint{-0.014911in}{0.008660in}}{\pgfqpoint{-0.016667in}{0.004420in}}{\pgfqpoint{-0.016667in}{0.000000in}}%
\pgfpathcurveto{\pgfqpoint{-0.016667in}{-0.004420in}}{\pgfqpoint{-0.014911in}{-0.008660in}}{\pgfqpoint{-0.011785in}{-0.011785in}}%
\pgfpathcurveto{\pgfqpoint{-0.008660in}{-0.014911in}}{\pgfqpoint{-0.004420in}{-0.016667in}}{\pgfqpoint{0.000000in}{-0.016667in}}%
\pgfpathclose%
\pgfpathmoveto{\pgfqpoint{0.166667in}{-0.016667in}}%
\pgfpathcurveto{\pgfqpoint{0.171087in}{-0.016667in}}{\pgfqpoint{0.175326in}{-0.014911in}}{\pgfqpoint{0.178452in}{-0.011785in}}%
\pgfpathcurveto{\pgfqpoint{0.181577in}{-0.008660in}}{\pgfqpoint{0.183333in}{-0.004420in}}{\pgfqpoint{0.183333in}{0.000000in}}%
\pgfpathcurveto{\pgfqpoint{0.183333in}{0.004420in}}{\pgfqpoint{0.181577in}{0.008660in}}{\pgfqpoint{0.178452in}{0.011785in}}%
\pgfpathcurveto{\pgfqpoint{0.175326in}{0.014911in}}{\pgfqpoint{0.171087in}{0.016667in}}{\pgfqpoint{0.166667in}{0.016667in}}%
\pgfpathcurveto{\pgfqpoint{0.162247in}{0.016667in}}{\pgfqpoint{0.158007in}{0.014911in}}{\pgfqpoint{0.154882in}{0.011785in}}%
\pgfpathcurveto{\pgfqpoint{0.151756in}{0.008660in}}{\pgfqpoint{0.150000in}{0.004420in}}{\pgfqpoint{0.150000in}{0.000000in}}%
\pgfpathcurveto{\pgfqpoint{0.150000in}{-0.004420in}}{\pgfqpoint{0.151756in}{-0.008660in}}{\pgfqpoint{0.154882in}{-0.011785in}}%
\pgfpathcurveto{\pgfqpoint{0.158007in}{-0.014911in}}{\pgfqpoint{0.162247in}{-0.016667in}}{\pgfqpoint{0.166667in}{-0.016667in}}%
\pgfpathclose%
\pgfpathmoveto{\pgfqpoint{0.333333in}{-0.016667in}}%
\pgfpathcurveto{\pgfqpoint{0.337753in}{-0.016667in}}{\pgfqpoint{0.341993in}{-0.014911in}}{\pgfqpoint{0.345118in}{-0.011785in}}%
\pgfpathcurveto{\pgfqpoint{0.348244in}{-0.008660in}}{\pgfqpoint{0.350000in}{-0.004420in}}{\pgfqpoint{0.350000in}{0.000000in}}%
\pgfpathcurveto{\pgfqpoint{0.350000in}{0.004420in}}{\pgfqpoint{0.348244in}{0.008660in}}{\pgfqpoint{0.345118in}{0.011785in}}%
\pgfpathcurveto{\pgfqpoint{0.341993in}{0.014911in}}{\pgfqpoint{0.337753in}{0.016667in}}{\pgfqpoint{0.333333in}{0.016667in}}%
\pgfpathcurveto{\pgfqpoint{0.328913in}{0.016667in}}{\pgfqpoint{0.324674in}{0.014911in}}{\pgfqpoint{0.321548in}{0.011785in}}%
\pgfpathcurveto{\pgfqpoint{0.318423in}{0.008660in}}{\pgfqpoint{0.316667in}{0.004420in}}{\pgfqpoint{0.316667in}{0.000000in}}%
\pgfpathcurveto{\pgfqpoint{0.316667in}{-0.004420in}}{\pgfqpoint{0.318423in}{-0.008660in}}{\pgfqpoint{0.321548in}{-0.011785in}}%
\pgfpathcurveto{\pgfqpoint{0.324674in}{-0.014911in}}{\pgfqpoint{0.328913in}{-0.016667in}}{\pgfqpoint{0.333333in}{-0.016667in}}%
\pgfpathclose%
\pgfpathmoveto{\pgfqpoint{0.500000in}{-0.016667in}}%
\pgfpathcurveto{\pgfqpoint{0.504420in}{-0.016667in}}{\pgfqpoint{0.508660in}{-0.014911in}}{\pgfqpoint{0.511785in}{-0.011785in}}%
\pgfpathcurveto{\pgfqpoint{0.514911in}{-0.008660in}}{\pgfqpoint{0.516667in}{-0.004420in}}{\pgfqpoint{0.516667in}{0.000000in}}%
\pgfpathcurveto{\pgfqpoint{0.516667in}{0.004420in}}{\pgfqpoint{0.514911in}{0.008660in}}{\pgfqpoint{0.511785in}{0.011785in}}%
\pgfpathcurveto{\pgfqpoint{0.508660in}{0.014911in}}{\pgfqpoint{0.504420in}{0.016667in}}{\pgfqpoint{0.500000in}{0.016667in}}%
\pgfpathcurveto{\pgfqpoint{0.495580in}{0.016667in}}{\pgfqpoint{0.491340in}{0.014911in}}{\pgfqpoint{0.488215in}{0.011785in}}%
\pgfpathcurveto{\pgfqpoint{0.485089in}{0.008660in}}{\pgfqpoint{0.483333in}{0.004420in}}{\pgfqpoint{0.483333in}{0.000000in}}%
\pgfpathcurveto{\pgfqpoint{0.483333in}{-0.004420in}}{\pgfqpoint{0.485089in}{-0.008660in}}{\pgfqpoint{0.488215in}{-0.011785in}}%
\pgfpathcurveto{\pgfqpoint{0.491340in}{-0.014911in}}{\pgfqpoint{0.495580in}{-0.016667in}}{\pgfqpoint{0.500000in}{-0.016667in}}%
\pgfpathclose%
\pgfpathmoveto{\pgfqpoint{0.666667in}{-0.016667in}}%
\pgfpathcurveto{\pgfqpoint{0.671087in}{-0.016667in}}{\pgfqpoint{0.675326in}{-0.014911in}}{\pgfqpoint{0.678452in}{-0.011785in}}%
\pgfpathcurveto{\pgfqpoint{0.681577in}{-0.008660in}}{\pgfqpoint{0.683333in}{-0.004420in}}{\pgfqpoint{0.683333in}{0.000000in}}%
\pgfpathcurveto{\pgfqpoint{0.683333in}{0.004420in}}{\pgfqpoint{0.681577in}{0.008660in}}{\pgfqpoint{0.678452in}{0.011785in}}%
\pgfpathcurveto{\pgfqpoint{0.675326in}{0.014911in}}{\pgfqpoint{0.671087in}{0.016667in}}{\pgfqpoint{0.666667in}{0.016667in}}%
\pgfpathcurveto{\pgfqpoint{0.662247in}{0.016667in}}{\pgfqpoint{0.658007in}{0.014911in}}{\pgfqpoint{0.654882in}{0.011785in}}%
\pgfpathcurveto{\pgfqpoint{0.651756in}{0.008660in}}{\pgfqpoint{0.650000in}{0.004420in}}{\pgfqpoint{0.650000in}{0.000000in}}%
\pgfpathcurveto{\pgfqpoint{0.650000in}{-0.004420in}}{\pgfqpoint{0.651756in}{-0.008660in}}{\pgfqpoint{0.654882in}{-0.011785in}}%
\pgfpathcurveto{\pgfqpoint{0.658007in}{-0.014911in}}{\pgfqpoint{0.662247in}{-0.016667in}}{\pgfqpoint{0.666667in}{-0.016667in}}%
\pgfpathclose%
\pgfpathmoveto{\pgfqpoint{0.833333in}{-0.016667in}}%
\pgfpathcurveto{\pgfqpoint{0.837753in}{-0.016667in}}{\pgfqpoint{0.841993in}{-0.014911in}}{\pgfqpoint{0.845118in}{-0.011785in}}%
\pgfpathcurveto{\pgfqpoint{0.848244in}{-0.008660in}}{\pgfqpoint{0.850000in}{-0.004420in}}{\pgfqpoint{0.850000in}{0.000000in}}%
\pgfpathcurveto{\pgfqpoint{0.850000in}{0.004420in}}{\pgfqpoint{0.848244in}{0.008660in}}{\pgfqpoint{0.845118in}{0.011785in}}%
\pgfpathcurveto{\pgfqpoint{0.841993in}{0.014911in}}{\pgfqpoint{0.837753in}{0.016667in}}{\pgfqpoint{0.833333in}{0.016667in}}%
\pgfpathcurveto{\pgfqpoint{0.828913in}{0.016667in}}{\pgfqpoint{0.824674in}{0.014911in}}{\pgfqpoint{0.821548in}{0.011785in}}%
\pgfpathcurveto{\pgfqpoint{0.818423in}{0.008660in}}{\pgfqpoint{0.816667in}{0.004420in}}{\pgfqpoint{0.816667in}{0.000000in}}%
\pgfpathcurveto{\pgfqpoint{0.816667in}{-0.004420in}}{\pgfqpoint{0.818423in}{-0.008660in}}{\pgfqpoint{0.821548in}{-0.011785in}}%
\pgfpathcurveto{\pgfqpoint{0.824674in}{-0.014911in}}{\pgfqpoint{0.828913in}{-0.016667in}}{\pgfqpoint{0.833333in}{-0.016667in}}%
\pgfpathclose%
\pgfpathmoveto{\pgfqpoint{1.000000in}{-0.016667in}}%
\pgfpathcurveto{\pgfqpoint{1.004420in}{-0.016667in}}{\pgfqpoint{1.008660in}{-0.014911in}}{\pgfqpoint{1.011785in}{-0.011785in}}%
\pgfpathcurveto{\pgfqpoint{1.014911in}{-0.008660in}}{\pgfqpoint{1.016667in}{-0.004420in}}{\pgfqpoint{1.016667in}{0.000000in}}%
\pgfpathcurveto{\pgfqpoint{1.016667in}{0.004420in}}{\pgfqpoint{1.014911in}{0.008660in}}{\pgfqpoint{1.011785in}{0.011785in}}%
\pgfpathcurveto{\pgfqpoint{1.008660in}{0.014911in}}{\pgfqpoint{1.004420in}{0.016667in}}{\pgfqpoint{1.000000in}{0.016667in}}%
\pgfpathcurveto{\pgfqpoint{0.995580in}{0.016667in}}{\pgfqpoint{0.991340in}{0.014911in}}{\pgfqpoint{0.988215in}{0.011785in}}%
\pgfpathcurveto{\pgfqpoint{0.985089in}{0.008660in}}{\pgfqpoint{0.983333in}{0.004420in}}{\pgfqpoint{0.983333in}{0.000000in}}%
\pgfpathcurveto{\pgfqpoint{0.983333in}{-0.004420in}}{\pgfqpoint{0.985089in}{-0.008660in}}{\pgfqpoint{0.988215in}{-0.011785in}}%
\pgfpathcurveto{\pgfqpoint{0.991340in}{-0.014911in}}{\pgfqpoint{0.995580in}{-0.016667in}}{\pgfqpoint{1.000000in}{-0.016667in}}%
\pgfpathclose%
\pgfpathmoveto{\pgfqpoint{0.083333in}{0.150000in}}%
\pgfpathcurveto{\pgfqpoint{0.087753in}{0.150000in}}{\pgfqpoint{0.091993in}{0.151756in}}{\pgfqpoint{0.095118in}{0.154882in}}%
\pgfpathcurveto{\pgfqpoint{0.098244in}{0.158007in}}{\pgfqpoint{0.100000in}{0.162247in}}{\pgfqpoint{0.100000in}{0.166667in}}%
\pgfpathcurveto{\pgfqpoint{0.100000in}{0.171087in}}{\pgfqpoint{0.098244in}{0.175326in}}{\pgfqpoint{0.095118in}{0.178452in}}%
\pgfpathcurveto{\pgfqpoint{0.091993in}{0.181577in}}{\pgfqpoint{0.087753in}{0.183333in}}{\pgfqpoint{0.083333in}{0.183333in}}%
\pgfpathcurveto{\pgfqpoint{0.078913in}{0.183333in}}{\pgfqpoint{0.074674in}{0.181577in}}{\pgfqpoint{0.071548in}{0.178452in}}%
\pgfpathcurveto{\pgfqpoint{0.068423in}{0.175326in}}{\pgfqpoint{0.066667in}{0.171087in}}{\pgfqpoint{0.066667in}{0.166667in}}%
\pgfpathcurveto{\pgfqpoint{0.066667in}{0.162247in}}{\pgfqpoint{0.068423in}{0.158007in}}{\pgfqpoint{0.071548in}{0.154882in}}%
\pgfpathcurveto{\pgfqpoint{0.074674in}{0.151756in}}{\pgfqpoint{0.078913in}{0.150000in}}{\pgfqpoint{0.083333in}{0.150000in}}%
\pgfpathclose%
\pgfpathmoveto{\pgfqpoint{0.250000in}{0.150000in}}%
\pgfpathcurveto{\pgfqpoint{0.254420in}{0.150000in}}{\pgfqpoint{0.258660in}{0.151756in}}{\pgfqpoint{0.261785in}{0.154882in}}%
\pgfpathcurveto{\pgfqpoint{0.264911in}{0.158007in}}{\pgfqpoint{0.266667in}{0.162247in}}{\pgfqpoint{0.266667in}{0.166667in}}%
\pgfpathcurveto{\pgfqpoint{0.266667in}{0.171087in}}{\pgfqpoint{0.264911in}{0.175326in}}{\pgfqpoint{0.261785in}{0.178452in}}%
\pgfpathcurveto{\pgfqpoint{0.258660in}{0.181577in}}{\pgfqpoint{0.254420in}{0.183333in}}{\pgfqpoint{0.250000in}{0.183333in}}%
\pgfpathcurveto{\pgfqpoint{0.245580in}{0.183333in}}{\pgfqpoint{0.241340in}{0.181577in}}{\pgfqpoint{0.238215in}{0.178452in}}%
\pgfpathcurveto{\pgfqpoint{0.235089in}{0.175326in}}{\pgfqpoint{0.233333in}{0.171087in}}{\pgfqpoint{0.233333in}{0.166667in}}%
\pgfpathcurveto{\pgfqpoint{0.233333in}{0.162247in}}{\pgfqpoint{0.235089in}{0.158007in}}{\pgfqpoint{0.238215in}{0.154882in}}%
\pgfpathcurveto{\pgfqpoint{0.241340in}{0.151756in}}{\pgfqpoint{0.245580in}{0.150000in}}{\pgfqpoint{0.250000in}{0.150000in}}%
\pgfpathclose%
\pgfpathmoveto{\pgfqpoint{0.416667in}{0.150000in}}%
\pgfpathcurveto{\pgfqpoint{0.421087in}{0.150000in}}{\pgfqpoint{0.425326in}{0.151756in}}{\pgfqpoint{0.428452in}{0.154882in}}%
\pgfpathcurveto{\pgfqpoint{0.431577in}{0.158007in}}{\pgfqpoint{0.433333in}{0.162247in}}{\pgfqpoint{0.433333in}{0.166667in}}%
\pgfpathcurveto{\pgfqpoint{0.433333in}{0.171087in}}{\pgfqpoint{0.431577in}{0.175326in}}{\pgfqpoint{0.428452in}{0.178452in}}%
\pgfpathcurveto{\pgfqpoint{0.425326in}{0.181577in}}{\pgfqpoint{0.421087in}{0.183333in}}{\pgfqpoint{0.416667in}{0.183333in}}%
\pgfpathcurveto{\pgfqpoint{0.412247in}{0.183333in}}{\pgfqpoint{0.408007in}{0.181577in}}{\pgfqpoint{0.404882in}{0.178452in}}%
\pgfpathcurveto{\pgfqpoint{0.401756in}{0.175326in}}{\pgfqpoint{0.400000in}{0.171087in}}{\pgfqpoint{0.400000in}{0.166667in}}%
\pgfpathcurveto{\pgfqpoint{0.400000in}{0.162247in}}{\pgfqpoint{0.401756in}{0.158007in}}{\pgfqpoint{0.404882in}{0.154882in}}%
\pgfpathcurveto{\pgfqpoint{0.408007in}{0.151756in}}{\pgfqpoint{0.412247in}{0.150000in}}{\pgfqpoint{0.416667in}{0.150000in}}%
\pgfpathclose%
\pgfpathmoveto{\pgfqpoint{0.583333in}{0.150000in}}%
\pgfpathcurveto{\pgfqpoint{0.587753in}{0.150000in}}{\pgfqpoint{0.591993in}{0.151756in}}{\pgfqpoint{0.595118in}{0.154882in}}%
\pgfpathcurveto{\pgfqpoint{0.598244in}{0.158007in}}{\pgfqpoint{0.600000in}{0.162247in}}{\pgfqpoint{0.600000in}{0.166667in}}%
\pgfpathcurveto{\pgfqpoint{0.600000in}{0.171087in}}{\pgfqpoint{0.598244in}{0.175326in}}{\pgfqpoint{0.595118in}{0.178452in}}%
\pgfpathcurveto{\pgfqpoint{0.591993in}{0.181577in}}{\pgfqpoint{0.587753in}{0.183333in}}{\pgfqpoint{0.583333in}{0.183333in}}%
\pgfpathcurveto{\pgfqpoint{0.578913in}{0.183333in}}{\pgfqpoint{0.574674in}{0.181577in}}{\pgfqpoint{0.571548in}{0.178452in}}%
\pgfpathcurveto{\pgfqpoint{0.568423in}{0.175326in}}{\pgfqpoint{0.566667in}{0.171087in}}{\pgfqpoint{0.566667in}{0.166667in}}%
\pgfpathcurveto{\pgfqpoint{0.566667in}{0.162247in}}{\pgfqpoint{0.568423in}{0.158007in}}{\pgfqpoint{0.571548in}{0.154882in}}%
\pgfpathcurveto{\pgfqpoint{0.574674in}{0.151756in}}{\pgfqpoint{0.578913in}{0.150000in}}{\pgfqpoint{0.583333in}{0.150000in}}%
\pgfpathclose%
\pgfpathmoveto{\pgfqpoint{0.750000in}{0.150000in}}%
\pgfpathcurveto{\pgfqpoint{0.754420in}{0.150000in}}{\pgfqpoint{0.758660in}{0.151756in}}{\pgfqpoint{0.761785in}{0.154882in}}%
\pgfpathcurveto{\pgfqpoint{0.764911in}{0.158007in}}{\pgfqpoint{0.766667in}{0.162247in}}{\pgfqpoint{0.766667in}{0.166667in}}%
\pgfpathcurveto{\pgfqpoint{0.766667in}{0.171087in}}{\pgfqpoint{0.764911in}{0.175326in}}{\pgfqpoint{0.761785in}{0.178452in}}%
\pgfpathcurveto{\pgfqpoint{0.758660in}{0.181577in}}{\pgfqpoint{0.754420in}{0.183333in}}{\pgfqpoint{0.750000in}{0.183333in}}%
\pgfpathcurveto{\pgfqpoint{0.745580in}{0.183333in}}{\pgfqpoint{0.741340in}{0.181577in}}{\pgfqpoint{0.738215in}{0.178452in}}%
\pgfpathcurveto{\pgfqpoint{0.735089in}{0.175326in}}{\pgfqpoint{0.733333in}{0.171087in}}{\pgfqpoint{0.733333in}{0.166667in}}%
\pgfpathcurveto{\pgfqpoint{0.733333in}{0.162247in}}{\pgfqpoint{0.735089in}{0.158007in}}{\pgfqpoint{0.738215in}{0.154882in}}%
\pgfpathcurveto{\pgfqpoint{0.741340in}{0.151756in}}{\pgfqpoint{0.745580in}{0.150000in}}{\pgfqpoint{0.750000in}{0.150000in}}%
\pgfpathclose%
\pgfpathmoveto{\pgfqpoint{0.916667in}{0.150000in}}%
\pgfpathcurveto{\pgfqpoint{0.921087in}{0.150000in}}{\pgfqpoint{0.925326in}{0.151756in}}{\pgfqpoint{0.928452in}{0.154882in}}%
\pgfpathcurveto{\pgfqpoint{0.931577in}{0.158007in}}{\pgfqpoint{0.933333in}{0.162247in}}{\pgfqpoint{0.933333in}{0.166667in}}%
\pgfpathcurveto{\pgfqpoint{0.933333in}{0.171087in}}{\pgfqpoint{0.931577in}{0.175326in}}{\pgfqpoint{0.928452in}{0.178452in}}%
\pgfpathcurveto{\pgfqpoint{0.925326in}{0.181577in}}{\pgfqpoint{0.921087in}{0.183333in}}{\pgfqpoint{0.916667in}{0.183333in}}%
\pgfpathcurveto{\pgfqpoint{0.912247in}{0.183333in}}{\pgfqpoint{0.908007in}{0.181577in}}{\pgfqpoint{0.904882in}{0.178452in}}%
\pgfpathcurveto{\pgfqpoint{0.901756in}{0.175326in}}{\pgfqpoint{0.900000in}{0.171087in}}{\pgfqpoint{0.900000in}{0.166667in}}%
\pgfpathcurveto{\pgfqpoint{0.900000in}{0.162247in}}{\pgfqpoint{0.901756in}{0.158007in}}{\pgfqpoint{0.904882in}{0.154882in}}%
\pgfpathcurveto{\pgfqpoint{0.908007in}{0.151756in}}{\pgfqpoint{0.912247in}{0.150000in}}{\pgfqpoint{0.916667in}{0.150000in}}%
\pgfpathclose%
\pgfpathmoveto{\pgfqpoint{0.000000in}{0.316667in}}%
\pgfpathcurveto{\pgfqpoint{0.004420in}{0.316667in}}{\pgfqpoint{0.008660in}{0.318423in}}{\pgfqpoint{0.011785in}{0.321548in}}%
\pgfpathcurveto{\pgfqpoint{0.014911in}{0.324674in}}{\pgfqpoint{0.016667in}{0.328913in}}{\pgfqpoint{0.016667in}{0.333333in}}%
\pgfpathcurveto{\pgfqpoint{0.016667in}{0.337753in}}{\pgfqpoint{0.014911in}{0.341993in}}{\pgfqpoint{0.011785in}{0.345118in}}%
\pgfpathcurveto{\pgfqpoint{0.008660in}{0.348244in}}{\pgfqpoint{0.004420in}{0.350000in}}{\pgfqpoint{0.000000in}{0.350000in}}%
\pgfpathcurveto{\pgfqpoint{-0.004420in}{0.350000in}}{\pgfqpoint{-0.008660in}{0.348244in}}{\pgfqpoint{-0.011785in}{0.345118in}}%
\pgfpathcurveto{\pgfqpoint{-0.014911in}{0.341993in}}{\pgfqpoint{-0.016667in}{0.337753in}}{\pgfqpoint{-0.016667in}{0.333333in}}%
\pgfpathcurveto{\pgfqpoint{-0.016667in}{0.328913in}}{\pgfqpoint{-0.014911in}{0.324674in}}{\pgfqpoint{-0.011785in}{0.321548in}}%
\pgfpathcurveto{\pgfqpoint{-0.008660in}{0.318423in}}{\pgfqpoint{-0.004420in}{0.316667in}}{\pgfqpoint{0.000000in}{0.316667in}}%
\pgfpathclose%
\pgfpathmoveto{\pgfqpoint{0.166667in}{0.316667in}}%
\pgfpathcurveto{\pgfqpoint{0.171087in}{0.316667in}}{\pgfqpoint{0.175326in}{0.318423in}}{\pgfqpoint{0.178452in}{0.321548in}}%
\pgfpathcurveto{\pgfqpoint{0.181577in}{0.324674in}}{\pgfqpoint{0.183333in}{0.328913in}}{\pgfqpoint{0.183333in}{0.333333in}}%
\pgfpathcurveto{\pgfqpoint{0.183333in}{0.337753in}}{\pgfqpoint{0.181577in}{0.341993in}}{\pgfqpoint{0.178452in}{0.345118in}}%
\pgfpathcurveto{\pgfqpoint{0.175326in}{0.348244in}}{\pgfqpoint{0.171087in}{0.350000in}}{\pgfqpoint{0.166667in}{0.350000in}}%
\pgfpathcurveto{\pgfqpoint{0.162247in}{0.350000in}}{\pgfqpoint{0.158007in}{0.348244in}}{\pgfqpoint{0.154882in}{0.345118in}}%
\pgfpathcurveto{\pgfqpoint{0.151756in}{0.341993in}}{\pgfqpoint{0.150000in}{0.337753in}}{\pgfqpoint{0.150000in}{0.333333in}}%
\pgfpathcurveto{\pgfqpoint{0.150000in}{0.328913in}}{\pgfqpoint{0.151756in}{0.324674in}}{\pgfqpoint{0.154882in}{0.321548in}}%
\pgfpathcurveto{\pgfqpoint{0.158007in}{0.318423in}}{\pgfqpoint{0.162247in}{0.316667in}}{\pgfqpoint{0.166667in}{0.316667in}}%
\pgfpathclose%
\pgfpathmoveto{\pgfqpoint{0.333333in}{0.316667in}}%
\pgfpathcurveto{\pgfqpoint{0.337753in}{0.316667in}}{\pgfqpoint{0.341993in}{0.318423in}}{\pgfqpoint{0.345118in}{0.321548in}}%
\pgfpathcurveto{\pgfqpoint{0.348244in}{0.324674in}}{\pgfqpoint{0.350000in}{0.328913in}}{\pgfqpoint{0.350000in}{0.333333in}}%
\pgfpathcurveto{\pgfqpoint{0.350000in}{0.337753in}}{\pgfqpoint{0.348244in}{0.341993in}}{\pgfqpoint{0.345118in}{0.345118in}}%
\pgfpathcurveto{\pgfqpoint{0.341993in}{0.348244in}}{\pgfqpoint{0.337753in}{0.350000in}}{\pgfqpoint{0.333333in}{0.350000in}}%
\pgfpathcurveto{\pgfqpoint{0.328913in}{0.350000in}}{\pgfqpoint{0.324674in}{0.348244in}}{\pgfqpoint{0.321548in}{0.345118in}}%
\pgfpathcurveto{\pgfqpoint{0.318423in}{0.341993in}}{\pgfqpoint{0.316667in}{0.337753in}}{\pgfqpoint{0.316667in}{0.333333in}}%
\pgfpathcurveto{\pgfqpoint{0.316667in}{0.328913in}}{\pgfqpoint{0.318423in}{0.324674in}}{\pgfqpoint{0.321548in}{0.321548in}}%
\pgfpathcurveto{\pgfqpoint{0.324674in}{0.318423in}}{\pgfqpoint{0.328913in}{0.316667in}}{\pgfqpoint{0.333333in}{0.316667in}}%
\pgfpathclose%
\pgfpathmoveto{\pgfqpoint{0.500000in}{0.316667in}}%
\pgfpathcurveto{\pgfqpoint{0.504420in}{0.316667in}}{\pgfqpoint{0.508660in}{0.318423in}}{\pgfqpoint{0.511785in}{0.321548in}}%
\pgfpathcurveto{\pgfqpoint{0.514911in}{0.324674in}}{\pgfqpoint{0.516667in}{0.328913in}}{\pgfqpoint{0.516667in}{0.333333in}}%
\pgfpathcurveto{\pgfqpoint{0.516667in}{0.337753in}}{\pgfqpoint{0.514911in}{0.341993in}}{\pgfqpoint{0.511785in}{0.345118in}}%
\pgfpathcurveto{\pgfqpoint{0.508660in}{0.348244in}}{\pgfqpoint{0.504420in}{0.350000in}}{\pgfqpoint{0.500000in}{0.350000in}}%
\pgfpathcurveto{\pgfqpoint{0.495580in}{0.350000in}}{\pgfqpoint{0.491340in}{0.348244in}}{\pgfqpoint{0.488215in}{0.345118in}}%
\pgfpathcurveto{\pgfqpoint{0.485089in}{0.341993in}}{\pgfqpoint{0.483333in}{0.337753in}}{\pgfqpoint{0.483333in}{0.333333in}}%
\pgfpathcurveto{\pgfqpoint{0.483333in}{0.328913in}}{\pgfqpoint{0.485089in}{0.324674in}}{\pgfqpoint{0.488215in}{0.321548in}}%
\pgfpathcurveto{\pgfqpoint{0.491340in}{0.318423in}}{\pgfqpoint{0.495580in}{0.316667in}}{\pgfqpoint{0.500000in}{0.316667in}}%
\pgfpathclose%
\pgfpathmoveto{\pgfqpoint{0.666667in}{0.316667in}}%
\pgfpathcurveto{\pgfqpoint{0.671087in}{0.316667in}}{\pgfqpoint{0.675326in}{0.318423in}}{\pgfqpoint{0.678452in}{0.321548in}}%
\pgfpathcurveto{\pgfqpoint{0.681577in}{0.324674in}}{\pgfqpoint{0.683333in}{0.328913in}}{\pgfqpoint{0.683333in}{0.333333in}}%
\pgfpathcurveto{\pgfqpoint{0.683333in}{0.337753in}}{\pgfqpoint{0.681577in}{0.341993in}}{\pgfqpoint{0.678452in}{0.345118in}}%
\pgfpathcurveto{\pgfqpoint{0.675326in}{0.348244in}}{\pgfqpoint{0.671087in}{0.350000in}}{\pgfqpoint{0.666667in}{0.350000in}}%
\pgfpathcurveto{\pgfqpoint{0.662247in}{0.350000in}}{\pgfqpoint{0.658007in}{0.348244in}}{\pgfqpoint{0.654882in}{0.345118in}}%
\pgfpathcurveto{\pgfqpoint{0.651756in}{0.341993in}}{\pgfqpoint{0.650000in}{0.337753in}}{\pgfqpoint{0.650000in}{0.333333in}}%
\pgfpathcurveto{\pgfqpoint{0.650000in}{0.328913in}}{\pgfqpoint{0.651756in}{0.324674in}}{\pgfqpoint{0.654882in}{0.321548in}}%
\pgfpathcurveto{\pgfqpoint{0.658007in}{0.318423in}}{\pgfqpoint{0.662247in}{0.316667in}}{\pgfqpoint{0.666667in}{0.316667in}}%
\pgfpathclose%
\pgfpathmoveto{\pgfqpoint{0.833333in}{0.316667in}}%
\pgfpathcurveto{\pgfqpoint{0.837753in}{0.316667in}}{\pgfqpoint{0.841993in}{0.318423in}}{\pgfqpoint{0.845118in}{0.321548in}}%
\pgfpathcurveto{\pgfqpoint{0.848244in}{0.324674in}}{\pgfqpoint{0.850000in}{0.328913in}}{\pgfqpoint{0.850000in}{0.333333in}}%
\pgfpathcurveto{\pgfqpoint{0.850000in}{0.337753in}}{\pgfqpoint{0.848244in}{0.341993in}}{\pgfqpoint{0.845118in}{0.345118in}}%
\pgfpathcurveto{\pgfqpoint{0.841993in}{0.348244in}}{\pgfqpoint{0.837753in}{0.350000in}}{\pgfqpoint{0.833333in}{0.350000in}}%
\pgfpathcurveto{\pgfqpoint{0.828913in}{0.350000in}}{\pgfqpoint{0.824674in}{0.348244in}}{\pgfqpoint{0.821548in}{0.345118in}}%
\pgfpathcurveto{\pgfqpoint{0.818423in}{0.341993in}}{\pgfqpoint{0.816667in}{0.337753in}}{\pgfqpoint{0.816667in}{0.333333in}}%
\pgfpathcurveto{\pgfqpoint{0.816667in}{0.328913in}}{\pgfqpoint{0.818423in}{0.324674in}}{\pgfqpoint{0.821548in}{0.321548in}}%
\pgfpathcurveto{\pgfqpoint{0.824674in}{0.318423in}}{\pgfqpoint{0.828913in}{0.316667in}}{\pgfqpoint{0.833333in}{0.316667in}}%
\pgfpathclose%
\pgfpathmoveto{\pgfqpoint{1.000000in}{0.316667in}}%
\pgfpathcurveto{\pgfqpoint{1.004420in}{0.316667in}}{\pgfqpoint{1.008660in}{0.318423in}}{\pgfqpoint{1.011785in}{0.321548in}}%
\pgfpathcurveto{\pgfqpoint{1.014911in}{0.324674in}}{\pgfqpoint{1.016667in}{0.328913in}}{\pgfqpoint{1.016667in}{0.333333in}}%
\pgfpathcurveto{\pgfqpoint{1.016667in}{0.337753in}}{\pgfqpoint{1.014911in}{0.341993in}}{\pgfqpoint{1.011785in}{0.345118in}}%
\pgfpathcurveto{\pgfqpoint{1.008660in}{0.348244in}}{\pgfqpoint{1.004420in}{0.350000in}}{\pgfqpoint{1.000000in}{0.350000in}}%
\pgfpathcurveto{\pgfqpoint{0.995580in}{0.350000in}}{\pgfqpoint{0.991340in}{0.348244in}}{\pgfqpoint{0.988215in}{0.345118in}}%
\pgfpathcurveto{\pgfqpoint{0.985089in}{0.341993in}}{\pgfqpoint{0.983333in}{0.337753in}}{\pgfqpoint{0.983333in}{0.333333in}}%
\pgfpathcurveto{\pgfqpoint{0.983333in}{0.328913in}}{\pgfqpoint{0.985089in}{0.324674in}}{\pgfqpoint{0.988215in}{0.321548in}}%
\pgfpathcurveto{\pgfqpoint{0.991340in}{0.318423in}}{\pgfqpoint{0.995580in}{0.316667in}}{\pgfqpoint{1.000000in}{0.316667in}}%
\pgfpathclose%
\pgfpathmoveto{\pgfqpoint{0.083333in}{0.483333in}}%
\pgfpathcurveto{\pgfqpoint{0.087753in}{0.483333in}}{\pgfqpoint{0.091993in}{0.485089in}}{\pgfqpoint{0.095118in}{0.488215in}}%
\pgfpathcurveto{\pgfqpoint{0.098244in}{0.491340in}}{\pgfqpoint{0.100000in}{0.495580in}}{\pgfqpoint{0.100000in}{0.500000in}}%
\pgfpathcurveto{\pgfqpoint{0.100000in}{0.504420in}}{\pgfqpoint{0.098244in}{0.508660in}}{\pgfqpoint{0.095118in}{0.511785in}}%
\pgfpathcurveto{\pgfqpoint{0.091993in}{0.514911in}}{\pgfqpoint{0.087753in}{0.516667in}}{\pgfqpoint{0.083333in}{0.516667in}}%
\pgfpathcurveto{\pgfqpoint{0.078913in}{0.516667in}}{\pgfqpoint{0.074674in}{0.514911in}}{\pgfqpoint{0.071548in}{0.511785in}}%
\pgfpathcurveto{\pgfqpoint{0.068423in}{0.508660in}}{\pgfqpoint{0.066667in}{0.504420in}}{\pgfqpoint{0.066667in}{0.500000in}}%
\pgfpathcurveto{\pgfqpoint{0.066667in}{0.495580in}}{\pgfqpoint{0.068423in}{0.491340in}}{\pgfqpoint{0.071548in}{0.488215in}}%
\pgfpathcurveto{\pgfqpoint{0.074674in}{0.485089in}}{\pgfqpoint{0.078913in}{0.483333in}}{\pgfqpoint{0.083333in}{0.483333in}}%
\pgfpathclose%
\pgfpathmoveto{\pgfqpoint{0.250000in}{0.483333in}}%
\pgfpathcurveto{\pgfqpoint{0.254420in}{0.483333in}}{\pgfqpoint{0.258660in}{0.485089in}}{\pgfqpoint{0.261785in}{0.488215in}}%
\pgfpathcurveto{\pgfqpoint{0.264911in}{0.491340in}}{\pgfqpoint{0.266667in}{0.495580in}}{\pgfqpoint{0.266667in}{0.500000in}}%
\pgfpathcurveto{\pgfqpoint{0.266667in}{0.504420in}}{\pgfqpoint{0.264911in}{0.508660in}}{\pgfqpoint{0.261785in}{0.511785in}}%
\pgfpathcurveto{\pgfqpoint{0.258660in}{0.514911in}}{\pgfqpoint{0.254420in}{0.516667in}}{\pgfqpoint{0.250000in}{0.516667in}}%
\pgfpathcurveto{\pgfqpoint{0.245580in}{0.516667in}}{\pgfqpoint{0.241340in}{0.514911in}}{\pgfqpoint{0.238215in}{0.511785in}}%
\pgfpathcurveto{\pgfqpoint{0.235089in}{0.508660in}}{\pgfqpoint{0.233333in}{0.504420in}}{\pgfqpoint{0.233333in}{0.500000in}}%
\pgfpathcurveto{\pgfqpoint{0.233333in}{0.495580in}}{\pgfqpoint{0.235089in}{0.491340in}}{\pgfqpoint{0.238215in}{0.488215in}}%
\pgfpathcurveto{\pgfqpoint{0.241340in}{0.485089in}}{\pgfqpoint{0.245580in}{0.483333in}}{\pgfqpoint{0.250000in}{0.483333in}}%
\pgfpathclose%
\pgfpathmoveto{\pgfqpoint{0.416667in}{0.483333in}}%
\pgfpathcurveto{\pgfqpoint{0.421087in}{0.483333in}}{\pgfqpoint{0.425326in}{0.485089in}}{\pgfqpoint{0.428452in}{0.488215in}}%
\pgfpathcurveto{\pgfqpoint{0.431577in}{0.491340in}}{\pgfqpoint{0.433333in}{0.495580in}}{\pgfqpoint{0.433333in}{0.500000in}}%
\pgfpathcurveto{\pgfqpoint{0.433333in}{0.504420in}}{\pgfqpoint{0.431577in}{0.508660in}}{\pgfqpoint{0.428452in}{0.511785in}}%
\pgfpathcurveto{\pgfqpoint{0.425326in}{0.514911in}}{\pgfqpoint{0.421087in}{0.516667in}}{\pgfqpoint{0.416667in}{0.516667in}}%
\pgfpathcurveto{\pgfqpoint{0.412247in}{0.516667in}}{\pgfqpoint{0.408007in}{0.514911in}}{\pgfqpoint{0.404882in}{0.511785in}}%
\pgfpathcurveto{\pgfqpoint{0.401756in}{0.508660in}}{\pgfqpoint{0.400000in}{0.504420in}}{\pgfqpoint{0.400000in}{0.500000in}}%
\pgfpathcurveto{\pgfqpoint{0.400000in}{0.495580in}}{\pgfqpoint{0.401756in}{0.491340in}}{\pgfqpoint{0.404882in}{0.488215in}}%
\pgfpathcurveto{\pgfqpoint{0.408007in}{0.485089in}}{\pgfqpoint{0.412247in}{0.483333in}}{\pgfqpoint{0.416667in}{0.483333in}}%
\pgfpathclose%
\pgfpathmoveto{\pgfqpoint{0.583333in}{0.483333in}}%
\pgfpathcurveto{\pgfqpoint{0.587753in}{0.483333in}}{\pgfqpoint{0.591993in}{0.485089in}}{\pgfqpoint{0.595118in}{0.488215in}}%
\pgfpathcurveto{\pgfqpoint{0.598244in}{0.491340in}}{\pgfqpoint{0.600000in}{0.495580in}}{\pgfqpoint{0.600000in}{0.500000in}}%
\pgfpathcurveto{\pgfqpoint{0.600000in}{0.504420in}}{\pgfqpoint{0.598244in}{0.508660in}}{\pgfqpoint{0.595118in}{0.511785in}}%
\pgfpathcurveto{\pgfqpoint{0.591993in}{0.514911in}}{\pgfqpoint{0.587753in}{0.516667in}}{\pgfqpoint{0.583333in}{0.516667in}}%
\pgfpathcurveto{\pgfqpoint{0.578913in}{0.516667in}}{\pgfqpoint{0.574674in}{0.514911in}}{\pgfqpoint{0.571548in}{0.511785in}}%
\pgfpathcurveto{\pgfqpoint{0.568423in}{0.508660in}}{\pgfqpoint{0.566667in}{0.504420in}}{\pgfqpoint{0.566667in}{0.500000in}}%
\pgfpathcurveto{\pgfqpoint{0.566667in}{0.495580in}}{\pgfqpoint{0.568423in}{0.491340in}}{\pgfqpoint{0.571548in}{0.488215in}}%
\pgfpathcurveto{\pgfqpoint{0.574674in}{0.485089in}}{\pgfqpoint{0.578913in}{0.483333in}}{\pgfqpoint{0.583333in}{0.483333in}}%
\pgfpathclose%
\pgfpathmoveto{\pgfqpoint{0.750000in}{0.483333in}}%
\pgfpathcurveto{\pgfqpoint{0.754420in}{0.483333in}}{\pgfqpoint{0.758660in}{0.485089in}}{\pgfqpoint{0.761785in}{0.488215in}}%
\pgfpathcurveto{\pgfqpoint{0.764911in}{0.491340in}}{\pgfqpoint{0.766667in}{0.495580in}}{\pgfqpoint{0.766667in}{0.500000in}}%
\pgfpathcurveto{\pgfqpoint{0.766667in}{0.504420in}}{\pgfqpoint{0.764911in}{0.508660in}}{\pgfqpoint{0.761785in}{0.511785in}}%
\pgfpathcurveto{\pgfqpoint{0.758660in}{0.514911in}}{\pgfqpoint{0.754420in}{0.516667in}}{\pgfqpoint{0.750000in}{0.516667in}}%
\pgfpathcurveto{\pgfqpoint{0.745580in}{0.516667in}}{\pgfqpoint{0.741340in}{0.514911in}}{\pgfqpoint{0.738215in}{0.511785in}}%
\pgfpathcurveto{\pgfqpoint{0.735089in}{0.508660in}}{\pgfqpoint{0.733333in}{0.504420in}}{\pgfqpoint{0.733333in}{0.500000in}}%
\pgfpathcurveto{\pgfqpoint{0.733333in}{0.495580in}}{\pgfqpoint{0.735089in}{0.491340in}}{\pgfqpoint{0.738215in}{0.488215in}}%
\pgfpathcurveto{\pgfqpoint{0.741340in}{0.485089in}}{\pgfqpoint{0.745580in}{0.483333in}}{\pgfqpoint{0.750000in}{0.483333in}}%
\pgfpathclose%
\pgfpathmoveto{\pgfqpoint{0.916667in}{0.483333in}}%
\pgfpathcurveto{\pgfqpoint{0.921087in}{0.483333in}}{\pgfqpoint{0.925326in}{0.485089in}}{\pgfqpoint{0.928452in}{0.488215in}}%
\pgfpathcurveto{\pgfqpoint{0.931577in}{0.491340in}}{\pgfqpoint{0.933333in}{0.495580in}}{\pgfqpoint{0.933333in}{0.500000in}}%
\pgfpathcurveto{\pgfqpoint{0.933333in}{0.504420in}}{\pgfqpoint{0.931577in}{0.508660in}}{\pgfqpoint{0.928452in}{0.511785in}}%
\pgfpathcurveto{\pgfqpoint{0.925326in}{0.514911in}}{\pgfqpoint{0.921087in}{0.516667in}}{\pgfqpoint{0.916667in}{0.516667in}}%
\pgfpathcurveto{\pgfqpoint{0.912247in}{0.516667in}}{\pgfqpoint{0.908007in}{0.514911in}}{\pgfqpoint{0.904882in}{0.511785in}}%
\pgfpathcurveto{\pgfqpoint{0.901756in}{0.508660in}}{\pgfqpoint{0.900000in}{0.504420in}}{\pgfqpoint{0.900000in}{0.500000in}}%
\pgfpathcurveto{\pgfqpoint{0.900000in}{0.495580in}}{\pgfqpoint{0.901756in}{0.491340in}}{\pgfqpoint{0.904882in}{0.488215in}}%
\pgfpathcurveto{\pgfqpoint{0.908007in}{0.485089in}}{\pgfqpoint{0.912247in}{0.483333in}}{\pgfqpoint{0.916667in}{0.483333in}}%
\pgfpathclose%
\pgfpathmoveto{\pgfqpoint{0.000000in}{0.650000in}}%
\pgfpathcurveto{\pgfqpoint{0.004420in}{0.650000in}}{\pgfqpoint{0.008660in}{0.651756in}}{\pgfqpoint{0.011785in}{0.654882in}}%
\pgfpathcurveto{\pgfqpoint{0.014911in}{0.658007in}}{\pgfqpoint{0.016667in}{0.662247in}}{\pgfqpoint{0.016667in}{0.666667in}}%
\pgfpathcurveto{\pgfqpoint{0.016667in}{0.671087in}}{\pgfqpoint{0.014911in}{0.675326in}}{\pgfqpoint{0.011785in}{0.678452in}}%
\pgfpathcurveto{\pgfqpoint{0.008660in}{0.681577in}}{\pgfqpoint{0.004420in}{0.683333in}}{\pgfqpoint{0.000000in}{0.683333in}}%
\pgfpathcurveto{\pgfqpoint{-0.004420in}{0.683333in}}{\pgfqpoint{-0.008660in}{0.681577in}}{\pgfqpoint{-0.011785in}{0.678452in}}%
\pgfpathcurveto{\pgfqpoint{-0.014911in}{0.675326in}}{\pgfqpoint{-0.016667in}{0.671087in}}{\pgfqpoint{-0.016667in}{0.666667in}}%
\pgfpathcurveto{\pgfqpoint{-0.016667in}{0.662247in}}{\pgfqpoint{-0.014911in}{0.658007in}}{\pgfqpoint{-0.011785in}{0.654882in}}%
\pgfpathcurveto{\pgfqpoint{-0.008660in}{0.651756in}}{\pgfqpoint{-0.004420in}{0.650000in}}{\pgfqpoint{0.000000in}{0.650000in}}%
\pgfpathclose%
\pgfpathmoveto{\pgfqpoint{0.166667in}{0.650000in}}%
\pgfpathcurveto{\pgfqpoint{0.171087in}{0.650000in}}{\pgfqpoint{0.175326in}{0.651756in}}{\pgfqpoint{0.178452in}{0.654882in}}%
\pgfpathcurveto{\pgfqpoint{0.181577in}{0.658007in}}{\pgfqpoint{0.183333in}{0.662247in}}{\pgfqpoint{0.183333in}{0.666667in}}%
\pgfpathcurveto{\pgfqpoint{0.183333in}{0.671087in}}{\pgfqpoint{0.181577in}{0.675326in}}{\pgfqpoint{0.178452in}{0.678452in}}%
\pgfpathcurveto{\pgfqpoint{0.175326in}{0.681577in}}{\pgfqpoint{0.171087in}{0.683333in}}{\pgfqpoint{0.166667in}{0.683333in}}%
\pgfpathcurveto{\pgfqpoint{0.162247in}{0.683333in}}{\pgfqpoint{0.158007in}{0.681577in}}{\pgfqpoint{0.154882in}{0.678452in}}%
\pgfpathcurveto{\pgfqpoint{0.151756in}{0.675326in}}{\pgfqpoint{0.150000in}{0.671087in}}{\pgfqpoint{0.150000in}{0.666667in}}%
\pgfpathcurveto{\pgfqpoint{0.150000in}{0.662247in}}{\pgfqpoint{0.151756in}{0.658007in}}{\pgfqpoint{0.154882in}{0.654882in}}%
\pgfpathcurveto{\pgfqpoint{0.158007in}{0.651756in}}{\pgfqpoint{0.162247in}{0.650000in}}{\pgfqpoint{0.166667in}{0.650000in}}%
\pgfpathclose%
\pgfpathmoveto{\pgfqpoint{0.333333in}{0.650000in}}%
\pgfpathcurveto{\pgfqpoint{0.337753in}{0.650000in}}{\pgfqpoint{0.341993in}{0.651756in}}{\pgfqpoint{0.345118in}{0.654882in}}%
\pgfpathcurveto{\pgfqpoint{0.348244in}{0.658007in}}{\pgfqpoint{0.350000in}{0.662247in}}{\pgfqpoint{0.350000in}{0.666667in}}%
\pgfpathcurveto{\pgfqpoint{0.350000in}{0.671087in}}{\pgfqpoint{0.348244in}{0.675326in}}{\pgfqpoint{0.345118in}{0.678452in}}%
\pgfpathcurveto{\pgfqpoint{0.341993in}{0.681577in}}{\pgfqpoint{0.337753in}{0.683333in}}{\pgfqpoint{0.333333in}{0.683333in}}%
\pgfpathcurveto{\pgfqpoint{0.328913in}{0.683333in}}{\pgfqpoint{0.324674in}{0.681577in}}{\pgfqpoint{0.321548in}{0.678452in}}%
\pgfpathcurveto{\pgfqpoint{0.318423in}{0.675326in}}{\pgfqpoint{0.316667in}{0.671087in}}{\pgfqpoint{0.316667in}{0.666667in}}%
\pgfpathcurveto{\pgfqpoint{0.316667in}{0.662247in}}{\pgfqpoint{0.318423in}{0.658007in}}{\pgfqpoint{0.321548in}{0.654882in}}%
\pgfpathcurveto{\pgfqpoint{0.324674in}{0.651756in}}{\pgfqpoint{0.328913in}{0.650000in}}{\pgfqpoint{0.333333in}{0.650000in}}%
\pgfpathclose%
\pgfpathmoveto{\pgfqpoint{0.500000in}{0.650000in}}%
\pgfpathcurveto{\pgfqpoint{0.504420in}{0.650000in}}{\pgfqpoint{0.508660in}{0.651756in}}{\pgfqpoint{0.511785in}{0.654882in}}%
\pgfpathcurveto{\pgfqpoint{0.514911in}{0.658007in}}{\pgfqpoint{0.516667in}{0.662247in}}{\pgfqpoint{0.516667in}{0.666667in}}%
\pgfpathcurveto{\pgfqpoint{0.516667in}{0.671087in}}{\pgfqpoint{0.514911in}{0.675326in}}{\pgfqpoint{0.511785in}{0.678452in}}%
\pgfpathcurveto{\pgfqpoint{0.508660in}{0.681577in}}{\pgfqpoint{0.504420in}{0.683333in}}{\pgfqpoint{0.500000in}{0.683333in}}%
\pgfpathcurveto{\pgfqpoint{0.495580in}{0.683333in}}{\pgfqpoint{0.491340in}{0.681577in}}{\pgfqpoint{0.488215in}{0.678452in}}%
\pgfpathcurveto{\pgfqpoint{0.485089in}{0.675326in}}{\pgfqpoint{0.483333in}{0.671087in}}{\pgfqpoint{0.483333in}{0.666667in}}%
\pgfpathcurveto{\pgfqpoint{0.483333in}{0.662247in}}{\pgfqpoint{0.485089in}{0.658007in}}{\pgfqpoint{0.488215in}{0.654882in}}%
\pgfpathcurveto{\pgfqpoint{0.491340in}{0.651756in}}{\pgfqpoint{0.495580in}{0.650000in}}{\pgfqpoint{0.500000in}{0.650000in}}%
\pgfpathclose%
\pgfpathmoveto{\pgfqpoint{0.666667in}{0.650000in}}%
\pgfpathcurveto{\pgfqpoint{0.671087in}{0.650000in}}{\pgfqpoint{0.675326in}{0.651756in}}{\pgfqpoint{0.678452in}{0.654882in}}%
\pgfpathcurveto{\pgfqpoint{0.681577in}{0.658007in}}{\pgfqpoint{0.683333in}{0.662247in}}{\pgfqpoint{0.683333in}{0.666667in}}%
\pgfpathcurveto{\pgfqpoint{0.683333in}{0.671087in}}{\pgfqpoint{0.681577in}{0.675326in}}{\pgfqpoint{0.678452in}{0.678452in}}%
\pgfpathcurveto{\pgfqpoint{0.675326in}{0.681577in}}{\pgfqpoint{0.671087in}{0.683333in}}{\pgfqpoint{0.666667in}{0.683333in}}%
\pgfpathcurveto{\pgfqpoint{0.662247in}{0.683333in}}{\pgfqpoint{0.658007in}{0.681577in}}{\pgfqpoint{0.654882in}{0.678452in}}%
\pgfpathcurveto{\pgfqpoint{0.651756in}{0.675326in}}{\pgfqpoint{0.650000in}{0.671087in}}{\pgfqpoint{0.650000in}{0.666667in}}%
\pgfpathcurveto{\pgfqpoint{0.650000in}{0.662247in}}{\pgfqpoint{0.651756in}{0.658007in}}{\pgfqpoint{0.654882in}{0.654882in}}%
\pgfpathcurveto{\pgfqpoint{0.658007in}{0.651756in}}{\pgfqpoint{0.662247in}{0.650000in}}{\pgfqpoint{0.666667in}{0.650000in}}%
\pgfpathclose%
\pgfpathmoveto{\pgfqpoint{0.833333in}{0.650000in}}%
\pgfpathcurveto{\pgfqpoint{0.837753in}{0.650000in}}{\pgfqpoint{0.841993in}{0.651756in}}{\pgfqpoint{0.845118in}{0.654882in}}%
\pgfpathcurveto{\pgfqpoint{0.848244in}{0.658007in}}{\pgfqpoint{0.850000in}{0.662247in}}{\pgfqpoint{0.850000in}{0.666667in}}%
\pgfpathcurveto{\pgfqpoint{0.850000in}{0.671087in}}{\pgfqpoint{0.848244in}{0.675326in}}{\pgfqpoint{0.845118in}{0.678452in}}%
\pgfpathcurveto{\pgfqpoint{0.841993in}{0.681577in}}{\pgfqpoint{0.837753in}{0.683333in}}{\pgfqpoint{0.833333in}{0.683333in}}%
\pgfpathcurveto{\pgfqpoint{0.828913in}{0.683333in}}{\pgfqpoint{0.824674in}{0.681577in}}{\pgfqpoint{0.821548in}{0.678452in}}%
\pgfpathcurveto{\pgfqpoint{0.818423in}{0.675326in}}{\pgfqpoint{0.816667in}{0.671087in}}{\pgfqpoint{0.816667in}{0.666667in}}%
\pgfpathcurveto{\pgfqpoint{0.816667in}{0.662247in}}{\pgfqpoint{0.818423in}{0.658007in}}{\pgfqpoint{0.821548in}{0.654882in}}%
\pgfpathcurveto{\pgfqpoint{0.824674in}{0.651756in}}{\pgfqpoint{0.828913in}{0.650000in}}{\pgfqpoint{0.833333in}{0.650000in}}%
\pgfpathclose%
\pgfpathmoveto{\pgfqpoint{1.000000in}{0.650000in}}%
\pgfpathcurveto{\pgfqpoint{1.004420in}{0.650000in}}{\pgfqpoint{1.008660in}{0.651756in}}{\pgfqpoint{1.011785in}{0.654882in}}%
\pgfpathcurveto{\pgfqpoint{1.014911in}{0.658007in}}{\pgfqpoint{1.016667in}{0.662247in}}{\pgfqpoint{1.016667in}{0.666667in}}%
\pgfpathcurveto{\pgfqpoint{1.016667in}{0.671087in}}{\pgfqpoint{1.014911in}{0.675326in}}{\pgfqpoint{1.011785in}{0.678452in}}%
\pgfpathcurveto{\pgfqpoint{1.008660in}{0.681577in}}{\pgfqpoint{1.004420in}{0.683333in}}{\pgfqpoint{1.000000in}{0.683333in}}%
\pgfpathcurveto{\pgfqpoint{0.995580in}{0.683333in}}{\pgfqpoint{0.991340in}{0.681577in}}{\pgfqpoint{0.988215in}{0.678452in}}%
\pgfpathcurveto{\pgfqpoint{0.985089in}{0.675326in}}{\pgfqpoint{0.983333in}{0.671087in}}{\pgfqpoint{0.983333in}{0.666667in}}%
\pgfpathcurveto{\pgfqpoint{0.983333in}{0.662247in}}{\pgfqpoint{0.985089in}{0.658007in}}{\pgfqpoint{0.988215in}{0.654882in}}%
\pgfpathcurveto{\pgfqpoint{0.991340in}{0.651756in}}{\pgfqpoint{0.995580in}{0.650000in}}{\pgfqpoint{1.000000in}{0.650000in}}%
\pgfpathclose%
\pgfpathmoveto{\pgfqpoint{0.083333in}{0.816667in}}%
\pgfpathcurveto{\pgfqpoint{0.087753in}{0.816667in}}{\pgfqpoint{0.091993in}{0.818423in}}{\pgfqpoint{0.095118in}{0.821548in}}%
\pgfpathcurveto{\pgfqpoint{0.098244in}{0.824674in}}{\pgfqpoint{0.100000in}{0.828913in}}{\pgfqpoint{0.100000in}{0.833333in}}%
\pgfpathcurveto{\pgfqpoint{0.100000in}{0.837753in}}{\pgfqpoint{0.098244in}{0.841993in}}{\pgfqpoint{0.095118in}{0.845118in}}%
\pgfpathcurveto{\pgfqpoint{0.091993in}{0.848244in}}{\pgfqpoint{0.087753in}{0.850000in}}{\pgfqpoint{0.083333in}{0.850000in}}%
\pgfpathcurveto{\pgfqpoint{0.078913in}{0.850000in}}{\pgfqpoint{0.074674in}{0.848244in}}{\pgfqpoint{0.071548in}{0.845118in}}%
\pgfpathcurveto{\pgfqpoint{0.068423in}{0.841993in}}{\pgfqpoint{0.066667in}{0.837753in}}{\pgfqpoint{0.066667in}{0.833333in}}%
\pgfpathcurveto{\pgfqpoint{0.066667in}{0.828913in}}{\pgfqpoint{0.068423in}{0.824674in}}{\pgfqpoint{0.071548in}{0.821548in}}%
\pgfpathcurveto{\pgfqpoint{0.074674in}{0.818423in}}{\pgfqpoint{0.078913in}{0.816667in}}{\pgfqpoint{0.083333in}{0.816667in}}%
\pgfpathclose%
\pgfpathmoveto{\pgfqpoint{0.250000in}{0.816667in}}%
\pgfpathcurveto{\pgfqpoint{0.254420in}{0.816667in}}{\pgfqpoint{0.258660in}{0.818423in}}{\pgfqpoint{0.261785in}{0.821548in}}%
\pgfpathcurveto{\pgfqpoint{0.264911in}{0.824674in}}{\pgfqpoint{0.266667in}{0.828913in}}{\pgfqpoint{0.266667in}{0.833333in}}%
\pgfpathcurveto{\pgfqpoint{0.266667in}{0.837753in}}{\pgfqpoint{0.264911in}{0.841993in}}{\pgfqpoint{0.261785in}{0.845118in}}%
\pgfpathcurveto{\pgfqpoint{0.258660in}{0.848244in}}{\pgfqpoint{0.254420in}{0.850000in}}{\pgfqpoint{0.250000in}{0.850000in}}%
\pgfpathcurveto{\pgfqpoint{0.245580in}{0.850000in}}{\pgfqpoint{0.241340in}{0.848244in}}{\pgfqpoint{0.238215in}{0.845118in}}%
\pgfpathcurveto{\pgfqpoint{0.235089in}{0.841993in}}{\pgfqpoint{0.233333in}{0.837753in}}{\pgfqpoint{0.233333in}{0.833333in}}%
\pgfpathcurveto{\pgfqpoint{0.233333in}{0.828913in}}{\pgfqpoint{0.235089in}{0.824674in}}{\pgfqpoint{0.238215in}{0.821548in}}%
\pgfpathcurveto{\pgfqpoint{0.241340in}{0.818423in}}{\pgfqpoint{0.245580in}{0.816667in}}{\pgfqpoint{0.250000in}{0.816667in}}%
\pgfpathclose%
\pgfpathmoveto{\pgfqpoint{0.416667in}{0.816667in}}%
\pgfpathcurveto{\pgfqpoint{0.421087in}{0.816667in}}{\pgfqpoint{0.425326in}{0.818423in}}{\pgfqpoint{0.428452in}{0.821548in}}%
\pgfpathcurveto{\pgfqpoint{0.431577in}{0.824674in}}{\pgfqpoint{0.433333in}{0.828913in}}{\pgfqpoint{0.433333in}{0.833333in}}%
\pgfpathcurveto{\pgfqpoint{0.433333in}{0.837753in}}{\pgfqpoint{0.431577in}{0.841993in}}{\pgfqpoint{0.428452in}{0.845118in}}%
\pgfpathcurveto{\pgfqpoint{0.425326in}{0.848244in}}{\pgfqpoint{0.421087in}{0.850000in}}{\pgfqpoint{0.416667in}{0.850000in}}%
\pgfpathcurveto{\pgfqpoint{0.412247in}{0.850000in}}{\pgfqpoint{0.408007in}{0.848244in}}{\pgfqpoint{0.404882in}{0.845118in}}%
\pgfpathcurveto{\pgfqpoint{0.401756in}{0.841993in}}{\pgfqpoint{0.400000in}{0.837753in}}{\pgfqpoint{0.400000in}{0.833333in}}%
\pgfpathcurveto{\pgfqpoint{0.400000in}{0.828913in}}{\pgfqpoint{0.401756in}{0.824674in}}{\pgfqpoint{0.404882in}{0.821548in}}%
\pgfpathcurveto{\pgfqpoint{0.408007in}{0.818423in}}{\pgfqpoint{0.412247in}{0.816667in}}{\pgfqpoint{0.416667in}{0.816667in}}%
\pgfpathclose%
\pgfpathmoveto{\pgfqpoint{0.583333in}{0.816667in}}%
\pgfpathcurveto{\pgfqpoint{0.587753in}{0.816667in}}{\pgfqpoint{0.591993in}{0.818423in}}{\pgfqpoint{0.595118in}{0.821548in}}%
\pgfpathcurveto{\pgfqpoint{0.598244in}{0.824674in}}{\pgfqpoint{0.600000in}{0.828913in}}{\pgfqpoint{0.600000in}{0.833333in}}%
\pgfpathcurveto{\pgfqpoint{0.600000in}{0.837753in}}{\pgfqpoint{0.598244in}{0.841993in}}{\pgfqpoint{0.595118in}{0.845118in}}%
\pgfpathcurveto{\pgfqpoint{0.591993in}{0.848244in}}{\pgfqpoint{0.587753in}{0.850000in}}{\pgfqpoint{0.583333in}{0.850000in}}%
\pgfpathcurveto{\pgfqpoint{0.578913in}{0.850000in}}{\pgfqpoint{0.574674in}{0.848244in}}{\pgfqpoint{0.571548in}{0.845118in}}%
\pgfpathcurveto{\pgfqpoint{0.568423in}{0.841993in}}{\pgfqpoint{0.566667in}{0.837753in}}{\pgfqpoint{0.566667in}{0.833333in}}%
\pgfpathcurveto{\pgfqpoint{0.566667in}{0.828913in}}{\pgfqpoint{0.568423in}{0.824674in}}{\pgfqpoint{0.571548in}{0.821548in}}%
\pgfpathcurveto{\pgfqpoint{0.574674in}{0.818423in}}{\pgfqpoint{0.578913in}{0.816667in}}{\pgfqpoint{0.583333in}{0.816667in}}%
\pgfpathclose%
\pgfpathmoveto{\pgfqpoint{0.750000in}{0.816667in}}%
\pgfpathcurveto{\pgfqpoint{0.754420in}{0.816667in}}{\pgfqpoint{0.758660in}{0.818423in}}{\pgfqpoint{0.761785in}{0.821548in}}%
\pgfpathcurveto{\pgfqpoint{0.764911in}{0.824674in}}{\pgfqpoint{0.766667in}{0.828913in}}{\pgfqpoint{0.766667in}{0.833333in}}%
\pgfpathcurveto{\pgfqpoint{0.766667in}{0.837753in}}{\pgfqpoint{0.764911in}{0.841993in}}{\pgfqpoint{0.761785in}{0.845118in}}%
\pgfpathcurveto{\pgfqpoint{0.758660in}{0.848244in}}{\pgfqpoint{0.754420in}{0.850000in}}{\pgfqpoint{0.750000in}{0.850000in}}%
\pgfpathcurveto{\pgfqpoint{0.745580in}{0.850000in}}{\pgfqpoint{0.741340in}{0.848244in}}{\pgfqpoint{0.738215in}{0.845118in}}%
\pgfpathcurveto{\pgfqpoint{0.735089in}{0.841993in}}{\pgfqpoint{0.733333in}{0.837753in}}{\pgfqpoint{0.733333in}{0.833333in}}%
\pgfpathcurveto{\pgfqpoint{0.733333in}{0.828913in}}{\pgfqpoint{0.735089in}{0.824674in}}{\pgfqpoint{0.738215in}{0.821548in}}%
\pgfpathcurveto{\pgfqpoint{0.741340in}{0.818423in}}{\pgfqpoint{0.745580in}{0.816667in}}{\pgfqpoint{0.750000in}{0.816667in}}%
\pgfpathclose%
\pgfpathmoveto{\pgfqpoint{0.916667in}{0.816667in}}%
\pgfpathcurveto{\pgfqpoint{0.921087in}{0.816667in}}{\pgfqpoint{0.925326in}{0.818423in}}{\pgfqpoint{0.928452in}{0.821548in}}%
\pgfpathcurveto{\pgfqpoint{0.931577in}{0.824674in}}{\pgfqpoint{0.933333in}{0.828913in}}{\pgfqpoint{0.933333in}{0.833333in}}%
\pgfpathcurveto{\pgfqpoint{0.933333in}{0.837753in}}{\pgfqpoint{0.931577in}{0.841993in}}{\pgfqpoint{0.928452in}{0.845118in}}%
\pgfpathcurveto{\pgfqpoint{0.925326in}{0.848244in}}{\pgfqpoint{0.921087in}{0.850000in}}{\pgfqpoint{0.916667in}{0.850000in}}%
\pgfpathcurveto{\pgfqpoint{0.912247in}{0.850000in}}{\pgfqpoint{0.908007in}{0.848244in}}{\pgfqpoint{0.904882in}{0.845118in}}%
\pgfpathcurveto{\pgfqpoint{0.901756in}{0.841993in}}{\pgfqpoint{0.900000in}{0.837753in}}{\pgfqpoint{0.900000in}{0.833333in}}%
\pgfpathcurveto{\pgfqpoint{0.900000in}{0.828913in}}{\pgfqpoint{0.901756in}{0.824674in}}{\pgfqpoint{0.904882in}{0.821548in}}%
\pgfpathcurveto{\pgfqpoint{0.908007in}{0.818423in}}{\pgfqpoint{0.912247in}{0.816667in}}{\pgfqpoint{0.916667in}{0.816667in}}%
\pgfpathclose%
\pgfpathmoveto{\pgfqpoint{0.000000in}{0.983333in}}%
\pgfpathcurveto{\pgfqpoint{0.004420in}{0.983333in}}{\pgfqpoint{0.008660in}{0.985089in}}{\pgfqpoint{0.011785in}{0.988215in}}%
\pgfpathcurveto{\pgfqpoint{0.014911in}{0.991340in}}{\pgfqpoint{0.016667in}{0.995580in}}{\pgfqpoint{0.016667in}{1.000000in}}%
\pgfpathcurveto{\pgfqpoint{0.016667in}{1.004420in}}{\pgfqpoint{0.014911in}{1.008660in}}{\pgfqpoint{0.011785in}{1.011785in}}%
\pgfpathcurveto{\pgfqpoint{0.008660in}{1.014911in}}{\pgfqpoint{0.004420in}{1.016667in}}{\pgfqpoint{0.000000in}{1.016667in}}%
\pgfpathcurveto{\pgfqpoint{-0.004420in}{1.016667in}}{\pgfqpoint{-0.008660in}{1.014911in}}{\pgfqpoint{-0.011785in}{1.011785in}}%
\pgfpathcurveto{\pgfqpoint{-0.014911in}{1.008660in}}{\pgfqpoint{-0.016667in}{1.004420in}}{\pgfqpoint{-0.016667in}{1.000000in}}%
\pgfpathcurveto{\pgfqpoint{-0.016667in}{0.995580in}}{\pgfqpoint{-0.014911in}{0.991340in}}{\pgfqpoint{-0.011785in}{0.988215in}}%
\pgfpathcurveto{\pgfqpoint{-0.008660in}{0.985089in}}{\pgfqpoint{-0.004420in}{0.983333in}}{\pgfqpoint{0.000000in}{0.983333in}}%
\pgfpathclose%
\pgfpathmoveto{\pgfqpoint{0.166667in}{0.983333in}}%
\pgfpathcurveto{\pgfqpoint{0.171087in}{0.983333in}}{\pgfqpoint{0.175326in}{0.985089in}}{\pgfqpoint{0.178452in}{0.988215in}}%
\pgfpathcurveto{\pgfqpoint{0.181577in}{0.991340in}}{\pgfqpoint{0.183333in}{0.995580in}}{\pgfqpoint{0.183333in}{1.000000in}}%
\pgfpathcurveto{\pgfqpoint{0.183333in}{1.004420in}}{\pgfqpoint{0.181577in}{1.008660in}}{\pgfqpoint{0.178452in}{1.011785in}}%
\pgfpathcurveto{\pgfqpoint{0.175326in}{1.014911in}}{\pgfqpoint{0.171087in}{1.016667in}}{\pgfqpoint{0.166667in}{1.016667in}}%
\pgfpathcurveto{\pgfqpoint{0.162247in}{1.016667in}}{\pgfqpoint{0.158007in}{1.014911in}}{\pgfqpoint{0.154882in}{1.011785in}}%
\pgfpathcurveto{\pgfqpoint{0.151756in}{1.008660in}}{\pgfqpoint{0.150000in}{1.004420in}}{\pgfqpoint{0.150000in}{1.000000in}}%
\pgfpathcurveto{\pgfqpoint{0.150000in}{0.995580in}}{\pgfqpoint{0.151756in}{0.991340in}}{\pgfqpoint{0.154882in}{0.988215in}}%
\pgfpathcurveto{\pgfqpoint{0.158007in}{0.985089in}}{\pgfqpoint{0.162247in}{0.983333in}}{\pgfqpoint{0.166667in}{0.983333in}}%
\pgfpathclose%
\pgfpathmoveto{\pgfqpoint{0.333333in}{0.983333in}}%
\pgfpathcurveto{\pgfqpoint{0.337753in}{0.983333in}}{\pgfqpoint{0.341993in}{0.985089in}}{\pgfqpoint{0.345118in}{0.988215in}}%
\pgfpathcurveto{\pgfqpoint{0.348244in}{0.991340in}}{\pgfqpoint{0.350000in}{0.995580in}}{\pgfqpoint{0.350000in}{1.000000in}}%
\pgfpathcurveto{\pgfqpoint{0.350000in}{1.004420in}}{\pgfqpoint{0.348244in}{1.008660in}}{\pgfqpoint{0.345118in}{1.011785in}}%
\pgfpathcurveto{\pgfqpoint{0.341993in}{1.014911in}}{\pgfqpoint{0.337753in}{1.016667in}}{\pgfqpoint{0.333333in}{1.016667in}}%
\pgfpathcurveto{\pgfqpoint{0.328913in}{1.016667in}}{\pgfqpoint{0.324674in}{1.014911in}}{\pgfqpoint{0.321548in}{1.011785in}}%
\pgfpathcurveto{\pgfqpoint{0.318423in}{1.008660in}}{\pgfqpoint{0.316667in}{1.004420in}}{\pgfqpoint{0.316667in}{1.000000in}}%
\pgfpathcurveto{\pgfqpoint{0.316667in}{0.995580in}}{\pgfqpoint{0.318423in}{0.991340in}}{\pgfqpoint{0.321548in}{0.988215in}}%
\pgfpathcurveto{\pgfqpoint{0.324674in}{0.985089in}}{\pgfqpoint{0.328913in}{0.983333in}}{\pgfqpoint{0.333333in}{0.983333in}}%
\pgfpathclose%
\pgfpathmoveto{\pgfqpoint{0.500000in}{0.983333in}}%
\pgfpathcurveto{\pgfqpoint{0.504420in}{0.983333in}}{\pgfqpoint{0.508660in}{0.985089in}}{\pgfqpoint{0.511785in}{0.988215in}}%
\pgfpathcurveto{\pgfqpoint{0.514911in}{0.991340in}}{\pgfqpoint{0.516667in}{0.995580in}}{\pgfqpoint{0.516667in}{1.000000in}}%
\pgfpathcurveto{\pgfqpoint{0.516667in}{1.004420in}}{\pgfqpoint{0.514911in}{1.008660in}}{\pgfqpoint{0.511785in}{1.011785in}}%
\pgfpathcurveto{\pgfqpoint{0.508660in}{1.014911in}}{\pgfqpoint{0.504420in}{1.016667in}}{\pgfqpoint{0.500000in}{1.016667in}}%
\pgfpathcurveto{\pgfqpoint{0.495580in}{1.016667in}}{\pgfqpoint{0.491340in}{1.014911in}}{\pgfqpoint{0.488215in}{1.011785in}}%
\pgfpathcurveto{\pgfqpoint{0.485089in}{1.008660in}}{\pgfqpoint{0.483333in}{1.004420in}}{\pgfqpoint{0.483333in}{1.000000in}}%
\pgfpathcurveto{\pgfqpoint{0.483333in}{0.995580in}}{\pgfqpoint{0.485089in}{0.991340in}}{\pgfqpoint{0.488215in}{0.988215in}}%
\pgfpathcurveto{\pgfqpoint{0.491340in}{0.985089in}}{\pgfqpoint{0.495580in}{0.983333in}}{\pgfqpoint{0.500000in}{0.983333in}}%
\pgfpathclose%
\pgfpathmoveto{\pgfqpoint{0.666667in}{0.983333in}}%
\pgfpathcurveto{\pgfqpoint{0.671087in}{0.983333in}}{\pgfqpoint{0.675326in}{0.985089in}}{\pgfqpoint{0.678452in}{0.988215in}}%
\pgfpathcurveto{\pgfqpoint{0.681577in}{0.991340in}}{\pgfqpoint{0.683333in}{0.995580in}}{\pgfqpoint{0.683333in}{1.000000in}}%
\pgfpathcurveto{\pgfqpoint{0.683333in}{1.004420in}}{\pgfqpoint{0.681577in}{1.008660in}}{\pgfqpoint{0.678452in}{1.011785in}}%
\pgfpathcurveto{\pgfqpoint{0.675326in}{1.014911in}}{\pgfqpoint{0.671087in}{1.016667in}}{\pgfqpoint{0.666667in}{1.016667in}}%
\pgfpathcurveto{\pgfqpoint{0.662247in}{1.016667in}}{\pgfqpoint{0.658007in}{1.014911in}}{\pgfqpoint{0.654882in}{1.011785in}}%
\pgfpathcurveto{\pgfqpoint{0.651756in}{1.008660in}}{\pgfqpoint{0.650000in}{1.004420in}}{\pgfqpoint{0.650000in}{1.000000in}}%
\pgfpathcurveto{\pgfqpoint{0.650000in}{0.995580in}}{\pgfqpoint{0.651756in}{0.991340in}}{\pgfqpoint{0.654882in}{0.988215in}}%
\pgfpathcurveto{\pgfqpoint{0.658007in}{0.985089in}}{\pgfqpoint{0.662247in}{0.983333in}}{\pgfqpoint{0.666667in}{0.983333in}}%
\pgfpathclose%
\pgfpathmoveto{\pgfqpoint{0.833333in}{0.983333in}}%
\pgfpathcurveto{\pgfqpoint{0.837753in}{0.983333in}}{\pgfqpoint{0.841993in}{0.985089in}}{\pgfqpoint{0.845118in}{0.988215in}}%
\pgfpathcurveto{\pgfqpoint{0.848244in}{0.991340in}}{\pgfqpoint{0.850000in}{0.995580in}}{\pgfqpoint{0.850000in}{1.000000in}}%
\pgfpathcurveto{\pgfqpoint{0.850000in}{1.004420in}}{\pgfqpoint{0.848244in}{1.008660in}}{\pgfqpoint{0.845118in}{1.011785in}}%
\pgfpathcurveto{\pgfqpoint{0.841993in}{1.014911in}}{\pgfqpoint{0.837753in}{1.016667in}}{\pgfqpoint{0.833333in}{1.016667in}}%
\pgfpathcurveto{\pgfqpoint{0.828913in}{1.016667in}}{\pgfqpoint{0.824674in}{1.014911in}}{\pgfqpoint{0.821548in}{1.011785in}}%
\pgfpathcurveto{\pgfqpoint{0.818423in}{1.008660in}}{\pgfqpoint{0.816667in}{1.004420in}}{\pgfqpoint{0.816667in}{1.000000in}}%
\pgfpathcurveto{\pgfqpoint{0.816667in}{0.995580in}}{\pgfqpoint{0.818423in}{0.991340in}}{\pgfqpoint{0.821548in}{0.988215in}}%
\pgfpathcurveto{\pgfqpoint{0.824674in}{0.985089in}}{\pgfqpoint{0.828913in}{0.983333in}}{\pgfqpoint{0.833333in}{0.983333in}}%
\pgfpathclose%
\pgfpathmoveto{\pgfqpoint{1.000000in}{0.983333in}}%
\pgfpathcurveto{\pgfqpoint{1.004420in}{0.983333in}}{\pgfqpoint{1.008660in}{0.985089in}}{\pgfqpoint{1.011785in}{0.988215in}}%
\pgfpathcurveto{\pgfqpoint{1.014911in}{0.991340in}}{\pgfqpoint{1.016667in}{0.995580in}}{\pgfqpoint{1.016667in}{1.000000in}}%
\pgfpathcurveto{\pgfqpoint{1.016667in}{1.004420in}}{\pgfqpoint{1.014911in}{1.008660in}}{\pgfqpoint{1.011785in}{1.011785in}}%
\pgfpathcurveto{\pgfqpoint{1.008660in}{1.014911in}}{\pgfqpoint{1.004420in}{1.016667in}}{\pgfqpoint{1.000000in}{1.016667in}}%
\pgfpathcurveto{\pgfqpoint{0.995580in}{1.016667in}}{\pgfqpoint{0.991340in}{1.014911in}}{\pgfqpoint{0.988215in}{1.011785in}}%
\pgfpathcurveto{\pgfqpoint{0.985089in}{1.008660in}}{\pgfqpoint{0.983333in}{1.004420in}}{\pgfqpoint{0.983333in}{1.000000in}}%
\pgfpathcurveto{\pgfqpoint{0.983333in}{0.995580in}}{\pgfqpoint{0.985089in}{0.991340in}}{\pgfqpoint{0.988215in}{0.988215in}}%
\pgfpathcurveto{\pgfqpoint{0.991340in}{0.985089in}}{\pgfqpoint{0.995580in}{0.983333in}}{\pgfqpoint{1.000000in}{0.983333in}}%
\pgfpathclose%
\pgfusepath{stroke}%
\end{pgfscope}%
}%
\pgfsys@transformshift{5.908038in}{3.636133in}%
\pgfsys@useobject{currentpattern}{}%
\pgfsys@transformshift{1in}{0in}%
\pgfsys@transformshift{-1in}{0in}%
\pgfsys@transformshift{0in}{1in}%
\end{pgfscope}%
\begin{pgfscope}%
\pgfpathrectangle{\pgfqpoint{0.870538in}{0.637495in}}{\pgfqpoint{9.300000in}{9.060000in}}%
\pgfusepath{clip}%
\pgfsetbuttcap%
\pgfsetmiterjoin%
\definecolor{currentfill}{rgb}{0.172549,0.627451,0.172549}%
\pgfsetfillcolor{currentfill}%
\pgfsetfillopacity{0.990000}%
\pgfsetlinewidth{0.000000pt}%
\definecolor{currentstroke}{rgb}{0.000000,0.000000,0.000000}%
\pgfsetstrokecolor{currentstroke}%
\pgfsetstrokeopacity{0.990000}%
\pgfsetdash{}{0pt}%
\pgfpathmoveto{\pgfqpoint{7.458038in}{0.637495in}}%
\pgfpathlineto{\pgfqpoint{8.233038in}{0.637495in}}%
\pgfpathlineto{\pgfqpoint{8.233038in}{0.637495in}}%
\pgfpathlineto{\pgfqpoint{7.458038in}{0.637495in}}%
\pgfpathclose%
\pgfusepath{fill}%
\end{pgfscope}%
\begin{pgfscope}%
\pgfsetbuttcap%
\pgfsetmiterjoin%
\definecolor{currentfill}{rgb}{0.172549,0.627451,0.172549}%
\pgfsetfillcolor{currentfill}%
\pgfsetfillopacity{0.990000}%
\pgfsetlinewidth{0.000000pt}%
\definecolor{currentstroke}{rgb}{0.000000,0.000000,0.000000}%
\pgfsetstrokecolor{currentstroke}%
\pgfsetstrokeopacity{0.990000}%
\pgfsetdash{}{0pt}%
\pgfpathrectangle{\pgfqpoint{0.870538in}{0.637495in}}{\pgfqpoint{9.300000in}{9.060000in}}%
\pgfusepath{clip}%
\pgfpathmoveto{\pgfqpoint{7.458038in}{0.637495in}}%
\pgfpathlineto{\pgfqpoint{8.233038in}{0.637495in}}%
\pgfpathlineto{\pgfqpoint{8.233038in}{0.637495in}}%
\pgfpathlineto{\pgfqpoint{7.458038in}{0.637495in}}%
\pgfpathclose%
\pgfusepath{clip}%
\pgfsys@defobject{currentpattern}{\pgfqpoint{0in}{0in}}{\pgfqpoint{1in}{1in}}{%
\begin{pgfscope}%
\pgfpathrectangle{\pgfqpoint{0in}{0in}}{\pgfqpoint{1in}{1in}}%
\pgfusepath{clip}%
\pgfpathmoveto{\pgfqpoint{0.000000in}{-0.016667in}}%
\pgfpathcurveto{\pgfqpoint{0.004420in}{-0.016667in}}{\pgfqpoint{0.008660in}{-0.014911in}}{\pgfqpoint{0.011785in}{-0.011785in}}%
\pgfpathcurveto{\pgfqpoint{0.014911in}{-0.008660in}}{\pgfqpoint{0.016667in}{-0.004420in}}{\pgfqpoint{0.016667in}{0.000000in}}%
\pgfpathcurveto{\pgfqpoint{0.016667in}{0.004420in}}{\pgfqpoint{0.014911in}{0.008660in}}{\pgfqpoint{0.011785in}{0.011785in}}%
\pgfpathcurveto{\pgfqpoint{0.008660in}{0.014911in}}{\pgfqpoint{0.004420in}{0.016667in}}{\pgfqpoint{0.000000in}{0.016667in}}%
\pgfpathcurveto{\pgfqpoint{-0.004420in}{0.016667in}}{\pgfqpoint{-0.008660in}{0.014911in}}{\pgfqpoint{-0.011785in}{0.011785in}}%
\pgfpathcurveto{\pgfqpoint{-0.014911in}{0.008660in}}{\pgfqpoint{-0.016667in}{0.004420in}}{\pgfqpoint{-0.016667in}{0.000000in}}%
\pgfpathcurveto{\pgfqpoint{-0.016667in}{-0.004420in}}{\pgfqpoint{-0.014911in}{-0.008660in}}{\pgfqpoint{-0.011785in}{-0.011785in}}%
\pgfpathcurveto{\pgfqpoint{-0.008660in}{-0.014911in}}{\pgfqpoint{-0.004420in}{-0.016667in}}{\pgfqpoint{0.000000in}{-0.016667in}}%
\pgfpathclose%
\pgfpathmoveto{\pgfqpoint{0.166667in}{-0.016667in}}%
\pgfpathcurveto{\pgfqpoint{0.171087in}{-0.016667in}}{\pgfqpoint{0.175326in}{-0.014911in}}{\pgfqpoint{0.178452in}{-0.011785in}}%
\pgfpathcurveto{\pgfqpoint{0.181577in}{-0.008660in}}{\pgfqpoint{0.183333in}{-0.004420in}}{\pgfqpoint{0.183333in}{0.000000in}}%
\pgfpathcurveto{\pgfqpoint{0.183333in}{0.004420in}}{\pgfqpoint{0.181577in}{0.008660in}}{\pgfqpoint{0.178452in}{0.011785in}}%
\pgfpathcurveto{\pgfqpoint{0.175326in}{0.014911in}}{\pgfqpoint{0.171087in}{0.016667in}}{\pgfqpoint{0.166667in}{0.016667in}}%
\pgfpathcurveto{\pgfqpoint{0.162247in}{0.016667in}}{\pgfqpoint{0.158007in}{0.014911in}}{\pgfqpoint{0.154882in}{0.011785in}}%
\pgfpathcurveto{\pgfqpoint{0.151756in}{0.008660in}}{\pgfqpoint{0.150000in}{0.004420in}}{\pgfqpoint{0.150000in}{0.000000in}}%
\pgfpathcurveto{\pgfqpoint{0.150000in}{-0.004420in}}{\pgfqpoint{0.151756in}{-0.008660in}}{\pgfqpoint{0.154882in}{-0.011785in}}%
\pgfpathcurveto{\pgfqpoint{0.158007in}{-0.014911in}}{\pgfqpoint{0.162247in}{-0.016667in}}{\pgfqpoint{0.166667in}{-0.016667in}}%
\pgfpathclose%
\pgfpathmoveto{\pgfqpoint{0.333333in}{-0.016667in}}%
\pgfpathcurveto{\pgfqpoint{0.337753in}{-0.016667in}}{\pgfqpoint{0.341993in}{-0.014911in}}{\pgfqpoint{0.345118in}{-0.011785in}}%
\pgfpathcurveto{\pgfqpoint{0.348244in}{-0.008660in}}{\pgfqpoint{0.350000in}{-0.004420in}}{\pgfqpoint{0.350000in}{0.000000in}}%
\pgfpathcurveto{\pgfqpoint{0.350000in}{0.004420in}}{\pgfqpoint{0.348244in}{0.008660in}}{\pgfqpoint{0.345118in}{0.011785in}}%
\pgfpathcurveto{\pgfqpoint{0.341993in}{0.014911in}}{\pgfqpoint{0.337753in}{0.016667in}}{\pgfqpoint{0.333333in}{0.016667in}}%
\pgfpathcurveto{\pgfqpoint{0.328913in}{0.016667in}}{\pgfqpoint{0.324674in}{0.014911in}}{\pgfqpoint{0.321548in}{0.011785in}}%
\pgfpathcurveto{\pgfqpoint{0.318423in}{0.008660in}}{\pgfqpoint{0.316667in}{0.004420in}}{\pgfqpoint{0.316667in}{0.000000in}}%
\pgfpathcurveto{\pgfqpoint{0.316667in}{-0.004420in}}{\pgfqpoint{0.318423in}{-0.008660in}}{\pgfqpoint{0.321548in}{-0.011785in}}%
\pgfpathcurveto{\pgfqpoint{0.324674in}{-0.014911in}}{\pgfqpoint{0.328913in}{-0.016667in}}{\pgfqpoint{0.333333in}{-0.016667in}}%
\pgfpathclose%
\pgfpathmoveto{\pgfqpoint{0.500000in}{-0.016667in}}%
\pgfpathcurveto{\pgfqpoint{0.504420in}{-0.016667in}}{\pgfqpoint{0.508660in}{-0.014911in}}{\pgfqpoint{0.511785in}{-0.011785in}}%
\pgfpathcurveto{\pgfqpoint{0.514911in}{-0.008660in}}{\pgfqpoint{0.516667in}{-0.004420in}}{\pgfqpoint{0.516667in}{0.000000in}}%
\pgfpathcurveto{\pgfqpoint{0.516667in}{0.004420in}}{\pgfqpoint{0.514911in}{0.008660in}}{\pgfqpoint{0.511785in}{0.011785in}}%
\pgfpathcurveto{\pgfqpoint{0.508660in}{0.014911in}}{\pgfqpoint{0.504420in}{0.016667in}}{\pgfqpoint{0.500000in}{0.016667in}}%
\pgfpathcurveto{\pgfqpoint{0.495580in}{0.016667in}}{\pgfqpoint{0.491340in}{0.014911in}}{\pgfqpoint{0.488215in}{0.011785in}}%
\pgfpathcurveto{\pgfqpoint{0.485089in}{0.008660in}}{\pgfqpoint{0.483333in}{0.004420in}}{\pgfqpoint{0.483333in}{0.000000in}}%
\pgfpathcurveto{\pgfqpoint{0.483333in}{-0.004420in}}{\pgfqpoint{0.485089in}{-0.008660in}}{\pgfqpoint{0.488215in}{-0.011785in}}%
\pgfpathcurveto{\pgfqpoint{0.491340in}{-0.014911in}}{\pgfqpoint{0.495580in}{-0.016667in}}{\pgfqpoint{0.500000in}{-0.016667in}}%
\pgfpathclose%
\pgfpathmoveto{\pgfqpoint{0.666667in}{-0.016667in}}%
\pgfpathcurveto{\pgfqpoint{0.671087in}{-0.016667in}}{\pgfqpoint{0.675326in}{-0.014911in}}{\pgfqpoint{0.678452in}{-0.011785in}}%
\pgfpathcurveto{\pgfqpoint{0.681577in}{-0.008660in}}{\pgfqpoint{0.683333in}{-0.004420in}}{\pgfqpoint{0.683333in}{0.000000in}}%
\pgfpathcurveto{\pgfqpoint{0.683333in}{0.004420in}}{\pgfqpoint{0.681577in}{0.008660in}}{\pgfqpoint{0.678452in}{0.011785in}}%
\pgfpathcurveto{\pgfqpoint{0.675326in}{0.014911in}}{\pgfqpoint{0.671087in}{0.016667in}}{\pgfqpoint{0.666667in}{0.016667in}}%
\pgfpathcurveto{\pgfqpoint{0.662247in}{0.016667in}}{\pgfqpoint{0.658007in}{0.014911in}}{\pgfqpoint{0.654882in}{0.011785in}}%
\pgfpathcurveto{\pgfqpoint{0.651756in}{0.008660in}}{\pgfqpoint{0.650000in}{0.004420in}}{\pgfqpoint{0.650000in}{0.000000in}}%
\pgfpathcurveto{\pgfqpoint{0.650000in}{-0.004420in}}{\pgfqpoint{0.651756in}{-0.008660in}}{\pgfqpoint{0.654882in}{-0.011785in}}%
\pgfpathcurveto{\pgfqpoint{0.658007in}{-0.014911in}}{\pgfqpoint{0.662247in}{-0.016667in}}{\pgfqpoint{0.666667in}{-0.016667in}}%
\pgfpathclose%
\pgfpathmoveto{\pgfqpoint{0.833333in}{-0.016667in}}%
\pgfpathcurveto{\pgfqpoint{0.837753in}{-0.016667in}}{\pgfqpoint{0.841993in}{-0.014911in}}{\pgfqpoint{0.845118in}{-0.011785in}}%
\pgfpathcurveto{\pgfqpoint{0.848244in}{-0.008660in}}{\pgfqpoint{0.850000in}{-0.004420in}}{\pgfqpoint{0.850000in}{0.000000in}}%
\pgfpathcurveto{\pgfqpoint{0.850000in}{0.004420in}}{\pgfqpoint{0.848244in}{0.008660in}}{\pgfqpoint{0.845118in}{0.011785in}}%
\pgfpathcurveto{\pgfqpoint{0.841993in}{0.014911in}}{\pgfqpoint{0.837753in}{0.016667in}}{\pgfqpoint{0.833333in}{0.016667in}}%
\pgfpathcurveto{\pgfqpoint{0.828913in}{0.016667in}}{\pgfqpoint{0.824674in}{0.014911in}}{\pgfqpoint{0.821548in}{0.011785in}}%
\pgfpathcurveto{\pgfqpoint{0.818423in}{0.008660in}}{\pgfqpoint{0.816667in}{0.004420in}}{\pgfqpoint{0.816667in}{0.000000in}}%
\pgfpathcurveto{\pgfqpoint{0.816667in}{-0.004420in}}{\pgfqpoint{0.818423in}{-0.008660in}}{\pgfqpoint{0.821548in}{-0.011785in}}%
\pgfpathcurveto{\pgfqpoint{0.824674in}{-0.014911in}}{\pgfqpoint{0.828913in}{-0.016667in}}{\pgfqpoint{0.833333in}{-0.016667in}}%
\pgfpathclose%
\pgfpathmoveto{\pgfqpoint{1.000000in}{-0.016667in}}%
\pgfpathcurveto{\pgfqpoint{1.004420in}{-0.016667in}}{\pgfqpoint{1.008660in}{-0.014911in}}{\pgfqpoint{1.011785in}{-0.011785in}}%
\pgfpathcurveto{\pgfqpoint{1.014911in}{-0.008660in}}{\pgfqpoint{1.016667in}{-0.004420in}}{\pgfqpoint{1.016667in}{0.000000in}}%
\pgfpathcurveto{\pgfqpoint{1.016667in}{0.004420in}}{\pgfqpoint{1.014911in}{0.008660in}}{\pgfqpoint{1.011785in}{0.011785in}}%
\pgfpathcurveto{\pgfqpoint{1.008660in}{0.014911in}}{\pgfqpoint{1.004420in}{0.016667in}}{\pgfqpoint{1.000000in}{0.016667in}}%
\pgfpathcurveto{\pgfqpoint{0.995580in}{0.016667in}}{\pgfqpoint{0.991340in}{0.014911in}}{\pgfqpoint{0.988215in}{0.011785in}}%
\pgfpathcurveto{\pgfqpoint{0.985089in}{0.008660in}}{\pgfqpoint{0.983333in}{0.004420in}}{\pgfqpoint{0.983333in}{0.000000in}}%
\pgfpathcurveto{\pgfqpoint{0.983333in}{-0.004420in}}{\pgfqpoint{0.985089in}{-0.008660in}}{\pgfqpoint{0.988215in}{-0.011785in}}%
\pgfpathcurveto{\pgfqpoint{0.991340in}{-0.014911in}}{\pgfqpoint{0.995580in}{-0.016667in}}{\pgfqpoint{1.000000in}{-0.016667in}}%
\pgfpathclose%
\pgfpathmoveto{\pgfqpoint{0.083333in}{0.150000in}}%
\pgfpathcurveto{\pgfqpoint{0.087753in}{0.150000in}}{\pgfqpoint{0.091993in}{0.151756in}}{\pgfqpoint{0.095118in}{0.154882in}}%
\pgfpathcurveto{\pgfqpoint{0.098244in}{0.158007in}}{\pgfqpoint{0.100000in}{0.162247in}}{\pgfqpoint{0.100000in}{0.166667in}}%
\pgfpathcurveto{\pgfqpoint{0.100000in}{0.171087in}}{\pgfqpoint{0.098244in}{0.175326in}}{\pgfqpoint{0.095118in}{0.178452in}}%
\pgfpathcurveto{\pgfqpoint{0.091993in}{0.181577in}}{\pgfqpoint{0.087753in}{0.183333in}}{\pgfqpoint{0.083333in}{0.183333in}}%
\pgfpathcurveto{\pgfqpoint{0.078913in}{0.183333in}}{\pgfqpoint{0.074674in}{0.181577in}}{\pgfqpoint{0.071548in}{0.178452in}}%
\pgfpathcurveto{\pgfqpoint{0.068423in}{0.175326in}}{\pgfqpoint{0.066667in}{0.171087in}}{\pgfqpoint{0.066667in}{0.166667in}}%
\pgfpathcurveto{\pgfqpoint{0.066667in}{0.162247in}}{\pgfqpoint{0.068423in}{0.158007in}}{\pgfqpoint{0.071548in}{0.154882in}}%
\pgfpathcurveto{\pgfqpoint{0.074674in}{0.151756in}}{\pgfqpoint{0.078913in}{0.150000in}}{\pgfqpoint{0.083333in}{0.150000in}}%
\pgfpathclose%
\pgfpathmoveto{\pgfqpoint{0.250000in}{0.150000in}}%
\pgfpathcurveto{\pgfqpoint{0.254420in}{0.150000in}}{\pgfqpoint{0.258660in}{0.151756in}}{\pgfqpoint{0.261785in}{0.154882in}}%
\pgfpathcurveto{\pgfqpoint{0.264911in}{0.158007in}}{\pgfqpoint{0.266667in}{0.162247in}}{\pgfqpoint{0.266667in}{0.166667in}}%
\pgfpathcurveto{\pgfqpoint{0.266667in}{0.171087in}}{\pgfqpoint{0.264911in}{0.175326in}}{\pgfqpoint{0.261785in}{0.178452in}}%
\pgfpathcurveto{\pgfqpoint{0.258660in}{0.181577in}}{\pgfqpoint{0.254420in}{0.183333in}}{\pgfqpoint{0.250000in}{0.183333in}}%
\pgfpathcurveto{\pgfqpoint{0.245580in}{0.183333in}}{\pgfqpoint{0.241340in}{0.181577in}}{\pgfqpoint{0.238215in}{0.178452in}}%
\pgfpathcurveto{\pgfqpoint{0.235089in}{0.175326in}}{\pgfqpoint{0.233333in}{0.171087in}}{\pgfqpoint{0.233333in}{0.166667in}}%
\pgfpathcurveto{\pgfqpoint{0.233333in}{0.162247in}}{\pgfqpoint{0.235089in}{0.158007in}}{\pgfqpoint{0.238215in}{0.154882in}}%
\pgfpathcurveto{\pgfqpoint{0.241340in}{0.151756in}}{\pgfqpoint{0.245580in}{0.150000in}}{\pgfqpoint{0.250000in}{0.150000in}}%
\pgfpathclose%
\pgfpathmoveto{\pgfqpoint{0.416667in}{0.150000in}}%
\pgfpathcurveto{\pgfqpoint{0.421087in}{0.150000in}}{\pgfqpoint{0.425326in}{0.151756in}}{\pgfqpoint{0.428452in}{0.154882in}}%
\pgfpathcurveto{\pgfqpoint{0.431577in}{0.158007in}}{\pgfqpoint{0.433333in}{0.162247in}}{\pgfqpoint{0.433333in}{0.166667in}}%
\pgfpathcurveto{\pgfqpoint{0.433333in}{0.171087in}}{\pgfqpoint{0.431577in}{0.175326in}}{\pgfqpoint{0.428452in}{0.178452in}}%
\pgfpathcurveto{\pgfqpoint{0.425326in}{0.181577in}}{\pgfqpoint{0.421087in}{0.183333in}}{\pgfqpoint{0.416667in}{0.183333in}}%
\pgfpathcurveto{\pgfqpoint{0.412247in}{0.183333in}}{\pgfqpoint{0.408007in}{0.181577in}}{\pgfqpoint{0.404882in}{0.178452in}}%
\pgfpathcurveto{\pgfqpoint{0.401756in}{0.175326in}}{\pgfqpoint{0.400000in}{0.171087in}}{\pgfqpoint{0.400000in}{0.166667in}}%
\pgfpathcurveto{\pgfqpoint{0.400000in}{0.162247in}}{\pgfqpoint{0.401756in}{0.158007in}}{\pgfqpoint{0.404882in}{0.154882in}}%
\pgfpathcurveto{\pgfqpoint{0.408007in}{0.151756in}}{\pgfqpoint{0.412247in}{0.150000in}}{\pgfqpoint{0.416667in}{0.150000in}}%
\pgfpathclose%
\pgfpathmoveto{\pgfqpoint{0.583333in}{0.150000in}}%
\pgfpathcurveto{\pgfqpoint{0.587753in}{0.150000in}}{\pgfqpoint{0.591993in}{0.151756in}}{\pgfqpoint{0.595118in}{0.154882in}}%
\pgfpathcurveto{\pgfqpoint{0.598244in}{0.158007in}}{\pgfqpoint{0.600000in}{0.162247in}}{\pgfqpoint{0.600000in}{0.166667in}}%
\pgfpathcurveto{\pgfqpoint{0.600000in}{0.171087in}}{\pgfqpoint{0.598244in}{0.175326in}}{\pgfqpoint{0.595118in}{0.178452in}}%
\pgfpathcurveto{\pgfqpoint{0.591993in}{0.181577in}}{\pgfqpoint{0.587753in}{0.183333in}}{\pgfqpoint{0.583333in}{0.183333in}}%
\pgfpathcurveto{\pgfqpoint{0.578913in}{0.183333in}}{\pgfqpoint{0.574674in}{0.181577in}}{\pgfqpoint{0.571548in}{0.178452in}}%
\pgfpathcurveto{\pgfqpoint{0.568423in}{0.175326in}}{\pgfqpoint{0.566667in}{0.171087in}}{\pgfqpoint{0.566667in}{0.166667in}}%
\pgfpathcurveto{\pgfqpoint{0.566667in}{0.162247in}}{\pgfqpoint{0.568423in}{0.158007in}}{\pgfqpoint{0.571548in}{0.154882in}}%
\pgfpathcurveto{\pgfqpoint{0.574674in}{0.151756in}}{\pgfqpoint{0.578913in}{0.150000in}}{\pgfqpoint{0.583333in}{0.150000in}}%
\pgfpathclose%
\pgfpathmoveto{\pgfqpoint{0.750000in}{0.150000in}}%
\pgfpathcurveto{\pgfqpoint{0.754420in}{0.150000in}}{\pgfqpoint{0.758660in}{0.151756in}}{\pgfqpoint{0.761785in}{0.154882in}}%
\pgfpathcurveto{\pgfqpoint{0.764911in}{0.158007in}}{\pgfqpoint{0.766667in}{0.162247in}}{\pgfqpoint{0.766667in}{0.166667in}}%
\pgfpathcurveto{\pgfqpoint{0.766667in}{0.171087in}}{\pgfqpoint{0.764911in}{0.175326in}}{\pgfqpoint{0.761785in}{0.178452in}}%
\pgfpathcurveto{\pgfqpoint{0.758660in}{0.181577in}}{\pgfqpoint{0.754420in}{0.183333in}}{\pgfqpoint{0.750000in}{0.183333in}}%
\pgfpathcurveto{\pgfqpoint{0.745580in}{0.183333in}}{\pgfqpoint{0.741340in}{0.181577in}}{\pgfqpoint{0.738215in}{0.178452in}}%
\pgfpathcurveto{\pgfqpoint{0.735089in}{0.175326in}}{\pgfqpoint{0.733333in}{0.171087in}}{\pgfqpoint{0.733333in}{0.166667in}}%
\pgfpathcurveto{\pgfqpoint{0.733333in}{0.162247in}}{\pgfqpoint{0.735089in}{0.158007in}}{\pgfqpoint{0.738215in}{0.154882in}}%
\pgfpathcurveto{\pgfqpoint{0.741340in}{0.151756in}}{\pgfqpoint{0.745580in}{0.150000in}}{\pgfqpoint{0.750000in}{0.150000in}}%
\pgfpathclose%
\pgfpathmoveto{\pgfqpoint{0.916667in}{0.150000in}}%
\pgfpathcurveto{\pgfqpoint{0.921087in}{0.150000in}}{\pgfqpoint{0.925326in}{0.151756in}}{\pgfqpoint{0.928452in}{0.154882in}}%
\pgfpathcurveto{\pgfqpoint{0.931577in}{0.158007in}}{\pgfqpoint{0.933333in}{0.162247in}}{\pgfqpoint{0.933333in}{0.166667in}}%
\pgfpathcurveto{\pgfqpoint{0.933333in}{0.171087in}}{\pgfqpoint{0.931577in}{0.175326in}}{\pgfqpoint{0.928452in}{0.178452in}}%
\pgfpathcurveto{\pgfqpoint{0.925326in}{0.181577in}}{\pgfqpoint{0.921087in}{0.183333in}}{\pgfqpoint{0.916667in}{0.183333in}}%
\pgfpathcurveto{\pgfqpoint{0.912247in}{0.183333in}}{\pgfqpoint{0.908007in}{0.181577in}}{\pgfqpoint{0.904882in}{0.178452in}}%
\pgfpathcurveto{\pgfqpoint{0.901756in}{0.175326in}}{\pgfqpoint{0.900000in}{0.171087in}}{\pgfqpoint{0.900000in}{0.166667in}}%
\pgfpathcurveto{\pgfqpoint{0.900000in}{0.162247in}}{\pgfqpoint{0.901756in}{0.158007in}}{\pgfqpoint{0.904882in}{0.154882in}}%
\pgfpathcurveto{\pgfqpoint{0.908007in}{0.151756in}}{\pgfqpoint{0.912247in}{0.150000in}}{\pgfqpoint{0.916667in}{0.150000in}}%
\pgfpathclose%
\pgfpathmoveto{\pgfqpoint{0.000000in}{0.316667in}}%
\pgfpathcurveto{\pgfqpoint{0.004420in}{0.316667in}}{\pgfqpoint{0.008660in}{0.318423in}}{\pgfqpoint{0.011785in}{0.321548in}}%
\pgfpathcurveto{\pgfqpoint{0.014911in}{0.324674in}}{\pgfqpoint{0.016667in}{0.328913in}}{\pgfqpoint{0.016667in}{0.333333in}}%
\pgfpathcurveto{\pgfqpoint{0.016667in}{0.337753in}}{\pgfqpoint{0.014911in}{0.341993in}}{\pgfqpoint{0.011785in}{0.345118in}}%
\pgfpathcurveto{\pgfqpoint{0.008660in}{0.348244in}}{\pgfqpoint{0.004420in}{0.350000in}}{\pgfqpoint{0.000000in}{0.350000in}}%
\pgfpathcurveto{\pgfqpoint{-0.004420in}{0.350000in}}{\pgfqpoint{-0.008660in}{0.348244in}}{\pgfqpoint{-0.011785in}{0.345118in}}%
\pgfpathcurveto{\pgfqpoint{-0.014911in}{0.341993in}}{\pgfqpoint{-0.016667in}{0.337753in}}{\pgfqpoint{-0.016667in}{0.333333in}}%
\pgfpathcurveto{\pgfqpoint{-0.016667in}{0.328913in}}{\pgfqpoint{-0.014911in}{0.324674in}}{\pgfqpoint{-0.011785in}{0.321548in}}%
\pgfpathcurveto{\pgfqpoint{-0.008660in}{0.318423in}}{\pgfqpoint{-0.004420in}{0.316667in}}{\pgfqpoint{0.000000in}{0.316667in}}%
\pgfpathclose%
\pgfpathmoveto{\pgfqpoint{0.166667in}{0.316667in}}%
\pgfpathcurveto{\pgfqpoint{0.171087in}{0.316667in}}{\pgfqpoint{0.175326in}{0.318423in}}{\pgfqpoint{0.178452in}{0.321548in}}%
\pgfpathcurveto{\pgfqpoint{0.181577in}{0.324674in}}{\pgfqpoint{0.183333in}{0.328913in}}{\pgfqpoint{0.183333in}{0.333333in}}%
\pgfpathcurveto{\pgfqpoint{0.183333in}{0.337753in}}{\pgfqpoint{0.181577in}{0.341993in}}{\pgfqpoint{0.178452in}{0.345118in}}%
\pgfpathcurveto{\pgfqpoint{0.175326in}{0.348244in}}{\pgfqpoint{0.171087in}{0.350000in}}{\pgfqpoint{0.166667in}{0.350000in}}%
\pgfpathcurveto{\pgfqpoint{0.162247in}{0.350000in}}{\pgfqpoint{0.158007in}{0.348244in}}{\pgfqpoint{0.154882in}{0.345118in}}%
\pgfpathcurveto{\pgfqpoint{0.151756in}{0.341993in}}{\pgfqpoint{0.150000in}{0.337753in}}{\pgfqpoint{0.150000in}{0.333333in}}%
\pgfpathcurveto{\pgfqpoint{0.150000in}{0.328913in}}{\pgfqpoint{0.151756in}{0.324674in}}{\pgfqpoint{0.154882in}{0.321548in}}%
\pgfpathcurveto{\pgfqpoint{0.158007in}{0.318423in}}{\pgfqpoint{0.162247in}{0.316667in}}{\pgfqpoint{0.166667in}{0.316667in}}%
\pgfpathclose%
\pgfpathmoveto{\pgfqpoint{0.333333in}{0.316667in}}%
\pgfpathcurveto{\pgfqpoint{0.337753in}{0.316667in}}{\pgfqpoint{0.341993in}{0.318423in}}{\pgfqpoint{0.345118in}{0.321548in}}%
\pgfpathcurveto{\pgfqpoint{0.348244in}{0.324674in}}{\pgfqpoint{0.350000in}{0.328913in}}{\pgfqpoint{0.350000in}{0.333333in}}%
\pgfpathcurveto{\pgfqpoint{0.350000in}{0.337753in}}{\pgfqpoint{0.348244in}{0.341993in}}{\pgfqpoint{0.345118in}{0.345118in}}%
\pgfpathcurveto{\pgfqpoint{0.341993in}{0.348244in}}{\pgfqpoint{0.337753in}{0.350000in}}{\pgfqpoint{0.333333in}{0.350000in}}%
\pgfpathcurveto{\pgfqpoint{0.328913in}{0.350000in}}{\pgfqpoint{0.324674in}{0.348244in}}{\pgfqpoint{0.321548in}{0.345118in}}%
\pgfpathcurveto{\pgfqpoint{0.318423in}{0.341993in}}{\pgfqpoint{0.316667in}{0.337753in}}{\pgfqpoint{0.316667in}{0.333333in}}%
\pgfpathcurveto{\pgfqpoint{0.316667in}{0.328913in}}{\pgfqpoint{0.318423in}{0.324674in}}{\pgfqpoint{0.321548in}{0.321548in}}%
\pgfpathcurveto{\pgfqpoint{0.324674in}{0.318423in}}{\pgfqpoint{0.328913in}{0.316667in}}{\pgfqpoint{0.333333in}{0.316667in}}%
\pgfpathclose%
\pgfpathmoveto{\pgfqpoint{0.500000in}{0.316667in}}%
\pgfpathcurveto{\pgfqpoint{0.504420in}{0.316667in}}{\pgfqpoint{0.508660in}{0.318423in}}{\pgfqpoint{0.511785in}{0.321548in}}%
\pgfpathcurveto{\pgfqpoint{0.514911in}{0.324674in}}{\pgfqpoint{0.516667in}{0.328913in}}{\pgfqpoint{0.516667in}{0.333333in}}%
\pgfpathcurveto{\pgfqpoint{0.516667in}{0.337753in}}{\pgfqpoint{0.514911in}{0.341993in}}{\pgfqpoint{0.511785in}{0.345118in}}%
\pgfpathcurveto{\pgfqpoint{0.508660in}{0.348244in}}{\pgfqpoint{0.504420in}{0.350000in}}{\pgfqpoint{0.500000in}{0.350000in}}%
\pgfpathcurveto{\pgfqpoint{0.495580in}{0.350000in}}{\pgfqpoint{0.491340in}{0.348244in}}{\pgfqpoint{0.488215in}{0.345118in}}%
\pgfpathcurveto{\pgfqpoint{0.485089in}{0.341993in}}{\pgfqpoint{0.483333in}{0.337753in}}{\pgfqpoint{0.483333in}{0.333333in}}%
\pgfpathcurveto{\pgfqpoint{0.483333in}{0.328913in}}{\pgfqpoint{0.485089in}{0.324674in}}{\pgfqpoint{0.488215in}{0.321548in}}%
\pgfpathcurveto{\pgfqpoint{0.491340in}{0.318423in}}{\pgfqpoint{0.495580in}{0.316667in}}{\pgfqpoint{0.500000in}{0.316667in}}%
\pgfpathclose%
\pgfpathmoveto{\pgfqpoint{0.666667in}{0.316667in}}%
\pgfpathcurveto{\pgfqpoint{0.671087in}{0.316667in}}{\pgfqpoint{0.675326in}{0.318423in}}{\pgfqpoint{0.678452in}{0.321548in}}%
\pgfpathcurveto{\pgfqpoint{0.681577in}{0.324674in}}{\pgfqpoint{0.683333in}{0.328913in}}{\pgfqpoint{0.683333in}{0.333333in}}%
\pgfpathcurveto{\pgfqpoint{0.683333in}{0.337753in}}{\pgfqpoint{0.681577in}{0.341993in}}{\pgfqpoint{0.678452in}{0.345118in}}%
\pgfpathcurveto{\pgfqpoint{0.675326in}{0.348244in}}{\pgfqpoint{0.671087in}{0.350000in}}{\pgfqpoint{0.666667in}{0.350000in}}%
\pgfpathcurveto{\pgfqpoint{0.662247in}{0.350000in}}{\pgfqpoint{0.658007in}{0.348244in}}{\pgfqpoint{0.654882in}{0.345118in}}%
\pgfpathcurveto{\pgfqpoint{0.651756in}{0.341993in}}{\pgfqpoint{0.650000in}{0.337753in}}{\pgfqpoint{0.650000in}{0.333333in}}%
\pgfpathcurveto{\pgfqpoint{0.650000in}{0.328913in}}{\pgfqpoint{0.651756in}{0.324674in}}{\pgfqpoint{0.654882in}{0.321548in}}%
\pgfpathcurveto{\pgfqpoint{0.658007in}{0.318423in}}{\pgfqpoint{0.662247in}{0.316667in}}{\pgfqpoint{0.666667in}{0.316667in}}%
\pgfpathclose%
\pgfpathmoveto{\pgfqpoint{0.833333in}{0.316667in}}%
\pgfpathcurveto{\pgfqpoint{0.837753in}{0.316667in}}{\pgfqpoint{0.841993in}{0.318423in}}{\pgfqpoint{0.845118in}{0.321548in}}%
\pgfpathcurveto{\pgfqpoint{0.848244in}{0.324674in}}{\pgfqpoint{0.850000in}{0.328913in}}{\pgfqpoint{0.850000in}{0.333333in}}%
\pgfpathcurveto{\pgfqpoint{0.850000in}{0.337753in}}{\pgfqpoint{0.848244in}{0.341993in}}{\pgfqpoint{0.845118in}{0.345118in}}%
\pgfpathcurveto{\pgfqpoint{0.841993in}{0.348244in}}{\pgfqpoint{0.837753in}{0.350000in}}{\pgfqpoint{0.833333in}{0.350000in}}%
\pgfpathcurveto{\pgfqpoint{0.828913in}{0.350000in}}{\pgfqpoint{0.824674in}{0.348244in}}{\pgfqpoint{0.821548in}{0.345118in}}%
\pgfpathcurveto{\pgfqpoint{0.818423in}{0.341993in}}{\pgfqpoint{0.816667in}{0.337753in}}{\pgfqpoint{0.816667in}{0.333333in}}%
\pgfpathcurveto{\pgfqpoint{0.816667in}{0.328913in}}{\pgfqpoint{0.818423in}{0.324674in}}{\pgfqpoint{0.821548in}{0.321548in}}%
\pgfpathcurveto{\pgfqpoint{0.824674in}{0.318423in}}{\pgfqpoint{0.828913in}{0.316667in}}{\pgfqpoint{0.833333in}{0.316667in}}%
\pgfpathclose%
\pgfpathmoveto{\pgfqpoint{1.000000in}{0.316667in}}%
\pgfpathcurveto{\pgfqpoint{1.004420in}{0.316667in}}{\pgfqpoint{1.008660in}{0.318423in}}{\pgfqpoint{1.011785in}{0.321548in}}%
\pgfpathcurveto{\pgfqpoint{1.014911in}{0.324674in}}{\pgfqpoint{1.016667in}{0.328913in}}{\pgfqpoint{1.016667in}{0.333333in}}%
\pgfpathcurveto{\pgfqpoint{1.016667in}{0.337753in}}{\pgfqpoint{1.014911in}{0.341993in}}{\pgfqpoint{1.011785in}{0.345118in}}%
\pgfpathcurveto{\pgfqpoint{1.008660in}{0.348244in}}{\pgfqpoint{1.004420in}{0.350000in}}{\pgfqpoint{1.000000in}{0.350000in}}%
\pgfpathcurveto{\pgfqpoint{0.995580in}{0.350000in}}{\pgfqpoint{0.991340in}{0.348244in}}{\pgfqpoint{0.988215in}{0.345118in}}%
\pgfpathcurveto{\pgfqpoint{0.985089in}{0.341993in}}{\pgfqpoint{0.983333in}{0.337753in}}{\pgfqpoint{0.983333in}{0.333333in}}%
\pgfpathcurveto{\pgfqpoint{0.983333in}{0.328913in}}{\pgfqpoint{0.985089in}{0.324674in}}{\pgfqpoint{0.988215in}{0.321548in}}%
\pgfpathcurveto{\pgfqpoint{0.991340in}{0.318423in}}{\pgfqpoint{0.995580in}{0.316667in}}{\pgfqpoint{1.000000in}{0.316667in}}%
\pgfpathclose%
\pgfpathmoveto{\pgfqpoint{0.083333in}{0.483333in}}%
\pgfpathcurveto{\pgfqpoint{0.087753in}{0.483333in}}{\pgfqpoint{0.091993in}{0.485089in}}{\pgfqpoint{0.095118in}{0.488215in}}%
\pgfpathcurveto{\pgfqpoint{0.098244in}{0.491340in}}{\pgfqpoint{0.100000in}{0.495580in}}{\pgfqpoint{0.100000in}{0.500000in}}%
\pgfpathcurveto{\pgfqpoint{0.100000in}{0.504420in}}{\pgfqpoint{0.098244in}{0.508660in}}{\pgfqpoint{0.095118in}{0.511785in}}%
\pgfpathcurveto{\pgfqpoint{0.091993in}{0.514911in}}{\pgfqpoint{0.087753in}{0.516667in}}{\pgfqpoint{0.083333in}{0.516667in}}%
\pgfpathcurveto{\pgfqpoint{0.078913in}{0.516667in}}{\pgfqpoint{0.074674in}{0.514911in}}{\pgfqpoint{0.071548in}{0.511785in}}%
\pgfpathcurveto{\pgfqpoint{0.068423in}{0.508660in}}{\pgfqpoint{0.066667in}{0.504420in}}{\pgfqpoint{0.066667in}{0.500000in}}%
\pgfpathcurveto{\pgfqpoint{0.066667in}{0.495580in}}{\pgfqpoint{0.068423in}{0.491340in}}{\pgfqpoint{0.071548in}{0.488215in}}%
\pgfpathcurveto{\pgfqpoint{0.074674in}{0.485089in}}{\pgfqpoint{0.078913in}{0.483333in}}{\pgfqpoint{0.083333in}{0.483333in}}%
\pgfpathclose%
\pgfpathmoveto{\pgfqpoint{0.250000in}{0.483333in}}%
\pgfpathcurveto{\pgfqpoint{0.254420in}{0.483333in}}{\pgfqpoint{0.258660in}{0.485089in}}{\pgfqpoint{0.261785in}{0.488215in}}%
\pgfpathcurveto{\pgfqpoint{0.264911in}{0.491340in}}{\pgfqpoint{0.266667in}{0.495580in}}{\pgfqpoint{0.266667in}{0.500000in}}%
\pgfpathcurveto{\pgfqpoint{0.266667in}{0.504420in}}{\pgfqpoint{0.264911in}{0.508660in}}{\pgfqpoint{0.261785in}{0.511785in}}%
\pgfpathcurveto{\pgfqpoint{0.258660in}{0.514911in}}{\pgfqpoint{0.254420in}{0.516667in}}{\pgfqpoint{0.250000in}{0.516667in}}%
\pgfpathcurveto{\pgfqpoint{0.245580in}{0.516667in}}{\pgfqpoint{0.241340in}{0.514911in}}{\pgfqpoint{0.238215in}{0.511785in}}%
\pgfpathcurveto{\pgfqpoint{0.235089in}{0.508660in}}{\pgfqpoint{0.233333in}{0.504420in}}{\pgfqpoint{0.233333in}{0.500000in}}%
\pgfpathcurveto{\pgfqpoint{0.233333in}{0.495580in}}{\pgfqpoint{0.235089in}{0.491340in}}{\pgfqpoint{0.238215in}{0.488215in}}%
\pgfpathcurveto{\pgfqpoint{0.241340in}{0.485089in}}{\pgfqpoint{0.245580in}{0.483333in}}{\pgfqpoint{0.250000in}{0.483333in}}%
\pgfpathclose%
\pgfpathmoveto{\pgfqpoint{0.416667in}{0.483333in}}%
\pgfpathcurveto{\pgfqpoint{0.421087in}{0.483333in}}{\pgfqpoint{0.425326in}{0.485089in}}{\pgfqpoint{0.428452in}{0.488215in}}%
\pgfpathcurveto{\pgfqpoint{0.431577in}{0.491340in}}{\pgfqpoint{0.433333in}{0.495580in}}{\pgfqpoint{0.433333in}{0.500000in}}%
\pgfpathcurveto{\pgfqpoint{0.433333in}{0.504420in}}{\pgfqpoint{0.431577in}{0.508660in}}{\pgfqpoint{0.428452in}{0.511785in}}%
\pgfpathcurveto{\pgfqpoint{0.425326in}{0.514911in}}{\pgfqpoint{0.421087in}{0.516667in}}{\pgfqpoint{0.416667in}{0.516667in}}%
\pgfpathcurveto{\pgfqpoint{0.412247in}{0.516667in}}{\pgfqpoint{0.408007in}{0.514911in}}{\pgfqpoint{0.404882in}{0.511785in}}%
\pgfpathcurveto{\pgfqpoint{0.401756in}{0.508660in}}{\pgfqpoint{0.400000in}{0.504420in}}{\pgfqpoint{0.400000in}{0.500000in}}%
\pgfpathcurveto{\pgfqpoint{0.400000in}{0.495580in}}{\pgfqpoint{0.401756in}{0.491340in}}{\pgfqpoint{0.404882in}{0.488215in}}%
\pgfpathcurveto{\pgfqpoint{0.408007in}{0.485089in}}{\pgfqpoint{0.412247in}{0.483333in}}{\pgfqpoint{0.416667in}{0.483333in}}%
\pgfpathclose%
\pgfpathmoveto{\pgfqpoint{0.583333in}{0.483333in}}%
\pgfpathcurveto{\pgfqpoint{0.587753in}{0.483333in}}{\pgfqpoint{0.591993in}{0.485089in}}{\pgfqpoint{0.595118in}{0.488215in}}%
\pgfpathcurveto{\pgfqpoint{0.598244in}{0.491340in}}{\pgfqpoint{0.600000in}{0.495580in}}{\pgfqpoint{0.600000in}{0.500000in}}%
\pgfpathcurveto{\pgfqpoint{0.600000in}{0.504420in}}{\pgfqpoint{0.598244in}{0.508660in}}{\pgfqpoint{0.595118in}{0.511785in}}%
\pgfpathcurveto{\pgfqpoint{0.591993in}{0.514911in}}{\pgfqpoint{0.587753in}{0.516667in}}{\pgfqpoint{0.583333in}{0.516667in}}%
\pgfpathcurveto{\pgfqpoint{0.578913in}{0.516667in}}{\pgfqpoint{0.574674in}{0.514911in}}{\pgfqpoint{0.571548in}{0.511785in}}%
\pgfpathcurveto{\pgfqpoint{0.568423in}{0.508660in}}{\pgfqpoint{0.566667in}{0.504420in}}{\pgfqpoint{0.566667in}{0.500000in}}%
\pgfpathcurveto{\pgfqpoint{0.566667in}{0.495580in}}{\pgfqpoint{0.568423in}{0.491340in}}{\pgfqpoint{0.571548in}{0.488215in}}%
\pgfpathcurveto{\pgfqpoint{0.574674in}{0.485089in}}{\pgfqpoint{0.578913in}{0.483333in}}{\pgfqpoint{0.583333in}{0.483333in}}%
\pgfpathclose%
\pgfpathmoveto{\pgfqpoint{0.750000in}{0.483333in}}%
\pgfpathcurveto{\pgfqpoint{0.754420in}{0.483333in}}{\pgfqpoint{0.758660in}{0.485089in}}{\pgfqpoint{0.761785in}{0.488215in}}%
\pgfpathcurveto{\pgfqpoint{0.764911in}{0.491340in}}{\pgfqpoint{0.766667in}{0.495580in}}{\pgfqpoint{0.766667in}{0.500000in}}%
\pgfpathcurveto{\pgfqpoint{0.766667in}{0.504420in}}{\pgfqpoint{0.764911in}{0.508660in}}{\pgfqpoint{0.761785in}{0.511785in}}%
\pgfpathcurveto{\pgfqpoint{0.758660in}{0.514911in}}{\pgfqpoint{0.754420in}{0.516667in}}{\pgfqpoint{0.750000in}{0.516667in}}%
\pgfpathcurveto{\pgfqpoint{0.745580in}{0.516667in}}{\pgfqpoint{0.741340in}{0.514911in}}{\pgfqpoint{0.738215in}{0.511785in}}%
\pgfpathcurveto{\pgfqpoint{0.735089in}{0.508660in}}{\pgfqpoint{0.733333in}{0.504420in}}{\pgfqpoint{0.733333in}{0.500000in}}%
\pgfpathcurveto{\pgfqpoint{0.733333in}{0.495580in}}{\pgfqpoint{0.735089in}{0.491340in}}{\pgfqpoint{0.738215in}{0.488215in}}%
\pgfpathcurveto{\pgfqpoint{0.741340in}{0.485089in}}{\pgfqpoint{0.745580in}{0.483333in}}{\pgfqpoint{0.750000in}{0.483333in}}%
\pgfpathclose%
\pgfpathmoveto{\pgfqpoint{0.916667in}{0.483333in}}%
\pgfpathcurveto{\pgfqpoint{0.921087in}{0.483333in}}{\pgfqpoint{0.925326in}{0.485089in}}{\pgfqpoint{0.928452in}{0.488215in}}%
\pgfpathcurveto{\pgfqpoint{0.931577in}{0.491340in}}{\pgfqpoint{0.933333in}{0.495580in}}{\pgfqpoint{0.933333in}{0.500000in}}%
\pgfpathcurveto{\pgfqpoint{0.933333in}{0.504420in}}{\pgfqpoint{0.931577in}{0.508660in}}{\pgfqpoint{0.928452in}{0.511785in}}%
\pgfpathcurveto{\pgfqpoint{0.925326in}{0.514911in}}{\pgfqpoint{0.921087in}{0.516667in}}{\pgfqpoint{0.916667in}{0.516667in}}%
\pgfpathcurveto{\pgfqpoint{0.912247in}{0.516667in}}{\pgfqpoint{0.908007in}{0.514911in}}{\pgfqpoint{0.904882in}{0.511785in}}%
\pgfpathcurveto{\pgfqpoint{0.901756in}{0.508660in}}{\pgfqpoint{0.900000in}{0.504420in}}{\pgfqpoint{0.900000in}{0.500000in}}%
\pgfpathcurveto{\pgfqpoint{0.900000in}{0.495580in}}{\pgfqpoint{0.901756in}{0.491340in}}{\pgfqpoint{0.904882in}{0.488215in}}%
\pgfpathcurveto{\pgfqpoint{0.908007in}{0.485089in}}{\pgfqpoint{0.912247in}{0.483333in}}{\pgfqpoint{0.916667in}{0.483333in}}%
\pgfpathclose%
\pgfpathmoveto{\pgfqpoint{0.000000in}{0.650000in}}%
\pgfpathcurveto{\pgfqpoint{0.004420in}{0.650000in}}{\pgfqpoint{0.008660in}{0.651756in}}{\pgfqpoint{0.011785in}{0.654882in}}%
\pgfpathcurveto{\pgfqpoint{0.014911in}{0.658007in}}{\pgfqpoint{0.016667in}{0.662247in}}{\pgfqpoint{0.016667in}{0.666667in}}%
\pgfpathcurveto{\pgfqpoint{0.016667in}{0.671087in}}{\pgfqpoint{0.014911in}{0.675326in}}{\pgfqpoint{0.011785in}{0.678452in}}%
\pgfpathcurveto{\pgfqpoint{0.008660in}{0.681577in}}{\pgfqpoint{0.004420in}{0.683333in}}{\pgfqpoint{0.000000in}{0.683333in}}%
\pgfpathcurveto{\pgfqpoint{-0.004420in}{0.683333in}}{\pgfqpoint{-0.008660in}{0.681577in}}{\pgfqpoint{-0.011785in}{0.678452in}}%
\pgfpathcurveto{\pgfqpoint{-0.014911in}{0.675326in}}{\pgfqpoint{-0.016667in}{0.671087in}}{\pgfqpoint{-0.016667in}{0.666667in}}%
\pgfpathcurveto{\pgfqpoint{-0.016667in}{0.662247in}}{\pgfqpoint{-0.014911in}{0.658007in}}{\pgfqpoint{-0.011785in}{0.654882in}}%
\pgfpathcurveto{\pgfqpoint{-0.008660in}{0.651756in}}{\pgfqpoint{-0.004420in}{0.650000in}}{\pgfqpoint{0.000000in}{0.650000in}}%
\pgfpathclose%
\pgfpathmoveto{\pgfqpoint{0.166667in}{0.650000in}}%
\pgfpathcurveto{\pgfqpoint{0.171087in}{0.650000in}}{\pgfqpoint{0.175326in}{0.651756in}}{\pgfqpoint{0.178452in}{0.654882in}}%
\pgfpathcurveto{\pgfqpoint{0.181577in}{0.658007in}}{\pgfqpoint{0.183333in}{0.662247in}}{\pgfqpoint{0.183333in}{0.666667in}}%
\pgfpathcurveto{\pgfqpoint{0.183333in}{0.671087in}}{\pgfqpoint{0.181577in}{0.675326in}}{\pgfqpoint{0.178452in}{0.678452in}}%
\pgfpathcurveto{\pgfqpoint{0.175326in}{0.681577in}}{\pgfqpoint{0.171087in}{0.683333in}}{\pgfqpoint{0.166667in}{0.683333in}}%
\pgfpathcurveto{\pgfqpoint{0.162247in}{0.683333in}}{\pgfqpoint{0.158007in}{0.681577in}}{\pgfqpoint{0.154882in}{0.678452in}}%
\pgfpathcurveto{\pgfqpoint{0.151756in}{0.675326in}}{\pgfqpoint{0.150000in}{0.671087in}}{\pgfqpoint{0.150000in}{0.666667in}}%
\pgfpathcurveto{\pgfqpoint{0.150000in}{0.662247in}}{\pgfqpoint{0.151756in}{0.658007in}}{\pgfqpoint{0.154882in}{0.654882in}}%
\pgfpathcurveto{\pgfqpoint{0.158007in}{0.651756in}}{\pgfqpoint{0.162247in}{0.650000in}}{\pgfqpoint{0.166667in}{0.650000in}}%
\pgfpathclose%
\pgfpathmoveto{\pgfqpoint{0.333333in}{0.650000in}}%
\pgfpathcurveto{\pgfqpoint{0.337753in}{0.650000in}}{\pgfqpoint{0.341993in}{0.651756in}}{\pgfqpoint{0.345118in}{0.654882in}}%
\pgfpathcurveto{\pgfqpoint{0.348244in}{0.658007in}}{\pgfqpoint{0.350000in}{0.662247in}}{\pgfqpoint{0.350000in}{0.666667in}}%
\pgfpathcurveto{\pgfqpoint{0.350000in}{0.671087in}}{\pgfqpoint{0.348244in}{0.675326in}}{\pgfqpoint{0.345118in}{0.678452in}}%
\pgfpathcurveto{\pgfqpoint{0.341993in}{0.681577in}}{\pgfqpoint{0.337753in}{0.683333in}}{\pgfqpoint{0.333333in}{0.683333in}}%
\pgfpathcurveto{\pgfqpoint{0.328913in}{0.683333in}}{\pgfqpoint{0.324674in}{0.681577in}}{\pgfqpoint{0.321548in}{0.678452in}}%
\pgfpathcurveto{\pgfqpoint{0.318423in}{0.675326in}}{\pgfqpoint{0.316667in}{0.671087in}}{\pgfqpoint{0.316667in}{0.666667in}}%
\pgfpathcurveto{\pgfqpoint{0.316667in}{0.662247in}}{\pgfqpoint{0.318423in}{0.658007in}}{\pgfqpoint{0.321548in}{0.654882in}}%
\pgfpathcurveto{\pgfqpoint{0.324674in}{0.651756in}}{\pgfqpoint{0.328913in}{0.650000in}}{\pgfqpoint{0.333333in}{0.650000in}}%
\pgfpathclose%
\pgfpathmoveto{\pgfqpoint{0.500000in}{0.650000in}}%
\pgfpathcurveto{\pgfqpoint{0.504420in}{0.650000in}}{\pgfqpoint{0.508660in}{0.651756in}}{\pgfqpoint{0.511785in}{0.654882in}}%
\pgfpathcurveto{\pgfqpoint{0.514911in}{0.658007in}}{\pgfqpoint{0.516667in}{0.662247in}}{\pgfqpoint{0.516667in}{0.666667in}}%
\pgfpathcurveto{\pgfqpoint{0.516667in}{0.671087in}}{\pgfqpoint{0.514911in}{0.675326in}}{\pgfqpoint{0.511785in}{0.678452in}}%
\pgfpathcurveto{\pgfqpoint{0.508660in}{0.681577in}}{\pgfqpoint{0.504420in}{0.683333in}}{\pgfqpoint{0.500000in}{0.683333in}}%
\pgfpathcurveto{\pgfqpoint{0.495580in}{0.683333in}}{\pgfqpoint{0.491340in}{0.681577in}}{\pgfqpoint{0.488215in}{0.678452in}}%
\pgfpathcurveto{\pgfqpoint{0.485089in}{0.675326in}}{\pgfqpoint{0.483333in}{0.671087in}}{\pgfqpoint{0.483333in}{0.666667in}}%
\pgfpathcurveto{\pgfqpoint{0.483333in}{0.662247in}}{\pgfqpoint{0.485089in}{0.658007in}}{\pgfqpoint{0.488215in}{0.654882in}}%
\pgfpathcurveto{\pgfqpoint{0.491340in}{0.651756in}}{\pgfqpoint{0.495580in}{0.650000in}}{\pgfqpoint{0.500000in}{0.650000in}}%
\pgfpathclose%
\pgfpathmoveto{\pgfqpoint{0.666667in}{0.650000in}}%
\pgfpathcurveto{\pgfqpoint{0.671087in}{0.650000in}}{\pgfqpoint{0.675326in}{0.651756in}}{\pgfqpoint{0.678452in}{0.654882in}}%
\pgfpathcurveto{\pgfqpoint{0.681577in}{0.658007in}}{\pgfqpoint{0.683333in}{0.662247in}}{\pgfqpoint{0.683333in}{0.666667in}}%
\pgfpathcurveto{\pgfqpoint{0.683333in}{0.671087in}}{\pgfqpoint{0.681577in}{0.675326in}}{\pgfqpoint{0.678452in}{0.678452in}}%
\pgfpathcurveto{\pgfqpoint{0.675326in}{0.681577in}}{\pgfqpoint{0.671087in}{0.683333in}}{\pgfqpoint{0.666667in}{0.683333in}}%
\pgfpathcurveto{\pgfqpoint{0.662247in}{0.683333in}}{\pgfqpoint{0.658007in}{0.681577in}}{\pgfqpoint{0.654882in}{0.678452in}}%
\pgfpathcurveto{\pgfqpoint{0.651756in}{0.675326in}}{\pgfqpoint{0.650000in}{0.671087in}}{\pgfqpoint{0.650000in}{0.666667in}}%
\pgfpathcurveto{\pgfqpoint{0.650000in}{0.662247in}}{\pgfqpoint{0.651756in}{0.658007in}}{\pgfqpoint{0.654882in}{0.654882in}}%
\pgfpathcurveto{\pgfqpoint{0.658007in}{0.651756in}}{\pgfqpoint{0.662247in}{0.650000in}}{\pgfqpoint{0.666667in}{0.650000in}}%
\pgfpathclose%
\pgfpathmoveto{\pgfqpoint{0.833333in}{0.650000in}}%
\pgfpathcurveto{\pgfqpoint{0.837753in}{0.650000in}}{\pgfqpoint{0.841993in}{0.651756in}}{\pgfqpoint{0.845118in}{0.654882in}}%
\pgfpathcurveto{\pgfqpoint{0.848244in}{0.658007in}}{\pgfqpoint{0.850000in}{0.662247in}}{\pgfqpoint{0.850000in}{0.666667in}}%
\pgfpathcurveto{\pgfqpoint{0.850000in}{0.671087in}}{\pgfqpoint{0.848244in}{0.675326in}}{\pgfqpoint{0.845118in}{0.678452in}}%
\pgfpathcurveto{\pgfqpoint{0.841993in}{0.681577in}}{\pgfqpoint{0.837753in}{0.683333in}}{\pgfqpoint{0.833333in}{0.683333in}}%
\pgfpathcurveto{\pgfqpoint{0.828913in}{0.683333in}}{\pgfqpoint{0.824674in}{0.681577in}}{\pgfqpoint{0.821548in}{0.678452in}}%
\pgfpathcurveto{\pgfqpoint{0.818423in}{0.675326in}}{\pgfqpoint{0.816667in}{0.671087in}}{\pgfqpoint{0.816667in}{0.666667in}}%
\pgfpathcurveto{\pgfqpoint{0.816667in}{0.662247in}}{\pgfqpoint{0.818423in}{0.658007in}}{\pgfqpoint{0.821548in}{0.654882in}}%
\pgfpathcurveto{\pgfqpoint{0.824674in}{0.651756in}}{\pgfqpoint{0.828913in}{0.650000in}}{\pgfqpoint{0.833333in}{0.650000in}}%
\pgfpathclose%
\pgfpathmoveto{\pgfqpoint{1.000000in}{0.650000in}}%
\pgfpathcurveto{\pgfqpoint{1.004420in}{0.650000in}}{\pgfqpoint{1.008660in}{0.651756in}}{\pgfqpoint{1.011785in}{0.654882in}}%
\pgfpathcurveto{\pgfqpoint{1.014911in}{0.658007in}}{\pgfqpoint{1.016667in}{0.662247in}}{\pgfqpoint{1.016667in}{0.666667in}}%
\pgfpathcurveto{\pgfqpoint{1.016667in}{0.671087in}}{\pgfqpoint{1.014911in}{0.675326in}}{\pgfqpoint{1.011785in}{0.678452in}}%
\pgfpathcurveto{\pgfqpoint{1.008660in}{0.681577in}}{\pgfqpoint{1.004420in}{0.683333in}}{\pgfqpoint{1.000000in}{0.683333in}}%
\pgfpathcurveto{\pgfqpoint{0.995580in}{0.683333in}}{\pgfqpoint{0.991340in}{0.681577in}}{\pgfqpoint{0.988215in}{0.678452in}}%
\pgfpathcurveto{\pgfqpoint{0.985089in}{0.675326in}}{\pgfqpoint{0.983333in}{0.671087in}}{\pgfqpoint{0.983333in}{0.666667in}}%
\pgfpathcurveto{\pgfqpoint{0.983333in}{0.662247in}}{\pgfqpoint{0.985089in}{0.658007in}}{\pgfqpoint{0.988215in}{0.654882in}}%
\pgfpathcurveto{\pgfqpoint{0.991340in}{0.651756in}}{\pgfqpoint{0.995580in}{0.650000in}}{\pgfqpoint{1.000000in}{0.650000in}}%
\pgfpathclose%
\pgfpathmoveto{\pgfqpoint{0.083333in}{0.816667in}}%
\pgfpathcurveto{\pgfqpoint{0.087753in}{0.816667in}}{\pgfqpoint{0.091993in}{0.818423in}}{\pgfqpoint{0.095118in}{0.821548in}}%
\pgfpathcurveto{\pgfqpoint{0.098244in}{0.824674in}}{\pgfqpoint{0.100000in}{0.828913in}}{\pgfqpoint{0.100000in}{0.833333in}}%
\pgfpathcurveto{\pgfqpoint{0.100000in}{0.837753in}}{\pgfqpoint{0.098244in}{0.841993in}}{\pgfqpoint{0.095118in}{0.845118in}}%
\pgfpathcurveto{\pgfqpoint{0.091993in}{0.848244in}}{\pgfqpoint{0.087753in}{0.850000in}}{\pgfqpoint{0.083333in}{0.850000in}}%
\pgfpathcurveto{\pgfqpoint{0.078913in}{0.850000in}}{\pgfqpoint{0.074674in}{0.848244in}}{\pgfqpoint{0.071548in}{0.845118in}}%
\pgfpathcurveto{\pgfqpoint{0.068423in}{0.841993in}}{\pgfqpoint{0.066667in}{0.837753in}}{\pgfqpoint{0.066667in}{0.833333in}}%
\pgfpathcurveto{\pgfqpoint{0.066667in}{0.828913in}}{\pgfqpoint{0.068423in}{0.824674in}}{\pgfqpoint{0.071548in}{0.821548in}}%
\pgfpathcurveto{\pgfqpoint{0.074674in}{0.818423in}}{\pgfqpoint{0.078913in}{0.816667in}}{\pgfqpoint{0.083333in}{0.816667in}}%
\pgfpathclose%
\pgfpathmoveto{\pgfqpoint{0.250000in}{0.816667in}}%
\pgfpathcurveto{\pgfqpoint{0.254420in}{0.816667in}}{\pgfqpoint{0.258660in}{0.818423in}}{\pgfqpoint{0.261785in}{0.821548in}}%
\pgfpathcurveto{\pgfqpoint{0.264911in}{0.824674in}}{\pgfqpoint{0.266667in}{0.828913in}}{\pgfqpoint{0.266667in}{0.833333in}}%
\pgfpathcurveto{\pgfqpoint{0.266667in}{0.837753in}}{\pgfqpoint{0.264911in}{0.841993in}}{\pgfqpoint{0.261785in}{0.845118in}}%
\pgfpathcurveto{\pgfqpoint{0.258660in}{0.848244in}}{\pgfqpoint{0.254420in}{0.850000in}}{\pgfqpoint{0.250000in}{0.850000in}}%
\pgfpathcurveto{\pgfqpoint{0.245580in}{0.850000in}}{\pgfqpoint{0.241340in}{0.848244in}}{\pgfqpoint{0.238215in}{0.845118in}}%
\pgfpathcurveto{\pgfqpoint{0.235089in}{0.841993in}}{\pgfqpoint{0.233333in}{0.837753in}}{\pgfqpoint{0.233333in}{0.833333in}}%
\pgfpathcurveto{\pgfqpoint{0.233333in}{0.828913in}}{\pgfqpoint{0.235089in}{0.824674in}}{\pgfqpoint{0.238215in}{0.821548in}}%
\pgfpathcurveto{\pgfqpoint{0.241340in}{0.818423in}}{\pgfqpoint{0.245580in}{0.816667in}}{\pgfqpoint{0.250000in}{0.816667in}}%
\pgfpathclose%
\pgfpathmoveto{\pgfqpoint{0.416667in}{0.816667in}}%
\pgfpathcurveto{\pgfqpoint{0.421087in}{0.816667in}}{\pgfqpoint{0.425326in}{0.818423in}}{\pgfqpoint{0.428452in}{0.821548in}}%
\pgfpathcurveto{\pgfqpoint{0.431577in}{0.824674in}}{\pgfqpoint{0.433333in}{0.828913in}}{\pgfqpoint{0.433333in}{0.833333in}}%
\pgfpathcurveto{\pgfqpoint{0.433333in}{0.837753in}}{\pgfqpoint{0.431577in}{0.841993in}}{\pgfqpoint{0.428452in}{0.845118in}}%
\pgfpathcurveto{\pgfqpoint{0.425326in}{0.848244in}}{\pgfqpoint{0.421087in}{0.850000in}}{\pgfqpoint{0.416667in}{0.850000in}}%
\pgfpathcurveto{\pgfqpoint{0.412247in}{0.850000in}}{\pgfqpoint{0.408007in}{0.848244in}}{\pgfqpoint{0.404882in}{0.845118in}}%
\pgfpathcurveto{\pgfqpoint{0.401756in}{0.841993in}}{\pgfqpoint{0.400000in}{0.837753in}}{\pgfqpoint{0.400000in}{0.833333in}}%
\pgfpathcurveto{\pgfqpoint{0.400000in}{0.828913in}}{\pgfqpoint{0.401756in}{0.824674in}}{\pgfqpoint{0.404882in}{0.821548in}}%
\pgfpathcurveto{\pgfqpoint{0.408007in}{0.818423in}}{\pgfqpoint{0.412247in}{0.816667in}}{\pgfqpoint{0.416667in}{0.816667in}}%
\pgfpathclose%
\pgfpathmoveto{\pgfqpoint{0.583333in}{0.816667in}}%
\pgfpathcurveto{\pgfqpoint{0.587753in}{0.816667in}}{\pgfqpoint{0.591993in}{0.818423in}}{\pgfqpoint{0.595118in}{0.821548in}}%
\pgfpathcurveto{\pgfqpoint{0.598244in}{0.824674in}}{\pgfqpoint{0.600000in}{0.828913in}}{\pgfqpoint{0.600000in}{0.833333in}}%
\pgfpathcurveto{\pgfqpoint{0.600000in}{0.837753in}}{\pgfqpoint{0.598244in}{0.841993in}}{\pgfqpoint{0.595118in}{0.845118in}}%
\pgfpathcurveto{\pgfqpoint{0.591993in}{0.848244in}}{\pgfqpoint{0.587753in}{0.850000in}}{\pgfqpoint{0.583333in}{0.850000in}}%
\pgfpathcurveto{\pgfqpoint{0.578913in}{0.850000in}}{\pgfqpoint{0.574674in}{0.848244in}}{\pgfqpoint{0.571548in}{0.845118in}}%
\pgfpathcurveto{\pgfqpoint{0.568423in}{0.841993in}}{\pgfqpoint{0.566667in}{0.837753in}}{\pgfqpoint{0.566667in}{0.833333in}}%
\pgfpathcurveto{\pgfqpoint{0.566667in}{0.828913in}}{\pgfqpoint{0.568423in}{0.824674in}}{\pgfqpoint{0.571548in}{0.821548in}}%
\pgfpathcurveto{\pgfqpoint{0.574674in}{0.818423in}}{\pgfqpoint{0.578913in}{0.816667in}}{\pgfqpoint{0.583333in}{0.816667in}}%
\pgfpathclose%
\pgfpathmoveto{\pgfqpoint{0.750000in}{0.816667in}}%
\pgfpathcurveto{\pgfqpoint{0.754420in}{0.816667in}}{\pgfqpoint{0.758660in}{0.818423in}}{\pgfqpoint{0.761785in}{0.821548in}}%
\pgfpathcurveto{\pgfqpoint{0.764911in}{0.824674in}}{\pgfqpoint{0.766667in}{0.828913in}}{\pgfqpoint{0.766667in}{0.833333in}}%
\pgfpathcurveto{\pgfqpoint{0.766667in}{0.837753in}}{\pgfqpoint{0.764911in}{0.841993in}}{\pgfqpoint{0.761785in}{0.845118in}}%
\pgfpathcurveto{\pgfqpoint{0.758660in}{0.848244in}}{\pgfqpoint{0.754420in}{0.850000in}}{\pgfqpoint{0.750000in}{0.850000in}}%
\pgfpathcurveto{\pgfqpoint{0.745580in}{0.850000in}}{\pgfqpoint{0.741340in}{0.848244in}}{\pgfqpoint{0.738215in}{0.845118in}}%
\pgfpathcurveto{\pgfqpoint{0.735089in}{0.841993in}}{\pgfqpoint{0.733333in}{0.837753in}}{\pgfqpoint{0.733333in}{0.833333in}}%
\pgfpathcurveto{\pgfqpoint{0.733333in}{0.828913in}}{\pgfqpoint{0.735089in}{0.824674in}}{\pgfqpoint{0.738215in}{0.821548in}}%
\pgfpathcurveto{\pgfqpoint{0.741340in}{0.818423in}}{\pgfqpoint{0.745580in}{0.816667in}}{\pgfqpoint{0.750000in}{0.816667in}}%
\pgfpathclose%
\pgfpathmoveto{\pgfqpoint{0.916667in}{0.816667in}}%
\pgfpathcurveto{\pgfqpoint{0.921087in}{0.816667in}}{\pgfqpoint{0.925326in}{0.818423in}}{\pgfqpoint{0.928452in}{0.821548in}}%
\pgfpathcurveto{\pgfqpoint{0.931577in}{0.824674in}}{\pgfqpoint{0.933333in}{0.828913in}}{\pgfqpoint{0.933333in}{0.833333in}}%
\pgfpathcurveto{\pgfqpoint{0.933333in}{0.837753in}}{\pgfqpoint{0.931577in}{0.841993in}}{\pgfqpoint{0.928452in}{0.845118in}}%
\pgfpathcurveto{\pgfqpoint{0.925326in}{0.848244in}}{\pgfqpoint{0.921087in}{0.850000in}}{\pgfqpoint{0.916667in}{0.850000in}}%
\pgfpathcurveto{\pgfqpoint{0.912247in}{0.850000in}}{\pgfqpoint{0.908007in}{0.848244in}}{\pgfqpoint{0.904882in}{0.845118in}}%
\pgfpathcurveto{\pgfqpoint{0.901756in}{0.841993in}}{\pgfqpoint{0.900000in}{0.837753in}}{\pgfqpoint{0.900000in}{0.833333in}}%
\pgfpathcurveto{\pgfqpoint{0.900000in}{0.828913in}}{\pgfqpoint{0.901756in}{0.824674in}}{\pgfqpoint{0.904882in}{0.821548in}}%
\pgfpathcurveto{\pgfqpoint{0.908007in}{0.818423in}}{\pgfqpoint{0.912247in}{0.816667in}}{\pgfqpoint{0.916667in}{0.816667in}}%
\pgfpathclose%
\pgfpathmoveto{\pgfqpoint{0.000000in}{0.983333in}}%
\pgfpathcurveto{\pgfqpoint{0.004420in}{0.983333in}}{\pgfqpoint{0.008660in}{0.985089in}}{\pgfqpoint{0.011785in}{0.988215in}}%
\pgfpathcurveto{\pgfqpoint{0.014911in}{0.991340in}}{\pgfqpoint{0.016667in}{0.995580in}}{\pgfqpoint{0.016667in}{1.000000in}}%
\pgfpathcurveto{\pgfqpoint{0.016667in}{1.004420in}}{\pgfqpoint{0.014911in}{1.008660in}}{\pgfqpoint{0.011785in}{1.011785in}}%
\pgfpathcurveto{\pgfqpoint{0.008660in}{1.014911in}}{\pgfqpoint{0.004420in}{1.016667in}}{\pgfqpoint{0.000000in}{1.016667in}}%
\pgfpathcurveto{\pgfqpoint{-0.004420in}{1.016667in}}{\pgfqpoint{-0.008660in}{1.014911in}}{\pgfqpoint{-0.011785in}{1.011785in}}%
\pgfpathcurveto{\pgfqpoint{-0.014911in}{1.008660in}}{\pgfqpoint{-0.016667in}{1.004420in}}{\pgfqpoint{-0.016667in}{1.000000in}}%
\pgfpathcurveto{\pgfqpoint{-0.016667in}{0.995580in}}{\pgfqpoint{-0.014911in}{0.991340in}}{\pgfqpoint{-0.011785in}{0.988215in}}%
\pgfpathcurveto{\pgfqpoint{-0.008660in}{0.985089in}}{\pgfqpoint{-0.004420in}{0.983333in}}{\pgfqpoint{0.000000in}{0.983333in}}%
\pgfpathclose%
\pgfpathmoveto{\pgfqpoint{0.166667in}{0.983333in}}%
\pgfpathcurveto{\pgfqpoint{0.171087in}{0.983333in}}{\pgfqpoint{0.175326in}{0.985089in}}{\pgfqpoint{0.178452in}{0.988215in}}%
\pgfpathcurveto{\pgfqpoint{0.181577in}{0.991340in}}{\pgfqpoint{0.183333in}{0.995580in}}{\pgfqpoint{0.183333in}{1.000000in}}%
\pgfpathcurveto{\pgfqpoint{0.183333in}{1.004420in}}{\pgfqpoint{0.181577in}{1.008660in}}{\pgfqpoint{0.178452in}{1.011785in}}%
\pgfpathcurveto{\pgfqpoint{0.175326in}{1.014911in}}{\pgfqpoint{0.171087in}{1.016667in}}{\pgfqpoint{0.166667in}{1.016667in}}%
\pgfpathcurveto{\pgfqpoint{0.162247in}{1.016667in}}{\pgfqpoint{0.158007in}{1.014911in}}{\pgfqpoint{0.154882in}{1.011785in}}%
\pgfpathcurveto{\pgfqpoint{0.151756in}{1.008660in}}{\pgfqpoint{0.150000in}{1.004420in}}{\pgfqpoint{0.150000in}{1.000000in}}%
\pgfpathcurveto{\pgfqpoint{0.150000in}{0.995580in}}{\pgfqpoint{0.151756in}{0.991340in}}{\pgfqpoint{0.154882in}{0.988215in}}%
\pgfpathcurveto{\pgfqpoint{0.158007in}{0.985089in}}{\pgfqpoint{0.162247in}{0.983333in}}{\pgfqpoint{0.166667in}{0.983333in}}%
\pgfpathclose%
\pgfpathmoveto{\pgfqpoint{0.333333in}{0.983333in}}%
\pgfpathcurveto{\pgfqpoint{0.337753in}{0.983333in}}{\pgfqpoint{0.341993in}{0.985089in}}{\pgfqpoint{0.345118in}{0.988215in}}%
\pgfpathcurveto{\pgfqpoint{0.348244in}{0.991340in}}{\pgfqpoint{0.350000in}{0.995580in}}{\pgfqpoint{0.350000in}{1.000000in}}%
\pgfpathcurveto{\pgfqpoint{0.350000in}{1.004420in}}{\pgfqpoint{0.348244in}{1.008660in}}{\pgfqpoint{0.345118in}{1.011785in}}%
\pgfpathcurveto{\pgfqpoint{0.341993in}{1.014911in}}{\pgfqpoint{0.337753in}{1.016667in}}{\pgfqpoint{0.333333in}{1.016667in}}%
\pgfpathcurveto{\pgfqpoint{0.328913in}{1.016667in}}{\pgfqpoint{0.324674in}{1.014911in}}{\pgfqpoint{0.321548in}{1.011785in}}%
\pgfpathcurveto{\pgfqpoint{0.318423in}{1.008660in}}{\pgfqpoint{0.316667in}{1.004420in}}{\pgfqpoint{0.316667in}{1.000000in}}%
\pgfpathcurveto{\pgfqpoint{0.316667in}{0.995580in}}{\pgfqpoint{0.318423in}{0.991340in}}{\pgfqpoint{0.321548in}{0.988215in}}%
\pgfpathcurveto{\pgfqpoint{0.324674in}{0.985089in}}{\pgfqpoint{0.328913in}{0.983333in}}{\pgfqpoint{0.333333in}{0.983333in}}%
\pgfpathclose%
\pgfpathmoveto{\pgfqpoint{0.500000in}{0.983333in}}%
\pgfpathcurveto{\pgfqpoint{0.504420in}{0.983333in}}{\pgfqpoint{0.508660in}{0.985089in}}{\pgfqpoint{0.511785in}{0.988215in}}%
\pgfpathcurveto{\pgfqpoint{0.514911in}{0.991340in}}{\pgfqpoint{0.516667in}{0.995580in}}{\pgfqpoint{0.516667in}{1.000000in}}%
\pgfpathcurveto{\pgfqpoint{0.516667in}{1.004420in}}{\pgfqpoint{0.514911in}{1.008660in}}{\pgfqpoint{0.511785in}{1.011785in}}%
\pgfpathcurveto{\pgfqpoint{0.508660in}{1.014911in}}{\pgfqpoint{0.504420in}{1.016667in}}{\pgfqpoint{0.500000in}{1.016667in}}%
\pgfpathcurveto{\pgfqpoint{0.495580in}{1.016667in}}{\pgfqpoint{0.491340in}{1.014911in}}{\pgfqpoint{0.488215in}{1.011785in}}%
\pgfpathcurveto{\pgfqpoint{0.485089in}{1.008660in}}{\pgfqpoint{0.483333in}{1.004420in}}{\pgfqpoint{0.483333in}{1.000000in}}%
\pgfpathcurveto{\pgfqpoint{0.483333in}{0.995580in}}{\pgfqpoint{0.485089in}{0.991340in}}{\pgfqpoint{0.488215in}{0.988215in}}%
\pgfpathcurveto{\pgfqpoint{0.491340in}{0.985089in}}{\pgfqpoint{0.495580in}{0.983333in}}{\pgfqpoint{0.500000in}{0.983333in}}%
\pgfpathclose%
\pgfpathmoveto{\pgfqpoint{0.666667in}{0.983333in}}%
\pgfpathcurveto{\pgfqpoint{0.671087in}{0.983333in}}{\pgfqpoint{0.675326in}{0.985089in}}{\pgfqpoint{0.678452in}{0.988215in}}%
\pgfpathcurveto{\pgfqpoint{0.681577in}{0.991340in}}{\pgfqpoint{0.683333in}{0.995580in}}{\pgfqpoint{0.683333in}{1.000000in}}%
\pgfpathcurveto{\pgfqpoint{0.683333in}{1.004420in}}{\pgfqpoint{0.681577in}{1.008660in}}{\pgfqpoint{0.678452in}{1.011785in}}%
\pgfpathcurveto{\pgfqpoint{0.675326in}{1.014911in}}{\pgfqpoint{0.671087in}{1.016667in}}{\pgfqpoint{0.666667in}{1.016667in}}%
\pgfpathcurveto{\pgfqpoint{0.662247in}{1.016667in}}{\pgfqpoint{0.658007in}{1.014911in}}{\pgfqpoint{0.654882in}{1.011785in}}%
\pgfpathcurveto{\pgfqpoint{0.651756in}{1.008660in}}{\pgfqpoint{0.650000in}{1.004420in}}{\pgfqpoint{0.650000in}{1.000000in}}%
\pgfpathcurveto{\pgfqpoint{0.650000in}{0.995580in}}{\pgfqpoint{0.651756in}{0.991340in}}{\pgfqpoint{0.654882in}{0.988215in}}%
\pgfpathcurveto{\pgfqpoint{0.658007in}{0.985089in}}{\pgfqpoint{0.662247in}{0.983333in}}{\pgfqpoint{0.666667in}{0.983333in}}%
\pgfpathclose%
\pgfpathmoveto{\pgfqpoint{0.833333in}{0.983333in}}%
\pgfpathcurveto{\pgfqpoint{0.837753in}{0.983333in}}{\pgfqpoint{0.841993in}{0.985089in}}{\pgfqpoint{0.845118in}{0.988215in}}%
\pgfpathcurveto{\pgfqpoint{0.848244in}{0.991340in}}{\pgfqpoint{0.850000in}{0.995580in}}{\pgfqpoint{0.850000in}{1.000000in}}%
\pgfpathcurveto{\pgfqpoint{0.850000in}{1.004420in}}{\pgfqpoint{0.848244in}{1.008660in}}{\pgfqpoint{0.845118in}{1.011785in}}%
\pgfpathcurveto{\pgfqpoint{0.841993in}{1.014911in}}{\pgfqpoint{0.837753in}{1.016667in}}{\pgfqpoint{0.833333in}{1.016667in}}%
\pgfpathcurveto{\pgfqpoint{0.828913in}{1.016667in}}{\pgfqpoint{0.824674in}{1.014911in}}{\pgfqpoint{0.821548in}{1.011785in}}%
\pgfpathcurveto{\pgfqpoint{0.818423in}{1.008660in}}{\pgfqpoint{0.816667in}{1.004420in}}{\pgfqpoint{0.816667in}{1.000000in}}%
\pgfpathcurveto{\pgfqpoint{0.816667in}{0.995580in}}{\pgfqpoint{0.818423in}{0.991340in}}{\pgfqpoint{0.821548in}{0.988215in}}%
\pgfpathcurveto{\pgfqpoint{0.824674in}{0.985089in}}{\pgfqpoint{0.828913in}{0.983333in}}{\pgfqpoint{0.833333in}{0.983333in}}%
\pgfpathclose%
\pgfpathmoveto{\pgfqpoint{1.000000in}{0.983333in}}%
\pgfpathcurveto{\pgfqpoint{1.004420in}{0.983333in}}{\pgfqpoint{1.008660in}{0.985089in}}{\pgfqpoint{1.011785in}{0.988215in}}%
\pgfpathcurveto{\pgfqpoint{1.014911in}{0.991340in}}{\pgfqpoint{1.016667in}{0.995580in}}{\pgfqpoint{1.016667in}{1.000000in}}%
\pgfpathcurveto{\pgfqpoint{1.016667in}{1.004420in}}{\pgfqpoint{1.014911in}{1.008660in}}{\pgfqpoint{1.011785in}{1.011785in}}%
\pgfpathcurveto{\pgfqpoint{1.008660in}{1.014911in}}{\pgfqpoint{1.004420in}{1.016667in}}{\pgfqpoint{1.000000in}{1.016667in}}%
\pgfpathcurveto{\pgfqpoint{0.995580in}{1.016667in}}{\pgfqpoint{0.991340in}{1.014911in}}{\pgfqpoint{0.988215in}{1.011785in}}%
\pgfpathcurveto{\pgfqpoint{0.985089in}{1.008660in}}{\pgfqpoint{0.983333in}{1.004420in}}{\pgfqpoint{0.983333in}{1.000000in}}%
\pgfpathcurveto{\pgfqpoint{0.983333in}{0.995580in}}{\pgfqpoint{0.985089in}{0.991340in}}{\pgfqpoint{0.988215in}{0.988215in}}%
\pgfpathcurveto{\pgfqpoint{0.991340in}{0.985089in}}{\pgfqpoint{0.995580in}{0.983333in}}{\pgfqpoint{1.000000in}{0.983333in}}%
\pgfpathclose%
\pgfusepath{stroke}%
\end{pgfscope}%
}%
\pgfsys@transformshift{7.458038in}{0.637495in}%
\end{pgfscope}%
\begin{pgfscope}%
\pgfpathrectangle{\pgfqpoint{0.870538in}{0.637495in}}{\pgfqpoint{9.300000in}{9.060000in}}%
\pgfusepath{clip}%
\pgfsetbuttcap%
\pgfsetmiterjoin%
\definecolor{currentfill}{rgb}{0.172549,0.627451,0.172549}%
\pgfsetfillcolor{currentfill}%
\pgfsetfillopacity{0.990000}%
\pgfsetlinewidth{0.000000pt}%
\definecolor{currentstroke}{rgb}{0.000000,0.000000,0.000000}%
\pgfsetstrokecolor{currentstroke}%
\pgfsetstrokeopacity{0.990000}%
\pgfsetdash{}{0pt}%
\pgfpathmoveto{\pgfqpoint{9.008038in}{3.751700in}}%
\pgfpathlineto{\pgfqpoint{9.783038in}{3.751700in}}%
\pgfpathlineto{\pgfqpoint{9.783038in}{4.114678in}}%
\pgfpathlineto{\pgfqpoint{9.008038in}{4.114678in}}%
\pgfpathclose%
\pgfusepath{fill}%
\end{pgfscope}%
\begin{pgfscope}%
\pgfsetbuttcap%
\pgfsetmiterjoin%
\definecolor{currentfill}{rgb}{0.172549,0.627451,0.172549}%
\pgfsetfillcolor{currentfill}%
\pgfsetfillopacity{0.990000}%
\pgfsetlinewidth{0.000000pt}%
\definecolor{currentstroke}{rgb}{0.000000,0.000000,0.000000}%
\pgfsetstrokecolor{currentstroke}%
\pgfsetstrokeopacity{0.990000}%
\pgfsetdash{}{0pt}%
\pgfpathrectangle{\pgfqpoint{0.870538in}{0.637495in}}{\pgfqpoint{9.300000in}{9.060000in}}%
\pgfusepath{clip}%
\pgfpathmoveto{\pgfqpoint{9.008038in}{3.751700in}}%
\pgfpathlineto{\pgfqpoint{9.783038in}{3.751700in}}%
\pgfpathlineto{\pgfqpoint{9.783038in}{4.114678in}}%
\pgfpathlineto{\pgfqpoint{9.008038in}{4.114678in}}%
\pgfpathclose%
\pgfusepath{clip}%
\pgfsys@defobject{currentpattern}{\pgfqpoint{0in}{0in}}{\pgfqpoint{1in}{1in}}{%
\begin{pgfscope}%
\pgfpathrectangle{\pgfqpoint{0in}{0in}}{\pgfqpoint{1in}{1in}}%
\pgfusepath{clip}%
\pgfpathmoveto{\pgfqpoint{0.000000in}{-0.016667in}}%
\pgfpathcurveto{\pgfqpoint{0.004420in}{-0.016667in}}{\pgfqpoint{0.008660in}{-0.014911in}}{\pgfqpoint{0.011785in}{-0.011785in}}%
\pgfpathcurveto{\pgfqpoint{0.014911in}{-0.008660in}}{\pgfqpoint{0.016667in}{-0.004420in}}{\pgfqpoint{0.016667in}{0.000000in}}%
\pgfpathcurveto{\pgfqpoint{0.016667in}{0.004420in}}{\pgfqpoint{0.014911in}{0.008660in}}{\pgfqpoint{0.011785in}{0.011785in}}%
\pgfpathcurveto{\pgfqpoint{0.008660in}{0.014911in}}{\pgfqpoint{0.004420in}{0.016667in}}{\pgfqpoint{0.000000in}{0.016667in}}%
\pgfpathcurveto{\pgfqpoint{-0.004420in}{0.016667in}}{\pgfqpoint{-0.008660in}{0.014911in}}{\pgfqpoint{-0.011785in}{0.011785in}}%
\pgfpathcurveto{\pgfqpoint{-0.014911in}{0.008660in}}{\pgfqpoint{-0.016667in}{0.004420in}}{\pgfqpoint{-0.016667in}{0.000000in}}%
\pgfpathcurveto{\pgfqpoint{-0.016667in}{-0.004420in}}{\pgfqpoint{-0.014911in}{-0.008660in}}{\pgfqpoint{-0.011785in}{-0.011785in}}%
\pgfpathcurveto{\pgfqpoint{-0.008660in}{-0.014911in}}{\pgfqpoint{-0.004420in}{-0.016667in}}{\pgfqpoint{0.000000in}{-0.016667in}}%
\pgfpathclose%
\pgfpathmoveto{\pgfqpoint{0.166667in}{-0.016667in}}%
\pgfpathcurveto{\pgfqpoint{0.171087in}{-0.016667in}}{\pgfqpoint{0.175326in}{-0.014911in}}{\pgfqpoint{0.178452in}{-0.011785in}}%
\pgfpathcurveto{\pgfqpoint{0.181577in}{-0.008660in}}{\pgfqpoint{0.183333in}{-0.004420in}}{\pgfqpoint{0.183333in}{0.000000in}}%
\pgfpathcurveto{\pgfqpoint{0.183333in}{0.004420in}}{\pgfqpoint{0.181577in}{0.008660in}}{\pgfqpoint{0.178452in}{0.011785in}}%
\pgfpathcurveto{\pgfqpoint{0.175326in}{0.014911in}}{\pgfqpoint{0.171087in}{0.016667in}}{\pgfqpoint{0.166667in}{0.016667in}}%
\pgfpathcurveto{\pgfqpoint{0.162247in}{0.016667in}}{\pgfqpoint{0.158007in}{0.014911in}}{\pgfqpoint{0.154882in}{0.011785in}}%
\pgfpathcurveto{\pgfqpoint{0.151756in}{0.008660in}}{\pgfqpoint{0.150000in}{0.004420in}}{\pgfqpoint{0.150000in}{0.000000in}}%
\pgfpathcurveto{\pgfqpoint{0.150000in}{-0.004420in}}{\pgfqpoint{0.151756in}{-0.008660in}}{\pgfqpoint{0.154882in}{-0.011785in}}%
\pgfpathcurveto{\pgfqpoint{0.158007in}{-0.014911in}}{\pgfqpoint{0.162247in}{-0.016667in}}{\pgfqpoint{0.166667in}{-0.016667in}}%
\pgfpathclose%
\pgfpathmoveto{\pgfqpoint{0.333333in}{-0.016667in}}%
\pgfpathcurveto{\pgfqpoint{0.337753in}{-0.016667in}}{\pgfqpoint{0.341993in}{-0.014911in}}{\pgfqpoint{0.345118in}{-0.011785in}}%
\pgfpathcurveto{\pgfqpoint{0.348244in}{-0.008660in}}{\pgfqpoint{0.350000in}{-0.004420in}}{\pgfqpoint{0.350000in}{0.000000in}}%
\pgfpathcurveto{\pgfqpoint{0.350000in}{0.004420in}}{\pgfqpoint{0.348244in}{0.008660in}}{\pgfqpoint{0.345118in}{0.011785in}}%
\pgfpathcurveto{\pgfqpoint{0.341993in}{0.014911in}}{\pgfqpoint{0.337753in}{0.016667in}}{\pgfqpoint{0.333333in}{0.016667in}}%
\pgfpathcurveto{\pgfqpoint{0.328913in}{0.016667in}}{\pgfqpoint{0.324674in}{0.014911in}}{\pgfqpoint{0.321548in}{0.011785in}}%
\pgfpathcurveto{\pgfqpoint{0.318423in}{0.008660in}}{\pgfqpoint{0.316667in}{0.004420in}}{\pgfqpoint{0.316667in}{0.000000in}}%
\pgfpathcurveto{\pgfqpoint{0.316667in}{-0.004420in}}{\pgfqpoint{0.318423in}{-0.008660in}}{\pgfqpoint{0.321548in}{-0.011785in}}%
\pgfpathcurveto{\pgfqpoint{0.324674in}{-0.014911in}}{\pgfqpoint{0.328913in}{-0.016667in}}{\pgfqpoint{0.333333in}{-0.016667in}}%
\pgfpathclose%
\pgfpathmoveto{\pgfqpoint{0.500000in}{-0.016667in}}%
\pgfpathcurveto{\pgfqpoint{0.504420in}{-0.016667in}}{\pgfqpoint{0.508660in}{-0.014911in}}{\pgfqpoint{0.511785in}{-0.011785in}}%
\pgfpathcurveto{\pgfqpoint{0.514911in}{-0.008660in}}{\pgfqpoint{0.516667in}{-0.004420in}}{\pgfqpoint{0.516667in}{0.000000in}}%
\pgfpathcurveto{\pgfqpoint{0.516667in}{0.004420in}}{\pgfqpoint{0.514911in}{0.008660in}}{\pgfqpoint{0.511785in}{0.011785in}}%
\pgfpathcurveto{\pgfqpoint{0.508660in}{0.014911in}}{\pgfqpoint{0.504420in}{0.016667in}}{\pgfqpoint{0.500000in}{0.016667in}}%
\pgfpathcurveto{\pgfqpoint{0.495580in}{0.016667in}}{\pgfqpoint{0.491340in}{0.014911in}}{\pgfqpoint{0.488215in}{0.011785in}}%
\pgfpathcurveto{\pgfqpoint{0.485089in}{0.008660in}}{\pgfqpoint{0.483333in}{0.004420in}}{\pgfqpoint{0.483333in}{0.000000in}}%
\pgfpathcurveto{\pgfqpoint{0.483333in}{-0.004420in}}{\pgfqpoint{0.485089in}{-0.008660in}}{\pgfqpoint{0.488215in}{-0.011785in}}%
\pgfpathcurveto{\pgfqpoint{0.491340in}{-0.014911in}}{\pgfqpoint{0.495580in}{-0.016667in}}{\pgfqpoint{0.500000in}{-0.016667in}}%
\pgfpathclose%
\pgfpathmoveto{\pgfqpoint{0.666667in}{-0.016667in}}%
\pgfpathcurveto{\pgfqpoint{0.671087in}{-0.016667in}}{\pgfqpoint{0.675326in}{-0.014911in}}{\pgfqpoint{0.678452in}{-0.011785in}}%
\pgfpathcurveto{\pgfqpoint{0.681577in}{-0.008660in}}{\pgfqpoint{0.683333in}{-0.004420in}}{\pgfqpoint{0.683333in}{0.000000in}}%
\pgfpathcurveto{\pgfqpoint{0.683333in}{0.004420in}}{\pgfqpoint{0.681577in}{0.008660in}}{\pgfqpoint{0.678452in}{0.011785in}}%
\pgfpathcurveto{\pgfqpoint{0.675326in}{0.014911in}}{\pgfqpoint{0.671087in}{0.016667in}}{\pgfqpoint{0.666667in}{0.016667in}}%
\pgfpathcurveto{\pgfqpoint{0.662247in}{0.016667in}}{\pgfqpoint{0.658007in}{0.014911in}}{\pgfqpoint{0.654882in}{0.011785in}}%
\pgfpathcurveto{\pgfqpoint{0.651756in}{0.008660in}}{\pgfqpoint{0.650000in}{0.004420in}}{\pgfqpoint{0.650000in}{0.000000in}}%
\pgfpathcurveto{\pgfqpoint{0.650000in}{-0.004420in}}{\pgfqpoint{0.651756in}{-0.008660in}}{\pgfqpoint{0.654882in}{-0.011785in}}%
\pgfpathcurveto{\pgfqpoint{0.658007in}{-0.014911in}}{\pgfqpoint{0.662247in}{-0.016667in}}{\pgfqpoint{0.666667in}{-0.016667in}}%
\pgfpathclose%
\pgfpathmoveto{\pgfqpoint{0.833333in}{-0.016667in}}%
\pgfpathcurveto{\pgfqpoint{0.837753in}{-0.016667in}}{\pgfqpoint{0.841993in}{-0.014911in}}{\pgfqpoint{0.845118in}{-0.011785in}}%
\pgfpathcurveto{\pgfqpoint{0.848244in}{-0.008660in}}{\pgfqpoint{0.850000in}{-0.004420in}}{\pgfqpoint{0.850000in}{0.000000in}}%
\pgfpathcurveto{\pgfqpoint{0.850000in}{0.004420in}}{\pgfqpoint{0.848244in}{0.008660in}}{\pgfqpoint{0.845118in}{0.011785in}}%
\pgfpathcurveto{\pgfqpoint{0.841993in}{0.014911in}}{\pgfqpoint{0.837753in}{0.016667in}}{\pgfqpoint{0.833333in}{0.016667in}}%
\pgfpathcurveto{\pgfqpoint{0.828913in}{0.016667in}}{\pgfqpoint{0.824674in}{0.014911in}}{\pgfqpoint{0.821548in}{0.011785in}}%
\pgfpathcurveto{\pgfqpoint{0.818423in}{0.008660in}}{\pgfqpoint{0.816667in}{0.004420in}}{\pgfqpoint{0.816667in}{0.000000in}}%
\pgfpathcurveto{\pgfqpoint{0.816667in}{-0.004420in}}{\pgfqpoint{0.818423in}{-0.008660in}}{\pgfqpoint{0.821548in}{-0.011785in}}%
\pgfpathcurveto{\pgfqpoint{0.824674in}{-0.014911in}}{\pgfqpoint{0.828913in}{-0.016667in}}{\pgfqpoint{0.833333in}{-0.016667in}}%
\pgfpathclose%
\pgfpathmoveto{\pgfqpoint{1.000000in}{-0.016667in}}%
\pgfpathcurveto{\pgfqpoint{1.004420in}{-0.016667in}}{\pgfqpoint{1.008660in}{-0.014911in}}{\pgfqpoint{1.011785in}{-0.011785in}}%
\pgfpathcurveto{\pgfqpoint{1.014911in}{-0.008660in}}{\pgfqpoint{1.016667in}{-0.004420in}}{\pgfqpoint{1.016667in}{0.000000in}}%
\pgfpathcurveto{\pgfqpoint{1.016667in}{0.004420in}}{\pgfqpoint{1.014911in}{0.008660in}}{\pgfqpoint{1.011785in}{0.011785in}}%
\pgfpathcurveto{\pgfqpoint{1.008660in}{0.014911in}}{\pgfqpoint{1.004420in}{0.016667in}}{\pgfqpoint{1.000000in}{0.016667in}}%
\pgfpathcurveto{\pgfqpoint{0.995580in}{0.016667in}}{\pgfqpoint{0.991340in}{0.014911in}}{\pgfqpoint{0.988215in}{0.011785in}}%
\pgfpathcurveto{\pgfqpoint{0.985089in}{0.008660in}}{\pgfqpoint{0.983333in}{0.004420in}}{\pgfqpoint{0.983333in}{0.000000in}}%
\pgfpathcurveto{\pgfqpoint{0.983333in}{-0.004420in}}{\pgfqpoint{0.985089in}{-0.008660in}}{\pgfqpoint{0.988215in}{-0.011785in}}%
\pgfpathcurveto{\pgfqpoint{0.991340in}{-0.014911in}}{\pgfqpoint{0.995580in}{-0.016667in}}{\pgfqpoint{1.000000in}{-0.016667in}}%
\pgfpathclose%
\pgfpathmoveto{\pgfqpoint{0.083333in}{0.150000in}}%
\pgfpathcurveto{\pgfqpoint{0.087753in}{0.150000in}}{\pgfqpoint{0.091993in}{0.151756in}}{\pgfqpoint{0.095118in}{0.154882in}}%
\pgfpathcurveto{\pgfqpoint{0.098244in}{0.158007in}}{\pgfqpoint{0.100000in}{0.162247in}}{\pgfqpoint{0.100000in}{0.166667in}}%
\pgfpathcurveto{\pgfqpoint{0.100000in}{0.171087in}}{\pgfqpoint{0.098244in}{0.175326in}}{\pgfqpoint{0.095118in}{0.178452in}}%
\pgfpathcurveto{\pgfqpoint{0.091993in}{0.181577in}}{\pgfqpoint{0.087753in}{0.183333in}}{\pgfqpoint{0.083333in}{0.183333in}}%
\pgfpathcurveto{\pgfqpoint{0.078913in}{0.183333in}}{\pgfqpoint{0.074674in}{0.181577in}}{\pgfqpoint{0.071548in}{0.178452in}}%
\pgfpathcurveto{\pgfqpoint{0.068423in}{0.175326in}}{\pgfqpoint{0.066667in}{0.171087in}}{\pgfqpoint{0.066667in}{0.166667in}}%
\pgfpathcurveto{\pgfqpoint{0.066667in}{0.162247in}}{\pgfqpoint{0.068423in}{0.158007in}}{\pgfqpoint{0.071548in}{0.154882in}}%
\pgfpathcurveto{\pgfqpoint{0.074674in}{0.151756in}}{\pgfqpoint{0.078913in}{0.150000in}}{\pgfqpoint{0.083333in}{0.150000in}}%
\pgfpathclose%
\pgfpathmoveto{\pgfqpoint{0.250000in}{0.150000in}}%
\pgfpathcurveto{\pgfqpoint{0.254420in}{0.150000in}}{\pgfqpoint{0.258660in}{0.151756in}}{\pgfqpoint{0.261785in}{0.154882in}}%
\pgfpathcurveto{\pgfqpoint{0.264911in}{0.158007in}}{\pgfqpoint{0.266667in}{0.162247in}}{\pgfqpoint{0.266667in}{0.166667in}}%
\pgfpathcurveto{\pgfqpoint{0.266667in}{0.171087in}}{\pgfqpoint{0.264911in}{0.175326in}}{\pgfqpoint{0.261785in}{0.178452in}}%
\pgfpathcurveto{\pgfqpoint{0.258660in}{0.181577in}}{\pgfqpoint{0.254420in}{0.183333in}}{\pgfqpoint{0.250000in}{0.183333in}}%
\pgfpathcurveto{\pgfqpoint{0.245580in}{0.183333in}}{\pgfqpoint{0.241340in}{0.181577in}}{\pgfqpoint{0.238215in}{0.178452in}}%
\pgfpathcurveto{\pgfqpoint{0.235089in}{0.175326in}}{\pgfqpoint{0.233333in}{0.171087in}}{\pgfqpoint{0.233333in}{0.166667in}}%
\pgfpathcurveto{\pgfqpoint{0.233333in}{0.162247in}}{\pgfqpoint{0.235089in}{0.158007in}}{\pgfqpoint{0.238215in}{0.154882in}}%
\pgfpathcurveto{\pgfqpoint{0.241340in}{0.151756in}}{\pgfqpoint{0.245580in}{0.150000in}}{\pgfqpoint{0.250000in}{0.150000in}}%
\pgfpathclose%
\pgfpathmoveto{\pgfqpoint{0.416667in}{0.150000in}}%
\pgfpathcurveto{\pgfqpoint{0.421087in}{0.150000in}}{\pgfqpoint{0.425326in}{0.151756in}}{\pgfqpoint{0.428452in}{0.154882in}}%
\pgfpathcurveto{\pgfqpoint{0.431577in}{0.158007in}}{\pgfqpoint{0.433333in}{0.162247in}}{\pgfqpoint{0.433333in}{0.166667in}}%
\pgfpathcurveto{\pgfqpoint{0.433333in}{0.171087in}}{\pgfqpoint{0.431577in}{0.175326in}}{\pgfqpoint{0.428452in}{0.178452in}}%
\pgfpathcurveto{\pgfqpoint{0.425326in}{0.181577in}}{\pgfqpoint{0.421087in}{0.183333in}}{\pgfqpoint{0.416667in}{0.183333in}}%
\pgfpathcurveto{\pgfqpoint{0.412247in}{0.183333in}}{\pgfqpoint{0.408007in}{0.181577in}}{\pgfqpoint{0.404882in}{0.178452in}}%
\pgfpathcurveto{\pgfqpoint{0.401756in}{0.175326in}}{\pgfqpoint{0.400000in}{0.171087in}}{\pgfqpoint{0.400000in}{0.166667in}}%
\pgfpathcurveto{\pgfqpoint{0.400000in}{0.162247in}}{\pgfqpoint{0.401756in}{0.158007in}}{\pgfqpoint{0.404882in}{0.154882in}}%
\pgfpathcurveto{\pgfqpoint{0.408007in}{0.151756in}}{\pgfqpoint{0.412247in}{0.150000in}}{\pgfqpoint{0.416667in}{0.150000in}}%
\pgfpathclose%
\pgfpathmoveto{\pgfqpoint{0.583333in}{0.150000in}}%
\pgfpathcurveto{\pgfqpoint{0.587753in}{0.150000in}}{\pgfqpoint{0.591993in}{0.151756in}}{\pgfqpoint{0.595118in}{0.154882in}}%
\pgfpathcurveto{\pgfqpoint{0.598244in}{0.158007in}}{\pgfqpoint{0.600000in}{0.162247in}}{\pgfqpoint{0.600000in}{0.166667in}}%
\pgfpathcurveto{\pgfqpoint{0.600000in}{0.171087in}}{\pgfqpoint{0.598244in}{0.175326in}}{\pgfqpoint{0.595118in}{0.178452in}}%
\pgfpathcurveto{\pgfqpoint{0.591993in}{0.181577in}}{\pgfqpoint{0.587753in}{0.183333in}}{\pgfqpoint{0.583333in}{0.183333in}}%
\pgfpathcurveto{\pgfqpoint{0.578913in}{0.183333in}}{\pgfqpoint{0.574674in}{0.181577in}}{\pgfqpoint{0.571548in}{0.178452in}}%
\pgfpathcurveto{\pgfqpoint{0.568423in}{0.175326in}}{\pgfqpoint{0.566667in}{0.171087in}}{\pgfqpoint{0.566667in}{0.166667in}}%
\pgfpathcurveto{\pgfqpoint{0.566667in}{0.162247in}}{\pgfqpoint{0.568423in}{0.158007in}}{\pgfqpoint{0.571548in}{0.154882in}}%
\pgfpathcurveto{\pgfqpoint{0.574674in}{0.151756in}}{\pgfqpoint{0.578913in}{0.150000in}}{\pgfqpoint{0.583333in}{0.150000in}}%
\pgfpathclose%
\pgfpathmoveto{\pgfqpoint{0.750000in}{0.150000in}}%
\pgfpathcurveto{\pgfqpoint{0.754420in}{0.150000in}}{\pgfqpoint{0.758660in}{0.151756in}}{\pgfqpoint{0.761785in}{0.154882in}}%
\pgfpathcurveto{\pgfqpoint{0.764911in}{0.158007in}}{\pgfqpoint{0.766667in}{0.162247in}}{\pgfqpoint{0.766667in}{0.166667in}}%
\pgfpathcurveto{\pgfqpoint{0.766667in}{0.171087in}}{\pgfqpoint{0.764911in}{0.175326in}}{\pgfqpoint{0.761785in}{0.178452in}}%
\pgfpathcurveto{\pgfqpoint{0.758660in}{0.181577in}}{\pgfqpoint{0.754420in}{0.183333in}}{\pgfqpoint{0.750000in}{0.183333in}}%
\pgfpathcurveto{\pgfqpoint{0.745580in}{0.183333in}}{\pgfqpoint{0.741340in}{0.181577in}}{\pgfqpoint{0.738215in}{0.178452in}}%
\pgfpathcurveto{\pgfqpoint{0.735089in}{0.175326in}}{\pgfqpoint{0.733333in}{0.171087in}}{\pgfqpoint{0.733333in}{0.166667in}}%
\pgfpathcurveto{\pgfqpoint{0.733333in}{0.162247in}}{\pgfqpoint{0.735089in}{0.158007in}}{\pgfqpoint{0.738215in}{0.154882in}}%
\pgfpathcurveto{\pgfqpoint{0.741340in}{0.151756in}}{\pgfqpoint{0.745580in}{0.150000in}}{\pgfqpoint{0.750000in}{0.150000in}}%
\pgfpathclose%
\pgfpathmoveto{\pgfqpoint{0.916667in}{0.150000in}}%
\pgfpathcurveto{\pgfqpoint{0.921087in}{0.150000in}}{\pgfqpoint{0.925326in}{0.151756in}}{\pgfqpoint{0.928452in}{0.154882in}}%
\pgfpathcurveto{\pgfqpoint{0.931577in}{0.158007in}}{\pgfqpoint{0.933333in}{0.162247in}}{\pgfqpoint{0.933333in}{0.166667in}}%
\pgfpathcurveto{\pgfqpoint{0.933333in}{0.171087in}}{\pgfqpoint{0.931577in}{0.175326in}}{\pgfqpoint{0.928452in}{0.178452in}}%
\pgfpathcurveto{\pgfqpoint{0.925326in}{0.181577in}}{\pgfqpoint{0.921087in}{0.183333in}}{\pgfqpoint{0.916667in}{0.183333in}}%
\pgfpathcurveto{\pgfqpoint{0.912247in}{0.183333in}}{\pgfqpoint{0.908007in}{0.181577in}}{\pgfqpoint{0.904882in}{0.178452in}}%
\pgfpathcurveto{\pgfqpoint{0.901756in}{0.175326in}}{\pgfqpoint{0.900000in}{0.171087in}}{\pgfqpoint{0.900000in}{0.166667in}}%
\pgfpathcurveto{\pgfqpoint{0.900000in}{0.162247in}}{\pgfqpoint{0.901756in}{0.158007in}}{\pgfqpoint{0.904882in}{0.154882in}}%
\pgfpathcurveto{\pgfqpoint{0.908007in}{0.151756in}}{\pgfqpoint{0.912247in}{0.150000in}}{\pgfqpoint{0.916667in}{0.150000in}}%
\pgfpathclose%
\pgfpathmoveto{\pgfqpoint{0.000000in}{0.316667in}}%
\pgfpathcurveto{\pgfqpoint{0.004420in}{0.316667in}}{\pgfqpoint{0.008660in}{0.318423in}}{\pgfqpoint{0.011785in}{0.321548in}}%
\pgfpathcurveto{\pgfqpoint{0.014911in}{0.324674in}}{\pgfqpoint{0.016667in}{0.328913in}}{\pgfqpoint{0.016667in}{0.333333in}}%
\pgfpathcurveto{\pgfqpoint{0.016667in}{0.337753in}}{\pgfqpoint{0.014911in}{0.341993in}}{\pgfqpoint{0.011785in}{0.345118in}}%
\pgfpathcurveto{\pgfqpoint{0.008660in}{0.348244in}}{\pgfqpoint{0.004420in}{0.350000in}}{\pgfqpoint{0.000000in}{0.350000in}}%
\pgfpathcurveto{\pgfqpoint{-0.004420in}{0.350000in}}{\pgfqpoint{-0.008660in}{0.348244in}}{\pgfqpoint{-0.011785in}{0.345118in}}%
\pgfpathcurveto{\pgfqpoint{-0.014911in}{0.341993in}}{\pgfqpoint{-0.016667in}{0.337753in}}{\pgfqpoint{-0.016667in}{0.333333in}}%
\pgfpathcurveto{\pgfqpoint{-0.016667in}{0.328913in}}{\pgfqpoint{-0.014911in}{0.324674in}}{\pgfqpoint{-0.011785in}{0.321548in}}%
\pgfpathcurveto{\pgfqpoint{-0.008660in}{0.318423in}}{\pgfqpoint{-0.004420in}{0.316667in}}{\pgfqpoint{0.000000in}{0.316667in}}%
\pgfpathclose%
\pgfpathmoveto{\pgfqpoint{0.166667in}{0.316667in}}%
\pgfpathcurveto{\pgfqpoint{0.171087in}{0.316667in}}{\pgfqpoint{0.175326in}{0.318423in}}{\pgfqpoint{0.178452in}{0.321548in}}%
\pgfpathcurveto{\pgfqpoint{0.181577in}{0.324674in}}{\pgfqpoint{0.183333in}{0.328913in}}{\pgfqpoint{0.183333in}{0.333333in}}%
\pgfpathcurveto{\pgfqpoint{0.183333in}{0.337753in}}{\pgfqpoint{0.181577in}{0.341993in}}{\pgfqpoint{0.178452in}{0.345118in}}%
\pgfpathcurveto{\pgfqpoint{0.175326in}{0.348244in}}{\pgfqpoint{0.171087in}{0.350000in}}{\pgfqpoint{0.166667in}{0.350000in}}%
\pgfpathcurveto{\pgfqpoint{0.162247in}{0.350000in}}{\pgfqpoint{0.158007in}{0.348244in}}{\pgfqpoint{0.154882in}{0.345118in}}%
\pgfpathcurveto{\pgfqpoint{0.151756in}{0.341993in}}{\pgfqpoint{0.150000in}{0.337753in}}{\pgfqpoint{0.150000in}{0.333333in}}%
\pgfpathcurveto{\pgfqpoint{0.150000in}{0.328913in}}{\pgfqpoint{0.151756in}{0.324674in}}{\pgfqpoint{0.154882in}{0.321548in}}%
\pgfpathcurveto{\pgfqpoint{0.158007in}{0.318423in}}{\pgfqpoint{0.162247in}{0.316667in}}{\pgfqpoint{0.166667in}{0.316667in}}%
\pgfpathclose%
\pgfpathmoveto{\pgfqpoint{0.333333in}{0.316667in}}%
\pgfpathcurveto{\pgfqpoint{0.337753in}{0.316667in}}{\pgfqpoint{0.341993in}{0.318423in}}{\pgfqpoint{0.345118in}{0.321548in}}%
\pgfpathcurveto{\pgfqpoint{0.348244in}{0.324674in}}{\pgfqpoint{0.350000in}{0.328913in}}{\pgfqpoint{0.350000in}{0.333333in}}%
\pgfpathcurveto{\pgfqpoint{0.350000in}{0.337753in}}{\pgfqpoint{0.348244in}{0.341993in}}{\pgfqpoint{0.345118in}{0.345118in}}%
\pgfpathcurveto{\pgfqpoint{0.341993in}{0.348244in}}{\pgfqpoint{0.337753in}{0.350000in}}{\pgfqpoint{0.333333in}{0.350000in}}%
\pgfpathcurveto{\pgfqpoint{0.328913in}{0.350000in}}{\pgfqpoint{0.324674in}{0.348244in}}{\pgfqpoint{0.321548in}{0.345118in}}%
\pgfpathcurveto{\pgfqpoint{0.318423in}{0.341993in}}{\pgfqpoint{0.316667in}{0.337753in}}{\pgfqpoint{0.316667in}{0.333333in}}%
\pgfpathcurveto{\pgfqpoint{0.316667in}{0.328913in}}{\pgfqpoint{0.318423in}{0.324674in}}{\pgfqpoint{0.321548in}{0.321548in}}%
\pgfpathcurveto{\pgfqpoint{0.324674in}{0.318423in}}{\pgfqpoint{0.328913in}{0.316667in}}{\pgfqpoint{0.333333in}{0.316667in}}%
\pgfpathclose%
\pgfpathmoveto{\pgfqpoint{0.500000in}{0.316667in}}%
\pgfpathcurveto{\pgfqpoint{0.504420in}{0.316667in}}{\pgfqpoint{0.508660in}{0.318423in}}{\pgfqpoint{0.511785in}{0.321548in}}%
\pgfpathcurveto{\pgfqpoint{0.514911in}{0.324674in}}{\pgfqpoint{0.516667in}{0.328913in}}{\pgfqpoint{0.516667in}{0.333333in}}%
\pgfpathcurveto{\pgfqpoint{0.516667in}{0.337753in}}{\pgfqpoint{0.514911in}{0.341993in}}{\pgfqpoint{0.511785in}{0.345118in}}%
\pgfpathcurveto{\pgfqpoint{0.508660in}{0.348244in}}{\pgfqpoint{0.504420in}{0.350000in}}{\pgfqpoint{0.500000in}{0.350000in}}%
\pgfpathcurveto{\pgfqpoint{0.495580in}{0.350000in}}{\pgfqpoint{0.491340in}{0.348244in}}{\pgfqpoint{0.488215in}{0.345118in}}%
\pgfpathcurveto{\pgfqpoint{0.485089in}{0.341993in}}{\pgfqpoint{0.483333in}{0.337753in}}{\pgfqpoint{0.483333in}{0.333333in}}%
\pgfpathcurveto{\pgfqpoint{0.483333in}{0.328913in}}{\pgfqpoint{0.485089in}{0.324674in}}{\pgfqpoint{0.488215in}{0.321548in}}%
\pgfpathcurveto{\pgfqpoint{0.491340in}{0.318423in}}{\pgfqpoint{0.495580in}{0.316667in}}{\pgfqpoint{0.500000in}{0.316667in}}%
\pgfpathclose%
\pgfpathmoveto{\pgfqpoint{0.666667in}{0.316667in}}%
\pgfpathcurveto{\pgfqpoint{0.671087in}{0.316667in}}{\pgfqpoint{0.675326in}{0.318423in}}{\pgfqpoint{0.678452in}{0.321548in}}%
\pgfpathcurveto{\pgfqpoint{0.681577in}{0.324674in}}{\pgfqpoint{0.683333in}{0.328913in}}{\pgfqpoint{0.683333in}{0.333333in}}%
\pgfpathcurveto{\pgfqpoint{0.683333in}{0.337753in}}{\pgfqpoint{0.681577in}{0.341993in}}{\pgfqpoint{0.678452in}{0.345118in}}%
\pgfpathcurveto{\pgfqpoint{0.675326in}{0.348244in}}{\pgfqpoint{0.671087in}{0.350000in}}{\pgfqpoint{0.666667in}{0.350000in}}%
\pgfpathcurveto{\pgfqpoint{0.662247in}{0.350000in}}{\pgfqpoint{0.658007in}{0.348244in}}{\pgfqpoint{0.654882in}{0.345118in}}%
\pgfpathcurveto{\pgfqpoint{0.651756in}{0.341993in}}{\pgfqpoint{0.650000in}{0.337753in}}{\pgfqpoint{0.650000in}{0.333333in}}%
\pgfpathcurveto{\pgfqpoint{0.650000in}{0.328913in}}{\pgfqpoint{0.651756in}{0.324674in}}{\pgfqpoint{0.654882in}{0.321548in}}%
\pgfpathcurveto{\pgfqpoint{0.658007in}{0.318423in}}{\pgfqpoint{0.662247in}{0.316667in}}{\pgfqpoint{0.666667in}{0.316667in}}%
\pgfpathclose%
\pgfpathmoveto{\pgfqpoint{0.833333in}{0.316667in}}%
\pgfpathcurveto{\pgfqpoint{0.837753in}{0.316667in}}{\pgfqpoint{0.841993in}{0.318423in}}{\pgfqpoint{0.845118in}{0.321548in}}%
\pgfpathcurveto{\pgfqpoint{0.848244in}{0.324674in}}{\pgfqpoint{0.850000in}{0.328913in}}{\pgfqpoint{0.850000in}{0.333333in}}%
\pgfpathcurveto{\pgfqpoint{0.850000in}{0.337753in}}{\pgfqpoint{0.848244in}{0.341993in}}{\pgfqpoint{0.845118in}{0.345118in}}%
\pgfpathcurveto{\pgfqpoint{0.841993in}{0.348244in}}{\pgfqpoint{0.837753in}{0.350000in}}{\pgfqpoint{0.833333in}{0.350000in}}%
\pgfpathcurveto{\pgfqpoint{0.828913in}{0.350000in}}{\pgfqpoint{0.824674in}{0.348244in}}{\pgfqpoint{0.821548in}{0.345118in}}%
\pgfpathcurveto{\pgfqpoint{0.818423in}{0.341993in}}{\pgfqpoint{0.816667in}{0.337753in}}{\pgfqpoint{0.816667in}{0.333333in}}%
\pgfpathcurveto{\pgfqpoint{0.816667in}{0.328913in}}{\pgfqpoint{0.818423in}{0.324674in}}{\pgfqpoint{0.821548in}{0.321548in}}%
\pgfpathcurveto{\pgfqpoint{0.824674in}{0.318423in}}{\pgfqpoint{0.828913in}{0.316667in}}{\pgfqpoint{0.833333in}{0.316667in}}%
\pgfpathclose%
\pgfpathmoveto{\pgfqpoint{1.000000in}{0.316667in}}%
\pgfpathcurveto{\pgfqpoint{1.004420in}{0.316667in}}{\pgfqpoint{1.008660in}{0.318423in}}{\pgfqpoint{1.011785in}{0.321548in}}%
\pgfpathcurveto{\pgfqpoint{1.014911in}{0.324674in}}{\pgfqpoint{1.016667in}{0.328913in}}{\pgfqpoint{1.016667in}{0.333333in}}%
\pgfpathcurveto{\pgfqpoint{1.016667in}{0.337753in}}{\pgfqpoint{1.014911in}{0.341993in}}{\pgfqpoint{1.011785in}{0.345118in}}%
\pgfpathcurveto{\pgfqpoint{1.008660in}{0.348244in}}{\pgfqpoint{1.004420in}{0.350000in}}{\pgfqpoint{1.000000in}{0.350000in}}%
\pgfpathcurveto{\pgfqpoint{0.995580in}{0.350000in}}{\pgfqpoint{0.991340in}{0.348244in}}{\pgfqpoint{0.988215in}{0.345118in}}%
\pgfpathcurveto{\pgfqpoint{0.985089in}{0.341993in}}{\pgfqpoint{0.983333in}{0.337753in}}{\pgfqpoint{0.983333in}{0.333333in}}%
\pgfpathcurveto{\pgfqpoint{0.983333in}{0.328913in}}{\pgfqpoint{0.985089in}{0.324674in}}{\pgfqpoint{0.988215in}{0.321548in}}%
\pgfpathcurveto{\pgfqpoint{0.991340in}{0.318423in}}{\pgfqpoint{0.995580in}{0.316667in}}{\pgfqpoint{1.000000in}{0.316667in}}%
\pgfpathclose%
\pgfpathmoveto{\pgfqpoint{0.083333in}{0.483333in}}%
\pgfpathcurveto{\pgfqpoint{0.087753in}{0.483333in}}{\pgfqpoint{0.091993in}{0.485089in}}{\pgfqpoint{0.095118in}{0.488215in}}%
\pgfpathcurveto{\pgfqpoint{0.098244in}{0.491340in}}{\pgfqpoint{0.100000in}{0.495580in}}{\pgfqpoint{0.100000in}{0.500000in}}%
\pgfpathcurveto{\pgfqpoint{0.100000in}{0.504420in}}{\pgfqpoint{0.098244in}{0.508660in}}{\pgfqpoint{0.095118in}{0.511785in}}%
\pgfpathcurveto{\pgfqpoint{0.091993in}{0.514911in}}{\pgfqpoint{0.087753in}{0.516667in}}{\pgfqpoint{0.083333in}{0.516667in}}%
\pgfpathcurveto{\pgfqpoint{0.078913in}{0.516667in}}{\pgfqpoint{0.074674in}{0.514911in}}{\pgfqpoint{0.071548in}{0.511785in}}%
\pgfpathcurveto{\pgfqpoint{0.068423in}{0.508660in}}{\pgfqpoint{0.066667in}{0.504420in}}{\pgfqpoint{0.066667in}{0.500000in}}%
\pgfpathcurveto{\pgfqpoint{0.066667in}{0.495580in}}{\pgfqpoint{0.068423in}{0.491340in}}{\pgfqpoint{0.071548in}{0.488215in}}%
\pgfpathcurveto{\pgfqpoint{0.074674in}{0.485089in}}{\pgfqpoint{0.078913in}{0.483333in}}{\pgfqpoint{0.083333in}{0.483333in}}%
\pgfpathclose%
\pgfpathmoveto{\pgfqpoint{0.250000in}{0.483333in}}%
\pgfpathcurveto{\pgfqpoint{0.254420in}{0.483333in}}{\pgfqpoint{0.258660in}{0.485089in}}{\pgfqpoint{0.261785in}{0.488215in}}%
\pgfpathcurveto{\pgfqpoint{0.264911in}{0.491340in}}{\pgfqpoint{0.266667in}{0.495580in}}{\pgfqpoint{0.266667in}{0.500000in}}%
\pgfpathcurveto{\pgfqpoint{0.266667in}{0.504420in}}{\pgfqpoint{0.264911in}{0.508660in}}{\pgfqpoint{0.261785in}{0.511785in}}%
\pgfpathcurveto{\pgfqpoint{0.258660in}{0.514911in}}{\pgfqpoint{0.254420in}{0.516667in}}{\pgfqpoint{0.250000in}{0.516667in}}%
\pgfpathcurveto{\pgfqpoint{0.245580in}{0.516667in}}{\pgfqpoint{0.241340in}{0.514911in}}{\pgfqpoint{0.238215in}{0.511785in}}%
\pgfpathcurveto{\pgfqpoint{0.235089in}{0.508660in}}{\pgfqpoint{0.233333in}{0.504420in}}{\pgfqpoint{0.233333in}{0.500000in}}%
\pgfpathcurveto{\pgfqpoint{0.233333in}{0.495580in}}{\pgfqpoint{0.235089in}{0.491340in}}{\pgfqpoint{0.238215in}{0.488215in}}%
\pgfpathcurveto{\pgfqpoint{0.241340in}{0.485089in}}{\pgfqpoint{0.245580in}{0.483333in}}{\pgfqpoint{0.250000in}{0.483333in}}%
\pgfpathclose%
\pgfpathmoveto{\pgfqpoint{0.416667in}{0.483333in}}%
\pgfpathcurveto{\pgfqpoint{0.421087in}{0.483333in}}{\pgfqpoint{0.425326in}{0.485089in}}{\pgfqpoint{0.428452in}{0.488215in}}%
\pgfpathcurveto{\pgfqpoint{0.431577in}{0.491340in}}{\pgfqpoint{0.433333in}{0.495580in}}{\pgfqpoint{0.433333in}{0.500000in}}%
\pgfpathcurveto{\pgfqpoint{0.433333in}{0.504420in}}{\pgfqpoint{0.431577in}{0.508660in}}{\pgfqpoint{0.428452in}{0.511785in}}%
\pgfpathcurveto{\pgfqpoint{0.425326in}{0.514911in}}{\pgfqpoint{0.421087in}{0.516667in}}{\pgfqpoint{0.416667in}{0.516667in}}%
\pgfpathcurveto{\pgfqpoint{0.412247in}{0.516667in}}{\pgfqpoint{0.408007in}{0.514911in}}{\pgfqpoint{0.404882in}{0.511785in}}%
\pgfpathcurveto{\pgfqpoint{0.401756in}{0.508660in}}{\pgfqpoint{0.400000in}{0.504420in}}{\pgfqpoint{0.400000in}{0.500000in}}%
\pgfpathcurveto{\pgfqpoint{0.400000in}{0.495580in}}{\pgfqpoint{0.401756in}{0.491340in}}{\pgfqpoint{0.404882in}{0.488215in}}%
\pgfpathcurveto{\pgfqpoint{0.408007in}{0.485089in}}{\pgfqpoint{0.412247in}{0.483333in}}{\pgfqpoint{0.416667in}{0.483333in}}%
\pgfpathclose%
\pgfpathmoveto{\pgfqpoint{0.583333in}{0.483333in}}%
\pgfpathcurveto{\pgfqpoint{0.587753in}{0.483333in}}{\pgfqpoint{0.591993in}{0.485089in}}{\pgfqpoint{0.595118in}{0.488215in}}%
\pgfpathcurveto{\pgfqpoint{0.598244in}{0.491340in}}{\pgfqpoint{0.600000in}{0.495580in}}{\pgfqpoint{0.600000in}{0.500000in}}%
\pgfpathcurveto{\pgfqpoint{0.600000in}{0.504420in}}{\pgfqpoint{0.598244in}{0.508660in}}{\pgfqpoint{0.595118in}{0.511785in}}%
\pgfpathcurveto{\pgfqpoint{0.591993in}{0.514911in}}{\pgfqpoint{0.587753in}{0.516667in}}{\pgfqpoint{0.583333in}{0.516667in}}%
\pgfpathcurveto{\pgfqpoint{0.578913in}{0.516667in}}{\pgfqpoint{0.574674in}{0.514911in}}{\pgfqpoint{0.571548in}{0.511785in}}%
\pgfpathcurveto{\pgfqpoint{0.568423in}{0.508660in}}{\pgfqpoint{0.566667in}{0.504420in}}{\pgfqpoint{0.566667in}{0.500000in}}%
\pgfpathcurveto{\pgfqpoint{0.566667in}{0.495580in}}{\pgfqpoint{0.568423in}{0.491340in}}{\pgfqpoint{0.571548in}{0.488215in}}%
\pgfpathcurveto{\pgfqpoint{0.574674in}{0.485089in}}{\pgfqpoint{0.578913in}{0.483333in}}{\pgfqpoint{0.583333in}{0.483333in}}%
\pgfpathclose%
\pgfpathmoveto{\pgfqpoint{0.750000in}{0.483333in}}%
\pgfpathcurveto{\pgfqpoint{0.754420in}{0.483333in}}{\pgfqpoint{0.758660in}{0.485089in}}{\pgfqpoint{0.761785in}{0.488215in}}%
\pgfpathcurveto{\pgfqpoint{0.764911in}{0.491340in}}{\pgfqpoint{0.766667in}{0.495580in}}{\pgfqpoint{0.766667in}{0.500000in}}%
\pgfpathcurveto{\pgfqpoint{0.766667in}{0.504420in}}{\pgfqpoint{0.764911in}{0.508660in}}{\pgfqpoint{0.761785in}{0.511785in}}%
\pgfpathcurveto{\pgfqpoint{0.758660in}{0.514911in}}{\pgfqpoint{0.754420in}{0.516667in}}{\pgfqpoint{0.750000in}{0.516667in}}%
\pgfpathcurveto{\pgfqpoint{0.745580in}{0.516667in}}{\pgfqpoint{0.741340in}{0.514911in}}{\pgfqpoint{0.738215in}{0.511785in}}%
\pgfpathcurveto{\pgfqpoint{0.735089in}{0.508660in}}{\pgfqpoint{0.733333in}{0.504420in}}{\pgfqpoint{0.733333in}{0.500000in}}%
\pgfpathcurveto{\pgfqpoint{0.733333in}{0.495580in}}{\pgfqpoint{0.735089in}{0.491340in}}{\pgfqpoint{0.738215in}{0.488215in}}%
\pgfpathcurveto{\pgfqpoint{0.741340in}{0.485089in}}{\pgfqpoint{0.745580in}{0.483333in}}{\pgfqpoint{0.750000in}{0.483333in}}%
\pgfpathclose%
\pgfpathmoveto{\pgfqpoint{0.916667in}{0.483333in}}%
\pgfpathcurveto{\pgfqpoint{0.921087in}{0.483333in}}{\pgfqpoint{0.925326in}{0.485089in}}{\pgfqpoint{0.928452in}{0.488215in}}%
\pgfpathcurveto{\pgfqpoint{0.931577in}{0.491340in}}{\pgfqpoint{0.933333in}{0.495580in}}{\pgfqpoint{0.933333in}{0.500000in}}%
\pgfpathcurveto{\pgfqpoint{0.933333in}{0.504420in}}{\pgfqpoint{0.931577in}{0.508660in}}{\pgfqpoint{0.928452in}{0.511785in}}%
\pgfpathcurveto{\pgfqpoint{0.925326in}{0.514911in}}{\pgfqpoint{0.921087in}{0.516667in}}{\pgfqpoint{0.916667in}{0.516667in}}%
\pgfpathcurveto{\pgfqpoint{0.912247in}{0.516667in}}{\pgfqpoint{0.908007in}{0.514911in}}{\pgfqpoint{0.904882in}{0.511785in}}%
\pgfpathcurveto{\pgfqpoint{0.901756in}{0.508660in}}{\pgfqpoint{0.900000in}{0.504420in}}{\pgfqpoint{0.900000in}{0.500000in}}%
\pgfpathcurveto{\pgfqpoint{0.900000in}{0.495580in}}{\pgfqpoint{0.901756in}{0.491340in}}{\pgfqpoint{0.904882in}{0.488215in}}%
\pgfpathcurveto{\pgfqpoint{0.908007in}{0.485089in}}{\pgfqpoint{0.912247in}{0.483333in}}{\pgfqpoint{0.916667in}{0.483333in}}%
\pgfpathclose%
\pgfpathmoveto{\pgfqpoint{0.000000in}{0.650000in}}%
\pgfpathcurveto{\pgfqpoint{0.004420in}{0.650000in}}{\pgfqpoint{0.008660in}{0.651756in}}{\pgfqpoint{0.011785in}{0.654882in}}%
\pgfpathcurveto{\pgfqpoint{0.014911in}{0.658007in}}{\pgfqpoint{0.016667in}{0.662247in}}{\pgfqpoint{0.016667in}{0.666667in}}%
\pgfpathcurveto{\pgfqpoint{0.016667in}{0.671087in}}{\pgfqpoint{0.014911in}{0.675326in}}{\pgfqpoint{0.011785in}{0.678452in}}%
\pgfpathcurveto{\pgfqpoint{0.008660in}{0.681577in}}{\pgfqpoint{0.004420in}{0.683333in}}{\pgfqpoint{0.000000in}{0.683333in}}%
\pgfpathcurveto{\pgfqpoint{-0.004420in}{0.683333in}}{\pgfqpoint{-0.008660in}{0.681577in}}{\pgfqpoint{-0.011785in}{0.678452in}}%
\pgfpathcurveto{\pgfqpoint{-0.014911in}{0.675326in}}{\pgfqpoint{-0.016667in}{0.671087in}}{\pgfqpoint{-0.016667in}{0.666667in}}%
\pgfpathcurveto{\pgfqpoint{-0.016667in}{0.662247in}}{\pgfqpoint{-0.014911in}{0.658007in}}{\pgfqpoint{-0.011785in}{0.654882in}}%
\pgfpathcurveto{\pgfqpoint{-0.008660in}{0.651756in}}{\pgfqpoint{-0.004420in}{0.650000in}}{\pgfqpoint{0.000000in}{0.650000in}}%
\pgfpathclose%
\pgfpathmoveto{\pgfqpoint{0.166667in}{0.650000in}}%
\pgfpathcurveto{\pgfqpoint{0.171087in}{0.650000in}}{\pgfqpoint{0.175326in}{0.651756in}}{\pgfqpoint{0.178452in}{0.654882in}}%
\pgfpathcurveto{\pgfqpoint{0.181577in}{0.658007in}}{\pgfqpoint{0.183333in}{0.662247in}}{\pgfqpoint{0.183333in}{0.666667in}}%
\pgfpathcurveto{\pgfqpoint{0.183333in}{0.671087in}}{\pgfqpoint{0.181577in}{0.675326in}}{\pgfqpoint{0.178452in}{0.678452in}}%
\pgfpathcurveto{\pgfqpoint{0.175326in}{0.681577in}}{\pgfqpoint{0.171087in}{0.683333in}}{\pgfqpoint{0.166667in}{0.683333in}}%
\pgfpathcurveto{\pgfqpoint{0.162247in}{0.683333in}}{\pgfqpoint{0.158007in}{0.681577in}}{\pgfqpoint{0.154882in}{0.678452in}}%
\pgfpathcurveto{\pgfqpoint{0.151756in}{0.675326in}}{\pgfqpoint{0.150000in}{0.671087in}}{\pgfqpoint{0.150000in}{0.666667in}}%
\pgfpathcurveto{\pgfqpoint{0.150000in}{0.662247in}}{\pgfqpoint{0.151756in}{0.658007in}}{\pgfqpoint{0.154882in}{0.654882in}}%
\pgfpathcurveto{\pgfqpoint{0.158007in}{0.651756in}}{\pgfqpoint{0.162247in}{0.650000in}}{\pgfqpoint{0.166667in}{0.650000in}}%
\pgfpathclose%
\pgfpathmoveto{\pgfqpoint{0.333333in}{0.650000in}}%
\pgfpathcurveto{\pgfqpoint{0.337753in}{0.650000in}}{\pgfqpoint{0.341993in}{0.651756in}}{\pgfqpoint{0.345118in}{0.654882in}}%
\pgfpathcurveto{\pgfqpoint{0.348244in}{0.658007in}}{\pgfqpoint{0.350000in}{0.662247in}}{\pgfqpoint{0.350000in}{0.666667in}}%
\pgfpathcurveto{\pgfqpoint{0.350000in}{0.671087in}}{\pgfqpoint{0.348244in}{0.675326in}}{\pgfqpoint{0.345118in}{0.678452in}}%
\pgfpathcurveto{\pgfqpoint{0.341993in}{0.681577in}}{\pgfqpoint{0.337753in}{0.683333in}}{\pgfqpoint{0.333333in}{0.683333in}}%
\pgfpathcurveto{\pgfqpoint{0.328913in}{0.683333in}}{\pgfqpoint{0.324674in}{0.681577in}}{\pgfqpoint{0.321548in}{0.678452in}}%
\pgfpathcurveto{\pgfqpoint{0.318423in}{0.675326in}}{\pgfqpoint{0.316667in}{0.671087in}}{\pgfqpoint{0.316667in}{0.666667in}}%
\pgfpathcurveto{\pgfqpoint{0.316667in}{0.662247in}}{\pgfqpoint{0.318423in}{0.658007in}}{\pgfqpoint{0.321548in}{0.654882in}}%
\pgfpathcurveto{\pgfqpoint{0.324674in}{0.651756in}}{\pgfqpoint{0.328913in}{0.650000in}}{\pgfqpoint{0.333333in}{0.650000in}}%
\pgfpathclose%
\pgfpathmoveto{\pgfqpoint{0.500000in}{0.650000in}}%
\pgfpathcurveto{\pgfqpoint{0.504420in}{0.650000in}}{\pgfqpoint{0.508660in}{0.651756in}}{\pgfqpoint{0.511785in}{0.654882in}}%
\pgfpathcurveto{\pgfqpoint{0.514911in}{0.658007in}}{\pgfqpoint{0.516667in}{0.662247in}}{\pgfqpoint{0.516667in}{0.666667in}}%
\pgfpathcurveto{\pgfqpoint{0.516667in}{0.671087in}}{\pgfqpoint{0.514911in}{0.675326in}}{\pgfqpoint{0.511785in}{0.678452in}}%
\pgfpathcurveto{\pgfqpoint{0.508660in}{0.681577in}}{\pgfqpoint{0.504420in}{0.683333in}}{\pgfqpoint{0.500000in}{0.683333in}}%
\pgfpathcurveto{\pgfqpoint{0.495580in}{0.683333in}}{\pgfqpoint{0.491340in}{0.681577in}}{\pgfqpoint{0.488215in}{0.678452in}}%
\pgfpathcurveto{\pgfqpoint{0.485089in}{0.675326in}}{\pgfqpoint{0.483333in}{0.671087in}}{\pgfqpoint{0.483333in}{0.666667in}}%
\pgfpathcurveto{\pgfqpoint{0.483333in}{0.662247in}}{\pgfqpoint{0.485089in}{0.658007in}}{\pgfqpoint{0.488215in}{0.654882in}}%
\pgfpathcurveto{\pgfqpoint{0.491340in}{0.651756in}}{\pgfqpoint{0.495580in}{0.650000in}}{\pgfqpoint{0.500000in}{0.650000in}}%
\pgfpathclose%
\pgfpathmoveto{\pgfqpoint{0.666667in}{0.650000in}}%
\pgfpathcurveto{\pgfqpoint{0.671087in}{0.650000in}}{\pgfqpoint{0.675326in}{0.651756in}}{\pgfqpoint{0.678452in}{0.654882in}}%
\pgfpathcurveto{\pgfqpoint{0.681577in}{0.658007in}}{\pgfqpoint{0.683333in}{0.662247in}}{\pgfqpoint{0.683333in}{0.666667in}}%
\pgfpathcurveto{\pgfqpoint{0.683333in}{0.671087in}}{\pgfqpoint{0.681577in}{0.675326in}}{\pgfqpoint{0.678452in}{0.678452in}}%
\pgfpathcurveto{\pgfqpoint{0.675326in}{0.681577in}}{\pgfqpoint{0.671087in}{0.683333in}}{\pgfqpoint{0.666667in}{0.683333in}}%
\pgfpathcurveto{\pgfqpoint{0.662247in}{0.683333in}}{\pgfqpoint{0.658007in}{0.681577in}}{\pgfqpoint{0.654882in}{0.678452in}}%
\pgfpathcurveto{\pgfqpoint{0.651756in}{0.675326in}}{\pgfqpoint{0.650000in}{0.671087in}}{\pgfqpoint{0.650000in}{0.666667in}}%
\pgfpathcurveto{\pgfqpoint{0.650000in}{0.662247in}}{\pgfqpoint{0.651756in}{0.658007in}}{\pgfqpoint{0.654882in}{0.654882in}}%
\pgfpathcurveto{\pgfqpoint{0.658007in}{0.651756in}}{\pgfqpoint{0.662247in}{0.650000in}}{\pgfqpoint{0.666667in}{0.650000in}}%
\pgfpathclose%
\pgfpathmoveto{\pgfqpoint{0.833333in}{0.650000in}}%
\pgfpathcurveto{\pgfqpoint{0.837753in}{0.650000in}}{\pgfqpoint{0.841993in}{0.651756in}}{\pgfqpoint{0.845118in}{0.654882in}}%
\pgfpathcurveto{\pgfqpoint{0.848244in}{0.658007in}}{\pgfqpoint{0.850000in}{0.662247in}}{\pgfqpoint{0.850000in}{0.666667in}}%
\pgfpathcurveto{\pgfqpoint{0.850000in}{0.671087in}}{\pgfqpoint{0.848244in}{0.675326in}}{\pgfqpoint{0.845118in}{0.678452in}}%
\pgfpathcurveto{\pgfqpoint{0.841993in}{0.681577in}}{\pgfqpoint{0.837753in}{0.683333in}}{\pgfqpoint{0.833333in}{0.683333in}}%
\pgfpathcurveto{\pgfqpoint{0.828913in}{0.683333in}}{\pgfqpoint{0.824674in}{0.681577in}}{\pgfqpoint{0.821548in}{0.678452in}}%
\pgfpathcurveto{\pgfqpoint{0.818423in}{0.675326in}}{\pgfqpoint{0.816667in}{0.671087in}}{\pgfqpoint{0.816667in}{0.666667in}}%
\pgfpathcurveto{\pgfqpoint{0.816667in}{0.662247in}}{\pgfqpoint{0.818423in}{0.658007in}}{\pgfqpoint{0.821548in}{0.654882in}}%
\pgfpathcurveto{\pgfqpoint{0.824674in}{0.651756in}}{\pgfqpoint{0.828913in}{0.650000in}}{\pgfqpoint{0.833333in}{0.650000in}}%
\pgfpathclose%
\pgfpathmoveto{\pgfqpoint{1.000000in}{0.650000in}}%
\pgfpathcurveto{\pgfqpoint{1.004420in}{0.650000in}}{\pgfqpoint{1.008660in}{0.651756in}}{\pgfqpoint{1.011785in}{0.654882in}}%
\pgfpathcurveto{\pgfqpoint{1.014911in}{0.658007in}}{\pgfqpoint{1.016667in}{0.662247in}}{\pgfqpoint{1.016667in}{0.666667in}}%
\pgfpathcurveto{\pgfqpoint{1.016667in}{0.671087in}}{\pgfqpoint{1.014911in}{0.675326in}}{\pgfqpoint{1.011785in}{0.678452in}}%
\pgfpathcurveto{\pgfqpoint{1.008660in}{0.681577in}}{\pgfqpoint{1.004420in}{0.683333in}}{\pgfqpoint{1.000000in}{0.683333in}}%
\pgfpathcurveto{\pgfqpoint{0.995580in}{0.683333in}}{\pgfqpoint{0.991340in}{0.681577in}}{\pgfqpoint{0.988215in}{0.678452in}}%
\pgfpathcurveto{\pgfqpoint{0.985089in}{0.675326in}}{\pgfqpoint{0.983333in}{0.671087in}}{\pgfqpoint{0.983333in}{0.666667in}}%
\pgfpathcurveto{\pgfqpoint{0.983333in}{0.662247in}}{\pgfqpoint{0.985089in}{0.658007in}}{\pgfqpoint{0.988215in}{0.654882in}}%
\pgfpathcurveto{\pgfqpoint{0.991340in}{0.651756in}}{\pgfqpoint{0.995580in}{0.650000in}}{\pgfqpoint{1.000000in}{0.650000in}}%
\pgfpathclose%
\pgfpathmoveto{\pgfqpoint{0.083333in}{0.816667in}}%
\pgfpathcurveto{\pgfqpoint{0.087753in}{0.816667in}}{\pgfqpoint{0.091993in}{0.818423in}}{\pgfqpoint{0.095118in}{0.821548in}}%
\pgfpathcurveto{\pgfqpoint{0.098244in}{0.824674in}}{\pgfqpoint{0.100000in}{0.828913in}}{\pgfqpoint{0.100000in}{0.833333in}}%
\pgfpathcurveto{\pgfqpoint{0.100000in}{0.837753in}}{\pgfqpoint{0.098244in}{0.841993in}}{\pgfqpoint{0.095118in}{0.845118in}}%
\pgfpathcurveto{\pgfqpoint{0.091993in}{0.848244in}}{\pgfqpoint{0.087753in}{0.850000in}}{\pgfqpoint{0.083333in}{0.850000in}}%
\pgfpathcurveto{\pgfqpoint{0.078913in}{0.850000in}}{\pgfqpoint{0.074674in}{0.848244in}}{\pgfqpoint{0.071548in}{0.845118in}}%
\pgfpathcurveto{\pgfqpoint{0.068423in}{0.841993in}}{\pgfqpoint{0.066667in}{0.837753in}}{\pgfqpoint{0.066667in}{0.833333in}}%
\pgfpathcurveto{\pgfqpoint{0.066667in}{0.828913in}}{\pgfqpoint{0.068423in}{0.824674in}}{\pgfqpoint{0.071548in}{0.821548in}}%
\pgfpathcurveto{\pgfqpoint{0.074674in}{0.818423in}}{\pgfqpoint{0.078913in}{0.816667in}}{\pgfqpoint{0.083333in}{0.816667in}}%
\pgfpathclose%
\pgfpathmoveto{\pgfqpoint{0.250000in}{0.816667in}}%
\pgfpathcurveto{\pgfqpoint{0.254420in}{0.816667in}}{\pgfqpoint{0.258660in}{0.818423in}}{\pgfqpoint{0.261785in}{0.821548in}}%
\pgfpathcurveto{\pgfqpoint{0.264911in}{0.824674in}}{\pgfqpoint{0.266667in}{0.828913in}}{\pgfqpoint{0.266667in}{0.833333in}}%
\pgfpathcurveto{\pgfqpoint{0.266667in}{0.837753in}}{\pgfqpoint{0.264911in}{0.841993in}}{\pgfqpoint{0.261785in}{0.845118in}}%
\pgfpathcurveto{\pgfqpoint{0.258660in}{0.848244in}}{\pgfqpoint{0.254420in}{0.850000in}}{\pgfqpoint{0.250000in}{0.850000in}}%
\pgfpathcurveto{\pgfqpoint{0.245580in}{0.850000in}}{\pgfqpoint{0.241340in}{0.848244in}}{\pgfqpoint{0.238215in}{0.845118in}}%
\pgfpathcurveto{\pgfqpoint{0.235089in}{0.841993in}}{\pgfqpoint{0.233333in}{0.837753in}}{\pgfqpoint{0.233333in}{0.833333in}}%
\pgfpathcurveto{\pgfqpoint{0.233333in}{0.828913in}}{\pgfqpoint{0.235089in}{0.824674in}}{\pgfqpoint{0.238215in}{0.821548in}}%
\pgfpathcurveto{\pgfqpoint{0.241340in}{0.818423in}}{\pgfqpoint{0.245580in}{0.816667in}}{\pgfqpoint{0.250000in}{0.816667in}}%
\pgfpathclose%
\pgfpathmoveto{\pgfqpoint{0.416667in}{0.816667in}}%
\pgfpathcurveto{\pgfqpoint{0.421087in}{0.816667in}}{\pgfqpoint{0.425326in}{0.818423in}}{\pgfqpoint{0.428452in}{0.821548in}}%
\pgfpathcurveto{\pgfqpoint{0.431577in}{0.824674in}}{\pgfqpoint{0.433333in}{0.828913in}}{\pgfqpoint{0.433333in}{0.833333in}}%
\pgfpathcurveto{\pgfqpoint{0.433333in}{0.837753in}}{\pgfqpoint{0.431577in}{0.841993in}}{\pgfqpoint{0.428452in}{0.845118in}}%
\pgfpathcurveto{\pgfqpoint{0.425326in}{0.848244in}}{\pgfqpoint{0.421087in}{0.850000in}}{\pgfqpoint{0.416667in}{0.850000in}}%
\pgfpathcurveto{\pgfqpoint{0.412247in}{0.850000in}}{\pgfqpoint{0.408007in}{0.848244in}}{\pgfqpoint{0.404882in}{0.845118in}}%
\pgfpathcurveto{\pgfqpoint{0.401756in}{0.841993in}}{\pgfqpoint{0.400000in}{0.837753in}}{\pgfqpoint{0.400000in}{0.833333in}}%
\pgfpathcurveto{\pgfqpoint{0.400000in}{0.828913in}}{\pgfqpoint{0.401756in}{0.824674in}}{\pgfqpoint{0.404882in}{0.821548in}}%
\pgfpathcurveto{\pgfqpoint{0.408007in}{0.818423in}}{\pgfqpoint{0.412247in}{0.816667in}}{\pgfqpoint{0.416667in}{0.816667in}}%
\pgfpathclose%
\pgfpathmoveto{\pgfqpoint{0.583333in}{0.816667in}}%
\pgfpathcurveto{\pgfqpoint{0.587753in}{0.816667in}}{\pgfqpoint{0.591993in}{0.818423in}}{\pgfqpoint{0.595118in}{0.821548in}}%
\pgfpathcurveto{\pgfqpoint{0.598244in}{0.824674in}}{\pgfqpoint{0.600000in}{0.828913in}}{\pgfqpoint{0.600000in}{0.833333in}}%
\pgfpathcurveto{\pgfqpoint{0.600000in}{0.837753in}}{\pgfqpoint{0.598244in}{0.841993in}}{\pgfqpoint{0.595118in}{0.845118in}}%
\pgfpathcurveto{\pgfqpoint{0.591993in}{0.848244in}}{\pgfqpoint{0.587753in}{0.850000in}}{\pgfqpoint{0.583333in}{0.850000in}}%
\pgfpathcurveto{\pgfqpoint{0.578913in}{0.850000in}}{\pgfqpoint{0.574674in}{0.848244in}}{\pgfqpoint{0.571548in}{0.845118in}}%
\pgfpathcurveto{\pgfqpoint{0.568423in}{0.841993in}}{\pgfqpoint{0.566667in}{0.837753in}}{\pgfqpoint{0.566667in}{0.833333in}}%
\pgfpathcurveto{\pgfqpoint{0.566667in}{0.828913in}}{\pgfqpoint{0.568423in}{0.824674in}}{\pgfqpoint{0.571548in}{0.821548in}}%
\pgfpathcurveto{\pgfqpoint{0.574674in}{0.818423in}}{\pgfqpoint{0.578913in}{0.816667in}}{\pgfqpoint{0.583333in}{0.816667in}}%
\pgfpathclose%
\pgfpathmoveto{\pgfqpoint{0.750000in}{0.816667in}}%
\pgfpathcurveto{\pgfqpoint{0.754420in}{0.816667in}}{\pgfqpoint{0.758660in}{0.818423in}}{\pgfqpoint{0.761785in}{0.821548in}}%
\pgfpathcurveto{\pgfqpoint{0.764911in}{0.824674in}}{\pgfqpoint{0.766667in}{0.828913in}}{\pgfqpoint{0.766667in}{0.833333in}}%
\pgfpathcurveto{\pgfqpoint{0.766667in}{0.837753in}}{\pgfqpoint{0.764911in}{0.841993in}}{\pgfqpoint{0.761785in}{0.845118in}}%
\pgfpathcurveto{\pgfqpoint{0.758660in}{0.848244in}}{\pgfqpoint{0.754420in}{0.850000in}}{\pgfqpoint{0.750000in}{0.850000in}}%
\pgfpathcurveto{\pgfqpoint{0.745580in}{0.850000in}}{\pgfqpoint{0.741340in}{0.848244in}}{\pgfqpoint{0.738215in}{0.845118in}}%
\pgfpathcurveto{\pgfqpoint{0.735089in}{0.841993in}}{\pgfqpoint{0.733333in}{0.837753in}}{\pgfqpoint{0.733333in}{0.833333in}}%
\pgfpathcurveto{\pgfqpoint{0.733333in}{0.828913in}}{\pgfqpoint{0.735089in}{0.824674in}}{\pgfqpoint{0.738215in}{0.821548in}}%
\pgfpathcurveto{\pgfqpoint{0.741340in}{0.818423in}}{\pgfqpoint{0.745580in}{0.816667in}}{\pgfqpoint{0.750000in}{0.816667in}}%
\pgfpathclose%
\pgfpathmoveto{\pgfqpoint{0.916667in}{0.816667in}}%
\pgfpathcurveto{\pgfqpoint{0.921087in}{0.816667in}}{\pgfqpoint{0.925326in}{0.818423in}}{\pgfqpoint{0.928452in}{0.821548in}}%
\pgfpathcurveto{\pgfqpoint{0.931577in}{0.824674in}}{\pgfqpoint{0.933333in}{0.828913in}}{\pgfqpoint{0.933333in}{0.833333in}}%
\pgfpathcurveto{\pgfqpoint{0.933333in}{0.837753in}}{\pgfqpoint{0.931577in}{0.841993in}}{\pgfqpoint{0.928452in}{0.845118in}}%
\pgfpathcurveto{\pgfqpoint{0.925326in}{0.848244in}}{\pgfqpoint{0.921087in}{0.850000in}}{\pgfqpoint{0.916667in}{0.850000in}}%
\pgfpathcurveto{\pgfqpoint{0.912247in}{0.850000in}}{\pgfqpoint{0.908007in}{0.848244in}}{\pgfqpoint{0.904882in}{0.845118in}}%
\pgfpathcurveto{\pgfqpoint{0.901756in}{0.841993in}}{\pgfqpoint{0.900000in}{0.837753in}}{\pgfqpoint{0.900000in}{0.833333in}}%
\pgfpathcurveto{\pgfqpoint{0.900000in}{0.828913in}}{\pgfqpoint{0.901756in}{0.824674in}}{\pgfqpoint{0.904882in}{0.821548in}}%
\pgfpathcurveto{\pgfqpoint{0.908007in}{0.818423in}}{\pgfqpoint{0.912247in}{0.816667in}}{\pgfqpoint{0.916667in}{0.816667in}}%
\pgfpathclose%
\pgfpathmoveto{\pgfqpoint{0.000000in}{0.983333in}}%
\pgfpathcurveto{\pgfqpoint{0.004420in}{0.983333in}}{\pgfqpoint{0.008660in}{0.985089in}}{\pgfqpoint{0.011785in}{0.988215in}}%
\pgfpathcurveto{\pgfqpoint{0.014911in}{0.991340in}}{\pgfqpoint{0.016667in}{0.995580in}}{\pgfqpoint{0.016667in}{1.000000in}}%
\pgfpathcurveto{\pgfqpoint{0.016667in}{1.004420in}}{\pgfqpoint{0.014911in}{1.008660in}}{\pgfqpoint{0.011785in}{1.011785in}}%
\pgfpathcurveto{\pgfqpoint{0.008660in}{1.014911in}}{\pgfqpoint{0.004420in}{1.016667in}}{\pgfqpoint{0.000000in}{1.016667in}}%
\pgfpathcurveto{\pgfqpoint{-0.004420in}{1.016667in}}{\pgfqpoint{-0.008660in}{1.014911in}}{\pgfqpoint{-0.011785in}{1.011785in}}%
\pgfpathcurveto{\pgfqpoint{-0.014911in}{1.008660in}}{\pgfqpoint{-0.016667in}{1.004420in}}{\pgfqpoint{-0.016667in}{1.000000in}}%
\pgfpathcurveto{\pgfqpoint{-0.016667in}{0.995580in}}{\pgfqpoint{-0.014911in}{0.991340in}}{\pgfqpoint{-0.011785in}{0.988215in}}%
\pgfpathcurveto{\pgfqpoint{-0.008660in}{0.985089in}}{\pgfqpoint{-0.004420in}{0.983333in}}{\pgfqpoint{0.000000in}{0.983333in}}%
\pgfpathclose%
\pgfpathmoveto{\pgfqpoint{0.166667in}{0.983333in}}%
\pgfpathcurveto{\pgfqpoint{0.171087in}{0.983333in}}{\pgfqpoint{0.175326in}{0.985089in}}{\pgfqpoint{0.178452in}{0.988215in}}%
\pgfpathcurveto{\pgfqpoint{0.181577in}{0.991340in}}{\pgfqpoint{0.183333in}{0.995580in}}{\pgfqpoint{0.183333in}{1.000000in}}%
\pgfpathcurveto{\pgfqpoint{0.183333in}{1.004420in}}{\pgfqpoint{0.181577in}{1.008660in}}{\pgfqpoint{0.178452in}{1.011785in}}%
\pgfpathcurveto{\pgfqpoint{0.175326in}{1.014911in}}{\pgfqpoint{0.171087in}{1.016667in}}{\pgfqpoint{0.166667in}{1.016667in}}%
\pgfpathcurveto{\pgfqpoint{0.162247in}{1.016667in}}{\pgfqpoint{0.158007in}{1.014911in}}{\pgfqpoint{0.154882in}{1.011785in}}%
\pgfpathcurveto{\pgfqpoint{0.151756in}{1.008660in}}{\pgfqpoint{0.150000in}{1.004420in}}{\pgfqpoint{0.150000in}{1.000000in}}%
\pgfpathcurveto{\pgfqpoint{0.150000in}{0.995580in}}{\pgfqpoint{0.151756in}{0.991340in}}{\pgfqpoint{0.154882in}{0.988215in}}%
\pgfpathcurveto{\pgfqpoint{0.158007in}{0.985089in}}{\pgfqpoint{0.162247in}{0.983333in}}{\pgfqpoint{0.166667in}{0.983333in}}%
\pgfpathclose%
\pgfpathmoveto{\pgfqpoint{0.333333in}{0.983333in}}%
\pgfpathcurveto{\pgfqpoint{0.337753in}{0.983333in}}{\pgfqpoint{0.341993in}{0.985089in}}{\pgfqpoint{0.345118in}{0.988215in}}%
\pgfpathcurveto{\pgfqpoint{0.348244in}{0.991340in}}{\pgfqpoint{0.350000in}{0.995580in}}{\pgfqpoint{0.350000in}{1.000000in}}%
\pgfpathcurveto{\pgfqpoint{0.350000in}{1.004420in}}{\pgfqpoint{0.348244in}{1.008660in}}{\pgfqpoint{0.345118in}{1.011785in}}%
\pgfpathcurveto{\pgfqpoint{0.341993in}{1.014911in}}{\pgfqpoint{0.337753in}{1.016667in}}{\pgfqpoint{0.333333in}{1.016667in}}%
\pgfpathcurveto{\pgfqpoint{0.328913in}{1.016667in}}{\pgfqpoint{0.324674in}{1.014911in}}{\pgfqpoint{0.321548in}{1.011785in}}%
\pgfpathcurveto{\pgfqpoint{0.318423in}{1.008660in}}{\pgfqpoint{0.316667in}{1.004420in}}{\pgfqpoint{0.316667in}{1.000000in}}%
\pgfpathcurveto{\pgfqpoint{0.316667in}{0.995580in}}{\pgfqpoint{0.318423in}{0.991340in}}{\pgfqpoint{0.321548in}{0.988215in}}%
\pgfpathcurveto{\pgfqpoint{0.324674in}{0.985089in}}{\pgfqpoint{0.328913in}{0.983333in}}{\pgfqpoint{0.333333in}{0.983333in}}%
\pgfpathclose%
\pgfpathmoveto{\pgfqpoint{0.500000in}{0.983333in}}%
\pgfpathcurveto{\pgfqpoint{0.504420in}{0.983333in}}{\pgfqpoint{0.508660in}{0.985089in}}{\pgfqpoint{0.511785in}{0.988215in}}%
\pgfpathcurveto{\pgfqpoint{0.514911in}{0.991340in}}{\pgfqpoint{0.516667in}{0.995580in}}{\pgfqpoint{0.516667in}{1.000000in}}%
\pgfpathcurveto{\pgfqpoint{0.516667in}{1.004420in}}{\pgfqpoint{0.514911in}{1.008660in}}{\pgfqpoint{0.511785in}{1.011785in}}%
\pgfpathcurveto{\pgfqpoint{0.508660in}{1.014911in}}{\pgfqpoint{0.504420in}{1.016667in}}{\pgfqpoint{0.500000in}{1.016667in}}%
\pgfpathcurveto{\pgfqpoint{0.495580in}{1.016667in}}{\pgfqpoint{0.491340in}{1.014911in}}{\pgfqpoint{0.488215in}{1.011785in}}%
\pgfpathcurveto{\pgfqpoint{0.485089in}{1.008660in}}{\pgfqpoint{0.483333in}{1.004420in}}{\pgfqpoint{0.483333in}{1.000000in}}%
\pgfpathcurveto{\pgfqpoint{0.483333in}{0.995580in}}{\pgfqpoint{0.485089in}{0.991340in}}{\pgfqpoint{0.488215in}{0.988215in}}%
\pgfpathcurveto{\pgfqpoint{0.491340in}{0.985089in}}{\pgfqpoint{0.495580in}{0.983333in}}{\pgfqpoint{0.500000in}{0.983333in}}%
\pgfpathclose%
\pgfpathmoveto{\pgfqpoint{0.666667in}{0.983333in}}%
\pgfpathcurveto{\pgfqpoint{0.671087in}{0.983333in}}{\pgfqpoint{0.675326in}{0.985089in}}{\pgfqpoint{0.678452in}{0.988215in}}%
\pgfpathcurveto{\pgfqpoint{0.681577in}{0.991340in}}{\pgfqpoint{0.683333in}{0.995580in}}{\pgfqpoint{0.683333in}{1.000000in}}%
\pgfpathcurveto{\pgfqpoint{0.683333in}{1.004420in}}{\pgfqpoint{0.681577in}{1.008660in}}{\pgfqpoint{0.678452in}{1.011785in}}%
\pgfpathcurveto{\pgfqpoint{0.675326in}{1.014911in}}{\pgfqpoint{0.671087in}{1.016667in}}{\pgfqpoint{0.666667in}{1.016667in}}%
\pgfpathcurveto{\pgfqpoint{0.662247in}{1.016667in}}{\pgfqpoint{0.658007in}{1.014911in}}{\pgfqpoint{0.654882in}{1.011785in}}%
\pgfpathcurveto{\pgfqpoint{0.651756in}{1.008660in}}{\pgfqpoint{0.650000in}{1.004420in}}{\pgfqpoint{0.650000in}{1.000000in}}%
\pgfpathcurveto{\pgfqpoint{0.650000in}{0.995580in}}{\pgfqpoint{0.651756in}{0.991340in}}{\pgfqpoint{0.654882in}{0.988215in}}%
\pgfpathcurveto{\pgfqpoint{0.658007in}{0.985089in}}{\pgfqpoint{0.662247in}{0.983333in}}{\pgfqpoint{0.666667in}{0.983333in}}%
\pgfpathclose%
\pgfpathmoveto{\pgfqpoint{0.833333in}{0.983333in}}%
\pgfpathcurveto{\pgfqpoint{0.837753in}{0.983333in}}{\pgfqpoint{0.841993in}{0.985089in}}{\pgfqpoint{0.845118in}{0.988215in}}%
\pgfpathcurveto{\pgfqpoint{0.848244in}{0.991340in}}{\pgfqpoint{0.850000in}{0.995580in}}{\pgfqpoint{0.850000in}{1.000000in}}%
\pgfpathcurveto{\pgfqpoint{0.850000in}{1.004420in}}{\pgfqpoint{0.848244in}{1.008660in}}{\pgfqpoint{0.845118in}{1.011785in}}%
\pgfpathcurveto{\pgfqpoint{0.841993in}{1.014911in}}{\pgfqpoint{0.837753in}{1.016667in}}{\pgfqpoint{0.833333in}{1.016667in}}%
\pgfpathcurveto{\pgfqpoint{0.828913in}{1.016667in}}{\pgfqpoint{0.824674in}{1.014911in}}{\pgfqpoint{0.821548in}{1.011785in}}%
\pgfpathcurveto{\pgfqpoint{0.818423in}{1.008660in}}{\pgfqpoint{0.816667in}{1.004420in}}{\pgfqpoint{0.816667in}{1.000000in}}%
\pgfpathcurveto{\pgfqpoint{0.816667in}{0.995580in}}{\pgfqpoint{0.818423in}{0.991340in}}{\pgfqpoint{0.821548in}{0.988215in}}%
\pgfpathcurveto{\pgfqpoint{0.824674in}{0.985089in}}{\pgfqpoint{0.828913in}{0.983333in}}{\pgfqpoint{0.833333in}{0.983333in}}%
\pgfpathclose%
\pgfpathmoveto{\pgfqpoint{1.000000in}{0.983333in}}%
\pgfpathcurveto{\pgfqpoint{1.004420in}{0.983333in}}{\pgfqpoint{1.008660in}{0.985089in}}{\pgfqpoint{1.011785in}{0.988215in}}%
\pgfpathcurveto{\pgfqpoint{1.014911in}{0.991340in}}{\pgfqpoint{1.016667in}{0.995580in}}{\pgfqpoint{1.016667in}{1.000000in}}%
\pgfpathcurveto{\pgfqpoint{1.016667in}{1.004420in}}{\pgfqpoint{1.014911in}{1.008660in}}{\pgfqpoint{1.011785in}{1.011785in}}%
\pgfpathcurveto{\pgfqpoint{1.008660in}{1.014911in}}{\pgfqpoint{1.004420in}{1.016667in}}{\pgfqpoint{1.000000in}{1.016667in}}%
\pgfpathcurveto{\pgfqpoint{0.995580in}{1.016667in}}{\pgfqpoint{0.991340in}{1.014911in}}{\pgfqpoint{0.988215in}{1.011785in}}%
\pgfpathcurveto{\pgfqpoint{0.985089in}{1.008660in}}{\pgfqpoint{0.983333in}{1.004420in}}{\pgfqpoint{0.983333in}{1.000000in}}%
\pgfpathcurveto{\pgfqpoint{0.983333in}{0.995580in}}{\pgfqpoint{0.985089in}{0.991340in}}{\pgfqpoint{0.988215in}{0.988215in}}%
\pgfpathcurveto{\pgfqpoint{0.991340in}{0.985089in}}{\pgfqpoint{0.995580in}{0.983333in}}{\pgfqpoint{1.000000in}{0.983333in}}%
\pgfpathclose%
\pgfusepath{stroke}%
\end{pgfscope}%
}%
\pgfsys@transformshift{9.008038in}{3.751700in}%
\pgfsys@useobject{currentpattern}{}%
\pgfsys@transformshift{1in}{0in}%
\pgfsys@transformshift{-1in}{0in}%
\pgfsys@transformshift{0in}{1in}%
\end{pgfscope}%
\begin{pgfscope}%
\pgfpathrectangle{\pgfqpoint{0.870538in}{0.637495in}}{\pgfqpoint{9.300000in}{9.060000in}}%
\pgfusepath{clip}%
\pgfsetbuttcap%
\pgfsetmiterjoin%
\definecolor{currentfill}{rgb}{1.000000,1.000000,0.000000}%
\pgfsetfillcolor{currentfill}%
\pgfsetfillopacity{0.990000}%
\pgfsetlinewidth{0.000000pt}%
\definecolor{currentstroke}{rgb}{0.000000,0.000000,0.000000}%
\pgfsetstrokecolor{currentstroke}%
\pgfsetstrokeopacity{0.990000}%
\pgfsetdash{}{0pt}%
\pgfpathmoveto{\pgfqpoint{1.258038in}{3.502115in}}%
\pgfpathlineto{\pgfqpoint{2.033038in}{3.502115in}}%
\pgfpathlineto{\pgfqpoint{2.033038in}{3.881164in}}%
\pgfpathlineto{\pgfqpoint{1.258038in}{3.881164in}}%
\pgfpathclose%
\pgfusepath{fill}%
\end{pgfscope}%
\begin{pgfscope}%
\pgfsetbuttcap%
\pgfsetmiterjoin%
\definecolor{currentfill}{rgb}{1.000000,1.000000,0.000000}%
\pgfsetfillcolor{currentfill}%
\pgfsetfillopacity{0.990000}%
\pgfsetlinewidth{0.000000pt}%
\definecolor{currentstroke}{rgb}{0.000000,0.000000,0.000000}%
\pgfsetstrokecolor{currentstroke}%
\pgfsetstrokeopacity{0.990000}%
\pgfsetdash{}{0pt}%
\pgfpathrectangle{\pgfqpoint{0.870538in}{0.637495in}}{\pgfqpoint{9.300000in}{9.060000in}}%
\pgfusepath{clip}%
\pgfpathmoveto{\pgfqpoint{1.258038in}{3.502115in}}%
\pgfpathlineto{\pgfqpoint{2.033038in}{3.502115in}}%
\pgfpathlineto{\pgfqpoint{2.033038in}{3.881164in}}%
\pgfpathlineto{\pgfqpoint{1.258038in}{3.881164in}}%
\pgfpathclose%
\pgfusepath{clip}%
\pgfsys@defobject{currentpattern}{\pgfqpoint{0in}{0in}}{\pgfqpoint{1in}{1in}}{%
\begin{pgfscope}%
\pgfpathrectangle{\pgfqpoint{0in}{0in}}{\pgfqpoint{1in}{1in}}%
\pgfusepath{clip}%
\pgfpathmoveto{\pgfqpoint{0.000000in}{0.055556in}}%
\pgfpathlineto{\pgfqpoint{-0.016327in}{0.022473in}}%
\pgfpathlineto{\pgfqpoint{-0.052836in}{0.017168in}}%
\pgfpathlineto{\pgfqpoint{-0.026418in}{-0.008584in}}%
\pgfpathlineto{\pgfqpoint{-0.032655in}{-0.044945in}}%
\pgfpathlineto{\pgfqpoint{-0.000000in}{-0.027778in}}%
\pgfpathlineto{\pgfqpoint{0.032655in}{-0.044945in}}%
\pgfpathlineto{\pgfqpoint{0.026418in}{-0.008584in}}%
\pgfpathlineto{\pgfqpoint{0.052836in}{0.017168in}}%
\pgfpathlineto{\pgfqpoint{0.016327in}{0.022473in}}%
\pgfpathlineto{\pgfqpoint{0.000000in}{0.055556in}}%
\pgfpathmoveto{\pgfqpoint{0.166667in}{0.055556in}}%
\pgfpathlineto{\pgfqpoint{0.150339in}{0.022473in}}%
\pgfpathlineto{\pgfqpoint{0.113830in}{0.017168in}}%
\pgfpathlineto{\pgfqpoint{0.140248in}{-0.008584in}}%
\pgfpathlineto{\pgfqpoint{0.134012in}{-0.044945in}}%
\pgfpathlineto{\pgfqpoint{0.166667in}{-0.027778in}}%
\pgfpathlineto{\pgfqpoint{0.199321in}{-0.044945in}}%
\pgfpathlineto{\pgfqpoint{0.193085in}{-0.008584in}}%
\pgfpathlineto{\pgfqpoint{0.219503in}{0.017168in}}%
\pgfpathlineto{\pgfqpoint{0.182994in}{0.022473in}}%
\pgfpathlineto{\pgfqpoint{0.166667in}{0.055556in}}%
\pgfpathmoveto{\pgfqpoint{0.333333in}{0.055556in}}%
\pgfpathlineto{\pgfqpoint{0.317006in}{0.022473in}}%
\pgfpathlineto{\pgfqpoint{0.280497in}{0.017168in}}%
\pgfpathlineto{\pgfqpoint{0.306915in}{-0.008584in}}%
\pgfpathlineto{\pgfqpoint{0.300679in}{-0.044945in}}%
\pgfpathlineto{\pgfqpoint{0.333333in}{-0.027778in}}%
\pgfpathlineto{\pgfqpoint{0.365988in}{-0.044945in}}%
\pgfpathlineto{\pgfqpoint{0.359752in}{-0.008584in}}%
\pgfpathlineto{\pgfqpoint{0.386170in}{0.017168in}}%
\pgfpathlineto{\pgfqpoint{0.349661in}{0.022473in}}%
\pgfpathlineto{\pgfqpoint{0.333333in}{0.055556in}}%
\pgfpathmoveto{\pgfqpoint{0.500000in}{0.055556in}}%
\pgfpathlineto{\pgfqpoint{0.483673in}{0.022473in}}%
\pgfpathlineto{\pgfqpoint{0.447164in}{0.017168in}}%
\pgfpathlineto{\pgfqpoint{0.473582in}{-0.008584in}}%
\pgfpathlineto{\pgfqpoint{0.467345in}{-0.044945in}}%
\pgfpathlineto{\pgfqpoint{0.500000in}{-0.027778in}}%
\pgfpathlineto{\pgfqpoint{0.532655in}{-0.044945in}}%
\pgfpathlineto{\pgfqpoint{0.526418in}{-0.008584in}}%
\pgfpathlineto{\pgfqpoint{0.552836in}{0.017168in}}%
\pgfpathlineto{\pgfqpoint{0.516327in}{0.022473in}}%
\pgfpathlineto{\pgfqpoint{0.500000in}{0.055556in}}%
\pgfpathmoveto{\pgfqpoint{0.666667in}{0.055556in}}%
\pgfpathlineto{\pgfqpoint{0.650339in}{0.022473in}}%
\pgfpathlineto{\pgfqpoint{0.613830in}{0.017168in}}%
\pgfpathlineto{\pgfqpoint{0.640248in}{-0.008584in}}%
\pgfpathlineto{\pgfqpoint{0.634012in}{-0.044945in}}%
\pgfpathlineto{\pgfqpoint{0.666667in}{-0.027778in}}%
\pgfpathlineto{\pgfqpoint{0.699321in}{-0.044945in}}%
\pgfpathlineto{\pgfqpoint{0.693085in}{-0.008584in}}%
\pgfpathlineto{\pgfqpoint{0.719503in}{0.017168in}}%
\pgfpathlineto{\pgfqpoint{0.682994in}{0.022473in}}%
\pgfpathlineto{\pgfqpoint{0.666667in}{0.055556in}}%
\pgfpathmoveto{\pgfqpoint{0.833333in}{0.055556in}}%
\pgfpathlineto{\pgfqpoint{0.817006in}{0.022473in}}%
\pgfpathlineto{\pgfqpoint{0.780497in}{0.017168in}}%
\pgfpathlineto{\pgfqpoint{0.806915in}{-0.008584in}}%
\pgfpathlineto{\pgfqpoint{0.800679in}{-0.044945in}}%
\pgfpathlineto{\pgfqpoint{0.833333in}{-0.027778in}}%
\pgfpathlineto{\pgfqpoint{0.865988in}{-0.044945in}}%
\pgfpathlineto{\pgfqpoint{0.859752in}{-0.008584in}}%
\pgfpathlineto{\pgfqpoint{0.886170in}{0.017168in}}%
\pgfpathlineto{\pgfqpoint{0.849661in}{0.022473in}}%
\pgfpathlineto{\pgfqpoint{0.833333in}{0.055556in}}%
\pgfpathmoveto{\pgfqpoint{1.000000in}{0.055556in}}%
\pgfpathlineto{\pgfqpoint{0.983673in}{0.022473in}}%
\pgfpathlineto{\pgfqpoint{0.947164in}{0.017168in}}%
\pgfpathlineto{\pgfqpoint{0.973582in}{-0.008584in}}%
\pgfpathlineto{\pgfqpoint{0.967345in}{-0.044945in}}%
\pgfpathlineto{\pgfqpoint{1.000000in}{-0.027778in}}%
\pgfpathlineto{\pgfqpoint{1.032655in}{-0.044945in}}%
\pgfpathlineto{\pgfqpoint{1.026418in}{-0.008584in}}%
\pgfpathlineto{\pgfqpoint{1.052836in}{0.017168in}}%
\pgfpathlineto{\pgfqpoint{1.016327in}{0.022473in}}%
\pgfpathlineto{\pgfqpoint{1.000000in}{0.055556in}}%
\pgfpathmoveto{\pgfqpoint{0.083333in}{0.222222in}}%
\pgfpathlineto{\pgfqpoint{0.067006in}{0.189139in}}%
\pgfpathlineto{\pgfqpoint{0.030497in}{0.183834in}}%
\pgfpathlineto{\pgfqpoint{0.056915in}{0.158083in}}%
\pgfpathlineto{\pgfqpoint{0.050679in}{0.121721in}}%
\pgfpathlineto{\pgfqpoint{0.083333in}{0.138889in}}%
\pgfpathlineto{\pgfqpoint{0.115988in}{0.121721in}}%
\pgfpathlineto{\pgfqpoint{0.109752in}{0.158083in}}%
\pgfpathlineto{\pgfqpoint{0.136170in}{0.183834in}}%
\pgfpathlineto{\pgfqpoint{0.099661in}{0.189139in}}%
\pgfpathlineto{\pgfqpoint{0.083333in}{0.222222in}}%
\pgfpathmoveto{\pgfqpoint{0.250000in}{0.222222in}}%
\pgfpathlineto{\pgfqpoint{0.233673in}{0.189139in}}%
\pgfpathlineto{\pgfqpoint{0.197164in}{0.183834in}}%
\pgfpathlineto{\pgfqpoint{0.223582in}{0.158083in}}%
\pgfpathlineto{\pgfqpoint{0.217345in}{0.121721in}}%
\pgfpathlineto{\pgfqpoint{0.250000in}{0.138889in}}%
\pgfpathlineto{\pgfqpoint{0.282655in}{0.121721in}}%
\pgfpathlineto{\pgfqpoint{0.276418in}{0.158083in}}%
\pgfpathlineto{\pgfqpoint{0.302836in}{0.183834in}}%
\pgfpathlineto{\pgfqpoint{0.266327in}{0.189139in}}%
\pgfpathlineto{\pgfqpoint{0.250000in}{0.222222in}}%
\pgfpathmoveto{\pgfqpoint{0.416667in}{0.222222in}}%
\pgfpathlineto{\pgfqpoint{0.400339in}{0.189139in}}%
\pgfpathlineto{\pgfqpoint{0.363830in}{0.183834in}}%
\pgfpathlineto{\pgfqpoint{0.390248in}{0.158083in}}%
\pgfpathlineto{\pgfqpoint{0.384012in}{0.121721in}}%
\pgfpathlineto{\pgfqpoint{0.416667in}{0.138889in}}%
\pgfpathlineto{\pgfqpoint{0.449321in}{0.121721in}}%
\pgfpathlineto{\pgfqpoint{0.443085in}{0.158083in}}%
\pgfpathlineto{\pgfqpoint{0.469503in}{0.183834in}}%
\pgfpathlineto{\pgfqpoint{0.432994in}{0.189139in}}%
\pgfpathlineto{\pgfqpoint{0.416667in}{0.222222in}}%
\pgfpathmoveto{\pgfqpoint{0.583333in}{0.222222in}}%
\pgfpathlineto{\pgfqpoint{0.567006in}{0.189139in}}%
\pgfpathlineto{\pgfqpoint{0.530497in}{0.183834in}}%
\pgfpathlineto{\pgfqpoint{0.556915in}{0.158083in}}%
\pgfpathlineto{\pgfqpoint{0.550679in}{0.121721in}}%
\pgfpathlineto{\pgfqpoint{0.583333in}{0.138889in}}%
\pgfpathlineto{\pgfqpoint{0.615988in}{0.121721in}}%
\pgfpathlineto{\pgfqpoint{0.609752in}{0.158083in}}%
\pgfpathlineto{\pgfqpoint{0.636170in}{0.183834in}}%
\pgfpathlineto{\pgfqpoint{0.599661in}{0.189139in}}%
\pgfpathlineto{\pgfqpoint{0.583333in}{0.222222in}}%
\pgfpathmoveto{\pgfqpoint{0.750000in}{0.222222in}}%
\pgfpathlineto{\pgfqpoint{0.733673in}{0.189139in}}%
\pgfpathlineto{\pgfqpoint{0.697164in}{0.183834in}}%
\pgfpathlineto{\pgfqpoint{0.723582in}{0.158083in}}%
\pgfpathlineto{\pgfqpoint{0.717345in}{0.121721in}}%
\pgfpathlineto{\pgfqpoint{0.750000in}{0.138889in}}%
\pgfpathlineto{\pgfqpoint{0.782655in}{0.121721in}}%
\pgfpathlineto{\pgfqpoint{0.776418in}{0.158083in}}%
\pgfpathlineto{\pgfqpoint{0.802836in}{0.183834in}}%
\pgfpathlineto{\pgfqpoint{0.766327in}{0.189139in}}%
\pgfpathlineto{\pgfqpoint{0.750000in}{0.222222in}}%
\pgfpathmoveto{\pgfqpoint{0.916667in}{0.222222in}}%
\pgfpathlineto{\pgfqpoint{0.900339in}{0.189139in}}%
\pgfpathlineto{\pgfqpoint{0.863830in}{0.183834in}}%
\pgfpathlineto{\pgfqpoint{0.890248in}{0.158083in}}%
\pgfpathlineto{\pgfqpoint{0.884012in}{0.121721in}}%
\pgfpathlineto{\pgfqpoint{0.916667in}{0.138889in}}%
\pgfpathlineto{\pgfqpoint{0.949321in}{0.121721in}}%
\pgfpathlineto{\pgfqpoint{0.943085in}{0.158083in}}%
\pgfpathlineto{\pgfqpoint{0.969503in}{0.183834in}}%
\pgfpathlineto{\pgfqpoint{0.932994in}{0.189139in}}%
\pgfpathlineto{\pgfqpoint{0.916667in}{0.222222in}}%
\pgfpathmoveto{\pgfqpoint{0.000000in}{0.388889in}}%
\pgfpathlineto{\pgfqpoint{-0.016327in}{0.355806in}}%
\pgfpathlineto{\pgfqpoint{-0.052836in}{0.350501in}}%
\pgfpathlineto{\pgfqpoint{-0.026418in}{0.324750in}}%
\pgfpathlineto{\pgfqpoint{-0.032655in}{0.288388in}}%
\pgfpathlineto{\pgfqpoint{-0.000000in}{0.305556in}}%
\pgfpathlineto{\pgfqpoint{0.032655in}{0.288388in}}%
\pgfpathlineto{\pgfqpoint{0.026418in}{0.324750in}}%
\pgfpathlineto{\pgfqpoint{0.052836in}{0.350501in}}%
\pgfpathlineto{\pgfqpoint{0.016327in}{0.355806in}}%
\pgfpathlineto{\pgfqpoint{0.000000in}{0.388889in}}%
\pgfpathmoveto{\pgfqpoint{0.166667in}{0.388889in}}%
\pgfpathlineto{\pgfqpoint{0.150339in}{0.355806in}}%
\pgfpathlineto{\pgfqpoint{0.113830in}{0.350501in}}%
\pgfpathlineto{\pgfqpoint{0.140248in}{0.324750in}}%
\pgfpathlineto{\pgfqpoint{0.134012in}{0.288388in}}%
\pgfpathlineto{\pgfqpoint{0.166667in}{0.305556in}}%
\pgfpathlineto{\pgfqpoint{0.199321in}{0.288388in}}%
\pgfpathlineto{\pgfqpoint{0.193085in}{0.324750in}}%
\pgfpathlineto{\pgfqpoint{0.219503in}{0.350501in}}%
\pgfpathlineto{\pgfqpoint{0.182994in}{0.355806in}}%
\pgfpathlineto{\pgfqpoint{0.166667in}{0.388889in}}%
\pgfpathmoveto{\pgfqpoint{0.333333in}{0.388889in}}%
\pgfpathlineto{\pgfqpoint{0.317006in}{0.355806in}}%
\pgfpathlineto{\pgfqpoint{0.280497in}{0.350501in}}%
\pgfpathlineto{\pgfqpoint{0.306915in}{0.324750in}}%
\pgfpathlineto{\pgfqpoint{0.300679in}{0.288388in}}%
\pgfpathlineto{\pgfqpoint{0.333333in}{0.305556in}}%
\pgfpathlineto{\pgfqpoint{0.365988in}{0.288388in}}%
\pgfpathlineto{\pgfqpoint{0.359752in}{0.324750in}}%
\pgfpathlineto{\pgfqpoint{0.386170in}{0.350501in}}%
\pgfpathlineto{\pgfqpoint{0.349661in}{0.355806in}}%
\pgfpathlineto{\pgfqpoint{0.333333in}{0.388889in}}%
\pgfpathmoveto{\pgfqpoint{0.500000in}{0.388889in}}%
\pgfpathlineto{\pgfqpoint{0.483673in}{0.355806in}}%
\pgfpathlineto{\pgfqpoint{0.447164in}{0.350501in}}%
\pgfpathlineto{\pgfqpoint{0.473582in}{0.324750in}}%
\pgfpathlineto{\pgfqpoint{0.467345in}{0.288388in}}%
\pgfpathlineto{\pgfqpoint{0.500000in}{0.305556in}}%
\pgfpathlineto{\pgfqpoint{0.532655in}{0.288388in}}%
\pgfpathlineto{\pgfqpoint{0.526418in}{0.324750in}}%
\pgfpathlineto{\pgfqpoint{0.552836in}{0.350501in}}%
\pgfpathlineto{\pgfqpoint{0.516327in}{0.355806in}}%
\pgfpathlineto{\pgfqpoint{0.500000in}{0.388889in}}%
\pgfpathmoveto{\pgfqpoint{0.666667in}{0.388889in}}%
\pgfpathlineto{\pgfqpoint{0.650339in}{0.355806in}}%
\pgfpathlineto{\pgfqpoint{0.613830in}{0.350501in}}%
\pgfpathlineto{\pgfqpoint{0.640248in}{0.324750in}}%
\pgfpathlineto{\pgfqpoint{0.634012in}{0.288388in}}%
\pgfpathlineto{\pgfqpoint{0.666667in}{0.305556in}}%
\pgfpathlineto{\pgfqpoint{0.699321in}{0.288388in}}%
\pgfpathlineto{\pgfqpoint{0.693085in}{0.324750in}}%
\pgfpathlineto{\pgfqpoint{0.719503in}{0.350501in}}%
\pgfpathlineto{\pgfqpoint{0.682994in}{0.355806in}}%
\pgfpathlineto{\pgfqpoint{0.666667in}{0.388889in}}%
\pgfpathmoveto{\pgfqpoint{0.833333in}{0.388889in}}%
\pgfpathlineto{\pgfqpoint{0.817006in}{0.355806in}}%
\pgfpathlineto{\pgfqpoint{0.780497in}{0.350501in}}%
\pgfpathlineto{\pgfqpoint{0.806915in}{0.324750in}}%
\pgfpathlineto{\pgfqpoint{0.800679in}{0.288388in}}%
\pgfpathlineto{\pgfqpoint{0.833333in}{0.305556in}}%
\pgfpathlineto{\pgfqpoint{0.865988in}{0.288388in}}%
\pgfpathlineto{\pgfqpoint{0.859752in}{0.324750in}}%
\pgfpathlineto{\pgfqpoint{0.886170in}{0.350501in}}%
\pgfpathlineto{\pgfqpoint{0.849661in}{0.355806in}}%
\pgfpathlineto{\pgfqpoint{0.833333in}{0.388889in}}%
\pgfpathmoveto{\pgfqpoint{1.000000in}{0.388889in}}%
\pgfpathlineto{\pgfqpoint{0.983673in}{0.355806in}}%
\pgfpathlineto{\pgfqpoint{0.947164in}{0.350501in}}%
\pgfpathlineto{\pgfqpoint{0.973582in}{0.324750in}}%
\pgfpathlineto{\pgfqpoint{0.967345in}{0.288388in}}%
\pgfpathlineto{\pgfqpoint{1.000000in}{0.305556in}}%
\pgfpathlineto{\pgfqpoint{1.032655in}{0.288388in}}%
\pgfpathlineto{\pgfqpoint{1.026418in}{0.324750in}}%
\pgfpathlineto{\pgfqpoint{1.052836in}{0.350501in}}%
\pgfpathlineto{\pgfqpoint{1.016327in}{0.355806in}}%
\pgfpathlineto{\pgfqpoint{1.000000in}{0.388889in}}%
\pgfpathmoveto{\pgfqpoint{0.083333in}{0.555556in}}%
\pgfpathlineto{\pgfqpoint{0.067006in}{0.522473in}}%
\pgfpathlineto{\pgfqpoint{0.030497in}{0.517168in}}%
\pgfpathlineto{\pgfqpoint{0.056915in}{0.491416in}}%
\pgfpathlineto{\pgfqpoint{0.050679in}{0.455055in}}%
\pgfpathlineto{\pgfqpoint{0.083333in}{0.472222in}}%
\pgfpathlineto{\pgfqpoint{0.115988in}{0.455055in}}%
\pgfpathlineto{\pgfqpoint{0.109752in}{0.491416in}}%
\pgfpathlineto{\pgfqpoint{0.136170in}{0.517168in}}%
\pgfpathlineto{\pgfqpoint{0.099661in}{0.522473in}}%
\pgfpathlineto{\pgfqpoint{0.083333in}{0.555556in}}%
\pgfpathmoveto{\pgfqpoint{0.250000in}{0.555556in}}%
\pgfpathlineto{\pgfqpoint{0.233673in}{0.522473in}}%
\pgfpathlineto{\pgfqpoint{0.197164in}{0.517168in}}%
\pgfpathlineto{\pgfqpoint{0.223582in}{0.491416in}}%
\pgfpathlineto{\pgfqpoint{0.217345in}{0.455055in}}%
\pgfpathlineto{\pgfqpoint{0.250000in}{0.472222in}}%
\pgfpathlineto{\pgfqpoint{0.282655in}{0.455055in}}%
\pgfpathlineto{\pgfqpoint{0.276418in}{0.491416in}}%
\pgfpathlineto{\pgfqpoint{0.302836in}{0.517168in}}%
\pgfpathlineto{\pgfqpoint{0.266327in}{0.522473in}}%
\pgfpathlineto{\pgfqpoint{0.250000in}{0.555556in}}%
\pgfpathmoveto{\pgfqpoint{0.416667in}{0.555556in}}%
\pgfpathlineto{\pgfqpoint{0.400339in}{0.522473in}}%
\pgfpathlineto{\pgfqpoint{0.363830in}{0.517168in}}%
\pgfpathlineto{\pgfqpoint{0.390248in}{0.491416in}}%
\pgfpathlineto{\pgfqpoint{0.384012in}{0.455055in}}%
\pgfpathlineto{\pgfqpoint{0.416667in}{0.472222in}}%
\pgfpathlineto{\pgfqpoint{0.449321in}{0.455055in}}%
\pgfpathlineto{\pgfqpoint{0.443085in}{0.491416in}}%
\pgfpathlineto{\pgfqpoint{0.469503in}{0.517168in}}%
\pgfpathlineto{\pgfqpoint{0.432994in}{0.522473in}}%
\pgfpathlineto{\pgfqpoint{0.416667in}{0.555556in}}%
\pgfpathmoveto{\pgfqpoint{0.583333in}{0.555556in}}%
\pgfpathlineto{\pgfqpoint{0.567006in}{0.522473in}}%
\pgfpathlineto{\pgfqpoint{0.530497in}{0.517168in}}%
\pgfpathlineto{\pgfqpoint{0.556915in}{0.491416in}}%
\pgfpathlineto{\pgfqpoint{0.550679in}{0.455055in}}%
\pgfpathlineto{\pgfqpoint{0.583333in}{0.472222in}}%
\pgfpathlineto{\pgfqpoint{0.615988in}{0.455055in}}%
\pgfpathlineto{\pgfqpoint{0.609752in}{0.491416in}}%
\pgfpathlineto{\pgfqpoint{0.636170in}{0.517168in}}%
\pgfpathlineto{\pgfqpoint{0.599661in}{0.522473in}}%
\pgfpathlineto{\pgfqpoint{0.583333in}{0.555556in}}%
\pgfpathmoveto{\pgfqpoint{0.750000in}{0.555556in}}%
\pgfpathlineto{\pgfqpoint{0.733673in}{0.522473in}}%
\pgfpathlineto{\pgfqpoint{0.697164in}{0.517168in}}%
\pgfpathlineto{\pgfqpoint{0.723582in}{0.491416in}}%
\pgfpathlineto{\pgfqpoint{0.717345in}{0.455055in}}%
\pgfpathlineto{\pgfqpoint{0.750000in}{0.472222in}}%
\pgfpathlineto{\pgfqpoint{0.782655in}{0.455055in}}%
\pgfpathlineto{\pgfqpoint{0.776418in}{0.491416in}}%
\pgfpathlineto{\pgfqpoint{0.802836in}{0.517168in}}%
\pgfpathlineto{\pgfqpoint{0.766327in}{0.522473in}}%
\pgfpathlineto{\pgfqpoint{0.750000in}{0.555556in}}%
\pgfpathmoveto{\pgfqpoint{0.916667in}{0.555556in}}%
\pgfpathlineto{\pgfqpoint{0.900339in}{0.522473in}}%
\pgfpathlineto{\pgfqpoint{0.863830in}{0.517168in}}%
\pgfpathlineto{\pgfqpoint{0.890248in}{0.491416in}}%
\pgfpathlineto{\pgfqpoint{0.884012in}{0.455055in}}%
\pgfpathlineto{\pgfqpoint{0.916667in}{0.472222in}}%
\pgfpathlineto{\pgfqpoint{0.949321in}{0.455055in}}%
\pgfpathlineto{\pgfqpoint{0.943085in}{0.491416in}}%
\pgfpathlineto{\pgfqpoint{0.969503in}{0.517168in}}%
\pgfpathlineto{\pgfqpoint{0.932994in}{0.522473in}}%
\pgfpathlineto{\pgfqpoint{0.916667in}{0.555556in}}%
\pgfpathmoveto{\pgfqpoint{0.000000in}{0.722222in}}%
\pgfpathlineto{\pgfqpoint{-0.016327in}{0.689139in}}%
\pgfpathlineto{\pgfqpoint{-0.052836in}{0.683834in}}%
\pgfpathlineto{\pgfqpoint{-0.026418in}{0.658083in}}%
\pgfpathlineto{\pgfqpoint{-0.032655in}{0.621721in}}%
\pgfpathlineto{\pgfqpoint{-0.000000in}{0.638889in}}%
\pgfpathlineto{\pgfqpoint{0.032655in}{0.621721in}}%
\pgfpathlineto{\pgfqpoint{0.026418in}{0.658083in}}%
\pgfpathlineto{\pgfqpoint{0.052836in}{0.683834in}}%
\pgfpathlineto{\pgfqpoint{0.016327in}{0.689139in}}%
\pgfpathlineto{\pgfqpoint{0.000000in}{0.722222in}}%
\pgfpathmoveto{\pgfqpoint{0.166667in}{0.722222in}}%
\pgfpathlineto{\pgfqpoint{0.150339in}{0.689139in}}%
\pgfpathlineto{\pgfqpoint{0.113830in}{0.683834in}}%
\pgfpathlineto{\pgfqpoint{0.140248in}{0.658083in}}%
\pgfpathlineto{\pgfqpoint{0.134012in}{0.621721in}}%
\pgfpathlineto{\pgfqpoint{0.166667in}{0.638889in}}%
\pgfpathlineto{\pgfqpoint{0.199321in}{0.621721in}}%
\pgfpathlineto{\pgfqpoint{0.193085in}{0.658083in}}%
\pgfpathlineto{\pgfqpoint{0.219503in}{0.683834in}}%
\pgfpathlineto{\pgfqpoint{0.182994in}{0.689139in}}%
\pgfpathlineto{\pgfqpoint{0.166667in}{0.722222in}}%
\pgfpathmoveto{\pgfqpoint{0.333333in}{0.722222in}}%
\pgfpathlineto{\pgfqpoint{0.317006in}{0.689139in}}%
\pgfpathlineto{\pgfqpoint{0.280497in}{0.683834in}}%
\pgfpathlineto{\pgfqpoint{0.306915in}{0.658083in}}%
\pgfpathlineto{\pgfqpoint{0.300679in}{0.621721in}}%
\pgfpathlineto{\pgfqpoint{0.333333in}{0.638889in}}%
\pgfpathlineto{\pgfqpoint{0.365988in}{0.621721in}}%
\pgfpathlineto{\pgfqpoint{0.359752in}{0.658083in}}%
\pgfpathlineto{\pgfqpoint{0.386170in}{0.683834in}}%
\pgfpathlineto{\pgfqpoint{0.349661in}{0.689139in}}%
\pgfpathlineto{\pgfqpoint{0.333333in}{0.722222in}}%
\pgfpathmoveto{\pgfqpoint{0.500000in}{0.722222in}}%
\pgfpathlineto{\pgfqpoint{0.483673in}{0.689139in}}%
\pgfpathlineto{\pgfqpoint{0.447164in}{0.683834in}}%
\pgfpathlineto{\pgfqpoint{0.473582in}{0.658083in}}%
\pgfpathlineto{\pgfqpoint{0.467345in}{0.621721in}}%
\pgfpathlineto{\pgfqpoint{0.500000in}{0.638889in}}%
\pgfpathlineto{\pgfqpoint{0.532655in}{0.621721in}}%
\pgfpathlineto{\pgfqpoint{0.526418in}{0.658083in}}%
\pgfpathlineto{\pgfqpoint{0.552836in}{0.683834in}}%
\pgfpathlineto{\pgfqpoint{0.516327in}{0.689139in}}%
\pgfpathlineto{\pgfqpoint{0.500000in}{0.722222in}}%
\pgfpathmoveto{\pgfqpoint{0.666667in}{0.722222in}}%
\pgfpathlineto{\pgfqpoint{0.650339in}{0.689139in}}%
\pgfpathlineto{\pgfqpoint{0.613830in}{0.683834in}}%
\pgfpathlineto{\pgfqpoint{0.640248in}{0.658083in}}%
\pgfpathlineto{\pgfqpoint{0.634012in}{0.621721in}}%
\pgfpathlineto{\pgfqpoint{0.666667in}{0.638889in}}%
\pgfpathlineto{\pgfqpoint{0.699321in}{0.621721in}}%
\pgfpathlineto{\pgfqpoint{0.693085in}{0.658083in}}%
\pgfpathlineto{\pgfqpoint{0.719503in}{0.683834in}}%
\pgfpathlineto{\pgfqpoint{0.682994in}{0.689139in}}%
\pgfpathlineto{\pgfqpoint{0.666667in}{0.722222in}}%
\pgfpathmoveto{\pgfqpoint{0.833333in}{0.722222in}}%
\pgfpathlineto{\pgfqpoint{0.817006in}{0.689139in}}%
\pgfpathlineto{\pgfqpoint{0.780497in}{0.683834in}}%
\pgfpathlineto{\pgfqpoint{0.806915in}{0.658083in}}%
\pgfpathlineto{\pgfqpoint{0.800679in}{0.621721in}}%
\pgfpathlineto{\pgfqpoint{0.833333in}{0.638889in}}%
\pgfpathlineto{\pgfqpoint{0.865988in}{0.621721in}}%
\pgfpathlineto{\pgfqpoint{0.859752in}{0.658083in}}%
\pgfpathlineto{\pgfqpoint{0.886170in}{0.683834in}}%
\pgfpathlineto{\pgfqpoint{0.849661in}{0.689139in}}%
\pgfpathlineto{\pgfqpoint{0.833333in}{0.722222in}}%
\pgfpathmoveto{\pgfqpoint{1.000000in}{0.722222in}}%
\pgfpathlineto{\pgfqpoint{0.983673in}{0.689139in}}%
\pgfpathlineto{\pgfqpoint{0.947164in}{0.683834in}}%
\pgfpathlineto{\pgfqpoint{0.973582in}{0.658083in}}%
\pgfpathlineto{\pgfqpoint{0.967345in}{0.621721in}}%
\pgfpathlineto{\pgfqpoint{1.000000in}{0.638889in}}%
\pgfpathlineto{\pgfqpoint{1.032655in}{0.621721in}}%
\pgfpathlineto{\pgfqpoint{1.026418in}{0.658083in}}%
\pgfpathlineto{\pgfqpoint{1.052836in}{0.683834in}}%
\pgfpathlineto{\pgfqpoint{1.016327in}{0.689139in}}%
\pgfpathlineto{\pgfqpoint{1.000000in}{0.722222in}}%
\pgfpathmoveto{\pgfqpoint{0.083333in}{0.888889in}}%
\pgfpathlineto{\pgfqpoint{0.067006in}{0.855806in}}%
\pgfpathlineto{\pgfqpoint{0.030497in}{0.850501in}}%
\pgfpathlineto{\pgfqpoint{0.056915in}{0.824750in}}%
\pgfpathlineto{\pgfqpoint{0.050679in}{0.788388in}}%
\pgfpathlineto{\pgfqpoint{0.083333in}{0.805556in}}%
\pgfpathlineto{\pgfqpoint{0.115988in}{0.788388in}}%
\pgfpathlineto{\pgfqpoint{0.109752in}{0.824750in}}%
\pgfpathlineto{\pgfqpoint{0.136170in}{0.850501in}}%
\pgfpathlineto{\pgfqpoint{0.099661in}{0.855806in}}%
\pgfpathlineto{\pgfqpoint{0.083333in}{0.888889in}}%
\pgfpathmoveto{\pgfqpoint{0.250000in}{0.888889in}}%
\pgfpathlineto{\pgfqpoint{0.233673in}{0.855806in}}%
\pgfpathlineto{\pgfqpoint{0.197164in}{0.850501in}}%
\pgfpathlineto{\pgfqpoint{0.223582in}{0.824750in}}%
\pgfpathlineto{\pgfqpoint{0.217345in}{0.788388in}}%
\pgfpathlineto{\pgfqpoint{0.250000in}{0.805556in}}%
\pgfpathlineto{\pgfqpoint{0.282655in}{0.788388in}}%
\pgfpathlineto{\pgfqpoint{0.276418in}{0.824750in}}%
\pgfpathlineto{\pgfqpoint{0.302836in}{0.850501in}}%
\pgfpathlineto{\pgfqpoint{0.266327in}{0.855806in}}%
\pgfpathlineto{\pgfqpoint{0.250000in}{0.888889in}}%
\pgfpathmoveto{\pgfqpoint{0.416667in}{0.888889in}}%
\pgfpathlineto{\pgfqpoint{0.400339in}{0.855806in}}%
\pgfpathlineto{\pgfqpoint{0.363830in}{0.850501in}}%
\pgfpathlineto{\pgfqpoint{0.390248in}{0.824750in}}%
\pgfpathlineto{\pgfqpoint{0.384012in}{0.788388in}}%
\pgfpathlineto{\pgfqpoint{0.416667in}{0.805556in}}%
\pgfpathlineto{\pgfqpoint{0.449321in}{0.788388in}}%
\pgfpathlineto{\pgfqpoint{0.443085in}{0.824750in}}%
\pgfpathlineto{\pgfqpoint{0.469503in}{0.850501in}}%
\pgfpathlineto{\pgfqpoint{0.432994in}{0.855806in}}%
\pgfpathlineto{\pgfqpoint{0.416667in}{0.888889in}}%
\pgfpathmoveto{\pgfqpoint{0.583333in}{0.888889in}}%
\pgfpathlineto{\pgfqpoint{0.567006in}{0.855806in}}%
\pgfpathlineto{\pgfqpoint{0.530497in}{0.850501in}}%
\pgfpathlineto{\pgfqpoint{0.556915in}{0.824750in}}%
\pgfpathlineto{\pgfqpoint{0.550679in}{0.788388in}}%
\pgfpathlineto{\pgfqpoint{0.583333in}{0.805556in}}%
\pgfpathlineto{\pgfqpoint{0.615988in}{0.788388in}}%
\pgfpathlineto{\pgfqpoint{0.609752in}{0.824750in}}%
\pgfpathlineto{\pgfqpoint{0.636170in}{0.850501in}}%
\pgfpathlineto{\pgfqpoint{0.599661in}{0.855806in}}%
\pgfpathlineto{\pgfqpoint{0.583333in}{0.888889in}}%
\pgfpathmoveto{\pgfqpoint{0.750000in}{0.888889in}}%
\pgfpathlineto{\pgfqpoint{0.733673in}{0.855806in}}%
\pgfpathlineto{\pgfqpoint{0.697164in}{0.850501in}}%
\pgfpathlineto{\pgfqpoint{0.723582in}{0.824750in}}%
\pgfpathlineto{\pgfqpoint{0.717345in}{0.788388in}}%
\pgfpathlineto{\pgfqpoint{0.750000in}{0.805556in}}%
\pgfpathlineto{\pgfqpoint{0.782655in}{0.788388in}}%
\pgfpathlineto{\pgfqpoint{0.776418in}{0.824750in}}%
\pgfpathlineto{\pgfqpoint{0.802836in}{0.850501in}}%
\pgfpathlineto{\pgfqpoint{0.766327in}{0.855806in}}%
\pgfpathlineto{\pgfqpoint{0.750000in}{0.888889in}}%
\pgfpathmoveto{\pgfqpoint{0.916667in}{0.888889in}}%
\pgfpathlineto{\pgfqpoint{0.900339in}{0.855806in}}%
\pgfpathlineto{\pgfqpoint{0.863830in}{0.850501in}}%
\pgfpathlineto{\pgfqpoint{0.890248in}{0.824750in}}%
\pgfpathlineto{\pgfqpoint{0.884012in}{0.788388in}}%
\pgfpathlineto{\pgfqpoint{0.916667in}{0.805556in}}%
\pgfpathlineto{\pgfqpoint{0.949321in}{0.788388in}}%
\pgfpathlineto{\pgfqpoint{0.943085in}{0.824750in}}%
\pgfpathlineto{\pgfqpoint{0.969503in}{0.850501in}}%
\pgfpathlineto{\pgfqpoint{0.932994in}{0.855806in}}%
\pgfpathlineto{\pgfqpoint{0.916667in}{0.888889in}}%
\pgfpathmoveto{\pgfqpoint{0.000000in}{1.055556in}}%
\pgfpathlineto{\pgfqpoint{-0.016327in}{1.022473in}}%
\pgfpathlineto{\pgfqpoint{-0.052836in}{1.017168in}}%
\pgfpathlineto{\pgfqpoint{-0.026418in}{0.991416in}}%
\pgfpathlineto{\pgfqpoint{-0.032655in}{0.955055in}}%
\pgfpathlineto{\pgfqpoint{-0.000000in}{0.972222in}}%
\pgfpathlineto{\pgfqpoint{0.032655in}{0.955055in}}%
\pgfpathlineto{\pgfqpoint{0.026418in}{0.991416in}}%
\pgfpathlineto{\pgfqpoint{0.052836in}{1.017168in}}%
\pgfpathlineto{\pgfqpoint{0.016327in}{1.022473in}}%
\pgfpathlineto{\pgfqpoint{0.000000in}{1.055556in}}%
\pgfpathmoveto{\pgfqpoint{0.166667in}{1.055556in}}%
\pgfpathlineto{\pgfqpoint{0.150339in}{1.022473in}}%
\pgfpathlineto{\pgfqpoint{0.113830in}{1.017168in}}%
\pgfpathlineto{\pgfqpoint{0.140248in}{0.991416in}}%
\pgfpathlineto{\pgfqpoint{0.134012in}{0.955055in}}%
\pgfpathlineto{\pgfqpoint{0.166667in}{0.972222in}}%
\pgfpathlineto{\pgfqpoint{0.199321in}{0.955055in}}%
\pgfpathlineto{\pgfqpoint{0.193085in}{0.991416in}}%
\pgfpathlineto{\pgfqpoint{0.219503in}{1.017168in}}%
\pgfpathlineto{\pgfqpoint{0.182994in}{1.022473in}}%
\pgfpathlineto{\pgfqpoint{0.166667in}{1.055556in}}%
\pgfpathmoveto{\pgfqpoint{0.333333in}{1.055556in}}%
\pgfpathlineto{\pgfqpoint{0.317006in}{1.022473in}}%
\pgfpathlineto{\pgfqpoint{0.280497in}{1.017168in}}%
\pgfpathlineto{\pgfqpoint{0.306915in}{0.991416in}}%
\pgfpathlineto{\pgfqpoint{0.300679in}{0.955055in}}%
\pgfpathlineto{\pgfqpoint{0.333333in}{0.972222in}}%
\pgfpathlineto{\pgfqpoint{0.365988in}{0.955055in}}%
\pgfpathlineto{\pgfqpoint{0.359752in}{0.991416in}}%
\pgfpathlineto{\pgfqpoint{0.386170in}{1.017168in}}%
\pgfpathlineto{\pgfqpoint{0.349661in}{1.022473in}}%
\pgfpathlineto{\pgfqpoint{0.333333in}{1.055556in}}%
\pgfpathmoveto{\pgfqpoint{0.500000in}{1.055556in}}%
\pgfpathlineto{\pgfqpoint{0.483673in}{1.022473in}}%
\pgfpathlineto{\pgfqpoint{0.447164in}{1.017168in}}%
\pgfpathlineto{\pgfqpoint{0.473582in}{0.991416in}}%
\pgfpathlineto{\pgfqpoint{0.467345in}{0.955055in}}%
\pgfpathlineto{\pgfqpoint{0.500000in}{0.972222in}}%
\pgfpathlineto{\pgfqpoint{0.532655in}{0.955055in}}%
\pgfpathlineto{\pgfqpoint{0.526418in}{0.991416in}}%
\pgfpathlineto{\pgfqpoint{0.552836in}{1.017168in}}%
\pgfpathlineto{\pgfqpoint{0.516327in}{1.022473in}}%
\pgfpathlineto{\pgfqpoint{0.500000in}{1.055556in}}%
\pgfpathmoveto{\pgfqpoint{0.666667in}{1.055556in}}%
\pgfpathlineto{\pgfqpoint{0.650339in}{1.022473in}}%
\pgfpathlineto{\pgfqpoint{0.613830in}{1.017168in}}%
\pgfpathlineto{\pgfqpoint{0.640248in}{0.991416in}}%
\pgfpathlineto{\pgfqpoint{0.634012in}{0.955055in}}%
\pgfpathlineto{\pgfqpoint{0.666667in}{0.972222in}}%
\pgfpathlineto{\pgfqpoint{0.699321in}{0.955055in}}%
\pgfpathlineto{\pgfqpoint{0.693085in}{0.991416in}}%
\pgfpathlineto{\pgfqpoint{0.719503in}{1.017168in}}%
\pgfpathlineto{\pgfqpoint{0.682994in}{1.022473in}}%
\pgfpathlineto{\pgfqpoint{0.666667in}{1.055556in}}%
\pgfpathmoveto{\pgfqpoint{0.833333in}{1.055556in}}%
\pgfpathlineto{\pgfqpoint{0.817006in}{1.022473in}}%
\pgfpathlineto{\pgfqpoint{0.780497in}{1.017168in}}%
\pgfpathlineto{\pgfqpoint{0.806915in}{0.991416in}}%
\pgfpathlineto{\pgfqpoint{0.800679in}{0.955055in}}%
\pgfpathlineto{\pgfqpoint{0.833333in}{0.972222in}}%
\pgfpathlineto{\pgfqpoint{0.865988in}{0.955055in}}%
\pgfpathlineto{\pgfqpoint{0.859752in}{0.991416in}}%
\pgfpathlineto{\pgfqpoint{0.886170in}{1.017168in}}%
\pgfpathlineto{\pgfqpoint{0.849661in}{1.022473in}}%
\pgfpathlineto{\pgfqpoint{0.833333in}{1.055556in}}%
\pgfpathmoveto{\pgfqpoint{1.000000in}{1.055556in}}%
\pgfpathlineto{\pgfqpoint{0.983673in}{1.022473in}}%
\pgfpathlineto{\pgfqpoint{0.947164in}{1.017168in}}%
\pgfpathlineto{\pgfqpoint{0.973582in}{0.991416in}}%
\pgfpathlineto{\pgfqpoint{0.967345in}{0.955055in}}%
\pgfpathlineto{\pgfqpoint{1.000000in}{0.972222in}}%
\pgfpathlineto{\pgfqpoint{1.032655in}{0.955055in}}%
\pgfpathlineto{\pgfqpoint{1.026418in}{0.991416in}}%
\pgfpathlineto{\pgfqpoint{1.052836in}{1.017168in}}%
\pgfpathlineto{\pgfqpoint{1.016327in}{1.022473in}}%
\pgfpathlineto{\pgfqpoint{1.000000in}{1.055556in}}%
\pgfpathlineto{\pgfqpoint{1.000000in}{1.055556in}}%
\pgfusepath{stroke}%
\end{pgfscope}%
}%
\pgfsys@transformshift{1.258038in}{3.502115in}%
\pgfsys@useobject{currentpattern}{}%
\pgfsys@transformshift{1in}{0in}%
\pgfsys@transformshift{-1in}{0in}%
\pgfsys@transformshift{0in}{1in}%
\end{pgfscope}%
\begin{pgfscope}%
\pgfpathrectangle{\pgfqpoint{0.870538in}{0.637495in}}{\pgfqpoint{9.300000in}{9.060000in}}%
\pgfusepath{clip}%
\pgfsetbuttcap%
\pgfsetmiterjoin%
\definecolor{currentfill}{rgb}{1.000000,1.000000,0.000000}%
\pgfsetfillcolor{currentfill}%
\pgfsetfillopacity{0.990000}%
\pgfsetlinewidth{0.000000pt}%
\definecolor{currentstroke}{rgb}{0.000000,0.000000,0.000000}%
\pgfsetstrokecolor{currentstroke}%
\pgfsetstrokeopacity{0.990000}%
\pgfsetdash{}{0pt}%
\pgfpathmoveto{\pgfqpoint{2.808038in}{3.468565in}}%
\pgfpathlineto{\pgfqpoint{3.583038in}{3.468565in}}%
\pgfpathlineto{\pgfqpoint{3.583038in}{4.019322in}}%
\pgfpathlineto{\pgfqpoint{2.808038in}{4.019322in}}%
\pgfpathclose%
\pgfusepath{fill}%
\end{pgfscope}%
\begin{pgfscope}%
\pgfsetbuttcap%
\pgfsetmiterjoin%
\definecolor{currentfill}{rgb}{1.000000,1.000000,0.000000}%
\pgfsetfillcolor{currentfill}%
\pgfsetfillopacity{0.990000}%
\pgfsetlinewidth{0.000000pt}%
\definecolor{currentstroke}{rgb}{0.000000,0.000000,0.000000}%
\pgfsetstrokecolor{currentstroke}%
\pgfsetstrokeopacity{0.990000}%
\pgfsetdash{}{0pt}%
\pgfpathrectangle{\pgfqpoint{0.870538in}{0.637495in}}{\pgfqpoint{9.300000in}{9.060000in}}%
\pgfusepath{clip}%
\pgfpathmoveto{\pgfqpoint{2.808038in}{3.468565in}}%
\pgfpathlineto{\pgfqpoint{3.583038in}{3.468565in}}%
\pgfpathlineto{\pgfqpoint{3.583038in}{4.019322in}}%
\pgfpathlineto{\pgfqpoint{2.808038in}{4.019322in}}%
\pgfpathclose%
\pgfusepath{clip}%
\pgfsys@defobject{currentpattern}{\pgfqpoint{0in}{0in}}{\pgfqpoint{1in}{1in}}{%
\begin{pgfscope}%
\pgfpathrectangle{\pgfqpoint{0in}{0in}}{\pgfqpoint{1in}{1in}}%
\pgfusepath{clip}%
\pgfpathmoveto{\pgfqpoint{0.000000in}{0.055556in}}%
\pgfpathlineto{\pgfqpoint{-0.016327in}{0.022473in}}%
\pgfpathlineto{\pgfqpoint{-0.052836in}{0.017168in}}%
\pgfpathlineto{\pgfqpoint{-0.026418in}{-0.008584in}}%
\pgfpathlineto{\pgfqpoint{-0.032655in}{-0.044945in}}%
\pgfpathlineto{\pgfqpoint{-0.000000in}{-0.027778in}}%
\pgfpathlineto{\pgfqpoint{0.032655in}{-0.044945in}}%
\pgfpathlineto{\pgfqpoint{0.026418in}{-0.008584in}}%
\pgfpathlineto{\pgfqpoint{0.052836in}{0.017168in}}%
\pgfpathlineto{\pgfqpoint{0.016327in}{0.022473in}}%
\pgfpathlineto{\pgfqpoint{0.000000in}{0.055556in}}%
\pgfpathmoveto{\pgfqpoint{0.166667in}{0.055556in}}%
\pgfpathlineto{\pgfqpoint{0.150339in}{0.022473in}}%
\pgfpathlineto{\pgfqpoint{0.113830in}{0.017168in}}%
\pgfpathlineto{\pgfqpoint{0.140248in}{-0.008584in}}%
\pgfpathlineto{\pgfqpoint{0.134012in}{-0.044945in}}%
\pgfpathlineto{\pgfqpoint{0.166667in}{-0.027778in}}%
\pgfpathlineto{\pgfqpoint{0.199321in}{-0.044945in}}%
\pgfpathlineto{\pgfqpoint{0.193085in}{-0.008584in}}%
\pgfpathlineto{\pgfqpoint{0.219503in}{0.017168in}}%
\pgfpathlineto{\pgfqpoint{0.182994in}{0.022473in}}%
\pgfpathlineto{\pgfqpoint{0.166667in}{0.055556in}}%
\pgfpathmoveto{\pgfqpoint{0.333333in}{0.055556in}}%
\pgfpathlineto{\pgfqpoint{0.317006in}{0.022473in}}%
\pgfpathlineto{\pgfqpoint{0.280497in}{0.017168in}}%
\pgfpathlineto{\pgfqpoint{0.306915in}{-0.008584in}}%
\pgfpathlineto{\pgfqpoint{0.300679in}{-0.044945in}}%
\pgfpathlineto{\pgfqpoint{0.333333in}{-0.027778in}}%
\pgfpathlineto{\pgfqpoint{0.365988in}{-0.044945in}}%
\pgfpathlineto{\pgfqpoint{0.359752in}{-0.008584in}}%
\pgfpathlineto{\pgfqpoint{0.386170in}{0.017168in}}%
\pgfpathlineto{\pgfqpoint{0.349661in}{0.022473in}}%
\pgfpathlineto{\pgfqpoint{0.333333in}{0.055556in}}%
\pgfpathmoveto{\pgfqpoint{0.500000in}{0.055556in}}%
\pgfpathlineto{\pgfqpoint{0.483673in}{0.022473in}}%
\pgfpathlineto{\pgfqpoint{0.447164in}{0.017168in}}%
\pgfpathlineto{\pgfqpoint{0.473582in}{-0.008584in}}%
\pgfpathlineto{\pgfqpoint{0.467345in}{-0.044945in}}%
\pgfpathlineto{\pgfqpoint{0.500000in}{-0.027778in}}%
\pgfpathlineto{\pgfqpoint{0.532655in}{-0.044945in}}%
\pgfpathlineto{\pgfqpoint{0.526418in}{-0.008584in}}%
\pgfpathlineto{\pgfqpoint{0.552836in}{0.017168in}}%
\pgfpathlineto{\pgfqpoint{0.516327in}{0.022473in}}%
\pgfpathlineto{\pgfqpoint{0.500000in}{0.055556in}}%
\pgfpathmoveto{\pgfqpoint{0.666667in}{0.055556in}}%
\pgfpathlineto{\pgfqpoint{0.650339in}{0.022473in}}%
\pgfpathlineto{\pgfqpoint{0.613830in}{0.017168in}}%
\pgfpathlineto{\pgfqpoint{0.640248in}{-0.008584in}}%
\pgfpathlineto{\pgfqpoint{0.634012in}{-0.044945in}}%
\pgfpathlineto{\pgfqpoint{0.666667in}{-0.027778in}}%
\pgfpathlineto{\pgfqpoint{0.699321in}{-0.044945in}}%
\pgfpathlineto{\pgfqpoint{0.693085in}{-0.008584in}}%
\pgfpathlineto{\pgfqpoint{0.719503in}{0.017168in}}%
\pgfpathlineto{\pgfqpoint{0.682994in}{0.022473in}}%
\pgfpathlineto{\pgfqpoint{0.666667in}{0.055556in}}%
\pgfpathmoveto{\pgfqpoint{0.833333in}{0.055556in}}%
\pgfpathlineto{\pgfqpoint{0.817006in}{0.022473in}}%
\pgfpathlineto{\pgfqpoint{0.780497in}{0.017168in}}%
\pgfpathlineto{\pgfqpoint{0.806915in}{-0.008584in}}%
\pgfpathlineto{\pgfqpoint{0.800679in}{-0.044945in}}%
\pgfpathlineto{\pgfqpoint{0.833333in}{-0.027778in}}%
\pgfpathlineto{\pgfqpoint{0.865988in}{-0.044945in}}%
\pgfpathlineto{\pgfqpoint{0.859752in}{-0.008584in}}%
\pgfpathlineto{\pgfqpoint{0.886170in}{0.017168in}}%
\pgfpathlineto{\pgfqpoint{0.849661in}{0.022473in}}%
\pgfpathlineto{\pgfqpoint{0.833333in}{0.055556in}}%
\pgfpathmoveto{\pgfqpoint{1.000000in}{0.055556in}}%
\pgfpathlineto{\pgfqpoint{0.983673in}{0.022473in}}%
\pgfpathlineto{\pgfqpoint{0.947164in}{0.017168in}}%
\pgfpathlineto{\pgfqpoint{0.973582in}{-0.008584in}}%
\pgfpathlineto{\pgfqpoint{0.967345in}{-0.044945in}}%
\pgfpathlineto{\pgfqpoint{1.000000in}{-0.027778in}}%
\pgfpathlineto{\pgfqpoint{1.032655in}{-0.044945in}}%
\pgfpathlineto{\pgfqpoint{1.026418in}{-0.008584in}}%
\pgfpathlineto{\pgfqpoint{1.052836in}{0.017168in}}%
\pgfpathlineto{\pgfqpoint{1.016327in}{0.022473in}}%
\pgfpathlineto{\pgfqpoint{1.000000in}{0.055556in}}%
\pgfpathmoveto{\pgfqpoint{0.083333in}{0.222222in}}%
\pgfpathlineto{\pgfqpoint{0.067006in}{0.189139in}}%
\pgfpathlineto{\pgfqpoint{0.030497in}{0.183834in}}%
\pgfpathlineto{\pgfqpoint{0.056915in}{0.158083in}}%
\pgfpathlineto{\pgfqpoint{0.050679in}{0.121721in}}%
\pgfpathlineto{\pgfqpoint{0.083333in}{0.138889in}}%
\pgfpathlineto{\pgfqpoint{0.115988in}{0.121721in}}%
\pgfpathlineto{\pgfqpoint{0.109752in}{0.158083in}}%
\pgfpathlineto{\pgfqpoint{0.136170in}{0.183834in}}%
\pgfpathlineto{\pgfqpoint{0.099661in}{0.189139in}}%
\pgfpathlineto{\pgfqpoint{0.083333in}{0.222222in}}%
\pgfpathmoveto{\pgfqpoint{0.250000in}{0.222222in}}%
\pgfpathlineto{\pgfqpoint{0.233673in}{0.189139in}}%
\pgfpathlineto{\pgfqpoint{0.197164in}{0.183834in}}%
\pgfpathlineto{\pgfqpoint{0.223582in}{0.158083in}}%
\pgfpathlineto{\pgfqpoint{0.217345in}{0.121721in}}%
\pgfpathlineto{\pgfqpoint{0.250000in}{0.138889in}}%
\pgfpathlineto{\pgfqpoint{0.282655in}{0.121721in}}%
\pgfpathlineto{\pgfqpoint{0.276418in}{0.158083in}}%
\pgfpathlineto{\pgfqpoint{0.302836in}{0.183834in}}%
\pgfpathlineto{\pgfqpoint{0.266327in}{0.189139in}}%
\pgfpathlineto{\pgfqpoint{0.250000in}{0.222222in}}%
\pgfpathmoveto{\pgfqpoint{0.416667in}{0.222222in}}%
\pgfpathlineto{\pgfqpoint{0.400339in}{0.189139in}}%
\pgfpathlineto{\pgfqpoint{0.363830in}{0.183834in}}%
\pgfpathlineto{\pgfqpoint{0.390248in}{0.158083in}}%
\pgfpathlineto{\pgfqpoint{0.384012in}{0.121721in}}%
\pgfpathlineto{\pgfqpoint{0.416667in}{0.138889in}}%
\pgfpathlineto{\pgfqpoint{0.449321in}{0.121721in}}%
\pgfpathlineto{\pgfqpoint{0.443085in}{0.158083in}}%
\pgfpathlineto{\pgfqpoint{0.469503in}{0.183834in}}%
\pgfpathlineto{\pgfqpoint{0.432994in}{0.189139in}}%
\pgfpathlineto{\pgfqpoint{0.416667in}{0.222222in}}%
\pgfpathmoveto{\pgfqpoint{0.583333in}{0.222222in}}%
\pgfpathlineto{\pgfqpoint{0.567006in}{0.189139in}}%
\pgfpathlineto{\pgfqpoint{0.530497in}{0.183834in}}%
\pgfpathlineto{\pgfqpoint{0.556915in}{0.158083in}}%
\pgfpathlineto{\pgfqpoint{0.550679in}{0.121721in}}%
\pgfpathlineto{\pgfqpoint{0.583333in}{0.138889in}}%
\pgfpathlineto{\pgfqpoint{0.615988in}{0.121721in}}%
\pgfpathlineto{\pgfqpoint{0.609752in}{0.158083in}}%
\pgfpathlineto{\pgfqpoint{0.636170in}{0.183834in}}%
\pgfpathlineto{\pgfqpoint{0.599661in}{0.189139in}}%
\pgfpathlineto{\pgfqpoint{0.583333in}{0.222222in}}%
\pgfpathmoveto{\pgfqpoint{0.750000in}{0.222222in}}%
\pgfpathlineto{\pgfqpoint{0.733673in}{0.189139in}}%
\pgfpathlineto{\pgfqpoint{0.697164in}{0.183834in}}%
\pgfpathlineto{\pgfqpoint{0.723582in}{0.158083in}}%
\pgfpathlineto{\pgfqpoint{0.717345in}{0.121721in}}%
\pgfpathlineto{\pgfqpoint{0.750000in}{0.138889in}}%
\pgfpathlineto{\pgfqpoint{0.782655in}{0.121721in}}%
\pgfpathlineto{\pgfqpoint{0.776418in}{0.158083in}}%
\pgfpathlineto{\pgfqpoint{0.802836in}{0.183834in}}%
\pgfpathlineto{\pgfqpoint{0.766327in}{0.189139in}}%
\pgfpathlineto{\pgfqpoint{0.750000in}{0.222222in}}%
\pgfpathmoveto{\pgfqpoint{0.916667in}{0.222222in}}%
\pgfpathlineto{\pgfqpoint{0.900339in}{0.189139in}}%
\pgfpathlineto{\pgfqpoint{0.863830in}{0.183834in}}%
\pgfpathlineto{\pgfqpoint{0.890248in}{0.158083in}}%
\pgfpathlineto{\pgfqpoint{0.884012in}{0.121721in}}%
\pgfpathlineto{\pgfqpoint{0.916667in}{0.138889in}}%
\pgfpathlineto{\pgfqpoint{0.949321in}{0.121721in}}%
\pgfpathlineto{\pgfqpoint{0.943085in}{0.158083in}}%
\pgfpathlineto{\pgfqpoint{0.969503in}{0.183834in}}%
\pgfpathlineto{\pgfqpoint{0.932994in}{0.189139in}}%
\pgfpathlineto{\pgfqpoint{0.916667in}{0.222222in}}%
\pgfpathmoveto{\pgfqpoint{0.000000in}{0.388889in}}%
\pgfpathlineto{\pgfqpoint{-0.016327in}{0.355806in}}%
\pgfpathlineto{\pgfqpoint{-0.052836in}{0.350501in}}%
\pgfpathlineto{\pgfqpoint{-0.026418in}{0.324750in}}%
\pgfpathlineto{\pgfqpoint{-0.032655in}{0.288388in}}%
\pgfpathlineto{\pgfqpoint{-0.000000in}{0.305556in}}%
\pgfpathlineto{\pgfqpoint{0.032655in}{0.288388in}}%
\pgfpathlineto{\pgfqpoint{0.026418in}{0.324750in}}%
\pgfpathlineto{\pgfqpoint{0.052836in}{0.350501in}}%
\pgfpathlineto{\pgfqpoint{0.016327in}{0.355806in}}%
\pgfpathlineto{\pgfqpoint{0.000000in}{0.388889in}}%
\pgfpathmoveto{\pgfqpoint{0.166667in}{0.388889in}}%
\pgfpathlineto{\pgfqpoint{0.150339in}{0.355806in}}%
\pgfpathlineto{\pgfqpoint{0.113830in}{0.350501in}}%
\pgfpathlineto{\pgfqpoint{0.140248in}{0.324750in}}%
\pgfpathlineto{\pgfqpoint{0.134012in}{0.288388in}}%
\pgfpathlineto{\pgfqpoint{0.166667in}{0.305556in}}%
\pgfpathlineto{\pgfqpoint{0.199321in}{0.288388in}}%
\pgfpathlineto{\pgfqpoint{0.193085in}{0.324750in}}%
\pgfpathlineto{\pgfqpoint{0.219503in}{0.350501in}}%
\pgfpathlineto{\pgfqpoint{0.182994in}{0.355806in}}%
\pgfpathlineto{\pgfqpoint{0.166667in}{0.388889in}}%
\pgfpathmoveto{\pgfqpoint{0.333333in}{0.388889in}}%
\pgfpathlineto{\pgfqpoint{0.317006in}{0.355806in}}%
\pgfpathlineto{\pgfqpoint{0.280497in}{0.350501in}}%
\pgfpathlineto{\pgfqpoint{0.306915in}{0.324750in}}%
\pgfpathlineto{\pgfqpoint{0.300679in}{0.288388in}}%
\pgfpathlineto{\pgfqpoint{0.333333in}{0.305556in}}%
\pgfpathlineto{\pgfqpoint{0.365988in}{0.288388in}}%
\pgfpathlineto{\pgfqpoint{0.359752in}{0.324750in}}%
\pgfpathlineto{\pgfqpoint{0.386170in}{0.350501in}}%
\pgfpathlineto{\pgfqpoint{0.349661in}{0.355806in}}%
\pgfpathlineto{\pgfqpoint{0.333333in}{0.388889in}}%
\pgfpathmoveto{\pgfqpoint{0.500000in}{0.388889in}}%
\pgfpathlineto{\pgfqpoint{0.483673in}{0.355806in}}%
\pgfpathlineto{\pgfqpoint{0.447164in}{0.350501in}}%
\pgfpathlineto{\pgfqpoint{0.473582in}{0.324750in}}%
\pgfpathlineto{\pgfqpoint{0.467345in}{0.288388in}}%
\pgfpathlineto{\pgfqpoint{0.500000in}{0.305556in}}%
\pgfpathlineto{\pgfqpoint{0.532655in}{0.288388in}}%
\pgfpathlineto{\pgfqpoint{0.526418in}{0.324750in}}%
\pgfpathlineto{\pgfqpoint{0.552836in}{0.350501in}}%
\pgfpathlineto{\pgfqpoint{0.516327in}{0.355806in}}%
\pgfpathlineto{\pgfqpoint{0.500000in}{0.388889in}}%
\pgfpathmoveto{\pgfqpoint{0.666667in}{0.388889in}}%
\pgfpathlineto{\pgfqpoint{0.650339in}{0.355806in}}%
\pgfpathlineto{\pgfqpoint{0.613830in}{0.350501in}}%
\pgfpathlineto{\pgfqpoint{0.640248in}{0.324750in}}%
\pgfpathlineto{\pgfqpoint{0.634012in}{0.288388in}}%
\pgfpathlineto{\pgfqpoint{0.666667in}{0.305556in}}%
\pgfpathlineto{\pgfqpoint{0.699321in}{0.288388in}}%
\pgfpathlineto{\pgfqpoint{0.693085in}{0.324750in}}%
\pgfpathlineto{\pgfqpoint{0.719503in}{0.350501in}}%
\pgfpathlineto{\pgfqpoint{0.682994in}{0.355806in}}%
\pgfpathlineto{\pgfqpoint{0.666667in}{0.388889in}}%
\pgfpathmoveto{\pgfqpoint{0.833333in}{0.388889in}}%
\pgfpathlineto{\pgfqpoint{0.817006in}{0.355806in}}%
\pgfpathlineto{\pgfqpoint{0.780497in}{0.350501in}}%
\pgfpathlineto{\pgfqpoint{0.806915in}{0.324750in}}%
\pgfpathlineto{\pgfqpoint{0.800679in}{0.288388in}}%
\pgfpathlineto{\pgfqpoint{0.833333in}{0.305556in}}%
\pgfpathlineto{\pgfqpoint{0.865988in}{0.288388in}}%
\pgfpathlineto{\pgfqpoint{0.859752in}{0.324750in}}%
\pgfpathlineto{\pgfqpoint{0.886170in}{0.350501in}}%
\pgfpathlineto{\pgfqpoint{0.849661in}{0.355806in}}%
\pgfpathlineto{\pgfqpoint{0.833333in}{0.388889in}}%
\pgfpathmoveto{\pgfqpoint{1.000000in}{0.388889in}}%
\pgfpathlineto{\pgfqpoint{0.983673in}{0.355806in}}%
\pgfpathlineto{\pgfqpoint{0.947164in}{0.350501in}}%
\pgfpathlineto{\pgfqpoint{0.973582in}{0.324750in}}%
\pgfpathlineto{\pgfqpoint{0.967345in}{0.288388in}}%
\pgfpathlineto{\pgfqpoint{1.000000in}{0.305556in}}%
\pgfpathlineto{\pgfqpoint{1.032655in}{0.288388in}}%
\pgfpathlineto{\pgfqpoint{1.026418in}{0.324750in}}%
\pgfpathlineto{\pgfqpoint{1.052836in}{0.350501in}}%
\pgfpathlineto{\pgfqpoint{1.016327in}{0.355806in}}%
\pgfpathlineto{\pgfqpoint{1.000000in}{0.388889in}}%
\pgfpathmoveto{\pgfqpoint{0.083333in}{0.555556in}}%
\pgfpathlineto{\pgfqpoint{0.067006in}{0.522473in}}%
\pgfpathlineto{\pgfqpoint{0.030497in}{0.517168in}}%
\pgfpathlineto{\pgfqpoint{0.056915in}{0.491416in}}%
\pgfpathlineto{\pgfqpoint{0.050679in}{0.455055in}}%
\pgfpathlineto{\pgfqpoint{0.083333in}{0.472222in}}%
\pgfpathlineto{\pgfqpoint{0.115988in}{0.455055in}}%
\pgfpathlineto{\pgfqpoint{0.109752in}{0.491416in}}%
\pgfpathlineto{\pgfqpoint{0.136170in}{0.517168in}}%
\pgfpathlineto{\pgfqpoint{0.099661in}{0.522473in}}%
\pgfpathlineto{\pgfqpoint{0.083333in}{0.555556in}}%
\pgfpathmoveto{\pgfqpoint{0.250000in}{0.555556in}}%
\pgfpathlineto{\pgfqpoint{0.233673in}{0.522473in}}%
\pgfpathlineto{\pgfqpoint{0.197164in}{0.517168in}}%
\pgfpathlineto{\pgfqpoint{0.223582in}{0.491416in}}%
\pgfpathlineto{\pgfqpoint{0.217345in}{0.455055in}}%
\pgfpathlineto{\pgfqpoint{0.250000in}{0.472222in}}%
\pgfpathlineto{\pgfqpoint{0.282655in}{0.455055in}}%
\pgfpathlineto{\pgfqpoint{0.276418in}{0.491416in}}%
\pgfpathlineto{\pgfqpoint{0.302836in}{0.517168in}}%
\pgfpathlineto{\pgfqpoint{0.266327in}{0.522473in}}%
\pgfpathlineto{\pgfqpoint{0.250000in}{0.555556in}}%
\pgfpathmoveto{\pgfqpoint{0.416667in}{0.555556in}}%
\pgfpathlineto{\pgfqpoint{0.400339in}{0.522473in}}%
\pgfpathlineto{\pgfqpoint{0.363830in}{0.517168in}}%
\pgfpathlineto{\pgfqpoint{0.390248in}{0.491416in}}%
\pgfpathlineto{\pgfqpoint{0.384012in}{0.455055in}}%
\pgfpathlineto{\pgfqpoint{0.416667in}{0.472222in}}%
\pgfpathlineto{\pgfqpoint{0.449321in}{0.455055in}}%
\pgfpathlineto{\pgfqpoint{0.443085in}{0.491416in}}%
\pgfpathlineto{\pgfqpoint{0.469503in}{0.517168in}}%
\pgfpathlineto{\pgfqpoint{0.432994in}{0.522473in}}%
\pgfpathlineto{\pgfqpoint{0.416667in}{0.555556in}}%
\pgfpathmoveto{\pgfqpoint{0.583333in}{0.555556in}}%
\pgfpathlineto{\pgfqpoint{0.567006in}{0.522473in}}%
\pgfpathlineto{\pgfqpoint{0.530497in}{0.517168in}}%
\pgfpathlineto{\pgfqpoint{0.556915in}{0.491416in}}%
\pgfpathlineto{\pgfqpoint{0.550679in}{0.455055in}}%
\pgfpathlineto{\pgfqpoint{0.583333in}{0.472222in}}%
\pgfpathlineto{\pgfqpoint{0.615988in}{0.455055in}}%
\pgfpathlineto{\pgfqpoint{0.609752in}{0.491416in}}%
\pgfpathlineto{\pgfqpoint{0.636170in}{0.517168in}}%
\pgfpathlineto{\pgfqpoint{0.599661in}{0.522473in}}%
\pgfpathlineto{\pgfqpoint{0.583333in}{0.555556in}}%
\pgfpathmoveto{\pgfqpoint{0.750000in}{0.555556in}}%
\pgfpathlineto{\pgfqpoint{0.733673in}{0.522473in}}%
\pgfpathlineto{\pgfqpoint{0.697164in}{0.517168in}}%
\pgfpathlineto{\pgfqpoint{0.723582in}{0.491416in}}%
\pgfpathlineto{\pgfqpoint{0.717345in}{0.455055in}}%
\pgfpathlineto{\pgfqpoint{0.750000in}{0.472222in}}%
\pgfpathlineto{\pgfqpoint{0.782655in}{0.455055in}}%
\pgfpathlineto{\pgfqpoint{0.776418in}{0.491416in}}%
\pgfpathlineto{\pgfqpoint{0.802836in}{0.517168in}}%
\pgfpathlineto{\pgfqpoint{0.766327in}{0.522473in}}%
\pgfpathlineto{\pgfqpoint{0.750000in}{0.555556in}}%
\pgfpathmoveto{\pgfqpoint{0.916667in}{0.555556in}}%
\pgfpathlineto{\pgfqpoint{0.900339in}{0.522473in}}%
\pgfpathlineto{\pgfqpoint{0.863830in}{0.517168in}}%
\pgfpathlineto{\pgfqpoint{0.890248in}{0.491416in}}%
\pgfpathlineto{\pgfqpoint{0.884012in}{0.455055in}}%
\pgfpathlineto{\pgfqpoint{0.916667in}{0.472222in}}%
\pgfpathlineto{\pgfqpoint{0.949321in}{0.455055in}}%
\pgfpathlineto{\pgfqpoint{0.943085in}{0.491416in}}%
\pgfpathlineto{\pgfqpoint{0.969503in}{0.517168in}}%
\pgfpathlineto{\pgfqpoint{0.932994in}{0.522473in}}%
\pgfpathlineto{\pgfqpoint{0.916667in}{0.555556in}}%
\pgfpathmoveto{\pgfqpoint{0.000000in}{0.722222in}}%
\pgfpathlineto{\pgfqpoint{-0.016327in}{0.689139in}}%
\pgfpathlineto{\pgfqpoint{-0.052836in}{0.683834in}}%
\pgfpathlineto{\pgfqpoint{-0.026418in}{0.658083in}}%
\pgfpathlineto{\pgfqpoint{-0.032655in}{0.621721in}}%
\pgfpathlineto{\pgfqpoint{-0.000000in}{0.638889in}}%
\pgfpathlineto{\pgfqpoint{0.032655in}{0.621721in}}%
\pgfpathlineto{\pgfqpoint{0.026418in}{0.658083in}}%
\pgfpathlineto{\pgfqpoint{0.052836in}{0.683834in}}%
\pgfpathlineto{\pgfqpoint{0.016327in}{0.689139in}}%
\pgfpathlineto{\pgfqpoint{0.000000in}{0.722222in}}%
\pgfpathmoveto{\pgfqpoint{0.166667in}{0.722222in}}%
\pgfpathlineto{\pgfqpoint{0.150339in}{0.689139in}}%
\pgfpathlineto{\pgfqpoint{0.113830in}{0.683834in}}%
\pgfpathlineto{\pgfqpoint{0.140248in}{0.658083in}}%
\pgfpathlineto{\pgfqpoint{0.134012in}{0.621721in}}%
\pgfpathlineto{\pgfqpoint{0.166667in}{0.638889in}}%
\pgfpathlineto{\pgfqpoint{0.199321in}{0.621721in}}%
\pgfpathlineto{\pgfqpoint{0.193085in}{0.658083in}}%
\pgfpathlineto{\pgfqpoint{0.219503in}{0.683834in}}%
\pgfpathlineto{\pgfqpoint{0.182994in}{0.689139in}}%
\pgfpathlineto{\pgfqpoint{0.166667in}{0.722222in}}%
\pgfpathmoveto{\pgfqpoint{0.333333in}{0.722222in}}%
\pgfpathlineto{\pgfqpoint{0.317006in}{0.689139in}}%
\pgfpathlineto{\pgfqpoint{0.280497in}{0.683834in}}%
\pgfpathlineto{\pgfqpoint{0.306915in}{0.658083in}}%
\pgfpathlineto{\pgfqpoint{0.300679in}{0.621721in}}%
\pgfpathlineto{\pgfqpoint{0.333333in}{0.638889in}}%
\pgfpathlineto{\pgfqpoint{0.365988in}{0.621721in}}%
\pgfpathlineto{\pgfqpoint{0.359752in}{0.658083in}}%
\pgfpathlineto{\pgfqpoint{0.386170in}{0.683834in}}%
\pgfpathlineto{\pgfqpoint{0.349661in}{0.689139in}}%
\pgfpathlineto{\pgfqpoint{0.333333in}{0.722222in}}%
\pgfpathmoveto{\pgfqpoint{0.500000in}{0.722222in}}%
\pgfpathlineto{\pgfqpoint{0.483673in}{0.689139in}}%
\pgfpathlineto{\pgfqpoint{0.447164in}{0.683834in}}%
\pgfpathlineto{\pgfqpoint{0.473582in}{0.658083in}}%
\pgfpathlineto{\pgfqpoint{0.467345in}{0.621721in}}%
\pgfpathlineto{\pgfqpoint{0.500000in}{0.638889in}}%
\pgfpathlineto{\pgfqpoint{0.532655in}{0.621721in}}%
\pgfpathlineto{\pgfqpoint{0.526418in}{0.658083in}}%
\pgfpathlineto{\pgfqpoint{0.552836in}{0.683834in}}%
\pgfpathlineto{\pgfqpoint{0.516327in}{0.689139in}}%
\pgfpathlineto{\pgfqpoint{0.500000in}{0.722222in}}%
\pgfpathmoveto{\pgfqpoint{0.666667in}{0.722222in}}%
\pgfpathlineto{\pgfqpoint{0.650339in}{0.689139in}}%
\pgfpathlineto{\pgfqpoint{0.613830in}{0.683834in}}%
\pgfpathlineto{\pgfqpoint{0.640248in}{0.658083in}}%
\pgfpathlineto{\pgfqpoint{0.634012in}{0.621721in}}%
\pgfpathlineto{\pgfqpoint{0.666667in}{0.638889in}}%
\pgfpathlineto{\pgfqpoint{0.699321in}{0.621721in}}%
\pgfpathlineto{\pgfqpoint{0.693085in}{0.658083in}}%
\pgfpathlineto{\pgfqpoint{0.719503in}{0.683834in}}%
\pgfpathlineto{\pgfqpoint{0.682994in}{0.689139in}}%
\pgfpathlineto{\pgfqpoint{0.666667in}{0.722222in}}%
\pgfpathmoveto{\pgfqpoint{0.833333in}{0.722222in}}%
\pgfpathlineto{\pgfqpoint{0.817006in}{0.689139in}}%
\pgfpathlineto{\pgfqpoint{0.780497in}{0.683834in}}%
\pgfpathlineto{\pgfqpoint{0.806915in}{0.658083in}}%
\pgfpathlineto{\pgfqpoint{0.800679in}{0.621721in}}%
\pgfpathlineto{\pgfqpoint{0.833333in}{0.638889in}}%
\pgfpathlineto{\pgfqpoint{0.865988in}{0.621721in}}%
\pgfpathlineto{\pgfqpoint{0.859752in}{0.658083in}}%
\pgfpathlineto{\pgfqpoint{0.886170in}{0.683834in}}%
\pgfpathlineto{\pgfqpoint{0.849661in}{0.689139in}}%
\pgfpathlineto{\pgfqpoint{0.833333in}{0.722222in}}%
\pgfpathmoveto{\pgfqpoint{1.000000in}{0.722222in}}%
\pgfpathlineto{\pgfqpoint{0.983673in}{0.689139in}}%
\pgfpathlineto{\pgfqpoint{0.947164in}{0.683834in}}%
\pgfpathlineto{\pgfqpoint{0.973582in}{0.658083in}}%
\pgfpathlineto{\pgfqpoint{0.967345in}{0.621721in}}%
\pgfpathlineto{\pgfqpoint{1.000000in}{0.638889in}}%
\pgfpathlineto{\pgfqpoint{1.032655in}{0.621721in}}%
\pgfpathlineto{\pgfqpoint{1.026418in}{0.658083in}}%
\pgfpathlineto{\pgfqpoint{1.052836in}{0.683834in}}%
\pgfpathlineto{\pgfqpoint{1.016327in}{0.689139in}}%
\pgfpathlineto{\pgfqpoint{1.000000in}{0.722222in}}%
\pgfpathmoveto{\pgfqpoint{0.083333in}{0.888889in}}%
\pgfpathlineto{\pgfqpoint{0.067006in}{0.855806in}}%
\pgfpathlineto{\pgfqpoint{0.030497in}{0.850501in}}%
\pgfpathlineto{\pgfqpoint{0.056915in}{0.824750in}}%
\pgfpathlineto{\pgfqpoint{0.050679in}{0.788388in}}%
\pgfpathlineto{\pgfqpoint{0.083333in}{0.805556in}}%
\pgfpathlineto{\pgfqpoint{0.115988in}{0.788388in}}%
\pgfpathlineto{\pgfqpoint{0.109752in}{0.824750in}}%
\pgfpathlineto{\pgfqpoint{0.136170in}{0.850501in}}%
\pgfpathlineto{\pgfqpoint{0.099661in}{0.855806in}}%
\pgfpathlineto{\pgfqpoint{0.083333in}{0.888889in}}%
\pgfpathmoveto{\pgfqpoint{0.250000in}{0.888889in}}%
\pgfpathlineto{\pgfqpoint{0.233673in}{0.855806in}}%
\pgfpathlineto{\pgfqpoint{0.197164in}{0.850501in}}%
\pgfpathlineto{\pgfqpoint{0.223582in}{0.824750in}}%
\pgfpathlineto{\pgfqpoint{0.217345in}{0.788388in}}%
\pgfpathlineto{\pgfqpoint{0.250000in}{0.805556in}}%
\pgfpathlineto{\pgfqpoint{0.282655in}{0.788388in}}%
\pgfpathlineto{\pgfqpoint{0.276418in}{0.824750in}}%
\pgfpathlineto{\pgfqpoint{0.302836in}{0.850501in}}%
\pgfpathlineto{\pgfqpoint{0.266327in}{0.855806in}}%
\pgfpathlineto{\pgfqpoint{0.250000in}{0.888889in}}%
\pgfpathmoveto{\pgfqpoint{0.416667in}{0.888889in}}%
\pgfpathlineto{\pgfqpoint{0.400339in}{0.855806in}}%
\pgfpathlineto{\pgfqpoint{0.363830in}{0.850501in}}%
\pgfpathlineto{\pgfqpoint{0.390248in}{0.824750in}}%
\pgfpathlineto{\pgfqpoint{0.384012in}{0.788388in}}%
\pgfpathlineto{\pgfqpoint{0.416667in}{0.805556in}}%
\pgfpathlineto{\pgfqpoint{0.449321in}{0.788388in}}%
\pgfpathlineto{\pgfqpoint{0.443085in}{0.824750in}}%
\pgfpathlineto{\pgfqpoint{0.469503in}{0.850501in}}%
\pgfpathlineto{\pgfqpoint{0.432994in}{0.855806in}}%
\pgfpathlineto{\pgfqpoint{0.416667in}{0.888889in}}%
\pgfpathmoveto{\pgfqpoint{0.583333in}{0.888889in}}%
\pgfpathlineto{\pgfqpoint{0.567006in}{0.855806in}}%
\pgfpathlineto{\pgfqpoint{0.530497in}{0.850501in}}%
\pgfpathlineto{\pgfqpoint{0.556915in}{0.824750in}}%
\pgfpathlineto{\pgfqpoint{0.550679in}{0.788388in}}%
\pgfpathlineto{\pgfqpoint{0.583333in}{0.805556in}}%
\pgfpathlineto{\pgfqpoint{0.615988in}{0.788388in}}%
\pgfpathlineto{\pgfqpoint{0.609752in}{0.824750in}}%
\pgfpathlineto{\pgfqpoint{0.636170in}{0.850501in}}%
\pgfpathlineto{\pgfqpoint{0.599661in}{0.855806in}}%
\pgfpathlineto{\pgfqpoint{0.583333in}{0.888889in}}%
\pgfpathmoveto{\pgfqpoint{0.750000in}{0.888889in}}%
\pgfpathlineto{\pgfqpoint{0.733673in}{0.855806in}}%
\pgfpathlineto{\pgfqpoint{0.697164in}{0.850501in}}%
\pgfpathlineto{\pgfqpoint{0.723582in}{0.824750in}}%
\pgfpathlineto{\pgfqpoint{0.717345in}{0.788388in}}%
\pgfpathlineto{\pgfqpoint{0.750000in}{0.805556in}}%
\pgfpathlineto{\pgfqpoint{0.782655in}{0.788388in}}%
\pgfpathlineto{\pgfqpoint{0.776418in}{0.824750in}}%
\pgfpathlineto{\pgfqpoint{0.802836in}{0.850501in}}%
\pgfpathlineto{\pgfqpoint{0.766327in}{0.855806in}}%
\pgfpathlineto{\pgfqpoint{0.750000in}{0.888889in}}%
\pgfpathmoveto{\pgfqpoint{0.916667in}{0.888889in}}%
\pgfpathlineto{\pgfqpoint{0.900339in}{0.855806in}}%
\pgfpathlineto{\pgfqpoint{0.863830in}{0.850501in}}%
\pgfpathlineto{\pgfqpoint{0.890248in}{0.824750in}}%
\pgfpathlineto{\pgfqpoint{0.884012in}{0.788388in}}%
\pgfpathlineto{\pgfqpoint{0.916667in}{0.805556in}}%
\pgfpathlineto{\pgfqpoint{0.949321in}{0.788388in}}%
\pgfpathlineto{\pgfqpoint{0.943085in}{0.824750in}}%
\pgfpathlineto{\pgfqpoint{0.969503in}{0.850501in}}%
\pgfpathlineto{\pgfqpoint{0.932994in}{0.855806in}}%
\pgfpathlineto{\pgfqpoint{0.916667in}{0.888889in}}%
\pgfpathmoveto{\pgfqpoint{0.000000in}{1.055556in}}%
\pgfpathlineto{\pgfqpoint{-0.016327in}{1.022473in}}%
\pgfpathlineto{\pgfqpoint{-0.052836in}{1.017168in}}%
\pgfpathlineto{\pgfqpoint{-0.026418in}{0.991416in}}%
\pgfpathlineto{\pgfqpoint{-0.032655in}{0.955055in}}%
\pgfpathlineto{\pgfqpoint{-0.000000in}{0.972222in}}%
\pgfpathlineto{\pgfqpoint{0.032655in}{0.955055in}}%
\pgfpathlineto{\pgfqpoint{0.026418in}{0.991416in}}%
\pgfpathlineto{\pgfqpoint{0.052836in}{1.017168in}}%
\pgfpathlineto{\pgfqpoint{0.016327in}{1.022473in}}%
\pgfpathlineto{\pgfqpoint{0.000000in}{1.055556in}}%
\pgfpathmoveto{\pgfqpoint{0.166667in}{1.055556in}}%
\pgfpathlineto{\pgfqpoint{0.150339in}{1.022473in}}%
\pgfpathlineto{\pgfqpoint{0.113830in}{1.017168in}}%
\pgfpathlineto{\pgfqpoint{0.140248in}{0.991416in}}%
\pgfpathlineto{\pgfqpoint{0.134012in}{0.955055in}}%
\pgfpathlineto{\pgfqpoint{0.166667in}{0.972222in}}%
\pgfpathlineto{\pgfqpoint{0.199321in}{0.955055in}}%
\pgfpathlineto{\pgfqpoint{0.193085in}{0.991416in}}%
\pgfpathlineto{\pgfqpoint{0.219503in}{1.017168in}}%
\pgfpathlineto{\pgfqpoint{0.182994in}{1.022473in}}%
\pgfpathlineto{\pgfqpoint{0.166667in}{1.055556in}}%
\pgfpathmoveto{\pgfqpoint{0.333333in}{1.055556in}}%
\pgfpathlineto{\pgfqpoint{0.317006in}{1.022473in}}%
\pgfpathlineto{\pgfqpoint{0.280497in}{1.017168in}}%
\pgfpathlineto{\pgfqpoint{0.306915in}{0.991416in}}%
\pgfpathlineto{\pgfqpoint{0.300679in}{0.955055in}}%
\pgfpathlineto{\pgfqpoint{0.333333in}{0.972222in}}%
\pgfpathlineto{\pgfqpoint{0.365988in}{0.955055in}}%
\pgfpathlineto{\pgfqpoint{0.359752in}{0.991416in}}%
\pgfpathlineto{\pgfqpoint{0.386170in}{1.017168in}}%
\pgfpathlineto{\pgfqpoint{0.349661in}{1.022473in}}%
\pgfpathlineto{\pgfqpoint{0.333333in}{1.055556in}}%
\pgfpathmoveto{\pgfqpoint{0.500000in}{1.055556in}}%
\pgfpathlineto{\pgfqpoint{0.483673in}{1.022473in}}%
\pgfpathlineto{\pgfqpoint{0.447164in}{1.017168in}}%
\pgfpathlineto{\pgfqpoint{0.473582in}{0.991416in}}%
\pgfpathlineto{\pgfqpoint{0.467345in}{0.955055in}}%
\pgfpathlineto{\pgfqpoint{0.500000in}{0.972222in}}%
\pgfpathlineto{\pgfqpoint{0.532655in}{0.955055in}}%
\pgfpathlineto{\pgfqpoint{0.526418in}{0.991416in}}%
\pgfpathlineto{\pgfqpoint{0.552836in}{1.017168in}}%
\pgfpathlineto{\pgfqpoint{0.516327in}{1.022473in}}%
\pgfpathlineto{\pgfqpoint{0.500000in}{1.055556in}}%
\pgfpathmoveto{\pgfqpoint{0.666667in}{1.055556in}}%
\pgfpathlineto{\pgfqpoint{0.650339in}{1.022473in}}%
\pgfpathlineto{\pgfqpoint{0.613830in}{1.017168in}}%
\pgfpathlineto{\pgfqpoint{0.640248in}{0.991416in}}%
\pgfpathlineto{\pgfqpoint{0.634012in}{0.955055in}}%
\pgfpathlineto{\pgfqpoint{0.666667in}{0.972222in}}%
\pgfpathlineto{\pgfqpoint{0.699321in}{0.955055in}}%
\pgfpathlineto{\pgfqpoint{0.693085in}{0.991416in}}%
\pgfpathlineto{\pgfqpoint{0.719503in}{1.017168in}}%
\pgfpathlineto{\pgfqpoint{0.682994in}{1.022473in}}%
\pgfpathlineto{\pgfqpoint{0.666667in}{1.055556in}}%
\pgfpathmoveto{\pgfqpoint{0.833333in}{1.055556in}}%
\pgfpathlineto{\pgfqpoint{0.817006in}{1.022473in}}%
\pgfpathlineto{\pgfqpoint{0.780497in}{1.017168in}}%
\pgfpathlineto{\pgfqpoint{0.806915in}{0.991416in}}%
\pgfpathlineto{\pgfqpoint{0.800679in}{0.955055in}}%
\pgfpathlineto{\pgfqpoint{0.833333in}{0.972222in}}%
\pgfpathlineto{\pgfqpoint{0.865988in}{0.955055in}}%
\pgfpathlineto{\pgfqpoint{0.859752in}{0.991416in}}%
\pgfpathlineto{\pgfqpoint{0.886170in}{1.017168in}}%
\pgfpathlineto{\pgfqpoint{0.849661in}{1.022473in}}%
\pgfpathlineto{\pgfqpoint{0.833333in}{1.055556in}}%
\pgfpathmoveto{\pgfqpoint{1.000000in}{1.055556in}}%
\pgfpathlineto{\pgfqpoint{0.983673in}{1.022473in}}%
\pgfpathlineto{\pgfqpoint{0.947164in}{1.017168in}}%
\pgfpathlineto{\pgfqpoint{0.973582in}{0.991416in}}%
\pgfpathlineto{\pgfqpoint{0.967345in}{0.955055in}}%
\pgfpathlineto{\pgfqpoint{1.000000in}{0.972222in}}%
\pgfpathlineto{\pgfqpoint{1.032655in}{0.955055in}}%
\pgfpathlineto{\pgfqpoint{1.026418in}{0.991416in}}%
\pgfpathlineto{\pgfqpoint{1.052836in}{1.017168in}}%
\pgfpathlineto{\pgfqpoint{1.016327in}{1.022473in}}%
\pgfpathlineto{\pgfqpoint{1.000000in}{1.055556in}}%
\pgfpathlineto{\pgfqpoint{1.000000in}{1.055556in}}%
\pgfusepath{stroke}%
\end{pgfscope}%
}%
\pgfsys@transformshift{2.808038in}{3.468565in}%
\pgfsys@useobject{currentpattern}{}%
\pgfsys@transformshift{1in}{0in}%
\pgfsys@transformshift{-1in}{0in}%
\pgfsys@transformshift{0in}{1in}%
\end{pgfscope}%
\begin{pgfscope}%
\pgfpathrectangle{\pgfqpoint{0.870538in}{0.637495in}}{\pgfqpoint{9.300000in}{9.060000in}}%
\pgfusepath{clip}%
\pgfsetbuttcap%
\pgfsetmiterjoin%
\definecolor{currentfill}{rgb}{1.000000,1.000000,0.000000}%
\pgfsetfillcolor{currentfill}%
\pgfsetfillopacity{0.990000}%
\pgfsetlinewidth{0.000000pt}%
\definecolor{currentstroke}{rgb}{0.000000,0.000000,0.000000}%
\pgfsetstrokecolor{currentstroke}%
\pgfsetstrokeopacity{0.990000}%
\pgfsetdash{}{0pt}%
\pgfpathmoveto{\pgfqpoint{4.358038in}{3.454505in}}%
\pgfpathlineto{\pgfqpoint{5.133038in}{3.454505in}}%
\pgfpathlineto{\pgfqpoint{5.133038in}{4.364668in}}%
\pgfpathlineto{\pgfqpoint{4.358038in}{4.364668in}}%
\pgfpathclose%
\pgfusepath{fill}%
\end{pgfscope}%
\begin{pgfscope}%
\pgfsetbuttcap%
\pgfsetmiterjoin%
\definecolor{currentfill}{rgb}{1.000000,1.000000,0.000000}%
\pgfsetfillcolor{currentfill}%
\pgfsetfillopacity{0.990000}%
\pgfsetlinewidth{0.000000pt}%
\definecolor{currentstroke}{rgb}{0.000000,0.000000,0.000000}%
\pgfsetstrokecolor{currentstroke}%
\pgfsetstrokeopacity{0.990000}%
\pgfsetdash{}{0pt}%
\pgfpathrectangle{\pgfqpoint{0.870538in}{0.637495in}}{\pgfqpoint{9.300000in}{9.060000in}}%
\pgfusepath{clip}%
\pgfpathmoveto{\pgfqpoint{4.358038in}{3.454505in}}%
\pgfpathlineto{\pgfqpoint{5.133038in}{3.454505in}}%
\pgfpathlineto{\pgfqpoint{5.133038in}{4.364668in}}%
\pgfpathlineto{\pgfqpoint{4.358038in}{4.364668in}}%
\pgfpathclose%
\pgfusepath{clip}%
\pgfsys@defobject{currentpattern}{\pgfqpoint{0in}{0in}}{\pgfqpoint{1in}{1in}}{%
\begin{pgfscope}%
\pgfpathrectangle{\pgfqpoint{0in}{0in}}{\pgfqpoint{1in}{1in}}%
\pgfusepath{clip}%
\pgfpathmoveto{\pgfqpoint{0.000000in}{0.055556in}}%
\pgfpathlineto{\pgfqpoint{-0.016327in}{0.022473in}}%
\pgfpathlineto{\pgfqpoint{-0.052836in}{0.017168in}}%
\pgfpathlineto{\pgfqpoint{-0.026418in}{-0.008584in}}%
\pgfpathlineto{\pgfqpoint{-0.032655in}{-0.044945in}}%
\pgfpathlineto{\pgfqpoint{-0.000000in}{-0.027778in}}%
\pgfpathlineto{\pgfqpoint{0.032655in}{-0.044945in}}%
\pgfpathlineto{\pgfqpoint{0.026418in}{-0.008584in}}%
\pgfpathlineto{\pgfqpoint{0.052836in}{0.017168in}}%
\pgfpathlineto{\pgfqpoint{0.016327in}{0.022473in}}%
\pgfpathlineto{\pgfqpoint{0.000000in}{0.055556in}}%
\pgfpathmoveto{\pgfqpoint{0.166667in}{0.055556in}}%
\pgfpathlineto{\pgfqpoint{0.150339in}{0.022473in}}%
\pgfpathlineto{\pgfqpoint{0.113830in}{0.017168in}}%
\pgfpathlineto{\pgfqpoint{0.140248in}{-0.008584in}}%
\pgfpathlineto{\pgfqpoint{0.134012in}{-0.044945in}}%
\pgfpathlineto{\pgfqpoint{0.166667in}{-0.027778in}}%
\pgfpathlineto{\pgfqpoint{0.199321in}{-0.044945in}}%
\pgfpathlineto{\pgfqpoint{0.193085in}{-0.008584in}}%
\pgfpathlineto{\pgfqpoint{0.219503in}{0.017168in}}%
\pgfpathlineto{\pgfqpoint{0.182994in}{0.022473in}}%
\pgfpathlineto{\pgfqpoint{0.166667in}{0.055556in}}%
\pgfpathmoveto{\pgfqpoint{0.333333in}{0.055556in}}%
\pgfpathlineto{\pgfqpoint{0.317006in}{0.022473in}}%
\pgfpathlineto{\pgfqpoint{0.280497in}{0.017168in}}%
\pgfpathlineto{\pgfqpoint{0.306915in}{-0.008584in}}%
\pgfpathlineto{\pgfqpoint{0.300679in}{-0.044945in}}%
\pgfpathlineto{\pgfqpoint{0.333333in}{-0.027778in}}%
\pgfpathlineto{\pgfqpoint{0.365988in}{-0.044945in}}%
\pgfpathlineto{\pgfqpoint{0.359752in}{-0.008584in}}%
\pgfpathlineto{\pgfqpoint{0.386170in}{0.017168in}}%
\pgfpathlineto{\pgfqpoint{0.349661in}{0.022473in}}%
\pgfpathlineto{\pgfqpoint{0.333333in}{0.055556in}}%
\pgfpathmoveto{\pgfqpoint{0.500000in}{0.055556in}}%
\pgfpathlineto{\pgfqpoint{0.483673in}{0.022473in}}%
\pgfpathlineto{\pgfqpoint{0.447164in}{0.017168in}}%
\pgfpathlineto{\pgfqpoint{0.473582in}{-0.008584in}}%
\pgfpathlineto{\pgfqpoint{0.467345in}{-0.044945in}}%
\pgfpathlineto{\pgfqpoint{0.500000in}{-0.027778in}}%
\pgfpathlineto{\pgfqpoint{0.532655in}{-0.044945in}}%
\pgfpathlineto{\pgfqpoint{0.526418in}{-0.008584in}}%
\pgfpathlineto{\pgfqpoint{0.552836in}{0.017168in}}%
\pgfpathlineto{\pgfqpoint{0.516327in}{0.022473in}}%
\pgfpathlineto{\pgfqpoint{0.500000in}{0.055556in}}%
\pgfpathmoveto{\pgfqpoint{0.666667in}{0.055556in}}%
\pgfpathlineto{\pgfqpoint{0.650339in}{0.022473in}}%
\pgfpathlineto{\pgfqpoint{0.613830in}{0.017168in}}%
\pgfpathlineto{\pgfqpoint{0.640248in}{-0.008584in}}%
\pgfpathlineto{\pgfqpoint{0.634012in}{-0.044945in}}%
\pgfpathlineto{\pgfqpoint{0.666667in}{-0.027778in}}%
\pgfpathlineto{\pgfqpoint{0.699321in}{-0.044945in}}%
\pgfpathlineto{\pgfqpoint{0.693085in}{-0.008584in}}%
\pgfpathlineto{\pgfqpoint{0.719503in}{0.017168in}}%
\pgfpathlineto{\pgfqpoint{0.682994in}{0.022473in}}%
\pgfpathlineto{\pgfqpoint{0.666667in}{0.055556in}}%
\pgfpathmoveto{\pgfqpoint{0.833333in}{0.055556in}}%
\pgfpathlineto{\pgfqpoint{0.817006in}{0.022473in}}%
\pgfpathlineto{\pgfqpoint{0.780497in}{0.017168in}}%
\pgfpathlineto{\pgfqpoint{0.806915in}{-0.008584in}}%
\pgfpathlineto{\pgfqpoint{0.800679in}{-0.044945in}}%
\pgfpathlineto{\pgfqpoint{0.833333in}{-0.027778in}}%
\pgfpathlineto{\pgfqpoint{0.865988in}{-0.044945in}}%
\pgfpathlineto{\pgfqpoint{0.859752in}{-0.008584in}}%
\pgfpathlineto{\pgfqpoint{0.886170in}{0.017168in}}%
\pgfpathlineto{\pgfqpoint{0.849661in}{0.022473in}}%
\pgfpathlineto{\pgfqpoint{0.833333in}{0.055556in}}%
\pgfpathmoveto{\pgfqpoint{1.000000in}{0.055556in}}%
\pgfpathlineto{\pgfqpoint{0.983673in}{0.022473in}}%
\pgfpathlineto{\pgfqpoint{0.947164in}{0.017168in}}%
\pgfpathlineto{\pgfqpoint{0.973582in}{-0.008584in}}%
\pgfpathlineto{\pgfqpoint{0.967345in}{-0.044945in}}%
\pgfpathlineto{\pgfqpoint{1.000000in}{-0.027778in}}%
\pgfpathlineto{\pgfqpoint{1.032655in}{-0.044945in}}%
\pgfpathlineto{\pgfqpoint{1.026418in}{-0.008584in}}%
\pgfpathlineto{\pgfqpoint{1.052836in}{0.017168in}}%
\pgfpathlineto{\pgfqpoint{1.016327in}{0.022473in}}%
\pgfpathlineto{\pgfqpoint{1.000000in}{0.055556in}}%
\pgfpathmoveto{\pgfqpoint{0.083333in}{0.222222in}}%
\pgfpathlineto{\pgfqpoint{0.067006in}{0.189139in}}%
\pgfpathlineto{\pgfqpoint{0.030497in}{0.183834in}}%
\pgfpathlineto{\pgfqpoint{0.056915in}{0.158083in}}%
\pgfpathlineto{\pgfqpoint{0.050679in}{0.121721in}}%
\pgfpathlineto{\pgfqpoint{0.083333in}{0.138889in}}%
\pgfpathlineto{\pgfqpoint{0.115988in}{0.121721in}}%
\pgfpathlineto{\pgfqpoint{0.109752in}{0.158083in}}%
\pgfpathlineto{\pgfqpoint{0.136170in}{0.183834in}}%
\pgfpathlineto{\pgfqpoint{0.099661in}{0.189139in}}%
\pgfpathlineto{\pgfqpoint{0.083333in}{0.222222in}}%
\pgfpathmoveto{\pgfqpoint{0.250000in}{0.222222in}}%
\pgfpathlineto{\pgfqpoint{0.233673in}{0.189139in}}%
\pgfpathlineto{\pgfqpoint{0.197164in}{0.183834in}}%
\pgfpathlineto{\pgfqpoint{0.223582in}{0.158083in}}%
\pgfpathlineto{\pgfqpoint{0.217345in}{0.121721in}}%
\pgfpathlineto{\pgfqpoint{0.250000in}{0.138889in}}%
\pgfpathlineto{\pgfqpoint{0.282655in}{0.121721in}}%
\pgfpathlineto{\pgfqpoint{0.276418in}{0.158083in}}%
\pgfpathlineto{\pgfqpoint{0.302836in}{0.183834in}}%
\pgfpathlineto{\pgfqpoint{0.266327in}{0.189139in}}%
\pgfpathlineto{\pgfqpoint{0.250000in}{0.222222in}}%
\pgfpathmoveto{\pgfqpoint{0.416667in}{0.222222in}}%
\pgfpathlineto{\pgfqpoint{0.400339in}{0.189139in}}%
\pgfpathlineto{\pgfqpoint{0.363830in}{0.183834in}}%
\pgfpathlineto{\pgfqpoint{0.390248in}{0.158083in}}%
\pgfpathlineto{\pgfqpoint{0.384012in}{0.121721in}}%
\pgfpathlineto{\pgfqpoint{0.416667in}{0.138889in}}%
\pgfpathlineto{\pgfqpoint{0.449321in}{0.121721in}}%
\pgfpathlineto{\pgfqpoint{0.443085in}{0.158083in}}%
\pgfpathlineto{\pgfqpoint{0.469503in}{0.183834in}}%
\pgfpathlineto{\pgfqpoint{0.432994in}{0.189139in}}%
\pgfpathlineto{\pgfqpoint{0.416667in}{0.222222in}}%
\pgfpathmoveto{\pgfqpoint{0.583333in}{0.222222in}}%
\pgfpathlineto{\pgfqpoint{0.567006in}{0.189139in}}%
\pgfpathlineto{\pgfqpoint{0.530497in}{0.183834in}}%
\pgfpathlineto{\pgfqpoint{0.556915in}{0.158083in}}%
\pgfpathlineto{\pgfqpoint{0.550679in}{0.121721in}}%
\pgfpathlineto{\pgfqpoint{0.583333in}{0.138889in}}%
\pgfpathlineto{\pgfqpoint{0.615988in}{0.121721in}}%
\pgfpathlineto{\pgfqpoint{0.609752in}{0.158083in}}%
\pgfpathlineto{\pgfqpoint{0.636170in}{0.183834in}}%
\pgfpathlineto{\pgfqpoint{0.599661in}{0.189139in}}%
\pgfpathlineto{\pgfqpoint{0.583333in}{0.222222in}}%
\pgfpathmoveto{\pgfqpoint{0.750000in}{0.222222in}}%
\pgfpathlineto{\pgfqpoint{0.733673in}{0.189139in}}%
\pgfpathlineto{\pgfqpoint{0.697164in}{0.183834in}}%
\pgfpathlineto{\pgfqpoint{0.723582in}{0.158083in}}%
\pgfpathlineto{\pgfqpoint{0.717345in}{0.121721in}}%
\pgfpathlineto{\pgfqpoint{0.750000in}{0.138889in}}%
\pgfpathlineto{\pgfqpoint{0.782655in}{0.121721in}}%
\pgfpathlineto{\pgfqpoint{0.776418in}{0.158083in}}%
\pgfpathlineto{\pgfqpoint{0.802836in}{0.183834in}}%
\pgfpathlineto{\pgfqpoint{0.766327in}{0.189139in}}%
\pgfpathlineto{\pgfqpoint{0.750000in}{0.222222in}}%
\pgfpathmoveto{\pgfqpoint{0.916667in}{0.222222in}}%
\pgfpathlineto{\pgfqpoint{0.900339in}{0.189139in}}%
\pgfpathlineto{\pgfqpoint{0.863830in}{0.183834in}}%
\pgfpathlineto{\pgfqpoint{0.890248in}{0.158083in}}%
\pgfpathlineto{\pgfqpoint{0.884012in}{0.121721in}}%
\pgfpathlineto{\pgfqpoint{0.916667in}{0.138889in}}%
\pgfpathlineto{\pgfqpoint{0.949321in}{0.121721in}}%
\pgfpathlineto{\pgfqpoint{0.943085in}{0.158083in}}%
\pgfpathlineto{\pgfqpoint{0.969503in}{0.183834in}}%
\pgfpathlineto{\pgfqpoint{0.932994in}{0.189139in}}%
\pgfpathlineto{\pgfqpoint{0.916667in}{0.222222in}}%
\pgfpathmoveto{\pgfqpoint{0.000000in}{0.388889in}}%
\pgfpathlineto{\pgfqpoint{-0.016327in}{0.355806in}}%
\pgfpathlineto{\pgfqpoint{-0.052836in}{0.350501in}}%
\pgfpathlineto{\pgfqpoint{-0.026418in}{0.324750in}}%
\pgfpathlineto{\pgfqpoint{-0.032655in}{0.288388in}}%
\pgfpathlineto{\pgfqpoint{-0.000000in}{0.305556in}}%
\pgfpathlineto{\pgfqpoint{0.032655in}{0.288388in}}%
\pgfpathlineto{\pgfqpoint{0.026418in}{0.324750in}}%
\pgfpathlineto{\pgfqpoint{0.052836in}{0.350501in}}%
\pgfpathlineto{\pgfqpoint{0.016327in}{0.355806in}}%
\pgfpathlineto{\pgfqpoint{0.000000in}{0.388889in}}%
\pgfpathmoveto{\pgfqpoint{0.166667in}{0.388889in}}%
\pgfpathlineto{\pgfqpoint{0.150339in}{0.355806in}}%
\pgfpathlineto{\pgfqpoint{0.113830in}{0.350501in}}%
\pgfpathlineto{\pgfqpoint{0.140248in}{0.324750in}}%
\pgfpathlineto{\pgfqpoint{0.134012in}{0.288388in}}%
\pgfpathlineto{\pgfqpoint{0.166667in}{0.305556in}}%
\pgfpathlineto{\pgfqpoint{0.199321in}{0.288388in}}%
\pgfpathlineto{\pgfqpoint{0.193085in}{0.324750in}}%
\pgfpathlineto{\pgfqpoint{0.219503in}{0.350501in}}%
\pgfpathlineto{\pgfqpoint{0.182994in}{0.355806in}}%
\pgfpathlineto{\pgfqpoint{0.166667in}{0.388889in}}%
\pgfpathmoveto{\pgfqpoint{0.333333in}{0.388889in}}%
\pgfpathlineto{\pgfqpoint{0.317006in}{0.355806in}}%
\pgfpathlineto{\pgfqpoint{0.280497in}{0.350501in}}%
\pgfpathlineto{\pgfqpoint{0.306915in}{0.324750in}}%
\pgfpathlineto{\pgfqpoint{0.300679in}{0.288388in}}%
\pgfpathlineto{\pgfqpoint{0.333333in}{0.305556in}}%
\pgfpathlineto{\pgfqpoint{0.365988in}{0.288388in}}%
\pgfpathlineto{\pgfqpoint{0.359752in}{0.324750in}}%
\pgfpathlineto{\pgfqpoint{0.386170in}{0.350501in}}%
\pgfpathlineto{\pgfqpoint{0.349661in}{0.355806in}}%
\pgfpathlineto{\pgfqpoint{0.333333in}{0.388889in}}%
\pgfpathmoveto{\pgfqpoint{0.500000in}{0.388889in}}%
\pgfpathlineto{\pgfqpoint{0.483673in}{0.355806in}}%
\pgfpathlineto{\pgfqpoint{0.447164in}{0.350501in}}%
\pgfpathlineto{\pgfqpoint{0.473582in}{0.324750in}}%
\pgfpathlineto{\pgfqpoint{0.467345in}{0.288388in}}%
\pgfpathlineto{\pgfqpoint{0.500000in}{0.305556in}}%
\pgfpathlineto{\pgfqpoint{0.532655in}{0.288388in}}%
\pgfpathlineto{\pgfqpoint{0.526418in}{0.324750in}}%
\pgfpathlineto{\pgfqpoint{0.552836in}{0.350501in}}%
\pgfpathlineto{\pgfqpoint{0.516327in}{0.355806in}}%
\pgfpathlineto{\pgfqpoint{0.500000in}{0.388889in}}%
\pgfpathmoveto{\pgfqpoint{0.666667in}{0.388889in}}%
\pgfpathlineto{\pgfqpoint{0.650339in}{0.355806in}}%
\pgfpathlineto{\pgfqpoint{0.613830in}{0.350501in}}%
\pgfpathlineto{\pgfqpoint{0.640248in}{0.324750in}}%
\pgfpathlineto{\pgfqpoint{0.634012in}{0.288388in}}%
\pgfpathlineto{\pgfqpoint{0.666667in}{0.305556in}}%
\pgfpathlineto{\pgfqpoint{0.699321in}{0.288388in}}%
\pgfpathlineto{\pgfqpoint{0.693085in}{0.324750in}}%
\pgfpathlineto{\pgfqpoint{0.719503in}{0.350501in}}%
\pgfpathlineto{\pgfqpoint{0.682994in}{0.355806in}}%
\pgfpathlineto{\pgfqpoint{0.666667in}{0.388889in}}%
\pgfpathmoveto{\pgfqpoint{0.833333in}{0.388889in}}%
\pgfpathlineto{\pgfqpoint{0.817006in}{0.355806in}}%
\pgfpathlineto{\pgfqpoint{0.780497in}{0.350501in}}%
\pgfpathlineto{\pgfqpoint{0.806915in}{0.324750in}}%
\pgfpathlineto{\pgfqpoint{0.800679in}{0.288388in}}%
\pgfpathlineto{\pgfqpoint{0.833333in}{0.305556in}}%
\pgfpathlineto{\pgfqpoint{0.865988in}{0.288388in}}%
\pgfpathlineto{\pgfqpoint{0.859752in}{0.324750in}}%
\pgfpathlineto{\pgfqpoint{0.886170in}{0.350501in}}%
\pgfpathlineto{\pgfqpoint{0.849661in}{0.355806in}}%
\pgfpathlineto{\pgfqpoint{0.833333in}{0.388889in}}%
\pgfpathmoveto{\pgfqpoint{1.000000in}{0.388889in}}%
\pgfpathlineto{\pgfqpoint{0.983673in}{0.355806in}}%
\pgfpathlineto{\pgfqpoint{0.947164in}{0.350501in}}%
\pgfpathlineto{\pgfqpoint{0.973582in}{0.324750in}}%
\pgfpathlineto{\pgfqpoint{0.967345in}{0.288388in}}%
\pgfpathlineto{\pgfqpoint{1.000000in}{0.305556in}}%
\pgfpathlineto{\pgfqpoint{1.032655in}{0.288388in}}%
\pgfpathlineto{\pgfqpoint{1.026418in}{0.324750in}}%
\pgfpathlineto{\pgfqpoint{1.052836in}{0.350501in}}%
\pgfpathlineto{\pgfqpoint{1.016327in}{0.355806in}}%
\pgfpathlineto{\pgfqpoint{1.000000in}{0.388889in}}%
\pgfpathmoveto{\pgfqpoint{0.083333in}{0.555556in}}%
\pgfpathlineto{\pgfqpoint{0.067006in}{0.522473in}}%
\pgfpathlineto{\pgfqpoint{0.030497in}{0.517168in}}%
\pgfpathlineto{\pgfqpoint{0.056915in}{0.491416in}}%
\pgfpathlineto{\pgfqpoint{0.050679in}{0.455055in}}%
\pgfpathlineto{\pgfqpoint{0.083333in}{0.472222in}}%
\pgfpathlineto{\pgfqpoint{0.115988in}{0.455055in}}%
\pgfpathlineto{\pgfqpoint{0.109752in}{0.491416in}}%
\pgfpathlineto{\pgfqpoint{0.136170in}{0.517168in}}%
\pgfpathlineto{\pgfqpoint{0.099661in}{0.522473in}}%
\pgfpathlineto{\pgfqpoint{0.083333in}{0.555556in}}%
\pgfpathmoveto{\pgfqpoint{0.250000in}{0.555556in}}%
\pgfpathlineto{\pgfqpoint{0.233673in}{0.522473in}}%
\pgfpathlineto{\pgfqpoint{0.197164in}{0.517168in}}%
\pgfpathlineto{\pgfqpoint{0.223582in}{0.491416in}}%
\pgfpathlineto{\pgfqpoint{0.217345in}{0.455055in}}%
\pgfpathlineto{\pgfqpoint{0.250000in}{0.472222in}}%
\pgfpathlineto{\pgfqpoint{0.282655in}{0.455055in}}%
\pgfpathlineto{\pgfqpoint{0.276418in}{0.491416in}}%
\pgfpathlineto{\pgfqpoint{0.302836in}{0.517168in}}%
\pgfpathlineto{\pgfqpoint{0.266327in}{0.522473in}}%
\pgfpathlineto{\pgfqpoint{0.250000in}{0.555556in}}%
\pgfpathmoveto{\pgfqpoint{0.416667in}{0.555556in}}%
\pgfpathlineto{\pgfqpoint{0.400339in}{0.522473in}}%
\pgfpathlineto{\pgfqpoint{0.363830in}{0.517168in}}%
\pgfpathlineto{\pgfqpoint{0.390248in}{0.491416in}}%
\pgfpathlineto{\pgfqpoint{0.384012in}{0.455055in}}%
\pgfpathlineto{\pgfqpoint{0.416667in}{0.472222in}}%
\pgfpathlineto{\pgfqpoint{0.449321in}{0.455055in}}%
\pgfpathlineto{\pgfqpoint{0.443085in}{0.491416in}}%
\pgfpathlineto{\pgfqpoint{0.469503in}{0.517168in}}%
\pgfpathlineto{\pgfqpoint{0.432994in}{0.522473in}}%
\pgfpathlineto{\pgfqpoint{0.416667in}{0.555556in}}%
\pgfpathmoveto{\pgfqpoint{0.583333in}{0.555556in}}%
\pgfpathlineto{\pgfqpoint{0.567006in}{0.522473in}}%
\pgfpathlineto{\pgfqpoint{0.530497in}{0.517168in}}%
\pgfpathlineto{\pgfqpoint{0.556915in}{0.491416in}}%
\pgfpathlineto{\pgfqpoint{0.550679in}{0.455055in}}%
\pgfpathlineto{\pgfqpoint{0.583333in}{0.472222in}}%
\pgfpathlineto{\pgfqpoint{0.615988in}{0.455055in}}%
\pgfpathlineto{\pgfqpoint{0.609752in}{0.491416in}}%
\pgfpathlineto{\pgfqpoint{0.636170in}{0.517168in}}%
\pgfpathlineto{\pgfqpoint{0.599661in}{0.522473in}}%
\pgfpathlineto{\pgfqpoint{0.583333in}{0.555556in}}%
\pgfpathmoveto{\pgfqpoint{0.750000in}{0.555556in}}%
\pgfpathlineto{\pgfqpoint{0.733673in}{0.522473in}}%
\pgfpathlineto{\pgfqpoint{0.697164in}{0.517168in}}%
\pgfpathlineto{\pgfqpoint{0.723582in}{0.491416in}}%
\pgfpathlineto{\pgfqpoint{0.717345in}{0.455055in}}%
\pgfpathlineto{\pgfqpoint{0.750000in}{0.472222in}}%
\pgfpathlineto{\pgfqpoint{0.782655in}{0.455055in}}%
\pgfpathlineto{\pgfqpoint{0.776418in}{0.491416in}}%
\pgfpathlineto{\pgfqpoint{0.802836in}{0.517168in}}%
\pgfpathlineto{\pgfqpoint{0.766327in}{0.522473in}}%
\pgfpathlineto{\pgfqpoint{0.750000in}{0.555556in}}%
\pgfpathmoveto{\pgfqpoint{0.916667in}{0.555556in}}%
\pgfpathlineto{\pgfqpoint{0.900339in}{0.522473in}}%
\pgfpathlineto{\pgfqpoint{0.863830in}{0.517168in}}%
\pgfpathlineto{\pgfqpoint{0.890248in}{0.491416in}}%
\pgfpathlineto{\pgfqpoint{0.884012in}{0.455055in}}%
\pgfpathlineto{\pgfqpoint{0.916667in}{0.472222in}}%
\pgfpathlineto{\pgfqpoint{0.949321in}{0.455055in}}%
\pgfpathlineto{\pgfqpoint{0.943085in}{0.491416in}}%
\pgfpathlineto{\pgfqpoint{0.969503in}{0.517168in}}%
\pgfpathlineto{\pgfqpoint{0.932994in}{0.522473in}}%
\pgfpathlineto{\pgfqpoint{0.916667in}{0.555556in}}%
\pgfpathmoveto{\pgfqpoint{0.000000in}{0.722222in}}%
\pgfpathlineto{\pgfqpoint{-0.016327in}{0.689139in}}%
\pgfpathlineto{\pgfqpoint{-0.052836in}{0.683834in}}%
\pgfpathlineto{\pgfqpoint{-0.026418in}{0.658083in}}%
\pgfpathlineto{\pgfqpoint{-0.032655in}{0.621721in}}%
\pgfpathlineto{\pgfqpoint{-0.000000in}{0.638889in}}%
\pgfpathlineto{\pgfqpoint{0.032655in}{0.621721in}}%
\pgfpathlineto{\pgfqpoint{0.026418in}{0.658083in}}%
\pgfpathlineto{\pgfqpoint{0.052836in}{0.683834in}}%
\pgfpathlineto{\pgfqpoint{0.016327in}{0.689139in}}%
\pgfpathlineto{\pgfqpoint{0.000000in}{0.722222in}}%
\pgfpathmoveto{\pgfqpoint{0.166667in}{0.722222in}}%
\pgfpathlineto{\pgfqpoint{0.150339in}{0.689139in}}%
\pgfpathlineto{\pgfqpoint{0.113830in}{0.683834in}}%
\pgfpathlineto{\pgfqpoint{0.140248in}{0.658083in}}%
\pgfpathlineto{\pgfqpoint{0.134012in}{0.621721in}}%
\pgfpathlineto{\pgfqpoint{0.166667in}{0.638889in}}%
\pgfpathlineto{\pgfqpoint{0.199321in}{0.621721in}}%
\pgfpathlineto{\pgfqpoint{0.193085in}{0.658083in}}%
\pgfpathlineto{\pgfqpoint{0.219503in}{0.683834in}}%
\pgfpathlineto{\pgfqpoint{0.182994in}{0.689139in}}%
\pgfpathlineto{\pgfqpoint{0.166667in}{0.722222in}}%
\pgfpathmoveto{\pgfqpoint{0.333333in}{0.722222in}}%
\pgfpathlineto{\pgfqpoint{0.317006in}{0.689139in}}%
\pgfpathlineto{\pgfqpoint{0.280497in}{0.683834in}}%
\pgfpathlineto{\pgfqpoint{0.306915in}{0.658083in}}%
\pgfpathlineto{\pgfqpoint{0.300679in}{0.621721in}}%
\pgfpathlineto{\pgfqpoint{0.333333in}{0.638889in}}%
\pgfpathlineto{\pgfqpoint{0.365988in}{0.621721in}}%
\pgfpathlineto{\pgfqpoint{0.359752in}{0.658083in}}%
\pgfpathlineto{\pgfqpoint{0.386170in}{0.683834in}}%
\pgfpathlineto{\pgfqpoint{0.349661in}{0.689139in}}%
\pgfpathlineto{\pgfqpoint{0.333333in}{0.722222in}}%
\pgfpathmoveto{\pgfqpoint{0.500000in}{0.722222in}}%
\pgfpathlineto{\pgfqpoint{0.483673in}{0.689139in}}%
\pgfpathlineto{\pgfqpoint{0.447164in}{0.683834in}}%
\pgfpathlineto{\pgfqpoint{0.473582in}{0.658083in}}%
\pgfpathlineto{\pgfqpoint{0.467345in}{0.621721in}}%
\pgfpathlineto{\pgfqpoint{0.500000in}{0.638889in}}%
\pgfpathlineto{\pgfqpoint{0.532655in}{0.621721in}}%
\pgfpathlineto{\pgfqpoint{0.526418in}{0.658083in}}%
\pgfpathlineto{\pgfqpoint{0.552836in}{0.683834in}}%
\pgfpathlineto{\pgfqpoint{0.516327in}{0.689139in}}%
\pgfpathlineto{\pgfqpoint{0.500000in}{0.722222in}}%
\pgfpathmoveto{\pgfqpoint{0.666667in}{0.722222in}}%
\pgfpathlineto{\pgfqpoint{0.650339in}{0.689139in}}%
\pgfpathlineto{\pgfqpoint{0.613830in}{0.683834in}}%
\pgfpathlineto{\pgfqpoint{0.640248in}{0.658083in}}%
\pgfpathlineto{\pgfqpoint{0.634012in}{0.621721in}}%
\pgfpathlineto{\pgfqpoint{0.666667in}{0.638889in}}%
\pgfpathlineto{\pgfqpoint{0.699321in}{0.621721in}}%
\pgfpathlineto{\pgfqpoint{0.693085in}{0.658083in}}%
\pgfpathlineto{\pgfqpoint{0.719503in}{0.683834in}}%
\pgfpathlineto{\pgfqpoint{0.682994in}{0.689139in}}%
\pgfpathlineto{\pgfqpoint{0.666667in}{0.722222in}}%
\pgfpathmoveto{\pgfqpoint{0.833333in}{0.722222in}}%
\pgfpathlineto{\pgfqpoint{0.817006in}{0.689139in}}%
\pgfpathlineto{\pgfqpoint{0.780497in}{0.683834in}}%
\pgfpathlineto{\pgfqpoint{0.806915in}{0.658083in}}%
\pgfpathlineto{\pgfqpoint{0.800679in}{0.621721in}}%
\pgfpathlineto{\pgfqpoint{0.833333in}{0.638889in}}%
\pgfpathlineto{\pgfqpoint{0.865988in}{0.621721in}}%
\pgfpathlineto{\pgfqpoint{0.859752in}{0.658083in}}%
\pgfpathlineto{\pgfqpoint{0.886170in}{0.683834in}}%
\pgfpathlineto{\pgfqpoint{0.849661in}{0.689139in}}%
\pgfpathlineto{\pgfqpoint{0.833333in}{0.722222in}}%
\pgfpathmoveto{\pgfqpoint{1.000000in}{0.722222in}}%
\pgfpathlineto{\pgfqpoint{0.983673in}{0.689139in}}%
\pgfpathlineto{\pgfqpoint{0.947164in}{0.683834in}}%
\pgfpathlineto{\pgfqpoint{0.973582in}{0.658083in}}%
\pgfpathlineto{\pgfqpoint{0.967345in}{0.621721in}}%
\pgfpathlineto{\pgfqpoint{1.000000in}{0.638889in}}%
\pgfpathlineto{\pgfqpoint{1.032655in}{0.621721in}}%
\pgfpathlineto{\pgfqpoint{1.026418in}{0.658083in}}%
\pgfpathlineto{\pgfqpoint{1.052836in}{0.683834in}}%
\pgfpathlineto{\pgfqpoint{1.016327in}{0.689139in}}%
\pgfpathlineto{\pgfqpoint{1.000000in}{0.722222in}}%
\pgfpathmoveto{\pgfqpoint{0.083333in}{0.888889in}}%
\pgfpathlineto{\pgfqpoint{0.067006in}{0.855806in}}%
\pgfpathlineto{\pgfqpoint{0.030497in}{0.850501in}}%
\pgfpathlineto{\pgfqpoint{0.056915in}{0.824750in}}%
\pgfpathlineto{\pgfqpoint{0.050679in}{0.788388in}}%
\pgfpathlineto{\pgfqpoint{0.083333in}{0.805556in}}%
\pgfpathlineto{\pgfqpoint{0.115988in}{0.788388in}}%
\pgfpathlineto{\pgfqpoint{0.109752in}{0.824750in}}%
\pgfpathlineto{\pgfqpoint{0.136170in}{0.850501in}}%
\pgfpathlineto{\pgfqpoint{0.099661in}{0.855806in}}%
\pgfpathlineto{\pgfqpoint{0.083333in}{0.888889in}}%
\pgfpathmoveto{\pgfqpoint{0.250000in}{0.888889in}}%
\pgfpathlineto{\pgfqpoint{0.233673in}{0.855806in}}%
\pgfpathlineto{\pgfqpoint{0.197164in}{0.850501in}}%
\pgfpathlineto{\pgfqpoint{0.223582in}{0.824750in}}%
\pgfpathlineto{\pgfqpoint{0.217345in}{0.788388in}}%
\pgfpathlineto{\pgfqpoint{0.250000in}{0.805556in}}%
\pgfpathlineto{\pgfqpoint{0.282655in}{0.788388in}}%
\pgfpathlineto{\pgfqpoint{0.276418in}{0.824750in}}%
\pgfpathlineto{\pgfqpoint{0.302836in}{0.850501in}}%
\pgfpathlineto{\pgfqpoint{0.266327in}{0.855806in}}%
\pgfpathlineto{\pgfqpoint{0.250000in}{0.888889in}}%
\pgfpathmoveto{\pgfqpoint{0.416667in}{0.888889in}}%
\pgfpathlineto{\pgfqpoint{0.400339in}{0.855806in}}%
\pgfpathlineto{\pgfqpoint{0.363830in}{0.850501in}}%
\pgfpathlineto{\pgfqpoint{0.390248in}{0.824750in}}%
\pgfpathlineto{\pgfqpoint{0.384012in}{0.788388in}}%
\pgfpathlineto{\pgfqpoint{0.416667in}{0.805556in}}%
\pgfpathlineto{\pgfqpoint{0.449321in}{0.788388in}}%
\pgfpathlineto{\pgfqpoint{0.443085in}{0.824750in}}%
\pgfpathlineto{\pgfqpoint{0.469503in}{0.850501in}}%
\pgfpathlineto{\pgfqpoint{0.432994in}{0.855806in}}%
\pgfpathlineto{\pgfqpoint{0.416667in}{0.888889in}}%
\pgfpathmoveto{\pgfqpoint{0.583333in}{0.888889in}}%
\pgfpathlineto{\pgfqpoint{0.567006in}{0.855806in}}%
\pgfpathlineto{\pgfqpoint{0.530497in}{0.850501in}}%
\pgfpathlineto{\pgfqpoint{0.556915in}{0.824750in}}%
\pgfpathlineto{\pgfqpoint{0.550679in}{0.788388in}}%
\pgfpathlineto{\pgfqpoint{0.583333in}{0.805556in}}%
\pgfpathlineto{\pgfqpoint{0.615988in}{0.788388in}}%
\pgfpathlineto{\pgfqpoint{0.609752in}{0.824750in}}%
\pgfpathlineto{\pgfqpoint{0.636170in}{0.850501in}}%
\pgfpathlineto{\pgfqpoint{0.599661in}{0.855806in}}%
\pgfpathlineto{\pgfqpoint{0.583333in}{0.888889in}}%
\pgfpathmoveto{\pgfqpoint{0.750000in}{0.888889in}}%
\pgfpathlineto{\pgfqpoint{0.733673in}{0.855806in}}%
\pgfpathlineto{\pgfqpoint{0.697164in}{0.850501in}}%
\pgfpathlineto{\pgfqpoint{0.723582in}{0.824750in}}%
\pgfpathlineto{\pgfqpoint{0.717345in}{0.788388in}}%
\pgfpathlineto{\pgfqpoint{0.750000in}{0.805556in}}%
\pgfpathlineto{\pgfqpoint{0.782655in}{0.788388in}}%
\pgfpathlineto{\pgfqpoint{0.776418in}{0.824750in}}%
\pgfpathlineto{\pgfqpoint{0.802836in}{0.850501in}}%
\pgfpathlineto{\pgfqpoint{0.766327in}{0.855806in}}%
\pgfpathlineto{\pgfqpoint{0.750000in}{0.888889in}}%
\pgfpathmoveto{\pgfqpoint{0.916667in}{0.888889in}}%
\pgfpathlineto{\pgfqpoint{0.900339in}{0.855806in}}%
\pgfpathlineto{\pgfqpoint{0.863830in}{0.850501in}}%
\pgfpathlineto{\pgfqpoint{0.890248in}{0.824750in}}%
\pgfpathlineto{\pgfqpoint{0.884012in}{0.788388in}}%
\pgfpathlineto{\pgfqpoint{0.916667in}{0.805556in}}%
\pgfpathlineto{\pgfqpoint{0.949321in}{0.788388in}}%
\pgfpathlineto{\pgfqpoint{0.943085in}{0.824750in}}%
\pgfpathlineto{\pgfqpoint{0.969503in}{0.850501in}}%
\pgfpathlineto{\pgfqpoint{0.932994in}{0.855806in}}%
\pgfpathlineto{\pgfqpoint{0.916667in}{0.888889in}}%
\pgfpathmoveto{\pgfqpoint{0.000000in}{1.055556in}}%
\pgfpathlineto{\pgfqpoint{-0.016327in}{1.022473in}}%
\pgfpathlineto{\pgfqpoint{-0.052836in}{1.017168in}}%
\pgfpathlineto{\pgfqpoint{-0.026418in}{0.991416in}}%
\pgfpathlineto{\pgfqpoint{-0.032655in}{0.955055in}}%
\pgfpathlineto{\pgfqpoint{-0.000000in}{0.972222in}}%
\pgfpathlineto{\pgfqpoint{0.032655in}{0.955055in}}%
\pgfpathlineto{\pgfqpoint{0.026418in}{0.991416in}}%
\pgfpathlineto{\pgfqpoint{0.052836in}{1.017168in}}%
\pgfpathlineto{\pgfqpoint{0.016327in}{1.022473in}}%
\pgfpathlineto{\pgfqpoint{0.000000in}{1.055556in}}%
\pgfpathmoveto{\pgfqpoint{0.166667in}{1.055556in}}%
\pgfpathlineto{\pgfqpoint{0.150339in}{1.022473in}}%
\pgfpathlineto{\pgfqpoint{0.113830in}{1.017168in}}%
\pgfpathlineto{\pgfqpoint{0.140248in}{0.991416in}}%
\pgfpathlineto{\pgfqpoint{0.134012in}{0.955055in}}%
\pgfpathlineto{\pgfqpoint{0.166667in}{0.972222in}}%
\pgfpathlineto{\pgfqpoint{0.199321in}{0.955055in}}%
\pgfpathlineto{\pgfqpoint{0.193085in}{0.991416in}}%
\pgfpathlineto{\pgfqpoint{0.219503in}{1.017168in}}%
\pgfpathlineto{\pgfqpoint{0.182994in}{1.022473in}}%
\pgfpathlineto{\pgfqpoint{0.166667in}{1.055556in}}%
\pgfpathmoveto{\pgfqpoint{0.333333in}{1.055556in}}%
\pgfpathlineto{\pgfqpoint{0.317006in}{1.022473in}}%
\pgfpathlineto{\pgfqpoint{0.280497in}{1.017168in}}%
\pgfpathlineto{\pgfqpoint{0.306915in}{0.991416in}}%
\pgfpathlineto{\pgfqpoint{0.300679in}{0.955055in}}%
\pgfpathlineto{\pgfqpoint{0.333333in}{0.972222in}}%
\pgfpathlineto{\pgfqpoint{0.365988in}{0.955055in}}%
\pgfpathlineto{\pgfqpoint{0.359752in}{0.991416in}}%
\pgfpathlineto{\pgfqpoint{0.386170in}{1.017168in}}%
\pgfpathlineto{\pgfqpoint{0.349661in}{1.022473in}}%
\pgfpathlineto{\pgfqpoint{0.333333in}{1.055556in}}%
\pgfpathmoveto{\pgfqpoint{0.500000in}{1.055556in}}%
\pgfpathlineto{\pgfqpoint{0.483673in}{1.022473in}}%
\pgfpathlineto{\pgfqpoint{0.447164in}{1.017168in}}%
\pgfpathlineto{\pgfqpoint{0.473582in}{0.991416in}}%
\pgfpathlineto{\pgfqpoint{0.467345in}{0.955055in}}%
\pgfpathlineto{\pgfqpoint{0.500000in}{0.972222in}}%
\pgfpathlineto{\pgfqpoint{0.532655in}{0.955055in}}%
\pgfpathlineto{\pgfqpoint{0.526418in}{0.991416in}}%
\pgfpathlineto{\pgfqpoint{0.552836in}{1.017168in}}%
\pgfpathlineto{\pgfqpoint{0.516327in}{1.022473in}}%
\pgfpathlineto{\pgfqpoint{0.500000in}{1.055556in}}%
\pgfpathmoveto{\pgfqpoint{0.666667in}{1.055556in}}%
\pgfpathlineto{\pgfqpoint{0.650339in}{1.022473in}}%
\pgfpathlineto{\pgfqpoint{0.613830in}{1.017168in}}%
\pgfpathlineto{\pgfqpoint{0.640248in}{0.991416in}}%
\pgfpathlineto{\pgfqpoint{0.634012in}{0.955055in}}%
\pgfpathlineto{\pgfqpoint{0.666667in}{0.972222in}}%
\pgfpathlineto{\pgfqpoint{0.699321in}{0.955055in}}%
\pgfpathlineto{\pgfqpoint{0.693085in}{0.991416in}}%
\pgfpathlineto{\pgfqpoint{0.719503in}{1.017168in}}%
\pgfpathlineto{\pgfqpoint{0.682994in}{1.022473in}}%
\pgfpathlineto{\pgfqpoint{0.666667in}{1.055556in}}%
\pgfpathmoveto{\pgfqpoint{0.833333in}{1.055556in}}%
\pgfpathlineto{\pgfqpoint{0.817006in}{1.022473in}}%
\pgfpathlineto{\pgfqpoint{0.780497in}{1.017168in}}%
\pgfpathlineto{\pgfqpoint{0.806915in}{0.991416in}}%
\pgfpathlineto{\pgfqpoint{0.800679in}{0.955055in}}%
\pgfpathlineto{\pgfqpoint{0.833333in}{0.972222in}}%
\pgfpathlineto{\pgfqpoint{0.865988in}{0.955055in}}%
\pgfpathlineto{\pgfqpoint{0.859752in}{0.991416in}}%
\pgfpathlineto{\pgfqpoint{0.886170in}{1.017168in}}%
\pgfpathlineto{\pgfqpoint{0.849661in}{1.022473in}}%
\pgfpathlineto{\pgfqpoint{0.833333in}{1.055556in}}%
\pgfpathmoveto{\pgfqpoint{1.000000in}{1.055556in}}%
\pgfpathlineto{\pgfqpoint{0.983673in}{1.022473in}}%
\pgfpathlineto{\pgfqpoint{0.947164in}{1.017168in}}%
\pgfpathlineto{\pgfqpoint{0.973582in}{0.991416in}}%
\pgfpathlineto{\pgfqpoint{0.967345in}{0.955055in}}%
\pgfpathlineto{\pgfqpoint{1.000000in}{0.972222in}}%
\pgfpathlineto{\pgfqpoint{1.032655in}{0.955055in}}%
\pgfpathlineto{\pgfqpoint{1.026418in}{0.991416in}}%
\pgfpathlineto{\pgfqpoint{1.052836in}{1.017168in}}%
\pgfpathlineto{\pgfqpoint{1.016327in}{1.022473in}}%
\pgfpathlineto{\pgfqpoint{1.000000in}{1.055556in}}%
\pgfpathlineto{\pgfqpoint{1.000000in}{1.055556in}}%
\pgfusepath{stroke}%
\end{pgfscope}%
}%
\pgfsys@transformshift{4.358038in}{3.454505in}%
\pgfsys@useobject{currentpattern}{}%
\pgfsys@transformshift{1in}{0in}%
\pgfsys@transformshift{-1in}{0in}%
\pgfsys@transformshift{0in}{1in}%
\end{pgfscope}%
\begin{pgfscope}%
\pgfpathrectangle{\pgfqpoint{0.870538in}{0.637495in}}{\pgfqpoint{9.300000in}{9.060000in}}%
\pgfusepath{clip}%
\pgfsetbuttcap%
\pgfsetmiterjoin%
\definecolor{currentfill}{rgb}{1.000000,1.000000,0.000000}%
\pgfsetfillcolor{currentfill}%
\pgfsetfillopacity{0.990000}%
\pgfsetlinewidth{0.000000pt}%
\definecolor{currentstroke}{rgb}{0.000000,0.000000,0.000000}%
\pgfsetstrokecolor{currentstroke}%
\pgfsetstrokeopacity{0.990000}%
\pgfsetdash{}{0pt}%
\pgfpathmoveto{\pgfqpoint{5.908038in}{3.636205in}}%
\pgfpathlineto{\pgfqpoint{6.683038in}{3.636205in}}%
\pgfpathlineto{\pgfqpoint{6.683038in}{4.713204in}}%
\pgfpathlineto{\pgfqpoint{5.908038in}{4.713204in}}%
\pgfpathclose%
\pgfusepath{fill}%
\end{pgfscope}%
\begin{pgfscope}%
\pgfsetbuttcap%
\pgfsetmiterjoin%
\definecolor{currentfill}{rgb}{1.000000,1.000000,0.000000}%
\pgfsetfillcolor{currentfill}%
\pgfsetfillopacity{0.990000}%
\pgfsetlinewidth{0.000000pt}%
\definecolor{currentstroke}{rgb}{0.000000,0.000000,0.000000}%
\pgfsetstrokecolor{currentstroke}%
\pgfsetstrokeopacity{0.990000}%
\pgfsetdash{}{0pt}%
\pgfpathrectangle{\pgfqpoint{0.870538in}{0.637495in}}{\pgfqpoint{9.300000in}{9.060000in}}%
\pgfusepath{clip}%
\pgfpathmoveto{\pgfqpoint{5.908038in}{3.636205in}}%
\pgfpathlineto{\pgfqpoint{6.683038in}{3.636205in}}%
\pgfpathlineto{\pgfqpoint{6.683038in}{4.713204in}}%
\pgfpathlineto{\pgfqpoint{5.908038in}{4.713204in}}%
\pgfpathclose%
\pgfusepath{clip}%
\pgfsys@defobject{currentpattern}{\pgfqpoint{0in}{0in}}{\pgfqpoint{1in}{1in}}{%
\begin{pgfscope}%
\pgfpathrectangle{\pgfqpoint{0in}{0in}}{\pgfqpoint{1in}{1in}}%
\pgfusepath{clip}%
\pgfpathmoveto{\pgfqpoint{0.000000in}{0.055556in}}%
\pgfpathlineto{\pgfqpoint{-0.016327in}{0.022473in}}%
\pgfpathlineto{\pgfqpoint{-0.052836in}{0.017168in}}%
\pgfpathlineto{\pgfqpoint{-0.026418in}{-0.008584in}}%
\pgfpathlineto{\pgfqpoint{-0.032655in}{-0.044945in}}%
\pgfpathlineto{\pgfqpoint{-0.000000in}{-0.027778in}}%
\pgfpathlineto{\pgfqpoint{0.032655in}{-0.044945in}}%
\pgfpathlineto{\pgfqpoint{0.026418in}{-0.008584in}}%
\pgfpathlineto{\pgfqpoint{0.052836in}{0.017168in}}%
\pgfpathlineto{\pgfqpoint{0.016327in}{0.022473in}}%
\pgfpathlineto{\pgfqpoint{0.000000in}{0.055556in}}%
\pgfpathmoveto{\pgfqpoint{0.166667in}{0.055556in}}%
\pgfpathlineto{\pgfqpoint{0.150339in}{0.022473in}}%
\pgfpathlineto{\pgfqpoint{0.113830in}{0.017168in}}%
\pgfpathlineto{\pgfqpoint{0.140248in}{-0.008584in}}%
\pgfpathlineto{\pgfqpoint{0.134012in}{-0.044945in}}%
\pgfpathlineto{\pgfqpoint{0.166667in}{-0.027778in}}%
\pgfpathlineto{\pgfqpoint{0.199321in}{-0.044945in}}%
\pgfpathlineto{\pgfqpoint{0.193085in}{-0.008584in}}%
\pgfpathlineto{\pgfqpoint{0.219503in}{0.017168in}}%
\pgfpathlineto{\pgfqpoint{0.182994in}{0.022473in}}%
\pgfpathlineto{\pgfqpoint{0.166667in}{0.055556in}}%
\pgfpathmoveto{\pgfqpoint{0.333333in}{0.055556in}}%
\pgfpathlineto{\pgfqpoint{0.317006in}{0.022473in}}%
\pgfpathlineto{\pgfqpoint{0.280497in}{0.017168in}}%
\pgfpathlineto{\pgfqpoint{0.306915in}{-0.008584in}}%
\pgfpathlineto{\pgfqpoint{0.300679in}{-0.044945in}}%
\pgfpathlineto{\pgfqpoint{0.333333in}{-0.027778in}}%
\pgfpathlineto{\pgfqpoint{0.365988in}{-0.044945in}}%
\pgfpathlineto{\pgfqpoint{0.359752in}{-0.008584in}}%
\pgfpathlineto{\pgfqpoint{0.386170in}{0.017168in}}%
\pgfpathlineto{\pgfqpoint{0.349661in}{0.022473in}}%
\pgfpathlineto{\pgfqpoint{0.333333in}{0.055556in}}%
\pgfpathmoveto{\pgfqpoint{0.500000in}{0.055556in}}%
\pgfpathlineto{\pgfqpoint{0.483673in}{0.022473in}}%
\pgfpathlineto{\pgfqpoint{0.447164in}{0.017168in}}%
\pgfpathlineto{\pgfqpoint{0.473582in}{-0.008584in}}%
\pgfpathlineto{\pgfqpoint{0.467345in}{-0.044945in}}%
\pgfpathlineto{\pgfqpoint{0.500000in}{-0.027778in}}%
\pgfpathlineto{\pgfqpoint{0.532655in}{-0.044945in}}%
\pgfpathlineto{\pgfqpoint{0.526418in}{-0.008584in}}%
\pgfpathlineto{\pgfqpoint{0.552836in}{0.017168in}}%
\pgfpathlineto{\pgfqpoint{0.516327in}{0.022473in}}%
\pgfpathlineto{\pgfqpoint{0.500000in}{0.055556in}}%
\pgfpathmoveto{\pgfqpoint{0.666667in}{0.055556in}}%
\pgfpathlineto{\pgfqpoint{0.650339in}{0.022473in}}%
\pgfpathlineto{\pgfqpoint{0.613830in}{0.017168in}}%
\pgfpathlineto{\pgfqpoint{0.640248in}{-0.008584in}}%
\pgfpathlineto{\pgfqpoint{0.634012in}{-0.044945in}}%
\pgfpathlineto{\pgfqpoint{0.666667in}{-0.027778in}}%
\pgfpathlineto{\pgfqpoint{0.699321in}{-0.044945in}}%
\pgfpathlineto{\pgfqpoint{0.693085in}{-0.008584in}}%
\pgfpathlineto{\pgfqpoint{0.719503in}{0.017168in}}%
\pgfpathlineto{\pgfqpoint{0.682994in}{0.022473in}}%
\pgfpathlineto{\pgfqpoint{0.666667in}{0.055556in}}%
\pgfpathmoveto{\pgfqpoint{0.833333in}{0.055556in}}%
\pgfpathlineto{\pgfqpoint{0.817006in}{0.022473in}}%
\pgfpathlineto{\pgfqpoint{0.780497in}{0.017168in}}%
\pgfpathlineto{\pgfqpoint{0.806915in}{-0.008584in}}%
\pgfpathlineto{\pgfqpoint{0.800679in}{-0.044945in}}%
\pgfpathlineto{\pgfqpoint{0.833333in}{-0.027778in}}%
\pgfpathlineto{\pgfqpoint{0.865988in}{-0.044945in}}%
\pgfpathlineto{\pgfqpoint{0.859752in}{-0.008584in}}%
\pgfpathlineto{\pgfqpoint{0.886170in}{0.017168in}}%
\pgfpathlineto{\pgfqpoint{0.849661in}{0.022473in}}%
\pgfpathlineto{\pgfqpoint{0.833333in}{0.055556in}}%
\pgfpathmoveto{\pgfqpoint{1.000000in}{0.055556in}}%
\pgfpathlineto{\pgfqpoint{0.983673in}{0.022473in}}%
\pgfpathlineto{\pgfqpoint{0.947164in}{0.017168in}}%
\pgfpathlineto{\pgfqpoint{0.973582in}{-0.008584in}}%
\pgfpathlineto{\pgfqpoint{0.967345in}{-0.044945in}}%
\pgfpathlineto{\pgfqpoint{1.000000in}{-0.027778in}}%
\pgfpathlineto{\pgfqpoint{1.032655in}{-0.044945in}}%
\pgfpathlineto{\pgfqpoint{1.026418in}{-0.008584in}}%
\pgfpathlineto{\pgfqpoint{1.052836in}{0.017168in}}%
\pgfpathlineto{\pgfqpoint{1.016327in}{0.022473in}}%
\pgfpathlineto{\pgfqpoint{1.000000in}{0.055556in}}%
\pgfpathmoveto{\pgfqpoint{0.083333in}{0.222222in}}%
\pgfpathlineto{\pgfqpoint{0.067006in}{0.189139in}}%
\pgfpathlineto{\pgfqpoint{0.030497in}{0.183834in}}%
\pgfpathlineto{\pgfqpoint{0.056915in}{0.158083in}}%
\pgfpathlineto{\pgfqpoint{0.050679in}{0.121721in}}%
\pgfpathlineto{\pgfqpoint{0.083333in}{0.138889in}}%
\pgfpathlineto{\pgfqpoint{0.115988in}{0.121721in}}%
\pgfpathlineto{\pgfqpoint{0.109752in}{0.158083in}}%
\pgfpathlineto{\pgfqpoint{0.136170in}{0.183834in}}%
\pgfpathlineto{\pgfqpoint{0.099661in}{0.189139in}}%
\pgfpathlineto{\pgfqpoint{0.083333in}{0.222222in}}%
\pgfpathmoveto{\pgfqpoint{0.250000in}{0.222222in}}%
\pgfpathlineto{\pgfqpoint{0.233673in}{0.189139in}}%
\pgfpathlineto{\pgfqpoint{0.197164in}{0.183834in}}%
\pgfpathlineto{\pgfqpoint{0.223582in}{0.158083in}}%
\pgfpathlineto{\pgfqpoint{0.217345in}{0.121721in}}%
\pgfpathlineto{\pgfqpoint{0.250000in}{0.138889in}}%
\pgfpathlineto{\pgfqpoint{0.282655in}{0.121721in}}%
\pgfpathlineto{\pgfqpoint{0.276418in}{0.158083in}}%
\pgfpathlineto{\pgfqpoint{0.302836in}{0.183834in}}%
\pgfpathlineto{\pgfqpoint{0.266327in}{0.189139in}}%
\pgfpathlineto{\pgfqpoint{0.250000in}{0.222222in}}%
\pgfpathmoveto{\pgfqpoint{0.416667in}{0.222222in}}%
\pgfpathlineto{\pgfqpoint{0.400339in}{0.189139in}}%
\pgfpathlineto{\pgfqpoint{0.363830in}{0.183834in}}%
\pgfpathlineto{\pgfqpoint{0.390248in}{0.158083in}}%
\pgfpathlineto{\pgfqpoint{0.384012in}{0.121721in}}%
\pgfpathlineto{\pgfqpoint{0.416667in}{0.138889in}}%
\pgfpathlineto{\pgfqpoint{0.449321in}{0.121721in}}%
\pgfpathlineto{\pgfqpoint{0.443085in}{0.158083in}}%
\pgfpathlineto{\pgfqpoint{0.469503in}{0.183834in}}%
\pgfpathlineto{\pgfqpoint{0.432994in}{0.189139in}}%
\pgfpathlineto{\pgfqpoint{0.416667in}{0.222222in}}%
\pgfpathmoveto{\pgfqpoint{0.583333in}{0.222222in}}%
\pgfpathlineto{\pgfqpoint{0.567006in}{0.189139in}}%
\pgfpathlineto{\pgfqpoint{0.530497in}{0.183834in}}%
\pgfpathlineto{\pgfqpoint{0.556915in}{0.158083in}}%
\pgfpathlineto{\pgfqpoint{0.550679in}{0.121721in}}%
\pgfpathlineto{\pgfqpoint{0.583333in}{0.138889in}}%
\pgfpathlineto{\pgfqpoint{0.615988in}{0.121721in}}%
\pgfpathlineto{\pgfqpoint{0.609752in}{0.158083in}}%
\pgfpathlineto{\pgfqpoint{0.636170in}{0.183834in}}%
\pgfpathlineto{\pgfqpoint{0.599661in}{0.189139in}}%
\pgfpathlineto{\pgfqpoint{0.583333in}{0.222222in}}%
\pgfpathmoveto{\pgfqpoint{0.750000in}{0.222222in}}%
\pgfpathlineto{\pgfqpoint{0.733673in}{0.189139in}}%
\pgfpathlineto{\pgfqpoint{0.697164in}{0.183834in}}%
\pgfpathlineto{\pgfqpoint{0.723582in}{0.158083in}}%
\pgfpathlineto{\pgfqpoint{0.717345in}{0.121721in}}%
\pgfpathlineto{\pgfqpoint{0.750000in}{0.138889in}}%
\pgfpathlineto{\pgfqpoint{0.782655in}{0.121721in}}%
\pgfpathlineto{\pgfqpoint{0.776418in}{0.158083in}}%
\pgfpathlineto{\pgfqpoint{0.802836in}{0.183834in}}%
\pgfpathlineto{\pgfqpoint{0.766327in}{0.189139in}}%
\pgfpathlineto{\pgfqpoint{0.750000in}{0.222222in}}%
\pgfpathmoveto{\pgfqpoint{0.916667in}{0.222222in}}%
\pgfpathlineto{\pgfqpoint{0.900339in}{0.189139in}}%
\pgfpathlineto{\pgfqpoint{0.863830in}{0.183834in}}%
\pgfpathlineto{\pgfqpoint{0.890248in}{0.158083in}}%
\pgfpathlineto{\pgfqpoint{0.884012in}{0.121721in}}%
\pgfpathlineto{\pgfqpoint{0.916667in}{0.138889in}}%
\pgfpathlineto{\pgfqpoint{0.949321in}{0.121721in}}%
\pgfpathlineto{\pgfqpoint{0.943085in}{0.158083in}}%
\pgfpathlineto{\pgfqpoint{0.969503in}{0.183834in}}%
\pgfpathlineto{\pgfqpoint{0.932994in}{0.189139in}}%
\pgfpathlineto{\pgfqpoint{0.916667in}{0.222222in}}%
\pgfpathmoveto{\pgfqpoint{0.000000in}{0.388889in}}%
\pgfpathlineto{\pgfqpoint{-0.016327in}{0.355806in}}%
\pgfpathlineto{\pgfqpoint{-0.052836in}{0.350501in}}%
\pgfpathlineto{\pgfqpoint{-0.026418in}{0.324750in}}%
\pgfpathlineto{\pgfqpoint{-0.032655in}{0.288388in}}%
\pgfpathlineto{\pgfqpoint{-0.000000in}{0.305556in}}%
\pgfpathlineto{\pgfqpoint{0.032655in}{0.288388in}}%
\pgfpathlineto{\pgfqpoint{0.026418in}{0.324750in}}%
\pgfpathlineto{\pgfqpoint{0.052836in}{0.350501in}}%
\pgfpathlineto{\pgfqpoint{0.016327in}{0.355806in}}%
\pgfpathlineto{\pgfqpoint{0.000000in}{0.388889in}}%
\pgfpathmoveto{\pgfqpoint{0.166667in}{0.388889in}}%
\pgfpathlineto{\pgfqpoint{0.150339in}{0.355806in}}%
\pgfpathlineto{\pgfqpoint{0.113830in}{0.350501in}}%
\pgfpathlineto{\pgfqpoint{0.140248in}{0.324750in}}%
\pgfpathlineto{\pgfqpoint{0.134012in}{0.288388in}}%
\pgfpathlineto{\pgfqpoint{0.166667in}{0.305556in}}%
\pgfpathlineto{\pgfqpoint{0.199321in}{0.288388in}}%
\pgfpathlineto{\pgfqpoint{0.193085in}{0.324750in}}%
\pgfpathlineto{\pgfqpoint{0.219503in}{0.350501in}}%
\pgfpathlineto{\pgfqpoint{0.182994in}{0.355806in}}%
\pgfpathlineto{\pgfqpoint{0.166667in}{0.388889in}}%
\pgfpathmoveto{\pgfqpoint{0.333333in}{0.388889in}}%
\pgfpathlineto{\pgfqpoint{0.317006in}{0.355806in}}%
\pgfpathlineto{\pgfqpoint{0.280497in}{0.350501in}}%
\pgfpathlineto{\pgfqpoint{0.306915in}{0.324750in}}%
\pgfpathlineto{\pgfqpoint{0.300679in}{0.288388in}}%
\pgfpathlineto{\pgfqpoint{0.333333in}{0.305556in}}%
\pgfpathlineto{\pgfqpoint{0.365988in}{0.288388in}}%
\pgfpathlineto{\pgfqpoint{0.359752in}{0.324750in}}%
\pgfpathlineto{\pgfqpoint{0.386170in}{0.350501in}}%
\pgfpathlineto{\pgfqpoint{0.349661in}{0.355806in}}%
\pgfpathlineto{\pgfqpoint{0.333333in}{0.388889in}}%
\pgfpathmoveto{\pgfqpoint{0.500000in}{0.388889in}}%
\pgfpathlineto{\pgfqpoint{0.483673in}{0.355806in}}%
\pgfpathlineto{\pgfqpoint{0.447164in}{0.350501in}}%
\pgfpathlineto{\pgfqpoint{0.473582in}{0.324750in}}%
\pgfpathlineto{\pgfqpoint{0.467345in}{0.288388in}}%
\pgfpathlineto{\pgfqpoint{0.500000in}{0.305556in}}%
\pgfpathlineto{\pgfqpoint{0.532655in}{0.288388in}}%
\pgfpathlineto{\pgfqpoint{0.526418in}{0.324750in}}%
\pgfpathlineto{\pgfqpoint{0.552836in}{0.350501in}}%
\pgfpathlineto{\pgfqpoint{0.516327in}{0.355806in}}%
\pgfpathlineto{\pgfqpoint{0.500000in}{0.388889in}}%
\pgfpathmoveto{\pgfqpoint{0.666667in}{0.388889in}}%
\pgfpathlineto{\pgfqpoint{0.650339in}{0.355806in}}%
\pgfpathlineto{\pgfqpoint{0.613830in}{0.350501in}}%
\pgfpathlineto{\pgfqpoint{0.640248in}{0.324750in}}%
\pgfpathlineto{\pgfqpoint{0.634012in}{0.288388in}}%
\pgfpathlineto{\pgfqpoint{0.666667in}{0.305556in}}%
\pgfpathlineto{\pgfqpoint{0.699321in}{0.288388in}}%
\pgfpathlineto{\pgfqpoint{0.693085in}{0.324750in}}%
\pgfpathlineto{\pgfqpoint{0.719503in}{0.350501in}}%
\pgfpathlineto{\pgfqpoint{0.682994in}{0.355806in}}%
\pgfpathlineto{\pgfqpoint{0.666667in}{0.388889in}}%
\pgfpathmoveto{\pgfqpoint{0.833333in}{0.388889in}}%
\pgfpathlineto{\pgfqpoint{0.817006in}{0.355806in}}%
\pgfpathlineto{\pgfqpoint{0.780497in}{0.350501in}}%
\pgfpathlineto{\pgfqpoint{0.806915in}{0.324750in}}%
\pgfpathlineto{\pgfqpoint{0.800679in}{0.288388in}}%
\pgfpathlineto{\pgfqpoint{0.833333in}{0.305556in}}%
\pgfpathlineto{\pgfqpoint{0.865988in}{0.288388in}}%
\pgfpathlineto{\pgfqpoint{0.859752in}{0.324750in}}%
\pgfpathlineto{\pgfqpoint{0.886170in}{0.350501in}}%
\pgfpathlineto{\pgfqpoint{0.849661in}{0.355806in}}%
\pgfpathlineto{\pgfqpoint{0.833333in}{0.388889in}}%
\pgfpathmoveto{\pgfqpoint{1.000000in}{0.388889in}}%
\pgfpathlineto{\pgfqpoint{0.983673in}{0.355806in}}%
\pgfpathlineto{\pgfqpoint{0.947164in}{0.350501in}}%
\pgfpathlineto{\pgfqpoint{0.973582in}{0.324750in}}%
\pgfpathlineto{\pgfqpoint{0.967345in}{0.288388in}}%
\pgfpathlineto{\pgfqpoint{1.000000in}{0.305556in}}%
\pgfpathlineto{\pgfqpoint{1.032655in}{0.288388in}}%
\pgfpathlineto{\pgfqpoint{1.026418in}{0.324750in}}%
\pgfpathlineto{\pgfqpoint{1.052836in}{0.350501in}}%
\pgfpathlineto{\pgfqpoint{1.016327in}{0.355806in}}%
\pgfpathlineto{\pgfqpoint{1.000000in}{0.388889in}}%
\pgfpathmoveto{\pgfqpoint{0.083333in}{0.555556in}}%
\pgfpathlineto{\pgfqpoint{0.067006in}{0.522473in}}%
\pgfpathlineto{\pgfqpoint{0.030497in}{0.517168in}}%
\pgfpathlineto{\pgfqpoint{0.056915in}{0.491416in}}%
\pgfpathlineto{\pgfqpoint{0.050679in}{0.455055in}}%
\pgfpathlineto{\pgfqpoint{0.083333in}{0.472222in}}%
\pgfpathlineto{\pgfqpoint{0.115988in}{0.455055in}}%
\pgfpathlineto{\pgfqpoint{0.109752in}{0.491416in}}%
\pgfpathlineto{\pgfqpoint{0.136170in}{0.517168in}}%
\pgfpathlineto{\pgfqpoint{0.099661in}{0.522473in}}%
\pgfpathlineto{\pgfqpoint{0.083333in}{0.555556in}}%
\pgfpathmoveto{\pgfqpoint{0.250000in}{0.555556in}}%
\pgfpathlineto{\pgfqpoint{0.233673in}{0.522473in}}%
\pgfpathlineto{\pgfqpoint{0.197164in}{0.517168in}}%
\pgfpathlineto{\pgfqpoint{0.223582in}{0.491416in}}%
\pgfpathlineto{\pgfqpoint{0.217345in}{0.455055in}}%
\pgfpathlineto{\pgfqpoint{0.250000in}{0.472222in}}%
\pgfpathlineto{\pgfqpoint{0.282655in}{0.455055in}}%
\pgfpathlineto{\pgfqpoint{0.276418in}{0.491416in}}%
\pgfpathlineto{\pgfqpoint{0.302836in}{0.517168in}}%
\pgfpathlineto{\pgfqpoint{0.266327in}{0.522473in}}%
\pgfpathlineto{\pgfqpoint{0.250000in}{0.555556in}}%
\pgfpathmoveto{\pgfqpoint{0.416667in}{0.555556in}}%
\pgfpathlineto{\pgfqpoint{0.400339in}{0.522473in}}%
\pgfpathlineto{\pgfqpoint{0.363830in}{0.517168in}}%
\pgfpathlineto{\pgfqpoint{0.390248in}{0.491416in}}%
\pgfpathlineto{\pgfqpoint{0.384012in}{0.455055in}}%
\pgfpathlineto{\pgfqpoint{0.416667in}{0.472222in}}%
\pgfpathlineto{\pgfqpoint{0.449321in}{0.455055in}}%
\pgfpathlineto{\pgfqpoint{0.443085in}{0.491416in}}%
\pgfpathlineto{\pgfqpoint{0.469503in}{0.517168in}}%
\pgfpathlineto{\pgfqpoint{0.432994in}{0.522473in}}%
\pgfpathlineto{\pgfqpoint{0.416667in}{0.555556in}}%
\pgfpathmoveto{\pgfqpoint{0.583333in}{0.555556in}}%
\pgfpathlineto{\pgfqpoint{0.567006in}{0.522473in}}%
\pgfpathlineto{\pgfqpoint{0.530497in}{0.517168in}}%
\pgfpathlineto{\pgfqpoint{0.556915in}{0.491416in}}%
\pgfpathlineto{\pgfqpoint{0.550679in}{0.455055in}}%
\pgfpathlineto{\pgfqpoint{0.583333in}{0.472222in}}%
\pgfpathlineto{\pgfqpoint{0.615988in}{0.455055in}}%
\pgfpathlineto{\pgfqpoint{0.609752in}{0.491416in}}%
\pgfpathlineto{\pgfqpoint{0.636170in}{0.517168in}}%
\pgfpathlineto{\pgfqpoint{0.599661in}{0.522473in}}%
\pgfpathlineto{\pgfqpoint{0.583333in}{0.555556in}}%
\pgfpathmoveto{\pgfqpoint{0.750000in}{0.555556in}}%
\pgfpathlineto{\pgfqpoint{0.733673in}{0.522473in}}%
\pgfpathlineto{\pgfqpoint{0.697164in}{0.517168in}}%
\pgfpathlineto{\pgfqpoint{0.723582in}{0.491416in}}%
\pgfpathlineto{\pgfqpoint{0.717345in}{0.455055in}}%
\pgfpathlineto{\pgfqpoint{0.750000in}{0.472222in}}%
\pgfpathlineto{\pgfqpoint{0.782655in}{0.455055in}}%
\pgfpathlineto{\pgfqpoint{0.776418in}{0.491416in}}%
\pgfpathlineto{\pgfqpoint{0.802836in}{0.517168in}}%
\pgfpathlineto{\pgfqpoint{0.766327in}{0.522473in}}%
\pgfpathlineto{\pgfqpoint{0.750000in}{0.555556in}}%
\pgfpathmoveto{\pgfqpoint{0.916667in}{0.555556in}}%
\pgfpathlineto{\pgfqpoint{0.900339in}{0.522473in}}%
\pgfpathlineto{\pgfqpoint{0.863830in}{0.517168in}}%
\pgfpathlineto{\pgfqpoint{0.890248in}{0.491416in}}%
\pgfpathlineto{\pgfqpoint{0.884012in}{0.455055in}}%
\pgfpathlineto{\pgfqpoint{0.916667in}{0.472222in}}%
\pgfpathlineto{\pgfqpoint{0.949321in}{0.455055in}}%
\pgfpathlineto{\pgfqpoint{0.943085in}{0.491416in}}%
\pgfpathlineto{\pgfqpoint{0.969503in}{0.517168in}}%
\pgfpathlineto{\pgfqpoint{0.932994in}{0.522473in}}%
\pgfpathlineto{\pgfqpoint{0.916667in}{0.555556in}}%
\pgfpathmoveto{\pgfqpoint{0.000000in}{0.722222in}}%
\pgfpathlineto{\pgfqpoint{-0.016327in}{0.689139in}}%
\pgfpathlineto{\pgfqpoint{-0.052836in}{0.683834in}}%
\pgfpathlineto{\pgfqpoint{-0.026418in}{0.658083in}}%
\pgfpathlineto{\pgfqpoint{-0.032655in}{0.621721in}}%
\pgfpathlineto{\pgfqpoint{-0.000000in}{0.638889in}}%
\pgfpathlineto{\pgfqpoint{0.032655in}{0.621721in}}%
\pgfpathlineto{\pgfqpoint{0.026418in}{0.658083in}}%
\pgfpathlineto{\pgfqpoint{0.052836in}{0.683834in}}%
\pgfpathlineto{\pgfqpoint{0.016327in}{0.689139in}}%
\pgfpathlineto{\pgfqpoint{0.000000in}{0.722222in}}%
\pgfpathmoveto{\pgfqpoint{0.166667in}{0.722222in}}%
\pgfpathlineto{\pgfqpoint{0.150339in}{0.689139in}}%
\pgfpathlineto{\pgfqpoint{0.113830in}{0.683834in}}%
\pgfpathlineto{\pgfqpoint{0.140248in}{0.658083in}}%
\pgfpathlineto{\pgfqpoint{0.134012in}{0.621721in}}%
\pgfpathlineto{\pgfqpoint{0.166667in}{0.638889in}}%
\pgfpathlineto{\pgfqpoint{0.199321in}{0.621721in}}%
\pgfpathlineto{\pgfqpoint{0.193085in}{0.658083in}}%
\pgfpathlineto{\pgfqpoint{0.219503in}{0.683834in}}%
\pgfpathlineto{\pgfqpoint{0.182994in}{0.689139in}}%
\pgfpathlineto{\pgfqpoint{0.166667in}{0.722222in}}%
\pgfpathmoveto{\pgfqpoint{0.333333in}{0.722222in}}%
\pgfpathlineto{\pgfqpoint{0.317006in}{0.689139in}}%
\pgfpathlineto{\pgfqpoint{0.280497in}{0.683834in}}%
\pgfpathlineto{\pgfqpoint{0.306915in}{0.658083in}}%
\pgfpathlineto{\pgfqpoint{0.300679in}{0.621721in}}%
\pgfpathlineto{\pgfqpoint{0.333333in}{0.638889in}}%
\pgfpathlineto{\pgfqpoint{0.365988in}{0.621721in}}%
\pgfpathlineto{\pgfqpoint{0.359752in}{0.658083in}}%
\pgfpathlineto{\pgfqpoint{0.386170in}{0.683834in}}%
\pgfpathlineto{\pgfqpoint{0.349661in}{0.689139in}}%
\pgfpathlineto{\pgfqpoint{0.333333in}{0.722222in}}%
\pgfpathmoveto{\pgfqpoint{0.500000in}{0.722222in}}%
\pgfpathlineto{\pgfqpoint{0.483673in}{0.689139in}}%
\pgfpathlineto{\pgfqpoint{0.447164in}{0.683834in}}%
\pgfpathlineto{\pgfqpoint{0.473582in}{0.658083in}}%
\pgfpathlineto{\pgfqpoint{0.467345in}{0.621721in}}%
\pgfpathlineto{\pgfqpoint{0.500000in}{0.638889in}}%
\pgfpathlineto{\pgfqpoint{0.532655in}{0.621721in}}%
\pgfpathlineto{\pgfqpoint{0.526418in}{0.658083in}}%
\pgfpathlineto{\pgfqpoint{0.552836in}{0.683834in}}%
\pgfpathlineto{\pgfqpoint{0.516327in}{0.689139in}}%
\pgfpathlineto{\pgfqpoint{0.500000in}{0.722222in}}%
\pgfpathmoveto{\pgfqpoint{0.666667in}{0.722222in}}%
\pgfpathlineto{\pgfqpoint{0.650339in}{0.689139in}}%
\pgfpathlineto{\pgfqpoint{0.613830in}{0.683834in}}%
\pgfpathlineto{\pgfqpoint{0.640248in}{0.658083in}}%
\pgfpathlineto{\pgfqpoint{0.634012in}{0.621721in}}%
\pgfpathlineto{\pgfqpoint{0.666667in}{0.638889in}}%
\pgfpathlineto{\pgfqpoint{0.699321in}{0.621721in}}%
\pgfpathlineto{\pgfqpoint{0.693085in}{0.658083in}}%
\pgfpathlineto{\pgfqpoint{0.719503in}{0.683834in}}%
\pgfpathlineto{\pgfqpoint{0.682994in}{0.689139in}}%
\pgfpathlineto{\pgfqpoint{0.666667in}{0.722222in}}%
\pgfpathmoveto{\pgfqpoint{0.833333in}{0.722222in}}%
\pgfpathlineto{\pgfqpoint{0.817006in}{0.689139in}}%
\pgfpathlineto{\pgfqpoint{0.780497in}{0.683834in}}%
\pgfpathlineto{\pgfqpoint{0.806915in}{0.658083in}}%
\pgfpathlineto{\pgfqpoint{0.800679in}{0.621721in}}%
\pgfpathlineto{\pgfqpoint{0.833333in}{0.638889in}}%
\pgfpathlineto{\pgfqpoint{0.865988in}{0.621721in}}%
\pgfpathlineto{\pgfqpoint{0.859752in}{0.658083in}}%
\pgfpathlineto{\pgfqpoint{0.886170in}{0.683834in}}%
\pgfpathlineto{\pgfqpoint{0.849661in}{0.689139in}}%
\pgfpathlineto{\pgfqpoint{0.833333in}{0.722222in}}%
\pgfpathmoveto{\pgfqpoint{1.000000in}{0.722222in}}%
\pgfpathlineto{\pgfqpoint{0.983673in}{0.689139in}}%
\pgfpathlineto{\pgfqpoint{0.947164in}{0.683834in}}%
\pgfpathlineto{\pgfqpoint{0.973582in}{0.658083in}}%
\pgfpathlineto{\pgfqpoint{0.967345in}{0.621721in}}%
\pgfpathlineto{\pgfqpoint{1.000000in}{0.638889in}}%
\pgfpathlineto{\pgfqpoint{1.032655in}{0.621721in}}%
\pgfpathlineto{\pgfqpoint{1.026418in}{0.658083in}}%
\pgfpathlineto{\pgfqpoint{1.052836in}{0.683834in}}%
\pgfpathlineto{\pgfqpoint{1.016327in}{0.689139in}}%
\pgfpathlineto{\pgfqpoint{1.000000in}{0.722222in}}%
\pgfpathmoveto{\pgfqpoint{0.083333in}{0.888889in}}%
\pgfpathlineto{\pgfqpoint{0.067006in}{0.855806in}}%
\pgfpathlineto{\pgfqpoint{0.030497in}{0.850501in}}%
\pgfpathlineto{\pgfqpoint{0.056915in}{0.824750in}}%
\pgfpathlineto{\pgfqpoint{0.050679in}{0.788388in}}%
\pgfpathlineto{\pgfqpoint{0.083333in}{0.805556in}}%
\pgfpathlineto{\pgfqpoint{0.115988in}{0.788388in}}%
\pgfpathlineto{\pgfqpoint{0.109752in}{0.824750in}}%
\pgfpathlineto{\pgfqpoint{0.136170in}{0.850501in}}%
\pgfpathlineto{\pgfqpoint{0.099661in}{0.855806in}}%
\pgfpathlineto{\pgfqpoint{0.083333in}{0.888889in}}%
\pgfpathmoveto{\pgfqpoint{0.250000in}{0.888889in}}%
\pgfpathlineto{\pgfqpoint{0.233673in}{0.855806in}}%
\pgfpathlineto{\pgfqpoint{0.197164in}{0.850501in}}%
\pgfpathlineto{\pgfqpoint{0.223582in}{0.824750in}}%
\pgfpathlineto{\pgfqpoint{0.217345in}{0.788388in}}%
\pgfpathlineto{\pgfqpoint{0.250000in}{0.805556in}}%
\pgfpathlineto{\pgfqpoint{0.282655in}{0.788388in}}%
\pgfpathlineto{\pgfqpoint{0.276418in}{0.824750in}}%
\pgfpathlineto{\pgfqpoint{0.302836in}{0.850501in}}%
\pgfpathlineto{\pgfqpoint{0.266327in}{0.855806in}}%
\pgfpathlineto{\pgfqpoint{0.250000in}{0.888889in}}%
\pgfpathmoveto{\pgfqpoint{0.416667in}{0.888889in}}%
\pgfpathlineto{\pgfqpoint{0.400339in}{0.855806in}}%
\pgfpathlineto{\pgfqpoint{0.363830in}{0.850501in}}%
\pgfpathlineto{\pgfqpoint{0.390248in}{0.824750in}}%
\pgfpathlineto{\pgfqpoint{0.384012in}{0.788388in}}%
\pgfpathlineto{\pgfqpoint{0.416667in}{0.805556in}}%
\pgfpathlineto{\pgfqpoint{0.449321in}{0.788388in}}%
\pgfpathlineto{\pgfqpoint{0.443085in}{0.824750in}}%
\pgfpathlineto{\pgfqpoint{0.469503in}{0.850501in}}%
\pgfpathlineto{\pgfqpoint{0.432994in}{0.855806in}}%
\pgfpathlineto{\pgfqpoint{0.416667in}{0.888889in}}%
\pgfpathmoveto{\pgfqpoint{0.583333in}{0.888889in}}%
\pgfpathlineto{\pgfqpoint{0.567006in}{0.855806in}}%
\pgfpathlineto{\pgfqpoint{0.530497in}{0.850501in}}%
\pgfpathlineto{\pgfqpoint{0.556915in}{0.824750in}}%
\pgfpathlineto{\pgfqpoint{0.550679in}{0.788388in}}%
\pgfpathlineto{\pgfqpoint{0.583333in}{0.805556in}}%
\pgfpathlineto{\pgfqpoint{0.615988in}{0.788388in}}%
\pgfpathlineto{\pgfqpoint{0.609752in}{0.824750in}}%
\pgfpathlineto{\pgfqpoint{0.636170in}{0.850501in}}%
\pgfpathlineto{\pgfqpoint{0.599661in}{0.855806in}}%
\pgfpathlineto{\pgfqpoint{0.583333in}{0.888889in}}%
\pgfpathmoveto{\pgfqpoint{0.750000in}{0.888889in}}%
\pgfpathlineto{\pgfqpoint{0.733673in}{0.855806in}}%
\pgfpathlineto{\pgfqpoint{0.697164in}{0.850501in}}%
\pgfpathlineto{\pgfqpoint{0.723582in}{0.824750in}}%
\pgfpathlineto{\pgfqpoint{0.717345in}{0.788388in}}%
\pgfpathlineto{\pgfqpoint{0.750000in}{0.805556in}}%
\pgfpathlineto{\pgfqpoint{0.782655in}{0.788388in}}%
\pgfpathlineto{\pgfqpoint{0.776418in}{0.824750in}}%
\pgfpathlineto{\pgfqpoint{0.802836in}{0.850501in}}%
\pgfpathlineto{\pgfqpoint{0.766327in}{0.855806in}}%
\pgfpathlineto{\pgfqpoint{0.750000in}{0.888889in}}%
\pgfpathmoveto{\pgfqpoint{0.916667in}{0.888889in}}%
\pgfpathlineto{\pgfqpoint{0.900339in}{0.855806in}}%
\pgfpathlineto{\pgfqpoint{0.863830in}{0.850501in}}%
\pgfpathlineto{\pgfqpoint{0.890248in}{0.824750in}}%
\pgfpathlineto{\pgfqpoint{0.884012in}{0.788388in}}%
\pgfpathlineto{\pgfqpoint{0.916667in}{0.805556in}}%
\pgfpathlineto{\pgfqpoint{0.949321in}{0.788388in}}%
\pgfpathlineto{\pgfqpoint{0.943085in}{0.824750in}}%
\pgfpathlineto{\pgfqpoint{0.969503in}{0.850501in}}%
\pgfpathlineto{\pgfqpoint{0.932994in}{0.855806in}}%
\pgfpathlineto{\pgfqpoint{0.916667in}{0.888889in}}%
\pgfpathmoveto{\pgfqpoint{0.000000in}{1.055556in}}%
\pgfpathlineto{\pgfqpoint{-0.016327in}{1.022473in}}%
\pgfpathlineto{\pgfqpoint{-0.052836in}{1.017168in}}%
\pgfpathlineto{\pgfqpoint{-0.026418in}{0.991416in}}%
\pgfpathlineto{\pgfqpoint{-0.032655in}{0.955055in}}%
\pgfpathlineto{\pgfqpoint{-0.000000in}{0.972222in}}%
\pgfpathlineto{\pgfqpoint{0.032655in}{0.955055in}}%
\pgfpathlineto{\pgfqpoint{0.026418in}{0.991416in}}%
\pgfpathlineto{\pgfqpoint{0.052836in}{1.017168in}}%
\pgfpathlineto{\pgfqpoint{0.016327in}{1.022473in}}%
\pgfpathlineto{\pgfqpoint{0.000000in}{1.055556in}}%
\pgfpathmoveto{\pgfqpoint{0.166667in}{1.055556in}}%
\pgfpathlineto{\pgfqpoint{0.150339in}{1.022473in}}%
\pgfpathlineto{\pgfqpoint{0.113830in}{1.017168in}}%
\pgfpathlineto{\pgfqpoint{0.140248in}{0.991416in}}%
\pgfpathlineto{\pgfqpoint{0.134012in}{0.955055in}}%
\pgfpathlineto{\pgfqpoint{0.166667in}{0.972222in}}%
\pgfpathlineto{\pgfqpoint{0.199321in}{0.955055in}}%
\pgfpathlineto{\pgfqpoint{0.193085in}{0.991416in}}%
\pgfpathlineto{\pgfqpoint{0.219503in}{1.017168in}}%
\pgfpathlineto{\pgfqpoint{0.182994in}{1.022473in}}%
\pgfpathlineto{\pgfqpoint{0.166667in}{1.055556in}}%
\pgfpathmoveto{\pgfqpoint{0.333333in}{1.055556in}}%
\pgfpathlineto{\pgfqpoint{0.317006in}{1.022473in}}%
\pgfpathlineto{\pgfqpoint{0.280497in}{1.017168in}}%
\pgfpathlineto{\pgfqpoint{0.306915in}{0.991416in}}%
\pgfpathlineto{\pgfqpoint{0.300679in}{0.955055in}}%
\pgfpathlineto{\pgfqpoint{0.333333in}{0.972222in}}%
\pgfpathlineto{\pgfqpoint{0.365988in}{0.955055in}}%
\pgfpathlineto{\pgfqpoint{0.359752in}{0.991416in}}%
\pgfpathlineto{\pgfqpoint{0.386170in}{1.017168in}}%
\pgfpathlineto{\pgfqpoint{0.349661in}{1.022473in}}%
\pgfpathlineto{\pgfqpoint{0.333333in}{1.055556in}}%
\pgfpathmoveto{\pgfqpoint{0.500000in}{1.055556in}}%
\pgfpathlineto{\pgfqpoint{0.483673in}{1.022473in}}%
\pgfpathlineto{\pgfqpoint{0.447164in}{1.017168in}}%
\pgfpathlineto{\pgfqpoint{0.473582in}{0.991416in}}%
\pgfpathlineto{\pgfqpoint{0.467345in}{0.955055in}}%
\pgfpathlineto{\pgfqpoint{0.500000in}{0.972222in}}%
\pgfpathlineto{\pgfqpoint{0.532655in}{0.955055in}}%
\pgfpathlineto{\pgfqpoint{0.526418in}{0.991416in}}%
\pgfpathlineto{\pgfqpoint{0.552836in}{1.017168in}}%
\pgfpathlineto{\pgfqpoint{0.516327in}{1.022473in}}%
\pgfpathlineto{\pgfqpoint{0.500000in}{1.055556in}}%
\pgfpathmoveto{\pgfqpoint{0.666667in}{1.055556in}}%
\pgfpathlineto{\pgfqpoint{0.650339in}{1.022473in}}%
\pgfpathlineto{\pgfqpoint{0.613830in}{1.017168in}}%
\pgfpathlineto{\pgfqpoint{0.640248in}{0.991416in}}%
\pgfpathlineto{\pgfqpoint{0.634012in}{0.955055in}}%
\pgfpathlineto{\pgfqpoint{0.666667in}{0.972222in}}%
\pgfpathlineto{\pgfqpoint{0.699321in}{0.955055in}}%
\pgfpathlineto{\pgfqpoint{0.693085in}{0.991416in}}%
\pgfpathlineto{\pgfqpoint{0.719503in}{1.017168in}}%
\pgfpathlineto{\pgfqpoint{0.682994in}{1.022473in}}%
\pgfpathlineto{\pgfqpoint{0.666667in}{1.055556in}}%
\pgfpathmoveto{\pgfqpoint{0.833333in}{1.055556in}}%
\pgfpathlineto{\pgfqpoint{0.817006in}{1.022473in}}%
\pgfpathlineto{\pgfqpoint{0.780497in}{1.017168in}}%
\pgfpathlineto{\pgfqpoint{0.806915in}{0.991416in}}%
\pgfpathlineto{\pgfqpoint{0.800679in}{0.955055in}}%
\pgfpathlineto{\pgfqpoint{0.833333in}{0.972222in}}%
\pgfpathlineto{\pgfqpoint{0.865988in}{0.955055in}}%
\pgfpathlineto{\pgfqpoint{0.859752in}{0.991416in}}%
\pgfpathlineto{\pgfqpoint{0.886170in}{1.017168in}}%
\pgfpathlineto{\pgfqpoint{0.849661in}{1.022473in}}%
\pgfpathlineto{\pgfqpoint{0.833333in}{1.055556in}}%
\pgfpathmoveto{\pgfqpoint{1.000000in}{1.055556in}}%
\pgfpathlineto{\pgfqpoint{0.983673in}{1.022473in}}%
\pgfpathlineto{\pgfqpoint{0.947164in}{1.017168in}}%
\pgfpathlineto{\pgfqpoint{0.973582in}{0.991416in}}%
\pgfpathlineto{\pgfqpoint{0.967345in}{0.955055in}}%
\pgfpathlineto{\pgfqpoint{1.000000in}{0.972222in}}%
\pgfpathlineto{\pgfqpoint{1.032655in}{0.955055in}}%
\pgfpathlineto{\pgfqpoint{1.026418in}{0.991416in}}%
\pgfpathlineto{\pgfqpoint{1.052836in}{1.017168in}}%
\pgfpathlineto{\pgfqpoint{1.016327in}{1.022473in}}%
\pgfpathlineto{\pgfqpoint{1.000000in}{1.055556in}}%
\pgfpathlineto{\pgfqpoint{1.000000in}{1.055556in}}%
\pgfusepath{stroke}%
\end{pgfscope}%
}%
\pgfsys@transformshift{5.908038in}{3.636205in}%
\pgfsys@useobject{currentpattern}{}%
\pgfsys@transformshift{1in}{0in}%
\pgfsys@transformshift{-1in}{0in}%
\pgfsys@transformshift{0in}{1in}%
\pgfsys@useobject{currentpattern}{}%
\pgfsys@transformshift{1in}{0in}%
\pgfsys@transformshift{-1in}{0in}%
\pgfsys@transformshift{0in}{1in}%
\end{pgfscope}%
\begin{pgfscope}%
\pgfpathrectangle{\pgfqpoint{0.870538in}{0.637495in}}{\pgfqpoint{9.300000in}{9.060000in}}%
\pgfusepath{clip}%
\pgfsetbuttcap%
\pgfsetmiterjoin%
\definecolor{currentfill}{rgb}{1.000000,1.000000,0.000000}%
\pgfsetfillcolor{currentfill}%
\pgfsetfillopacity{0.990000}%
\pgfsetlinewidth{0.000000pt}%
\definecolor{currentstroke}{rgb}{0.000000,0.000000,0.000000}%
\pgfsetstrokecolor{currentstroke}%
\pgfsetstrokeopacity{0.990000}%
\pgfsetdash{}{0pt}%
\pgfpathmoveto{\pgfqpoint{7.458038in}{3.877341in}}%
\pgfpathlineto{\pgfqpoint{8.233038in}{3.877341in}}%
\pgfpathlineto{\pgfqpoint{8.233038in}{5.048106in}}%
\pgfpathlineto{\pgfqpoint{7.458038in}{5.048106in}}%
\pgfpathclose%
\pgfusepath{fill}%
\end{pgfscope}%
\begin{pgfscope}%
\pgfsetbuttcap%
\pgfsetmiterjoin%
\definecolor{currentfill}{rgb}{1.000000,1.000000,0.000000}%
\pgfsetfillcolor{currentfill}%
\pgfsetfillopacity{0.990000}%
\pgfsetlinewidth{0.000000pt}%
\definecolor{currentstroke}{rgb}{0.000000,0.000000,0.000000}%
\pgfsetstrokecolor{currentstroke}%
\pgfsetstrokeopacity{0.990000}%
\pgfsetdash{}{0pt}%
\pgfpathrectangle{\pgfqpoint{0.870538in}{0.637495in}}{\pgfqpoint{9.300000in}{9.060000in}}%
\pgfusepath{clip}%
\pgfpathmoveto{\pgfqpoint{7.458038in}{3.877341in}}%
\pgfpathlineto{\pgfqpoint{8.233038in}{3.877341in}}%
\pgfpathlineto{\pgfqpoint{8.233038in}{5.048106in}}%
\pgfpathlineto{\pgfqpoint{7.458038in}{5.048106in}}%
\pgfpathclose%
\pgfusepath{clip}%
\pgfsys@defobject{currentpattern}{\pgfqpoint{0in}{0in}}{\pgfqpoint{1in}{1in}}{%
\begin{pgfscope}%
\pgfpathrectangle{\pgfqpoint{0in}{0in}}{\pgfqpoint{1in}{1in}}%
\pgfusepath{clip}%
\pgfpathmoveto{\pgfqpoint{0.000000in}{0.055556in}}%
\pgfpathlineto{\pgfqpoint{-0.016327in}{0.022473in}}%
\pgfpathlineto{\pgfqpoint{-0.052836in}{0.017168in}}%
\pgfpathlineto{\pgfqpoint{-0.026418in}{-0.008584in}}%
\pgfpathlineto{\pgfqpoint{-0.032655in}{-0.044945in}}%
\pgfpathlineto{\pgfqpoint{-0.000000in}{-0.027778in}}%
\pgfpathlineto{\pgfqpoint{0.032655in}{-0.044945in}}%
\pgfpathlineto{\pgfqpoint{0.026418in}{-0.008584in}}%
\pgfpathlineto{\pgfqpoint{0.052836in}{0.017168in}}%
\pgfpathlineto{\pgfqpoint{0.016327in}{0.022473in}}%
\pgfpathlineto{\pgfqpoint{0.000000in}{0.055556in}}%
\pgfpathmoveto{\pgfqpoint{0.166667in}{0.055556in}}%
\pgfpathlineto{\pgfqpoint{0.150339in}{0.022473in}}%
\pgfpathlineto{\pgfqpoint{0.113830in}{0.017168in}}%
\pgfpathlineto{\pgfqpoint{0.140248in}{-0.008584in}}%
\pgfpathlineto{\pgfqpoint{0.134012in}{-0.044945in}}%
\pgfpathlineto{\pgfqpoint{0.166667in}{-0.027778in}}%
\pgfpathlineto{\pgfqpoint{0.199321in}{-0.044945in}}%
\pgfpathlineto{\pgfqpoint{0.193085in}{-0.008584in}}%
\pgfpathlineto{\pgfqpoint{0.219503in}{0.017168in}}%
\pgfpathlineto{\pgfqpoint{0.182994in}{0.022473in}}%
\pgfpathlineto{\pgfqpoint{0.166667in}{0.055556in}}%
\pgfpathmoveto{\pgfqpoint{0.333333in}{0.055556in}}%
\pgfpathlineto{\pgfqpoint{0.317006in}{0.022473in}}%
\pgfpathlineto{\pgfqpoint{0.280497in}{0.017168in}}%
\pgfpathlineto{\pgfqpoint{0.306915in}{-0.008584in}}%
\pgfpathlineto{\pgfqpoint{0.300679in}{-0.044945in}}%
\pgfpathlineto{\pgfqpoint{0.333333in}{-0.027778in}}%
\pgfpathlineto{\pgfqpoint{0.365988in}{-0.044945in}}%
\pgfpathlineto{\pgfqpoint{0.359752in}{-0.008584in}}%
\pgfpathlineto{\pgfqpoint{0.386170in}{0.017168in}}%
\pgfpathlineto{\pgfqpoint{0.349661in}{0.022473in}}%
\pgfpathlineto{\pgfqpoint{0.333333in}{0.055556in}}%
\pgfpathmoveto{\pgfqpoint{0.500000in}{0.055556in}}%
\pgfpathlineto{\pgfqpoint{0.483673in}{0.022473in}}%
\pgfpathlineto{\pgfqpoint{0.447164in}{0.017168in}}%
\pgfpathlineto{\pgfqpoint{0.473582in}{-0.008584in}}%
\pgfpathlineto{\pgfqpoint{0.467345in}{-0.044945in}}%
\pgfpathlineto{\pgfqpoint{0.500000in}{-0.027778in}}%
\pgfpathlineto{\pgfqpoint{0.532655in}{-0.044945in}}%
\pgfpathlineto{\pgfqpoint{0.526418in}{-0.008584in}}%
\pgfpathlineto{\pgfqpoint{0.552836in}{0.017168in}}%
\pgfpathlineto{\pgfqpoint{0.516327in}{0.022473in}}%
\pgfpathlineto{\pgfqpoint{0.500000in}{0.055556in}}%
\pgfpathmoveto{\pgfqpoint{0.666667in}{0.055556in}}%
\pgfpathlineto{\pgfqpoint{0.650339in}{0.022473in}}%
\pgfpathlineto{\pgfqpoint{0.613830in}{0.017168in}}%
\pgfpathlineto{\pgfqpoint{0.640248in}{-0.008584in}}%
\pgfpathlineto{\pgfqpoint{0.634012in}{-0.044945in}}%
\pgfpathlineto{\pgfqpoint{0.666667in}{-0.027778in}}%
\pgfpathlineto{\pgfqpoint{0.699321in}{-0.044945in}}%
\pgfpathlineto{\pgfqpoint{0.693085in}{-0.008584in}}%
\pgfpathlineto{\pgfqpoint{0.719503in}{0.017168in}}%
\pgfpathlineto{\pgfqpoint{0.682994in}{0.022473in}}%
\pgfpathlineto{\pgfqpoint{0.666667in}{0.055556in}}%
\pgfpathmoveto{\pgfqpoint{0.833333in}{0.055556in}}%
\pgfpathlineto{\pgfqpoint{0.817006in}{0.022473in}}%
\pgfpathlineto{\pgfqpoint{0.780497in}{0.017168in}}%
\pgfpathlineto{\pgfqpoint{0.806915in}{-0.008584in}}%
\pgfpathlineto{\pgfqpoint{0.800679in}{-0.044945in}}%
\pgfpathlineto{\pgfqpoint{0.833333in}{-0.027778in}}%
\pgfpathlineto{\pgfqpoint{0.865988in}{-0.044945in}}%
\pgfpathlineto{\pgfqpoint{0.859752in}{-0.008584in}}%
\pgfpathlineto{\pgfqpoint{0.886170in}{0.017168in}}%
\pgfpathlineto{\pgfqpoint{0.849661in}{0.022473in}}%
\pgfpathlineto{\pgfqpoint{0.833333in}{0.055556in}}%
\pgfpathmoveto{\pgfqpoint{1.000000in}{0.055556in}}%
\pgfpathlineto{\pgfqpoint{0.983673in}{0.022473in}}%
\pgfpathlineto{\pgfqpoint{0.947164in}{0.017168in}}%
\pgfpathlineto{\pgfqpoint{0.973582in}{-0.008584in}}%
\pgfpathlineto{\pgfqpoint{0.967345in}{-0.044945in}}%
\pgfpathlineto{\pgfqpoint{1.000000in}{-0.027778in}}%
\pgfpathlineto{\pgfqpoint{1.032655in}{-0.044945in}}%
\pgfpathlineto{\pgfqpoint{1.026418in}{-0.008584in}}%
\pgfpathlineto{\pgfqpoint{1.052836in}{0.017168in}}%
\pgfpathlineto{\pgfqpoint{1.016327in}{0.022473in}}%
\pgfpathlineto{\pgfqpoint{1.000000in}{0.055556in}}%
\pgfpathmoveto{\pgfqpoint{0.083333in}{0.222222in}}%
\pgfpathlineto{\pgfqpoint{0.067006in}{0.189139in}}%
\pgfpathlineto{\pgfqpoint{0.030497in}{0.183834in}}%
\pgfpathlineto{\pgfqpoint{0.056915in}{0.158083in}}%
\pgfpathlineto{\pgfqpoint{0.050679in}{0.121721in}}%
\pgfpathlineto{\pgfqpoint{0.083333in}{0.138889in}}%
\pgfpathlineto{\pgfqpoint{0.115988in}{0.121721in}}%
\pgfpathlineto{\pgfqpoint{0.109752in}{0.158083in}}%
\pgfpathlineto{\pgfqpoint{0.136170in}{0.183834in}}%
\pgfpathlineto{\pgfqpoint{0.099661in}{0.189139in}}%
\pgfpathlineto{\pgfqpoint{0.083333in}{0.222222in}}%
\pgfpathmoveto{\pgfqpoint{0.250000in}{0.222222in}}%
\pgfpathlineto{\pgfqpoint{0.233673in}{0.189139in}}%
\pgfpathlineto{\pgfqpoint{0.197164in}{0.183834in}}%
\pgfpathlineto{\pgfqpoint{0.223582in}{0.158083in}}%
\pgfpathlineto{\pgfqpoint{0.217345in}{0.121721in}}%
\pgfpathlineto{\pgfqpoint{0.250000in}{0.138889in}}%
\pgfpathlineto{\pgfqpoint{0.282655in}{0.121721in}}%
\pgfpathlineto{\pgfqpoint{0.276418in}{0.158083in}}%
\pgfpathlineto{\pgfqpoint{0.302836in}{0.183834in}}%
\pgfpathlineto{\pgfqpoint{0.266327in}{0.189139in}}%
\pgfpathlineto{\pgfqpoint{0.250000in}{0.222222in}}%
\pgfpathmoveto{\pgfqpoint{0.416667in}{0.222222in}}%
\pgfpathlineto{\pgfqpoint{0.400339in}{0.189139in}}%
\pgfpathlineto{\pgfqpoint{0.363830in}{0.183834in}}%
\pgfpathlineto{\pgfqpoint{0.390248in}{0.158083in}}%
\pgfpathlineto{\pgfqpoint{0.384012in}{0.121721in}}%
\pgfpathlineto{\pgfqpoint{0.416667in}{0.138889in}}%
\pgfpathlineto{\pgfqpoint{0.449321in}{0.121721in}}%
\pgfpathlineto{\pgfqpoint{0.443085in}{0.158083in}}%
\pgfpathlineto{\pgfqpoint{0.469503in}{0.183834in}}%
\pgfpathlineto{\pgfqpoint{0.432994in}{0.189139in}}%
\pgfpathlineto{\pgfqpoint{0.416667in}{0.222222in}}%
\pgfpathmoveto{\pgfqpoint{0.583333in}{0.222222in}}%
\pgfpathlineto{\pgfqpoint{0.567006in}{0.189139in}}%
\pgfpathlineto{\pgfqpoint{0.530497in}{0.183834in}}%
\pgfpathlineto{\pgfqpoint{0.556915in}{0.158083in}}%
\pgfpathlineto{\pgfqpoint{0.550679in}{0.121721in}}%
\pgfpathlineto{\pgfqpoint{0.583333in}{0.138889in}}%
\pgfpathlineto{\pgfqpoint{0.615988in}{0.121721in}}%
\pgfpathlineto{\pgfqpoint{0.609752in}{0.158083in}}%
\pgfpathlineto{\pgfqpoint{0.636170in}{0.183834in}}%
\pgfpathlineto{\pgfqpoint{0.599661in}{0.189139in}}%
\pgfpathlineto{\pgfqpoint{0.583333in}{0.222222in}}%
\pgfpathmoveto{\pgfqpoint{0.750000in}{0.222222in}}%
\pgfpathlineto{\pgfqpoint{0.733673in}{0.189139in}}%
\pgfpathlineto{\pgfqpoint{0.697164in}{0.183834in}}%
\pgfpathlineto{\pgfqpoint{0.723582in}{0.158083in}}%
\pgfpathlineto{\pgfqpoint{0.717345in}{0.121721in}}%
\pgfpathlineto{\pgfqpoint{0.750000in}{0.138889in}}%
\pgfpathlineto{\pgfqpoint{0.782655in}{0.121721in}}%
\pgfpathlineto{\pgfqpoint{0.776418in}{0.158083in}}%
\pgfpathlineto{\pgfqpoint{0.802836in}{0.183834in}}%
\pgfpathlineto{\pgfqpoint{0.766327in}{0.189139in}}%
\pgfpathlineto{\pgfqpoint{0.750000in}{0.222222in}}%
\pgfpathmoveto{\pgfqpoint{0.916667in}{0.222222in}}%
\pgfpathlineto{\pgfqpoint{0.900339in}{0.189139in}}%
\pgfpathlineto{\pgfqpoint{0.863830in}{0.183834in}}%
\pgfpathlineto{\pgfqpoint{0.890248in}{0.158083in}}%
\pgfpathlineto{\pgfqpoint{0.884012in}{0.121721in}}%
\pgfpathlineto{\pgfqpoint{0.916667in}{0.138889in}}%
\pgfpathlineto{\pgfqpoint{0.949321in}{0.121721in}}%
\pgfpathlineto{\pgfqpoint{0.943085in}{0.158083in}}%
\pgfpathlineto{\pgfqpoint{0.969503in}{0.183834in}}%
\pgfpathlineto{\pgfqpoint{0.932994in}{0.189139in}}%
\pgfpathlineto{\pgfqpoint{0.916667in}{0.222222in}}%
\pgfpathmoveto{\pgfqpoint{0.000000in}{0.388889in}}%
\pgfpathlineto{\pgfqpoint{-0.016327in}{0.355806in}}%
\pgfpathlineto{\pgfqpoint{-0.052836in}{0.350501in}}%
\pgfpathlineto{\pgfqpoint{-0.026418in}{0.324750in}}%
\pgfpathlineto{\pgfqpoint{-0.032655in}{0.288388in}}%
\pgfpathlineto{\pgfqpoint{-0.000000in}{0.305556in}}%
\pgfpathlineto{\pgfqpoint{0.032655in}{0.288388in}}%
\pgfpathlineto{\pgfqpoint{0.026418in}{0.324750in}}%
\pgfpathlineto{\pgfqpoint{0.052836in}{0.350501in}}%
\pgfpathlineto{\pgfqpoint{0.016327in}{0.355806in}}%
\pgfpathlineto{\pgfqpoint{0.000000in}{0.388889in}}%
\pgfpathmoveto{\pgfqpoint{0.166667in}{0.388889in}}%
\pgfpathlineto{\pgfqpoint{0.150339in}{0.355806in}}%
\pgfpathlineto{\pgfqpoint{0.113830in}{0.350501in}}%
\pgfpathlineto{\pgfqpoint{0.140248in}{0.324750in}}%
\pgfpathlineto{\pgfqpoint{0.134012in}{0.288388in}}%
\pgfpathlineto{\pgfqpoint{0.166667in}{0.305556in}}%
\pgfpathlineto{\pgfqpoint{0.199321in}{0.288388in}}%
\pgfpathlineto{\pgfqpoint{0.193085in}{0.324750in}}%
\pgfpathlineto{\pgfqpoint{0.219503in}{0.350501in}}%
\pgfpathlineto{\pgfqpoint{0.182994in}{0.355806in}}%
\pgfpathlineto{\pgfqpoint{0.166667in}{0.388889in}}%
\pgfpathmoveto{\pgfqpoint{0.333333in}{0.388889in}}%
\pgfpathlineto{\pgfqpoint{0.317006in}{0.355806in}}%
\pgfpathlineto{\pgfqpoint{0.280497in}{0.350501in}}%
\pgfpathlineto{\pgfqpoint{0.306915in}{0.324750in}}%
\pgfpathlineto{\pgfqpoint{0.300679in}{0.288388in}}%
\pgfpathlineto{\pgfqpoint{0.333333in}{0.305556in}}%
\pgfpathlineto{\pgfqpoint{0.365988in}{0.288388in}}%
\pgfpathlineto{\pgfqpoint{0.359752in}{0.324750in}}%
\pgfpathlineto{\pgfqpoint{0.386170in}{0.350501in}}%
\pgfpathlineto{\pgfqpoint{0.349661in}{0.355806in}}%
\pgfpathlineto{\pgfqpoint{0.333333in}{0.388889in}}%
\pgfpathmoveto{\pgfqpoint{0.500000in}{0.388889in}}%
\pgfpathlineto{\pgfqpoint{0.483673in}{0.355806in}}%
\pgfpathlineto{\pgfqpoint{0.447164in}{0.350501in}}%
\pgfpathlineto{\pgfqpoint{0.473582in}{0.324750in}}%
\pgfpathlineto{\pgfqpoint{0.467345in}{0.288388in}}%
\pgfpathlineto{\pgfqpoint{0.500000in}{0.305556in}}%
\pgfpathlineto{\pgfqpoint{0.532655in}{0.288388in}}%
\pgfpathlineto{\pgfqpoint{0.526418in}{0.324750in}}%
\pgfpathlineto{\pgfqpoint{0.552836in}{0.350501in}}%
\pgfpathlineto{\pgfqpoint{0.516327in}{0.355806in}}%
\pgfpathlineto{\pgfqpoint{0.500000in}{0.388889in}}%
\pgfpathmoveto{\pgfqpoint{0.666667in}{0.388889in}}%
\pgfpathlineto{\pgfqpoint{0.650339in}{0.355806in}}%
\pgfpathlineto{\pgfqpoint{0.613830in}{0.350501in}}%
\pgfpathlineto{\pgfqpoint{0.640248in}{0.324750in}}%
\pgfpathlineto{\pgfqpoint{0.634012in}{0.288388in}}%
\pgfpathlineto{\pgfqpoint{0.666667in}{0.305556in}}%
\pgfpathlineto{\pgfqpoint{0.699321in}{0.288388in}}%
\pgfpathlineto{\pgfqpoint{0.693085in}{0.324750in}}%
\pgfpathlineto{\pgfqpoint{0.719503in}{0.350501in}}%
\pgfpathlineto{\pgfqpoint{0.682994in}{0.355806in}}%
\pgfpathlineto{\pgfqpoint{0.666667in}{0.388889in}}%
\pgfpathmoveto{\pgfqpoint{0.833333in}{0.388889in}}%
\pgfpathlineto{\pgfqpoint{0.817006in}{0.355806in}}%
\pgfpathlineto{\pgfqpoint{0.780497in}{0.350501in}}%
\pgfpathlineto{\pgfqpoint{0.806915in}{0.324750in}}%
\pgfpathlineto{\pgfqpoint{0.800679in}{0.288388in}}%
\pgfpathlineto{\pgfqpoint{0.833333in}{0.305556in}}%
\pgfpathlineto{\pgfqpoint{0.865988in}{0.288388in}}%
\pgfpathlineto{\pgfqpoint{0.859752in}{0.324750in}}%
\pgfpathlineto{\pgfqpoint{0.886170in}{0.350501in}}%
\pgfpathlineto{\pgfqpoint{0.849661in}{0.355806in}}%
\pgfpathlineto{\pgfqpoint{0.833333in}{0.388889in}}%
\pgfpathmoveto{\pgfqpoint{1.000000in}{0.388889in}}%
\pgfpathlineto{\pgfqpoint{0.983673in}{0.355806in}}%
\pgfpathlineto{\pgfqpoint{0.947164in}{0.350501in}}%
\pgfpathlineto{\pgfqpoint{0.973582in}{0.324750in}}%
\pgfpathlineto{\pgfqpoint{0.967345in}{0.288388in}}%
\pgfpathlineto{\pgfqpoint{1.000000in}{0.305556in}}%
\pgfpathlineto{\pgfqpoint{1.032655in}{0.288388in}}%
\pgfpathlineto{\pgfqpoint{1.026418in}{0.324750in}}%
\pgfpathlineto{\pgfqpoint{1.052836in}{0.350501in}}%
\pgfpathlineto{\pgfqpoint{1.016327in}{0.355806in}}%
\pgfpathlineto{\pgfqpoint{1.000000in}{0.388889in}}%
\pgfpathmoveto{\pgfqpoint{0.083333in}{0.555556in}}%
\pgfpathlineto{\pgfqpoint{0.067006in}{0.522473in}}%
\pgfpathlineto{\pgfqpoint{0.030497in}{0.517168in}}%
\pgfpathlineto{\pgfqpoint{0.056915in}{0.491416in}}%
\pgfpathlineto{\pgfqpoint{0.050679in}{0.455055in}}%
\pgfpathlineto{\pgfqpoint{0.083333in}{0.472222in}}%
\pgfpathlineto{\pgfqpoint{0.115988in}{0.455055in}}%
\pgfpathlineto{\pgfqpoint{0.109752in}{0.491416in}}%
\pgfpathlineto{\pgfqpoint{0.136170in}{0.517168in}}%
\pgfpathlineto{\pgfqpoint{0.099661in}{0.522473in}}%
\pgfpathlineto{\pgfqpoint{0.083333in}{0.555556in}}%
\pgfpathmoveto{\pgfqpoint{0.250000in}{0.555556in}}%
\pgfpathlineto{\pgfqpoint{0.233673in}{0.522473in}}%
\pgfpathlineto{\pgfqpoint{0.197164in}{0.517168in}}%
\pgfpathlineto{\pgfqpoint{0.223582in}{0.491416in}}%
\pgfpathlineto{\pgfqpoint{0.217345in}{0.455055in}}%
\pgfpathlineto{\pgfqpoint{0.250000in}{0.472222in}}%
\pgfpathlineto{\pgfqpoint{0.282655in}{0.455055in}}%
\pgfpathlineto{\pgfqpoint{0.276418in}{0.491416in}}%
\pgfpathlineto{\pgfqpoint{0.302836in}{0.517168in}}%
\pgfpathlineto{\pgfqpoint{0.266327in}{0.522473in}}%
\pgfpathlineto{\pgfqpoint{0.250000in}{0.555556in}}%
\pgfpathmoveto{\pgfqpoint{0.416667in}{0.555556in}}%
\pgfpathlineto{\pgfqpoint{0.400339in}{0.522473in}}%
\pgfpathlineto{\pgfqpoint{0.363830in}{0.517168in}}%
\pgfpathlineto{\pgfqpoint{0.390248in}{0.491416in}}%
\pgfpathlineto{\pgfqpoint{0.384012in}{0.455055in}}%
\pgfpathlineto{\pgfqpoint{0.416667in}{0.472222in}}%
\pgfpathlineto{\pgfqpoint{0.449321in}{0.455055in}}%
\pgfpathlineto{\pgfqpoint{0.443085in}{0.491416in}}%
\pgfpathlineto{\pgfqpoint{0.469503in}{0.517168in}}%
\pgfpathlineto{\pgfqpoint{0.432994in}{0.522473in}}%
\pgfpathlineto{\pgfqpoint{0.416667in}{0.555556in}}%
\pgfpathmoveto{\pgfqpoint{0.583333in}{0.555556in}}%
\pgfpathlineto{\pgfqpoint{0.567006in}{0.522473in}}%
\pgfpathlineto{\pgfqpoint{0.530497in}{0.517168in}}%
\pgfpathlineto{\pgfqpoint{0.556915in}{0.491416in}}%
\pgfpathlineto{\pgfqpoint{0.550679in}{0.455055in}}%
\pgfpathlineto{\pgfqpoint{0.583333in}{0.472222in}}%
\pgfpathlineto{\pgfqpoint{0.615988in}{0.455055in}}%
\pgfpathlineto{\pgfqpoint{0.609752in}{0.491416in}}%
\pgfpathlineto{\pgfqpoint{0.636170in}{0.517168in}}%
\pgfpathlineto{\pgfqpoint{0.599661in}{0.522473in}}%
\pgfpathlineto{\pgfqpoint{0.583333in}{0.555556in}}%
\pgfpathmoveto{\pgfqpoint{0.750000in}{0.555556in}}%
\pgfpathlineto{\pgfqpoint{0.733673in}{0.522473in}}%
\pgfpathlineto{\pgfqpoint{0.697164in}{0.517168in}}%
\pgfpathlineto{\pgfqpoint{0.723582in}{0.491416in}}%
\pgfpathlineto{\pgfqpoint{0.717345in}{0.455055in}}%
\pgfpathlineto{\pgfqpoint{0.750000in}{0.472222in}}%
\pgfpathlineto{\pgfqpoint{0.782655in}{0.455055in}}%
\pgfpathlineto{\pgfqpoint{0.776418in}{0.491416in}}%
\pgfpathlineto{\pgfqpoint{0.802836in}{0.517168in}}%
\pgfpathlineto{\pgfqpoint{0.766327in}{0.522473in}}%
\pgfpathlineto{\pgfqpoint{0.750000in}{0.555556in}}%
\pgfpathmoveto{\pgfqpoint{0.916667in}{0.555556in}}%
\pgfpathlineto{\pgfqpoint{0.900339in}{0.522473in}}%
\pgfpathlineto{\pgfqpoint{0.863830in}{0.517168in}}%
\pgfpathlineto{\pgfqpoint{0.890248in}{0.491416in}}%
\pgfpathlineto{\pgfqpoint{0.884012in}{0.455055in}}%
\pgfpathlineto{\pgfqpoint{0.916667in}{0.472222in}}%
\pgfpathlineto{\pgfqpoint{0.949321in}{0.455055in}}%
\pgfpathlineto{\pgfqpoint{0.943085in}{0.491416in}}%
\pgfpathlineto{\pgfqpoint{0.969503in}{0.517168in}}%
\pgfpathlineto{\pgfqpoint{0.932994in}{0.522473in}}%
\pgfpathlineto{\pgfqpoint{0.916667in}{0.555556in}}%
\pgfpathmoveto{\pgfqpoint{0.000000in}{0.722222in}}%
\pgfpathlineto{\pgfqpoint{-0.016327in}{0.689139in}}%
\pgfpathlineto{\pgfqpoint{-0.052836in}{0.683834in}}%
\pgfpathlineto{\pgfqpoint{-0.026418in}{0.658083in}}%
\pgfpathlineto{\pgfqpoint{-0.032655in}{0.621721in}}%
\pgfpathlineto{\pgfqpoint{-0.000000in}{0.638889in}}%
\pgfpathlineto{\pgfqpoint{0.032655in}{0.621721in}}%
\pgfpathlineto{\pgfqpoint{0.026418in}{0.658083in}}%
\pgfpathlineto{\pgfqpoint{0.052836in}{0.683834in}}%
\pgfpathlineto{\pgfqpoint{0.016327in}{0.689139in}}%
\pgfpathlineto{\pgfqpoint{0.000000in}{0.722222in}}%
\pgfpathmoveto{\pgfqpoint{0.166667in}{0.722222in}}%
\pgfpathlineto{\pgfqpoint{0.150339in}{0.689139in}}%
\pgfpathlineto{\pgfqpoint{0.113830in}{0.683834in}}%
\pgfpathlineto{\pgfqpoint{0.140248in}{0.658083in}}%
\pgfpathlineto{\pgfqpoint{0.134012in}{0.621721in}}%
\pgfpathlineto{\pgfqpoint{0.166667in}{0.638889in}}%
\pgfpathlineto{\pgfqpoint{0.199321in}{0.621721in}}%
\pgfpathlineto{\pgfqpoint{0.193085in}{0.658083in}}%
\pgfpathlineto{\pgfqpoint{0.219503in}{0.683834in}}%
\pgfpathlineto{\pgfqpoint{0.182994in}{0.689139in}}%
\pgfpathlineto{\pgfqpoint{0.166667in}{0.722222in}}%
\pgfpathmoveto{\pgfqpoint{0.333333in}{0.722222in}}%
\pgfpathlineto{\pgfqpoint{0.317006in}{0.689139in}}%
\pgfpathlineto{\pgfqpoint{0.280497in}{0.683834in}}%
\pgfpathlineto{\pgfqpoint{0.306915in}{0.658083in}}%
\pgfpathlineto{\pgfqpoint{0.300679in}{0.621721in}}%
\pgfpathlineto{\pgfqpoint{0.333333in}{0.638889in}}%
\pgfpathlineto{\pgfqpoint{0.365988in}{0.621721in}}%
\pgfpathlineto{\pgfqpoint{0.359752in}{0.658083in}}%
\pgfpathlineto{\pgfqpoint{0.386170in}{0.683834in}}%
\pgfpathlineto{\pgfqpoint{0.349661in}{0.689139in}}%
\pgfpathlineto{\pgfqpoint{0.333333in}{0.722222in}}%
\pgfpathmoveto{\pgfqpoint{0.500000in}{0.722222in}}%
\pgfpathlineto{\pgfqpoint{0.483673in}{0.689139in}}%
\pgfpathlineto{\pgfqpoint{0.447164in}{0.683834in}}%
\pgfpathlineto{\pgfqpoint{0.473582in}{0.658083in}}%
\pgfpathlineto{\pgfqpoint{0.467345in}{0.621721in}}%
\pgfpathlineto{\pgfqpoint{0.500000in}{0.638889in}}%
\pgfpathlineto{\pgfqpoint{0.532655in}{0.621721in}}%
\pgfpathlineto{\pgfqpoint{0.526418in}{0.658083in}}%
\pgfpathlineto{\pgfqpoint{0.552836in}{0.683834in}}%
\pgfpathlineto{\pgfqpoint{0.516327in}{0.689139in}}%
\pgfpathlineto{\pgfqpoint{0.500000in}{0.722222in}}%
\pgfpathmoveto{\pgfqpoint{0.666667in}{0.722222in}}%
\pgfpathlineto{\pgfqpoint{0.650339in}{0.689139in}}%
\pgfpathlineto{\pgfqpoint{0.613830in}{0.683834in}}%
\pgfpathlineto{\pgfqpoint{0.640248in}{0.658083in}}%
\pgfpathlineto{\pgfqpoint{0.634012in}{0.621721in}}%
\pgfpathlineto{\pgfqpoint{0.666667in}{0.638889in}}%
\pgfpathlineto{\pgfqpoint{0.699321in}{0.621721in}}%
\pgfpathlineto{\pgfqpoint{0.693085in}{0.658083in}}%
\pgfpathlineto{\pgfqpoint{0.719503in}{0.683834in}}%
\pgfpathlineto{\pgfqpoint{0.682994in}{0.689139in}}%
\pgfpathlineto{\pgfqpoint{0.666667in}{0.722222in}}%
\pgfpathmoveto{\pgfqpoint{0.833333in}{0.722222in}}%
\pgfpathlineto{\pgfqpoint{0.817006in}{0.689139in}}%
\pgfpathlineto{\pgfqpoint{0.780497in}{0.683834in}}%
\pgfpathlineto{\pgfqpoint{0.806915in}{0.658083in}}%
\pgfpathlineto{\pgfqpoint{0.800679in}{0.621721in}}%
\pgfpathlineto{\pgfqpoint{0.833333in}{0.638889in}}%
\pgfpathlineto{\pgfqpoint{0.865988in}{0.621721in}}%
\pgfpathlineto{\pgfqpoint{0.859752in}{0.658083in}}%
\pgfpathlineto{\pgfqpoint{0.886170in}{0.683834in}}%
\pgfpathlineto{\pgfqpoint{0.849661in}{0.689139in}}%
\pgfpathlineto{\pgfqpoint{0.833333in}{0.722222in}}%
\pgfpathmoveto{\pgfqpoint{1.000000in}{0.722222in}}%
\pgfpathlineto{\pgfqpoint{0.983673in}{0.689139in}}%
\pgfpathlineto{\pgfqpoint{0.947164in}{0.683834in}}%
\pgfpathlineto{\pgfqpoint{0.973582in}{0.658083in}}%
\pgfpathlineto{\pgfqpoint{0.967345in}{0.621721in}}%
\pgfpathlineto{\pgfqpoint{1.000000in}{0.638889in}}%
\pgfpathlineto{\pgfqpoint{1.032655in}{0.621721in}}%
\pgfpathlineto{\pgfqpoint{1.026418in}{0.658083in}}%
\pgfpathlineto{\pgfqpoint{1.052836in}{0.683834in}}%
\pgfpathlineto{\pgfqpoint{1.016327in}{0.689139in}}%
\pgfpathlineto{\pgfqpoint{1.000000in}{0.722222in}}%
\pgfpathmoveto{\pgfqpoint{0.083333in}{0.888889in}}%
\pgfpathlineto{\pgfqpoint{0.067006in}{0.855806in}}%
\pgfpathlineto{\pgfqpoint{0.030497in}{0.850501in}}%
\pgfpathlineto{\pgfqpoint{0.056915in}{0.824750in}}%
\pgfpathlineto{\pgfqpoint{0.050679in}{0.788388in}}%
\pgfpathlineto{\pgfqpoint{0.083333in}{0.805556in}}%
\pgfpathlineto{\pgfqpoint{0.115988in}{0.788388in}}%
\pgfpathlineto{\pgfqpoint{0.109752in}{0.824750in}}%
\pgfpathlineto{\pgfqpoint{0.136170in}{0.850501in}}%
\pgfpathlineto{\pgfqpoint{0.099661in}{0.855806in}}%
\pgfpathlineto{\pgfqpoint{0.083333in}{0.888889in}}%
\pgfpathmoveto{\pgfqpoint{0.250000in}{0.888889in}}%
\pgfpathlineto{\pgfqpoint{0.233673in}{0.855806in}}%
\pgfpathlineto{\pgfqpoint{0.197164in}{0.850501in}}%
\pgfpathlineto{\pgfqpoint{0.223582in}{0.824750in}}%
\pgfpathlineto{\pgfqpoint{0.217345in}{0.788388in}}%
\pgfpathlineto{\pgfqpoint{0.250000in}{0.805556in}}%
\pgfpathlineto{\pgfqpoint{0.282655in}{0.788388in}}%
\pgfpathlineto{\pgfqpoint{0.276418in}{0.824750in}}%
\pgfpathlineto{\pgfqpoint{0.302836in}{0.850501in}}%
\pgfpathlineto{\pgfqpoint{0.266327in}{0.855806in}}%
\pgfpathlineto{\pgfqpoint{0.250000in}{0.888889in}}%
\pgfpathmoveto{\pgfqpoint{0.416667in}{0.888889in}}%
\pgfpathlineto{\pgfqpoint{0.400339in}{0.855806in}}%
\pgfpathlineto{\pgfqpoint{0.363830in}{0.850501in}}%
\pgfpathlineto{\pgfqpoint{0.390248in}{0.824750in}}%
\pgfpathlineto{\pgfqpoint{0.384012in}{0.788388in}}%
\pgfpathlineto{\pgfqpoint{0.416667in}{0.805556in}}%
\pgfpathlineto{\pgfqpoint{0.449321in}{0.788388in}}%
\pgfpathlineto{\pgfqpoint{0.443085in}{0.824750in}}%
\pgfpathlineto{\pgfqpoint{0.469503in}{0.850501in}}%
\pgfpathlineto{\pgfqpoint{0.432994in}{0.855806in}}%
\pgfpathlineto{\pgfqpoint{0.416667in}{0.888889in}}%
\pgfpathmoveto{\pgfqpoint{0.583333in}{0.888889in}}%
\pgfpathlineto{\pgfqpoint{0.567006in}{0.855806in}}%
\pgfpathlineto{\pgfqpoint{0.530497in}{0.850501in}}%
\pgfpathlineto{\pgfqpoint{0.556915in}{0.824750in}}%
\pgfpathlineto{\pgfqpoint{0.550679in}{0.788388in}}%
\pgfpathlineto{\pgfqpoint{0.583333in}{0.805556in}}%
\pgfpathlineto{\pgfqpoint{0.615988in}{0.788388in}}%
\pgfpathlineto{\pgfqpoint{0.609752in}{0.824750in}}%
\pgfpathlineto{\pgfqpoint{0.636170in}{0.850501in}}%
\pgfpathlineto{\pgfqpoint{0.599661in}{0.855806in}}%
\pgfpathlineto{\pgfqpoint{0.583333in}{0.888889in}}%
\pgfpathmoveto{\pgfqpoint{0.750000in}{0.888889in}}%
\pgfpathlineto{\pgfqpoint{0.733673in}{0.855806in}}%
\pgfpathlineto{\pgfqpoint{0.697164in}{0.850501in}}%
\pgfpathlineto{\pgfqpoint{0.723582in}{0.824750in}}%
\pgfpathlineto{\pgfqpoint{0.717345in}{0.788388in}}%
\pgfpathlineto{\pgfqpoint{0.750000in}{0.805556in}}%
\pgfpathlineto{\pgfqpoint{0.782655in}{0.788388in}}%
\pgfpathlineto{\pgfqpoint{0.776418in}{0.824750in}}%
\pgfpathlineto{\pgfqpoint{0.802836in}{0.850501in}}%
\pgfpathlineto{\pgfqpoint{0.766327in}{0.855806in}}%
\pgfpathlineto{\pgfqpoint{0.750000in}{0.888889in}}%
\pgfpathmoveto{\pgfqpoint{0.916667in}{0.888889in}}%
\pgfpathlineto{\pgfqpoint{0.900339in}{0.855806in}}%
\pgfpathlineto{\pgfqpoint{0.863830in}{0.850501in}}%
\pgfpathlineto{\pgfqpoint{0.890248in}{0.824750in}}%
\pgfpathlineto{\pgfqpoint{0.884012in}{0.788388in}}%
\pgfpathlineto{\pgfqpoint{0.916667in}{0.805556in}}%
\pgfpathlineto{\pgfqpoint{0.949321in}{0.788388in}}%
\pgfpathlineto{\pgfqpoint{0.943085in}{0.824750in}}%
\pgfpathlineto{\pgfqpoint{0.969503in}{0.850501in}}%
\pgfpathlineto{\pgfqpoint{0.932994in}{0.855806in}}%
\pgfpathlineto{\pgfqpoint{0.916667in}{0.888889in}}%
\pgfpathmoveto{\pgfqpoint{0.000000in}{1.055556in}}%
\pgfpathlineto{\pgfqpoint{-0.016327in}{1.022473in}}%
\pgfpathlineto{\pgfqpoint{-0.052836in}{1.017168in}}%
\pgfpathlineto{\pgfqpoint{-0.026418in}{0.991416in}}%
\pgfpathlineto{\pgfqpoint{-0.032655in}{0.955055in}}%
\pgfpathlineto{\pgfqpoint{-0.000000in}{0.972222in}}%
\pgfpathlineto{\pgfqpoint{0.032655in}{0.955055in}}%
\pgfpathlineto{\pgfqpoint{0.026418in}{0.991416in}}%
\pgfpathlineto{\pgfqpoint{0.052836in}{1.017168in}}%
\pgfpathlineto{\pgfqpoint{0.016327in}{1.022473in}}%
\pgfpathlineto{\pgfqpoint{0.000000in}{1.055556in}}%
\pgfpathmoveto{\pgfqpoint{0.166667in}{1.055556in}}%
\pgfpathlineto{\pgfqpoint{0.150339in}{1.022473in}}%
\pgfpathlineto{\pgfqpoint{0.113830in}{1.017168in}}%
\pgfpathlineto{\pgfqpoint{0.140248in}{0.991416in}}%
\pgfpathlineto{\pgfqpoint{0.134012in}{0.955055in}}%
\pgfpathlineto{\pgfqpoint{0.166667in}{0.972222in}}%
\pgfpathlineto{\pgfqpoint{0.199321in}{0.955055in}}%
\pgfpathlineto{\pgfqpoint{0.193085in}{0.991416in}}%
\pgfpathlineto{\pgfqpoint{0.219503in}{1.017168in}}%
\pgfpathlineto{\pgfqpoint{0.182994in}{1.022473in}}%
\pgfpathlineto{\pgfqpoint{0.166667in}{1.055556in}}%
\pgfpathmoveto{\pgfqpoint{0.333333in}{1.055556in}}%
\pgfpathlineto{\pgfqpoint{0.317006in}{1.022473in}}%
\pgfpathlineto{\pgfqpoint{0.280497in}{1.017168in}}%
\pgfpathlineto{\pgfqpoint{0.306915in}{0.991416in}}%
\pgfpathlineto{\pgfqpoint{0.300679in}{0.955055in}}%
\pgfpathlineto{\pgfqpoint{0.333333in}{0.972222in}}%
\pgfpathlineto{\pgfqpoint{0.365988in}{0.955055in}}%
\pgfpathlineto{\pgfqpoint{0.359752in}{0.991416in}}%
\pgfpathlineto{\pgfqpoint{0.386170in}{1.017168in}}%
\pgfpathlineto{\pgfqpoint{0.349661in}{1.022473in}}%
\pgfpathlineto{\pgfqpoint{0.333333in}{1.055556in}}%
\pgfpathmoveto{\pgfqpoint{0.500000in}{1.055556in}}%
\pgfpathlineto{\pgfqpoint{0.483673in}{1.022473in}}%
\pgfpathlineto{\pgfqpoint{0.447164in}{1.017168in}}%
\pgfpathlineto{\pgfqpoint{0.473582in}{0.991416in}}%
\pgfpathlineto{\pgfqpoint{0.467345in}{0.955055in}}%
\pgfpathlineto{\pgfqpoint{0.500000in}{0.972222in}}%
\pgfpathlineto{\pgfqpoint{0.532655in}{0.955055in}}%
\pgfpathlineto{\pgfqpoint{0.526418in}{0.991416in}}%
\pgfpathlineto{\pgfqpoint{0.552836in}{1.017168in}}%
\pgfpathlineto{\pgfqpoint{0.516327in}{1.022473in}}%
\pgfpathlineto{\pgfqpoint{0.500000in}{1.055556in}}%
\pgfpathmoveto{\pgfqpoint{0.666667in}{1.055556in}}%
\pgfpathlineto{\pgfqpoint{0.650339in}{1.022473in}}%
\pgfpathlineto{\pgfqpoint{0.613830in}{1.017168in}}%
\pgfpathlineto{\pgfqpoint{0.640248in}{0.991416in}}%
\pgfpathlineto{\pgfqpoint{0.634012in}{0.955055in}}%
\pgfpathlineto{\pgfqpoint{0.666667in}{0.972222in}}%
\pgfpathlineto{\pgfqpoint{0.699321in}{0.955055in}}%
\pgfpathlineto{\pgfqpoint{0.693085in}{0.991416in}}%
\pgfpathlineto{\pgfqpoint{0.719503in}{1.017168in}}%
\pgfpathlineto{\pgfqpoint{0.682994in}{1.022473in}}%
\pgfpathlineto{\pgfqpoint{0.666667in}{1.055556in}}%
\pgfpathmoveto{\pgfqpoint{0.833333in}{1.055556in}}%
\pgfpathlineto{\pgfqpoint{0.817006in}{1.022473in}}%
\pgfpathlineto{\pgfqpoint{0.780497in}{1.017168in}}%
\pgfpathlineto{\pgfqpoint{0.806915in}{0.991416in}}%
\pgfpathlineto{\pgfqpoint{0.800679in}{0.955055in}}%
\pgfpathlineto{\pgfqpoint{0.833333in}{0.972222in}}%
\pgfpathlineto{\pgfqpoint{0.865988in}{0.955055in}}%
\pgfpathlineto{\pgfqpoint{0.859752in}{0.991416in}}%
\pgfpathlineto{\pgfqpoint{0.886170in}{1.017168in}}%
\pgfpathlineto{\pgfqpoint{0.849661in}{1.022473in}}%
\pgfpathlineto{\pgfqpoint{0.833333in}{1.055556in}}%
\pgfpathmoveto{\pgfqpoint{1.000000in}{1.055556in}}%
\pgfpathlineto{\pgfqpoint{0.983673in}{1.022473in}}%
\pgfpathlineto{\pgfqpoint{0.947164in}{1.017168in}}%
\pgfpathlineto{\pgfqpoint{0.973582in}{0.991416in}}%
\pgfpathlineto{\pgfqpoint{0.967345in}{0.955055in}}%
\pgfpathlineto{\pgfqpoint{1.000000in}{0.972222in}}%
\pgfpathlineto{\pgfqpoint{1.032655in}{0.955055in}}%
\pgfpathlineto{\pgfqpoint{1.026418in}{0.991416in}}%
\pgfpathlineto{\pgfqpoint{1.052836in}{1.017168in}}%
\pgfpathlineto{\pgfqpoint{1.016327in}{1.022473in}}%
\pgfpathlineto{\pgfqpoint{1.000000in}{1.055556in}}%
\pgfpathlineto{\pgfqpoint{1.000000in}{1.055556in}}%
\pgfusepath{stroke}%
\end{pgfscope}%
}%
\pgfsys@transformshift{7.458038in}{3.877341in}%
\pgfsys@useobject{currentpattern}{}%
\pgfsys@transformshift{1in}{0in}%
\pgfsys@transformshift{-1in}{0in}%
\pgfsys@transformshift{0in}{1in}%
\pgfsys@useobject{currentpattern}{}%
\pgfsys@transformshift{1in}{0in}%
\pgfsys@transformshift{-1in}{0in}%
\pgfsys@transformshift{0in}{1in}%
\end{pgfscope}%
\begin{pgfscope}%
\pgfpathrectangle{\pgfqpoint{0.870538in}{0.637495in}}{\pgfqpoint{9.300000in}{9.060000in}}%
\pgfusepath{clip}%
\pgfsetbuttcap%
\pgfsetmiterjoin%
\definecolor{currentfill}{rgb}{1.000000,1.000000,0.000000}%
\pgfsetfillcolor{currentfill}%
\pgfsetfillopacity{0.990000}%
\pgfsetlinewidth{0.000000pt}%
\definecolor{currentstroke}{rgb}{0.000000,0.000000,0.000000}%
\pgfsetstrokecolor{currentstroke}%
\pgfsetstrokeopacity{0.990000}%
\pgfsetdash{}{0pt}%
\pgfpathmoveto{\pgfqpoint{9.008038in}{4.114678in}}%
\pgfpathlineto{\pgfqpoint{9.783038in}{4.114678in}}%
\pgfpathlineto{\pgfqpoint{9.783038in}{5.392180in}}%
\pgfpathlineto{\pgfqpoint{9.008038in}{5.392180in}}%
\pgfpathclose%
\pgfusepath{fill}%
\end{pgfscope}%
\begin{pgfscope}%
\pgfsetbuttcap%
\pgfsetmiterjoin%
\definecolor{currentfill}{rgb}{1.000000,1.000000,0.000000}%
\pgfsetfillcolor{currentfill}%
\pgfsetfillopacity{0.990000}%
\pgfsetlinewidth{0.000000pt}%
\definecolor{currentstroke}{rgb}{0.000000,0.000000,0.000000}%
\pgfsetstrokecolor{currentstroke}%
\pgfsetstrokeopacity{0.990000}%
\pgfsetdash{}{0pt}%
\pgfpathrectangle{\pgfqpoint{0.870538in}{0.637495in}}{\pgfqpoint{9.300000in}{9.060000in}}%
\pgfusepath{clip}%
\pgfpathmoveto{\pgfqpoint{9.008038in}{4.114678in}}%
\pgfpathlineto{\pgfqpoint{9.783038in}{4.114678in}}%
\pgfpathlineto{\pgfqpoint{9.783038in}{5.392180in}}%
\pgfpathlineto{\pgfqpoint{9.008038in}{5.392180in}}%
\pgfpathclose%
\pgfusepath{clip}%
\pgfsys@defobject{currentpattern}{\pgfqpoint{0in}{0in}}{\pgfqpoint{1in}{1in}}{%
\begin{pgfscope}%
\pgfpathrectangle{\pgfqpoint{0in}{0in}}{\pgfqpoint{1in}{1in}}%
\pgfusepath{clip}%
\pgfpathmoveto{\pgfqpoint{0.000000in}{0.055556in}}%
\pgfpathlineto{\pgfqpoint{-0.016327in}{0.022473in}}%
\pgfpathlineto{\pgfqpoint{-0.052836in}{0.017168in}}%
\pgfpathlineto{\pgfqpoint{-0.026418in}{-0.008584in}}%
\pgfpathlineto{\pgfqpoint{-0.032655in}{-0.044945in}}%
\pgfpathlineto{\pgfqpoint{-0.000000in}{-0.027778in}}%
\pgfpathlineto{\pgfqpoint{0.032655in}{-0.044945in}}%
\pgfpathlineto{\pgfqpoint{0.026418in}{-0.008584in}}%
\pgfpathlineto{\pgfqpoint{0.052836in}{0.017168in}}%
\pgfpathlineto{\pgfqpoint{0.016327in}{0.022473in}}%
\pgfpathlineto{\pgfqpoint{0.000000in}{0.055556in}}%
\pgfpathmoveto{\pgfqpoint{0.166667in}{0.055556in}}%
\pgfpathlineto{\pgfqpoint{0.150339in}{0.022473in}}%
\pgfpathlineto{\pgfqpoint{0.113830in}{0.017168in}}%
\pgfpathlineto{\pgfqpoint{0.140248in}{-0.008584in}}%
\pgfpathlineto{\pgfqpoint{0.134012in}{-0.044945in}}%
\pgfpathlineto{\pgfqpoint{0.166667in}{-0.027778in}}%
\pgfpathlineto{\pgfqpoint{0.199321in}{-0.044945in}}%
\pgfpathlineto{\pgfqpoint{0.193085in}{-0.008584in}}%
\pgfpathlineto{\pgfqpoint{0.219503in}{0.017168in}}%
\pgfpathlineto{\pgfqpoint{0.182994in}{0.022473in}}%
\pgfpathlineto{\pgfqpoint{0.166667in}{0.055556in}}%
\pgfpathmoveto{\pgfqpoint{0.333333in}{0.055556in}}%
\pgfpathlineto{\pgfqpoint{0.317006in}{0.022473in}}%
\pgfpathlineto{\pgfqpoint{0.280497in}{0.017168in}}%
\pgfpathlineto{\pgfqpoint{0.306915in}{-0.008584in}}%
\pgfpathlineto{\pgfqpoint{0.300679in}{-0.044945in}}%
\pgfpathlineto{\pgfqpoint{0.333333in}{-0.027778in}}%
\pgfpathlineto{\pgfqpoint{0.365988in}{-0.044945in}}%
\pgfpathlineto{\pgfqpoint{0.359752in}{-0.008584in}}%
\pgfpathlineto{\pgfqpoint{0.386170in}{0.017168in}}%
\pgfpathlineto{\pgfqpoint{0.349661in}{0.022473in}}%
\pgfpathlineto{\pgfqpoint{0.333333in}{0.055556in}}%
\pgfpathmoveto{\pgfqpoint{0.500000in}{0.055556in}}%
\pgfpathlineto{\pgfqpoint{0.483673in}{0.022473in}}%
\pgfpathlineto{\pgfqpoint{0.447164in}{0.017168in}}%
\pgfpathlineto{\pgfqpoint{0.473582in}{-0.008584in}}%
\pgfpathlineto{\pgfqpoint{0.467345in}{-0.044945in}}%
\pgfpathlineto{\pgfqpoint{0.500000in}{-0.027778in}}%
\pgfpathlineto{\pgfqpoint{0.532655in}{-0.044945in}}%
\pgfpathlineto{\pgfqpoint{0.526418in}{-0.008584in}}%
\pgfpathlineto{\pgfqpoint{0.552836in}{0.017168in}}%
\pgfpathlineto{\pgfqpoint{0.516327in}{0.022473in}}%
\pgfpathlineto{\pgfqpoint{0.500000in}{0.055556in}}%
\pgfpathmoveto{\pgfqpoint{0.666667in}{0.055556in}}%
\pgfpathlineto{\pgfqpoint{0.650339in}{0.022473in}}%
\pgfpathlineto{\pgfqpoint{0.613830in}{0.017168in}}%
\pgfpathlineto{\pgfqpoint{0.640248in}{-0.008584in}}%
\pgfpathlineto{\pgfqpoint{0.634012in}{-0.044945in}}%
\pgfpathlineto{\pgfqpoint{0.666667in}{-0.027778in}}%
\pgfpathlineto{\pgfqpoint{0.699321in}{-0.044945in}}%
\pgfpathlineto{\pgfqpoint{0.693085in}{-0.008584in}}%
\pgfpathlineto{\pgfqpoint{0.719503in}{0.017168in}}%
\pgfpathlineto{\pgfqpoint{0.682994in}{0.022473in}}%
\pgfpathlineto{\pgfqpoint{0.666667in}{0.055556in}}%
\pgfpathmoveto{\pgfqpoint{0.833333in}{0.055556in}}%
\pgfpathlineto{\pgfqpoint{0.817006in}{0.022473in}}%
\pgfpathlineto{\pgfqpoint{0.780497in}{0.017168in}}%
\pgfpathlineto{\pgfqpoint{0.806915in}{-0.008584in}}%
\pgfpathlineto{\pgfqpoint{0.800679in}{-0.044945in}}%
\pgfpathlineto{\pgfqpoint{0.833333in}{-0.027778in}}%
\pgfpathlineto{\pgfqpoint{0.865988in}{-0.044945in}}%
\pgfpathlineto{\pgfqpoint{0.859752in}{-0.008584in}}%
\pgfpathlineto{\pgfqpoint{0.886170in}{0.017168in}}%
\pgfpathlineto{\pgfqpoint{0.849661in}{0.022473in}}%
\pgfpathlineto{\pgfqpoint{0.833333in}{0.055556in}}%
\pgfpathmoveto{\pgfqpoint{1.000000in}{0.055556in}}%
\pgfpathlineto{\pgfqpoint{0.983673in}{0.022473in}}%
\pgfpathlineto{\pgfqpoint{0.947164in}{0.017168in}}%
\pgfpathlineto{\pgfqpoint{0.973582in}{-0.008584in}}%
\pgfpathlineto{\pgfqpoint{0.967345in}{-0.044945in}}%
\pgfpathlineto{\pgfqpoint{1.000000in}{-0.027778in}}%
\pgfpathlineto{\pgfqpoint{1.032655in}{-0.044945in}}%
\pgfpathlineto{\pgfqpoint{1.026418in}{-0.008584in}}%
\pgfpathlineto{\pgfqpoint{1.052836in}{0.017168in}}%
\pgfpathlineto{\pgfqpoint{1.016327in}{0.022473in}}%
\pgfpathlineto{\pgfqpoint{1.000000in}{0.055556in}}%
\pgfpathmoveto{\pgfqpoint{0.083333in}{0.222222in}}%
\pgfpathlineto{\pgfqpoint{0.067006in}{0.189139in}}%
\pgfpathlineto{\pgfqpoint{0.030497in}{0.183834in}}%
\pgfpathlineto{\pgfqpoint{0.056915in}{0.158083in}}%
\pgfpathlineto{\pgfqpoint{0.050679in}{0.121721in}}%
\pgfpathlineto{\pgfqpoint{0.083333in}{0.138889in}}%
\pgfpathlineto{\pgfqpoint{0.115988in}{0.121721in}}%
\pgfpathlineto{\pgfqpoint{0.109752in}{0.158083in}}%
\pgfpathlineto{\pgfqpoint{0.136170in}{0.183834in}}%
\pgfpathlineto{\pgfqpoint{0.099661in}{0.189139in}}%
\pgfpathlineto{\pgfqpoint{0.083333in}{0.222222in}}%
\pgfpathmoveto{\pgfqpoint{0.250000in}{0.222222in}}%
\pgfpathlineto{\pgfqpoint{0.233673in}{0.189139in}}%
\pgfpathlineto{\pgfqpoint{0.197164in}{0.183834in}}%
\pgfpathlineto{\pgfqpoint{0.223582in}{0.158083in}}%
\pgfpathlineto{\pgfqpoint{0.217345in}{0.121721in}}%
\pgfpathlineto{\pgfqpoint{0.250000in}{0.138889in}}%
\pgfpathlineto{\pgfqpoint{0.282655in}{0.121721in}}%
\pgfpathlineto{\pgfqpoint{0.276418in}{0.158083in}}%
\pgfpathlineto{\pgfqpoint{0.302836in}{0.183834in}}%
\pgfpathlineto{\pgfqpoint{0.266327in}{0.189139in}}%
\pgfpathlineto{\pgfqpoint{0.250000in}{0.222222in}}%
\pgfpathmoveto{\pgfqpoint{0.416667in}{0.222222in}}%
\pgfpathlineto{\pgfqpoint{0.400339in}{0.189139in}}%
\pgfpathlineto{\pgfqpoint{0.363830in}{0.183834in}}%
\pgfpathlineto{\pgfqpoint{0.390248in}{0.158083in}}%
\pgfpathlineto{\pgfqpoint{0.384012in}{0.121721in}}%
\pgfpathlineto{\pgfqpoint{0.416667in}{0.138889in}}%
\pgfpathlineto{\pgfqpoint{0.449321in}{0.121721in}}%
\pgfpathlineto{\pgfqpoint{0.443085in}{0.158083in}}%
\pgfpathlineto{\pgfqpoint{0.469503in}{0.183834in}}%
\pgfpathlineto{\pgfqpoint{0.432994in}{0.189139in}}%
\pgfpathlineto{\pgfqpoint{0.416667in}{0.222222in}}%
\pgfpathmoveto{\pgfqpoint{0.583333in}{0.222222in}}%
\pgfpathlineto{\pgfqpoint{0.567006in}{0.189139in}}%
\pgfpathlineto{\pgfqpoint{0.530497in}{0.183834in}}%
\pgfpathlineto{\pgfqpoint{0.556915in}{0.158083in}}%
\pgfpathlineto{\pgfqpoint{0.550679in}{0.121721in}}%
\pgfpathlineto{\pgfqpoint{0.583333in}{0.138889in}}%
\pgfpathlineto{\pgfqpoint{0.615988in}{0.121721in}}%
\pgfpathlineto{\pgfqpoint{0.609752in}{0.158083in}}%
\pgfpathlineto{\pgfqpoint{0.636170in}{0.183834in}}%
\pgfpathlineto{\pgfqpoint{0.599661in}{0.189139in}}%
\pgfpathlineto{\pgfqpoint{0.583333in}{0.222222in}}%
\pgfpathmoveto{\pgfqpoint{0.750000in}{0.222222in}}%
\pgfpathlineto{\pgfqpoint{0.733673in}{0.189139in}}%
\pgfpathlineto{\pgfqpoint{0.697164in}{0.183834in}}%
\pgfpathlineto{\pgfqpoint{0.723582in}{0.158083in}}%
\pgfpathlineto{\pgfqpoint{0.717345in}{0.121721in}}%
\pgfpathlineto{\pgfqpoint{0.750000in}{0.138889in}}%
\pgfpathlineto{\pgfqpoint{0.782655in}{0.121721in}}%
\pgfpathlineto{\pgfqpoint{0.776418in}{0.158083in}}%
\pgfpathlineto{\pgfqpoint{0.802836in}{0.183834in}}%
\pgfpathlineto{\pgfqpoint{0.766327in}{0.189139in}}%
\pgfpathlineto{\pgfqpoint{0.750000in}{0.222222in}}%
\pgfpathmoveto{\pgfqpoint{0.916667in}{0.222222in}}%
\pgfpathlineto{\pgfqpoint{0.900339in}{0.189139in}}%
\pgfpathlineto{\pgfqpoint{0.863830in}{0.183834in}}%
\pgfpathlineto{\pgfqpoint{0.890248in}{0.158083in}}%
\pgfpathlineto{\pgfqpoint{0.884012in}{0.121721in}}%
\pgfpathlineto{\pgfqpoint{0.916667in}{0.138889in}}%
\pgfpathlineto{\pgfqpoint{0.949321in}{0.121721in}}%
\pgfpathlineto{\pgfqpoint{0.943085in}{0.158083in}}%
\pgfpathlineto{\pgfqpoint{0.969503in}{0.183834in}}%
\pgfpathlineto{\pgfqpoint{0.932994in}{0.189139in}}%
\pgfpathlineto{\pgfqpoint{0.916667in}{0.222222in}}%
\pgfpathmoveto{\pgfqpoint{0.000000in}{0.388889in}}%
\pgfpathlineto{\pgfqpoint{-0.016327in}{0.355806in}}%
\pgfpathlineto{\pgfqpoint{-0.052836in}{0.350501in}}%
\pgfpathlineto{\pgfqpoint{-0.026418in}{0.324750in}}%
\pgfpathlineto{\pgfqpoint{-0.032655in}{0.288388in}}%
\pgfpathlineto{\pgfqpoint{-0.000000in}{0.305556in}}%
\pgfpathlineto{\pgfqpoint{0.032655in}{0.288388in}}%
\pgfpathlineto{\pgfqpoint{0.026418in}{0.324750in}}%
\pgfpathlineto{\pgfqpoint{0.052836in}{0.350501in}}%
\pgfpathlineto{\pgfqpoint{0.016327in}{0.355806in}}%
\pgfpathlineto{\pgfqpoint{0.000000in}{0.388889in}}%
\pgfpathmoveto{\pgfqpoint{0.166667in}{0.388889in}}%
\pgfpathlineto{\pgfqpoint{0.150339in}{0.355806in}}%
\pgfpathlineto{\pgfqpoint{0.113830in}{0.350501in}}%
\pgfpathlineto{\pgfqpoint{0.140248in}{0.324750in}}%
\pgfpathlineto{\pgfqpoint{0.134012in}{0.288388in}}%
\pgfpathlineto{\pgfqpoint{0.166667in}{0.305556in}}%
\pgfpathlineto{\pgfqpoint{0.199321in}{0.288388in}}%
\pgfpathlineto{\pgfqpoint{0.193085in}{0.324750in}}%
\pgfpathlineto{\pgfqpoint{0.219503in}{0.350501in}}%
\pgfpathlineto{\pgfqpoint{0.182994in}{0.355806in}}%
\pgfpathlineto{\pgfqpoint{0.166667in}{0.388889in}}%
\pgfpathmoveto{\pgfqpoint{0.333333in}{0.388889in}}%
\pgfpathlineto{\pgfqpoint{0.317006in}{0.355806in}}%
\pgfpathlineto{\pgfqpoint{0.280497in}{0.350501in}}%
\pgfpathlineto{\pgfqpoint{0.306915in}{0.324750in}}%
\pgfpathlineto{\pgfqpoint{0.300679in}{0.288388in}}%
\pgfpathlineto{\pgfqpoint{0.333333in}{0.305556in}}%
\pgfpathlineto{\pgfqpoint{0.365988in}{0.288388in}}%
\pgfpathlineto{\pgfqpoint{0.359752in}{0.324750in}}%
\pgfpathlineto{\pgfqpoint{0.386170in}{0.350501in}}%
\pgfpathlineto{\pgfqpoint{0.349661in}{0.355806in}}%
\pgfpathlineto{\pgfqpoint{0.333333in}{0.388889in}}%
\pgfpathmoveto{\pgfqpoint{0.500000in}{0.388889in}}%
\pgfpathlineto{\pgfqpoint{0.483673in}{0.355806in}}%
\pgfpathlineto{\pgfqpoint{0.447164in}{0.350501in}}%
\pgfpathlineto{\pgfqpoint{0.473582in}{0.324750in}}%
\pgfpathlineto{\pgfqpoint{0.467345in}{0.288388in}}%
\pgfpathlineto{\pgfqpoint{0.500000in}{0.305556in}}%
\pgfpathlineto{\pgfqpoint{0.532655in}{0.288388in}}%
\pgfpathlineto{\pgfqpoint{0.526418in}{0.324750in}}%
\pgfpathlineto{\pgfqpoint{0.552836in}{0.350501in}}%
\pgfpathlineto{\pgfqpoint{0.516327in}{0.355806in}}%
\pgfpathlineto{\pgfqpoint{0.500000in}{0.388889in}}%
\pgfpathmoveto{\pgfqpoint{0.666667in}{0.388889in}}%
\pgfpathlineto{\pgfqpoint{0.650339in}{0.355806in}}%
\pgfpathlineto{\pgfqpoint{0.613830in}{0.350501in}}%
\pgfpathlineto{\pgfqpoint{0.640248in}{0.324750in}}%
\pgfpathlineto{\pgfqpoint{0.634012in}{0.288388in}}%
\pgfpathlineto{\pgfqpoint{0.666667in}{0.305556in}}%
\pgfpathlineto{\pgfqpoint{0.699321in}{0.288388in}}%
\pgfpathlineto{\pgfqpoint{0.693085in}{0.324750in}}%
\pgfpathlineto{\pgfqpoint{0.719503in}{0.350501in}}%
\pgfpathlineto{\pgfqpoint{0.682994in}{0.355806in}}%
\pgfpathlineto{\pgfqpoint{0.666667in}{0.388889in}}%
\pgfpathmoveto{\pgfqpoint{0.833333in}{0.388889in}}%
\pgfpathlineto{\pgfqpoint{0.817006in}{0.355806in}}%
\pgfpathlineto{\pgfqpoint{0.780497in}{0.350501in}}%
\pgfpathlineto{\pgfqpoint{0.806915in}{0.324750in}}%
\pgfpathlineto{\pgfqpoint{0.800679in}{0.288388in}}%
\pgfpathlineto{\pgfqpoint{0.833333in}{0.305556in}}%
\pgfpathlineto{\pgfqpoint{0.865988in}{0.288388in}}%
\pgfpathlineto{\pgfqpoint{0.859752in}{0.324750in}}%
\pgfpathlineto{\pgfqpoint{0.886170in}{0.350501in}}%
\pgfpathlineto{\pgfqpoint{0.849661in}{0.355806in}}%
\pgfpathlineto{\pgfqpoint{0.833333in}{0.388889in}}%
\pgfpathmoveto{\pgfqpoint{1.000000in}{0.388889in}}%
\pgfpathlineto{\pgfqpoint{0.983673in}{0.355806in}}%
\pgfpathlineto{\pgfqpoint{0.947164in}{0.350501in}}%
\pgfpathlineto{\pgfqpoint{0.973582in}{0.324750in}}%
\pgfpathlineto{\pgfqpoint{0.967345in}{0.288388in}}%
\pgfpathlineto{\pgfqpoint{1.000000in}{0.305556in}}%
\pgfpathlineto{\pgfqpoint{1.032655in}{0.288388in}}%
\pgfpathlineto{\pgfqpoint{1.026418in}{0.324750in}}%
\pgfpathlineto{\pgfqpoint{1.052836in}{0.350501in}}%
\pgfpathlineto{\pgfqpoint{1.016327in}{0.355806in}}%
\pgfpathlineto{\pgfqpoint{1.000000in}{0.388889in}}%
\pgfpathmoveto{\pgfqpoint{0.083333in}{0.555556in}}%
\pgfpathlineto{\pgfqpoint{0.067006in}{0.522473in}}%
\pgfpathlineto{\pgfqpoint{0.030497in}{0.517168in}}%
\pgfpathlineto{\pgfqpoint{0.056915in}{0.491416in}}%
\pgfpathlineto{\pgfqpoint{0.050679in}{0.455055in}}%
\pgfpathlineto{\pgfqpoint{0.083333in}{0.472222in}}%
\pgfpathlineto{\pgfqpoint{0.115988in}{0.455055in}}%
\pgfpathlineto{\pgfqpoint{0.109752in}{0.491416in}}%
\pgfpathlineto{\pgfqpoint{0.136170in}{0.517168in}}%
\pgfpathlineto{\pgfqpoint{0.099661in}{0.522473in}}%
\pgfpathlineto{\pgfqpoint{0.083333in}{0.555556in}}%
\pgfpathmoveto{\pgfqpoint{0.250000in}{0.555556in}}%
\pgfpathlineto{\pgfqpoint{0.233673in}{0.522473in}}%
\pgfpathlineto{\pgfqpoint{0.197164in}{0.517168in}}%
\pgfpathlineto{\pgfqpoint{0.223582in}{0.491416in}}%
\pgfpathlineto{\pgfqpoint{0.217345in}{0.455055in}}%
\pgfpathlineto{\pgfqpoint{0.250000in}{0.472222in}}%
\pgfpathlineto{\pgfqpoint{0.282655in}{0.455055in}}%
\pgfpathlineto{\pgfqpoint{0.276418in}{0.491416in}}%
\pgfpathlineto{\pgfqpoint{0.302836in}{0.517168in}}%
\pgfpathlineto{\pgfqpoint{0.266327in}{0.522473in}}%
\pgfpathlineto{\pgfqpoint{0.250000in}{0.555556in}}%
\pgfpathmoveto{\pgfqpoint{0.416667in}{0.555556in}}%
\pgfpathlineto{\pgfqpoint{0.400339in}{0.522473in}}%
\pgfpathlineto{\pgfqpoint{0.363830in}{0.517168in}}%
\pgfpathlineto{\pgfqpoint{0.390248in}{0.491416in}}%
\pgfpathlineto{\pgfqpoint{0.384012in}{0.455055in}}%
\pgfpathlineto{\pgfqpoint{0.416667in}{0.472222in}}%
\pgfpathlineto{\pgfqpoint{0.449321in}{0.455055in}}%
\pgfpathlineto{\pgfqpoint{0.443085in}{0.491416in}}%
\pgfpathlineto{\pgfqpoint{0.469503in}{0.517168in}}%
\pgfpathlineto{\pgfqpoint{0.432994in}{0.522473in}}%
\pgfpathlineto{\pgfqpoint{0.416667in}{0.555556in}}%
\pgfpathmoveto{\pgfqpoint{0.583333in}{0.555556in}}%
\pgfpathlineto{\pgfqpoint{0.567006in}{0.522473in}}%
\pgfpathlineto{\pgfqpoint{0.530497in}{0.517168in}}%
\pgfpathlineto{\pgfqpoint{0.556915in}{0.491416in}}%
\pgfpathlineto{\pgfqpoint{0.550679in}{0.455055in}}%
\pgfpathlineto{\pgfqpoint{0.583333in}{0.472222in}}%
\pgfpathlineto{\pgfqpoint{0.615988in}{0.455055in}}%
\pgfpathlineto{\pgfqpoint{0.609752in}{0.491416in}}%
\pgfpathlineto{\pgfqpoint{0.636170in}{0.517168in}}%
\pgfpathlineto{\pgfqpoint{0.599661in}{0.522473in}}%
\pgfpathlineto{\pgfqpoint{0.583333in}{0.555556in}}%
\pgfpathmoveto{\pgfqpoint{0.750000in}{0.555556in}}%
\pgfpathlineto{\pgfqpoint{0.733673in}{0.522473in}}%
\pgfpathlineto{\pgfqpoint{0.697164in}{0.517168in}}%
\pgfpathlineto{\pgfqpoint{0.723582in}{0.491416in}}%
\pgfpathlineto{\pgfqpoint{0.717345in}{0.455055in}}%
\pgfpathlineto{\pgfqpoint{0.750000in}{0.472222in}}%
\pgfpathlineto{\pgfqpoint{0.782655in}{0.455055in}}%
\pgfpathlineto{\pgfqpoint{0.776418in}{0.491416in}}%
\pgfpathlineto{\pgfqpoint{0.802836in}{0.517168in}}%
\pgfpathlineto{\pgfqpoint{0.766327in}{0.522473in}}%
\pgfpathlineto{\pgfqpoint{0.750000in}{0.555556in}}%
\pgfpathmoveto{\pgfqpoint{0.916667in}{0.555556in}}%
\pgfpathlineto{\pgfqpoint{0.900339in}{0.522473in}}%
\pgfpathlineto{\pgfqpoint{0.863830in}{0.517168in}}%
\pgfpathlineto{\pgfqpoint{0.890248in}{0.491416in}}%
\pgfpathlineto{\pgfqpoint{0.884012in}{0.455055in}}%
\pgfpathlineto{\pgfqpoint{0.916667in}{0.472222in}}%
\pgfpathlineto{\pgfqpoint{0.949321in}{0.455055in}}%
\pgfpathlineto{\pgfqpoint{0.943085in}{0.491416in}}%
\pgfpathlineto{\pgfqpoint{0.969503in}{0.517168in}}%
\pgfpathlineto{\pgfqpoint{0.932994in}{0.522473in}}%
\pgfpathlineto{\pgfqpoint{0.916667in}{0.555556in}}%
\pgfpathmoveto{\pgfqpoint{0.000000in}{0.722222in}}%
\pgfpathlineto{\pgfqpoint{-0.016327in}{0.689139in}}%
\pgfpathlineto{\pgfqpoint{-0.052836in}{0.683834in}}%
\pgfpathlineto{\pgfqpoint{-0.026418in}{0.658083in}}%
\pgfpathlineto{\pgfqpoint{-0.032655in}{0.621721in}}%
\pgfpathlineto{\pgfqpoint{-0.000000in}{0.638889in}}%
\pgfpathlineto{\pgfqpoint{0.032655in}{0.621721in}}%
\pgfpathlineto{\pgfqpoint{0.026418in}{0.658083in}}%
\pgfpathlineto{\pgfqpoint{0.052836in}{0.683834in}}%
\pgfpathlineto{\pgfqpoint{0.016327in}{0.689139in}}%
\pgfpathlineto{\pgfqpoint{0.000000in}{0.722222in}}%
\pgfpathmoveto{\pgfqpoint{0.166667in}{0.722222in}}%
\pgfpathlineto{\pgfqpoint{0.150339in}{0.689139in}}%
\pgfpathlineto{\pgfqpoint{0.113830in}{0.683834in}}%
\pgfpathlineto{\pgfqpoint{0.140248in}{0.658083in}}%
\pgfpathlineto{\pgfqpoint{0.134012in}{0.621721in}}%
\pgfpathlineto{\pgfqpoint{0.166667in}{0.638889in}}%
\pgfpathlineto{\pgfqpoint{0.199321in}{0.621721in}}%
\pgfpathlineto{\pgfqpoint{0.193085in}{0.658083in}}%
\pgfpathlineto{\pgfqpoint{0.219503in}{0.683834in}}%
\pgfpathlineto{\pgfqpoint{0.182994in}{0.689139in}}%
\pgfpathlineto{\pgfqpoint{0.166667in}{0.722222in}}%
\pgfpathmoveto{\pgfqpoint{0.333333in}{0.722222in}}%
\pgfpathlineto{\pgfqpoint{0.317006in}{0.689139in}}%
\pgfpathlineto{\pgfqpoint{0.280497in}{0.683834in}}%
\pgfpathlineto{\pgfqpoint{0.306915in}{0.658083in}}%
\pgfpathlineto{\pgfqpoint{0.300679in}{0.621721in}}%
\pgfpathlineto{\pgfqpoint{0.333333in}{0.638889in}}%
\pgfpathlineto{\pgfqpoint{0.365988in}{0.621721in}}%
\pgfpathlineto{\pgfqpoint{0.359752in}{0.658083in}}%
\pgfpathlineto{\pgfqpoint{0.386170in}{0.683834in}}%
\pgfpathlineto{\pgfqpoint{0.349661in}{0.689139in}}%
\pgfpathlineto{\pgfqpoint{0.333333in}{0.722222in}}%
\pgfpathmoveto{\pgfqpoint{0.500000in}{0.722222in}}%
\pgfpathlineto{\pgfqpoint{0.483673in}{0.689139in}}%
\pgfpathlineto{\pgfqpoint{0.447164in}{0.683834in}}%
\pgfpathlineto{\pgfqpoint{0.473582in}{0.658083in}}%
\pgfpathlineto{\pgfqpoint{0.467345in}{0.621721in}}%
\pgfpathlineto{\pgfqpoint{0.500000in}{0.638889in}}%
\pgfpathlineto{\pgfqpoint{0.532655in}{0.621721in}}%
\pgfpathlineto{\pgfqpoint{0.526418in}{0.658083in}}%
\pgfpathlineto{\pgfqpoint{0.552836in}{0.683834in}}%
\pgfpathlineto{\pgfqpoint{0.516327in}{0.689139in}}%
\pgfpathlineto{\pgfqpoint{0.500000in}{0.722222in}}%
\pgfpathmoveto{\pgfqpoint{0.666667in}{0.722222in}}%
\pgfpathlineto{\pgfqpoint{0.650339in}{0.689139in}}%
\pgfpathlineto{\pgfqpoint{0.613830in}{0.683834in}}%
\pgfpathlineto{\pgfqpoint{0.640248in}{0.658083in}}%
\pgfpathlineto{\pgfqpoint{0.634012in}{0.621721in}}%
\pgfpathlineto{\pgfqpoint{0.666667in}{0.638889in}}%
\pgfpathlineto{\pgfqpoint{0.699321in}{0.621721in}}%
\pgfpathlineto{\pgfqpoint{0.693085in}{0.658083in}}%
\pgfpathlineto{\pgfqpoint{0.719503in}{0.683834in}}%
\pgfpathlineto{\pgfqpoint{0.682994in}{0.689139in}}%
\pgfpathlineto{\pgfqpoint{0.666667in}{0.722222in}}%
\pgfpathmoveto{\pgfqpoint{0.833333in}{0.722222in}}%
\pgfpathlineto{\pgfqpoint{0.817006in}{0.689139in}}%
\pgfpathlineto{\pgfqpoint{0.780497in}{0.683834in}}%
\pgfpathlineto{\pgfqpoint{0.806915in}{0.658083in}}%
\pgfpathlineto{\pgfqpoint{0.800679in}{0.621721in}}%
\pgfpathlineto{\pgfqpoint{0.833333in}{0.638889in}}%
\pgfpathlineto{\pgfqpoint{0.865988in}{0.621721in}}%
\pgfpathlineto{\pgfqpoint{0.859752in}{0.658083in}}%
\pgfpathlineto{\pgfqpoint{0.886170in}{0.683834in}}%
\pgfpathlineto{\pgfqpoint{0.849661in}{0.689139in}}%
\pgfpathlineto{\pgfqpoint{0.833333in}{0.722222in}}%
\pgfpathmoveto{\pgfqpoint{1.000000in}{0.722222in}}%
\pgfpathlineto{\pgfqpoint{0.983673in}{0.689139in}}%
\pgfpathlineto{\pgfqpoint{0.947164in}{0.683834in}}%
\pgfpathlineto{\pgfqpoint{0.973582in}{0.658083in}}%
\pgfpathlineto{\pgfqpoint{0.967345in}{0.621721in}}%
\pgfpathlineto{\pgfqpoint{1.000000in}{0.638889in}}%
\pgfpathlineto{\pgfqpoint{1.032655in}{0.621721in}}%
\pgfpathlineto{\pgfqpoint{1.026418in}{0.658083in}}%
\pgfpathlineto{\pgfqpoint{1.052836in}{0.683834in}}%
\pgfpathlineto{\pgfqpoint{1.016327in}{0.689139in}}%
\pgfpathlineto{\pgfqpoint{1.000000in}{0.722222in}}%
\pgfpathmoveto{\pgfqpoint{0.083333in}{0.888889in}}%
\pgfpathlineto{\pgfqpoint{0.067006in}{0.855806in}}%
\pgfpathlineto{\pgfqpoint{0.030497in}{0.850501in}}%
\pgfpathlineto{\pgfqpoint{0.056915in}{0.824750in}}%
\pgfpathlineto{\pgfqpoint{0.050679in}{0.788388in}}%
\pgfpathlineto{\pgfqpoint{0.083333in}{0.805556in}}%
\pgfpathlineto{\pgfqpoint{0.115988in}{0.788388in}}%
\pgfpathlineto{\pgfqpoint{0.109752in}{0.824750in}}%
\pgfpathlineto{\pgfqpoint{0.136170in}{0.850501in}}%
\pgfpathlineto{\pgfqpoint{0.099661in}{0.855806in}}%
\pgfpathlineto{\pgfqpoint{0.083333in}{0.888889in}}%
\pgfpathmoveto{\pgfqpoint{0.250000in}{0.888889in}}%
\pgfpathlineto{\pgfqpoint{0.233673in}{0.855806in}}%
\pgfpathlineto{\pgfqpoint{0.197164in}{0.850501in}}%
\pgfpathlineto{\pgfqpoint{0.223582in}{0.824750in}}%
\pgfpathlineto{\pgfqpoint{0.217345in}{0.788388in}}%
\pgfpathlineto{\pgfqpoint{0.250000in}{0.805556in}}%
\pgfpathlineto{\pgfqpoint{0.282655in}{0.788388in}}%
\pgfpathlineto{\pgfqpoint{0.276418in}{0.824750in}}%
\pgfpathlineto{\pgfqpoint{0.302836in}{0.850501in}}%
\pgfpathlineto{\pgfqpoint{0.266327in}{0.855806in}}%
\pgfpathlineto{\pgfqpoint{0.250000in}{0.888889in}}%
\pgfpathmoveto{\pgfqpoint{0.416667in}{0.888889in}}%
\pgfpathlineto{\pgfqpoint{0.400339in}{0.855806in}}%
\pgfpathlineto{\pgfqpoint{0.363830in}{0.850501in}}%
\pgfpathlineto{\pgfqpoint{0.390248in}{0.824750in}}%
\pgfpathlineto{\pgfqpoint{0.384012in}{0.788388in}}%
\pgfpathlineto{\pgfqpoint{0.416667in}{0.805556in}}%
\pgfpathlineto{\pgfqpoint{0.449321in}{0.788388in}}%
\pgfpathlineto{\pgfqpoint{0.443085in}{0.824750in}}%
\pgfpathlineto{\pgfqpoint{0.469503in}{0.850501in}}%
\pgfpathlineto{\pgfqpoint{0.432994in}{0.855806in}}%
\pgfpathlineto{\pgfqpoint{0.416667in}{0.888889in}}%
\pgfpathmoveto{\pgfqpoint{0.583333in}{0.888889in}}%
\pgfpathlineto{\pgfqpoint{0.567006in}{0.855806in}}%
\pgfpathlineto{\pgfqpoint{0.530497in}{0.850501in}}%
\pgfpathlineto{\pgfqpoint{0.556915in}{0.824750in}}%
\pgfpathlineto{\pgfqpoint{0.550679in}{0.788388in}}%
\pgfpathlineto{\pgfqpoint{0.583333in}{0.805556in}}%
\pgfpathlineto{\pgfqpoint{0.615988in}{0.788388in}}%
\pgfpathlineto{\pgfqpoint{0.609752in}{0.824750in}}%
\pgfpathlineto{\pgfqpoint{0.636170in}{0.850501in}}%
\pgfpathlineto{\pgfqpoint{0.599661in}{0.855806in}}%
\pgfpathlineto{\pgfqpoint{0.583333in}{0.888889in}}%
\pgfpathmoveto{\pgfqpoint{0.750000in}{0.888889in}}%
\pgfpathlineto{\pgfqpoint{0.733673in}{0.855806in}}%
\pgfpathlineto{\pgfqpoint{0.697164in}{0.850501in}}%
\pgfpathlineto{\pgfqpoint{0.723582in}{0.824750in}}%
\pgfpathlineto{\pgfqpoint{0.717345in}{0.788388in}}%
\pgfpathlineto{\pgfqpoint{0.750000in}{0.805556in}}%
\pgfpathlineto{\pgfqpoint{0.782655in}{0.788388in}}%
\pgfpathlineto{\pgfqpoint{0.776418in}{0.824750in}}%
\pgfpathlineto{\pgfqpoint{0.802836in}{0.850501in}}%
\pgfpathlineto{\pgfqpoint{0.766327in}{0.855806in}}%
\pgfpathlineto{\pgfqpoint{0.750000in}{0.888889in}}%
\pgfpathmoveto{\pgfqpoint{0.916667in}{0.888889in}}%
\pgfpathlineto{\pgfqpoint{0.900339in}{0.855806in}}%
\pgfpathlineto{\pgfqpoint{0.863830in}{0.850501in}}%
\pgfpathlineto{\pgfqpoint{0.890248in}{0.824750in}}%
\pgfpathlineto{\pgfqpoint{0.884012in}{0.788388in}}%
\pgfpathlineto{\pgfqpoint{0.916667in}{0.805556in}}%
\pgfpathlineto{\pgfqpoint{0.949321in}{0.788388in}}%
\pgfpathlineto{\pgfqpoint{0.943085in}{0.824750in}}%
\pgfpathlineto{\pgfqpoint{0.969503in}{0.850501in}}%
\pgfpathlineto{\pgfqpoint{0.932994in}{0.855806in}}%
\pgfpathlineto{\pgfqpoint{0.916667in}{0.888889in}}%
\pgfpathmoveto{\pgfqpoint{0.000000in}{1.055556in}}%
\pgfpathlineto{\pgfqpoint{-0.016327in}{1.022473in}}%
\pgfpathlineto{\pgfqpoint{-0.052836in}{1.017168in}}%
\pgfpathlineto{\pgfqpoint{-0.026418in}{0.991416in}}%
\pgfpathlineto{\pgfqpoint{-0.032655in}{0.955055in}}%
\pgfpathlineto{\pgfqpoint{-0.000000in}{0.972222in}}%
\pgfpathlineto{\pgfqpoint{0.032655in}{0.955055in}}%
\pgfpathlineto{\pgfqpoint{0.026418in}{0.991416in}}%
\pgfpathlineto{\pgfqpoint{0.052836in}{1.017168in}}%
\pgfpathlineto{\pgfqpoint{0.016327in}{1.022473in}}%
\pgfpathlineto{\pgfqpoint{0.000000in}{1.055556in}}%
\pgfpathmoveto{\pgfqpoint{0.166667in}{1.055556in}}%
\pgfpathlineto{\pgfqpoint{0.150339in}{1.022473in}}%
\pgfpathlineto{\pgfqpoint{0.113830in}{1.017168in}}%
\pgfpathlineto{\pgfqpoint{0.140248in}{0.991416in}}%
\pgfpathlineto{\pgfqpoint{0.134012in}{0.955055in}}%
\pgfpathlineto{\pgfqpoint{0.166667in}{0.972222in}}%
\pgfpathlineto{\pgfqpoint{0.199321in}{0.955055in}}%
\pgfpathlineto{\pgfqpoint{0.193085in}{0.991416in}}%
\pgfpathlineto{\pgfqpoint{0.219503in}{1.017168in}}%
\pgfpathlineto{\pgfqpoint{0.182994in}{1.022473in}}%
\pgfpathlineto{\pgfqpoint{0.166667in}{1.055556in}}%
\pgfpathmoveto{\pgfqpoint{0.333333in}{1.055556in}}%
\pgfpathlineto{\pgfqpoint{0.317006in}{1.022473in}}%
\pgfpathlineto{\pgfqpoint{0.280497in}{1.017168in}}%
\pgfpathlineto{\pgfqpoint{0.306915in}{0.991416in}}%
\pgfpathlineto{\pgfqpoint{0.300679in}{0.955055in}}%
\pgfpathlineto{\pgfqpoint{0.333333in}{0.972222in}}%
\pgfpathlineto{\pgfqpoint{0.365988in}{0.955055in}}%
\pgfpathlineto{\pgfqpoint{0.359752in}{0.991416in}}%
\pgfpathlineto{\pgfqpoint{0.386170in}{1.017168in}}%
\pgfpathlineto{\pgfqpoint{0.349661in}{1.022473in}}%
\pgfpathlineto{\pgfqpoint{0.333333in}{1.055556in}}%
\pgfpathmoveto{\pgfqpoint{0.500000in}{1.055556in}}%
\pgfpathlineto{\pgfqpoint{0.483673in}{1.022473in}}%
\pgfpathlineto{\pgfqpoint{0.447164in}{1.017168in}}%
\pgfpathlineto{\pgfqpoint{0.473582in}{0.991416in}}%
\pgfpathlineto{\pgfqpoint{0.467345in}{0.955055in}}%
\pgfpathlineto{\pgfqpoint{0.500000in}{0.972222in}}%
\pgfpathlineto{\pgfqpoint{0.532655in}{0.955055in}}%
\pgfpathlineto{\pgfqpoint{0.526418in}{0.991416in}}%
\pgfpathlineto{\pgfqpoint{0.552836in}{1.017168in}}%
\pgfpathlineto{\pgfqpoint{0.516327in}{1.022473in}}%
\pgfpathlineto{\pgfqpoint{0.500000in}{1.055556in}}%
\pgfpathmoveto{\pgfqpoint{0.666667in}{1.055556in}}%
\pgfpathlineto{\pgfqpoint{0.650339in}{1.022473in}}%
\pgfpathlineto{\pgfqpoint{0.613830in}{1.017168in}}%
\pgfpathlineto{\pgfqpoint{0.640248in}{0.991416in}}%
\pgfpathlineto{\pgfqpoint{0.634012in}{0.955055in}}%
\pgfpathlineto{\pgfqpoint{0.666667in}{0.972222in}}%
\pgfpathlineto{\pgfqpoint{0.699321in}{0.955055in}}%
\pgfpathlineto{\pgfqpoint{0.693085in}{0.991416in}}%
\pgfpathlineto{\pgfqpoint{0.719503in}{1.017168in}}%
\pgfpathlineto{\pgfqpoint{0.682994in}{1.022473in}}%
\pgfpathlineto{\pgfqpoint{0.666667in}{1.055556in}}%
\pgfpathmoveto{\pgfqpoint{0.833333in}{1.055556in}}%
\pgfpathlineto{\pgfqpoint{0.817006in}{1.022473in}}%
\pgfpathlineto{\pgfqpoint{0.780497in}{1.017168in}}%
\pgfpathlineto{\pgfqpoint{0.806915in}{0.991416in}}%
\pgfpathlineto{\pgfqpoint{0.800679in}{0.955055in}}%
\pgfpathlineto{\pgfqpoint{0.833333in}{0.972222in}}%
\pgfpathlineto{\pgfqpoint{0.865988in}{0.955055in}}%
\pgfpathlineto{\pgfqpoint{0.859752in}{0.991416in}}%
\pgfpathlineto{\pgfqpoint{0.886170in}{1.017168in}}%
\pgfpathlineto{\pgfqpoint{0.849661in}{1.022473in}}%
\pgfpathlineto{\pgfqpoint{0.833333in}{1.055556in}}%
\pgfpathmoveto{\pgfqpoint{1.000000in}{1.055556in}}%
\pgfpathlineto{\pgfqpoint{0.983673in}{1.022473in}}%
\pgfpathlineto{\pgfqpoint{0.947164in}{1.017168in}}%
\pgfpathlineto{\pgfqpoint{0.973582in}{0.991416in}}%
\pgfpathlineto{\pgfqpoint{0.967345in}{0.955055in}}%
\pgfpathlineto{\pgfqpoint{1.000000in}{0.972222in}}%
\pgfpathlineto{\pgfqpoint{1.032655in}{0.955055in}}%
\pgfpathlineto{\pgfqpoint{1.026418in}{0.991416in}}%
\pgfpathlineto{\pgfqpoint{1.052836in}{1.017168in}}%
\pgfpathlineto{\pgfqpoint{1.016327in}{1.022473in}}%
\pgfpathlineto{\pgfqpoint{1.000000in}{1.055556in}}%
\pgfpathlineto{\pgfqpoint{1.000000in}{1.055556in}}%
\pgfusepath{stroke}%
\end{pgfscope}%
}%
\pgfsys@transformshift{9.008038in}{4.114678in}%
\pgfsys@useobject{currentpattern}{}%
\pgfsys@transformshift{1in}{0in}%
\pgfsys@transformshift{-1in}{0in}%
\pgfsys@transformshift{0in}{1in}%
\pgfsys@useobject{currentpattern}{}%
\pgfsys@transformshift{1in}{0in}%
\pgfsys@transformshift{-1in}{0in}%
\pgfsys@transformshift{0in}{1in}%
\end{pgfscope}%
\begin{pgfscope}%
\pgfpathrectangle{\pgfqpoint{0.870538in}{0.637495in}}{\pgfqpoint{9.300000in}{9.060000in}}%
\pgfusepath{clip}%
\pgfsetbuttcap%
\pgfsetmiterjoin%
\definecolor{currentfill}{rgb}{0.121569,0.466667,0.705882}%
\pgfsetfillcolor{currentfill}%
\pgfsetfillopacity{0.990000}%
\pgfsetlinewidth{0.000000pt}%
\definecolor{currentstroke}{rgb}{0.000000,0.000000,0.000000}%
\pgfsetstrokecolor{currentstroke}%
\pgfsetstrokeopacity{0.990000}%
\pgfsetdash{}{0pt}%
\pgfpathmoveto{\pgfqpoint{1.258038in}{3.881164in}}%
\pgfpathlineto{\pgfqpoint{2.033038in}{3.881164in}}%
\pgfpathlineto{\pgfqpoint{2.033038in}{7.540352in}}%
\pgfpathlineto{\pgfqpoint{1.258038in}{7.540352in}}%
\pgfpathclose%
\pgfusepath{fill}%
\end{pgfscope}%
\begin{pgfscope}%
\pgfsetbuttcap%
\pgfsetmiterjoin%
\definecolor{currentfill}{rgb}{0.121569,0.466667,0.705882}%
\pgfsetfillcolor{currentfill}%
\pgfsetfillopacity{0.990000}%
\pgfsetlinewidth{0.000000pt}%
\definecolor{currentstroke}{rgb}{0.000000,0.000000,0.000000}%
\pgfsetstrokecolor{currentstroke}%
\pgfsetstrokeopacity{0.990000}%
\pgfsetdash{}{0pt}%
\pgfpathrectangle{\pgfqpoint{0.870538in}{0.637495in}}{\pgfqpoint{9.300000in}{9.060000in}}%
\pgfusepath{clip}%
\pgfpathmoveto{\pgfqpoint{1.258038in}{3.881164in}}%
\pgfpathlineto{\pgfqpoint{2.033038in}{3.881164in}}%
\pgfpathlineto{\pgfqpoint{2.033038in}{7.540352in}}%
\pgfpathlineto{\pgfqpoint{1.258038in}{7.540352in}}%
\pgfpathclose%
\pgfusepath{clip}%
\pgfsys@defobject{currentpattern}{\pgfqpoint{0in}{0in}}{\pgfqpoint{1in}{1in}}{%
\begin{pgfscope}%
\pgfpathrectangle{\pgfqpoint{0in}{0in}}{\pgfqpoint{1in}{1in}}%
\pgfusepath{clip}%
\pgfpathmoveto{\pgfqpoint{0.000000in}{0.083333in}}%
\pgfpathlineto{\pgfqpoint{1.000000in}{0.083333in}}%
\pgfpathmoveto{\pgfqpoint{0.000000in}{0.250000in}}%
\pgfpathlineto{\pgfqpoint{1.000000in}{0.250000in}}%
\pgfpathmoveto{\pgfqpoint{0.000000in}{0.416667in}}%
\pgfpathlineto{\pgfqpoint{1.000000in}{0.416667in}}%
\pgfpathmoveto{\pgfqpoint{0.000000in}{0.583333in}}%
\pgfpathlineto{\pgfqpoint{1.000000in}{0.583333in}}%
\pgfpathmoveto{\pgfqpoint{0.000000in}{0.750000in}}%
\pgfpathlineto{\pgfqpoint{1.000000in}{0.750000in}}%
\pgfpathmoveto{\pgfqpoint{0.000000in}{0.916667in}}%
\pgfpathlineto{\pgfqpoint{1.000000in}{0.916667in}}%
\pgfpathmoveto{\pgfqpoint{0.083333in}{0.000000in}}%
\pgfpathlineto{\pgfqpoint{0.083333in}{1.000000in}}%
\pgfpathmoveto{\pgfqpoint{0.250000in}{0.000000in}}%
\pgfpathlineto{\pgfqpoint{0.250000in}{1.000000in}}%
\pgfpathmoveto{\pgfqpoint{0.416667in}{0.000000in}}%
\pgfpathlineto{\pgfqpoint{0.416667in}{1.000000in}}%
\pgfpathmoveto{\pgfqpoint{0.583333in}{0.000000in}}%
\pgfpathlineto{\pgfqpoint{0.583333in}{1.000000in}}%
\pgfpathmoveto{\pgfqpoint{0.750000in}{0.000000in}}%
\pgfpathlineto{\pgfqpoint{0.750000in}{1.000000in}}%
\pgfpathmoveto{\pgfqpoint{0.916667in}{0.000000in}}%
\pgfpathlineto{\pgfqpoint{0.916667in}{1.000000in}}%
\pgfusepath{stroke}%
\end{pgfscope}%
}%
\pgfsys@transformshift{1.258038in}{3.881164in}%
\pgfsys@useobject{currentpattern}{}%
\pgfsys@transformshift{1in}{0in}%
\pgfsys@transformshift{-1in}{0in}%
\pgfsys@transformshift{0in}{1in}%
\pgfsys@useobject{currentpattern}{}%
\pgfsys@transformshift{1in}{0in}%
\pgfsys@transformshift{-1in}{0in}%
\pgfsys@transformshift{0in}{1in}%
\pgfsys@useobject{currentpattern}{}%
\pgfsys@transformshift{1in}{0in}%
\pgfsys@transformshift{-1in}{0in}%
\pgfsys@transformshift{0in}{1in}%
\pgfsys@useobject{currentpattern}{}%
\pgfsys@transformshift{1in}{0in}%
\pgfsys@transformshift{-1in}{0in}%
\pgfsys@transformshift{0in}{1in}%
\end{pgfscope}%
\begin{pgfscope}%
\pgfpathrectangle{\pgfqpoint{0.870538in}{0.637495in}}{\pgfqpoint{9.300000in}{9.060000in}}%
\pgfusepath{clip}%
\pgfsetbuttcap%
\pgfsetmiterjoin%
\definecolor{currentfill}{rgb}{0.121569,0.466667,0.705882}%
\pgfsetfillcolor{currentfill}%
\pgfsetfillopacity{0.990000}%
\pgfsetlinewidth{0.000000pt}%
\definecolor{currentstroke}{rgb}{0.000000,0.000000,0.000000}%
\pgfsetstrokecolor{currentstroke}%
\pgfsetstrokeopacity{0.990000}%
\pgfsetdash{}{0pt}%
\pgfpathmoveto{\pgfqpoint{2.808038in}{4.019322in}}%
\pgfpathlineto{\pgfqpoint{3.583038in}{4.019322in}}%
\pgfpathlineto{\pgfqpoint{3.583038in}{7.885495in}}%
\pgfpathlineto{\pgfqpoint{2.808038in}{7.885495in}}%
\pgfpathclose%
\pgfusepath{fill}%
\end{pgfscope}%
\begin{pgfscope}%
\pgfsetbuttcap%
\pgfsetmiterjoin%
\definecolor{currentfill}{rgb}{0.121569,0.466667,0.705882}%
\pgfsetfillcolor{currentfill}%
\pgfsetfillopacity{0.990000}%
\pgfsetlinewidth{0.000000pt}%
\definecolor{currentstroke}{rgb}{0.000000,0.000000,0.000000}%
\pgfsetstrokecolor{currentstroke}%
\pgfsetstrokeopacity{0.990000}%
\pgfsetdash{}{0pt}%
\pgfpathrectangle{\pgfqpoint{0.870538in}{0.637495in}}{\pgfqpoint{9.300000in}{9.060000in}}%
\pgfusepath{clip}%
\pgfpathmoveto{\pgfqpoint{2.808038in}{4.019322in}}%
\pgfpathlineto{\pgfqpoint{3.583038in}{4.019322in}}%
\pgfpathlineto{\pgfqpoint{3.583038in}{7.885495in}}%
\pgfpathlineto{\pgfqpoint{2.808038in}{7.885495in}}%
\pgfpathclose%
\pgfusepath{clip}%
\pgfsys@defobject{currentpattern}{\pgfqpoint{0in}{0in}}{\pgfqpoint{1in}{1in}}{%
\begin{pgfscope}%
\pgfpathrectangle{\pgfqpoint{0in}{0in}}{\pgfqpoint{1in}{1in}}%
\pgfusepath{clip}%
\pgfpathmoveto{\pgfqpoint{0.000000in}{0.083333in}}%
\pgfpathlineto{\pgfqpoint{1.000000in}{0.083333in}}%
\pgfpathmoveto{\pgfqpoint{0.000000in}{0.250000in}}%
\pgfpathlineto{\pgfqpoint{1.000000in}{0.250000in}}%
\pgfpathmoveto{\pgfqpoint{0.000000in}{0.416667in}}%
\pgfpathlineto{\pgfqpoint{1.000000in}{0.416667in}}%
\pgfpathmoveto{\pgfqpoint{0.000000in}{0.583333in}}%
\pgfpathlineto{\pgfqpoint{1.000000in}{0.583333in}}%
\pgfpathmoveto{\pgfqpoint{0.000000in}{0.750000in}}%
\pgfpathlineto{\pgfqpoint{1.000000in}{0.750000in}}%
\pgfpathmoveto{\pgfqpoint{0.000000in}{0.916667in}}%
\pgfpathlineto{\pgfqpoint{1.000000in}{0.916667in}}%
\pgfpathmoveto{\pgfqpoint{0.083333in}{0.000000in}}%
\pgfpathlineto{\pgfqpoint{0.083333in}{1.000000in}}%
\pgfpathmoveto{\pgfqpoint{0.250000in}{0.000000in}}%
\pgfpathlineto{\pgfqpoint{0.250000in}{1.000000in}}%
\pgfpathmoveto{\pgfqpoint{0.416667in}{0.000000in}}%
\pgfpathlineto{\pgfqpoint{0.416667in}{1.000000in}}%
\pgfpathmoveto{\pgfqpoint{0.583333in}{0.000000in}}%
\pgfpathlineto{\pgfqpoint{0.583333in}{1.000000in}}%
\pgfpathmoveto{\pgfqpoint{0.750000in}{0.000000in}}%
\pgfpathlineto{\pgfqpoint{0.750000in}{1.000000in}}%
\pgfpathmoveto{\pgfqpoint{0.916667in}{0.000000in}}%
\pgfpathlineto{\pgfqpoint{0.916667in}{1.000000in}}%
\pgfusepath{stroke}%
\end{pgfscope}%
}%
\pgfsys@transformshift{2.808038in}{4.019322in}%
\pgfsys@useobject{currentpattern}{}%
\pgfsys@transformshift{1in}{0in}%
\pgfsys@transformshift{-1in}{0in}%
\pgfsys@transformshift{0in}{1in}%
\pgfsys@useobject{currentpattern}{}%
\pgfsys@transformshift{1in}{0in}%
\pgfsys@transformshift{-1in}{0in}%
\pgfsys@transformshift{0in}{1in}%
\pgfsys@useobject{currentpattern}{}%
\pgfsys@transformshift{1in}{0in}%
\pgfsys@transformshift{-1in}{0in}%
\pgfsys@transformshift{0in}{1in}%
\pgfsys@useobject{currentpattern}{}%
\pgfsys@transformshift{1in}{0in}%
\pgfsys@transformshift{-1in}{0in}%
\pgfsys@transformshift{0in}{1in}%
\end{pgfscope}%
\begin{pgfscope}%
\pgfpathrectangle{\pgfqpoint{0.870538in}{0.637495in}}{\pgfqpoint{9.300000in}{9.060000in}}%
\pgfusepath{clip}%
\pgfsetbuttcap%
\pgfsetmiterjoin%
\definecolor{currentfill}{rgb}{0.121569,0.466667,0.705882}%
\pgfsetfillcolor{currentfill}%
\pgfsetfillopacity{0.990000}%
\pgfsetlinewidth{0.000000pt}%
\definecolor{currentstroke}{rgb}{0.000000,0.000000,0.000000}%
\pgfsetstrokecolor{currentstroke}%
\pgfsetstrokeopacity{0.990000}%
\pgfsetdash{}{0pt}%
\pgfpathmoveto{\pgfqpoint{4.358038in}{4.364668in}}%
\pgfpathlineto{\pgfqpoint{5.133038in}{4.364668in}}%
\pgfpathlineto{\pgfqpoint{5.133038in}{8.230638in}}%
\pgfpathlineto{\pgfqpoint{4.358038in}{8.230638in}}%
\pgfpathclose%
\pgfusepath{fill}%
\end{pgfscope}%
\begin{pgfscope}%
\pgfsetbuttcap%
\pgfsetmiterjoin%
\definecolor{currentfill}{rgb}{0.121569,0.466667,0.705882}%
\pgfsetfillcolor{currentfill}%
\pgfsetfillopacity{0.990000}%
\pgfsetlinewidth{0.000000pt}%
\definecolor{currentstroke}{rgb}{0.000000,0.000000,0.000000}%
\pgfsetstrokecolor{currentstroke}%
\pgfsetstrokeopacity{0.990000}%
\pgfsetdash{}{0pt}%
\pgfpathrectangle{\pgfqpoint{0.870538in}{0.637495in}}{\pgfqpoint{9.300000in}{9.060000in}}%
\pgfusepath{clip}%
\pgfpathmoveto{\pgfqpoint{4.358038in}{4.364668in}}%
\pgfpathlineto{\pgfqpoint{5.133038in}{4.364668in}}%
\pgfpathlineto{\pgfqpoint{5.133038in}{8.230638in}}%
\pgfpathlineto{\pgfqpoint{4.358038in}{8.230638in}}%
\pgfpathclose%
\pgfusepath{clip}%
\pgfsys@defobject{currentpattern}{\pgfqpoint{0in}{0in}}{\pgfqpoint{1in}{1in}}{%
\begin{pgfscope}%
\pgfpathrectangle{\pgfqpoint{0in}{0in}}{\pgfqpoint{1in}{1in}}%
\pgfusepath{clip}%
\pgfpathmoveto{\pgfqpoint{0.000000in}{0.083333in}}%
\pgfpathlineto{\pgfqpoint{1.000000in}{0.083333in}}%
\pgfpathmoveto{\pgfqpoint{0.000000in}{0.250000in}}%
\pgfpathlineto{\pgfqpoint{1.000000in}{0.250000in}}%
\pgfpathmoveto{\pgfqpoint{0.000000in}{0.416667in}}%
\pgfpathlineto{\pgfqpoint{1.000000in}{0.416667in}}%
\pgfpathmoveto{\pgfqpoint{0.000000in}{0.583333in}}%
\pgfpathlineto{\pgfqpoint{1.000000in}{0.583333in}}%
\pgfpathmoveto{\pgfqpoint{0.000000in}{0.750000in}}%
\pgfpathlineto{\pgfqpoint{1.000000in}{0.750000in}}%
\pgfpathmoveto{\pgfqpoint{0.000000in}{0.916667in}}%
\pgfpathlineto{\pgfqpoint{1.000000in}{0.916667in}}%
\pgfpathmoveto{\pgfqpoint{0.083333in}{0.000000in}}%
\pgfpathlineto{\pgfqpoint{0.083333in}{1.000000in}}%
\pgfpathmoveto{\pgfqpoint{0.250000in}{0.000000in}}%
\pgfpathlineto{\pgfqpoint{0.250000in}{1.000000in}}%
\pgfpathmoveto{\pgfqpoint{0.416667in}{0.000000in}}%
\pgfpathlineto{\pgfqpoint{0.416667in}{1.000000in}}%
\pgfpathmoveto{\pgfqpoint{0.583333in}{0.000000in}}%
\pgfpathlineto{\pgfqpoint{0.583333in}{1.000000in}}%
\pgfpathmoveto{\pgfqpoint{0.750000in}{0.000000in}}%
\pgfpathlineto{\pgfqpoint{0.750000in}{1.000000in}}%
\pgfpathmoveto{\pgfqpoint{0.916667in}{0.000000in}}%
\pgfpathlineto{\pgfqpoint{0.916667in}{1.000000in}}%
\pgfusepath{stroke}%
\end{pgfscope}%
}%
\pgfsys@transformshift{4.358038in}{4.364668in}%
\pgfsys@useobject{currentpattern}{}%
\pgfsys@transformshift{1in}{0in}%
\pgfsys@transformshift{-1in}{0in}%
\pgfsys@transformshift{0in}{1in}%
\pgfsys@useobject{currentpattern}{}%
\pgfsys@transformshift{1in}{0in}%
\pgfsys@transformshift{-1in}{0in}%
\pgfsys@transformshift{0in}{1in}%
\pgfsys@useobject{currentpattern}{}%
\pgfsys@transformshift{1in}{0in}%
\pgfsys@transformshift{-1in}{0in}%
\pgfsys@transformshift{0in}{1in}%
\pgfsys@useobject{currentpattern}{}%
\pgfsys@transformshift{1in}{0in}%
\pgfsys@transformshift{-1in}{0in}%
\pgfsys@transformshift{0in}{1in}%
\end{pgfscope}%
\begin{pgfscope}%
\pgfpathrectangle{\pgfqpoint{0.870538in}{0.637495in}}{\pgfqpoint{9.300000in}{9.060000in}}%
\pgfusepath{clip}%
\pgfsetbuttcap%
\pgfsetmiterjoin%
\definecolor{currentfill}{rgb}{0.121569,0.466667,0.705882}%
\pgfsetfillcolor{currentfill}%
\pgfsetfillopacity{0.990000}%
\pgfsetlinewidth{0.000000pt}%
\definecolor{currentstroke}{rgb}{0.000000,0.000000,0.000000}%
\pgfsetstrokecolor{currentstroke}%
\pgfsetstrokeopacity{0.990000}%
\pgfsetdash{}{0pt}%
\pgfpathmoveto{\pgfqpoint{5.908038in}{4.713204in}}%
\pgfpathlineto{\pgfqpoint{6.683038in}{4.713204in}}%
\pgfpathlineto{\pgfqpoint{6.683038in}{8.575781in}}%
\pgfpathlineto{\pgfqpoint{5.908038in}{8.575781in}}%
\pgfpathclose%
\pgfusepath{fill}%
\end{pgfscope}%
\begin{pgfscope}%
\pgfsetbuttcap%
\pgfsetmiterjoin%
\definecolor{currentfill}{rgb}{0.121569,0.466667,0.705882}%
\pgfsetfillcolor{currentfill}%
\pgfsetfillopacity{0.990000}%
\pgfsetlinewidth{0.000000pt}%
\definecolor{currentstroke}{rgb}{0.000000,0.000000,0.000000}%
\pgfsetstrokecolor{currentstroke}%
\pgfsetstrokeopacity{0.990000}%
\pgfsetdash{}{0pt}%
\pgfpathrectangle{\pgfqpoint{0.870538in}{0.637495in}}{\pgfqpoint{9.300000in}{9.060000in}}%
\pgfusepath{clip}%
\pgfpathmoveto{\pgfqpoint{5.908038in}{4.713204in}}%
\pgfpathlineto{\pgfqpoint{6.683038in}{4.713204in}}%
\pgfpathlineto{\pgfqpoint{6.683038in}{8.575781in}}%
\pgfpathlineto{\pgfqpoint{5.908038in}{8.575781in}}%
\pgfpathclose%
\pgfusepath{clip}%
\pgfsys@defobject{currentpattern}{\pgfqpoint{0in}{0in}}{\pgfqpoint{1in}{1in}}{%
\begin{pgfscope}%
\pgfpathrectangle{\pgfqpoint{0in}{0in}}{\pgfqpoint{1in}{1in}}%
\pgfusepath{clip}%
\pgfpathmoveto{\pgfqpoint{0.000000in}{0.083333in}}%
\pgfpathlineto{\pgfqpoint{1.000000in}{0.083333in}}%
\pgfpathmoveto{\pgfqpoint{0.000000in}{0.250000in}}%
\pgfpathlineto{\pgfqpoint{1.000000in}{0.250000in}}%
\pgfpathmoveto{\pgfqpoint{0.000000in}{0.416667in}}%
\pgfpathlineto{\pgfqpoint{1.000000in}{0.416667in}}%
\pgfpathmoveto{\pgfqpoint{0.000000in}{0.583333in}}%
\pgfpathlineto{\pgfqpoint{1.000000in}{0.583333in}}%
\pgfpathmoveto{\pgfqpoint{0.000000in}{0.750000in}}%
\pgfpathlineto{\pgfqpoint{1.000000in}{0.750000in}}%
\pgfpathmoveto{\pgfqpoint{0.000000in}{0.916667in}}%
\pgfpathlineto{\pgfqpoint{1.000000in}{0.916667in}}%
\pgfpathmoveto{\pgfqpoint{0.083333in}{0.000000in}}%
\pgfpathlineto{\pgfqpoint{0.083333in}{1.000000in}}%
\pgfpathmoveto{\pgfqpoint{0.250000in}{0.000000in}}%
\pgfpathlineto{\pgfqpoint{0.250000in}{1.000000in}}%
\pgfpathmoveto{\pgfqpoint{0.416667in}{0.000000in}}%
\pgfpathlineto{\pgfqpoint{0.416667in}{1.000000in}}%
\pgfpathmoveto{\pgfqpoint{0.583333in}{0.000000in}}%
\pgfpathlineto{\pgfqpoint{0.583333in}{1.000000in}}%
\pgfpathmoveto{\pgfqpoint{0.750000in}{0.000000in}}%
\pgfpathlineto{\pgfqpoint{0.750000in}{1.000000in}}%
\pgfpathmoveto{\pgfqpoint{0.916667in}{0.000000in}}%
\pgfpathlineto{\pgfqpoint{0.916667in}{1.000000in}}%
\pgfusepath{stroke}%
\end{pgfscope}%
}%
\pgfsys@transformshift{5.908038in}{4.713204in}%
\pgfsys@useobject{currentpattern}{}%
\pgfsys@transformshift{1in}{0in}%
\pgfsys@transformshift{-1in}{0in}%
\pgfsys@transformshift{0in}{1in}%
\pgfsys@useobject{currentpattern}{}%
\pgfsys@transformshift{1in}{0in}%
\pgfsys@transformshift{-1in}{0in}%
\pgfsys@transformshift{0in}{1in}%
\pgfsys@useobject{currentpattern}{}%
\pgfsys@transformshift{1in}{0in}%
\pgfsys@transformshift{-1in}{0in}%
\pgfsys@transformshift{0in}{1in}%
\pgfsys@useobject{currentpattern}{}%
\pgfsys@transformshift{1in}{0in}%
\pgfsys@transformshift{-1in}{0in}%
\pgfsys@transformshift{0in}{1in}%
\end{pgfscope}%
\begin{pgfscope}%
\pgfpathrectangle{\pgfqpoint{0.870538in}{0.637495in}}{\pgfqpoint{9.300000in}{9.060000in}}%
\pgfusepath{clip}%
\pgfsetbuttcap%
\pgfsetmiterjoin%
\definecolor{currentfill}{rgb}{0.121569,0.466667,0.705882}%
\pgfsetfillcolor{currentfill}%
\pgfsetfillopacity{0.990000}%
\pgfsetlinewidth{0.000000pt}%
\definecolor{currentstroke}{rgb}{0.000000,0.000000,0.000000}%
\pgfsetstrokecolor{currentstroke}%
\pgfsetstrokeopacity{0.990000}%
\pgfsetdash{}{0pt}%
\pgfpathmoveto{\pgfqpoint{7.458038in}{5.048106in}}%
\pgfpathlineto{\pgfqpoint{8.233038in}{5.048106in}}%
\pgfpathlineto{\pgfqpoint{8.233038in}{8.920924in}}%
\pgfpathlineto{\pgfqpoint{7.458038in}{8.920924in}}%
\pgfpathclose%
\pgfusepath{fill}%
\end{pgfscope}%
\begin{pgfscope}%
\pgfsetbuttcap%
\pgfsetmiterjoin%
\definecolor{currentfill}{rgb}{0.121569,0.466667,0.705882}%
\pgfsetfillcolor{currentfill}%
\pgfsetfillopacity{0.990000}%
\pgfsetlinewidth{0.000000pt}%
\definecolor{currentstroke}{rgb}{0.000000,0.000000,0.000000}%
\pgfsetstrokecolor{currentstroke}%
\pgfsetstrokeopacity{0.990000}%
\pgfsetdash{}{0pt}%
\pgfpathrectangle{\pgfqpoint{0.870538in}{0.637495in}}{\pgfqpoint{9.300000in}{9.060000in}}%
\pgfusepath{clip}%
\pgfpathmoveto{\pgfqpoint{7.458038in}{5.048106in}}%
\pgfpathlineto{\pgfqpoint{8.233038in}{5.048106in}}%
\pgfpathlineto{\pgfqpoint{8.233038in}{8.920924in}}%
\pgfpathlineto{\pgfqpoint{7.458038in}{8.920924in}}%
\pgfpathclose%
\pgfusepath{clip}%
\pgfsys@defobject{currentpattern}{\pgfqpoint{0in}{0in}}{\pgfqpoint{1in}{1in}}{%
\begin{pgfscope}%
\pgfpathrectangle{\pgfqpoint{0in}{0in}}{\pgfqpoint{1in}{1in}}%
\pgfusepath{clip}%
\pgfpathmoveto{\pgfqpoint{0.000000in}{0.083333in}}%
\pgfpathlineto{\pgfqpoint{1.000000in}{0.083333in}}%
\pgfpathmoveto{\pgfqpoint{0.000000in}{0.250000in}}%
\pgfpathlineto{\pgfqpoint{1.000000in}{0.250000in}}%
\pgfpathmoveto{\pgfqpoint{0.000000in}{0.416667in}}%
\pgfpathlineto{\pgfqpoint{1.000000in}{0.416667in}}%
\pgfpathmoveto{\pgfqpoint{0.000000in}{0.583333in}}%
\pgfpathlineto{\pgfqpoint{1.000000in}{0.583333in}}%
\pgfpathmoveto{\pgfqpoint{0.000000in}{0.750000in}}%
\pgfpathlineto{\pgfqpoint{1.000000in}{0.750000in}}%
\pgfpathmoveto{\pgfqpoint{0.000000in}{0.916667in}}%
\pgfpathlineto{\pgfqpoint{1.000000in}{0.916667in}}%
\pgfpathmoveto{\pgfqpoint{0.083333in}{0.000000in}}%
\pgfpathlineto{\pgfqpoint{0.083333in}{1.000000in}}%
\pgfpathmoveto{\pgfqpoint{0.250000in}{0.000000in}}%
\pgfpathlineto{\pgfqpoint{0.250000in}{1.000000in}}%
\pgfpathmoveto{\pgfqpoint{0.416667in}{0.000000in}}%
\pgfpathlineto{\pgfqpoint{0.416667in}{1.000000in}}%
\pgfpathmoveto{\pgfqpoint{0.583333in}{0.000000in}}%
\pgfpathlineto{\pgfqpoint{0.583333in}{1.000000in}}%
\pgfpathmoveto{\pgfqpoint{0.750000in}{0.000000in}}%
\pgfpathlineto{\pgfqpoint{0.750000in}{1.000000in}}%
\pgfpathmoveto{\pgfqpoint{0.916667in}{0.000000in}}%
\pgfpathlineto{\pgfqpoint{0.916667in}{1.000000in}}%
\pgfusepath{stroke}%
\end{pgfscope}%
}%
\pgfsys@transformshift{7.458038in}{5.048106in}%
\pgfsys@useobject{currentpattern}{}%
\pgfsys@transformshift{1in}{0in}%
\pgfsys@transformshift{-1in}{0in}%
\pgfsys@transformshift{0in}{1in}%
\pgfsys@useobject{currentpattern}{}%
\pgfsys@transformshift{1in}{0in}%
\pgfsys@transformshift{-1in}{0in}%
\pgfsys@transformshift{0in}{1in}%
\pgfsys@useobject{currentpattern}{}%
\pgfsys@transformshift{1in}{0in}%
\pgfsys@transformshift{-1in}{0in}%
\pgfsys@transformshift{0in}{1in}%
\pgfsys@useobject{currentpattern}{}%
\pgfsys@transformshift{1in}{0in}%
\pgfsys@transformshift{-1in}{0in}%
\pgfsys@transformshift{0in}{1in}%
\end{pgfscope}%
\begin{pgfscope}%
\pgfpathrectangle{\pgfqpoint{0.870538in}{0.637495in}}{\pgfqpoint{9.300000in}{9.060000in}}%
\pgfusepath{clip}%
\pgfsetbuttcap%
\pgfsetmiterjoin%
\definecolor{currentfill}{rgb}{0.121569,0.466667,0.705882}%
\pgfsetfillcolor{currentfill}%
\pgfsetfillopacity{0.990000}%
\pgfsetlinewidth{0.000000pt}%
\definecolor{currentstroke}{rgb}{0.000000,0.000000,0.000000}%
\pgfsetstrokecolor{currentstroke}%
\pgfsetstrokeopacity{0.990000}%
\pgfsetdash{}{0pt}%
\pgfpathmoveto{\pgfqpoint{9.008038in}{5.392180in}}%
\pgfpathlineto{\pgfqpoint{9.783038in}{5.392180in}}%
\pgfpathlineto{\pgfqpoint{9.783038in}{9.266067in}}%
\pgfpathlineto{\pgfqpoint{9.008038in}{9.266067in}}%
\pgfpathclose%
\pgfusepath{fill}%
\end{pgfscope}%
\begin{pgfscope}%
\pgfsetbuttcap%
\pgfsetmiterjoin%
\definecolor{currentfill}{rgb}{0.121569,0.466667,0.705882}%
\pgfsetfillcolor{currentfill}%
\pgfsetfillopacity{0.990000}%
\pgfsetlinewidth{0.000000pt}%
\definecolor{currentstroke}{rgb}{0.000000,0.000000,0.000000}%
\pgfsetstrokecolor{currentstroke}%
\pgfsetstrokeopacity{0.990000}%
\pgfsetdash{}{0pt}%
\pgfpathrectangle{\pgfqpoint{0.870538in}{0.637495in}}{\pgfqpoint{9.300000in}{9.060000in}}%
\pgfusepath{clip}%
\pgfpathmoveto{\pgfqpoint{9.008038in}{5.392180in}}%
\pgfpathlineto{\pgfqpoint{9.783038in}{5.392180in}}%
\pgfpathlineto{\pgfqpoint{9.783038in}{9.266067in}}%
\pgfpathlineto{\pgfqpoint{9.008038in}{9.266067in}}%
\pgfpathclose%
\pgfusepath{clip}%
\pgfsys@defobject{currentpattern}{\pgfqpoint{0in}{0in}}{\pgfqpoint{1in}{1in}}{%
\begin{pgfscope}%
\pgfpathrectangle{\pgfqpoint{0in}{0in}}{\pgfqpoint{1in}{1in}}%
\pgfusepath{clip}%
\pgfpathmoveto{\pgfqpoint{0.000000in}{0.083333in}}%
\pgfpathlineto{\pgfqpoint{1.000000in}{0.083333in}}%
\pgfpathmoveto{\pgfqpoint{0.000000in}{0.250000in}}%
\pgfpathlineto{\pgfqpoint{1.000000in}{0.250000in}}%
\pgfpathmoveto{\pgfqpoint{0.000000in}{0.416667in}}%
\pgfpathlineto{\pgfqpoint{1.000000in}{0.416667in}}%
\pgfpathmoveto{\pgfqpoint{0.000000in}{0.583333in}}%
\pgfpathlineto{\pgfqpoint{1.000000in}{0.583333in}}%
\pgfpathmoveto{\pgfqpoint{0.000000in}{0.750000in}}%
\pgfpathlineto{\pgfqpoint{1.000000in}{0.750000in}}%
\pgfpathmoveto{\pgfqpoint{0.000000in}{0.916667in}}%
\pgfpathlineto{\pgfqpoint{1.000000in}{0.916667in}}%
\pgfpathmoveto{\pgfqpoint{0.083333in}{0.000000in}}%
\pgfpathlineto{\pgfqpoint{0.083333in}{1.000000in}}%
\pgfpathmoveto{\pgfqpoint{0.250000in}{0.000000in}}%
\pgfpathlineto{\pgfqpoint{0.250000in}{1.000000in}}%
\pgfpathmoveto{\pgfqpoint{0.416667in}{0.000000in}}%
\pgfpathlineto{\pgfqpoint{0.416667in}{1.000000in}}%
\pgfpathmoveto{\pgfqpoint{0.583333in}{0.000000in}}%
\pgfpathlineto{\pgfqpoint{0.583333in}{1.000000in}}%
\pgfpathmoveto{\pgfqpoint{0.750000in}{0.000000in}}%
\pgfpathlineto{\pgfqpoint{0.750000in}{1.000000in}}%
\pgfpathmoveto{\pgfqpoint{0.916667in}{0.000000in}}%
\pgfpathlineto{\pgfqpoint{0.916667in}{1.000000in}}%
\pgfusepath{stroke}%
\end{pgfscope}%
}%
\pgfsys@transformshift{9.008038in}{5.392180in}%
\pgfsys@useobject{currentpattern}{}%
\pgfsys@transformshift{1in}{0in}%
\pgfsys@transformshift{-1in}{0in}%
\pgfsys@transformshift{0in}{1in}%
\pgfsys@useobject{currentpattern}{}%
\pgfsys@transformshift{1in}{0in}%
\pgfsys@transformshift{-1in}{0in}%
\pgfsys@transformshift{0in}{1in}%
\pgfsys@useobject{currentpattern}{}%
\pgfsys@transformshift{1in}{0in}%
\pgfsys@transformshift{-1in}{0in}%
\pgfsys@transformshift{0in}{1in}%
\pgfsys@useobject{currentpattern}{}%
\pgfsys@transformshift{1in}{0in}%
\pgfsys@transformshift{-1in}{0in}%
\pgfsys@transformshift{0in}{1in}%
\end{pgfscope}%
\begin{pgfscope}%
\pgfsetrectcap%
\pgfsetmiterjoin%
\pgfsetlinewidth{1.003750pt}%
\definecolor{currentstroke}{rgb}{1.000000,1.000000,1.000000}%
\pgfsetstrokecolor{currentstroke}%
\pgfsetdash{}{0pt}%
\pgfpathmoveto{\pgfqpoint{0.870538in}{0.637495in}}%
\pgfpathlineto{\pgfqpoint{0.870538in}{9.697495in}}%
\pgfusepath{stroke}%
\end{pgfscope}%
\begin{pgfscope}%
\pgfsetrectcap%
\pgfsetmiterjoin%
\pgfsetlinewidth{1.003750pt}%
\definecolor{currentstroke}{rgb}{1.000000,1.000000,1.000000}%
\pgfsetstrokecolor{currentstroke}%
\pgfsetdash{}{0pt}%
\pgfpathmoveto{\pgfqpoint{10.170538in}{0.637495in}}%
\pgfpathlineto{\pgfqpoint{10.170538in}{9.697495in}}%
\pgfusepath{stroke}%
\end{pgfscope}%
\begin{pgfscope}%
\pgfsetrectcap%
\pgfsetmiterjoin%
\pgfsetlinewidth{1.003750pt}%
\definecolor{currentstroke}{rgb}{1.000000,1.000000,1.000000}%
\pgfsetstrokecolor{currentstroke}%
\pgfsetdash{}{0pt}%
\pgfpathmoveto{\pgfqpoint{0.870538in}{0.637495in}}%
\pgfpathlineto{\pgfqpoint{10.170538in}{0.637495in}}%
\pgfusepath{stroke}%
\end{pgfscope}%
\begin{pgfscope}%
\pgfsetrectcap%
\pgfsetmiterjoin%
\pgfsetlinewidth{1.003750pt}%
\definecolor{currentstroke}{rgb}{1.000000,1.000000,1.000000}%
\pgfsetstrokecolor{currentstroke}%
\pgfsetdash{}{0pt}%
\pgfpathmoveto{\pgfqpoint{0.870538in}{9.697495in}}%
\pgfpathlineto{\pgfqpoint{10.170538in}{9.697495in}}%
\pgfusepath{stroke}%
\end{pgfscope}%
\begin{pgfscope}%
\definecolor{textcolor}{rgb}{0.000000,0.000000,0.000000}%
\pgfsetstrokecolor{textcolor}%
\pgfsetfillcolor{textcolor}%
\pgftext[x=5.520538in,y=9.780828in,,base]{\color{textcolor}\rmfamily\fontsize{24.000000}{28.800000}\selectfont UIUC Annual Electric Generation}%
\end{pgfscope}%
\begin{pgfscope}%
\pgfsetbuttcap%
\pgfsetmiterjoin%
\definecolor{currentfill}{rgb}{0.269412,0.269412,0.269412}%
\pgfsetfillcolor{currentfill}%
\pgfsetfillopacity{0.500000}%
\pgfsetlinewidth{0.501875pt}%
\definecolor{currentstroke}{rgb}{0.269412,0.269412,0.269412}%
\pgfsetstrokecolor{currentstroke}%
\pgfsetstrokeopacity{0.500000}%
\pgfsetdash{}{0pt}%
\pgfpathmoveto{\pgfqpoint{1.053871in}{7.545181in}}%
\pgfpathlineto{\pgfqpoint{3.273212in}{7.545181in}}%
\pgfpathquadraticcurveto{\pgfqpoint{3.317657in}{7.545181in}}{\pgfqpoint{3.317657in}{7.589626in}}%
\pgfpathlineto{\pgfqpoint{3.317657in}{9.514162in}}%
\pgfpathquadraticcurveto{\pgfqpoint{3.317657in}{9.558606in}}{\pgfqpoint{3.273212in}{9.558606in}}%
\pgfpathlineto{\pgfqpoint{1.053871in}{9.558606in}}%
\pgfpathquadraticcurveto{\pgfqpoint{1.009427in}{9.558606in}}{\pgfqpoint{1.009427in}{9.514162in}}%
\pgfpathlineto{\pgfqpoint{1.009427in}{7.589626in}}%
\pgfpathquadraticcurveto{\pgfqpoint{1.009427in}{7.545181in}}{\pgfqpoint{1.053871in}{7.545181in}}%
\pgfpathclose%
\pgfusepath{stroke,fill}%
\end{pgfscope}%
\begin{pgfscope}%
\pgfsetbuttcap%
\pgfsetmiterjoin%
\definecolor{currentfill}{rgb}{0.898039,0.898039,0.898039}%
\pgfsetfillcolor{currentfill}%
\pgfsetlinewidth{0.501875pt}%
\definecolor{currentstroke}{rgb}{0.800000,0.800000,0.800000}%
\pgfsetstrokecolor{currentstroke}%
\pgfsetdash{}{0pt}%
\pgfpathmoveto{\pgfqpoint{1.026093in}{7.572959in}}%
\pgfpathlineto{\pgfqpoint{3.245434in}{7.572959in}}%
\pgfpathquadraticcurveto{\pgfqpoint{3.289879in}{7.572959in}}{\pgfqpoint{3.289879in}{7.617403in}}%
\pgfpathlineto{\pgfqpoint{3.289879in}{9.541940in}}%
\pgfpathquadraticcurveto{\pgfqpoint{3.289879in}{9.586384in}}{\pgfqpoint{3.245434in}{9.586384in}}%
\pgfpathlineto{\pgfqpoint{1.026093in}{9.586384in}}%
\pgfpathquadraticcurveto{\pgfqpoint{0.981649in}{9.586384in}}{\pgfqpoint{0.981649in}{9.541940in}}%
\pgfpathlineto{\pgfqpoint{0.981649in}{7.617403in}}%
\pgfpathquadraticcurveto{\pgfqpoint{0.981649in}{7.572959in}}{\pgfqpoint{1.026093in}{7.572959in}}%
\pgfpathclose%
\pgfusepath{stroke,fill}%
\end{pgfscope}%
\begin{pgfscope}%
\pgfsetbuttcap%
\pgfsetmiterjoin%
\definecolor{currentfill}{rgb}{0.839216,0.152941,0.156863}%
\pgfsetfillcolor{currentfill}%
\pgfsetfillopacity{0.990000}%
\pgfsetlinewidth{0.000000pt}%
\definecolor{currentstroke}{rgb}{0.000000,0.000000,0.000000}%
\pgfsetstrokecolor{currentstroke}%
\pgfsetstrokeopacity{0.990000}%
\pgfsetdash{}{0pt}%
\pgfpathmoveto{\pgfqpoint{1.070538in}{9.330828in}}%
\pgfpathlineto{\pgfqpoint{1.514982in}{9.330828in}}%
\pgfpathlineto{\pgfqpoint{1.514982in}{9.486384in}}%
\pgfpathlineto{\pgfqpoint{1.070538in}{9.486384in}}%
\pgfpathclose%
\pgfusepath{fill}%
\end{pgfscope}%
\begin{pgfscope}%
\pgfsetbuttcap%
\pgfsetmiterjoin%
\definecolor{currentfill}{rgb}{0.839216,0.152941,0.156863}%
\pgfsetfillcolor{currentfill}%
\pgfsetfillopacity{0.990000}%
\pgfsetlinewidth{0.000000pt}%
\definecolor{currentstroke}{rgb}{0.000000,0.000000,0.000000}%
\pgfsetstrokecolor{currentstroke}%
\pgfsetstrokeopacity{0.990000}%
\pgfsetdash{}{0pt}%
\pgfpathmoveto{\pgfqpoint{1.070538in}{9.330828in}}%
\pgfpathlineto{\pgfqpoint{1.514982in}{9.330828in}}%
\pgfpathlineto{\pgfqpoint{1.514982in}{9.486384in}}%
\pgfpathlineto{\pgfqpoint{1.070538in}{9.486384in}}%
\pgfpathclose%
\pgfusepath{clip}%
\pgfsys@defobject{currentpattern}{\pgfqpoint{0in}{0in}}{\pgfqpoint{1in}{1in}}{%
\begin{pgfscope}%
\pgfpathrectangle{\pgfqpoint{0in}{0in}}{\pgfqpoint{1in}{1in}}%
\pgfusepath{clip}%
\pgfpathmoveto{\pgfqpoint{-0.500000in}{0.500000in}}%
\pgfpathlineto{\pgfqpoint{0.500000in}{1.500000in}}%
\pgfpathmoveto{\pgfqpoint{-0.333333in}{0.333333in}}%
\pgfpathlineto{\pgfqpoint{0.666667in}{1.333333in}}%
\pgfpathmoveto{\pgfqpoint{-0.166667in}{0.166667in}}%
\pgfpathlineto{\pgfqpoint{0.833333in}{1.166667in}}%
\pgfpathmoveto{\pgfqpoint{0.000000in}{0.000000in}}%
\pgfpathlineto{\pgfqpoint{1.000000in}{1.000000in}}%
\pgfpathmoveto{\pgfqpoint{0.166667in}{-0.166667in}}%
\pgfpathlineto{\pgfqpoint{1.166667in}{0.833333in}}%
\pgfpathmoveto{\pgfqpoint{0.333333in}{-0.333333in}}%
\pgfpathlineto{\pgfqpoint{1.333333in}{0.666667in}}%
\pgfpathmoveto{\pgfqpoint{0.500000in}{-0.500000in}}%
\pgfpathlineto{\pgfqpoint{1.500000in}{0.500000in}}%
\pgfpathmoveto{\pgfqpoint{-0.500000in}{0.500000in}}%
\pgfpathlineto{\pgfqpoint{0.500000in}{-0.500000in}}%
\pgfpathmoveto{\pgfqpoint{-0.333333in}{0.666667in}}%
\pgfpathlineto{\pgfqpoint{0.666667in}{-0.333333in}}%
\pgfpathmoveto{\pgfqpoint{-0.166667in}{0.833333in}}%
\pgfpathlineto{\pgfqpoint{0.833333in}{-0.166667in}}%
\pgfpathmoveto{\pgfqpoint{0.000000in}{1.000000in}}%
\pgfpathlineto{\pgfqpoint{1.000000in}{0.000000in}}%
\pgfpathmoveto{\pgfqpoint{0.166667in}{1.166667in}}%
\pgfpathlineto{\pgfqpoint{1.166667in}{0.166667in}}%
\pgfpathmoveto{\pgfqpoint{0.333333in}{1.333333in}}%
\pgfpathlineto{\pgfqpoint{1.333333in}{0.333333in}}%
\pgfpathmoveto{\pgfqpoint{0.500000in}{1.500000in}}%
\pgfpathlineto{\pgfqpoint{1.500000in}{0.500000in}}%
\pgfusepath{stroke}%
\end{pgfscope}%
}%
\pgfsys@transformshift{1.070538in}{9.330828in}%
\pgfsys@useobject{currentpattern}{}%
\pgfsys@transformshift{1in}{0in}%
\pgfsys@transformshift{-1in}{0in}%
\pgfsys@transformshift{0in}{1in}%
\end{pgfscope}%
\begin{pgfscope}%
\definecolor{textcolor}{rgb}{0.000000,0.000000,0.000000}%
\pgfsetstrokecolor{textcolor}%
\pgfsetfillcolor{textcolor}%
\pgftext[x=1.692760in,y=9.330828in,left,base]{\color{textcolor}\rmfamily\fontsize{16.000000}{19.200000}\selectfont ABBOTT\_TB}%
\end{pgfscope}%
\begin{pgfscope}%
\pgfsetbuttcap%
\pgfsetmiterjoin%
\definecolor{currentfill}{rgb}{0.549020,0.337255,0.294118}%
\pgfsetfillcolor{currentfill}%
\pgfsetfillopacity{0.990000}%
\pgfsetlinewidth{0.000000pt}%
\definecolor{currentstroke}{rgb}{0.000000,0.000000,0.000000}%
\pgfsetstrokecolor{currentstroke}%
\pgfsetstrokeopacity{0.990000}%
\pgfsetdash{}{0pt}%
\pgfpathmoveto{\pgfqpoint{1.070538in}{9.006369in}}%
\pgfpathlineto{\pgfqpoint{1.514982in}{9.006369in}}%
\pgfpathlineto{\pgfqpoint{1.514982in}{9.161924in}}%
\pgfpathlineto{\pgfqpoint{1.070538in}{9.161924in}}%
\pgfpathclose%
\pgfusepath{fill}%
\end{pgfscope}%
\begin{pgfscope}%
\pgfsetbuttcap%
\pgfsetmiterjoin%
\definecolor{currentfill}{rgb}{0.549020,0.337255,0.294118}%
\pgfsetfillcolor{currentfill}%
\pgfsetfillopacity{0.990000}%
\pgfsetlinewidth{0.000000pt}%
\definecolor{currentstroke}{rgb}{0.000000,0.000000,0.000000}%
\pgfsetstrokecolor{currentstroke}%
\pgfsetstrokeopacity{0.990000}%
\pgfsetdash{}{0pt}%
\pgfpathmoveto{\pgfqpoint{1.070538in}{9.006369in}}%
\pgfpathlineto{\pgfqpoint{1.514982in}{9.006369in}}%
\pgfpathlineto{\pgfqpoint{1.514982in}{9.161924in}}%
\pgfpathlineto{\pgfqpoint{1.070538in}{9.161924in}}%
\pgfpathclose%
\pgfusepath{clip}%
\pgfsys@defobject{currentpattern}{\pgfqpoint{0in}{0in}}{\pgfqpoint{1in}{1in}}{%
\begin{pgfscope}%
\pgfpathrectangle{\pgfqpoint{0in}{0in}}{\pgfqpoint{1in}{1in}}%
\pgfusepath{clip}%
\pgfpathmoveto{\pgfqpoint{0.000000in}{-0.058333in}}%
\pgfpathcurveto{\pgfqpoint{0.015470in}{-0.058333in}}{\pgfqpoint{0.030309in}{-0.052187in}}{\pgfqpoint{0.041248in}{-0.041248in}}%
\pgfpathcurveto{\pgfqpoint{0.052187in}{-0.030309in}}{\pgfqpoint{0.058333in}{-0.015470in}}{\pgfqpoint{0.058333in}{0.000000in}}%
\pgfpathcurveto{\pgfqpoint{0.058333in}{0.015470in}}{\pgfqpoint{0.052187in}{0.030309in}}{\pgfqpoint{0.041248in}{0.041248in}}%
\pgfpathcurveto{\pgfqpoint{0.030309in}{0.052187in}}{\pgfqpoint{0.015470in}{0.058333in}}{\pgfqpoint{0.000000in}{0.058333in}}%
\pgfpathcurveto{\pgfqpoint{-0.015470in}{0.058333in}}{\pgfqpoint{-0.030309in}{0.052187in}}{\pgfqpoint{-0.041248in}{0.041248in}}%
\pgfpathcurveto{\pgfqpoint{-0.052187in}{0.030309in}}{\pgfqpoint{-0.058333in}{0.015470in}}{\pgfqpoint{-0.058333in}{0.000000in}}%
\pgfpathcurveto{\pgfqpoint{-0.058333in}{-0.015470in}}{\pgfqpoint{-0.052187in}{-0.030309in}}{\pgfqpoint{-0.041248in}{-0.041248in}}%
\pgfpathcurveto{\pgfqpoint{-0.030309in}{-0.052187in}}{\pgfqpoint{-0.015470in}{-0.058333in}}{\pgfqpoint{0.000000in}{-0.058333in}}%
\pgfpathclose%
\pgfpathmoveto{\pgfqpoint{0.000000in}{-0.052500in}}%
\pgfpathcurveto{\pgfqpoint{0.000000in}{-0.052500in}}{\pgfqpoint{-0.013923in}{-0.052500in}}{\pgfqpoint{-0.027278in}{-0.046968in}}%
\pgfpathcurveto{\pgfqpoint{-0.037123in}{-0.037123in}}{\pgfqpoint{-0.046968in}{-0.027278in}}{\pgfqpoint{-0.052500in}{-0.013923in}}%
\pgfpathcurveto{\pgfqpoint{-0.052500in}{0.000000in}}{\pgfqpoint{-0.052500in}{0.013923in}}{\pgfqpoint{-0.046968in}{0.027278in}}%
\pgfpathcurveto{\pgfqpoint{-0.037123in}{0.037123in}}{\pgfqpoint{-0.027278in}{0.046968in}}{\pgfqpoint{-0.013923in}{0.052500in}}%
\pgfpathcurveto{\pgfqpoint{0.000000in}{0.052500in}}{\pgfqpoint{0.013923in}{0.052500in}}{\pgfqpoint{0.027278in}{0.046968in}}%
\pgfpathcurveto{\pgfqpoint{0.037123in}{0.037123in}}{\pgfqpoint{0.046968in}{0.027278in}}{\pgfqpoint{0.052500in}{0.013923in}}%
\pgfpathcurveto{\pgfqpoint{0.052500in}{0.000000in}}{\pgfqpoint{0.052500in}{-0.013923in}}{\pgfqpoint{0.046968in}{-0.027278in}}%
\pgfpathcurveto{\pgfqpoint{0.037123in}{-0.037123in}}{\pgfqpoint{0.027278in}{-0.046968in}}{\pgfqpoint{0.013923in}{-0.052500in}}%
\pgfpathclose%
\pgfpathmoveto{\pgfqpoint{0.166667in}{-0.058333in}}%
\pgfpathcurveto{\pgfqpoint{0.182137in}{-0.058333in}}{\pgfqpoint{0.196975in}{-0.052187in}}{\pgfqpoint{0.207915in}{-0.041248in}}%
\pgfpathcurveto{\pgfqpoint{0.218854in}{-0.030309in}}{\pgfqpoint{0.225000in}{-0.015470in}}{\pgfqpoint{0.225000in}{0.000000in}}%
\pgfpathcurveto{\pgfqpoint{0.225000in}{0.015470in}}{\pgfqpoint{0.218854in}{0.030309in}}{\pgfqpoint{0.207915in}{0.041248in}}%
\pgfpathcurveto{\pgfqpoint{0.196975in}{0.052187in}}{\pgfqpoint{0.182137in}{0.058333in}}{\pgfqpoint{0.166667in}{0.058333in}}%
\pgfpathcurveto{\pgfqpoint{0.151196in}{0.058333in}}{\pgfqpoint{0.136358in}{0.052187in}}{\pgfqpoint{0.125419in}{0.041248in}}%
\pgfpathcurveto{\pgfqpoint{0.114480in}{0.030309in}}{\pgfqpoint{0.108333in}{0.015470in}}{\pgfqpoint{0.108333in}{0.000000in}}%
\pgfpathcurveto{\pgfqpoint{0.108333in}{-0.015470in}}{\pgfqpoint{0.114480in}{-0.030309in}}{\pgfqpoint{0.125419in}{-0.041248in}}%
\pgfpathcurveto{\pgfqpoint{0.136358in}{-0.052187in}}{\pgfqpoint{0.151196in}{-0.058333in}}{\pgfqpoint{0.166667in}{-0.058333in}}%
\pgfpathclose%
\pgfpathmoveto{\pgfqpoint{0.166667in}{-0.052500in}}%
\pgfpathcurveto{\pgfqpoint{0.166667in}{-0.052500in}}{\pgfqpoint{0.152744in}{-0.052500in}}{\pgfqpoint{0.139389in}{-0.046968in}}%
\pgfpathcurveto{\pgfqpoint{0.129544in}{-0.037123in}}{\pgfqpoint{0.119698in}{-0.027278in}}{\pgfqpoint{0.114167in}{-0.013923in}}%
\pgfpathcurveto{\pgfqpoint{0.114167in}{0.000000in}}{\pgfqpoint{0.114167in}{0.013923in}}{\pgfqpoint{0.119698in}{0.027278in}}%
\pgfpathcurveto{\pgfqpoint{0.129544in}{0.037123in}}{\pgfqpoint{0.139389in}{0.046968in}}{\pgfqpoint{0.152744in}{0.052500in}}%
\pgfpathcurveto{\pgfqpoint{0.166667in}{0.052500in}}{\pgfqpoint{0.180590in}{0.052500in}}{\pgfqpoint{0.193945in}{0.046968in}}%
\pgfpathcurveto{\pgfqpoint{0.203790in}{0.037123in}}{\pgfqpoint{0.213635in}{0.027278in}}{\pgfqpoint{0.219167in}{0.013923in}}%
\pgfpathcurveto{\pgfqpoint{0.219167in}{0.000000in}}{\pgfqpoint{0.219167in}{-0.013923in}}{\pgfqpoint{0.213635in}{-0.027278in}}%
\pgfpathcurveto{\pgfqpoint{0.203790in}{-0.037123in}}{\pgfqpoint{0.193945in}{-0.046968in}}{\pgfqpoint{0.180590in}{-0.052500in}}%
\pgfpathclose%
\pgfpathmoveto{\pgfqpoint{0.333333in}{-0.058333in}}%
\pgfpathcurveto{\pgfqpoint{0.348804in}{-0.058333in}}{\pgfqpoint{0.363642in}{-0.052187in}}{\pgfqpoint{0.374581in}{-0.041248in}}%
\pgfpathcurveto{\pgfqpoint{0.385520in}{-0.030309in}}{\pgfqpoint{0.391667in}{-0.015470in}}{\pgfqpoint{0.391667in}{0.000000in}}%
\pgfpathcurveto{\pgfqpoint{0.391667in}{0.015470in}}{\pgfqpoint{0.385520in}{0.030309in}}{\pgfqpoint{0.374581in}{0.041248in}}%
\pgfpathcurveto{\pgfqpoint{0.363642in}{0.052187in}}{\pgfqpoint{0.348804in}{0.058333in}}{\pgfqpoint{0.333333in}{0.058333in}}%
\pgfpathcurveto{\pgfqpoint{0.317863in}{0.058333in}}{\pgfqpoint{0.303025in}{0.052187in}}{\pgfqpoint{0.292085in}{0.041248in}}%
\pgfpathcurveto{\pgfqpoint{0.281146in}{0.030309in}}{\pgfqpoint{0.275000in}{0.015470in}}{\pgfqpoint{0.275000in}{0.000000in}}%
\pgfpathcurveto{\pgfqpoint{0.275000in}{-0.015470in}}{\pgfqpoint{0.281146in}{-0.030309in}}{\pgfqpoint{0.292085in}{-0.041248in}}%
\pgfpathcurveto{\pgfqpoint{0.303025in}{-0.052187in}}{\pgfqpoint{0.317863in}{-0.058333in}}{\pgfqpoint{0.333333in}{-0.058333in}}%
\pgfpathclose%
\pgfpathmoveto{\pgfqpoint{0.333333in}{-0.052500in}}%
\pgfpathcurveto{\pgfqpoint{0.333333in}{-0.052500in}}{\pgfqpoint{0.319410in}{-0.052500in}}{\pgfqpoint{0.306055in}{-0.046968in}}%
\pgfpathcurveto{\pgfqpoint{0.296210in}{-0.037123in}}{\pgfqpoint{0.286365in}{-0.027278in}}{\pgfqpoint{0.280833in}{-0.013923in}}%
\pgfpathcurveto{\pgfqpoint{0.280833in}{0.000000in}}{\pgfqpoint{0.280833in}{0.013923in}}{\pgfqpoint{0.286365in}{0.027278in}}%
\pgfpathcurveto{\pgfqpoint{0.296210in}{0.037123in}}{\pgfqpoint{0.306055in}{0.046968in}}{\pgfqpoint{0.319410in}{0.052500in}}%
\pgfpathcurveto{\pgfqpoint{0.333333in}{0.052500in}}{\pgfqpoint{0.347256in}{0.052500in}}{\pgfqpoint{0.360611in}{0.046968in}}%
\pgfpathcurveto{\pgfqpoint{0.370456in}{0.037123in}}{\pgfqpoint{0.380302in}{0.027278in}}{\pgfqpoint{0.385833in}{0.013923in}}%
\pgfpathcurveto{\pgfqpoint{0.385833in}{0.000000in}}{\pgfqpoint{0.385833in}{-0.013923in}}{\pgfqpoint{0.380302in}{-0.027278in}}%
\pgfpathcurveto{\pgfqpoint{0.370456in}{-0.037123in}}{\pgfqpoint{0.360611in}{-0.046968in}}{\pgfqpoint{0.347256in}{-0.052500in}}%
\pgfpathclose%
\pgfpathmoveto{\pgfqpoint{0.500000in}{-0.058333in}}%
\pgfpathcurveto{\pgfqpoint{0.515470in}{-0.058333in}}{\pgfqpoint{0.530309in}{-0.052187in}}{\pgfqpoint{0.541248in}{-0.041248in}}%
\pgfpathcurveto{\pgfqpoint{0.552187in}{-0.030309in}}{\pgfqpoint{0.558333in}{-0.015470in}}{\pgfqpoint{0.558333in}{0.000000in}}%
\pgfpathcurveto{\pgfqpoint{0.558333in}{0.015470in}}{\pgfqpoint{0.552187in}{0.030309in}}{\pgfqpoint{0.541248in}{0.041248in}}%
\pgfpathcurveto{\pgfqpoint{0.530309in}{0.052187in}}{\pgfqpoint{0.515470in}{0.058333in}}{\pgfqpoint{0.500000in}{0.058333in}}%
\pgfpathcurveto{\pgfqpoint{0.484530in}{0.058333in}}{\pgfqpoint{0.469691in}{0.052187in}}{\pgfqpoint{0.458752in}{0.041248in}}%
\pgfpathcurveto{\pgfqpoint{0.447813in}{0.030309in}}{\pgfqpoint{0.441667in}{0.015470in}}{\pgfqpoint{0.441667in}{0.000000in}}%
\pgfpathcurveto{\pgfqpoint{0.441667in}{-0.015470in}}{\pgfqpoint{0.447813in}{-0.030309in}}{\pgfqpoint{0.458752in}{-0.041248in}}%
\pgfpathcurveto{\pgfqpoint{0.469691in}{-0.052187in}}{\pgfqpoint{0.484530in}{-0.058333in}}{\pgfqpoint{0.500000in}{-0.058333in}}%
\pgfpathclose%
\pgfpathmoveto{\pgfqpoint{0.500000in}{-0.052500in}}%
\pgfpathcurveto{\pgfqpoint{0.500000in}{-0.052500in}}{\pgfqpoint{0.486077in}{-0.052500in}}{\pgfqpoint{0.472722in}{-0.046968in}}%
\pgfpathcurveto{\pgfqpoint{0.462877in}{-0.037123in}}{\pgfqpoint{0.453032in}{-0.027278in}}{\pgfqpoint{0.447500in}{-0.013923in}}%
\pgfpathcurveto{\pgfqpoint{0.447500in}{0.000000in}}{\pgfqpoint{0.447500in}{0.013923in}}{\pgfqpoint{0.453032in}{0.027278in}}%
\pgfpathcurveto{\pgfqpoint{0.462877in}{0.037123in}}{\pgfqpoint{0.472722in}{0.046968in}}{\pgfqpoint{0.486077in}{0.052500in}}%
\pgfpathcurveto{\pgfqpoint{0.500000in}{0.052500in}}{\pgfqpoint{0.513923in}{0.052500in}}{\pgfqpoint{0.527278in}{0.046968in}}%
\pgfpathcurveto{\pgfqpoint{0.537123in}{0.037123in}}{\pgfqpoint{0.546968in}{0.027278in}}{\pgfqpoint{0.552500in}{0.013923in}}%
\pgfpathcurveto{\pgfqpoint{0.552500in}{0.000000in}}{\pgfqpoint{0.552500in}{-0.013923in}}{\pgfqpoint{0.546968in}{-0.027278in}}%
\pgfpathcurveto{\pgfqpoint{0.537123in}{-0.037123in}}{\pgfqpoint{0.527278in}{-0.046968in}}{\pgfqpoint{0.513923in}{-0.052500in}}%
\pgfpathclose%
\pgfpathmoveto{\pgfqpoint{0.666667in}{-0.058333in}}%
\pgfpathcurveto{\pgfqpoint{0.682137in}{-0.058333in}}{\pgfqpoint{0.696975in}{-0.052187in}}{\pgfqpoint{0.707915in}{-0.041248in}}%
\pgfpathcurveto{\pgfqpoint{0.718854in}{-0.030309in}}{\pgfqpoint{0.725000in}{-0.015470in}}{\pgfqpoint{0.725000in}{0.000000in}}%
\pgfpathcurveto{\pgfqpoint{0.725000in}{0.015470in}}{\pgfqpoint{0.718854in}{0.030309in}}{\pgfqpoint{0.707915in}{0.041248in}}%
\pgfpathcurveto{\pgfqpoint{0.696975in}{0.052187in}}{\pgfqpoint{0.682137in}{0.058333in}}{\pgfqpoint{0.666667in}{0.058333in}}%
\pgfpathcurveto{\pgfqpoint{0.651196in}{0.058333in}}{\pgfqpoint{0.636358in}{0.052187in}}{\pgfqpoint{0.625419in}{0.041248in}}%
\pgfpathcurveto{\pgfqpoint{0.614480in}{0.030309in}}{\pgfqpoint{0.608333in}{0.015470in}}{\pgfqpoint{0.608333in}{0.000000in}}%
\pgfpathcurveto{\pgfqpoint{0.608333in}{-0.015470in}}{\pgfqpoint{0.614480in}{-0.030309in}}{\pgfqpoint{0.625419in}{-0.041248in}}%
\pgfpathcurveto{\pgfqpoint{0.636358in}{-0.052187in}}{\pgfqpoint{0.651196in}{-0.058333in}}{\pgfqpoint{0.666667in}{-0.058333in}}%
\pgfpathclose%
\pgfpathmoveto{\pgfqpoint{0.666667in}{-0.052500in}}%
\pgfpathcurveto{\pgfqpoint{0.666667in}{-0.052500in}}{\pgfqpoint{0.652744in}{-0.052500in}}{\pgfqpoint{0.639389in}{-0.046968in}}%
\pgfpathcurveto{\pgfqpoint{0.629544in}{-0.037123in}}{\pgfqpoint{0.619698in}{-0.027278in}}{\pgfqpoint{0.614167in}{-0.013923in}}%
\pgfpathcurveto{\pgfqpoint{0.614167in}{0.000000in}}{\pgfqpoint{0.614167in}{0.013923in}}{\pgfqpoint{0.619698in}{0.027278in}}%
\pgfpathcurveto{\pgfqpoint{0.629544in}{0.037123in}}{\pgfqpoint{0.639389in}{0.046968in}}{\pgfqpoint{0.652744in}{0.052500in}}%
\pgfpathcurveto{\pgfqpoint{0.666667in}{0.052500in}}{\pgfqpoint{0.680590in}{0.052500in}}{\pgfqpoint{0.693945in}{0.046968in}}%
\pgfpathcurveto{\pgfqpoint{0.703790in}{0.037123in}}{\pgfqpoint{0.713635in}{0.027278in}}{\pgfqpoint{0.719167in}{0.013923in}}%
\pgfpathcurveto{\pgfqpoint{0.719167in}{0.000000in}}{\pgfqpoint{0.719167in}{-0.013923in}}{\pgfqpoint{0.713635in}{-0.027278in}}%
\pgfpathcurveto{\pgfqpoint{0.703790in}{-0.037123in}}{\pgfqpoint{0.693945in}{-0.046968in}}{\pgfqpoint{0.680590in}{-0.052500in}}%
\pgfpathclose%
\pgfpathmoveto{\pgfqpoint{0.833333in}{-0.058333in}}%
\pgfpathcurveto{\pgfqpoint{0.848804in}{-0.058333in}}{\pgfqpoint{0.863642in}{-0.052187in}}{\pgfqpoint{0.874581in}{-0.041248in}}%
\pgfpathcurveto{\pgfqpoint{0.885520in}{-0.030309in}}{\pgfqpoint{0.891667in}{-0.015470in}}{\pgfqpoint{0.891667in}{0.000000in}}%
\pgfpathcurveto{\pgfqpoint{0.891667in}{0.015470in}}{\pgfqpoint{0.885520in}{0.030309in}}{\pgfqpoint{0.874581in}{0.041248in}}%
\pgfpathcurveto{\pgfqpoint{0.863642in}{0.052187in}}{\pgfqpoint{0.848804in}{0.058333in}}{\pgfqpoint{0.833333in}{0.058333in}}%
\pgfpathcurveto{\pgfqpoint{0.817863in}{0.058333in}}{\pgfqpoint{0.803025in}{0.052187in}}{\pgfqpoint{0.792085in}{0.041248in}}%
\pgfpathcurveto{\pgfqpoint{0.781146in}{0.030309in}}{\pgfqpoint{0.775000in}{0.015470in}}{\pgfqpoint{0.775000in}{0.000000in}}%
\pgfpathcurveto{\pgfqpoint{0.775000in}{-0.015470in}}{\pgfqpoint{0.781146in}{-0.030309in}}{\pgfqpoint{0.792085in}{-0.041248in}}%
\pgfpathcurveto{\pgfqpoint{0.803025in}{-0.052187in}}{\pgfqpoint{0.817863in}{-0.058333in}}{\pgfqpoint{0.833333in}{-0.058333in}}%
\pgfpathclose%
\pgfpathmoveto{\pgfqpoint{0.833333in}{-0.052500in}}%
\pgfpathcurveto{\pgfqpoint{0.833333in}{-0.052500in}}{\pgfqpoint{0.819410in}{-0.052500in}}{\pgfqpoint{0.806055in}{-0.046968in}}%
\pgfpathcurveto{\pgfqpoint{0.796210in}{-0.037123in}}{\pgfqpoint{0.786365in}{-0.027278in}}{\pgfqpoint{0.780833in}{-0.013923in}}%
\pgfpathcurveto{\pgfqpoint{0.780833in}{0.000000in}}{\pgfqpoint{0.780833in}{0.013923in}}{\pgfqpoint{0.786365in}{0.027278in}}%
\pgfpathcurveto{\pgfqpoint{0.796210in}{0.037123in}}{\pgfqpoint{0.806055in}{0.046968in}}{\pgfqpoint{0.819410in}{0.052500in}}%
\pgfpathcurveto{\pgfqpoint{0.833333in}{0.052500in}}{\pgfqpoint{0.847256in}{0.052500in}}{\pgfqpoint{0.860611in}{0.046968in}}%
\pgfpathcurveto{\pgfqpoint{0.870456in}{0.037123in}}{\pgfqpoint{0.880302in}{0.027278in}}{\pgfqpoint{0.885833in}{0.013923in}}%
\pgfpathcurveto{\pgfqpoint{0.885833in}{0.000000in}}{\pgfqpoint{0.885833in}{-0.013923in}}{\pgfqpoint{0.880302in}{-0.027278in}}%
\pgfpathcurveto{\pgfqpoint{0.870456in}{-0.037123in}}{\pgfqpoint{0.860611in}{-0.046968in}}{\pgfqpoint{0.847256in}{-0.052500in}}%
\pgfpathclose%
\pgfpathmoveto{\pgfqpoint{1.000000in}{-0.058333in}}%
\pgfpathcurveto{\pgfqpoint{1.015470in}{-0.058333in}}{\pgfqpoint{1.030309in}{-0.052187in}}{\pgfqpoint{1.041248in}{-0.041248in}}%
\pgfpathcurveto{\pgfqpoint{1.052187in}{-0.030309in}}{\pgfqpoint{1.058333in}{-0.015470in}}{\pgfqpoint{1.058333in}{0.000000in}}%
\pgfpathcurveto{\pgfqpoint{1.058333in}{0.015470in}}{\pgfqpoint{1.052187in}{0.030309in}}{\pgfqpoint{1.041248in}{0.041248in}}%
\pgfpathcurveto{\pgfqpoint{1.030309in}{0.052187in}}{\pgfqpoint{1.015470in}{0.058333in}}{\pgfqpoint{1.000000in}{0.058333in}}%
\pgfpathcurveto{\pgfqpoint{0.984530in}{0.058333in}}{\pgfqpoint{0.969691in}{0.052187in}}{\pgfqpoint{0.958752in}{0.041248in}}%
\pgfpathcurveto{\pgfqpoint{0.947813in}{0.030309in}}{\pgfqpoint{0.941667in}{0.015470in}}{\pgfqpoint{0.941667in}{0.000000in}}%
\pgfpathcurveto{\pgfqpoint{0.941667in}{-0.015470in}}{\pgfqpoint{0.947813in}{-0.030309in}}{\pgfqpoint{0.958752in}{-0.041248in}}%
\pgfpathcurveto{\pgfqpoint{0.969691in}{-0.052187in}}{\pgfqpoint{0.984530in}{-0.058333in}}{\pgfqpoint{1.000000in}{-0.058333in}}%
\pgfpathclose%
\pgfpathmoveto{\pgfqpoint{1.000000in}{-0.052500in}}%
\pgfpathcurveto{\pgfqpoint{1.000000in}{-0.052500in}}{\pgfqpoint{0.986077in}{-0.052500in}}{\pgfqpoint{0.972722in}{-0.046968in}}%
\pgfpathcurveto{\pgfqpoint{0.962877in}{-0.037123in}}{\pgfqpoint{0.953032in}{-0.027278in}}{\pgfqpoint{0.947500in}{-0.013923in}}%
\pgfpathcurveto{\pgfqpoint{0.947500in}{0.000000in}}{\pgfqpoint{0.947500in}{0.013923in}}{\pgfqpoint{0.953032in}{0.027278in}}%
\pgfpathcurveto{\pgfqpoint{0.962877in}{0.037123in}}{\pgfqpoint{0.972722in}{0.046968in}}{\pgfqpoint{0.986077in}{0.052500in}}%
\pgfpathcurveto{\pgfqpoint{1.000000in}{0.052500in}}{\pgfqpoint{1.013923in}{0.052500in}}{\pgfqpoint{1.027278in}{0.046968in}}%
\pgfpathcurveto{\pgfqpoint{1.037123in}{0.037123in}}{\pgfqpoint{1.046968in}{0.027278in}}{\pgfqpoint{1.052500in}{0.013923in}}%
\pgfpathcurveto{\pgfqpoint{1.052500in}{0.000000in}}{\pgfqpoint{1.052500in}{-0.013923in}}{\pgfqpoint{1.046968in}{-0.027278in}}%
\pgfpathcurveto{\pgfqpoint{1.037123in}{-0.037123in}}{\pgfqpoint{1.027278in}{-0.046968in}}{\pgfqpoint{1.013923in}{-0.052500in}}%
\pgfpathclose%
\pgfpathmoveto{\pgfqpoint{0.083333in}{0.108333in}}%
\pgfpathcurveto{\pgfqpoint{0.098804in}{0.108333in}}{\pgfqpoint{0.113642in}{0.114480in}}{\pgfqpoint{0.124581in}{0.125419in}}%
\pgfpathcurveto{\pgfqpoint{0.135520in}{0.136358in}}{\pgfqpoint{0.141667in}{0.151196in}}{\pgfqpoint{0.141667in}{0.166667in}}%
\pgfpathcurveto{\pgfqpoint{0.141667in}{0.182137in}}{\pgfqpoint{0.135520in}{0.196975in}}{\pgfqpoint{0.124581in}{0.207915in}}%
\pgfpathcurveto{\pgfqpoint{0.113642in}{0.218854in}}{\pgfqpoint{0.098804in}{0.225000in}}{\pgfqpoint{0.083333in}{0.225000in}}%
\pgfpathcurveto{\pgfqpoint{0.067863in}{0.225000in}}{\pgfqpoint{0.053025in}{0.218854in}}{\pgfqpoint{0.042085in}{0.207915in}}%
\pgfpathcurveto{\pgfqpoint{0.031146in}{0.196975in}}{\pgfqpoint{0.025000in}{0.182137in}}{\pgfqpoint{0.025000in}{0.166667in}}%
\pgfpathcurveto{\pgfqpoint{0.025000in}{0.151196in}}{\pgfqpoint{0.031146in}{0.136358in}}{\pgfqpoint{0.042085in}{0.125419in}}%
\pgfpathcurveto{\pgfqpoint{0.053025in}{0.114480in}}{\pgfqpoint{0.067863in}{0.108333in}}{\pgfqpoint{0.083333in}{0.108333in}}%
\pgfpathclose%
\pgfpathmoveto{\pgfqpoint{0.083333in}{0.114167in}}%
\pgfpathcurveto{\pgfqpoint{0.083333in}{0.114167in}}{\pgfqpoint{0.069410in}{0.114167in}}{\pgfqpoint{0.056055in}{0.119698in}}%
\pgfpathcurveto{\pgfqpoint{0.046210in}{0.129544in}}{\pgfqpoint{0.036365in}{0.139389in}}{\pgfqpoint{0.030833in}{0.152744in}}%
\pgfpathcurveto{\pgfqpoint{0.030833in}{0.166667in}}{\pgfqpoint{0.030833in}{0.180590in}}{\pgfqpoint{0.036365in}{0.193945in}}%
\pgfpathcurveto{\pgfqpoint{0.046210in}{0.203790in}}{\pgfqpoint{0.056055in}{0.213635in}}{\pgfqpoint{0.069410in}{0.219167in}}%
\pgfpathcurveto{\pgfqpoint{0.083333in}{0.219167in}}{\pgfqpoint{0.097256in}{0.219167in}}{\pgfqpoint{0.110611in}{0.213635in}}%
\pgfpathcurveto{\pgfqpoint{0.120456in}{0.203790in}}{\pgfqpoint{0.130302in}{0.193945in}}{\pgfqpoint{0.135833in}{0.180590in}}%
\pgfpathcurveto{\pgfqpoint{0.135833in}{0.166667in}}{\pgfqpoint{0.135833in}{0.152744in}}{\pgfqpoint{0.130302in}{0.139389in}}%
\pgfpathcurveto{\pgfqpoint{0.120456in}{0.129544in}}{\pgfqpoint{0.110611in}{0.119698in}}{\pgfqpoint{0.097256in}{0.114167in}}%
\pgfpathclose%
\pgfpathmoveto{\pgfqpoint{0.250000in}{0.108333in}}%
\pgfpathcurveto{\pgfqpoint{0.265470in}{0.108333in}}{\pgfqpoint{0.280309in}{0.114480in}}{\pgfqpoint{0.291248in}{0.125419in}}%
\pgfpathcurveto{\pgfqpoint{0.302187in}{0.136358in}}{\pgfqpoint{0.308333in}{0.151196in}}{\pgfqpoint{0.308333in}{0.166667in}}%
\pgfpathcurveto{\pgfqpoint{0.308333in}{0.182137in}}{\pgfqpoint{0.302187in}{0.196975in}}{\pgfqpoint{0.291248in}{0.207915in}}%
\pgfpathcurveto{\pgfqpoint{0.280309in}{0.218854in}}{\pgfqpoint{0.265470in}{0.225000in}}{\pgfqpoint{0.250000in}{0.225000in}}%
\pgfpathcurveto{\pgfqpoint{0.234530in}{0.225000in}}{\pgfqpoint{0.219691in}{0.218854in}}{\pgfqpoint{0.208752in}{0.207915in}}%
\pgfpathcurveto{\pgfqpoint{0.197813in}{0.196975in}}{\pgfqpoint{0.191667in}{0.182137in}}{\pgfqpoint{0.191667in}{0.166667in}}%
\pgfpathcurveto{\pgfqpoint{0.191667in}{0.151196in}}{\pgfqpoint{0.197813in}{0.136358in}}{\pgfqpoint{0.208752in}{0.125419in}}%
\pgfpathcurveto{\pgfqpoint{0.219691in}{0.114480in}}{\pgfqpoint{0.234530in}{0.108333in}}{\pgfqpoint{0.250000in}{0.108333in}}%
\pgfpathclose%
\pgfpathmoveto{\pgfqpoint{0.250000in}{0.114167in}}%
\pgfpathcurveto{\pgfqpoint{0.250000in}{0.114167in}}{\pgfqpoint{0.236077in}{0.114167in}}{\pgfqpoint{0.222722in}{0.119698in}}%
\pgfpathcurveto{\pgfqpoint{0.212877in}{0.129544in}}{\pgfqpoint{0.203032in}{0.139389in}}{\pgfqpoint{0.197500in}{0.152744in}}%
\pgfpathcurveto{\pgfqpoint{0.197500in}{0.166667in}}{\pgfqpoint{0.197500in}{0.180590in}}{\pgfqpoint{0.203032in}{0.193945in}}%
\pgfpathcurveto{\pgfqpoint{0.212877in}{0.203790in}}{\pgfqpoint{0.222722in}{0.213635in}}{\pgfqpoint{0.236077in}{0.219167in}}%
\pgfpathcurveto{\pgfqpoint{0.250000in}{0.219167in}}{\pgfqpoint{0.263923in}{0.219167in}}{\pgfqpoint{0.277278in}{0.213635in}}%
\pgfpathcurveto{\pgfqpoint{0.287123in}{0.203790in}}{\pgfqpoint{0.296968in}{0.193945in}}{\pgfqpoint{0.302500in}{0.180590in}}%
\pgfpathcurveto{\pgfqpoint{0.302500in}{0.166667in}}{\pgfqpoint{0.302500in}{0.152744in}}{\pgfqpoint{0.296968in}{0.139389in}}%
\pgfpathcurveto{\pgfqpoint{0.287123in}{0.129544in}}{\pgfqpoint{0.277278in}{0.119698in}}{\pgfqpoint{0.263923in}{0.114167in}}%
\pgfpathclose%
\pgfpathmoveto{\pgfqpoint{0.416667in}{0.108333in}}%
\pgfpathcurveto{\pgfqpoint{0.432137in}{0.108333in}}{\pgfqpoint{0.446975in}{0.114480in}}{\pgfqpoint{0.457915in}{0.125419in}}%
\pgfpathcurveto{\pgfqpoint{0.468854in}{0.136358in}}{\pgfqpoint{0.475000in}{0.151196in}}{\pgfqpoint{0.475000in}{0.166667in}}%
\pgfpathcurveto{\pgfqpoint{0.475000in}{0.182137in}}{\pgfqpoint{0.468854in}{0.196975in}}{\pgfqpoint{0.457915in}{0.207915in}}%
\pgfpathcurveto{\pgfqpoint{0.446975in}{0.218854in}}{\pgfqpoint{0.432137in}{0.225000in}}{\pgfqpoint{0.416667in}{0.225000in}}%
\pgfpathcurveto{\pgfqpoint{0.401196in}{0.225000in}}{\pgfqpoint{0.386358in}{0.218854in}}{\pgfqpoint{0.375419in}{0.207915in}}%
\pgfpathcurveto{\pgfqpoint{0.364480in}{0.196975in}}{\pgfqpoint{0.358333in}{0.182137in}}{\pgfqpoint{0.358333in}{0.166667in}}%
\pgfpathcurveto{\pgfqpoint{0.358333in}{0.151196in}}{\pgfqpoint{0.364480in}{0.136358in}}{\pgfqpoint{0.375419in}{0.125419in}}%
\pgfpathcurveto{\pgfqpoint{0.386358in}{0.114480in}}{\pgfqpoint{0.401196in}{0.108333in}}{\pgfqpoint{0.416667in}{0.108333in}}%
\pgfpathclose%
\pgfpathmoveto{\pgfqpoint{0.416667in}{0.114167in}}%
\pgfpathcurveto{\pgfqpoint{0.416667in}{0.114167in}}{\pgfqpoint{0.402744in}{0.114167in}}{\pgfqpoint{0.389389in}{0.119698in}}%
\pgfpathcurveto{\pgfqpoint{0.379544in}{0.129544in}}{\pgfqpoint{0.369698in}{0.139389in}}{\pgfqpoint{0.364167in}{0.152744in}}%
\pgfpathcurveto{\pgfqpoint{0.364167in}{0.166667in}}{\pgfqpoint{0.364167in}{0.180590in}}{\pgfqpoint{0.369698in}{0.193945in}}%
\pgfpathcurveto{\pgfqpoint{0.379544in}{0.203790in}}{\pgfqpoint{0.389389in}{0.213635in}}{\pgfqpoint{0.402744in}{0.219167in}}%
\pgfpathcurveto{\pgfqpoint{0.416667in}{0.219167in}}{\pgfqpoint{0.430590in}{0.219167in}}{\pgfqpoint{0.443945in}{0.213635in}}%
\pgfpathcurveto{\pgfqpoint{0.453790in}{0.203790in}}{\pgfqpoint{0.463635in}{0.193945in}}{\pgfqpoint{0.469167in}{0.180590in}}%
\pgfpathcurveto{\pgfqpoint{0.469167in}{0.166667in}}{\pgfqpoint{0.469167in}{0.152744in}}{\pgfqpoint{0.463635in}{0.139389in}}%
\pgfpathcurveto{\pgfqpoint{0.453790in}{0.129544in}}{\pgfqpoint{0.443945in}{0.119698in}}{\pgfqpoint{0.430590in}{0.114167in}}%
\pgfpathclose%
\pgfpathmoveto{\pgfqpoint{0.583333in}{0.108333in}}%
\pgfpathcurveto{\pgfqpoint{0.598804in}{0.108333in}}{\pgfqpoint{0.613642in}{0.114480in}}{\pgfqpoint{0.624581in}{0.125419in}}%
\pgfpathcurveto{\pgfqpoint{0.635520in}{0.136358in}}{\pgfqpoint{0.641667in}{0.151196in}}{\pgfqpoint{0.641667in}{0.166667in}}%
\pgfpathcurveto{\pgfqpoint{0.641667in}{0.182137in}}{\pgfqpoint{0.635520in}{0.196975in}}{\pgfqpoint{0.624581in}{0.207915in}}%
\pgfpathcurveto{\pgfqpoint{0.613642in}{0.218854in}}{\pgfqpoint{0.598804in}{0.225000in}}{\pgfqpoint{0.583333in}{0.225000in}}%
\pgfpathcurveto{\pgfqpoint{0.567863in}{0.225000in}}{\pgfqpoint{0.553025in}{0.218854in}}{\pgfqpoint{0.542085in}{0.207915in}}%
\pgfpathcurveto{\pgfqpoint{0.531146in}{0.196975in}}{\pgfqpoint{0.525000in}{0.182137in}}{\pgfqpoint{0.525000in}{0.166667in}}%
\pgfpathcurveto{\pgfqpoint{0.525000in}{0.151196in}}{\pgfqpoint{0.531146in}{0.136358in}}{\pgfqpoint{0.542085in}{0.125419in}}%
\pgfpathcurveto{\pgfqpoint{0.553025in}{0.114480in}}{\pgfqpoint{0.567863in}{0.108333in}}{\pgfqpoint{0.583333in}{0.108333in}}%
\pgfpathclose%
\pgfpathmoveto{\pgfqpoint{0.583333in}{0.114167in}}%
\pgfpathcurveto{\pgfqpoint{0.583333in}{0.114167in}}{\pgfqpoint{0.569410in}{0.114167in}}{\pgfqpoint{0.556055in}{0.119698in}}%
\pgfpathcurveto{\pgfqpoint{0.546210in}{0.129544in}}{\pgfqpoint{0.536365in}{0.139389in}}{\pgfqpoint{0.530833in}{0.152744in}}%
\pgfpathcurveto{\pgfqpoint{0.530833in}{0.166667in}}{\pgfqpoint{0.530833in}{0.180590in}}{\pgfqpoint{0.536365in}{0.193945in}}%
\pgfpathcurveto{\pgfqpoint{0.546210in}{0.203790in}}{\pgfqpoint{0.556055in}{0.213635in}}{\pgfqpoint{0.569410in}{0.219167in}}%
\pgfpathcurveto{\pgfqpoint{0.583333in}{0.219167in}}{\pgfqpoint{0.597256in}{0.219167in}}{\pgfqpoint{0.610611in}{0.213635in}}%
\pgfpathcurveto{\pgfqpoint{0.620456in}{0.203790in}}{\pgfqpoint{0.630302in}{0.193945in}}{\pgfqpoint{0.635833in}{0.180590in}}%
\pgfpathcurveto{\pgfqpoint{0.635833in}{0.166667in}}{\pgfqpoint{0.635833in}{0.152744in}}{\pgfqpoint{0.630302in}{0.139389in}}%
\pgfpathcurveto{\pgfqpoint{0.620456in}{0.129544in}}{\pgfqpoint{0.610611in}{0.119698in}}{\pgfqpoint{0.597256in}{0.114167in}}%
\pgfpathclose%
\pgfpathmoveto{\pgfqpoint{0.750000in}{0.108333in}}%
\pgfpathcurveto{\pgfqpoint{0.765470in}{0.108333in}}{\pgfqpoint{0.780309in}{0.114480in}}{\pgfqpoint{0.791248in}{0.125419in}}%
\pgfpathcurveto{\pgfqpoint{0.802187in}{0.136358in}}{\pgfqpoint{0.808333in}{0.151196in}}{\pgfqpoint{0.808333in}{0.166667in}}%
\pgfpathcurveto{\pgfqpoint{0.808333in}{0.182137in}}{\pgfqpoint{0.802187in}{0.196975in}}{\pgfqpoint{0.791248in}{0.207915in}}%
\pgfpathcurveto{\pgfqpoint{0.780309in}{0.218854in}}{\pgfqpoint{0.765470in}{0.225000in}}{\pgfqpoint{0.750000in}{0.225000in}}%
\pgfpathcurveto{\pgfqpoint{0.734530in}{0.225000in}}{\pgfqpoint{0.719691in}{0.218854in}}{\pgfqpoint{0.708752in}{0.207915in}}%
\pgfpathcurveto{\pgfqpoint{0.697813in}{0.196975in}}{\pgfqpoint{0.691667in}{0.182137in}}{\pgfqpoint{0.691667in}{0.166667in}}%
\pgfpathcurveto{\pgfqpoint{0.691667in}{0.151196in}}{\pgfqpoint{0.697813in}{0.136358in}}{\pgfqpoint{0.708752in}{0.125419in}}%
\pgfpathcurveto{\pgfqpoint{0.719691in}{0.114480in}}{\pgfqpoint{0.734530in}{0.108333in}}{\pgfqpoint{0.750000in}{0.108333in}}%
\pgfpathclose%
\pgfpathmoveto{\pgfqpoint{0.750000in}{0.114167in}}%
\pgfpathcurveto{\pgfqpoint{0.750000in}{0.114167in}}{\pgfqpoint{0.736077in}{0.114167in}}{\pgfqpoint{0.722722in}{0.119698in}}%
\pgfpathcurveto{\pgfqpoint{0.712877in}{0.129544in}}{\pgfqpoint{0.703032in}{0.139389in}}{\pgfqpoint{0.697500in}{0.152744in}}%
\pgfpathcurveto{\pgfqpoint{0.697500in}{0.166667in}}{\pgfqpoint{0.697500in}{0.180590in}}{\pgfqpoint{0.703032in}{0.193945in}}%
\pgfpathcurveto{\pgfqpoint{0.712877in}{0.203790in}}{\pgfqpoint{0.722722in}{0.213635in}}{\pgfqpoint{0.736077in}{0.219167in}}%
\pgfpathcurveto{\pgfqpoint{0.750000in}{0.219167in}}{\pgfqpoint{0.763923in}{0.219167in}}{\pgfqpoint{0.777278in}{0.213635in}}%
\pgfpathcurveto{\pgfqpoint{0.787123in}{0.203790in}}{\pgfqpoint{0.796968in}{0.193945in}}{\pgfqpoint{0.802500in}{0.180590in}}%
\pgfpathcurveto{\pgfqpoint{0.802500in}{0.166667in}}{\pgfqpoint{0.802500in}{0.152744in}}{\pgfqpoint{0.796968in}{0.139389in}}%
\pgfpathcurveto{\pgfqpoint{0.787123in}{0.129544in}}{\pgfqpoint{0.777278in}{0.119698in}}{\pgfqpoint{0.763923in}{0.114167in}}%
\pgfpathclose%
\pgfpathmoveto{\pgfqpoint{0.916667in}{0.108333in}}%
\pgfpathcurveto{\pgfqpoint{0.932137in}{0.108333in}}{\pgfqpoint{0.946975in}{0.114480in}}{\pgfqpoint{0.957915in}{0.125419in}}%
\pgfpathcurveto{\pgfqpoint{0.968854in}{0.136358in}}{\pgfqpoint{0.975000in}{0.151196in}}{\pgfqpoint{0.975000in}{0.166667in}}%
\pgfpathcurveto{\pgfqpoint{0.975000in}{0.182137in}}{\pgfqpoint{0.968854in}{0.196975in}}{\pgfqpoint{0.957915in}{0.207915in}}%
\pgfpathcurveto{\pgfqpoint{0.946975in}{0.218854in}}{\pgfqpoint{0.932137in}{0.225000in}}{\pgfqpoint{0.916667in}{0.225000in}}%
\pgfpathcurveto{\pgfqpoint{0.901196in}{0.225000in}}{\pgfqpoint{0.886358in}{0.218854in}}{\pgfqpoint{0.875419in}{0.207915in}}%
\pgfpathcurveto{\pgfqpoint{0.864480in}{0.196975in}}{\pgfqpoint{0.858333in}{0.182137in}}{\pgfqpoint{0.858333in}{0.166667in}}%
\pgfpathcurveto{\pgfqpoint{0.858333in}{0.151196in}}{\pgfqpoint{0.864480in}{0.136358in}}{\pgfqpoint{0.875419in}{0.125419in}}%
\pgfpathcurveto{\pgfqpoint{0.886358in}{0.114480in}}{\pgfqpoint{0.901196in}{0.108333in}}{\pgfqpoint{0.916667in}{0.108333in}}%
\pgfpathclose%
\pgfpathmoveto{\pgfqpoint{0.916667in}{0.114167in}}%
\pgfpathcurveto{\pgfqpoint{0.916667in}{0.114167in}}{\pgfqpoint{0.902744in}{0.114167in}}{\pgfqpoint{0.889389in}{0.119698in}}%
\pgfpathcurveto{\pgfqpoint{0.879544in}{0.129544in}}{\pgfqpoint{0.869698in}{0.139389in}}{\pgfqpoint{0.864167in}{0.152744in}}%
\pgfpathcurveto{\pgfqpoint{0.864167in}{0.166667in}}{\pgfqpoint{0.864167in}{0.180590in}}{\pgfqpoint{0.869698in}{0.193945in}}%
\pgfpathcurveto{\pgfqpoint{0.879544in}{0.203790in}}{\pgfqpoint{0.889389in}{0.213635in}}{\pgfqpoint{0.902744in}{0.219167in}}%
\pgfpathcurveto{\pgfqpoint{0.916667in}{0.219167in}}{\pgfqpoint{0.930590in}{0.219167in}}{\pgfqpoint{0.943945in}{0.213635in}}%
\pgfpathcurveto{\pgfqpoint{0.953790in}{0.203790in}}{\pgfqpoint{0.963635in}{0.193945in}}{\pgfqpoint{0.969167in}{0.180590in}}%
\pgfpathcurveto{\pgfqpoint{0.969167in}{0.166667in}}{\pgfqpoint{0.969167in}{0.152744in}}{\pgfqpoint{0.963635in}{0.139389in}}%
\pgfpathcurveto{\pgfqpoint{0.953790in}{0.129544in}}{\pgfqpoint{0.943945in}{0.119698in}}{\pgfqpoint{0.930590in}{0.114167in}}%
\pgfpathclose%
\pgfpathmoveto{\pgfqpoint{0.000000in}{0.275000in}}%
\pgfpathcurveto{\pgfqpoint{0.015470in}{0.275000in}}{\pgfqpoint{0.030309in}{0.281146in}}{\pgfqpoint{0.041248in}{0.292085in}}%
\pgfpathcurveto{\pgfqpoint{0.052187in}{0.303025in}}{\pgfqpoint{0.058333in}{0.317863in}}{\pgfqpoint{0.058333in}{0.333333in}}%
\pgfpathcurveto{\pgfqpoint{0.058333in}{0.348804in}}{\pgfqpoint{0.052187in}{0.363642in}}{\pgfqpoint{0.041248in}{0.374581in}}%
\pgfpathcurveto{\pgfqpoint{0.030309in}{0.385520in}}{\pgfqpoint{0.015470in}{0.391667in}}{\pgfqpoint{0.000000in}{0.391667in}}%
\pgfpathcurveto{\pgfqpoint{-0.015470in}{0.391667in}}{\pgfqpoint{-0.030309in}{0.385520in}}{\pgfqpoint{-0.041248in}{0.374581in}}%
\pgfpathcurveto{\pgfqpoint{-0.052187in}{0.363642in}}{\pgfqpoint{-0.058333in}{0.348804in}}{\pgfqpoint{-0.058333in}{0.333333in}}%
\pgfpathcurveto{\pgfqpoint{-0.058333in}{0.317863in}}{\pgfqpoint{-0.052187in}{0.303025in}}{\pgfqpoint{-0.041248in}{0.292085in}}%
\pgfpathcurveto{\pgfqpoint{-0.030309in}{0.281146in}}{\pgfqpoint{-0.015470in}{0.275000in}}{\pgfqpoint{0.000000in}{0.275000in}}%
\pgfpathclose%
\pgfpathmoveto{\pgfqpoint{0.000000in}{0.280833in}}%
\pgfpathcurveto{\pgfqpoint{0.000000in}{0.280833in}}{\pgfqpoint{-0.013923in}{0.280833in}}{\pgfqpoint{-0.027278in}{0.286365in}}%
\pgfpathcurveto{\pgfqpoint{-0.037123in}{0.296210in}}{\pgfqpoint{-0.046968in}{0.306055in}}{\pgfqpoint{-0.052500in}{0.319410in}}%
\pgfpathcurveto{\pgfqpoint{-0.052500in}{0.333333in}}{\pgfqpoint{-0.052500in}{0.347256in}}{\pgfqpoint{-0.046968in}{0.360611in}}%
\pgfpathcurveto{\pgfqpoint{-0.037123in}{0.370456in}}{\pgfqpoint{-0.027278in}{0.380302in}}{\pgfqpoint{-0.013923in}{0.385833in}}%
\pgfpathcurveto{\pgfqpoint{0.000000in}{0.385833in}}{\pgfqpoint{0.013923in}{0.385833in}}{\pgfqpoint{0.027278in}{0.380302in}}%
\pgfpathcurveto{\pgfqpoint{0.037123in}{0.370456in}}{\pgfqpoint{0.046968in}{0.360611in}}{\pgfqpoint{0.052500in}{0.347256in}}%
\pgfpathcurveto{\pgfqpoint{0.052500in}{0.333333in}}{\pgfqpoint{0.052500in}{0.319410in}}{\pgfqpoint{0.046968in}{0.306055in}}%
\pgfpathcurveto{\pgfqpoint{0.037123in}{0.296210in}}{\pgfqpoint{0.027278in}{0.286365in}}{\pgfqpoint{0.013923in}{0.280833in}}%
\pgfpathclose%
\pgfpathmoveto{\pgfqpoint{0.166667in}{0.275000in}}%
\pgfpathcurveto{\pgfqpoint{0.182137in}{0.275000in}}{\pgfqpoint{0.196975in}{0.281146in}}{\pgfqpoint{0.207915in}{0.292085in}}%
\pgfpathcurveto{\pgfqpoint{0.218854in}{0.303025in}}{\pgfqpoint{0.225000in}{0.317863in}}{\pgfqpoint{0.225000in}{0.333333in}}%
\pgfpathcurveto{\pgfqpoint{0.225000in}{0.348804in}}{\pgfqpoint{0.218854in}{0.363642in}}{\pgfqpoint{0.207915in}{0.374581in}}%
\pgfpathcurveto{\pgfqpoint{0.196975in}{0.385520in}}{\pgfqpoint{0.182137in}{0.391667in}}{\pgfqpoint{0.166667in}{0.391667in}}%
\pgfpathcurveto{\pgfqpoint{0.151196in}{0.391667in}}{\pgfqpoint{0.136358in}{0.385520in}}{\pgfqpoint{0.125419in}{0.374581in}}%
\pgfpathcurveto{\pgfqpoint{0.114480in}{0.363642in}}{\pgfqpoint{0.108333in}{0.348804in}}{\pgfqpoint{0.108333in}{0.333333in}}%
\pgfpathcurveto{\pgfqpoint{0.108333in}{0.317863in}}{\pgfqpoint{0.114480in}{0.303025in}}{\pgfqpoint{0.125419in}{0.292085in}}%
\pgfpathcurveto{\pgfqpoint{0.136358in}{0.281146in}}{\pgfqpoint{0.151196in}{0.275000in}}{\pgfqpoint{0.166667in}{0.275000in}}%
\pgfpathclose%
\pgfpathmoveto{\pgfqpoint{0.166667in}{0.280833in}}%
\pgfpathcurveto{\pgfqpoint{0.166667in}{0.280833in}}{\pgfqpoint{0.152744in}{0.280833in}}{\pgfqpoint{0.139389in}{0.286365in}}%
\pgfpathcurveto{\pgfqpoint{0.129544in}{0.296210in}}{\pgfqpoint{0.119698in}{0.306055in}}{\pgfqpoint{0.114167in}{0.319410in}}%
\pgfpathcurveto{\pgfqpoint{0.114167in}{0.333333in}}{\pgfqpoint{0.114167in}{0.347256in}}{\pgfqpoint{0.119698in}{0.360611in}}%
\pgfpathcurveto{\pgfqpoint{0.129544in}{0.370456in}}{\pgfqpoint{0.139389in}{0.380302in}}{\pgfqpoint{0.152744in}{0.385833in}}%
\pgfpathcurveto{\pgfqpoint{0.166667in}{0.385833in}}{\pgfqpoint{0.180590in}{0.385833in}}{\pgfqpoint{0.193945in}{0.380302in}}%
\pgfpathcurveto{\pgfqpoint{0.203790in}{0.370456in}}{\pgfqpoint{0.213635in}{0.360611in}}{\pgfqpoint{0.219167in}{0.347256in}}%
\pgfpathcurveto{\pgfqpoint{0.219167in}{0.333333in}}{\pgfqpoint{0.219167in}{0.319410in}}{\pgfqpoint{0.213635in}{0.306055in}}%
\pgfpathcurveto{\pgfqpoint{0.203790in}{0.296210in}}{\pgfqpoint{0.193945in}{0.286365in}}{\pgfqpoint{0.180590in}{0.280833in}}%
\pgfpathclose%
\pgfpathmoveto{\pgfqpoint{0.333333in}{0.275000in}}%
\pgfpathcurveto{\pgfqpoint{0.348804in}{0.275000in}}{\pgfqpoint{0.363642in}{0.281146in}}{\pgfqpoint{0.374581in}{0.292085in}}%
\pgfpathcurveto{\pgfqpoint{0.385520in}{0.303025in}}{\pgfqpoint{0.391667in}{0.317863in}}{\pgfqpoint{0.391667in}{0.333333in}}%
\pgfpathcurveto{\pgfqpoint{0.391667in}{0.348804in}}{\pgfqpoint{0.385520in}{0.363642in}}{\pgfqpoint{0.374581in}{0.374581in}}%
\pgfpathcurveto{\pgfqpoint{0.363642in}{0.385520in}}{\pgfqpoint{0.348804in}{0.391667in}}{\pgfqpoint{0.333333in}{0.391667in}}%
\pgfpathcurveto{\pgfqpoint{0.317863in}{0.391667in}}{\pgfqpoint{0.303025in}{0.385520in}}{\pgfqpoint{0.292085in}{0.374581in}}%
\pgfpathcurveto{\pgfqpoint{0.281146in}{0.363642in}}{\pgfqpoint{0.275000in}{0.348804in}}{\pgfqpoint{0.275000in}{0.333333in}}%
\pgfpathcurveto{\pgfqpoint{0.275000in}{0.317863in}}{\pgfqpoint{0.281146in}{0.303025in}}{\pgfqpoint{0.292085in}{0.292085in}}%
\pgfpathcurveto{\pgfqpoint{0.303025in}{0.281146in}}{\pgfqpoint{0.317863in}{0.275000in}}{\pgfqpoint{0.333333in}{0.275000in}}%
\pgfpathclose%
\pgfpathmoveto{\pgfqpoint{0.333333in}{0.280833in}}%
\pgfpathcurveto{\pgfqpoint{0.333333in}{0.280833in}}{\pgfqpoint{0.319410in}{0.280833in}}{\pgfqpoint{0.306055in}{0.286365in}}%
\pgfpathcurveto{\pgfqpoint{0.296210in}{0.296210in}}{\pgfqpoint{0.286365in}{0.306055in}}{\pgfqpoint{0.280833in}{0.319410in}}%
\pgfpathcurveto{\pgfqpoint{0.280833in}{0.333333in}}{\pgfqpoint{0.280833in}{0.347256in}}{\pgfqpoint{0.286365in}{0.360611in}}%
\pgfpathcurveto{\pgfqpoint{0.296210in}{0.370456in}}{\pgfqpoint{0.306055in}{0.380302in}}{\pgfqpoint{0.319410in}{0.385833in}}%
\pgfpathcurveto{\pgfqpoint{0.333333in}{0.385833in}}{\pgfqpoint{0.347256in}{0.385833in}}{\pgfqpoint{0.360611in}{0.380302in}}%
\pgfpathcurveto{\pgfqpoint{0.370456in}{0.370456in}}{\pgfqpoint{0.380302in}{0.360611in}}{\pgfqpoint{0.385833in}{0.347256in}}%
\pgfpathcurveto{\pgfqpoint{0.385833in}{0.333333in}}{\pgfqpoint{0.385833in}{0.319410in}}{\pgfqpoint{0.380302in}{0.306055in}}%
\pgfpathcurveto{\pgfqpoint{0.370456in}{0.296210in}}{\pgfqpoint{0.360611in}{0.286365in}}{\pgfqpoint{0.347256in}{0.280833in}}%
\pgfpathclose%
\pgfpathmoveto{\pgfqpoint{0.500000in}{0.275000in}}%
\pgfpathcurveto{\pgfqpoint{0.515470in}{0.275000in}}{\pgfqpoint{0.530309in}{0.281146in}}{\pgfqpoint{0.541248in}{0.292085in}}%
\pgfpathcurveto{\pgfqpoint{0.552187in}{0.303025in}}{\pgfqpoint{0.558333in}{0.317863in}}{\pgfqpoint{0.558333in}{0.333333in}}%
\pgfpathcurveto{\pgfqpoint{0.558333in}{0.348804in}}{\pgfqpoint{0.552187in}{0.363642in}}{\pgfqpoint{0.541248in}{0.374581in}}%
\pgfpathcurveto{\pgfqpoint{0.530309in}{0.385520in}}{\pgfqpoint{0.515470in}{0.391667in}}{\pgfqpoint{0.500000in}{0.391667in}}%
\pgfpathcurveto{\pgfqpoint{0.484530in}{0.391667in}}{\pgfqpoint{0.469691in}{0.385520in}}{\pgfqpoint{0.458752in}{0.374581in}}%
\pgfpathcurveto{\pgfqpoint{0.447813in}{0.363642in}}{\pgfqpoint{0.441667in}{0.348804in}}{\pgfqpoint{0.441667in}{0.333333in}}%
\pgfpathcurveto{\pgfqpoint{0.441667in}{0.317863in}}{\pgfqpoint{0.447813in}{0.303025in}}{\pgfqpoint{0.458752in}{0.292085in}}%
\pgfpathcurveto{\pgfqpoint{0.469691in}{0.281146in}}{\pgfqpoint{0.484530in}{0.275000in}}{\pgfqpoint{0.500000in}{0.275000in}}%
\pgfpathclose%
\pgfpathmoveto{\pgfqpoint{0.500000in}{0.280833in}}%
\pgfpathcurveto{\pgfqpoint{0.500000in}{0.280833in}}{\pgfqpoint{0.486077in}{0.280833in}}{\pgfqpoint{0.472722in}{0.286365in}}%
\pgfpathcurveto{\pgfqpoint{0.462877in}{0.296210in}}{\pgfqpoint{0.453032in}{0.306055in}}{\pgfqpoint{0.447500in}{0.319410in}}%
\pgfpathcurveto{\pgfqpoint{0.447500in}{0.333333in}}{\pgfqpoint{0.447500in}{0.347256in}}{\pgfqpoint{0.453032in}{0.360611in}}%
\pgfpathcurveto{\pgfqpoint{0.462877in}{0.370456in}}{\pgfqpoint{0.472722in}{0.380302in}}{\pgfqpoint{0.486077in}{0.385833in}}%
\pgfpathcurveto{\pgfqpoint{0.500000in}{0.385833in}}{\pgfqpoint{0.513923in}{0.385833in}}{\pgfqpoint{0.527278in}{0.380302in}}%
\pgfpathcurveto{\pgfqpoint{0.537123in}{0.370456in}}{\pgfqpoint{0.546968in}{0.360611in}}{\pgfqpoint{0.552500in}{0.347256in}}%
\pgfpathcurveto{\pgfqpoint{0.552500in}{0.333333in}}{\pgfqpoint{0.552500in}{0.319410in}}{\pgfqpoint{0.546968in}{0.306055in}}%
\pgfpathcurveto{\pgfqpoint{0.537123in}{0.296210in}}{\pgfqpoint{0.527278in}{0.286365in}}{\pgfqpoint{0.513923in}{0.280833in}}%
\pgfpathclose%
\pgfpathmoveto{\pgfqpoint{0.666667in}{0.275000in}}%
\pgfpathcurveto{\pgfqpoint{0.682137in}{0.275000in}}{\pgfqpoint{0.696975in}{0.281146in}}{\pgfqpoint{0.707915in}{0.292085in}}%
\pgfpathcurveto{\pgfqpoint{0.718854in}{0.303025in}}{\pgfqpoint{0.725000in}{0.317863in}}{\pgfqpoint{0.725000in}{0.333333in}}%
\pgfpathcurveto{\pgfqpoint{0.725000in}{0.348804in}}{\pgfqpoint{0.718854in}{0.363642in}}{\pgfqpoint{0.707915in}{0.374581in}}%
\pgfpathcurveto{\pgfqpoint{0.696975in}{0.385520in}}{\pgfqpoint{0.682137in}{0.391667in}}{\pgfqpoint{0.666667in}{0.391667in}}%
\pgfpathcurveto{\pgfqpoint{0.651196in}{0.391667in}}{\pgfqpoint{0.636358in}{0.385520in}}{\pgfqpoint{0.625419in}{0.374581in}}%
\pgfpathcurveto{\pgfqpoint{0.614480in}{0.363642in}}{\pgfqpoint{0.608333in}{0.348804in}}{\pgfqpoint{0.608333in}{0.333333in}}%
\pgfpathcurveto{\pgfqpoint{0.608333in}{0.317863in}}{\pgfqpoint{0.614480in}{0.303025in}}{\pgfqpoint{0.625419in}{0.292085in}}%
\pgfpathcurveto{\pgfqpoint{0.636358in}{0.281146in}}{\pgfqpoint{0.651196in}{0.275000in}}{\pgfqpoint{0.666667in}{0.275000in}}%
\pgfpathclose%
\pgfpathmoveto{\pgfqpoint{0.666667in}{0.280833in}}%
\pgfpathcurveto{\pgfqpoint{0.666667in}{0.280833in}}{\pgfqpoint{0.652744in}{0.280833in}}{\pgfqpoint{0.639389in}{0.286365in}}%
\pgfpathcurveto{\pgfqpoint{0.629544in}{0.296210in}}{\pgfqpoint{0.619698in}{0.306055in}}{\pgfqpoint{0.614167in}{0.319410in}}%
\pgfpathcurveto{\pgfqpoint{0.614167in}{0.333333in}}{\pgfqpoint{0.614167in}{0.347256in}}{\pgfqpoint{0.619698in}{0.360611in}}%
\pgfpathcurveto{\pgfqpoint{0.629544in}{0.370456in}}{\pgfqpoint{0.639389in}{0.380302in}}{\pgfqpoint{0.652744in}{0.385833in}}%
\pgfpathcurveto{\pgfqpoint{0.666667in}{0.385833in}}{\pgfqpoint{0.680590in}{0.385833in}}{\pgfqpoint{0.693945in}{0.380302in}}%
\pgfpathcurveto{\pgfqpoint{0.703790in}{0.370456in}}{\pgfqpoint{0.713635in}{0.360611in}}{\pgfqpoint{0.719167in}{0.347256in}}%
\pgfpathcurveto{\pgfqpoint{0.719167in}{0.333333in}}{\pgfqpoint{0.719167in}{0.319410in}}{\pgfqpoint{0.713635in}{0.306055in}}%
\pgfpathcurveto{\pgfqpoint{0.703790in}{0.296210in}}{\pgfqpoint{0.693945in}{0.286365in}}{\pgfqpoint{0.680590in}{0.280833in}}%
\pgfpathclose%
\pgfpathmoveto{\pgfqpoint{0.833333in}{0.275000in}}%
\pgfpathcurveto{\pgfqpoint{0.848804in}{0.275000in}}{\pgfqpoint{0.863642in}{0.281146in}}{\pgfqpoint{0.874581in}{0.292085in}}%
\pgfpathcurveto{\pgfqpoint{0.885520in}{0.303025in}}{\pgfqpoint{0.891667in}{0.317863in}}{\pgfqpoint{0.891667in}{0.333333in}}%
\pgfpathcurveto{\pgfqpoint{0.891667in}{0.348804in}}{\pgfqpoint{0.885520in}{0.363642in}}{\pgfqpoint{0.874581in}{0.374581in}}%
\pgfpathcurveto{\pgfqpoint{0.863642in}{0.385520in}}{\pgfqpoint{0.848804in}{0.391667in}}{\pgfqpoint{0.833333in}{0.391667in}}%
\pgfpathcurveto{\pgfqpoint{0.817863in}{0.391667in}}{\pgfqpoint{0.803025in}{0.385520in}}{\pgfqpoint{0.792085in}{0.374581in}}%
\pgfpathcurveto{\pgfqpoint{0.781146in}{0.363642in}}{\pgfqpoint{0.775000in}{0.348804in}}{\pgfqpoint{0.775000in}{0.333333in}}%
\pgfpathcurveto{\pgfqpoint{0.775000in}{0.317863in}}{\pgfqpoint{0.781146in}{0.303025in}}{\pgfqpoint{0.792085in}{0.292085in}}%
\pgfpathcurveto{\pgfqpoint{0.803025in}{0.281146in}}{\pgfqpoint{0.817863in}{0.275000in}}{\pgfqpoint{0.833333in}{0.275000in}}%
\pgfpathclose%
\pgfpathmoveto{\pgfqpoint{0.833333in}{0.280833in}}%
\pgfpathcurveto{\pgfqpoint{0.833333in}{0.280833in}}{\pgfqpoint{0.819410in}{0.280833in}}{\pgfqpoint{0.806055in}{0.286365in}}%
\pgfpathcurveto{\pgfqpoint{0.796210in}{0.296210in}}{\pgfqpoint{0.786365in}{0.306055in}}{\pgfqpoint{0.780833in}{0.319410in}}%
\pgfpathcurveto{\pgfqpoint{0.780833in}{0.333333in}}{\pgfqpoint{0.780833in}{0.347256in}}{\pgfqpoint{0.786365in}{0.360611in}}%
\pgfpathcurveto{\pgfqpoint{0.796210in}{0.370456in}}{\pgfqpoint{0.806055in}{0.380302in}}{\pgfqpoint{0.819410in}{0.385833in}}%
\pgfpathcurveto{\pgfqpoint{0.833333in}{0.385833in}}{\pgfqpoint{0.847256in}{0.385833in}}{\pgfqpoint{0.860611in}{0.380302in}}%
\pgfpathcurveto{\pgfqpoint{0.870456in}{0.370456in}}{\pgfqpoint{0.880302in}{0.360611in}}{\pgfqpoint{0.885833in}{0.347256in}}%
\pgfpathcurveto{\pgfqpoint{0.885833in}{0.333333in}}{\pgfqpoint{0.885833in}{0.319410in}}{\pgfqpoint{0.880302in}{0.306055in}}%
\pgfpathcurveto{\pgfqpoint{0.870456in}{0.296210in}}{\pgfqpoint{0.860611in}{0.286365in}}{\pgfqpoint{0.847256in}{0.280833in}}%
\pgfpathclose%
\pgfpathmoveto{\pgfqpoint{1.000000in}{0.275000in}}%
\pgfpathcurveto{\pgfqpoint{1.015470in}{0.275000in}}{\pgfqpoint{1.030309in}{0.281146in}}{\pgfqpoint{1.041248in}{0.292085in}}%
\pgfpathcurveto{\pgfqpoint{1.052187in}{0.303025in}}{\pgfqpoint{1.058333in}{0.317863in}}{\pgfqpoint{1.058333in}{0.333333in}}%
\pgfpathcurveto{\pgfqpoint{1.058333in}{0.348804in}}{\pgfqpoint{1.052187in}{0.363642in}}{\pgfqpoint{1.041248in}{0.374581in}}%
\pgfpathcurveto{\pgfqpoint{1.030309in}{0.385520in}}{\pgfqpoint{1.015470in}{0.391667in}}{\pgfqpoint{1.000000in}{0.391667in}}%
\pgfpathcurveto{\pgfqpoint{0.984530in}{0.391667in}}{\pgfqpoint{0.969691in}{0.385520in}}{\pgfqpoint{0.958752in}{0.374581in}}%
\pgfpathcurveto{\pgfqpoint{0.947813in}{0.363642in}}{\pgfqpoint{0.941667in}{0.348804in}}{\pgfqpoint{0.941667in}{0.333333in}}%
\pgfpathcurveto{\pgfqpoint{0.941667in}{0.317863in}}{\pgfqpoint{0.947813in}{0.303025in}}{\pgfqpoint{0.958752in}{0.292085in}}%
\pgfpathcurveto{\pgfqpoint{0.969691in}{0.281146in}}{\pgfqpoint{0.984530in}{0.275000in}}{\pgfqpoint{1.000000in}{0.275000in}}%
\pgfpathclose%
\pgfpathmoveto{\pgfqpoint{1.000000in}{0.280833in}}%
\pgfpathcurveto{\pgfqpoint{1.000000in}{0.280833in}}{\pgfqpoint{0.986077in}{0.280833in}}{\pgfqpoint{0.972722in}{0.286365in}}%
\pgfpathcurveto{\pgfqpoint{0.962877in}{0.296210in}}{\pgfqpoint{0.953032in}{0.306055in}}{\pgfqpoint{0.947500in}{0.319410in}}%
\pgfpathcurveto{\pgfqpoint{0.947500in}{0.333333in}}{\pgfqpoint{0.947500in}{0.347256in}}{\pgfqpoint{0.953032in}{0.360611in}}%
\pgfpathcurveto{\pgfqpoint{0.962877in}{0.370456in}}{\pgfqpoint{0.972722in}{0.380302in}}{\pgfqpoint{0.986077in}{0.385833in}}%
\pgfpathcurveto{\pgfqpoint{1.000000in}{0.385833in}}{\pgfqpoint{1.013923in}{0.385833in}}{\pgfqpoint{1.027278in}{0.380302in}}%
\pgfpathcurveto{\pgfqpoint{1.037123in}{0.370456in}}{\pgfqpoint{1.046968in}{0.360611in}}{\pgfqpoint{1.052500in}{0.347256in}}%
\pgfpathcurveto{\pgfqpoint{1.052500in}{0.333333in}}{\pgfqpoint{1.052500in}{0.319410in}}{\pgfqpoint{1.046968in}{0.306055in}}%
\pgfpathcurveto{\pgfqpoint{1.037123in}{0.296210in}}{\pgfqpoint{1.027278in}{0.286365in}}{\pgfqpoint{1.013923in}{0.280833in}}%
\pgfpathclose%
\pgfpathmoveto{\pgfqpoint{0.083333in}{0.441667in}}%
\pgfpathcurveto{\pgfqpoint{0.098804in}{0.441667in}}{\pgfqpoint{0.113642in}{0.447813in}}{\pgfqpoint{0.124581in}{0.458752in}}%
\pgfpathcurveto{\pgfqpoint{0.135520in}{0.469691in}}{\pgfqpoint{0.141667in}{0.484530in}}{\pgfqpoint{0.141667in}{0.500000in}}%
\pgfpathcurveto{\pgfqpoint{0.141667in}{0.515470in}}{\pgfqpoint{0.135520in}{0.530309in}}{\pgfqpoint{0.124581in}{0.541248in}}%
\pgfpathcurveto{\pgfqpoint{0.113642in}{0.552187in}}{\pgfqpoint{0.098804in}{0.558333in}}{\pgfqpoint{0.083333in}{0.558333in}}%
\pgfpathcurveto{\pgfqpoint{0.067863in}{0.558333in}}{\pgfqpoint{0.053025in}{0.552187in}}{\pgfqpoint{0.042085in}{0.541248in}}%
\pgfpathcurveto{\pgfqpoint{0.031146in}{0.530309in}}{\pgfqpoint{0.025000in}{0.515470in}}{\pgfqpoint{0.025000in}{0.500000in}}%
\pgfpathcurveto{\pgfqpoint{0.025000in}{0.484530in}}{\pgfqpoint{0.031146in}{0.469691in}}{\pgfqpoint{0.042085in}{0.458752in}}%
\pgfpathcurveto{\pgfqpoint{0.053025in}{0.447813in}}{\pgfqpoint{0.067863in}{0.441667in}}{\pgfqpoint{0.083333in}{0.441667in}}%
\pgfpathclose%
\pgfpathmoveto{\pgfqpoint{0.083333in}{0.447500in}}%
\pgfpathcurveto{\pgfqpoint{0.083333in}{0.447500in}}{\pgfqpoint{0.069410in}{0.447500in}}{\pgfqpoint{0.056055in}{0.453032in}}%
\pgfpathcurveto{\pgfqpoint{0.046210in}{0.462877in}}{\pgfqpoint{0.036365in}{0.472722in}}{\pgfqpoint{0.030833in}{0.486077in}}%
\pgfpathcurveto{\pgfqpoint{0.030833in}{0.500000in}}{\pgfqpoint{0.030833in}{0.513923in}}{\pgfqpoint{0.036365in}{0.527278in}}%
\pgfpathcurveto{\pgfqpoint{0.046210in}{0.537123in}}{\pgfqpoint{0.056055in}{0.546968in}}{\pgfqpoint{0.069410in}{0.552500in}}%
\pgfpathcurveto{\pgfqpoint{0.083333in}{0.552500in}}{\pgfqpoint{0.097256in}{0.552500in}}{\pgfqpoint{0.110611in}{0.546968in}}%
\pgfpathcurveto{\pgfqpoint{0.120456in}{0.537123in}}{\pgfqpoint{0.130302in}{0.527278in}}{\pgfqpoint{0.135833in}{0.513923in}}%
\pgfpathcurveto{\pgfqpoint{0.135833in}{0.500000in}}{\pgfqpoint{0.135833in}{0.486077in}}{\pgfqpoint{0.130302in}{0.472722in}}%
\pgfpathcurveto{\pgfqpoint{0.120456in}{0.462877in}}{\pgfqpoint{0.110611in}{0.453032in}}{\pgfqpoint{0.097256in}{0.447500in}}%
\pgfpathclose%
\pgfpathmoveto{\pgfqpoint{0.250000in}{0.441667in}}%
\pgfpathcurveto{\pgfqpoint{0.265470in}{0.441667in}}{\pgfqpoint{0.280309in}{0.447813in}}{\pgfqpoint{0.291248in}{0.458752in}}%
\pgfpathcurveto{\pgfqpoint{0.302187in}{0.469691in}}{\pgfqpoint{0.308333in}{0.484530in}}{\pgfqpoint{0.308333in}{0.500000in}}%
\pgfpathcurveto{\pgfqpoint{0.308333in}{0.515470in}}{\pgfqpoint{0.302187in}{0.530309in}}{\pgfqpoint{0.291248in}{0.541248in}}%
\pgfpathcurveto{\pgfqpoint{0.280309in}{0.552187in}}{\pgfqpoint{0.265470in}{0.558333in}}{\pgfqpoint{0.250000in}{0.558333in}}%
\pgfpathcurveto{\pgfqpoint{0.234530in}{0.558333in}}{\pgfqpoint{0.219691in}{0.552187in}}{\pgfqpoint{0.208752in}{0.541248in}}%
\pgfpathcurveto{\pgfqpoint{0.197813in}{0.530309in}}{\pgfqpoint{0.191667in}{0.515470in}}{\pgfqpoint{0.191667in}{0.500000in}}%
\pgfpathcurveto{\pgfqpoint{0.191667in}{0.484530in}}{\pgfqpoint{0.197813in}{0.469691in}}{\pgfqpoint{0.208752in}{0.458752in}}%
\pgfpathcurveto{\pgfqpoint{0.219691in}{0.447813in}}{\pgfqpoint{0.234530in}{0.441667in}}{\pgfqpoint{0.250000in}{0.441667in}}%
\pgfpathclose%
\pgfpathmoveto{\pgfqpoint{0.250000in}{0.447500in}}%
\pgfpathcurveto{\pgfqpoint{0.250000in}{0.447500in}}{\pgfqpoint{0.236077in}{0.447500in}}{\pgfqpoint{0.222722in}{0.453032in}}%
\pgfpathcurveto{\pgfqpoint{0.212877in}{0.462877in}}{\pgfqpoint{0.203032in}{0.472722in}}{\pgfqpoint{0.197500in}{0.486077in}}%
\pgfpathcurveto{\pgfqpoint{0.197500in}{0.500000in}}{\pgfqpoint{0.197500in}{0.513923in}}{\pgfqpoint{0.203032in}{0.527278in}}%
\pgfpathcurveto{\pgfqpoint{0.212877in}{0.537123in}}{\pgfqpoint{0.222722in}{0.546968in}}{\pgfqpoint{0.236077in}{0.552500in}}%
\pgfpathcurveto{\pgfqpoint{0.250000in}{0.552500in}}{\pgfqpoint{0.263923in}{0.552500in}}{\pgfqpoint{0.277278in}{0.546968in}}%
\pgfpathcurveto{\pgfqpoint{0.287123in}{0.537123in}}{\pgfqpoint{0.296968in}{0.527278in}}{\pgfqpoint{0.302500in}{0.513923in}}%
\pgfpathcurveto{\pgfqpoint{0.302500in}{0.500000in}}{\pgfqpoint{0.302500in}{0.486077in}}{\pgfqpoint{0.296968in}{0.472722in}}%
\pgfpathcurveto{\pgfqpoint{0.287123in}{0.462877in}}{\pgfqpoint{0.277278in}{0.453032in}}{\pgfqpoint{0.263923in}{0.447500in}}%
\pgfpathclose%
\pgfpathmoveto{\pgfqpoint{0.416667in}{0.441667in}}%
\pgfpathcurveto{\pgfqpoint{0.432137in}{0.441667in}}{\pgfqpoint{0.446975in}{0.447813in}}{\pgfqpoint{0.457915in}{0.458752in}}%
\pgfpathcurveto{\pgfqpoint{0.468854in}{0.469691in}}{\pgfqpoint{0.475000in}{0.484530in}}{\pgfqpoint{0.475000in}{0.500000in}}%
\pgfpathcurveto{\pgfqpoint{0.475000in}{0.515470in}}{\pgfqpoint{0.468854in}{0.530309in}}{\pgfqpoint{0.457915in}{0.541248in}}%
\pgfpathcurveto{\pgfqpoint{0.446975in}{0.552187in}}{\pgfqpoint{0.432137in}{0.558333in}}{\pgfqpoint{0.416667in}{0.558333in}}%
\pgfpathcurveto{\pgfqpoint{0.401196in}{0.558333in}}{\pgfqpoint{0.386358in}{0.552187in}}{\pgfqpoint{0.375419in}{0.541248in}}%
\pgfpathcurveto{\pgfqpoint{0.364480in}{0.530309in}}{\pgfqpoint{0.358333in}{0.515470in}}{\pgfqpoint{0.358333in}{0.500000in}}%
\pgfpathcurveto{\pgfqpoint{0.358333in}{0.484530in}}{\pgfqpoint{0.364480in}{0.469691in}}{\pgfqpoint{0.375419in}{0.458752in}}%
\pgfpathcurveto{\pgfqpoint{0.386358in}{0.447813in}}{\pgfqpoint{0.401196in}{0.441667in}}{\pgfqpoint{0.416667in}{0.441667in}}%
\pgfpathclose%
\pgfpathmoveto{\pgfqpoint{0.416667in}{0.447500in}}%
\pgfpathcurveto{\pgfqpoint{0.416667in}{0.447500in}}{\pgfqpoint{0.402744in}{0.447500in}}{\pgfqpoint{0.389389in}{0.453032in}}%
\pgfpathcurveto{\pgfqpoint{0.379544in}{0.462877in}}{\pgfqpoint{0.369698in}{0.472722in}}{\pgfqpoint{0.364167in}{0.486077in}}%
\pgfpathcurveto{\pgfqpoint{0.364167in}{0.500000in}}{\pgfqpoint{0.364167in}{0.513923in}}{\pgfqpoint{0.369698in}{0.527278in}}%
\pgfpathcurveto{\pgfqpoint{0.379544in}{0.537123in}}{\pgfqpoint{0.389389in}{0.546968in}}{\pgfqpoint{0.402744in}{0.552500in}}%
\pgfpathcurveto{\pgfqpoint{0.416667in}{0.552500in}}{\pgfqpoint{0.430590in}{0.552500in}}{\pgfqpoint{0.443945in}{0.546968in}}%
\pgfpathcurveto{\pgfqpoint{0.453790in}{0.537123in}}{\pgfqpoint{0.463635in}{0.527278in}}{\pgfqpoint{0.469167in}{0.513923in}}%
\pgfpathcurveto{\pgfqpoint{0.469167in}{0.500000in}}{\pgfqpoint{0.469167in}{0.486077in}}{\pgfqpoint{0.463635in}{0.472722in}}%
\pgfpathcurveto{\pgfqpoint{0.453790in}{0.462877in}}{\pgfqpoint{0.443945in}{0.453032in}}{\pgfqpoint{0.430590in}{0.447500in}}%
\pgfpathclose%
\pgfpathmoveto{\pgfqpoint{0.583333in}{0.441667in}}%
\pgfpathcurveto{\pgfqpoint{0.598804in}{0.441667in}}{\pgfqpoint{0.613642in}{0.447813in}}{\pgfqpoint{0.624581in}{0.458752in}}%
\pgfpathcurveto{\pgfqpoint{0.635520in}{0.469691in}}{\pgfqpoint{0.641667in}{0.484530in}}{\pgfqpoint{0.641667in}{0.500000in}}%
\pgfpathcurveto{\pgfqpoint{0.641667in}{0.515470in}}{\pgfqpoint{0.635520in}{0.530309in}}{\pgfqpoint{0.624581in}{0.541248in}}%
\pgfpathcurveto{\pgfqpoint{0.613642in}{0.552187in}}{\pgfqpoint{0.598804in}{0.558333in}}{\pgfqpoint{0.583333in}{0.558333in}}%
\pgfpathcurveto{\pgfqpoint{0.567863in}{0.558333in}}{\pgfqpoint{0.553025in}{0.552187in}}{\pgfqpoint{0.542085in}{0.541248in}}%
\pgfpathcurveto{\pgfqpoint{0.531146in}{0.530309in}}{\pgfqpoint{0.525000in}{0.515470in}}{\pgfqpoint{0.525000in}{0.500000in}}%
\pgfpathcurveto{\pgfqpoint{0.525000in}{0.484530in}}{\pgfqpoint{0.531146in}{0.469691in}}{\pgfqpoint{0.542085in}{0.458752in}}%
\pgfpathcurveto{\pgfqpoint{0.553025in}{0.447813in}}{\pgfqpoint{0.567863in}{0.441667in}}{\pgfqpoint{0.583333in}{0.441667in}}%
\pgfpathclose%
\pgfpathmoveto{\pgfqpoint{0.583333in}{0.447500in}}%
\pgfpathcurveto{\pgfqpoint{0.583333in}{0.447500in}}{\pgfqpoint{0.569410in}{0.447500in}}{\pgfqpoint{0.556055in}{0.453032in}}%
\pgfpathcurveto{\pgfqpoint{0.546210in}{0.462877in}}{\pgfqpoint{0.536365in}{0.472722in}}{\pgfqpoint{0.530833in}{0.486077in}}%
\pgfpathcurveto{\pgfqpoint{0.530833in}{0.500000in}}{\pgfqpoint{0.530833in}{0.513923in}}{\pgfqpoint{0.536365in}{0.527278in}}%
\pgfpathcurveto{\pgfqpoint{0.546210in}{0.537123in}}{\pgfqpoint{0.556055in}{0.546968in}}{\pgfqpoint{0.569410in}{0.552500in}}%
\pgfpathcurveto{\pgfqpoint{0.583333in}{0.552500in}}{\pgfqpoint{0.597256in}{0.552500in}}{\pgfqpoint{0.610611in}{0.546968in}}%
\pgfpathcurveto{\pgfqpoint{0.620456in}{0.537123in}}{\pgfqpoint{0.630302in}{0.527278in}}{\pgfqpoint{0.635833in}{0.513923in}}%
\pgfpathcurveto{\pgfqpoint{0.635833in}{0.500000in}}{\pgfqpoint{0.635833in}{0.486077in}}{\pgfqpoint{0.630302in}{0.472722in}}%
\pgfpathcurveto{\pgfqpoint{0.620456in}{0.462877in}}{\pgfqpoint{0.610611in}{0.453032in}}{\pgfqpoint{0.597256in}{0.447500in}}%
\pgfpathclose%
\pgfpathmoveto{\pgfqpoint{0.750000in}{0.441667in}}%
\pgfpathcurveto{\pgfqpoint{0.765470in}{0.441667in}}{\pgfqpoint{0.780309in}{0.447813in}}{\pgfqpoint{0.791248in}{0.458752in}}%
\pgfpathcurveto{\pgfqpoint{0.802187in}{0.469691in}}{\pgfqpoint{0.808333in}{0.484530in}}{\pgfqpoint{0.808333in}{0.500000in}}%
\pgfpathcurveto{\pgfqpoint{0.808333in}{0.515470in}}{\pgfqpoint{0.802187in}{0.530309in}}{\pgfqpoint{0.791248in}{0.541248in}}%
\pgfpathcurveto{\pgfqpoint{0.780309in}{0.552187in}}{\pgfqpoint{0.765470in}{0.558333in}}{\pgfqpoint{0.750000in}{0.558333in}}%
\pgfpathcurveto{\pgfqpoint{0.734530in}{0.558333in}}{\pgfqpoint{0.719691in}{0.552187in}}{\pgfqpoint{0.708752in}{0.541248in}}%
\pgfpathcurveto{\pgfqpoint{0.697813in}{0.530309in}}{\pgfqpoint{0.691667in}{0.515470in}}{\pgfqpoint{0.691667in}{0.500000in}}%
\pgfpathcurveto{\pgfqpoint{0.691667in}{0.484530in}}{\pgfqpoint{0.697813in}{0.469691in}}{\pgfqpoint{0.708752in}{0.458752in}}%
\pgfpathcurveto{\pgfqpoint{0.719691in}{0.447813in}}{\pgfqpoint{0.734530in}{0.441667in}}{\pgfqpoint{0.750000in}{0.441667in}}%
\pgfpathclose%
\pgfpathmoveto{\pgfqpoint{0.750000in}{0.447500in}}%
\pgfpathcurveto{\pgfqpoint{0.750000in}{0.447500in}}{\pgfqpoint{0.736077in}{0.447500in}}{\pgfqpoint{0.722722in}{0.453032in}}%
\pgfpathcurveto{\pgfqpoint{0.712877in}{0.462877in}}{\pgfqpoint{0.703032in}{0.472722in}}{\pgfqpoint{0.697500in}{0.486077in}}%
\pgfpathcurveto{\pgfqpoint{0.697500in}{0.500000in}}{\pgfqpoint{0.697500in}{0.513923in}}{\pgfqpoint{0.703032in}{0.527278in}}%
\pgfpathcurveto{\pgfqpoint{0.712877in}{0.537123in}}{\pgfqpoint{0.722722in}{0.546968in}}{\pgfqpoint{0.736077in}{0.552500in}}%
\pgfpathcurveto{\pgfqpoint{0.750000in}{0.552500in}}{\pgfqpoint{0.763923in}{0.552500in}}{\pgfqpoint{0.777278in}{0.546968in}}%
\pgfpathcurveto{\pgfqpoint{0.787123in}{0.537123in}}{\pgfqpoint{0.796968in}{0.527278in}}{\pgfqpoint{0.802500in}{0.513923in}}%
\pgfpathcurveto{\pgfqpoint{0.802500in}{0.500000in}}{\pgfqpoint{0.802500in}{0.486077in}}{\pgfqpoint{0.796968in}{0.472722in}}%
\pgfpathcurveto{\pgfqpoint{0.787123in}{0.462877in}}{\pgfqpoint{0.777278in}{0.453032in}}{\pgfqpoint{0.763923in}{0.447500in}}%
\pgfpathclose%
\pgfpathmoveto{\pgfqpoint{0.916667in}{0.441667in}}%
\pgfpathcurveto{\pgfqpoint{0.932137in}{0.441667in}}{\pgfqpoint{0.946975in}{0.447813in}}{\pgfqpoint{0.957915in}{0.458752in}}%
\pgfpathcurveto{\pgfqpoint{0.968854in}{0.469691in}}{\pgfqpoint{0.975000in}{0.484530in}}{\pgfqpoint{0.975000in}{0.500000in}}%
\pgfpathcurveto{\pgfqpoint{0.975000in}{0.515470in}}{\pgfqpoint{0.968854in}{0.530309in}}{\pgfqpoint{0.957915in}{0.541248in}}%
\pgfpathcurveto{\pgfqpoint{0.946975in}{0.552187in}}{\pgfqpoint{0.932137in}{0.558333in}}{\pgfqpoint{0.916667in}{0.558333in}}%
\pgfpathcurveto{\pgfqpoint{0.901196in}{0.558333in}}{\pgfqpoint{0.886358in}{0.552187in}}{\pgfqpoint{0.875419in}{0.541248in}}%
\pgfpathcurveto{\pgfqpoint{0.864480in}{0.530309in}}{\pgfqpoint{0.858333in}{0.515470in}}{\pgfqpoint{0.858333in}{0.500000in}}%
\pgfpathcurveto{\pgfqpoint{0.858333in}{0.484530in}}{\pgfqpoint{0.864480in}{0.469691in}}{\pgfqpoint{0.875419in}{0.458752in}}%
\pgfpathcurveto{\pgfqpoint{0.886358in}{0.447813in}}{\pgfqpoint{0.901196in}{0.441667in}}{\pgfqpoint{0.916667in}{0.441667in}}%
\pgfpathclose%
\pgfpathmoveto{\pgfqpoint{0.916667in}{0.447500in}}%
\pgfpathcurveto{\pgfqpoint{0.916667in}{0.447500in}}{\pgfqpoint{0.902744in}{0.447500in}}{\pgfqpoint{0.889389in}{0.453032in}}%
\pgfpathcurveto{\pgfqpoint{0.879544in}{0.462877in}}{\pgfqpoint{0.869698in}{0.472722in}}{\pgfqpoint{0.864167in}{0.486077in}}%
\pgfpathcurveto{\pgfqpoint{0.864167in}{0.500000in}}{\pgfqpoint{0.864167in}{0.513923in}}{\pgfqpoint{0.869698in}{0.527278in}}%
\pgfpathcurveto{\pgfqpoint{0.879544in}{0.537123in}}{\pgfqpoint{0.889389in}{0.546968in}}{\pgfqpoint{0.902744in}{0.552500in}}%
\pgfpathcurveto{\pgfqpoint{0.916667in}{0.552500in}}{\pgfqpoint{0.930590in}{0.552500in}}{\pgfqpoint{0.943945in}{0.546968in}}%
\pgfpathcurveto{\pgfqpoint{0.953790in}{0.537123in}}{\pgfqpoint{0.963635in}{0.527278in}}{\pgfqpoint{0.969167in}{0.513923in}}%
\pgfpathcurveto{\pgfqpoint{0.969167in}{0.500000in}}{\pgfqpoint{0.969167in}{0.486077in}}{\pgfqpoint{0.963635in}{0.472722in}}%
\pgfpathcurveto{\pgfqpoint{0.953790in}{0.462877in}}{\pgfqpoint{0.943945in}{0.453032in}}{\pgfqpoint{0.930590in}{0.447500in}}%
\pgfpathclose%
\pgfpathmoveto{\pgfqpoint{0.000000in}{0.608333in}}%
\pgfpathcurveto{\pgfqpoint{0.015470in}{0.608333in}}{\pgfqpoint{0.030309in}{0.614480in}}{\pgfqpoint{0.041248in}{0.625419in}}%
\pgfpathcurveto{\pgfqpoint{0.052187in}{0.636358in}}{\pgfqpoint{0.058333in}{0.651196in}}{\pgfqpoint{0.058333in}{0.666667in}}%
\pgfpathcurveto{\pgfqpoint{0.058333in}{0.682137in}}{\pgfqpoint{0.052187in}{0.696975in}}{\pgfqpoint{0.041248in}{0.707915in}}%
\pgfpathcurveto{\pgfqpoint{0.030309in}{0.718854in}}{\pgfqpoint{0.015470in}{0.725000in}}{\pgfqpoint{0.000000in}{0.725000in}}%
\pgfpathcurveto{\pgfqpoint{-0.015470in}{0.725000in}}{\pgfqpoint{-0.030309in}{0.718854in}}{\pgfqpoint{-0.041248in}{0.707915in}}%
\pgfpathcurveto{\pgfqpoint{-0.052187in}{0.696975in}}{\pgfqpoint{-0.058333in}{0.682137in}}{\pgfqpoint{-0.058333in}{0.666667in}}%
\pgfpathcurveto{\pgfqpoint{-0.058333in}{0.651196in}}{\pgfqpoint{-0.052187in}{0.636358in}}{\pgfqpoint{-0.041248in}{0.625419in}}%
\pgfpathcurveto{\pgfqpoint{-0.030309in}{0.614480in}}{\pgfqpoint{-0.015470in}{0.608333in}}{\pgfqpoint{0.000000in}{0.608333in}}%
\pgfpathclose%
\pgfpathmoveto{\pgfqpoint{0.000000in}{0.614167in}}%
\pgfpathcurveto{\pgfqpoint{0.000000in}{0.614167in}}{\pgfqpoint{-0.013923in}{0.614167in}}{\pgfqpoint{-0.027278in}{0.619698in}}%
\pgfpathcurveto{\pgfqpoint{-0.037123in}{0.629544in}}{\pgfqpoint{-0.046968in}{0.639389in}}{\pgfqpoint{-0.052500in}{0.652744in}}%
\pgfpathcurveto{\pgfqpoint{-0.052500in}{0.666667in}}{\pgfqpoint{-0.052500in}{0.680590in}}{\pgfqpoint{-0.046968in}{0.693945in}}%
\pgfpathcurveto{\pgfqpoint{-0.037123in}{0.703790in}}{\pgfqpoint{-0.027278in}{0.713635in}}{\pgfqpoint{-0.013923in}{0.719167in}}%
\pgfpathcurveto{\pgfqpoint{0.000000in}{0.719167in}}{\pgfqpoint{0.013923in}{0.719167in}}{\pgfqpoint{0.027278in}{0.713635in}}%
\pgfpathcurveto{\pgfqpoint{0.037123in}{0.703790in}}{\pgfqpoint{0.046968in}{0.693945in}}{\pgfqpoint{0.052500in}{0.680590in}}%
\pgfpathcurveto{\pgfqpoint{0.052500in}{0.666667in}}{\pgfqpoint{0.052500in}{0.652744in}}{\pgfqpoint{0.046968in}{0.639389in}}%
\pgfpathcurveto{\pgfqpoint{0.037123in}{0.629544in}}{\pgfqpoint{0.027278in}{0.619698in}}{\pgfqpoint{0.013923in}{0.614167in}}%
\pgfpathclose%
\pgfpathmoveto{\pgfqpoint{0.166667in}{0.608333in}}%
\pgfpathcurveto{\pgfqpoint{0.182137in}{0.608333in}}{\pgfqpoint{0.196975in}{0.614480in}}{\pgfqpoint{0.207915in}{0.625419in}}%
\pgfpathcurveto{\pgfqpoint{0.218854in}{0.636358in}}{\pgfqpoint{0.225000in}{0.651196in}}{\pgfqpoint{0.225000in}{0.666667in}}%
\pgfpathcurveto{\pgfqpoint{0.225000in}{0.682137in}}{\pgfqpoint{0.218854in}{0.696975in}}{\pgfqpoint{0.207915in}{0.707915in}}%
\pgfpathcurveto{\pgfqpoint{0.196975in}{0.718854in}}{\pgfqpoint{0.182137in}{0.725000in}}{\pgfqpoint{0.166667in}{0.725000in}}%
\pgfpathcurveto{\pgfqpoint{0.151196in}{0.725000in}}{\pgfqpoint{0.136358in}{0.718854in}}{\pgfqpoint{0.125419in}{0.707915in}}%
\pgfpathcurveto{\pgfqpoint{0.114480in}{0.696975in}}{\pgfqpoint{0.108333in}{0.682137in}}{\pgfqpoint{0.108333in}{0.666667in}}%
\pgfpathcurveto{\pgfqpoint{0.108333in}{0.651196in}}{\pgfqpoint{0.114480in}{0.636358in}}{\pgfqpoint{0.125419in}{0.625419in}}%
\pgfpathcurveto{\pgfqpoint{0.136358in}{0.614480in}}{\pgfqpoint{0.151196in}{0.608333in}}{\pgfqpoint{0.166667in}{0.608333in}}%
\pgfpathclose%
\pgfpathmoveto{\pgfqpoint{0.166667in}{0.614167in}}%
\pgfpathcurveto{\pgfqpoint{0.166667in}{0.614167in}}{\pgfqpoint{0.152744in}{0.614167in}}{\pgfqpoint{0.139389in}{0.619698in}}%
\pgfpathcurveto{\pgfqpoint{0.129544in}{0.629544in}}{\pgfqpoint{0.119698in}{0.639389in}}{\pgfqpoint{0.114167in}{0.652744in}}%
\pgfpathcurveto{\pgfqpoint{0.114167in}{0.666667in}}{\pgfqpoint{0.114167in}{0.680590in}}{\pgfqpoint{0.119698in}{0.693945in}}%
\pgfpathcurveto{\pgfqpoint{0.129544in}{0.703790in}}{\pgfqpoint{0.139389in}{0.713635in}}{\pgfqpoint{0.152744in}{0.719167in}}%
\pgfpathcurveto{\pgfqpoint{0.166667in}{0.719167in}}{\pgfqpoint{0.180590in}{0.719167in}}{\pgfqpoint{0.193945in}{0.713635in}}%
\pgfpathcurveto{\pgfqpoint{0.203790in}{0.703790in}}{\pgfqpoint{0.213635in}{0.693945in}}{\pgfqpoint{0.219167in}{0.680590in}}%
\pgfpathcurveto{\pgfqpoint{0.219167in}{0.666667in}}{\pgfqpoint{0.219167in}{0.652744in}}{\pgfqpoint{0.213635in}{0.639389in}}%
\pgfpathcurveto{\pgfqpoint{0.203790in}{0.629544in}}{\pgfqpoint{0.193945in}{0.619698in}}{\pgfqpoint{0.180590in}{0.614167in}}%
\pgfpathclose%
\pgfpathmoveto{\pgfqpoint{0.333333in}{0.608333in}}%
\pgfpathcurveto{\pgfqpoint{0.348804in}{0.608333in}}{\pgfqpoint{0.363642in}{0.614480in}}{\pgfqpoint{0.374581in}{0.625419in}}%
\pgfpathcurveto{\pgfqpoint{0.385520in}{0.636358in}}{\pgfqpoint{0.391667in}{0.651196in}}{\pgfqpoint{0.391667in}{0.666667in}}%
\pgfpathcurveto{\pgfqpoint{0.391667in}{0.682137in}}{\pgfqpoint{0.385520in}{0.696975in}}{\pgfqpoint{0.374581in}{0.707915in}}%
\pgfpathcurveto{\pgfqpoint{0.363642in}{0.718854in}}{\pgfqpoint{0.348804in}{0.725000in}}{\pgfqpoint{0.333333in}{0.725000in}}%
\pgfpathcurveto{\pgfqpoint{0.317863in}{0.725000in}}{\pgfqpoint{0.303025in}{0.718854in}}{\pgfqpoint{0.292085in}{0.707915in}}%
\pgfpathcurveto{\pgfqpoint{0.281146in}{0.696975in}}{\pgfqpoint{0.275000in}{0.682137in}}{\pgfqpoint{0.275000in}{0.666667in}}%
\pgfpathcurveto{\pgfqpoint{0.275000in}{0.651196in}}{\pgfqpoint{0.281146in}{0.636358in}}{\pgfqpoint{0.292085in}{0.625419in}}%
\pgfpathcurveto{\pgfqpoint{0.303025in}{0.614480in}}{\pgfqpoint{0.317863in}{0.608333in}}{\pgfqpoint{0.333333in}{0.608333in}}%
\pgfpathclose%
\pgfpathmoveto{\pgfqpoint{0.333333in}{0.614167in}}%
\pgfpathcurveto{\pgfqpoint{0.333333in}{0.614167in}}{\pgfqpoint{0.319410in}{0.614167in}}{\pgfqpoint{0.306055in}{0.619698in}}%
\pgfpathcurveto{\pgfqpoint{0.296210in}{0.629544in}}{\pgfqpoint{0.286365in}{0.639389in}}{\pgfqpoint{0.280833in}{0.652744in}}%
\pgfpathcurveto{\pgfqpoint{0.280833in}{0.666667in}}{\pgfqpoint{0.280833in}{0.680590in}}{\pgfqpoint{0.286365in}{0.693945in}}%
\pgfpathcurveto{\pgfqpoint{0.296210in}{0.703790in}}{\pgfqpoint{0.306055in}{0.713635in}}{\pgfqpoint{0.319410in}{0.719167in}}%
\pgfpathcurveto{\pgfqpoint{0.333333in}{0.719167in}}{\pgfqpoint{0.347256in}{0.719167in}}{\pgfqpoint{0.360611in}{0.713635in}}%
\pgfpathcurveto{\pgfqpoint{0.370456in}{0.703790in}}{\pgfqpoint{0.380302in}{0.693945in}}{\pgfqpoint{0.385833in}{0.680590in}}%
\pgfpathcurveto{\pgfqpoint{0.385833in}{0.666667in}}{\pgfqpoint{0.385833in}{0.652744in}}{\pgfqpoint{0.380302in}{0.639389in}}%
\pgfpathcurveto{\pgfqpoint{0.370456in}{0.629544in}}{\pgfqpoint{0.360611in}{0.619698in}}{\pgfqpoint{0.347256in}{0.614167in}}%
\pgfpathclose%
\pgfpathmoveto{\pgfqpoint{0.500000in}{0.608333in}}%
\pgfpathcurveto{\pgfqpoint{0.515470in}{0.608333in}}{\pgfqpoint{0.530309in}{0.614480in}}{\pgfqpoint{0.541248in}{0.625419in}}%
\pgfpathcurveto{\pgfqpoint{0.552187in}{0.636358in}}{\pgfqpoint{0.558333in}{0.651196in}}{\pgfqpoint{0.558333in}{0.666667in}}%
\pgfpathcurveto{\pgfqpoint{0.558333in}{0.682137in}}{\pgfqpoint{0.552187in}{0.696975in}}{\pgfqpoint{0.541248in}{0.707915in}}%
\pgfpathcurveto{\pgfqpoint{0.530309in}{0.718854in}}{\pgfqpoint{0.515470in}{0.725000in}}{\pgfqpoint{0.500000in}{0.725000in}}%
\pgfpathcurveto{\pgfqpoint{0.484530in}{0.725000in}}{\pgfqpoint{0.469691in}{0.718854in}}{\pgfqpoint{0.458752in}{0.707915in}}%
\pgfpathcurveto{\pgfqpoint{0.447813in}{0.696975in}}{\pgfqpoint{0.441667in}{0.682137in}}{\pgfqpoint{0.441667in}{0.666667in}}%
\pgfpathcurveto{\pgfqpoint{0.441667in}{0.651196in}}{\pgfqpoint{0.447813in}{0.636358in}}{\pgfqpoint{0.458752in}{0.625419in}}%
\pgfpathcurveto{\pgfqpoint{0.469691in}{0.614480in}}{\pgfqpoint{0.484530in}{0.608333in}}{\pgfqpoint{0.500000in}{0.608333in}}%
\pgfpathclose%
\pgfpathmoveto{\pgfqpoint{0.500000in}{0.614167in}}%
\pgfpathcurveto{\pgfqpoint{0.500000in}{0.614167in}}{\pgfqpoint{0.486077in}{0.614167in}}{\pgfqpoint{0.472722in}{0.619698in}}%
\pgfpathcurveto{\pgfqpoint{0.462877in}{0.629544in}}{\pgfqpoint{0.453032in}{0.639389in}}{\pgfqpoint{0.447500in}{0.652744in}}%
\pgfpathcurveto{\pgfqpoint{0.447500in}{0.666667in}}{\pgfqpoint{0.447500in}{0.680590in}}{\pgfqpoint{0.453032in}{0.693945in}}%
\pgfpathcurveto{\pgfqpoint{0.462877in}{0.703790in}}{\pgfqpoint{0.472722in}{0.713635in}}{\pgfqpoint{0.486077in}{0.719167in}}%
\pgfpathcurveto{\pgfqpoint{0.500000in}{0.719167in}}{\pgfqpoint{0.513923in}{0.719167in}}{\pgfqpoint{0.527278in}{0.713635in}}%
\pgfpathcurveto{\pgfqpoint{0.537123in}{0.703790in}}{\pgfqpoint{0.546968in}{0.693945in}}{\pgfqpoint{0.552500in}{0.680590in}}%
\pgfpathcurveto{\pgfqpoint{0.552500in}{0.666667in}}{\pgfqpoint{0.552500in}{0.652744in}}{\pgfqpoint{0.546968in}{0.639389in}}%
\pgfpathcurveto{\pgfqpoint{0.537123in}{0.629544in}}{\pgfqpoint{0.527278in}{0.619698in}}{\pgfqpoint{0.513923in}{0.614167in}}%
\pgfpathclose%
\pgfpathmoveto{\pgfqpoint{0.666667in}{0.608333in}}%
\pgfpathcurveto{\pgfqpoint{0.682137in}{0.608333in}}{\pgfqpoint{0.696975in}{0.614480in}}{\pgfqpoint{0.707915in}{0.625419in}}%
\pgfpathcurveto{\pgfqpoint{0.718854in}{0.636358in}}{\pgfqpoint{0.725000in}{0.651196in}}{\pgfqpoint{0.725000in}{0.666667in}}%
\pgfpathcurveto{\pgfqpoint{0.725000in}{0.682137in}}{\pgfqpoint{0.718854in}{0.696975in}}{\pgfqpoint{0.707915in}{0.707915in}}%
\pgfpathcurveto{\pgfqpoint{0.696975in}{0.718854in}}{\pgfqpoint{0.682137in}{0.725000in}}{\pgfqpoint{0.666667in}{0.725000in}}%
\pgfpathcurveto{\pgfqpoint{0.651196in}{0.725000in}}{\pgfqpoint{0.636358in}{0.718854in}}{\pgfqpoint{0.625419in}{0.707915in}}%
\pgfpathcurveto{\pgfqpoint{0.614480in}{0.696975in}}{\pgfqpoint{0.608333in}{0.682137in}}{\pgfqpoint{0.608333in}{0.666667in}}%
\pgfpathcurveto{\pgfqpoint{0.608333in}{0.651196in}}{\pgfqpoint{0.614480in}{0.636358in}}{\pgfqpoint{0.625419in}{0.625419in}}%
\pgfpathcurveto{\pgfqpoint{0.636358in}{0.614480in}}{\pgfqpoint{0.651196in}{0.608333in}}{\pgfqpoint{0.666667in}{0.608333in}}%
\pgfpathclose%
\pgfpathmoveto{\pgfqpoint{0.666667in}{0.614167in}}%
\pgfpathcurveto{\pgfqpoint{0.666667in}{0.614167in}}{\pgfqpoint{0.652744in}{0.614167in}}{\pgfqpoint{0.639389in}{0.619698in}}%
\pgfpathcurveto{\pgfqpoint{0.629544in}{0.629544in}}{\pgfqpoint{0.619698in}{0.639389in}}{\pgfqpoint{0.614167in}{0.652744in}}%
\pgfpathcurveto{\pgfqpoint{0.614167in}{0.666667in}}{\pgfqpoint{0.614167in}{0.680590in}}{\pgfqpoint{0.619698in}{0.693945in}}%
\pgfpathcurveto{\pgfqpoint{0.629544in}{0.703790in}}{\pgfqpoint{0.639389in}{0.713635in}}{\pgfqpoint{0.652744in}{0.719167in}}%
\pgfpathcurveto{\pgfqpoint{0.666667in}{0.719167in}}{\pgfqpoint{0.680590in}{0.719167in}}{\pgfqpoint{0.693945in}{0.713635in}}%
\pgfpathcurveto{\pgfqpoint{0.703790in}{0.703790in}}{\pgfqpoint{0.713635in}{0.693945in}}{\pgfqpoint{0.719167in}{0.680590in}}%
\pgfpathcurveto{\pgfqpoint{0.719167in}{0.666667in}}{\pgfqpoint{0.719167in}{0.652744in}}{\pgfqpoint{0.713635in}{0.639389in}}%
\pgfpathcurveto{\pgfqpoint{0.703790in}{0.629544in}}{\pgfqpoint{0.693945in}{0.619698in}}{\pgfqpoint{0.680590in}{0.614167in}}%
\pgfpathclose%
\pgfpathmoveto{\pgfqpoint{0.833333in}{0.608333in}}%
\pgfpathcurveto{\pgfqpoint{0.848804in}{0.608333in}}{\pgfqpoint{0.863642in}{0.614480in}}{\pgfqpoint{0.874581in}{0.625419in}}%
\pgfpathcurveto{\pgfqpoint{0.885520in}{0.636358in}}{\pgfqpoint{0.891667in}{0.651196in}}{\pgfqpoint{0.891667in}{0.666667in}}%
\pgfpathcurveto{\pgfqpoint{0.891667in}{0.682137in}}{\pgfqpoint{0.885520in}{0.696975in}}{\pgfqpoint{0.874581in}{0.707915in}}%
\pgfpathcurveto{\pgfqpoint{0.863642in}{0.718854in}}{\pgfqpoint{0.848804in}{0.725000in}}{\pgfqpoint{0.833333in}{0.725000in}}%
\pgfpathcurveto{\pgfqpoint{0.817863in}{0.725000in}}{\pgfqpoint{0.803025in}{0.718854in}}{\pgfqpoint{0.792085in}{0.707915in}}%
\pgfpathcurveto{\pgfqpoint{0.781146in}{0.696975in}}{\pgfqpoint{0.775000in}{0.682137in}}{\pgfqpoint{0.775000in}{0.666667in}}%
\pgfpathcurveto{\pgfqpoint{0.775000in}{0.651196in}}{\pgfqpoint{0.781146in}{0.636358in}}{\pgfqpoint{0.792085in}{0.625419in}}%
\pgfpathcurveto{\pgfqpoint{0.803025in}{0.614480in}}{\pgfqpoint{0.817863in}{0.608333in}}{\pgfqpoint{0.833333in}{0.608333in}}%
\pgfpathclose%
\pgfpathmoveto{\pgfqpoint{0.833333in}{0.614167in}}%
\pgfpathcurveto{\pgfqpoint{0.833333in}{0.614167in}}{\pgfqpoint{0.819410in}{0.614167in}}{\pgfqpoint{0.806055in}{0.619698in}}%
\pgfpathcurveto{\pgfqpoint{0.796210in}{0.629544in}}{\pgfqpoint{0.786365in}{0.639389in}}{\pgfqpoint{0.780833in}{0.652744in}}%
\pgfpathcurveto{\pgfqpoint{0.780833in}{0.666667in}}{\pgfqpoint{0.780833in}{0.680590in}}{\pgfqpoint{0.786365in}{0.693945in}}%
\pgfpathcurveto{\pgfqpoint{0.796210in}{0.703790in}}{\pgfqpoint{0.806055in}{0.713635in}}{\pgfqpoint{0.819410in}{0.719167in}}%
\pgfpathcurveto{\pgfqpoint{0.833333in}{0.719167in}}{\pgfqpoint{0.847256in}{0.719167in}}{\pgfqpoint{0.860611in}{0.713635in}}%
\pgfpathcurveto{\pgfqpoint{0.870456in}{0.703790in}}{\pgfqpoint{0.880302in}{0.693945in}}{\pgfqpoint{0.885833in}{0.680590in}}%
\pgfpathcurveto{\pgfqpoint{0.885833in}{0.666667in}}{\pgfqpoint{0.885833in}{0.652744in}}{\pgfqpoint{0.880302in}{0.639389in}}%
\pgfpathcurveto{\pgfqpoint{0.870456in}{0.629544in}}{\pgfqpoint{0.860611in}{0.619698in}}{\pgfqpoint{0.847256in}{0.614167in}}%
\pgfpathclose%
\pgfpathmoveto{\pgfqpoint{1.000000in}{0.608333in}}%
\pgfpathcurveto{\pgfqpoint{1.015470in}{0.608333in}}{\pgfqpoint{1.030309in}{0.614480in}}{\pgfqpoint{1.041248in}{0.625419in}}%
\pgfpathcurveto{\pgfqpoint{1.052187in}{0.636358in}}{\pgfqpoint{1.058333in}{0.651196in}}{\pgfqpoint{1.058333in}{0.666667in}}%
\pgfpathcurveto{\pgfqpoint{1.058333in}{0.682137in}}{\pgfqpoint{1.052187in}{0.696975in}}{\pgfqpoint{1.041248in}{0.707915in}}%
\pgfpathcurveto{\pgfqpoint{1.030309in}{0.718854in}}{\pgfqpoint{1.015470in}{0.725000in}}{\pgfqpoint{1.000000in}{0.725000in}}%
\pgfpathcurveto{\pgfqpoint{0.984530in}{0.725000in}}{\pgfqpoint{0.969691in}{0.718854in}}{\pgfqpoint{0.958752in}{0.707915in}}%
\pgfpathcurveto{\pgfqpoint{0.947813in}{0.696975in}}{\pgfqpoint{0.941667in}{0.682137in}}{\pgfqpoint{0.941667in}{0.666667in}}%
\pgfpathcurveto{\pgfqpoint{0.941667in}{0.651196in}}{\pgfqpoint{0.947813in}{0.636358in}}{\pgfqpoint{0.958752in}{0.625419in}}%
\pgfpathcurveto{\pgfqpoint{0.969691in}{0.614480in}}{\pgfqpoint{0.984530in}{0.608333in}}{\pgfqpoint{1.000000in}{0.608333in}}%
\pgfpathclose%
\pgfpathmoveto{\pgfqpoint{1.000000in}{0.614167in}}%
\pgfpathcurveto{\pgfqpoint{1.000000in}{0.614167in}}{\pgfqpoint{0.986077in}{0.614167in}}{\pgfqpoint{0.972722in}{0.619698in}}%
\pgfpathcurveto{\pgfqpoint{0.962877in}{0.629544in}}{\pgfqpoint{0.953032in}{0.639389in}}{\pgfqpoint{0.947500in}{0.652744in}}%
\pgfpathcurveto{\pgfqpoint{0.947500in}{0.666667in}}{\pgfqpoint{0.947500in}{0.680590in}}{\pgfqpoint{0.953032in}{0.693945in}}%
\pgfpathcurveto{\pgfqpoint{0.962877in}{0.703790in}}{\pgfqpoint{0.972722in}{0.713635in}}{\pgfqpoint{0.986077in}{0.719167in}}%
\pgfpathcurveto{\pgfqpoint{1.000000in}{0.719167in}}{\pgfqpoint{1.013923in}{0.719167in}}{\pgfqpoint{1.027278in}{0.713635in}}%
\pgfpathcurveto{\pgfqpoint{1.037123in}{0.703790in}}{\pgfqpoint{1.046968in}{0.693945in}}{\pgfqpoint{1.052500in}{0.680590in}}%
\pgfpathcurveto{\pgfqpoint{1.052500in}{0.666667in}}{\pgfqpoint{1.052500in}{0.652744in}}{\pgfqpoint{1.046968in}{0.639389in}}%
\pgfpathcurveto{\pgfqpoint{1.037123in}{0.629544in}}{\pgfqpoint{1.027278in}{0.619698in}}{\pgfqpoint{1.013923in}{0.614167in}}%
\pgfpathclose%
\pgfpathmoveto{\pgfqpoint{0.083333in}{0.775000in}}%
\pgfpathcurveto{\pgfqpoint{0.098804in}{0.775000in}}{\pgfqpoint{0.113642in}{0.781146in}}{\pgfqpoint{0.124581in}{0.792085in}}%
\pgfpathcurveto{\pgfqpoint{0.135520in}{0.803025in}}{\pgfqpoint{0.141667in}{0.817863in}}{\pgfqpoint{0.141667in}{0.833333in}}%
\pgfpathcurveto{\pgfqpoint{0.141667in}{0.848804in}}{\pgfqpoint{0.135520in}{0.863642in}}{\pgfqpoint{0.124581in}{0.874581in}}%
\pgfpathcurveto{\pgfqpoint{0.113642in}{0.885520in}}{\pgfqpoint{0.098804in}{0.891667in}}{\pgfqpoint{0.083333in}{0.891667in}}%
\pgfpathcurveto{\pgfqpoint{0.067863in}{0.891667in}}{\pgfqpoint{0.053025in}{0.885520in}}{\pgfqpoint{0.042085in}{0.874581in}}%
\pgfpathcurveto{\pgfqpoint{0.031146in}{0.863642in}}{\pgfqpoint{0.025000in}{0.848804in}}{\pgfqpoint{0.025000in}{0.833333in}}%
\pgfpathcurveto{\pgfqpoint{0.025000in}{0.817863in}}{\pgfqpoint{0.031146in}{0.803025in}}{\pgfqpoint{0.042085in}{0.792085in}}%
\pgfpathcurveto{\pgfqpoint{0.053025in}{0.781146in}}{\pgfqpoint{0.067863in}{0.775000in}}{\pgfqpoint{0.083333in}{0.775000in}}%
\pgfpathclose%
\pgfpathmoveto{\pgfqpoint{0.083333in}{0.780833in}}%
\pgfpathcurveto{\pgfqpoint{0.083333in}{0.780833in}}{\pgfqpoint{0.069410in}{0.780833in}}{\pgfqpoint{0.056055in}{0.786365in}}%
\pgfpathcurveto{\pgfqpoint{0.046210in}{0.796210in}}{\pgfqpoint{0.036365in}{0.806055in}}{\pgfqpoint{0.030833in}{0.819410in}}%
\pgfpathcurveto{\pgfqpoint{0.030833in}{0.833333in}}{\pgfqpoint{0.030833in}{0.847256in}}{\pgfqpoint{0.036365in}{0.860611in}}%
\pgfpathcurveto{\pgfqpoint{0.046210in}{0.870456in}}{\pgfqpoint{0.056055in}{0.880302in}}{\pgfqpoint{0.069410in}{0.885833in}}%
\pgfpathcurveto{\pgfqpoint{0.083333in}{0.885833in}}{\pgfqpoint{0.097256in}{0.885833in}}{\pgfqpoint{0.110611in}{0.880302in}}%
\pgfpathcurveto{\pgfqpoint{0.120456in}{0.870456in}}{\pgfqpoint{0.130302in}{0.860611in}}{\pgfqpoint{0.135833in}{0.847256in}}%
\pgfpathcurveto{\pgfqpoint{0.135833in}{0.833333in}}{\pgfqpoint{0.135833in}{0.819410in}}{\pgfqpoint{0.130302in}{0.806055in}}%
\pgfpathcurveto{\pgfqpoint{0.120456in}{0.796210in}}{\pgfqpoint{0.110611in}{0.786365in}}{\pgfqpoint{0.097256in}{0.780833in}}%
\pgfpathclose%
\pgfpathmoveto{\pgfqpoint{0.250000in}{0.775000in}}%
\pgfpathcurveto{\pgfqpoint{0.265470in}{0.775000in}}{\pgfqpoint{0.280309in}{0.781146in}}{\pgfqpoint{0.291248in}{0.792085in}}%
\pgfpathcurveto{\pgfqpoint{0.302187in}{0.803025in}}{\pgfqpoint{0.308333in}{0.817863in}}{\pgfqpoint{0.308333in}{0.833333in}}%
\pgfpathcurveto{\pgfqpoint{0.308333in}{0.848804in}}{\pgfqpoint{0.302187in}{0.863642in}}{\pgfqpoint{0.291248in}{0.874581in}}%
\pgfpathcurveto{\pgfqpoint{0.280309in}{0.885520in}}{\pgfqpoint{0.265470in}{0.891667in}}{\pgfqpoint{0.250000in}{0.891667in}}%
\pgfpathcurveto{\pgfqpoint{0.234530in}{0.891667in}}{\pgfqpoint{0.219691in}{0.885520in}}{\pgfqpoint{0.208752in}{0.874581in}}%
\pgfpathcurveto{\pgfqpoint{0.197813in}{0.863642in}}{\pgfqpoint{0.191667in}{0.848804in}}{\pgfqpoint{0.191667in}{0.833333in}}%
\pgfpathcurveto{\pgfqpoint{0.191667in}{0.817863in}}{\pgfqpoint{0.197813in}{0.803025in}}{\pgfqpoint{0.208752in}{0.792085in}}%
\pgfpathcurveto{\pgfqpoint{0.219691in}{0.781146in}}{\pgfqpoint{0.234530in}{0.775000in}}{\pgfqpoint{0.250000in}{0.775000in}}%
\pgfpathclose%
\pgfpathmoveto{\pgfqpoint{0.250000in}{0.780833in}}%
\pgfpathcurveto{\pgfqpoint{0.250000in}{0.780833in}}{\pgfqpoint{0.236077in}{0.780833in}}{\pgfqpoint{0.222722in}{0.786365in}}%
\pgfpathcurveto{\pgfqpoint{0.212877in}{0.796210in}}{\pgfqpoint{0.203032in}{0.806055in}}{\pgfqpoint{0.197500in}{0.819410in}}%
\pgfpathcurveto{\pgfqpoint{0.197500in}{0.833333in}}{\pgfqpoint{0.197500in}{0.847256in}}{\pgfqpoint{0.203032in}{0.860611in}}%
\pgfpathcurveto{\pgfqpoint{0.212877in}{0.870456in}}{\pgfqpoint{0.222722in}{0.880302in}}{\pgfqpoint{0.236077in}{0.885833in}}%
\pgfpathcurveto{\pgfqpoint{0.250000in}{0.885833in}}{\pgfqpoint{0.263923in}{0.885833in}}{\pgfqpoint{0.277278in}{0.880302in}}%
\pgfpathcurveto{\pgfqpoint{0.287123in}{0.870456in}}{\pgfqpoint{0.296968in}{0.860611in}}{\pgfqpoint{0.302500in}{0.847256in}}%
\pgfpathcurveto{\pgfqpoint{0.302500in}{0.833333in}}{\pgfqpoint{0.302500in}{0.819410in}}{\pgfqpoint{0.296968in}{0.806055in}}%
\pgfpathcurveto{\pgfqpoint{0.287123in}{0.796210in}}{\pgfqpoint{0.277278in}{0.786365in}}{\pgfqpoint{0.263923in}{0.780833in}}%
\pgfpathclose%
\pgfpathmoveto{\pgfqpoint{0.416667in}{0.775000in}}%
\pgfpathcurveto{\pgfqpoint{0.432137in}{0.775000in}}{\pgfqpoint{0.446975in}{0.781146in}}{\pgfqpoint{0.457915in}{0.792085in}}%
\pgfpathcurveto{\pgfqpoint{0.468854in}{0.803025in}}{\pgfqpoint{0.475000in}{0.817863in}}{\pgfqpoint{0.475000in}{0.833333in}}%
\pgfpathcurveto{\pgfqpoint{0.475000in}{0.848804in}}{\pgfqpoint{0.468854in}{0.863642in}}{\pgfqpoint{0.457915in}{0.874581in}}%
\pgfpathcurveto{\pgfqpoint{0.446975in}{0.885520in}}{\pgfqpoint{0.432137in}{0.891667in}}{\pgfqpoint{0.416667in}{0.891667in}}%
\pgfpathcurveto{\pgfqpoint{0.401196in}{0.891667in}}{\pgfqpoint{0.386358in}{0.885520in}}{\pgfqpoint{0.375419in}{0.874581in}}%
\pgfpathcurveto{\pgfqpoint{0.364480in}{0.863642in}}{\pgfqpoint{0.358333in}{0.848804in}}{\pgfqpoint{0.358333in}{0.833333in}}%
\pgfpathcurveto{\pgfqpoint{0.358333in}{0.817863in}}{\pgfqpoint{0.364480in}{0.803025in}}{\pgfqpoint{0.375419in}{0.792085in}}%
\pgfpathcurveto{\pgfqpoint{0.386358in}{0.781146in}}{\pgfqpoint{0.401196in}{0.775000in}}{\pgfqpoint{0.416667in}{0.775000in}}%
\pgfpathclose%
\pgfpathmoveto{\pgfqpoint{0.416667in}{0.780833in}}%
\pgfpathcurveto{\pgfqpoint{0.416667in}{0.780833in}}{\pgfqpoint{0.402744in}{0.780833in}}{\pgfqpoint{0.389389in}{0.786365in}}%
\pgfpathcurveto{\pgfqpoint{0.379544in}{0.796210in}}{\pgfqpoint{0.369698in}{0.806055in}}{\pgfqpoint{0.364167in}{0.819410in}}%
\pgfpathcurveto{\pgfqpoint{0.364167in}{0.833333in}}{\pgfqpoint{0.364167in}{0.847256in}}{\pgfqpoint{0.369698in}{0.860611in}}%
\pgfpathcurveto{\pgfqpoint{0.379544in}{0.870456in}}{\pgfqpoint{0.389389in}{0.880302in}}{\pgfqpoint{0.402744in}{0.885833in}}%
\pgfpathcurveto{\pgfqpoint{0.416667in}{0.885833in}}{\pgfqpoint{0.430590in}{0.885833in}}{\pgfqpoint{0.443945in}{0.880302in}}%
\pgfpathcurveto{\pgfqpoint{0.453790in}{0.870456in}}{\pgfqpoint{0.463635in}{0.860611in}}{\pgfqpoint{0.469167in}{0.847256in}}%
\pgfpathcurveto{\pgfqpoint{0.469167in}{0.833333in}}{\pgfqpoint{0.469167in}{0.819410in}}{\pgfqpoint{0.463635in}{0.806055in}}%
\pgfpathcurveto{\pgfqpoint{0.453790in}{0.796210in}}{\pgfqpoint{0.443945in}{0.786365in}}{\pgfqpoint{0.430590in}{0.780833in}}%
\pgfpathclose%
\pgfpathmoveto{\pgfqpoint{0.583333in}{0.775000in}}%
\pgfpathcurveto{\pgfqpoint{0.598804in}{0.775000in}}{\pgfqpoint{0.613642in}{0.781146in}}{\pgfqpoint{0.624581in}{0.792085in}}%
\pgfpathcurveto{\pgfqpoint{0.635520in}{0.803025in}}{\pgfqpoint{0.641667in}{0.817863in}}{\pgfqpoint{0.641667in}{0.833333in}}%
\pgfpathcurveto{\pgfqpoint{0.641667in}{0.848804in}}{\pgfqpoint{0.635520in}{0.863642in}}{\pgfqpoint{0.624581in}{0.874581in}}%
\pgfpathcurveto{\pgfqpoint{0.613642in}{0.885520in}}{\pgfqpoint{0.598804in}{0.891667in}}{\pgfqpoint{0.583333in}{0.891667in}}%
\pgfpathcurveto{\pgfqpoint{0.567863in}{0.891667in}}{\pgfqpoint{0.553025in}{0.885520in}}{\pgfqpoint{0.542085in}{0.874581in}}%
\pgfpathcurveto{\pgfqpoint{0.531146in}{0.863642in}}{\pgfqpoint{0.525000in}{0.848804in}}{\pgfqpoint{0.525000in}{0.833333in}}%
\pgfpathcurveto{\pgfqpoint{0.525000in}{0.817863in}}{\pgfqpoint{0.531146in}{0.803025in}}{\pgfqpoint{0.542085in}{0.792085in}}%
\pgfpathcurveto{\pgfqpoint{0.553025in}{0.781146in}}{\pgfqpoint{0.567863in}{0.775000in}}{\pgfqpoint{0.583333in}{0.775000in}}%
\pgfpathclose%
\pgfpathmoveto{\pgfqpoint{0.583333in}{0.780833in}}%
\pgfpathcurveto{\pgfqpoint{0.583333in}{0.780833in}}{\pgfqpoint{0.569410in}{0.780833in}}{\pgfqpoint{0.556055in}{0.786365in}}%
\pgfpathcurveto{\pgfqpoint{0.546210in}{0.796210in}}{\pgfqpoint{0.536365in}{0.806055in}}{\pgfqpoint{0.530833in}{0.819410in}}%
\pgfpathcurveto{\pgfqpoint{0.530833in}{0.833333in}}{\pgfqpoint{0.530833in}{0.847256in}}{\pgfqpoint{0.536365in}{0.860611in}}%
\pgfpathcurveto{\pgfqpoint{0.546210in}{0.870456in}}{\pgfqpoint{0.556055in}{0.880302in}}{\pgfqpoint{0.569410in}{0.885833in}}%
\pgfpathcurveto{\pgfqpoint{0.583333in}{0.885833in}}{\pgfqpoint{0.597256in}{0.885833in}}{\pgfqpoint{0.610611in}{0.880302in}}%
\pgfpathcurveto{\pgfqpoint{0.620456in}{0.870456in}}{\pgfqpoint{0.630302in}{0.860611in}}{\pgfqpoint{0.635833in}{0.847256in}}%
\pgfpathcurveto{\pgfqpoint{0.635833in}{0.833333in}}{\pgfqpoint{0.635833in}{0.819410in}}{\pgfqpoint{0.630302in}{0.806055in}}%
\pgfpathcurveto{\pgfqpoint{0.620456in}{0.796210in}}{\pgfqpoint{0.610611in}{0.786365in}}{\pgfqpoint{0.597256in}{0.780833in}}%
\pgfpathclose%
\pgfpathmoveto{\pgfqpoint{0.750000in}{0.775000in}}%
\pgfpathcurveto{\pgfqpoint{0.765470in}{0.775000in}}{\pgfqpoint{0.780309in}{0.781146in}}{\pgfqpoint{0.791248in}{0.792085in}}%
\pgfpathcurveto{\pgfqpoint{0.802187in}{0.803025in}}{\pgfqpoint{0.808333in}{0.817863in}}{\pgfqpoint{0.808333in}{0.833333in}}%
\pgfpathcurveto{\pgfqpoint{0.808333in}{0.848804in}}{\pgfqpoint{0.802187in}{0.863642in}}{\pgfqpoint{0.791248in}{0.874581in}}%
\pgfpathcurveto{\pgfqpoint{0.780309in}{0.885520in}}{\pgfqpoint{0.765470in}{0.891667in}}{\pgfqpoint{0.750000in}{0.891667in}}%
\pgfpathcurveto{\pgfqpoint{0.734530in}{0.891667in}}{\pgfqpoint{0.719691in}{0.885520in}}{\pgfqpoint{0.708752in}{0.874581in}}%
\pgfpathcurveto{\pgfqpoint{0.697813in}{0.863642in}}{\pgfqpoint{0.691667in}{0.848804in}}{\pgfqpoint{0.691667in}{0.833333in}}%
\pgfpathcurveto{\pgfqpoint{0.691667in}{0.817863in}}{\pgfqpoint{0.697813in}{0.803025in}}{\pgfqpoint{0.708752in}{0.792085in}}%
\pgfpathcurveto{\pgfqpoint{0.719691in}{0.781146in}}{\pgfqpoint{0.734530in}{0.775000in}}{\pgfqpoint{0.750000in}{0.775000in}}%
\pgfpathclose%
\pgfpathmoveto{\pgfqpoint{0.750000in}{0.780833in}}%
\pgfpathcurveto{\pgfqpoint{0.750000in}{0.780833in}}{\pgfqpoint{0.736077in}{0.780833in}}{\pgfqpoint{0.722722in}{0.786365in}}%
\pgfpathcurveto{\pgfqpoint{0.712877in}{0.796210in}}{\pgfqpoint{0.703032in}{0.806055in}}{\pgfqpoint{0.697500in}{0.819410in}}%
\pgfpathcurveto{\pgfqpoint{0.697500in}{0.833333in}}{\pgfqpoint{0.697500in}{0.847256in}}{\pgfqpoint{0.703032in}{0.860611in}}%
\pgfpathcurveto{\pgfqpoint{0.712877in}{0.870456in}}{\pgfqpoint{0.722722in}{0.880302in}}{\pgfqpoint{0.736077in}{0.885833in}}%
\pgfpathcurveto{\pgfqpoint{0.750000in}{0.885833in}}{\pgfqpoint{0.763923in}{0.885833in}}{\pgfqpoint{0.777278in}{0.880302in}}%
\pgfpathcurveto{\pgfqpoint{0.787123in}{0.870456in}}{\pgfqpoint{0.796968in}{0.860611in}}{\pgfqpoint{0.802500in}{0.847256in}}%
\pgfpathcurveto{\pgfqpoint{0.802500in}{0.833333in}}{\pgfqpoint{0.802500in}{0.819410in}}{\pgfqpoint{0.796968in}{0.806055in}}%
\pgfpathcurveto{\pgfqpoint{0.787123in}{0.796210in}}{\pgfqpoint{0.777278in}{0.786365in}}{\pgfqpoint{0.763923in}{0.780833in}}%
\pgfpathclose%
\pgfpathmoveto{\pgfqpoint{0.916667in}{0.775000in}}%
\pgfpathcurveto{\pgfqpoint{0.932137in}{0.775000in}}{\pgfqpoint{0.946975in}{0.781146in}}{\pgfqpoint{0.957915in}{0.792085in}}%
\pgfpathcurveto{\pgfqpoint{0.968854in}{0.803025in}}{\pgfqpoint{0.975000in}{0.817863in}}{\pgfqpoint{0.975000in}{0.833333in}}%
\pgfpathcurveto{\pgfqpoint{0.975000in}{0.848804in}}{\pgfqpoint{0.968854in}{0.863642in}}{\pgfqpoint{0.957915in}{0.874581in}}%
\pgfpathcurveto{\pgfqpoint{0.946975in}{0.885520in}}{\pgfqpoint{0.932137in}{0.891667in}}{\pgfqpoint{0.916667in}{0.891667in}}%
\pgfpathcurveto{\pgfqpoint{0.901196in}{0.891667in}}{\pgfqpoint{0.886358in}{0.885520in}}{\pgfqpoint{0.875419in}{0.874581in}}%
\pgfpathcurveto{\pgfqpoint{0.864480in}{0.863642in}}{\pgfqpoint{0.858333in}{0.848804in}}{\pgfqpoint{0.858333in}{0.833333in}}%
\pgfpathcurveto{\pgfqpoint{0.858333in}{0.817863in}}{\pgfqpoint{0.864480in}{0.803025in}}{\pgfqpoint{0.875419in}{0.792085in}}%
\pgfpathcurveto{\pgfqpoint{0.886358in}{0.781146in}}{\pgfqpoint{0.901196in}{0.775000in}}{\pgfqpoint{0.916667in}{0.775000in}}%
\pgfpathclose%
\pgfpathmoveto{\pgfqpoint{0.916667in}{0.780833in}}%
\pgfpathcurveto{\pgfqpoint{0.916667in}{0.780833in}}{\pgfqpoint{0.902744in}{0.780833in}}{\pgfqpoint{0.889389in}{0.786365in}}%
\pgfpathcurveto{\pgfqpoint{0.879544in}{0.796210in}}{\pgfqpoint{0.869698in}{0.806055in}}{\pgfqpoint{0.864167in}{0.819410in}}%
\pgfpathcurveto{\pgfqpoint{0.864167in}{0.833333in}}{\pgfqpoint{0.864167in}{0.847256in}}{\pgfqpoint{0.869698in}{0.860611in}}%
\pgfpathcurveto{\pgfqpoint{0.879544in}{0.870456in}}{\pgfqpoint{0.889389in}{0.880302in}}{\pgfqpoint{0.902744in}{0.885833in}}%
\pgfpathcurveto{\pgfqpoint{0.916667in}{0.885833in}}{\pgfqpoint{0.930590in}{0.885833in}}{\pgfqpoint{0.943945in}{0.880302in}}%
\pgfpathcurveto{\pgfqpoint{0.953790in}{0.870456in}}{\pgfqpoint{0.963635in}{0.860611in}}{\pgfqpoint{0.969167in}{0.847256in}}%
\pgfpathcurveto{\pgfqpoint{0.969167in}{0.833333in}}{\pgfqpoint{0.969167in}{0.819410in}}{\pgfqpoint{0.963635in}{0.806055in}}%
\pgfpathcurveto{\pgfqpoint{0.953790in}{0.796210in}}{\pgfqpoint{0.943945in}{0.786365in}}{\pgfqpoint{0.930590in}{0.780833in}}%
\pgfpathclose%
\pgfpathmoveto{\pgfqpoint{0.000000in}{0.941667in}}%
\pgfpathcurveto{\pgfqpoint{0.015470in}{0.941667in}}{\pgfqpoint{0.030309in}{0.947813in}}{\pgfqpoint{0.041248in}{0.958752in}}%
\pgfpathcurveto{\pgfqpoint{0.052187in}{0.969691in}}{\pgfqpoint{0.058333in}{0.984530in}}{\pgfqpoint{0.058333in}{1.000000in}}%
\pgfpathcurveto{\pgfqpoint{0.058333in}{1.015470in}}{\pgfqpoint{0.052187in}{1.030309in}}{\pgfqpoint{0.041248in}{1.041248in}}%
\pgfpathcurveto{\pgfqpoint{0.030309in}{1.052187in}}{\pgfqpoint{0.015470in}{1.058333in}}{\pgfqpoint{0.000000in}{1.058333in}}%
\pgfpathcurveto{\pgfqpoint{-0.015470in}{1.058333in}}{\pgfqpoint{-0.030309in}{1.052187in}}{\pgfqpoint{-0.041248in}{1.041248in}}%
\pgfpathcurveto{\pgfqpoint{-0.052187in}{1.030309in}}{\pgfqpoint{-0.058333in}{1.015470in}}{\pgfqpoint{-0.058333in}{1.000000in}}%
\pgfpathcurveto{\pgfqpoint{-0.058333in}{0.984530in}}{\pgfqpoint{-0.052187in}{0.969691in}}{\pgfqpoint{-0.041248in}{0.958752in}}%
\pgfpathcurveto{\pgfqpoint{-0.030309in}{0.947813in}}{\pgfqpoint{-0.015470in}{0.941667in}}{\pgfqpoint{0.000000in}{0.941667in}}%
\pgfpathclose%
\pgfpathmoveto{\pgfqpoint{0.000000in}{0.947500in}}%
\pgfpathcurveto{\pgfqpoint{0.000000in}{0.947500in}}{\pgfqpoint{-0.013923in}{0.947500in}}{\pgfqpoint{-0.027278in}{0.953032in}}%
\pgfpathcurveto{\pgfqpoint{-0.037123in}{0.962877in}}{\pgfqpoint{-0.046968in}{0.972722in}}{\pgfqpoint{-0.052500in}{0.986077in}}%
\pgfpathcurveto{\pgfqpoint{-0.052500in}{1.000000in}}{\pgfqpoint{-0.052500in}{1.013923in}}{\pgfqpoint{-0.046968in}{1.027278in}}%
\pgfpathcurveto{\pgfqpoint{-0.037123in}{1.037123in}}{\pgfqpoint{-0.027278in}{1.046968in}}{\pgfqpoint{-0.013923in}{1.052500in}}%
\pgfpathcurveto{\pgfqpoint{0.000000in}{1.052500in}}{\pgfqpoint{0.013923in}{1.052500in}}{\pgfqpoint{0.027278in}{1.046968in}}%
\pgfpathcurveto{\pgfqpoint{0.037123in}{1.037123in}}{\pgfqpoint{0.046968in}{1.027278in}}{\pgfqpoint{0.052500in}{1.013923in}}%
\pgfpathcurveto{\pgfqpoint{0.052500in}{1.000000in}}{\pgfqpoint{0.052500in}{0.986077in}}{\pgfqpoint{0.046968in}{0.972722in}}%
\pgfpathcurveto{\pgfqpoint{0.037123in}{0.962877in}}{\pgfqpoint{0.027278in}{0.953032in}}{\pgfqpoint{0.013923in}{0.947500in}}%
\pgfpathclose%
\pgfpathmoveto{\pgfqpoint{0.166667in}{0.941667in}}%
\pgfpathcurveto{\pgfqpoint{0.182137in}{0.941667in}}{\pgfqpoint{0.196975in}{0.947813in}}{\pgfqpoint{0.207915in}{0.958752in}}%
\pgfpathcurveto{\pgfqpoint{0.218854in}{0.969691in}}{\pgfqpoint{0.225000in}{0.984530in}}{\pgfqpoint{0.225000in}{1.000000in}}%
\pgfpathcurveto{\pgfqpoint{0.225000in}{1.015470in}}{\pgfqpoint{0.218854in}{1.030309in}}{\pgfqpoint{0.207915in}{1.041248in}}%
\pgfpathcurveto{\pgfqpoint{0.196975in}{1.052187in}}{\pgfqpoint{0.182137in}{1.058333in}}{\pgfqpoint{0.166667in}{1.058333in}}%
\pgfpathcurveto{\pgfqpoint{0.151196in}{1.058333in}}{\pgfqpoint{0.136358in}{1.052187in}}{\pgfqpoint{0.125419in}{1.041248in}}%
\pgfpathcurveto{\pgfqpoint{0.114480in}{1.030309in}}{\pgfqpoint{0.108333in}{1.015470in}}{\pgfqpoint{0.108333in}{1.000000in}}%
\pgfpathcurveto{\pgfqpoint{0.108333in}{0.984530in}}{\pgfqpoint{0.114480in}{0.969691in}}{\pgfqpoint{0.125419in}{0.958752in}}%
\pgfpathcurveto{\pgfqpoint{0.136358in}{0.947813in}}{\pgfqpoint{0.151196in}{0.941667in}}{\pgfqpoint{0.166667in}{0.941667in}}%
\pgfpathclose%
\pgfpathmoveto{\pgfqpoint{0.166667in}{0.947500in}}%
\pgfpathcurveto{\pgfqpoint{0.166667in}{0.947500in}}{\pgfqpoint{0.152744in}{0.947500in}}{\pgfqpoint{0.139389in}{0.953032in}}%
\pgfpathcurveto{\pgfqpoint{0.129544in}{0.962877in}}{\pgfqpoint{0.119698in}{0.972722in}}{\pgfqpoint{0.114167in}{0.986077in}}%
\pgfpathcurveto{\pgfqpoint{0.114167in}{1.000000in}}{\pgfqpoint{0.114167in}{1.013923in}}{\pgfqpoint{0.119698in}{1.027278in}}%
\pgfpathcurveto{\pgfqpoint{0.129544in}{1.037123in}}{\pgfqpoint{0.139389in}{1.046968in}}{\pgfqpoint{0.152744in}{1.052500in}}%
\pgfpathcurveto{\pgfqpoint{0.166667in}{1.052500in}}{\pgfqpoint{0.180590in}{1.052500in}}{\pgfqpoint{0.193945in}{1.046968in}}%
\pgfpathcurveto{\pgfqpoint{0.203790in}{1.037123in}}{\pgfqpoint{0.213635in}{1.027278in}}{\pgfqpoint{0.219167in}{1.013923in}}%
\pgfpathcurveto{\pgfqpoint{0.219167in}{1.000000in}}{\pgfqpoint{0.219167in}{0.986077in}}{\pgfqpoint{0.213635in}{0.972722in}}%
\pgfpathcurveto{\pgfqpoint{0.203790in}{0.962877in}}{\pgfqpoint{0.193945in}{0.953032in}}{\pgfqpoint{0.180590in}{0.947500in}}%
\pgfpathclose%
\pgfpathmoveto{\pgfqpoint{0.333333in}{0.941667in}}%
\pgfpathcurveto{\pgfqpoint{0.348804in}{0.941667in}}{\pgfqpoint{0.363642in}{0.947813in}}{\pgfqpoint{0.374581in}{0.958752in}}%
\pgfpathcurveto{\pgfqpoint{0.385520in}{0.969691in}}{\pgfqpoint{0.391667in}{0.984530in}}{\pgfqpoint{0.391667in}{1.000000in}}%
\pgfpathcurveto{\pgfqpoint{0.391667in}{1.015470in}}{\pgfqpoint{0.385520in}{1.030309in}}{\pgfqpoint{0.374581in}{1.041248in}}%
\pgfpathcurveto{\pgfqpoint{0.363642in}{1.052187in}}{\pgfqpoint{0.348804in}{1.058333in}}{\pgfqpoint{0.333333in}{1.058333in}}%
\pgfpathcurveto{\pgfqpoint{0.317863in}{1.058333in}}{\pgfqpoint{0.303025in}{1.052187in}}{\pgfqpoint{0.292085in}{1.041248in}}%
\pgfpathcurveto{\pgfqpoint{0.281146in}{1.030309in}}{\pgfqpoint{0.275000in}{1.015470in}}{\pgfqpoint{0.275000in}{1.000000in}}%
\pgfpathcurveto{\pgfqpoint{0.275000in}{0.984530in}}{\pgfqpoint{0.281146in}{0.969691in}}{\pgfqpoint{0.292085in}{0.958752in}}%
\pgfpathcurveto{\pgfqpoint{0.303025in}{0.947813in}}{\pgfqpoint{0.317863in}{0.941667in}}{\pgfqpoint{0.333333in}{0.941667in}}%
\pgfpathclose%
\pgfpathmoveto{\pgfqpoint{0.333333in}{0.947500in}}%
\pgfpathcurveto{\pgfqpoint{0.333333in}{0.947500in}}{\pgfqpoint{0.319410in}{0.947500in}}{\pgfqpoint{0.306055in}{0.953032in}}%
\pgfpathcurveto{\pgfqpoint{0.296210in}{0.962877in}}{\pgfqpoint{0.286365in}{0.972722in}}{\pgfqpoint{0.280833in}{0.986077in}}%
\pgfpathcurveto{\pgfqpoint{0.280833in}{1.000000in}}{\pgfqpoint{0.280833in}{1.013923in}}{\pgfqpoint{0.286365in}{1.027278in}}%
\pgfpathcurveto{\pgfqpoint{0.296210in}{1.037123in}}{\pgfqpoint{0.306055in}{1.046968in}}{\pgfqpoint{0.319410in}{1.052500in}}%
\pgfpathcurveto{\pgfqpoint{0.333333in}{1.052500in}}{\pgfqpoint{0.347256in}{1.052500in}}{\pgfqpoint{0.360611in}{1.046968in}}%
\pgfpathcurveto{\pgfqpoint{0.370456in}{1.037123in}}{\pgfqpoint{0.380302in}{1.027278in}}{\pgfqpoint{0.385833in}{1.013923in}}%
\pgfpathcurveto{\pgfqpoint{0.385833in}{1.000000in}}{\pgfqpoint{0.385833in}{0.986077in}}{\pgfqpoint{0.380302in}{0.972722in}}%
\pgfpathcurveto{\pgfqpoint{0.370456in}{0.962877in}}{\pgfqpoint{0.360611in}{0.953032in}}{\pgfqpoint{0.347256in}{0.947500in}}%
\pgfpathclose%
\pgfpathmoveto{\pgfqpoint{0.500000in}{0.941667in}}%
\pgfpathcurveto{\pgfqpoint{0.515470in}{0.941667in}}{\pgfqpoint{0.530309in}{0.947813in}}{\pgfqpoint{0.541248in}{0.958752in}}%
\pgfpathcurveto{\pgfqpoint{0.552187in}{0.969691in}}{\pgfqpoint{0.558333in}{0.984530in}}{\pgfqpoint{0.558333in}{1.000000in}}%
\pgfpathcurveto{\pgfqpoint{0.558333in}{1.015470in}}{\pgfqpoint{0.552187in}{1.030309in}}{\pgfqpoint{0.541248in}{1.041248in}}%
\pgfpathcurveto{\pgfqpoint{0.530309in}{1.052187in}}{\pgfqpoint{0.515470in}{1.058333in}}{\pgfqpoint{0.500000in}{1.058333in}}%
\pgfpathcurveto{\pgfqpoint{0.484530in}{1.058333in}}{\pgfqpoint{0.469691in}{1.052187in}}{\pgfqpoint{0.458752in}{1.041248in}}%
\pgfpathcurveto{\pgfqpoint{0.447813in}{1.030309in}}{\pgfqpoint{0.441667in}{1.015470in}}{\pgfqpoint{0.441667in}{1.000000in}}%
\pgfpathcurveto{\pgfqpoint{0.441667in}{0.984530in}}{\pgfqpoint{0.447813in}{0.969691in}}{\pgfqpoint{0.458752in}{0.958752in}}%
\pgfpathcurveto{\pgfqpoint{0.469691in}{0.947813in}}{\pgfqpoint{0.484530in}{0.941667in}}{\pgfqpoint{0.500000in}{0.941667in}}%
\pgfpathclose%
\pgfpathmoveto{\pgfqpoint{0.500000in}{0.947500in}}%
\pgfpathcurveto{\pgfqpoint{0.500000in}{0.947500in}}{\pgfqpoint{0.486077in}{0.947500in}}{\pgfqpoint{0.472722in}{0.953032in}}%
\pgfpathcurveto{\pgfqpoint{0.462877in}{0.962877in}}{\pgfqpoint{0.453032in}{0.972722in}}{\pgfqpoint{0.447500in}{0.986077in}}%
\pgfpathcurveto{\pgfqpoint{0.447500in}{1.000000in}}{\pgfqpoint{0.447500in}{1.013923in}}{\pgfqpoint{0.453032in}{1.027278in}}%
\pgfpathcurveto{\pgfqpoint{0.462877in}{1.037123in}}{\pgfqpoint{0.472722in}{1.046968in}}{\pgfqpoint{0.486077in}{1.052500in}}%
\pgfpathcurveto{\pgfqpoint{0.500000in}{1.052500in}}{\pgfqpoint{0.513923in}{1.052500in}}{\pgfqpoint{0.527278in}{1.046968in}}%
\pgfpathcurveto{\pgfqpoint{0.537123in}{1.037123in}}{\pgfqpoint{0.546968in}{1.027278in}}{\pgfqpoint{0.552500in}{1.013923in}}%
\pgfpathcurveto{\pgfqpoint{0.552500in}{1.000000in}}{\pgfqpoint{0.552500in}{0.986077in}}{\pgfqpoint{0.546968in}{0.972722in}}%
\pgfpathcurveto{\pgfqpoint{0.537123in}{0.962877in}}{\pgfqpoint{0.527278in}{0.953032in}}{\pgfqpoint{0.513923in}{0.947500in}}%
\pgfpathclose%
\pgfpathmoveto{\pgfqpoint{0.666667in}{0.941667in}}%
\pgfpathcurveto{\pgfqpoint{0.682137in}{0.941667in}}{\pgfqpoint{0.696975in}{0.947813in}}{\pgfqpoint{0.707915in}{0.958752in}}%
\pgfpathcurveto{\pgfqpoint{0.718854in}{0.969691in}}{\pgfqpoint{0.725000in}{0.984530in}}{\pgfqpoint{0.725000in}{1.000000in}}%
\pgfpathcurveto{\pgfqpoint{0.725000in}{1.015470in}}{\pgfqpoint{0.718854in}{1.030309in}}{\pgfqpoint{0.707915in}{1.041248in}}%
\pgfpathcurveto{\pgfqpoint{0.696975in}{1.052187in}}{\pgfqpoint{0.682137in}{1.058333in}}{\pgfqpoint{0.666667in}{1.058333in}}%
\pgfpathcurveto{\pgfqpoint{0.651196in}{1.058333in}}{\pgfqpoint{0.636358in}{1.052187in}}{\pgfqpoint{0.625419in}{1.041248in}}%
\pgfpathcurveto{\pgfqpoint{0.614480in}{1.030309in}}{\pgfqpoint{0.608333in}{1.015470in}}{\pgfqpoint{0.608333in}{1.000000in}}%
\pgfpathcurveto{\pgfqpoint{0.608333in}{0.984530in}}{\pgfqpoint{0.614480in}{0.969691in}}{\pgfqpoint{0.625419in}{0.958752in}}%
\pgfpathcurveto{\pgfqpoint{0.636358in}{0.947813in}}{\pgfqpoint{0.651196in}{0.941667in}}{\pgfqpoint{0.666667in}{0.941667in}}%
\pgfpathclose%
\pgfpathmoveto{\pgfqpoint{0.666667in}{0.947500in}}%
\pgfpathcurveto{\pgfqpoint{0.666667in}{0.947500in}}{\pgfqpoint{0.652744in}{0.947500in}}{\pgfqpoint{0.639389in}{0.953032in}}%
\pgfpathcurveto{\pgfqpoint{0.629544in}{0.962877in}}{\pgfqpoint{0.619698in}{0.972722in}}{\pgfqpoint{0.614167in}{0.986077in}}%
\pgfpathcurveto{\pgfqpoint{0.614167in}{1.000000in}}{\pgfqpoint{0.614167in}{1.013923in}}{\pgfqpoint{0.619698in}{1.027278in}}%
\pgfpathcurveto{\pgfqpoint{0.629544in}{1.037123in}}{\pgfqpoint{0.639389in}{1.046968in}}{\pgfqpoint{0.652744in}{1.052500in}}%
\pgfpathcurveto{\pgfqpoint{0.666667in}{1.052500in}}{\pgfqpoint{0.680590in}{1.052500in}}{\pgfqpoint{0.693945in}{1.046968in}}%
\pgfpathcurveto{\pgfqpoint{0.703790in}{1.037123in}}{\pgfqpoint{0.713635in}{1.027278in}}{\pgfqpoint{0.719167in}{1.013923in}}%
\pgfpathcurveto{\pgfqpoint{0.719167in}{1.000000in}}{\pgfqpoint{0.719167in}{0.986077in}}{\pgfqpoint{0.713635in}{0.972722in}}%
\pgfpathcurveto{\pgfqpoint{0.703790in}{0.962877in}}{\pgfqpoint{0.693945in}{0.953032in}}{\pgfqpoint{0.680590in}{0.947500in}}%
\pgfpathclose%
\pgfpathmoveto{\pgfqpoint{0.833333in}{0.941667in}}%
\pgfpathcurveto{\pgfqpoint{0.848804in}{0.941667in}}{\pgfqpoint{0.863642in}{0.947813in}}{\pgfqpoint{0.874581in}{0.958752in}}%
\pgfpathcurveto{\pgfqpoint{0.885520in}{0.969691in}}{\pgfqpoint{0.891667in}{0.984530in}}{\pgfqpoint{0.891667in}{1.000000in}}%
\pgfpathcurveto{\pgfqpoint{0.891667in}{1.015470in}}{\pgfqpoint{0.885520in}{1.030309in}}{\pgfqpoint{0.874581in}{1.041248in}}%
\pgfpathcurveto{\pgfqpoint{0.863642in}{1.052187in}}{\pgfqpoint{0.848804in}{1.058333in}}{\pgfqpoint{0.833333in}{1.058333in}}%
\pgfpathcurveto{\pgfqpoint{0.817863in}{1.058333in}}{\pgfqpoint{0.803025in}{1.052187in}}{\pgfqpoint{0.792085in}{1.041248in}}%
\pgfpathcurveto{\pgfqpoint{0.781146in}{1.030309in}}{\pgfqpoint{0.775000in}{1.015470in}}{\pgfqpoint{0.775000in}{1.000000in}}%
\pgfpathcurveto{\pgfqpoint{0.775000in}{0.984530in}}{\pgfqpoint{0.781146in}{0.969691in}}{\pgfqpoint{0.792085in}{0.958752in}}%
\pgfpathcurveto{\pgfqpoint{0.803025in}{0.947813in}}{\pgfqpoint{0.817863in}{0.941667in}}{\pgfqpoint{0.833333in}{0.941667in}}%
\pgfpathclose%
\pgfpathmoveto{\pgfqpoint{0.833333in}{0.947500in}}%
\pgfpathcurveto{\pgfqpoint{0.833333in}{0.947500in}}{\pgfqpoint{0.819410in}{0.947500in}}{\pgfqpoint{0.806055in}{0.953032in}}%
\pgfpathcurveto{\pgfqpoint{0.796210in}{0.962877in}}{\pgfqpoint{0.786365in}{0.972722in}}{\pgfqpoint{0.780833in}{0.986077in}}%
\pgfpathcurveto{\pgfqpoint{0.780833in}{1.000000in}}{\pgfqpoint{0.780833in}{1.013923in}}{\pgfqpoint{0.786365in}{1.027278in}}%
\pgfpathcurveto{\pgfqpoint{0.796210in}{1.037123in}}{\pgfqpoint{0.806055in}{1.046968in}}{\pgfqpoint{0.819410in}{1.052500in}}%
\pgfpathcurveto{\pgfqpoint{0.833333in}{1.052500in}}{\pgfqpoint{0.847256in}{1.052500in}}{\pgfqpoint{0.860611in}{1.046968in}}%
\pgfpathcurveto{\pgfqpoint{0.870456in}{1.037123in}}{\pgfqpoint{0.880302in}{1.027278in}}{\pgfqpoint{0.885833in}{1.013923in}}%
\pgfpathcurveto{\pgfqpoint{0.885833in}{1.000000in}}{\pgfqpoint{0.885833in}{0.986077in}}{\pgfqpoint{0.880302in}{0.972722in}}%
\pgfpathcurveto{\pgfqpoint{0.870456in}{0.962877in}}{\pgfqpoint{0.860611in}{0.953032in}}{\pgfqpoint{0.847256in}{0.947500in}}%
\pgfpathclose%
\pgfpathmoveto{\pgfqpoint{1.000000in}{0.941667in}}%
\pgfpathcurveto{\pgfqpoint{1.015470in}{0.941667in}}{\pgfqpoint{1.030309in}{0.947813in}}{\pgfqpoint{1.041248in}{0.958752in}}%
\pgfpathcurveto{\pgfqpoint{1.052187in}{0.969691in}}{\pgfqpoint{1.058333in}{0.984530in}}{\pgfqpoint{1.058333in}{1.000000in}}%
\pgfpathcurveto{\pgfqpoint{1.058333in}{1.015470in}}{\pgfqpoint{1.052187in}{1.030309in}}{\pgfqpoint{1.041248in}{1.041248in}}%
\pgfpathcurveto{\pgfqpoint{1.030309in}{1.052187in}}{\pgfqpoint{1.015470in}{1.058333in}}{\pgfqpoint{1.000000in}{1.058333in}}%
\pgfpathcurveto{\pgfqpoint{0.984530in}{1.058333in}}{\pgfqpoint{0.969691in}{1.052187in}}{\pgfqpoint{0.958752in}{1.041248in}}%
\pgfpathcurveto{\pgfqpoint{0.947813in}{1.030309in}}{\pgfqpoint{0.941667in}{1.015470in}}{\pgfqpoint{0.941667in}{1.000000in}}%
\pgfpathcurveto{\pgfqpoint{0.941667in}{0.984530in}}{\pgfqpoint{0.947813in}{0.969691in}}{\pgfqpoint{0.958752in}{0.958752in}}%
\pgfpathcurveto{\pgfqpoint{0.969691in}{0.947813in}}{\pgfqpoint{0.984530in}{0.941667in}}{\pgfqpoint{1.000000in}{0.941667in}}%
\pgfpathclose%
\pgfpathmoveto{\pgfqpoint{1.000000in}{0.947500in}}%
\pgfpathcurveto{\pgfqpoint{1.000000in}{0.947500in}}{\pgfqpoint{0.986077in}{0.947500in}}{\pgfqpoint{0.972722in}{0.953032in}}%
\pgfpathcurveto{\pgfqpoint{0.962877in}{0.962877in}}{\pgfqpoint{0.953032in}{0.972722in}}{\pgfqpoint{0.947500in}{0.986077in}}%
\pgfpathcurveto{\pgfqpoint{0.947500in}{1.000000in}}{\pgfqpoint{0.947500in}{1.013923in}}{\pgfqpoint{0.953032in}{1.027278in}}%
\pgfpathcurveto{\pgfqpoint{0.962877in}{1.037123in}}{\pgfqpoint{0.972722in}{1.046968in}}{\pgfqpoint{0.986077in}{1.052500in}}%
\pgfpathcurveto{\pgfqpoint{1.000000in}{1.052500in}}{\pgfqpoint{1.013923in}{1.052500in}}{\pgfqpoint{1.027278in}{1.046968in}}%
\pgfpathcurveto{\pgfqpoint{1.037123in}{1.037123in}}{\pgfqpoint{1.046968in}{1.027278in}}{\pgfqpoint{1.052500in}{1.013923in}}%
\pgfpathcurveto{\pgfqpoint{1.052500in}{1.000000in}}{\pgfqpoint{1.052500in}{0.986077in}}{\pgfqpoint{1.046968in}{0.972722in}}%
\pgfpathcurveto{\pgfqpoint{1.037123in}{0.962877in}}{\pgfqpoint{1.027278in}{0.953032in}}{\pgfqpoint{1.013923in}{0.947500in}}%
\pgfpathclose%
\pgfusepath{stroke}%
\end{pgfscope}%
}%
\pgfsys@transformshift{1.070538in}{9.006369in}%
\pgfsys@useobject{currentpattern}{}%
\pgfsys@transformshift{1in}{0in}%
\pgfsys@transformshift{-1in}{0in}%
\pgfsys@transformshift{0in}{1in}%
\end{pgfscope}%
\begin{pgfscope}%
\definecolor{textcolor}{rgb}{0.000000,0.000000,0.000000}%
\pgfsetstrokecolor{textcolor}%
\pgfsetfillcolor{textcolor}%
\pgftext[x=1.692760in,y=9.006369in,left,base]{\color{textcolor}\rmfamily\fontsize{16.000000}{19.200000}\selectfont IMP\_ELC}%
\end{pgfscope}%
\begin{pgfscope}%
\pgfsetbuttcap%
\pgfsetmiterjoin%
\definecolor{currentfill}{rgb}{0.411765,0.411765,0.411765}%
\pgfsetfillcolor{currentfill}%
\pgfsetfillopacity{0.990000}%
\pgfsetlinewidth{0.000000pt}%
\definecolor{currentstroke}{rgb}{0.000000,0.000000,0.000000}%
\pgfsetstrokecolor{currentstroke}%
\pgfsetstrokeopacity{0.990000}%
\pgfsetdash{}{0pt}%
\pgfpathmoveto{\pgfqpoint{1.070538in}{8.681909in}}%
\pgfpathlineto{\pgfqpoint{1.514982in}{8.681909in}}%
\pgfpathlineto{\pgfqpoint{1.514982in}{8.837465in}}%
\pgfpathlineto{\pgfqpoint{1.070538in}{8.837465in}}%
\pgfpathclose%
\pgfusepath{fill}%
\end{pgfscope}%
\begin{pgfscope}%
\pgfsetbuttcap%
\pgfsetmiterjoin%
\definecolor{currentfill}{rgb}{0.411765,0.411765,0.411765}%
\pgfsetfillcolor{currentfill}%
\pgfsetfillopacity{0.990000}%
\pgfsetlinewidth{0.000000pt}%
\definecolor{currentstroke}{rgb}{0.000000,0.000000,0.000000}%
\pgfsetstrokecolor{currentstroke}%
\pgfsetstrokeopacity{0.990000}%
\pgfsetdash{}{0pt}%
\pgfpathmoveto{\pgfqpoint{1.070538in}{8.681909in}}%
\pgfpathlineto{\pgfqpoint{1.514982in}{8.681909in}}%
\pgfpathlineto{\pgfqpoint{1.514982in}{8.837465in}}%
\pgfpathlineto{\pgfqpoint{1.070538in}{8.837465in}}%
\pgfpathclose%
\pgfusepath{clip}%
\pgfsys@defobject{currentpattern}{\pgfqpoint{0in}{0in}}{\pgfqpoint{1in}{1in}}{%
\begin{pgfscope}%
\pgfpathrectangle{\pgfqpoint{0in}{0in}}{\pgfqpoint{1in}{1in}}%
\pgfusepath{clip}%
\pgfpathmoveto{\pgfqpoint{-0.500000in}{0.500000in}}%
\pgfpathlineto{\pgfqpoint{0.500000in}{1.500000in}}%
\pgfpathmoveto{\pgfqpoint{-0.333333in}{0.333333in}}%
\pgfpathlineto{\pgfqpoint{0.666667in}{1.333333in}}%
\pgfpathmoveto{\pgfqpoint{-0.166667in}{0.166667in}}%
\pgfpathlineto{\pgfqpoint{0.833333in}{1.166667in}}%
\pgfpathmoveto{\pgfqpoint{0.000000in}{0.000000in}}%
\pgfpathlineto{\pgfqpoint{1.000000in}{1.000000in}}%
\pgfpathmoveto{\pgfqpoint{0.166667in}{-0.166667in}}%
\pgfpathlineto{\pgfqpoint{1.166667in}{0.833333in}}%
\pgfpathmoveto{\pgfqpoint{0.333333in}{-0.333333in}}%
\pgfpathlineto{\pgfqpoint{1.333333in}{0.666667in}}%
\pgfpathmoveto{\pgfqpoint{0.500000in}{-0.500000in}}%
\pgfpathlineto{\pgfqpoint{1.500000in}{0.500000in}}%
\pgfusepath{stroke}%
\end{pgfscope}%
}%
\pgfsys@transformshift{1.070538in}{8.681909in}%
\pgfsys@useobject{currentpattern}{}%
\pgfsys@transformshift{1in}{0in}%
\pgfsys@transformshift{-1in}{0in}%
\pgfsys@transformshift{0in}{1in}%
\end{pgfscope}%
\begin{pgfscope}%
\definecolor{textcolor}{rgb}{0.000000,0.000000,0.000000}%
\pgfsetstrokecolor{textcolor}%
\pgfsetfillcolor{textcolor}%
\pgftext[x=1.692760in,y=8.681909in,left,base]{\color{textcolor}\rmfamily\fontsize{16.000000}{19.200000}\selectfont LI\_BATTERY}%
\end{pgfscope}%
\begin{pgfscope}%
\pgfsetbuttcap%
\pgfsetmiterjoin%
\definecolor{currentfill}{rgb}{0.172549,0.627451,0.172549}%
\pgfsetfillcolor{currentfill}%
\pgfsetfillopacity{0.990000}%
\pgfsetlinewidth{0.000000pt}%
\definecolor{currentstroke}{rgb}{0.000000,0.000000,0.000000}%
\pgfsetstrokecolor{currentstroke}%
\pgfsetstrokeopacity{0.990000}%
\pgfsetdash{}{0pt}%
\pgfpathmoveto{\pgfqpoint{1.070538in}{8.357449in}}%
\pgfpathlineto{\pgfqpoint{1.514982in}{8.357449in}}%
\pgfpathlineto{\pgfqpoint{1.514982in}{8.513005in}}%
\pgfpathlineto{\pgfqpoint{1.070538in}{8.513005in}}%
\pgfpathclose%
\pgfusepath{fill}%
\end{pgfscope}%
\begin{pgfscope}%
\pgfsetbuttcap%
\pgfsetmiterjoin%
\definecolor{currentfill}{rgb}{0.172549,0.627451,0.172549}%
\pgfsetfillcolor{currentfill}%
\pgfsetfillopacity{0.990000}%
\pgfsetlinewidth{0.000000pt}%
\definecolor{currentstroke}{rgb}{0.000000,0.000000,0.000000}%
\pgfsetstrokecolor{currentstroke}%
\pgfsetstrokeopacity{0.990000}%
\pgfsetdash{}{0pt}%
\pgfpathmoveto{\pgfqpoint{1.070538in}{8.357449in}}%
\pgfpathlineto{\pgfqpoint{1.514982in}{8.357449in}}%
\pgfpathlineto{\pgfqpoint{1.514982in}{8.513005in}}%
\pgfpathlineto{\pgfqpoint{1.070538in}{8.513005in}}%
\pgfpathclose%
\pgfusepath{clip}%
\pgfsys@defobject{currentpattern}{\pgfqpoint{0in}{0in}}{\pgfqpoint{1in}{1in}}{%
\begin{pgfscope}%
\pgfpathrectangle{\pgfqpoint{0in}{0in}}{\pgfqpoint{1in}{1in}}%
\pgfusepath{clip}%
\pgfpathmoveto{\pgfqpoint{0.000000in}{-0.016667in}}%
\pgfpathcurveto{\pgfqpoint{0.004420in}{-0.016667in}}{\pgfqpoint{0.008660in}{-0.014911in}}{\pgfqpoint{0.011785in}{-0.011785in}}%
\pgfpathcurveto{\pgfqpoint{0.014911in}{-0.008660in}}{\pgfqpoint{0.016667in}{-0.004420in}}{\pgfqpoint{0.016667in}{0.000000in}}%
\pgfpathcurveto{\pgfqpoint{0.016667in}{0.004420in}}{\pgfqpoint{0.014911in}{0.008660in}}{\pgfqpoint{0.011785in}{0.011785in}}%
\pgfpathcurveto{\pgfqpoint{0.008660in}{0.014911in}}{\pgfqpoint{0.004420in}{0.016667in}}{\pgfqpoint{0.000000in}{0.016667in}}%
\pgfpathcurveto{\pgfqpoint{-0.004420in}{0.016667in}}{\pgfqpoint{-0.008660in}{0.014911in}}{\pgfqpoint{-0.011785in}{0.011785in}}%
\pgfpathcurveto{\pgfqpoint{-0.014911in}{0.008660in}}{\pgfqpoint{-0.016667in}{0.004420in}}{\pgfqpoint{-0.016667in}{0.000000in}}%
\pgfpathcurveto{\pgfqpoint{-0.016667in}{-0.004420in}}{\pgfqpoint{-0.014911in}{-0.008660in}}{\pgfqpoint{-0.011785in}{-0.011785in}}%
\pgfpathcurveto{\pgfqpoint{-0.008660in}{-0.014911in}}{\pgfqpoint{-0.004420in}{-0.016667in}}{\pgfqpoint{0.000000in}{-0.016667in}}%
\pgfpathclose%
\pgfpathmoveto{\pgfqpoint{0.166667in}{-0.016667in}}%
\pgfpathcurveto{\pgfqpoint{0.171087in}{-0.016667in}}{\pgfqpoint{0.175326in}{-0.014911in}}{\pgfqpoint{0.178452in}{-0.011785in}}%
\pgfpathcurveto{\pgfqpoint{0.181577in}{-0.008660in}}{\pgfqpoint{0.183333in}{-0.004420in}}{\pgfqpoint{0.183333in}{0.000000in}}%
\pgfpathcurveto{\pgfqpoint{0.183333in}{0.004420in}}{\pgfqpoint{0.181577in}{0.008660in}}{\pgfqpoint{0.178452in}{0.011785in}}%
\pgfpathcurveto{\pgfqpoint{0.175326in}{0.014911in}}{\pgfqpoint{0.171087in}{0.016667in}}{\pgfqpoint{0.166667in}{0.016667in}}%
\pgfpathcurveto{\pgfqpoint{0.162247in}{0.016667in}}{\pgfqpoint{0.158007in}{0.014911in}}{\pgfqpoint{0.154882in}{0.011785in}}%
\pgfpathcurveto{\pgfqpoint{0.151756in}{0.008660in}}{\pgfqpoint{0.150000in}{0.004420in}}{\pgfqpoint{0.150000in}{0.000000in}}%
\pgfpathcurveto{\pgfqpoint{0.150000in}{-0.004420in}}{\pgfqpoint{0.151756in}{-0.008660in}}{\pgfqpoint{0.154882in}{-0.011785in}}%
\pgfpathcurveto{\pgfqpoint{0.158007in}{-0.014911in}}{\pgfqpoint{0.162247in}{-0.016667in}}{\pgfqpoint{0.166667in}{-0.016667in}}%
\pgfpathclose%
\pgfpathmoveto{\pgfqpoint{0.333333in}{-0.016667in}}%
\pgfpathcurveto{\pgfqpoint{0.337753in}{-0.016667in}}{\pgfqpoint{0.341993in}{-0.014911in}}{\pgfqpoint{0.345118in}{-0.011785in}}%
\pgfpathcurveto{\pgfqpoint{0.348244in}{-0.008660in}}{\pgfqpoint{0.350000in}{-0.004420in}}{\pgfqpoint{0.350000in}{0.000000in}}%
\pgfpathcurveto{\pgfqpoint{0.350000in}{0.004420in}}{\pgfqpoint{0.348244in}{0.008660in}}{\pgfqpoint{0.345118in}{0.011785in}}%
\pgfpathcurveto{\pgfqpoint{0.341993in}{0.014911in}}{\pgfqpoint{0.337753in}{0.016667in}}{\pgfqpoint{0.333333in}{0.016667in}}%
\pgfpathcurveto{\pgfqpoint{0.328913in}{0.016667in}}{\pgfqpoint{0.324674in}{0.014911in}}{\pgfqpoint{0.321548in}{0.011785in}}%
\pgfpathcurveto{\pgfqpoint{0.318423in}{0.008660in}}{\pgfqpoint{0.316667in}{0.004420in}}{\pgfqpoint{0.316667in}{0.000000in}}%
\pgfpathcurveto{\pgfqpoint{0.316667in}{-0.004420in}}{\pgfqpoint{0.318423in}{-0.008660in}}{\pgfqpoint{0.321548in}{-0.011785in}}%
\pgfpathcurveto{\pgfqpoint{0.324674in}{-0.014911in}}{\pgfqpoint{0.328913in}{-0.016667in}}{\pgfqpoint{0.333333in}{-0.016667in}}%
\pgfpathclose%
\pgfpathmoveto{\pgfqpoint{0.500000in}{-0.016667in}}%
\pgfpathcurveto{\pgfqpoint{0.504420in}{-0.016667in}}{\pgfqpoint{0.508660in}{-0.014911in}}{\pgfqpoint{0.511785in}{-0.011785in}}%
\pgfpathcurveto{\pgfqpoint{0.514911in}{-0.008660in}}{\pgfqpoint{0.516667in}{-0.004420in}}{\pgfqpoint{0.516667in}{0.000000in}}%
\pgfpathcurveto{\pgfqpoint{0.516667in}{0.004420in}}{\pgfqpoint{0.514911in}{0.008660in}}{\pgfqpoint{0.511785in}{0.011785in}}%
\pgfpathcurveto{\pgfqpoint{0.508660in}{0.014911in}}{\pgfqpoint{0.504420in}{0.016667in}}{\pgfqpoint{0.500000in}{0.016667in}}%
\pgfpathcurveto{\pgfqpoint{0.495580in}{0.016667in}}{\pgfqpoint{0.491340in}{0.014911in}}{\pgfqpoint{0.488215in}{0.011785in}}%
\pgfpathcurveto{\pgfqpoint{0.485089in}{0.008660in}}{\pgfqpoint{0.483333in}{0.004420in}}{\pgfqpoint{0.483333in}{0.000000in}}%
\pgfpathcurveto{\pgfqpoint{0.483333in}{-0.004420in}}{\pgfqpoint{0.485089in}{-0.008660in}}{\pgfqpoint{0.488215in}{-0.011785in}}%
\pgfpathcurveto{\pgfqpoint{0.491340in}{-0.014911in}}{\pgfqpoint{0.495580in}{-0.016667in}}{\pgfqpoint{0.500000in}{-0.016667in}}%
\pgfpathclose%
\pgfpathmoveto{\pgfqpoint{0.666667in}{-0.016667in}}%
\pgfpathcurveto{\pgfqpoint{0.671087in}{-0.016667in}}{\pgfqpoint{0.675326in}{-0.014911in}}{\pgfqpoint{0.678452in}{-0.011785in}}%
\pgfpathcurveto{\pgfqpoint{0.681577in}{-0.008660in}}{\pgfqpoint{0.683333in}{-0.004420in}}{\pgfqpoint{0.683333in}{0.000000in}}%
\pgfpathcurveto{\pgfqpoint{0.683333in}{0.004420in}}{\pgfqpoint{0.681577in}{0.008660in}}{\pgfqpoint{0.678452in}{0.011785in}}%
\pgfpathcurveto{\pgfqpoint{0.675326in}{0.014911in}}{\pgfqpoint{0.671087in}{0.016667in}}{\pgfqpoint{0.666667in}{0.016667in}}%
\pgfpathcurveto{\pgfqpoint{0.662247in}{0.016667in}}{\pgfqpoint{0.658007in}{0.014911in}}{\pgfqpoint{0.654882in}{0.011785in}}%
\pgfpathcurveto{\pgfqpoint{0.651756in}{0.008660in}}{\pgfqpoint{0.650000in}{0.004420in}}{\pgfqpoint{0.650000in}{0.000000in}}%
\pgfpathcurveto{\pgfqpoint{0.650000in}{-0.004420in}}{\pgfqpoint{0.651756in}{-0.008660in}}{\pgfqpoint{0.654882in}{-0.011785in}}%
\pgfpathcurveto{\pgfqpoint{0.658007in}{-0.014911in}}{\pgfqpoint{0.662247in}{-0.016667in}}{\pgfqpoint{0.666667in}{-0.016667in}}%
\pgfpathclose%
\pgfpathmoveto{\pgfqpoint{0.833333in}{-0.016667in}}%
\pgfpathcurveto{\pgfqpoint{0.837753in}{-0.016667in}}{\pgfqpoint{0.841993in}{-0.014911in}}{\pgfqpoint{0.845118in}{-0.011785in}}%
\pgfpathcurveto{\pgfqpoint{0.848244in}{-0.008660in}}{\pgfqpoint{0.850000in}{-0.004420in}}{\pgfqpoint{0.850000in}{0.000000in}}%
\pgfpathcurveto{\pgfqpoint{0.850000in}{0.004420in}}{\pgfqpoint{0.848244in}{0.008660in}}{\pgfqpoint{0.845118in}{0.011785in}}%
\pgfpathcurveto{\pgfqpoint{0.841993in}{0.014911in}}{\pgfqpoint{0.837753in}{0.016667in}}{\pgfqpoint{0.833333in}{0.016667in}}%
\pgfpathcurveto{\pgfqpoint{0.828913in}{0.016667in}}{\pgfqpoint{0.824674in}{0.014911in}}{\pgfqpoint{0.821548in}{0.011785in}}%
\pgfpathcurveto{\pgfqpoint{0.818423in}{0.008660in}}{\pgfqpoint{0.816667in}{0.004420in}}{\pgfqpoint{0.816667in}{0.000000in}}%
\pgfpathcurveto{\pgfqpoint{0.816667in}{-0.004420in}}{\pgfqpoint{0.818423in}{-0.008660in}}{\pgfqpoint{0.821548in}{-0.011785in}}%
\pgfpathcurveto{\pgfqpoint{0.824674in}{-0.014911in}}{\pgfqpoint{0.828913in}{-0.016667in}}{\pgfqpoint{0.833333in}{-0.016667in}}%
\pgfpathclose%
\pgfpathmoveto{\pgfqpoint{1.000000in}{-0.016667in}}%
\pgfpathcurveto{\pgfqpoint{1.004420in}{-0.016667in}}{\pgfqpoint{1.008660in}{-0.014911in}}{\pgfqpoint{1.011785in}{-0.011785in}}%
\pgfpathcurveto{\pgfqpoint{1.014911in}{-0.008660in}}{\pgfqpoint{1.016667in}{-0.004420in}}{\pgfqpoint{1.016667in}{0.000000in}}%
\pgfpathcurveto{\pgfqpoint{1.016667in}{0.004420in}}{\pgfqpoint{1.014911in}{0.008660in}}{\pgfqpoint{1.011785in}{0.011785in}}%
\pgfpathcurveto{\pgfqpoint{1.008660in}{0.014911in}}{\pgfqpoint{1.004420in}{0.016667in}}{\pgfqpoint{1.000000in}{0.016667in}}%
\pgfpathcurveto{\pgfqpoint{0.995580in}{0.016667in}}{\pgfqpoint{0.991340in}{0.014911in}}{\pgfqpoint{0.988215in}{0.011785in}}%
\pgfpathcurveto{\pgfqpoint{0.985089in}{0.008660in}}{\pgfqpoint{0.983333in}{0.004420in}}{\pgfqpoint{0.983333in}{0.000000in}}%
\pgfpathcurveto{\pgfqpoint{0.983333in}{-0.004420in}}{\pgfqpoint{0.985089in}{-0.008660in}}{\pgfqpoint{0.988215in}{-0.011785in}}%
\pgfpathcurveto{\pgfqpoint{0.991340in}{-0.014911in}}{\pgfqpoint{0.995580in}{-0.016667in}}{\pgfqpoint{1.000000in}{-0.016667in}}%
\pgfpathclose%
\pgfpathmoveto{\pgfqpoint{0.083333in}{0.150000in}}%
\pgfpathcurveto{\pgfqpoint{0.087753in}{0.150000in}}{\pgfqpoint{0.091993in}{0.151756in}}{\pgfqpoint{0.095118in}{0.154882in}}%
\pgfpathcurveto{\pgfqpoint{0.098244in}{0.158007in}}{\pgfqpoint{0.100000in}{0.162247in}}{\pgfqpoint{0.100000in}{0.166667in}}%
\pgfpathcurveto{\pgfqpoint{0.100000in}{0.171087in}}{\pgfqpoint{0.098244in}{0.175326in}}{\pgfqpoint{0.095118in}{0.178452in}}%
\pgfpathcurveto{\pgfqpoint{0.091993in}{0.181577in}}{\pgfqpoint{0.087753in}{0.183333in}}{\pgfqpoint{0.083333in}{0.183333in}}%
\pgfpathcurveto{\pgfqpoint{0.078913in}{0.183333in}}{\pgfqpoint{0.074674in}{0.181577in}}{\pgfqpoint{0.071548in}{0.178452in}}%
\pgfpathcurveto{\pgfqpoint{0.068423in}{0.175326in}}{\pgfqpoint{0.066667in}{0.171087in}}{\pgfqpoint{0.066667in}{0.166667in}}%
\pgfpathcurveto{\pgfqpoint{0.066667in}{0.162247in}}{\pgfqpoint{0.068423in}{0.158007in}}{\pgfqpoint{0.071548in}{0.154882in}}%
\pgfpathcurveto{\pgfqpoint{0.074674in}{0.151756in}}{\pgfqpoint{0.078913in}{0.150000in}}{\pgfqpoint{0.083333in}{0.150000in}}%
\pgfpathclose%
\pgfpathmoveto{\pgfqpoint{0.250000in}{0.150000in}}%
\pgfpathcurveto{\pgfqpoint{0.254420in}{0.150000in}}{\pgfqpoint{0.258660in}{0.151756in}}{\pgfqpoint{0.261785in}{0.154882in}}%
\pgfpathcurveto{\pgfqpoint{0.264911in}{0.158007in}}{\pgfqpoint{0.266667in}{0.162247in}}{\pgfqpoint{0.266667in}{0.166667in}}%
\pgfpathcurveto{\pgfqpoint{0.266667in}{0.171087in}}{\pgfqpoint{0.264911in}{0.175326in}}{\pgfqpoint{0.261785in}{0.178452in}}%
\pgfpathcurveto{\pgfqpoint{0.258660in}{0.181577in}}{\pgfqpoint{0.254420in}{0.183333in}}{\pgfqpoint{0.250000in}{0.183333in}}%
\pgfpathcurveto{\pgfqpoint{0.245580in}{0.183333in}}{\pgfqpoint{0.241340in}{0.181577in}}{\pgfqpoint{0.238215in}{0.178452in}}%
\pgfpathcurveto{\pgfqpoint{0.235089in}{0.175326in}}{\pgfqpoint{0.233333in}{0.171087in}}{\pgfqpoint{0.233333in}{0.166667in}}%
\pgfpathcurveto{\pgfqpoint{0.233333in}{0.162247in}}{\pgfqpoint{0.235089in}{0.158007in}}{\pgfqpoint{0.238215in}{0.154882in}}%
\pgfpathcurveto{\pgfqpoint{0.241340in}{0.151756in}}{\pgfqpoint{0.245580in}{0.150000in}}{\pgfqpoint{0.250000in}{0.150000in}}%
\pgfpathclose%
\pgfpathmoveto{\pgfqpoint{0.416667in}{0.150000in}}%
\pgfpathcurveto{\pgfqpoint{0.421087in}{0.150000in}}{\pgfqpoint{0.425326in}{0.151756in}}{\pgfqpoint{0.428452in}{0.154882in}}%
\pgfpathcurveto{\pgfqpoint{0.431577in}{0.158007in}}{\pgfqpoint{0.433333in}{0.162247in}}{\pgfqpoint{0.433333in}{0.166667in}}%
\pgfpathcurveto{\pgfqpoint{0.433333in}{0.171087in}}{\pgfqpoint{0.431577in}{0.175326in}}{\pgfqpoint{0.428452in}{0.178452in}}%
\pgfpathcurveto{\pgfqpoint{0.425326in}{0.181577in}}{\pgfqpoint{0.421087in}{0.183333in}}{\pgfqpoint{0.416667in}{0.183333in}}%
\pgfpathcurveto{\pgfqpoint{0.412247in}{0.183333in}}{\pgfqpoint{0.408007in}{0.181577in}}{\pgfqpoint{0.404882in}{0.178452in}}%
\pgfpathcurveto{\pgfqpoint{0.401756in}{0.175326in}}{\pgfqpoint{0.400000in}{0.171087in}}{\pgfqpoint{0.400000in}{0.166667in}}%
\pgfpathcurveto{\pgfqpoint{0.400000in}{0.162247in}}{\pgfqpoint{0.401756in}{0.158007in}}{\pgfqpoint{0.404882in}{0.154882in}}%
\pgfpathcurveto{\pgfqpoint{0.408007in}{0.151756in}}{\pgfqpoint{0.412247in}{0.150000in}}{\pgfqpoint{0.416667in}{0.150000in}}%
\pgfpathclose%
\pgfpathmoveto{\pgfqpoint{0.583333in}{0.150000in}}%
\pgfpathcurveto{\pgfqpoint{0.587753in}{0.150000in}}{\pgfqpoint{0.591993in}{0.151756in}}{\pgfqpoint{0.595118in}{0.154882in}}%
\pgfpathcurveto{\pgfqpoint{0.598244in}{0.158007in}}{\pgfqpoint{0.600000in}{0.162247in}}{\pgfqpoint{0.600000in}{0.166667in}}%
\pgfpathcurveto{\pgfqpoint{0.600000in}{0.171087in}}{\pgfqpoint{0.598244in}{0.175326in}}{\pgfqpoint{0.595118in}{0.178452in}}%
\pgfpathcurveto{\pgfqpoint{0.591993in}{0.181577in}}{\pgfqpoint{0.587753in}{0.183333in}}{\pgfqpoint{0.583333in}{0.183333in}}%
\pgfpathcurveto{\pgfqpoint{0.578913in}{0.183333in}}{\pgfqpoint{0.574674in}{0.181577in}}{\pgfqpoint{0.571548in}{0.178452in}}%
\pgfpathcurveto{\pgfqpoint{0.568423in}{0.175326in}}{\pgfqpoint{0.566667in}{0.171087in}}{\pgfqpoint{0.566667in}{0.166667in}}%
\pgfpathcurveto{\pgfqpoint{0.566667in}{0.162247in}}{\pgfqpoint{0.568423in}{0.158007in}}{\pgfqpoint{0.571548in}{0.154882in}}%
\pgfpathcurveto{\pgfqpoint{0.574674in}{0.151756in}}{\pgfqpoint{0.578913in}{0.150000in}}{\pgfqpoint{0.583333in}{0.150000in}}%
\pgfpathclose%
\pgfpathmoveto{\pgfqpoint{0.750000in}{0.150000in}}%
\pgfpathcurveto{\pgfqpoint{0.754420in}{0.150000in}}{\pgfqpoint{0.758660in}{0.151756in}}{\pgfqpoint{0.761785in}{0.154882in}}%
\pgfpathcurveto{\pgfqpoint{0.764911in}{0.158007in}}{\pgfqpoint{0.766667in}{0.162247in}}{\pgfqpoint{0.766667in}{0.166667in}}%
\pgfpathcurveto{\pgfqpoint{0.766667in}{0.171087in}}{\pgfqpoint{0.764911in}{0.175326in}}{\pgfqpoint{0.761785in}{0.178452in}}%
\pgfpathcurveto{\pgfqpoint{0.758660in}{0.181577in}}{\pgfqpoint{0.754420in}{0.183333in}}{\pgfqpoint{0.750000in}{0.183333in}}%
\pgfpathcurveto{\pgfqpoint{0.745580in}{0.183333in}}{\pgfqpoint{0.741340in}{0.181577in}}{\pgfqpoint{0.738215in}{0.178452in}}%
\pgfpathcurveto{\pgfqpoint{0.735089in}{0.175326in}}{\pgfqpoint{0.733333in}{0.171087in}}{\pgfqpoint{0.733333in}{0.166667in}}%
\pgfpathcurveto{\pgfqpoint{0.733333in}{0.162247in}}{\pgfqpoint{0.735089in}{0.158007in}}{\pgfqpoint{0.738215in}{0.154882in}}%
\pgfpathcurveto{\pgfqpoint{0.741340in}{0.151756in}}{\pgfqpoint{0.745580in}{0.150000in}}{\pgfqpoint{0.750000in}{0.150000in}}%
\pgfpathclose%
\pgfpathmoveto{\pgfqpoint{0.916667in}{0.150000in}}%
\pgfpathcurveto{\pgfqpoint{0.921087in}{0.150000in}}{\pgfqpoint{0.925326in}{0.151756in}}{\pgfqpoint{0.928452in}{0.154882in}}%
\pgfpathcurveto{\pgfqpoint{0.931577in}{0.158007in}}{\pgfqpoint{0.933333in}{0.162247in}}{\pgfqpoint{0.933333in}{0.166667in}}%
\pgfpathcurveto{\pgfqpoint{0.933333in}{0.171087in}}{\pgfqpoint{0.931577in}{0.175326in}}{\pgfqpoint{0.928452in}{0.178452in}}%
\pgfpathcurveto{\pgfqpoint{0.925326in}{0.181577in}}{\pgfqpoint{0.921087in}{0.183333in}}{\pgfqpoint{0.916667in}{0.183333in}}%
\pgfpathcurveto{\pgfqpoint{0.912247in}{0.183333in}}{\pgfqpoint{0.908007in}{0.181577in}}{\pgfqpoint{0.904882in}{0.178452in}}%
\pgfpathcurveto{\pgfqpoint{0.901756in}{0.175326in}}{\pgfqpoint{0.900000in}{0.171087in}}{\pgfqpoint{0.900000in}{0.166667in}}%
\pgfpathcurveto{\pgfqpoint{0.900000in}{0.162247in}}{\pgfqpoint{0.901756in}{0.158007in}}{\pgfqpoint{0.904882in}{0.154882in}}%
\pgfpathcurveto{\pgfqpoint{0.908007in}{0.151756in}}{\pgfqpoint{0.912247in}{0.150000in}}{\pgfqpoint{0.916667in}{0.150000in}}%
\pgfpathclose%
\pgfpathmoveto{\pgfqpoint{0.000000in}{0.316667in}}%
\pgfpathcurveto{\pgfqpoint{0.004420in}{0.316667in}}{\pgfqpoint{0.008660in}{0.318423in}}{\pgfqpoint{0.011785in}{0.321548in}}%
\pgfpathcurveto{\pgfqpoint{0.014911in}{0.324674in}}{\pgfqpoint{0.016667in}{0.328913in}}{\pgfqpoint{0.016667in}{0.333333in}}%
\pgfpathcurveto{\pgfqpoint{0.016667in}{0.337753in}}{\pgfqpoint{0.014911in}{0.341993in}}{\pgfqpoint{0.011785in}{0.345118in}}%
\pgfpathcurveto{\pgfqpoint{0.008660in}{0.348244in}}{\pgfqpoint{0.004420in}{0.350000in}}{\pgfqpoint{0.000000in}{0.350000in}}%
\pgfpathcurveto{\pgfqpoint{-0.004420in}{0.350000in}}{\pgfqpoint{-0.008660in}{0.348244in}}{\pgfqpoint{-0.011785in}{0.345118in}}%
\pgfpathcurveto{\pgfqpoint{-0.014911in}{0.341993in}}{\pgfqpoint{-0.016667in}{0.337753in}}{\pgfqpoint{-0.016667in}{0.333333in}}%
\pgfpathcurveto{\pgfqpoint{-0.016667in}{0.328913in}}{\pgfqpoint{-0.014911in}{0.324674in}}{\pgfqpoint{-0.011785in}{0.321548in}}%
\pgfpathcurveto{\pgfqpoint{-0.008660in}{0.318423in}}{\pgfqpoint{-0.004420in}{0.316667in}}{\pgfqpoint{0.000000in}{0.316667in}}%
\pgfpathclose%
\pgfpathmoveto{\pgfqpoint{0.166667in}{0.316667in}}%
\pgfpathcurveto{\pgfqpoint{0.171087in}{0.316667in}}{\pgfqpoint{0.175326in}{0.318423in}}{\pgfqpoint{0.178452in}{0.321548in}}%
\pgfpathcurveto{\pgfqpoint{0.181577in}{0.324674in}}{\pgfqpoint{0.183333in}{0.328913in}}{\pgfqpoint{0.183333in}{0.333333in}}%
\pgfpathcurveto{\pgfqpoint{0.183333in}{0.337753in}}{\pgfqpoint{0.181577in}{0.341993in}}{\pgfqpoint{0.178452in}{0.345118in}}%
\pgfpathcurveto{\pgfqpoint{0.175326in}{0.348244in}}{\pgfqpoint{0.171087in}{0.350000in}}{\pgfqpoint{0.166667in}{0.350000in}}%
\pgfpathcurveto{\pgfqpoint{0.162247in}{0.350000in}}{\pgfqpoint{0.158007in}{0.348244in}}{\pgfqpoint{0.154882in}{0.345118in}}%
\pgfpathcurveto{\pgfqpoint{0.151756in}{0.341993in}}{\pgfqpoint{0.150000in}{0.337753in}}{\pgfqpoint{0.150000in}{0.333333in}}%
\pgfpathcurveto{\pgfqpoint{0.150000in}{0.328913in}}{\pgfqpoint{0.151756in}{0.324674in}}{\pgfqpoint{0.154882in}{0.321548in}}%
\pgfpathcurveto{\pgfqpoint{0.158007in}{0.318423in}}{\pgfqpoint{0.162247in}{0.316667in}}{\pgfqpoint{0.166667in}{0.316667in}}%
\pgfpathclose%
\pgfpathmoveto{\pgfqpoint{0.333333in}{0.316667in}}%
\pgfpathcurveto{\pgfqpoint{0.337753in}{0.316667in}}{\pgfqpoint{0.341993in}{0.318423in}}{\pgfqpoint{0.345118in}{0.321548in}}%
\pgfpathcurveto{\pgfqpoint{0.348244in}{0.324674in}}{\pgfqpoint{0.350000in}{0.328913in}}{\pgfqpoint{0.350000in}{0.333333in}}%
\pgfpathcurveto{\pgfqpoint{0.350000in}{0.337753in}}{\pgfqpoint{0.348244in}{0.341993in}}{\pgfqpoint{0.345118in}{0.345118in}}%
\pgfpathcurveto{\pgfqpoint{0.341993in}{0.348244in}}{\pgfqpoint{0.337753in}{0.350000in}}{\pgfqpoint{0.333333in}{0.350000in}}%
\pgfpathcurveto{\pgfqpoint{0.328913in}{0.350000in}}{\pgfqpoint{0.324674in}{0.348244in}}{\pgfqpoint{0.321548in}{0.345118in}}%
\pgfpathcurveto{\pgfqpoint{0.318423in}{0.341993in}}{\pgfqpoint{0.316667in}{0.337753in}}{\pgfqpoint{0.316667in}{0.333333in}}%
\pgfpathcurveto{\pgfqpoint{0.316667in}{0.328913in}}{\pgfqpoint{0.318423in}{0.324674in}}{\pgfqpoint{0.321548in}{0.321548in}}%
\pgfpathcurveto{\pgfqpoint{0.324674in}{0.318423in}}{\pgfqpoint{0.328913in}{0.316667in}}{\pgfqpoint{0.333333in}{0.316667in}}%
\pgfpathclose%
\pgfpathmoveto{\pgfqpoint{0.500000in}{0.316667in}}%
\pgfpathcurveto{\pgfqpoint{0.504420in}{0.316667in}}{\pgfqpoint{0.508660in}{0.318423in}}{\pgfqpoint{0.511785in}{0.321548in}}%
\pgfpathcurveto{\pgfqpoint{0.514911in}{0.324674in}}{\pgfqpoint{0.516667in}{0.328913in}}{\pgfqpoint{0.516667in}{0.333333in}}%
\pgfpathcurveto{\pgfqpoint{0.516667in}{0.337753in}}{\pgfqpoint{0.514911in}{0.341993in}}{\pgfqpoint{0.511785in}{0.345118in}}%
\pgfpathcurveto{\pgfqpoint{0.508660in}{0.348244in}}{\pgfqpoint{0.504420in}{0.350000in}}{\pgfqpoint{0.500000in}{0.350000in}}%
\pgfpathcurveto{\pgfqpoint{0.495580in}{0.350000in}}{\pgfqpoint{0.491340in}{0.348244in}}{\pgfqpoint{0.488215in}{0.345118in}}%
\pgfpathcurveto{\pgfqpoint{0.485089in}{0.341993in}}{\pgfqpoint{0.483333in}{0.337753in}}{\pgfqpoint{0.483333in}{0.333333in}}%
\pgfpathcurveto{\pgfqpoint{0.483333in}{0.328913in}}{\pgfqpoint{0.485089in}{0.324674in}}{\pgfqpoint{0.488215in}{0.321548in}}%
\pgfpathcurveto{\pgfqpoint{0.491340in}{0.318423in}}{\pgfqpoint{0.495580in}{0.316667in}}{\pgfqpoint{0.500000in}{0.316667in}}%
\pgfpathclose%
\pgfpathmoveto{\pgfqpoint{0.666667in}{0.316667in}}%
\pgfpathcurveto{\pgfqpoint{0.671087in}{0.316667in}}{\pgfqpoint{0.675326in}{0.318423in}}{\pgfqpoint{0.678452in}{0.321548in}}%
\pgfpathcurveto{\pgfqpoint{0.681577in}{0.324674in}}{\pgfqpoint{0.683333in}{0.328913in}}{\pgfqpoint{0.683333in}{0.333333in}}%
\pgfpathcurveto{\pgfqpoint{0.683333in}{0.337753in}}{\pgfqpoint{0.681577in}{0.341993in}}{\pgfqpoint{0.678452in}{0.345118in}}%
\pgfpathcurveto{\pgfqpoint{0.675326in}{0.348244in}}{\pgfqpoint{0.671087in}{0.350000in}}{\pgfqpoint{0.666667in}{0.350000in}}%
\pgfpathcurveto{\pgfqpoint{0.662247in}{0.350000in}}{\pgfqpoint{0.658007in}{0.348244in}}{\pgfqpoint{0.654882in}{0.345118in}}%
\pgfpathcurveto{\pgfqpoint{0.651756in}{0.341993in}}{\pgfqpoint{0.650000in}{0.337753in}}{\pgfqpoint{0.650000in}{0.333333in}}%
\pgfpathcurveto{\pgfqpoint{0.650000in}{0.328913in}}{\pgfqpoint{0.651756in}{0.324674in}}{\pgfqpoint{0.654882in}{0.321548in}}%
\pgfpathcurveto{\pgfqpoint{0.658007in}{0.318423in}}{\pgfqpoint{0.662247in}{0.316667in}}{\pgfqpoint{0.666667in}{0.316667in}}%
\pgfpathclose%
\pgfpathmoveto{\pgfqpoint{0.833333in}{0.316667in}}%
\pgfpathcurveto{\pgfqpoint{0.837753in}{0.316667in}}{\pgfqpoint{0.841993in}{0.318423in}}{\pgfqpoint{0.845118in}{0.321548in}}%
\pgfpathcurveto{\pgfqpoint{0.848244in}{0.324674in}}{\pgfqpoint{0.850000in}{0.328913in}}{\pgfqpoint{0.850000in}{0.333333in}}%
\pgfpathcurveto{\pgfqpoint{0.850000in}{0.337753in}}{\pgfqpoint{0.848244in}{0.341993in}}{\pgfqpoint{0.845118in}{0.345118in}}%
\pgfpathcurveto{\pgfqpoint{0.841993in}{0.348244in}}{\pgfqpoint{0.837753in}{0.350000in}}{\pgfqpoint{0.833333in}{0.350000in}}%
\pgfpathcurveto{\pgfqpoint{0.828913in}{0.350000in}}{\pgfqpoint{0.824674in}{0.348244in}}{\pgfqpoint{0.821548in}{0.345118in}}%
\pgfpathcurveto{\pgfqpoint{0.818423in}{0.341993in}}{\pgfqpoint{0.816667in}{0.337753in}}{\pgfqpoint{0.816667in}{0.333333in}}%
\pgfpathcurveto{\pgfqpoint{0.816667in}{0.328913in}}{\pgfqpoint{0.818423in}{0.324674in}}{\pgfqpoint{0.821548in}{0.321548in}}%
\pgfpathcurveto{\pgfqpoint{0.824674in}{0.318423in}}{\pgfqpoint{0.828913in}{0.316667in}}{\pgfqpoint{0.833333in}{0.316667in}}%
\pgfpathclose%
\pgfpathmoveto{\pgfqpoint{1.000000in}{0.316667in}}%
\pgfpathcurveto{\pgfqpoint{1.004420in}{0.316667in}}{\pgfqpoint{1.008660in}{0.318423in}}{\pgfqpoint{1.011785in}{0.321548in}}%
\pgfpathcurveto{\pgfqpoint{1.014911in}{0.324674in}}{\pgfqpoint{1.016667in}{0.328913in}}{\pgfqpoint{1.016667in}{0.333333in}}%
\pgfpathcurveto{\pgfqpoint{1.016667in}{0.337753in}}{\pgfqpoint{1.014911in}{0.341993in}}{\pgfqpoint{1.011785in}{0.345118in}}%
\pgfpathcurveto{\pgfqpoint{1.008660in}{0.348244in}}{\pgfqpoint{1.004420in}{0.350000in}}{\pgfqpoint{1.000000in}{0.350000in}}%
\pgfpathcurveto{\pgfqpoint{0.995580in}{0.350000in}}{\pgfqpoint{0.991340in}{0.348244in}}{\pgfqpoint{0.988215in}{0.345118in}}%
\pgfpathcurveto{\pgfqpoint{0.985089in}{0.341993in}}{\pgfqpoint{0.983333in}{0.337753in}}{\pgfqpoint{0.983333in}{0.333333in}}%
\pgfpathcurveto{\pgfqpoint{0.983333in}{0.328913in}}{\pgfqpoint{0.985089in}{0.324674in}}{\pgfqpoint{0.988215in}{0.321548in}}%
\pgfpathcurveto{\pgfqpoint{0.991340in}{0.318423in}}{\pgfqpoint{0.995580in}{0.316667in}}{\pgfqpoint{1.000000in}{0.316667in}}%
\pgfpathclose%
\pgfpathmoveto{\pgfqpoint{0.083333in}{0.483333in}}%
\pgfpathcurveto{\pgfqpoint{0.087753in}{0.483333in}}{\pgfqpoint{0.091993in}{0.485089in}}{\pgfqpoint{0.095118in}{0.488215in}}%
\pgfpathcurveto{\pgfqpoint{0.098244in}{0.491340in}}{\pgfqpoint{0.100000in}{0.495580in}}{\pgfqpoint{0.100000in}{0.500000in}}%
\pgfpathcurveto{\pgfqpoint{0.100000in}{0.504420in}}{\pgfqpoint{0.098244in}{0.508660in}}{\pgfqpoint{0.095118in}{0.511785in}}%
\pgfpathcurveto{\pgfqpoint{0.091993in}{0.514911in}}{\pgfqpoint{0.087753in}{0.516667in}}{\pgfqpoint{0.083333in}{0.516667in}}%
\pgfpathcurveto{\pgfqpoint{0.078913in}{0.516667in}}{\pgfqpoint{0.074674in}{0.514911in}}{\pgfqpoint{0.071548in}{0.511785in}}%
\pgfpathcurveto{\pgfqpoint{0.068423in}{0.508660in}}{\pgfqpoint{0.066667in}{0.504420in}}{\pgfqpoint{0.066667in}{0.500000in}}%
\pgfpathcurveto{\pgfqpoint{0.066667in}{0.495580in}}{\pgfqpoint{0.068423in}{0.491340in}}{\pgfqpoint{0.071548in}{0.488215in}}%
\pgfpathcurveto{\pgfqpoint{0.074674in}{0.485089in}}{\pgfqpoint{0.078913in}{0.483333in}}{\pgfqpoint{0.083333in}{0.483333in}}%
\pgfpathclose%
\pgfpathmoveto{\pgfqpoint{0.250000in}{0.483333in}}%
\pgfpathcurveto{\pgfqpoint{0.254420in}{0.483333in}}{\pgfqpoint{0.258660in}{0.485089in}}{\pgfqpoint{0.261785in}{0.488215in}}%
\pgfpathcurveto{\pgfqpoint{0.264911in}{0.491340in}}{\pgfqpoint{0.266667in}{0.495580in}}{\pgfqpoint{0.266667in}{0.500000in}}%
\pgfpathcurveto{\pgfqpoint{0.266667in}{0.504420in}}{\pgfqpoint{0.264911in}{0.508660in}}{\pgfqpoint{0.261785in}{0.511785in}}%
\pgfpathcurveto{\pgfqpoint{0.258660in}{0.514911in}}{\pgfqpoint{0.254420in}{0.516667in}}{\pgfqpoint{0.250000in}{0.516667in}}%
\pgfpathcurveto{\pgfqpoint{0.245580in}{0.516667in}}{\pgfqpoint{0.241340in}{0.514911in}}{\pgfqpoint{0.238215in}{0.511785in}}%
\pgfpathcurveto{\pgfqpoint{0.235089in}{0.508660in}}{\pgfqpoint{0.233333in}{0.504420in}}{\pgfqpoint{0.233333in}{0.500000in}}%
\pgfpathcurveto{\pgfqpoint{0.233333in}{0.495580in}}{\pgfqpoint{0.235089in}{0.491340in}}{\pgfqpoint{0.238215in}{0.488215in}}%
\pgfpathcurveto{\pgfqpoint{0.241340in}{0.485089in}}{\pgfqpoint{0.245580in}{0.483333in}}{\pgfqpoint{0.250000in}{0.483333in}}%
\pgfpathclose%
\pgfpathmoveto{\pgfqpoint{0.416667in}{0.483333in}}%
\pgfpathcurveto{\pgfqpoint{0.421087in}{0.483333in}}{\pgfqpoint{0.425326in}{0.485089in}}{\pgfqpoint{0.428452in}{0.488215in}}%
\pgfpathcurveto{\pgfqpoint{0.431577in}{0.491340in}}{\pgfqpoint{0.433333in}{0.495580in}}{\pgfqpoint{0.433333in}{0.500000in}}%
\pgfpathcurveto{\pgfqpoint{0.433333in}{0.504420in}}{\pgfqpoint{0.431577in}{0.508660in}}{\pgfqpoint{0.428452in}{0.511785in}}%
\pgfpathcurveto{\pgfqpoint{0.425326in}{0.514911in}}{\pgfqpoint{0.421087in}{0.516667in}}{\pgfqpoint{0.416667in}{0.516667in}}%
\pgfpathcurveto{\pgfqpoint{0.412247in}{0.516667in}}{\pgfqpoint{0.408007in}{0.514911in}}{\pgfqpoint{0.404882in}{0.511785in}}%
\pgfpathcurveto{\pgfqpoint{0.401756in}{0.508660in}}{\pgfqpoint{0.400000in}{0.504420in}}{\pgfqpoint{0.400000in}{0.500000in}}%
\pgfpathcurveto{\pgfqpoint{0.400000in}{0.495580in}}{\pgfqpoint{0.401756in}{0.491340in}}{\pgfqpoint{0.404882in}{0.488215in}}%
\pgfpathcurveto{\pgfqpoint{0.408007in}{0.485089in}}{\pgfqpoint{0.412247in}{0.483333in}}{\pgfqpoint{0.416667in}{0.483333in}}%
\pgfpathclose%
\pgfpathmoveto{\pgfqpoint{0.583333in}{0.483333in}}%
\pgfpathcurveto{\pgfqpoint{0.587753in}{0.483333in}}{\pgfqpoint{0.591993in}{0.485089in}}{\pgfqpoint{0.595118in}{0.488215in}}%
\pgfpathcurveto{\pgfqpoint{0.598244in}{0.491340in}}{\pgfqpoint{0.600000in}{0.495580in}}{\pgfqpoint{0.600000in}{0.500000in}}%
\pgfpathcurveto{\pgfqpoint{0.600000in}{0.504420in}}{\pgfqpoint{0.598244in}{0.508660in}}{\pgfqpoint{0.595118in}{0.511785in}}%
\pgfpathcurveto{\pgfqpoint{0.591993in}{0.514911in}}{\pgfqpoint{0.587753in}{0.516667in}}{\pgfqpoint{0.583333in}{0.516667in}}%
\pgfpathcurveto{\pgfqpoint{0.578913in}{0.516667in}}{\pgfqpoint{0.574674in}{0.514911in}}{\pgfqpoint{0.571548in}{0.511785in}}%
\pgfpathcurveto{\pgfqpoint{0.568423in}{0.508660in}}{\pgfqpoint{0.566667in}{0.504420in}}{\pgfqpoint{0.566667in}{0.500000in}}%
\pgfpathcurveto{\pgfqpoint{0.566667in}{0.495580in}}{\pgfqpoint{0.568423in}{0.491340in}}{\pgfqpoint{0.571548in}{0.488215in}}%
\pgfpathcurveto{\pgfqpoint{0.574674in}{0.485089in}}{\pgfqpoint{0.578913in}{0.483333in}}{\pgfqpoint{0.583333in}{0.483333in}}%
\pgfpathclose%
\pgfpathmoveto{\pgfqpoint{0.750000in}{0.483333in}}%
\pgfpathcurveto{\pgfqpoint{0.754420in}{0.483333in}}{\pgfqpoint{0.758660in}{0.485089in}}{\pgfqpoint{0.761785in}{0.488215in}}%
\pgfpathcurveto{\pgfqpoint{0.764911in}{0.491340in}}{\pgfqpoint{0.766667in}{0.495580in}}{\pgfqpoint{0.766667in}{0.500000in}}%
\pgfpathcurveto{\pgfqpoint{0.766667in}{0.504420in}}{\pgfqpoint{0.764911in}{0.508660in}}{\pgfqpoint{0.761785in}{0.511785in}}%
\pgfpathcurveto{\pgfqpoint{0.758660in}{0.514911in}}{\pgfqpoint{0.754420in}{0.516667in}}{\pgfqpoint{0.750000in}{0.516667in}}%
\pgfpathcurveto{\pgfqpoint{0.745580in}{0.516667in}}{\pgfqpoint{0.741340in}{0.514911in}}{\pgfqpoint{0.738215in}{0.511785in}}%
\pgfpathcurveto{\pgfqpoint{0.735089in}{0.508660in}}{\pgfqpoint{0.733333in}{0.504420in}}{\pgfqpoint{0.733333in}{0.500000in}}%
\pgfpathcurveto{\pgfqpoint{0.733333in}{0.495580in}}{\pgfqpoint{0.735089in}{0.491340in}}{\pgfqpoint{0.738215in}{0.488215in}}%
\pgfpathcurveto{\pgfqpoint{0.741340in}{0.485089in}}{\pgfqpoint{0.745580in}{0.483333in}}{\pgfqpoint{0.750000in}{0.483333in}}%
\pgfpathclose%
\pgfpathmoveto{\pgfqpoint{0.916667in}{0.483333in}}%
\pgfpathcurveto{\pgfqpoint{0.921087in}{0.483333in}}{\pgfqpoint{0.925326in}{0.485089in}}{\pgfqpoint{0.928452in}{0.488215in}}%
\pgfpathcurveto{\pgfqpoint{0.931577in}{0.491340in}}{\pgfqpoint{0.933333in}{0.495580in}}{\pgfqpoint{0.933333in}{0.500000in}}%
\pgfpathcurveto{\pgfqpoint{0.933333in}{0.504420in}}{\pgfqpoint{0.931577in}{0.508660in}}{\pgfqpoint{0.928452in}{0.511785in}}%
\pgfpathcurveto{\pgfqpoint{0.925326in}{0.514911in}}{\pgfqpoint{0.921087in}{0.516667in}}{\pgfqpoint{0.916667in}{0.516667in}}%
\pgfpathcurveto{\pgfqpoint{0.912247in}{0.516667in}}{\pgfqpoint{0.908007in}{0.514911in}}{\pgfqpoint{0.904882in}{0.511785in}}%
\pgfpathcurveto{\pgfqpoint{0.901756in}{0.508660in}}{\pgfqpoint{0.900000in}{0.504420in}}{\pgfqpoint{0.900000in}{0.500000in}}%
\pgfpathcurveto{\pgfqpoint{0.900000in}{0.495580in}}{\pgfqpoint{0.901756in}{0.491340in}}{\pgfqpoint{0.904882in}{0.488215in}}%
\pgfpathcurveto{\pgfqpoint{0.908007in}{0.485089in}}{\pgfqpoint{0.912247in}{0.483333in}}{\pgfqpoint{0.916667in}{0.483333in}}%
\pgfpathclose%
\pgfpathmoveto{\pgfqpoint{0.000000in}{0.650000in}}%
\pgfpathcurveto{\pgfqpoint{0.004420in}{0.650000in}}{\pgfqpoint{0.008660in}{0.651756in}}{\pgfqpoint{0.011785in}{0.654882in}}%
\pgfpathcurveto{\pgfqpoint{0.014911in}{0.658007in}}{\pgfqpoint{0.016667in}{0.662247in}}{\pgfqpoint{0.016667in}{0.666667in}}%
\pgfpathcurveto{\pgfqpoint{0.016667in}{0.671087in}}{\pgfqpoint{0.014911in}{0.675326in}}{\pgfqpoint{0.011785in}{0.678452in}}%
\pgfpathcurveto{\pgfqpoint{0.008660in}{0.681577in}}{\pgfqpoint{0.004420in}{0.683333in}}{\pgfqpoint{0.000000in}{0.683333in}}%
\pgfpathcurveto{\pgfqpoint{-0.004420in}{0.683333in}}{\pgfqpoint{-0.008660in}{0.681577in}}{\pgfqpoint{-0.011785in}{0.678452in}}%
\pgfpathcurveto{\pgfqpoint{-0.014911in}{0.675326in}}{\pgfqpoint{-0.016667in}{0.671087in}}{\pgfqpoint{-0.016667in}{0.666667in}}%
\pgfpathcurveto{\pgfqpoint{-0.016667in}{0.662247in}}{\pgfqpoint{-0.014911in}{0.658007in}}{\pgfqpoint{-0.011785in}{0.654882in}}%
\pgfpathcurveto{\pgfqpoint{-0.008660in}{0.651756in}}{\pgfqpoint{-0.004420in}{0.650000in}}{\pgfqpoint{0.000000in}{0.650000in}}%
\pgfpathclose%
\pgfpathmoveto{\pgfqpoint{0.166667in}{0.650000in}}%
\pgfpathcurveto{\pgfqpoint{0.171087in}{0.650000in}}{\pgfqpoint{0.175326in}{0.651756in}}{\pgfqpoint{0.178452in}{0.654882in}}%
\pgfpathcurveto{\pgfqpoint{0.181577in}{0.658007in}}{\pgfqpoint{0.183333in}{0.662247in}}{\pgfqpoint{0.183333in}{0.666667in}}%
\pgfpathcurveto{\pgfqpoint{0.183333in}{0.671087in}}{\pgfqpoint{0.181577in}{0.675326in}}{\pgfqpoint{0.178452in}{0.678452in}}%
\pgfpathcurveto{\pgfqpoint{0.175326in}{0.681577in}}{\pgfqpoint{0.171087in}{0.683333in}}{\pgfqpoint{0.166667in}{0.683333in}}%
\pgfpathcurveto{\pgfqpoint{0.162247in}{0.683333in}}{\pgfqpoint{0.158007in}{0.681577in}}{\pgfqpoint{0.154882in}{0.678452in}}%
\pgfpathcurveto{\pgfqpoint{0.151756in}{0.675326in}}{\pgfqpoint{0.150000in}{0.671087in}}{\pgfqpoint{0.150000in}{0.666667in}}%
\pgfpathcurveto{\pgfqpoint{0.150000in}{0.662247in}}{\pgfqpoint{0.151756in}{0.658007in}}{\pgfqpoint{0.154882in}{0.654882in}}%
\pgfpathcurveto{\pgfqpoint{0.158007in}{0.651756in}}{\pgfqpoint{0.162247in}{0.650000in}}{\pgfqpoint{0.166667in}{0.650000in}}%
\pgfpathclose%
\pgfpathmoveto{\pgfqpoint{0.333333in}{0.650000in}}%
\pgfpathcurveto{\pgfqpoint{0.337753in}{0.650000in}}{\pgfqpoint{0.341993in}{0.651756in}}{\pgfqpoint{0.345118in}{0.654882in}}%
\pgfpathcurveto{\pgfqpoint{0.348244in}{0.658007in}}{\pgfqpoint{0.350000in}{0.662247in}}{\pgfqpoint{0.350000in}{0.666667in}}%
\pgfpathcurveto{\pgfqpoint{0.350000in}{0.671087in}}{\pgfqpoint{0.348244in}{0.675326in}}{\pgfqpoint{0.345118in}{0.678452in}}%
\pgfpathcurveto{\pgfqpoint{0.341993in}{0.681577in}}{\pgfqpoint{0.337753in}{0.683333in}}{\pgfqpoint{0.333333in}{0.683333in}}%
\pgfpathcurveto{\pgfqpoint{0.328913in}{0.683333in}}{\pgfqpoint{0.324674in}{0.681577in}}{\pgfqpoint{0.321548in}{0.678452in}}%
\pgfpathcurveto{\pgfqpoint{0.318423in}{0.675326in}}{\pgfqpoint{0.316667in}{0.671087in}}{\pgfqpoint{0.316667in}{0.666667in}}%
\pgfpathcurveto{\pgfqpoint{0.316667in}{0.662247in}}{\pgfqpoint{0.318423in}{0.658007in}}{\pgfqpoint{0.321548in}{0.654882in}}%
\pgfpathcurveto{\pgfqpoint{0.324674in}{0.651756in}}{\pgfqpoint{0.328913in}{0.650000in}}{\pgfqpoint{0.333333in}{0.650000in}}%
\pgfpathclose%
\pgfpathmoveto{\pgfqpoint{0.500000in}{0.650000in}}%
\pgfpathcurveto{\pgfqpoint{0.504420in}{0.650000in}}{\pgfqpoint{0.508660in}{0.651756in}}{\pgfqpoint{0.511785in}{0.654882in}}%
\pgfpathcurveto{\pgfqpoint{0.514911in}{0.658007in}}{\pgfqpoint{0.516667in}{0.662247in}}{\pgfqpoint{0.516667in}{0.666667in}}%
\pgfpathcurveto{\pgfqpoint{0.516667in}{0.671087in}}{\pgfqpoint{0.514911in}{0.675326in}}{\pgfqpoint{0.511785in}{0.678452in}}%
\pgfpathcurveto{\pgfqpoint{0.508660in}{0.681577in}}{\pgfqpoint{0.504420in}{0.683333in}}{\pgfqpoint{0.500000in}{0.683333in}}%
\pgfpathcurveto{\pgfqpoint{0.495580in}{0.683333in}}{\pgfqpoint{0.491340in}{0.681577in}}{\pgfqpoint{0.488215in}{0.678452in}}%
\pgfpathcurveto{\pgfqpoint{0.485089in}{0.675326in}}{\pgfqpoint{0.483333in}{0.671087in}}{\pgfqpoint{0.483333in}{0.666667in}}%
\pgfpathcurveto{\pgfqpoint{0.483333in}{0.662247in}}{\pgfqpoint{0.485089in}{0.658007in}}{\pgfqpoint{0.488215in}{0.654882in}}%
\pgfpathcurveto{\pgfqpoint{0.491340in}{0.651756in}}{\pgfqpoint{0.495580in}{0.650000in}}{\pgfqpoint{0.500000in}{0.650000in}}%
\pgfpathclose%
\pgfpathmoveto{\pgfqpoint{0.666667in}{0.650000in}}%
\pgfpathcurveto{\pgfqpoint{0.671087in}{0.650000in}}{\pgfqpoint{0.675326in}{0.651756in}}{\pgfqpoint{0.678452in}{0.654882in}}%
\pgfpathcurveto{\pgfqpoint{0.681577in}{0.658007in}}{\pgfqpoint{0.683333in}{0.662247in}}{\pgfqpoint{0.683333in}{0.666667in}}%
\pgfpathcurveto{\pgfqpoint{0.683333in}{0.671087in}}{\pgfqpoint{0.681577in}{0.675326in}}{\pgfqpoint{0.678452in}{0.678452in}}%
\pgfpathcurveto{\pgfqpoint{0.675326in}{0.681577in}}{\pgfqpoint{0.671087in}{0.683333in}}{\pgfqpoint{0.666667in}{0.683333in}}%
\pgfpathcurveto{\pgfqpoint{0.662247in}{0.683333in}}{\pgfqpoint{0.658007in}{0.681577in}}{\pgfqpoint{0.654882in}{0.678452in}}%
\pgfpathcurveto{\pgfqpoint{0.651756in}{0.675326in}}{\pgfqpoint{0.650000in}{0.671087in}}{\pgfqpoint{0.650000in}{0.666667in}}%
\pgfpathcurveto{\pgfqpoint{0.650000in}{0.662247in}}{\pgfqpoint{0.651756in}{0.658007in}}{\pgfqpoint{0.654882in}{0.654882in}}%
\pgfpathcurveto{\pgfqpoint{0.658007in}{0.651756in}}{\pgfqpoint{0.662247in}{0.650000in}}{\pgfqpoint{0.666667in}{0.650000in}}%
\pgfpathclose%
\pgfpathmoveto{\pgfqpoint{0.833333in}{0.650000in}}%
\pgfpathcurveto{\pgfqpoint{0.837753in}{0.650000in}}{\pgfqpoint{0.841993in}{0.651756in}}{\pgfqpoint{0.845118in}{0.654882in}}%
\pgfpathcurveto{\pgfqpoint{0.848244in}{0.658007in}}{\pgfqpoint{0.850000in}{0.662247in}}{\pgfqpoint{0.850000in}{0.666667in}}%
\pgfpathcurveto{\pgfqpoint{0.850000in}{0.671087in}}{\pgfqpoint{0.848244in}{0.675326in}}{\pgfqpoint{0.845118in}{0.678452in}}%
\pgfpathcurveto{\pgfqpoint{0.841993in}{0.681577in}}{\pgfqpoint{0.837753in}{0.683333in}}{\pgfqpoint{0.833333in}{0.683333in}}%
\pgfpathcurveto{\pgfqpoint{0.828913in}{0.683333in}}{\pgfqpoint{0.824674in}{0.681577in}}{\pgfqpoint{0.821548in}{0.678452in}}%
\pgfpathcurveto{\pgfqpoint{0.818423in}{0.675326in}}{\pgfqpoint{0.816667in}{0.671087in}}{\pgfqpoint{0.816667in}{0.666667in}}%
\pgfpathcurveto{\pgfqpoint{0.816667in}{0.662247in}}{\pgfqpoint{0.818423in}{0.658007in}}{\pgfqpoint{0.821548in}{0.654882in}}%
\pgfpathcurveto{\pgfqpoint{0.824674in}{0.651756in}}{\pgfqpoint{0.828913in}{0.650000in}}{\pgfqpoint{0.833333in}{0.650000in}}%
\pgfpathclose%
\pgfpathmoveto{\pgfqpoint{1.000000in}{0.650000in}}%
\pgfpathcurveto{\pgfqpoint{1.004420in}{0.650000in}}{\pgfqpoint{1.008660in}{0.651756in}}{\pgfqpoint{1.011785in}{0.654882in}}%
\pgfpathcurveto{\pgfqpoint{1.014911in}{0.658007in}}{\pgfqpoint{1.016667in}{0.662247in}}{\pgfqpoint{1.016667in}{0.666667in}}%
\pgfpathcurveto{\pgfqpoint{1.016667in}{0.671087in}}{\pgfqpoint{1.014911in}{0.675326in}}{\pgfqpoint{1.011785in}{0.678452in}}%
\pgfpathcurveto{\pgfqpoint{1.008660in}{0.681577in}}{\pgfqpoint{1.004420in}{0.683333in}}{\pgfqpoint{1.000000in}{0.683333in}}%
\pgfpathcurveto{\pgfqpoint{0.995580in}{0.683333in}}{\pgfqpoint{0.991340in}{0.681577in}}{\pgfqpoint{0.988215in}{0.678452in}}%
\pgfpathcurveto{\pgfqpoint{0.985089in}{0.675326in}}{\pgfqpoint{0.983333in}{0.671087in}}{\pgfqpoint{0.983333in}{0.666667in}}%
\pgfpathcurveto{\pgfqpoint{0.983333in}{0.662247in}}{\pgfqpoint{0.985089in}{0.658007in}}{\pgfqpoint{0.988215in}{0.654882in}}%
\pgfpathcurveto{\pgfqpoint{0.991340in}{0.651756in}}{\pgfqpoint{0.995580in}{0.650000in}}{\pgfqpoint{1.000000in}{0.650000in}}%
\pgfpathclose%
\pgfpathmoveto{\pgfqpoint{0.083333in}{0.816667in}}%
\pgfpathcurveto{\pgfqpoint{0.087753in}{0.816667in}}{\pgfqpoint{0.091993in}{0.818423in}}{\pgfqpoint{0.095118in}{0.821548in}}%
\pgfpathcurveto{\pgfqpoint{0.098244in}{0.824674in}}{\pgfqpoint{0.100000in}{0.828913in}}{\pgfqpoint{0.100000in}{0.833333in}}%
\pgfpathcurveto{\pgfqpoint{0.100000in}{0.837753in}}{\pgfqpoint{0.098244in}{0.841993in}}{\pgfqpoint{0.095118in}{0.845118in}}%
\pgfpathcurveto{\pgfqpoint{0.091993in}{0.848244in}}{\pgfqpoint{0.087753in}{0.850000in}}{\pgfqpoint{0.083333in}{0.850000in}}%
\pgfpathcurveto{\pgfqpoint{0.078913in}{0.850000in}}{\pgfqpoint{0.074674in}{0.848244in}}{\pgfqpoint{0.071548in}{0.845118in}}%
\pgfpathcurveto{\pgfqpoint{0.068423in}{0.841993in}}{\pgfqpoint{0.066667in}{0.837753in}}{\pgfqpoint{0.066667in}{0.833333in}}%
\pgfpathcurveto{\pgfqpoint{0.066667in}{0.828913in}}{\pgfqpoint{0.068423in}{0.824674in}}{\pgfqpoint{0.071548in}{0.821548in}}%
\pgfpathcurveto{\pgfqpoint{0.074674in}{0.818423in}}{\pgfqpoint{0.078913in}{0.816667in}}{\pgfqpoint{0.083333in}{0.816667in}}%
\pgfpathclose%
\pgfpathmoveto{\pgfqpoint{0.250000in}{0.816667in}}%
\pgfpathcurveto{\pgfqpoint{0.254420in}{0.816667in}}{\pgfqpoint{0.258660in}{0.818423in}}{\pgfqpoint{0.261785in}{0.821548in}}%
\pgfpathcurveto{\pgfqpoint{0.264911in}{0.824674in}}{\pgfqpoint{0.266667in}{0.828913in}}{\pgfqpoint{0.266667in}{0.833333in}}%
\pgfpathcurveto{\pgfqpoint{0.266667in}{0.837753in}}{\pgfqpoint{0.264911in}{0.841993in}}{\pgfqpoint{0.261785in}{0.845118in}}%
\pgfpathcurveto{\pgfqpoint{0.258660in}{0.848244in}}{\pgfqpoint{0.254420in}{0.850000in}}{\pgfqpoint{0.250000in}{0.850000in}}%
\pgfpathcurveto{\pgfqpoint{0.245580in}{0.850000in}}{\pgfqpoint{0.241340in}{0.848244in}}{\pgfqpoint{0.238215in}{0.845118in}}%
\pgfpathcurveto{\pgfqpoint{0.235089in}{0.841993in}}{\pgfqpoint{0.233333in}{0.837753in}}{\pgfqpoint{0.233333in}{0.833333in}}%
\pgfpathcurveto{\pgfqpoint{0.233333in}{0.828913in}}{\pgfqpoint{0.235089in}{0.824674in}}{\pgfqpoint{0.238215in}{0.821548in}}%
\pgfpathcurveto{\pgfqpoint{0.241340in}{0.818423in}}{\pgfqpoint{0.245580in}{0.816667in}}{\pgfqpoint{0.250000in}{0.816667in}}%
\pgfpathclose%
\pgfpathmoveto{\pgfqpoint{0.416667in}{0.816667in}}%
\pgfpathcurveto{\pgfqpoint{0.421087in}{0.816667in}}{\pgfqpoint{0.425326in}{0.818423in}}{\pgfqpoint{0.428452in}{0.821548in}}%
\pgfpathcurveto{\pgfqpoint{0.431577in}{0.824674in}}{\pgfqpoint{0.433333in}{0.828913in}}{\pgfqpoint{0.433333in}{0.833333in}}%
\pgfpathcurveto{\pgfqpoint{0.433333in}{0.837753in}}{\pgfqpoint{0.431577in}{0.841993in}}{\pgfqpoint{0.428452in}{0.845118in}}%
\pgfpathcurveto{\pgfqpoint{0.425326in}{0.848244in}}{\pgfqpoint{0.421087in}{0.850000in}}{\pgfqpoint{0.416667in}{0.850000in}}%
\pgfpathcurveto{\pgfqpoint{0.412247in}{0.850000in}}{\pgfqpoint{0.408007in}{0.848244in}}{\pgfqpoint{0.404882in}{0.845118in}}%
\pgfpathcurveto{\pgfqpoint{0.401756in}{0.841993in}}{\pgfqpoint{0.400000in}{0.837753in}}{\pgfqpoint{0.400000in}{0.833333in}}%
\pgfpathcurveto{\pgfqpoint{0.400000in}{0.828913in}}{\pgfqpoint{0.401756in}{0.824674in}}{\pgfqpoint{0.404882in}{0.821548in}}%
\pgfpathcurveto{\pgfqpoint{0.408007in}{0.818423in}}{\pgfqpoint{0.412247in}{0.816667in}}{\pgfqpoint{0.416667in}{0.816667in}}%
\pgfpathclose%
\pgfpathmoveto{\pgfqpoint{0.583333in}{0.816667in}}%
\pgfpathcurveto{\pgfqpoint{0.587753in}{0.816667in}}{\pgfqpoint{0.591993in}{0.818423in}}{\pgfqpoint{0.595118in}{0.821548in}}%
\pgfpathcurveto{\pgfqpoint{0.598244in}{0.824674in}}{\pgfqpoint{0.600000in}{0.828913in}}{\pgfqpoint{0.600000in}{0.833333in}}%
\pgfpathcurveto{\pgfqpoint{0.600000in}{0.837753in}}{\pgfqpoint{0.598244in}{0.841993in}}{\pgfqpoint{0.595118in}{0.845118in}}%
\pgfpathcurveto{\pgfqpoint{0.591993in}{0.848244in}}{\pgfqpoint{0.587753in}{0.850000in}}{\pgfqpoint{0.583333in}{0.850000in}}%
\pgfpathcurveto{\pgfqpoint{0.578913in}{0.850000in}}{\pgfqpoint{0.574674in}{0.848244in}}{\pgfqpoint{0.571548in}{0.845118in}}%
\pgfpathcurveto{\pgfqpoint{0.568423in}{0.841993in}}{\pgfqpoint{0.566667in}{0.837753in}}{\pgfqpoint{0.566667in}{0.833333in}}%
\pgfpathcurveto{\pgfqpoint{0.566667in}{0.828913in}}{\pgfqpoint{0.568423in}{0.824674in}}{\pgfqpoint{0.571548in}{0.821548in}}%
\pgfpathcurveto{\pgfqpoint{0.574674in}{0.818423in}}{\pgfqpoint{0.578913in}{0.816667in}}{\pgfqpoint{0.583333in}{0.816667in}}%
\pgfpathclose%
\pgfpathmoveto{\pgfqpoint{0.750000in}{0.816667in}}%
\pgfpathcurveto{\pgfqpoint{0.754420in}{0.816667in}}{\pgfqpoint{0.758660in}{0.818423in}}{\pgfqpoint{0.761785in}{0.821548in}}%
\pgfpathcurveto{\pgfqpoint{0.764911in}{0.824674in}}{\pgfqpoint{0.766667in}{0.828913in}}{\pgfqpoint{0.766667in}{0.833333in}}%
\pgfpathcurveto{\pgfqpoint{0.766667in}{0.837753in}}{\pgfqpoint{0.764911in}{0.841993in}}{\pgfqpoint{0.761785in}{0.845118in}}%
\pgfpathcurveto{\pgfqpoint{0.758660in}{0.848244in}}{\pgfqpoint{0.754420in}{0.850000in}}{\pgfqpoint{0.750000in}{0.850000in}}%
\pgfpathcurveto{\pgfqpoint{0.745580in}{0.850000in}}{\pgfqpoint{0.741340in}{0.848244in}}{\pgfqpoint{0.738215in}{0.845118in}}%
\pgfpathcurveto{\pgfqpoint{0.735089in}{0.841993in}}{\pgfqpoint{0.733333in}{0.837753in}}{\pgfqpoint{0.733333in}{0.833333in}}%
\pgfpathcurveto{\pgfqpoint{0.733333in}{0.828913in}}{\pgfqpoint{0.735089in}{0.824674in}}{\pgfqpoint{0.738215in}{0.821548in}}%
\pgfpathcurveto{\pgfqpoint{0.741340in}{0.818423in}}{\pgfqpoint{0.745580in}{0.816667in}}{\pgfqpoint{0.750000in}{0.816667in}}%
\pgfpathclose%
\pgfpathmoveto{\pgfqpoint{0.916667in}{0.816667in}}%
\pgfpathcurveto{\pgfqpoint{0.921087in}{0.816667in}}{\pgfqpoint{0.925326in}{0.818423in}}{\pgfqpoint{0.928452in}{0.821548in}}%
\pgfpathcurveto{\pgfqpoint{0.931577in}{0.824674in}}{\pgfqpoint{0.933333in}{0.828913in}}{\pgfqpoint{0.933333in}{0.833333in}}%
\pgfpathcurveto{\pgfqpoint{0.933333in}{0.837753in}}{\pgfqpoint{0.931577in}{0.841993in}}{\pgfqpoint{0.928452in}{0.845118in}}%
\pgfpathcurveto{\pgfqpoint{0.925326in}{0.848244in}}{\pgfqpoint{0.921087in}{0.850000in}}{\pgfqpoint{0.916667in}{0.850000in}}%
\pgfpathcurveto{\pgfqpoint{0.912247in}{0.850000in}}{\pgfqpoint{0.908007in}{0.848244in}}{\pgfqpoint{0.904882in}{0.845118in}}%
\pgfpathcurveto{\pgfqpoint{0.901756in}{0.841993in}}{\pgfqpoint{0.900000in}{0.837753in}}{\pgfqpoint{0.900000in}{0.833333in}}%
\pgfpathcurveto{\pgfqpoint{0.900000in}{0.828913in}}{\pgfqpoint{0.901756in}{0.824674in}}{\pgfqpoint{0.904882in}{0.821548in}}%
\pgfpathcurveto{\pgfqpoint{0.908007in}{0.818423in}}{\pgfqpoint{0.912247in}{0.816667in}}{\pgfqpoint{0.916667in}{0.816667in}}%
\pgfpathclose%
\pgfpathmoveto{\pgfqpoint{0.000000in}{0.983333in}}%
\pgfpathcurveto{\pgfqpoint{0.004420in}{0.983333in}}{\pgfqpoint{0.008660in}{0.985089in}}{\pgfqpoint{0.011785in}{0.988215in}}%
\pgfpathcurveto{\pgfqpoint{0.014911in}{0.991340in}}{\pgfqpoint{0.016667in}{0.995580in}}{\pgfqpoint{0.016667in}{1.000000in}}%
\pgfpathcurveto{\pgfqpoint{0.016667in}{1.004420in}}{\pgfqpoint{0.014911in}{1.008660in}}{\pgfqpoint{0.011785in}{1.011785in}}%
\pgfpathcurveto{\pgfqpoint{0.008660in}{1.014911in}}{\pgfqpoint{0.004420in}{1.016667in}}{\pgfqpoint{0.000000in}{1.016667in}}%
\pgfpathcurveto{\pgfqpoint{-0.004420in}{1.016667in}}{\pgfqpoint{-0.008660in}{1.014911in}}{\pgfqpoint{-0.011785in}{1.011785in}}%
\pgfpathcurveto{\pgfqpoint{-0.014911in}{1.008660in}}{\pgfqpoint{-0.016667in}{1.004420in}}{\pgfqpoint{-0.016667in}{1.000000in}}%
\pgfpathcurveto{\pgfqpoint{-0.016667in}{0.995580in}}{\pgfqpoint{-0.014911in}{0.991340in}}{\pgfqpoint{-0.011785in}{0.988215in}}%
\pgfpathcurveto{\pgfqpoint{-0.008660in}{0.985089in}}{\pgfqpoint{-0.004420in}{0.983333in}}{\pgfqpoint{0.000000in}{0.983333in}}%
\pgfpathclose%
\pgfpathmoveto{\pgfqpoint{0.166667in}{0.983333in}}%
\pgfpathcurveto{\pgfqpoint{0.171087in}{0.983333in}}{\pgfqpoint{0.175326in}{0.985089in}}{\pgfqpoint{0.178452in}{0.988215in}}%
\pgfpathcurveto{\pgfqpoint{0.181577in}{0.991340in}}{\pgfqpoint{0.183333in}{0.995580in}}{\pgfqpoint{0.183333in}{1.000000in}}%
\pgfpathcurveto{\pgfqpoint{0.183333in}{1.004420in}}{\pgfqpoint{0.181577in}{1.008660in}}{\pgfqpoint{0.178452in}{1.011785in}}%
\pgfpathcurveto{\pgfqpoint{0.175326in}{1.014911in}}{\pgfqpoint{0.171087in}{1.016667in}}{\pgfqpoint{0.166667in}{1.016667in}}%
\pgfpathcurveto{\pgfqpoint{0.162247in}{1.016667in}}{\pgfqpoint{0.158007in}{1.014911in}}{\pgfqpoint{0.154882in}{1.011785in}}%
\pgfpathcurveto{\pgfqpoint{0.151756in}{1.008660in}}{\pgfqpoint{0.150000in}{1.004420in}}{\pgfqpoint{0.150000in}{1.000000in}}%
\pgfpathcurveto{\pgfqpoint{0.150000in}{0.995580in}}{\pgfqpoint{0.151756in}{0.991340in}}{\pgfqpoint{0.154882in}{0.988215in}}%
\pgfpathcurveto{\pgfqpoint{0.158007in}{0.985089in}}{\pgfqpoint{0.162247in}{0.983333in}}{\pgfqpoint{0.166667in}{0.983333in}}%
\pgfpathclose%
\pgfpathmoveto{\pgfqpoint{0.333333in}{0.983333in}}%
\pgfpathcurveto{\pgfqpoint{0.337753in}{0.983333in}}{\pgfqpoint{0.341993in}{0.985089in}}{\pgfqpoint{0.345118in}{0.988215in}}%
\pgfpathcurveto{\pgfqpoint{0.348244in}{0.991340in}}{\pgfqpoint{0.350000in}{0.995580in}}{\pgfqpoint{0.350000in}{1.000000in}}%
\pgfpathcurveto{\pgfqpoint{0.350000in}{1.004420in}}{\pgfqpoint{0.348244in}{1.008660in}}{\pgfqpoint{0.345118in}{1.011785in}}%
\pgfpathcurveto{\pgfqpoint{0.341993in}{1.014911in}}{\pgfqpoint{0.337753in}{1.016667in}}{\pgfqpoint{0.333333in}{1.016667in}}%
\pgfpathcurveto{\pgfqpoint{0.328913in}{1.016667in}}{\pgfqpoint{0.324674in}{1.014911in}}{\pgfqpoint{0.321548in}{1.011785in}}%
\pgfpathcurveto{\pgfqpoint{0.318423in}{1.008660in}}{\pgfqpoint{0.316667in}{1.004420in}}{\pgfqpoint{0.316667in}{1.000000in}}%
\pgfpathcurveto{\pgfqpoint{0.316667in}{0.995580in}}{\pgfqpoint{0.318423in}{0.991340in}}{\pgfqpoint{0.321548in}{0.988215in}}%
\pgfpathcurveto{\pgfqpoint{0.324674in}{0.985089in}}{\pgfqpoint{0.328913in}{0.983333in}}{\pgfqpoint{0.333333in}{0.983333in}}%
\pgfpathclose%
\pgfpathmoveto{\pgfqpoint{0.500000in}{0.983333in}}%
\pgfpathcurveto{\pgfqpoint{0.504420in}{0.983333in}}{\pgfqpoint{0.508660in}{0.985089in}}{\pgfqpoint{0.511785in}{0.988215in}}%
\pgfpathcurveto{\pgfqpoint{0.514911in}{0.991340in}}{\pgfqpoint{0.516667in}{0.995580in}}{\pgfqpoint{0.516667in}{1.000000in}}%
\pgfpathcurveto{\pgfqpoint{0.516667in}{1.004420in}}{\pgfqpoint{0.514911in}{1.008660in}}{\pgfqpoint{0.511785in}{1.011785in}}%
\pgfpathcurveto{\pgfqpoint{0.508660in}{1.014911in}}{\pgfqpoint{0.504420in}{1.016667in}}{\pgfqpoint{0.500000in}{1.016667in}}%
\pgfpathcurveto{\pgfqpoint{0.495580in}{1.016667in}}{\pgfqpoint{0.491340in}{1.014911in}}{\pgfqpoint{0.488215in}{1.011785in}}%
\pgfpathcurveto{\pgfqpoint{0.485089in}{1.008660in}}{\pgfqpoint{0.483333in}{1.004420in}}{\pgfqpoint{0.483333in}{1.000000in}}%
\pgfpathcurveto{\pgfqpoint{0.483333in}{0.995580in}}{\pgfqpoint{0.485089in}{0.991340in}}{\pgfqpoint{0.488215in}{0.988215in}}%
\pgfpathcurveto{\pgfqpoint{0.491340in}{0.985089in}}{\pgfqpoint{0.495580in}{0.983333in}}{\pgfqpoint{0.500000in}{0.983333in}}%
\pgfpathclose%
\pgfpathmoveto{\pgfqpoint{0.666667in}{0.983333in}}%
\pgfpathcurveto{\pgfqpoint{0.671087in}{0.983333in}}{\pgfqpoint{0.675326in}{0.985089in}}{\pgfqpoint{0.678452in}{0.988215in}}%
\pgfpathcurveto{\pgfqpoint{0.681577in}{0.991340in}}{\pgfqpoint{0.683333in}{0.995580in}}{\pgfqpoint{0.683333in}{1.000000in}}%
\pgfpathcurveto{\pgfqpoint{0.683333in}{1.004420in}}{\pgfqpoint{0.681577in}{1.008660in}}{\pgfqpoint{0.678452in}{1.011785in}}%
\pgfpathcurveto{\pgfqpoint{0.675326in}{1.014911in}}{\pgfqpoint{0.671087in}{1.016667in}}{\pgfqpoint{0.666667in}{1.016667in}}%
\pgfpathcurveto{\pgfqpoint{0.662247in}{1.016667in}}{\pgfqpoint{0.658007in}{1.014911in}}{\pgfqpoint{0.654882in}{1.011785in}}%
\pgfpathcurveto{\pgfqpoint{0.651756in}{1.008660in}}{\pgfqpoint{0.650000in}{1.004420in}}{\pgfqpoint{0.650000in}{1.000000in}}%
\pgfpathcurveto{\pgfqpoint{0.650000in}{0.995580in}}{\pgfqpoint{0.651756in}{0.991340in}}{\pgfqpoint{0.654882in}{0.988215in}}%
\pgfpathcurveto{\pgfqpoint{0.658007in}{0.985089in}}{\pgfqpoint{0.662247in}{0.983333in}}{\pgfqpoint{0.666667in}{0.983333in}}%
\pgfpathclose%
\pgfpathmoveto{\pgfqpoint{0.833333in}{0.983333in}}%
\pgfpathcurveto{\pgfqpoint{0.837753in}{0.983333in}}{\pgfqpoint{0.841993in}{0.985089in}}{\pgfqpoint{0.845118in}{0.988215in}}%
\pgfpathcurveto{\pgfqpoint{0.848244in}{0.991340in}}{\pgfqpoint{0.850000in}{0.995580in}}{\pgfqpoint{0.850000in}{1.000000in}}%
\pgfpathcurveto{\pgfqpoint{0.850000in}{1.004420in}}{\pgfqpoint{0.848244in}{1.008660in}}{\pgfqpoint{0.845118in}{1.011785in}}%
\pgfpathcurveto{\pgfqpoint{0.841993in}{1.014911in}}{\pgfqpoint{0.837753in}{1.016667in}}{\pgfqpoint{0.833333in}{1.016667in}}%
\pgfpathcurveto{\pgfqpoint{0.828913in}{1.016667in}}{\pgfqpoint{0.824674in}{1.014911in}}{\pgfqpoint{0.821548in}{1.011785in}}%
\pgfpathcurveto{\pgfqpoint{0.818423in}{1.008660in}}{\pgfqpoint{0.816667in}{1.004420in}}{\pgfqpoint{0.816667in}{1.000000in}}%
\pgfpathcurveto{\pgfqpoint{0.816667in}{0.995580in}}{\pgfqpoint{0.818423in}{0.991340in}}{\pgfqpoint{0.821548in}{0.988215in}}%
\pgfpathcurveto{\pgfqpoint{0.824674in}{0.985089in}}{\pgfqpoint{0.828913in}{0.983333in}}{\pgfqpoint{0.833333in}{0.983333in}}%
\pgfpathclose%
\pgfpathmoveto{\pgfqpoint{1.000000in}{0.983333in}}%
\pgfpathcurveto{\pgfqpoint{1.004420in}{0.983333in}}{\pgfqpoint{1.008660in}{0.985089in}}{\pgfqpoint{1.011785in}{0.988215in}}%
\pgfpathcurveto{\pgfqpoint{1.014911in}{0.991340in}}{\pgfqpoint{1.016667in}{0.995580in}}{\pgfqpoint{1.016667in}{1.000000in}}%
\pgfpathcurveto{\pgfqpoint{1.016667in}{1.004420in}}{\pgfqpoint{1.014911in}{1.008660in}}{\pgfqpoint{1.011785in}{1.011785in}}%
\pgfpathcurveto{\pgfqpoint{1.008660in}{1.014911in}}{\pgfqpoint{1.004420in}{1.016667in}}{\pgfqpoint{1.000000in}{1.016667in}}%
\pgfpathcurveto{\pgfqpoint{0.995580in}{1.016667in}}{\pgfqpoint{0.991340in}{1.014911in}}{\pgfqpoint{0.988215in}{1.011785in}}%
\pgfpathcurveto{\pgfqpoint{0.985089in}{1.008660in}}{\pgfqpoint{0.983333in}{1.004420in}}{\pgfqpoint{0.983333in}{1.000000in}}%
\pgfpathcurveto{\pgfqpoint{0.983333in}{0.995580in}}{\pgfqpoint{0.985089in}{0.991340in}}{\pgfqpoint{0.988215in}{0.988215in}}%
\pgfpathcurveto{\pgfqpoint{0.991340in}{0.985089in}}{\pgfqpoint{0.995580in}{0.983333in}}{\pgfqpoint{1.000000in}{0.983333in}}%
\pgfpathclose%
\pgfusepath{stroke}%
\end{pgfscope}%
}%
\pgfsys@transformshift{1.070538in}{8.357449in}%
\pgfsys@useobject{currentpattern}{}%
\pgfsys@transformshift{1in}{0in}%
\pgfsys@transformshift{-1in}{0in}%
\pgfsys@transformshift{0in}{1in}%
\end{pgfscope}%
\begin{pgfscope}%
\definecolor{textcolor}{rgb}{0.000000,0.000000,0.000000}%
\pgfsetstrokecolor{textcolor}%
\pgfsetfillcolor{textcolor}%
\pgftext[x=1.692760in,y=8.357449in,left,base]{\color{textcolor}\rmfamily\fontsize{16.000000}{19.200000}\selectfont NUCLEAR\_TB}%
\end{pgfscope}%
\begin{pgfscope}%
\pgfsetbuttcap%
\pgfsetmiterjoin%
\definecolor{currentfill}{rgb}{1.000000,1.000000,0.000000}%
\pgfsetfillcolor{currentfill}%
\pgfsetfillopacity{0.990000}%
\pgfsetlinewidth{0.000000pt}%
\definecolor{currentstroke}{rgb}{0.000000,0.000000,0.000000}%
\pgfsetstrokecolor{currentstroke}%
\pgfsetstrokeopacity{0.990000}%
\pgfsetdash{}{0pt}%
\pgfpathmoveto{\pgfqpoint{1.070538in}{8.032990in}}%
\pgfpathlineto{\pgfqpoint{1.514982in}{8.032990in}}%
\pgfpathlineto{\pgfqpoint{1.514982in}{8.188545in}}%
\pgfpathlineto{\pgfqpoint{1.070538in}{8.188545in}}%
\pgfpathclose%
\pgfusepath{fill}%
\end{pgfscope}%
\begin{pgfscope}%
\pgfsetbuttcap%
\pgfsetmiterjoin%
\definecolor{currentfill}{rgb}{1.000000,1.000000,0.000000}%
\pgfsetfillcolor{currentfill}%
\pgfsetfillopacity{0.990000}%
\pgfsetlinewidth{0.000000pt}%
\definecolor{currentstroke}{rgb}{0.000000,0.000000,0.000000}%
\pgfsetstrokecolor{currentstroke}%
\pgfsetstrokeopacity{0.990000}%
\pgfsetdash{}{0pt}%
\pgfpathmoveto{\pgfqpoint{1.070538in}{8.032990in}}%
\pgfpathlineto{\pgfqpoint{1.514982in}{8.032990in}}%
\pgfpathlineto{\pgfqpoint{1.514982in}{8.188545in}}%
\pgfpathlineto{\pgfqpoint{1.070538in}{8.188545in}}%
\pgfpathclose%
\pgfusepath{clip}%
\pgfsys@defobject{currentpattern}{\pgfqpoint{0in}{0in}}{\pgfqpoint{1in}{1in}}{%
\begin{pgfscope}%
\pgfpathrectangle{\pgfqpoint{0in}{0in}}{\pgfqpoint{1in}{1in}}%
\pgfusepath{clip}%
\pgfpathmoveto{\pgfqpoint{0.000000in}{0.055556in}}%
\pgfpathlineto{\pgfqpoint{-0.016327in}{0.022473in}}%
\pgfpathlineto{\pgfqpoint{-0.052836in}{0.017168in}}%
\pgfpathlineto{\pgfqpoint{-0.026418in}{-0.008584in}}%
\pgfpathlineto{\pgfqpoint{-0.032655in}{-0.044945in}}%
\pgfpathlineto{\pgfqpoint{-0.000000in}{-0.027778in}}%
\pgfpathlineto{\pgfqpoint{0.032655in}{-0.044945in}}%
\pgfpathlineto{\pgfqpoint{0.026418in}{-0.008584in}}%
\pgfpathlineto{\pgfqpoint{0.052836in}{0.017168in}}%
\pgfpathlineto{\pgfqpoint{0.016327in}{0.022473in}}%
\pgfpathlineto{\pgfqpoint{0.000000in}{0.055556in}}%
\pgfpathmoveto{\pgfqpoint{0.166667in}{0.055556in}}%
\pgfpathlineto{\pgfqpoint{0.150339in}{0.022473in}}%
\pgfpathlineto{\pgfqpoint{0.113830in}{0.017168in}}%
\pgfpathlineto{\pgfqpoint{0.140248in}{-0.008584in}}%
\pgfpathlineto{\pgfqpoint{0.134012in}{-0.044945in}}%
\pgfpathlineto{\pgfqpoint{0.166667in}{-0.027778in}}%
\pgfpathlineto{\pgfqpoint{0.199321in}{-0.044945in}}%
\pgfpathlineto{\pgfqpoint{0.193085in}{-0.008584in}}%
\pgfpathlineto{\pgfqpoint{0.219503in}{0.017168in}}%
\pgfpathlineto{\pgfqpoint{0.182994in}{0.022473in}}%
\pgfpathlineto{\pgfqpoint{0.166667in}{0.055556in}}%
\pgfpathmoveto{\pgfqpoint{0.333333in}{0.055556in}}%
\pgfpathlineto{\pgfqpoint{0.317006in}{0.022473in}}%
\pgfpathlineto{\pgfqpoint{0.280497in}{0.017168in}}%
\pgfpathlineto{\pgfqpoint{0.306915in}{-0.008584in}}%
\pgfpathlineto{\pgfqpoint{0.300679in}{-0.044945in}}%
\pgfpathlineto{\pgfqpoint{0.333333in}{-0.027778in}}%
\pgfpathlineto{\pgfqpoint{0.365988in}{-0.044945in}}%
\pgfpathlineto{\pgfqpoint{0.359752in}{-0.008584in}}%
\pgfpathlineto{\pgfqpoint{0.386170in}{0.017168in}}%
\pgfpathlineto{\pgfqpoint{0.349661in}{0.022473in}}%
\pgfpathlineto{\pgfqpoint{0.333333in}{0.055556in}}%
\pgfpathmoveto{\pgfqpoint{0.500000in}{0.055556in}}%
\pgfpathlineto{\pgfqpoint{0.483673in}{0.022473in}}%
\pgfpathlineto{\pgfqpoint{0.447164in}{0.017168in}}%
\pgfpathlineto{\pgfqpoint{0.473582in}{-0.008584in}}%
\pgfpathlineto{\pgfqpoint{0.467345in}{-0.044945in}}%
\pgfpathlineto{\pgfqpoint{0.500000in}{-0.027778in}}%
\pgfpathlineto{\pgfqpoint{0.532655in}{-0.044945in}}%
\pgfpathlineto{\pgfqpoint{0.526418in}{-0.008584in}}%
\pgfpathlineto{\pgfqpoint{0.552836in}{0.017168in}}%
\pgfpathlineto{\pgfqpoint{0.516327in}{0.022473in}}%
\pgfpathlineto{\pgfqpoint{0.500000in}{0.055556in}}%
\pgfpathmoveto{\pgfqpoint{0.666667in}{0.055556in}}%
\pgfpathlineto{\pgfqpoint{0.650339in}{0.022473in}}%
\pgfpathlineto{\pgfqpoint{0.613830in}{0.017168in}}%
\pgfpathlineto{\pgfqpoint{0.640248in}{-0.008584in}}%
\pgfpathlineto{\pgfqpoint{0.634012in}{-0.044945in}}%
\pgfpathlineto{\pgfqpoint{0.666667in}{-0.027778in}}%
\pgfpathlineto{\pgfqpoint{0.699321in}{-0.044945in}}%
\pgfpathlineto{\pgfqpoint{0.693085in}{-0.008584in}}%
\pgfpathlineto{\pgfqpoint{0.719503in}{0.017168in}}%
\pgfpathlineto{\pgfqpoint{0.682994in}{0.022473in}}%
\pgfpathlineto{\pgfqpoint{0.666667in}{0.055556in}}%
\pgfpathmoveto{\pgfqpoint{0.833333in}{0.055556in}}%
\pgfpathlineto{\pgfqpoint{0.817006in}{0.022473in}}%
\pgfpathlineto{\pgfqpoint{0.780497in}{0.017168in}}%
\pgfpathlineto{\pgfqpoint{0.806915in}{-0.008584in}}%
\pgfpathlineto{\pgfqpoint{0.800679in}{-0.044945in}}%
\pgfpathlineto{\pgfqpoint{0.833333in}{-0.027778in}}%
\pgfpathlineto{\pgfqpoint{0.865988in}{-0.044945in}}%
\pgfpathlineto{\pgfqpoint{0.859752in}{-0.008584in}}%
\pgfpathlineto{\pgfqpoint{0.886170in}{0.017168in}}%
\pgfpathlineto{\pgfqpoint{0.849661in}{0.022473in}}%
\pgfpathlineto{\pgfqpoint{0.833333in}{0.055556in}}%
\pgfpathmoveto{\pgfqpoint{1.000000in}{0.055556in}}%
\pgfpathlineto{\pgfqpoint{0.983673in}{0.022473in}}%
\pgfpathlineto{\pgfqpoint{0.947164in}{0.017168in}}%
\pgfpathlineto{\pgfqpoint{0.973582in}{-0.008584in}}%
\pgfpathlineto{\pgfqpoint{0.967345in}{-0.044945in}}%
\pgfpathlineto{\pgfqpoint{1.000000in}{-0.027778in}}%
\pgfpathlineto{\pgfqpoint{1.032655in}{-0.044945in}}%
\pgfpathlineto{\pgfqpoint{1.026418in}{-0.008584in}}%
\pgfpathlineto{\pgfqpoint{1.052836in}{0.017168in}}%
\pgfpathlineto{\pgfqpoint{1.016327in}{0.022473in}}%
\pgfpathlineto{\pgfqpoint{1.000000in}{0.055556in}}%
\pgfpathmoveto{\pgfqpoint{0.083333in}{0.222222in}}%
\pgfpathlineto{\pgfqpoint{0.067006in}{0.189139in}}%
\pgfpathlineto{\pgfqpoint{0.030497in}{0.183834in}}%
\pgfpathlineto{\pgfqpoint{0.056915in}{0.158083in}}%
\pgfpathlineto{\pgfqpoint{0.050679in}{0.121721in}}%
\pgfpathlineto{\pgfqpoint{0.083333in}{0.138889in}}%
\pgfpathlineto{\pgfqpoint{0.115988in}{0.121721in}}%
\pgfpathlineto{\pgfqpoint{0.109752in}{0.158083in}}%
\pgfpathlineto{\pgfqpoint{0.136170in}{0.183834in}}%
\pgfpathlineto{\pgfqpoint{0.099661in}{0.189139in}}%
\pgfpathlineto{\pgfqpoint{0.083333in}{0.222222in}}%
\pgfpathmoveto{\pgfqpoint{0.250000in}{0.222222in}}%
\pgfpathlineto{\pgfqpoint{0.233673in}{0.189139in}}%
\pgfpathlineto{\pgfqpoint{0.197164in}{0.183834in}}%
\pgfpathlineto{\pgfqpoint{0.223582in}{0.158083in}}%
\pgfpathlineto{\pgfqpoint{0.217345in}{0.121721in}}%
\pgfpathlineto{\pgfqpoint{0.250000in}{0.138889in}}%
\pgfpathlineto{\pgfqpoint{0.282655in}{0.121721in}}%
\pgfpathlineto{\pgfqpoint{0.276418in}{0.158083in}}%
\pgfpathlineto{\pgfqpoint{0.302836in}{0.183834in}}%
\pgfpathlineto{\pgfqpoint{0.266327in}{0.189139in}}%
\pgfpathlineto{\pgfqpoint{0.250000in}{0.222222in}}%
\pgfpathmoveto{\pgfqpoint{0.416667in}{0.222222in}}%
\pgfpathlineto{\pgfqpoint{0.400339in}{0.189139in}}%
\pgfpathlineto{\pgfqpoint{0.363830in}{0.183834in}}%
\pgfpathlineto{\pgfqpoint{0.390248in}{0.158083in}}%
\pgfpathlineto{\pgfqpoint{0.384012in}{0.121721in}}%
\pgfpathlineto{\pgfqpoint{0.416667in}{0.138889in}}%
\pgfpathlineto{\pgfqpoint{0.449321in}{0.121721in}}%
\pgfpathlineto{\pgfqpoint{0.443085in}{0.158083in}}%
\pgfpathlineto{\pgfqpoint{0.469503in}{0.183834in}}%
\pgfpathlineto{\pgfqpoint{0.432994in}{0.189139in}}%
\pgfpathlineto{\pgfqpoint{0.416667in}{0.222222in}}%
\pgfpathmoveto{\pgfqpoint{0.583333in}{0.222222in}}%
\pgfpathlineto{\pgfqpoint{0.567006in}{0.189139in}}%
\pgfpathlineto{\pgfqpoint{0.530497in}{0.183834in}}%
\pgfpathlineto{\pgfqpoint{0.556915in}{0.158083in}}%
\pgfpathlineto{\pgfqpoint{0.550679in}{0.121721in}}%
\pgfpathlineto{\pgfqpoint{0.583333in}{0.138889in}}%
\pgfpathlineto{\pgfqpoint{0.615988in}{0.121721in}}%
\pgfpathlineto{\pgfqpoint{0.609752in}{0.158083in}}%
\pgfpathlineto{\pgfqpoint{0.636170in}{0.183834in}}%
\pgfpathlineto{\pgfqpoint{0.599661in}{0.189139in}}%
\pgfpathlineto{\pgfqpoint{0.583333in}{0.222222in}}%
\pgfpathmoveto{\pgfqpoint{0.750000in}{0.222222in}}%
\pgfpathlineto{\pgfqpoint{0.733673in}{0.189139in}}%
\pgfpathlineto{\pgfqpoint{0.697164in}{0.183834in}}%
\pgfpathlineto{\pgfqpoint{0.723582in}{0.158083in}}%
\pgfpathlineto{\pgfqpoint{0.717345in}{0.121721in}}%
\pgfpathlineto{\pgfqpoint{0.750000in}{0.138889in}}%
\pgfpathlineto{\pgfqpoint{0.782655in}{0.121721in}}%
\pgfpathlineto{\pgfqpoint{0.776418in}{0.158083in}}%
\pgfpathlineto{\pgfqpoint{0.802836in}{0.183834in}}%
\pgfpathlineto{\pgfqpoint{0.766327in}{0.189139in}}%
\pgfpathlineto{\pgfqpoint{0.750000in}{0.222222in}}%
\pgfpathmoveto{\pgfqpoint{0.916667in}{0.222222in}}%
\pgfpathlineto{\pgfqpoint{0.900339in}{0.189139in}}%
\pgfpathlineto{\pgfqpoint{0.863830in}{0.183834in}}%
\pgfpathlineto{\pgfqpoint{0.890248in}{0.158083in}}%
\pgfpathlineto{\pgfqpoint{0.884012in}{0.121721in}}%
\pgfpathlineto{\pgfqpoint{0.916667in}{0.138889in}}%
\pgfpathlineto{\pgfqpoint{0.949321in}{0.121721in}}%
\pgfpathlineto{\pgfqpoint{0.943085in}{0.158083in}}%
\pgfpathlineto{\pgfqpoint{0.969503in}{0.183834in}}%
\pgfpathlineto{\pgfqpoint{0.932994in}{0.189139in}}%
\pgfpathlineto{\pgfqpoint{0.916667in}{0.222222in}}%
\pgfpathmoveto{\pgfqpoint{0.000000in}{0.388889in}}%
\pgfpathlineto{\pgfqpoint{-0.016327in}{0.355806in}}%
\pgfpathlineto{\pgfqpoint{-0.052836in}{0.350501in}}%
\pgfpathlineto{\pgfqpoint{-0.026418in}{0.324750in}}%
\pgfpathlineto{\pgfqpoint{-0.032655in}{0.288388in}}%
\pgfpathlineto{\pgfqpoint{-0.000000in}{0.305556in}}%
\pgfpathlineto{\pgfqpoint{0.032655in}{0.288388in}}%
\pgfpathlineto{\pgfqpoint{0.026418in}{0.324750in}}%
\pgfpathlineto{\pgfqpoint{0.052836in}{0.350501in}}%
\pgfpathlineto{\pgfqpoint{0.016327in}{0.355806in}}%
\pgfpathlineto{\pgfqpoint{0.000000in}{0.388889in}}%
\pgfpathmoveto{\pgfqpoint{0.166667in}{0.388889in}}%
\pgfpathlineto{\pgfqpoint{0.150339in}{0.355806in}}%
\pgfpathlineto{\pgfqpoint{0.113830in}{0.350501in}}%
\pgfpathlineto{\pgfqpoint{0.140248in}{0.324750in}}%
\pgfpathlineto{\pgfqpoint{0.134012in}{0.288388in}}%
\pgfpathlineto{\pgfqpoint{0.166667in}{0.305556in}}%
\pgfpathlineto{\pgfqpoint{0.199321in}{0.288388in}}%
\pgfpathlineto{\pgfqpoint{0.193085in}{0.324750in}}%
\pgfpathlineto{\pgfqpoint{0.219503in}{0.350501in}}%
\pgfpathlineto{\pgfqpoint{0.182994in}{0.355806in}}%
\pgfpathlineto{\pgfqpoint{0.166667in}{0.388889in}}%
\pgfpathmoveto{\pgfqpoint{0.333333in}{0.388889in}}%
\pgfpathlineto{\pgfqpoint{0.317006in}{0.355806in}}%
\pgfpathlineto{\pgfqpoint{0.280497in}{0.350501in}}%
\pgfpathlineto{\pgfqpoint{0.306915in}{0.324750in}}%
\pgfpathlineto{\pgfqpoint{0.300679in}{0.288388in}}%
\pgfpathlineto{\pgfqpoint{0.333333in}{0.305556in}}%
\pgfpathlineto{\pgfqpoint{0.365988in}{0.288388in}}%
\pgfpathlineto{\pgfqpoint{0.359752in}{0.324750in}}%
\pgfpathlineto{\pgfqpoint{0.386170in}{0.350501in}}%
\pgfpathlineto{\pgfqpoint{0.349661in}{0.355806in}}%
\pgfpathlineto{\pgfqpoint{0.333333in}{0.388889in}}%
\pgfpathmoveto{\pgfqpoint{0.500000in}{0.388889in}}%
\pgfpathlineto{\pgfqpoint{0.483673in}{0.355806in}}%
\pgfpathlineto{\pgfqpoint{0.447164in}{0.350501in}}%
\pgfpathlineto{\pgfqpoint{0.473582in}{0.324750in}}%
\pgfpathlineto{\pgfqpoint{0.467345in}{0.288388in}}%
\pgfpathlineto{\pgfqpoint{0.500000in}{0.305556in}}%
\pgfpathlineto{\pgfqpoint{0.532655in}{0.288388in}}%
\pgfpathlineto{\pgfqpoint{0.526418in}{0.324750in}}%
\pgfpathlineto{\pgfqpoint{0.552836in}{0.350501in}}%
\pgfpathlineto{\pgfqpoint{0.516327in}{0.355806in}}%
\pgfpathlineto{\pgfqpoint{0.500000in}{0.388889in}}%
\pgfpathmoveto{\pgfqpoint{0.666667in}{0.388889in}}%
\pgfpathlineto{\pgfqpoint{0.650339in}{0.355806in}}%
\pgfpathlineto{\pgfqpoint{0.613830in}{0.350501in}}%
\pgfpathlineto{\pgfqpoint{0.640248in}{0.324750in}}%
\pgfpathlineto{\pgfqpoint{0.634012in}{0.288388in}}%
\pgfpathlineto{\pgfqpoint{0.666667in}{0.305556in}}%
\pgfpathlineto{\pgfqpoint{0.699321in}{0.288388in}}%
\pgfpathlineto{\pgfqpoint{0.693085in}{0.324750in}}%
\pgfpathlineto{\pgfqpoint{0.719503in}{0.350501in}}%
\pgfpathlineto{\pgfqpoint{0.682994in}{0.355806in}}%
\pgfpathlineto{\pgfqpoint{0.666667in}{0.388889in}}%
\pgfpathmoveto{\pgfqpoint{0.833333in}{0.388889in}}%
\pgfpathlineto{\pgfqpoint{0.817006in}{0.355806in}}%
\pgfpathlineto{\pgfqpoint{0.780497in}{0.350501in}}%
\pgfpathlineto{\pgfqpoint{0.806915in}{0.324750in}}%
\pgfpathlineto{\pgfqpoint{0.800679in}{0.288388in}}%
\pgfpathlineto{\pgfqpoint{0.833333in}{0.305556in}}%
\pgfpathlineto{\pgfqpoint{0.865988in}{0.288388in}}%
\pgfpathlineto{\pgfqpoint{0.859752in}{0.324750in}}%
\pgfpathlineto{\pgfqpoint{0.886170in}{0.350501in}}%
\pgfpathlineto{\pgfqpoint{0.849661in}{0.355806in}}%
\pgfpathlineto{\pgfqpoint{0.833333in}{0.388889in}}%
\pgfpathmoveto{\pgfqpoint{1.000000in}{0.388889in}}%
\pgfpathlineto{\pgfqpoint{0.983673in}{0.355806in}}%
\pgfpathlineto{\pgfqpoint{0.947164in}{0.350501in}}%
\pgfpathlineto{\pgfqpoint{0.973582in}{0.324750in}}%
\pgfpathlineto{\pgfqpoint{0.967345in}{0.288388in}}%
\pgfpathlineto{\pgfqpoint{1.000000in}{0.305556in}}%
\pgfpathlineto{\pgfqpoint{1.032655in}{0.288388in}}%
\pgfpathlineto{\pgfqpoint{1.026418in}{0.324750in}}%
\pgfpathlineto{\pgfqpoint{1.052836in}{0.350501in}}%
\pgfpathlineto{\pgfqpoint{1.016327in}{0.355806in}}%
\pgfpathlineto{\pgfqpoint{1.000000in}{0.388889in}}%
\pgfpathmoveto{\pgfqpoint{0.083333in}{0.555556in}}%
\pgfpathlineto{\pgfqpoint{0.067006in}{0.522473in}}%
\pgfpathlineto{\pgfqpoint{0.030497in}{0.517168in}}%
\pgfpathlineto{\pgfqpoint{0.056915in}{0.491416in}}%
\pgfpathlineto{\pgfqpoint{0.050679in}{0.455055in}}%
\pgfpathlineto{\pgfqpoint{0.083333in}{0.472222in}}%
\pgfpathlineto{\pgfqpoint{0.115988in}{0.455055in}}%
\pgfpathlineto{\pgfqpoint{0.109752in}{0.491416in}}%
\pgfpathlineto{\pgfqpoint{0.136170in}{0.517168in}}%
\pgfpathlineto{\pgfqpoint{0.099661in}{0.522473in}}%
\pgfpathlineto{\pgfqpoint{0.083333in}{0.555556in}}%
\pgfpathmoveto{\pgfqpoint{0.250000in}{0.555556in}}%
\pgfpathlineto{\pgfqpoint{0.233673in}{0.522473in}}%
\pgfpathlineto{\pgfqpoint{0.197164in}{0.517168in}}%
\pgfpathlineto{\pgfqpoint{0.223582in}{0.491416in}}%
\pgfpathlineto{\pgfqpoint{0.217345in}{0.455055in}}%
\pgfpathlineto{\pgfqpoint{0.250000in}{0.472222in}}%
\pgfpathlineto{\pgfqpoint{0.282655in}{0.455055in}}%
\pgfpathlineto{\pgfqpoint{0.276418in}{0.491416in}}%
\pgfpathlineto{\pgfqpoint{0.302836in}{0.517168in}}%
\pgfpathlineto{\pgfqpoint{0.266327in}{0.522473in}}%
\pgfpathlineto{\pgfqpoint{0.250000in}{0.555556in}}%
\pgfpathmoveto{\pgfqpoint{0.416667in}{0.555556in}}%
\pgfpathlineto{\pgfqpoint{0.400339in}{0.522473in}}%
\pgfpathlineto{\pgfqpoint{0.363830in}{0.517168in}}%
\pgfpathlineto{\pgfqpoint{0.390248in}{0.491416in}}%
\pgfpathlineto{\pgfqpoint{0.384012in}{0.455055in}}%
\pgfpathlineto{\pgfqpoint{0.416667in}{0.472222in}}%
\pgfpathlineto{\pgfqpoint{0.449321in}{0.455055in}}%
\pgfpathlineto{\pgfqpoint{0.443085in}{0.491416in}}%
\pgfpathlineto{\pgfqpoint{0.469503in}{0.517168in}}%
\pgfpathlineto{\pgfqpoint{0.432994in}{0.522473in}}%
\pgfpathlineto{\pgfqpoint{0.416667in}{0.555556in}}%
\pgfpathmoveto{\pgfqpoint{0.583333in}{0.555556in}}%
\pgfpathlineto{\pgfqpoint{0.567006in}{0.522473in}}%
\pgfpathlineto{\pgfqpoint{0.530497in}{0.517168in}}%
\pgfpathlineto{\pgfqpoint{0.556915in}{0.491416in}}%
\pgfpathlineto{\pgfqpoint{0.550679in}{0.455055in}}%
\pgfpathlineto{\pgfqpoint{0.583333in}{0.472222in}}%
\pgfpathlineto{\pgfqpoint{0.615988in}{0.455055in}}%
\pgfpathlineto{\pgfqpoint{0.609752in}{0.491416in}}%
\pgfpathlineto{\pgfqpoint{0.636170in}{0.517168in}}%
\pgfpathlineto{\pgfqpoint{0.599661in}{0.522473in}}%
\pgfpathlineto{\pgfqpoint{0.583333in}{0.555556in}}%
\pgfpathmoveto{\pgfqpoint{0.750000in}{0.555556in}}%
\pgfpathlineto{\pgfqpoint{0.733673in}{0.522473in}}%
\pgfpathlineto{\pgfqpoint{0.697164in}{0.517168in}}%
\pgfpathlineto{\pgfqpoint{0.723582in}{0.491416in}}%
\pgfpathlineto{\pgfqpoint{0.717345in}{0.455055in}}%
\pgfpathlineto{\pgfqpoint{0.750000in}{0.472222in}}%
\pgfpathlineto{\pgfqpoint{0.782655in}{0.455055in}}%
\pgfpathlineto{\pgfqpoint{0.776418in}{0.491416in}}%
\pgfpathlineto{\pgfqpoint{0.802836in}{0.517168in}}%
\pgfpathlineto{\pgfqpoint{0.766327in}{0.522473in}}%
\pgfpathlineto{\pgfqpoint{0.750000in}{0.555556in}}%
\pgfpathmoveto{\pgfqpoint{0.916667in}{0.555556in}}%
\pgfpathlineto{\pgfqpoint{0.900339in}{0.522473in}}%
\pgfpathlineto{\pgfqpoint{0.863830in}{0.517168in}}%
\pgfpathlineto{\pgfqpoint{0.890248in}{0.491416in}}%
\pgfpathlineto{\pgfqpoint{0.884012in}{0.455055in}}%
\pgfpathlineto{\pgfqpoint{0.916667in}{0.472222in}}%
\pgfpathlineto{\pgfqpoint{0.949321in}{0.455055in}}%
\pgfpathlineto{\pgfqpoint{0.943085in}{0.491416in}}%
\pgfpathlineto{\pgfqpoint{0.969503in}{0.517168in}}%
\pgfpathlineto{\pgfqpoint{0.932994in}{0.522473in}}%
\pgfpathlineto{\pgfqpoint{0.916667in}{0.555556in}}%
\pgfpathmoveto{\pgfqpoint{0.000000in}{0.722222in}}%
\pgfpathlineto{\pgfqpoint{-0.016327in}{0.689139in}}%
\pgfpathlineto{\pgfqpoint{-0.052836in}{0.683834in}}%
\pgfpathlineto{\pgfqpoint{-0.026418in}{0.658083in}}%
\pgfpathlineto{\pgfqpoint{-0.032655in}{0.621721in}}%
\pgfpathlineto{\pgfqpoint{-0.000000in}{0.638889in}}%
\pgfpathlineto{\pgfqpoint{0.032655in}{0.621721in}}%
\pgfpathlineto{\pgfqpoint{0.026418in}{0.658083in}}%
\pgfpathlineto{\pgfqpoint{0.052836in}{0.683834in}}%
\pgfpathlineto{\pgfqpoint{0.016327in}{0.689139in}}%
\pgfpathlineto{\pgfqpoint{0.000000in}{0.722222in}}%
\pgfpathmoveto{\pgfqpoint{0.166667in}{0.722222in}}%
\pgfpathlineto{\pgfqpoint{0.150339in}{0.689139in}}%
\pgfpathlineto{\pgfqpoint{0.113830in}{0.683834in}}%
\pgfpathlineto{\pgfqpoint{0.140248in}{0.658083in}}%
\pgfpathlineto{\pgfqpoint{0.134012in}{0.621721in}}%
\pgfpathlineto{\pgfqpoint{0.166667in}{0.638889in}}%
\pgfpathlineto{\pgfqpoint{0.199321in}{0.621721in}}%
\pgfpathlineto{\pgfqpoint{0.193085in}{0.658083in}}%
\pgfpathlineto{\pgfqpoint{0.219503in}{0.683834in}}%
\pgfpathlineto{\pgfqpoint{0.182994in}{0.689139in}}%
\pgfpathlineto{\pgfqpoint{0.166667in}{0.722222in}}%
\pgfpathmoveto{\pgfqpoint{0.333333in}{0.722222in}}%
\pgfpathlineto{\pgfqpoint{0.317006in}{0.689139in}}%
\pgfpathlineto{\pgfqpoint{0.280497in}{0.683834in}}%
\pgfpathlineto{\pgfqpoint{0.306915in}{0.658083in}}%
\pgfpathlineto{\pgfqpoint{0.300679in}{0.621721in}}%
\pgfpathlineto{\pgfqpoint{0.333333in}{0.638889in}}%
\pgfpathlineto{\pgfqpoint{0.365988in}{0.621721in}}%
\pgfpathlineto{\pgfqpoint{0.359752in}{0.658083in}}%
\pgfpathlineto{\pgfqpoint{0.386170in}{0.683834in}}%
\pgfpathlineto{\pgfqpoint{0.349661in}{0.689139in}}%
\pgfpathlineto{\pgfqpoint{0.333333in}{0.722222in}}%
\pgfpathmoveto{\pgfqpoint{0.500000in}{0.722222in}}%
\pgfpathlineto{\pgfqpoint{0.483673in}{0.689139in}}%
\pgfpathlineto{\pgfqpoint{0.447164in}{0.683834in}}%
\pgfpathlineto{\pgfqpoint{0.473582in}{0.658083in}}%
\pgfpathlineto{\pgfqpoint{0.467345in}{0.621721in}}%
\pgfpathlineto{\pgfqpoint{0.500000in}{0.638889in}}%
\pgfpathlineto{\pgfqpoint{0.532655in}{0.621721in}}%
\pgfpathlineto{\pgfqpoint{0.526418in}{0.658083in}}%
\pgfpathlineto{\pgfqpoint{0.552836in}{0.683834in}}%
\pgfpathlineto{\pgfqpoint{0.516327in}{0.689139in}}%
\pgfpathlineto{\pgfqpoint{0.500000in}{0.722222in}}%
\pgfpathmoveto{\pgfqpoint{0.666667in}{0.722222in}}%
\pgfpathlineto{\pgfqpoint{0.650339in}{0.689139in}}%
\pgfpathlineto{\pgfqpoint{0.613830in}{0.683834in}}%
\pgfpathlineto{\pgfqpoint{0.640248in}{0.658083in}}%
\pgfpathlineto{\pgfqpoint{0.634012in}{0.621721in}}%
\pgfpathlineto{\pgfqpoint{0.666667in}{0.638889in}}%
\pgfpathlineto{\pgfqpoint{0.699321in}{0.621721in}}%
\pgfpathlineto{\pgfqpoint{0.693085in}{0.658083in}}%
\pgfpathlineto{\pgfqpoint{0.719503in}{0.683834in}}%
\pgfpathlineto{\pgfqpoint{0.682994in}{0.689139in}}%
\pgfpathlineto{\pgfqpoint{0.666667in}{0.722222in}}%
\pgfpathmoveto{\pgfqpoint{0.833333in}{0.722222in}}%
\pgfpathlineto{\pgfqpoint{0.817006in}{0.689139in}}%
\pgfpathlineto{\pgfqpoint{0.780497in}{0.683834in}}%
\pgfpathlineto{\pgfqpoint{0.806915in}{0.658083in}}%
\pgfpathlineto{\pgfqpoint{0.800679in}{0.621721in}}%
\pgfpathlineto{\pgfqpoint{0.833333in}{0.638889in}}%
\pgfpathlineto{\pgfqpoint{0.865988in}{0.621721in}}%
\pgfpathlineto{\pgfqpoint{0.859752in}{0.658083in}}%
\pgfpathlineto{\pgfqpoint{0.886170in}{0.683834in}}%
\pgfpathlineto{\pgfqpoint{0.849661in}{0.689139in}}%
\pgfpathlineto{\pgfqpoint{0.833333in}{0.722222in}}%
\pgfpathmoveto{\pgfqpoint{1.000000in}{0.722222in}}%
\pgfpathlineto{\pgfqpoint{0.983673in}{0.689139in}}%
\pgfpathlineto{\pgfqpoint{0.947164in}{0.683834in}}%
\pgfpathlineto{\pgfqpoint{0.973582in}{0.658083in}}%
\pgfpathlineto{\pgfqpoint{0.967345in}{0.621721in}}%
\pgfpathlineto{\pgfqpoint{1.000000in}{0.638889in}}%
\pgfpathlineto{\pgfqpoint{1.032655in}{0.621721in}}%
\pgfpathlineto{\pgfqpoint{1.026418in}{0.658083in}}%
\pgfpathlineto{\pgfqpoint{1.052836in}{0.683834in}}%
\pgfpathlineto{\pgfqpoint{1.016327in}{0.689139in}}%
\pgfpathlineto{\pgfqpoint{1.000000in}{0.722222in}}%
\pgfpathmoveto{\pgfqpoint{0.083333in}{0.888889in}}%
\pgfpathlineto{\pgfqpoint{0.067006in}{0.855806in}}%
\pgfpathlineto{\pgfqpoint{0.030497in}{0.850501in}}%
\pgfpathlineto{\pgfqpoint{0.056915in}{0.824750in}}%
\pgfpathlineto{\pgfqpoint{0.050679in}{0.788388in}}%
\pgfpathlineto{\pgfqpoint{0.083333in}{0.805556in}}%
\pgfpathlineto{\pgfqpoint{0.115988in}{0.788388in}}%
\pgfpathlineto{\pgfqpoint{0.109752in}{0.824750in}}%
\pgfpathlineto{\pgfqpoint{0.136170in}{0.850501in}}%
\pgfpathlineto{\pgfqpoint{0.099661in}{0.855806in}}%
\pgfpathlineto{\pgfqpoint{0.083333in}{0.888889in}}%
\pgfpathmoveto{\pgfqpoint{0.250000in}{0.888889in}}%
\pgfpathlineto{\pgfqpoint{0.233673in}{0.855806in}}%
\pgfpathlineto{\pgfqpoint{0.197164in}{0.850501in}}%
\pgfpathlineto{\pgfqpoint{0.223582in}{0.824750in}}%
\pgfpathlineto{\pgfqpoint{0.217345in}{0.788388in}}%
\pgfpathlineto{\pgfqpoint{0.250000in}{0.805556in}}%
\pgfpathlineto{\pgfqpoint{0.282655in}{0.788388in}}%
\pgfpathlineto{\pgfqpoint{0.276418in}{0.824750in}}%
\pgfpathlineto{\pgfqpoint{0.302836in}{0.850501in}}%
\pgfpathlineto{\pgfqpoint{0.266327in}{0.855806in}}%
\pgfpathlineto{\pgfqpoint{0.250000in}{0.888889in}}%
\pgfpathmoveto{\pgfqpoint{0.416667in}{0.888889in}}%
\pgfpathlineto{\pgfqpoint{0.400339in}{0.855806in}}%
\pgfpathlineto{\pgfqpoint{0.363830in}{0.850501in}}%
\pgfpathlineto{\pgfqpoint{0.390248in}{0.824750in}}%
\pgfpathlineto{\pgfqpoint{0.384012in}{0.788388in}}%
\pgfpathlineto{\pgfqpoint{0.416667in}{0.805556in}}%
\pgfpathlineto{\pgfqpoint{0.449321in}{0.788388in}}%
\pgfpathlineto{\pgfqpoint{0.443085in}{0.824750in}}%
\pgfpathlineto{\pgfqpoint{0.469503in}{0.850501in}}%
\pgfpathlineto{\pgfqpoint{0.432994in}{0.855806in}}%
\pgfpathlineto{\pgfqpoint{0.416667in}{0.888889in}}%
\pgfpathmoveto{\pgfqpoint{0.583333in}{0.888889in}}%
\pgfpathlineto{\pgfqpoint{0.567006in}{0.855806in}}%
\pgfpathlineto{\pgfqpoint{0.530497in}{0.850501in}}%
\pgfpathlineto{\pgfqpoint{0.556915in}{0.824750in}}%
\pgfpathlineto{\pgfqpoint{0.550679in}{0.788388in}}%
\pgfpathlineto{\pgfqpoint{0.583333in}{0.805556in}}%
\pgfpathlineto{\pgfqpoint{0.615988in}{0.788388in}}%
\pgfpathlineto{\pgfqpoint{0.609752in}{0.824750in}}%
\pgfpathlineto{\pgfqpoint{0.636170in}{0.850501in}}%
\pgfpathlineto{\pgfqpoint{0.599661in}{0.855806in}}%
\pgfpathlineto{\pgfqpoint{0.583333in}{0.888889in}}%
\pgfpathmoveto{\pgfqpoint{0.750000in}{0.888889in}}%
\pgfpathlineto{\pgfqpoint{0.733673in}{0.855806in}}%
\pgfpathlineto{\pgfqpoint{0.697164in}{0.850501in}}%
\pgfpathlineto{\pgfqpoint{0.723582in}{0.824750in}}%
\pgfpathlineto{\pgfqpoint{0.717345in}{0.788388in}}%
\pgfpathlineto{\pgfqpoint{0.750000in}{0.805556in}}%
\pgfpathlineto{\pgfqpoint{0.782655in}{0.788388in}}%
\pgfpathlineto{\pgfqpoint{0.776418in}{0.824750in}}%
\pgfpathlineto{\pgfqpoint{0.802836in}{0.850501in}}%
\pgfpathlineto{\pgfqpoint{0.766327in}{0.855806in}}%
\pgfpathlineto{\pgfqpoint{0.750000in}{0.888889in}}%
\pgfpathmoveto{\pgfqpoint{0.916667in}{0.888889in}}%
\pgfpathlineto{\pgfqpoint{0.900339in}{0.855806in}}%
\pgfpathlineto{\pgfqpoint{0.863830in}{0.850501in}}%
\pgfpathlineto{\pgfqpoint{0.890248in}{0.824750in}}%
\pgfpathlineto{\pgfqpoint{0.884012in}{0.788388in}}%
\pgfpathlineto{\pgfqpoint{0.916667in}{0.805556in}}%
\pgfpathlineto{\pgfqpoint{0.949321in}{0.788388in}}%
\pgfpathlineto{\pgfqpoint{0.943085in}{0.824750in}}%
\pgfpathlineto{\pgfqpoint{0.969503in}{0.850501in}}%
\pgfpathlineto{\pgfqpoint{0.932994in}{0.855806in}}%
\pgfpathlineto{\pgfqpoint{0.916667in}{0.888889in}}%
\pgfpathmoveto{\pgfqpoint{0.000000in}{1.055556in}}%
\pgfpathlineto{\pgfqpoint{-0.016327in}{1.022473in}}%
\pgfpathlineto{\pgfqpoint{-0.052836in}{1.017168in}}%
\pgfpathlineto{\pgfqpoint{-0.026418in}{0.991416in}}%
\pgfpathlineto{\pgfqpoint{-0.032655in}{0.955055in}}%
\pgfpathlineto{\pgfqpoint{-0.000000in}{0.972222in}}%
\pgfpathlineto{\pgfqpoint{0.032655in}{0.955055in}}%
\pgfpathlineto{\pgfqpoint{0.026418in}{0.991416in}}%
\pgfpathlineto{\pgfqpoint{0.052836in}{1.017168in}}%
\pgfpathlineto{\pgfqpoint{0.016327in}{1.022473in}}%
\pgfpathlineto{\pgfqpoint{0.000000in}{1.055556in}}%
\pgfpathmoveto{\pgfqpoint{0.166667in}{1.055556in}}%
\pgfpathlineto{\pgfqpoint{0.150339in}{1.022473in}}%
\pgfpathlineto{\pgfqpoint{0.113830in}{1.017168in}}%
\pgfpathlineto{\pgfqpoint{0.140248in}{0.991416in}}%
\pgfpathlineto{\pgfqpoint{0.134012in}{0.955055in}}%
\pgfpathlineto{\pgfqpoint{0.166667in}{0.972222in}}%
\pgfpathlineto{\pgfqpoint{0.199321in}{0.955055in}}%
\pgfpathlineto{\pgfqpoint{0.193085in}{0.991416in}}%
\pgfpathlineto{\pgfqpoint{0.219503in}{1.017168in}}%
\pgfpathlineto{\pgfqpoint{0.182994in}{1.022473in}}%
\pgfpathlineto{\pgfqpoint{0.166667in}{1.055556in}}%
\pgfpathmoveto{\pgfqpoint{0.333333in}{1.055556in}}%
\pgfpathlineto{\pgfqpoint{0.317006in}{1.022473in}}%
\pgfpathlineto{\pgfqpoint{0.280497in}{1.017168in}}%
\pgfpathlineto{\pgfqpoint{0.306915in}{0.991416in}}%
\pgfpathlineto{\pgfqpoint{0.300679in}{0.955055in}}%
\pgfpathlineto{\pgfqpoint{0.333333in}{0.972222in}}%
\pgfpathlineto{\pgfqpoint{0.365988in}{0.955055in}}%
\pgfpathlineto{\pgfqpoint{0.359752in}{0.991416in}}%
\pgfpathlineto{\pgfqpoint{0.386170in}{1.017168in}}%
\pgfpathlineto{\pgfqpoint{0.349661in}{1.022473in}}%
\pgfpathlineto{\pgfqpoint{0.333333in}{1.055556in}}%
\pgfpathmoveto{\pgfqpoint{0.500000in}{1.055556in}}%
\pgfpathlineto{\pgfqpoint{0.483673in}{1.022473in}}%
\pgfpathlineto{\pgfqpoint{0.447164in}{1.017168in}}%
\pgfpathlineto{\pgfqpoint{0.473582in}{0.991416in}}%
\pgfpathlineto{\pgfqpoint{0.467345in}{0.955055in}}%
\pgfpathlineto{\pgfqpoint{0.500000in}{0.972222in}}%
\pgfpathlineto{\pgfqpoint{0.532655in}{0.955055in}}%
\pgfpathlineto{\pgfqpoint{0.526418in}{0.991416in}}%
\pgfpathlineto{\pgfqpoint{0.552836in}{1.017168in}}%
\pgfpathlineto{\pgfqpoint{0.516327in}{1.022473in}}%
\pgfpathlineto{\pgfqpoint{0.500000in}{1.055556in}}%
\pgfpathmoveto{\pgfqpoint{0.666667in}{1.055556in}}%
\pgfpathlineto{\pgfqpoint{0.650339in}{1.022473in}}%
\pgfpathlineto{\pgfqpoint{0.613830in}{1.017168in}}%
\pgfpathlineto{\pgfqpoint{0.640248in}{0.991416in}}%
\pgfpathlineto{\pgfqpoint{0.634012in}{0.955055in}}%
\pgfpathlineto{\pgfqpoint{0.666667in}{0.972222in}}%
\pgfpathlineto{\pgfqpoint{0.699321in}{0.955055in}}%
\pgfpathlineto{\pgfqpoint{0.693085in}{0.991416in}}%
\pgfpathlineto{\pgfqpoint{0.719503in}{1.017168in}}%
\pgfpathlineto{\pgfqpoint{0.682994in}{1.022473in}}%
\pgfpathlineto{\pgfqpoint{0.666667in}{1.055556in}}%
\pgfpathmoveto{\pgfqpoint{0.833333in}{1.055556in}}%
\pgfpathlineto{\pgfqpoint{0.817006in}{1.022473in}}%
\pgfpathlineto{\pgfqpoint{0.780497in}{1.017168in}}%
\pgfpathlineto{\pgfqpoint{0.806915in}{0.991416in}}%
\pgfpathlineto{\pgfqpoint{0.800679in}{0.955055in}}%
\pgfpathlineto{\pgfqpoint{0.833333in}{0.972222in}}%
\pgfpathlineto{\pgfqpoint{0.865988in}{0.955055in}}%
\pgfpathlineto{\pgfqpoint{0.859752in}{0.991416in}}%
\pgfpathlineto{\pgfqpoint{0.886170in}{1.017168in}}%
\pgfpathlineto{\pgfqpoint{0.849661in}{1.022473in}}%
\pgfpathlineto{\pgfqpoint{0.833333in}{1.055556in}}%
\pgfpathmoveto{\pgfqpoint{1.000000in}{1.055556in}}%
\pgfpathlineto{\pgfqpoint{0.983673in}{1.022473in}}%
\pgfpathlineto{\pgfqpoint{0.947164in}{1.017168in}}%
\pgfpathlineto{\pgfqpoint{0.973582in}{0.991416in}}%
\pgfpathlineto{\pgfqpoint{0.967345in}{0.955055in}}%
\pgfpathlineto{\pgfqpoint{1.000000in}{0.972222in}}%
\pgfpathlineto{\pgfqpoint{1.032655in}{0.955055in}}%
\pgfpathlineto{\pgfqpoint{1.026418in}{0.991416in}}%
\pgfpathlineto{\pgfqpoint{1.052836in}{1.017168in}}%
\pgfpathlineto{\pgfqpoint{1.016327in}{1.022473in}}%
\pgfpathlineto{\pgfqpoint{1.000000in}{1.055556in}}%
\pgfpathlineto{\pgfqpoint{1.000000in}{1.055556in}}%
\pgfusepath{stroke}%
\end{pgfscope}%
}%
\pgfsys@transformshift{1.070538in}{8.032990in}%
\pgfsys@useobject{currentpattern}{}%
\pgfsys@transformshift{1in}{0in}%
\pgfsys@transformshift{-1in}{0in}%
\pgfsys@transformshift{0in}{1in}%
\end{pgfscope}%
\begin{pgfscope}%
\definecolor{textcolor}{rgb}{0.000000,0.000000,0.000000}%
\pgfsetstrokecolor{textcolor}%
\pgfsetfillcolor{textcolor}%
\pgftext[x=1.692760in,y=8.032990in,left,base]{\color{textcolor}\rmfamily\fontsize{16.000000}{19.200000}\selectfont SOLAR\_FARM}%
\end{pgfscope}%
\begin{pgfscope}%
\pgfsetbuttcap%
\pgfsetmiterjoin%
\definecolor{currentfill}{rgb}{0.121569,0.466667,0.705882}%
\pgfsetfillcolor{currentfill}%
\pgfsetfillopacity{0.990000}%
\pgfsetlinewidth{0.000000pt}%
\definecolor{currentstroke}{rgb}{0.000000,0.000000,0.000000}%
\pgfsetstrokecolor{currentstroke}%
\pgfsetstrokeopacity{0.990000}%
\pgfsetdash{}{0pt}%
\pgfpathmoveto{\pgfqpoint{1.070538in}{7.708530in}}%
\pgfpathlineto{\pgfqpoint{1.514982in}{7.708530in}}%
\pgfpathlineto{\pgfqpoint{1.514982in}{7.864085in}}%
\pgfpathlineto{\pgfqpoint{1.070538in}{7.864085in}}%
\pgfpathclose%
\pgfusepath{fill}%
\end{pgfscope}%
\begin{pgfscope}%
\pgfsetbuttcap%
\pgfsetmiterjoin%
\definecolor{currentfill}{rgb}{0.121569,0.466667,0.705882}%
\pgfsetfillcolor{currentfill}%
\pgfsetfillopacity{0.990000}%
\pgfsetlinewidth{0.000000pt}%
\definecolor{currentstroke}{rgb}{0.000000,0.000000,0.000000}%
\pgfsetstrokecolor{currentstroke}%
\pgfsetstrokeopacity{0.990000}%
\pgfsetdash{}{0pt}%
\pgfpathmoveto{\pgfqpoint{1.070538in}{7.708530in}}%
\pgfpathlineto{\pgfqpoint{1.514982in}{7.708530in}}%
\pgfpathlineto{\pgfqpoint{1.514982in}{7.864085in}}%
\pgfpathlineto{\pgfqpoint{1.070538in}{7.864085in}}%
\pgfpathclose%
\pgfusepath{clip}%
\pgfsys@defobject{currentpattern}{\pgfqpoint{0in}{0in}}{\pgfqpoint{1in}{1in}}{%
\begin{pgfscope}%
\pgfpathrectangle{\pgfqpoint{0in}{0in}}{\pgfqpoint{1in}{1in}}%
\pgfusepath{clip}%
\pgfpathmoveto{\pgfqpoint{0.000000in}{0.083333in}}%
\pgfpathlineto{\pgfqpoint{1.000000in}{0.083333in}}%
\pgfpathmoveto{\pgfqpoint{0.000000in}{0.250000in}}%
\pgfpathlineto{\pgfqpoint{1.000000in}{0.250000in}}%
\pgfpathmoveto{\pgfqpoint{0.000000in}{0.416667in}}%
\pgfpathlineto{\pgfqpoint{1.000000in}{0.416667in}}%
\pgfpathmoveto{\pgfqpoint{0.000000in}{0.583333in}}%
\pgfpathlineto{\pgfqpoint{1.000000in}{0.583333in}}%
\pgfpathmoveto{\pgfqpoint{0.000000in}{0.750000in}}%
\pgfpathlineto{\pgfqpoint{1.000000in}{0.750000in}}%
\pgfpathmoveto{\pgfqpoint{0.000000in}{0.916667in}}%
\pgfpathlineto{\pgfqpoint{1.000000in}{0.916667in}}%
\pgfpathmoveto{\pgfqpoint{0.083333in}{0.000000in}}%
\pgfpathlineto{\pgfqpoint{0.083333in}{1.000000in}}%
\pgfpathmoveto{\pgfqpoint{0.250000in}{0.000000in}}%
\pgfpathlineto{\pgfqpoint{0.250000in}{1.000000in}}%
\pgfpathmoveto{\pgfqpoint{0.416667in}{0.000000in}}%
\pgfpathlineto{\pgfqpoint{0.416667in}{1.000000in}}%
\pgfpathmoveto{\pgfqpoint{0.583333in}{0.000000in}}%
\pgfpathlineto{\pgfqpoint{0.583333in}{1.000000in}}%
\pgfpathmoveto{\pgfqpoint{0.750000in}{0.000000in}}%
\pgfpathlineto{\pgfqpoint{0.750000in}{1.000000in}}%
\pgfpathmoveto{\pgfqpoint{0.916667in}{0.000000in}}%
\pgfpathlineto{\pgfqpoint{0.916667in}{1.000000in}}%
\pgfusepath{stroke}%
\end{pgfscope}%
}%
\pgfsys@transformshift{1.070538in}{7.708530in}%
\pgfsys@useobject{currentpattern}{}%
\pgfsys@transformshift{1in}{0in}%
\pgfsys@transformshift{-1in}{0in}%
\pgfsys@transformshift{0in}{1in}%
\end{pgfscope}%
\begin{pgfscope}%
\definecolor{textcolor}{rgb}{0.000000,0.000000,0.000000}%
\pgfsetstrokecolor{textcolor}%
\pgfsetfillcolor{textcolor}%
\pgftext[x=1.692760in,y=7.708530in,left,base]{\color{textcolor}\rmfamily\fontsize{16.000000}{19.200000}\selectfont WIND\_FARM}%
\end{pgfscope}%
\end{pgfpicture}%
\makeatother%
\endgroup%
}
    \caption[]{Least cost electricity generation.}
    \label{fig:uiuc_elc_gen}
  \end{minipage}
\end{figure}

Lastly, the wind capacity immediately jumps to 100 MW by 2025. The model was
constrained to 100.5 MW because \gls{uiuc} has a power purchase agreement with
Rail Splitter Wind Farm, for 8\% of its output, which has an existing capacity of
100.5 MW  \cite{breitweiser_wind_2016}. The model recommends 100 MW of wind
capacity because wind energy is cheap and has significant built-in flexibility
through other electricity imports.

Figure \ref{fig:uiuc_thm_cap} and Figure \ref{fig:uiuc_chw_cap} show the capacities
for steam production and refrigeration, respectively. There is no carbon-free
alternative to produce steam besides a nuclear reactor using the technologies
in this model. Figure
\ref{fig:uiuc_chw_cap} shows that chilled water capacity increases with annual
cooling demand, and the chilled water storage is roughly constant.

\begin{figure}[H]
  \begin{minipage}{0.48\textwidth}
    \captionsetup{type=figure}
    \centering
    \resizebox{\columnwidth}{!}{%% Creator: Matplotlib, PGF backend
%%
%% To include the figure in your LaTeX document, write
%%   \input{<filename>.pgf}
%%
%% Make sure the required packages are loaded in your preamble
%%   \usepackage{pgf}
%%
%% Figures using additional raster images can only be included by \input if
%% they are in the same directory as the main LaTeX file. For loading figures
%% from other directories you can use the `import` package
%%   \usepackage{import}
%%
%% and then include the figures with
%%   \import{<path to file>}{<filename>.pgf}
%%
%% Matplotlib used the following preamble
%%
\begingroup%
\makeatletter%
\begin{pgfpicture}%
\pgfpathrectangle{\pgfpointorigin}{\pgfqpoint{10.335815in}{10.120798in}}%
\pgfusepath{use as bounding box, clip}%
\begin{pgfscope}%
\pgfsetbuttcap%
\pgfsetmiterjoin%
\definecolor{currentfill}{rgb}{1.000000,1.000000,1.000000}%
\pgfsetfillcolor{currentfill}%
\pgfsetlinewidth{0.000000pt}%
\definecolor{currentstroke}{rgb}{0.000000,0.000000,0.000000}%
\pgfsetstrokecolor{currentstroke}%
\pgfsetdash{}{0pt}%
\pgfpathmoveto{\pgfqpoint{0.000000in}{0.000000in}}%
\pgfpathlineto{\pgfqpoint{10.335815in}{0.000000in}}%
\pgfpathlineto{\pgfqpoint{10.335815in}{10.120798in}}%
\pgfpathlineto{\pgfqpoint{0.000000in}{10.120798in}}%
\pgfpathclose%
\pgfusepath{fill}%
\end{pgfscope}%
\begin{pgfscope}%
\pgfsetbuttcap%
\pgfsetmiterjoin%
\definecolor{currentfill}{rgb}{0.898039,0.898039,0.898039}%
\pgfsetfillcolor{currentfill}%
\pgfsetlinewidth{0.000000pt}%
\definecolor{currentstroke}{rgb}{0.000000,0.000000,0.000000}%
\pgfsetstrokecolor{currentstroke}%
\pgfsetstrokeopacity{0.000000}%
\pgfsetdash{}{0pt}%
\pgfpathmoveto{\pgfqpoint{0.935815in}{0.637495in}}%
\pgfpathlineto{\pgfqpoint{10.235815in}{0.637495in}}%
\pgfpathlineto{\pgfqpoint{10.235815in}{9.697495in}}%
\pgfpathlineto{\pgfqpoint{0.935815in}{9.697495in}}%
\pgfpathclose%
\pgfusepath{fill}%
\end{pgfscope}%
\begin{pgfscope}%
\pgfpathrectangle{\pgfqpoint{0.935815in}{0.637495in}}{\pgfqpoint{9.300000in}{9.060000in}}%
\pgfusepath{clip}%
\pgfsetrectcap%
\pgfsetroundjoin%
\pgfsetlinewidth{0.803000pt}%
\definecolor{currentstroke}{rgb}{1.000000,1.000000,1.000000}%
\pgfsetstrokecolor{currentstroke}%
\pgfsetdash{}{0pt}%
\pgfpathmoveto{\pgfqpoint{1.710815in}{0.637495in}}%
\pgfpathlineto{\pgfqpoint{1.710815in}{9.697495in}}%
\pgfusepath{stroke}%
\end{pgfscope}%
\begin{pgfscope}%
\pgfsetbuttcap%
\pgfsetroundjoin%
\definecolor{currentfill}{rgb}{0.333333,0.333333,0.333333}%
\pgfsetfillcolor{currentfill}%
\pgfsetlinewidth{0.803000pt}%
\definecolor{currentstroke}{rgb}{0.333333,0.333333,0.333333}%
\pgfsetstrokecolor{currentstroke}%
\pgfsetdash{}{0pt}%
\pgfsys@defobject{currentmarker}{\pgfqpoint{0.000000in}{-0.048611in}}{\pgfqpoint{0.000000in}{0.000000in}}{%
\pgfpathmoveto{\pgfqpoint{0.000000in}{0.000000in}}%
\pgfpathlineto{\pgfqpoint{0.000000in}{-0.048611in}}%
\pgfusepath{stroke,fill}%
}%
\begin{pgfscope}%
\pgfsys@transformshift{1.710815in}{0.637495in}%
\pgfsys@useobject{currentmarker}{}%
\end{pgfscope}%
\end{pgfscope}%
\begin{pgfscope}%
\definecolor{textcolor}{rgb}{0.333333,0.333333,0.333333}%
\pgfsetstrokecolor{textcolor}%
\pgfsetfillcolor{textcolor}%
\pgftext[x=1.770807in, y=0.100000in, left, base,rotate=90.000000]{\color{textcolor}\rmfamily\fontsize{16.000000}{19.200000}\selectfont 2025}%
\end{pgfscope}%
\begin{pgfscope}%
\pgfpathrectangle{\pgfqpoint{0.935815in}{0.637495in}}{\pgfqpoint{9.300000in}{9.060000in}}%
\pgfusepath{clip}%
\pgfsetrectcap%
\pgfsetroundjoin%
\pgfsetlinewidth{0.803000pt}%
\definecolor{currentstroke}{rgb}{1.000000,1.000000,1.000000}%
\pgfsetstrokecolor{currentstroke}%
\pgfsetdash{}{0pt}%
\pgfpathmoveto{\pgfqpoint{3.260815in}{0.637495in}}%
\pgfpathlineto{\pgfqpoint{3.260815in}{9.697495in}}%
\pgfusepath{stroke}%
\end{pgfscope}%
\begin{pgfscope}%
\pgfsetbuttcap%
\pgfsetroundjoin%
\definecolor{currentfill}{rgb}{0.333333,0.333333,0.333333}%
\pgfsetfillcolor{currentfill}%
\pgfsetlinewidth{0.803000pt}%
\definecolor{currentstroke}{rgb}{0.333333,0.333333,0.333333}%
\pgfsetstrokecolor{currentstroke}%
\pgfsetdash{}{0pt}%
\pgfsys@defobject{currentmarker}{\pgfqpoint{0.000000in}{-0.048611in}}{\pgfqpoint{0.000000in}{0.000000in}}{%
\pgfpathmoveto{\pgfqpoint{0.000000in}{0.000000in}}%
\pgfpathlineto{\pgfqpoint{0.000000in}{-0.048611in}}%
\pgfusepath{stroke,fill}%
}%
\begin{pgfscope}%
\pgfsys@transformshift{3.260815in}{0.637495in}%
\pgfsys@useobject{currentmarker}{}%
\end{pgfscope}%
\end{pgfscope}%
\begin{pgfscope}%
\definecolor{textcolor}{rgb}{0.333333,0.333333,0.333333}%
\pgfsetstrokecolor{textcolor}%
\pgfsetfillcolor{textcolor}%
\pgftext[x=3.320807in, y=0.100000in, left, base,rotate=90.000000]{\color{textcolor}\rmfamily\fontsize{16.000000}{19.200000}\selectfont 2030}%
\end{pgfscope}%
\begin{pgfscope}%
\pgfpathrectangle{\pgfqpoint{0.935815in}{0.637495in}}{\pgfqpoint{9.300000in}{9.060000in}}%
\pgfusepath{clip}%
\pgfsetrectcap%
\pgfsetroundjoin%
\pgfsetlinewidth{0.803000pt}%
\definecolor{currentstroke}{rgb}{1.000000,1.000000,1.000000}%
\pgfsetstrokecolor{currentstroke}%
\pgfsetdash{}{0pt}%
\pgfpathmoveto{\pgfqpoint{4.810815in}{0.637495in}}%
\pgfpathlineto{\pgfqpoint{4.810815in}{9.697495in}}%
\pgfusepath{stroke}%
\end{pgfscope}%
\begin{pgfscope}%
\pgfsetbuttcap%
\pgfsetroundjoin%
\definecolor{currentfill}{rgb}{0.333333,0.333333,0.333333}%
\pgfsetfillcolor{currentfill}%
\pgfsetlinewidth{0.803000pt}%
\definecolor{currentstroke}{rgb}{0.333333,0.333333,0.333333}%
\pgfsetstrokecolor{currentstroke}%
\pgfsetdash{}{0pt}%
\pgfsys@defobject{currentmarker}{\pgfqpoint{0.000000in}{-0.048611in}}{\pgfqpoint{0.000000in}{0.000000in}}{%
\pgfpathmoveto{\pgfqpoint{0.000000in}{0.000000in}}%
\pgfpathlineto{\pgfqpoint{0.000000in}{-0.048611in}}%
\pgfusepath{stroke,fill}%
}%
\begin{pgfscope}%
\pgfsys@transformshift{4.810815in}{0.637495in}%
\pgfsys@useobject{currentmarker}{}%
\end{pgfscope}%
\end{pgfscope}%
\begin{pgfscope}%
\definecolor{textcolor}{rgb}{0.333333,0.333333,0.333333}%
\pgfsetstrokecolor{textcolor}%
\pgfsetfillcolor{textcolor}%
\pgftext[x=4.870807in, y=0.100000in, left, base,rotate=90.000000]{\color{textcolor}\rmfamily\fontsize{16.000000}{19.200000}\selectfont 2035}%
\end{pgfscope}%
\begin{pgfscope}%
\pgfpathrectangle{\pgfqpoint{0.935815in}{0.637495in}}{\pgfqpoint{9.300000in}{9.060000in}}%
\pgfusepath{clip}%
\pgfsetrectcap%
\pgfsetroundjoin%
\pgfsetlinewidth{0.803000pt}%
\definecolor{currentstroke}{rgb}{1.000000,1.000000,1.000000}%
\pgfsetstrokecolor{currentstroke}%
\pgfsetdash{}{0pt}%
\pgfpathmoveto{\pgfqpoint{6.360815in}{0.637495in}}%
\pgfpathlineto{\pgfqpoint{6.360815in}{9.697495in}}%
\pgfusepath{stroke}%
\end{pgfscope}%
\begin{pgfscope}%
\pgfsetbuttcap%
\pgfsetroundjoin%
\definecolor{currentfill}{rgb}{0.333333,0.333333,0.333333}%
\pgfsetfillcolor{currentfill}%
\pgfsetlinewidth{0.803000pt}%
\definecolor{currentstroke}{rgb}{0.333333,0.333333,0.333333}%
\pgfsetstrokecolor{currentstroke}%
\pgfsetdash{}{0pt}%
\pgfsys@defobject{currentmarker}{\pgfqpoint{0.000000in}{-0.048611in}}{\pgfqpoint{0.000000in}{0.000000in}}{%
\pgfpathmoveto{\pgfqpoint{0.000000in}{0.000000in}}%
\pgfpathlineto{\pgfqpoint{0.000000in}{-0.048611in}}%
\pgfusepath{stroke,fill}%
}%
\begin{pgfscope}%
\pgfsys@transformshift{6.360815in}{0.637495in}%
\pgfsys@useobject{currentmarker}{}%
\end{pgfscope}%
\end{pgfscope}%
\begin{pgfscope}%
\definecolor{textcolor}{rgb}{0.333333,0.333333,0.333333}%
\pgfsetstrokecolor{textcolor}%
\pgfsetfillcolor{textcolor}%
\pgftext[x=6.420807in, y=0.100000in, left, base,rotate=90.000000]{\color{textcolor}\rmfamily\fontsize{16.000000}{19.200000}\selectfont 2040}%
\end{pgfscope}%
\begin{pgfscope}%
\pgfpathrectangle{\pgfqpoint{0.935815in}{0.637495in}}{\pgfqpoint{9.300000in}{9.060000in}}%
\pgfusepath{clip}%
\pgfsetrectcap%
\pgfsetroundjoin%
\pgfsetlinewidth{0.803000pt}%
\definecolor{currentstroke}{rgb}{1.000000,1.000000,1.000000}%
\pgfsetstrokecolor{currentstroke}%
\pgfsetdash{}{0pt}%
\pgfpathmoveto{\pgfqpoint{7.910815in}{0.637495in}}%
\pgfpathlineto{\pgfqpoint{7.910815in}{9.697495in}}%
\pgfusepath{stroke}%
\end{pgfscope}%
\begin{pgfscope}%
\pgfsetbuttcap%
\pgfsetroundjoin%
\definecolor{currentfill}{rgb}{0.333333,0.333333,0.333333}%
\pgfsetfillcolor{currentfill}%
\pgfsetlinewidth{0.803000pt}%
\definecolor{currentstroke}{rgb}{0.333333,0.333333,0.333333}%
\pgfsetstrokecolor{currentstroke}%
\pgfsetdash{}{0pt}%
\pgfsys@defobject{currentmarker}{\pgfqpoint{0.000000in}{-0.048611in}}{\pgfqpoint{0.000000in}{0.000000in}}{%
\pgfpathmoveto{\pgfqpoint{0.000000in}{0.000000in}}%
\pgfpathlineto{\pgfqpoint{0.000000in}{-0.048611in}}%
\pgfusepath{stroke,fill}%
}%
\begin{pgfscope}%
\pgfsys@transformshift{7.910815in}{0.637495in}%
\pgfsys@useobject{currentmarker}{}%
\end{pgfscope}%
\end{pgfscope}%
\begin{pgfscope}%
\definecolor{textcolor}{rgb}{0.333333,0.333333,0.333333}%
\pgfsetstrokecolor{textcolor}%
\pgfsetfillcolor{textcolor}%
\pgftext[x=7.970807in, y=0.100000in, left, base,rotate=90.000000]{\color{textcolor}\rmfamily\fontsize{16.000000}{19.200000}\selectfont 2045}%
\end{pgfscope}%
\begin{pgfscope}%
\pgfpathrectangle{\pgfqpoint{0.935815in}{0.637495in}}{\pgfqpoint{9.300000in}{9.060000in}}%
\pgfusepath{clip}%
\pgfsetrectcap%
\pgfsetroundjoin%
\pgfsetlinewidth{0.803000pt}%
\definecolor{currentstroke}{rgb}{1.000000,1.000000,1.000000}%
\pgfsetstrokecolor{currentstroke}%
\pgfsetdash{}{0pt}%
\pgfpathmoveto{\pgfqpoint{9.460815in}{0.637495in}}%
\pgfpathlineto{\pgfqpoint{9.460815in}{9.697495in}}%
\pgfusepath{stroke}%
\end{pgfscope}%
\begin{pgfscope}%
\pgfsetbuttcap%
\pgfsetroundjoin%
\definecolor{currentfill}{rgb}{0.333333,0.333333,0.333333}%
\pgfsetfillcolor{currentfill}%
\pgfsetlinewidth{0.803000pt}%
\definecolor{currentstroke}{rgb}{0.333333,0.333333,0.333333}%
\pgfsetstrokecolor{currentstroke}%
\pgfsetdash{}{0pt}%
\pgfsys@defobject{currentmarker}{\pgfqpoint{0.000000in}{-0.048611in}}{\pgfqpoint{0.000000in}{0.000000in}}{%
\pgfpathmoveto{\pgfqpoint{0.000000in}{0.000000in}}%
\pgfpathlineto{\pgfqpoint{0.000000in}{-0.048611in}}%
\pgfusepath{stroke,fill}%
}%
\begin{pgfscope}%
\pgfsys@transformshift{9.460815in}{0.637495in}%
\pgfsys@useobject{currentmarker}{}%
\end{pgfscope}%
\end{pgfscope}%
\begin{pgfscope}%
\definecolor{textcolor}{rgb}{0.333333,0.333333,0.333333}%
\pgfsetstrokecolor{textcolor}%
\pgfsetfillcolor{textcolor}%
\pgftext[x=9.520807in, y=0.100000in, left, base,rotate=90.000000]{\color{textcolor}\rmfamily\fontsize{16.000000}{19.200000}\selectfont 2050}%
\end{pgfscope}%
\begin{pgfscope}%
\pgfpathrectangle{\pgfqpoint{0.935815in}{0.637495in}}{\pgfqpoint{9.300000in}{9.060000in}}%
\pgfusepath{clip}%
\pgfsetrectcap%
\pgfsetroundjoin%
\pgfsetlinewidth{0.803000pt}%
\definecolor{currentstroke}{rgb}{1.000000,1.000000,1.000000}%
\pgfsetstrokecolor{currentstroke}%
\pgfsetdash{}{0pt}%
\pgfpathmoveto{\pgfqpoint{0.935815in}{0.637495in}}%
\pgfpathlineto{\pgfqpoint{10.235815in}{0.637495in}}%
\pgfusepath{stroke}%
\end{pgfscope}%
\begin{pgfscope}%
\pgfsetbuttcap%
\pgfsetroundjoin%
\definecolor{currentfill}{rgb}{0.333333,0.333333,0.333333}%
\pgfsetfillcolor{currentfill}%
\pgfsetlinewidth{0.803000pt}%
\definecolor{currentstroke}{rgb}{0.333333,0.333333,0.333333}%
\pgfsetstrokecolor{currentstroke}%
\pgfsetdash{}{0pt}%
\pgfsys@defobject{currentmarker}{\pgfqpoint{-0.048611in}{0.000000in}}{\pgfqpoint{-0.000000in}{0.000000in}}{%
\pgfpathmoveto{\pgfqpoint{-0.000000in}{0.000000in}}%
\pgfpathlineto{\pgfqpoint{-0.048611in}{0.000000in}}%
\pgfusepath{stroke,fill}%
}%
\begin{pgfscope}%
\pgfsys@transformshift{0.935815in}{0.637495in}%
\pgfsys@useobject{currentmarker}{}%
\end{pgfscope}%
\end{pgfscope}%
\begin{pgfscope}%
\definecolor{textcolor}{rgb}{0.333333,0.333333,0.333333}%
\pgfsetstrokecolor{textcolor}%
\pgfsetfillcolor{textcolor}%
\pgftext[x=0.443111in, y=0.554162in, left, base]{\color{textcolor}\rmfamily\fontsize{16.000000}{19.200000}\selectfont \(\displaystyle {0.00}\)}%
\end{pgfscope}%
\begin{pgfscope}%
\pgfpathrectangle{\pgfqpoint{0.935815in}{0.637495in}}{\pgfqpoint{9.300000in}{9.060000in}}%
\pgfusepath{clip}%
\pgfsetrectcap%
\pgfsetroundjoin%
\pgfsetlinewidth{0.803000pt}%
\definecolor{currentstroke}{rgb}{1.000000,1.000000,1.000000}%
\pgfsetstrokecolor{currentstroke}%
\pgfsetdash{}{0pt}%
\pgfpathmoveto{\pgfqpoint{0.935815in}{2.290288in}}%
\pgfpathlineto{\pgfqpoint{10.235815in}{2.290288in}}%
\pgfusepath{stroke}%
\end{pgfscope}%
\begin{pgfscope}%
\pgfsetbuttcap%
\pgfsetroundjoin%
\definecolor{currentfill}{rgb}{0.333333,0.333333,0.333333}%
\pgfsetfillcolor{currentfill}%
\pgfsetlinewidth{0.803000pt}%
\definecolor{currentstroke}{rgb}{0.333333,0.333333,0.333333}%
\pgfsetstrokecolor{currentstroke}%
\pgfsetdash{}{0pt}%
\pgfsys@defobject{currentmarker}{\pgfqpoint{-0.048611in}{0.000000in}}{\pgfqpoint{-0.000000in}{0.000000in}}{%
\pgfpathmoveto{\pgfqpoint{-0.000000in}{0.000000in}}%
\pgfpathlineto{\pgfqpoint{-0.048611in}{0.000000in}}%
\pgfusepath{stroke,fill}%
}%
\begin{pgfscope}%
\pgfsys@transformshift{0.935815in}{2.290288in}%
\pgfsys@useobject{currentmarker}{}%
\end{pgfscope}%
\end{pgfscope}%
\begin{pgfscope}%
\definecolor{textcolor}{rgb}{0.333333,0.333333,0.333333}%
\pgfsetstrokecolor{textcolor}%
\pgfsetfillcolor{textcolor}%
\pgftext[x=0.443111in, y=2.206955in, left, base]{\color{textcolor}\rmfamily\fontsize{16.000000}{19.200000}\selectfont \(\displaystyle {0.05}\)}%
\end{pgfscope}%
\begin{pgfscope}%
\pgfpathrectangle{\pgfqpoint{0.935815in}{0.637495in}}{\pgfqpoint{9.300000in}{9.060000in}}%
\pgfusepath{clip}%
\pgfsetrectcap%
\pgfsetroundjoin%
\pgfsetlinewidth{0.803000pt}%
\definecolor{currentstroke}{rgb}{1.000000,1.000000,1.000000}%
\pgfsetstrokecolor{currentstroke}%
\pgfsetdash{}{0pt}%
\pgfpathmoveto{\pgfqpoint{0.935815in}{3.943081in}}%
\pgfpathlineto{\pgfqpoint{10.235815in}{3.943081in}}%
\pgfusepath{stroke}%
\end{pgfscope}%
\begin{pgfscope}%
\pgfsetbuttcap%
\pgfsetroundjoin%
\definecolor{currentfill}{rgb}{0.333333,0.333333,0.333333}%
\pgfsetfillcolor{currentfill}%
\pgfsetlinewidth{0.803000pt}%
\definecolor{currentstroke}{rgb}{0.333333,0.333333,0.333333}%
\pgfsetstrokecolor{currentstroke}%
\pgfsetdash{}{0pt}%
\pgfsys@defobject{currentmarker}{\pgfqpoint{-0.048611in}{0.000000in}}{\pgfqpoint{-0.000000in}{0.000000in}}{%
\pgfpathmoveto{\pgfqpoint{-0.000000in}{0.000000in}}%
\pgfpathlineto{\pgfqpoint{-0.048611in}{0.000000in}}%
\pgfusepath{stroke,fill}%
}%
\begin{pgfscope}%
\pgfsys@transformshift{0.935815in}{3.943081in}%
\pgfsys@useobject{currentmarker}{}%
\end{pgfscope}%
\end{pgfscope}%
\begin{pgfscope}%
\definecolor{textcolor}{rgb}{0.333333,0.333333,0.333333}%
\pgfsetstrokecolor{textcolor}%
\pgfsetfillcolor{textcolor}%
\pgftext[x=0.443111in, y=3.859748in, left, base]{\color{textcolor}\rmfamily\fontsize{16.000000}{19.200000}\selectfont \(\displaystyle {0.10}\)}%
\end{pgfscope}%
\begin{pgfscope}%
\pgfpathrectangle{\pgfqpoint{0.935815in}{0.637495in}}{\pgfqpoint{9.300000in}{9.060000in}}%
\pgfusepath{clip}%
\pgfsetrectcap%
\pgfsetroundjoin%
\pgfsetlinewidth{0.803000pt}%
\definecolor{currentstroke}{rgb}{1.000000,1.000000,1.000000}%
\pgfsetstrokecolor{currentstroke}%
\pgfsetdash{}{0pt}%
\pgfpathmoveto{\pgfqpoint{0.935815in}{5.595874in}}%
\pgfpathlineto{\pgfqpoint{10.235815in}{5.595874in}}%
\pgfusepath{stroke}%
\end{pgfscope}%
\begin{pgfscope}%
\pgfsetbuttcap%
\pgfsetroundjoin%
\definecolor{currentfill}{rgb}{0.333333,0.333333,0.333333}%
\pgfsetfillcolor{currentfill}%
\pgfsetlinewidth{0.803000pt}%
\definecolor{currentstroke}{rgb}{0.333333,0.333333,0.333333}%
\pgfsetstrokecolor{currentstroke}%
\pgfsetdash{}{0pt}%
\pgfsys@defobject{currentmarker}{\pgfqpoint{-0.048611in}{0.000000in}}{\pgfqpoint{-0.000000in}{0.000000in}}{%
\pgfpathmoveto{\pgfqpoint{-0.000000in}{0.000000in}}%
\pgfpathlineto{\pgfqpoint{-0.048611in}{0.000000in}}%
\pgfusepath{stroke,fill}%
}%
\begin{pgfscope}%
\pgfsys@transformshift{0.935815in}{5.595874in}%
\pgfsys@useobject{currentmarker}{}%
\end{pgfscope}%
\end{pgfscope}%
\begin{pgfscope}%
\definecolor{textcolor}{rgb}{0.333333,0.333333,0.333333}%
\pgfsetstrokecolor{textcolor}%
\pgfsetfillcolor{textcolor}%
\pgftext[x=0.443111in, y=5.512541in, left, base]{\color{textcolor}\rmfamily\fontsize{16.000000}{19.200000}\selectfont \(\displaystyle {0.15}\)}%
\end{pgfscope}%
\begin{pgfscope}%
\pgfpathrectangle{\pgfqpoint{0.935815in}{0.637495in}}{\pgfqpoint{9.300000in}{9.060000in}}%
\pgfusepath{clip}%
\pgfsetrectcap%
\pgfsetroundjoin%
\pgfsetlinewidth{0.803000pt}%
\definecolor{currentstroke}{rgb}{1.000000,1.000000,1.000000}%
\pgfsetstrokecolor{currentstroke}%
\pgfsetdash{}{0pt}%
\pgfpathmoveto{\pgfqpoint{0.935815in}{7.248667in}}%
\pgfpathlineto{\pgfqpoint{10.235815in}{7.248667in}}%
\pgfusepath{stroke}%
\end{pgfscope}%
\begin{pgfscope}%
\pgfsetbuttcap%
\pgfsetroundjoin%
\definecolor{currentfill}{rgb}{0.333333,0.333333,0.333333}%
\pgfsetfillcolor{currentfill}%
\pgfsetlinewidth{0.803000pt}%
\definecolor{currentstroke}{rgb}{0.333333,0.333333,0.333333}%
\pgfsetstrokecolor{currentstroke}%
\pgfsetdash{}{0pt}%
\pgfsys@defobject{currentmarker}{\pgfqpoint{-0.048611in}{0.000000in}}{\pgfqpoint{-0.000000in}{0.000000in}}{%
\pgfpathmoveto{\pgfqpoint{-0.000000in}{0.000000in}}%
\pgfpathlineto{\pgfqpoint{-0.048611in}{0.000000in}}%
\pgfusepath{stroke,fill}%
}%
\begin{pgfscope}%
\pgfsys@transformshift{0.935815in}{7.248667in}%
\pgfsys@useobject{currentmarker}{}%
\end{pgfscope}%
\end{pgfscope}%
\begin{pgfscope}%
\definecolor{textcolor}{rgb}{0.333333,0.333333,0.333333}%
\pgfsetstrokecolor{textcolor}%
\pgfsetfillcolor{textcolor}%
\pgftext[x=0.443111in, y=7.165334in, left, base]{\color{textcolor}\rmfamily\fontsize{16.000000}{19.200000}\selectfont \(\displaystyle {0.20}\)}%
\end{pgfscope}%
\begin{pgfscope}%
\pgfpathrectangle{\pgfqpoint{0.935815in}{0.637495in}}{\pgfqpoint{9.300000in}{9.060000in}}%
\pgfusepath{clip}%
\pgfsetrectcap%
\pgfsetroundjoin%
\pgfsetlinewidth{0.803000pt}%
\definecolor{currentstroke}{rgb}{1.000000,1.000000,1.000000}%
\pgfsetstrokecolor{currentstroke}%
\pgfsetdash{}{0pt}%
\pgfpathmoveto{\pgfqpoint{0.935815in}{8.901460in}}%
\pgfpathlineto{\pgfqpoint{10.235815in}{8.901460in}}%
\pgfusepath{stroke}%
\end{pgfscope}%
\begin{pgfscope}%
\pgfsetbuttcap%
\pgfsetroundjoin%
\definecolor{currentfill}{rgb}{0.333333,0.333333,0.333333}%
\pgfsetfillcolor{currentfill}%
\pgfsetlinewidth{0.803000pt}%
\definecolor{currentstroke}{rgb}{0.333333,0.333333,0.333333}%
\pgfsetstrokecolor{currentstroke}%
\pgfsetdash{}{0pt}%
\pgfsys@defobject{currentmarker}{\pgfqpoint{-0.048611in}{0.000000in}}{\pgfqpoint{-0.000000in}{0.000000in}}{%
\pgfpathmoveto{\pgfqpoint{-0.000000in}{0.000000in}}%
\pgfpathlineto{\pgfqpoint{-0.048611in}{0.000000in}}%
\pgfusepath{stroke,fill}%
}%
\begin{pgfscope}%
\pgfsys@transformshift{0.935815in}{8.901460in}%
\pgfsys@useobject{currentmarker}{}%
\end{pgfscope}%
\end{pgfscope}%
\begin{pgfscope}%
\definecolor{textcolor}{rgb}{0.333333,0.333333,0.333333}%
\pgfsetstrokecolor{textcolor}%
\pgfsetfillcolor{textcolor}%
\pgftext[x=0.443111in, y=8.818127in, left, base]{\color{textcolor}\rmfamily\fontsize{16.000000}{19.200000}\selectfont \(\displaystyle {0.25}\)}%
\end{pgfscope}%
\begin{pgfscope}%
\definecolor{textcolor}{rgb}{0.333333,0.333333,0.333333}%
\pgfsetstrokecolor{textcolor}%
\pgfsetfillcolor{textcolor}%
\pgftext[x=0.387555in,y=5.167495in,,bottom,rotate=90.000000]{\color{textcolor}\rmfamily\fontsize{20.000000}{24.000000}\selectfont Capacity [GW\(\displaystyle _{th}\)]}%
\end{pgfscope}%
\begin{pgfscope}%
\pgfpathrectangle{\pgfqpoint{0.935815in}{0.637495in}}{\pgfqpoint{9.300000in}{9.060000in}}%
\pgfusepath{clip}%
\pgfsetbuttcap%
\pgfsetmiterjoin%
\definecolor{currentfill}{rgb}{0.839216,0.152941,0.156863}%
\pgfsetfillcolor{currentfill}%
\pgfsetlinewidth{0.000000pt}%
\definecolor{currentstroke}{rgb}{0.000000,0.000000,0.000000}%
\pgfsetstrokecolor{currentstroke}%
\pgfsetstrokeopacity{0.000000}%
\pgfsetdash{}{0pt}%
\pgfpathmoveto{\pgfqpoint{1.323315in}{0.637495in}}%
\pgfpathlineto{\pgfqpoint{2.098315in}{0.637495in}}%
\pgfpathlineto{\pgfqpoint{2.098315in}{9.266067in}}%
\pgfpathlineto{\pgfqpoint{1.323315in}{9.266067in}}%
\pgfpathclose%
\pgfusepath{fill}%
\end{pgfscope}%
\begin{pgfscope}%
\pgfpathrectangle{\pgfqpoint{0.935815in}{0.637495in}}{\pgfqpoint{9.300000in}{9.060000in}}%
\pgfusepath{clip}%
\pgfsetbuttcap%
\pgfsetmiterjoin%
\definecolor{currentfill}{rgb}{0.839216,0.152941,0.156863}%
\pgfsetfillcolor{currentfill}%
\pgfsetlinewidth{0.000000pt}%
\definecolor{currentstroke}{rgb}{0.000000,0.000000,0.000000}%
\pgfsetstrokecolor{currentstroke}%
\pgfsetstrokeopacity{0.000000}%
\pgfsetdash{}{0pt}%
\pgfpathmoveto{\pgfqpoint{2.873315in}{0.637495in}}%
\pgfpathlineto{\pgfqpoint{3.648315in}{0.637495in}}%
\pgfpathlineto{\pgfqpoint{3.648315in}{9.266067in}}%
\pgfpathlineto{\pgfqpoint{2.873315in}{9.266067in}}%
\pgfpathclose%
\pgfusepath{fill}%
\end{pgfscope}%
\begin{pgfscope}%
\pgfpathrectangle{\pgfqpoint{0.935815in}{0.637495in}}{\pgfqpoint{9.300000in}{9.060000in}}%
\pgfusepath{clip}%
\pgfsetbuttcap%
\pgfsetmiterjoin%
\definecolor{currentfill}{rgb}{0.839216,0.152941,0.156863}%
\pgfsetfillcolor{currentfill}%
\pgfsetlinewidth{0.000000pt}%
\definecolor{currentstroke}{rgb}{0.000000,0.000000,0.000000}%
\pgfsetstrokecolor{currentstroke}%
\pgfsetstrokeopacity{0.000000}%
\pgfsetdash{}{0pt}%
\pgfpathmoveto{\pgfqpoint{4.423315in}{0.637495in}}%
\pgfpathlineto{\pgfqpoint{5.198315in}{0.637495in}}%
\pgfpathlineto{\pgfqpoint{5.198315in}{9.266067in}}%
\pgfpathlineto{\pgfqpoint{4.423315in}{9.266067in}}%
\pgfpathclose%
\pgfusepath{fill}%
\end{pgfscope}%
\begin{pgfscope}%
\pgfpathrectangle{\pgfqpoint{0.935815in}{0.637495in}}{\pgfqpoint{9.300000in}{9.060000in}}%
\pgfusepath{clip}%
\pgfsetbuttcap%
\pgfsetmiterjoin%
\definecolor{currentfill}{rgb}{0.839216,0.152941,0.156863}%
\pgfsetfillcolor{currentfill}%
\pgfsetlinewidth{0.000000pt}%
\definecolor{currentstroke}{rgb}{0.000000,0.000000,0.000000}%
\pgfsetstrokecolor{currentstroke}%
\pgfsetstrokeopacity{0.000000}%
\pgfsetdash{}{0pt}%
\pgfpathmoveto{\pgfqpoint{5.973315in}{0.637495in}}%
\pgfpathlineto{\pgfqpoint{6.748315in}{0.637495in}}%
\pgfpathlineto{\pgfqpoint{6.748315in}{5.814638in}}%
\pgfpathlineto{\pgfqpoint{5.973315in}{5.814638in}}%
\pgfpathclose%
\pgfusepath{fill}%
\end{pgfscope}%
\begin{pgfscope}%
\pgfpathrectangle{\pgfqpoint{0.935815in}{0.637495in}}{\pgfqpoint{9.300000in}{9.060000in}}%
\pgfusepath{clip}%
\pgfsetbuttcap%
\pgfsetmiterjoin%
\definecolor{currentfill}{rgb}{0.839216,0.152941,0.156863}%
\pgfsetfillcolor{currentfill}%
\pgfsetlinewidth{0.000000pt}%
\definecolor{currentstroke}{rgb}{0.000000,0.000000,0.000000}%
\pgfsetstrokecolor{currentstroke}%
\pgfsetstrokeopacity{0.000000}%
\pgfsetdash{}{0pt}%
\pgfpathmoveto{\pgfqpoint{7.523315in}{0.637495in}}%
\pgfpathlineto{\pgfqpoint{8.298315in}{0.637495in}}%
\pgfpathlineto{\pgfqpoint{8.298315in}{2.592649in}}%
\pgfpathlineto{\pgfqpoint{7.523315in}{2.592649in}}%
\pgfpathclose%
\pgfusepath{fill}%
\end{pgfscope}%
\begin{pgfscope}%
\pgfpathrectangle{\pgfqpoint{0.935815in}{0.637495in}}{\pgfqpoint{9.300000in}{9.060000in}}%
\pgfusepath{clip}%
\pgfsetbuttcap%
\pgfsetmiterjoin%
\definecolor{currentfill}{rgb}{0.839216,0.152941,0.156863}%
\pgfsetfillcolor{currentfill}%
\pgfsetlinewidth{0.000000pt}%
\definecolor{currentstroke}{rgb}{0.000000,0.000000,0.000000}%
\pgfsetstrokecolor{currentstroke}%
\pgfsetstrokeopacity{0.000000}%
\pgfsetdash{}{0pt}%
\pgfpathmoveto{\pgfqpoint{9.073315in}{0.637495in}}%
\pgfpathlineto{\pgfqpoint{9.848315in}{0.637495in}}%
\pgfpathlineto{\pgfqpoint{9.848315in}{2.592649in}}%
\pgfpathlineto{\pgfqpoint{9.073315in}{2.592649in}}%
\pgfpathclose%
\pgfusepath{fill}%
\end{pgfscope}%
\begin{pgfscope}%
\pgfpathrectangle{\pgfqpoint{0.935815in}{0.637495in}}{\pgfqpoint{9.300000in}{9.060000in}}%
\pgfusepath{clip}%
\pgfsetbuttcap%
\pgfsetmiterjoin%
\definecolor{currentfill}{rgb}{0.172549,0.627451,0.172549}%
\pgfsetfillcolor{currentfill}%
\pgfsetlinewidth{0.000000pt}%
\definecolor{currentstroke}{rgb}{0.000000,0.000000,0.000000}%
\pgfsetstrokecolor{currentstroke}%
\pgfsetstrokeopacity{0.000000}%
\pgfsetdash{}{0pt}%
\pgfpathmoveto{\pgfqpoint{1.323315in}{0.637495in}}%
\pgfpathlineto{\pgfqpoint{2.098315in}{0.637495in}}%
\pgfpathlineto{\pgfqpoint{2.098315in}{0.637495in}}%
\pgfpathlineto{\pgfqpoint{1.323315in}{0.637495in}}%
\pgfpathclose%
\pgfusepath{fill}%
\end{pgfscope}%
\begin{pgfscope}%
\pgfpathrectangle{\pgfqpoint{0.935815in}{0.637495in}}{\pgfqpoint{9.300000in}{9.060000in}}%
\pgfusepath{clip}%
\pgfsetbuttcap%
\pgfsetmiterjoin%
\definecolor{currentfill}{rgb}{0.172549,0.627451,0.172549}%
\pgfsetfillcolor{currentfill}%
\pgfsetlinewidth{0.000000pt}%
\definecolor{currentstroke}{rgb}{0.000000,0.000000,0.000000}%
\pgfsetstrokecolor{currentstroke}%
\pgfsetstrokeopacity{0.000000}%
\pgfsetdash{}{0pt}%
\pgfpathmoveto{\pgfqpoint{2.873315in}{0.637495in}}%
\pgfpathlineto{\pgfqpoint{3.648315in}{0.637495in}}%
\pgfpathlineto{\pgfqpoint{3.648315in}{0.637495in}}%
\pgfpathlineto{\pgfqpoint{2.873315in}{0.637495in}}%
\pgfpathclose%
\pgfusepath{fill}%
\end{pgfscope}%
\begin{pgfscope}%
\pgfpathrectangle{\pgfqpoint{0.935815in}{0.637495in}}{\pgfqpoint{9.300000in}{9.060000in}}%
\pgfusepath{clip}%
\pgfsetbuttcap%
\pgfsetmiterjoin%
\definecolor{currentfill}{rgb}{0.172549,0.627451,0.172549}%
\pgfsetfillcolor{currentfill}%
\pgfsetlinewidth{0.000000pt}%
\definecolor{currentstroke}{rgb}{0.000000,0.000000,0.000000}%
\pgfsetstrokecolor{currentstroke}%
\pgfsetstrokeopacity{0.000000}%
\pgfsetdash{}{0pt}%
\pgfpathmoveto{\pgfqpoint{4.423315in}{0.637495in}}%
\pgfpathlineto{\pgfqpoint{5.198315in}{0.637495in}}%
\pgfpathlineto{\pgfqpoint{5.198315in}{0.637495in}}%
\pgfpathlineto{\pgfqpoint{4.423315in}{0.637495in}}%
\pgfpathclose%
\pgfusepath{fill}%
\end{pgfscope}%
\begin{pgfscope}%
\pgfpathrectangle{\pgfqpoint{0.935815in}{0.637495in}}{\pgfqpoint{9.300000in}{9.060000in}}%
\pgfusepath{clip}%
\pgfsetbuttcap%
\pgfsetmiterjoin%
\definecolor{currentfill}{rgb}{0.172549,0.627451,0.172549}%
\pgfsetfillcolor{currentfill}%
\pgfsetlinewidth{0.000000pt}%
\definecolor{currentstroke}{rgb}{0.000000,0.000000,0.000000}%
\pgfsetstrokecolor{currentstroke}%
\pgfsetstrokeopacity{0.000000}%
\pgfsetdash{}{0pt}%
\pgfpathmoveto{\pgfqpoint{5.973315in}{5.814638in}}%
\pgfpathlineto{\pgfqpoint{6.748315in}{5.814638in}}%
\pgfpathlineto{\pgfqpoint{6.748315in}{5.938205in}}%
\pgfpathlineto{\pgfqpoint{5.973315in}{5.938205in}}%
\pgfpathclose%
\pgfusepath{fill}%
\end{pgfscope}%
\begin{pgfscope}%
\pgfpathrectangle{\pgfqpoint{0.935815in}{0.637495in}}{\pgfqpoint{9.300000in}{9.060000in}}%
\pgfusepath{clip}%
\pgfsetbuttcap%
\pgfsetmiterjoin%
\definecolor{currentfill}{rgb}{0.172549,0.627451,0.172549}%
\pgfsetfillcolor{currentfill}%
\pgfsetlinewidth{0.000000pt}%
\definecolor{currentstroke}{rgb}{0.000000,0.000000,0.000000}%
\pgfsetstrokecolor{currentstroke}%
\pgfsetstrokeopacity{0.000000}%
\pgfsetdash{}{0pt}%
\pgfpathmoveto{\pgfqpoint{7.523315in}{2.592649in}}%
\pgfpathlineto{\pgfqpoint{8.298315in}{2.592649in}}%
\pgfpathlineto{\pgfqpoint{8.298315in}{4.180022in}}%
\pgfpathlineto{\pgfqpoint{7.523315in}{4.180022in}}%
\pgfpathclose%
\pgfusepath{fill}%
\end{pgfscope}%
\begin{pgfscope}%
\pgfpathrectangle{\pgfqpoint{0.935815in}{0.637495in}}{\pgfqpoint{9.300000in}{9.060000in}}%
\pgfusepath{clip}%
\pgfsetbuttcap%
\pgfsetmiterjoin%
\definecolor{currentfill}{rgb}{0.172549,0.627451,0.172549}%
\pgfsetfillcolor{currentfill}%
\pgfsetlinewidth{0.000000pt}%
\definecolor{currentstroke}{rgb}{0.000000,0.000000,0.000000}%
\pgfsetstrokecolor{currentstroke}%
\pgfsetstrokeopacity{0.000000}%
\pgfsetdash{}{0pt}%
\pgfpathmoveto{\pgfqpoint{9.073315in}{2.592649in}}%
\pgfpathlineto{\pgfqpoint{9.848315in}{2.592649in}}%
\pgfpathlineto{\pgfqpoint{9.848315in}{5.494413in}}%
\pgfpathlineto{\pgfqpoint{9.073315in}{5.494413in}}%
\pgfpathclose%
\pgfusepath{fill}%
\end{pgfscope}%
\begin{pgfscope}%
\pgfsetrectcap%
\pgfsetmiterjoin%
\pgfsetlinewidth{1.003750pt}%
\definecolor{currentstroke}{rgb}{1.000000,1.000000,1.000000}%
\pgfsetstrokecolor{currentstroke}%
\pgfsetdash{}{0pt}%
\pgfpathmoveto{\pgfqpoint{0.935815in}{0.637495in}}%
\pgfpathlineto{\pgfqpoint{0.935815in}{9.697495in}}%
\pgfusepath{stroke}%
\end{pgfscope}%
\begin{pgfscope}%
\pgfsetrectcap%
\pgfsetmiterjoin%
\pgfsetlinewidth{1.003750pt}%
\definecolor{currentstroke}{rgb}{1.000000,1.000000,1.000000}%
\pgfsetstrokecolor{currentstroke}%
\pgfsetdash{}{0pt}%
\pgfpathmoveto{\pgfqpoint{10.235815in}{0.637495in}}%
\pgfpathlineto{\pgfqpoint{10.235815in}{9.697495in}}%
\pgfusepath{stroke}%
\end{pgfscope}%
\begin{pgfscope}%
\pgfsetrectcap%
\pgfsetmiterjoin%
\pgfsetlinewidth{1.003750pt}%
\definecolor{currentstroke}{rgb}{1.000000,1.000000,1.000000}%
\pgfsetstrokecolor{currentstroke}%
\pgfsetdash{}{0pt}%
\pgfpathmoveto{\pgfqpoint{0.935815in}{0.637495in}}%
\pgfpathlineto{\pgfqpoint{10.235815in}{0.637495in}}%
\pgfusepath{stroke}%
\end{pgfscope}%
\begin{pgfscope}%
\pgfsetrectcap%
\pgfsetmiterjoin%
\pgfsetlinewidth{1.003750pt}%
\definecolor{currentstroke}{rgb}{1.000000,1.000000,1.000000}%
\pgfsetstrokecolor{currentstroke}%
\pgfsetdash{}{0pt}%
\pgfpathmoveto{\pgfqpoint{0.935815in}{9.697495in}}%
\pgfpathlineto{\pgfqpoint{10.235815in}{9.697495in}}%
\pgfusepath{stroke}%
\end{pgfscope}%
\begin{pgfscope}%
\definecolor{textcolor}{rgb}{0.000000,0.000000,0.000000}%
\pgfsetstrokecolor{textcolor}%
\pgfsetfillcolor{textcolor}%
\pgftext[x=5.585815in,y=9.780828in,,base]{\color{textcolor}\rmfamily\fontsize{24.000000}{28.800000}\selectfont UIUC Thermal Capacity}%
\end{pgfscope}%
\begin{pgfscope}%
\pgfsetbuttcap%
\pgfsetmiterjoin%
\definecolor{currentfill}{rgb}{0.269412,0.269412,0.269412}%
\pgfsetfillcolor{currentfill}%
\pgfsetfillopacity{0.500000}%
\pgfsetlinewidth{0.501875pt}%
\definecolor{currentstroke}{rgb}{0.269412,0.269412,0.269412}%
\pgfsetstrokecolor{currentstroke}%
\pgfsetstrokeopacity{0.500000}%
\pgfsetdash{}{0pt}%
\pgfpathmoveto{\pgfqpoint{7.675329in}{8.843020in}}%
\pgfpathlineto{\pgfqpoint{10.108037in}{8.843020in}}%
\pgfpathquadraticcurveto{\pgfqpoint{10.152481in}{8.843020in}}{\pgfqpoint{10.152481in}{8.887465in}}%
\pgfpathlineto{\pgfqpoint{10.152481in}{9.514162in}}%
\pgfpathquadraticcurveto{\pgfqpoint{10.152481in}{9.558606in}}{\pgfqpoint{10.108037in}{9.558606in}}%
\pgfpathlineto{\pgfqpoint{7.675329in}{9.558606in}}%
\pgfpathquadraticcurveto{\pgfqpoint{7.630885in}{9.558606in}}{\pgfqpoint{7.630885in}{9.514162in}}%
\pgfpathlineto{\pgfqpoint{7.630885in}{8.887465in}}%
\pgfpathquadraticcurveto{\pgfqpoint{7.630885in}{8.843020in}}{\pgfqpoint{7.675329in}{8.843020in}}%
\pgfpathclose%
\pgfusepath{stroke,fill}%
\end{pgfscope}%
\begin{pgfscope}%
\pgfsetbuttcap%
\pgfsetmiterjoin%
\definecolor{currentfill}{rgb}{0.898039,0.898039,0.898039}%
\pgfsetfillcolor{currentfill}%
\pgfsetlinewidth{0.501875pt}%
\definecolor{currentstroke}{rgb}{0.800000,0.800000,0.800000}%
\pgfsetstrokecolor{currentstroke}%
\pgfsetdash{}{0pt}%
\pgfpathmoveto{\pgfqpoint{7.647552in}{8.870798in}}%
\pgfpathlineto{\pgfqpoint{10.080259in}{8.870798in}}%
\pgfpathquadraticcurveto{\pgfqpoint{10.124703in}{8.870798in}}{\pgfqpoint{10.124703in}{8.915242in}}%
\pgfpathlineto{\pgfqpoint{10.124703in}{9.541940in}}%
\pgfpathquadraticcurveto{\pgfqpoint{10.124703in}{9.586384in}}{\pgfqpoint{10.080259in}{9.586384in}}%
\pgfpathlineto{\pgfqpoint{7.647552in}{9.586384in}}%
\pgfpathquadraticcurveto{\pgfqpoint{7.603107in}{9.586384in}}{\pgfqpoint{7.603107in}{9.541940in}}%
\pgfpathlineto{\pgfqpoint{7.603107in}{8.915242in}}%
\pgfpathquadraticcurveto{\pgfqpoint{7.603107in}{8.870798in}}{\pgfqpoint{7.647552in}{8.870798in}}%
\pgfpathclose%
\pgfusepath{stroke,fill}%
\end{pgfscope}%
\begin{pgfscope}%
\pgfsetbuttcap%
\pgfsetmiterjoin%
\definecolor{currentfill}{rgb}{0.839216,0.152941,0.156863}%
\pgfsetfillcolor{currentfill}%
\pgfsetlinewidth{0.000000pt}%
\definecolor{currentstroke}{rgb}{0.000000,0.000000,0.000000}%
\pgfsetstrokecolor{currentstroke}%
\pgfsetstrokeopacity{0.000000}%
\pgfsetdash{}{0pt}%
\pgfpathmoveto{\pgfqpoint{7.691996in}{9.330828in}}%
\pgfpathlineto{\pgfqpoint{8.136440in}{9.330828in}}%
\pgfpathlineto{\pgfqpoint{8.136440in}{9.486384in}}%
\pgfpathlineto{\pgfqpoint{7.691996in}{9.486384in}}%
\pgfpathclose%
\pgfusepath{fill}%
\end{pgfscope}%
\begin{pgfscope}%
\definecolor{textcolor}{rgb}{0.000000,0.000000,0.000000}%
\pgfsetstrokecolor{textcolor}%
\pgfsetfillcolor{textcolor}%
\pgftext[x=8.314218in,y=9.330828in,left,base]{\color{textcolor}\rmfamily\fontsize{16.000000}{19.200000}\selectfont ABBOTT}%
\end{pgfscope}%
\begin{pgfscope}%
\pgfsetbuttcap%
\pgfsetmiterjoin%
\definecolor{currentfill}{rgb}{0.172549,0.627451,0.172549}%
\pgfsetfillcolor{currentfill}%
\pgfsetlinewidth{0.000000pt}%
\definecolor{currentstroke}{rgb}{0.000000,0.000000,0.000000}%
\pgfsetstrokecolor{currentstroke}%
\pgfsetstrokeopacity{0.000000}%
\pgfsetdash{}{0pt}%
\pgfpathmoveto{\pgfqpoint{7.691996in}{9.006369in}}%
\pgfpathlineto{\pgfqpoint{8.136440in}{9.006369in}}%
\pgfpathlineto{\pgfqpoint{8.136440in}{9.161924in}}%
\pgfpathlineto{\pgfqpoint{7.691996in}{9.161924in}}%
\pgfpathclose%
\pgfusepath{fill}%
\end{pgfscope}%
\begin{pgfscope}%
\definecolor{textcolor}{rgb}{0.000000,0.000000,0.000000}%
\pgfsetstrokecolor{textcolor}%
\pgfsetfillcolor{textcolor}%
\pgftext[x=8.314218in,y=9.006369in,left,base]{\color{textcolor}\rmfamily\fontsize{16.000000}{19.200000}\selectfont NUCLEAR\_THM}%
\end{pgfscope}%
\end{pgfpicture}%
\makeatother%
\endgroup%
}
    \caption[]{Least cost thermal capacity.}
    \label{fig:uiuc_thm_cap}
  \end{minipage}
  \begin{minipage}{0.48\textwidth}
    \centering
    \resizebox{\columnwidth}{!}{%% Creator: Matplotlib, PGF backend
%%
%% To include the figure in your LaTeX document, write
%%   \input{<filename>.pgf}
%%
%% Make sure the required packages are loaded in your preamble
%%   \usepackage{pgf}
%%
%% Figures using additional raster images can only be included by \input if
%% they are in the same directory as the main LaTeX file. For loading figures
%% from other directories you can use the `import` package
%%   \usepackage{import}
%%
%% and then include the figures with
%%   \import{<path to file>}{<filename>.pgf}
%%
%% Matplotlib used the following preamble
%%
\begingroup%
\makeatletter%
\begin{pgfpicture}%
\pgfpathrectangle{\pgfpointorigin}{\pgfqpoint{10.490674in}{10.120798in}}%
\pgfusepath{use as bounding box, clip}%
\begin{pgfscope}%
\pgfsetbuttcap%
\pgfsetmiterjoin%
\definecolor{currentfill}{rgb}{1.000000,1.000000,1.000000}%
\pgfsetfillcolor{currentfill}%
\pgfsetlinewidth{0.000000pt}%
\definecolor{currentstroke}{rgb}{0.000000,0.000000,0.000000}%
\pgfsetstrokecolor{currentstroke}%
\pgfsetdash{}{0pt}%
\pgfpathmoveto{\pgfqpoint{0.000000in}{0.000000in}}%
\pgfpathlineto{\pgfqpoint{10.490674in}{0.000000in}}%
\pgfpathlineto{\pgfqpoint{10.490674in}{10.120798in}}%
\pgfpathlineto{\pgfqpoint{0.000000in}{10.120798in}}%
\pgfpathclose%
\pgfusepath{fill}%
\end{pgfscope}%
\begin{pgfscope}%
\pgfsetbuttcap%
\pgfsetmiterjoin%
\definecolor{currentfill}{rgb}{0.898039,0.898039,0.898039}%
\pgfsetfillcolor{currentfill}%
\pgfsetlinewidth{0.000000pt}%
\definecolor{currentstroke}{rgb}{0.000000,0.000000,0.000000}%
\pgfsetstrokecolor{currentstroke}%
\pgfsetstrokeopacity{0.000000}%
\pgfsetdash{}{0pt}%
\pgfpathmoveto{\pgfqpoint{1.090674in}{0.637495in}}%
\pgfpathlineto{\pgfqpoint{10.390674in}{0.637495in}}%
\pgfpathlineto{\pgfqpoint{10.390674in}{9.697495in}}%
\pgfpathlineto{\pgfqpoint{1.090674in}{9.697495in}}%
\pgfpathclose%
\pgfusepath{fill}%
\end{pgfscope}%
\begin{pgfscope}%
\pgfpathrectangle{\pgfqpoint{1.090674in}{0.637495in}}{\pgfqpoint{9.300000in}{9.060000in}}%
\pgfusepath{clip}%
\pgfsetrectcap%
\pgfsetroundjoin%
\pgfsetlinewidth{0.803000pt}%
\definecolor{currentstroke}{rgb}{1.000000,1.000000,1.000000}%
\pgfsetstrokecolor{currentstroke}%
\pgfsetdash{}{0pt}%
\pgfpathmoveto{\pgfqpoint{1.865674in}{0.637495in}}%
\pgfpathlineto{\pgfqpoint{1.865674in}{9.697495in}}%
\pgfusepath{stroke}%
\end{pgfscope}%
\begin{pgfscope}%
\pgfsetbuttcap%
\pgfsetroundjoin%
\definecolor{currentfill}{rgb}{0.333333,0.333333,0.333333}%
\pgfsetfillcolor{currentfill}%
\pgfsetlinewidth{0.803000pt}%
\definecolor{currentstroke}{rgb}{0.333333,0.333333,0.333333}%
\pgfsetstrokecolor{currentstroke}%
\pgfsetdash{}{0pt}%
\pgfsys@defobject{currentmarker}{\pgfqpoint{0.000000in}{-0.048611in}}{\pgfqpoint{0.000000in}{0.000000in}}{%
\pgfpathmoveto{\pgfqpoint{0.000000in}{0.000000in}}%
\pgfpathlineto{\pgfqpoint{0.000000in}{-0.048611in}}%
\pgfusepath{stroke,fill}%
}%
\begin{pgfscope}%
\pgfsys@transformshift{1.865674in}{0.637495in}%
\pgfsys@useobject{currentmarker}{}%
\end{pgfscope}%
\end{pgfscope}%
\begin{pgfscope}%
\definecolor{textcolor}{rgb}{0.333333,0.333333,0.333333}%
\pgfsetstrokecolor{textcolor}%
\pgfsetfillcolor{textcolor}%
\pgftext[x=1.925667in, y=0.100000in, left, base,rotate=90.000000]{\color{textcolor}\rmfamily\fontsize{16.000000}{19.200000}\selectfont 2025}%
\end{pgfscope}%
\begin{pgfscope}%
\pgfpathrectangle{\pgfqpoint{1.090674in}{0.637495in}}{\pgfqpoint{9.300000in}{9.060000in}}%
\pgfusepath{clip}%
\pgfsetrectcap%
\pgfsetroundjoin%
\pgfsetlinewidth{0.803000pt}%
\definecolor{currentstroke}{rgb}{1.000000,1.000000,1.000000}%
\pgfsetstrokecolor{currentstroke}%
\pgfsetdash{}{0pt}%
\pgfpathmoveto{\pgfqpoint{3.415674in}{0.637495in}}%
\pgfpathlineto{\pgfqpoint{3.415674in}{9.697495in}}%
\pgfusepath{stroke}%
\end{pgfscope}%
\begin{pgfscope}%
\pgfsetbuttcap%
\pgfsetroundjoin%
\definecolor{currentfill}{rgb}{0.333333,0.333333,0.333333}%
\pgfsetfillcolor{currentfill}%
\pgfsetlinewidth{0.803000pt}%
\definecolor{currentstroke}{rgb}{0.333333,0.333333,0.333333}%
\pgfsetstrokecolor{currentstroke}%
\pgfsetdash{}{0pt}%
\pgfsys@defobject{currentmarker}{\pgfqpoint{0.000000in}{-0.048611in}}{\pgfqpoint{0.000000in}{0.000000in}}{%
\pgfpathmoveto{\pgfqpoint{0.000000in}{0.000000in}}%
\pgfpathlineto{\pgfqpoint{0.000000in}{-0.048611in}}%
\pgfusepath{stroke,fill}%
}%
\begin{pgfscope}%
\pgfsys@transformshift{3.415674in}{0.637495in}%
\pgfsys@useobject{currentmarker}{}%
\end{pgfscope}%
\end{pgfscope}%
\begin{pgfscope}%
\definecolor{textcolor}{rgb}{0.333333,0.333333,0.333333}%
\pgfsetstrokecolor{textcolor}%
\pgfsetfillcolor{textcolor}%
\pgftext[x=3.475667in, y=0.100000in, left, base,rotate=90.000000]{\color{textcolor}\rmfamily\fontsize{16.000000}{19.200000}\selectfont 2030}%
\end{pgfscope}%
\begin{pgfscope}%
\pgfpathrectangle{\pgfqpoint{1.090674in}{0.637495in}}{\pgfqpoint{9.300000in}{9.060000in}}%
\pgfusepath{clip}%
\pgfsetrectcap%
\pgfsetroundjoin%
\pgfsetlinewidth{0.803000pt}%
\definecolor{currentstroke}{rgb}{1.000000,1.000000,1.000000}%
\pgfsetstrokecolor{currentstroke}%
\pgfsetdash{}{0pt}%
\pgfpathmoveto{\pgfqpoint{4.965674in}{0.637495in}}%
\pgfpathlineto{\pgfqpoint{4.965674in}{9.697495in}}%
\pgfusepath{stroke}%
\end{pgfscope}%
\begin{pgfscope}%
\pgfsetbuttcap%
\pgfsetroundjoin%
\definecolor{currentfill}{rgb}{0.333333,0.333333,0.333333}%
\pgfsetfillcolor{currentfill}%
\pgfsetlinewidth{0.803000pt}%
\definecolor{currentstroke}{rgb}{0.333333,0.333333,0.333333}%
\pgfsetstrokecolor{currentstroke}%
\pgfsetdash{}{0pt}%
\pgfsys@defobject{currentmarker}{\pgfqpoint{0.000000in}{-0.048611in}}{\pgfqpoint{0.000000in}{0.000000in}}{%
\pgfpathmoveto{\pgfqpoint{0.000000in}{0.000000in}}%
\pgfpathlineto{\pgfqpoint{0.000000in}{-0.048611in}}%
\pgfusepath{stroke,fill}%
}%
\begin{pgfscope}%
\pgfsys@transformshift{4.965674in}{0.637495in}%
\pgfsys@useobject{currentmarker}{}%
\end{pgfscope}%
\end{pgfscope}%
\begin{pgfscope}%
\definecolor{textcolor}{rgb}{0.333333,0.333333,0.333333}%
\pgfsetstrokecolor{textcolor}%
\pgfsetfillcolor{textcolor}%
\pgftext[x=5.025667in, y=0.100000in, left, base,rotate=90.000000]{\color{textcolor}\rmfamily\fontsize{16.000000}{19.200000}\selectfont 2035}%
\end{pgfscope}%
\begin{pgfscope}%
\pgfpathrectangle{\pgfqpoint{1.090674in}{0.637495in}}{\pgfqpoint{9.300000in}{9.060000in}}%
\pgfusepath{clip}%
\pgfsetrectcap%
\pgfsetroundjoin%
\pgfsetlinewidth{0.803000pt}%
\definecolor{currentstroke}{rgb}{1.000000,1.000000,1.000000}%
\pgfsetstrokecolor{currentstroke}%
\pgfsetdash{}{0pt}%
\pgfpathmoveto{\pgfqpoint{6.515674in}{0.637495in}}%
\pgfpathlineto{\pgfqpoint{6.515674in}{9.697495in}}%
\pgfusepath{stroke}%
\end{pgfscope}%
\begin{pgfscope}%
\pgfsetbuttcap%
\pgfsetroundjoin%
\definecolor{currentfill}{rgb}{0.333333,0.333333,0.333333}%
\pgfsetfillcolor{currentfill}%
\pgfsetlinewidth{0.803000pt}%
\definecolor{currentstroke}{rgb}{0.333333,0.333333,0.333333}%
\pgfsetstrokecolor{currentstroke}%
\pgfsetdash{}{0pt}%
\pgfsys@defobject{currentmarker}{\pgfqpoint{0.000000in}{-0.048611in}}{\pgfqpoint{0.000000in}{0.000000in}}{%
\pgfpathmoveto{\pgfqpoint{0.000000in}{0.000000in}}%
\pgfpathlineto{\pgfqpoint{0.000000in}{-0.048611in}}%
\pgfusepath{stroke,fill}%
}%
\begin{pgfscope}%
\pgfsys@transformshift{6.515674in}{0.637495in}%
\pgfsys@useobject{currentmarker}{}%
\end{pgfscope}%
\end{pgfscope}%
\begin{pgfscope}%
\definecolor{textcolor}{rgb}{0.333333,0.333333,0.333333}%
\pgfsetstrokecolor{textcolor}%
\pgfsetfillcolor{textcolor}%
\pgftext[x=6.575667in, y=0.100000in, left, base,rotate=90.000000]{\color{textcolor}\rmfamily\fontsize{16.000000}{19.200000}\selectfont 2040}%
\end{pgfscope}%
\begin{pgfscope}%
\pgfpathrectangle{\pgfqpoint{1.090674in}{0.637495in}}{\pgfqpoint{9.300000in}{9.060000in}}%
\pgfusepath{clip}%
\pgfsetrectcap%
\pgfsetroundjoin%
\pgfsetlinewidth{0.803000pt}%
\definecolor{currentstroke}{rgb}{1.000000,1.000000,1.000000}%
\pgfsetstrokecolor{currentstroke}%
\pgfsetdash{}{0pt}%
\pgfpathmoveto{\pgfqpoint{8.065674in}{0.637495in}}%
\pgfpathlineto{\pgfqpoint{8.065674in}{9.697495in}}%
\pgfusepath{stroke}%
\end{pgfscope}%
\begin{pgfscope}%
\pgfsetbuttcap%
\pgfsetroundjoin%
\definecolor{currentfill}{rgb}{0.333333,0.333333,0.333333}%
\pgfsetfillcolor{currentfill}%
\pgfsetlinewidth{0.803000pt}%
\definecolor{currentstroke}{rgb}{0.333333,0.333333,0.333333}%
\pgfsetstrokecolor{currentstroke}%
\pgfsetdash{}{0pt}%
\pgfsys@defobject{currentmarker}{\pgfqpoint{0.000000in}{-0.048611in}}{\pgfqpoint{0.000000in}{0.000000in}}{%
\pgfpathmoveto{\pgfqpoint{0.000000in}{0.000000in}}%
\pgfpathlineto{\pgfqpoint{0.000000in}{-0.048611in}}%
\pgfusepath{stroke,fill}%
}%
\begin{pgfscope}%
\pgfsys@transformshift{8.065674in}{0.637495in}%
\pgfsys@useobject{currentmarker}{}%
\end{pgfscope}%
\end{pgfscope}%
\begin{pgfscope}%
\definecolor{textcolor}{rgb}{0.333333,0.333333,0.333333}%
\pgfsetstrokecolor{textcolor}%
\pgfsetfillcolor{textcolor}%
\pgftext[x=8.125667in, y=0.100000in, left, base,rotate=90.000000]{\color{textcolor}\rmfamily\fontsize{16.000000}{19.200000}\selectfont 2045}%
\end{pgfscope}%
\begin{pgfscope}%
\pgfpathrectangle{\pgfqpoint{1.090674in}{0.637495in}}{\pgfqpoint{9.300000in}{9.060000in}}%
\pgfusepath{clip}%
\pgfsetrectcap%
\pgfsetroundjoin%
\pgfsetlinewidth{0.803000pt}%
\definecolor{currentstroke}{rgb}{1.000000,1.000000,1.000000}%
\pgfsetstrokecolor{currentstroke}%
\pgfsetdash{}{0pt}%
\pgfpathmoveto{\pgfqpoint{9.615674in}{0.637495in}}%
\pgfpathlineto{\pgfqpoint{9.615674in}{9.697495in}}%
\pgfusepath{stroke}%
\end{pgfscope}%
\begin{pgfscope}%
\pgfsetbuttcap%
\pgfsetroundjoin%
\definecolor{currentfill}{rgb}{0.333333,0.333333,0.333333}%
\pgfsetfillcolor{currentfill}%
\pgfsetlinewidth{0.803000pt}%
\definecolor{currentstroke}{rgb}{0.333333,0.333333,0.333333}%
\pgfsetstrokecolor{currentstroke}%
\pgfsetdash{}{0pt}%
\pgfsys@defobject{currentmarker}{\pgfqpoint{0.000000in}{-0.048611in}}{\pgfqpoint{0.000000in}{0.000000in}}{%
\pgfpathmoveto{\pgfqpoint{0.000000in}{0.000000in}}%
\pgfpathlineto{\pgfqpoint{0.000000in}{-0.048611in}}%
\pgfusepath{stroke,fill}%
}%
\begin{pgfscope}%
\pgfsys@transformshift{9.615674in}{0.637495in}%
\pgfsys@useobject{currentmarker}{}%
\end{pgfscope}%
\end{pgfscope}%
\begin{pgfscope}%
\definecolor{textcolor}{rgb}{0.333333,0.333333,0.333333}%
\pgfsetstrokecolor{textcolor}%
\pgfsetfillcolor{textcolor}%
\pgftext[x=9.675667in, y=0.100000in, left, base,rotate=90.000000]{\color{textcolor}\rmfamily\fontsize{16.000000}{19.200000}\selectfont 2050}%
\end{pgfscope}%
\begin{pgfscope}%
\pgfpathrectangle{\pgfqpoint{1.090674in}{0.637495in}}{\pgfqpoint{9.300000in}{9.060000in}}%
\pgfusepath{clip}%
\pgfsetrectcap%
\pgfsetroundjoin%
\pgfsetlinewidth{0.803000pt}%
\definecolor{currentstroke}{rgb}{1.000000,1.000000,1.000000}%
\pgfsetstrokecolor{currentstroke}%
\pgfsetdash{}{0pt}%
\pgfpathmoveto{\pgfqpoint{1.090674in}{0.637495in}}%
\pgfpathlineto{\pgfqpoint{10.390674in}{0.637495in}}%
\pgfusepath{stroke}%
\end{pgfscope}%
\begin{pgfscope}%
\pgfsetbuttcap%
\pgfsetroundjoin%
\definecolor{currentfill}{rgb}{0.333333,0.333333,0.333333}%
\pgfsetfillcolor{currentfill}%
\pgfsetlinewidth{0.803000pt}%
\definecolor{currentstroke}{rgb}{0.333333,0.333333,0.333333}%
\pgfsetstrokecolor{currentstroke}%
\pgfsetdash{}{0pt}%
\pgfsys@defobject{currentmarker}{\pgfqpoint{-0.048611in}{0.000000in}}{\pgfqpoint{-0.000000in}{0.000000in}}{%
\pgfpathmoveto{\pgfqpoint{-0.000000in}{0.000000in}}%
\pgfpathlineto{\pgfqpoint{-0.048611in}{0.000000in}}%
\pgfusepath{stroke,fill}%
}%
\begin{pgfscope}%
\pgfsys@transformshift{1.090674in}{0.637495in}%
\pgfsys@useobject{currentmarker}{}%
\end{pgfscope}%
\end{pgfscope}%
\begin{pgfscope}%
\definecolor{textcolor}{rgb}{0.333333,0.333333,0.333333}%
\pgfsetstrokecolor{textcolor}%
\pgfsetfillcolor{textcolor}%
\pgftext[x=0.883384in, y=0.554162in, left, base]{\color{textcolor}\rmfamily\fontsize{16.000000}{19.200000}\selectfont \(\displaystyle {0}\)}%
\end{pgfscope}%
\begin{pgfscope}%
\pgfpathrectangle{\pgfqpoint{1.090674in}{0.637495in}}{\pgfqpoint{9.300000in}{9.060000in}}%
\pgfusepath{clip}%
\pgfsetrectcap%
\pgfsetroundjoin%
\pgfsetlinewidth{0.803000pt}%
\definecolor{currentstroke}{rgb}{1.000000,1.000000,1.000000}%
\pgfsetstrokecolor{currentstroke}%
\pgfsetdash{}{0pt}%
\pgfpathmoveto{\pgfqpoint{1.090674in}{2.645961in}}%
\pgfpathlineto{\pgfqpoint{10.390674in}{2.645961in}}%
\pgfusepath{stroke}%
\end{pgfscope}%
\begin{pgfscope}%
\pgfsetbuttcap%
\pgfsetroundjoin%
\definecolor{currentfill}{rgb}{0.333333,0.333333,0.333333}%
\pgfsetfillcolor{currentfill}%
\pgfsetlinewidth{0.803000pt}%
\definecolor{currentstroke}{rgb}{0.333333,0.333333,0.333333}%
\pgfsetstrokecolor{currentstroke}%
\pgfsetdash{}{0pt}%
\pgfsys@defobject{currentmarker}{\pgfqpoint{-0.048611in}{0.000000in}}{\pgfqpoint{-0.000000in}{0.000000in}}{%
\pgfpathmoveto{\pgfqpoint{-0.000000in}{0.000000in}}%
\pgfpathlineto{\pgfqpoint{-0.048611in}{0.000000in}}%
\pgfusepath{stroke,fill}%
}%
\begin{pgfscope}%
\pgfsys@transformshift{1.090674in}{2.645961in}%
\pgfsys@useobject{currentmarker}{}%
\end{pgfscope}%
\end{pgfscope}%
\begin{pgfscope}%
\definecolor{textcolor}{rgb}{0.333333,0.333333,0.333333}%
\pgfsetstrokecolor{textcolor}%
\pgfsetfillcolor{textcolor}%
\pgftext[x=0.443111in, y=2.562628in, left, base]{\color{textcolor}\rmfamily\fontsize{16.000000}{19.200000}\selectfont \(\displaystyle {10000}\)}%
\end{pgfscope}%
\begin{pgfscope}%
\pgfpathrectangle{\pgfqpoint{1.090674in}{0.637495in}}{\pgfqpoint{9.300000in}{9.060000in}}%
\pgfusepath{clip}%
\pgfsetrectcap%
\pgfsetroundjoin%
\pgfsetlinewidth{0.803000pt}%
\definecolor{currentstroke}{rgb}{1.000000,1.000000,1.000000}%
\pgfsetstrokecolor{currentstroke}%
\pgfsetdash{}{0pt}%
\pgfpathmoveto{\pgfqpoint{1.090674in}{4.654427in}}%
\pgfpathlineto{\pgfqpoint{10.390674in}{4.654427in}}%
\pgfusepath{stroke}%
\end{pgfscope}%
\begin{pgfscope}%
\pgfsetbuttcap%
\pgfsetroundjoin%
\definecolor{currentfill}{rgb}{0.333333,0.333333,0.333333}%
\pgfsetfillcolor{currentfill}%
\pgfsetlinewidth{0.803000pt}%
\definecolor{currentstroke}{rgb}{0.333333,0.333333,0.333333}%
\pgfsetstrokecolor{currentstroke}%
\pgfsetdash{}{0pt}%
\pgfsys@defobject{currentmarker}{\pgfqpoint{-0.048611in}{0.000000in}}{\pgfqpoint{-0.000000in}{0.000000in}}{%
\pgfpathmoveto{\pgfqpoint{-0.000000in}{0.000000in}}%
\pgfpathlineto{\pgfqpoint{-0.048611in}{0.000000in}}%
\pgfusepath{stroke,fill}%
}%
\begin{pgfscope}%
\pgfsys@transformshift{1.090674in}{4.654427in}%
\pgfsys@useobject{currentmarker}{}%
\end{pgfscope}%
\end{pgfscope}%
\begin{pgfscope}%
\definecolor{textcolor}{rgb}{0.333333,0.333333,0.333333}%
\pgfsetstrokecolor{textcolor}%
\pgfsetfillcolor{textcolor}%
\pgftext[x=0.443111in, y=4.571094in, left, base]{\color{textcolor}\rmfamily\fontsize{16.000000}{19.200000}\selectfont \(\displaystyle {20000}\)}%
\end{pgfscope}%
\begin{pgfscope}%
\pgfpathrectangle{\pgfqpoint{1.090674in}{0.637495in}}{\pgfqpoint{9.300000in}{9.060000in}}%
\pgfusepath{clip}%
\pgfsetrectcap%
\pgfsetroundjoin%
\pgfsetlinewidth{0.803000pt}%
\definecolor{currentstroke}{rgb}{1.000000,1.000000,1.000000}%
\pgfsetstrokecolor{currentstroke}%
\pgfsetdash{}{0pt}%
\pgfpathmoveto{\pgfqpoint{1.090674in}{6.662893in}}%
\pgfpathlineto{\pgfqpoint{10.390674in}{6.662893in}}%
\pgfusepath{stroke}%
\end{pgfscope}%
\begin{pgfscope}%
\pgfsetbuttcap%
\pgfsetroundjoin%
\definecolor{currentfill}{rgb}{0.333333,0.333333,0.333333}%
\pgfsetfillcolor{currentfill}%
\pgfsetlinewidth{0.803000pt}%
\definecolor{currentstroke}{rgb}{0.333333,0.333333,0.333333}%
\pgfsetstrokecolor{currentstroke}%
\pgfsetdash{}{0pt}%
\pgfsys@defobject{currentmarker}{\pgfqpoint{-0.048611in}{0.000000in}}{\pgfqpoint{-0.000000in}{0.000000in}}{%
\pgfpathmoveto{\pgfqpoint{-0.000000in}{0.000000in}}%
\pgfpathlineto{\pgfqpoint{-0.048611in}{0.000000in}}%
\pgfusepath{stroke,fill}%
}%
\begin{pgfscope}%
\pgfsys@transformshift{1.090674in}{6.662893in}%
\pgfsys@useobject{currentmarker}{}%
\end{pgfscope}%
\end{pgfscope}%
\begin{pgfscope}%
\definecolor{textcolor}{rgb}{0.333333,0.333333,0.333333}%
\pgfsetstrokecolor{textcolor}%
\pgfsetfillcolor{textcolor}%
\pgftext[x=0.443111in, y=6.579559in, left, base]{\color{textcolor}\rmfamily\fontsize{16.000000}{19.200000}\selectfont \(\displaystyle {30000}\)}%
\end{pgfscope}%
\begin{pgfscope}%
\pgfpathrectangle{\pgfqpoint{1.090674in}{0.637495in}}{\pgfqpoint{9.300000in}{9.060000in}}%
\pgfusepath{clip}%
\pgfsetrectcap%
\pgfsetroundjoin%
\pgfsetlinewidth{0.803000pt}%
\definecolor{currentstroke}{rgb}{1.000000,1.000000,1.000000}%
\pgfsetstrokecolor{currentstroke}%
\pgfsetdash{}{0pt}%
\pgfpathmoveto{\pgfqpoint{1.090674in}{8.671359in}}%
\pgfpathlineto{\pgfqpoint{10.390674in}{8.671359in}}%
\pgfusepath{stroke}%
\end{pgfscope}%
\begin{pgfscope}%
\pgfsetbuttcap%
\pgfsetroundjoin%
\definecolor{currentfill}{rgb}{0.333333,0.333333,0.333333}%
\pgfsetfillcolor{currentfill}%
\pgfsetlinewidth{0.803000pt}%
\definecolor{currentstroke}{rgb}{0.333333,0.333333,0.333333}%
\pgfsetstrokecolor{currentstroke}%
\pgfsetdash{}{0pt}%
\pgfsys@defobject{currentmarker}{\pgfqpoint{-0.048611in}{0.000000in}}{\pgfqpoint{-0.000000in}{0.000000in}}{%
\pgfpathmoveto{\pgfqpoint{-0.000000in}{0.000000in}}%
\pgfpathlineto{\pgfqpoint{-0.048611in}{0.000000in}}%
\pgfusepath{stroke,fill}%
}%
\begin{pgfscope}%
\pgfsys@transformshift{1.090674in}{8.671359in}%
\pgfsys@useobject{currentmarker}{}%
\end{pgfscope}%
\end{pgfscope}%
\begin{pgfscope}%
\definecolor{textcolor}{rgb}{0.333333,0.333333,0.333333}%
\pgfsetstrokecolor{textcolor}%
\pgfsetfillcolor{textcolor}%
\pgftext[x=0.443111in, y=8.588025in, left, base]{\color{textcolor}\rmfamily\fontsize{16.000000}{19.200000}\selectfont \(\displaystyle {40000}\)}%
\end{pgfscope}%
\begin{pgfscope}%
\definecolor{textcolor}{rgb}{0.333333,0.333333,0.333333}%
\pgfsetstrokecolor{textcolor}%
\pgfsetfillcolor{textcolor}%
\pgftext[x=0.387555in,y=5.167495in,,bottom,rotate=90.000000]{\color{textcolor}\rmfamily\fontsize{20.000000}{24.000000}\selectfont Capacity [Tons of Refrigeration]}%
\end{pgfscope}%
\begin{pgfscope}%
\pgfpathrectangle{\pgfqpoint{1.090674in}{0.637495in}}{\pgfqpoint{9.300000in}{9.060000in}}%
\pgfusepath{clip}%
\pgfsetbuttcap%
\pgfsetmiterjoin%
\definecolor{currentfill}{rgb}{0.580392,0.403922,0.741176}%
\pgfsetfillcolor{currentfill}%
\pgfsetfillopacity{0.990000}%
\pgfsetlinewidth{0.000000pt}%
\definecolor{currentstroke}{rgb}{0.000000,0.000000,0.000000}%
\pgfsetstrokecolor{currentstroke}%
\pgfsetstrokeopacity{0.990000}%
\pgfsetdash{}{0pt}%
\pgfpathmoveto{\pgfqpoint{1.478174in}{0.637495in}}%
\pgfpathlineto{\pgfqpoint{2.253174in}{0.637495in}}%
\pgfpathlineto{\pgfqpoint{2.253174in}{5.779271in}}%
\pgfpathlineto{\pgfqpoint{1.478174in}{5.779271in}}%
\pgfpathclose%
\pgfusepath{fill}%
\end{pgfscope}%
\begin{pgfscope}%
\pgfsetbuttcap%
\pgfsetmiterjoin%
\definecolor{currentfill}{rgb}{0.580392,0.403922,0.741176}%
\pgfsetfillcolor{currentfill}%
\pgfsetfillopacity{0.990000}%
\pgfsetlinewidth{0.000000pt}%
\definecolor{currentstroke}{rgb}{0.000000,0.000000,0.000000}%
\pgfsetstrokecolor{currentstroke}%
\pgfsetstrokeopacity{0.990000}%
\pgfsetdash{}{0pt}%
\pgfpathrectangle{\pgfqpoint{1.090674in}{0.637495in}}{\pgfqpoint{9.300000in}{9.060000in}}%
\pgfusepath{clip}%
\pgfpathmoveto{\pgfqpoint{1.478174in}{0.637495in}}%
\pgfpathlineto{\pgfqpoint{2.253174in}{0.637495in}}%
\pgfpathlineto{\pgfqpoint{2.253174in}{5.779271in}}%
\pgfpathlineto{\pgfqpoint{1.478174in}{5.779271in}}%
\pgfpathclose%
\pgfusepath{clip}%
\pgfsys@defobject{currentpattern}{\pgfqpoint{0in}{0in}}{\pgfqpoint{1in}{1in}}{%
\begin{pgfscope}%
\pgfpathrectangle{\pgfqpoint{0in}{0in}}{\pgfqpoint{1in}{1in}}%
\pgfusepath{clip}%
\pgfpathmoveto{\pgfqpoint{-0.500000in}{0.500000in}}%
\pgfpathlineto{\pgfqpoint{0.500000in}{1.500000in}}%
\pgfpathmoveto{\pgfqpoint{-0.333333in}{0.333333in}}%
\pgfpathlineto{\pgfqpoint{0.666667in}{1.333333in}}%
\pgfpathmoveto{\pgfqpoint{-0.166667in}{0.166667in}}%
\pgfpathlineto{\pgfqpoint{0.833333in}{1.166667in}}%
\pgfpathmoveto{\pgfqpoint{0.000000in}{0.000000in}}%
\pgfpathlineto{\pgfqpoint{1.000000in}{1.000000in}}%
\pgfpathmoveto{\pgfqpoint{0.166667in}{-0.166667in}}%
\pgfpathlineto{\pgfqpoint{1.166667in}{0.833333in}}%
\pgfpathmoveto{\pgfqpoint{0.333333in}{-0.333333in}}%
\pgfpathlineto{\pgfqpoint{1.333333in}{0.666667in}}%
\pgfpathmoveto{\pgfqpoint{0.500000in}{-0.500000in}}%
\pgfpathlineto{\pgfqpoint{1.500000in}{0.500000in}}%
\pgfpathmoveto{\pgfqpoint{-0.500000in}{0.500000in}}%
\pgfpathlineto{\pgfqpoint{0.500000in}{-0.500000in}}%
\pgfpathmoveto{\pgfqpoint{-0.333333in}{0.666667in}}%
\pgfpathlineto{\pgfqpoint{0.666667in}{-0.333333in}}%
\pgfpathmoveto{\pgfqpoint{-0.166667in}{0.833333in}}%
\pgfpathlineto{\pgfqpoint{0.833333in}{-0.166667in}}%
\pgfpathmoveto{\pgfqpoint{0.000000in}{1.000000in}}%
\pgfpathlineto{\pgfqpoint{1.000000in}{0.000000in}}%
\pgfpathmoveto{\pgfqpoint{0.166667in}{1.166667in}}%
\pgfpathlineto{\pgfqpoint{1.166667in}{0.166667in}}%
\pgfpathmoveto{\pgfqpoint{0.333333in}{1.333333in}}%
\pgfpathlineto{\pgfqpoint{1.333333in}{0.333333in}}%
\pgfpathmoveto{\pgfqpoint{0.500000in}{1.500000in}}%
\pgfpathlineto{\pgfqpoint{1.500000in}{0.500000in}}%
\pgfusepath{stroke}%
\end{pgfscope}%
}%
\pgfsys@transformshift{1.478174in}{0.637495in}%
\pgfsys@useobject{currentpattern}{}%
\pgfsys@transformshift{1in}{0in}%
\pgfsys@transformshift{-1in}{0in}%
\pgfsys@transformshift{0in}{1in}%
\pgfsys@useobject{currentpattern}{}%
\pgfsys@transformshift{1in}{0in}%
\pgfsys@transformshift{-1in}{0in}%
\pgfsys@transformshift{0in}{1in}%
\pgfsys@useobject{currentpattern}{}%
\pgfsys@transformshift{1in}{0in}%
\pgfsys@transformshift{-1in}{0in}%
\pgfsys@transformshift{0in}{1in}%
\pgfsys@useobject{currentpattern}{}%
\pgfsys@transformshift{1in}{0in}%
\pgfsys@transformshift{-1in}{0in}%
\pgfsys@transformshift{0in}{1in}%
\pgfsys@useobject{currentpattern}{}%
\pgfsys@transformshift{1in}{0in}%
\pgfsys@transformshift{-1in}{0in}%
\pgfsys@transformshift{0in}{1in}%
\pgfsys@useobject{currentpattern}{}%
\pgfsys@transformshift{1in}{0in}%
\pgfsys@transformshift{-1in}{0in}%
\pgfsys@transformshift{0in}{1in}%
\end{pgfscope}%
\begin{pgfscope}%
\pgfpathrectangle{\pgfqpoint{1.090674in}{0.637495in}}{\pgfqpoint{9.300000in}{9.060000in}}%
\pgfusepath{clip}%
\pgfsetbuttcap%
\pgfsetmiterjoin%
\definecolor{currentfill}{rgb}{0.580392,0.403922,0.741176}%
\pgfsetfillcolor{currentfill}%
\pgfsetfillopacity{0.990000}%
\pgfsetlinewidth{0.000000pt}%
\definecolor{currentstroke}{rgb}{0.000000,0.000000,0.000000}%
\pgfsetstrokecolor{currentstroke}%
\pgfsetstrokeopacity{0.990000}%
\pgfsetdash{}{0pt}%
\pgfpathmoveto{\pgfqpoint{3.028174in}{0.637495in}}%
\pgfpathlineto{\pgfqpoint{3.803174in}{0.637495in}}%
\pgfpathlineto{\pgfqpoint{3.803174in}{6.036360in}}%
\pgfpathlineto{\pgfqpoint{3.028174in}{6.036360in}}%
\pgfpathclose%
\pgfusepath{fill}%
\end{pgfscope}%
\begin{pgfscope}%
\pgfsetbuttcap%
\pgfsetmiterjoin%
\definecolor{currentfill}{rgb}{0.580392,0.403922,0.741176}%
\pgfsetfillcolor{currentfill}%
\pgfsetfillopacity{0.990000}%
\pgfsetlinewidth{0.000000pt}%
\definecolor{currentstroke}{rgb}{0.000000,0.000000,0.000000}%
\pgfsetstrokecolor{currentstroke}%
\pgfsetstrokeopacity{0.990000}%
\pgfsetdash{}{0pt}%
\pgfpathrectangle{\pgfqpoint{1.090674in}{0.637495in}}{\pgfqpoint{9.300000in}{9.060000in}}%
\pgfusepath{clip}%
\pgfpathmoveto{\pgfqpoint{3.028174in}{0.637495in}}%
\pgfpathlineto{\pgfqpoint{3.803174in}{0.637495in}}%
\pgfpathlineto{\pgfqpoint{3.803174in}{6.036360in}}%
\pgfpathlineto{\pgfqpoint{3.028174in}{6.036360in}}%
\pgfpathclose%
\pgfusepath{clip}%
\pgfsys@defobject{currentpattern}{\pgfqpoint{0in}{0in}}{\pgfqpoint{1in}{1in}}{%
\begin{pgfscope}%
\pgfpathrectangle{\pgfqpoint{0in}{0in}}{\pgfqpoint{1in}{1in}}%
\pgfusepath{clip}%
\pgfpathmoveto{\pgfqpoint{-0.500000in}{0.500000in}}%
\pgfpathlineto{\pgfqpoint{0.500000in}{1.500000in}}%
\pgfpathmoveto{\pgfqpoint{-0.333333in}{0.333333in}}%
\pgfpathlineto{\pgfqpoint{0.666667in}{1.333333in}}%
\pgfpathmoveto{\pgfqpoint{-0.166667in}{0.166667in}}%
\pgfpathlineto{\pgfqpoint{0.833333in}{1.166667in}}%
\pgfpathmoveto{\pgfqpoint{0.000000in}{0.000000in}}%
\pgfpathlineto{\pgfqpoint{1.000000in}{1.000000in}}%
\pgfpathmoveto{\pgfqpoint{0.166667in}{-0.166667in}}%
\pgfpathlineto{\pgfqpoint{1.166667in}{0.833333in}}%
\pgfpathmoveto{\pgfqpoint{0.333333in}{-0.333333in}}%
\pgfpathlineto{\pgfqpoint{1.333333in}{0.666667in}}%
\pgfpathmoveto{\pgfqpoint{0.500000in}{-0.500000in}}%
\pgfpathlineto{\pgfqpoint{1.500000in}{0.500000in}}%
\pgfpathmoveto{\pgfqpoint{-0.500000in}{0.500000in}}%
\pgfpathlineto{\pgfqpoint{0.500000in}{-0.500000in}}%
\pgfpathmoveto{\pgfqpoint{-0.333333in}{0.666667in}}%
\pgfpathlineto{\pgfqpoint{0.666667in}{-0.333333in}}%
\pgfpathmoveto{\pgfqpoint{-0.166667in}{0.833333in}}%
\pgfpathlineto{\pgfqpoint{0.833333in}{-0.166667in}}%
\pgfpathmoveto{\pgfqpoint{0.000000in}{1.000000in}}%
\pgfpathlineto{\pgfqpoint{1.000000in}{0.000000in}}%
\pgfpathmoveto{\pgfqpoint{0.166667in}{1.166667in}}%
\pgfpathlineto{\pgfqpoint{1.166667in}{0.166667in}}%
\pgfpathmoveto{\pgfqpoint{0.333333in}{1.333333in}}%
\pgfpathlineto{\pgfqpoint{1.333333in}{0.333333in}}%
\pgfpathmoveto{\pgfqpoint{0.500000in}{1.500000in}}%
\pgfpathlineto{\pgfqpoint{1.500000in}{0.500000in}}%
\pgfusepath{stroke}%
\end{pgfscope}%
}%
\pgfsys@transformshift{3.028174in}{0.637495in}%
\pgfsys@useobject{currentpattern}{}%
\pgfsys@transformshift{1in}{0in}%
\pgfsys@transformshift{-1in}{0in}%
\pgfsys@transformshift{0in}{1in}%
\pgfsys@useobject{currentpattern}{}%
\pgfsys@transformshift{1in}{0in}%
\pgfsys@transformshift{-1in}{0in}%
\pgfsys@transformshift{0in}{1in}%
\pgfsys@useobject{currentpattern}{}%
\pgfsys@transformshift{1in}{0in}%
\pgfsys@transformshift{-1in}{0in}%
\pgfsys@transformshift{0in}{1in}%
\pgfsys@useobject{currentpattern}{}%
\pgfsys@transformshift{1in}{0in}%
\pgfsys@transformshift{-1in}{0in}%
\pgfsys@transformshift{0in}{1in}%
\pgfsys@useobject{currentpattern}{}%
\pgfsys@transformshift{1in}{0in}%
\pgfsys@transformshift{-1in}{0in}%
\pgfsys@transformshift{0in}{1in}%
\pgfsys@useobject{currentpattern}{}%
\pgfsys@transformshift{1in}{0in}%
\pgfsys@transformshift{-1in}{0in}%
\pgfsys@transformshift{0in}{1in}%
\end{pgfscope}%
\begin{pgfscope}%
\pgfpathrectangle{\pgfqpoint{1.090674in}{0.637495in}}{\pgfqpoint{9.300000in}{9.060000in}}%
\pgfusepath{clip}%
\pgfsetbuttcap%
\pgfsetmiterjoin%
\definecolor{currentfill}{rgb}{0.580392,0.403922,0.741176}%
\pgfsetfillcolor{currentfill}%
\pgfsetfillopacity{0.990000}%
\pgfsetlinewidth{0.000000pt}%
\definecolor{currentstroke}{rgb}{0.000000,0.000000,0.000000}%
\pgfsetstrokecolor{currentstroke}%
\pgfsetstrokeopacity{0.990000}%
\pgfsetdash{}{0pt}%
\pgfpathmoveto{\pgfqpoint{4.578174in}{0.637495in}}%
\pgfpathlineto{\pgfqpoint{5.353174in}{0.637495in}}%
\pgfpathlineto{\pgfqpoint{5.353174in}{6.293449in}}%
\pgfpathlineto{\pgfqpoint{4.578174in}{6.293449in}}%
\pgfpathclose%
\pgfusepath{fill}%
\end{pgfscope}%
\begin{pgfscope}%
\pgfsetbuttcap%
\pgfsetmiterjoin%
\definecolor{currentfill}{rgb}{0.580392,0.403922,0.741176}%
\pgfsetfillcolor{currentfill}%
\pgfsetfillopacity{0.990000}%
\pgfsetlinewidth{0.000000pt}%
\definecolor{currentstroke}{rgb}{0.000000,0.000000,0.000000}%
\pgfsetstrokecolor{currentstroke}%
\pgfsetstrokeopacity{0.990000}%
\pgfsetdash{}{0pt}%
\pgfpathrectangle{\pgfqpoint{1.090674in}{0.637495in}}{\pgfqpoint{9.300000in}{9.060000in}}%
\pgfusepath{clip}%
\pgfpathmoveto{\pgfqpoint{4.578174in}{0.637495in}}%
\pgfpathlineto{\pgfqpoint{5.353174in}{0.637495in}}%
\pgfpathlineto{\pgfqpoint{5.353174in}{6.293449in}}%
\pgfpathlineto{\pgfqpoint{4.578174in}{6.293449in}}%
\pgfpathclose%
\pgfusepath{clip}%
\pgfsys@defobject{currentpattern}{\pgfqpoint{0in}{0in}}{\pgfqpoint{1in}{1in}}{%
\begin{pgfscope}%
\pgfpathrectangle{\pgfqpoint{0in}{0in}}{\pgfqpoint{1in}{1in}}%
\pgfusepath{clip}%
\pgfpathmoveto{\pgfqpoint{-0.500000in}{0.500000in}}%
\pgfpathlineto{\pgfqpoint{0.500000in}{1.500000in}}%
\pgfpathmoveto{\pgfqpoint{-0.333333in}{0.333333in}}%
\pgfpathlineto{\pgfqpoint{0.666667in}{1.333333in}}%
\pgfpathmoveto{\pgfqpoint{-0.166667in}{0.166667in}}%
\pgfpathlineto{\pgfqpoint{0.833333in}{1.166667in}}%
\pgfpathmoveto{\pgfqpoint{0.000000in}{0.000000in}}%
\pgfpathlineto{\pgfqpoint{1.000000in}{1.000000in}}%
\pgfpathmoveto{\pgfqpoint{0.166667in}{-0.166667in}}%
\pgfpathlineto{\pgfqpoint{1.166667in}{0.833333in}}%
\pgfpathmoveto{\pgfqpoint{0.333333in}{-0.333333in}}%
\pgfpathlineto{\pgfqpoint{1.333333in}{0.666667in}}%
\pgfpathmoveto{\pgfqpoint{0.500000in}{-0.500000in}}%
\pgfpathlineto{\pgfqpoint{1.500000in}{0.500000in}}%
\pgfpathmoveto{\pgfqpoint{-0.500000in}{0.500000in}}%
\pgfpathlineto{\pgfqpoint{0.500000in}{-0.500000in}}%
\pgfpathmoveto{\pgfqpoint{-0.333333in}{0.666667in}}%
\pgfpathlineto{\pgfqpoint{0.666667in}{-0.333333in}}%
\pgfpathmoveto{\pgfqpoint{-0.166667in}{0.833333in}}%
\pgfpathlineto{\pgfqpoint{0.833333in}{-0.166667in}}%
\pgfpathmoveto{\pgfqpoint{0.000000in}{1.000000in}}%
\pgfpathlineto{\pgfqpoint{1.000000in}{0.000000in}}%
\pgfpathmoveto{\pgfqpoint{0.166667in}{1.166667in}}%
\pgfpathlineto{\pgfqpoint{1.166667in}{0.166667in}}%
\pgfpathmoveto{\pgfqpoint{0.333333in}{1.333333in}}%
\pgfpathlineto{\pgfqpoint{1.333333in}{0.333333in}}%
\pgfpathmoveto{\pgfqpoint{0.500000in}{1.500000in}}%
\pgfpathlineto{\pgfqpoint{1.500000in}{0.500000in}}%
\pgfusepath{stroke}%
\end{pgfscope}%
}%
\pgfsys@transformshift{4.578174in}{0.637495in}%
\pgfsys@useobject{currentpattern}{}%
\pgfsys@transformshift{1in}{0in}%
\pgfsys@transformshift{-1in}{0in}%
\pgfsys@transformshift{0in}{1in}%
\pgfsys@useobject{currentpattern}{}%
\pgfsys@transformshift{1in}{0in}%
\pgfsys@transformshift{-1in}{0in}%
\pgfsys@transformshift{0in}{1in}%
\pgfsys@useobject{currentpattern}{}%
\pgfsys@transformshift{1in}{0in}%
\pgfsys@transformshift{-1in}{0in}%
\pgfsys@transformshift{0in}{1in}%
\pgfsys@useobject{currentpattern}{}%
\pgfsys@transformshift{1in}{0in}%
\pgfsys@transformshift{-1in}{0in}%
\pgfsys@transformshift{0in}{1in}%
\pgfsys@useobject{currentpattern}{}%
\pgfsys@transformshift{1in}{0in}%
\pgfsys@transformshift{-1in}{0in}%
\pgfsys@transformshift{0in}{1in}%
\pgfsys@useobject{currentpattern}{}%
\pgfsys@transformshift{1in}{0in}%
\pgfsys@transformshift{-1in}{0in}%
\pgfsys@transformshift{0in}{1in}%
\end{pgfscope}%
\begin{pgfscope}%
\pgfpathrectangle{\pgfqpoint{1.090674in}{0.637495in}}{\pgfqpoint{9.300000in}{9.060000in}}%
\pgfusepath{clip}%
\pgfsetbuttcap%
\pgfsetmiterjoin%
\definecolor{currentfill}{rgb}{0.580392,0.403922,0.741176}%
\pgfsetfillcolor{currentfill}%
\pgfsetfillopacity{0.990000}%
\pgfsetlinewidth{0.000000pt}%
\definecolor{currentstroke}{rgb}{0.000000,0.000000,0.000000}%
\pgfsetstrokecolor{currentstroke}%
\pgfsetstrokeopacity{0.990000}%
\pgfsetdash{}{0pt}%
\pgfpathmoveto{\pgfqpoint{6.128174in}{0.637495in}}%
\pgfpathlineto{\pgfqpoint{6.903174in}{0.637495in}}%
\pgfpathlineto{\pgfqpoint{6.903174in}{6.550538in}}%
\pgfpathlineto{\pgfqpoint{6.128174in}{6.550538in}}%
\pgfpathclose%
\pgfusepath{fill}%
\end{pgfscope}%
\begin{pgfscope}%
\pgfsetbuttcap%
\pgfsetmiterjoin%
\definecolor{currentfill}{rgb}{0.580392,0.403922,0.741176}%
\pgfsetfillcolor{currentfill}%
\pgfsetfillopacity{0.990000}%
\pgfsetlinewidth{0.000000pt}%
\definecolor{currentstroke}{rgb}{0.000000,0.000000,0.000000}%
\pgfsetstrokecolor{currentstroke}%
\pgfsetstrokeopacity{0.990000}%
\pgfsetdash{}{0pt}%
\pgfpathrectangle{\pgfqpoint{1.090674in}{0.637495in}}{\pgfqpoint{9.300000in}{9.060000in}}%
\pgfusepath{clip}%
\pgfpathmoveto{\pgfqpoint{6.128174in}{0.637495in}}%
\pgfpathlineto{\pgfqpoint{6.903174in}{0.637495in}}%
\pgfpathlineto{\pgfqpoint{6.903174in}{6.550538in}}%
\pgfpathlineto{\pgfqpoint{6.128174in}{6.550538in}}%
\pgfpathclose%
\pgfusepath{clip}%
\pgfsys@defobject{currentpattern}{\pgfqpoint{0in}{0in}}{\pgfqpoint{1in}{1in}}{%
\begin{pgfscope}%
\pgfpathrectangle{\pgfqpoint{0in}{0in}}{\pgfqpoint{1in}{1in}}%
\pgfusepath{clip}%
\pgfpathmoveto{\pgfqpoint{-0.500000in}{0.500000in}}%
\pgfpathlineto{\pgfqpoint{0.500000in}{1.500000in}}%
\pgfpathmoveto{\pgfqpoint{-0.333333in}{0.333333in}}%
\pgfpathlineto{\pgfqpoint{0.666667in}{1.333333in}}%
\pgfpathmoveto{\pgfqpoint{-0.166667in}{0.166667in}}%
\pgfpathlineto{\pgfqpoint{0.833333in}{1.166667in}}%
\pgfpathmoveto{\pgfqpoint{0.000000in}{0.000000in}}%
\pgfpathlineto{\pgfqpoint{1.000000in}{1.000000in}}%
\pgfpathmoveto{\pgfqpoint{0.166667in}{-0.166667in}}%
\pgfpathlineto{\pgfqpoint{1.166667in}{0.833333in}}%
\pgfpathmoveto{\pgfqpoint{0.333333in}{-0.333333in}}%
\pgfpathlineto{\pgfqpoint{1.333333in}{0.666667in}}%
\pgfpathmoveto{\pgfqpoint{0.500000in}{-0.500000in}}%
\pgfpathlineto{\pgfqpoint{1.500000in}{0.500000in}}%
\pgfpathmoveto{\pgfqpoint{-0.500000in}{0.500000in}}%
\pgfpathlineto{\pgfqpoint{0.500000in}{-0.500000in}}%
\pgfpathmoveto{\pgfqpoint{-0.333333in}{0.666667in}}%
\pgfpathlineto{\pgfqpoint{0.666667in}{-0.333333in}}%
\pgfpathmoveto{\pgfqpoint{-0.166667in}{0.833333in}}%
\pgfpathlineto{\pgfqpoint{0.833333in}{-0.166667in}}%
\pgfpathmoveto{\pgfqpoint{0.000000in}{1.000000in}}%
\pgfpathlineto{\pgfqpoint{1.000000in}{0.000000in}}%
\pgfpathmoveto{\pgfqpoint{0.166667in}{1.166667in}}%
\pgfpathlineto{\pgfqpoint{1.166667in}{0.166667in}}%
\pgfpathmoveto{\pgfqpoint{0.333333in}{1.333333in}}%
\pgfpathlineto{\pgfqpoint{1.333333in}{0.333333in}}%
\pgfpathmoveto{\pgfqpoint{0.500000in}{1.500000in}}%
\pgfpathlineto{\pgfqpoint{1.500000in}{0.500000in}}%
\pgfusepath{stroke}%
\end{pgfscope}%
}%
\pgfsys@transformshift{6.128174in}{0.637495in}%
\pgfsys@useobject{currentpattern}{}%
\pgfsys@transformshift{1in}{0in}%
\pgfsys@transformshift{-1in}{0in}%
\pgfsys@transformshift{0in}{1in}%
\pgfsys@useobject{currentpattern}{}%
\pgfsys@transformshift{1in}{0in}%
\pgfsys@transformshift{-1in}{0in}%
\pgfsys@transformshift{0in}{1in}%
\pgfsys@useobject{currentpattern}{}%
\pgfsys@transformshift{1in}{0in}%
\pgfsys@transformshift{-1in}{0in}%
\pgfsys@transformshift{0in}{1in}%
\pgfsys@useobject{currentpattern}{}%
\pgfsys@transformshift{1in}{0in}%
\pgfsys@transformshift{-1in}{0in}%
\pgfsys@transformshift{0in}{1in}%
\pgfsys@useobject{currentpattern}{}%
\pgfsys@transformshift{1in}{0in}%
\pgfsys@transformshift{-1in}{0in}%
\pgfsys@transformshift{0in}{1in}%
\pgfsys@useobject{currentpattern}{}%
\pgfsys@transformshift{1in}{0in}%
\pgfsys@transformshift{-1in}{0in}%
\pgfsys@transformshift{0in}{1in}%
\end{pgfscope}%
\begin{pgfscope}%
\pgfpathrectangle{\pgfqpoint{1.090674in}{0.637495in}}{\pgfqpoint{9.300000in}{9.060000in}}%
\pgfusepath{clip}%
\pgfsetbuttcap%
\pgfsetmiterjoin%
\definecolor{currentfill}{rgb}{0.580392,0.403922,0.741176}%
\pgfsetfillcolor{currentfill}%
\pgfsetfillopacity{0.990000}%
\pgfsetlinewidth{0.000000pt}%
\definecolor{currentstroke}{rgb}{0.000000,0.000000,0.000000}%
\pgfsetstrokecolor{currentstroke}%
\pgfsetstrokeopacity{0.990000}%
\pgfsetdash{}{0pt}%
\pgfpathmoveto{\pgfqpoint{7.678174in}{0.637495in}}%
\pgfpathlineto{\pgfqpoint{8.453174in}{0.637495in}}%
\pgfpathlineto{\pgfqpoint{8.453174in}{6.807627in}}%
\pgfpathlineto{\pgfqpoint{7.678174in}{6.807627in}}%
\pgfpathclose%
\pgfusepath{fill}%
\end{pgfscope}%
\begin{pgfscope}%
\pgfsetbuttcap%
\pgfsetmiterjoin%
\definecolor{currentfill}{rgb}{0.580392,0.403922,0.741176}%
\pgfsetfillcolor{currentfill}%
\pgfsetfillopacity{0.990000}%
\pgfsetlinewidth{0.000000pt}%
\definecolor{currentstroke}{rgb}{0.000000,0.000000,0.000000}%
\pgfsetstrokecolor{currentstroke}%
\pgfsetstrokeopacity{0.990000}%
\pgfsetdash{}{0pt}%
\pgfpathrectangle{\pgfqpoint{1.090674in}{0.637495in}}{\pgfqpoint{9.300000in}{9.060000in}}%
\pgfusepath{clip}%
\pgfpathmoveto{\pgfqpoint{7.678174in}{0.637495in}}%
\pgfpathlineto{\pgfqpoint{8.453174in}{0.637495in}}%
\pgfpathlineto{\pgfqpoint{8.453174in}{6.807627in}}%
\pgfpathlineto{\pgfqpoint{7.678174in}{6.807627in}}%
\pgfpathclose%
\pgfusepath{clip}%
\pgfsys@defobject{currentpattern}{\pgfqpoint{0in}{0in}}{\pgfqpoint{1in}{1in}}{%
\begin{pgfscope}%
\pgfpathrectangle{\pgfqpoint{0in}{0in}}{\pgfqpoint{1in}{1in}}%
\pgfusepath{clip}%
\pgfpathmoveto{\pgfqpoint{-0.500000in}{0.500000in}}%
\pgfpathlineto{\pgfqpoint{0.500000in}{1.500000in}}%
\pgfpathmoveto{\pgfqpoint{-0.333333in}{0.333333in}}%
\pgfpathlineto{\pgfqpoint{0.666667in}{1.333333in}}%
\pgfpathmoveto{\pgfqpoint{-0.166667in}{0.166667in}}%
\pgfpathlineto{\pgfqpoint{0.833333in}{1.166667in}}%
\pgfpathmoveto{\pgfqpoint{0.000000in}{0.000000in}}%
\pgfpathlineto{\pgfqpoint{1.000000in}{1.000000in}}%
\pgfpathmoveto{\pgfqpoint{0.166667in}{-0.166667in}}%
\pgfpathlineto{\pgfqpoint{1.166667in}{0.833333in}}%
\pgfpathmoveto{\pgfqpoint{0.333333in}{-0.333333in}}%
\pgfpathlineto{\pgfqpoint{1.333333in}{0.666667in}}%
\pgfpathmoveto{\pgfqpoint{0.500000in}{-0.500000in}}%
\pgfpathlineto{\pgfqpoint{1.500000in}{0.500000in}}%
\pgfpathmoveto{\pgfqpoint{-0.500000in}{0.500000in}}%
\pgfpathlineto{\pgfqpoint{0.500000in}{-0.500000in}}%
\pgfpathmoveto{\pgfqpoint{-0.333333in}{0.666667in}}%
\pgfpathlineto{\pgfqpoint{0.666667in}{-0.333333in}}%
\pgfpathmoveto{\pgfqpoint{-0.166667in}{0.833333in}}%
\pgfpathlineto{\pgfqpoint{0.833333in}{-0.166667in}}%
\pgfpathmoveto{\pgfqpoint{0.000000in}{1.000000in}}%
\pgfpathlineto{\pgfqpoint{1.000000in}{0.000000in}}%
\pgfpathmoveto{\pgfqpoint{0.166667in}{1.166667in}}%
\pgfpathlineto{\pgfqpoint{1.166667in}{0.166667in}}%
\pgfpathmoveto{\pgfqpoint{0.333333in}{1.333333in}}%
\pgfpathlineto{\pgfqpoint{1.333333in}{0.333333in}}%
\pgfpathmoveto{\pgfqpoint{0.500000in}{1.500000in}}%
\pgfpathlineto{\pgfqpoint{1.500000in}{0.500000in}}%
\pgfusepath{stroke}%
\end{pgfscope}%
}%
\pgfsys@transformshift{7.678174in}{0.637495in}%
\pgfsys@useobject{currentpattern}{}%
\pgfsys@transformshift{1in}{0in}%
\pgfsys@transformshift{-1in}{0in}%
\pgfsys@transformshift{0in}{1in}%
\pgfsys@useobject{currentpattern}{}%
\pgfsys@transformshift{1in}{0in}%
\pgfsys@transformshift{-1in}{0in}%
\pgfsys@transformshift{0in}{1in}%
\pgfsys@useobject{currentpattern}{}%
\pgfsys@transformshift{1in}{0in}%
\pgfsys@transformshift{-1in}{0in}%
\pgfsys@transformshift{0in}{1in}%
\pgfsys@useobject{currentpattern}{}%
\pgfsys@transformshift{1in}{0in}%
\pgfsys@transformshift{-1in}{0in}%
\pgfsys@transformshift{0in}{1in}%
\pgfsys@useobject{currentpattern}{}%
\pgfsys@transformshift{1in}{0in}%
\pgfsys@transformshift{-1in}{0in}%
\pgfsys@transformshift{0in}{1in}%
\pgfsys@useobject{currentpattern}{}%
\pgfsys@transformshift{1in}{0in}%
\pgfsys@transformshift{-1in}{0in}%
\pgfsys@transformshift{0in}{1in}%
\pgfsys@useobject{currentpattern}{}%
\pgfsys@transformshift{1in}{0in}%
\pgfsys@transformshift{-1in}{0in}%
\pgfsys@transformshift{0in}{1in}%
\end{pgfscope}%
\begin{pgfscope}%
\pgfpathrectangle{\pgfqpoint{1.090674in}{0.637495in}}{\pgfqpoint{9.300000in}{9.060000in}}%
\pgfusepath{clip}%
\pgfsetbuttcap%
\pgfsetmiterjoin%
\definecolor{currentfill}{rgb}{0.580392,0.403922,0.741176}%
\pgfsetfillcolor{currentfill}%
\pgfsetfillopacity{0.990000}%
\pgfsetlinewidth{0.000000pt}%
\definecolor{currentstroke}{rgb}{0.000000,0.000000,0.000000}%
\pgfsetstrokecolor{currentstroke}%
\pgfsetstrokeopacity{0.990000}%
\pgfsetdash{}{0pt}%
\pgfpathmoveto{\pgfqpoint{9.228174in}{0.637495in}}%
\pgfpathlineto{\pgfqpoint{10.003174in}{0.637495in}}%
\pgfpathlineto{\pgfqpoint{10.003174in}{7.064715in}}%
\pgfpathlineto{\pgfqpoint{9.228174in}{7.064715in}}%
\pgfpathclose%
\pgfusepath{fill}%
\end{pgfscope}%
\begin{pgfscope}%
\pgfsetbuttcap%
\pgfsetmiterjoin%
\definecolor{currentfill}{rgb}{0.580392,0.403922,0.741176}%
\pgfsetfillcolor{currentfill}%
\pgfsetfillopacity{0.990000}%
\pgfsetlinewidth{0.000000pt}%
\definecolor{currentstroke}{rgb}{0.000000,0.000000,0.000000}%
\pgfsetstrokecolor{currentstroke}%
\pgfsetstrokeopacity{0.990000}%
\pgfsetdash{}{0pt}%
\pgfpathrectangle{\pgfqpoint{1.090674in}{0.637495in}}{\pgfqpoint{9.300000in}{9.060000in}}%
\pgfusepath{clip}%
\pgfpathmoveto{\pgfqpoint{9.228174in}{0.637495in}}%
\pgfpathlineto{\pgfqpoint{10.003174in}{0.637495in}}%
\pgfpathlineto{\pgfqpoint{10.003174in}{7.064715in}}%
\pgfpathlineto{\pgfqpoint{9.228174in}{7.064715in}}%
\pgfpathclose%
\pgfusepath{clip}%
\pgfsys@defobject{currentpattern}{\pgfqpoint{0in}{0in}}{\pgfqpoint{1in}{1in}}{%
\begin{pgfscope}%
\pgfpathrectangle{\pgfqpoint{0in}{0in}}{\pgfqpoint{1in}{1in}}%
\pgfusepath{clip}%
\pgfpathmoveto{\pgfqpoint{-0.500000in}{0.500000in}}%
\pgfpathlineto{\pgfqpoint{0.500000in}{1.500000in}}%
\pgfpathmoveto{\pgfqpoint{-0.333333in}{0.333333in}}%
\pgfpathlineto{\pgfqpoint{0.666667in}{1.333333in}}%
\pgfpathmoveto{\pgfqpoint{-0.166667in}{0.166667in}}%
\pgfpathlineto{\pgfqpoint{0.833333in}{1.166667in}}%
\pgfpathmoveto{\pgfqpoint{0.000000in}{0.000000in}}%
\pgfpathlineto{\pgfqpoint{1.000000in}{1.000000in}}%
\pgfpathmoveto{\pgfqpoint{0.166667in}{-0.166667in}}%
\pgfpathlineto{\pgfqpoint{1.166667in}{0.833333in}}%
\pgfpathmoveto{\pgfqpoint{0.333333in}{-0.333333in}}%
\pgfpathlineto{\pgfqpoint{1.333333in}{0.666667in}}%
\pgfpathmoveto{\pgfqpoint{0.500000in}{-0.500000in}}%
\pgfpathlineto{\pgfqpoint{1.500000in}{0.500000in}}%
\pgfpathmoveto{\pgfqpoint{-0.500000in}{0.500000in}}%
\pgfpathlineto{\pgfqpoint{0.500000in}{-0.500000in}}%
\pgfpathmoveto{\pgfqpoint{-0.333333in}{0.666667in}}%
\pgfpathlineto{\pgfqpoint{0.666667in}{-0.333333in}}%
\pgfpathmoveto{\pgfqpoint{-0.166667in}{0.833333in}}%
\pgfpathlineto{\pgfqpoint{0.833333in}{-0.166667in}}%
\pgfpathmoveto{\pgfqpoint{0.000000in}{1.000000in}}%
\pgfpathlineto{\pgfqpoint{1.000000in}{0.000000in}}%
\pgfpathmoveto{\pgfqpoint{0.166667in}{1.166667in}}%
\pgfpathlineto{\pgfqpoint{1.166667in}{0.166667in}}%
\pgfpathmoveto{\pgfqpoint{0.333333in}{1.333333in}}%
\pgfpathlineto{\pgfqpoint{1.333333in}{0.333333in}}%
\pgfpathmoveto{\pgfqpoint{0.500000in}{1.500000in}}%
\pgfpathlineto{\pgfqpoint{1.500000in}{0.500000in}}%
\pgfusepath{stroke}%
\end{pgfscope}%
}%
\pgfsys@transformshift{9.228174in}{0.637495in}%
\pgfsys@useobject{currentpattern}{}%
\pgfsys@transformshift{1in}{0in}%
\pgfsys@transformshift{-1in}{0in}%
\pgfsys@transformshift{0in}{1in}%
\pgfsys@useobject{currentpattern}{}%
\pgfsys@transformshift{1in}{0in}%
\pgfsys@transformshift{-1in}{0in}%
\pgfsys@transformshift{0in}{1in}%
\pgfsys@useobject{currentpattern}{}%
\pgfsys@transformshift{1in}{0in}%
\pgfsys@transformshift{-1in}{0in}%
\pgfsys@transformshift{0in}{1in}%
\pgfsys@useobject{currentpattern}{}%
\pgfsys@transformshift{1in}{0in}%
\pgfsys@transformshift{-1in}{0in}%
\pgfsys@transformshift{0in}{1in}%
\pgfsys@useobject{currentpattern}{}%
\pgfsys@transformshift{1in}{0in}%
\pgfsys@transformshift{-1in}{0in}%
\pgfsys@transformshift{0in}{1in}%
\pgfsys@useobject{currentpattern}{}%
\pgfsys@transformshift{1in}{0in}%
\pgfsys@transformshift{-1in}{0in}%
\pgfsys@transformshift{0in}{1in}%
\pgfsys@useobject{currentpattern}{}%
\pgfsys@transformshift{1in}{0in}%
\pgfsys@transformshift{-1in}{0in}%
\pgfsys@transformshift{0in}{1in}%
\end{pgfscope}%
\begin{pgfscope}%
\pgfpathrectangle{\pgfqpoint{1.090674in}{0.637495in}}{\pgfqpoint{9.300000in}{9.060000in}}%
\pgfusepath{clip}%
\pgfsetbuttcap%
\pgfsetmiterjoin%
\definecolor{currentfill}{rgb}{0.890196,0.466667,0.760784}%
\pgfsetfillcolor{currentfill}%
\pgfsetfillopacity{0.990000}%
\pgfsetlinewidth{0.000000pt}%
\definecolor{currentstroke}{rgb}{0.000000,0.000000,0.000000}%
\pgfsetstrokecolor{currentstroke}%
\pgfsetstrokeopacity{0.990000}%
\pgfsetdash{}{0pt}%
\pgfpathmoveto{\pgfqpoint{1.478174in}{5.779271in}}%
\pgfpathlineto{\pgfqpoint{2.253174in}{5.779271in}}%
\pgfpathlineto{\pgfqpoint{2.253174in}{7.540352in}}%
\pgfpathlineto{\pgfqpoint{1.478174in}{7.540352in}}%
\pgfpathclose%
\pgfusepath{fill}%
\end{pgfscope}%
\begin{pgfscope}%
\pgfsetbuttcap%
\pgfsetmiterjoin%
\definecolor{currentfill}{rgb}{0.890196,0.466667,0.760784}%
\pgfsetfillcolor{currentfill}%
\pgfsetfillopacity{0.990000}%
\pgfsetlinewidth{0.000000pt}%
\definecolor{currentstroke}{rgb}{0.000000,0.000000,0.000000}%
\pgfsetstrokecolor{currentstroke}%
\pgfsetstrokeopacity{0.990000}%
\pgfsetdash{}{0pt}%
\pgfpathrectangle{\pgfqpoint{1.090674in}{0.637495in}}{\pgfqpoint{9.300000in}{9.060000in}}%
\pgfusepath{clip}%
\pgfpathmoveto{\pgfqpoint{1.478174in}{5.779271in}}%
\pgfpathlineto{\pgfqpoint{2.253174in}{5.779271in}}%
\pgfpathlineto{\pgfqpoint{2.253174in}{7.540352in}}%
\pgfpathlineto{\pgfqpoint{1.478174in}{7.540352in}}%
\pgfpathclose%
\pgfusepath{clip}%
\pgfsys@defobject{currentpattern}{\pgfqpoint{0in}{0in}}{\pgfqpoint{1in}{1in}}{%
\begin{pgfscope}%
\pgfpathrectangle{\pgfqpoint{0in}{0in}}{\pgfqpoint{1in}{1in}}%
\pgfusepath{clip}%
\pgfpathmoveto{\pgfqpoint{0.000000in}{-0.058333in}}%
\pgfpathcurveto{\pgfqpoint{0.015470in}{-0.058333in}}{\pgfqpoint{0.030309in}{-0.052187in}}{\pgfqpoint{0.041248in}{-0.041248in}}%
\pgfpathcurveto{\pgfqpoint{0.052187in}{-0.030309in}}{\pgfqpoint{0.058333in}{-0.015470in}}{\pgfqpoint{0.058333in}{0.000000in}}%
\pgfpathcurveto{\pgfqpoint{0.058333in}{0.015470in}}{\pgfqpoint{0.052187in}{0.030309in}}{\pgfqpoint{0.041248in}{0.041248in}}%
\pgfpathcurveto{\pgfqpoint{0.030309in}{0.052187in}}{\pgfqpoint{0.015470in}{0.058333in}}{\pgfqpoint{0.000000in}{0.058333in}}%
\pgfpathcurveto{\pgfqpoint{-0.015470in}{0.058333in}}{\pgfqpoint{-0.030309in}{0.052187in}}{\pgfqpoint{-0.041248in}{0.041248in}}%
\pgfpathcurveto{\pgfqpoint{-0.052187in}{0.030309in}}{\pgfqpoint{-0.058333in}{0.015470in}}{\pgfqpoint{-0.058333in}{0.000000in}}%
\pgfpathcurveto{\pgfqpoint{-0.058333in}{-0.015470in}}{\pgfqpoint{-0.052187in}{-0.030309in}}{\pgfqpoint{-0.041248in}{-0.041248in}}%
\pgfpathcurveto{\pgfqpoint{-0.030309in}{-0.052187in}}{\pgfqpoint{-0.015470in}{-0.058333in}}{\pgfqpoint{0.000000in}{-0.058333in}}%
\pgfpathclose%
\pgfpathmoveto{\pgfqpoint{0.000000in}{-0.052500in}}%
\pgfpathcurveto{\pgfqpoint{0.000000in}{-0.052500in}}{\pgfqpoint{-0.013923in}{-0.052500in}}{\pgfqpoint{-0.027278in}{-0.046968in}}%
\pgfpathcurveto{\pgfqpoint{-0.037123in}{-0.037123in}}{\pgfqpoint{-0.046968in}{-0.027278in}}{\pgfqpoint{-0.052500in}{-0.013923in}}%
\pgfpathcurveto{\pgfqpoint{-0.052500in}{0.000000in}}{\pgfqpoint{-0.052500in}{0.013923in}}{\pgfqpoint{-0.046968in}{0.027278in}}%
\pgfpathcurveto{\pgfqpoint{-0.037123in}{0.037123in}}{\pgfqpoint{-0.027278in}{0.046968in}}{\pgfqpoint{-0.013923in}{0.052500in}}%
\pgfpathcurveto{\pgfqpoint{0.000000in}{0.052500in}}{\pgfqpoint{0.013923in}{0.052500in}}{\pgfqpoint{0.027278in}{0.046968in}}%
\pgfpathcurveto{\pgfqpoint{0.037123in}{0.037123in}}{\pgfqpoint{0.046968in}{0.027278in}}{\pgfqpoint{0.052500in}{0.013923in}}%
\pgfpathcurveto{\pgfqpoint{0.052500in}{0.000000in}}{\pgfqpoint{0.052500in}{-0.013923in}}{\pgfqpoint{0.046968in}{-0.027278in}}%
\pgfpathcurveto{\pgfqpoint{0.037123in}{-0.037123in}}{\pgfqpoint{0.027278in}{-0.046968in}}{\pgfqpoint{0.013923in}{-0.052500in}}%
\pgfpathclose%
\pgfpathmoveto{\pgfqpoint{0.166667in}{-0.058333in}}%
\pgfpathcurveto{\pgfqpoint{0.182137in}{-0.058333in}}{\pgfqpoint{0.196975in}{-0.052187in}}{\pgfqpoint{0.207915in}{-0.041248in}}%
\pgfpathcurveto{\pgfqpoint{0.218854in}{-0.030309in}}{\pgfqpoint{0.225000in}{-0.015470in}}{\pgfqpoint{0.225000in}{0.000000in}}%
\pgfpathcurveto{\pgfqpoint{0.225000in}{0.015470in}}{\pgfqpoint{0.218854in}{0.030309in}}{\pgfqpoint{0.207915in}{0.041248in}}%
\pgfpathcurveto{\pgfqpoint{0.196975in}{0.052187in}}{\pgfqpoint{0.182137in}{0.058333in}}{\pgfqpoint{0.166667in}{0.058333in}}%
\pgfpathcurveto{\pgfqpoint{0.151196in}{0.058333in}}{\pgfqpoint{0.136358in}{0.052187in}}{\pgfqpoint{0.125419in}{0.041248in}}%
\pgfpathcurveto{\pgfqpoint{0.114480in}{0.030309in}}{\pgfqpoint{0.108333in}{0.015470in}}{\pgfqpoint{0.108333in}{0.000000in}}%
\pgfpathcurveto{\pgfqpoint{0.108333in}{-0.015470in}}{\pgfqpoint{0.114480in}{-0.030309in}}{\pgfqpoint{0.125419in}{-0.041248in}}%
\pgfpathcurveto{\pgfqpoint{0.136358in}{-0.052187in}}{\pgfqpoint{0.151196in}{-0.058333in}}{\pgfqpoint{0.166667in}{-0.058333in}}%
\pgfpathclose%
\pgfpathmoveto{\pgfqpoint{0.166667in}{-0.052500in}}%
\pgfpathcurveto{\pgfqpoint{0.166667in}{-0.052500in}}{\pgfqpoint{0.152744in}{-0.052500in}}{\pgfqpoint{0.139389in}{-0.046968in}}%
\pgfpathcurveto{\pgfqpoint{0.129544in}{-0.037123in}}{\pgfqpoint{0.119698in}{-0.027278in}}{\pgfqpoint{0.114167in}{-0.013923in}}%
\pgfpathcurveto{\pgfqpoint{0.114167in}{0.000000in}}{\pgfqpoint{0.114167in}{0.013923in}}{\pgfqpoint{0.119698in}{0.027278in}}%
\pgfpathcurveto{\pgfqpoint{0.129544in}{0.037123in}}{\pgfqpoint{0.139389in}{0.046968in}}{\pgfqpoint{0.152744in}{0.052500in}}%
\pgfpathcurveto{\pgfqpoint{0.166667in}{0.052500in}}{\pgfqpoint{0.180590in}{0.052500in}}{\pgfqpoint{0.193945in}{0.046968in}}%
\pgfpathcurveto{\pgfqpoint{0.203790in}{0.037123in}}{\pgfqpoint{0.213635in}{0.027278in}}{\pgfqpoint{0.219167in}{0.013923in}}%
\pgfpathcurveto{\pgfqpoint{0.219167in}{0.000000in}}{\pgfqpoint{0.219167in}{-0.013923in}}{\pgfqpoint{0.213635in}{-0.027278in}}%
\pgfpathcurveto{\pgfqpoint{0.203790in}{-0.037123in}}{\pgfqpoint{0.193945in}{-0.046968in}}{\pgfqpoint{0.180590in}{-0.052500in}}%
\pgfpathclose%
\pgfpathmoveto{\pgfqpoint{0.333333in}{-0.058333in}}%
\pgfpathcurveto{\pgfqpoint{0.348804in}{-0.058333in}}{\pgfqpoint{0.363642in}{-0.052187in}}{\pgfqpoint{0.374581in}{-0.041248in}}%
\pgfpathcurveto{\pgfqpoint{0.385520in}{-0.030309in}}{\pgfqpoint{0.391667in}{-0.015470in}}{\pgfqpoint{0.391667in}{0.000000in}}%
\pgfpathcurveto{\pgfqpoint{0.391667in}{0.015470in}}{\pgfqpoint{0.385520in}{0.030309in}}{\pgfqpoint{0.374581in}{0.041248in}}%
\pgfpathcurveto{\pgfqpoint{0.363642in}{0.052187in}}{\pgfqpoint{0.348804in}{0.058333in}}{\pgfqpoint{0.333333in}{0.058333in}}%
\pgfpathcurveto{\pgfqpoint{0.317863in}{0.058333in}}{\pgfqpoint{0.303025in}{0.052187in}}{\pgfqpoint{0.292085in}{0.041248in}}%
\pgfpathcurveto{\pgfqpoint{0.281146in}{0.030309in}}{\pgfqpoint{0.275000in}{0.015470in}}{\pgfqpoint{0.275000in}{0.000000in}}%
\pgfpathcurveto{\pgfqpoint{0.275000in}{-0.015470in}}{\pgfqpoint{0.281146in}{-0.030309in}}{\pgfqpoint{0.292085in}{-0.041248in}}%
\pgfpathcurveto{\pgfqpoint{0.303025in}{-0.052187in}}{\pgfqpoint{0.317863in}{-0.058333in}}{\pgfqpoint{0.333333in}{-0.058333in}}%
\pgfpathclose%
\pgfpathmoveto{\pgfqpoint{0.333333in}{-0.052500in}}%
\pgfpathcurveto{\pgfqpoint{0.333333in}{-0.052500in}}{\pgfqpoint{0.319410in}{-0.052500in}}{\pgfqpoint{0.306055in}{-0.046968in}}%
\pgfpathcurveto{\pgfqpoint{0.296210in}{-0.037123in}}{\pgfqpoint{0.286365in}{-0.027278in}}{\pgfqpoint{0.280833in}{-0.013923in}}%
\pgfpathcurveto{\pgfqpoint{0.280833in}{0.000000in}}{\pgfqpoint{0.280833in}{0.013923in}}{\pgfqpoint{0.286365in}{0.027278in}}%
\pgfpathcurveto{\pgfqpoint{0.296210in}{0.037123in}}{\pgfqpoint{0.306055in}{0.046968in}}{\pgfqpoint{0.319410in}{0.052500in}}%
\pgfpathcurveto{\pgfqpoint{0.333333in}{0.052500in}}{\pgfqpoint{0.347256in}{0.052500in}}{\pgfqpoint{0.360611in}{0.046968in}}%
\pgfpathcurveto{\pgfqpoint{0.370456in}{0.037123in}}{\pgfqpoint{0.380302in}{0.027278in}}{\pgfqpoint{0.385833in}{0.013923in}}%
\pgfpathcurveto{\pgfqpoint{0.385833in}{0.000000in}}{\pgfqpoint{0.385833in}{-0.013923in}}{\pgfqpoint{0.380302in}{-0.027278in}}%
\pgfpathcurveto{\pgfqpoint{0.370456in}{-0.037123in}}{\pgfqpoint{0.360611in}{-0.046968in}}{\pgfqpoint{0.347256in}{-0.052500in}}%
\pgfpathclose%
\pgfpathmoveto{\pgfqpoint{0.500000in}{-0.058333in}}%
\pgfpathcurveto{\pgfqpoint{0.515470in}{-0.058333in}}{\pgfqpoint{0.530309in}{-0.052187in}}{\pgfqpoint{0.541248in}{-0.041248in}}%
\pgfpathcurveto{\pgfqpoint{0.552187in}{-0.030309in}}{\pgfqpoint{0.558333in}{-0.015470in}}{\pgfqpoint{0.558333in}{0.000000in}}%
\pgfpathcurveto{\pgfqpoint{0.558333in}{0.015470in}}{\pgfqpoint{0.552187in}{0.030309in}}{\pgfqpoint{0.541248in}{0.041248in}}%
\pgfpathcurveto{\pgfqpoint{0.530309in}{0.052187in}}{\pgfqpoint{0.515470in}{0.058333in}}{\pgfqpoint{0.500000in}{0.058333in}}%
\pgfpathcurveto{\pgfqpoint{0.484530in}{0.058333in}}{\pgfqpoint{0.469691in}{0.052187in}}{\pgfqpoint{0.458752in}{0.041248in}}%
\pgfpathcurveto{\pgfqpoint{0.447813in}{0.030309in}}{\pgfqpoint{0.441667in}{0.015470in}}{\pgfqpoint{0.441667in}{0.000000in}}%
\pgfpathcurveto{\pgfqpoint{0.441667in}{-0.015470in}}{\pgfqpoint{0.447813in}{-0.030309in}}{\pgfqpoint{0.458752in}{-0.041248in}}%
\pgfpathcurveto{\pgfqpoint{0.469691in}{-0.052187in}}{\pgfqpoint{0.484530in}{-0.058333in}}{\pgfqpoint{0.500000in}{-0.058333in}}%
\pgfpathclose%
\pgfpathmoveto{\pgfqpoint{0.500000in}{-0.052500in}}%
\pgfpathcurveto{\pgfqpoint{0.500000in}{-0.052500in}}{\pgfqpoint{0.486077in}{-0.052500in}}{\pgfqpoint{0.472722in}{-0.046968in}}%
\pgfpathcurveto{\pgfqpoint{0.462877in}{-0.037123in}}{\pgfqpoint{0.453032in}{-0.027278in}}{\pgfqpoint{0.447500in}{-0.013923in}}%
\pgfpathcurveto{\pgfqpoint{0.447500in}{0.000000in}}{\pgfqpoint{0.447500in}{0.013923in}}{\pgfqpoint{0.453032in}{0.027278in}}%
\pgfpathcurveto{\pgfqpoint{0.462877in}{0.037123in}}{\pgfqpoint{0.472722in}{0.046968in}}{\pgfqpoint{0.486077in}{0.052500in}}%
\pgfpathcurveto{\pgfqpoint{0.500000in}{0.052500in}}{\pgfqpoint{0.513923in}{0.052500in}}{\pgfqpoint{0.527278in}{0.046968in}}%
\pgfpathcurveto{\pgfqpoint{0.537123in}{0.037123in}}{\pgfqpoint{0.546968in}{0.027278in}}{\pgfqpoint{0.552500in}{0.013923in}}%
\pgfpathcurveto{\pgfqpoint{0.552500in}{0.000000in}}{\pgfqpoint{0.552500in}{-0.013923in}}{\pgfqpoint{0.546968in}{-0.027278in}}%
\pgfpathcurveto{\pgfqpoint{0.537123in}{-0.037123in}}{\pgfqpoint{0.527278in}{-0.046968in}}{\pgfqpoint{0.513923in}{-0.052500in}}%
\pgfpathclose%
\pgfpathmoveto{\pgfqpoint{0.666667in}{-0.058333in}}%
\pgfpathcurveto{\pgfqpoint{0.682137in}{-0.058333in}}{\pgfqpoint{0.696975in}{-0.052187in}}{\pgfqpoint{0.707915in}{-0.041248in}}%
\pgfpathcurveto{\pgfqpoint{0.718854in}{-0.030309in}}{\pgfqpoint{0.725000in}{-0.015470in}}{\pgfqpoint{0.725000in}{0.000000in}}%
\pgfpathcurveto{\pgfqpoint{0.725000in}{0.015470in}}{\pgfqpoint{0.718854in}{0.030309in}}{\pgfqpoint{0.707915in}{0.041248in}}%
\pgfpathcurveto{\pgfqpoint{0.696975in}{0.052187in}}{\pgfqpoint{0.682137in}{0.058333in}}{\pgfqpoint{0.666667in}{0.058333in}}%
\pgfpathcurveto{\pgfqpoint{0.651196in}{0.058333in}}{\pgfqpoint{0.636358in}{0.052187in}}{\pgfqpoint{0.625419in}{0.041248in}}%
\pgfpathcurveto{\pgfqpoint{0.614480in}{0.030309in}}{\pgfqpoint{0.608333in}{0.015470in}}{\pgfqpoint{0.608333in}{0.000000in}}%
\pgfpathcurveto{\pgfqpoint{0.608333in}{-0.015470in}}{\pgfqpoint{0.614480in}{-0.030309in}}{\pgfqpoint{0.625419in}{-0.041248in}}%
\pgfpathcurveto{\pgfqpoint{0.636358in}{-0.052187in}}{\pgfqpoint{0.651196in}{-0.058333in}}{\pgfqpoint{0.666667in}{-0.058333in}}%
\pgfpathclose%
\pgfpathmoveto{\pgfqpoint{0.666667in}{-0.052500in}}%
\pgfpathcurveto{\pgfqpoint{0.666667in}{-0.052500in}}{\pgfqpoint{0.652744in}{-0.052500in}}{\pgfqpoint{0.639389in}{-0.046968in}}%
\pgfpathcurveto{\pgfqpoint{0.629544in}{-0.037123in}}{\pgfqpoint{0.619698in}{-0.027278in}}{\pgfqpoint{0.614167in}{-0.013923in}}%
\pgfpathcurveto{\pgfqpoint{0.614167in}{0.000000in}}{\pgfqpoint{0.614167in}{0.013923in}}{\pgfqpoint{0.619698in}{0.027278in}}%
\pgfpathcurveto{\pgfqpoint{0.629544in}{0.037123in}}{\pgfqpoint{0.639389in}{0.046968in}}{\pgfqpoint{0.652744in}{0.052500in}}%
\pgfpathcurveto{\pgfqpoint{0.666667in}{0.052500in}}{\pgfqpoint{0.680590in}{0.052500in}}{\pgfqpoint{0.693945in}{0.046968in}}%
\pgfpathcurveto{\pgfqpoint{0.703790in}{0.037123in}}{\pgfqpoint{0.713635in}{0.027278in}}{\pgfqpoint{0.719167in}{0.013923in}}%
\pgfpathcurveto{\pgfqpoint{0.719167in}{0.000000in}}{\pgfqpoint{0.719167in}{-0.013923in}}{\pgfqpoint{0.713635in}{-0.027278in}}%
\pgfpathcurveto{\pgfqpoint{0.703790in}{-0.037123in}}{\pgfqpoint{0.693945in}{-0.046968in}}{\pgfqpoint{0.680590in}{-0.052500in}}%
\pgfpathclose%
\pgfpathmoveto{\pgfqpoint{0.833333in}{-0.058333in}}%
\pgfpathcurveto{\pgfqpoint{0.848804in}{-0.058333in}}{\pgfqpoint{0.863642in}{-0.052187in}}{\pgfqpoint{0.874581in}{-0.041248in}}%
\pgfpathcurveto{\pgfqpoint{0.885520in}{-0.030309in}}{\pgfqpoint{0.891667in}{-0.015470in}}{\pgfqpoint{0.891667in}{0.000000in}}%
\pgfpathcurveto{\pgfqpoint{0.891667in}{0.015470in}}{\pgfqpoint{0.885520in}{0.030309in}}{\pgfqpoint{0.874581in}{0.041248in}}%
\pgfpathcurveto{\pgfqpoint{0.863642in}{0.052187in}}{\pgfqpoint{0.848804in}{0.058333in}}{\pgfqpoint{0.833333in}{0.058333in}}%
\pgfpathcurveto{\pgfqpoint{0.817863in}{0.058333in}}{\pgfqpoint{0.803025in}{0.052187in}}{\pgfqpoint{0.792085in}{0.041248in}}%
\pgfpathcurveto{\pgfqpoint{0.781146in}{0.030309in}}{\pgfqpoint{0.775000in}{0.015470in}}{\pgfqpoint{0.775000in}{0.000000in}}%
\pgfpathcurveto{\pgfqpoint{0.775000in}{-0.015470in}}{\pgfqpoint{0.781146in}{-0.030309in}}{\pgfqpoint{0.792085in}{-0.041248in}}%
\pgfpathcurveto{\pgfqpoint{0.803025in}{-0.052187in}}{\pgfqpoint{0.817863in}{-0.058333in}}{\pgfqpoint{0.833333in}{-0.058333in}}%
\pgfpathclose%
\pgfpathmoveto{\pgfqpoint{0.833333in}{-0.052500in}}%
\pgfpathcurveto{\pgfqpoint{0.833333in}{-0.052500in}}{\pgfqpoint{0.819410in}{-0.052500in}}{\pgfqpoint{0.806055in}{-0.046968in}}%
\pgfpathcurveto{\pgfqpoint{0.796210in}{-0.037123in}}{\pgfqpoint{0.786365in}{-0.027278in}}{\pgfqpoint{0.780833in}{-0.013923in}}%
\pgfpathcurveto{\pgfqpoint{0.780833in}{0.000000in}}{\pgfqpoint{0.780833in}{0.013923in}}{\pgfqpoint{0.786365in}{0.027278in}}%
\pgfpathcurveto{\pgfqpoint{0.796210in}{0.037123in}}{\pgfqpoint{0.806055in}{0.046968in}}{\pgfqpoint{0.819410in}{0.052500in}}%
\pgfpathcurveto{\pgfqpoint{0.833333in}{0.052500in}}{\pgfqpoint{0.847256in}{0.052500in}}{\pgfqpoint{0.860611in}{0.046968in}}%
\pgfpathcurveto{\pgfqpoint{0.870456in}{0.037123in}}{\pgfqpoint{0.880302in}{0.027278in}}{\pgfqpoint{0.885833in}{0.013923in}}%
\pgfpathcurveto{\pgfqpoint{0.885833in}{0.000000in}}{\pgfqpoint{0.885833in}{-0.013923in}}{\pgfqpoint{0.880302in}{-0.027278in}}%
\pgfpathcurveto{\pgfqpoint{0.870456in}{-0.037123in}}{\pgfqpoint{0.860611in}{-0.046968in}}{\pgfqpoint{0.847256in}{-0.052500in}}%
\pgfpathclose%
\pgfpathmoveto{\pgfqpoint{1.000000in}{-0.058333in}}%
\pgfpathcurveto{\pgfqpoint{1.015470in}{-0.058333in}}{\pgfqpoint{1.030309in}{-0.052187in}}{\pgfqpoint{1.041248in}{-0.041248in}}%
\pgfpathcurveto{\pgfqpoint{1.052187in}{-0.030309in}}{\pgfqpoint{1.058333in}{-0.015470in}}{\pgfqpoint{1.058333in}{0.000000in}}%
\pgfpathcurveto{\pgfqpoint{1.058333in}{0.015470in}}{\pgfqpoint{1.052187in}{0.030309in}}{\pgfqpoint{1.041248in}{0.041248in}}%
\pgfpathcurveto{\pgfqpoint{1.030309in}{0.052187in}}{\pgfqpoint{1.015470in}{0.058333in}}{\pgfqpoint{1.000000in}{0.058333in}}%
\pgfpathcurveto{\pgfqpoint{0.984530in}{0.058333in}}{\pgfqpoint{0.969691in}{0.052187in}}{\pgfqpoint{0.958752in}{0.041248in}}%
\pgfpathcurveto{\pgfqpoint{0.947813in}{0.030309in}}{\pgfqpoint{0.941667in}{0.015470in}}{\pgfqpoint{0.941667in}{0.000000in}}%
\pgfpathcurveto{\pgfqpoint{0.941667in}{-0.015470in}}{\pgfqpoint{0.947813in}{-0.030309in}}{\pgfqpoint{0.958752in}{-0.041248in}}%
\pgfpathcurveto{\pgfqpoint{0.969691in}{-0.052187in}}{\pgfqpoint{0.984530in}{-0.058333in}}{\pgfqpoint{1.000000in}{-0.058333in}}%
\pgfpathclose%
\pgfpathmoveto{\pgfqpoint{1.000000in}{-0.052500in}}%
\pgfpathcurveto{\pgfqpoint{1.000000in}{-0.052500in}}{\pgfqpoint{0.986077in}{-0.052500in}}{\pgfqpoint{0.972722in}{-0.046968in}}%
\pgfpathcurveto{\pgfqpoint{0.962877in}{-0.037123in}}{\pgfqpoint{0.953032in}{-0.027278in}}{\pgfqpoint{0.947500in}{-0.013923in}}%
\pgfpathcurveto{\pgfqpoint{0.947500in}{0.000000in}}{\pgfqpoint{0.947500in}{0.013923in}}{\pgfqpoint{0.953032in}{0.027278in}}%
\pgfpathcurveto{\pgfqpoint{0.962877in}{0.037123in}}{\pgfqpoint{0.972722in}{0.046968in}}{\pgfqpoint{0.986077in}{0.052500in}}%
\pgfpathcurveto{\pgfqpoint{1.000000in}{0.052500in}}{\pgfqpoint{1.013923in}{0.052500in}}{\pgfqpoint{1.027278in}{0.046968in}}%
\pgfpathcurveto{\pgfqpoint{1.037123in}{0.037123in}}{\pgfqpoint{1.046968in}{0.027278in}}{\pgfqpoint{1.052500in}{0.013923in}}%
\pgfpathcurveto{\pgfqpoint{1.052500in}{0.000000in}}{\pgfqpoint{1.052500in}{-0.013923in}}{\pgfqpoint{1.046968in}{-0.027278in}}%
\pgfpathcurveto{\pgfqpoint{1.037123in}{-0.037123in}}{\pgfqpoint{1.027278in}{-0.046968in}}{\pgfqpoint{1.013923in}{-0.052500in}}%
\pgfpathclose%
\pgfpathmoveto{\pgfqpoint{0.083333in}{0.108333in}}%
\pgfpathcurveto{\pgfqpoint{0.098804in}{0.108333in}}{\pgfqpoint{0.113642in}{0.114480in}}{\pgfqpoint{0.124581in}{0.125419in}}%
\pgfpathcurveto{\pgfqpoint{0.135520in}{0.136358in}}{\pgfqpoint{0.141667in}{0.151196in}}{\pgfqpoint{0.141667in}{0.166667in}}%
\pgfpathcurveto{\pgfqpoint{0.141667in}{0.182137in}}{\pgfqpoint{0.135520in}{0.196975in}}{\pgfqpoint{0.124581in}{0.207915in}}%
\pgfpathcurveto{\pgfqpoint{0.113642in}{0.218854in}}{\pgfqpoint{0.098804in}{0.225000in}}{\pgfqpoint{0.083333in}{0.225000in}}%
\pgfpathcurveto{\pgfqpoint{0.067863in}{0.225000in}}{\pgfqpoint{0.053025in}{0.218854in}}{\pgfqpoint{0.042085in}{0.207915in}}%
\pgfpathcurveto{\pgfqpoint{0.031146in}{0.196975in}}{\pgfqpoint{0.025000in}{0.182137in}}{\pgfqpoint{0.025000in}{0.166667in}}%
\pgfpathcurveto{\pgfqpoint{0.025000in}{0.151196in}}{\pgfqpoint{0.031146in}{0.136358in}}{\pgfqpoint{0.042085in}{0.125419in}}%
\pgfpathcurveto{\pgfqpoint{0.053025in}{0.114480in}}{\pgfqpoint{0.067863in}{0.108333in}}{\pgfqpoint{0.083333in}{0.108333in}}%
\pgfpathclose%
\pgfpathmoveto{\pgfqpoint{0.083333in}{0.114167in}}%
\pgfpathcurveto{\pgfqpoint{0.083333in}{0.114167in}}{\pgfqpoint{0.069410in}{0.114167in}}{\pgfqpoint{0.056055in}{0.119698in}}%
\pgfpathcurveto{\pgfqpoint{0.046210in}{0.129544in}}{\pgfqpoint{0.036365in}{0.139389in}}{\pgfqpoint{0.030833in}{0.152744in}}%
\pgfpathcurveto{\pgfqpoint{0.030833in}{0.166667in}}{\pgfqpoint{0.030833in}{0.180590in}}{\pgfqpoint{0.036365in}{0.193945in}}%
\pgfpathcurveto{\pgfqpoint{0.046210in}{0.203790in}}{\pgfqpoint{0.056055in}{0.213635in}}{\pgfqpoint{0.069410in}{0.219167in}}%
\pgfpathcurveto{\pgfqpoint{0.083333in}{0.219167in}}{\pgfqpoint{0.097256in}{0.219167in}}{\pgfqpoint{0.110611in}{0.213635in}}%
\pgfpathcurveto{\pgfqpoint{0.120456in}{0.203790in}}{\pgfqpoint{0.130302in}{0.193945in}}{\pgfqpoint{0.135833in}{0.180590in}}%
\pgfpathcurveto{\pgfqpoint{0.135833in}{0.166667in}}{\pgfqpoint{0.135833in}{0.152744in}}{\pgfqpoint{0.130302in}{0.139389in}}%
\pgfpathcurveto{\pgfqpoint{0.120456in}{0.129544in}}{\pgfqpoint{0.110611in}{0.119698in}}{\pgfqpoint{0.097256in}{0.114167in}}%
\pgfpathclose%
\pgfpathmoveto{\pgfqpoint{0.250000in}{0.108333in}}%
\pgfpathcurveto{\pgfqpoint{0.265470in}{0.108333in}}{\pgfqpoint{0.280309in}{0.114480in}}{\pgfqpoint{0.291248in}{0.125419in}}%
\pgfpathcurveto{\pgfqpoint{0.302187in}{0.136358in}}{\pgfqpoint{0.308333in}{0.151196in}}{\pgfqpoint{0.308333in}{0.166667in}}%
\pgfpathcurveto{\pgfqpoint{0.308333in}{0.182137in}}{\pgfqpoint{0.302187in}{0.196975in}}{\pgfqpoint{0.291248in}{0.207915in}}%
\pgfpathcurveto{\pgfqpoint{0.280309in}{0.218854in}}{\pgfqpoint{0.265470in}{0.225000in}}{\pgfqpoint{0.250000in}{0.225000in}}%
\pgfpathcurveto{\pgfqpoint{0.234530in}{0.225000in}}{\pgfqpoint{0.219691in}{0.218854in}}{\pgfqpoint{0.208752in}{0.207915in}}%
\pgfpathcurveto{\pgfqpoint{0.197813in}{0.196975in}}{\pgfqpoint{0.191667in}{0.182137in}}{\pgfqpoint{0.191667in}{0.166667in}}%
\pgfpathcurveto{\pgfqpoint{0.191667in}{0.151196in}}{\pgfqpoint{0.197813in}{0.136358in}}{\pgfqpoint{0.208752in}{0.125419in}}%
\pgfpathcurveto{\pgfqpoint{0.219691in}{0.114480in}}{\pgfqpoint{0.234530in}{0.108333in}}{\pgfqpoint{0.250000in}{0.108333in}}%
\pgfpathclose%
\pgfpathmoveto{\pgfqpoint{0.250000in}{0.114167in}}%
\pgfpathcurveto{\pgfqpoint{0.250000in}{0.114167in}}{\pgfqpoint{0.236077in}{0.114167in}}{\pgfqpoint{0.222722in}{0.119698in}}%
\pgfpathcurveto{\pgfqpoint{0.212877in}{0.129544in}}{\pgfqpoint{0.203032in}{0.139389in}}{\pgfqpoint{0.197500in}{0.152744in}}%
\pgfpathcurveto{\pgfqpoint{0.197500in}{0.166667in}}{\pgfqpoint{0.197500in}{0.180590in}}{\pgfqpoint{0.203032in}{0.193945in}}%
\pgfpathcurveto{\pgfqpoint{0.212877in}{0.203790in}}{\pgfqpoint{0.222722in}{0.213635in}}{\pgfqpoint{0.236077in}{0.219167in}}%
\pgfpathcurveto{\pgfqpoint{0.250000in}{0.219167in}}{\pgfqpoint{0.263923in}{0.219167in}}{\pgfqpoint{0.277278in}{0.213635in}}%
\pgfpathcurveto{\pgfqpoint{0.287123in}{0.203790in}}{\pgfqpoint{0.296968in}{0.193945in}}{\pgfqpoint{0.302500in}{0.180590in}}%
\pgfpathcurveto{\pgfqpoint{0.302500in}{0.166667in}}{\pgfqpoint{0.302500in}{0.152744in}}{\pgfqpoint{0.296968in}{0.139389in}}%
\pgfpathcurveto{\pgfqpoint{0.287123in}{0.129544in}}{\pgfqpoint{0.277278in}{0.119698in}}{\pgfqpoint{0.263923in}{0.114167in}}%
\pgfpathclose%
\pgfpathmoveto{\pgfqpoint{0.416667in}{0.108333in}}%
\pgfpathcurveto{\pgfqpoint{0.432137in}{0.108333in}}{\pgfqpoint{0.446975in}{0.114480in}}{\pgfqpoint{0.457915in}{0.125419in}}%
\pgfpathcurveto{\pgfqpoint{0.468854in}{0.136358in}}{\pgfqpoint{0.475000in}{0.151196in}}{\pgfqpoint{0.475000in}{0.166667in}}%
\pgfpathcurveto{\pgfqpoint{0.475000in}{0.182137in}}{\pgfqpoint{0.468854in}{0.196975in}}{\pgfqpoint{0.457915in}{0.207915in}}%
\pgfpathcurveto{\pgfqpoint{0.446975in}{0.218854in}}{\pgfqpoint{0.432137in}{0.225000in}}{\pgfqpoint{0.416667in}{0.225000in}}%
\pgfpathcurveto{\pgfqpoint{0.401196in}{0.225000in}}{\pgfqpoint{0.386358in}{0.218854in}}{\pgfqpoint{0.375419in}{0.207915in}}%
\pgfpathcurveto{\pgfqpoint{0.364480in}{0.196975in}}{\pgfqpoint{0.358333in}{0.182137in}}{\pgfqpoint{0.358333in}{0.166667in}}%
\pgfpathcurveto{\pgfqpoint{0.358333in}{0.151196in}}{\pgfqpoint{0.364480in}{0.136358in}}{\pgfqpoint{0.375419in}{0.125419in}}%
\pgfpathcurveto{\pgfqpoint{0.386358in}{0.114480in}}{\pgfqpoint{0.401196in}{0.108333in}}{\pgfqpoint{0.416667in}{0.108333in}}%
\pgfpathclose%
\pgfpathmoveto{\pgfqpoint{0.416667in}{0.114167in}}%
\pgfpathcurveto{\pgfqpoint{0.416667in}{0.114167in}}{\pgfqpoint{0.402744in}{0.114167in}}{\pgfqpoint{0.389389in}{0.119698in}}%
\pgfpathcurveto{\pgfqpoint{0.379544in}{0.129544in}}{\pgfqpoint{0.369698in}{0.139389in}}{\pgfqpoint{0.364167in}{0.152744in}}%
\pgfpathcurveto{\pgfqpoint{0.364167in}{0.166667in}}{\pgfqpoint{0.364167in}{0.180590in}}{\pgfqpoint{0.369698in}{0.193945in}}%
\pgfpathcurveto{\pgfqpoint{0.379544in}{0.203790in}}{\pgfqpoint{0.389389in}{0.213635in}}{\pgfqpoint{0.402744in}{0.219167in}}%
\pgfpathcurveto{\pgfqpoint{0.416667in}{0.219167in}}{\pgfqpoint{0.430590in}{0.219167in}}{\pgfqpoint{0.443945in}{0.213635in}}%
\pgfpathcurveto{\pgfqpoint{0.453790in}{0.203790in}}{\pgfqpoint{0.463635in}{0.193945in}}{\pgfqpoint{0.469167in}{0.180590in}}%
\pgfpathcurveto{\pgfqpoint{0.469167in}{0.166667in}}{\pgfqpoint{0.469167in}{0.152744in}}{\pgfqpoint{0.463635in}{0.139389in}}%
\pgfpathcurveto{\pgfqpoint{0.453790in}{0.129544in}}{\pgfqpoint{0.443945in}{0.119698in}}{\pgfqpoint{0.430590in}{0.114167in}}%
\pgfpathclose%
\pgfpathmoveto{\pgfqpoint{0.583333in}{0.108333in}}%
\pgfpathcurveto{\pgfqpoint{0.598804in}{0.108333in}}{\pgfqpoint{0.613642in}{0.114480in}}{\pgfqpoint{0.624581in}{0.125419in}}%
\pgfpathcurveto{\pgfqpoint{0.635520in}{0.136358in}}{\pgfqpoint{0.641667in}{0.151196in}}{\pgfqpoint{0.641667in}{0.166667in}}%
\pgfpathcurveto{\pgfqpoint{0.641667in}{0.182137in}}{\pgfqpoint{0.635520in}{0.196975in}}{\pgfqpoint{0.624581in}{0.207915in}}%
\pgfpathcurveto{\pgfqpoint{0.613642in}{0.218854in}}{\pgfqpoint{0.598804in}{0.225000in}}{\pgfqpoint{0.583333in}{0.225000in}}%
\pgfpathcurveto{\pgfqpoint{0.567863in}{0.225000in}}{\pgfqpoint{0.553025in}{0.218854in}}{\pgfqpoint{0.542085in}{0.207915in}}%
\pgfpathcurveto{\pgfqpoint{0.531146in}{0.196975in}}{\pgfqpoint{0.525000in}{0.182137in}}{\pgfqpoint{0.525000in}{0.166667in}}%
\pgfpathcurveto{\pgfqpoint{0.525000in}{0.151196in}}{\pgfqpoint{0.531146in}{0.136358in}}{\pgfqpoint{0.542085in}{0.125419in}}%
\pgfpathcurveto{\pgfqpoint{0.553025in}{0.114480in}}{\pgfqpoint{0.567863in}{0.108333in}}{\pgfqpoint{0.583333in}{0.108333in}}%
\pgfpathclose%
\pgfpathmoveto{\pgfqpoint{0.583333in}{0.114167in}}%
\pgfpathcurveto{\pgfqpoint{0.583333in}{0.114167in}}{\pgfqpoint{0.569410in}{0.114167in}}{\pgfqpoint{0.556055in}{0.119698in}}%
\pgfpathcurveto{\pgfqpoint{0.546210in}{0.129544in}}{\pgfqpoint{0.536365in}{0.139389in}}{\pgfqpoint{0.530833in}{0.152744in}}%
\pgfpathcurveto{\pgfqpoint{0.530833in}{0.166667in}}{\pgfqpoint{0.530833in}{0.180590in}}{\pgfqpoint{0.536365in}{0.193945in}}%
\pgfpathcurveto{\pgfqpoint{0.546210in}{0.203790in}}{\pgfqpoint{0.556055in}{0.213635in}}{\pgfqpoint{0.569410in}{0.219167in}}%
\pgfpathcurveto{\pgfqpoint{0.583333in}{0.219167in}}{\pgfqpoint{0.597256in}{0.219167in}}{\pgfqpoint{0.610611in}{0.213635in}}%
\pgfpathcurveto{\pgfqpoint{0.620456in}{0.203790in}}{\pgfqpoint{0.630302in}{0.193945in}}{\pgfqpoint{0.635833in}{0.180590in}}%
\pgfpathcurveto{\pgfqpoint{0.635833in}{0.166667in}}{\pgfqpoint{0.635833in}{0.152744in}}{\pgfqpoint{0.630302in}{0.139389in}}%
\pgfpathcurveto{\pgfqpoint{0.620456in}{0.129544in}}{\pgfqpoint{0.610611in}{0.119698in}}{\pgfqpoint{0.597256in}{0.114167in}}%
\pgfpathclose%
\pgfpathmoveto{\pgfqpoint{0.750000in}{0.108333in}}%
\pgfpathcurveto{\pgfqpoint{0.765470in}{0.108333in}}{\pgfqpoint{0.780309in}{0.114480in}}{\pgfqpoint{0.791248in}{0.125419in}}%
\pgfpathcurveto{\pgfqpoint{0.802187in}{0.136358in}}{\pgfqpoint{0.808333in}{0.151196in}}{\pgfqpoint{0.808333in}{0.166667in}}%
\pgfpathcurveto{\pgfqpoint{0.808333in}{0.182137in}}{\pgfqpoint{0.802187in}{0.196975in}}{\pgfqpoint{0.791248in}{0.207915in}}%
\pgfpathcurveto{\pgfqpoint{0.780309in}{0.218854in}}{\pgfqpoint{0.765470in}{0.225000in}}{\pgfqpoint{0.750000in}{0.225000in}}%
\pgfpathcurveto{\pgfqpoint{0.734530in}{0.225000in}}{\pgfqpoint{0.719691in}{0.218854in}}{\pgfqpoint{0.708752in}{0.207915in}}%
\pgfpathcurveto{\pgfqpoint{0.697813in}{0.196975in}}{\pgfqpoint{0.691667in}{0.182137in}}{\pgfqpoint{0.691667in}{0.166667in}}%
\pgfpathcurveto{\pgfqpoint{0.691667in}{0.151196in}}{\pgfqpoint{0.697813in}{0.136358in}}{\pgfqpoint{0.708752in}{0.125419in}}%
\pgfpathcurveto{\pgfqpoint{0.719691in}{0.114480in}}{\pgfqpoint{0.734530in}{0.108333in}}{\pgfqpoint{0.750000in}{0.108333in}}%
\pgfpathclose%
\pgfpathmoveto{\pgfqpoint{0.750000in}{0.114167in}}%
\pgfpathcurveto{\pgfqpoint{0.750000in}{0.114167in}}{\pgfqpoint{0.736077in}{0.114167in}}{\pgfqpoint{0.722722in}{0.119698in}}%
\pgfpathcurveto{\pgfqpoint{0.712877in}{0.129544in}}{\pgfqpoint{0.703032in}{0.139389in}}{\pgfqpoint{0.697500in}{0.152744in}}%
\pgfpathcurveto{\pgfqpoint{0.697500in}{0.166667in}}{\pgfqpoint{0.697500in}{0.180590in}}{\pgfqpoint{0.703032in}{0.193945in}}%
\pgfpathcurveto{\pgfqpoint{0.712877in}{0.203790in}}{\pgfqpoint{0.722722in}{0.213635in}}{\pgfqpoint{0.736077in}{0.219167in}}%
\pgfpathcurveto{\pgfqpoint{0.750000in}{0.219167in}}{\pgfqpoint{0.763923in}{0.219167in}}{\pgfqpoint{0.777278in}{0.213635in}}%
\pgfpathcurveto{\pgfqpoint{0.787123in}{0.203790in}}{\pgfqpoint{0.796968in}{0.193945in}}{\pgfqpoint{0.802500in}{0.180590in}}%
\pgfpathcurveto{\pgfqpoint{0.802500in}{0.166667in}}{\pgfqpoint{0.802500in}{0.152744in}}{\pgfqpoint{0.796968in}{0.139389in}}%
\pgfpathcurveto{\pgfqpoint{0.787123in}{0.129544in}}{\pgfqpoint{0.777278in}{0.119698in}}{\pgfqpoint{0.763923in}{0.114167in}}%
\pgfpathclose%
\pgfpathmoveto{\pgfqpoint{0.916667in}{0.108333in}}%
\pgfpathcurveto{\pgfqpoint{0.932137in}{0.108333in}}{\pgfqpoint{0.946975in}{0.114480in}}{\pgfqpoint{0.957915in}{0.125419in}}%
\pgfpathcurveto{\pgfqpoint{0.968854in}{0.136358in}}{\pgfqpoint{0.975000in}{0.151196in}}{\pgfqpoint{0.975000in}{0.166667in}}%
\pgfpathcurveto{\pgfqpoint{0.975000in}{0.182137in}}{\pgfqpoint{0.968854in}{0.196975in}}{\pgfqpoint{0.957915in}{0.207915in}}%
\pgfpathcurveto{\pgfqpoint{0.946975in}{0.218854in}}{\pgfqpoint{0.932137in}{0.225000in}}{\pgfqpoint{0.916667in}{0.225000in}}%
\pgfpathcurveto{\pgfqpoint{0.901196in}{0.225000in}}{\pgfqpoint{0.886358in}{0.218854in}}{\pgfqpoint{0.875419in}{0.207915in}}%
\pgfpathcurveto{\pgfqpoint{0.864480in}{0.196975in}}{\pgfqpoint{0.858333in}{0.182137in}}{\pgfqpoint{0.858333in}{0.166667in}}%
\pgfpathcurveto{\pgfqpoint{0.858333in}{0.151196in}}{\pgfqpoint{0.864480in}{0.136358in}}{\pgfqpoint{0.875419in}{0.125419in}}%
\pgfpathcurveto{\pgfqpoint{0.886358in}{0.114480in}}{\pgfqpoint{0.901196in}{0.108333in}}{\pgfqpoint{0.916667in}{0.108333in}}%
\pgfpathclose%
\pgfpathmoveto{\pgfqpoint{0.916667in}{0.114167in}}%
\pgfpathcurveto{\pgfqpoint{0.916667in}{0.114167in}}{\pgfqpoint{0.902744in}{0.114167in}}{\pgfqpoint{0.889389in}{0.119698in}}%
\pgfpathcurveto{\pgfqpoint{0.879544in}{0.129544in}}{\pgfqpoint{0.869698in}{0.139389in}}{\pgfqpoint{0.864167in}{0.152744in}}%
\pgfpathcurveto{\pgfqpoint{0.864167in}{0.166667in}}{\pgfqpoint{0.864167in}{0.180590in}}{\pgfqpoint{0.869698in}{0.193945in}}%
\pgfpathcurveto{\pgfqpoint{0.879544in}{0.203790in}}{\pgfqpoint{0.889389in}{0.213635in}}{\pgfqpoint{0.902744in}{0.219167in}}%
\pgfpathcurveto{\pgfqpoint{0.916667in}{0.219167in}}{\pgfqpoint{0.930590in}{0.219167in}}{\pgfqpoint{0.943945in}{0.213635in}}%
\pgfpathcurveto{\pgfqpoint{0.953790in}{0.203790in}}{\pgfqpoint{0.963635in}{0.193945in}}{\pgfqpoint{0.969167in}{0.180590in}}%
\pgfpathcurveto{\pgfqpoint{0.969167in}{0.166667in}}{\pgfqpoint{0.969167in}{0.152744in}}{\pgfqpoint{0.963635in}{0.139389in}}%
\pgfpathcurveto{\pgfqpoint{0.953790in}{0.129544in}}{\pgfqpoint{0.943945in}{0.119698in}}{\pgfqpoint{0.930590in}{0.114167in}}%
\pgfpathclose%
\pgfpathmoveto{\pgfqpoint{0.000000in}{0.275000in}}%
\pgfpathcurveto{\pgfqpoint{0.015470in}{0.275000in}}{\pgfqpoint{0.030309in}{0.281146in}}{\pgfqpoint{0.041248in}{0.292085in}}%
\pgfpathcurveto{\pgfqpoint{0.052187in}{0.303025in}}{\pgfqpoint{0.058333in}{0.317863in}}{\pgfqpoint{0.058333in}{0.333333in}}%
\pgfpathcurveto{\pgfqpoint{0.058333in}{0.348804in}}{\pgfqpoint{0.052187in}{0.363642in}}{\pgfqpoint{0.041248in}{0.374581in}}%
\pgfpathcurveto{\pgfqpoint{0.030309in}{0.385520in}}{\pgfqpoint{0.015470in}{0.391667in}}{\pgfqpoint{0.000000in}{0.391667in}}%
\pgfpathcurveto{\pgfqpoint{-0.015470in}{0.391667in}}{\pgfqpoint{-0.030309in}{0.385520in}}{\pgfqpoint{-0.041248in}{0.374581in}}%
\pgfpathcurveto{\pgfqpoint{-0.052187in}{0.363642in}}{\pgfqpoint{-0.058333in}{0.348804in}}{\pgfqpoint{-0.058333in}{0.333333in}}%
\pgfpathcurveto{\pgfqpoint{-0.058333in}{0.317863in}}{\pgfqpoint{-0.052187in}{0.303025in}}{\pgfqpoint{-0.041248in}{0.292085in}}%
\pgfpathcurveto{\pgfqpoint{-0.030309in}{0.281146in}}{\pgfqpoint{-0.015470in}{0.275000in}}{\pgfqpoint{0.000000in}{0.275000in}}%
\pgfpathclose%
\pgfpathmoveto{\pgfqpoint{0.000000in}{0.280833in}}%
\pgfpathcurveto{\pgfqpoint{0.000000in}{0.280833in}}{\pgfqpoint{-0.013923in}{0.280833in}}{\pgfqpoint{-0.027278in}{0.286365in}}%
\pgfpathcurveto{\pgfqpoint{-0.037123in}{0.296210in}}{\pgfqpoint{-0.046968in}{0.306055in}}{\pgfqpoint{-0.052500in}{0.319410in}}%
\pgfpathcurveto{\pgfqpoint{-0.052500in}{0.333333in}}{\pgfqpoint{-0.052500in}{0.347256in}}{\pgfqpoint{-0.046968in}{0.360611in}}%
\pgfpathcurveto{\pgfqpoint{-0.037123in}{0.370456in}}{\pgfqpoint{-0.027278in}{0.380302in}}{\pgfqpoint{-0.013923in}{0.385833in}}%
\pgfpathcurveto{\pgfqpoint{0.000000in}{0.385833in}}{\pgfqpoint{0.013923in}{0.385833in}}{\pgfqpoint{0.027278in}{0.380302in}}%
\pgfpathcurveto{\pgfqpoint{0.037123in}{0.370456in}}{\pgfqpoint{0.046968in}{0.360611in}}{\pgfqpoint{0.052500in}{0.347256in}}%
\pgfpathcurveto{\pgfqpoint{0.052500in}{0.333333in}}{\pgfqpoint{0.052500in}{0.319410in}}{\pgfqpoint{0.046968in}{0.306055in}}%
\pgfpathcurveto{\pgfqpoint{0.037123in}{0.296210in}}{\pgfqpoint{0.027278in}{0.286365in}}{\pgfqpoint{0.013923in}{0.280833in}}%
\pgfpathclose%
\pgfpathmoveto{\pgfqpoint{0.166667in}{0.275000in}}%
\pgfpathcurveto{\pgfqpoint{0.182137in}{0.275000in}}{\pgfqpoint{0.196975in}{0.281146in}}{\pgfqpoint{0.207915in}{0.292085in}}%
\pgfpathcurveto{\pgfqpoint{0.218854in}{0.303025in}}{\pgfqpoint{0.225000in}{0.317863in}}{\pgfqpoint{0.225000in}{0.333333in}}%
\pgfpathcurveto{\pgfqpoint{0.225000in}{0.348804in}}{\pgfqpoint{0.218854in}{0.363642in}}{\pgfqpoint{0.207915in}{0.374581in}}%
\pgfpathcurveto{\pgfqpoint{0.196975in}{0.385520in}}{\pgfqpoint{0.182137in}{0.391667in}}{\pgfqpoint{0.166667in}{0.391667in}}%
\pgfpathcurveto{\pgfqpoint{0.151196in}{0.391667in}}{\pgfqpoint{0.136358in}{0.385520in}}{\pgfqpoint{0.125419in}{0.374581in}}%
\pgfpathcurveto{\pgfqpoint{0.114480in}{0.363642in}}{\pgfqpoint{0.108333in}{0.348804in}}{\pgfqpoint{0.108333in}{0.333333in}}%
\pgfpathcurveto{\pgfqpoint{0.108333in}{0.317863in}}{\pgfqpoint{0.114480in}{0.303025in}}{\pgfqpoint{0.125419in}{0.292085in}}%
\pgfpathcurveto{\pgfqpoint{0.136358in}{0.281146in}}{\pgfqpoint{0.151196in}{0.275000in}}{\pgfqpoint{0.166667in}{0.275000in}}%
\pgfpathclose%
\pgfpathmoveto{\pgfqpoint{0.166667in}{0.280833in}}%
\pgfpathcurveto{\pgfqpoint{0.166667in}{0.280833in}}{\pgfqpoint{0.152744in}{0.280833in}}{\pgfqpoint{0.139389in}{0.286365in}}%
\pgfpathcurveto{\pgfqpoint{0.129544in}{0.296210in}}{\pgfqpoint{0.119698in}{0.306055in}}{\pgfqpoint{0.114167in}{0.319410in}}%
\pgfpathcurveto{\pgfqpoint{0.114167in}{0.333333in}}{\pgfqpoint{0.114167in}{0.347256in}}{\pgfqpoint{0.119698in}{0.360611in}}%
\pgfpathcurveto{\pgfqpoint{0.129544in}{0.370456in}}{\pgfqpoint{0.139389in}{0.380302in}}{\pgfqpoint{0.152744in}{0.385833in}}%
\pgfpathcurveto{\pgfqpoint{0.166667in}{0.385833in}}{\pgfqpoint{0.180590in}{0.385833in}}{\pgfqpoint{0.193945in}{0.380302in}}%
\pgfpathcurveto{\pgfqpoint{0.203790in}{0.370456in}}{\pgfqpoint{0.213635in}{0.360611in}}{\pgfqpoint{0.219167in}{0.347256in}}%
\pgfpathcurveto{\pgfqpoint{0.219167in}{0.333333in}}{\pgfqpoint{0.219167in}{0.319410in}}{\pgfqpoint{0.213635in}{0.306055in}}%
\pgfpathcurveto{\pgfqpoint{0.203790in}{0.296210in}}{\pgfqpoint{0.193945in}{0.286365in}}{\pgfqpoint{0.180590in}{0.280833in}}%
\pgfpathclose%
\pgfpathmoveto{\pgfqpoint{0.333333in}{0.275000in}}%
\pgfpathcurveto{\pgfqpoint{0.348804in}{0.275000in}}{\pgfqpoint{0.363642in}{0.281146in}}{\pgfqpoint{0.374581in}{0.292085in}}%
\pgfpathcurveto{\pgfqpoint{0.385520in}{0.303025in}}{\pgfqpoint{0.391667in}{0.317863in}}{\pgfqpoint{0.391667in}{0.333333in}}%
\pgfpathcurveto{\pgfqpoint{0.391667in}{0.348804in}}{\pgfqpoint{0.385520in}{0.363642in}}{\pgfqpoint{0.374581in}{0.374581in}}%
\pgfpathcurveto{\pgfqpoint{0.363642in}{0.385520in}}{\pgfqpoint{0.348804in}{0.391667in}}{\pgfqpoint{0.333333in}{0.391667in}}%
\pgfpathcurveto{\pgfqpoint{0.317863in}{0.391667in}}{\pgfqpoint{0.303025in}{0.385520in}}{\pgfqpoint{0.292085in}{0.374581in}}%
\pgfpathcurveto{\pgfqpoint{0.281146in}{0.363642in}}{\pgfqpoint{0.275000in}{0.348804in}}{\pgfqpoint{0.275000in}{0.333333in}}%
\pgfpathcurveto{\pgfqpoint{0.275000in}{0.317863in}}{\pgfqpoint{0.281146in}{0.303025in}}{\pgfqpoint{0.292085in}{0.292085in}}%
\pgfpathcurveto{\pgfqpoint{0.303025in}{0.281146in}}{\pgfqpoint{0.317863in}{0.275000in}}{\pgfqpoint{0.333333in}{0.275000in}}%
\pgfpathclose%
\pgfpathmoveto{\pgfqpoint{0.333333in}{0.280833in}}%
\pgfpathcurveto{\pgfqpoint{0.333333in}{0.280833in}}{\pgfqpoint{0.319410in}{0.280833in}}{\pgfqpoint{0.306055in}{0.286365in}}%
\pgfpathcurveto{\pgfqpoint{0.296210in}{0.296210in}}{\pgfqpoint{0.286365in}{0.306055in}}{\pgfqpoint{0.280833in}{0.319410in}}%
\pgfpathcurveto{\pgfqpoint{0.280833in}{0.333333in}}{\pgfqpoint{0.280833in}{0.347256in}}{\pgfqpoint{0.286365in}{0.360611in}}%
\pgfpathcurveto{\pgfqpoint{0.296210in}{0.370456in}}{\pgfqpoint{0.306055in}{0.380302in}}{\pgfqpoint{0.319410in}{0.385833in}}%
\pgfpathcurveto{\pgfqpoint{0.333333in}{0.385833in}}{\pgfqpoint{0.347256in}{0.385833in}}{\pgfqpoint{0.360611in}{0.380302in}}%
\pgfpathcurveto{\pgfqpoint{0.370456in}{0.370456in}}{\pgfqpoint{0.380302in}{0.360611in}}{\pgfqpoint{0.385833in}{0.347256in}}%
\pgfpathcurveto{\pgfqpoint{0.385833in}{0.333333in}}{\pgfqpoint{0.385833in}{0.319410in}}{\pgfqpoint{0.380302in}{0.306055in}}%
\pgfpathcurveto{\pgfqpoint{0.370456in}{0.296210in}}{\pgfqpoint{0.360611in}{0.286365in}}{\pgfqpoint{0.347256in}{0.280833in}}%
\pgfpathclose%
\pgfpathmoveto{\pgfqpoint{0.500000in}{0.275000in}}%
\pgfpathcurveto{\pgfqpoint{0.515470in}{0.275000in}}{\pgfqpoint{0.530309in}{0.281146in}}{\pgfqpoint{0.541248in}{0.292085in}}%
\pgfpathcurveto{\pgfqpoint{0.552187in}{0.303025in}}{\pgfqpoint{0.558333in}{0.317863in}}{\pgfqpoint{0.558333in}{0.333333in}}%
\pgfpathcurveto{\pgfqpoint{0.558333in}{0.348804in}}{\pgfqpoint{0.552187in}{0.363642in}}{\pgfqpoint{0.541248in}{0.374581in}}%
\pgfpathcurveto{\pgfqpoint{0.530309in}{0.385520in}}{\pgfqpoint{0.515470in}{0.391667in}}{\pgfqpoint{0.500000in}{0.391667in}}%
\pgfpathcurveto{\pgfqpoint{0.484530in}{0.391667in}}{\pgfqpoint{0.469691in}{0.385520in}}{\pgfqpoint{0.458752in}{0.374581in}}%
\pgfpathcurveto{\pgfqpoint{0.447813in}{0.363642in}}{\pgfqpoint{0.441667in}{0.348804in}}{\pgfqpoint{0.441667in}{0.333333in}}%
\pgfpathcurveto{\pgfqpoint{0.441667in}{0.317863in}}{\pgfqpoint{0.447813in}{0.303025in}}{\pgfqpoint{0.458752in}{0.292085in}}%
\pgfpathcurveto{\pgfqpoint{0.469691in}{0.281146in}}{\pgfqpoint{0.484530in}{0.275000in}}{\pgfqpoint{0.500000in}{0.275000in}}%
\pgfpathclose%
\pgfpathmoveto{\pgfqpoint{0.500000in}{0.280833in}}%
\pgfpathcurveto{\pgfqpoint{0.500000in}{0.280833in}}{\pgfqpoint{0.486077in}{0.280833in}}{\pgfqpoint{0.472722in}{0.286365in}}%
\pgfpathcurveto{\pgfqpoint{0.462877in}{0.296210in}}{\pgfqpoint{0.453032in}{0.306055in}}{\pgfqpoint{0.447500in}{0.319410in}}%
\pgfpathcurveto{\pgfqpoint{0.447500in}{0.333333in}}{\pgfqpoint{0.447500in}{0.347256in}}{\pgfqpoint{0.453032in}{0.360611in}}%
\pgfpathcurveto{\pgfqpoint{0.462877in}{0.370456in}}{\pgfqpoint{0.472722in}{0.380302in}}{\pgfqpoint{0.486077in}{0.385833in}}%
\pgfpathcurveto{\pgfqpoint{0.500000in}{0.385833in}}{\pgfqpoint{0.513923in}{0.385833in}}{\pgfqpoint{0.527278in}{0.380302in}}%
\pgfpathcurveto{\pgfqpoint{0.537123in}{0.370456in}}{\pgfqpoint{0.546968in}{0.360611in}}{\pgfqpoint{0.552500in}{0.347256in}}%
\pgfpathcurveto{\pgfqpoint{0.552500in}{0.333333in}}{\pgfqpoint{0.552500in}{0.319410in}}{\pgfqpoint{0.546968in}{0.306055in}}%
\pgfpathcurveto{\pgfqpoint{0.537123in}{0.296210in}}{\pgfqpoint{0.527278in}{0.286365in}}{\pgfqpoint{0.513923in}{0.280833in}}%
\pgfpathclose%
\pgfpathmoveto{\pgfqpoint{0.666667in}{0.275000in}}%
\pgfpathcurveto{\pgfqpoint{0.682137in}{0.275000in}}{\pgfqpoint{0.696975in}{0.281146in}}{\pgfqpoint{0.707915in}{0.292085in}}%
\pgfpathcurveto{\pgfqpoint{0.718854in}{0.303025in}}{\pgfqpoint{0.725000in}{0.317863in}}{\pgfqpoint{0.725000in}{0.333333in}}%
\pgfpathcurveto{\pgfqpoint{0.725000in}{0.348804in}}{\pgfqpoint{0.718854in}{0.363642in}}{\pgfqpoint{0.707915in}{0.374581in}}%
\pgfpathcurveto{\pgfqpoint{0.696975in}{0.385520in}}{\pgfqpoint{0.682137in}{0.391667in}}{\pgfqpoint{0.666667in}{0.391667in}}%
\pgfpathcurveto{\pgfqpoint{0.651196in}{0.391667in}}{\pgfqpoint{0.636358in}{0.385520in}}{\pgfqpoint{0.625419in}{0.374581in}}%
\pgfpathcurveto{\pgfqpoint{0.614480in}{0.363642in}}{\pgfqpoint{0.608333in}{0.348804in}}{\pgfqpoint{0.608333in}{0.333333in}}%
\pgfpathcurveto{\pgfqpoint{0.608333in}{0.317863in}}{\pgfqpoint{0.614480in}{0.303025in}}{\pgfqpoint{0.625419in}{0.292085in}}%
\pgfpathcurveto{\pgfqpoint{0.636358in}{0.281146in}}{\pgfqpoint{0.651196in}{0.275000in}}{\pgfqpoint{0.666667in}{0.275000in}}%
\pgfpathclose%
\pgfpathmoveto{\pgfqpoint{0.666667in}{0.280833in}}%
\pgfpathcurveto{\pgfqpoint{0.666667in}{0.280833in}}{\pgfqpoint{0.652744in}{0.280833in}}{\pgfqpoint{0.639389in}{0.286365in}}%
\pgfpathcurveto{\pgfqpoint{0.629544in}{0.296210in}}{\pgfqpoint{0.619698in}{0.306055in}}{\pgfqpoint{0.614167in}{0.319410in}}%
\pgfpathcurveto{\pgfqpoint{0.614167in}{0.333333in}}{\pgfqpoint{0.614167in}{0.347256in}}{\pgfqpoint{0.619698in}{0.360611in}}%
\pgfpathcurveto{\pgfqpoint{0.629544in}{0.370456in}}{\pgfqpoint{0.639389in}{0.380302in}}{\pgfqpoint{0.652744in}{0.385833in}}%
\pgfpathcurveto{\pgfqpoint{0.666667in}{0.385833in}}{\pgfqpoint{0.680590in}{0.385833in}}{\pgfqpoint{0.693945in}{0.380302in}}%
\pgfpathcurveto{\pgfqpoint{0.703790in}{0.370456in}}{\pgfqpoint{0.713635in}{0.360611in}}{\pgfqpoint{0.719167in}{0.347256in}}%
\pgfpathcurveto{\pgfqpoint{0.719167in}{0.333333in}}{\pgfqpoint{0.719167in}{0.319410in}}{\pgfqpoint{0.713635in}{0.306055in}}%
\pgfpathcurveto{\pgfqpoint{0.703790in}{0.296210in}}{\pgfqpoint{0.693945in}{0.286365in}}{\pgfqpoint{0.680590in}{0.280833in}}%
\pgfpathclose%
\pgfpathmoveto{\pgfqpoint{0.833333in}{0.275000in}}%
\pgfpathcurveto{\pgfqpoint{0.848804in}{0.275000in}}{\pgfqpoint{0.863642in}{0.281146in}}{\pgfqpoint{0.874581in}{0.292085in}}%
\pgfpathcurveto{\pgfqpoint{0.885520in}{0.303025in}}{\pgfqpoint{0.891667in}{0.317863in}}{\pgfqpoint{0.891667in}{0.333333in}}%
\pgfpathcurveto{\pgfqpoint{0.891667in}{0.348804in}}{\pgfqpoint{0.885520in}{0.363642in}}{\pgfqpoint{0.874581in}{0.374581in}}%
\pgfpathcurveto{\pgfqpoint{0.863642in}{0.385520in}}{\pgfqpoint{0.848804in}{0.391667in}}{\pgfqpoint{0.833333in}{0.391667in}}%
\pgfpathcurveto{\pgfqpoint{0.817863in}{0.391667in}}{\pgfqpoint{0.803025in}{0.385520in}}{\pgfqpoint{0.792085in}{0.374581in}}%
\pgfpathcurveto{\pgfqpoint{0.781146in}{0.363642in}}{\pgfqpoint{0.775000in}{0.348804in}}{\pgfqpoint{0.775000in}{0.333333in}}%
\pgfpathcurveto{\pgfqpoint{0.775000in}{0.317863in}}{\pgfqpoint{0.781146in}{0.303025in}}{\pgfqpoint{0.792085in}{0.292085in}}%
\pgfpathcurveto{\pgfqpoint{0.803025in}{0.281146in}}{\pgfqpoint{0.817863in}{0.275000in}}{\pgfqpoint{0.833333in}{0.275000in}}%
\pgfpathclose%
\pgfpathmoveto{\pgfqpoint{0.833333in}{0.280833in}}%
\pgfpathcurveto{\pgfqpoint{0.833333in}{0.280833in}}{\pgfqpoint{0.819410in}{0.280833in}}{\pgfqpoint{0.806055in}{0.286365in}}%
\pgfpathcurveto{\pgfqpoint{0.796210in}{0.296210in}}{\pgfqpoint{0.786365in}{0.306055in}}{\pgfqpoint{0.780833in}{0.319410in}}%
\pgfpathcurveto{\pgfqpoint{0.780833in}{0.333333in}}{\pgfqpoint{0.780833in}{0.347256in}}{\pgfqpoint{0.786365in}{0.360611in}}%
\pgfpathcurveto{\pgfqpoint{0.796210in}{0.370456in}}{\pgfqpoint{0.806055in}{0.380302in}}{\pgfqpoint{0.819410in}{0.385833in}}%
\pgfpathcurveto{\pgfqpoint{0.833333in}{0.385833in}}{\pgfqpoint{0.847256in}{0.385833in}}{\pgfqpoint{0.860611in}{0.380302in}}%
\pgfpathcurveto{\pgfqpoint{0.870456in}{0.370456in}}{\pgfqpoint{0.880302in}{0.360611in}}{\pgfqpoint{0.885833in}{0.347256in}}%
\pgfpathcurveto{\pgfqpoint{0.885833in}{0.333333in}}{\pgfqpoint{0.885833in}{0.319410in}}{\pgfqpoint{0.880302in}{0.306055in}}%
\pgfpathcurveto{\pgfqpoint{0.870456in}{0.296210in}}{\pgfqpoint{0.860611in}{0.286365in}}{\pgfqpoint{0.847256in}{0.280833in}}%
\pgfpathclose%
\pgfpathmoveto{\pgfqpoint{1.000000in}{0.275000in}}%
\pgfpathcurveto{\pgfqpoint{1.015470in}{0.275000in}}{\pgfqpoint{1.030309in}{0.281146in}}{\pgfqpoint{1.041248in}{0.292085in}}%
\pgfpathcurveto{\pgfqpoint{1.052187in}{0.303025in}}{\pgfqpoint{1.058333in}{0.317863in}}{\pgfqpoint{1.058333in}{0.333333in}}%
\pgfpathcurveto{\pgfqpoint{1.058333in}{0.348804in}}{\pgfqpoint{1.052187in}{0.363642in}}{\pgfqpoint{1.041248in}{0.374581in}}%
\pgfpathcurveto{\pgfqpoint{1.030309in}{0.385520in}}{\pgfqpoint{1.015470in}{0.391667in}}{\pgfqpoint{1.000000in}{0.391667in}}%
\pgfpathcurveto{\pgfqpoint{0.984530in}{0.391667in}}{\pgfqpoint{0.969691in}{0.385520in}}{\pgfqpoint{0.958752in}{0.374581in}}%
\pgfpathcurveto{\pgfqpoint{0.947813in}{0.363642in}}{\pgfqpoint{0.941667in}{0.348804in}}{\pgfqpoint{0.941667in}{0.333333in}}%
\pgfpathcurveto{\pgfqpoint{0.941667in}{0.317863in}}{\pgfqpoint{0.947813in}{0.303025in}}{\pgfqpoint{0.958752in}{0.292085in}}%
\pgfpathcurveto{\pgfqpoint{0.969691in}{0.281146in}}{\pgfqpoint{0.984530in}{0.275000in}}{\pgfqpoint{1.000000in}{0.275000in}}%
\pgfpathclose%
\pgfpathmoveto{\pgfqpoint{1.000000in}{0.280833in}}%
\pgfpathcurveto{\pgfqpoint{1.000000in}{0.280833in}}{\pgfqpoint{0.986077in}{0.280833in}}{\pgfqpoint{0.972722in}{0.286365in}}%
\pgfpathcurveto{\pgfqpoint{0.962877in}{0.296210in}}{\pgfqpoint{0.953032in}{0.306055in}}{\pgfqpoint{0.947500in}{0.319410in}}%
\pgfpathcurveto{\pgfqpoint{0.947500in}{0.333333in}}{\pgfqpoint{0.947500in}{0.347256in}}{\pgfqpoint{0.953032in}{0.360611in}}%
\pgfpathcurveto{\pgfqpoint{0.962877in}{0.370456in}}{\pgfqpoint{0.972722in}{0.380302in}}{\pgfqpoint{0.986077in}{0.385833in}}%
\pgfpathcurveto{\pgfqpoint{1.000000in}{0.385833in}}{\pgfqpoint{1.013923in}{0.385833in}}{\pgfqpoint{1.027278in}{0.380302in}}%
\pgfpathcurveto{\pgfqpoint{1.037123in}{0.370456in}}{\pgfqpoint{1.046968in}{0.360611in}}{\pgfqpoint{1.052500in}{0.347256in}}%
\pgfpathcurveto{\pgfqpoint{1.052500in}{0.333333in}}{\pgfqpoint{1.052500in}{0.319410in}}{\pgfqpoint{1.046968in}{0.306055in}}%
\pgfpathcurveto{\pgfqpoint{1.037123in}{0.296210in}}{\pgfqpoint{1.027278in}{0.286365in}}{\pgfqpoint{1.013923in}{0.280833in}}%
\pgfpathclose%
\pgfpathmoveto{\pgfqpoint{0.083333in}{0.441667in}}%
\pgfpathcurveto{\pgfqpoint{0.098804in}{0.441667in}}{\pgfqpoint{0.113642in}{0.447813in}}{\pgfqpoint{0.124581in}{0.458752in}}%
\pgfpathcurveto{\pgfqpoint{0.135520in}{0.469691in}}{\pgfqpoint{0.141667in}{0.484530in}}{\pgfqpoint{0.141667in}{0.500000in}}%
\pgfpathcurveto{\pgfqpoint{0.141667in}{0.515470in}}{\pgfqpoint{0.135520in}{0.530309in}}{\pgfqpoint{0.124581in}{0.541248in}}%
\pgfpathcurveto{\pgfqpoint{0.113642in}{0.552187in}}{\pgfqpoint{0.098804in}{0.558333in}}{\pgfqpoint{0.083333in}{0.558333in}}%
\pgfpathcurveto{\pgfqpoint{0.067863in}{0.558333in}}{\pgfqpoint{0.053025in}{0.552187in}}{\pgfqpoint{0.042085in}{0.541248in}}%
\pgfpathcurveto{\pgfqpoint{0.031146in}{0.530309in}}{\pgfqpoint{0.025000in}{0.515470in}}{\pgfqpoint{0.025000in}{0.500000in}}%
\pgfpathcurveto{\pgfqpoint{0.025000in}{0.484530in}}{\pgfqpoint{0.031146in}{0.469691in}}{\pgfqpoint{0.042085in}{0.458752in}}%
\pgfpathcurveto{\pgfqpoint{0.053025in}{0.447813in}}{\pgfqpoint{0.067863in}{0.441667in}}{\pgfqpoint{0.083333in}{0.441667in}}%
\pgfpathclose%
\pgfpathmoveto{\pgfqpoint{0.083333in}{0.447500in}}%
\pgfpathcurveto{\pgfqpoint{0.083333in}{0.447500in}}{\pgfqpoint{0.069410in}{0.447500in}}{\pgfqpoint{0.056055in}{0.453032in}}%
\pgfpathcurveto{\pgfqpoint{0.046210in}{0.462877in}}{\pgfqpoint{0.036365in}{0.472722in}}{\pgfqpoint{0.030833in}{0.486077in}}%
\pgfpathcurveto{\pgfqpoint{0.030833in}{0.500000in}}{\pgfqpoint{0.030833in}{0.513923in}}{\pgfqpoint{0.036365in}{0.527278in}}%
\pgfpathcurveto{\pgfqpoint{0.046210in}{0.537123in}}{\pgfqpoint{0.056055in}{0.546968in}}{\pgfqpoint{0.069410in}{0.552500in}}%
\pgfpathcurveto{\pgfqpoint{0.083333in}{0.552500in}}{\pgfqpoint{0.097256in}{0.552500in}}{\pgfqpoint{0.110611in}{0.546968in}}%
\pgfpathcurveto{\pgfqpoint{0.120456in}{0.537123in}}{\pgfqpoint{0.130302in}{0.527278in}}{\pgfqpoint{0.135833in}{0.513923in}}%
\pgfpathcurveto{\pgfqpoint{0.135833in}{0.500000in}}{\pgfqpoint{0.135833in}{0.486077in}}{\pgfqpoint{0.130302in}{0.472722in}}%
\pgfpathcurveto{\pgfqpoint{0.120456in}{0.462877in}}{\pgfqpoint{0.110611in}{0.453032in}}{\pgfqpoint{0.097256in}{0.447500in}}%
\pgfpathclose%
\pgfpathmoveto{\pgfqpoint{0.250000in}{0.441667in}}%
\pgfpathcurveto{\pgfqpoint{0.265470in}{0.441667in}}{\pgfqpoint{0.280309in}{0.447813in}}{\pgfqpoint{0.291248in}{0.458752in}}%
\pgfpathcurveto{\pgfqpoint{0.302187in}{0.469691in}}{\pgfqpoint{0.308333in}{0.484530in}}{\pgfqpoint{0.308333in}{0.500000in}}%
\pgfpathcurveto{\pgfqpoint{0.308333in}{0.515470in}}{\pgfqpoint{0.302187in}{0.530309in}}{\pgfqpoint{0.291248in}{0.541248in}}%
\pgfpathcurveto{\pgfqpoint{0.280309in}{0.552187in}}{\pgfqpoint{0.265470in}{0.558333in}}{\pgfqpoint{0.250000in}{0.558333in}}%
\pgfpathcurveto{\pgfqpoint{0.234530in}{0.558333in}}{\pgfqpoint{0.219691in}{0.552187in}}{\pgfqpoint{0.208752in}{0.541248in}}%
\pgfpathcurveto{\pgfqpoint{0.197813in}{0.530309in}}{\pgfqpoint{0.191667in}{0.515470in}}{\pgfqpoint{0.191667in}{0.500000in}}%
\pgfpathcurveto{\pgfqpoint{0.191667in}{0.484530in}}{\pgfqpoint{0.197813in}{0.469691in}}{\pgfqpoint{0.208752in}{0.458752in}}%
\pgfpathcurveto{\pgfqpoint{0.219691in}{0.447813in}}{\pgfqpoint{0.234530in}{0.441667in}}{\pgfqpoint{0.250000in}{0.441667in}}%
\pgfpathclose%
\pgfpathmoveto{\pgfqpoint{0.250000in}{0.447500in}}%
\pgfpathcurveto{\pgfqpoint{0.250000in}{0.447500in}}{\pgfqpoint{0.236077in}{0.447500in}}{\pgfqpoint{0.222722in}{0.453032in}}%
\pgfpathcurveto{\pgfqpoint{0.212877in}{0.462877in}}{\pgfqpoint{0.203032in}{0.472722in}}{\pgfqpoint{0.197500in}{0.486077in}}%
\pgfpathcurveto{\pgfqpoint{0.197500in}{0.500000in}}{\pgfqpoint{0.197500in}{0.513923in}}{\pgfqpoint{0.203032in}{0.527278in}}%
\pgfpathcurveto{\pgfqpoint{0.212877in}{0.537123in}}{\pgfqpoint{0.222722in}{0.546968in}}{\pgfqpoint{0.236077in}{0.552500in}}%
\pgfpathcurveto{\pgfqpoint{0.250000in}{0.552500in}}{\pgfqpoint{0.263923in}{0.552500in}}{\pgfqpoint{0.277278in}{0.546968in}}%
\pgfpathcurveto{\pgfqpoint{0.287123in}{0.537123in}}{\pgfqpoint{0.296968in}{0.527278in}}{\pgfqpoint{0.302500in}{0.513923in}}%
\pgfpathcurveto{\pgfqpoint{0.302500in}{0.500000in}}{\pgfqpoint{0.302500in}{0.486077in}}{\pgfqpoint{0.296968in}{0.472722in}}%
\pgfpathcurveto{\pgfqpoint{0.287123in}{0.462877in}}{\pgfqpoint{0.277278in}{0.453032in}}{\pgfqpoint{0.263923in}{0.447500in}}%
\pgfpathclose%
\pgfpathmoveto{\pgfqpoint{0.416667in}{0.441667in}}%
\pgfpathcurveto{\pgfqpoint{0.432137in}{0.441667in}}{\pgfqpoint{0.446975in}{0.447813in}}{\pgfqpoint{0.457915in}{0.458752in}}%
\pgfpathcurveto{\pgfqpoint{0.468854in}{0.469691in}}{\pgfqpoint{0.475000in}{0.484530in}}{\pgfqpoint{0.475000in}{0.500000in}}%
\pgfpathcurveto{\pgfqpoint{0.475000in}{0.515470in}}{\pgfqpoint{0.468854in}{0.530309in}}{\pgfqpoint{0.457915in}{0.541248in}}%
\pgfpathcurveto{\pgfqpoint{0.446975in}{0.552187in}}{\pgfqpoint{0.432137in}{0.558333in}}{\pgfqpoint{0.416667in}{0.558333in}}%
\pgfpathcurveto{\pgfqpoint{0.401196in}{0.558333in}}{\pgfqpoint{0.386358in}{0.552187in}}{\pgfqpoint{0.375419in}{0.541248in}}%
\pgfpathcurveto{\pgfqpoint{0.364480in}{0.530309in}}{\pgfqpoint{0.358333in}{0.515470in}}{\pgfqpoint{0.358333in}{0.500000in}}%
\pgfpathcurveto{\pgfqpoint{0.358333in}{0.484530in}}{\pgfqpoint{0.364480in}{0.469691in}}{\pgfqpoint{0.375419in}{0.458752in}}%
\pgfpathcurveto{\pgfqpoint{0.386358in}{0.447813in}}{\pgfqpoint{0.401196in}{0.441667in}}{\pgfqpoint{0.416667in}{0.441667in}}%
\pgfpathclose%
\pgfpathmoveto{\pgfqpoint{0.416667in}{0.447500in}}%
\pgfpathcurveto{\pgfqpoint{0.416667in}{0.447500in}}{\pgfqpoint{0.402744in}{0.447500in}}{\pgfqpoint{0.389389in}{0.453032in}}%
\pgfpathcurveto{\pgfqpoint{0.379544in}{0.462877in}}{\pgfqpoint{0.369698in}{0.472722in}}{\pgfqpoint{0.364167in}{0.486077in}}%
\pgfpathcurveto{\pgfqpoint{0.364167in}{0.500000in}}{\pgfqpoint{0.364167in}{0.513923in}}{\pgfqpoint{0.369698in}{0.527278in}}%
\pgfpathcurveto{\pgfqpoint{0.379544in}{0.537123in}}{\pgfqpoint{0.389389in}{0.546968in}}{\pgfqpoint{0.402744in}{0.552500in}}%
\pgfpathcurveto{\pgfqpoint{0.416667in}{0.552500in}}{\pgfqpoint{0.430590in}{0.552500in}}{\pgfqpoint{0.443945in}{0.546968in}}%
\pgfpathcurveto{\pgfqpoint{0.453790in}{0.537123in}}{\pgfqpoint{0.463635in}{0.527278in}}{\pgfqpoint{0.469167in}{0.513923in}}%
\pgfpathcurveto{\pgfqpoint{0.469167in}{0.500000in}}{\pgfqpoint{0.469167in}{0.486077in}}{\pgfqpoint{0.463635in}{0.472722in}}%
\pgfpathcurveto{\pgfqpoint{0.453790in}{0.462877in}}{\pgfqpoint{0.443945in}{0.453032in}}{\pgfqpoint{0.430590in}{0.447500in}}%
\pgfpathclose%
\pgfpathmoveto{\pgfqpoint{0.583333in}{0.441667in}}%
\pgfpathcurveto{\pgfqpoint{0.598804in}{0.441667in}}{\pgfqpoint{0.613642in}{0.447813in}}{\pgfqpoint{0.624581in}{0.458752in}}%
\pgfpathcurveto{\pgfqpoint{0.635520in}{0.469691in}}{\pgfqpoint{0.641667in}{0.484530in}}{\pgfqpoint{0.641667in}{0.500000in}}%
\pgfpathcurveto{\pgfqpoint{0.641667in}{0.515470in}}{\pgfqpoint{0.635520in}{0.530309in}}{\pgfqpoint{0.624581in}{0.541248in}}%
\pgfpathcurveto{\pgfqpoint{0.613642in}{0.552187in}}{\pgfqpoint{0.598804in}{0.558333in}}{\pgfqpoint{0.583333in}{0.558333in}}%
\pgfpathcurveto{\pgfqpoint{0.567863in}{0.558333in}}{\pgfqpoint{0.553025in}{0.552187in}}{\pgfqpoint{0.542085in}{0.541248in}}%
\pgfpathcurveto{\pgfqpoint{0.531146in}{0.530309in}}{\pgfqpoint{0.525000in}{0.515470in}}{\pgfqpoint{0.525000in}{0.500000in}}%
\pgfpathcurveto{\pgfqpoint{0.525000in}{0.484530in}}{\pgfqpoint{0.531146in}{0.469691in}}{\pgfqpoint{0.542085in}{0.458752in}}%
\pgfpathcurveto{\pgfqpoint{0.553025in}{0.447813in}}{\pgfqpoint{0.567863in}{0.441667in}}{\pgfqpoint{0.583333in}{0.441667in}}%
\pgfpathclose%
\pgfpathmoveto{\pgfqpoint{0.583333in}{0.447500in}}%
\pgfpathcurveto{\pgfqpoint{0.583333in}{0.447500in}}{\pgfqpoint{0.569410in}{0.447500in}}{\pgfqpoint{0.556055in}{0.453032in}}%
\pgfpathcurveto{\pgfqpoint{0.546210in}{0.462877in}}{\pgfqpoint{0.536365in}{0.472722in}}{\pgfqpoint{0.530833in}{0.486077in}}%
\pgfpathcurveto{\pgfqpoint{0.530833in}{0.500000in}}{\pgfqpoint{0.530833in}{0.513923in}}{\pgfqpoint{0.536365in}{0.527278in}}%
\pgfpathcurveto{\pgfqpoint{0.546210in}{0.537123in}}{\pgfqpoint{0.556055in}{0.546968in}}{\pgfqpoint{0.569410in}{0.552500in}}%
\pgfpathcurveto{\pgfqpoint{0.583333in}{0.552500in}}{\pgfqpoint{0.597256in}{0.552500in}}{\pgfqpoint{0.610611in}{0.546968in}}%
\pgfpathcurveto{\pgfqpoint{0.620456in}{0.537123in}}{\pgfqpoint{0.630302in}{0.527278in}}{\pgfqpoint{0.635833in}{0.513923in}}%
\pgfpathcurveto{\pgfqpoint{0.635833in}{0.500000in}}{\pgfqpoint{0.635833in}{0.486077in}}{\pgfqpoint{0.630302in}{0.472722in}}%
\pgfpathcurveto{\pgfqpoint{0.620456in}{0.462877in}}{\pgfqpoint{0.610611in}{0.453032in}}{\pgfqpoint{0.597256in}{0.447500in}}%
\pgfpathclose%
\pgfpathmoveto{\pgfqpoint{0.750000in}{0.441667in}}%
\pgfpathcurveto{\pgfqpoint{0.765470in}{0.441667in}}{\pgfqpoint{0.780309in}{0.447813in}}{\pgfqpoint{0.791248in}{0.458752in}}%
\pgfpathcurveto{\pgfqpoint{0.802187in}{0.469691in}}{\pgfqpoint{0.808333in}{0.484530in}}{\pgfqpoint{0.808333in}{0.500000in}}%
\pgfpathcurveto{\pgfqpoint{0.808333in}{0.515470in}}{\pgfqpoint{0.802187in}{0.530309in}}{\pgfqpoint{0.791248in}{0.541248in}}%
\pgfpathcurveto{\pgfqpoint{0.780309in}{0.552187in}}{\pgfqpoint{0.765470in}{0.558333in}}{\pgfqpoint{0.750000in}{0.558333in}}%
\pgfpathcurveto{\pgfqpoint{0.734530in}{0.558333in}}{\pgfqpoint{0.719691in}{0.552187in}}{\pgfqpoint{0.708752in}{0.541248in}}%
\pgfpathcurveto{\pgfqpoint{0.697813in}{0.530309in}}{\pgfqpoint{0.691667in}{0.515470in}}{\pgfqpoint{0.691667in}{0.500000in}}%
\pgfpathcurveto{\pgfqpoint{0.691667in}{0.484530in}}{\pgfqpoint{0.697813in}{0.469691in}}{\pgfqpoint{0.708752in}{0.458752in}}%
\pgfpathcurveto{\pgfqpoint{0.719691in}{0.447813in}}{\pgfqpoint{0.734530in}{0.441667in}}{\pgfqpoint{0.750000in}{0.441667in}}%
\pgfpathclose%
\pgfpathmoveto{\pgfqpoint{0.750000in}{0.447500in}}%
\pgfpathcurveto{\pgfqpoint{0.750000in}{0.447500in}}{\pgfqpoint{0.736077in}{0.447500in}}{\pgfqpoint{0.722722in}{0.453032in}}%
\pgfpathcurveto{\pgfqpoint{0.712877in}{0.462877in}}{\pgfqpoint{0.703032in}{0.472722in}}{\pgfqpoint{0.697500in}{0.486077in}}%
\pgfpathcurveto{\pgfqpoint{0.697500in}{0.500000in}}{\pgfqpoint{0.697500in}{0.513923in}}{\pgfqpoint{0.703032in}{0.527278in}}%
\pgfpathcurveto{\pgfqpoint{0.712877in}{0.537123in}}{\pgfqpoint{0.722722in}{0.546968in}}{\pgfqpoint{0.736077in}{0.552500in}}%
\pgfpathcurveto{\pgfqpoint{0.750000in}{0.552500in}}{\pgfqpoint{0.763923in}{0.552500in}}{\pgfqpoint{0.777278in}{0.546968in}}%
\pgfpathcurveto{\pgfqpoint{0.787123in}{0.537123in}}{\pgfqpoint{0.796968in}{0.527278in}}{\pgfqpoint{0.802500in}{0.513923in}}%
\pgfpathcurveto{\pgfqpoint{0.802500in}{0.500000in}}{\pgfqpoint{0.802500in}{0.486077in}}{\pgfqpoint{0.796968in}{0.472722in}}%
\pgfpathcurveto{\pgfqpoint{0.787123in}{0.462877in}}{\pgfqpoint{0.777278in}{0.453032in}}{\pgfqpoint{0.763923in}{0.447500in}}%
\pgfpathclose%
\pgfpathmoveto{\pgfqpoint{0.916667in}{0.441667in}}%
\pgfpathcurveto{\pgfqpoint{0.932137in}{0.441667in}}{\pgfqpoint{0.946975in}{0.447813in}}{\pgfqpoint{0.957915in}{0.458752in}}%
\pgfpathcurveto{\pgfqpoint{0.968854in}{0.469691in}}{\pgfqpoint{0.975000in}{0.484530in}}{\pgfqpoint{0.975000in}{0.500000in}}%
\pgfpathcurveto{\pgfqpoint{0.975000in}{0.515470in}}{\pgfqpoint{0.968854in}{0.530309in}}{\pgfqpoint{0.957915in}{0.541248in}}%
\pgfpathcurveto{\pgfqpoint{0.946975in}{0.552187in}}{\pgfqpoint{0.932137in}{0.558333in}}{\pgfqpoint{0.916667in}{0.558333in}}%
\pgfpathcurveto{\pgfqpoint{0.901196in}{0.558333in}}{\pgfqpoint{0.886358in}{0.552187in}}{\pgfqpoint{0.875419in}{0.541248in}}%
\pgfpathcurveto{\pgfqpoint{0.864480in}{0.530309in}}{\pgfqpoint{0.858333in}{0.515470in}}{\pgfqpoint{0.858333in}{0.500000in}}%
\pgfpathcurveto{\pgfqpoint{0.858333in}{0.484530in}}{\pgfqpoint{0.864480in}{0.469691in}}{\pgfqpoint{0.875419in}{0.458752in}}%
\pgfpathcurveto{\pgfqpoint{0.886358in}{0.447813in}}{\pgfqpoint{0.901196in}{0.441667in}}{\pgfqpoint{0.916667in}{0.441667in}}%
\pgfpathclose%
\pgfpathmoveto{\pgfqpoint{0.916667in}{0.447500in}}%
\pgfpathcurveto{\pgfqpoint{0.916667in}{0.447500in}}{\pgfqpoint{0.902744in}{0.447500in}}{\pgfqpoint{0.889389in}{0.453032in}}%
\pgfpathcurveto{\pgfqpoint{0.879544in}{0.462877in}}{\pgfqpoint{0.869698in}{0.472722in}}{\pgfqpoint{0.864167in}{0.486077in}}%
\pgfpathcurveto{\pgfqpoint{0.864167in}{0.500000in}}{\pgfqpoint{0.864167in}{0.513923in}}{\pgfqpoint{0.869698in}{0.527278in}}%
\pgfpathcurveto{\pgfqpoint{0.879544in}{0.537123in}}{\pgfqpoint{0.889389in}{0.546968in}}{\pgfqpoint{0.902744in}{0.552500in}}%
\pgfpathcurveto{\pgfqpoint{0.916667in}{0.552500in}}{\pgfqpoint{0.930590in}{0.552500in}}{\pgfqpoint{0.943945in}{0.546968in}}%
\pgfpathcurveto{\pgfqpoint{0.953790in}{0.537123in}}{\pgfqpoint{0.963635in}{0.527278in}}{\pgfqpoint{0.969167in}{0.513923in}}%
\pgfpathcurveto{\pgfqpoint{0.969167in}{0.500000in}}{\pgfqpoint{0.969167in}{0.486077in}}{\pgfqpoint{0.963635in}{0.472722in}}%
\pgfpathcurveto{\pgfqpoint{0.953790in}{0.462877in}}{\pgfqpoint{0.943945in}{0.453032in}}{\pgfqpoint{0.930590in}{0.447500in}}%
\pgfpathclose%
\pgfpathmoveto{\pgfqpoint{0.000000in}{0.608333in}}%
\pgfpathcurveto{\pgfqpoint{0.015470in}{0.608333in}}{\pgfqpoint{0.030309in}{0.614480in}}{\pgfqpoint{0.041248in}{0.625419in}}%
\pgfpathcurveto{\pgfqpoint{0.052187in}{0.636358in}}{\pgfqpoint{0.058333in}{0.651196in}}{\pgfqpoint{0.058333in}{0.666667in}}%
\pgfpathcurveto{\pgfqpoint{0.058333in}{0.682137in}}{\pgfqpoint{0.052187in}{0.696975in}}{\pgfqpoint{0.041248in}{0.707915in}}%
\pgfpathcurveto{\pgfqpoint{0.030309in}{0.718854in}}{\pgfqpoint{0.015470in}{0.725000in}}{\pgfqpoint{0.000000in}{0.725000in}}%
\pgfpathcurveto{\pgfqpoint{-0.015470in}{0.725000in}}{\pgfqpoint{-0.030309in}{0.718854in}}{\pgfqpoint{-0.041248in}{0.707915in}}%
\pgfpathcurveto{\pgfqpoint{-0.052187in}{0.696975in}}{\pgfqpoint{-0.058333in}{0.682137in}}{\pgfqpoint{-0.058333in}{0.666667in}}%
\pgfpathcurveto{\pgfqpoint{-0.058333in}{0.651196in}}{\pgfqpoint{-0.052187in}{0.636358in}}{\pgfqpoint{-0.041248in}{0.625419in}}%
\pgfpathcurveto{\pgfqpoint{-0.030309in}{0.614480in}}{\pgfqpoint{-0.015470in}{0.608333in}}{\pgfqpoint{0.000000in}{0.608333in}}%
\pgfpathclose%
\pgfpathmoveto{\pgfqpoint{0.000000in}{0.614167in}}%
\pgfpathcurveto{\pgfqpoint{0.000000in}{0.614167in}}{\pgfqpoint{-0.013923in}{0.614167in}}{\pgfqpoint{-0.027278in}{0.619698in}}%
\pgfpathcurveto{\pgfqpoint{-0.037123in}{0.629544in}}{\pgfqpoint{-0.046968in}{0.639389in}}{\pgfqpoint{-0.052500in}{0.652744in}}%
\pgfpathcurveto{\pgfqpoint{-0.052500in}{0.666667in}}{\pgfqpoint{-0.052500in}{0.680590in}}{\pgfqpoint{-0.046968in}{0.693945in}}%
\pgfpathcurveto{\pgfqpoint{-0.037123in}{0.703790in}}{\pgfqpoint{-0.027278in}{0.713635in}}{\pgfqpoint{-0.013923in}{0.719167in}}%
\pgfpathcurveto{\pgfqpoint{0.000000in}{0.719167in}}{\pgfqpoint{0.013923in}{0.719167in}}{\pgfqpoint{0.027278in}{0.713635in}}%
\pgfpathcurveto{\pgfqpoint{0.037123in}{0.703790in}}{\pgfqpoint{0.046968in}{0.693945in}}{\pgfqpoint{0.052500in}{0.680590in}}%
\pgfpathcurveto{\pgfqpoint{0.052500in}{0.666667in}}{\pgfqpoint{0.052500in}{0.652744in}}{\pgfqpoint{0.046968in}{0.639389in}}%
\pgfpathcurveto{\pgfqpoint{0.037123in}{0.629544in}}{\pgfqpoint{0.027278in}{0.619698in}}{\pgfqpoint{0.013923in}{0.614167in}}%
\pgfpathclose%
\pgfpathmoveto{\pgfqpoint{0.166667in}{0.608333in}}%
\pgfpathcurveto{\pgfqpoint{0.182137in}{0.608333in}}{\pgfqpoint{0.196975in}{0.614480in}}{\pgfqpoint{0.207915in}{0.625419in}}%
\pgfpathcurveto{\pgfqpoint{0.218854in}{0.636358in}}{\pgfqpoint{0.225000in}{0.651196in}}{\pgfqpoint{0.225000in}{0.666667in}}%
\pgfpathcurveto{\pgfqpoint{0.225000in}{0.682137in}}{\pgfqpoint{0.218854in}{0.696975in}}{\pgfqpoint{0.207915in}{0.707915in}}%
\pgfpathcurveto{\pgfqpoint{0.196975in}{0.718854in}}{\pgfqpoint{0.182137in}{0.725000in}}{\pgfqpoint{0.166667in}{0.725000in}}%
\pgfpathcurveto{\pgfqpoint{0.151196in}{0.725000in}}{\pgfqpoint{0.136358in}{0.718854in}}{\pgfqpoint{0.125419in}{0.707915in}}%
\pgfpathcurveto{\pgfqpoint{0.114480in}{0.696975in}}{\pgfqpoint{0.108333in}{0.682137in}}{\pgfqpoint{0.108333in}{0.666667in}}%
\pgfpathcurveto{\pgfqpoint{0.108333in}{0.651196in}}{\pgfqpoint{0.114480in}{0.636358in}}{\pgfqpoint{0.125419in}{0.625419in}}%
\pgfpathcurveto{\pgfqpoint{0.136358in}{0.614480in}}{\pgfqpoint{0.151196in}{0.608333in}}{\pgfqpoint{0.166667in}{0.608333in}}%
\pgfpathclose%
\pgfpathmoveto{\pgfqpoint{0.166667in}{0.614167in}}%
\pgfpathcurveto{\pgfqpoint{0.166667in}{0.614167in}}{\pgfqpoint{0.152744in}{0.614167in}}{\pgfqpoint{0.139389in}{0.619698in}}%
\pgfpathcurveto{\pgfqpoint{0.129544in}{0.629544in}}{\pgfqpoint{0.119698in}{0.639389in}}{\pgfqpoint{0.114167in}{0.652744in}}%
\pgfpathcurveto{\pgfqpoint{0.114167in}{0.666667in}}{\pgfqpoint{0.114167in}{0.680590in}}{\pgfqpoint{0.119698in}{0.693945in}}%
\pgfpathcurveto{\pgfqpoint{0.129544in}{0.703790in}}{\pgfqpoint{0.139389in}{0.713635in}}{\pgfqpoint{0.152744in}{0.719167in}}%
\pgfpathcurveto{\pgfqpoint{0.166667in}{0.719167in}}{\pgfqpoint{0.180590in}{0.719167in}}{\pgfqpoint{0.193945in}{0.713635in}}%
\pgfpathcurveto{\pgfqpoint{0.203790in}{0.703790in}}{\pgfqpoint{0.213635in}{0.693945in}}{\pgfqpoint{0.219167in}{0.680590in}}%
\pgfpathcurveto{\pgfqpoint{0.219167in}{0.666667in}}{\pgfqpoint{0.219167in}{0.652744in}}{\pgfqpoint{0.213635in}{0.639389in}}%
\pgfpathcurveto{\pgfqpoint{0.203790in}{0.629544in}}{\pgfqpoint{0.193945in}{0.619698in}}{\pgfqpoint{0.180590in}{0.614167in}}%
\pgfpathclose%
\pgfpathmoveto{\pgfqpoint{0.333333in}{0.608333in}}%
\pgfpathcurveto{\pgfqpoint{0.348804in}{0.608333in}}{\pgfqpoint{0.363642in}{0.614480in}}{\pgfqpoint{0.374581in}{0.625419in}}%
\pgfpathcurveto{\pgfqpoint{0.385520in}{0.636358in}}{\pgfqpoint{0.391667in}{0.651196in}}{\pgfqpoint{0.391667in}{0.666667in}}%
\pgfpathcurveto{\pgfqpoint{0.391667in}{0.682137in}}{\pgfqpoint{0.385520in}{0.696975in}}{\pgfqpoint{0.374581in}{0.707915in}}%
\pgfpathcurveto{\pgfqpoint{0.363642in}{0.718854in}}{\pgfqpoint{0.348804in}{0.725000in}}{\pgfqpoint{0.333333in}{0.725000in}}%
\pgfpathcurveto{\pgfqpoint{0.317863in}{0.725000in}}{\pgfqpoint{0.303025in}{0.718854in}}{\pgfqpoint{0.292085in}{0.707915in}}%
\pgfpathcurveto{\pgfqpoint{0.281146in}{0.696975in}}{\pgfqpoint{0.275000in}{0.682137in}}{\pgfqpoint{0.275000in}{0.666667in}}%
\pgfpathcurveto{\pgfqpoint{0.275000in}{0.651196in}}{\pgfqpoint{0.281146in}{0.636358in}}{\pgfqpoint{0.292085in}{0.625419in}}%
\pgfpathcurveto{\pgfqpoint{0.303025in}{0.614480in}}{\pgfqpoint{0.317863in}{0.608333in}}{\pgfqpoint{0.333333in}{0.608333in}}%
\pgfpathclose%
\pgfpathmoveto{\pgfqpoint{0.333333in}{0.614167in}}%
\pgfpathcurveto{\pgfqpoint{0.333333in}{0.614167in}}{\pgfqpoint{0.319410in}{0.614167in}}{\pgfqpoint{0.306055in}{0.619698in}}%
\pgfpathcurveto{\pgfqpoint{0.296210in}{0.629544in}}{\pgfqpoint{0.286365in}{0.639389in}}{\pgfqpoint{0.280833in}{0.652744in}}%
\pgfpathcurveto{\pgfqpoint{0.280833in}{0.666667in}}{\pgfqpoint{0.280833in}{0.680590in}}{\pgfqpoint{0.286365in}{0.693945in}}%
\pgfpathcurveto{\pgfqpoint{0.296210in}{0.703790in}}{\pgfqpoint{0.306055in}{0.713635in}}{\pgfqpoint{0.319410in}{0.719167in}}%
\pgfpathcurveto{\pgfqpoint{0.333333in}{0.719167in}}{\pgfqpoint{0.347256in}{0.719167in}}{\pgfqpoint{0.360611in}{0.713635in}}%
\pgfpathcurveto{\pgfqpoint{0.370456in}{0.703790in}}{\pgfqpoint{0.380302in}{0.693945in}}{\pgfqpoint{0.385833in}{0.680590in}}%
\pgfpathcurveto{\pgfqpoint{0.385833in}{0.666667in}}{\pgfqpoint{0.385833in}{0.652744in}}{\pgfqpoint{0.380302in}{0.639389in}}%
\pgfpathcurveto{\pgfqpoint{0.370456in}{0.629544in}}{\pgfqpoint{0.360611in}{0.619698in}}{\pgfqpoint{0.347256in}{0.614167in}}%
\pgfpathclose%
\pgfpathmoveto{\pgfqpoint{0.500000in}{0.608333in}}%
\pgfpathcurveto{\pgfqpoint{0.515470in}{0.608333in}}{\pgfqpoint{0.530309in}{0.614480in}}{\pgfqpoint{0.541248in}{0.625419in}}%
\pgfpathcurveto{\pgfqpoint{0.552187in}{0.636358in}}{\pgfqpoint{0.558333in}{0.651196in}}{\pgfqpoint{0.558333in}{0.666667in}}%
\pgfpathcurveto{\pgfqpoint{0.558333in}{0.682137in}}{\pgfqpoint{0.552187in}{0.696975in}}{\pgfqpoint{0.541248in}{0.707915in}}%
\pgfpathcurveto{\pgfqpoint{0.530309in}{0.718854in}}{\pgfqpoint{0.515470in}{0.725000in}}{\pgfqpoint{0.500000in}{0.725000in}}%
\pgfpathcurveto{\pgfqpoint{0.484530in}{0.725000in}}{\pgfqpoint{0.469691in}{0.718854in}}{\pgfqpoint{0.458752in}{0.707915in}}%
\pgfpathcurveto{\pgfqpoint{0.447813in}{0.696975in}}{\pgfqpoint{0.441667in}{0.682137in}}{\pgfqpoint{0.441667in}{0.666667in}}%
\pgfpathcurveto{\pgfqpoint{0.441667in}{0.651196in}}{\pgfqpoint{0.447813in}{0.636358in}}{\pgfqpoint{0.458752in}{0.625419in}}%
\pgfpathcurveto{\pgfqpoint{0.469691in}{0.614480in}}{\pgfqpoint{0.484530in}{0.608333in}}{\pgfqpoint{0.500000in}{0.608333in}}%
\pgfpathclose%
\pgfpathmoveto{\pgfqpoint{0.500000in}{0.614167in}}%
\pgfpathcurveto{\pgfqpoint{0.500000in}{0.614167in}}{\pgfqpoint{0.486077in}{0.614167in}}{\pgfqpoint{0.472722in}{0.619698in}}%
\pgfpathcurveto{\pgfqpoint{0.462877in}{0.629544in}}{\pgfqpoint{0.453032in}{0.639389in}}{\pgfqpoint{0.447500in}{0.652744in}}%
\pgfpathcurveto{\pgfqpoint{0.447500in}{0.666667in}}{\pgfqpoint{0.447500in}{0.680590in}}{\pgfqpoint{0.453032in}{0.693945in}}%
\pgfpathcurveto{\pgfqpoint{0.462877in}{0.703790in}}{\pgfqpoint{0.472722in}{0.713635in}}{\pgfqpoint{0.486077in}{0.719167in}}%
\pgfpathcurveto{\pgfqpoint{0.500000in}{0.719167in}}{\pgfqpoint{0.513923in}{0.719167in}}{\pgfqpoint{0.527278in}{0.713635in}}%
\pgfpathcurveto{\pgfqpoint{0.537123in}{0.703790in}}{\pgfqpoint{0.546968in}{0.693945in}}{\pgfqpoint{0.552500in}{0.680590in}}%
\pgfpathcurveto{\pgfqpoint{0.552500in}{0.666667in}}{\pgfqpoint{0.552500in}{0.652744in}}{\pgfqpoint{0.546968in}{0.639389in}}%
\pgfpathcurveto{\pgfqpoint{0.537123in}{0.629544in}}{\pgfqpoint{0.527278in}{0.619698in}}{\pgfqpoint{0.513923in}{0.614167in}}%
\pgfpathclose%
\pgfpathmoveto{\pgfqpoint{0.666667in}{0.608333in}}%
\pgfpathcurveto{\pgfqpoint{0.682137in}{0.608333in}}{\pgfqpoint{0.696975in}{0.614480in}}{\pgfqpoint{0.707915in}{0.625419in}}%
\pgfpathcurveto{\pgfqpoint{0.718854in}{0.636358in}}{\pgfqpoint{0.725000in}{0.651196in}}{\pgfqpoint{0.725000in}{0.666667in}}%
\pgfpathcurveto{\pgfqpoint{0.725000in}{0.682137in}}{\pgfqpoint{0.718854in}{0.696975in}}{\pgfqpoint{0.707915in}{0.707915in}}%
\pgfpathcurveto{\pgfqpoint{0.696975in}{0.718854in}}{\pgfqpoint{0.682137in}{0.725000in}}{\pgfqpoint{0.666667in}{0.725000in}}%
\pgfpathcurveto{\pgfqpoint{0.651196in}{0.725000in}}{\pgfqpoint{0.636358in}{0.718854in}}{\pgfqpoint{0.625419in}{0.707915in}}%
\pgfpathcurveto{\pgfqpoint{0.614480in}{0.696975in}}{\pgfqpoint{0.608333in}{0.682137in}}{\pgfqpoint{0.608333in}{0.666667in}}%
\pgfpathcurveto{\pgfqpoint{0.608333in}{0.651196in}}{\pgfqpoint{0.614480in}{0.636358in}}{\pgfqpoint{0.625419in}{0.625419in}}%
\pgfpathcurveto{\pgfqpoint{0.636358in}{0.614480in}}{\pgfqpoint{0.651196in}{0.608333in}}{\pgfqpoint{0.666667in}{0.608333in}}%
\pgfpathclose%
\pgfpathmoveto{\pgfqpoint{0.666667in}{0.614167in}}%
\pgfpathcurveto{\pgfqpoint{0.666667in}{0.614167in}}{\pgfqpoint{0.652744in}{0.614167in}}{\pgfqpoint{0.639389in}{0.619698in}}%
\pgfpathcurveto{\pgfqpoint{0.629544in}{0.629544in}}{\pgfqpoint{0.619698in}{0.639389in}}{\pgfqpoint{0.614167in}{0.652744in}}%
\pgfpathcurveto{\pgfqpoint{0.614167in}{0.666667in}}{\pgfqpoint{0.614167in}{0.680590in}}{\pgfqpoint{0.619698in}{0.693945in}}%
\pgfpathcurveto{\pgfqpoint{0.629544in}{0.703790in}}{\pgfqpoint{0.639389in}{0.713635in}}{\pgfqpoint{0.652744in}{0.719167in}}%
\pgfpathcurveto{\pgfqpoint{0.666667in}{0.719167in}}{\pgfqpoint{0.680590in}{0.719167in}}{\pgfqpoint{0.693945in}{0.713635in}}%
\pgfpathcurveto{\pgfqpoint{0.703790in}{0.703790in}}{\pgfqpoint{0.713635in}{0.693945in}}{\pgfqpoint{0.719167in}{0.680590in}}%
\pgfpathcurveto{\pgfqpoint{0.719167in}{0.666667in}}{\pgfqpoint{0.719167in}{0.652744in}}{\pgfqpoint{0.713635in}{0.639389in}}%
\pgfpathcurveto{\pgfqpoint{0.703790in}{0.629544in}}{\pgfqpoint{0.693945in}{0.619698in}}{\pgfqpoint{0.680590in}{0.614167in}}%
\pgfpathclose%
\pgfpathmoveto{\pgfqpoint{0.833333in}{0.608333in}}%
\pgfpathcurveto{\pgfqpoint{0.848804in}{0.608333in}}{\pgfqpoint{0.863642in}{0.614480in}}{\pgfqpoint{0.874581in}{0.625419in}}%
\pgfpathcurveto{\pgfqpoint{0.885520in}{0.636358in}}{\pgfqpoint{0.891667in}{0.651196in}}{\pgfqpoint{0.891667in}{0.666667in}}%
\pgfpathcurveto{\pgfqpoint{0.891667in}{0.682137in}}{\pgfqpoint{0.885520in}{0.696975in}}{\pgfqpoint{0.874581in}{0.707915in}}%
\pgfpathcurveto{\pgfqpoint{0.863642in}{0.718854in}}{\pgfqpoint{0.848804in}{0.725000in}}{\pgfqpoint{0.833333in}{0.725000in}}%
\pgfpathcurveto{\pgfqpoint{0.817863in}{0.725000in}}{\pgfqpoint{0.803025in}{0.718854in}}{\pgfqpoint{0.792085in}{0.707915in}}%
\pgfpathcurveto{\pgfqpoint{0.781146in}{0.696975in}}{\pgfqpoint{0.775000in}{0.682137in}}{\pgfqpoint{0.775000in}{0.666667in}}%
\pgfpathcurveto{\pgfqpoint{0.775000in}{0.651196in}}{\pgfqpoint{0.781146in}{0.636358in}}{\pgfqpoint{0.792085in}{0.625419in}}%
\pgfpathcurveto{\pgfqpoint{0.803025in}{0.614480in}}{\pgfqpoint{0.817863in}{0.608333in}}{\pgfqpoint{0.833333in}{0.608333in}}%
\pgfpathclose%
\pgfpathmoveto{\pgfqpoint{0.833333in}{0.614167in}}%
\pgfpathcurveto{\pgfqpoint{0.833333in}{0.614167in}}{\pgfqpoint{0.819410in}{0.614167in}}{\pgfqpoint{0.806055in}{0.619698in}}%
\pgfpathcurveto{\pgfqpoint{0.796210in}{0.629544in}}{\pgfqpoint{0.786365in}{0.639389in}}{\pgfqpoint{0.780833in}{0.652744in}}%
\pgfpathcurveto{\pgfqpoint{0.780833in}{0.666667in}}{\pgfqpoint{0.780833in}{0.680590in}}{\pgfqpoint{0.786365in}{0.693945in}}%
\pgfpathcurveto{\pgfqpoint{0.796210in}{0.703790in}}{\pgfqpoint{0.806055in}{0.713635in}}{\pgfqpoint{0.819410in}{0.719167in}}%
\pgfpathcurveto{\pgfqpoint{0.833333in}{0.719167in}}{\pgfqpoint{0.847256in}{0.719167in}}{\pgfqpoint{0.860611in}{0.713635in}}%
\pgfpathcurveto{\pgfqpoint{0.870456in}{0.703790in}}{\pgfqpoint{0.880302in}{0.693945in}}{\pgfqpoint{0.885833in}{0.680590in}}%
\pgfpathcurveto{\pgfqpoint{0.885833in}{0.666667in}}{\pgfqpoint{0.885833in}{0.652744in}}{\pgfqpoint{0.880302in}{0.639389in}}%
\pgfpathcurveto{\pgfqpoint{0.870456in}{0.629544in}}{\pgfqpoint{0.860611in}{0.619698in}}{\pgfqpoint{0.847256in}{0.614167in}}%
\pgfpathclose%
\pgfpathmoveto{\pgfqpoint{1.000000in}{0.608333in}}%
\pgfpathcurveto{\pgfqpoint{1.015470in}{0.608333in}}{\pgfqpoint{1.030309in}{0.614480in}}{\pgfqpoint{1.041248in}{0.625419in}}%
\pgfpathcurveto{\pgfqpoint{1.052187in}{0.636358in}}{\pgfqpoint{1.058333in}{0.651196in}}{\pgfqpoint{1.058333in}{0.666667in}}%
\pgfpathcurveto{\pgfqpoint{1.058333in}{0.682137in}}{\pgfqpoint{1.052187in}{0.696975in}}{\pgfqpoint{1.041248in}{0.707915in}}%
\pgfpathcurveto{\pgfqpoint{1.030309in}{0.718854in}}{\pgfqpoint{1.015470in}{0.725000in}}{\pgfqpoint{1.000000in}{0.725000in}}%
\pgfpathcurveto{\pgfqpoint{0.984530in}{0.725000in}}{\pgfqpoint{0.969691in}{0.718854in}}{\pgfqpoint{0.958752in}{0.707915in}}%
\pgfpathcurveto{\pgfqpoint{0.947813in}{0.696975in}}{\pgfqpoint{0.941667in}{0.682137in}}{\pgfqpoint{0.941667in}{0.666667in}}%
\pgfpathcurveto{\pgfqpoint{0.941667in}{0.651196in}}{\pgfqpoint{0.947813in}{0.636358in}}{\pgfqpoint{0.958752in}{0.625419in}}%
\pgfpathcurveto{\pgfqpoint{0.969691in}{0.614480in}}{\pgfqpoint{0.984530in}{0.608333in}}{\pgfqpoint{1.000000in}{0.608333in}}%
\pgfpathclose%
\pgfpathmoveto{\pgfqpoint{1.000000in}{0.614167in}}%
\pgfpathcurveto{\pgfqpoint{1.000000in}{0.614167in}}{\pgfqpoint{0.986077in}{0.614167in}}{\pgfqpoint{0.972722in}{0.619698in}}%
\pgfpathcurveto{\pgfqpoint{0.962877in}{0.629544in}}{\pgfqpoint{0.953032in}{0.639389in}}{\pgfqpoint{0.947500in}{0.652744in}}%
\pgfpathcurveto{\pgfqpoint{0.947500in}{0.666667in}}{\pgfqpoint{0.947500in}{0.680590in}}{\pgfqpoint{0.953032in}{0.693945in}}%
\pgfpathcurveto{\pgfqpoint{0.962877in}{0.703790in}}{\pgfqpoint{0.972722in}{0.713635in}}{\pgfqpoint{0.986077in}{0.719167in}}%
\pgfpathcurveto{\pgfqpoint{1.000000in}{0.719167in}}{\pgfqpoint{1.013923in}{0.719167in}}{\pgfqpoint{1.027278in}{0.713635in}}%
\pgfpathcurveto{\pgfqpoint{1.037123in}{0.703790in}}{\pgfqpoint{1.046968in}{0.693945in}}{\pgfqpoint{1.052500in}{0.680590in}}%
\pgfpathcurveto{\pgfqpoint{1.052500in}{0.666667in}}{\pgfqpoint{1.052500in}{0.652744in}}{\pgfqpoint{1.046968in}{0.639389in}}%
\pgfpathcurveto{\pgfqpoint{1.037123in}{0.629544in}}{\pgfqpoint{1.027278in}{0.619698in}}{\pgfqpoint{1.013923in}{0.614167in}}%
\pgfpathclose%
\pgfpathmoveto{\pgfqpoint{0.083333in}{0.775000in}}%
\pgfpathcurveto{\pgfqpoint{0.098804in}{0.775000in}}{\pgfqpoint{0.113642in}{0.781146in}}{\pgfqpoint{0.124581in}{0.792085in}}%
\pgfpathcurveto{\pgfqpoint{0.135520in}{0.803025in}}{\pgfqpoint{0.141667in}{0.817863in}}{\pgfqpoint{0.141667in}{0.833333in}}%
\pgfpathcurveto{\pgfqpoint{0.141667in}{0.848804in}}{\pgfqpoint{0.135520in}{0.863642in}}{\pgfqpoint{0.124581in}{0.874581in}}%
\pgfpathcurveto{\pgfqpoint{0.113642in}{0.885520in}}{\pgfqpoint{0.098804in}{0.891667in}}{\pgfqpoint{0.083333in}{0.891667in}}%
\pgfpathcurveto{\pgfqpoint{0.067863in}{0.891667in}}{\pgfqpoint{0.053025in}{0.885520in}}{\pgfqpoint{0.042085in}{0.874581in}}%
\pgfpathcurveto{\pgfqpoint{0.031146in}{0.863642in}}{\pgfqpoint{0.025000in}{0.848804in}}{\pgfqpoint{0.025000in}{0.833333in}}%
\pgfpathcurveto{\pgfqpoint{0.025000in}{0.817863in}}{\pgfqpoint{0.031146in}{0.803025in}}{\pgfqpoint{0.042085in}{0.792085in}}%
\pgfpathcurveto{\pgfqpoint{0.053025in}{0.781146in}}{\pgfqpoint{0.067863in}{0.775000in}}{\pgfqpoint{0.083333in}{0.775000in}}%
\pgfpathclose%
\pgfpathmoveto{\pgfqpoint{0.083333in}{0.780833in}}%
\pgfpathcurveto{\pgfqpoint{0.083333in}{0.780833in}}{\pgfqpoint{0.069410in}{0.780833in}}{\pgfqpoint{0.056055in}{0.786365in}}%
\pgfpathcurveto{\pgfqpoint{0.046210in}{0.796210in}}{\pgfqpoint{0.036365in}{0.806055in}}{\pgfqpoint{0.030833in}{0.819410in}}%
\pgfpathcurveto{\pgfqpoint{0.030833in}{0.833333in}}{\pgfqpoint{0.030833in}{0.847256in}}{\pgfqpoint{0.036365in}{0.860611in}}%
\pgfpathcurveto{\pgfqpoint{0.046210in}{0.870456in}}{\pgfqpoint{0.056055in}{0.880302in}}{\pgfqpoint{0.069410in}{0.885833in}}%
\pgfpathcurveto{\pgfqpoint{0.083333in}{0.885833in}}{\pgfqpoint{0.097256in}{0.885833in}}{\pgfqpoint{0.110611in}{0.880302in}}%
\pgfpathcurveto{\pgfqpoint{0.120456in}{0.870456in}}{\pgfqpoint{0.130302in}{0.860611in}}{\pgfqpoint{0.135833in}{0.847256in}}%
\pgfpathcurveto{\pgfqpoint{0.135833in}{0.833333in}}{\pgfqpoint{0.135833in}{0.819410in}}{\pgfqpoint{0.130302in}{0.806055in}}%
\pgfpathcurveto{\pgfqpoint{0.120456in}{0.796210in}}{\pgfqpoint{0.110611in}{0.786365in}}{\pgfqpoint{0.097256in}{0.780833in}}%
\pgfpathclose%
\pgfpathmoveto{\pgfqpoint{0.250000in}{0.775000in}}%
\pgfpathcurveto{\pgfqpoint{0.265470in}{0.775000in}}{\pgfqpoint{0.280309in}{0.781146in}}{\pgfqpoint{0.291248in}{0.792085in}}%
\pgfpathcurveto{\pgfqpoint{0.302187in}{0.803025in}}{\pgfqpoint{0.308333in}{0.817863in}}{\pgfqpoint{0.308333in}{0.833333in}}%
\pgfpathcurveto{\pgfqpoint{0.308333in}{0.848804in}}{\pgfqpoint{0.302187in}{0.863642in}}{\pgfqpoint{0.291248in}{0.874581in}}%
\pgfpathcurveto{\pgfqpoint{0.280309in}{0.885520in}}{\pgfqpoint{0.265470in}{0.891667in}}{\pgfqpoint{0.250000in}{0.891667in}}%
\pgfpathcurveto{\pgfqpoint{0.234530in}{0.891667in}}{\pgfqpoint{0.219691in}{0.885520in}}{\pgfqpoint{0.208752in}{0.874581in}}%
\pgfpathcurveto{\pgfqpoint{0.197813in}{0.863642in}}{\pgfqpoint{0.191667in}{0.848804in}}{\pgfqpoint{0.191667in}{0.833333in}}%
\pgfpathcurveto{\pgfqpoint{0.191667in}{0.817863in}}{\pgfqpoint{0.197813in}{0.803025in}}{\pgfqpoint{0.208752in}{0.792085in}}%
\pgfpathcurveto{\pgfqpoint{0.219691in}{0.781146in}}{\pgfqpoint{0.234530in}{0.775000in}}{\pgfqpoint{0.250000in}{0.775000in}}%
\pgfpathclose%
\pgfpathmoveto{\pgfqpoint{0.250000in}{0.780833in}}%
\pgfpathcurveto{\pgfqpoint{0.250000in}{0.780833in}}{\pgfqpoint{0.236077in}{0.780833in}}{\pgfqpoint{0.222722in}{0.786365in}}%
\pgfpathcurveto{\pgfqpoint{0.212877in}{0.796210in}}{\pgfqpoint{0.203032in}{0.806055in}}{\pgfqpoint{0.197500in}{0.819410in}}%
\pgfpathcurveto{\pgfqpoint{0.197500in}{0.833333in}}{\pgfqpoint{0.197500in}{0.847256in}}{\pgfqpoint{0.203032in}{0.860611in}}%
\pgfpathcurveto{\pgfqpoint{0.212877in}{0.870456in}}{\pgfqpoint{0.222722in}{0.880302in}}{\pgfqpoint{0.236077in}{0.885833in}}%
\pgfpathcurveto{\pgfqpoint{0.250000in}{0.885833in}}{\pgfqpoint{0.263923in}{0.885833in}}{\pgfqpoint{0.277278in}{0.880302in}}%
\pgfpathcurveto{\pgfqpoint{0.287123in}{0.870456in}}{\pgfqpoint{0.296968in}{0.860611in}}{\pgfqpoint{0.302500in}{0.847256in}}%
\pgfpathcurveto{\pgfqpoint{0.302500in}{0.833333in}}{\pgfqpoint{0.302500in}{0.819410in}}{\pgfqpoint{0.296968in}{0.806055in}}%
\pgfpathcurveto{\pgfqpoint{0.287123in}{0.796210in}}{\pgfqpoint{0.277278in}{0.786365in}}{\pgfqpoint{0.263923in}{0.780833in}}%
\pgfpathclose%
\pgfpathmoveto{\pgfqpoint{0.416667in}{0.775000in}}%
\pgfpathcurveto{\pgfqpoint{0.432137in}{0.775000in}}{\pgfqpoint{0.446975in}{0.781146in}}{\pgfqpoint{0.457915in}{0.792085in}}%
\pgfpathcurveto{\pgfqpoint{0.468854in}{0.803025in}}{\pgfqpoint{0.475000in}{0.817863in}}{\pgfqpoint{0.475000in}{0.833333in}}%
\pgfpathcurveto{\pgfqpoint{0.475000in}{0.848804in}}{\pgfqpoint{0.468854in}{0.863642in}}{\pgfqpoint{0.457915in}{0.874581in}}%
\pgfpathcurveto{\pgfqpoint{0.446975in}{0.885520in}}{\pgfqpoint{0.432137in}{0.891667in}}{\pgfqpoint{0.416667in}{0.891667in}}%
\pgfpathcurveto{\pgfqpoint{0.401196in}{0.891667in}}{\pgfqpoint{0.386358in}{0.885520in}}{\pgfqpoint{0.375419in}{0.874581in}}%
\pgfpathcurveto{\pgfqpoint{0.364480in}{0.863642in}}{\pgfqpoint{0.358333in}{0.848804in}}{\pgfqpoint{0.358333in}{0.833333in}}%
\pgfpathcurveto{\pgfqpoint{0.358333in}{0.817863in}}{\pgfqpoint{0.364480in}{0.803025in}}{\pgfqpoint{0.375419in}{0.792085in}}%
\pgfpathcurveto{\pgfqpoint{0.386358in}{0.781146in}}{\pgfqpoint{0.401196in}{0.775000in}}{\pgfqpoint{0.416667in}{0.775000in}}%
\pgfpathclose%
\pgfpathmoveto{\pgfqpoint{0.416667in}{0.780833in}}%
\pgfpathcurveto{\pgfqpoint{0.416667in}{0.780833in}}{\pgfqpoint{0.402744in}{0.780833in}}{\pgfqpoint{0.389389in}{0.786365in}}%
\pgfpathcurveto{\pgfqpoint{0.379544in}{0.796210in}}{\pgfqpoint{0.369698in}{0.806055in}}{\pgfqpoint{0.364167in}{0.819410in}}%
\pgfpathcurveto{\pgfqpoint{0.364167in}{0.833333in}}{\pgfqpoint{0.364167in}{0.847256in}}{\pgfqpoint{0.369698in}{0.860611in}}%
\pgfpathcurveto{\pgfqpoint{0.379544in}{0.870456in}}{\pgfqpoint{0.389389in}{0.880302in}}{\pgfqpoint{0.402744in}{0.885833in}}%
\pgfpathcurveto{\pgfqpoint{0.416667in}{0.885833in}}{\pgfqpoint{0.430590in}{0.885833in}}{\pgfqpoint{0.443945in}{0.880302in}}%
\pgfpathcurveto{\pgfqpoint{0.453790in}{0.870456in}}{\pgfqpoint{0.463635in}{0.860611in}}{\pgfqpoint{0.469167in}{0.847256in}}%
\pgfpathcurveto{\pgfqpoint{0.469167in}{0.833333in}}{\pgfqpoint{0.469167in}{0.819410in}}{\pgfqpoint{0.463635in}{0.806055in}}%
\pgfpathcurveto{\pgfqpoint{0.453790in}{0.796210in}}{\pgfqpoint{0.443945in}{0.786365in}}{\pgfqpoint{0.430590in}{0.780833in}}%
\pgfpathclose%
\pgfpathmoveto{\pgfqpoint{0.583333in}{0.775000in}}%
\pgfpathcurveto{\pgfqpoint{0.598804in}{0.775000in}}{\pgfqpoint{0.613642in}{0.781146in}}{\pgfqpoint{0.624581in}{0.792085in}}%
\pgfpathcurveto{\pgfqpoint{0.635520in}{0.803025in}}{\pgfqpoint{0.641667in}{0.817863in}}{\pgfqpoint{0.641667in}{0.833333in}}%
\pgfpathcurveto{\pgfqpoint{0.641667in}{0.848804in}}{\pgfqpoint{0.635520in}{0.863642in}}{\pgfqpoint{0.624581in}{0.874581in}}%
\pgfpathcurveto{\pgfqpoint{0.613642in}{0.885520in}}{\pgfqpoint{0.598804in}{0.891667in}}{\pgfqpoint{0.583333in}{0.891667in}}%
\pgfpathcurveto{\pgfqpoint{0.567863in}{0.891667in}}{\pgfqpoint{0.553025in}{0.885520in}}{\pgfqpoint{0.542085in}{0.874581in}}%
\pgfpathcurveto{\pgfqpoint{0.531146in}{0.863642in}}{\pgfqpoint{0.525000in}{0.848804in}}{\pgfqpoint{0.525000in}{0.833333in}}%
\pgfpathcurveto{\pgfqpoint{0.525000in}{0.817863in}}{\pgfqpoint{0.531146in}{0.803025in}}{\pgfqpoint{0.542085in}{0.792085in}}%
\pgfpathcurveto{\pgfqpoint{0.553025in}{0.781146in}}{\pgfqpoint{0.567863in}{0.775000in}}{\pgfqpoint{0.583333in}{0.775000in}}%
\pgfpathclose%
\pgfpathmoveto{\pgfqpoint{0.583333in}{0.780833in}}%
\pgfpathcurveto{\pgfqpoint{0.583333in}{0.780833in}}{\pgfqpoint{0.569410in}{0.780833in}}{\pgfqpoint{0.556055in}{0.786365in}}%
\pgfpathcurveto{\pgfqpoint{0.546210in}{0.796210in}}{\pgfqpoint{0.536365in}{0.806055in}}{\pgfqpoint{0.530833in}{0.819410in}}%
\pgfpathcurveto{\pgfqpoint{0.530833in}{0.833333in}}{\pgfqpoint{0.530833in}{0.847256in}}{\pgfqpoint{0.536365in}{0.860611in}}%
\pgfpathcurveto{\pgfqpoint{0.546210in}{0.870456in}}{\pgfqpoint{0.556055in}{0.880302in}}{\pgfqpoint{0.569410in}{0.885833in}}%
\pgfpathcurveto{\pgfqpoint{0.583333in}{0.885833in}}{\pgfqpoint{0.597256in}{0.885833in}}{\pgfqpoint{0.610611in}{0.880302in}}%
\pgfpathcurveto{\pgfqpoint{0.620456in}{0.870456in}}{\pgfqpoint{0.630302in}{0.860611in}}{\pgfqpoint{0.635833in}{0.847256in}}%
\pgfpathcurveto{\pgfqpoint{0.635833in}{0.833333in}}{\pgfqpoint{0.635833in}{0.819410in}}{\pgfqpoint{0.630302in}{0.806055in}}%
\pgfpathcurveto{\pgfqpoint{0.620456in}{0.796210in}}{\pgfqpoint{0.610611in}{0.786365in}}{\pgfqpoint{0.597256in}{0.780833in}}%
\pgfpathclose%
\pgfpathmoveto{\pgfqpoint{0.750000in}{0.775000in}}%
\pgfpathcurveto{\pgfqpoint{0.765470in}{0.775000in}}{\pgfqpoint{0.780309in}{0.781146in}}{\pgfqpoint{0.791248in}{0.792085in}}%
\pgfpathcurveto{\pgfqpoint{0.802187in}{0.803025in}}{\pgfqpoint{0.808333in}{0.817863in}}{\pgfqpoint{0.808333in}{0.833333in}}%
\pgfpathcurveto{\pgfqpoint{0.808333in}{0.848804in}}{\pgfqpoint{0.802187in}{0.863642in}}{\pgfqpoint{0.791248in}{0.874581in}}%
\pgfpathcurveto{\pgfqpoint{0.780309in}{0.885520in}}{\pgfqpoint{0.765470in}{0.891667in}}{\pgfqpoint{0.750000in}{0.891667in}}%
\pgfpathcurveto{\pgfqpoint{0.734530in}{0.891667in}}{\pgfqpoint{0.719691in}{0.885520in}}{\pgfqpoint{0.708752in}{0.874581in}}%
\pgfpathcurveto{\pgfqpoint{0.697813in}{0.863642in}}{\pgfqpoint{0.691667in}{0.848804in}}{\pgfqpoint{0.691667in}{0.833333in}}%
\pgfpathcurveto{\pgfqpoint{0.691667in}{0.817863in}}{\pgfqpoint{0.697813in}{0.803025in}}{\pgfqpoint{0.708752in}{0.792085in}}%
\pgfpathcurveto{\pgfqpoint{0.719691in}{0.781146in}}{\pgfqpoint{0.734530in}{0.775000in}}{\pgfqpoint{0.750000in}{0.775000in}}%
\pgfpathclose%
\pgfpathmoveto{\pgfqpoint{0.750000in}{0.780833in}}%
\pgfpathcurveto{\pgfqpoint{0.750000in}{0.780833in}}{\pgfqpoint{0.736077in}{0.780833in}}{\pgfqpoint{0.722722in}{0.786365in}}%
\pgfpathcurveto{\pgfqpoint{0.712877in}{0.796210in}}{\pgfqpoint{0.703032in}{0.806055in}}{\pgfqpoint{0.697500in}{0.819410in}}%
\pgfpathcurveto{\pgfqpoint{0.697500in}{0.833333in}}{\pgfqpoint{0.697500in}{0.847256in}}{\pgfqpoint{0.703032in}{0.860611in}}%
\pgfpathcurveto{\pgfqpoint{0.712877in}{0.870456in}}{\pgfqpoint{0.722722in}{0.880302in}}{\pgfqpoint{0.736077in}{0.885833in}}%
\pgfpathcurveto{\pgfqpoint{0.750000in}{0.885833in}}{\pgfqpoint{0.763923in}{0.885833in}}{\pgfqpoint{0.777278in}{0.880302in}}%
\pgfpathcurveto{\pgfqpoint{0.787123in}{0.870456in}}{\pgfqpoint{0.796968in}{0.860611in}}{\pgfqpoint{0.802500in}{0.847256in}}%
\pgfpathcurveto{\pgfqpoint{0.802500in}{0.833333in}}{\pgfqpoint{0.802500in}{0.819410in}}{\pgfqpoint{0.796968in}{0.806055in}}%
\pgfpathcurveto{\pgfqpoint{0.787123in}{0.796210in}}{\pgfqpoint{0.777278in}{0.786365in}}{\pgfqpoint{0.763923in}{0.780833in}}%
\pgfpathclose%
\pgfpathmoveto{\pgfqpoint{0.916667in}{0.775000in}}%
\pgfpathcurveto{\pgfqpoint{0.932137in}{0.775000in}}{\pgfqpoint{0.946975in}{0.781146in}}{\pgfqpoint{0.957915in}{0.792085in}}%
\pgfpathcurveto{\pgfqpoint{0.968854in}{0.803025in}}{\pgfqpoint{0.975000in}{0.817863in}}{\pgfqpoint{0.975000in}{0.833333in}}%
\pgfpathcurveto{\pgfqpoint{0.975000in}{0.848804in}}{\pgfqpoint{0.968854in}{0.863642in}}{\pgfqpoint{0.957915in}{0.874581in}}%
\pgfpathcurveto{\pgfqpoint{0.946975in}{0.885520in}}{\pgfqpoint{0.932137in}{0.891667in}}{\pgfqpoint{0.916667in}{0.891667in}}%
\pgfpathcurveto{\pgfqpoint{0.901196in}{0.891667in}}{\pgfqpoint{0.886358in}{0.885520in}}{\pgfqpoint{0.875419in}{0.874581in}}%
\pgfpathcurveto{\pgfqpoint{0.864480in}{0.863642in}}{\pgfqpoint{0.858333in}{0.848804in}}{\pgfqpoint{0.858333in}{0.833333in}}%
\pgfpathcurveto{\pgfqpoint{0.858333in}{0.817863in}}{\pgfqpoint{0.864480in}{0.803025in}}{\pgfqpoint{0.875419in}{0.792085in}}%
\pgfpathcurveto{\pgfqpoint{0.886358in}{0.781146in}}{\pgfqpoint{0.901196in}{0.775000in}}{\pgfqpoint{0.916667in}{0.775000in}}%
\pgfpathclose%
\pgfpathmoveto{\pgfqpoint{0.916667in}{0.780833in}}%
\pgfpathcurveto{\pgfqpoint{0.916667in}{0.780833in}}{\pgfqpoint{0.902744in}{0.780833in}}{\pgfqpoint{0.889389in}{0.786365in}}%
\pgfpathcurveto{\pgfqpoint{0.879544in}{0.796210in}}{\pgfqpoint{0.869698in}{0.806055in}}{\pgfqpoint{0.864167in}{0.819410in}}%
\pgfpathcurveto{\pgfqpoint{0.864167in}{0.833333in}}{\pgfqpoint{0.864167in}{0.847256in}}{\pgfqpoint{0.869698in}{0.860611in}}%
\pgfpathcurveto{\pgfqpoint{0.879544in}{0.870456in}}{\pgfqpoint{0.889389in}{0.880302in}}{\pgfqpoint{0.902744in}{0.885833in}}%
\pgfpathcurveto{\pgfqpoint{0.916667in}{0.885833in}}{\pgfqpoint{0.930590in}{0.885833in}}{\pgfqpoint{0.943945in}{0.880302in}}%
\pgfpathcurveto{\pgfqpoint{0.953790in}{0.870456in}}{\pgfqpoint{0.963635in}{0.860611in}}{\pgfqpoint{0.969167in}{0.847256in}}%
\pgfpathcurveto{\pgfqpoint{0.969167in}{0.833333in}}{\pgfqpoint{0.969167in}{0.819410in}}{\pgfqpoint{0.963635in}{0.806055in}}%
\pgfpathcurveto{\pgfqpoint{0.953790in}{0.796210in}}{\pgfqpoint{0.943945in}{0.786365in}}{\pgfqpoint{0.930590in}{0.780833in}}%
\pgfpathclose%
\pgfpathmoveto{\pgfqpoint{0.000000in}{0.941667in}}%
\pgfpathcurveto{\pgfqpoint{0.015470in}{0.941667in}}{\pgfqpoint{0.030309in}{0.947813in}}{\pgfqpoint{0.041248in}{0.958752in}}%
\pgfpathcurveto{\pgfqpoint{0.052187in}{0.969691in}}{\pgfqpoint{0.058333in}{0.984530in}}{\pgfqpoint{0.058333in}{1.000000in}}%
\pgfpathcurveto{\pgfqpoint{0.058333in}{1.015470in}}{\pgfqpoint{0.052187in}{1.030309in}}{\pgfqpoint{0.041248in}{1.041248in}}%
\pgfpathcurveto{\pgfqpoint{0.030309in}{1.052187in}}{\pgfqpoint{0.015470in}{1.058333in}}{\pgfqpoint{0.000000in}{1.058333in}}%
\pgfpathcurveto{\pgfqpoint{-0.015470in}{1.058333in}}{\pgfqpoint{-0.030309in}{1.052187in}}{\pgfqpoint{-0.041248in}{1.041248in}}%
\pgfpathcurveto{\pgfqpoint{-0.052187in}{1.030309in}}{\pgfqpoint{-0.058333in}{1.015470in}}{\pgfqpoint{-0.058333in}{1.000000in}}%
\pgfpathcurveto{\pgfqpoint{-0.058333in}{0.984530in}}{\pgfqpoint{-0.052187in}{0.969691in}}{\pgfqpoint{-0.041248in}{0.958752in}}%
\pgfpathcurveto{\pgfqpoint{-0.030309in}{0.947813in}}{\pgfqpoint{-0.015470in}{0.941667in}}{\pgfqpoint{0.000000in}{0.941667in}}%
\pgfpathclose%
\pgfpathmoveto{\pgfqpoint{0.000000in}{0.947500in}}%
\pgfpathcurveto{\pgfqpoint{0.000000in}{0.947500in}}{\pgfqpoint{-0.013923in}{0.947500in}}{\pgfqpoint{-0.027278in}{0.953032in}}%
\pgfpathcurveto{\pgfqpoint{-0.037123in}{0.962877in}}{\pgfqpoint{-0.046968in}{0.972722in}}{\pgfqpoint{-0.052500in}{0.986077in}}%
\pgfpathcurveto{\pgfqpoint{-0.052500in}{1.000000in}}{\pgfqpoint{-0.052500in}{1.013923in}}{\pgfqpoint{-0.046968in}{1.027278in}}%
\pgfpathcurveto{\pgfqpoint{-0.037123in}{1.037123in}}{\pgfqpoint{-0.027278in}{1.046968in}}{\pgfqpoint{-0.013923in}{1.052500in}}%
\pgfpathcurveto{\pgfqpoint{0.000000in}{1.052500in}}{\pgfqpoint{0.013923in}{1.052500in}}{\pgfqpoint{0.027278in}{1.046968in}}%
\pgfpathcurveto{\pgfqpoint{0.037123in}{1.037123in}}{\pgfqpoint{0.046968in}{1.027278in}}{\pgfqpoint{0.052500in}{1.013923in}}%
\pgfpathcurveto{\pgfqpoint{0.052500in}{1.000000in}}{\pgfqpoint{0.052500in}{0.986077in}}{\pgfqpoint{0.046968in}{0.972722in}}%
\pgfpathcurveto{\pgfqpoint{0.037123in}{0.962877in}}{\pgfqpoint{0.027278in}{0.953032in}}{\pgfqpoint{0.013923in}{0.947500in}}%
\pgfpathclose%
\pgfpathmoveto{\pgfqpoint{0.166667in}{0.941667in}}%
\pgfpathcurveto{\pgfqpoint{0.182137in}{0.941667in}}{\pgfqpoint{0.196975in}{0.947813in}}{\pgfqpoint{0.207915in}{0.958752in}}%
\pgfpathcurveto{\pgfqpoint{0.218854in}{0.969691in}}{\pgfqpoint{0.225000in}{0.984530in}}{\pgfqpoint{0.225000in}{1.000000in}}%
\pgfpathcurveto{\pgfqpoint{0.225000in}{1.015470in}}{\pgfqpoint{0.218854in}{1.030309in}}{\pgfqpoint{0.207915in}{1.041248in}}%
\pgfpathcurveto{\pgfqpoint{0.196975in}{1.052187in}}{\pgfqpoint{0.182137in}{1.058333in}}{\pgfqpoint{0.166667in}{1.058333in}}%
\pgfpathcurveto{\pgfqpoint{0.151196in}{1.058333in}}{\pgfqpoint{0.136358in}{1.052187in}}{\pgfqpoint{0.125419in}{1.041248in}}%
\pgfpathcurveto{\pgfqpoint{0.114480in}{1.030309in}}{\pgfqpoint{0.108333in}{1.015470in}}{\pgfqpoint{0.108333in}{1.000000in}}%
\pgfpathcurveto{\pgfqpoint{0.108333in}{0.984530in}}{\pgfqpoint{0.114480in}{0.969691in}}{\pgfqpoint{0.125419in}{0.958752in}}%
\pgfpathcurveto{\pgfqpoint{0.136358in}{0.947813in}}{\pgfqpoint{0.151196in}{0.941667in}}{\pgfqpoint{0.166667in}{0.941667in}}%
\pgfpathclose%
\pgfpathmoveto{\pgfqpoint{0.166667in}{0.947500in}}%
\pgfpathcurveto{\pgfqpoint{0.166667in}{0.947500in}}{\pgfqpoint{0.152744in}{0.947500in}}{\pgfqpoint{0.139389in}{0.953032in}}%
\pgfpathcurveto{\pgfqpoint{0.129544in}{0.962877in}}{\pgfqpoint{0.119698in}{0.972722in}}{\pgfqpoint{0.114167in}{0.986077in}}%
\pgfpathcurveto{\pgfqpoint{0.114167in}{1.000000in}}{\pgfqpoint{0.114167in}{1.013923in}}{\pgfqpoint{0.119698in}{1.027278in}}%
\pgfpathcurveto{\pgfqpoint{0.129544in}{1.037123in}}{\pgfqpoint{0.139389in}{1.046968in}}{\pgfqpoint{0.152744in}{1.052500in}}%
\pgfpathcurveto{\pgfqpoint{0.166667in}{1.052500in}}{\pgfqpoint{0.180590in}{1.052500in}}{\pgfqpoint{0.193945in}{1.046968in}}%
\pgfpathcurveto{\pgfqpoint{0.203790in}{1.037123in}}{\pgfqpoint{0.213635in}{1.027278in}}{\pgfqpoint{0.219167in}{1.013923in}}%
\pgfpathcurveto{\pgfqpoint{0.219167in}{1.000000in}}{\pgfqpoint{0.219167in}{0.986077in}}{\pgfqpoint{0.213635in}{0.972722in}}%
\pgfpathcurveto{\pgfqpoint{0.203790in}{0.962877in}}{\pgfqpoint{0.193945in}{0.953032in}}{\pgfqpoint{0.180590in}{0.947500in}}%
\pgfpathclose%
\pgfpathmoveto{\pgfqpoint{0.333333in}{0.941667in}}%
\pgfpathcurveto{\pgfqpoint{0.348804in}{0.941667in}}{\pgfqpoint{0.363642in}{0.947813in}}{\pgfqpoint{0.374581in}{0.958752in}}%
\pgfpathcurveto{\pgfqpoint{0.385520in}{0.969691in}}{\pgfqpoint{0.391667in}{0.984530in}}{\pgfqpoint{0.391667in}{1.000000in}}%
\pgfpathcurveto{\pgfqpoint{0.391667in}{1.015470in}}{\pgfqpoint{0.385520in}{1.030309in}}{\pgfqpoint{0.374581in}{1.041248in}}%
\pgfpathcurveto{\pgfqpoint{0.363642in}{1.052187in}}{\pgfqpoint{0.348804in}{1.058333in}}{\pgfqpoint{0.333333in}{1.058333in}}%
\pgfpathcurveto{\pgfqpoint{0.317863in}{1.058333in}}{\pgfqpoint{0.303025in}{1.052187in}}{\pgfqpoint{0.292085in}{1.041248in}}%
\pgfpathcurveto{\pgfqpoint{0.281146in}{1.030309in}}{\pgfqpoint{0.275000in}{1.015470in}}{\pgfqpoint{0.275000in}{1.000000in}}%
\pgfpathcurveto{\pgfqpoint{0.275000in}{0.984530in}}{\pgfqpoint{0.281146in}{0.969691in}}{\pgfqpoint{0.292085in}{0.958752in}}%
\pgfpathcurveto{\pgfqpoint{0.303025in}{0.947813in}}{\pgfqpoint{0.317863in}{0.941667in}}{\pgfqpoint{0.333333in}{0.941667in}}%
\pgfpathclose%
\pgfpathmoveto{\pgfqpoint{0.333333in}{0.947500in}}%
\pgfpathcurveto{\pgfqpoint{0.333333in}{0.947500in}}{\pgfqpoint{0.319410in}{0.947500in}}{\pgfqpoint{0.306055in}{0.953032in}}%
\pgfpathcurveto{\pgfqpoint{0.296210in}{0.962877in}}{\pgfqpoint{0.286365in}{0.972722in}}{\pgfqpoint{0.280833in}{0.986077in}}%
\pgfpathcurveto{\pgfqpoint{0.280833in}{1.000000in}}{\pgfqpoint{0.280833in}{1.013923in}}{\pgfqpoint{0.286365in}{1.027278in}}%
\pgfpathcurveto{\pgfqpoint{0.296210in}{1.037123in}}{\pgfqpoint{0.306055in}{1.046968in}}{\pgfqpoint{0.319410in}{1.052500in}}%
\pgfpathcurveto{\pgfqpoint{0.333333in}{1.052500in}}{\pgfqpoint{0.347256in}{1.052500in}}{\pgfqpoint{0.360611in}{1.046968in}}%
\pgfpathcurveto{\pgfqpoint{0.370456in}{1.037123in}}{\pgfqpoint{0.380302in}{1.027278in}}{\pgfqpoint{0.385833in}{1.013923in}}%
\pgfpathcurveto{\pgfqpoint{0.385833in}{1.000000in}}{\pgfqpoint{0.385833in}{0.986077in}}{\pgfqpoint{0.380302in}{0.972722in}}%
\pgfpathcurveto{\pgfqpoint{0.370456in}{0.962877in}}{\pgfqpoint{0.360611in}{0.953032in}}{\pgfqpoint{0.347256in}{0.947500in}}%
\pgfpathclose%
\pgfpathmoveto{\pgfqpoint{0.500000in}{0.941667in}}%
\pgfpathcurveto{\pgfqpoint{0.515470in}{0.941667in}}{\pgfqpoint{0.530309in}{0.947813in}}{\pgfqpoint{0.541248in}{0.958752in}}%
\pgfpathcurveto{\pgfqpoint{0.552187in}{0.969691in}}{\pgfqpoint{0.558333in}{0.984530in}}{\pgfqpoint{0.558333in}{1.000000in}}%
\pgfpathcurveto{\pgfqpoint{0.558333in}{1.015470in}}{\pgfqpoint{0.552187in}{1.030309in}}{\pgfqpoint{0.541248in}{1.041248in}}%
\pgfpathcurveto{\pgfqpoint{0.530309in}{1.052187in}}{\pgfqpoint{0.515470in}{1.058333in}}{\pgfqpoint{0.500000in}{1.058333in}}%
\pgfpathcurveto{\pgfqpoint{0.484530in}{1.058333in}}{\pgfqpoint{0.469691in}{1.052187in}}{\pgfqpoint{0.458752in}{1.041248in}}%
\pgfpathcurveto{\pgfqpoint{0.447813in}{1.030309in}}{\pgfqpoint{0.441667in}{1.015470in}}{\pgfqpoint{0.441667in}{1.000000in}}%
\pgfpathcurveto{\pgfqpoint{0.441667in}{0.984530in}}{\pgfqpoint{0.447813in}{0.969691in}}{\pgfqpoint{0.458752in}{0.958752in}}%
\pgfpathcurveto{\pgfqpoint{0.469691in}{0.947813in}}{\pgfqpoint{0.484530in}{0.941667in}}{\pgfqpoint{0.500000in}{0.941667in}}%
\pgfpathclose%
\pgfpathmoveto{\pgfqpoint{0.500000in}{0.947500in}}%
\pgfpathcurveto{\pgfqpoint{0.500000in}{0.947500in}}{\pgfqpoint{0.486077in}{0.947500in}}{\pgfqpoint{0.472722in}{0.953032in}}%
\pgfpathcurveto{\pgfqpoint{0.462877in}{0.962877in}}{\pgfqpoint{0.453032in}{0.972722in}}{\pgfqpoint{0.447500in}{0.986077in}}%
\pgfpathcurveto{\pgfqpoint{0.447500in}{1.000000in}}{\pgfqpoint{0.447500in}{1.013923in}}{\pgfqpoint{0.453032in}{1.027278in}}%
\pgfpathcurveto{\pgfqpoint{0.462877in}{1.037123in}}{\pgfqpoint{0.472722in}{1.046968in}}{\pgfqpoint{0.486077in}{1.052500in}}%
\pgfpathcurveto{\pgfqpoint{0.500000in}{1.052500in}}{\pgfqpoint{0.513923in}{1.052500in}}{\pgfqpoint{0.527278in}{1.046968in}}%
\pgfpathcurveto{\pgfqpoint{0.537123in}{1.037123in}}{\pgfqpoint{0.546968in}{1.027278in}}{\pgfqpoint{0.552500in}{1.013923in}}%
\pgfpathcurveto{\pgfqpoint{0.552500in}{1.000000in}}{\pgfqpoint{0.552500in}{0.986077in}}{\pgfqpoint{0.546968in}{0.972722in}}%
\pgfpathcurveto{\pgfqpoint{0.537123in}{0.962877in}}{\pgfqpoint{0.527278in}{0.953032in}}{\pgfqpoint{0.513923in}{0.947500in}}%
\pgfpathclose%
\pgfpathmoveto{\pgfqpoint{0.666667in}{0.941667in}}%
\pgfpathcurveto{\pgfqpoint{0.682137in}{0.941667in}}{\pgfqpoint{0.696975in}{0.947813in}}{\pgfqpoint{0.707915in}{0.958752in}}%
\pgfpathcurveto{\pgfqpoint{0.718854in}{0.969691in}}{\pgfqpoint{0.725000in}{0.984530in}}{\pgfqpoint{0.725000in}{1.000000in}}%
\pgfpathcurveto{\pgfqpoint{0.725000in}{1.015470in}}{\pgfqpoint{0.718854in}{1.030309in}}{\pgfqpoint{0.707915in}{1.041248in}}%
\pgfpathcurveto{\pgfqpoint{0.696975in}{1.052187in}}{\pgfqpoint{0.682137in}{1.058333in}}{\pgfqpoint{0.666667in}{1.058333in}}%
\pgfpathcurveto{\pgfqpoint{0.651196in}{1.058333in}}{\pgfqpoint{0.636358in}{1.052187in}}{\pgfqpoint{0.625419in}{1.041248in}}%
\pgfpathcurveto{\pgfqpoint{0.614480in}{1.030309in}}{\pgfqpoint{0.608333in}{1.015470in}}{\pgfqpoint{0.608333in}{1.000000in}}%
\pgfpathcurveto{\pgfqpoint{0.608333in}{0.984530in}}{\pgfqpoint{0.614480in}{0.969691in}}{\pgfqpoint{0.625419in}{0.958752in}}%
\pgfpathcurveto{\pgfqpoint{0.636358in}{0.947813in}}{\pgfqpoint{0.651196in}{0.941667in}}{\pgfqpoint{0.666667in}{0.941667in}}%
\pgfpathclose%
\pgfpathmoveto{\pgfqpoint{0.666667in}{0.947500in}}%
\pgfpathcurveto{\pgfqpoint{0.666667in}{0.947500in}}{\pgfqpoint{0.652744in}{0.947500in}}{\pgfqpoint{0.639389in}{0.953032in}}%
\pgfpathcurveto{\pgfqpoint{0.629544in}{0.962877in}}{\pgfqpoint{0.619698in}{0.972722in}}{\pgfqpoint{0.614167in}{0.986077in}}%
\pgfpathcurveto{\pgfqpoint{0.614167in}{1.000000in}}{\pgfqpoint{0.614167in}{1.013923in}}{\pgfqpoint{0.619698in}{1.027278in}}%
\pgfpathcurveto{\pgfqpoint{0.629544in}{1.037123in}}{\pgfqpoint{0.639389in}{1.046968in}}{\pgfqpoint{0.652744in}{1.052500in}}%
\pgfpathcurveto{\pgfqpoint{0.666667in}{1.052500in}}{\pgfqpoint{0.680590in}{1.052500in}}{\pgfqpoint{0.693945in}{1.046968in}}%
\pgfpathcurveto{\pgfqpoint{0.703790in}{1.037123in}}{\pgfqpoint{0.713635in}{1.027278in}}{\pgfqpoint{0.719167in}{1.013923in}}%
\pgfpathcurveto{\pgfqpoint{0.719167in}{1.000000in}}{\pgfqpoint{0.719167in}{0.986077in}}{\pgfqpoint{0.713635in}{0.972722in}}%
\pgfpathcurveto{\pgfqpoint{0.703790in}{0.962877in}}{\pgfqpoint{0.693945in}{0.953032in}}{\pgfqpoint{0.680590in}{0.947500in}}%
\pgfpathclose%
\pgfpathmoveto{\pgfqpoint{0.833333in}{0.941667in}}%
\pgfpathcurveto{\pgfqpoint{0.848804in}{0.941667in}}{\pgfqpoint{0.863642in}{0.947813in}}{\pgfqpoint{0.874581in}{0.958752in}}%
\pgfpathcurveto{\pgfqpoint{0.885520in}{0.969691in}}{\pgfqpoint{0.891667in}{0.984530in}}{\pgfqpoint{0.891667in}{1.000000in}}%
\pgfpathcurveto{\pgfqpoint{0.891667in}{1.015470in}}{\pgfqpoint{0.885520in}{1.030309in}}{\pgfqpoint{0.874581in}{1.041248in}}%
\pgfpathcurveto{\pgfqpoint{0.863642in}{1.052187in}}{\pgfqpoint{0.848804in}{1.058333in}}{\pgfqpoint{0.833333in}{1.058333in}}%
\pgfpathcurveto{\pgfqpoint{0.817863in}{1.058333in}}{\pgfqpoint{0.803025in}{1.052187in}}{\pgfqpoint{0.792085in}{1.041248in}}%
\pgfpathcurveto{\pgfqpoint{0.781146in}{1.030309in}}{\pgfqpoint{0.775000in}{1.015470in}}{\pgfqpoint{0.775000in}{1.000000in}}%
\pgfpathcurveto{\pgfqpoint{0.775000in}{0.984530in}}{\pgfqpoint{0.781146in}{0.969691in}}{\pgfqpoint{0.792085in}{0.958752in}}%
\pgfpathcurveto{\pgfqpoint{0.803025in}{0.947813in}}{\pgfqpoint{0.817863in}{0.941667in}}{\pgfqpoint{0.833333in}{0.941667in}}%
\pgfpathclose%
\pgfpathmoveto{\pgfqpoint{0.833333in}{0.947500in}}%
\pgfpathcurveto{\pgfqpoint{0.833333in}{0.947500in}}{\pgfqpoint{0.819410in}{0.947500in}}{\pgfqpoint{0.806055in}{0.953032in}}%
\pgfpathcurveto{\pgfqpoint{0.796210in}{0.962877in}}{\pgfqpoint{0.786365in}{0.972722in}}{\pgfqpoint{0.780833in}{0.986077in}}%
\pgfpathcurveto{\pgfqpoint{0.780833in}{1.000000in}}{\pgfqpoint{0.780833in}{1.013923in}}{\pgfqpoint{0.786365in}{1.027278in}}%
\pgfpathcurveto{\pgfqpoint{0.796210in}{1.037123in}}{\pgfqpoint{0.806055in}{1.046968in}}{\pgfqpoint{0.819410in}{1.052500in}}%
\pgfpathcurveto{\pgfqpoint{0.833333in}{1.052500in}}{\pgfqpoint{0.847256in}{1.052500in}}{\pgfqpoint{0.860611in}{1.046968in}}%
\pgfpathcurveto{\pgfqpoint{0.870456in}{1.037123in}}{\pgfqpoint{0.880302in}{1.027278in}}{\pgfqpoint{0.885833in}{1.013923in}}%
\pgfpathcurveto{\pgfqpoint{0.885833in}{1.000000in}}{\pgfqpoint{0.885833in}{0.986077in}}{\pgfqpoint{0.880302in}{0.972722in}}%
\pgfpathcurveto{\pgfqpoint{0.870456in}{0.962877in}}{\pgfqpoint{0.860611in}{0.953032in}}{\pgfqpoint{0.847256in}{0.947500in}}%
\pgfpathclose%
\pgfpathmoveto{\pgfqpoint{1.000000in}{0.941667in}}%
\pgfpathcurveto{\pgfqpoint{1.015470in}{0.941667in}}{\pgfqpoint{1.030309in}{0.947813in}}{\pgfqpoint{1.041248in}{0.958752in}}%
\pgfpathcurveto{\pgfqpoint{1.052187in}{0.969691in}}{\pgfqpoint{1.058333in}{0.984530in}}{\pgfqpoint{1.058333in}{1.000000in}}%
\pgfpathcurveto{\pgfqpoint{1.058333in}{1.015470in}}{\pgfqpoint{1.052187in}{1.030309in}}{\pgfqpoint{1.041248in}{1.041248in}}%
\pgfpathcurveto{\pgfqpoint{1.030309in}{1.052187in}}{\pgfqpoint{1.015470in}{1.058333in}}{\pgfqpoint{1.000000in}{1.058333in}}%
\pgfpathcurveto{\pgfqpoint{0.984530in}{1.058333in}}{\pgfqpoint{0.969691in}{1.052187in}}{\pgfqpoint{0.958752in}{1.041248in}}%
\pgfpathcurveto{\pgfqpoint{0.947813in}{1.030309in}}{\pgfqpoint{0.941667in}{1.015470in}}{\pgfqpoint{0.941667in}{1.000000in}}%
\pgfpathcurveto{\pgfqpoint{0.941667in}{0.984530in}}{\pgfqpoint{0.947813in}{0.969691in}}{\pgfqpoint{0.958752in}{0.958752in}}%
\pgfpathcurveto{\pgfqpoint{0.969691in}{0.947813in}}{\pgfqpoint{0.984530in}{0.941667in}}{\pgfqpoint{1.000000in}{0.941667in}}%
\pgfpathclose%
\pgfpathmoveto{\pgfqpoint{1.000000in}{0.947500in}}%
\pgfpathcurveto{\pgfqpoint{1.000000in}{0.947500in}}{\pgfqpoint{0.986077in}{0.947500in}}{\pgfqpoint{0.972722in}{0.953032in}}%
\pgfpathcurveto{\pgfqpoint{0.962877in}{0.962877in}}{\pgfqpoint{0.953032in}{0.972722in}}{\pgfqpoint{0.947500in}{0.986077in}}%
\pgfpathcurveto{\pgfqpoint{0.947500in}{1.000000in}}{\pgfqpoint{0.947500in}{1.013923in}}{\pgfqpoint{0.953032in}{1.027278in}}%
\pgfpathcurveto{\pgfqpoint{0.962877in}{1.037123in}}{\pgfqpoint{0.972722in}{1.046968in}}{\pgfqpoint{0.986077in}{1.052500in}}%
\pgfpathcurveto{\pgfqpoint{1.000000in}{1.052500in}}{\pgfqpoint{1.013923in}{1.052500in}}{\pgfqpoint{1.027278in}{1.046968in}}%
\pgfpathcurveto{\pgfqpoint{1.037123in}{1.037123in}}{\pgfqpoint{1.046968in}{1.027278in}}{\pgfqpoint{1.052500in}{1.013923in}}%
\pgfpathcurveto{\pgfqpoint{1.052500in}{1.000000in}}{\pgfqpoint{1.052500in}{0.986077in}}{\pgfqpoint{1.046968in}{0.972722in}}%
\pgfpathcurveto{\pgfqpoint{1.037123in}{0.962877in}}{\pgfqpoint{1.027278in}{0.953032in}}{\pgfqpoint{1.013923in}{0.947500in}}%
\pgfpathclose%
\pgfusepath{stroke}%
\end{pgfscope}%
}%
\pgfsys@transformshift{1.478174in}{5.779271in}%
\pgfsys@useobject{currentpattern}{}%
\pgfsys@transformshift{1in}{0in}%
\pgfsys@transformshift{-1in}{0in}%
\pgfsys@transformshift{0in}{1in}%
\pgfsys@useobject{currentpattern}{}%
\pgfsys@transformshift{1in}{0in}%
\pgfsys@transformshift{-1in}{0in}%
\pgfsys@transformshift{0in}{1in}%
\end{pgfscope}%
\begin{pgfscope}%
\pgfpathrectangle{\pgfqpoint{1.090674in}{0.637495in}}{\pgfqpoint{9.300000in}{9.060000in}}%
\pgfusepath{clip}%
\pgfsetbuttcap%
\pgfsetmiterjoin%
\definecolor{currentfill}{rgb}{0.890196,0.466667,0.760784}%
\pgfsetfillcolor{currentfill}%
\pgfsetfillopacity{0.990000}%
\pgfsetlinewidth{0.000000pt}%
\definecolor{currentstroke}{rgb}{0.000000,0.000000,0.000000}%
\pgfsetstrokecolor{currentstroke}%
\pgfsetstrokeopacity{0.990000}%
\pgfsetdash{}{0pt}%
\pgfpathmoveto{\pgfqpoint{3.028174in}{6.036360in}}%
\pgfpathlineto{\pgfqpoint{3.803174in}{6.036360in}}%
\pgfpathlineto{\pgfqpoint{3.803174in}{7.885495in}}%
\pgfpathlineto{\pgfqpoint{3.028174in}{7.885495in}}%
\pgfpathclose%
\pgfusepath{fill}%
\end{pgfscope}%
\begin{pgfscope}%
\pgfsetbuttcap%
\pgfsetmiterjoin%
\definecolor{currentfill}{rgb}{0.890196,0.466667,0.760784}%
\pgfsetfillcolor{currentfill}%
\pgfsetfillopacity{0.990000}%
\pgfsetlinewidth{0.000000pt}%
\definecolor{currentstroke}{rgb}{0.000000,0.000000,0.000000}%
\pgfsetstrokecolor{currentstroke}%
\pgfsetstrokeopacity{0.990000}%
\pgfsetdash{}{0pt}%
\pgfpathrectangle{\pgfqpoint{1.090674in}{0.637495in}}{\pgfqpoint{9.300000in}{9.060000in}}%
\pgfusepath{clip}%
\pgfpathmoveto{\pgfqpoint{3.028174in}{6.036360in}}%
\pgfpathlineto{\pgfqpoint{3.803174in}{6.036360in}}%
\pgfpathlineto{\pgfqpoint{3.803174in}{7.885495in}}%
\pgfpathlineto{\pgfqpoint{3.028174in}{7.885495in}}%
\pgfpathclose%
\pgfusepath{clip}%
\pgfsys@defobject{currentpattern}{\pgfqpoint{0in}{0in}}{\pgfqpoint{1in}{1in}}{%
\begin{pgfscope}%
\pgfpathrectangle{\pgfqpoint{0in}{0in}}{\pgfqpoint{1in}{1in}}%
\pgfusepath{clip}%
\pgfpathmoveto{\pgfqpoint{0.000000in}{-0.058333in}}%
\pgfpathcurveto{\pgfqpoint{0.015470in}{-0.058333in}}{\pgfqpoint{0.030309in}{-0.052187in}}{\pgfqpoint{0.041248in}{-0.041248in}}%
\pgfpathcurveto{\pgfqpoint{0.052187in}{-0.030309in}}{\pgfqpoint{0.058333in}{-0.015470in}}{\pgfqpoint{0.058333in}{0.000000in}}%
\pgfpathcurveto{\pgfqpoint{0.058333in}{0.015470in}}{\pgfqpoint{0.052187in}{0.030309in}}{\pgfqpoint{0.041248in}{0.041248in}}%
\pgfpathcurveto{\pgfqpoint{0.030309in}{0.052187in}}{\pgfqpoint{0.015470in}{0.058333in}}{\pgfqpoint{0.000000in}{0.058333in}}%
\pgfpathcurveto{\pgfqpoint{-0.015470in}{0.058333in}}{\pgfqpoint{-0.030309in}{0.052187in}}{\pgfqpoint{-0.041248in}{0.041248in}}%
\pgfpathcurveto{\pgfqpoint{-0.052187in}{0.030309in}}{\pgfqpoint{-0.058333in}{0.015470in}}{\pgfqpoint{-0.058333in}{0.000000in}}%
\pgfpathcurveto{\pgfqpoint{-0.058333in}{-0.015470in}}{\pgfqpoint{-0.052187in}{-0.030309in}}{\pgfqpoint{-0.041248in}{-0.041248in}}%
\pgfpathcurveto{\pgfqpoint{-0.030309in}{-0.052187in}}{\pgfqpoint{-0.015470in}{-0.058333in}}{\pgfqpoint{0.000000in}{-0.058333in}}%
\pgfpathclose%
\pgfpathmoveto{\pgfqpoint{0.000000in}{-0.052500in}}%
\pgfpathcurveto{\pgfqpoint{0.000000in}{-0.052500in}}{\pgfqpoint{-0.013923in}{-0.052500in}}{\pgfqpoint{-0.027278in}{-0.046968in}}%
\pgfpathcurveto{\pgfqpoint{-0.037123in}{-0.037123in}}{\pgfqpoint{-0.046968in}{-0.027278in}}{\pgfqpoint{-0.052500in}{-0.013923in}}%
\pgfpathcurveto{\pgfqpoint{-0.052500in}{0.000000in}}{\pgfqpoint{-0.052500in}{0.013923in}}{\pgfqpoint{-0.046968in}{0.027278in}}%
\pgfpathcurveto{\pgfqpoint{-0.037123in}{0.037123in}}{\pgfqpoint{-0.027278in}{0.046968in}}{\pgfqpoint{-0.013923in}{0.052500in}}%
\pgfpathcurveto{\pgfqpoint{0.000000in}{0.052500in}}{\pgfqpoint{0.013923in}{0.052500in}}{\pgfqpoint{0.027278in}{0.046968in}}%
\pgfpathcurveto{\pgfqpoint{0.037123in}{0.037123in}}{\pgfqpoint{0.046968in}{0.027278in}}{\pgfqpoint{0.052500in}{0.013923in}}%
\pgfpathcurveto{\pgfqpoint{0.052500in}{0.000000in}}{\pgfqpoint{0.052500in}{-0.013923in}}{\pgfqpoint{0.046968in}{-0.027278in}}%
\pgfpathcurveto{\pgfqpoint{0.037123in}{-0.037123in}}{\pgfqpoint{0.027278in}{-0.046968in}}{\pgfqpoint{0.013923in}{-0.052500in}}%
\pgfpathclose%
\pgfpathmoveto{\pgfqpoint{0.166667in}{-0.058333in}}%
\pgfpathcurveto{\pgfqpoint{0.182137in}{-0.058333in}}{\pgfqpoint{0.196975in}{-0.052187in}}{\pgfqpoint{0.207915in}{-0.041248in}}%
\pgfpathcurveto{\pgfqpoint{0.218854in}{-0.030309in}}{\pgfqpoint{0.225000in}{-0.015470in}}{\pgfqpoint{0.225000in}{0.000000in}}%
\pgfpathcurveto{\pgfqpoint{0.225000in}{0.015470in}}{\pgfqpoint{0.218854in}{0.030309in}}{\pgfqpoint{0.207915in}{0.041248in}}%
\pgfpathcurveto{\pgfqpoint{0.196975in}{0.052187in}}{\pgfqpoint{0.182137in}{0.058333in}}{\pgfqpoint{0.166667in}{0.058333in}}%
\pgfpathcurveto{\pgfqpoint{0.151196in}{0.058333in}}{\pgfqpoint{0.136358in}{0.052187in}}{\pgfqpoint{0.125419in}{0.041248in}}%
\pgfpathcurveto{\pgfqpoint{0.114480in}{0.030309in}}{\pgfqpoint{0.108333in}{0.015470in}}{\pgfqpoint{0.108333in}{0.000000in}}%
\pgfpathcurveto{\pgfqpoint{0.108333in}{-0.015470in}}{\pgfqpoint{0.114480in}{-0.030309in}}{\pgfqpoint{0.125419in}{-0.041248in}}%
\pgfpathcurveto{\pgfqpoint{0.136358in}{-0.052187in}}{\pgfqpoint{0.151196in}{-0.058333in}}{\pgfqpoint{0.166667in}{-0.058333in}}%
\pgfpathclose%
\pgfpathmoveto{\pgfqpoint{0.166667in}{-0.052500in}}%
\pgfpathcurveto{\pgfqpoint{0.166667in}{-0.052500in}}{\pgfqpoint{0.152744in}{-0.052500in}}{\pgfqpoint{0.139389in}{-0.046968in}}%
\pgfpathcurveto{\pgfqpoint{0.129544in}{-0.037123in}}{\pgfqpoint{0.119698in}{-0.027278in}}{\pgfqpoint{0.114167in}{-0.013923in}}%
\pgfpathcurveto{\pgfqpoint{0.114167in}{0.000000in}}{\pgfqpoint{0.114167in}{0.013923in}}{\pgfqpoint{0.119698in}{0.027278in}}%
\pgfpathcurveto{\pgfqpoint{0.129544in}{0.037123in}}{\pgfqpoint{0.139389in}{0.046968in}}{\pgfqpoint{0.152744in}{0.052500in}}%
\pgfpathcurveto{\pgfqpoint{0.166667in}{0.052500in}}{\pgfqpoint{0.180590in}{0.052500in}}{\pgfqpoint{0.193945in}{0.046968in}}%
\pgfpathcurveto{\pgfqpoint{0.203790in}{0.037123in}}{\pgfqpoint{0.213635in}{0.027278in}}{\pgfqpoint{0.219167in}{0.013923in}}%
\pgfpathcurveto{\pgfqpoint{0.219167in}{0.000000in}}{\pgfqpoint{0.219167in}{-0.013923in}}{\pgfqpoint{0.213635in}{-0.027278in}}%
\pgfpathcurveto{\pgfqpoint{0.203790in}{-0.037123in}}{\pgfqpoint{0.193945in}{-0.046968in}}{\pgfqpoint{0.180590in}{-0.052500in}}%
\pgfpathclose%
\pgfpathmoveto{\pgfqpoint{0.333333in}{-0.058333in}}%
\pgfpathcurveto{\pgfqpoint{0.348804in}{-0.058333in}}{\pgfqpoint{0.363642in}{-0.052187in}}{\pgfqpoint{0.374581in}{-0.041248in}}%
\pgfpathcurveto{\pgfqpoint{0.385520in}{-0.030309in}}{\pgfqpoint{0.391667in}{-0.015470in}}{\pgfqpoint{0.391667in}{0.000000in}}%
\pgfpathcurveto{\pgfqpoint{0.391667in}{0.015470in}}{\pgfqpoint{0.385520in}{0.030309in}}{\pgfqpoint{0.374581in}{0.041248in}}%
\pgfpathcurveto{\pgfqpoint{0.363642in}{0.052187in}}{\pgfqpoint{0.348804in}{0.058333in}}{\pgfqpoint{0.333333in}{0.058333in}}%
\pgfpathcurveto{\pgfqpoint{0.317863in}{0.058333in}}{\pgfqpoint{0.303025in}{0.052187in}}{\pgfqpoint{0.292085in}{0.041248in}}%
\pgfpathcurveto{\pgfqpoint{0.281146in}{0.030309in}}{\pgfqpoint{0.275000in}{0.015470in}}{\pgfqpoint{0.275000in}{0.000000in}}%
\pgfpathcurveto{\pgfqpoint{0.275000in}{-0.015470in}}{\pgfqpoint{0.281146in}{-0.030309in}}{\pgfqpoint{0.292085in}{-0.041248in}}%
\pgfpathcurveto{\pgfqpoint{0.303025in}{-0.052187in}}{\pgfqpoint{0.317863in}{-0.058333in}}{\pgfqpoint{0.333333in}{-0.058333in}}%
\pgfpathclose%
\pgfpathmoveto{\pgfqpoint{0.333333in}{-0.052500in}}%
\pgfpathcurveto{\pgfqpoint{0.333333in}{-0.052500in}}{\pgfqpoint{0.319410in}{-0.052500in}}{\pgfqpoint{0.306055in}{-0.046968in}}%
\pgfpathcurveto{\pgfqpoint{0.296210in}{-0.037123in}}{\pgfqpoint{0.286365in}{-0.027278in}}{\pgfqpoint{0.280833in}{-0.013923in}}%
\pgfpathcurveto{\pgfqpoint{0.280833in}{0.000000in}}{\pgfqpoint{0.280833in}{0.013923in}}{\pgfqpoint{0.286365in}{0.027278in}}%
\pgfpathcurveto{\pgfqpoint{0.296210in}{0.037123in}}{\pgfqpoint{0.306055in}{0.046968in}}{\pgfqpoint{0.319410in}{0.052500in}}%
\pgfpathcurveto{\pgfqpoint{0.333333in}{0.052500in}}{\pgfqpoint{0.347256in}{0.052500in}}{\pgfqpoint{0.360611in}{0.046968in}}%
\pgfpathcurveto{\pgfqpoint{0.370456in}{0.037123in}}{\pgfqpoint{0.380302in}{0.027278in}}{\pgfqpoint{0.385833in}{0.013923in}}%
\pgfpathcurveto{\pgfqpoint{0.385833in}{0.000000in}}{\pgfqpoint{0.385833in}{-0.013923in}}{\pgfqpoint{0.380302in}{-0.027278in}}%
\pgfpathcurveto{\pgfqpoint{0.370456in}{-0.037123in}}{\pgfqpoint{0.360611in}{-0.046968in}}{\pgfqpoint{0.347256in}{-0.052500in}}%
\pgfpathclose%
\pgfpathmoveto{\pgfqpoint{0.500000in}{-0.058333in}}%
\pgfpathcurveto{\pgfqpoint{0.515470in}{-0.058333in}}{\pgfqpoint{0.530309in}{-0.052187in}}{\pgfqpoint{0.541248in}{-0.041248in}}%
\pgfpathcurveto{\pgfqpoint{0.552187in}{-0.030309in}}{\pgfqpoint{0.558333in}{-0.015470in}}{\pgfqpoint{0.558333in}{0.000000in}}%
\pgfpathcurveto{\pgfqpoint{0.558333in}{0.015470in}}{\pgfqpoint{0.552187in}{0.030309in}}{\pgfqpoint{0.541248in}{0.041248in}}%
\pgfpathcurveto{\pgfqpoint{0.530309in}{0.052187in}}{\pgfqpoint{0.515470in}{0.058333in}}{\pgfqpoint{0.500000in}{0.058333in}}%
\pgfpathcurveto{\pgfqpoint{0.484530in}{0.058333in}}{\pgfqpoint{0.469691in}{0.052187in}}{\pgfqpoint{0.458752in}{0.041248in}}%
\pgfpathcurveto{\pgfqpoint{0.447813in}{0.030309in}}{\pgfqpoint{0.441667in}{0.015470in}}{\pgfqpoint{0.441667in}{0.000000in}}%
\pgfpathcurveto{\pgfqpoint{0.441667in}{-0.015470in}}{\pgfqpoint{0.447813in}{-0.030309in}}{\pgfqpoint{0.458752in}{-0.041248in}}%
\pgfpathcurveto{\pgfqpoint{0.469691in}{-0.052187in}}{\pgfqpoint{0.484530in}{-0.058333in}}{\pgfqpoint{0.500000in}{-0.058333in}}%
\pgfpathclose%
\pgfpathmoveto{\pgfqpoint{0.500000in}{-0.052500in}}%
\pgfpathcurveto{\pgfqpoint{0.500000in}{-0.052500in}}{\pgfqpoint{0.486077in}{-0.052500in}}{\pgfqpoint{0.472722in}{-0.046968in}}%
\pgfpathcurveto{\pgfqpoint{0.462877in}{-0.037123in}}{\pgfqpoint{0.453032in}{-0.027278in}}{\pgfqpoint{0.447500in}{-0.013923in}}%
\pgfpathcurveto{\pgfqpoint{0.447500in}{0.000000in}}{\pgfqpoint{0.447500in}{0.013923in}}{\pgfqpoint{0.453032in}{0.027278in}}%
\pgfpathcurveto{\pgfqpoint{0.462877in}{0.037123in}}{\pgfqpoint{0.472722in}{0.046968in}}{\pgfqpoint{0.486077in}{0.052500in}}%
\pgfpathcurveto{\pgfqpoint{0.500000in}{0.052500in}}{\pgfqpoint{0.513923in}{0.052500in}}{\pgfqpoint{0.527278in}{0.046968in}}%
\pgfpathcurveto{\pgfqpoint{0.537123in}{0.037123in}}{\pgfqpoint{0.546968in}{0.027278in}}{\pgfqpoint{0.552500in}{0.013923in}}%
\pgfpathcurveto{\pgfqpoint{0.552500in}{0.000000in}}{\pgfqpoint{0.552500in}{-0.013923in}}{\pgfqpoint{0.546968in}{-0.027278in}}%
\pgfpathcurveto{\pgfqpoint{0.537123in}{-0.037123in}}{\pgfqpoint{0.527278in}{-0.046968in}}{\pgfqpoint{0.513923in}{-0.052500in}}%
\pgfpathclose%
\pgfpathmoveto{\pgfqpoint{0.666667in}{-0.058333in}}%
\pgfpathcurveto{\pgfqpoint{0.682137in}{-0.058333in}}{\pgfqpoint{0.696975in}{-0.052187in}}{\pgfqpoint{0.707915in}{-0.041248in}}%
\pgfpathcurveto{\pgfqpoint{0.718854in}{-0.030309in}}{\pgfqpoint{0.725000in}{-0.015470in}}{\pgfqpoint{0.725000in}{0.000000in}}%
\pgfpathcurveto{\pgfqpoint{0.725000in}{0.015470in}}{\pgfqpoint{0.718854in}{0.030309in}}{\pgfqpoint{0.707915in}{0.041248in}}%
\pgfpathcurveto{\pgfqpoint{0.696975in}{0.052187in}}{\pgfqpoint{0.682137in}{0.058333in}}{\pgfqpoint{0.666667in}{0.058333in}}%
\pgfpathcurveto{\pgfqpoint{0.651196in}{0.058333in}}{\pgfqpoint{0.636358in}{0.052187in}}{\pgfqpoint{0.625419in}{0.041248in}}%
\pgfpathcurveto{\pgfqpoint{0.614480in}{0.030309in}}{\pgfqpoint{0.608333in}{0.015470in}}{\pgfqpoint{0.608333in}{0.000000in}}%
\pgfpathcurveto{\pgfqpoint{0.608333in}{-0.015470in}}{\pgfqpoint{0.614480in}{-0.030309in}}{\pgfqpoint{0.625419in}{-0.041248in}}%
\pgfpathcurveto{\pgfqpoint{0.636358in}{-0.052187in}}{\pgfqpoint{0.651196in}{-0.058333in}}{\pgfqpoint{0.666667in}{-0.058333in}}%
\pgfpathclose%
\pgfpathmoveto{\pgfqpoint{0.666667in}{-0.052500in}}%
\pgfpathcurveto{\pgfqpoint{0.666667in}{-0.052500in}}{\pgfqpoint{0.652744in}{-0.052500in}}{\pgfqpoint{0.639389in}{-0.046968in}}%
\pgfpathcurveto{\pgfqpoint{0.629544in}{-0.037123in}}{\pgfqpoint{0.619698in}{-0.027278in}}{\pgfqpoint{0.614167in}{-0.013923in}}%
\pgfpathcurveto{\pgfqpoint{0.614167in}{0.000000in}}{\pgfqpoint{0.614167in}{0.013923in}}{\pgfqpoint{0.619698in}{0.027278in}}%
\pgfpathcurveto{\pgfqpoint{0.629544in}{0.037123in}}{\pgfqpoint{0.639389in}{0.046968in}}{\pgfqpoint{0.652744in}{0.052500in}}%
\pgfpathcurveto{\pgfqpoint{0.666667in}{0.052500in}}{\pgfqpoint{0.680590in}{0.052500in}}{\pgfqpoint{0.693945in}{0.046968in}}%
\pgfpathcurveto{\pgfqpoint{0.703790in}{0.037123in}}{\pgfqpoint{0.713635in}{0.027278in}}{\pgfqpoint{0.719167in}{0.013923in}}%
\pgfpathcurveto{\pgfqpoint{0.719167in}{0.000000in}}{\pgfqpoint{0.719167in}{-0.013923in}}{\pgfqpoint{0.713635in}{-0.027278in}}%
\pgfpathcurveto{\pgfqpoint{0.703790in}{-0.037123in}}{\pgfqpoint{0.693945in}{-0.046968in}}{\pgfqpoint{0.680590in}{-0.052500in}}%
\pgfpathclose%
\pgfpathmoveto{\pgfqpoint{0.833333in}{-0.058333in}}%
\pgfpathcurveto{\pgfqpoint{0.848804in}{-0.058333in}}{\pgfqpoint{0.863642in}{-0.052187in}}{\pgfqpoint{0.874581in}{-0.041248in}}%
\pgfpathcurveto{\pgfqpoint{0.885520in}{-0.030309in}}{\pgfqpoint{0.891667in}{-0.015470in}}{\pgfqpoint{0.891667in}{0.000000in}}%
\pgfpathcurveto{\pgfqpoint{0.891667in}{0.015470in}}{\pgfqpoint{0.885520in}{0.030309in}}{\pgfqpoint{0.874581in}{0.041248in}}%
\pgfpathcurveto{\pgfqpoint{0.863642in}{0.052187in}}{\pgfqpoint{0.848804in}{0.058333in}}{\pgfqpoint{0.833333in}{0.058333in}}%
\pgfpathcurveto{\pgfqpoint{0.817863in}{0.058333in}}{\pgfqpoint{0.803025in}{0.052187in}}{\pgfqpoint{0.792085in}{0.041248in}}%
\pgfpathcurveto{\pgfqpoint{0.781146in}{0.030309in}}{\pgfqpoint{0.775000in}{0.015470in}}{\pgfqpoint{0.775000in}{0.000000in}}%
\pgfpathcurveto{\pgfqpoint{0.775000in}{-0.015470in}}{\pgfqpoint{0.781146in}{-0.030309in}}{\pgfqpoint{0.792085in}{-0.041248in}}%
\pgfpathcurveto{\pgfqpoint{0.803025in}{-0.052187in}}{\pgfqpoint{0.817863in}{-0.058333in}}{\pgfqpoint{0.833333in}{-0.058333in}}%
\pgfpathclose%
\pgfpathmoveto{\pgfqpoint{0.833333in}{-0.052500in}}%
\pgfpathcurveto{\pgfqpoint{0.833333in}{-0.052500in}}{\pgfqpoint{0.819410in}{-0.052500in}}{\pgfqpoint{0.806055in}{-0.046968in}}%
\pgfpathcurveto{\pgfqpoint{0.796210in}{-0.037123in}}{\pgfqpoint{0.786365in}{-0.027278in}}{\pgfqpoint{0.780833in}{-0.013923in}}%
\pgfpathcurveto{\pgfqpoint{0.780833in}{0.000000in}}{\pgfqpoint{0.780833in}{0.013923in}}{\pgfqpoint{0.786365in}{0.027278in}}%
\pgfpathcurveto{\pgfqpoint{0.796210in}{0.037123in}}{\pgfqpoint{0.806055in}{0.046968in}}{\pgfqpoint{0.819410in}{0.052500in}}%
\pgfpathcurveto{\pgfqpoint{0.833333in}{0.052500in}}{\pgfqpoint{0.847256in}{0.052500in}}{\pgfqpoint{0.860611in}{0.046968in}}%
\pgfpathcurveto{\pgfqpoint{0.870456in}{0.037123in}}{\pgfqpoint{0.880302in}{0.027278in}}{\pgfqpoint{0.885833in}{0.013923in}}%
\pgfpathcurveto{\pgfqpoint{0.885833in}{0.000000in}}{\pgfqpoint{0.885833in}{-0.013923in}}{\pgfqpoint{0.880302in}{-0.027278in}}%
\pgfpathcurveto{\pgfqpoint{0.870456in}{-0.037123in}}{\pgfqpoint{0.860611in}{-0.046968in}}{\pgfqpoint{0.847256in}{-0.052500in}}%
\pgfpathclose%
\pgfpathmoveto{\pgfqpoint{1.000000in}{-0.058333in}}%
\pgfpathcurveto{\pgfqpoint{1.015470in}{-0.058333in}}{\pgfqpoint{1.030309in}{-0.052187in}}{\pgfqpoint{1.041248in}{-0.041248in}}%
\pgfpathcurveto{\pgfqpoint{1.052187in}{-0.030309in}}{\pgfqpoint{1.058333in}{-0.015470in}}{\pgfqpoint{1.058333in}{0.000000in}}%
\pgfpathcurveto{\pgfqpoint{1.058333in}{0.015470in}}{\pgfqpoint{1.052187in}{0.030309in}}{\pgfqpoint{1.041248in}{0.041248in}}%
\pgfpathcurveto{\pgfqpoint{1.030309in}{0.052187in}}{\pgfqpoint{1.015470in}{0.058333in}}{\pgfqpoint{1.000000in}{0.058333in}}%
\pgfpathcurveto{\pgfqpoint{0.984530in}{0.058333in}}{\pgfqpoint{0.969691in}{0.052187in}}{\pgfqpoint{0.958752in}{0.041248in}}%
\pgfpathcurveto{\pgfqpoint{0.947813in}{0.030309in}}{\pgfqpoint{0.941667in}{0.015470in}}{\pgfqpoint{0.941667in}{0.000000in}}%
\pgfpathcurveto{\pgfqpoint{0.941667in}{-0.015470in}}{\pgfqpoint{0.947813in}{-0.030309in}}{\pgfqpoint{0.958752in}{-0.041248in}}%
\pgfpathcurveto{\pgfqpoint{0.969691in}{-0.052187in}}{\pgfqpoint{0.984530in}{-0.058333in}}{\pgfqpoint{1.000000in}{-0.058333in}}%
\pgfpathclose%
\pgfpathmoveto{\pgfqpoint{1.000000in}{-0.052500in}}%
\pgfpathcurveto{\pgfqpoint{1.000000in}{-0.052500in}}{\pgfqpoint{0.986077in}{-0.052500in}}{\pgfqpoint{0.972722in}{-0.046968in}}%
\pgfpathcurveto{\pgfqpoint{0.962877in}{-0.037123in}}{\pgfqpoint{0.953032in}{-0.027278in}}{\pgfqpoint{0.947500in}{-0.013923in}}%
\pgfpathcurveto{\pgfqpoint{0.947500in}{0.000000in}}{\pgfqpoint{0.947500in}{0.013923in}}{\pgfqpoint{0.953032in}{0.027278in}}%
\pgfpathcurveto{\pgfqpoint{0.962877in}{0.037123in}}{\pgfqpoint{0.972722in}{0.046968in}}{\pgfqpoint{0.986077in}{0.052500in}}%
\pgfpathcurveto{\pgfqpoint{1.000000in}{0.052500in}}{\pgfqpoint{1.013923in}{0.052500in}}{\pgfqpoint{1.027278in}{0.046968in}}%
\pgfpathcurveto{\pgfqpoint{1.037123in}{0.037123in}}{\pgfqpoint{1.046968in}{0.027278in}}{\pgfqpoint{1.052500in}{0.013923in}}%
\pgfpathcurveto{\pgfqpoint{1.052500in}{0.000000in}}{\pgfqpoint{1.052500in}{-0.013923in}}{\pgfqpoint{1.046968in}{-0.027278in}}%
\pgfpathcurveto{\pgfqpoint{1.037123in}{-0.037123in}}{\pgfqpoint{1.027278in}{-0.046968in}}{\pgfqpoint{1.013923in}{-0.052500in}}%
\pgfpathclose%
\pgfpathmoveto{\pgfqpoint{0.083333in}{0.108333in}}%
\pgfpathcurveto{\pgfqpoint{0.098804in}{0.108333in}}{\pgfqpoint{0.113642in}{0.114480in}}{\pgfqpoint{0.124581in}{0.125419in}}%
\pgfpathcurveto{\pgfqpoint{0.135520in}{0.136358in}}{\pgfqpoint{0.141667in}{0.151196in}}{\pgfqpoint{0.141667in}{0.166667in}}%
\pgfpathcurveto{\pgfqpoint{0.141667in}{0.182137in}}{\pgfqpoint{0.135520in}{0.196975in}}{\pgfqpoint{0.124581in}{0.207915in}}%
\pgfpathcurveto{\pgfqpoint{0.113642in}{0.218854in}}{\pgfqpoint{0.098804in}{0.225000in}}{\pgfqpoint{0.083333in}{0.225000in}}%
\pgfpathcurveto{\pgfqpoint{0.067863in}{0.225000in}}{\pgfqpoint{0.053025in}{0.218854in}}{\pgfqpoint{0.042085in}{0.207915in}}%
\pgfpathcurveto{\pgfqpoint{0.031146in}{0.196975in}}{\pgfqpoint{0.025000in}{0.182137in}}{\pgfqpoint{0.025000in}{0.166667in}}%
\pgfpathcurveto{\pgfqpoint{0.025000in}{0.151196in}}{\pgfqpoint{0.031146in}{0.136358in}}{\pgfqpoint{0.042085in}{0.125419in}}%
\pgfpathcurveto{\pgfqpoint{0.053025in}{0.114480in}}{\pgfqpoint{0.067863in}{0.108333in}}{\pgfqpoint{0.083333in}{0.108333in}}%
\pgfpathclose%
\pgfpathmoveto{\pgfqpoint{0.083333in}{0.114167in}}%
\pgfpathcurveto{\pgfqpoint{0.083333in}{0.114167in}}{\pgfqpoint{0.069410in}{0.114167in}}{\pgfqpoint{0.056055in}{0.119698in}}%
\pgfpathcurveto{\pgfqpoint{0.046210in}{0.129544in}}{\pgfqpoint{0.036365in}{0.139389in}}{\pgfqpoint{0.030833in}{0.152744in}}%
\pgfpathcurveto{\pgfqpoint{0.030833in}{0.166667in}}{\pgfqpoint{0.030833in}{0.180590in}}{\pgfqpoint{0.036365in}{0.193945in}}%
\pgfpathcurveto{\pgfqpoint{0.046210in}{0.203790in}}{\pgfqpoint{0.056055in}{0.213635in}}{\pgfqpoint{0.069410in}{0.219167in}}%
\pgfpathcurveto{\pgfqpoint{0.083333in}{0.219167in}}{\pgfqpoint{0.097256in}{0.219167in}}{\pgfqpoint{0.110611in}{0.213635in}}%
\pgfpathcurveto{\pgfqpoint{0.120456in}{0.203790in}}{\pgfqpoint{0.130302in}{0.193945in}}{\pgfqpoint{0.135833in}{0.180590in}}%
\pgfpathcurveto{\pgfqpoint{0.135833in}{0.166667in}}{\pgfqpoint{0.135833in}{0.152744in}}{\pgfqpoint{0.130302in}{0.139389in}}%
\pgfpathcurveto{\pgfqpoint{0.120456in}{0.129544in}}{\pgfqpoint{0.110611in}{0.119698in}}{\pgfqpoint{0.097256in}{0.114167in}}%
\pgfpathclose%
\pgfpathmoveto{\pgfqpoint{0.250000in}{0.108333in}}%
\pgfpathcurveto{\pgfqpoint{0.265470in}{0.108333in}}{\pgfqpoint{0.280309in}{0.114480in}}{\pgfqpoint{0.291248in}{0.125419in}}%
\pgfpathcurveto{\pgfqpoint{0.302187in}{0.136358in}}{\pgfqpoint{0.308333in}{0.151196in}}{\pgfqpoint{0.308333in}{0.166667in}}%
\pgfpathcurveto{\pgfqpoint{0.308333in}{0.182137in}}{\pgfqpoint{0.302187in}{0.196975in}}{\pgfqpoint{0.291248in}{0.207915in}}%
\pgfpathcurveto{\pgfqpoint{0.280309in}{0.218854in}}{\pgfqpoint{0.265470in}{0.225000in}}{\pgfqpoint{0.250000in}{0.225000in}}%
\pgfpathcurveto{\pgfqpoint{0.234530in}{0.225000in}}{\pgfqpoint{0.219691in}{0.218854in}}{\pgfqpoint{0.208752in}{0.207915in}}%
\pgfpathcurveto{\pgfqpoint{0.197813in}{0.196975in}}{\pgfqpoint{0.191667in}{0.182137in}}{\pgfqpoint{0.191667in}{0.166667in}}%
\pgfpathcurveto{\pgfqpoint{0.191667in}{0.151196in}}{\pgfqpoint{0.197813in}{0.136358in}}{\pgfqpoint{0.208752in}{0.125419in}}%
\pgfpathcurveto{\pgfqpoint{0.219691in}{0.114480in}}{\pgfqpoint{0.234530in}{0.108333in}}{\pgfqpoint{0.250000in}{0.108333in}}%
\pgfpathclose%
\pgfpathmoveto{\pgfqpoint{0.250000in}{0.114167in}}%
\pgfpathcurveto{\pgfqpoint{0.250000in}{0.114167in}}{\pgfqpoint{0.236077in}{0.114167in}}{\pgfqpoint{0.222722in}{0.119698in}}%
\pgfpathcurveto{\pgfqpoint{0.212877in}{0.129544in}}{\pgfqpoint{0.203032in}{0.139389in}}{\pgfqpoint{0.197500in}{0.152744in}}%
\pgfpathcurveto{\pgfqpoint{0.197500in}{0.166667in}}{\pgfqpoint{0.197500in}{0.180590in}}{\pgfqpoint{0.203032in}{0.193945in}}%
\pgfpathcurveto{\pgfqpoint{0.212877in}{0.203790in}}{\pgfqpoint{0.222722in}{0.213635in}}{\pgfqpoint{0.236077in}{0.219167in}}%
\pgfpathcurveto{\pgfqpoint{0.250000in}{0.219167in}}{\pgfqpoint{0.263923in}{0.219167in}}{\pgfqpoint{0.277278in}{0.213635in}}%
\pgfpathcurveto{\pgfqpoint{0.287123in}{0.203790in}}{\pgfqpoint{0.296968in}{0.193945in}}{\pgfqpoint{0.302500in}{0.180590in}}%
\pgfpathcurveto{\pgfqpoint{0.302500in}{0.166667in}}{\pgfqpoint{0.302500in}{0.152744in}}{\pgfqpoint{0.296968in}{0.139389in}}%
\pgfpathcurveto{\pgfqpoint{0.287123in}{0.129544in}}{\pgfqpoint{0.277278in}{0.119698in}}{\pgfqpoint{0.263923in}{0.114167in}}%
\pgfpathclose%
\pgfpathmoveto{\pgfqpoint{0.416667in}{0.108333in}}%
\pgfpathcurveto{\pgfqpoint{0.432137in}{0.108333in}}{\pgfqpoint{0.446975in}{0.114480in}}{\pgfqpoint{0.457915in}{0.125419in}}%
\pgfpathcurveto{\pgfqpoint{0.468854in}{0.136358in}}{\pgfqpoint{0.475000in}{0.151196in}}{\pgfqpoint{0.475000in}{0.166667in}}%
\pgfpathcurveto{\pgfqpoint{0.475000in}{0.182137in}}{\pgfqpoint{0.468854in}{0.196975in}}{\pgfqpoint{0.457915in}{0.207915in}}%
\pgfpathcurveto{\pgfqpoint{0.446975in}{0.218854in}}{\pgfqpoint{0.432137in}{0.225000in}}{\pgfqpoint{0.416667in}{0.225000in}}%
\pgfpathcurveto{\pgfqpoint{0.401196in}{0.225000in}}{\pgfqpoint{0.386358in}{0.218854in}}{\pgfqpoint{0.375419in}{0.207915in}}%
\pgfpathcurveto{\pgfqpoint{0.364480in}{0.196975in}}{\pgfqpoint{0.358333in}{0.182137in}}{\pgfqpoint{0.358333in}{0.166667in}}%
\pgfpathcurveto{\pgfqpoint{0.358333in}{0.151196in}}{\pgfqpoint{0.364480in}{0.136358in}}{\pgfqpoint{0.375419in}{0.125419in}}%
\pgfpathcurveto{\pgfqpoint{0.386358in}{0.114480in}}{\pgfqpoint{0.401196in}{0.108333in}}{\pgfqpoint{0.416667in}{0.108333in}}%
\pgfpathclose%
\pgfpathmoveto{\pgfqpoint{0.416667in}{0.114167in}}%
\pgfpathcurveto{\pgfqpoint{0.416667in}{0.114167in}}{\pgfqpoint{0.402744in}{0.114167in}}{\pgfqpoint{0.389389in}{0.119698in}}%
\pgfpathcurveto{\pgfqpoint{0.379544in}{0.129544in}}{\pgfqpoint{0.369698in}{0.139389in}}{\pgfqpoint{0.364167in}{0.152744in}}%
\pgfpathcurveto{\pgfqpoint{0.364167in}{0.166667in}}{\pgfqpoint{0.364167in}{0.180590in}}{\pgfqpoint{0.369698in}{0.193945in}}%
\pgfpathcurveto{\pgfqpoint{0.379544in}{0.203790in}}{\pgfqpoint{0.389389in}{0.213635in}}{\pgfqpoint{0.402744in}{0.219167in}}%
\pgfpathcurveto{\pgfqpoint{0.416667in}{0.219167in}}{\pgfqpoint{0.430590in}{0.219167in}}{\pgfqpoint{0.443945in}{0.213635in}}%
\pgfpathcurveto{\pgfqpoint{0.453790in}{0.203790in}}{\pgfqpoint{0.463635in}{0.193945in}}{\pgfqpoint{0.469167in}{0.180590in}}%
\pgfpathcurveto{\pgfqpoint{0.469167in}{0.166667in}}{\pgfqpoint{0.469167in}{0.152744in}}{\pgfqpoint{0.463635in}{0.139389in}}%
\pgfpathcurveto{\pgfqpoint{0.453790in}{0.129544in}}{\pgfqpoint{0.443945in}{0.119698in}}{\pgfqpoint{0.430590in}{0.114167in}}%
\pgfpathclose%
\pgfpathmoveto{\pgfqpoint{0.583333in}{0.108333in}}%
\pgfpathcurveto{\pgfqpoint{0.598804in}{0.108333in}}{\pgfqpoint{0.613642in}{0.114480in}}{\pgfqpoint{0.624581in}{0.125419in}}%
\pgfpathcurveto{\pgfqpoint{0.635520in}{0.136358in}}{\pgfqpoint{0.641667in}{0.151196in}}{\pgfqpoint{0.641667in}{0.166667in}}%
\pgfpathcurveto{\pgfqpoint{0.641667in}{0.182137in}}{\pgfqpoint{0.635520in}{0.196975in}}{\pgfqpoint{0.624581in}{0.207915in}}%
\pgfpathcurveto{\pgfqpoint{0.613642in}{0.218854in}}{\pgfqpoint{0.598804in}{0.225000in}}{\pgfqpoint{0.583333in}{0.225000in}}%
\pgfpathcurveto{\pgfqpoint{0.567863in}{0.225000in}}{\pgfqpoint{0.553025in}{0.218854in}}{\pgfqpoint{0.542085in}{0.207915in}}%
\pgfpathcurveto{\pgfqpoint{0.531146in}{0.196975in}}{\pgfqpoint{0.525000in}{0.182137in}}{\pgfqpoint{0.525000in}{0.166667in}}%
\pgfpathcurveto{\pgfqpoint{0.525000in}{0.151196in}}{\pgfqpoint{0.531146in}{0.136358in}}{\pgfqpoint{0.542085in}{0.125419in}}%
\pgfpathcurveto{\pgfqpoint{0.553025in}{0.114480in}}{\pgfqpoint{0.567863in}{0.108333in}}{\pgfqpoint{0.583333in}{0.108333in}}%
\pgfpathclose%
\pgfpathmoveto{\pgfqpoint{0.583333in}{0.114167in}}%
\pgfpathcurveto{\pgfqpoint{0.583333in}{0.114167in}}{\pgfqpoint{0.569410in}{0.114167in}}{\pgfqpoint{0.556055in}{0.119698in}}%
\pgfpathcurveto{\pgfqpoint{0.546210in}{0.129544in}}{\pgfqpoint{0.536365in}{0.139389in}}{\pgfqpoint{0.530833in}{0.152744in}}%
\pgfpathcurveto{\pgfqpoint{0.530833in}{0.166667in}}{\pgfqpoint{0.530833in}{0.180590in}}{\pgfqpoint{0.536365in}{0.193945in}}%
\pgfpathcurveto{\pgfqpoint{0.546210in}{0.203790in}}{\pgfqpoint{0.556055in}{0.213635in}}{\pgfqpoint{0.569410in}{0.219167in}}%
\pgfpathcurveto{\pgfqpoint{0.583333in}{0.219167in}}{\pgfqpoint{0.597256in}{0.219167in}}{\pgfqpoint{0.610611in}{0.213635in}}%
\pgfpathcurveto{\pgfqpoint{0.620456in}{0.203790in}}{\pgfqpoint{0.630302in}{0.193945in}}{\pgfqpoint{0.635833in}{0.180590in}}%
\pgfpathcurveto{\pgfqpoint{0.635833in}{0.166667in}}{\pgfqpoint{0.635833in}{0.152744in}}{\pgfqpoint{0.630302in}{0.139389in}}%
\pgfpathcurveto{\pgfqpoint{0.620456in}{0.129544in}}{\pgfqpoint{0.610611in}{0.119698in}}{\pgfqpoint{0.597256in}{0.114167in}}%
\pgfpathclose%
\pgfpathmoveto{\pgfqpoint{0.750000in}{0.108333in}}%
\pgfpathcurveto{\pgfqpoint{0.765470in}{0.108333in}}{\pgfqpoint{0.780309in}{0.114480in}}{\pgfqpoint{0.791248in}{0.125419in}}%
\pgfpathcurveto{\pgfqpoint{0.802187in}{0.136358in}}{\pgfqpoint{0.808333in}{0.151196in}}{\pgfqpoint{0.808333in}{0.166667in}}%
\pgfpathcurveto{\pgfqpoint{0.808333in}{0.182137in}}{\pgfqpoint{0.802187in}{0.196975in}}{\pgfqpoint{0.791248in}{0.207915in}}%
\pgfpathcurveto{\pgfqpoint{0.780309in}{0.218854in}}{\pgfqpoint{0.765470in}{0.225000in}}{\pgfqpoint{0.750000in}{0.225000in}}%
\pgfpathcurveto{\pgfqpoint{0.734530in}{0.225000in}}{\pgfqpoint{0.719691in}{0.218854in}}{\pgfqpoint{0.708752in}{0.207915in}}%
\pgfpathcurveto{\pgfqpoint{0.697813in}{0.196975in}}{\pgfqpoint{0.691667in}{0.182137in}}{\pgfqpoint{0.691667in}{0.166667in}}%
\pgfpathcurveto{\pgfqpoint{0.691667in}{0.151196in}}{\pgfqpoint{0.697813in}{0.136358in}}{\pgfqpoint{0.708752in}{0.125419in}}%
\pgfpathcurveto{\pgfqpoint{0.719691in}{0.114480in}}{\pgfqpoint{0.734530in}{0.108333in}}{\pgfqpoint{0.750000in}{0.108333in}}%
\pgfpathclose%
\pgfpathmoveto{\pgfqpoint{0.750000in}{0.114167in}}%
\pgfpathcurveto{\pgfqpoint{0.750000in}{0.114167in}}{\pgfqpoint{0.736077in}{0.114167in}}{\pgfqpoint{0.722722in}{0.119698in}}%
\pgfpathcurveto{\pgfqpoint{0.712877in}{0.129544in}}{\pgfqpoint{0.703032in}{0.139389in}}{\pgfqpoint{0.697500in}{0.152744in}}%
\pgfpathcurveto{\pgfqpoint{0.697500in}{0.166667in}}{\pgfqpoint{0.697500in}{0.180590in}}{\pgfqpoint{0.703032in}{0.193945in}}%
\pgfpathcurveto{\pgfqpoint{0.712877in}{0.203790in}}{\pgfqpoint{0.722722in}{0.213635in}}{\pgfqpoint{0.736077in}{0.219167in}}%
\pgfpathcurveto{\pgfqpoint{0.750000in}{0.219167in}}{\pgfqpoint{0.763923in}{0.219167in}}{\pgfqpoint{0.777278in}{0.213635in}}%
\pgfpathcurveto{\pgfqpoint{0.787123in}{0.203790in}}{\pgfqpoint{0.796968in}{0.193945in}}{\pgfqpoint{0.802500in}{0.180590in}}%
\pgfpathcurveto{\pgfqpoint{0.802500in}{0.166667in}}{\pgfqpoint{0.802500in}{0.152744in}}{\pgfqpoint{0.796968in}{0.139389in}}%
\pgfpathcurveto{\pgfqpoint{0.787123in}{0.129544in}}{\pgfqpoint{0.777278in}{0.119698in}}{\pgfqpoint{0.763923in}{0.114167in}}%
\pgfpathclose%
\pgfpathmoveto{\pgfqpoint{0.916667in}{0.108333in}}%
\pgfpathcurveto{\pgfqpoint{0.932137in}{0.108333in}}{\pgfqpoint{0.946975in}{0.114480in}}{\pgfqpoint{0.957915in}{0.125419in}}%
\pgfpathcurveto{\pgfqpoint{0.968854in}{0.136358in}}{\pgfqpoint{0.975000in}{0.151196in}}{\pgfqpoint{0.975000in}{0.166667in}}%
\pgfpathcurveto{\pgfqpoint{0.975000in}{0.182137in}}{\pgfqpoint{0.968854in}{0.196975in}}{\pgfqpoint{0.957915in}{0.207915in}}%
\pgfpathcurveto{\pgfqpoint{0.946975in}{0.218854in}}{\pgfqpoint{0.932137in}{0.225000in}}{\pgfqpoint{0.916667in}{0.225000in}}%
\pgfpathcurveto{\pgfqpoint{0.901196in}{0.225000in}}{\pgfqpoint{0.886358in}{0.218854in}}{\pgfqpoint{0.875419in}{0.207915in}}%
\pgfpathcurveto{\pgfqpoint{0.864480in}{0.196975in}}{\pgfqpoint{0.858333in}{0.182137in}}{\pgfqpoint{0.858333in}{0.166667in}}%
\pgfpathcurveto{\pgfqpoint{0.858333in}{0.151196in}}{\pgfqpoint{0.864480in}{0.136358in}}{\pgfqpoint{0.875419in}{0.125419in}}%
\pgfpathcurveto{\pgfqpoint{0.886358in}{0.114480in}}{\pgfqpoint{0.901196in}{0.108333in}}{\pgfqpoint{0.916667in}{0.108333in}}%
\pgfpathclose%
\pgfpathmoveto{\pgfqpoint{0.916667in}{0.114167in}}%
\pgfpathcurveto{\pgfqpoint{0.916667in}{0.114167in}}{\pgfqpoint{0.902744in}{0.114167in}}{\pgfqpoint{0.889389in}{0.119698in}}%
\pgfpathcurveto{\pgfqpoint{0.879544in}{0.129544in}}{\pgfqpoint{0.869698in}{0.139389in}}{\pgfqpoint{0.864167in}{0.152744in}}%
\pgfpathcurveto{\pgfqpoint{0.864167in}{0.166667in}}{\pgfqpoint{0.864167in}{0.180590in}}{\pgfqpoint{0.869698in}{0.193945in}}%
\pgfpathcurveto{\pgfqpoint{0.879544in}{0.203790in}}{\pgfqpoint{0.889389in}{0.213635in}}{\pgfqpoint{0.902744in}{0.219167in}}%
\pgfpathcurveto{\pgfqpoint{0.916667in}{0.219167in}}{\pgfqpoint{0.930590in}{0.219167in}}{\pgfqpoint{0.943945in}{0.213635in}}%
\pgfpathcurveto{\pgfqpoint{0.953790in}{0.203790in}}{\pgfqpoint{0.963635in}{0.193945in}}{\pgfqpoint{0.969167in}{0.180590in}}%
\pgfpathcurveto{\pgfqpoint{0.969167in}{0.166667in}}{\pgfqpoint{0.969167in}{0.152744in}}{\pgfqpoint{0.963635in}{0.139389in}}%
\pgfpathcurveto{\pgfqpoint{0.953790in}{0.129544in}}{\pgfqpoint{0.943945in}{0.119698in}}{\pgfqpoint{0.930590in}{0.114167in}}%
\pgfpathclose%
\pgfpathmoveto{\pgfqpoint{0.000000in}{0.275000in}}%
\pgfpathcurveto{\pgfqpoint{0.015470in}{0.275000in}}{\pgfqpoint{0.030309in}{0.281146in}}{\pgfqpoint{0.041248in}{0.292085in}}%
\pgfpathcurveto{\pgfqpoint{0.052187in}{0.303025in}}{\pgfqpoint{0.058333in}{0.317863in}}{\pgfqpoint{0.058333in}{0.333333in}}%
\pgfpathcurveto{\pgfqpoint{0.058333in}{0.348804in}}{\pgfqpoint{0.052187in}{0.363642in}}{\pgfqpoint{0.041248in}{0.374581in}}%
\pgfpathcurveto{\pgfqpoint{0.030309in}{0.385520in}}{\pgfqpoint{0.015470in}{0.391667in}}{\pgfqpoint{0.000000in}{0.391667in}}%
\pgfpathcurveto{\pgfqpoint{-0.015470in}{0.391667in}}{\pgfqpoint{-0.030309in}{0.385520in}}{\pgfqpoint{-0.041248in}{0.374581in}}%
\pgfpathcurveto{\pgfqpoint{-0.052187in}{0.363642in}}{\pgfqpoint{-0.058333in}{0.348804in}}{\pgfqpoint{-0.058333in}{0.333333in}}%
\pgfpathcurveto{\pgfqpoint{-0.058333in}{0.317863in}}{\pgfqpoint{-0.052187in}{0.303025in}}{\pgfqpoint{-0.041248in}{0.292085in}}%
\pgfpathcurveto{\pgfqpoint{-0.030309in}{0.281146in}}{\pgfqpoint{-0.015470in}{0.275000in}}{\pgfqpoint{0.000000in}{0.275000in}}%
\pgfpathclose%
\pgfpathmoveto{\pgfqpoint{0.000000in}{0.280833in}}%
\pgfpathcurveto{\pgfqpoint{0.000000in}{0.280833in}}{\pgfqpoint{-0.013923in}{0.280833in}}{\pgfqpoint{-0.027278in}{0.286365in}}%
\pgfpathcurveto{\pgfqpoint{-0.037123in}{0.296210in}}{\pgfqpoint{-0.046968in}{0.306055in}}{\pgfqpoint{-0.052500in}{0.319410in}}%
\pgfpathcurveto{\pgfqpoint{-0.052500in}{0.333333in}}{\pgfqpoint{-0.052500in}{0.347256in}}{\pgfqpoint{-0.046968in}{0.360611in}}%
\pgfpathcurveto{\pgfqpoint{-0.037123in}{0.370456in}}{\pgfqpoint{-0.027278in}{0.380302in}}{\pgfqpoint{-0.013923in}{0.385833in}}%
\pgfpathcurveto{\pgfqpoint{0.000000in}{0.385833in}}{\pgfqpoint{0.013923in}{0.385833in}}{\pgfqpoint{0.027278in}{0.380302in}}%
\pgfpathcurveto{\pgfqpoint{0.037123in}{0.370456in}}{\pgfqpoint{0.046968in}{0.360611in}}{\pgfqpoint{0.052500in}{0.347256in}}%
\pgfpathcurveto{\pgfqpoint{0.052500in}{0.333333in}}{\pgfqpoint{0.052500in}{0.319410in}}{\pgfqpoint{0.046968in}{0.306055in}}%
\pgfpathcurveto{\pgfqpoint{0.037123in}{0.296210in}}{\pgfqpoint{0.027278in}{0.286365in}}{\pgfqpoint{0.013923in}{0.280833in}}%
\pgfpathclose%
\pgfpathmoveto{\pgfqpoint{0.166667in}{0.275000in}}%
\pgfpathcurveto{\pgfqpoint{0.182137in}{0.275000in}}{\pgfqpoint{0.196975in}{0.281146in}}{\pgfqpoint{0.207915in}{0.292085in}}%
\pgfpathcurveto{\pgfqpoint{0.218854in}{0.303025in}}{\pgfqpoint{0.225000in}{0.317863in}}{\pgfqpoint{0.225000in}{0.333333in}}%
\pgfpathcurveto{\pgfqpoint{0.225000in}{0.348804in}}{\pgfqpoint{0.218854in}{0.363642in}}{\pgfqpoint{0.207915in}{0.374581in}}%
\pgfpathcurveto{\pgfqpoint{0.196975in}{0.385520in}}{\pgfqpoint{0.182137in}{0.391667in}}{\pgfqpoint{0.166667in}{0.391667in}}%
\pgfpathcurveto{\pgfqpoint{0.151196in}{0.391667in}}{\pgfqpoint{0.136358in}{0.385520in}}{\pgfqpoint{0.125419in}{0.374581in}}%
\pgfpathcurveto{\pgfqpoint{0.114480in}{0.363642in}}{\pgfqpoint{0.108333in}{0.348804in}}{\pgfqpoint{0.108333in}{0.333333in}}%
\pgfpathcurveto{\pgfqpoint{0.108333in}{0.317863in}}{\pgfqpoint{0.114480in}{0.303025in}}{\pgfqpoint{0.125419in}{0.292085in}}%
\pgfpathcurveto{\pgfqpoint{0.136358in}{0.281146in}}{\pgfqpoint{0.151196in}{0.275000in}}{\pgfqpoint{0.166667in}{0.275000in}}%
\pgfpathclose%
\pgfpathmoveto{\pgfqpoint{0.166667in}{0.280833in}}%
\pgfpathcurveto{\pgfqpoint{0.166667in}{0.280833in}}{\pgfqpoint{0.152744in}{0.280833in}}{\pgfqpoint{0.139389in}{0.286365in}}%
\pgfpathcurveto{\pgfqpoint{0.129544in}{0.296210in}}{\pgfqpoint{0.119698in}{0.306055in}}{\pgfqpoint{0.114167in}{0.319410in}}%
\pgfpathcurveto{\pgfqpoint{0.114167in}{0.333333in}}{\pgfqpoint{0.114167in}{0.347256in}}{\pgfqpoint{0.119698in}{0.360611in}}%
\pgfpathcurveto{\pgfqpoint{0.129544in}{0.370456in}}{\pgfqpoint{0.139389in}{0.380302in}}{\pgfqpoint{0.152744in}{0.385833in}}%
\pgfpathcurveto{\pgfqpoint{0.166667in}{0.385833in}}{\pgfqpoint{0.180590in}{0.385833in}}{\pgfqpoint{0.193945in}{0.380302in}}%
\pgfpathcurveto{\pgfqpoint{0.203790in}{0.370456in}}{\pgfqpoint{0.213635in}{0.360611in}}{\pgfqpoint{0.219167in}{0.347256in}}%
\pgfpathcurveto{\pgfqpoint{0.219167in}{0.333333in}}{\pgfqpoint{0.219167in}{0.319410in}}{\pgfqpoint{0.213635in}{0.306055in}}%
\pgfpathcurveto{\pgfqpoint{0.203790in}{0.296210in}}{\pgfqpoint{0.193945in}{0.286365in}}{\pgfqpoint{0.180590in}{0.280833in}}%
\pgfpathclose%
\pgfpathmoveto{\pgfqpoint{0.333333in}{0.275000in}}%
\pgfpathcurveto{\pgfqpoint{0.348804in}{0.275000in}}{\pgfqpoint{0.363642in}{0.281146in}}{\pgfqpoint{0.374581in}{0.292085in}}%
\pgfpathcurveto{\pgfqpoint{0.385520in}{0.303025in}}{\pgfqpoint{0.391667in}{0.317863in}}{\pgfqpoint{0.391667in}{0.333333in}}%
\pgfpathcurveto{\pgfqpoint{0.391667in}{0.348804in}}{\pgfqpoint{0.385520in}{0.363642in}}{\pgfqpoint{0.374581in}{0.374581in}}%
\pgfpathcurveto{\pgfqpoint{0.363642in}{0.385520in}}{\pgfqpoint{0.348804in}{0.391667in}}{\pgfqpoint{0.333333in}{0.391667in}}%
\pgfpathcurveto{\pgfqpoint{0.317863in}{0.391667in}}{\pgfqpoint{0.303025in}{0.385520in}}{\pgfqpoint{0.292085in}{0.374581in}}%
\pgfpathcurveto{\pgfqpoint{0.281146in}{0.363642in}}{\pgfqpoint{0.275000in}{0.348804in}}{\pgfqpoint{0.275000in}{0.333333in}}%
\pgfpathcurveto{\pgfqpoint{0.275000in}{0.317863in}}{\pgfqpoint{0.281146in}{0.303025in}}{\pgfqpoint{0.292085in}{0.292085in}}%
\pgfpathcurveto{\pgfqpoint{0.303025in}{0.281146in}}{\pgfqpoint{0.317863in}{0.275000in}}{\pgfqpoint{0.333333in}{0.275000in}}%
\pgfpathclose%
\pgfpathmoveto{\pgfqpoint{0.333333in}{0.280833in}}%
\pgfpathcurveto{\pgfqpoint{0.333333in}{0.280833in}}{\pgfqpoint{0.319410in}{0.280833in}}{\pgfqpoint{0.306055in}{0.286365in}}%
\pgfpathcurveto{\pgfqpoint{0.296210in}{0.296210in}}{\pgfqpoint{0.286365in}{0.306055in}}{\pgfqpoint{0.280833in}{0.319410in}}%
\pgfpathcurveto{\pgfqpoint{0.280833in}{0.333333in}}{\pgfqpoint{0.280833in}{0.347256in}}{\pgfqpoint{0.286365in}{0.360611in}}%
\pgfpathcurveto{\pgfqpoint{0.296210in}{0.370456in}}{\pgfqpoint{0.306055in}{0.380302in}}{\pgfqpoint{0.319410in}{0.385833in}}%
\pgfpathcurveto{\pgfqpoint{0.333333in}{0.385833in}}{\pgfqpoint{0.347256in}{0.385833in}}{\pgfqpoint{0.360611in}{0.380302in}}%
\pgfpathcurveto{\pgfqpoint{0.370456in}{0.370456in}}{\pgfqpoint{0.380302in}{0.360611in}}{\pgfqpoint{0.385833in}{0.347256in}}%
\pgfpathcurveto{\pgfqpoint{0.385833in}{0.333333in}}{\pgfqpoint{0.385833in}{0.319410in}}{\pgfqpoint{0.380302in}{0.306055in}}%
\pgfpathcurveto{\pgfqpoint{0.370456in}{0.296210in}}{\pgfqpoint{0.360611in}{0.286365in}}{\pgfqpoint{0.347256in}{0.280833in}}%
\pgfpathclose%
\pgfpathmoveto{\pgfqpoint{0.500000in}{0.275000in}}%
\pgfpathcurveto{\pgfqpoint{0.515470in}{0.275000in}}{\pgfqpoint{0.530309in}{0.281146in}}{\pgfqpoint{0.541248in}{0.292085in}}%
\pgfpathcurveto{\pgfqpoint{0.552187in}{0.303025in}}{\pgfqpoint{0.558333in}{0.317863in}}{\pgfqpoint{0.558333in}{0.333333in}}%
\pgfpathcurveto{\pgfqpoint{0.558333in}{0.348804in}}{\pgfqpoint{0.552187in}{0.363642in}}{\pgfqpoint{0.541248in}{0.374581in}}%
\pgfpathcurveto{\pgfqpoint{0.530309in}{0.385520in}}{\pgfqpoint{0.515470in}{0.391667in}}{\pgfqpoint{0.500000in}{0.391667in}}%
\pgfpathcurveto{\pgfqpoint{0.484530in}{0.391667in}}{\pgfqpoint{0.469691in}{0.385520in}}{\pgfqpoint{0.458752in}{0.374581in}}%
\pgfpathcurveto{\pgfqpoint{0.447813in}{0.363642in}}{\pgfqpoint{0.441667in}{0.348804in}}{\pgfqpoint{0.441667in}{0.333333in}}%
\pgfpathcurveto{\pgfqpoint{0.441667in}{0.317863in}}{\pgfqpoint{0.447813in}{0.303025in}}{\pgfqpoint{0.458752in}{0.292085in}}%
\pgfpathcurveto{\pgfqpoint{0.469691in}{0.281146in}}{\pgfqpoint{0.484530in}{0.275000in}}{\pgfqpoint{0.500000in}{0.275000in}}%
\pgfpathclose%
\pgfpathmoveto{\pgfqpoint{0.500000in}{0.280833in}}%
\pgfpathcurveto{\pgfqpoint{0.500000in}{0.280833in}}{\pgfqpoint{0.486077in}{0.280833in}}{\pgfqpoint{0.472722in}{0.286365in}}%
\pgfpathcurveto{\pgfqpoint{0.462877in}{0.296210in}}{\pgfqpoint{0.453032in}{0.306055in}}{\pgfqpoint{0.447500in}{0.319410in}}%
\pgfpathcurveto{\pgfqpoint{0.447500in}{0.333333in}}{\pgfqpoint{0.447500in}{0.347256in}}{\pgfqpoint{0.453032in}{0.360611in}}%
\pgfpathcurveto{\pgfqpoint{0.462877in}{0.370456in}}{\pgfqpoint{0.472722in}{0.380302in}}{\pgfqpoint{0.486077in}{0.385833in}}%
\pgfpathcurveto{\pgfqpoint{0.500000in}{0.385833in}}{\pgfqpoint{0.513923in}{0.385833in}}{\pgfqpoint{0.527278in}{0.380302in}}%
\pgfpathcurveto{\pgfqpoint{0.537123in}{0.370456in}}{\pgfqpoint{0.546968in}{0.360611in}}{\pgfqpoint{0.552500in}{0.347256in}}%
\pgfpathcurveto{\pgfqpoint{0.552500in}{0.333333in}}{\pgfqpoint{0.552500in}{0.319410in}}{\pgfqpoint{0.546968in}{0.306055in}}%
\pgfpathcurveto{\pgfqpoint{0.537123in}{0.296210in}}{\pgfqpoint{0.527278in}{0.286365in}}{\pgfqpoint{0.513923in}{0.280833in}}%
\pgfpathclose%
\pgfpathmoveto{\pgfqpoint{0.666667in}{0.275000in}}%
\pgfpathcurveto{\pgfqpoint{0.682137in}{0.275000in}}{\pgfqpoint{0.696975in}{0.281146in}}{\pgfqpoint{0.707915in}{0.292085in}}%
\pgfpathcurveto{\pgfqpoint{0.718854in}{0.303025in}}{\pgfqpoint{0.725000in}{0.317863in}}{\pgfqpoint{0.725000in}{0.333333in}}%
\pgfpathcurveto{\pgfqpoint{0.725000in}{0.348804in}}{\pgfqpoint{0.718854in}{0.363642in}}{\pgfqpoint{0.707915in}{0.374581in}}%
\pgfpathcurveto{\pgfqpoint{0.696975in}{0.385520in}}{\pgfqpoint{0.682137in}{0.391667in}}{\pgfqpoint{0.666667in}{0.391667in}}%
\pgfpathcurveto{\pgfqpoint{0.651196in}{0.391667in}}{\pgfqpoint{0.636358in}{0.385520in}}{\pgfqpoint{0.625419in}{0.374581in}}%
\pgfpathcurveto{\pgfqpoint{0.614480in}{0.363642in}}{\pgfqpoint{0.608333in}{0.348804in}}{\pgfqpoint{0.608333in}{0.333333in}}%
\pgfpathcurveto{\pgfqpoint{0.608333in}{0.317863in}}{\pgfqpoint{0.614480in}{0.303025in}}{\pgfqpoint{0.625419in}{0.292085in}}%
\pgfpathcurveto{\pgfqpoint{0.636358in}{0.281146in}}{\pgfqpoint{0.651196in}{0.275000in}}{\pgfqpoint{0.666667in}{0.275000in}}%
\pgfpathclose%
\pgfpathmoveto{\pgfqpoint{0.666667in}{0.280833in}}%
\pgfpathcurveto{\pgfqpoint{0.666667in}{0.280833in}}{\pgfqpoint{0.652744in}{0.280833in}}{\pgfqpoint{0.639389in}{0.286365in}}%
\pgfpathcurveto{\pgfqpoint{0.629544in}{0.296210in}}{\pgfqpoint{0.619698in}{0.306055in}}{\pgfqpoint{0.614167in}{0.319410in}}%
\pgfpathcurveto{\pgfqpoint{0.614167in}{0.333333in}}{\pgfqpoint{0.614167in}{0.347256in}}{\pgfqpoint{0.619698in}{0.360611in}}%
\pgfpathcurveto{\pgfqpoint{0.629544in}{0.370456in}}{\pgfqpoint{0.639389in}{0.380302in}}{\pgfqpoint{0.652744in}{0.385833in}}%
\pgfpathcurveto{\pgfqpoint{0.666667in}{0.385833in}}{\pgfqpoint{0.680590in}{0.385833in}}{\pgfqpoint{0.693945in}{0.380302in}}%
\pgfpathcurveto{\pgfqpoint{0.703790in}{0.370456in}}{\pgfqpoint{0.713635in}{0.360611in}}{\pgfqpoint{0.719167in}{0.347256in}}%
\pgfpathcurveto{\pgfqpoint{0.719167in}{0.333333in}}{\pgfqpoint{0.719167in}{0.319410in}}{\pgfqpoint{0.713635in}{0.306055in}}%
\pgfpathcurveto{\pgfqpoint{0.703790in}{0.296210in}}{\pgfqpoint{0.693945in}{0.286365in}}{\pgfqpoint{0.680590in}{0.280833in}}%
\pgfpathclose%
\pgfpathmoveto{\pgfqpoint{0.833333in}{0.275000in}}%
\pgfpathcurveto{\pgfqpoint{0.848804in}{0.275000in}}{\pgfqpoint{0.863642in}{0.281146in}}{\pgfqpoint{0.874581in}{0.292085in}}%
\pgfpathcurveto{\pgfqpoint{0.885520in}{0.303025in}}{\pgfqpoint{0.891667in}{0.317863in}}{\pgfqpoint{0.891667in}{0.333333in}}%
\pgfpathcurveto{\pgfqpoint{0.891667in}{0.348804in}}{\pgfqpoint{0.885520in}{0.363642in}}{\pgfqpoint{0.874581in}{0.374581in}}%
\pgfpathcurveto{\pgfqpoint{0.863642in}{0.385520in}}{\pgfqpoint{0.848804in}{0.391667in}}{\pgfqpoint{0.833333in}{0.391667in}}%
\pgfpathcurveto{\pgfqpoint{0.817863in}{0.391667in}}{\pgfqpoint{0.803025in}{0.385520in}}{\pgfqpoint{0.792085in}{0.374581in}}%
\pgfpathcurveto{\pgfqpoint{0.781146in}{0.363642in}}{\pgfqpoint{0.775000in}{0.348804in}}{\pgfqpoint{0.775000in}{0.333333in}}%
\pgfpathcurveto{\pgfqpoint{0.775000in}{0.317863in}}{\pgfqpoint{0.781146in}{0.303025in}}{\pgfqpoint{0.792085in}{0.292085in}}%
\pgfpathcurveto{\pgfqpoint{0.803025in}{0.281146in}}{\pgfqpoint{0.817863in}{0.275000in}}{\pgfqpoint{0.833333in}{0.275000in}}%
\pgfpathclose%
\pgfpathmoveto{\pgfqpoint{0.833333in}{0.280833in}}%
\pgfpathcurveto{\pgfqpoint{0.833333in}{0.280833in}}{\pgfqpoint{0.819410in}{0.280833in}}{\pgfqpoint{0.806055in}{0.286365in}}%
\pgfpathcurveto{\pgfqpoint{0.796210in}{0.296210in}}{\pgfqpoint{0.786365in}{0.306055in}}{\pgfqpoint{0.780833in}{0.319410in}}%
\pgfpathcurveto{\pgfqpoint{0.780833in}{0.333333in}}{\pgfqpoint{0.780833in}{0.347256in}}{\pgfqpoint{0.786365in}{0.360611in}}%
\pgfpathcurveto{\pgfqpoint{0.796210in}{0.370456in}}{\pgfqpoint{0.806055in}{0.380302in}}{\pgfqpoint{0.819410in}{0.385833in}}%
\pgfpathcurveto{\pgfqpoint{0.833333in}{0.385833in}}{\pgfqpoint{0.847256in}{0.385833in}}{\pgfqpoint{0.860611in}{0.380302in}}%
\pgfpathcurveto{\pgfqpoint{0.870456in}{0.370456in}}{\pgfqpoint{0.880302in}{0.360611in}}{\pgfqpoint{0.885833in}{0.347256in}}%
\pgfpathcurveto{\pgfqpoint{0.885833in}{0.333333in}}{\pgfqpoint{0.885833in}{0.319410in}}{\pgfqpoint{0.880302in}{0.306055in}}%
\pgfpathcurveto{\pgfqpoint{0.870456in}{0.296210in}}{\pgfqpoint{0.860611in}{0.286365in}}{\pgfqpoint{0.847256in}{0.280833in}}%
\pgfpathclose%
\pgfpathmoveto{\pgfqpoint{1.000000in}{0.275000in}}%
\pgfpathcurveto{\pgfqpoint{1.015470in}{0.275000in}}{\pgfqpoint{1.030309in}{0.281146in}}{\pgfqpoint{1.041248in}{0.292085in}}%
\pgfpathcurveto{\pgfqpoint{1.052187in}{0.303025in}}{\pgfqpoint{1.058333in}{0.317863in}}{\pgfqpoint{1.058333in}{0.333333in}}%
\pgfpathcurveto{\pgfqpoint{1.058333in}{0.348804in}}{\pgfqpoint{1.052187in}{0.363642in}}{\pgfqpoint{1.041248in}{0.374581in}}%
\pgfpathcurveto{\pgfqpoint{1.030309in}{0.385520in}}{\pgfqpoint{1.015470in}{0.391667in}}{\pgfqpoint{1.000000in}{0.391667in}}%
\pgfpathcurveto{\pgfqpoint{0.984530in}{0.391667in}}{\pgfqpoint{0.969691in}{0.385520in}}{\pgfqpoint{0.958752in}{0.374581in}}%
\pgfpathcurveto{\pgfqpoint{0.947813in}{0.363642in}}{\pgfqpoint{0.941667in}{0.348804in}}{\pgfqpoint{0.941667in}{0.333333in}}%
\pgfpathcurveto{\pgfqpoint{0.941667in}{0.317863in}}{\pgfqpoint{0.947813in}{0.303025in}}{\pgfqpoint{0.958752in}{0.292085in}}%
\pgfpathcurveto{\pgfqpoint{0.969691in}{0.281146in}}{\pgfqpoint{0.984530in}{0.275000in}}{\pgfqpoint{1.000000in}{0.275000in}}%
\pgfpathclose%
\pgfpathmoveto{\pgfqpoint{1.000000in}{0.280833in}}%
\pgfpathcurveto{\pgfqpoint{1.000000in}{0.280833in}}{\pgfqpoint{0.986077in}{0.280833in}}{\pgfqpoint{0.972722in}{0.286365in}}%
\pgfpathcurveto{\pgfqpoint{0.962877in}{0.296210in}}{\pgfqpoint{0.953032in}{0.306055in}}{\pgfqpoint{0.947500in}{0.319410in}}%
\pgfpathcurveto{\pgfqpoint{0.947500in}{0.333333in}}{\pgfqpoint{0.947500in}{0.347256in}}{\pgfqpoint{0.953032in}{0.360611in}}%
\pgfpathcurveto{\pgfqpoint{0.962877in}{0.370456in}}{\pgfqpoint{0.972722in}{0.380302in}}{\pgfqpoint{0.986077in}{0.385833in}}%
\pgfpathcurveto{\pgfqpoint{1.000000in}{0.385833in}}{\pgfqpoint{1.013923in}{0.385833in}}{\pgfqpoint{1.027278in}{0.380302in}}%
\pgfpathcurveto{\pgfqpoint{1.037123in}{0.370456in}}{\pgfqpoint{1.046968in}{0.360611in}}{\pgfqpoint{1.052500in}{0.347256in}}%
\pgfpathcurveto{\pgfqpoint{1.052500in}{0.333333in}}{\pgfqpoint{1.052500in}{0.319410in}}{\pgfqpoint{1.046968in}{0.306055in}}%
\pgfpathcurveto{\pgfqpoint{1.037123in}{0.296210in}}{\pgfqpoint{1.027278in}{0.286365in}}{\pgfqpoint{1.013923in}{0.280833in}}%
\pgfpathclose%
\pgfpathmoveto{\pgfqpoint{0.083333in}{0.441667in}}%
\pgfpathcurveto{\pgfqpoint{0.098804in}{0.441667in}}{\pgfqpoint{0.113642in}{0.447813in}}{\pgfqpoint{0.124581in}{0.458752in}}%
\pgfpathcurveto{\pgfqpoint{0.135520in}{0.469691in}}{\pgfqpoint{0.141667in}{0.484530in}}{\pgfqpoint{0.141667in}{0.500000in}}%
\pgfpathcurveto{\pgfqpoint{0.141667in}{0.515470in}}{\pgfqpoint{0.135520in}{0.530309in}}{\pgfqpoint{0.124581in}{0.541248in}}%
\pgfpathcurveto{\pgfqpoint{0.113642in}{0.552187in}}{\pgfqpoint{0.098804in}{0.558333in}}{\pgfqpoint{0.083333in}{0.558333in}}%
\pgfpathcurveto{\pgfqpoint{0.067863in}{0.558333in}}{\pgfqpoint{0.053025in}{0.552187in}}{\pgfqpoint{0.042085in}{0.541248in}}%
\pgfpathcurveto{\pgfqpoint{0.031146in}{0.530309in}}{\pgfqpoint{0.025000in}{0.515470in}}{\pgfqpoint{0.025000in}{0.500000in}}%
\pgfpathcurveto{\pgfqpoint{0.025000in}{0.484530in}}{\pgfqpoint{0.031146in}{0.469691in}}{\pgfqpoint{0.042085in}{0.458752in}}%
\pgfpathcurveto{\pgfqpoint{0.053025in}{0.447813in}}{\pgfqpoint{0.067863in}{0.441667in}}{\pgfqpoint{0.083333in}{0.441667in}}%
\pgfpathclose%
\pgfpathmoveto{\pgfqpoint{0.083333in}{0.447500in}}%
\pgfpathcurveto{\pgfqpoint{0.083333in}{0.447500in}}{\pgfqpoint{0.069410in}{0.447500in}}{\pgfqpoint{0.056055in}{0.453032in}}%
\pgfpathcurveto{\pgfqpoint{0.046210in}{0.462877in}}{\pgfqpoint{0.036365in}{0.472722in}}{\pgfqpoint{0.030833in}{0.486077in}}%
\pgfpathcurveto{\pgfqpoint{0.030833in}{0.500000in}}{\pgfqpoint{0.030833in}{0.513923in}}{\pgfqpoint{0.036365in}{0.527278in}}%
\pgfpathcurveto{\pgfqpoint{0.046210in}{0.537123in}}{\pgfqpoint{0.056055in}{0.546968in}}{\pgfqpoint{0.069410in}{0.552500in}}%
\pgfpathcurveto{\pgfqpoint{0.083333in}{0.552500in}}{\pgfqpoint{0.097256in}{0.552500in}}{\pgfqpoint{0.110611in}{0.546968in}}%
\pgfpathcurveto{\pgfqpoint{0.120456in}{0.537123in}}{\pgfqpoint{0.130302in}{0.527278in}}{\pgfqpoint{0.135833in}{0.513923in}}%
\pgfpathcurveto{\pgfqpoint{0.135833in}{0.500000in}}{\pgfqpoint{0.135833in}{0.486077in}}{\pgfqpoint{0.130302in}{0.472722in}}%
\pgfpathcurveto{\pgfqpoint{0.120456in}{0.462877in}}{\pgfqpoint{0.110611in}{0.453032in}}{\pgfqpoint{0.097256in}{0.447500in}}%
\pgfpathclose%
\pgfpathmoveto{\pgfqpoint{0.250000in}{0.441667in}}%
\pgfpathcurveto{\pgfqpoint{0.265470in}{0.441667in}}{\pgfqpoint{0.280309in}{0.447813in}}{\pgfqpoint{0.291248in}{0.458752in}}%
\pgfpathcurveto{\pgfqpoint{0.302187in}{0.469691in}}{\pgfqpoint{0.308333in}{0.484530in}}{\pgfqpoint{0.308333in}{0.500000in}}%
\pgfpathcurveto{\pgfqpoint{0.308333in}{0.515470in}}{\pgfqpoint{0.302187in}{0.530309in}}{\pgfqpoint{0.291248in}{0.541248in}}%
\pgfpathcurveto{\pgfqpoint{0.280309in}{0.552187in}}{\pgfqpoint{0.265470in}{0.558333in}}{\pgfqpoint{0.250000in}{0.558333in}}%
\pgfpathcurveto{\pgfqpoint{0.234530in}{0.558333in}}{\pgfqpoint{0.219691in}{0.552187in}}{\pgfqpoint{0.208752in}{0.541248in}}%
\pgfpathcurveto{\pgfqpoint{0.197813in}{0.530309in}}{\pgfqpoint{0.191667in}{0.515470in}}{\pgfqpoint{0.191667in}{0.500000in}}%
\pgfpathcurveto{\pgfqpoint{0.191667in}{0.484530in}}{\pgfqpoint{0.197813in}{0.469691in}}{\pgfqpoint{0.208752in}{0.458752in}}%
\pgfpathcurveto{\pgfqpoint{0.219691in}{0.447813in}}{\pgfqpoint{0.234530in}{0.441667in}}{\pgfqpoint{0.250000in}{0.441667in}}%
\pgfpathclose%
\pgfpathmoveto{\pgfqpoint{0.250000in}{0.447500in}}%
\pgfpathcurveto{\pgfqpoint{0.250000in}{0.447500in}}{\pgfqpoint{0.236077in}{0.447500in}}{\pgfqpoint{0.222722in}{0.453032in}}%
\pgfpathcurveto{\pgfqpoint{0.212877in}{0.462877in}}{\pgfqpoint{0.203032in}{0.472722in}}{\pgfqpoint{0.197500in}{0.486077in}}%
\pgfpathcurveto{\pgfqpoint{0.197500in}{0.500000in}}{\pgfqpoint{0.197500in}{0.513923in}}{\pgfqpoint{0.203032in}{0.527278in}}%
\pgfpathcurveto{\pgfqpoint{0.212877in}{0.537123in}}{\pgfqpoint{0.222722in}{0.546968in}}{\pgfqpoint{0.236077in}{0.552500in}}%
\pgfpathcurveto{\pgfqpoint{0.250000in}{0.552500in}}{\pgfqpoint{0.263923in}{0.552500in}}{\pgfqpoint{0.277278in}{0.546968in}}%
\pgfpathcurveto{\pgfqpoint{0.287123in}{0.537123in}}{\pgfqpoint{0.296968in}{0.527278in}}{\pgfqpoint{0.302500in}{0.513923in}}%
\pgfpathcurveto{\pgfqpoint{0.302500in}{0.500000in}}{\pgfqpoint{0.302500in}{0.486077in}}{\pgfqpoint{0.296968in}{0.472722in}}%
\pgfpathcurveto{\pgfqpoint{0.287123in}{0.462877in}}{\pgfqpoint{0.277278in}{0.453032in}}{\pgfqpoint{0.263923in}{0.447500in}}%
\pgfpathclose%
\pgfpathmoveto{\pgfqpoint{0.416667in}{0.441667in}}%
\pgfpathcurveto{\pgfqpoint{0.432137in}{0.441667in}}{\pgfqpoint{0.446975in}{0.447813in}}{\pgfqpoint{0.457915in}{0.458752in}}%
\pgfpathcurveto{\pgfqpoint{0.468854in}{0.469691in}}{\pgfqpoint{0.475000in}{0.484530in}}{\pgfqpoint{0.475000in}{0.500000in}}%
\pgfpathcurveto{\pgfqpoint{0.475000in}{0.515470in}}{\pgfqpoint{0.468854in}{0.530309in}}{\pgfqpoint{0.457915in}{0.541248in}}%
\pgfpathcurveto{\pgfqpoint{0.446975in}{0.552187in}}{\pgfqpoint{0.432137in}{0.558333in}}{\pgfqpoint{0.416667in}{0.558333in}}%
\pgfpathcurveto{\pgfqpoint{0.401196in}{0.558333in}}{\pgfqpoint{0.386358in}{0.552187in}}{\pgfqpoint{0.375419in}{0.541248in}}%
\pgfpathcurveto{\pgfqpoint{0.364480in}{0.530309in}}{\pgfqpoint{0.358333in}{0.515470in}}{\pgfqpoint{0.358333in}{0.500000in}}%
\pgfpathcurveto{\pgfqpoint{0.358333in}{0.484530in}}{\pgfqpoint{0.364480in}{0.469691in}}{\pgfqpoint{0.375419in}{0.458752in}}%
\pgfpathcurveto{\pgfqpoint{0.386358in}{0.447813in}}{\pgfqpoint{0.401196in}{0.441667in}}{\pgfqpoint{0.416667in}{0.441667in}}%
\pgfpathclose%
\pgfpathmoveto{\pgfqpoint{0.416667in}{0.447500in}}%
\pgfpathcurveto{\pgfqpoint{0.416667in}{0.447500in}}{\pgfqpoint{0.402744in}{0.447500in}}{\pgfqpoint{0.389389in}{0.453032in}}%
\pgfpathcurveto{\pgfqpoint{0.379544in}{0.462877in}}{\pgfqpoint{0.369698in}{0.472722in}}{\pgfqpoint{0.364167in}{0.486077in}}%
\pgfpathcurveto{\pgfqpoint{0.364167in}{0.500000in}}{\pgfqpoint{0.364167in}{0.513923in}}{\pgfqpoint{0.369698in}{0.527278in}}%
\pgfpathcurveto{\pgfqpoint{0.379544in}{0.537123in}}{\pgfqpoint{0.389389in}{0.546968in}}{\pgfqpoint{0.402744in}{0.552500in}}%
\pgfpathcurveto{\pgfqpoint{0.416667in}{0.552500in}}{\pgfqpoint{0.430590in}{0.552500in}}{\pgfqpoint{0.443945in}{0.546968in}}%
\pgfpathcurveto{\pgfqpoint{0.453790in}{0.537123in}}{\pgfqpoint{0.463635in}{0.527278in}}{\pgfqpoint{0.469167in}{0.513923in}}%
\pgfpathcurveto{\pgfqpoint{0.469167in}{0.500000in}}{\pgfqpoint{0.469167in}{0.486077in}}{\pgfqpoint{0.463635in}{0.472722in}}%
\pgfpathcurveto{\pgfqpoint{0.453790in}{0.462877in}}{\pgfqpoint{0.443945in}{0.453032in}}{\pgfqpoint{0.430590in}{0.447500in}}%
\pgfpathclose%
\pgfpathmoveto{\pgfqpoint{0.583333in}{0.441667in}}%
\pgfpathcurveto{\pgfqpoint{0.598804in}{0.441667in}}{\pgfqpoint{0.613642in}{0.447813in}}{\pgfqpoint{0.624581in}{0.458752in}}%
\pgfpathcurveto{\pgfqpoint{0.635520in}{0.469691in}}{\pgfqpoint{0.641667in}{0.484530in}}{\pgfqpoint{0.641667in}{0.500000in}}%
\pgfpathcurveto{\pgfqpoint{0.641667in}{0.515470in}}{\pgfqpoint{0.635520in}{0.530309in}}{\pgfqpoint{0.624581in}{0.541248in}}%
\pgfpathcurveto{\pgfqpoint{0.613642in}{0.552187in}}{\pgfqpoint{0.598804in}{0.558333in}}{\pgfqpoint{0.583333in}{0.558333in}}%
\pgfpathcurveto{\pgfqpoint{0.567863in}{0.558333in}}{\pgfqpoint{0.553025in}{0.552187in}}{\pgfqpoint{0.542085in}{0.541248in}}%
\pgfpathcurveto{\pgfqpoint{0.531146in}{0.530309in}}{\pgfqpoint{0.525000in}{0.515470in}}{\pgfqpoint{0.525000in}{0.500000in}}%
\pgfpathcurveto{\pgfqpoint{0.525000in}{0.484530in}}{\pgfqpoint{0.531146in}{0.469691in}}{\pgfqpoint{0.542085in}{0.458752in}}%
\pgfpathcurveto{\pgfqpoint{0.553025in}{0.447813in}}{\pgfqpoint{0.567863in}{0.441667in}}{\pgfqpoint{0.583333in}{0.441667in}}%
\pgfpathclose%
\pgfpathmoveto{\pgfqpoint{0.583333in}{0.447500in}}%
\pgfpathcurveto{\pgfqpoint{0.583333in}{0.447500in}}{\pgfqpoint{0.569410in}{0.447500in}}{\pgfqpoint{0.556055in}{0.453032in}}%
\pgfpathcurveto{\pgfqpoint{0.546210in}{0.462877in}}{\pgfqpoint{0.536365in}{0.472722in}}{\pgfqpoint{0.530833in}{0.486077in}}%
\pgfpathcurveto{\pgfqpoint{0.530833in}{0.500000in}}{\pgfqpoint{0.530833in}{0.513923in}}{\pgfqpoint{0.536365in}{0.527278in}}%
\pgfpathcurveto{\pgfqpoint{0.546210in}{0.537123in}}{\pgfqpoint{0.556055in}{0.546968in}}{\pgfqpoint{0.569410in}{0.552500in}}%
\pgfpathcurveto{\pgfqpoint{0.583333in}{0.552500in}}{\pgfqpoint{0.597256in}{0.552500in}}{\pgfqpoint{0.610611in}{0.546968in}}%
\pgfpathcurveto{\pgfqpoint{0.620456in}{0.537123in}}{\pgfqpoint{0.630302in}{0.527278in}}{\pgfqpoint{0.635833in}{0.513923in}}%
\pgfpathcurveto{\pgfqpoint{0.635833in}{0.500000in}}{\pgfqpoint{0.635833in}{0.486077in}}{\pgfqpoint{0.630302in}{0.472722in}}%
\pgfpathcurveto{\pgfqpoint{0.620456in}{0.462877in}}{\pgfqpoint{0.610611in}{0.453032in}}{\pgfqpoint{0.597256in}{0.447500in}}%
\pgfpathclose%
\pgfpathmoveto{\pgfqpoint{0.750000in}{0.441667in}}%
\pgfpathcurveto{\pgfqpoint{0.765470in}{0.441667in}}{\pgfqpoint{0.780309in}{0.447813in}}{\pgfqpoint{0.791248in}{0.458752in}}%
\pgfpathcurveto{\pgfqpoint{0.802187in}{0.469691in}}{\pgfqpoint{0.808333in}{0.484530in}}{\pgfqpoint{0.808333in}{0.500000in}}%
\pgfpathcurveto{\pgfqpoint{0.808333in}{0.515470in}}{\pgfqpoint{0.802187in}{0.530309in}}{\pgfqpoint{0.791248in}{0.541248in}}%
\pgfpathcurveto{\pgfqpoint{0.780309in}{0.552187in}}{\pgfqpoint{0.765470in}{0.558333in}}{\pgfqpoint{0.750000in}{0.558333in}}%
\pgfpathcurveto{\pgfqpoint{0.734530in}{0.558333in}}{\pgfqpoint{0.719691in}{0.552187in}}{\pgfqpoint{0.708752in}{0.541248in}}%
\pgfpathcurveto{\pgfqpoint{0.697813in}{0.530309in}}{\pgfqpoint{0.691667in}{0.515470in}}{\pgfqpoint{0.691667in}{0.500000in}}%
\pgfpathcurveto{\pgfqpoint{0.691667in}{0.484530in}}{\pgfqpoint{0.697813in}{0.469691in}}{\pgfqpoint{0.708752in}{0.458752in}}%
\pgfpathcurveto{\pgfqpoint{0.719691in}{0.447813in}}{\pgfqpoint{0.734530in}{0.441667in}}{\pgfqpoint{0.750000in}{0.441667in}}%
\pgfpathclose%
\pgfpathmoveto{\pgfqpoint{0.750000in}{0.447500in}}%
\pgfpathcurveto{\pgfqpoint{0.750000in}{0.447500in}}{\pgfqpoint{0.736077in}{0.447500in}}{\pgfqpoint{0.722722in}{0.453032in}}%
\pgfpathcurveto{\pgfqpoint{0.712877in}{0.462877in}}{\pgfqpoint{0.703032in}{0.472722in}}{\pgfqpoint{0.697500in}{0.486077in}}%
\pgfpathcurveto{\pgfqpoint{0.697500in}{0.500000in}}{\pgfqpoint{0.697500in}{0.513923in}}{\pgfqpoint{0.703032in}{0.527278in}}%
\pgfpathcurveto{\pgfqpoint{0.712877in}{0.537123in}}{\pgfqpoint{0.722722in}{0.546968in}}{\pgfqpoint{0.736077in}{0.552500in}}%
\pgfpathcurveto{\pgfqpoint{0.750000in}{0.552500in}}{\pgfqpoint{0.763923in}{0.552500in}}{\pgfqpoint{0.777278in}{0.546968in}}%
\pgfpathcurveto{\pgfqpoint{0.787123in}{0.537123in}}{\pgfqpoint{0.796968in}{0.527278in}}{\pgfqpoint{0.802500in}{0.513923in}}%
\pgfpathcurveto{\pgfqpoint{0.802500in}{0.500000in}}{\pgfqpoint{0.802500in}{0.486077in}}{\pgfqpoint{0.796968in}{0.472722in}}%
\pgfpathcurveto{\pgfqpoint{0.787123in}{0.462877in}}{\pgfqpoint{0.777278in}{0.453032in}}{\pgfqpoint{0.763923in}{0.447500in}}%
\pgfpathclose%
\pgfpathmoveto{\pgfqpoint{0.916667in}{0.441667in}}%
\pgfpathcurveto{\pgfqpoint{0.932137in}{0.441667in}}{\pgfqpoint{0.946975in}{0.447813in}}{\pgfqpoint{0.957915in}{0.458752in}}%
\pgfpathcurveto{\pgfqpoint{0.968854in}{0.469691in}}{\pgfqpoint{0.975000in}{0.484530in}}{\pgfqpoint{0.975000in}{0.500000in}}%
\pgfpathcurveto{\pgfqpoint{0.975000in}{0.515470in}}{\pgfqpoint{0.968854in}{0.530309in}}{\pgfqpoint{0.957915in}{0.541248in}}%
\pgfpathcurveto{\pgfqpoint{0.946975in}{0.552187in}}{\pgfqpoint{0.932137in}{0.558333in}}{\pgfqpoint{0.916667in}{0.558333in}}%
\pgfpathcurveto{\pgfqpoint{0.901196in}{0.558333in}}{\pgfqpoint{0.886358in}{0.552187in}}{\pgfqpoint{0.875419in}{0.541248in}}%
\pgfpathcurveto{\pgfqpoint{0.864480in}{0.530309in}}{\pgfqpoint{0.858333in}{0.515470in}}{\pgfqpoint{0.858333in}{0.500000in}}%
\pgfpathcurveto{\pgfqpoint{0.858333in}{0.484530in}}{\pgfqpoint{0.864480in}{0.469691in}}{\pgfqpoint{0.875419in}{0.458752in}}%
\pgfpathcurveto{\pgfqpoint{0.886358in}{0.447813in}}{\pgfqpoint{0.901196in}{0.441667in}}{\pgfqpoint{0.916667in}{0.441667in}}%
\pgfpathclose%
\pgfpathmoveto{\pgfqpoint{0.916667in}{0.447500in}}%
\pgfpathcurveto{\pgfqpoint{0.916667in}{0.447500in}}{\pgfqpoint{0.902744in}{0.447500in}}{\pgfqpoint{0.889389in}{0.453032in}}%
\pgfpathcurveto{\pgfqpoint{0.879544in}{0.462877in}}{\pgfqpoint{0.869698in}{0.472722in}}{\pgfqpoint{0.864167in}{0.486077in}}%
\pgfpathcurveto{\pgfqpoint{0.864167in}{0.500000in}}{\pgfqpoint{0.864167in}{0.513923in}}{\pgfqpoint{0.869698in}{0.527278in}}%
\pgfpathcurveto{\pgfqpoint{0.879544in}{0.537123in}}{\pgfqpoint{0.889389in}{0.546968in}}{\pgfqpoint{0.902744in}{0.552500in}}%
\pgfpathcurveto{\pgfqpoint{0.916667in}{0.552500in}}{\pgfqpoint{0.930590in}{0.552500in}}{\pgfqpoint{0.943945in}{0.546968in}}%
\pgfpathcurveto{\pgfqpoint{0.953790in}{0.537123in}}{\pgfqpoint{0.963635in}{0.527278in}}{\pgfqpoint{0.969167in}{0.513923in}}%
\pgfpathcurveto{\pgfqpoint{0.969167in}{0.500000in}}{\pgfqpoint{0.969167in}{0.486077in}}{\pgfqpoint{0.963635in}{0.472722in}}%
\pgfpathcurveto{\pgfqpoint{0.953790in}{0.462877in}}{\pgfqpoint{0.943945in}{0.453032in}}{\pgfqpoint{0.930590in}{0.447500in}}%
\pgfpathclose%
\pgfpathmoveto{\pgfqpoint{0.000000in}{0.608333in}}%
\pgfpathcurveto{\pgfqpoint{0.015470in}{0.608333in}}{\pgfqpoint{0.030309in}{0.614480in}}{\pgfqpoint{0.041248in}{0.625419in}}%
\pgfpathcurveto{\pgfqpoint{0.052187in}{0.636358in}}{\pgfqpoint{0.058333in}{0.651196in}}{\pgfqpoint{0.058333in}{0.666667in}}%
\pgfpathcurveto{\pgfqpoint{0.058333in}{0.682137in}}{\pgfqpoint{0.052187in}{0.696975in}}{\pgfqpoint{0.041248in}{0.707915in}}%
\pgfpathcurveto{\pgfqpoint{0.030309in}{0.718854in}}{\pgfqpoint{0.015470in}{0.725000in}}{\pgfqpoint{0.000000in}{0.725000in}}%
\pgfpathcurveto{\pgfqpoint{-0.015470in}{0.725000in}}{\pgfqpoint{-0.030309in}{0.718854in}}{\pgfqpoint{-0.041248in}{0.707915in}}%
\pgfpathcurveto{\pgfqpoint{-0.052187in}{0.696975in}}{\pgfqpoint{-0.058333in}{0.682137in}}{\pgfqpoint{-0.058333in}{0.666667in}}%
\pgfpathcurveto{\pgfqpoint{-0.058333in}{0.651196in}}{\pgfqpoint{-0.052187in}{0.636358in}}{\pgfqpoint{-0.041248in}{0.625419in}}%
\pgfpathcurveto{\pgfqpoint{-0.030309in}{0.614480in}}{\pgfqpoint{-0.015470in}{0.608333in}}{\pgfqpoint{0.000000in}{0.608333in}}%
\pgfpathclose%
\pgfpathmoveto{\pgfqpoint{0.000000in}{0.614167in}}%
\pgfpathcurveto{\pgfqpoint{0.000000in}{0.614167in}}{\pgfqpoint{-0.013923in}{0.614167in}}{\pgfqpoint{-0.027278in}{0.619698in}}%
\pgfpathcurveto{\pgfqpoint{-0.037123in}{0.629544in}}{\pgfqpoint{-0.046968in}{0.639389in}}{\pgfqpoint{-0.052500in}{0.652744in}}%
\pgfpathcurveto{\pgfqpoint{-0.052500in}{0.666667in}}{\pgfqpoint{-0.052500in}{0.680590in}}{\pgfqpoint{-0.046968in}{0.693945in}}%
\pgfpathcurveto{\pgfqpoint{-0.037123in}{0.703790in}}{\pgfqpoint{-0.027278in}{0.713635in}}{\pgfqpoint{-0.013923in}{0.719167in}}%
\pgfpathcurveto{\pgfqpoint{0.000000in}{0.719167in}}{\pgfqpoint{0.013923in}{0.719167in}}{\pgfqpoint{0.027278in}{0.713635in}}%
\pgfpathcurveto{\pgfqpoint{0.037123in}{0.703790in}}{\pgfqpoint{0.046968in}{0.693945in}}{\pgfqpoint{0.052500in}{0.680590in}}%
\pgfpathcurveto{\pgfqpoint{0.052500in}{0.666667in}}{\pgfqpoint{0.052500in}{0.652744in}}{\pgfqpoint{0.046968in}{0.639389in}}%
\pgfpathcurveto{\pgfqpoint{0.037123in}{0.629544in}}{\pgfqpoint{0.027278in}{0.619698in}}{\pgfqpoint{0.013923in}{0.614167in}}%
\pgfpathclose%
\pgfpathmoveto{\pgfqpoint{0.166667in}{0.608333in}}%
\pgfpathcurveto{\pgfqpoint{0.182137in}{0.608333in}}{\pgfqpoint{0.196975in}{0.614480in}}{\pgfqpoint{0.207915in}{0.625419in}}%
\pgfpathcurveto{\pgfqpoint{0.218854in}{0.636358in}}{\pgfqpoint{0.225000in}{0.651196in}}{\pgfqpoint{0.225000in}{0.666667in}}%
\pgfpathcurveto{\pgfqpoint{0.225000in}{0.682137in}}{\pgfqpoint{0.218854in}{0.696975in}}{\pgfqpoint{0.207915in}{0.707915in}}%
\pgfpathcurveto{\pgfqpoint{0.196975in}{0.718854in}}{\pgfqpoint{0.182137in}{0.725000in}}{\pgfqpoint{0.166667in}{0.725000in}}%
\pgfpathcurveto{\pgfqpoint{0.151196in}{0.725000in}}{\pgfqpoint{0.136358in}{0.718854in}}{\pgfqpoint{0.125419in}{0.707915in}}%
\pgfpathcurveto{\pgfqpoint{0.114480in}{0.696975in}}{\pgfqpoint{0.108333in}{0.682137in}}{\pgfqpoint{0.108333in}{0.666667in}}%
\pgfpathcurveto{\pgfqpoint{0.108333in}{0.651196in}}{\pgfqpoint{0.114480in}{0.636358in}}{\pgfqpoint{0.125419in}{0.625419in}}%
\pgfpathcurveto{\pgfqpoint{0.136358in}{0.614480in}}{\pgfqpoint{0.151196in}{0.608333in}}{\pgfqpoint{0.166667in}{0.608333in}}%
\pgfpathclose%
\pgfpathmoveto{\pgfqpoint{0.166667in}{0.614167in}}%
\pgfpathcurveto{\pgfqpoint{0.166667in}{0.614167in}}{\pgfqpoint{0.152744in}{0.614167in}}{\pgfqpoint{0.139389in}{0.619698in}}%
\pgfpathcurveto{\pgfqpoint{0.129544in}{0.629544in}}{\pgfqpoint{0.119698in}{0.639389in}}{\pgfqpoint{0.114167in}{0.652744in}}%
\pgfpathcurveto{\pgfqpoint{0.114167in}{0.666667in}}{\pgfqpoint{0.114167in}{0.680590in}}{\pgfqpoint{0.119698in}{0.693945in}}%
\pgfpathcurveto{\pgfqpoint{0.129544in}{0.703790in}}{\pgfqpoint{0.139389in}{0.713635in}}{\pgfqpoint{0.152744in}{0.719167in}}%
\pgfpathcurveto{\pgfqpoint{0.166667in}{0.719167in}}{\pgfqpoint{0.180590in}{0.719167in}}{\pgfqpoint{0.193945in}{0.713635in}}%
\pgfpathcurveto{\pgfqpoint{0.203790in}{0.703790in}}{\pgfqpoint{0.213635in}{0.693945in}}{\pgfqpoint{0.219167in}{0.680590in}}%
\pgfpathcurveto{\pgfqpoint{0.219167in}{0.666667in}}{\pgfqpoint{0.219167in}{0.652744in}}{\pgfqpoint{0.213635in}{0.639389in}}%
\pgfpathcurveto{\pgfqpoint{0.203790in}{0.629544in}}{\pgfqpoint{0.193945in}{0.619698in}}{\pgfqpoint{0.180590in}{0.614167in}}%
\pgfpathclose%
\pgfpathmoveto{\pgfqpoint{0.333333in}{0.608333in}}%
\pgfpathcurveto{\pgfqpoint{0.348804in}{0.608333in}}{\pgfqpoint{0.363642in}{0.614480in}}{\pgfqpoint{0.374581in}{0.625419in}}%
\pgfpathcurveto{\pgfqpoint{0.385520in}{0.636358in}}{\pgfqpoint{0.391667in}{0.651196in}}{\pgfqpoint{0.391667in}{0.666667in}}%
\pgfpathcurveto{\pgfqpoint{0.391667in}{0.682137in}}{\pgfqpoint{0.385520in}{0.696975in}}{\pgfqpoint{0.374581in}{0.707915in}}%
\pgfpathcurveto{\pgfqpoint{0.363642in}{0.718854in}}{\pgfqpoint{0.348804in}{0.725000in}}{\pgfqpoint{0.333333in}{0.725000in}}%
\pgfpathcurveto{\pgfqpoint{0.317863in}{0.725000in}}{\pgfqpoint{0.303025in}{0.718854in}}{\pgfqpoint{0.292085in}{0.707915in}}%
\pgfpathcurveto{\pgfqpoint{0.281146in}{0.696975in}}{\pgfqpoint{0.275000in}{0.682137in}}{\pgfqpoint{0.275000in}{0.666667in}}%
\pgfpathcurveto{\pgfqpoint{0.275000in}{0.651196in}}{\pgfqpoint{0.281146in}{0.636358in}}{\pgfqpoint{0.292085in}{0.625419in}}%
\pgfpathcurveto{\pgfqpoint{0.303025in}{0.614480in}}{\pgfqpoint{0.317863in}{0.608333in}}{\pgfqpoint{0.333333in}{0.608333in}}%
\pgfpathclose%
\pgfpathmoveto{\pgfqpoint{0.333333in}{0.614167in}}%
\pgfpathcurveto{\pgfqpoint{0.333333in}{0.614167in}}{\pgfqpoint{0.319410in}{0.614167in}}{\pgfqpoint{0.306055in}{0.619698in}}%
\pgfpathcurveto{\pgfqpoint{0.296210in}{0.629544in}}{\pgfqpoint{0.286365in}{0.639389in}}{\pgfqpoint{0.280833in}{0.652744in}}%
\pgfpathcurveto{\pgfqpoint{0.280833in}{0.666667in}}{\pgfqpoint{0.280833in}{0.680590in}}{\pgfqpoint{0.286365in}{0.693945in}}%
\pgfpathcurveto{\pgfqpoint{0.296210in}{0.703790in}}{\pgfqpoint{0.306055in}{0.713635in}}{\pgfqpoint{0.319410in}{0.719167in}}%
\pgfpathcurveto{\pgfqpoint{0.333333in}{0.719167in}}{\pgfqpoint{0.347256in}{0.719167in}}{\pgfqpoint{0.360611in}{0.713635in}}%
\pgfpathcurveto{\pgfqpoint{0.370456in}{0.703790in}}{\pgfqpoint{0.380302in}{0.693945in}}{\pgfqpoint{0.385833in}{0.680590in}}%
\pgfpathcurveto{\pgfqpoint{0.385833in}{0.666667in}}{\pgfqpoint{0.385833in}{0.652744in}}{\pgfqpoint{0.380302in}{0.639389in}}%
\pgfpathcurveto{\pgfqpoint{0.370456in}{0.629544in}}{\pgfqpoint{0.360611in}{0.619698in}}{\pgfqpoint{0.347256in}{0.614167in}}%
\pgfpathclose%
\pgfpathmoveto{\pgfqpoint{0.500000in}{0.608333in}}%
\pgfpathcurveto{\pgfqpoint{0.515470in}{0.608333in}}{\pgfqpoint{0.530309in}{0.614480in}}{\pgfqpoint{0.541248in}{0.625419in}}%
\pgfpathcurveto{\pgfqpoint{0.552187in}{0.636358in}}{\pgfqpoint{0.558333in}{0.651196in}}{\pgfqpoint{0.558333in}{0.666667in}}%
\pgfpathcurveto{\pgfqpoint{0.558333in}{0.682137in}}{\pgfqpoint{0.552187in}{0.696975in}}{\pgfqpoint{0.541248in}{0.707915in}}%
\pgfpathcurveto{\pgfqpoint{0.530309in}{0.718854in}}{\pgfqpoint{0.515470in}{0.725000in}}{\pgfqpoint{0.500000in}{0.725000in}}%
\pgfpathcurveto{\pgfqpoint{0.484530in}{0.725000in}}{\pgfqpoint{0.469691in}{0.718854in}}{\pgfqpoint{0.458752in}{0.707915in}}%
\pgfpathcurveto{\pgfqpoint{0.447813in}{0.696975in}}{\pgfqpoint{0.441667in}{0.682137in}}{\pgfqpoint{0.441667in}{0.666667in}}%
\pgfpathcurveto{\pgfqpoint{0.441667in}{0.651196in}}{\pgfqpoint{0.447813in}{0.636358in}}{\pgfqpoint{0.458752in}{0.625419in}}%
\pgfpathcurveto{\pgfqpoint{0.469691in}{0.614480in}}{\pgfqpoint{0.484530in}{0.608333in}}{\pgfqpoint{0.500000in}{0.608333in}}%
\pgfpathclose%
\pgfpathmoveto{\pgfqpoint{0.500000in}{0.614167in}}%
\pgfpathcurveto{\pgfqpoint{0.500000in}{0.614167in}}{\pgfqpoint{0.486077in}{0.614167in}}{\pgfqpoint{0.472722in}{0.619698in}}%
\pgfpathcurveto{\pgfqpoint{0.462877in}{0.629544in}}{\pgfqpoint{0.453032in}{0.639389in}}{\pgfqpoint{0.447500in}{0.652744in}}%
\pgfpathcurveto{\pgfqpoint{0.447500in}{0.666667in}}{\pgfqpoint{0.447500in}{0.680590in}}{\pgfqpoint{0.453032in}{0.693945in}}%
\pgfpathcurveto{\pgfqpoint{0.462877in}{0.703790in}}{\pgfqpoint{0.472722in}{0.713635in}}{\pgfqpoint{0.486077in}{0.719167in}}%
\pgfpathcurveto{\pgfqpoint{0.500000in}{0.719167in}}{\pgfqpoint{0.513923in}{0.719167in}}{\pgfqpoint{0.527278in}{0.713635in}}%
\pgfpathcurveto{\pgfqpoint{0.537123in}{0.703790in}}{\pgfqpoint{0.546968in}{0.693945in}}{\pgfqpoint{0.552500in}{0.680590in}}%
\pgfpathcurveto{\pgfqpoint{0.552500in}{0.666667in}}{\pgfqpoint{0.552500in}{0.652744in}}{\pgfqpoint{0.546968in}{0.639389in}}%
\pgfpathcurveto{\pgfqpoint{0.537123in}{0.629544in}}{\pgfqpoint{0.527278in}{0.619698in}}{\pgfqpoint{0.513923in}{0.614167in}}%
\pgfpathclose%
\pgfpathmoveto{\pgfqpoint{0.666667in}{0.608333in}}%
\pgfpathcurveto{\pgfqpoint{0.682137in}{0.608333in}}{\pgfqpoint{0.696975in}{0.614480in}}{\pgfqpoint{0.707915in}{0.625419in}}%
\pgfpathcurveto{\pgfqpoint{0.718854in}{0.636358in}}{\pgfqpoint{0.725000in}{0.651196in}}{\pgfqpoint{0.725000in}{0.666667in}}%
\pgfpathcurveto{\pgfqpoint{0.725000in}{0.682137in}}{\pgfqpoint{0.718854in}{0.696975in}}{\pgfqpoint{0.707915in}{0.707915in}}%
\pgfpathcurveto{\pgfqpoint{0.696975in}{0.718854in}}{\pgfqpoint{0.682137in}{0.725000in}}{\pgfqpoint{0.666667in}{0.725000in}}%
\pgfpathcurveto{\pgfqpoint{0.651196in}{0.725000in}}{\pgfqpoint{0.636358in}{0.718854in}}{\pgfqpoint{0.625419in}{0.707915in}}%
\pgfpathcurveto{\pgfqpoint{0.614480in}{0.696975in}}{\pgfqpoint{0.608333in}{0.682137in}}{\pgfqpoint{0.608333in}{0.666667in}}%
\pgfpathcurveto{\pgfqpoint{0.608333in}{0.651196in}}{\pgfqpoint{0.614480in}{0.636358in}}{\pgfqpoint{0.625419in}{0.625419in}}%
\pgfpathcurveto{\pgfqpoint{0.636358in}{0.614480in}}{\pgfqpoint{0.651196in}{0.608333in}}{\pgfqpoint{0.666667in}{0.608333in}}%
\pgfpathclose%
\pgfpathmoveto{\pgfqpoint{0.666667in}{0.614167in}}%
\pgfpathcurveto{\pgfqpoint{0.666667in}{0.614167in}}{\pgfqpoint{0.652744in}{0.614167in}}{\pgfqpoint{0.639389in}{0.619698in}}%
\pgfpathcurveto{\pgfqpoint{0.629544in}{0.629544in}}{\pgfqpoint{0.619698in}{0.639389in}}{\pgfqpoint{0.614167in}{0.652744in}}%
\pgfpathcurveto{\pgfqpoint{0.614167in}{0.666667in}}{\pgfqpoint{0.614167in}{0.680590in}}{\pgfqpoint{0.619698in}{0.693945in}}%
\pgfpathcurveto{\pgfqpoint{0.629544in}{0.703790in}}{\pgfqpoint{0.639389in}{0.713635in}}{\pgfqpoint{0.652744in}{0.719167in}}%
\pgfpathcurveto{\pgfqpoint{0.666667in}{0.719167in}}{\pgfqpoint{0.680590in}{0.719167in}}{\pgfqpoint{0.693945in}{0.713635in}}%
\pgfpathcurveto{\pgfqpoint{0.703790in}{0.703790in}}{\pgfqpoint{0.713635in}{0.693945in}}{\pgfqpoint{0.719167in}{0.680590in}}%
\pgfpathcurveto{\pgfqpoint{0.719167in}{0.666667in}}{\pgfqpoint{0.719167in}{0.652744in}}{\pgfqpoint{0.713635in}{0.639389in}}%
\pgfpathcurveto{\pgfqpoint{0.703790in}{0.629544in}}{\pgfqpoint{0.693945in}{0.619698in}}{\pgfqpoint{0.680590in}{0.614167in}}%
\pgfpathclose%
\pgfpathmoveto{\pgfqpoint{0.833333in}{0.608333in}}%
\pgfpathcurveto{\pgfqpoint{0.848804in}{0.608333in}}{\pgfqpoint{0.863642in}{0.614480in}}{\pgfqpoint{0.874581in}{0.625419in}}%
\pgfpathcurveto{\pgfqpoint{0.885520in}{0.636358in}}{\pgfqpoint{0.891667in}{0.651196in}}{\pgfqpoint{0.891667in}{0.666667in}}%
\pgfpathcurveto{\pgfqpoint{0.891667in}{0.682137in}}{\pgfqpoint{0.885520in}{0.696975in}}{\pgfqpoint{0.874581in}{0.707915in}}%
\pgfpathcurveto{\pgfqpoint{0.863642in}{0.718854in}}{\pgfqpoint{0.848804in}{0.725000in}}{\pgfqpoint{0.833333in}{0.725000in}}%
\pgfpathcurveto{\pgfqpoint{0.817863in}{0.725000in}}{\pgfqpoint{0.803025in}{0.718854in}}{\pgfqpoint{0.792085in}{0.707915in}}%
\pgfpathcurveto{\pgfqpoint{0.781146in}{0.696975in}}{\pgfqpoint{0.775000in}{0.682137in}}{\pgfqpoint{0.775000in}{0.666667in}}%
\pgfpathcurveto{\pgfqpoint{0.775000in}{0.651196in}}{\pgfqpoint{0.781146in}{0.636358in}}{\pgfqpoint{0.792085in}{0.625419in}}%
\pgfpathcurveto{\pgfqpoint{0.803025in}{0.614480in}}{\pgfqpoint{0.817863in}{0.608333in}}{\pgfqpoint{0.833333in}{0.608333in}}%
\pgfpathclose%
\pgfpathmoveto{\pgfqpoint{0.833333in}{0.614167in}}%
\pgfpathcurveto{\pgfqpoint{0.833333in}{0.614167in}}{\pgfqpoint{0.819410in}{0.614167in}}{\pgfqpoint{0.806055in}{0.619698in}}%
\pgfpathcurveto{\pgfqpoint{0.796210in}{0.629544in}}{\pgfqpoint{0.786365in}{0.639389in}}{\pgfqpoint{0.780833in}{0.652744in}}%
\pgfpathcurveto{\pgfqpoint{0.780833in}{0.666667in}}{\pgfqpoint{0.780833in}{0.680590in}}{\pgfqpoint{0.786365in}{0.693945in}}%
\pgfpathcurveto{\pgfqpoint{0.796210in}{0.703790in}}{\pgfqpoint{0.806055in}{0.713635in}}{\pgfqpoint{0.819410in}{0.719167in}}%
\pgfpathcurveto{\pgfqpoint{0.833333in}{0.719167in}}{\pgfqpoint{0.847256in}{0.719167in}}{\pgfqpoint{0.860611in}{0.713635in}}%
\pgfpathcurveto{\pgfqpoint{0.870456in}{0.703790in}}{\pgfqpoint{0.880302in}{0.693945in}}{\pgfqpoint{0.885833in}{0.680590in}}%
\pgfpathcurveto{\pgfqpoint{0.885833in}{0.666667in}}{\pgfqpoint{0.885833in}{0.652744in}}{\pgfqpoint{0.880302in}{0.639389in}}%
\pgfpathcurveto{\pgfqpoint{0.870456in}{0.629544in}}{\pgfqpoint{0.860611in}{0.619698in}}{\pgfqpoint{0.847256in}{0.614167in}}%
\pgfpathclose%
\pgfpathmoveto{\pgfqpoint{1.000000in}{0.608333in}}%
\pgfpathcurveto{\pgfqpoint{1.015470in}{0.608333in}}{\pgfqpoint{1.030309in}{0.614480in}}{\pgfqpoint{1.041248in}{0.625419in}}%
\pgfpathcurveto{\pgfqpoint{1.052187in}{0.636358in}}{\pgfqpoint{1.058333in}{0.651196in}}{\pgfqpoint{1.058333in}{0.666667in}}%
\pgfpathcurveto{\pgfqpoint{1.058333in}{0.682137in}}{\pgfqpoint{1.052187in}{0.696975in}}{\pgfqpoint{1.041248in}{0.707915in}}%
\pgfpathcurveto{\pgfqpoint{1.030309in}{0.718854in}}{\pgfqpoint{1.015470in}{0.725000in}}{\pgfqpoint{1.000000in}{0.725000in}}%
\pgfpathcurveto{\pgfqpoint{0.984530in}{0.725000in}}{\pgfqpoint{0.969691in}{0.718854in}}{\pgfqpoint{0.958752in}{0.707915in}}%
\pgfpathcurveto{\pgfqpoint{0.947813in}{0.696975in}}{\pgfqpoint{0.941667in}{0.682137in}}{\pgfqpoint{0.941667in}{0.666667in}}%
\pgfpathcurveto{\pgfqpoint{0.941667in}{0.651196in}}{\pgfqpoint{0.947813in}{0.636358in}}{\pgfqpoint{0.958752in}{0.625419in}}%
\pgfpathcurveto{\pgfqpoint{0.969691in}{0.614480in}}{\pgfqpoint{0.984530in}{0.608333in}}{\pgfqpoint{1.000000in}{0.608333in}}%
\pgfpathclose%
\pgfpathmoveto{\pgfqpoint{1.000000in}{0.614167in}}%
\pgfpathcurveto{\pgfqpoint{1.000000in}{0.614167in}}{\pgfqpoint{0.986077in}{0.614167in}}{\pgfqpoint{0.972722in}{0.619698in}}%
\pgfpathcurveto{\pgfqpoint{0.962877in}{0.629544in}}{\pgfqpoint{0.953032in}{0.639389in}}{\pgfqpoint{0.947500in}{0.652744in}}%
\pgfpathcurveto{\pgfqpoint{0.947500in}{0.666667in}}{\pgfqpoint{0.947500in}{0.680590in}}{\pgfqpoint{0.953032in}{0.693945in}}%
\pgfpathcurveto{\pgfqpoint{0.962877in}{0.703790in}}{\pgfqpoint{0.972722in}{0.713635in}}{\pgfqpoint{0.986077in}{0.719167in}}%
\pgfpathcurveto{\pgfqpoint{1.000000in}{0.719167in}}{\pgfqpoint{1.013923in}{0.719167in}}{\pgfqpoint{1.027278in}{0.713635in}}%
\pgfpathcurveto{\pgfqpoint{1.037123in}{0.703790in}}{\pgfqpoint{1.046968in}{0.693945in}}{\pgfqpoint{1.052500in}{0.680590in}}%
\pgfpathcurveto{\pgfqpoint{1.052500in}{0.666667in}}{\pgfqpoint{1.052500in}{0.652744in}}{\pgfqpoint{1.046968in}{0.639389in}}%
\pgfpathcurveto{\pgfqpoint{1.037123in}{0.629544in}}{\pgfqpoint{1.027278in}{0.619698in}}{\pgfqpoint{1.013923in}{0.614167in}}%
\pgfpathclose%
\pgfpathmoveto{\pgfqpoint{0.083333in}{0.775000in}}%
\pgfpathcurveto{\pgfqpoint{0.098804in}{0.775000in}}{\pgfqpoint{0.113642in}{0.781146in}}{\pgfqpoint{0.124581in}{0.792085in}}%
\pgfpathcurveto{\pgfqpoint{0.135520in}{0.803025in}}{\pgfqpoint{0.141667in}{0.817863in}}{\pgfqpoint{0.141667in}{0.833333in}}%
\pgfpathcurveto{\pgfqpoint{0.141667in}{0.848804in}}{\pgfqpoint{0.135520in}{0.863642in}}{\pgfqpoint{0.124581in}{0.874581in}}%
\pgfpathcurveto{\pgfqpoint{0.113642in}{0.885520in}}{\pgfqpoint{0.098804in}{0.891667in}}{\pgfqpoint{0.083333in}{0.891667in}}%
\pgfpathcurveto{\pgfqpoint{0.067863in}{0.891667in}}{\pgfqpoint{0.053025in}{0.885520in}}{\pgfqpoint{0.042085in}{0.874581in}}%
\pgfpathcurveto{\pgfqpoint{0.031146in}{0.863642in}}{\pgfqpoint{0.025000in}{0.848804in}}{\pgfqpoint{0.025000in}{0.833333in}}%
\pgfpathcurveto{\pgfqpoint{0.025000in}{0.817863in}}{\pgfqpoint{0.031146in}{0.803025in}}{\pgfqpoint{0.042085in}{0.792085in}}%
\pgfpathcurveto{\pgfqpoint{0.053025in}{0.781146in}}{\pgfqpoint{0.067863in}{0.775000in}}{\pgfqpoint{0.083333in}{0.775000in}}%
\pgfpathclose%
\pgfpathmoveto{\pgfqpoint{0.083333in}{0.780833in}}%
\pgfpathcurveto{\pgfqpoint{0.083333in}{0.780833in}}{\pgfqpoint{0.069410in}{0.780833in}}{\pgfqpoint{0.056055in}{0.786365in}}%
\pgfpathcurveto{\pgfqpoint{0.046210in}{0.796210in}}{\pgfqpoint{0.036365in}{0.806055in}}{\pgfqpoint{0.030833in}{0.819410in}}%
\pgfpathcurveto{\pgfqpoint{0.030833in}{0.833333in}}{\pgfqpoint{0.030833in}{0.847256in}}{\pgfqpoint{0.036365in}{0.860611in}}%
\pgfpathcurveto{\pgfqpoint{0.046210in}{0.870456in}}{\pgfqpoint{0.056055in}{0.880302in}}{\pgfqpoint{0.069410in}{0.885833in}}%
\pgfpathcurveto{\pgfqpoint{0.083333in}{0.885833in}}{\pgfqpoint{0.097256in}{0.885833in}}{\pgfqpoint{0.110611in}{0.880302in}}%
\pgfpathcurveto{\pgfqpoint{0.120456in}{0.870456in}}{\pgfqpoint{0.130302in}{0.860611in}}{\pgfqpoint{0.135833in}{0.847256in}}%
\pgfpathcurveto{\pgfqpoint{0.135833in}{0.833333in}}{\pgfqpoint{0.135833in}{0.819410in}}{\pgfqpoint{0.130302in}{0.806055in}}%
\pgfpathcurveto{\pgfqpoint{0.120456in}{0.796210in}}{\pgfqpoint{0.110611in}{0.786365in}}{\pgfqpoint{0.097256in}{0.780833in}}%
\pgfpathclose%
\pgfpathmoveto{\pgfqpoint{0.250000in}{0.775000in}}%
\pgfpathcurveto{\pgfqpoint{0.265470in}{0.775000in}}{\pgfqpoint{0.280309in}{0.781146in}}{\pgfqpoint{0.291248in}{0.792085in}}%
\pgfpathcurveto{\pgfqpoint{0.302187in}{0.803025in}}{\pgfqpoint{0.308333in}{0.817863in}}{\pgfqpoint{0.308333in}{0.833333in}}%
\pgfpathcurveto{\pgfqpoint{0.308333in}{0.848804in}}{\pgfqpoint{0.302187in}{0.863642in}}{\pgfqpoint{0.291248in}{0.874581in}}%
\pgfpathcurveto{\pgfqpoint{0.280309in}{0.885520in}}{\pgfqpoint{0.265470in}{0.891667in}}{\pgfqpoint{0.250000in}{0.891667in}}%
\pgfpathcurveto{\pgfqpoint{0.234530in}{0.891667in}}{\pgfqpoint{0.219691in}{0.885520in}}{\pgfqpoint{0.208752in}{0.874581in}}%
\pgfpathcurveto{\pgfqpoint{0.197813in}{0.863642in}}{\pgfqpoint{0.191667in}{0.848804in}}{\pgfqpoint{0.191667in}{0.833333in}}%
\pgfpathcurveto{\pgfqpoint{0.191667in}{0.817863in}}{\pgfqpoint{0.197813in}{0.803025in}}{\pgfqpoint{0.208752in}{0.792085in}}%
\pgfpathcurveto{\pgfqpoint{0.219691in}{0.781146in}}{\pgfqpoint{0.234530in}{0.775000in}}{\pgfqpoint{0.250000in}{0.775000in}}%
\pgfpathclose%
\pgfpathmoveto{\pgfqpoint{0.250000in}{0.780833in}}%
\pgfpathcurveto{\pgfqpoint{0.250000in}{0.780833in}}{\pgfqpoint{0.236077in}{0.780833in}}{\pgfqpoint{0.222722in}{0.786365in}}%
\pgfpathcurveto{\pgfqpoint{0.212877in}{0.796210in}}{\pgfqpoint{0.203032in}{0.806055in}}{\pgfqpoint{0.197500in}{0.819410in}}%
\pgfpathcurveto{\pgfqpoint{0.197500in}{0.833333in}}{\pgfqpoint{0.197500in}{0.847256in}}{\pgfqpoint{0.203032in}{0.860611in}}%
\pgfpathcurveto{\pgfqpoint{0.212877in}{0.870456in}}{\pgfqpoint{0.222722in}{0.880302in}}{\pgfqpoint{0.236077in}{0.885833in}}%
\pgfpathcurveto{\pgfqpoint{0.250000in}{0.885833in}}{\pgfqpoint{0.263923in}{0.885833in}}{\pgfqpoint{0.277278in}{0.880302in}}%
\pgfpathcurveto{\pgfqpoint{0.287123in}{0.870456in}}{\pgfqpoint{0.296968in}{0.860611in}}{\pgfqpoint{0.302500in}{0.847256in}}%
\pgfpathcurveto{\pgfqpoint{0.302500in}{0.833333in}}{\pgfqpoint{0.302500in}{0.819410in}}{\pgfqpoint{0.296968in}{0.806055in}}%
\pgfpathcurveto{\pgfqpoint{0.287123in}{0.796210in}}{\pgfqpoint{0.277278in}{0.786365in}}{\pgfqpoint{0.263923in}{0.780833in}}%
\pgfpathclose%
\pgfpathmoveto{\pgfqpoint{0.416667in}{0.775000in}}%
\pgfpathcurveto{\pgfqpoint{0.432137in}{0.775000in}}{\pgfqpoint{0.446975in}{0.781146in}}{\pgfqpoint{0.457915in}{0.792085in}}%
\pgfpathcurveto{\pgfqpoint{0.468854in}{0.803025in}}{\pgfqpoint{0.475000in}{0.817863in}}{\pgfqpoint{0.475000in}{0.833333in}}%
\pgfpathcurveto{\pgfqpoint{0.475000in}{0.848804in}}{\pgfqpoint{0.468854in}{0.863642in}}{\pgfqpoint{0.457915in}{0.874581in}}%
\pgfpathcurveto{\pgfqpoint{0.446975in}{0.885520in}}{\pgfqpoint{0.432137in}{0.891667in}}{\pgfqpoint{0.416667in}{0.891667in}}%
\pgfpathcurveto{\pgfqpoint{0.401196in}{0.891667in}}{\pgfqpoint{0.386358in}{0.885520in}}{\pgfqpoint{0.375419in}{0.874581in}}%
\pgfpathcurveto{\pgfqpoint{0.364480in}{0.863642in}}{\pgfqpoint{0.358333in}{0.848804in}}{\pgfqpoint{0.358333in}{0.833333in}}%
\pgfpathcurveto{\pgfqpoint{0.358333in}{0.817863in}}{\pgfqpoint{0.364480in}{0.803025in}}{\pgfqpoint{0.375419in}{0.792085in}}%
\pgfpathcurveto{\pgfqpoint{0.386358in}{0.781146in}}{\pgfqpoint{0.401196in}{0.775000in}}{\pgfqpoint{0.416667in}{0.775000in}}%
\pgfpathclose%
\pgfpathmoveto{\pgfqpoint{0.416667in}{0.780833in}}%
\pgfpathcurveto{\pgfqpoint{0.416667in}{0.780833in}}{\pgfqpoint{0.402744in}{0.780833in}}{\pgfqpoint{0.389389in}{0.786365in}}%
\pgfpathcurveto{\pgfqpoint{0.379544in}{0.796210in}}{\pgfqpoint{0.369698in}{0.806055in}}{\pgfqpoint{0.364167in}{0.819410in}}%
\pgfpathcurveto{\pgfqpoint{0.364167in}{0.833333in}}{\pgfqpoint{0.364167in}{0.847256in}}{\pgfqpoint{0.369698in}{0.860611in}}%
\pgfpathcurveto{\pgfqpoint{0.379544in}{0.870456in}}{\pgfqpoint{0.389389in}{0.880302in}}{\pgfqpoint{0.402744in}{0.885833in}}%
\pgfpathcurveto{\pgfqpoint{0.416667in}{0.885833in}}{\pgfqpoint{0.430590in}{0.885833in}}{\pgfqpoint{0.443945in}{0.880302in}}%
\pgfpathcurveto{\pgfqpoint{0.453790in}{0.870456in}}{\pgfqpoint{0.463635in}{0.860611in}}{\pgfqpoint{0.469167in}{0.847256in}}%
\pgfpathcurveto{\pgfqpoint{0.469167in}{0.833333in}}{\pgfqpoint{0.469167in}{0.819410in}}{\pgfqpoint{0.463635in}{0.806055in}}%
\pgfpathcurveto{\pgfqpoint{0.453790in}{0.796210in}}{\pgfqpoint{0.443945in}{0.786365in}}{\pgfqpoint{0.430590in}{0.780833in}}%
\pgfpathclose%
\pgfpathmoveto{\pgfqpoint{0.583333in}{0.775000in}}%
\pgfpathcurveto{\pgfqpoint{0.598804in}{0.775000in}}{\pgfqpoint{0.613642in}{0.781146in}}{\pgfqpoint{0.624581in}{0.792085in}}%
\pgfpathcurveto{\pgfqpoint{0.635520in}{0.803025in}}{\pgfqpoint{0.641667in}{0.817863in}}{\pgfqpoint{0.641667in}{0.833333in}}%
\pgfpathcurveto{\pgfqpoint{0.641667in}{0.848804in}}{\pgfqpoint{0.635520in}{0.863642in}}{\pgfqpoint{0.624581in}{0.874581in}}%
\pgfpathcurveto{\pgfqpoint{0.613642in}{0.885520in}}{\pgfqpoint{0.598804in}{0.891667in}}{\pgfqpoint{0.583333in}{0.891667in}}%
\pgfpathcurveto{\pgfqpoint{0.567863in}{0.891667in}}{\pgfqpoint{0.553025in}{0.885520in}}{\pgfqpoint{0.542085in}{0.874581in}}%
\pgfpathcurveto{\pgfqpoint{0.531146in}{0.863642in}}{\pgfqpoint{0.525000in}{0.848804in}}{\pgfqpoint{0.525000in}{0.833333in}}%
\pgfpathcurveto{\pgfqpoint{0.525000in}{0.817863in}}{\pgfqpoint{0.531146in}{0.803025in}}{\pgfqpoint{0.542085in}{0.792085in}}%
\pgfpathcurveto{\pgfqpoint{0.553025in}{0.781146in}}{\pgfqpoint{0.567863in}{0.775000in}}{\pgfqpoint{0.583333in}{0.775000in}}%
\pgfpathclose%
\pgfpathmoveto{\pgfqpoint{0.583333in}{0.780833in}}%
\pgfpathcurveto{\pgfqpoint{0.583333in}{0.780833in}}{\pgfqpoint{0.569410in}{0.780833in}}{\pgfqpoint{0.556055in}{0.786365in}}%
\pgfpathcurveto{\pgfqpoint{0.546210in}{0.796210in}}{\pgfqpoint{0.536365in}{0.806055in}}{\pgfqpoint{0.530833in}{0.819410in}}%
\pgfpathcurveto{\pgfqpoint{0.530833in}{0.833333in}}{\pgfqpoint{0.530833in}{0.847256in}}{\pgfqpoint{0.536365in}{0.860611in}}%
\pgfpathcurveto{\pgfqpoint{0.546210in}{0.870456in}}{\pgfqpoint{0.556055in}{0.880302in}}{\pgfqpoint{0.569410in}{0.885833in}}%
\pgfpathcurveto{\pgfqpoint{0.583333in}{0.885833in}}{\pgfqpoint{0.597256in}{0.885833in}}{\pgfqpoint{0.610611in}{0.880302in}}%
\pgfpathcurveto{\pgfqpoint{0.620456in}{0.870456in}}{\pgfqpoint{0.630302in}{0.860611in}}{\pgfqpoint{0.635833in}{0.847256in}}%
\pgfpathcurveto{\pgfqpoint{0.635833in}{0.833333in}}{\pgfqpoint{0.635833in}{0.819410in}}{\pgfqpoint{0.630302in}{0.806055in}}%
\pgfpathcurveto{\pgfqpoint{0.620456in}{0.796210in}}{\pgfqpoint{0.610611in}{0.786365in}}{\pgfqpoint{0.597256in}{0.780833in}}%
\pgfpathclose%
\pgfpathmoveto{\pgfqpoint{0.750000in}{0.775000in}}%
\pgfpathcurveto{\pgfqpoint{0.765470in}{0.775000in}}{\pgfqpoint{0.780309in}{0.781146in}}{\pgfqpoint{0.791248in}{0.792085in}}%
\pgfpathcurveto{\pgfqpoint{0.802187in}{0.803025in}}{\pgfqpoint{0.808333in}{0.817863in}}{\pgfqpoint{0.808333in}{0.833333in}}%
\pgfpathcurveto{\pgfqpoint{0.808333in}{0.848804in}}{\pgfqpoint{0.802187in}{0.863642in}}{\pgfqpoint{0.791248in}{0.874581in}}%
\pgfpathcurveto{\pgfqpoint{0.780309in}{0.885520in}}{\pgfqpoint{0.765470in}{0.891667in}}{\pgfqpoint{0.750000in}{0.891667in}}%
\pgfpathcurveto{\pgfqpoint{0.734530in}{0.891667in}}{\pgfqpoint{0.719691in}{0.885520in}}{\pgfqpoint{0.708752in}{0.874581in}}%
\pgfpathcurveto{\pgfqpoint{0.697813in}{0.863642in}}{\pgfqpoint{0.691667in}{0.848804in}}{\pgfqpoint{0.691667in}{0.833333in}}%
\pgfpathcurveto{\pgfqpoint{0.691667in}{0.817863in}}{\pgfqpoint{0.697813in}{0.803025in}}{\pgfqpoint{0.708752in}{0.792085in}}%
\pgfpathcurveto{\pgfqpoint{0.719691in}{0.781146in}}{\pgfqpoint{0.734530in}{0.775000in}}{\pgfqpoint{0.750000in}{0.775000in}}%
\pgfpathclose%
\pgfpathmoveto{\pgfqpoint{0.750000in}{0.780833in}}%
\pgfpathcurveto{\pgfqpoint{0.750000in}{0.780833in}}{\pgfqpoint{0.736077in}{0.780833in}}{\pgfqpoint{0.722722in}{0.786365in}}%
\pgfpathcurveto{\pgfqpoint{0.712877in}{0.796210in}}{\pgfqpoint{0.703032in}{0.806055in}}{\pgfqpoint{0.697500in}{0.819410in}}%
\pgfpathcurveto{\pgfqpoint{0.697500in}{0.833333in}}{\pgfqpoint{0.697500in}{0.847256in}}{\pgfqpoint{0.703032in}{0.860611in}}%
\pgfpathcurveto{\pgfqpoint{0.712877in}{0.870456in}}{\pgfqpoint{0.722722in}{0.880302in}}{\pgfqpoint{0.736077in}{0.885833in}}%
\pgfpathcurveto{\pgfqpoint{0.750000in}{0.885833in}}{\pgfqpoint{0.763923in}{0.885833in}}{\pgfqpoint{0.777278in}{0.880302in}}%
\pgfpathcurveto{\pgfqpoint{0.787123in}{0.870456in}}{\pgfqpoint{0.796968in}{0.860611in}}{\pgfqpoint{0.802500in}{0.847256in}}%
\pgfpathcurveto{\pgfqpoint{0.802500in}{0.833333in}}{\pgfqpoint{0.802500in}{0.819410in}}{\pgfqpoint{0.796968in}{0.806055in}}%
\pgfpathcurveto{\pgfqpoint{0.787123in}{0.796210in}}{\pgfqpoint{0.777278in}{0.786365in}}{\pgfqpoint{0.763923in}{0.780833in}}%
\pgfpathclose%
\pgfpathmoveto{\pgfqpoint{0.916667in}{0.775000in}}%
\pgfpathcurveto{\pgfqpoint{0.932137in}{0.775000in}}{\pgfqpoint{0.946975in}{0.781146in}}{\pgfqpoint{0.957915in}{0.792085in}}%
\pgfpathcurveto{\pgfqpoint{0.968854in}{0.803025in}}{\pgfqpoint{0.975000in}{0.817863in}}{\pgfqpoint{0.975000in}{0.833333in}}%
\pgfpathcurveto{\pgfqpoint{0.975000in}{0.848804in}}{\pgfqpoint{0.968854in}{0.863642in}}{\pgfqpoint{0.957915in}{0.874581in}}%
\pgfpathcurveto{\pgfqpoint{0.946975in}{0.885520in}}{\pgfqpoint{0.932137in}{0.891667in}}{\pgfqpoint{0.916667in}{0.891667in}}%
\pgfpathcurveto{\pgfqpoint{0.901196in}{0.891667in}}{\pgfqpoint{0.886358in}{0.885520in}}{\pgfqpoint{0.875419in}{0.874581in}}%
\pgfpathcurveto{\pgfqpoint{0.864480in}{0.863642in}}{\pgfqpoint{0.858333in}{0.848804in}}{\pgfqpoint{0.858333in}{0.833333in}}%
\pgfpathcurveto{\pgfqpoint{0.858333in}{0.817863in}}{\pgfqpoint{0.864480in}{0.803025in}}{\pgfqpoint{0.875419in}{0.792085in}}%
\pgfpathcurveto{\pgfqpoint{0.886358in}{0.781146in}}{\pgfqpoint{0.901196in}{0.775000in}}{\pgfqpoint{0.916667in}{0.775000in}}%
\pgfpathclose%
\pgfpathmoveto{\pgfqpoint{0.916667in}{0.780833in}}%
\pgfpathcurveto{\pgfqpoint{0.916667in}{0.780833in}}{\pgfqpoint{0.902744in}{0.780833in}}{\pgfqpoint{0.889389in}{0.786365in}}%
\pgfpathcurveto{\pgfqpoint{0.879544in}{0.796210in}}{\pgfqpoint{0.869698in}{0.806055in}}{\pgfqpoint{0.864167in}{0.819410in}}%
\pgfpathcurveto{\pgfqpoint{0.864167in}{0.833333in}}{\pgfqpoint{0.864167in}{0.847256in}}{\pgfqpoint{0.869698in}{0.860611in}}%
\pgfpathcurveto{\pgfqpoint{0.879544in}{0.870456in}}{\pgfqpoint{0.889389in}{0.880302in}}{\pgfqpoint{0.902744in}{0.885833in}}%
\pgfpathcurveto{\pgfqpoint{0.916667in}{0.885833in}}{\pgfqpoint{0.930590in}{0.885833in}}{\pgfqpoint{0.943945in}{0.880302in}}%
\pgfpathcurveto{\pgfqpoint{0.953790in}{0.870456in}}{\pgfqpoint{0.963635in}{0.860611in}}{\pgfqpoint{0.969167in}{0.847256in}}%
\pgfpathcurveto{\pgfqpoint{0.969167in}{0.833333in}}{\pgfqpoint{0.969167in}{0.819410in}}{\pgfqpoint{0.963635in}{0.806055in}}%
\pgfpathcurveto{\pgfqpoint{0.953790in}{0.796210in}}{\pgfqpoint{0.943945in}{0.786365in}}{\pgfqpoint{0.930590in}{0.780833in}}%
\pgfpathclose%
\pgfpathmoveto{\pgfqpoint{0.000000in}{0.941667in}}%
\pgfpathcurveto{\pgfqpoint{0.015470in}{0.941667in}}{\pgfqpoint{0.030309in}{0.947813in}}{\pgfqpoint{0.041248in}{0.958752in}}%
\pgfpathcurveto{\pgfqpoint{0.052187in}{0.969691in}}{\pgfqpoint{0.058333in}{0.984530in}}{\pgfqpoint{0.058333in}{1.000000in}}%
\pgfpathcurveto{\pgfqpoint{0.058333in}{1.015470in}}{\pgfqpoint{0.052187in}{1.030309in}}{\pgfqpoint{0.041248in}{1.041248in}}%
\pgfpathcurveto{\pgfqpoint{0.030309in}{1.052187in}}{\pgfqpoint{0.015470in}{1.058333in}}{\pgfqpoint{0.000000in}{1.058333in}}%
\pgfpathcurveto{\pgfqpoint{-0.015470in}{1.058333in}}{\pgfqpoint{-0.030309in}{1.052187in}}{\pgfqpoint{-0.041248in}{1.041248in}}%
\pgfpathcurveto{\pgfqpoint{-0.052187in}{1.030309in}}{\pgfqpoint{-0.058333in}{1.015470in}}{\pgfqpoint{-0.058333in}{1.000000in}}%
\pgfpathcurveto{\pgfqpoint{-0.058333in}{0.984530in}}{\pgfqpoint{-0.052187in}{0.969691in}}{\pgfqpoint{-0.041248in}{0.958752in}}%
\pgfpathcurveto{\pgfqpoint{-0.030309in}{0.947813in}}{\pgfqpoint{-0.015470in}{0.941667in}}{\pgfqpoint{0.000000in}{0.941667in}}%
\pgfpathclose%
\pgfpathmoveto{\pgfqpoint{0.000000in}{0.947500in}}%
\pgfpathcurveto{\pgfqpoint{0.000000in}{0.947500in}}{\pgfqpoint{-0.013923in}{0.947500in}}{\pgfqpoint{-0.027278in}{0.953032in}}%
\pgfpathcurveto{\pgfqpoint{-0.037123in}{0.962877in}}{\pgfqpoint{-0.046968in}{0.972722in}}{\pgfqpoint{-0.052500in}{0.986077in}}%
\pgfpathcurveto{\pgfqpoint{-0.052500in}{1.000000in}}{\pgfqpoint{-0.052500in}{1.013923in}}{\pgfqpoint{-0.046968in}{1.027278in}}%
\pgfpathcurveto{\pgfqpoint{-0.037123in}{1.037123in}}{\pgfqpoint{-0.027278in}{1.046968in}}{\pgfqpoint{-0.013923in}{1.052500in}}%
\pgfpathcurveto{\pgfqpoint{0.000000in}{1.052500in}}{\pgfqpoint{0.013923in}{1.052500in}}{\pgfqpoint{0.027278in}{1.046968in}}%
\pgfpathcurveto{\pgfqpoint{0.037123in}{1.037123in}}{\pgfqpoint{0.046968in}{1.027278in}}{\pgfqpoint{0.052500in}{1.013923in}}%
\pgfpathcurveto{\pgfqpoint{0.052500in}{1.000000in}}{\pgfqpoint{0.052500in}{0.986077in}}{\pgfqpoint{0.046968in}{0.972722in}}%
\pgfpathcurveto{\pgfqpoint{0.037123in}{0.962877in}}{\pgfqpoint{0.027278in}{0.953032in}}{\pgfqpoint{0.013923in}{0.947500in}}%
\pgfpathclose%
\pgfpathmoveto{\pgfqpoint{0.166667in}{0.941667in}}%
\pgfpathcurveto{\pgfqpoint{0.182137in}{0.941667in}}{\pgfqpoint{0.196975in}{0.947813in}}{\pgfqpoint{0.207915in}{0.958752in}}%
\pgfpathcurveto{\pgfqpoint{0.218854in}{0.969691in}}{\pgfqpoint{0.225000in}{0.984530in}}{\pgfqpoint{0.225000in}{1.000000in}}%
\pgfpathcurveto{\pgfqpoint{0.225000in}{1.015470in}}{\pgfqpoint{0.218854in}{1.030309in}}{\pgfqpoint{0.207915in}{1.041248in}}%
\pgfpathcurveto{\pgfqpoint{0.196975in}{1.052187in}}{\pgfqpoint{0.182137in}{1.058333in}}{\pgfqpoint{0.166667in}{1.058333in}}%
\pgfpathcurveto{\pgfqpoint{0.151196in}{1.058333in}}{\pgfqpoint{0.136358in}{1.052187in}}{\pgfqpoint{0.125419in}{1.041248in}}%
\pgfpathcurveto{\pgfqpoint{0.114480in}{1.030309in}}{\pgfqpoint{0.108333in}{1.015470in}}{\pgfqpoint{0.108333in}{1.000000in}}%
\pgfpathcurveto{\pgfqpoint{0.108333in}{0.984530in}}{\pgfqpoint{0.114480in}{0.969691in}}{\pgfqpoint{0.125419in}{0.958752in}}%
\pgfpathcurveto{\pgfqpoint{0.136358in}{0.947813in}}{\pgfqpoint{0.151196in}{0.941667in}}{\pgfqpoint{0.166667in}{0.941667in}}%
\pgfpathclose%
\pgfpathmoveto{\pgfqpoint{0.166667in}{0.947500in}}%
\pgfpathcurveto{\pgfqpoint{0.166667in}{0.947500in}}{\pgfqpoint{0.152744in}{0.947500in}}{\pgfqpoint{0.139389in}{0.953032in}}%
\pgfpathcurveto{\pgfqpoint{0.129544in}{0.962877in}}{\pgfqpoint{0.119698in}{0.972722in}}{\pgfqpoint{0.114167in}{0.986077in}}%
\pgfpathcurveto{\pgfqpoint{0.114167in}{1.000000in}}{\pgfqpoint{0.114167in}{1.013923in}}{\pgfqpoint{0.119698in}{1.027278in}}%
\pgfpathcurveto{\pgfqpoint{0.129544in}{1.037123in}}{\pgfqpoint{0.139389in}{1.046968in}}{\pgfqpoint{0.152744in}{1.052500in}}%
\pgfpathcurveto{\pgfqpoint{0.166667in}{1.052500in}}{\pgfqpoint{0.180590in}{1.052500in}}{\pgfqpoint{0.193945in}{1.046968in}}%
\pgfpathcurveto{\pgfqpoint{0.203790in}{1.037123in}}{\pgfqpoint{0.213635in}{1.027278in}}{\pgfqpoint{0.219167in}{1.013923in}}%
\pgfpathcurveto{\pgfqpoint{0.219167in}{1.000000in}}{\pgfqpoint{0.219167in}{0.986077in}}{\pgfqpoint{0.213635in}{0.972722in}}%
\pgfpathcurveto{\pgfqpoint{0.203790in}{0.962877in}}{\pgfqpoint{0.193945in}{0.953032in}}{\pgfqpoint{0.180590in}{0.947500in}}%
\pgfpathclose%
\pgfpathmoveto{\pgfqpoint{0.333333in}{0.941667in}}%
\pgfpathcurveto{\pgfqpoint{0.348804in}{0.941667in}}{\pgfqpoint{0.363642in}{0.947813in}}{\pgfqpoint{0.374581in}{0.958752in}}%
\pgfpathcurveto{\pgfqpoint{0.385520in}{0.969691in}}{\pgfqpoint{0.391667in}{0.984530in}}{\pgfqpoint{0.391667in}{1.000000in}}%
\pgfpathcurveto{\pgfqpoint{0.391667in}{1.015470in}}{\pgfqpoint{0.385520in}{1.030309in}}{\pgfqpoint{0.374581in}{1.041248in}}%
\pgfpathcurveto{\pgfqpoint{0.363642in}{1.052187in}}{\pgfqpoint{0.348804in}{1.058333in}}{\pgfqpoint{0.333333in}{1.058333in}}%
\pgfpathcurveto{\pgfqpoint{0.317863in}{1.058333in}}{\pgfqpoint{0.303025in}{1.052187in}}{\pgfqpoint{0.292085in}{1.041248in}}%
\pgfpathcurveto{\pgfqpoint{0.281146in}{1.030309in}}{\pgfqpoint{0.275000in}{1.015470in}}{\pgfqpoint{0.275000in}{1.000000in}}%
\pgfpathcurveto{\pgfqpoint{0.275000in}{0.984530in}}{\pgfqpoint{0.281146in}{0.969691in}}{\pgfqpoint{0.292085in}{0.958752in}}%
\pgfpathcurveto{\pgfqpoint{0.303025in}{0.947813in}}{\pgfqpoint{0.317863in}{0.941667in}}{\pgfqpoint{0.333333in}{0.941667in}}%
\pgfpathclose%
\pgfpathmoveto{\pgfqpoint{0.333333in}{0.947500in}}%
\pgfpathcurveto{\pgfqpoint{0.333333in}{0.947500in}}{\pgfqpoint{0.319410in}{0.947500in}}{\pgfqpoint{0.306055in}{0.953032in}}%
\pgfpathcurveto{\pgfqpoint{0.296210in}{0.962877in}}{\pgfqpoint{0.286365in}{0.972722in}}{\pgfqpoint{0.280833in}{0.986077in}}%
\pgfpathcurveto{\pgfqpoint{0.280833in}{1.000000in}}{\pgfqpoint{0.280833in}{1.013923in}}{\pgfqpoint{0.286365in}{1.027278in}}%
\pgfpathcurveto{\pgfqpoint{0.296210in}{1.037123in}}{\pgfqpoint{0.306055in}{1.046968in}}{\pgfqpoint{0.319410in}{1.052500in}}%
\pgfpathcurveto{\pgfqpoint{0.333333in}{1.052500in}}{\pgfqpoint{0.347256in}{1.052500in}}{\pgfqpoint{0.360611in}{1.046968in}}%
\pgfpathcurveto{\pgfqpoint{0.370456in}{1.037123in}}{\pgfqpoint{0.380302in}{1.027278in}}{\pgfqpoint{0.385833in}{1.013923in}}%
\pgfpathcurveto{\pgfqpoint{0.385833in}{1.000000in}}{\pgfqpoint{0.385833in}{0.986077in}}{\pgfqpoint{0.380302in}{0.972722in}}%
\pgfpathcurveto{\pgfqpoint{0.370456in}{0.962877in}}{\pgfqpoint{0.360611in}{0.953032in}}{\pgfqpoint{0.347256in}{0.947500in}}%
\pgfpathclose%
\pgfpathmoveto{\pgfqpoint{0.500000in}{0.941667in}}%
\pgfpathcurveto{\pgfqpoint{0.515470in}{0.941667in}}{\pgfqpoint{0.530309in}{0.947813in}}{\pgfqpoint{0.541248in}{0.958752in}}%
\pgfpathcurveto{\pgfqpoint{0.552187in}{0.969691in}}{\pgfqpoint{0.558333in}{0.984530in}}{\pgfqpoint{0.558333in}{1.000000in}}%
\pgfpathcurveto{\pgfqpoint{0.558333in}{1.015470in}}{\pgfqpoint{0.552187in}{1.030309in}}{\pgfqpoint{0.541248in}{1.041248in}}%
\pgfpathcurveto{\pgfqpoint{0.530309in}{1.052187in}}{\pgfqpoint{0.515470in}{1.058333in}}{\pgfqpoint{0.500000in}{1.058333in}}%
\pgfpathcurveto{\pgfqpoint{0.484530in}{1.058333in}}{\pgfqpoint{0.469691in}{1.052187in}}{\pgfqpoint{0.458752in}{1.041248in}}%
\pgfpathcurveto{\pgfqpoint{0.447813in}{1.030309in}}{\pgfqpoint{0.441667in}{1.015470in}}{\pgfqpoint{0.441667in}{1.000000in}}%
\pgfpathcurveto{\pgfqpoint{0.441667in}{0.984530in}}{\pgfqpoint{0.447813in}{0.969691in}}{\pgfqpoint{0.458752in}{0.958752in}}%
\pgfpathcurveto{\pgfqpoint{0.469691in}{0.947813in}}{\pgfqpoint{0.484530in}{0.941667in}}{\pgfqpoint{0.500000in}{0.941667in}}%
\pgfpathclose%
\pgfpathmoveto{\pgfqpoint{0.500000in}{0.947500in}}%
\pgfpathcurveto{\pgfqpoint{0.500000in}{0.947500in}}{\pgfqpoint{0.486077in}{0.947500in}}{\pgfqpoint{0.472722in}{0.953032in}}%
\pgfpathcurveto{\pgfqpoint{0.462877in}{0.962877in}}{\pgfqpoint{0.453032in}{0.972722in}}{\pgfqpoint{0.447500in}{0.986077in}}%
\pgfpathcurveto{\pgfqpoint{0.447500in}{1.000000in}}{\pgfqpoint{0.447500in}{1.013923in}}{\pgfqpoint{0.453032in}{1.027278in}}%
\pgfpathcurveto{\pgfqpoint{0.462877in}{1.037123in}}{\pgfqpoint{0.472722in}{1.046968in}}{\pgfqpoint{0.486077in}{1.052500in}}%
\pgfpathcurveto{\pgfqpoint{0.500000in}{1.052500in}}{\pgfqpoint{0.513923in}{1.052500in}}{\pgfqpoint{0.527278in}{1.046968in}}%
\pgfpathcurveto{\pgfqpoint{0.537123in}{1.037123in}}{\pgfqpoint{0.546968in}{1.027278in}}{\pgfqpoint{0.552500in}{1.013923in}}%
\pgfpathcurveto{\pgfqpoint{0.552500in}{1.000000in}}{\pgfqpoint{0.552500in}{0.986077in}}{\pgfqpoint{0.546968in}{0.972722in}}%
\pgfpathcurveto{\pgfqpoint{0.537123in}{0.962877in}}{\pgfqpoint{0.527278in}{0.953032in}}{\pgfqpoint{0.513923in}{0.947500in}}%
\pgfpathclose%
\pgfpathmoveto{\pgfqpoint{0.666667in}{0.941667in}}%
\pgfpathcurveto{\pgfqpoint{0.682137in}{0.941667in}}{\pgfqpoint{0.696975in}{0.947813in}}{\pgfqpoint{0.707915in}{0.958752in}}%
\pgfpathcurveto{\pgfqpoint{0.718854in}{0.969691in}}{\pgfqpoint{0.725000in}{0.984530in}}{\pgfqpoint{0.725000in}{1.000000in}}%
\pgfpathcurveto{\pgfqpoint{0.725000in}{1.015470in}}{\pgfqpoint{0.718854in}{1.030309in}}{\pgfqpoint{0.707915in}{1.041248in}}%
\pgfpathcurveto{\pgfqpoint{0.696975in}{1.052187in}}{\pgfqpoint{0.682137in}{1.058333in}}{\pgfqpoint{0.666667in}{1.058333in}}%
\pgfpathcurveto{\pgfqpoint{0.651196in}{1.058333in}}{\pgfqpoint{0.636358in}{1.052187in}}{\pgfqpoint{0.625419in}{1.041248in}}%
\pgfpathcurveto{\pgfqpoint{0.614480in}{1.030309in}}{\pgfqpoint{0.608333in}{1.015470in}}{\pgfqpoint{0.608333in}{1.000000in}}%
\pgfpathcurveto{\pgfqpoint{0.608333in}{0.984530in}}{\pgfqpoint{0.614480in}{0.969691in}}{\pgfqpoint{0.625419in}{0.958752in}}%
\pgfpathcurveto{\pgfqpoint{0.636358in}{0.947813in}}{\pgfqpoint{0.651196in}{0.941667in}}{\pgfqpoint{0.666667in}{0.941667in}}%
\pgfpathclose%
\pgfpathmoveto{\pgfqpoint{0.666667in}{0.947500in}}%
\pgfpathcurveto{\pgfqpoint{0.666667in}{0.947500in}}{\pgfqpoint{0.652744in}{0.947500in}}{\pgfqpoint{0.639389in}{0.953032in}}%
\pgfpathcurveto{\pgfqpoint{0.629544in}{0.962877in}}{\pgfqpoint{0.619698in}{0.972722in}}{\pgfqpoint{0.614167in}{0.986077in}}%
\pgfpathcurveto{\pgfqpoint{0.614167in}{1.000000in}}{\pgfqpoint{0.614167in}{1.013923in}}{\pgfqpoint{0.619698in}{1.027278in}}%
\pgfpathcurveto{\pgfqpoint{0.629544in}{1.037123in}}{\pgfqpoint{0.639389in}{1.046968in}}{\pgfqpoint{0.652744in}{1.052500in}}%
\pgfpathcurveto{\pgfqpoint{0.666667in}{1.052500in}}{\pgfqpoint{0.680590in}{1.052500in}}{\pgfqpoint{0.693945in}{1.046968in}}%
\pgfpathcurveto{\pgfqpoint{0.703790in}{1.037123in}}{\pgfqpoint{0.713635in}{1.027278in}}{\pgfqpoint{0.719167in}{1.013923in}}%
\pgfpathcurveto{\pgfqpoint{0.719167in}{1.000000in}}{\pgfqpoint{0.719167in}{0.986077in}}{\pgfqpoint{0.713635in}{0.972722in}}%
\pgfpathcurveto{\pgfqpoint{0.703790in}{0.962877in}}{\pgfqpoint{0.693945in}{0.953032in}}{\pgfqpoint{0.680590in}{0.947500in}}%
\pgfpathclose%
\pgfpathmoveto{\pgfqpoint{0.833333in}{0.941667in}}%
\pgfpathcurveto{\pgfqpoint{0.848804in}{0.941667in}}{\pgfqpoint{0.863642in}{0.947813in}}{\pgfqpoint{0.874581in}{0.958752in}}%
\pgfpathcurveto{\pgfqpoint{0.885520in}{0.969691in}}{\pgfqpoint{0.891667in}{0.984530in}}{\pgfqpoint{0.891667in}{1.000000in}}%
\pgfpathcurveto{\pgfqpoint{0.891667in}{1.015470in}}{\pgfqpoint{0.885520in}{1.030309in}}{\pgfqpoint{0.874581in}{1.041248in}}%
\pgfpathcurveto{\pgfqpoint{0.863642in}{1.052187in}}{\pgfqpoint{0.848804in}{1.058333in}}{\pgfqpoint{0.833333in}{1.058333in}}%
\pgfpathcurveto{\pgfqpoint{0.817863in}{1.058333in}}{\pgfqpoint{0.803025in}{1.052187in}}{\pgfqpoint{0.792085in}{1.041248in}}%
\pgfpathcurveto{\pgfqpoint{0.781146in}{1.030309in}}{\pgfqpoint{0.775000in}{1.015470in}}{\pgfqpoint{0.775000in}{1.000000in}}%
\pgfpathcurveto{\pgfqpoint{0.775000in}{0.984530in}}{\pgfqpoint{0.781146in}{0.969691in}}{\pgfqpoint{0.792085in}{0.958752in}}%
\pgfpathcurveto{\pgfqpoint{0.803025in}{0.947813in}}{\pgfqpoint{0.817863in}{0.941667in}}{\pgfqpoint{0.833333in}{0.941667in}}%
\pgfpathclose%
\pgfpathmoveto{\pgfqpoint{0.833333in}{0.947500in}}%
\pgfpathcurveto{\pgfqpoint{0.833333in}{0.947500in}}{\pgfqpoint{0.819410in}{0.947500in}}{\pgfqpoint{0.806055in}{0.953032in}}%
\pgfpathcurveto{\pgfqpoint{0.796210in}{0.962877in}}{\pgfqpoint{0.786365in}{0.972722in}}{\pgfqpoint{0.780833in}{0.986077in}}%
\pgfpathcurveto{\pgfqpoint{0.780833in}{1.000000in}}{\pgfqpoint{0.780833in}{1.013923in}}{\pgfqpoint{0.786365in}{1.027278in}}%
\pgfpathcurveto{\pgfqpoint{0.796210in}{1.037123in}}{\pgfqpoint{0.806055in}{1.046968in}}{\pgfqpoint{0.819410in}{1.052500in}}%
\pgfpathcurveto{\pgfqpoint{0.833333in}{1.052500in}}{\pgfqpoint{0.847256in}{1.052500in}}{\pgfqpoint{0.860611in}{1.046968in}}%
\pgfpathcurveto{\pgfqpoint{0.870456in}{1.037123in}}{\pgfqpoint{0.880302in}{1.027278in}}{\pgfqpoint{0.885833in}{1.013923in}}%
\pgfpathcurveto{\pgfqpoint{0.885833in}{1.000000in}}{\pgfqpoint{0.885833in}{0.986077in}}{\pgfqpoint{0.880302in}{0.972722in}}%
\pgfpathcurveto{\pgfqpoint{0.870456in}{0.962877in}}{\pgfqpoint{0.860611in}{0.953032in}}{\pgfqpoint{0.847256in}{0.947500in}}%
\pgfpathclose%
\pgfpathmoveto{\pgfqpoint{1.000000in}{0.941667in}}%
\pgfpathcurveto{\pgfqpoint{1.015470in}{0.941667in}}{\pgfqpoint{1.030309in}{0.947813in}}{\pgfqpoint{1.041248in}{0.958752in}}%
\pgfpathcurveto{\pgfqpoint{1.052187in}{0.969691in}}{\pgfqpoint{1.058333in}{0.984530in}}{\pgfqpoint{1.058333in}{1.000000in}}%
\pgfpathcurveto{\pgfqpoint{1.058333in}{1.015470in}}{\pgfqpoint{1.052187in}{1.030309in}}{\pgfqpoint{1.041248in}{1.041248in}}%
\pgfpathcurveto{\pgfqpoint{1.030309in}{1.052187in}}{\pgfqpoint{1.015470in}{1.058333in}}{\pgfqpoint{1.000000in}{1.058333in}}%
\pgfpathcurveto{\pgfqpoint{0.984530in}{1.058333in}}{\pgfqpoint{0.969691in}{1.052187in}}{\pgfqpoint{0.958752in}{1.041248in}}%
\pgfpathcurveto{\pgfqpoint{0.947813in}{1.030309in}}{\pgfqpoint{0.941667in}{1.015470in}}{\pgfqpoint{0.941667in}{1.000000in}}%
\pgfpathcurveto{\pgfqpoint{0.941667in}{0.984530in}}{\pgfqpoint{0.947813in}{0.969691in}}{\pgfqpoint{0.958752in}{0.958752in}}%
\pgfpathcurveto{\pgfqpoint{0.969691in}{0.947813in}}{\pgfqpoint{0.984530in}{0.941667in}}{\pgfqpoint{1.000000in}{0.941667in}}%
\pgfpathclose%
\pgfpathmoveto{\pgfqpoint{1.000000in}{0.947500in}}%
\pgfpathcurveto{\pgfqpoint{1.000000in}{0.947500in}}{\pgfqpoint{0.986077in}{0.947500in}}{\pgfqpoint{0.972722in}{0.953032in}}%
\pgfpathcurveto{\pgfqpoint{0.962877in}{0.962877in}}{\pgfqpoint{0.953032in}{0.972722in}}{\pgfqpoint{0.947500in}{0.986077in}}%
\pgfpathcurveto{\pgfqpoint{0.947500in}{1.000000in}}{\pgfqpoint{0.947500in}{1.013923in}}{\pgfqpoint{0.953032in}{1.027278in}}%
\pgfpathcurveto{\pgfqpoint{0.962877in}{1.037123in}}{\pgfqpoint{0.972722in}{1.046968in}}{\pgfqpoint{0.986077in}{1.052500in}}%
\pgfpathcurveto{\pgfqpoint{1.000000in}{1.052500in}}{\pgfqpoint{1.013923in}{1.052500in}}{\pgfqpoint{1.027278in}{1.046968in}}%
\pgfpathcurveto{\pgfqpoint{1.037123in}{1.037123in}}{\pgfqpoint{1.046968in}{1.027278in}}{\pgfqpoint{1.052500in}{1.013923in}}%
\pgfpathcurveto{\pgfqpoint{1.052500in}{1.000000in}}{\pgfqpoint{1.052500in}{0.986077in}}{\pgfqpoint{1.046968in}{0.972722in}}%
\pgfpathcurveto{\pgfqpoint{1.037123in}{0.962877in}}{\pgfqpoint{1.027278in}{0.953032in}}{\pgfqpoint{1.013923in}{0.947500in}}%
\pgfpathclose%
\pgfusepath{stroke}%
\end{pgfscope}%
}%
\pgfsys@transformshift{3.028174in}{6.036360in}%
\pgfsys@useobject{currentpattern}{}%
\pgfsys@transformshift{1in}{0in}%
\pgfsys@transformshift{-1in}{0in}%
\pgfsys@transformshift{0in}{1in}%
\pgfsys@useobject{currentpattern}{}%
\pgfsys@transformshift{1in}{0in}%
\pgfsys@transformshift{-1in}{0in}%
\pgfsys@transformshift{0in}{1in}%
\end{pgfscope}%
\begin{pgfscope}%
\pgfpathrectangle{\pgfqpoint{1.090674in}{0.637495in}}{\pgfqpoint{9.300000in}{9.060000in}}%
\pgfusepath{clip}%
\pgfsetbuttcap%
\pgfsetmiterjoin%
\definecolor{currentfill}{rgb}{0.890196,0.466667,0.760784}%
\pgfsetfillcolor{currentfill}%
\pgfsetfillopacity{0.990000}%
\pgfsetlinewidth{0.000000pt}%
\definecolor{currentstroke}{rgb}{0.000000,0.000000,0.000000}%
\pgfsetstrokecolor{currentstroke}%
\pgfsetstrokeopacity{0.990000}%
\pgfsetdash{}{0pt}%
\pgfpathmoveto{\pgfqpoint{4.578174in}{6.293449in}}%
\pgfpathlineto{\pgfqpoint{5.353174in}{6.293449in}}%
\pgfpathlineto{\pgfqpoint{5.353174in}{8.230638in}}%
\pgfpathlineto{\pgfqpoint{4.578174in}{8.230638in}}%
\pgfpathclose%
\pgfusepath{fill}%
\end{pgfscope}%
\begin{pgfscope}%
\pgfsetbuttcap%
\pgfsetmiterjoin%
\definecolor{currentfill}{rgb}{0.890196,0.466667,0.760784}%
\pgfsetfillcolor{currentfill}%
\pgfsetfillopacity{0.990000}%
\pgfsetlinewidth{0.000000pt}%
\definecolor{currentstroke}{rgb}{0.000000,0.000000,0.000000}%
\pgfsetstrokecolor{currentstroke}%
\pgfsetstrokeopacity{0.990000}%
\pgfsetdash{}{0pt}%
\pgfpathrectangle{\pgfqpoint{1.090674in}{0.637495in}}{\pgfqpoint{9.300000in}{9.060000in}}%
\pgfusepath{clip}%
\pgfpathmoveto{\pgfqpoint{4.578174in}{6.293449in}}%
\pgfpathlineto{\pgfqpoint{5.353174in}{6.293449in}}%
\pgfpathlineto{\pgfqpoint{5.353174in}{8.230638in}}%
\pgfpathlineto{\pgfqpoint{4.578174in}{8.230638in}}%
\pgfpathclose%
\pgfusepath{clip}%
\pgfsys@defobject{currentpattern}{\pgfqpoint{0in}{0in}}{\pgfqpoint{1in}{1in}}{%
\begin{pgfscope}%
\pgfpathrectangle{\pgfqpoint{0in}{0in}}{\pgfqpoint{1in}{1in}}%
\pgfusepath{clip}%
\pgfpathmoveto{\pgfqpoint{0.000000in}{-0.058333in}}%
\pgfpathcurveto{\pgfqpoint{0.015470in}{-0.058333in}}{\pgfqpoint{0.030309in}{-0.052187in}}{\pgfqpoint{0.041248in}{-0.041248in}}%
\pgfpathcurveto{\pgfqpoint{0.052187in}{-0.030309in}}{\pgfqpoint{0.058333in}{-0.015470in}}{\pgfqpoint{0.058333in}{0.000000in}}%
\pgfpathcurveto{\pgfqpoint{0.058333in}{0.015470in}}{\pgfqpoint{0.052187in}{0.030309in}}{\pgfqpoint{0.041248in}{0.041248in}}%
\pgfpathcurveto{\pgfqpoint{0.030309in}{0.052187in}}{\pgfqpoint{0.015470in}{0.058333in}}{\pgfqpoint{0.000000in}{0.058333in}}%
\pgfpathcurveto{\pgfqpoint{-0.015470in}{0.058333in}}{\pgfqpoint{-0.030309in}{0.052187in}}{\pgfqpoint{-0.041248in}{0.041248in}}%
\pgfpathcurveto{\pgfqpoint{-0.052187in}{0.030309in}}{\pgfqpoint{-0.058333in}{0.015470in}}{\pgfqpoint{-0.058333in}{0.000000in}}%
\pgfpathcurveto{\pgfqpoint{-0.058333in}{-0.015470in}}{\pgfqpoint{-0.052187in}{-0.030309in}}{\pgfqpoint{-0.041248in}{-0.041248in}}%
\pgfpathcurveto{\pgfqpoint{-0.030309in}{-0.052187in}}{\pgfqpoint{-0.015470in}{-0.058333in}}{\pgfqpoint{0.000000in}{-0.058333in}}%
\pgfpathclose%
\pgfpathmoveto{\pgfqpoint{0.000000in}{-0.052500in}}%
\pgfpathcurveto{\pgfqpoint{0.000000in}{-0.052500in}}{\pgfqpoint{-0.013923in}{-0.052500in}}{\pgfqpoint{-0.027278in}{-0.046968in}}%
\pgfpathcurveto{\pgfqpoint{-0.037123in}{-0.037123in}}{\pgfqpoint{-0.046968in}{-0.027278in}}{\pgfqpoint{-0.052500in}{-0.013923in}}%
\pgfpathcurveto{\pgfqpoint{-0.052500in}{0.000000in}}{\pgfqpoint{-0.052500in}{0.013923in}}{\pgfqpoint{-0.046968in}{0.027278in}}%
\pgfpathcurveto{\pgfqpoint{-0.037123in}{0.037123in}}{\pgfqpoint{-0.027278in}{0.046968in}}{\pgfqpoint{-0.013923in}{0.052500in}}%
\pgfpathcurveto{\pgfqpoint{0.000000in}{0.052500in}}{\pgfqpoint{0.013923in}{0.052500in}}{\pgfqpoint{0.027278in}{0.046968in}}%
\pgfpathcurveto{\pgfqpoint{0.037123in}{0.037123in}}{\pgfqpoint{0.046968in}{0.027278in}}{\pgfqpoint{0.052500in}{0.013923in}}%
\pgfpathcurveto{\pgfqpoint{0.052500in}{0.000000in}}{\pgfqpoint{0.052500in}{-0.013923in}}{\pgfqpoint{0.046968in}{-0.027278in}}%
\pgfpathcurveto{\pgfqpoint{0.037123in}{-0.037123in}}{\pgfqpoint{0.027278in}{-0.046968in}}{\pgfqpoint{0.013923in}{-0.052500in}}%
\pgfpathclose%
\pgfpathmoveto{\pgfqpoint{0.166667in}{-0.058333in}}%
\pgfpathcurveto{\pgfqpoint{0.182137in}{-0.058333in}}{\pgfqpoint{0.196975in}{-0.052187in}}{\pgfqpoint{0.207915in}{-0.041248in}}%
\pgfpathcurveto{\pgfqpoint{0.218854in}{-0.030309in}}{\pgfqpoint{0.225000in}{-0.015470in}}{\pgfqpoint{0.225000in}{0.000000in}}%
\pgfpathcurveto{\pgfqpoint{0.225000in}{0.015470in}}{\pgfqpoint{0.218854in}{0.030309in}}{\pgfqpoint{0.207915in}{0.041248in}}%
\pgfpathcurveto{\pgfqpoint{0.196975in}{0.052187in}}{\pgfqpoint{0.182137in}{0.058333in}}{\pgfqpoint{0.166667in}{0.058333in}}%
\pgfpathcurveto{\pgfqpoint{0.151196in}{0.058333in}}{\pgfqpoint{0.136358in}{0.052187in}}{\pgfqpoint{0.125419in}{0.041248in}}%
\pgfpathcurveto{\pgfqpoint{0.114480in}{0.030309in}}{\pgfqpoint{0.108333in}{0.015470in}}{\pgfqpoint{0.108333in}{0.000000in}}%
\pgfpathcurveto{\pgfqpoint{0.108333in}{-0.015470in}}{\pgfqpoint{0.114480in}{-0.030309in}}{\pgfqpoint{0.125419in}{-0.041248in}}%
\pgfpathcurveto{\pgfqpoint{0.136358in}{-0.052187in}}{\pgfqpoint{0.151196in}{-0.058333in}}{\pgfqpoint{0.166667in}{-0.058333in}}%
\pgfpathclose%
\pgfpathmoveto{\pgfqpoint{0.166667in}{-0.052500in}}%
\pgfpathcurveto{\pgfqpoint{0.166667in}{-0.052500in}}{\pgfqpoint{0.152744in}{-0.052500in}}{\pgfqpoint{0.139389in}{-0.046968in}}%
\pgfpathcurveto{\pgfqpoint{0.129544in}{-0.037123in}}{\pgfqpoint{0.119698in}{-0.027278in}}{\pgfqpoint{0.114167in}{-0.013923in}}%
\pgfpathcurveto{\pgfqpoint{0.114167in}{0.000000in}}{\pgfqpoint{0.114167in}{0.013923in}}{\pgfqpoint{0.119698in}{0.027278in}}%
\pgfpathcurveto{\pgfqpoint{0.129544in}{0.037123in}}{\pgfqpoint{0.139389in}{0.046968in}}{\pgfqpoint{0.152744in}{0.052500in}}%
\pgfpathcurveto{\pgfqpoint{0.166667in}{0.052500in}}{\pgfqpoint{0.180590in}{0.052500in}}{\pgfqpoint{0.193945in}{0.046968in}}%
\pgfpathcurveto{\pgfqpoint{0.203790in}{0.037123in}}{\pgfqpoint{0.213635in}{0.027278in}}{\pgfqpoint{0.219167in}{0.013923in}}%
\pgfpathcurveto{\pgfqpoint{0.219167in}{0.000000in}}{\pgfqpoint{0.219167in}{-0.013923in}}{\pgfqpoint{0.213635in}{-0.027278in}}%
\pgfpathcurveto{\pgfqpoint{0.203790in}{-0.037123in}}{\pgfqpoint{0.193945in}{-0.046968in}}{\pgfqpoint{0.180590in}{-0.052500in}}%
\pgfpathclose%
\pgfpathmoveto{\pgfqpoint{0.333333in}{-0.058333in}}%
\pgfpathcurveto{\pgfqpoint{0.348804in}{-0.058333in}}{\pgfqpoint{0.363642in}{-0.052187in}}{\pgfqpoint{0.374581in}{-0.041248in}}%
\pgfpathcurveto{\pgfqpoint{0.385520in}{-0.030309in}}{\pgfqpoint{0.391667in}{-0.015470in}}{\pgfqpoint{0.391667in}{0.000000in}}%
\pgfpathcurveto{\pgfqpoint{0.391667in}{0.015470in}}{\pgfqpoint{0.385520in}{0.030309in}}{\pgfqpoint{0.374581in}{0.041248in}}%
\pgfpathcurveto{\pgfqpoint{0.363642in}{0.052187in}}{\pgfqpoint{0.348804in}{0.058333in}}{\pgfqpoint{0.333333in}{0.058333in}}%
\pgfpathcurveto{\pgfqpoint{0.317863in}{0.058333in}}{\pgfqpoint{0.303025in}{0.052187in}}{\pgfqpoint{0.292085in}{0.041248in}}%
\pgfpathcurveto{\pgfqpoint{0.281146in}{0.030309in}}{\pgfqpoint{0.275000in}{0.015470in}}{\pgfqpoint{0.275000in}{0.000000in}}%
\pgfpathcurveto{\pgfqpoint{0.275000in}{-0.015470in}}{\pgfqpoint{0.281146in}{-0.030309in}}{\pgfqpoint{0.292085in}{-0.041248in}}%
\pgfpathcurveto{\pgfqpoint{0.303025in}{-0.052187in}}{\pgfqpoint{0.317863in}{-0.058333in}}{\pgfqpoint{0.333333in}{-0.058333in}}%
\pgfpathclose%
\pgfpathmoveto{\pgfqpoint{0.333333in}{-0.052500in}}%
\pgfpathcurveto{\pgfqpoint{0.333333in}{-0.052500in}}{\pgfqpoint{0.319410in}{-0.052500in}}{\pgfqpoint{0.306055in}{-0.046968in}}%
\pgfpathcurveto{\pgfqpoint{0.296210in}{-0.037123in}}{\pgfqpoint{0.286365in}{-0.027278in}}{\pgfqpoint{0.280833in}{-0.013923in}}%
\pgfpathcurveto{\pgfqpoint{0.280833in}{0.000000in}}{\pgfqpoint{0.280833in}{0.013923in}}{\pgfqpoint{0.286365in}{0.027278in}}%
\pgfpathcurveto{\pgfqpoint{0.296210in}{0.037123in}}{\pgfqpoint{0.306055in}{0.046968in}}{\pgfqpoint{0.319410in}{0.052500in}}%
\pgfpathcurveto{\pgfqpoint{0.333333in}{0.052500in}}{\pgfqpoint{0.347256in}{0.052500in}}{\pgfqpoint{0.360611in}{0.046968in}}%
\pgfpathcurveto{\pgfqpoint{0.370456in}{0.037123in}}{\pgfqpoint{0.380302in}{0.027278in}}{\pgfqpoint{0.385833in}{0.013923in}}%
\pgfpathcurveto{\pgfqpoint{0.385833in}{0.000000in}}{\pgfqpoint{0.385833in}{-0.013923in}}{\pgfqpoint{0.380302in}{-0.027278in}}%
\pgfpathcurveto{\pgfqpoint{0.370456in}{-0.037123in}}{\pgfqpoint{0.360611in}{-0.046968in}}{\pgfqpoint{0.347256in}{-0.052500in}}%
\pgfpathclose%
\pgfpathmoveto{\pgfqpoint{0.500000in}{-0.058333in}}%
\pgfpathcurveto{\pgfqpoint{0.515470in}{-0.058333in}}{\pgfqpoint{0.530309in}{-0.052187in}}{\pgfqpoint{0.541248in}{-0.041248in}}%
\pgfpathcurveto{\pgfqpoint{0.552187in}{-0.030309in}}{\pgfqpoint{0.558333in}{-0.015470in}}{\pgfqpoint{0.558333in}{0.000000in}}%
\pgfpathcurveto{\pgfqpoint{0.558333in}{0.015470in}}{\pgfqpoint{0.552187in}{0.030309in}}{\pgfqpoint{0.541248in}{0.041248in}}%
\pgfpathcurveto{\pgfqpoint{0.530309in}{0.052187in}}{\pgfqpoint{0.515470in}{0.058333in}}{\pgfqpoint{0.500000in}{0.058333in}}%
\pgfpathcurveto{\pgfqpoint{0.484530in}{0.058333in}}{\pgfqpoint{0.469691in}{0.052187in}}{\pgfqpoint{0.458752in}{0.041248in}}%
\pgfpathcurveto{\pgfqpoint{0.447813in}{0.030309in}}{\pgfqpoint{0.441667in}{0.015470in}}{\pgfqpoint{0.441667in}{0.000000in}}%
\pgfpathcurveto{\pgfqpoint{0.441667in}{-0.015470in}}{\pgfqpoint{0.447813in}{-0.030309in}}{\pgfqpoint{0.458752in}{-0.041248in}}%
\pgfpathcurveto{\pgfqpoint{0.469691in}{-0.052187in}}{\pgfqpoint{0.484530in}{-0.058333in}}{\pgfqpoint{0.500000in}{-0.058333in}}%
\pgfpathclose%
\pgfpathmoveto{\pgfqpoint{0.500000in}{-0.052500in}}%
\pgfpathcurveto{\pgfqpoint{0.500000in}{-0.052500in}}{\pgfqpoint{0.486077in}{-0.052500in}}{\pgfqpoint{0.472722in}{-0.046968in}}%
\pgfpathcurveto{\pgfqpoint{0.462877in}{-0.037123in}}{\pgfqpoint{0.453032in}{-0.027278in}}{\pgfqpoint{0.447500in}{-0.013923in}}%
\pgfpathcurveto{\pgfqpoint{0.447500in}{0.000000in}}{\pgfqpoint{0.447500in}{0.013923in}}{\pgfqpoint{0.453032in}{0.027278in}}%
\pgfpathcurveto{\pgfqpoint{0.462877in}{0.037123in}}{\pgfqpoint{0.472722in}{0.046968in}}{\pgfqpoint{0.486077in}{0.052500in}}%
\pgfpathcurveto{\pgfqpoint{0.500000in}{0.052500in}}{\pgfqpoint{0.513923in}{0.052500in}}{\pgfqpoint{0.527278in}{0.046968in}}%
\pgfpathcurveto{\pgfqpoint{0.537123in}{0.037123in}}{\pgfqpoint{0.546968in}{0.027278in}}{\pgfqpoint{0.552500in}{0.013923in}}%
\pgfpathcurveto{\pgfqpoint{0.552500in}{0.000000in}}{\pgfqpoint{0.552500in}{-0.013923in}}{\pgfqpoint{0.546968in}{-0.027278in}}%
\pgfpathcurveto{\pgfqpoint{0.537123in}{-0.037123in}}{\pgfqpoint{0.527278in}{-0.046968in}}{\pgfqpoint{0.513923in}{-0.052500in}}%
\pgfpathclose%
\pgfpathmoveto{\pgfqpoint{0.666667in}{-0.058333in}}%
\pgfpathcurveto{\pgfqpoint{0.682137in}{-0.058333in}}{\pgfqpoint{0.696975in}{-0.052187in}}{\pgfqpoint{0.707915in}{-0.041248in}}%
\pgfpathcurveto{\pgfqpoint{0.718854in}{-0.030309in}}{\pgfqpoint{0.725000in}{-0.015470in}}{\pgfqpoint{0.725000in}{0.000000in}}%
\pgfpathcurveto{\pgfqpoint{0.725000in}{0.015470in}}{\pgfqpoint{0.718854in}{0.030309in}}{\pgfqpoint{0.707915in}{0.041248in}}%
\pgfpathcurveto{\pgfqpoint{0.696975in}{0.052187in}}{\pgfqpoint{0.682137in}{0.058333in}}{\pgfqpoint{0.666667in}{0.058333in}}%
\pgfpathcurveto{\pgfqpoint{0.651196in}{0.058333in}}{\pgfqpoint{0.636358in}{0.052187in}}{\pgfqpoint{0.625419in}{0.041248in}}%
\pgfpathcurveto{\pgfqpoint{0.614480in}{0.030309in}}{\pgfqpoint{0.608333in}{0.015470in}}{\pgfqpoint{0.608333in}{0.000000in}}%
\pgfpathcurveto{\pgfqpoint{0.608333in}{-0.015470in}}{\pgfqpoint{0.614480in}{-0.030309in}}{\pgfqpoint{0.625419in}{-0.041248in}}%
\pgfpathcurveto{\pgfqpoint{0.636358in}{-0.052187in}}{\pgfqpoint{0.651196in}{-0.058333in}}{\pgfqpoint{0.666667in}{-0.058333in}}%
\pgfpathclose%
\pgfpathmoveto{\pgfqpoint{0.666667in}{-0.052500in}}%
\pgfpathcurveto{\pgfqpoint{0.666667in}{-0.052500in}}{\pgfqpoint{0.652744in}{-0.052500in}}{\pgfqpoint{0.639389in}{-0.046968in}}%
\pgfpathcurveto{\pgfqpoint{0.629544in}{-0.037123in}}{\pgfqpoint{0.619698in}{-0.027278in}}{\pgfqpoint{0.614167in}{-0.013923in}}%
\pgfpathcurveto{\pgfqpoint{0.614167in}{0.000000in}}{\pgfqpoint{0.614167in}{0.013923in}}{\pgfqpoint{0.619698in}{0.027278in}}%
\pgfpathcurveto{\pgfqpoint{0.629544in}{0.037123in}}{\pgfqpoint{0.639389in}{0.046968in}}{\pgfqpoint{0.652744in}{0.052500in}}%
\pgfpathcurveto{\pgfqpoint{0.666667in}{0.052500in}}{\pgfqpoint{0.680590in}{0.052500in}}{\pgfqpoint{0.693945in}{0.046968in}}%
\pgfpathcurveto{\pgfqpoint{0.703790in}{0.037123in}}{\pgfqpoint{0.713635in}{0.027278in}}{\pgfqpoint{0.719167in}{0.013923in}}%
\pgfpathcurveto{\pgfqpoint{0.719167in}{0.000000in}}{\pgfqpoint{0.719167in}{-0.013923in}}{\pgfqpoint{0.713635in}{-0.027278in}}%
\pgfpathcurveto{\pgfqpoint{0.703790in}{-0.037123in}}{\pgfqpoint{0.693945in}{-0.046968in}}{\pgfqpoint{0.680590in}{-0.052500in}}%
\pgfpathclose%
\pgfpathmoveto{\pgfqpoint{0.833333in}{-0.058333in}}%
\pgfpathcurveto{\pgfqpoint{0.848804in}{-0.058333in}}{\pgfqpoint{0.863642in}{-0.052187in}}{\pgfqpoint{0.874581in}{-0.041248in}}%
\pgfpathcurveto{\pgfqpoint{0.885520in}{-0.030309in}}{\pgfqpoint{0.891667in}{-0.015470in}}{\pgfqpoint{0.891667in}{0.000000in}}%
\pgfpathcurveto{\pgfqpoint{0.891667in}{0.015470in}}{\pgfqpoint{0.885520in}{0.030309in}}{\pgfqpoint{0.874581in}{0.041248in}}%
\pgfpathcurveto{\pgfqpoint{0.863642in}{0.052187in}}{\pgfqpoint{0.848804in}{0.058333in}}{\pgfqpoint{0.833333in}{0.058333in}}%
\pgfpathcurveto{\pgfqpoint{0.817863in}{0.058333in}}{\pgfqpoint{0.803025in}{0.052187in}}{\pgfqpoint{0.792085in}{0.041248in}}%
\pgfpathcurveto{\pgfqpoint{0.781146in}{0.030309in}}{\pgfqpoint{0.775000in}{0.015470in}}{\pgfqpoint{0.775000in}{0.000000in}}%
\pgfpathcurveto{\pgfqpoint{0.775000in}{-0.015470in}}{\pgfqpoint{0.781146in}{-0.030309in}}{\pgfqpoint{0.792085in}{-0.041248in}}%
\pgfpathcurveto{\pgfqpoint{0.803025in}{-0.052187in}}{\pgfqpoint{0.817863in}{-0.058333in}}{\pgfqpoint{0.833333in}{-0.058333in}}%
\pgfpathclose%
\pgfpathmoveto{\pgfqpoint{0.833333in}{-0.052500in}}%
\pgfpathcurveto{\pgfqpoint{0.833333in}{-0.052500in}}{\pgfqpoint{0.819410in}{-0.052500in}}{\pgfqpoint{0.806055in}{-0.046968in}}%
\pgfpathcurveto{\pgfqpoint{0.796210in}{-0.037123in}}{\pgfqpoint{0.786365in}{-0.027278in}}{\pgfqpoint{0.780833in}{-0.013923in}}%
\pgfpathcurveto{\pgfqpoint{0.780833in}{0.000000in}}{\pgfqpoint{0.780833in}{0.013923in}}{\pgfqpoint{0.786365in}{0.027278in}}%
\pgfpathcurveto{\pgfqpoint{0.796210in}{0.037123in}}{\pgfqpoint{0.806055in}{0.046968in}}{\pgfqpoint{0.819410in}{0.052500in}}%
\pgfpathcurveto{\pgfqpoint{0.833333in}{0.052500in}}{\pgfqpoint{0.847256in}{0.052500in}}{\pgfqpoint{0.860611in}{0.046968in}}%
\pgfpathcurveto{\pgfqpoint{0.870456in}{0.037123in}}{\pgfqpoint{0.880302in}{0.027278in}}{\pgfqpoint{0.885833in}{0.013923in}}%
\pgfpathcurveto{\pgfqpoint{0.885833in}{0.000000in}}{\pgfqpoint{0.885833in}{-0.013923in}}{\pgfqpoint{0.880302in}{-0.027278in}}%
\pgfpathcurveto{\pgfqpoint{0.870456in}{-0.037123in}}{\pgfqpoint{0.860611in}{-0.046968in}}{\pgfqpoint{0.847256in}{-0.052500in}}%
\pgfpathclose%
\pgfpathmoveto{\pgfqpoint{1.000000in}{-0.058333in}}%
\pgfpathcurveto{\pgfqpoint{1.015470in}{-0.058333in}}{\pgfqpoint{1.030309in}{-0.052187in}}{\pgfqpoint{1.041248in}{-0.041248in}}%
\pgfpathcurveto{\pgfqpoint{1.052187in}{-0.030309in}}{\pgfqpoint{1.058333in}{-0.015470in}}{\pgfqpoint{1.058333in}{0.000000in}}%
\pgfpathcurveto{\pgfqpoint{1.058333in}{0.015470in}}{\pgfqpoint{1.052187in}{0.030309in}}{\pgfqpoint{1.041248in}{0.041248in}}%
\pgfpathcurveto{\pgfqpoint{1.030309in}{0.052187in}}{\pgfqpoint{1.015470in}{0.058333in}}{\pgfqpoint{1.000000in}{0.058333in}}%
\pgfpathcurveto{\pgfqpoint{0.984530in}{0.058333in}}{\pgfqpoint{0.969691in}{0.052187in}}{\pgfqpoint{0.958752in}{0.041248in}}%
\pgfpathcurveto{\pgfqpoint{0.947813in}{0.030309in}}{\pgfqpoint{0.941667in}{0.015470in}}{\pgfqpoint{0.941667in}{0.000000in}}%
\pgfpathcurveto{\pgfqpoint{0.941667in}{-0.015470in}}{\pgfqpoint{0.947813in}{-0.030309in}}{\pgfqpoint{0.958752in}{-0.041248in}}%
\pgfpathcurveto{\pgfqpoint{0.969691in}{-0.052187in}}{\pgfqpoint{0.984530in}{-0.058333in}}{\pgfqpoint{1.000000in}{-0.058333in}}%
\pgfpathclose%
\pgfpathmoveto{\pgfqpoint{1.000000in}{-0.052500in}}%
\pgfpathcurveto{\pgfqpoint{1.000000in}{-0.052500in}}{\pgfqpoint{0.986077in}{-0.052500in}}{\pgfqpoint{0.972722in}{-0.046968in}}%
\pgfpathcurveto{\pgfqpoint{0.962877in}{-0.037123in}}{\pgfqpoint{0.953032in}{-0.027278in}}{\pgfqpoint{0.947500in}{-0.013923in}}%
\pgfpathcurveto{\pgfqpoint{0.947500in}{0.000000in}}{\pgfqpoint{0.947500in}{0.013923in}}{\pgfqpoint{0.953032in}{0.027278in}}%
\pgfpathcurveto{\pgfqpoint{0.962877in}{0.037123in}}{\pgfqpoint{0.972722in}{0.046968in}}{\pgfqpoint{0.986077in}{0.052500in}}%
\pgfpathcurveto{\pgfqpoint{1.000000in}{0.052500in}}{\pgfqpoint{1.013923in}{0.052500in}}{\pgfqpoint{1.027278in}{0.046968in}}%
\pgfpathcurveto{\pgfqpoint{1.037123in}{0.037123in}}{\pgfqpoint{1.046968in}{0.027278in}}{\pgfqpoint{1.052500in}{0.013923in}}%
\pgfpathcurveto{\pgfqpoint{1.052500in}{0.000000in}}{\pgfqpoint{1.052500in}{-0.013923in}}{\pgfqpoint{1.046968in}{-0.027278in}}%
\pgfpathcurveto{\pgfqpoint{1.037123in}{-0.037123in}}{\pgfqpoint{1.027278in}{-0.046968in}}{\pgfqpoint{1.013923in}{-0.052500in}}%
\pgfpathclose%
\pgfpathmoveto{\pgfqpoint{0.083333in}{0.108333in}}%
\pgfpathcurveto{\pgfqpoint{0.098804in}{0.108333in}}{\pgfqpoint{0.113642in}{0.114480in}}{\pgfqpoint{0.124581in}{0.125419in}}%
\pgfpathcurveto{\pgfqpoint{0.135520in}{0.136358in}}{\pgfqpoint{0.141667in}{0.151196in}}{\pgfqpoint{0.141667in}{0.166667in}}%
\pgfpathcurveto{\pgfqpoint{0.141667in}{0.182137in}}{\pgfqpoint{0.135520in}{0.196975in}}{\pgfqpoint{0.124581in}{0.207915in}}%
\pgfpathcurveto{\pgfqpoint{0.113642in}{0.218854in}}{\pgfqpoint{0.098804in}{0.225000in}}{\pgfqpoint{0.083333in}{0.225000in}}%
\pgfpathcurveto{\pgfqpoint{0.067863in}{0.225000in}}{\pgfqpoint{0.053025in}{0.218854in}}{\pgfqpoint{0.042085in}{0.207915in}}%
\pgfpathcurveto{\pgfqpoint{0.031146in}{0.196975in}}{\pgfqpoint{0.025000in}{0.182137in}}{\pgfqpoint{0.025000in}{0.166667in}}%
\pgfpathcurveto{\pgfqpoint{0.025000in}{0.151196in}}{\pgfqpoint{0.031146in}{0.136358in}}{\pgfqpoint{0.042085in}{0.125419in}}%
\pgfpathcurveto{\pgfqpoint{0.053025in}{0.114480in}}{\pgfqpoint{0.067863in}{0.108333in}}{\pgfqpoint{0.083333in}{0.108333in}}%
\pgfpathclose%
\pgfpathmoveto{\pgfqpoint{0.083333in}{0.114167in}}%
\pgfpathcurveto{\pgfqpoint{0.083333in}{0.114167in}}{\pgfqpoint{0.069410in}{0.114167in}}{\pgfqpoint{0.056055in}{0.119698in}}%
\pgfpathcurveto{\pgfqpoint{0.046210in}{0.129544in}}{\pgfqpoint{0.036365in}{0.139389in}}{\pgfqpoint{0.030833in}{0.152744in}}%
\pgfpathcurveto{\pgfqpoint{0.030833in}{0.166667in}}{\pgfqpoint{0.030833in}{0.180590in}}{\pgfqpoint{0.036365in}{0.193945in}}%
\pgfpathcurveto{\pgfqpoint{0.046210in}{0.203790in}}{\pgfqpoint{0.056055in}{0.213635in}}{\pgfqpoint{0.069410in}{0.219167in}}%
\pgfpathcurveto{\pgfqpoint{0.083333in}{0.219167in}}{\pgfqpoint{0.097256in}{0.219167in}}{\pgfqpoint{0.110611in}{0.213635in}}%
\pgfpathcurveto{\pgfqpoint{0.120456in}{0.203790in}}{\pgfqpoint{0.130302in}{0.193945in}}{\pgfqpoint{0.135833in}{0.180590in}}%
\pgfpathcurveto{\pgfqpoint{0.135833in}{0.166667in}}{\pgfqpoint{0.135833in}{0.152744in}}{\pgfqpoint{0.130302in}{0.139389in}}%
\pgfpathcurveto{\pgfqpoint{0.120456in}{0.129544in}}{\pgfqpoint{0.110611in}{0.119698in}}{\pgfqpoint{0.097256in}{0.114167in}}%
\pgfpathclose%
\pgfpathmoveto{\pgfqpoint{0.250000in}{0.108333in}}%
\pgfpathcurveto{\pgfqpoint{0.265470in}{0.108333in}}{\pgfqpoint{0.280309in}{0.114480in}}{\pgfqpoint{0.291248in}{0.125419in}}%
\pgfpathcurveto{\pgfqpoint{0.302187in}{0.136358in}}{\pgfqpoint{0.308333in}{0.151196in}}{\pgfqpoint{0.308333in}{0.166667in}}%
\pgfpathcurveto{\pgfqpoint{0.308333in}{0.182137in}}{\pgfqpoint{0.302187in}{0.196975in}}{\pgfqpoint{0.291248in}{0.207915in}}%
\pgfpathcurveto{\pgfqpoint{0.280309in}{0.218854in}}{\pgfqpoint{0.265470in}{0.225000in}}{\pgfqpoint{0.250000in}{0.225000in}}%
\pgfpathcurveto{\pgfqpoint{0.234530in}{0.225000in}}{\pgfqpoint{0.219691in}{0.218854in}}{\pgfqpoint{0.208752in}{0.207915in}}%
\pgfpathcurveto{\pgfqpoint{0.197813in}{0.196975in}}{\pgfqpoint{0.191667in}{0.182137in}}{\pgfqpoint{0.191667in}{0.166667in}}%
\pgfpathcurveto{\pgfqpoint{0.191667in}{0.151196in}}{\pgfqpoint{0.197813in}{0.136358in}}{\pgfqpoint{0.208752in}{0.125419in}}%
\pgfpathcurveto{\pgfqpoint{0.219691in}{0.114480in}}{\pgfqpoint{0.234530in}{0.108333in}}{\pgfqpoint{0.250000in}{0.108333in}}%
\pgfpathclose%
\pgfpathmoveto{\pgfqpoint{0.250000in}{0.114167in}}%
\pgfpathcurveto{\pgfqpoint{0.250000in}{0.114167in}}{\pgfqpoint{0.236077in}{0.114167in}}{\pgfqpoint{0.222722in}{0.119698in}}%
\pgfpathcurveto{\pgfqpoint{0.212877in}{0.129544in}}{\pgfqpoint{0.203032in}{0.139389in}}{\pgfqpoint{0.197500in}{0.152744in}}%
\pgfpathcurveto{\pgfqpoint{0.197500in}{0.166667in}}{\pgfqpoint{0.197500in}{0.180590in}}{\pgfqpoint{0.203032in}{0.193945in}}%
\pgfpathcurveto{\pgfqpoint{0.212877in}{0.203790in}}{\pgfqpoint{0.222722in}{0.213635in}}{\pgfqpoint{0.236077in}{0.219167in}}%
\pgfpathcurveto{\pgfqpoint{0.250000in}{0.219167in}}{\pgfqpoint{0.263923in}{0.219167in}}{\pgfqpoint{0.277278in}{0.213635in}}%
\pgfpathcurveto{\pgfqpoint{0.287123in}{0.203790in}}{\pgfqpoint{0.296968in}{0.193945in}}{\pgfqpoint{0.302500in}{0.180590in}}%
\pgfpathcurveto{\pgfqpoint{0.302500in}{0.166667in}}{\pgfqpoint{0.302500in}{0.152744in}}{\pgfqpoint{0.296968in}{0.139389in}}%
\pgfpathcurveto{\pgfqpoint{0.287123in}{0.129544in}}{\pgfqpoint{0.277278in}{0.119698in}}{\pgfqpoint{0.263923in}{0.114167in}}%
\pgfpathclose%
\pgfpathmoveto{\pgfqpoint{0.416667in}{0.108333in}}%
\pgfpathcurveto{\pgfqpoint{0.432137in}{0.108333in}}{\pgfqpoint{0.446975in}{0.114480in}}{\pgfqpoint{0.457915in}{0.125419in}}%
\pgfpathcurveto{\pgfqpoint{0.468854in}{0.136358in}}{\pgfqpoint{0.475000in}{0.151196in}}{\pgfqpoint{0.475000in}{0.166667in}}%
\pgfpathcurveto{\pgfqpoint{0.475000in}{0.182137in}}{\pgfqpoint{0.468854in}{0.196975in}}{\pgfqpoint{0.457915in}{0.207915in}}%
\pgfpathcurveto{\pgfqpoint{0.446975in}{0.218854in}}{\pgfqpoint{0.432137in}{0.225000in}}{\pgfqpoint{0.416667in}{0.225000in}}%
\pgfpathcurveto{\pgfqpoint{0.401196in}{0.225000in}}{\pgfqpoint{0.386358in}{0.218854in}}{\pgfqpoint{0.375419in}{0.207915in}}%
\pgfpathcurveto{\pgfqpoint{0.364480in}{0.196975in}}{\pgfqpoint{0.358333in}{0.182137in}}{\pgfqpoint{0.358333in}{0.166667in}}%
\pgfpathcurveto{\pgfqpoint{0.358333in}{0.151196in}}{\pgfqpoint{0.364480in}{0.136358in}}{\pgfqpoint{0.375419in}{0.125419in}}%
\pgfpathcurveto{\pgfqpoint{0.386358in}{0.114480in}}{\pgfqpoint{0.401196in}{0.108333in}}{\pgfqpoint{0.416667in}{0.108333in}}%
\pgfpathclose%
\pgfpathmoveto{\pgfqpoint{0.416667in}{0.114167in}}%
\pgfpathcurveto{\pgfqpoint{0.416667in}{0.114167in}}{\pgfqpoint{0.402744in}{0.114167in}}{\pgfqpoint{0.389389in}{0.119698in}}%
\pgfpathcurveto{\pgfqpoint{0.379544in}{0.129544in}}{\pgfqpoint{0.369698in}{0.139389in}}{\pgfqpoint{0.364167in}{0.152744in}}%
\pgfpathcurveto{\pgfqpoint{0.364167in}{0.166667in}}{\pgfqpoint{0.364167in}{0.180590in}}{\pgfqpoint{0.369698in}{0.193945in}}%
\pgfpathcurveto{\pgfqpoint{0.379544in}{0.203790in}}{\pgfqpoint{0.389389in}{0.213635in}}{\pgfqpoint{0.402744in}{0.219167in}}%
\pgfpathcurveto{\pgfqpoint{0.416667in}{0.219167in}}{\pgfqpoint{0.430590in}{0.219167in}}{\pgfqpoint{0.443945in}{0.213635in}}%
\pgfpathcurveto{\pgfqpoint{0.453790in}{0.203790in}}{\pgfqpoint{0.463635in}{0.193945in}}{\pgfqpoint{0.469167in}{0.180590in}}%
\pgfpathcurveto{\pgfqpoint{0.469167in}{0.166667in}}{\pgfqpoint{0.469167in}{0.152744in}}{\pgfqpoint{0.463635in}{0.139389in}}%
\pgfpathcurveto{\pgfqpoint{0.453790in}{0.129544in}}{\pgfqpoint{0.443945in}{0.119698in}}{\pgfqpoint{0.430590in}{0.114167in}}%
\pgfpathclose%
\pgfpathmoveto{\pgfqpoint{0.583333in}{0.108333in}}%
\pgfpathcurveto{\pgfqpoint{0.598804in}{0.108333in}}{\pgfqpoint{0.613642in}{0.114480in}}{\pgfqpoint{0.624581in}{0.125419in}}%
\pgfpathcurveto{\pgfqpoint{0.635520in}{0.136358in}}{\pgfqpoint{0.641667in}{0.151196in}}{\pgfqpoint{0.641667in}{0.166667in}}%
\pgfpathcurveto{\pgfqpoint{0.641667in}{0.182137in}}{\pgfqpoint{0.635520in}{0.196975in}}{\pgfqpoint{0.624581in}{0.207915in}}%
\pgfpathcurveto{\pgfqpoint{0.613642in}{0.218854in}}{\pgfqpoint{0.598804in}{0.225000in}}{\pgfqpoint{0.583333in}{0.225000in}}%
\pgfpathcurveto{\pgfqpoint{0.567863in}{0.225000in}}{\pgfqpoint{0.553025in}{0.218854in}}{\pgfqpoint{0.542085in}{0.207915in}}%
\pgfpathcurveto{\pgfqpoint{0.531146in}{0.196975in}}{\pgfqpoint{0.525000in}{0.182137in}}{\pgfqpoint{0.525000in}{0.166667in}}%
\pgfpathcurveto{\pgfqpoint{0.525000in}{0.151196in}}{\pgfqpoint{0.531146in}{0.136358in}}{\pgfqpoint{0.542085in}{0.125419in}}%
\pgfpathcurveto{\pgfqpoint{0.553025in}{0.114480in}}{\pgfqpoint{0.567863in}{0.108333in}}{\pgfqpoint{0.583333in}{0.108333in}}%
\pgfpathclose%
\pgfpathmoveto{\pgfqpoint{0.583333in}{0.114167in}}%
\pgfpathcurveto{\pgfqpoint{0.583333in}{0.114167in}}{\pgfqpoint{0.569410in}{0.114167in}}{\pgfqpoint{0.556055in}{0.119698in}}%
\pgfpathcurveto{\pgfqpoint{0.546210in}{0.129544in}}{\pgfqpoint{0.536365in}{0.139389in}}{\pgfqpoint{0.530833in}{0.152744in}}%
\pgfpathcurveto{\pgfqpoint{0.530833in}{0.166667in}}{\pgfqpoint{0.530833in}{0.180590in}}{\pgfqpoint{0.536365in}{0.193945in}}%
\pgfpathcurveto{\pgfqpoint{0.546210in}{0.203790in}}{\pgfqpoint{0.556055in}{0.213635in}}{\pgfqpoint{0.569410in}{0.219167in}}%
\pgfpathcurveto{\pgfqpoint{0.583333in}{0.219167in}}{\pgfqpoint{0.597256in}{0.219167in}}{\pgfqpoint{0.610611in}{0.213635in}}%
\pgfpathcurveto{\pgfqpoint{0.620456in}{0.203790in}}{\pgfqpoint{0.630302in}{0.193945in}}{\pgfqpoint{0.635833in}{0.180590in}}%
\pgfpathcurveto{\pgfqpoint{0.635833in}{0.166667in}}{\pgfqpoint{0.635833in}{0.152744in}}{\pgfqpoint{0.630302in}{0.139389in}}%
\pgfpathcurveto{\pgfqpoint{0.620456in}{0.129544in}}{\pgfqpoint{0.610611in}{0.119698in}}{\pgfqpoint{0.597256in}{0.114167in}}%
\pgfpathclose%
\pgfpathmoveto{\pgfqpoint{0.750000in}{0.108333in}}%
\pgfpathcurveto{\pgfqpoint{0.765470in}{0.108333in}}{\pgfqpoint{0.780309in}{0.114480in}}{\pgfqpoint{0.791248in}{0.125419in}}%
\pgfpathcurveto{\pgfqpoint{0.802187in}{0.136358in}}{\pgfqpoint{0.808333in}{0.151196in}}{\pgfqpoint{0.808333in}{0.166667in}}%
\pgfpathcurveto{\pgfqpoint{0.808333in}{0.182137in}}{\pgfqpoint{0.802187in}{0.196975in}}{\pgfqpoint{0.791248in}{0.207915in}}%
\pgfpathcurveto{\pgfqpoint{0.780309in}{0.218854in}}{\pgfqpoint{0.765470in}{0.225000in}}{\pgfqpoint{0.750000in}{0.225000in}}%
\pgfpathcurveto{\pgfqpoint{0.734530in}{0.225000in}}{\pgfqpoint{0.719691in}{0.218854in}}{\pgfqpoint{0.708752in}{0.207915in}}%
\pgfpathcurveto{\pgfqpoint{0.697813in}{0.196975in}}{\pgfqpoint{0.691667in}{0.182137in}}{\pgfqpoint{0.691667in}{0.166667in}}%
\pgfpathcurveto{\pgfqpoint{0.691667in}{0.151196in}}{\pgfqpoint{0.697813in}{0.136358in}}{\pgfqpoint{0.708752in}{0.125419in}}%
\pgfpathcurveto{\pgfqpoint{0.719691in}{0.114480in}}{\pgfqpoint{0.734530in}{0.108333in}}{\pgfqpoint{0.750000in}{0.108333in}}%
\pgfpathclose%
\pgfpathmoveto{\pgfqpoint{0.750000in}{0.114167in}}%
\pgfpathcurveto{\pgfqpoint{0.750000in}{0.114167in}}{\pgfqpoint{0.736077in}{0.114167in}}{\pgfqpoint{0.722722in}{0.119698in}}%
\pgfpathcurveto{\pgfqpoint{0.712877in}{0.129544in}}{\pgfqpoint{0.703032in}{0.139389in}}{\pgfqpoint{0.697500in}{0.152744in}}%
\pgfpathcurveto{\pgfqpoint{0.697500in}{0.166667in}}{\pgfqpoint{0.697500in}{0.180590in}}{\pgfqpoint{0.703032in}{0.193945in}}%
\pgfpathcurveto{\pgfqpoint{0.712877in}{0.203790in}}{\pgfqpoint{0.722722in}{0.213635in}}{\pgfqpoint{0.736077in}{0.219167in}}%
\pgfpathcurveto{\pgfqpoint{0.750000in}{0.219167in}}{\pgfqpoint{0.763923in}{0.219167in}}{\pgfqpoint{0.777278in}{0.213635in}}%
\pgfpathcurveto{\pgfqpoint{0.787123in}{0.203790in}}{\pgfqpoint{0.796968in}{0.193945in}}{\pgfqpoint{0.802500in}{0.180590in}}%
\pgfpathcurveto{\pgfqpoint{0.802500in}{0.166667in}}{\pgfqpoint{0.802500in}{0.152744in}}{\pgfqpoint{0.796968in}{0.139389in}}%
\pgfpathcurveto{\pgfqpoint{0.787123in}{0.129544in}}{\pgfqpoint{0.777278in}{0.119698in}}{\pgfqpoint{0.763923in}{0.114167in}}%
\pgfpathclose%
\pgfpathmoveto{\pgfqpoint{0.916667in}{0.108333in}}%
\pgfpathcurveto{\pgfqpoint{0.932137in}{0.108333in}}{\pgfqpoint{0.946975in}{0.114480in}}{\pgfqpoint{0.957915in}{0.125419in}}%
\pgfpathcurveto{\pgfqpoint{0.968854in}{0.136358in}}{\pgfqpoint{0.975000in}{0.151196in}}{\pgfqpoint{0.975000in}{0.166667in}}%
\pgfpathcurveto{\pgfqpoint{0.975000in}{0.182137in}}{\pgfqpoint{0.968854in}{0.196975in}}{\pgfqpoint{0.957915in}{0.207915in}}%
\pgfpathcurveto{\pgfqpoint{0.946975in}{0.218854in}}{\pgfqpoint{0.932137in}{0.225000in}}{\pgfqpoint{0.916667in}{0.225000in}}%
\pgfpathcurveto{\pgfqpoint{0.901196in}{0.225000in}}{\pgfqpoint{0.886358in}{0.218854in}}{\pgfqpoint{0.875419in}{0.207915in}}%
\pgfpathcurveto{\pgfqpoint{0.864480in}{0.196975in}}{\pgfqpoint{0.858333in}{0.182137in}}{\pgfqpoint{0.858333in}{0.166667in}}%
\pgfpathcurveto{\pgfqpoint{0.858333in}{0.151196in}}{\pgfqpoint{0.864480in}{0.136358in}}{\pgfqpoint{0.875419in}{0.125419in}}%
\pgfpathcurveto{\pgfqpoint{0.886358in}{0.114480in}}{\pgfqpoint{0.901196in}{0.108333in}}{\pgfqpoint{0.916667in}{0.108333in}}%
\pgfpathclose%
\pgfpathmoveto{\pgfqpoint{0.916667in}{0.114167in}}%
\pgfpathcurveto{\pgfqpoint{0.916667in}{0.114167in}}{\pgfqpoint{0.902744in}{0.114167in}}{\pgfqpoint{0.889389in}{0.119698in}}%
\pgfpathcurveto{\pgfqpoint{0.879544in}{0.129544in}}{\pgfqpoint{0.869698in}{0.139389in}}{\pgfqpoint{0.864167in}{0.152744in}}%
\pgfpathcurveto{\pgfqpoint{0.864167in}{0.166667in}}{\pgfqpoint{0.864167in}{0.180590in}}{\pgfqpoint{0.869698in}{0.193945in}}%
\pgfpathcurveto{\pgfqpoint{0.879544in}{0.203790in}}{\pgfqpoint{0.889389in}{0.213635in}}{\pgfqpoint{0.902744in}{0.219167in}}%
\pgfpathcurveto{\pgfqpoint{0.916667in}{0.219167in}}{\pgfqpoint{0.930590in}{0.219167in}}{\pgfqpoint{0.943945in}{0.213635in}}%
\pgfpathcurveto{\pgfqpoint{0.953790in}{0.203790in}}{\pgfqpoint{0.963635in}{0.193945in}}{\pgfqpoint{0.969167in}{0.180590in}}%
\pgfpathcurveto{\pgfqpoint{0.969167in}{0.166667in}}{\pgfqpoint{0.969167in}{0.152744in}}{\pgfqpoint{0.963635in}{0.139389in}}%
\pgfpathcurveto{\pgfqpoint{0.953790in}{0.129544in}}{\pgfqpoint{0.943945in}{0.119698in}}{\pgfqpoint{0.930590in}{0.114167in}}%
\pgfpathclose%
\pgfpathmoveto{\pgfqpoint{0.000000in}{0.275000in}}%
\pgfpathcurveto{\pgfqpoint{0.015470in}{0.275000in}}{\pgfqpoint{0.030309in}{0.281146in}}{\pgfqpoint{0.041248in}{0.292085in}}%
\pgfpathcurveto{\pgfqpoint{0.052187in}{0.303025in}}{\pgfqpoint{0.058333in}{0.317863in}}{\pgfqpoint{0.058333in}{0.333333in}}%
\pgfpathcurveto{\pgfqpoint{0.058333in}{0.348804in}}{\pgfqpoint{0.052187in}{0.363642in}}{\pgfqpoint{0.041248in}{0.374581in}}%
\pgfpathcurveto{\pgfqpoint{0.030309in}{0.385520in}}{\pgfqpoint{0.015470in}{0.391667in}}{\pgfqpoint{0.000000in}{0.391667in}}%
\pgfpathcurveto{\pgfqpoint{-0.015470in}{0.391667in}}{\pgfqpoint{-0.030309in}{0.385520in}}{\pgfqpoint{-0.041248in}{0.374581in}}%
\pgfpathcurveto{\pgfqpoint{-0.052187in}{0.363642in}}{\pgfqpoint{-0.058333in}{0.348804in}}{\pgfqpoint{-0.058333in}{0.333333in}}%
\pgfpathcurveto{\pgfqpoint{-0.058333in}{0.317863in}}{\pgfqpoint{-0.052187in}{0.303025in}}{\pgfqpoint{-0.041248in}{0.292085in}}%
\pgfpathcurveto{\pgfqpoint{-0.030309in}{0.281146in}}{\pgfqpoint{-0.015470in}{0.275000in}}{\pgfqpoint{0.000000in}{0.275000in}}%
\pgfpathclose%
\pgfpathmoveto{\pgfqpoint{0.000000in}{0.280833in}}%
\pgfpathcurveto{\pgfqpoint{0.000000in}{0.280833in}}{\pgfqpoint{-0.013923in}{0.280833in}}{\pgfqpoint{-0.027278in}{0.286365in}}%
\pgfpathcurveto{\pgfqpoint{-0.037123in}{0.296210in}}{\pgfqpoint{-0.046968in}{0.306055in}}{\pgfqpoint{-0.052500in}{0.319410in}}%
\pgfpathcurveto{\pgfqpoint{-0.052500in}{0.333333in}}{\pgfqpoint{-0.052500in}{0.347256in}}{\pgfqpoint{-0.046968in}{0.360611in}}%
\pgfpathcurveto{\pgfqpoint{-0.037123in}{0.370456in}}{\pgfqpoint{-0.027278in}{0.380302in}}{\pgfqpoint{-0.013923in}{0.385833in}}%
\pgfpathcurveto{\pgfqpoint{0.000000in}{0.385833in}}{\pgfqpoint{0.013923in}{0.385833in}}{\pgfqpoint{0.027278in}{0.380302in}}%
\pgfpathcurveto{\pgfqpoint{0.037123in}{0.370456in}}{\pgfqpoint{0.046968in}{0.360611in}}{\pgfqpoint{0.052500in}{0.347256in}}%
\pgfpathcurveto{\pgfqpoint{0.052500in}{0.333333in}}{\pgfqpoint{0.052500in}{0.319410in}}{\pgfqpoint{0.046968in}{0.306055in}}%
\pgfpathcurveto{\pgfqpoint{0.037123in}{0.296210in}}{\pgfqpoint{0.027278in}{0.286365in}}{\pgfqpoint{0.013923in}{0.280833in}}%
\pgfpathclose%
\pgfpathmoveto{\pgfqpoint{0.166667in}{0.275000in}}%
\pgfpathcurveto{\pgfqpoint{0.182137in}{0.275000in}}{\pgfqpoint{0.196975in}{0.281146in}}{\pgfqpoint{0.207915in}{0.292085in}}%
\pgfpathcurveto{\pgfqpoint{0.218854in}{0.303025in}}{\pgfqpoint{0.225000in}{0.317863in}}{\pgfqpoint{0.225000in}{0.333333in}}%
\pgfpathcurveto{\pgfqpoint{0.225000in}{0.348804in}}{\pgfqpoint{0.218854in}{0.363642in}}{\pgfqpoint{0.207915in}{0.374581in}}%
\pgfpathcurveto{\pgfqpoint{0.196975in}{0.385520in}}{\pgfqpoint{0.182137in}{0.391667in}}{\pgfqpoint{0.166667in}{0.391667in}}%
\pgfpathcurveto{\pgfqpoint{0.151196in}{0.391667in}}{\pgfqpoint{0.136358in}{0.385520in}}{\pgfqpoint{0.125419in}{0.374581in}}%
\pgfpathcurveto{\pgfqpoint{0.114480in}{0.363642in}}{\pgfqpoint{0.108333in}{0.348804in}}{\pgfqpoint{0.108333in}{0.333333in}}%
\pgfpathcurveto{\pgfqpoint{0.108333in}{0.317863in}}{\pgfqpoint{0.114480in}{0.303025in}}{\pgfqpoint{0.125419in}{0.292085in}}%
\pgfpathcurveto{\pgfqpoint{0.136358in}{0.281146in}}{\pgfqpoint{0.151196in}{0.275000in}}{\pgfqpoint{0.166667in}{0.275000in}}%
\pgfpathclose%
\pgfpathmoveto{\pgfqpoint{0.166667in}{0.280833in}}%
\pgfpathcurveto{\pgfqpoint{0.166667in}{0.280833in}}{\pgfqpoint{0.152744in}{0.280833in}}{\pgfqpoint{0.139389in}{0.286365in}}%
\pgfpathcurveto{\pgfqpoint{0.129544in}{0.296210in}}{\pgfqpoint{0.119698in}{0.306055in}}{\pgfqpoint{0.114167in}{0.319410in}}%
\pgfpathcurveto{\pgfqpoint{0.114167in}{0.333333in}}{\pgfqpoint{0.114167in}{0.347256in}}{\pgfqpoint{0.119698in}{0.360611in}}%
\pgfpathcurveto{\pgfqpoint{0.129544in}{0.370456in}}{\pgfqpoint{0.139389in}{0.380302in}}{\pgfqpoint{0.152744in}{0.385833in}}%
\pgfpathcurveto{\pgfqpoint{0.166667in}{0.385833in}}{\pgfqpoint{0.180590in}{0.385833in}}{\pgfqpoint{0.193945in}{0.380302in}}%
\pgfpathcurveto{\pgfqpoint{0.203790in}{0.370456in}}{\pgfqpoint{0.213635in}{0.360611in}}{\pgfqpoint{0.219167in}{0.347256in}}%
\pgfpathcurveto{\pgfqpoint{0.219167in}{0.333333in}}{\pgfqpoint{0.219167in}{0.319410in}}{\pgfqpoint{0.213635in}{0.306055in}}%
\pgfpathcurveto{\pgfqpoint{0.203790in}{0.296210in}}{\pgfqpoint{0.193945in}{0.286365in}}{\pgfqpoint{0.180590in}{0.280833in}}%
\pgfpathclose%
\pgfpathmoveto{\pgfqpoint{0.333333in}{0.275000in}}%
\pgfpathcurveto{\pgfqpoint{0.348804in}{0.275000in}}{\pgfqpoint{0.363642in}{0.281146in}}{\pgfqpoint{0.374581in}{0.292085in}}%
\pgfpathcurveto{\pgfqpoint{0.385520in}{0.303025in}}{\pgfqpoint{0.391667in}{0.317863in}}{\pgfqpoint{0.391667in}{0.333333in}}%
\pgfpathcurveto{\pgfqpoint{0.391667in}{0.348804in}}{\pgfqpoint{0.385520in}{0.363642in}}{\pgfqpoint{0.374581in}{0.374581in}}%
\pgfpathcurveto{\pgfqpoint{0.363642in}{0.385520in}}{\pgfqpoint{0.348804in}{0.391667in}}{\pgfqpoint{0.333333in}{0.391667in}}%
\pgfpathcurveto{\pgfqpoint{0.317863in}{0.391667in}}{\pgfqpoint{0.303025in}{0.385520in}}{\pgfqpoint{0.292085in}{0.374581in}}%
\pgfpathcurveto{\pgfqpoint{0.281146in}{0.363642in}}{\pgfqpoint{0.275000in}{0.348804in}}{\pgfqpoint{0.275000in}{0.333333in}}%
\pgfpathcurveto{\pgfqpoint{0.275000in}{0.317863in}}{\pgfqpoint{0.281146in}{0.303025in}}{\pgfqpoint{0.292085in}{0.292085in}}%
\pgfpathcurveto{\pgfqpoint{0.303025in}{0.281146in}}{\pgfqpoint{0.317863in}{0.275000in}}{\pgfqpoint{0.333333in}{0.275000in}}%
\pgfpathclose%
\pgfpathmoveto{\pgfqpoint{0.333333in}{0.280833in}}%
\pgfpathcurveto{\pgfqpoint{0.333333in}{0.280833in}}{\pgfqpoint{0.319410in}{0.280833in}}{\pgfqpoint{0.306055in}{0.286365in}}%
\pgfpathcurveto{\pgfqpoint{0.296210in}{0.296210in}}{\pgfqpoint{0.286365in}{0.306055in}}{\pgfqpoint{0.280833in}{0.319410in}}%
\pgfpathcurveto{\pgfqpoint{0.280833in}{0.333333in}}{\pgfqpoint{0.280833in}{0.347256in}}{\pgfqpoint{0.286365in}{0.360611in}}%
\pgfpathcurveto{\pgfqpoint{0.296210in}{0.370456in}}{\pgfqpoint{0.306055in}{0.380302in}}{\pgfqpoint{0.319410in}{0.385833in}}%
\pgfpathcurveto{\pgfqpoint{0.333333in}{0.385833in}}{\pgfqpoint{0.347256in}{0.385833in}}{\pgfqpoint{0.360611in}{0.380302in}}%
\pgfpathcurveto{\pgfqpoint{0.370456in}{0.370456in}}{\pgfqpoint{0.380302in}{0.360611in}}{\pgfqpoint{0.385833in}{0.347256in}}%
\pgfpathcurveto{\pgfqpoint{0.385833in}{0.333333in}}{\pgfqpoint{0.385833in}{0.319410in}}{\pgfqpoint{0.380302in}{0.306055in}}%
\pgfpathcurveto{\pgfqpoint{0.370456in}{0.296210in}}{\pgfqpoint{0.360611in}{0.286365in}}{\pgfqpoint{0.347256in}{0.280833in}}%
\pgfpathclose%
\pgfpathmoveto{\pgfqpoint{0.500000in}{0.275000in}}%
\pgfpathcurveto{\pgfqpoint{0.515470in}{0.275000in}}{\pgfqpoint{0.530309in}{0.281146in}}{\pgfqpoint{0.541248in}{0.292085in}}%
\pgfpathcurveto{\pgfqpoint{0.552187in}{0.303025in}}{\pgfqpoint{0.558333in}{0.317863in}}{\pgfqpoint{0.558333in}{0.333333in}}%
\pgfpathcurveto{\pgfqpoint{0.558333in}{0.348804in}}{\pgfqpoint{0.552187in}{0.363642in}}{\pgfqpoint{0.541248in}{0.374581in}}%
\pgfpathcurveto{\pgfqpoint{0.530309in}{0.385520in}}{\pgfqpoint{0.515470in}{0.391667in}}{\pgfqpoint{0.500000in}{0.391667in}}%
\pgfpathcurveto{\pgfqpoint{0.484530in}{0.391667in}}{\pgfqpoint{0.469691in}{0.385520in}}{\pgfqpoint{0.458752in}{0.374581in}}%
\pgfpathcurveto{\pgfqpoint{0.447813in}{0.363642in}}{\pgfqpoint{0.441667in}{0.348804in}}{\pgfqpoint{0.441667in}{0.333333in}}%
\pgfpathcurveto{\pgfqpoint{0.441667in}{0.317863in}}{\pgfqpoint{0.447813in}{0.303025in}}{\pgfqpoint{0.458752in}{0.292085in}}%
\pgfpathcurveto{\pgfqpoint{0.469691in}{0.281146in}}{\pgfqpoint{0.484530in}{0.275000in}}{\pgfqpoint{0.500000in}{0.275000in}}%
\pgfpathclose%
\pgfpathmoveto{\pgfqpoint{0.500000in}{0.280833in}}%
\pgfpathcurveto{\pgfqpoint{0.500000in}{0.280833in}}{\pgfqpoint{0.486077in}{0.280833in}}{\pgfqpoint{0.472722in}{0.286365in}}%
\pgfpathcurveto{\pgfqpoint{0.462877in}{0.296210in}}{\pgfqpoint{0.453032in}{0.306055in}}{\pgfqpoint{0.447500in}{0.319410in}}%
\pgfpathcurveto{\pgfqpoint{0.447500in}{0.333333in}}{\pgfqpoint{0.447500in}{0.347256in}}{\pgfqpoint{0.453032in}{0.360611in}}%
\pgfpathcurveto{\pgfqpoint{0.462877in}{0.370456in}}{\pgfqpoint{0.472722in}{0.380302in}}{\pgfqpoint{0.486077in}{0.385833in}}%
\pgfpathcurveto{\pgfqpoint{0.500000in}{0.385833in}}{\pgfqpoint{0.513923in}{0.385833in}}{\pgfqpoint{0.527278in}{0.380302in}}%
\pgfpathcurveto{\pgfqpoint{0.537123in}{0.370456in}}{\pgfqpoint{0.546968in}{0.360611in}}{\pgfqpoint{0.552500in}{0.347256in}}%
\pgfpathcurveto{\pgfqpoint{0.552500in}{0.333333in}}{\pgfqpoint{0.552500in}{0.319410in}}{\pgfqpoint{0.546968in}{0.306055in}}%
\pgfpathcurveto{\pgfqpoint{0.537123in}{0.296210in}}{\pgfqpoint{0.527278in}{0.286365in}}{\pgfqpoint{0.513923in}{0.280833in}}%
\pgfpathclose%
\pgfpathmoveto{\pgfqpoint{0.666667in}{0.275000in}}%
\pgfpathcurveto{\pgfqpoint{0.682137in}{0.275000in}}{\pgfqpoint{0.696975in}{0.281146in}}{\pgfqpoint{0.707915in}{0.292085in}}%
\pgfpathcurveto{\pgfqpoint{0.718854in}{0.303025in}}{\pgfqpoint{0.725000in}{0.317863in}}{\pgfqpoint{0.725000in}{0.333333in}}%
\pgfpathcurveto{\pgfqpoint{0.725000in}{0.348804in}}{\pgfqpoint{0.718854in}{0.363642in}}{\pgfqpoint{0.707915in}{0.374581in}}%
\pgfpathcurveto{\pgfqpoint{0.696975in}{0.385520in}}{\pgfqpoint{0.682137in}{0.391667in}}{\pgfqpoint{0.666667in}{0.391667in}}%
\pgfpathcurveto{\pgfqpoint{0.651196in}{0.391667in}}{\pgfqpoint{0.636358in}{0.385520in}}{\pgfqpoint{0.625419in}{0.374581in}}%
\pgfpathcurveto{\pgfqpoint{0.614480in}{0.363642in}}{\pgfqpoint{0.608333in}{0.348804in}}{\pgfqpoint{0.608333in}{0.333333in}}%
\pgfpathcurveto{\pgfqpoint{0.608333in}{0.317863in}}{\pgfqpoint{0.614480in}{0.303025in}}{\pgfqpoint{0.625419in}{0.292085in}}%
\pgfpathcurveto{\pgfqpoint{0.636358in}{0.281146in}}{\pgfqpoint{0.651196in}{0.275000in}}{\pgfqpoint{0.666667in}{0.275000in}}%
\pgfpathclose%
\pgfpathmoveto{\pgfqpoint{0.666667in}{0.280833in}}%
\pgfpathcurveto{\pgfqpoint{0.666667in}{0.280833in}}{\pgfqpoint{0.652744in}{0.280833in}}{\pgfqpoint{0.639389in}{0.286365in}}%
\pgfpathcurveto{\pgfqpoint{0.629544in}{0.296210in}}{\pgfqpoint{0.619698in}{0.306055in}}{\pgfqpoint{0.614167in}{0.319410in}}%
\pgfpathcurveto{\pgfqpoint{0.614167in}{0.333333in}}{\pgfqpoint{0.614167in}{0.347256in}}{\pgfqpoint{0.619698in}{0.360611in}}%
\pgfpathcurveto{\pgfqpoint{0.629544in}{0.370456in}}{\pgfqpoint{0.639389in}{0.380302in}}{\pgfqpoint{0.652744in}{0.385833in}}%
\pgfpathcurveto{\pgfqpoint{0.666667in}{0.385833in}}{\pgfqpoint{0.680590in}{0.385833in}}{\pgfqpoint{0.693945in}{0.380302in}}%
\pgfpathcurveto{\pgfqpoint{0.703790in}{0.370456in}}{\pgfqpoint{0.713635in}{0.360611in}}{\pgfqpoint{0.719167in}{0.347256in}}%
\pgfpathcurveto{\pgfqpoint{0.719167in}{0.333333in}}{\pgfqpoint{0.719167in}{0.319410in}}{\pgfqpoint{0.713635in}{0.306055in}}%
\pgfpathcurveto{\pgfqpoint{0.703790in}{0.296210in}}{\pgfqpoint{0.693945in}{0.286365in}}{\pgfqpoint{0.680590in}{0.280833in}}%
\pgfpathclose%
\pgfpathmoveto{\pgfqpoint{0.833333in}{0.275000in}}%
\pgfpathcurveto{\pgfqpoint{0.848804in}{0.275000in}}{\pgfqpoint{0.863642in}{0.281146in}}{\pgfqpoint{0.874581in}{0.292085in}}%
\pgfpathcurveto{\pgfqpoint{0.885520in}{0.303025in}}{\pgfqpoint{0.891667in}{0.317863in}}{\pgfqpoint{0.891667in}{0.333333in}}%
\pgfpathcurveto{\pgfqpoint{0.891667in}{0.348804in}}{\pgfqpoint{0.885520in}{0.363642in}}{\pgfqpoint{0.874581in}{0.374581in}}%
\pgfpathcurveto{\pgfqpoint{0.863642in}{0.385520in}}{\pgfqpoint{0.848804in}{0.391667in}}{\pgfqpoint{0.833333in}{0.391667in}}%
\pgfpathcurveto{\pgfqpoint{0.817863in}{0.391667in}}{\pgfqpoint{0.803025in}{0.385520in}}{\pgfqpoint{0.792085in}{0.374581in}}%
\pgfpathcurveto{\pgfqpoint{0.781146in}{0.363642in}}{\pgfqpoint{0.775000in}{0.348804in}}{\pgfqpoint{0.775000in}{0.333333in}}%
\pgfpathcurveto{\pgfqpoint{0.775000in}{0.317863in}}{\pgfqpoint{0.781146in}{0.303025in}}{\pgfqpoint{0.792085in}{0.292085in}}%
\pgfpathcurveto{\pgfqpoint{0.803025in}{0.281146in}}{\pgfqpoint{0.817863in}{0.275000in}}{\pgfqpoint{0.833333in}{0.275000in}}%
\pgfpathclose%
\pgfpathmoveto{\pgfqpoint{0.833333in}{0.280833in}}%
\pgfpathcurveto{\pgfqpoint{0.833333in}{0.280833in}}{\pgfqpoint{0.819410in}{0.280833in}}{\pgfqpoint{0.806055in}{0.286365in}}%
\pgfpathcurveto{\pgfqpoint{0.796210in}{0.296210in}}{\pgfqpoint{0.786365in}{0.306055in}}{\pgfqpoint{0.780833in}{0.319410in}}%
\pgfpathcurveto{\pgfqpoint{0.780833in}{0.333333in}}{\pgfqpoint{0.780833in}{0.347256in}}{\pgfqpoint{0.786365in}{0.360611in}}%
\pgfpathcurveto{\pgfqpoint{0.796210in}{0.370456in}}{\pgfqpoint{0.806055in}{0.380302in}}{\pgfqpoint{0.819410in}{0.385833in}}%
\pgfpathcurveto{\pgfqpoint{0.833333in}{0.385833in}}{\pgfqpoint{0.847256in}{0.385833in}}{\pgfqpoint{0.860611in}{0.380302in}}%
\pgfpathcurveto{\pgfqpoint{0.870456in}{0.370456in}}{\pgfqpoint{0.880302in}{0.360611in}}{\pgfqpoint{0.885833in}{0.347256in}}%
\pgfpathcurveto{\pgfqpoint{0.885833in}{0.333333in}}{\pgfqpoint{0.885833in}{0.319410in}}{\pgfqpoint{0.880302in}{0.306055in}}%
\pgfpathcurveto{\pgfqpoint{0.870456in}{0.296210in}}{\pgfqpoint{0.860611in}{0.286365in}}{\pgfqpoint{0.847256in}{0.280833in}}%
\pgfpathclose%
\pgfpathmoveto{\pgfqpoint{1.000000in}{0.275000in}}%
\pgfpathcurveto{\pgfqpoint{1.015470in}{0.275000in}}{\pgfqpoint{1.030309in}{0.281146in}}{\pgfqpoint{1.041248in}{0.292085in}}%
\pgfpathcurveto{\pgfqpoint{1.052187in}{0.303025in}}{\pgfqpoint{1.058333in}{0.317863in}}{\pgfqpoint{1.058333in}{0.333333in}}%
\pgfpathcurveto{\pgfqpoint{1.058333in}{0.348804in}}{\pgfqpoint{1.052187in}{0.363642in}}{\pgfqpoint{1.041248in}{0.374581in}}%
\pgfpathcurveto{\pgfqpoint{1.030309in}{0.385520in}}{\pgfqpoint{1.015470in}{0.391667in}}{\pgfqpoint{1.000000in}{0.391667in}}%
\pgfpathcurveto{\pgfqpoint{0.984530in}{0.391667in}}{\pgfqpoint{0.969691in}{0.385520in}}{\pgfqpoint{0.958752in}{0.374581in}}%
\pgfpathcurveto{\pgfqpoint{0.947813in}{0.363642in}}{\pgfqpoint{0.941667in}{0.348804in}}{\pgfqpoint{0.941667in}{0.333333in}}%
\pgfpathcurveto{\pgfqpoint{0.941667in}{0.317863in}}{\pgfqpoint{0.947813in}{0.303025in}}{\pgfqpoint{0.958752in}{0.292085in}}%
\pgfpathcurveto{\pgfqpoint{0.969691in}{0.281146in}}{\pgfqpoint{0.984530in}{0.275000in}}{\pgfqpoint{1.000000in}{0.275000in}}%
\pgfpathclose%
\pgfpathmoveto{\pgfqpoint{1.000000in}{0.280833in}}%
\pgfpathcurveto{\pgfqpoint{1.000000in}{0.280833in}}{\pgfqpoint{0.986077in}{0.280833in}}{\pgfqpoint{0.972722in}{0.286365in}}%
\pgfpathcurveto{\pgfqpoint{0.962877in}{0.296210in}}{\pgfqpoint{0.953032in}{0.306055in}}{\pgfqpoint{0.947500in}{0.319410in}}%
\pgfpathcurveto{\pgfqpoint{0.947500in}{0.333333in}}{\pgfqpoint{0.947500in}{0.347256in}}{\pgfqpoint{0.953032in}{0.360611in}}%
\pgfpathcurveto{\pgfqpoint{0.962877in}{0.370456in}}{\pgfqpoint{0.972722in}{0.380302in}}{\pgfqpoint{0.986077in}{0.385833in}}%
\pgfpathcurveto{\pgfqpoint{1.000000in}{0.385833in}}{\pgfqpoint{1.013923in}{0.385833in}}{\pgfqpoint{1.027278in}{0.380302in}}%
\pgfpathcurveto{\pgfqpoint{1.037123in}{0.370456in}}{\pgfqpoint{1.046968in}{0.360611in}}{\pgfqpoint{1.052500in}{0.347256in}}%
\pgfpathcurveto{\pgfqpoint{1.052500in}{0.333333in}}{\pgfqpoint{1.052500in}{0.319410in}}{\pgfqpoint{1.046968in}{0.306055in}}%
\pgfpathcurveto{\pgfqpoint{1.037123in}{0.296210in}}{\pgfqpoint{1.027278in}{0.286365in}}{\pgfqpoint{1.013923in}{0.280833in}}%
\pgfpathclose%
\pgfpathmoveto{\pgfqpoint{0.083333in}{0.441667in}}%
\pgfpathcurveto{\pgfqpoint{0.098804in}{0.441667in}}{\pgfqpoint{0.113642in}{0.447813in}}{\pgfqpoint{0.124581in}{0.458752in}}%
\pgfpathcurveto{\pgfqpoint{0.135520in}{0.469691in}}{\pgfqpoint{0.141667in}{0.484530in}}{\pgfqpoint{0.141667in}{0.500000in}}%
\pgfpathcurveto{\pgfqpoint{0.141667in}{0.515470in}}{\pgfqpoint{0.135520in}{0.530309in}}{\pgfqpoint{0.124581in}{0.541248in}}%
\pgfpathcurveto{\pgfqpoint{0.113642in}{0.552187in}}{\pgfqpoint{0.098804in}{0.558333in}}{\pgfqpoint{0.083333in}{0.558333in}}%
\pgfpathcurveto{\pgfqpoint{0.067863in}{0.558333in}}{\pgfqpoint{0.053025in}{0.552187in}}{\pgfqpoint{0.042085in}{0.541248in}}%
\pgfpathcurveto{\pgfqpoint{0.031146in}{0.530309in}}{\pgfqpoint{0.025000in}{0.515470in}}{\pgfqpoint{0.025000in}{0.500000in}}%
\pgfpathcurveto{\pgfqpoint{0.025000in}{0.484530in}}{\pgfqpoint{0.031146in}{0.469691in}}{\pgfqpoint{0.042085in}{0.458752in}}%
\pgfpathcurveto{\pgfqpoint{0.053025in}{0.447813in}}{\pgfqpoint{0.067863in}{0.441667in}}{\pgfqpoint{0.083333in}{0.441667in}}%
\pgfpathclose%
\pgfpathmoveto{\pgfqpoint{0.083333in}{0.447500in}}%
\pgfpathcurveto{\pgfqpoint{0.083333in}{0.447500in}}{\pgfqpoint{0.069410in}{0.447500in}}{\pgfqpoint{0.056055in}{0.453032in}}%
\pgfpathcurveto{\pgfqpoint{0.046210in}{0.462877in}}{\pgfqpoint{0.036365in}{0.472722in}}{\pgfqpoint{0.030833in}{0.486077in}}%
\pgfpathcurveto{\pgfqpoint{0.030833in}{0.500000in}}{\pgfqpoint{0.030833in}{0.513923in}}{\pgfqpoint{0.036365in}{0.527278in}}%
\pgfpathcurveto{\pgfqpoint{0.046210in}{0.537123in}}{\pgfqpoint{0.056055in}{0.546968in}}{\pgfqpoint{0.069410in}{0.552500in}}%
\pgfpathcurveto{\pgfqpoint{0.083333in}{0.552500in}}{\pgfqpoint{0.097256in}{0.552500in}}{\pgfqpoint{0.110611in}{0.546968in}}%
\pgfpathcurveto{\pgfqpoint{0.120456in}{0.537123in}}{\pgfqpoint{0.130302in}{0.527278in}}{\pgfqpoint{0.135833in}{0.513923in}}%
\pgfpathcurveto{\pgfqpoint{0.135833in}{0.500000in}}{\pgfqpoint{0.135833in}{0.486077in}}{\pgfqpoint{0.130302in}{0.472722in}}%
\pgfpathcurveto{\pgfqpoint{0.120456in}{0.462877in}}{\pgfqpoint{0.110611in}{0.453032in}}{\pgfqpoint{0.097256in}{0.447500in}}%
\pgfpathclose%
\pgfpathmoveto{\pgfqpoint{0.250000in}{0.441667in}}%
\pgfpathcurveto{\pgfqpoint{0.265470in}{0.441667in}}{\pgfqpoint{0.280309in}{0.447813in}}{\pgfqpoint{0.291248in}{0.458752in}}%
\pgfpathcurveto{\pgfqpoint{0.302187in}{0.469691in}}{\pgfqpoint{0.308333in}{0.484530in}}{\pgfqpoint{0.308333in}{0.500000in}}%
\pgfpathcurveto{\pgfqpoint{0.308333in}{0.515470in}}{\pgfqpoint{0.302187in}{0.530309in}}{\pgfqpoint{0.291248in}{0.541248in}}%
\pgfpathcurveto{\pgfqpoint{0.280309in}{0.552187in}}{\pgfqpoint{0.265470in}{0.558333in}}{\pgfqpoint{0.250000in}{0.558333in}}%
\pgfpathcurveto{\pgfqpoint{0.234530in}{0.558333in}}{\pgfqpoint{0.219691in}{0.552187in}}{\pgfqpoint{0.208752in}{0.541248in}}%
\pgfpathcurveto{\pgfqpoint{0.197813in}{0.530309in}}{\pgfqpoint{0.191667in}{0.515470in}}{\pgfqpoint{0.191667in}{0.500000in}}%
\pgfpathcurveto{\pgfqpoint{0.191667in}{0.484530in}}{\pgfqpoint{0.197813in}{0.469691in}}{\pgfqpoint{0.208752in}{0.458752in}}%
\pgfpathcurveto{\pgfqpoint{0.219691in}{0.447813in}}{\pgfqpoint{0.234530in}{0.441667in}}{\pgfqpoint{0.250000in}{0.441667in}}%
\pgfpathclose%
\pgfpathmoveto{\pgfqpoint{0.250000in}{0.447500in}}%
\pgfpathcurveto{\pgfqpoint{0.250000in}{0.447500in}}{\pgfqpoint{0.236077in}{0.447500in}}{\pgfqpoint{0.222722in}{0.453032in}}%
\pgfpathcurveto{\pgfqpoint{0.212877in}{0.462877in}}{\pgfqpoint{0.203032in}{0.472722in}}{\pgfqpoint{0.197500in}{0.486077in}}%
\pgfpathcurveto{\pgfqpoint{0.197500in}{0.500000in}}{\pgfqpoint{0.197500in}{0.513923in}}{\pgfqpoint{0.203032in}{0.527278in}}%
\pgfpathcurveto{\pgfqpoint{0.212877in}{0.537123in}}{\pgfqpoint{0.222722in}{0.546968in}}{\pgfqpoint{0.236077in}{0.552500in}}%
\pgfpathcurveto{\pgfqpoint{0.250000in}{0.552500in}}{\pgfqpoint{0.263923in}{0.552500in}}{\pgfqpoint{0.277278in}{0.546968in}}%
\pgfpathcurveto{\pgfqpoint{0.287123in}{0.537123in}}{\pgfqpoint{0.296968in}{0.527278in}}{\pgfqpoint{0.302500in}{0.513923in}}%
\pgfpathcurveto{\pgfqpoint{0.302500in}{0.500000in}}{\pgfqpoint{0.302500in}{0.486077in}}{\pgfqpoint{0.296968in}{0.472722in}}%
\pgfpathcurveto{\pgfqpoint{0.287123in}{0.462877in}}{\pgfqpoint{0.277278in}{0.453032in}}{\pgfqpoint{0.263923in}{0.447500in}}%
\pgfpathclose%
\pgfpathmoveto{\pgfqpoint{0.416667in}{0.441667in}}%
\pgfpathcurveto{\pgfqpoint{0.432137in}{0.441667in}}{\pgfqpoint{0.446975in}{0.447813in}}{\pgfqpoint{0.457915in}{0.458752in}}%
\pgfpathcurveto{\pgfqpoint{0.468854in}{0.469691in}}{\pgfqpoint{0.475000in}{0.484530in}}{\pgfqpoint{0.475000in}{0.500000in}}%
\pgfpathcurveto{\pgfqpoint{0.475000in}{0.515470in}}{\pgfqpoint{0.468854in}{0.530309in}}{\pgfqpoint{0.457915in}{0.541248in}}%
\pgfpathcurveto{\pgfqpoint{0.446975in}{0.552187in}}{\pgfqpoint{0.432137in}{0.558333in}}{\pgfqpoint{0.416667in}{0.558333in}}%
\pgfpathcurveto{\pgfqpoint{0.401196in}{0.558333in}}{\pgfqpoint{0.386358in}{0.552187in}}{\pgfqpoint{0.375419in}{0.541248in}}%
\pgfpathcurveto{\pgfqpoint{0.364480in}{0.530309in}}{\pgfqpoint{0.358333in}{0.515470in}}{\pgfqpoint{0.358333in}{0.500000in}}%
\pgfpathcurveto{\pgfqpoint{0.358333in}{0.484530in}}{\pgfqpoint{0.364480in}{0.469691in}}{\pgfqpoint{0.375419in}{0.458752in}}%
\pgfpathcurveto{\pgfqpoint{0.386358in}{0.447813in}}{\pgfqpoint{0.401196in}{0.441667in}}{\pgfqpoint{0.416667in}{0.441667in}}%
\pgfpathclose%
\pgfpathmoveto{\pgfqpoint{0.416667in}{0.447500in}}%
\pgfpathcurveto{\pgfqpoint{0.416667in}{0.447500in}}{\pgfqpoint{0.402744in}{0.447500in}}{\pgfqpoint{0.389389in}{0.453032in}}%
\pgfpathcurveto{\pgfqpoint{0.379544in}{0.462877in}}{\pgfqpoint{0.369698in}{0.472722in}}{\pgfqpoint{0.364167in}{0.486077in}}%
\pgfpathcurveto{\pgfqpoint{0.364167in}{0.500000in}}{\pgfqpoint{0.364167in}{0.513923in}}{\pgfqpoint{0.369698in}{0.527278in}}%
\pgfpathcurveto{\pgfqpoint{0.379544in}{0.537123in}}{\pgfqpoint{0.389389in}{0.546968in}}{\pgfqpoint{0.402744in}{0.552500in}}%
\pgfpathcurveto{\pgfqpoint{0.416667in}{0.552500in}}{\pgfqpoint{0.430590in}{0.552500in}}{\pgfqpoint{0.443945in}{0.546968in}}%
\pgfpathcurveto{\pgfqpoint{0.453790in}{0.537123in}}{\pgfqpoint{0.463635in}{0.527278in}}{\pgfqpoint{0.469167in}{0.513923in}}%
\pgfpathcurveto{\pgfqpoint{0.469167in}{0.500000in}}{\pgfqpoint{0.469167in}{0.486077in}}{\pgfqpoint{0.463635in}{0.472722in}}%
\pgfpathcurveto{\pgfqpoint{0.453790in}{0.462877in}}{\pgfqpoint{0.443945in}{0.453032in}}{\pgfqpoint{0.430590in}{0.447500in}}%
\pgfpathclose%
\pgfpathmoveto{\pgfqpoint{0.583333in}{0.441667in}}%
\pgfpathcurveto{\pgfqpoint{0.598804in}{0.441667in}}{\pgfqpoint{0.613642in}{0.447813in}}{\pgfqpoint{0.624581in}{0.458752in}}%
\pgfpathcurveto{\pgfqpoint{0.635520in}{0.469691in}}{\pgfqpoint{0.641667in}{0.484530in}}{\pgfqpoint{0.641667in}{0.500000in}}%
\pgfpathcurveto{\pgfqpoint{0.641667in}{0.515470in}}{\pgfqpoint{0.635520in}{0.530309in}}{\pgfqpoint{0.624581in}{0.541248in}}%
\pgfpathcurveto{\pgfqpoint{0.613642in}{0.552187in}}{\pgfqpoint{0.598804in}{0.558333in}}{\pgfqpoint{0.583333in}{0.558333in}}%
\pgfpathcurveto{\pgfqpoint{0.567863in}{0.558333in}}{\pgfqpoint{0.553025in}{0.552187in}}{\pgfqpoint{0.542085in}{0.541248in}}%
\pgfpathcurveto{\pgfqpoint{0.531146in}{0.530309in}}{\pgfqpoint{0.525000in}{0.515470in}}{\pgfqpoint{0.525000in}{0.500000in}}%
\pgfpathcurveto{\pgfqpoint{0.525000in}{0.484530in}}{\pgfqpoint{0.531146in}{0.469691in}}{\pgfqpoint{0.542085in}{0.458752in}}%
\pgfpathcurveto{\pgfqpoint{0.553025in}{0.447813in}}{\pgfqpoint{0.567863in}{0.441667in}}{\pgfqpoint{0.583333in}{0.441667in}}%
\pgfpathclose%
\pgfpathmoveto{\pgfqpoint{0.583333in}{0.447500in}}%
\pgfpathcurveto{\pgfqpoint{0.583333in}{0.447500in}}{\pgfqpoint{0.569410in}{0.447500in}}{\pgfqpoint{0.556055in}{0.453032in}}%
\pgfpathcurveto{\pgfqpoint{0.546210in}{0.462877in}}{\pgfqpoint{0.536365in}{0.472722in}}{\pgfqpoint{0.530833in}{0.486077in}}%
\pgfpathcurveto{\pgfqpoint{0.530833in}{0.500000in}}{\pgfqpoint{0.530833in}{0.513923in}}{\pgfqpoint{0.536365in}{0.527278in}}%
\pgfpathcurveto{\pgfqpoint{0.546210in}{0.537123in}}{\pgfqpoint{0.556055in}{0.546968in}}{\pgfqpoint{0.569410in}{0.552500in}}%
\pgfpathcurveto{\pgfqpoint{0.583333in}{0.552500in}}{\pgfqpoint{0.597256in}{0.552500in}}{\pgfqpoint{0.610611in}{0.546968in}}%
\pgfpathcurveto{\pgfqpoint{0.620456in}{0.537123in}}{\pgfqpoint{0.630302in}{0.527278in}}{\pgfqpoint{0.635833in}{0.513923in}}%
\pgfpathcurveto{\pgfqpoint{0.635833in}{0.500000in}}{\pgfqpoint{0.635833in}{0.486077in}}{\pgfqpoint{0.630302in}{0.472722in}}%
\pgfpathcurveto{\pgfqpoint{0.620456in}{0.462877in}}{\pgfqpoint{0.610611in}{0.453032in}}{\pgfqpoint{0.597256in}{0.447500in}}%
\pgfpathclose%
\pgfpathmoveto{\pgfqpoint{0.750000in}{0.441667in}}%
\pgfpathcurveto{\pgfqpoint{0.765470in}{0.441667in}}{\pgfqpoint{0.780309in}{0.447813in}}{\pgfqpoint{0.791248in}{0.458752in}}%
\pgfpathcurveto{\pgfqpoint{0.802187in}{0.469691in}}{\pgfqpoint{0.808333in}{0.484530in}}{\pgfqpoint{0.808333in}{0.500000in}}%
\pgfpathcurveto{\pgfqpoint{0.808333in}{0.515470in}}{\pgfqpoint{0.802187in}{0.530309in}}{\pgfqpoint{0.791248in}{0.541248in}}%
\pgfpathcurveto{\pgfqpoint{0.780309in}{0.552187in}}{\pgfqpoint{0.765470in}{0.558333in}}{\pgfqpoint{0.750000in}{0.558333in}}%
\pgfpathcurveto{\pgfqpoint{0.734530in}{0.558333in}}{\pgfqpoint{0.719691in}{0.552187in}}{\pgfqpoint{0.708752in}{0.541248in}}%
\pgfpathcurveto{\pgfqpoint{0.697813in}{0.530309in}}{\pgfqpoint{0.691667in}{0.515470in}}{\pgfqpoint{0.691667in}{0.500000in}}%
\pgfpathcurveto{\pgfqpoint{0.691667in}{0.484530in}}{\pgfqpoint{0.697813in}{0.469691in}}{\pgfqpoint{0.708752in}{0.458752in}}%
\pgfpathcurveto{\pgfqpoint{0.719691in}{0.447813in}}{\pgfqpoint{0.734530in}{0.441667in}}{\pgfqpoint{0.750000in}{0.441667in}}%
\pgfpathclose%
\pgfpathmoveto{\pgfqpoint{0.750000in}{0.447500in}}%
\pgfpathcurveto{\pgfqpoint{0.750000in}{0.447500in}}{\pgfqpoint{0.736077in}{0.447500in}}{\pgfqpoint{0.722722in}{0.453032in}}%
\pgfpathcurveto{\pgfqpoint{0.712877in}{0.462877in}}{\pgfqpoint{0.703032in}{0.472722in}}{\pgfqpoint{0.697500in}{0.486077in}}%
\pgfpathcurveto{\pgfqpoint{0.697500in}{0.500000in}}{\pgfqpoint{0.697500in}{0.513923in}}{\pgfqpoint{0.703032in}{0.527278in}}%
\pgfpathcurveto{\pgfqpoint{0.712877in}{0.537123in}}{\pgfqpoint{0.722722in}{0.546968in}}{\pgfqpoint{0.736077in}{0.552500in}}%
\pgfpathcurveto{\pgfqpoint{0.750000in}{0.552500in}}{\pgfqpoint{0.763923in}{0.552500in}}{\pgfqpoint{0.777278in}{0.546968in}}%
\pgfpathcurveto{\pgfqpoint{0.787123in}{0.537123in}}{\pgfqpoint{0.796968in}{0.527278in}}{\pgfqpoint{0.802500in}{0.513923in}}%
\pgfpathcurveto{\pgfqpoint{0.802500in}{0.500000in}}{\pgfqpoint{0.802500in}{0.486077in}}{\pgfqpoint{0.796968in}{0.472722in}}%
\pgfpathcurveto{\pgfqpoint{0.787123in}{0.462877in}}{\pgfqpoint{0.777278in}{0.453032in}}{\pgfqpoint{0.763923in}{0.447500in}}%
\pgfpathclose%
\pgfpathmoveto{\pgfqpoint{0.916667in}{0.441667in}}%
\pgfpathcurveto{\pgfqpoint{0.932137in}{0.441667in}}{\pgfqpoint{0.946975in}{0.447813in}}{\pgfqpoint{0.957915in}{0.458752in}}%
\pgfpathcurveto{\pgfqpoint{0.968854in}{0.469691in}}{\pgfqpoint{0.975000in}{0.484530in}}{\pgfqpoint{0.975000in}{0.500000in}}%
\pgfpathcurveto{\pgfqpoint{0.975000in}{0.515470in}}{\pgfqpoint{0.968854in}{0.530309in}}{\pgfqpoint{0.957915in}{0.541248in}}%
\pgfpathcurveto{\pgfqpoint{0.946975in}{0.552187in}}{\pgfqpoint{0.932137in}{0.558333in}}{\pgfqpoint{0.916667in}{0.558333in}}%
\pgfpathcurveto{\pgfqpoint{0.901196in}{0.558333in}}{\pgfqpoint{0.886358in}{0.552187in}}{\pgfqpoint{0.875419in}{0.541248in}}%
\pgfpathcurveto{\pgfqpoint{0.864480in}{0.530309in}}{\pgfqpoint{0.858333in}{0.515470in}}{\pgfqpoint{0.858333in}{0.500000in}}%
\pgfpathcurveto{\pgfqpoint{0.858333in}{0.484530in}}{\pgfqpoint{0.864480in}{0.469691in}}{\pgfqpoint{0.875419in}{0.458752in}}%
\pgfpathcurveto{\pgfqpoint{0.886358in}{0.447813in}}{\pgfqpoint{0.901196in}{0.441667in}}{\pgfqpoint{0.916667in}{0.441667in}}%
\pgfpathclose%
\pgfpathmoveto{\pgfqpoint{0.916667in}{0.447500in}}%
\pgfpathcurveto{\pgfqpoint{0.916667in}{0.447500in}}{\pgfqpoint{0.902744in}{0.447500in}}{\pgfqpoint{0.889389in}{0.453032in}}%
\pgfpathcurveto{\pgfqpoint{0.879544in}{0.462877in}}{\pgfqpoint{0.869698in}{0.472722in}}{\pgfqpoint{0.864167in}{0.486077in}}%
\pgfpathcurveto{\pgfqpoint{0.864167in}{0.500000in}}{\pgfqpoint{0.864167in}{0.513923in}}{\pgfqpoint{0.869698in}{0.527278in}}%
\pgfpathcurveto{\pgfqpoint{0.879544in}{0.537123in}}{\pgfqpoint{0.889389in}{0.546968in}}{\pgfqpoint{0.902744in}{0.552500in}}%
\pgfpathcurveto{\pgfqpoint{0.916667in}{0.552500in}}{\pgfqpoint{0.930590in}{0.552500in}}{\pgfqpoint{0.943945in}{0.546968in}}%
\pgfpathcurveto{\pgfqpoint{0.953790in}{0.537123in}}{\pgfqpoint{0.963635in}{0.527278in}}{\pgfqpoint{0.969167in}{0.513923in}}%
\pgfpathcurveto{\pgfqpoint{0.969167in}{0.500000in}}{\pgfqpoint{0.969167in}{0.486077in}}{\pgfqpoint{0.963635in}{0.472722in}}%
\pgfpathcurveto{\pgfqpoint{0.953790in}{0.462877in}}{\pgfqpoint{0.943945in}{0.453032in}}{\pgfqpoint{0.930590in}{0.447500in}}%
\pgfpathclose%
\pgfpathmoveto{\pgfqpoint{0.000000in}{0.608333in}}%
\pgfpathcurveto{\pgfqpoint{0.015470in}{0.608333in}}{\pgfqpoint{0.030309in}{0.614480in}}{\pgfqpoint{0.041248in}{0.625419in}}%
\pgfpathcurveto{\pgfqpoint{0.052187in}{0.636358in}}{\pgfqpoint{0.058333in}{0.651196in}}{\pgfqpoint{0.058333in}{0.666667in}}%
\pgfpathcurveto{\pgfqpoint{0.058333in}{0.682137in}}{\pgfqpoint{0.052187in}{0.696975in}}{\pgfqpoint{0.041248in}{0.707915in}}%
\pgfpathcurveto{\pgfqpoint{0.030309in}{0.718854in}}{\pgfqpoint{0.015470in}{0.725000in}}{\pgfqpoint{0.000000in}{0.725000in}}%
\pgfpathcurveto{\pgfqpoint{-0.015470in}{0.725000in}}{\pgfqpoint{-0.030309in}{0.718854in}}{\pgfqpoint{-0.041248in}{0.707915in}}%
\pgfpathcurveto{\pgfqpoint{-0.052187in}{0.696975in}}{\pgfqpoint{-0.058333in}{0.682137in}}{\pgfqpoint{-0.058333in}{0.666667in}}%
\pgfpathcurveto{\pgfqpoint{-0.058333in}{0.651196in}}{\pgfqpoint{-0.052187in}{0.636358in}}{\pgfqpoint{-0.041248in}{0.625419in}}%
\pgfpathcurveto{\pgfqpoint{-0.030309in}{0.614480in}}{\pgfqpoint{-0.015470in}{0.608333in}}{\pgfqpoint{0.000000in}{0.608333in}}%
\pgfpathclose%
\pgfpathmoveto{\pgfqpoint{0.000000in}{0.614167in}}%
\pgfpathcurveto{\pgfqpoint{0.000000in}{0.614167in}}{\pgfqpoint{-0.013923in}{0.614167in}}{\pgfqpoint{-0.027278in}{0.619698in}}%
\pgfpathcurveto{\pgfqpoint{-0.037123in}{0.629544in}}{\pgfqpoint{-0.046968in}{0.639389in}}{\pgfqpoint{-0.052500in}{0.652744in}}%
\pgfpathcurveto{\pgfqpoint{-0.052500in}{0.666667in}}{\pgfqpoint{-0.052500in}{0.680590in}}{\pgfqpoint{-0.046968in}{0.693945in}}%
\pgfpathcurveto{\pgfqpoint{-0.037123in}{0.703790in}}{\pgfqpoint{-0.027278in}{0.713635in}}{\pgfqpoint{-0.013923in}{0.719167in}}%
\pgfpathcurveto{\pgfqpoint{0.000000in}{0.719167in}}{\pgfqpoint{0.013923in}{0.719167in}}{\pgfqpoint{0.027278in}{0.713635in}}%
\pgfpathcurveto{\pgfqpoint{0.037123in}{0.703790in}}{\pgfqpoint{0.046968in}{0.693945in}}{\pgfqpoint{0.052500in}{0.680590in}}%
\pgfpathcurveto{\pgfqpoint{0.052500in}{0.666667in}}{\pgfqpoint{0.052500in}{0.652744in}}{\pgfqpoint{0.046968in}{0.639389in}}%
\pgfpathcurveto{\pgfqpoint{0.037123in}{0.629544in}}{\pgfqpoint{0.027278in}{0.619698in}}{\pgfqpoint{0.013923in}{0.614167in}}%
\pgfpathclose%
\pgfpathmoveto{\pgfqpoint{0.166667in}{0.608333in}}%
\pgfpathcurveto{\pgfqpoint{0.182137in}{0.608333in}}{\pgfqpoint{0.196975in}{0.614480in}}{\pgfqpoint{0.207915in}{0.625419in}}%
\pgfpathcurveto{\pgfqpoint{0.218854in}{0.636358in}}{\pgfqpoint{0.225000in}{0.651196in}}{\pgfqpoint{0.225000in}{0.666667in}}%
\pgfpathcurveto{\pgfqpoint{0.225000in}{0.682137in}}{\pgfqpoint{0.218854in}{0.696975in}}{\pgfqpoint{0.207915in}{0.707915in}}%
\pgfpathcurveto{\pgfqpoint{0.196975in}{0.718854in}}{\pgfqpoint{0.182137in}{0.725000in}}{\pgfqpoint{0.166667in}{0.725000in}}%
\pgfpathcurveto{\pgfqpoint{0.151196in}{0.725000in}}{\pgfqpoint{0.136358in}{0.718854in}}{\pgfqpoint{0.125419in}{0.707915in}}%
\pgfpathcurveto{\pgfqpoint{0.114480in}{0.696975in}}{\pgfqpoint{0.108333in}{0.682137in}}{\pgfqpoint{0.108333in}{0.666667in}}%
\pgfpathcurveto{\pgfqpoint{0.108333in}{0.651196in}}{\pgfqpoint{0.114480in}{0.636358in}}{\pgfqpoint{0.125419in}{0.625419in}}%
\pgfpathcurveto{\pgfqpoint{0.136358in}{0.614480in}}{\pgfqpoint{0.151196in}{0.608333in}}{\pgfqpoint{0.166667in}{0.608333in}}%
\pgfpathclose%
\pgfpathmoveto{\pgfqpoint{0.166667in}{0.614167in}}%
\pgfpathcurveto{\pgfqpoint{0.166667in}{0.614167in}}{\pgfqpoint{0.152744in}{0.614167in}}{\pgfqpoint{0.139389in}{0.619698in}}%
\pgfpathcurveto{\pgfqpoint{0.129544in}{0.629544in}}{\pgfqpoint{0.119698in}{0.639389in}}{\pgfqpoint{0.114167in}{0.652744in}}%
\pgfpathcurveto{\pgfqpoint{0.114167in}{0.666667in}}{\pgfqpoint{0.114167in}{0.680590in}}{\pgfqpoint{0.119698in}{0.693945in}}%
\pgfpathcurveto{\pgfqpoint{0.129544in}{0.703790in}}{\pgfqpoint{0.139389in}{0.713635in}}{\pgfqpoint{0.152744in}{0.719167in}}%
\pgfpathcurveto{\pgfqpoint{0.166667in}{0.719167in}}{\pgfqpoint{0.180590in}{0.719167in}}{\pgfqpoint{0.193945in}{0.713635in}}%
\pgfpathcurveto{\pgfqpoint{0.203790in}{0.703790in}}{\pgfqpoint{0.213635in}{0.693945in}}{\pgfqpoint{0.219167in}{0.680590in}}%
\pgfpathcurveto{\pgfqpoint{0.219167in}{0.666667in}}{\pgfqpoint{0.219167in}{0.652744in}}{\pgfqpoint{0.213635in}{0.639389in}}%
\pgfpathcurveto{\pgfqpoint{0.203790in}{0.629544in}}{\pgfqpoint{0.193945in}{0.619698in}}{\pgfqpoint{0.180590in}{0.614167in}}%
\pgfpathclose%
\pgfpathmoveto{\pgfqpoint{0.333333in}{0.608333in}}%
\pgfpathcurveto{\pgfqpoint{0.348804in}{0.608333in}}{\pgfqpoint{0.363642in}{0.614480in}}{\pgfqpoint{0.374581in}{0.625419in}}%
\pgfpathcurveto{\pgfqpoint{0.385520in}{0.636358in}}{\pgfqpoint{0.391667in}{0.651196in}}{\pgfqpoint{0.391667in}{0.666667in}}%
\pgfpathcurveto{\pgfqpoint{0.391667in}{0.682137in}}{\pgfqpoint{0.385520in}{0.696975in}}{\pgfqpoint{0.374581in}{0.707915in}}%
\pgfpathcurveto{\pgfqpoint{0.363642in}{0.718854in}}{\pgfqpoint{0.348804in}{0.725000in}}{\pgfqpoint{0.333333in}{0.725000in}}%
\pgfpathcurveto{\pgfqpoint{0.317863in}{0.725000in}}{\pgfqpoint{0.303025in}{0.718854in}}{\pgfqpoint{0.292085in}{0.707915in}}%
\pgfpathcurveto{\pgfqpoint{0.281146in}{0.696975in}}{\pgfqpoint{0.275000in}{0.682137in}}{\pgfqpoint{0.275000in}{0.666667in}}%
\pgfpathcurveto{\pgfqpoint{0.275000in}{0.651196in}}{\pgfqpoint{0.281146in}{0.636358in}}{\pgfqpoint{0.292085in}{0.625419in}}%
\pgfpathcurveto{\pgfqpoint{0.303025in}{0.614480in}}{\pgfqpoint{0.317863in}{0.608333in}}{\pgfqpoint{0.333333in}{0.608333in}}%
\pgfpathclose%
\pgfpathmoveto{\pgfqpoint{0.333333in}{0.614167in}}%
\pgfpathcurveto{\pgfqpoint{0.333333in}{0.614167in}}{\pgfqpoint{0.319410in}{0.614167in}}{\pgfqpoint{0.306055in}{0.619698in}}%
\pgfpathcurveto{\pgfqpoint{0.296210in}{0.629544in}}{\pgfqpoint{0.286365in}{0.639389in}}{\pgfqpoint{0.280833in}{0.652744in}}%
\pgfpathcurveto{\pgfqpoint{0.280833in}{0.666667in}}{\pgfqpoint{0.280833in}{0.680590in}}{\pgfqpoint{0.286365in}{0.693945in}}%
\pgfpathcurveto{\pgfqpoint{0.296210in}{0.703790in}}{\pgfqpoint{0.306055in}{0.713635in}}{\pgfqpoint{0.319410in}{0.719167in}}%
\pgfpathcurveto{\pgfqpoint{0.333333in}{0.719167in}}{\pgfqpoint{0.347256in}{0.719167in}}{\pgfqpoint{0.360611in}{0.713635in}}%
\pgfpathcurveto{\pgfqpoint{0.370456in}{0.703790in}}{\pgfqpoint{0.380302in}{0.693945in}}{\pgfqpoint{0.385833in}{0.680590in}}%
\pgfpathcurveto{\pgfqpoint{0.385833in}{0.666667in}}{\pgfqpoint{0.385833in}{0.652744in}}{\pgfqpoint{0.380302in}{0.639389in}}%
\pgfpathcurveto{\pgfqpoint{0.370456in}{0.629544in}}{\pgfqpoint{0.360611in}{0.619698in}}{\pgfqpoint{0.347256in}{0.614167in}}%
\pgfpathclose%
\pgfpathmoveto{\pgfqpoint{0.500000in}{0.608333in}}%
\pgfpathcurveto{\pgfqpoint{0.515470in}{0.608333in}}{\pgfqpoint{0.530309in}{0.614480in}}{\pgfqpoint{0.541248in}{0.625419in}}%
\pgfpathcurveto{\pgfqpoint{0.552187in}{0.636358in}}{\pgfqpoint{0.558333in}{0.651196in}}{\pgfqpoint{0.558333in}{0.666667in}}%
\pgfpathcurveto{\pgfqpoint{0.558333in}{0.682137in}}{\pgfqpoint{0.552187in}{0.696975in}}{\pgfqpoint{0.541248in}{0.707915in}}%
\pgfpathcurveto{\pgfqpoint{0.530309in}{0.718854in}}{\pgfqpoint{0.515470in}{0.725000in}}{\pgfqpoint{0.500000in}{0.725000in}}%
\pgfpathcurveto{\pgfqpoint{0.484530in}{0.725000in}}{\pgfqpoint{0.469691in}{0.718854in}}{\pgfqpoint{0.458752in}{0.707915in}}%
\pgfpathcurveto{\pgfqpoint{0.447813in}{0.696975in}}{\pgfqpoint{0.441667in}{0.682137in}}{\pgfqpoint{0.441667in}{0.666667in}}%
\pgfpathcurveto{\pgfqpoint{0.441667in}{0.651196in}}{\pgfqpoint{0.447813in}{0.636358in}}{\pgfqpoint{0.458752in}{0.625419in}}%
\pgfpathcurveto{\pgfqpoint{0.469691in}{0.614480in}}{\pgfqpoint{0.484530in}{0.608333in}}{\pgfqpoint{0.500000in}{0.608333in}}%
\pgfpathclose%
\pgfpathmoveto{\pgfqpoint{0.500000in}{0.614167in}}%
\pgfpathcurveto{\pgfqpoint{0.500000in}{0.614167in}}{\pgfqpoint{0.486077in}{0.614167in}}{\pgfqpoint{0.472722in}{0.619698in}}%
\pgfpathcurveto{\pgfqpoint{0.462877in}{0.629544in}}{\pgfqpoint{0.453032in}{0.639389in}}{\pgfqpoint{0.447500in}{0.652744in}}%
\pgfpathcurveto{\pgfqpoint{0.447500in}{0.666667in}}{\pgfqpoint{0.447500in}{0.680590in}}{\pgfqpoint{0.453032in}{0.693945in}}%
\pgfpathcurveto{\pgfqpoint{0.462877in}{0.703790in}}{\pgfqpoint{0.472722in}{0.713635in}}{\pgfqpoint{0.486077in}{0.719167in}}%
\pgfpathcurveto{\pgfqpoint{0.500000in}{0.719167in}}{\pgfqpoint{0.513923in}{0.719167in}}{\pgfqpoint{0.527278in}{0.713635in}}%
\pgfpathcurveto{\pgfqpoint{0.537123in}{0.703790in}}{\pgfqpoint{0.546968in}{0.693945in}}{\pgfqpoint{0.552500in}{0.680590in}}%
\pgfpathcurveto{\pgfqpoint{0.552500in}{0.666667in}}{\pgfqpoint{0.552500in}{0.652744in}}{\pgfqpoint{0.546968in}{0.639389in}}%
\pgfpathcurveto{\pgfqpoint{0.537123in}{0.629544in}}{\pgfqpoint{0.527278in}{0.619698in}}{\pgfqpoint{0.513923in}{0.614167in}}%
\pgfpathclose%
\pgfpathmoveto{\pgfqpoint{0.666667in}{0.608333in}}%
\pgfpathcurveto{\pgfqpoint{0.682137in}{0.608333in}}{\pgfqpoint{0.696975in}{0.614480in}}{\pgfqpoint{0.707915in}{0.625419in}}%
\pgfpathcurveto{\pgfqpoint{0.718854in}{0.636358in}}{\pgfqpoint{0.725000in}{0.651196in}}{\pgfqpoint{0.725000in}{0.666667in}}%
\pgfpathcurveto{\pgfqpoint{0.725000in}{0.682137in}}{\pgfqpoint{0.718854in}{0.696975in}}{\pgfqpoint{0.707915in}{0.707915in}}%
\pgfpathcurveto{\pgfqpoint{0.696975in}{0.718854in}}{\pgfqpoint{0.682137in}{0.725000in}}{\pgfqpoint{0.666667in}{0.725000in}}%
\pgfpathcurveto{\pgfqpoint{0.651196in}{0.725000in}}{\pgfqpoint{0.636358in}{0.718854in}}{\pgfqpoint{0.625419in}{0.707915in}}%
\pgfpathcurveto{\pgfqpoint{0.614480in}{0.696975in}}{\pgfqpoint{0.608333in}{0.682137in}}{\pgfqpoint{0.608333in}{0.666667in}}%
\pgfpathcurveto{\pgfqpoint{0.608333in}{0.651196in}}{\pgfqpoint{0.614480in}{0.636358in}}{\pgfqpoint{0.625419in}{0.625419in}}%
\pgfpathcurveto{\pgfqpoint{0.636358in}{0.614480in}}{\pgfqpoint{0.651196in}{0.608333in}}{\pgfqpoint{0.666667in}{0.608333in}}%
\pgfpathclose%
\pgfpathmoveto{\pgfqpoint{0.666667in}{0.614167in}}%
\pgfpathcurveto{\pgfqpoint{0.666667in}{0.614167in}}{\pgfqpoint{0.652744in}{0.614167in}}{\pgfqpoint{0.639389in}{0.619698in}}%
\pgfpathcurveto{\pgfqpoint{0.629544in}{0.629544in}}{\pgfqpoint{0.619698in}{0.639389in}}{\pgfqpoint{0.614167in}{0.652744in}}%
\pgfpathcurveto{\pgfqpoint{0.614167in}{0.666667in}}{\pgfqpoint{0.614167in}{0.680590in}}{\pgfqpoint{0.619698in}{0.693945in}}%
\pgfpathcurveto{\pgfqpoint{0.629544in}{0.703790in}}{\pgfqpoint{0.639389in}{0.713635in}}{\pgfqpoint{0.652744in}{0.719167in}}%
\pgfpathcurveto{\pgfqpoint{0.666667in}{0.719167in}}{\pgfqpoint{0.680590in}{0.719167in}}{\pgfqpoint{0.693945in}{0.713635in}}%
\pgfpathcurveto{\pgfqpoint{0.703790in}{0.703790in}}{\pgfqpoint{0.713635in}{0.693945in}}{\pgfqpoint{0.719167in}{0.680590in}}%
\pgfpathcurveto{\pgfqpoint{0.719167in}{0.666667in}}{\pgfqpoint{0.719167in}{0.652744in}}{\pgfqpoint{0.713635in}{0.639389in}}%
\pgfpathcurveto{\pgfqpoint{0.703790in}{0.629544in}}{\pgfqpoint{0.693945in}{0.619698in}}{\pgfqpoint{0.680590in}{0.614167in}}%
\pgfpathclose%
\pgfpathmoveto{\pgfqpoint{0.833333in}{0.608333in}}%
\pgfpathcurveto{\pgfqpoint{0.848804in}{0.608333in}}{\pgfqpoint{0.863642in}{0.614480in}}{\pgfqpoint{0.874581in}{0.625419in}}%
\pgfpathcurveto{\pgfqpoint{0.885520in}{0.636358in}}{\pgfqpoint{0.891667in}{0.651196in}}{\pgfqpoint{0.891667in}{0.666667in}}%
\pgfpathcurveto{\pgfqpoint{0.891667in}{0.682137in}}{\pgfqpoint{0.885520in}{0.696975in}}{\pgfqpoint{0.874581in}{0.707915in}}%
\pgfpathcurveto{\pgfqpoint{0.863642in}{0.718854in}}{\pgfqpoint{0.848804in}{0.725000in}}{\pgfqpoint{0.833333in}{0.725000in}}%
\pgfpathcurveto{\pgfqpoint{0.817863in}{0.725000in}}{\pgfqpoint{0.803025in}{0.718854in}}{\pgfqpoint{0.792085in}{0.707915in}}%
\pgfpathcurveto{\pgfqpoint{0.781146in}{0.696975in}}{\pgfqpoint{0.775000in}{0.682137in}}{\pgfqpoint{0.775000in}{0.666667in}}%
\pgfpathcurveto{\pgfqpoint{0.775000in}{0.651196in}}{\pgfqpoint{0.781146in}{0.636358in}}{\pgfqpoint{0.792085in}{0.625419in}}%
\pgfpathcurveto{\pgfqpoint{0.803025in}{0.614480in}}{\pgfqpoint{0.817863in}{0.608333in}}{\pgfqpoint{0.833333in}{0.608333in}}%
\pgfpathclose%
\pgfpathmoveto{\pgfqpoint{0.833333in}{0.614167in}}%
\pgfpathcurveto{\pgfqpoint{0.833333in}{0.614167in}}{\pgfqpoint{0.819410in}{0.614167in}}{\pgfqpoint{0.806055in}{0.619698in}}%
\pgfpathcurveto{\pgfqpoint{0.796210in}{0.629544in}}{\pgfqpoint{0.786365in}{0.639389in}}{\pgfqpoint{0.780833in}{0.652744in}}%
\pgfpathcurveto{\pgfqpoint{0.780833in}{0.666667in}}{\pgfqpoint{0.780833in}{0.680590in}}{\pgfqpoint{0.786365in}{0.693945in}}%
\pgfpathcurveto{\pgfqpoint{0.796210in}{0.703790in}}{\pgfqpoint{0.806055in}{0.713635in}}{\pgfqpoint{0.819410in}{0.719167in}}%
\pgfpathcurveto{\pgfqpoint{0.833333in}{0.719167in}}{\pgfqpoint{0.847256in}{0.719167in}}{\pgfqpoint{0.860611in}{0.713635in}}%
\pgfpathcurveto{\pgfqpoint{0.870456in}{0.703790in}}{\pgfqpoint{0.880302in}{0.693945in}}{\pgfqpoint{0.885833in}{0.680590in}}%
\pgfpathcurveto{\pgfqpoint{0.885833in}{0.666667in}}{\pgfqpoint{0.885833in}{0.652744in}}{\pgfqpoint{0.880302in}{0.639389in}}%
\pgfpathcurveto{\pgfqpoint{0.870456in}{0.629544in}}{\pgfqpoint{0.860611in}{0.619698in}}{\pgfqpoint{0.847256in}{0.614167in}}%
\pgfpathclose%
\pgfpathmoveto{\pgfqpoint{1.000000in}{0.608333in}}%
\pgfpathcurveto{\pgfqpoint{1.015470in}{0.608333in}}{\pgfqpoint{1.030309in}{0.614480in}}{\pgfqpoint{1.041248in}{0.625419in}}%
\pgfpathcurveto{\pgfqpoint{1.052187in}{0.636358in}}{\pgfqpoint{1.058333in}{0.651196in}}{\pgfqpoint{1.058333in}{0.666667in}}%
\pgfpathcurveto{\pgfqpoint{1.058333in}{0.682137in}}{\pgfqpoint{1.052187in}{0.696975in}}{\pgfqpoint{1.041248in}{0.707915in}}%
\pgfpathcurveto{\pgfqpoint{1.030309in}{0.718854in}}{\pgfqpoint{1.015470in}{0.725000in}}{\pgfqpoint{1.000000in}{0.725000in}}%
\pgfpathcurveto{\pgfqpoint{0.984530in}{0.725000in}}{\pgfqpoint{0.969691in}{0.718854in}}{\pgfqpoint{0.958752in}{0.707915in}}%
\pgfpathcurveto{\pgfqpoint{0.947813in}{0.696975in}}{\pgfqpoint{0.941667in}{0.682137in}}{\pgfqpoint{0.941667in}{0.666667in}}%
\pgfpathcurveto{\pgfqpoint{0.941667in}{0.651196in}}{\pgfqpoint{0.947813in}{0.636358in}}{\pgfqpoint{0.958752in}{0.625419in}}%
\pgfpathcurveto{\pgfqpoint{0.969691in}{0.614480in}}{\pgfqpoint{0.984530in}{0.608333in}}{\pgfqpoint{1.000000in}{0.608333in}}%
\pgfpathclose%
\pgfpathmoveto{\pgfqpoint{1.000000in}{0.614167in}}%
\pgfpathcurveto{\pgfqpoint{1.000000in}{0.614167in}}{\pgfqpoint{0.986077in}{0.614167in}}{\pgfqpoint{0.972722in}{0.619698in}}%
\pgfpathcurveto{\pgfqpoint{0.962877in}{0.629544in}}{\pgfqpoint{0.953032in}{0.639389in}}{\pgfqpoint{0.947500in}{0.652744in}}%
\pgfpathcurveto{\pgfqpoint{0.947500in}{0.666667in}}{\pgfqpoint{0.947500in}{0.680590in}}{\pgfqpoint{0.953032in}{0.693945in}}%
\pgfpathcurveto{\pgfqpoint{0.962877in}{0.703790in}}{\pgfqpoint{0.972722in}{0.713635in}}{\pgfqpoint{0.986077in}{0.719167in}}%
\pgfpathcurveto{\pgfqpoint{1.000000in}{0.719167in}}{\pgfqpoint{1.013923in}{0.719167in}}{\pgfqpoint{1.027278in}{0.713635in}}%
\pgfpathcurveto{\pgfqpoint{1.037123in}{0.703790in}}{\pgfqpoint{1.046968in}{0.693945in}}{\pgfqpoint{1.052500in}{0.680590in}}%
\pgfpathcurveto{\pgfqpoint{1.052500in}{0.666667in}}{\pgfqpoint{1.052500in}{0.652744in}}{\pgfqpoint{1.046968in}{0.639389in}}%
\pgfpathcurveto{\pgfqpoint{1.037123in}{0.629544in}}{\pgfqpoint{1.027278in}{0.619698in}}{\pgfqpoint{1.013923in}{0.614167in}}%
\pgfpathclose%
\pgfpathmoveto{\pgfqpoint{0.083333in}{0.775000in}}%
\pgfpathcurveto{\pgfqpoint{0.098804in}{0.775000in}}{\pgfqpoint{0.113642in}{0.781146in}}{\pgfqpoint{0.124581in}{0.792085in}}%
\pgfpathcurveto{\pgfqpoint{0.135520in}{0.803025in}}{\pgfqpoint{0.141667in}{0.817863in}}{\pgfqpoint{0.141667in}{0.833333in}}%
\pgfpathcurveto{\pgfqpoint{0.141667in}{0.848804in}}{\pgfqpoint{0.135520in}{0.863642in}}{\pgfqpoint{0.124581in}{0.874581in}}%
\pgfpathcurveto{\pgfqpoint{0.113642in}{0.885520in}}{\pgfqpoint{0.098804in}{0.891667in}}{\pgfqpoint{0.083333in}{0.891667in}}%
\pgfpathcurveto{\pgfqpoint{0.067863in}{0.891667in}}{\pgfqpoint{0.053025in}{0.885520in}}{\pgfqpoint{0.042085in}{0.874581in}}%
\pgfpathcurveto{\pgfqpoint{0.031146in}{0.863642in}}{\pgfqpoint{0.025000in}{0.848804in}}{\pgfqpoint{0.025000in}{0.833333in}}%
\pgfpathcurveto{\pgfqpoint{0.025000in}{0.817863in}}{\pgfqpoint{0.031146in}{0.803025in}}{\pgfqpoint{0.042085in}{0.792085in}}%
\pgfpathcurveto{\pgfqpoint{0.053025in}{0.781146in}}{\pgfqpoint{0.067863in}{0.775000in}}{\pgfqpoint{0.083333in}{0.775000in}}%
\pgfpathclose%
\pgfpathmoveto{\pgfqpoint{0.083333in}{0.780833in}}%
\pgfpathcurveto{\pgfqpoint{0.083333in}{0.780833in}}{\pgfqpoint{0.069410in}{0.780833in}}{\pgfqpoint{0.056055in}{0.786365in}}%
\pgfpathcurveto{\pgfqpoint{0.046210in}{0.796210in}}{\pgfqpoint{0.036365in}{0.806055in}}{\pgfqpoint{0.030833in}{0.819410in}}%
\pgfpathcurveto{\pgfqpoint{0.030833in}{0.833333in}}{\pgfqpoint{0.030833in}{0.847256in}}{\pgfqpoint{0.036365in}{0.860611in}}%
\pgfpathcurveto{\pgfqpoint{0.046210in}{0.870456in}}{\pgfqpoint{0.056055in}{0.880302in}}{\pgfqpoint{0.069410in}{0.885833in}}%
\pgfpathcurveto{\pgfqpoint{0.083333in}{0.885833in}}{\pgfqpoint{0.097256in}{0.885833in}}{\pgfqpoint{0.110611in}{0.880302in}}%
\pgfpathcurveto{\pgfqpoint{0.120456in}{0.870456in}}{\pgfqpoint{0.130302in}{0.860611in}}{\pgfqpoint{0.135833in}{0.847256in}}%
\pgfpathcurveto{\pgfqpoint{0.135833in}{0.833333in}}{\pgfqpoint{0.135833in}{0.819410in}}{\pgfqpoint{0.130302in}{0.806055in}}%
\pgfpathcurveto{\pgfqpoint{0.120456in}{0.796210in}}{\pgfqpoint{0.110611in}{0.786365in}}{\pgfqpoint{0.097256in}{0.780833in}}%
\pgfpathclose%
\pgfpathmoveto{\pgfqpoint{0.250000in}{0.775000in}}%
\pgfpathcurveto{\pgfqpoint{0.265470in}{0.775000in}}{\pgfqpoint{0.280309in}{0.781146in}}{\pgfqpoint{0.291248in}{0.792085in}}%
\pgfpathcurveto{\pgfqpoint{0.302187in}{0.803025in}}{\pgfqpoint{0.308333in}{0.817863in}}{\pgfqpoint{0.308333in}{0.833333in}}%
\pgfpathcurveto{\pgfqpoint{0.308333in}{0.848804in}}{\pgfqpoint{0.302187in}{0.863642in}}{\pgfqpoint{0.291248in}{0.874581in}}%
\pgfpathcurveto{\pgfqpoint{0.280309in}{0.885520in}}{\pgfqpoint{0.265470in}{0.891667in}}{\pgfqpoint{0.250000in}{0.891667in}}%
\pgfpathcurveto{\pgfqpoint{0.234530in}{0.891667in}}{\pgfqpoint{0.219691in}{0.885520in}}{\pgfqpoint{0.208752in}{0.874581in}}%
\pgfpathcurveto{\pgfqpoint{0.197813in}{0.863642in}}{\pgfqpoint{0.191667in}{0.848804in}}{\pgfqpoint{0.191667in}{0.833333in}}%
\pgfpathcurveto{\pgfqpoint{0.191667in}{0.817863in}}{\pgfqpoint{0.197813in}{0.803025in}}{\pgfqpoint{0.208752in}{0.792085in}}%
\pgfpathcurveto{\pgfqpoint{0.219691in}{0.781146in}}{\pgfqpoint{0.234530in}{0.775000in}}{\pgfqpoint{0.250000in}{0.775000in}}%
\pgfpathclose%
\pgfpathmoveto{\pgfqpoint{0.250000in}{0.780833in}}%
\pgfpathcurveto{\pgfqpoint{0.250000in}{0.780833in}}{\pgfqpoint{0.236077in}{0.780833in}}{\pgfqpoint{0.222722in}{0.786365in}}%
\pgfpathcurveto{\pgfqpoint{0.212877in}{0.796210in}}{\pgfqpoint{0.203032in}{0.806055in}}{\pgfqpoint{0.197500in}{0.819410in}}%
\pgfpathcurveto{\pgfqpoint{0.197500in}{0.833333in}}{\pgfqpoint{0.197500in}{0.847256in}}{\pgfqpoint{0.203032in}{0.860611in}}%
\pgfpathcurveto{\pgfqpoint{0.212877in}{0.870456in}}{\pgfqpoint{0.222722in}{0.880302in}}{\pgfqpoint{0.236077in}{0.885833in}}%
\pgfpathcurveto{\pgfqpoint{0.250000in}{0.885833in}}{\pgfqpoint{0.263923in}{0.885833in}}{\pgfqpoint{0.277278in}{0.880302in}}%
\pgfpathcurveto{\pgfqpoint{0.287123in}{0.870456in}}{\pgfqpoint{0.296968in}{0.860611in}}{\pgfqpoint{0.302500in}{0.847256in}}%
\pgfpathcurveto{\pgfqpoint{0.302500in}{0.833333in}}{\pgfqpoint{0.302500in}{0.819410in}}{\pgfqpoint{0.296968in}{0.806055in}}%
\pgfpathcurveto{\pgfqpoint{0.287123in}{0.796210in}}{\pgfqpoint{0.277278in}{0.786365in}}{\pgfqpoint{0.263923in}{0.780833in}}%
\pgfpathclose%
\pgfpathmoveto{\pgfqpoint{0.416667in}{0.775000in}}%
\pgfpathcurveto{\pgfqpoint{0.432137in}{0.775000in}}{\pgfqpoint{0.446975in}{0.781146in}}{\pgfqpoint{0.457915in}{0.792085in}}%
\pgfpathcurveto{\pgfqpoint{0.468854in}{0.803025in}}{\pgfqpoint{0.475000in}{0.817863in}}{\pgfqpoint{0.475000in}{0.833333in}}%
\pgfpathcurveto{\pgfqpoint{0.475000in}{0.848804in}}{\pgfqpoint{0.468854in}{0.863642in}}{\pgfqpoint{0.457915in}{0.874581in}}%
\pgfpathcurveto{\pgfqpoint{0.446975in}{0.885520in}}{\pgfqpoint{0.432137in}{0.891667in}}{\pgfqpoint{0.416667in}{0.891667in}}%
\pgfpathcurveto{\pgfqpoint{0.401196in}{0.891667in}}{\pgfqpoint{0.386358in}{0.885520in}}{\pgfqpoint{0.375419in}{0.874581in}}%
\pgfpathcurveto{\pgfqpoint{0.364480in}{0.863642in}}{\pgfqpoint{0.358333in}{0.848804in}}{\pgfqpoint{0.358333in}{0.833333in}}%
\pgfpathcurveto{\pgfqpoint{0.358333in}{0.817863in}}{\pgfqpoint{0.364480in}{0.803025in}}{\pgfqpoint{0.375419in}{0.792085in}}%
\pgfpathcurveto{\pgfqpoint{0.386358in}{0.781146in}}{\pgfqpoint{0.401196in}{0.775000in}}{\pgfqpoint{0.416667in}{0.775000in}}%
\pgfpathclose%
\pgfpathmoveto{\pgfqpoint{0.416667in}{0.780833in}}%
\pgfpathcurveto{\pgfqpoint{0.416667in}{0.780833in}}{\pgfqpoint{0.402744in}{0.780833in}}{\pgfqpoint{0.389389in}{0.786365in}}%
\pgfpathcurveto{\pgfqpoint{0.379544in}{0.796210in}}{\pgfqpoint{0.369698in}{0.806055in}}{\pgfqpoint{0.364167in}{0.819410in}}%
\pgfpathcurveto{\pgfqpoint{0.364167in}{0.833333in}}{\pgfqpoint{0.364167in}{0.847256in}}{\pgfqpoint{0.369698in}{0.860611in}}%
\pgfpathcurveto{\pgfqpoint{0.379544in}{0.870456in}}{\pgfqpoint{0.389389in}{0.880302in}}{\pgfqpoint{0.402744in}{0.885833in}}%
\pgfpathcurveto{\pgfqpoint{0.416667in}{0.885833in}}{\pgfqpoint{0.430590in}{0.885833in}}{\pgfqpoint{0.443945in}{0.880302in}}%
\pgfpathcurveto{\pgfqpoint{0.453790in}{0.870456in}}{\pgfqpoint{0.463635in}{0.860611in}}{\pgfqpoint{0.469167in}{0.847256in}}%
\pgfpathcurveto{\pgfqpoint{0.469167in}{0.833333in}}{\pgfqpoint{0.469167in}{0.819410in}}{\pgfqpoint{0.463635in}{0.806055in}}%
\pgfpathcurveto{\pgfqpoint{0.453790in}{0.796210in}}{\pgfqpoint{0.443945in}{0.786365in}}{\pgfqpoint{0.430590in}{0.780833in}}%
\pgfpathclose%
\pgfpathmoveto{\pgfqpoint{0.583333in}{0.775000in}}%
\pgfpathcurveto{\pgfqpoint{0.598804in}{0.775000in}}{\pgfqpoint{0.613642in}{0.781146in}}{\pgfqpoint{0.624581in}{0.792085in}}%
\pgfpathcurveto{\pgfqpoint{0.635520in}{0.803025in}}{\pgfqpoint{0.641667in}{0.817863in}}{\pgfqpoint{0.641667in}{0.833333in}}%
\pgfpathcurveto{\pgfqpoint{0.641667in}{0.848804in}}{\pgfqpoint{0.635520in}{0.863642in}}{\pgfqpoint{0.624581in}{0.874581in}}%
\pgfpathcurveto{\pgfqpoint{0.613642in}{0.885520in}}{\pgfqpoint{0.598804in}{0.891667in}}{\pgfqpoint{0.583333in}{0.891667in}}%
\pgfpathcurveto{\pgfqpoint{0.567863in}{0.891667in}}{\pgfqpoint{0.553025in}{0.885520in}}{\pgfqpoint{0.542085in}{0.874581in}}%
\pgfpathcurveto{\pgfqpoint{0.531146in}{0.863642in}}{\pgfqpoint{0.525000in}{0.848804in}}{\pgfqpoint{0.525000in}{0.833333in}}%
\pgfpathcurveto{\pgfqpoint{0.525000in}{0.817863in}}{\pgfqpoint{0.531146in}{0.803025in}}{\pgfqpoint{0.542085in}{0.792085in}}%
\pgfpathcurveto{\pgfqpoint{0.553025in}{0.781146in}}{\pgfqpoint{0.567863in}{0.775000in}}{\pgfqpoint{0.583333in}{0.775000in}}%
\pgfpathclose%
\pgfpathmoveto{\pgfqpoint{0.583333in}{0.780833in}}%
\pgfpathcurveto{\pgfqpoint{0.583333in}{0.780833in}}{\pgfqpoint{0.569410in}{0.780833in}}{\pgfqpoint{0.556055in}{0.786365in}}%
\pgfpathcurveto{\pgfqpoint{0.546210in}{0.796210in}}{\pgfqpoint{0.536365in}{0.806055in}}{\pgfqpoint{0.530833in}{0.819410in}}%
\pgfpathcurveto{\pgfqpoint{0.530833in}{0.833333in}}{\pgfqpoint{0.530833in}{0.847256in}}{\pgfqpoint{0.536365in}{0.860611in}}%
\pgfpathcurveto{\pgfqpoint{0.546210in}{0.870456in}}{\pgfqpoint{0.556055in}{0.880302in}}{\pgfqpoint{0.569410in}{0.885833in}}%
\pgfpathcurveto{\pgfqpoint{0.583333in}{0.885833in}}{\pgfqpoint{0.597256in}{0.885833in}}{\pgfqpoint{0.610611in}{0.880302in}}%
\pgfpathcurveto{\pgfqpoint{0.620456in}{0.870456in}}{\pgfqpoint{0.630302in}{0.860611in}}{\pgfqpoint{0.635833in}{0.847256in}}%
\pgfpathcurveto{\pgfqpoint{0.635833in}{0.833333in}}{\pgfqpoint{0.635833in}{0.819410in}}{\pgfqpoint{0.630302in}{0.806055in}}%
\pgfpathcurveto{\pgfqpoint{0.620456in}{0.796210in}}{\pgfqpoint{0.610611in}{0.786365in}}{\pgfqpoint{0.597256in}{0.780833in}}%
\pgfpathclose%
\pgfpathmoveto{\pgfqpoint{0.750000in}{0.775000in}}%
\pgfpathcurveto{\pgfqpoint{0.765470in}{0.775000in}}{\pgfqpoint{0.780309in}{0.781146in}}{\pgfqpoint{0.791248in}{0.792085in}}%
\pgfpathcurveto{\pgfqpoint{0.802187in}{0.803025in}}{\pgfqpoint{0.808333in}{0.817863in}}{\pgfqpoint{0.808333in}{0.833333in}}%
\pgfpathcurveto{\pgfqpoint{0.808333in}{0.848804in}}{\pgfqpoint{0.802187in}{0.863642in}}{\pgfqpoint{0.791248in}{0.874581in}}%
\pgfpathcurveto{\pgfqpoint{0.780309in}{0.885520in}}{\pgfqpoint{0.765470in}{0.891667in}}{\pgfqpoint{0.750000in}{0.891667in}}%
\pgfpathcurveto{\pgfqpoint{0.734530in}{0.891667in}}{\pgfqpoint{0.719691in}{0.885520in}}{\pgfqpoint{0.708752in}{0.874581in}}%
\pgfpathcurveto{\pgfqpoint{0.697813in}{0.863642in}}{\pgfqpoint{0.691667in}{0.848804in}}{\pgfqpoint{0.691667in}{0.833333in}}%
\pgfpathcurveto{\pgfqpoint{0.691667in}{0.817863in}}{\pgfqpoint{0.697813in}{0.803025in}}{\pgfqpoint{0.708752in}{0.792085in}}%
\pgfpathcurveto{\pgfqpoint{0.719691in}{0.781146in}}{\pgfqpoint{0.734530in}{0.775000in}}{\pgfqpoint{0.750000in}{0.775000in}}%
\pgfpathclose%
\pgfpathmoveto{\pgfqpoint{0.750000in}{0.780833in}}%
\pgfpathcurveto{\pgfqpoint{0.750000in}{0.780833in}}{\pgfqpoint{0.736077in}{0.780833in}}{\pgfqpoint{0.722722in}{0.786365in}}%
\pgfpathcurveto{\pgfqpoint{0.712877in}{0.796210in}}{\pgfqpoint{0.703032in}{0.806055in}}{\pgfqpoint{0.697500in}{0.819410in}}%
\pgfpathcurveto{\pgfqpoint{0.697500in}{0.833333in}}{\pgfqpoint{0.697500in}{0.847256in}}{\pgfqpoint{0.703032in}{0.860611in}}%
\pgfpathcurveto{\pgfqpoint{0.712877in}{0.870456in}}{\pgfqpoint{0.722722in}{0.880302in}}{\pgfqpoint{0.736077in}{0.885833in}}%
\pgfpathcurveto{\pgfqpoint{0.750000in}{0.885833in}}{\pgfqpoint{0.763923in}{0.885833in}}{\pgfqpoint{0.777278in}{0.880302in}}%
\pgfpathcurveto{\pgfqpoint{0.787123in}{0.870456in}}{\pgfqpoint{0.796968in}{0.860611in}}{\pgfqpoint{0.802500in}{0.847256in}}%
\pgfpathcurveto{\pgfqpoint{0.802500in}{0.833333in}}{\pgfqpoint{0.802500in}{0.819410in}}{\pgfqpoint{0.796968in}{0.806055in}}%
\pgfpathcurveto{\pgfqpoint{0.787123in}{0.796210in}}{\pgfqpoint{0.777278in}{0.786365in}}{\pgfqpoint{0.763923in}{0.780833in}}%
\pgfpathclose%
\pgfpathmoveto{\pgfqpoint{0.916667in}{0.775000in}}%
\pgfpathcurveto{\pgfqpoint{0.932137in}{0.775000in}}{\pgfqpoint{0.946975in}{0.781146in}}{\pgfqpoint{0.957915in}{0.792085in}}%
\pgfpathcurveto{\pgfqpoint{0.968854in}{0.803025in}}{\pgfqpoint{0.975000in}{0.817863in}}{\pgfqpoint{0.975000in}{0.833333in}}%
\pgfpathcurveto{\pgfqpoint{0.975000in}{0.848804in}}{\pgfqpoint{0.968854in}{0.863642in}}{\pgfqpoint{0.957915in}{0.874581in}}%
\pgfpathcurveto{\pgfqpoint{0.946975in}{0.885520in}}{\pgfqpoint{0.932137in}{0.891667in}}{\pgfqpoint{0.916667in}{0.891667in}}%
\pgfpathcurveto{\pgfqpoint{0.901196in}{0.891667in}}{\pgfqpoint{0.886358in}{0.885520in}}{\pgfqpoint{0.875419in}{0.874581in}}%
\pgfpathcurveto{\pgfqpoint{0.864480in}{0.863642in}}{\pgfqpoint{0.858333in}{0.848804in}}{\pgfqpoint{0.858333in}{0.833333in}}%
\pgfpathcurveto{\pgfqpoint{0.858333in}{0.817863in}}{\pgfqpoint{0.864480in}{0.803025in}}{\pgfqpoint{0.875419in}{0.792085in}}%
\pgfpathcurveto{\pgfqpoint{0.886358in}{0.781146in}}{\pgfqpoint{0.901196in}{0.775000in}}{\pgfqpoint{0.916667in}{0.775000in}}%
\pgfpathclose%
\pgfpathmoveto{\pgfqpoint{0.916667in}{0.780833in}}%
\pgfpathcurveto{\pgfqpoint{0.916667in}{0.780833in}}{\pgfqpoint{0.902744in}{0.780833in}}{\pgfqpoint{0.889389in}{0.786365in}}%
\pgfpathcurveto{\pgfqpoint{0.879544in}{0.796210in}}{\pgfqpoint{0.869698in}{0.806055in}}{\pgfqpoint{0.864167in}{0.819410in}}%
\pgfpathcurveto{\pgfqpoint{0.864167in}{0.833333in}}{\pgfqpoint{0.864167in}{0.847256in}}{\pgfqpoint{0.869698in}{0.860611in}}%
\pgfpathcurveto{\pgfqpoint{0.879544in}{0.870456in}}{\pgfqpoint{0.889389in}{0.880302in}}{\pgfqpoint{0.902744in}{0.885833in}}%
\pgfpathcurveto{\pgfqpoint{0.916667in}{0.885833in}}{\pgfqpoint{0.930590in}{0.885833in}}{\pgfqpoint{0.943945in}{0.880302in}}%
\pgfpathcurveto{\pgfqpoint{0.953790in}{0.870456in}}{\pgfqpoint{0.963635in}{0.860611in}}{\pgfqpoint{0.969167in}{0.847256in}}%
\pgfpathcurveto{\pgfqpoint{0.969167in}{0.833333in}}{\pgfqpoint{0.969167in}{0.819410in}}{\pgfqpoint{0.963635in}{0.806055in}}%
\pgfpathcurveto{\pgfqpoint{0.953790in}{0.796210in}}{\pgfqpoint{0.943945in}{0.786365in}}{\pgfqpoint{0.930590in}{0.780833in}}%
\pgfpathclose%
\pgfpathmoveto{\pgfqpoint{0.000000in}{0.941667in}}%
\pgfpathcurveto{\pgfqpoint{0.015470in}{0.941667in}}{\pgfqpoint{0.030309in}{0.947813in}}{\pgfqpoint{0.041248in}{0.958752in}}%
\pgfpathcurveto{\pgfqpoint{0.052187in}{0.969691in}}{\pgfqpoint{0.058333in}{0.984530in}}{\pgfqpoint{0.058333in}{1.000000in}}%
\pgfpathcurveto{\pgfqpoint{0.058333in}{1.015470in}}{\pgfqpoint{0.052187in}{1.030309in}}{\pgfqpoint{0.041248in}{1.041248in}}%
\pgfpathcurveto{\pgfqpoint{0.030309in}{1.052187in}}{\pgfqpoint{0.015470in}{1.058333in}}{\pgfqpoint{0.000000in}{1.058333in}}%
\pgfpathcurveto{\pgfqpoint{-0.015470in}{1.058333in}}{\pgfqpoint{-0.030309in}{1.052187in}}{\pgfqpoint{-0.041248in}{1.041248in}}%
\pgfpathcurveto{\pgfqpoint{-0.052187in}{1.030309in}}{\pgfqpoint{-0.058333in}{1.015470in}}{\pgfqpoint{-0.058333in}{1.000000in}}%
\pgfpathcurveto{\pgfqpoint{-0.058333in}{0.984530in}}{\pgfqpoint{-0.052187in}{0.969691in}}{\pgfqpoint{-0.041248in}{0.958752in}}%
\pgfpathcurveto{\pgfqpoint{-0.030309in}{0.947813in}}{\pgfqpoint{-0.015470in}{0.941667in}}{\pgfqpoint{0.000000in}{0.941667in}}%
\pgfpathclose%
\pgfpathmoveto{\pgfqpoint{0.000000in}{0.947500in}}%
\pgfpathcurveto{\pgfqpoint{0.000000in}{0.947500in}}{\pgfqpoint{-0.013923in}{0.947500in}}{\pgfqpoint{-0.027278in}{0.953032in}}%
\pgfpathcurveto{\pgfqpoint{-0.037123in}{0.962877in}}{\pgfqpoint{-0.046968in}{0.972722in}}{\pgfqpoint{-0.052500in}{0.986077in}}%
\pgfpathcurveto{\pgfqpoint{-0.052500in}{1.000000in}}{\pgfqpoint{-0.052500in}{1.013923in}}{\pgfqpoint{-0.046968in}{1.027278in}}%
\pgfpathcurveto{\pgfqpoint{-0.037123in}{1.037123in}}{\pgfqpoint{-0.027278in}{1.046968in}}{\pgfqpoint{-0.013923in}{1.052500in}}%
\pgfpathcurveto{\pgfqpoint{0.000000in}{1.052500in}}{\pgfqpoint{0.013923in}{1.052500in}}{\pgfqpoint{0.027278in}{1.046968in}}%
\pgfpathcurveto{\pgfqpoint{0.037123in}{1.037123in}}{\pgfqpoint{0.046968in}{1.027278in}}{\pgfqpoint{0.052500in}{1.013923in}}%
\pgfpathcurveto{\pgfqpoint{0.052500in}{1.000000in}}{\pgfqpoint{0.052500in}{0.986077in}}{\pgfqpoint{0.046968in}{0.972722in}}%
\pgfpathcurveto{\pgfqpoint{0.037123in}{0.962877in}}{\pgfqpoint{0.027278in}{0.953032in}}{\pgfqpoint{0.013923in}{0.947500in}}%
\pgfpathclose%
\pgfpathmoveto{\pgfqpoint{0.166667in}{0.941667in}}%
\pgfpathcurveto{\pgfqpoint{0.182137in}{0.941667in}}{\pgfqpoint{0.196975in}{0.947813in}}{\pgfqpoint{0.207915in}{0.958752in}}%
\pgfpathcurveto{\pgfqpoint{0.218854in}{0.969691in}}{\pgfqpoint{0.225000in}{0.984530in}}{\pgfqpoint{0.225000in}{1.000000in}}%
\pgfpathcurveto{\pgfqpoint{0.225000in}{1.015470in}}{\pgfqpoint{0.218854in}{1.030309in}}{\pgfqpoint{0.207915in}{1.041248in}}%
\pgfpathcurveto{\pgfqpoint{0.196975in}{1.052187in}}{\pgfqpoint{0.182137in}{1.058333in}}{\pgfqpoint{0.166667in}{1.058333in}}%
\pgfpathcurveto{\pgfqpoint{0.151196in}{1.058333in}}{\pgfqpoint{0.136358in}{1.052187in}}{\pgfqpoint{0.125419in}{1.041248in}}%
\pgfpathcurveto{\pgfqpoint{0.114480in}{1.030309in}}{\pgfqpoint{0.108333in}{1.015470in}}{\pgfqpoint{0.108333in}{1.000000in}}%
\pgfpathcurveto{\pgfqpoint{0.108333in}{0.984530in}}{\pgfqpoint{0.114480in}{0.969691in}}{\pgfqpoint{0.125419in}{0.958752in}}%
\pgfpathcurveto{\pgfqpoint{0.136358in}{0.947813in}}{\pgfqpoint{0.151196in}{0.941667in}}{\pgfqpoint{0.166667in}{0.941667in}}%
\pgfpathclose%
\pgfpathmoveto{\pgfqpoint{0.166667in}{0.947500in}}%
\pgfpathcurveto{\pgfqpoint{0.166667in}{0.947500in}}{\pgfqpoint{0.152744in}{0.947500in}}{\pgfqpoint{0.139389in}{0.953032in}}%
\pgfpathcurveto{\pgfqpoint{0.129544in}{0.962877in}}{\pgfqpoint{0.119698in}{0.972722in}}{\pgfqpoint{0.114167in}{0.986077in}}%
\pgfpathcurveto{\pgfqpoint{0.114167in}{1.000000in}}{\pgfqpoint{0.114167in}{1.013923in}}{\pgfqpoint{0.119698in}{1.027278in}}%
\pgfpathcurveto{\pgfqpoint{0.129544in}{1.037123in}}{\pgfqpoint{0.139389in}{1.046968in}}{\pgfqpoint{0.152744in}{1.052500in}}%
\pgfpathcurveto{\pgfqpoint{0.166667in}{1.052500in}}{\pgfqpoint{0.180590in}{1.052500in}}{\pgfqpoint{0.193945in}{1.046968in}}%
\pgfpathcurveto{\pgfqpoint{0.203790in}{1.037123in}}{\pgfqpoint{0.213635in}{1.027278in}}{\pgfqpoint{0.219167in}{1.013923in}}%
\pgfpathcurveto{\pgfqpoint{0.219167in}{1.000000in}}{\pgfqpoint{0.219167in}{0.986077in}}{\pgfqpoint{0.213635in}{0.972722in}}%
\pgfpathcurveto{\pgfqpoint{0.203790in}{0.962877in}}{\pgfqpoint{0.193945in}{0.953032in}}{\pgfqpoint{0.180590in}{0.947500in}}%
\pgfpathclose%
\pgfpathmoveto{\pgfqpoint{0.333333in}{0.941667in}}%
\pgfpathcurveto{\pgfqpoint{0.348804in}{0.941667in}}{\pgfqpoint{0.363642in}{0.947813in}}{\pgfqpoint{0.374581in}{0.958752in}}%
\pgfpathcurveto{\pgfqpoint{0.385520in}{0.969691in}}{\pgfqpoint{0.391667in}{0.984530in}}{\pgfqpoint{0.391667in}{1.000000in}}%
\pgfpathcurveto{\pgfqpoint{0.391667in}{1.015470in}}{\pgfqpoint{0.385520in}{1.030309in}}{\pgfqpoint{0.374581in}{1.041248in}}%
\pgfpathcurveto{\pgfqpoint{0.363642in}{1.052187in}}{\pgfqpoint{0.348804in}{1.058333in}}{\pgfqpoint{0.333333in}{1.058333in}}%
\pgfpathcurveto{\pgfqpoint{0.317863in}{1.058333in}}{\pgfqpoint{0.303025in}{1.052187in}}{\pgfqpoint{0.292085in}{1.041248in}}%
\pgfpathcurveto{\pgfqpoint{0.281146in}{1.030309in}}{\pgfqpoint{0.275000in}{1.015470in}}{\pgfqpoint{0.275000in}{1.000000in}}%
\pgfpathcurveto{\pgfqpoint{0.275000in}{0.984530in}}{\pgfqpoint{0.281146in}{0.969691in}}{\pgfqpoint{0.292085in}{0.958752in}}%
\pgfpathcurveto{\pgfqpoint{0.303025in}{0.947813in}}{\pgfqpoint{0.317863in}{0.941667in}}{\pgfqpoint{0.333333in}{0.941667in}}%
\pgfpathclose%
\pgfpathmoveto{\pgfqpoint{0.333333in}{0.947500in}}%
\pgfpathcurveto{\pgfqpoint{0.333333in}{0.947500in}}{\pgfqpoint{0.319410in}{0.947500in}}{\pgfqpoint{0.306055in}{0.953032in}}%
\pgfpathcurveto{\pgfqpoint{0.296210in}{0.962877in}}{\pgfqpoint{0.286365in}{0.972722in}}{\pgfqpoint{0.280833in}{0.986077in}}%
\pgfpathcurveto{\pgfqpoint{0.280833in}{1.000000in}}{\pgfqpoint{0.280833in}{1.013923in}}{\pgfqpoint{0.286365in}{1.027278in}}%
\pgfpathcurveto{\pgfqpoint{0.296210in}{1.037123in}}{\pgfqpoint{0.306055in}{1.046968in}}{\pgfqpoint{0.319410in}{1.052500in}}%
\pgfpathcurveto{\pgfqpoint{0.333333in}{1.052500in}}{\pgfqpoint{0.347256in}{1.052500in}}{\pgfqpoint{0.360611in}{1.046968in}}%
\pgfpathcurveto{\pgfqpoint{0.370456in}{1.037123in}}{\pgfqpoint{0.380302in}{1.027278in}}{\pgfqpoint{0.385833in}{1.013923in}}%
\pgfpathcurveto{\pgfqpoint{0.385833in}{1.000000in}}{\pgfqpoint{0.385833in}{0.986077in}}{\pgfqpoint{0.380302in}{0.972722in}}%
\pgfpathcurveto{\pgfqpoint{0.370456in}{0.962877in}}{\pgfqpoint{0.360611in}{0.953032in}}{\pgfqpoint{0.347256in}{0.947500in}}%
\pgfpathclose%
\pgfpathmoveto{\pgfqpoint{0.500000in}{0.941667in}}%
\pgfpathcurveto{\pgfqpoint{0.515470in}{0.941667in}}{\pgfqpoint{0.530309in}{0.947813in}}{\pgfqpoint{0.541248in}{0.958752in}}%
\pgfpathcurveto{\pgfqpoint{0.552187in}{0.969691in}}{\pgfqpoint{0.558333in}{0.984530in}}{\pgfqpoint{0.558333in}{1.000000in}}%
\pgfpathcurveto{\pgfqpoint{0.558333in}{1.015470in}}{\pgfqpoint{0.552187in}{1.030309in}}{\pgfqpoint{0.541248in}{1.041248in}}%
\pgfpathcurveto{\pgfqpoint{0.530309in}{1.052187in}}{\pgfqpoint{0.515470in}{1.058333in}}{\pgfqpoint{0.500000in}{1.058333in}}%
\pgfpathcurveto{\pgfqpoint{0.484530in}{1.058333in}}{\pgfqpoint{0.469691in}{1.052187in}}{\pgfqpoint{0.458752in}{1.041248in}}%
\pgfpathcurveto{\pgfqpoint{0.447813in}{1.030309in}}{\pgfqpoint{0.441667in}{1.015470in}}{\pgfqpoint{0.441667in}{1.000000in}}%
\pgfpathcurveto{\pgfqpoint{0.441667in}{0.984530in}}{\pgfqpoint{0.447813in}{0.969691in}}{\pgfqpoint{0.458752in}{0.958752in}}%
\pgfpathcurveto{\pgfqpoint{0.469691in}{0.947813in}}{\pgfqpoint{0.484530in}{0.941667in}}{\pgfqpoint{0.500000in}{0.941667in}}%
\pgfpathclose%
\pgfpathmoveto{\pgfqpoint{0.500000in}{0.947500in}}%
\pgfpathcurveto{\pgfqpoint{0.500000in}{0.947500in}}{\pgfqpoint{0.486077in}{0.947500in}}{\pgfqpoint{0.472722in}{0.953032in}}%
\pgfpathcurveto{\pgfqpoint{0.462877in}{0.962877in}}{\pgfqpoint{0.453032in}{0.972722in}}{\pgfqpoint{0.447500in}{0.986077in}}%
\pgfpathcurveto{\pgfqpoint{0.447500in}{1.000000in}}{\pgfqpoint{0.447500in}{1.013923in}}{\pgfqpoint{0.453032in}{1.027278in}}%
\pgfpathcurveto{\pgfqpoint{0.462877in}{1.037123in}}{\pgfqpoint{0.472722in}{1.046968in}}{\pgfqpoint{0.486077in}{1.052500in}}%
\pgfpathcurveto{\pgfqpoint{0.500000in}{1.052500in}}{\pgfqpoint{0.513923in}{1.052500in}}{\pgfqpoint{0.527278in}{1.046968in}}%
\pgfpathcurveto{\pgfqpoint{0.537123in}{1.037123in}}{\pgfqpoint{0.546968in}{1.027278in}}{\pgfqpoint{0.552500in}{1.013923in}}%
\pgfpathcurveto{\pgfqpoint{0.552500in}{1.000000in}}{\pgfqpoint{0.552500in}{0.986077in}}{\pgfqpoint{0.546968in}{0.972722in}}%
\pgfpathcurveto{\pgfqpoint{0.537123in}{0.962877in}}{\pgfqpoint{0.527278in}{0.953032in}}{\pgfqpoint{0.513923in}{0.947500in}}%
\pgfpathclose%
\pgfpathmoveto{\pgfqpoint{0.666667in}{0.941667in}}%
\pgfpathcurveto{\pgfqpoint{0.682137in}{0.941667in}}{\pgfqpoint{0.696975in}{0.947813in}}{\pgfqpoint{0.707915in}{0.958752in}}%
\pgfpathcurveto{\pgfqpoint{0.718854in}{0.969691in}}{\pgfqpoint{0.725000in}{0.984530in}}{\pgfqpoint{0.725000in}{1.000000in}}%
\pgfpathcurveto{\pgfqpoint{0.725000in}{1.015470in}}{\pgfqpoint{0.718854in}{1.030309in}}{\pgfqpoint{0.707915in}{1.041248in}}%
\pgfpathcurveto{\pgfqpoint{0.696975in}{1.052187in}}{\pgfqpoint{0.682137in}{1.058333in}}{\pgfqpoint{0.666667in}{1.058333in}}%
\pgfpathcurveto{\pgfqpoint{0.651196in}{1.058333in}}{\pgfqpoint{0.636358in}{1.052187in}}{\pgfqpoint{0.625419in}{1.041248in}}%
\pgfpathcurveto{\pgfqpoint{0.614480in}{1.030309in}}{\pgfqpoint{0.608333in}{1.015470in}}{\pgfqpoint{0.608333in}{1.000000in}}%
\pgfpathcurveto{\pgfqpoint{0.608333in}{0.984530in}}{\pgfqpoint{0.614480in}{0.969691in}}{\pgfqpoint{0.625419in}{0.958752in}}%
\pgfpathcurveto{\pgfqpoint{0.636358in}{0.947813in}}{\pgfqpoint{0.651196in}{0.941667in}}{\pgfqpoint{0.666667in}{0.941667in}}%
\pgfpathclose%
\pgfpathmoveto{\pgfqpoint{0.666667in}{0.947500in}}%
\pgfpathcurveto{\pgfqpoint{0.666667in}{0.947500in}}{\pgfqpoint{0.652744in}{0.947500in}}{\pgfqpoint{0.639389in}{0.953032in}}%
\pgfpathcurveto{\pgfqpoint{0.629544in}{0.962877in}}{\pgfqpoint{0.619698in}{0.972722in}}{\pgfqpoint{0.614167in}{0.986077in}}%
\pgfpathcurveto{\pgfqpoint{0.614167in}{1.000000in}}{\pgfqpoint{0.614167in}{1.013923in}}{\pgfqpoint{0.619698in}{1.027278in}}%
\pgfpathcurveto{\pgfqpoint{0.629544in}{1.037123in}}{\pgfqpoint{0.639389in}{1.046968in}}{\pgfqpoint{0.652744in}{1.052500in}}%
\pgfpathcurveto{\pgfqpoint{0.666667in}{1.052500in}}{\pgfqpoint{0.680590in}{1.052500in}}{\pgfqpoint{0.693945in}{1.046968in}}%
\pgfpathcurveto{\pgfqpoint{0.703790in}{1.037123in}}{\pgfqpoint{0.713635in}{1.027278in}}{\pgfqpoint{0.719167in}{1.013923in}}%
\pgfpathcurveto{\pgfqpoint{0.719167in}{1.000000in}}{\pgfqpoint{0.719167in}{0.986077in}}{\pgfqpoint{0.713635in}{0.972722in}}%
\pgfpathcurveto{\pgfqpoint{0.703790in}{0.962877in}}{\pgfqpoint{0.693945in}{0.953032in}}{\pgfqpoint{0.680590in}{0.947500in}}%
\pgfpathclose%
\pgfpathmoveto{\pgfqpoint{0.833333in}{0.941667in}}%
\pgfpathcurveto{\pgfqpoint{0.848804in}{0.941667in}}{\pgfqpoint{0.863642in}{0.947813in}}{\pgfqpoint{0.874581in}{0.958752in}}%
\pgfpathcurveto{\pgfqpoint{0.885520in}{0.969691in}}{\pgfqpoint{0.891667in}{0.984530in}}{\pgfqpoint{0.891667in}{1.000000in}}%
\pgfpathcurveto{\pgfqpoint{0.891667in}{1.015470in}}{\pgfqpoint{0.885520in}{1.030309in}}{\pgfqpoint{0.874581in}{1.041248in}}%
\pgfpathcurveto{\pgfqpoint{0.863642in}{1.052187in}}{\pgfqpoint{0.848804in}{1.058333in}}{\pgfqpoint{0.833333in}{1.058333in}}%
\pgfpathcurveto{\pgfqpoint{0.817863in}{1.058333in}}{\pgfqpoint{0.803025in}{1.052187in}}{\pgfqpoint{0.792085in}{1.041248in}}%
\pgfpathcurveto{\pgfqpoint{0.781146in}{1.030309in}}{\pgfqpoint{0.775000in}{1.015470in}}{\pgfqpoint{0.775000in}{1.000000in}}%
\pgfpathcurveto{\pgfqpoint{0.775000in}{0.984530in}}{\pgfqpoint{0.781146in}{0.969691in}}{\pgfqpoint{0.792085in}{0.958752in}}%
\pgfpathcurveto{\pgfqpoint{0.803025in}{0.947813in}}{\pgfqpoint{0.817863in}{0.941667in}}{\pgfqpoint{0.833333in}{0.941667in}}%
\pgfpathclose%
\pgfpathmoveto{\pgfqpoint{0.833333in}{0.947500in}}%
\pgfpathcurveto{\pgfqpoint{0.833333in}{0.947500in}}{\pgfqpoint{0.819410in}{0.947500in}}{\pgfqpoint{0.806055in}{0.953032in}}%
\pgfpathcurveto{\pgfqpoint{0.796210in}{0.962877in}}{\pgfqpoint{0.786365in}{0.972722in}}{\pgfqpoint{0.780833in}{0.986077in}}%
\pgfpathcurveto{\pgfqpoint{0.780833in}{1.000000in}}{\pgfqpoint{0.780833in}{1.013923in}}{\pgfqpoint{0.786365in}{1.027278in}}%
\pgfpathcurveto{\pgfqpoint{0.796210in}{1.037123in}}{\pgfqpoint{0.806055in}{1.046968in}}{\pgfqpoint{0.819410in}{1.052500in}}%
\pgfpathcurveto{\pgfqpoint{0.833333in}{1.052500in}}{\pgfqpoint{0.847256in}{1.052500in}}{\pgfqpoint{0.860611in}{1.046968in}}%
\pgfpathcurveto{\pgfqpoint{0.870456in}{1.037123in}}{\pgfqpoint{0.880302in}{1.027278in}}{\pgfqpoint{0.885833in}{1.013923in}}%
\pgfpathcurveto{\pgfqpoint{0.885833in}{1.000000in}}{\pgfqpoint{0.885833in}{0.986077in}}{\pgfqpoint{0.880302in}{0.972722in}}%
\pgfpathcurveto{\pgfqpoint{0.870456in}{0.962877in}}{\pgfqpoint{0.860611in}{0.953032in}}{\pgfqpoint{0.847256in}{0.947500in}}%
\pgfpathclose%
\pgfpathmoveto{\pgfqpoint{1.000000in}{0.941667in}}%
\pgfpathcurveto{\pgfqpoint{1.015470in}{0.941667in}}{\pgfqpoint{1.030309in}{0.947813in}}{\pgfqpoint{1.041248in}{0.958752in}}%
\pgfpathcurveto{\pgfqpoint{1.052187in}{0.969691in}}{\pgfqpoint{1.058333in}{0.984530in}}{\pgfqpoint{1.058333in}{1.000000in}}%
\pgfpathcurveto{\pgfqpoint{1.058333in}{1.015470in}}{\pgfqpoint{1.052187in}{1.030309in}}{\pgfqpoint{1.041248in}{1.041248in}}%
\pgfpathcurveto{\pgfqpoint{1.030309in}{1.052187in}}{\pgfqpoint{1.015470in}{1.058333in}}{\pgfqpoint{1.000000in}{1.058333in}}%
\pgfpathcurveto{\pgfqpoint{0.984530in}{1.058333in}}{\pgfqpoint{0.969691in}{1.052187in}}{\pgfqpoint{0.958752in}{1.041248in}}%
\pgfpathcurveto{\pgfqpoint{0.947813in}{1.030309in}}{\pgfqpoint{0.941667in}{1.015470in}}{\pgfqpoint{0.941667in}{1.000000in}}%
\pgfpathcurveto{\pgfqpoint{0.941667in}{0.984530in}}{\pgfqpoint{0.947813in}{0.969691in}}{\pgfqpoint{0.958752in}{0.958752in}}%
\pgfpathcurveto{\pgfqpoint{0.969691in}{0.947813in}}{\pgfqpoint{0.984530in}{0.941667in}}{\pgfqpoint{1.000000in}{0.941667in}}%
\pgfpathclose%
\pgfpathmoveto{\pgfqpoint{1.000000in}{0.947500in}}%
\pgfpathcurveto{\pgfqpoint{1.000000in}{0.947500in}}{\pgfqpoint{0.986077in}{0.947500in}}{\pgfqpoint{0.972722in}{0.953032in}}%
\pgfpathcurveto{\pgfqpoint{0.962877in}{0.962877in}}{\pgfqpoint{0.953032in}{0.972722in}}{\pgfqpoint{0.947500in}{0.986077in}}%
\pgfpathcurveto{\pgfqpoint{0.947500in}{1.000000in}}{\pgfqpoint{0.947500in}{1.013923in}}{\pgfqpoint{0.953032in}{1.027278in}}%
\pgfpathcurveto{\pgfqpoint{0.962877in}{1.037123in}}{\pgfqpoint{0.972722in}{1.046968in}}{\pgfqpoint{0.986077in}{1.052500in}}%
\pgfpathcurveto{\pgfqpoint{1.000000in}{1.052500in}}{\pgfqpoint{1.013923in}{1.052500in}}{\pgfqpoint{1.027278in}{1.046968in}}%
\pgfpathcurveto{\pgfqpoint{1.037123in}{1.037123in}}{\pgfqpoint{1.046968in}{1.027278in}}{\pgfqpoint{1.052500in}{1.013923in}}%
\pgfpathcurveto{\pgfqpoint{1.052500in}{1.000000in}}{\pgfqpoint{1.052500in}{0.986077in}}{\pgfqpoint{1.046968in}{0.972722in}}%
\pgfpathcurveto{\pgfqpoint{1.037123in}{0.962877in}}{\pgfqpoint{1.027278in}{0.953032in}}{\pgfqpoint{1.013923in}{0.947500in}}%
\pgfpathclose%
\pgfusepath{stroke}%
\end{pgfscope}%
}%
\pgfsys@transformshift{4.578174in}{6.293449in}%
\pgfsys@useobject{currentpattern}{}%
\pgfsys@transformshift{1in}{0in}%
\pgfsys@transformshift{-1in}{0in}%
\pgfsys@transformshift{0in}{1in}%
\pgfsys@useobject{currentpattern}{}%
\pgfsys@transformshift{1in}{0in}%
\pgfsys@transformshift{-1in}{0in}%
\pgfsys@transformshift{0in}{1in}%
\end{pgfscope}%
\begin{pgfscope}%
\pgfpathrectangle{\pgfqpoint{1.090674in}{0.637495in}}{\pgfqpoint{9.300000in}{9.060000in}}%
\pgfusepath{clip}%
\pgfsetbuttcap%
\pgfsetmiterjoin%
\definecolor{currentfill}{rgb}{0.890196,0.466667,0.760784}%
\pgfsetfillcolor{currentfill}%
\pgfsetfillopacity{0.990000}%
\pgfsetlinewidth{0.000000pt}%
\definecolor{currentstroke}{rgb}{0.000000,0.000000,0.000000}%
\pgfsetstrokecolor{currentstroke}%
\pgfsetstrokeopacity{0.990000}%
\pgfsetdash{}{0pt}%
\pgfpathmoveto{\pgfqpoint{6.128174in}{6.550538in}}%
\pgfpathlineto{\pgfqpoint{6.903174in}{6.550538in}}%
\pgfpathlineto{\pgfqpoint{6.903174in}{8.575781in}}%
\pgfpathlineto{\pgfqpoint{6.128174in}{8.575781in}}%
\pgfpathclose%
\pgfusepath{fill}%
\end{pgfscope}%
\begin{pgfscope}%
\pgfsetbuttcap%
\pgfsetmiterjoin%
\definecolor{currentfill}{rgb}{0.890196,0.466667,0.760784}%
\pgfsetfillcolor{currentfill}%
\pgfsetfillopacity{0.990000}%
\pgfsetlinewidth{0.000000pt}%
\definecolor{currentstroke}{rgb}{0.000000,0.000000,0.000000}%
\pgfsetstrokecolor{currentstroke}%
\pgfsetstrokeopacity{0.990000}%
\pgfsetdash{}{0pt}%
\pgfpathrectangle{\pgfqpoint{1.090674in}{0.637495in}}{\pgfqpoint{9.300000in}{9.060000in}}%
\pgfusepath{clip}%
\pgfpathmoveto{\pgfqpoint{6.128174in}{6.550538in}}%
\pgfpathlineto{\pgfqpoint{6.903174in}{6.550538in}}%
\pgfpathlineto{\pgfqpoint{6.903174in}{8.575781in}}%
\pgfpathlineto{\pgfqpoint{6.128174in}{8.575781in}}%
\pgfpathclose%
\pgfusepath{clip}%
\pgfsys@defobject{currentpattern}{\pgfqpoint{0in}{0in}}{\pgfqpoint{1in}{1in}}{%
\begin{pgfscope}%
\pgfpathrectangle{\pgfqpoint{0in}{0in}}{\pgfqpoint{1in}{1in}}%
\pgfusepath{clip}%
\pgfpathmoveto{\pgfqpoint{0.000000in}{-0.058333in}}%
\pgfpathcurveto{\pgfqpoint{0.015470in}{-0.058333in}}{\pgfqpoint{0.030309in}{-0.052187in}}{\pgfqpoint{0.041248in}{-0.041248in}}%
\pgfpathcurveto{\pgfqpoint{0.052187in}{-0.030309in}}{\pgfqpoint{0.058333in}{-0.015470in}}{\pgfqpoint{0.058333in}{0.000000in}}%
\pgfpathcurveto{\pgfqpoint{0.058333in}{0.015470in}}{\pgfqpoint{0.052187in}{0.030309in}}{\pgfqpoint{0.041248in}{0.041248in}}%
\pgfpathcurveto{\pgfqpoint{0.030309in}{0.052187in}}{\pgfqpoint{0.015470in}{0.058333in}}{\pgfqpoint{0.000000in}{0.058333in}}%
\pgfpathcurveto{\pgfqpoint{-0.015470in}{0.058333in}}{\pgfqpoint{-0.030309in}{0.052187in}}{\pgfqpoint{-0.041248in}{0.041248in}}%
\pgfpathcurveto{\pgfqpoint{-0.052187in}{0.030309in}}{\pgfqpoint{-0.058333in}{0.015470in}}{\pgfqpoint{-0.058333in}{0.000000in}}%
\pgfpathcurveto{\pgfqpoint{-0.058333in}{-0.015470in}}{\pgfqpoint{-0.052187in}{-0.030309in}}{\pgfqpoint{-0.041248in}{-0.041248in}}%
\pgfpathcurveto{\pgfqpoint{-0.030309in}{-0.052187in}}{\pgfqpoint{-0.015470in}{-0.058333in}}{\pgfqpoint{0.000000in}{-0.058333in}}%
\pgfpathclose%
\pgfpathmoveto{\pgfqpoint{0.000000in}{-0.052500in}}%
\pgfpathcurveto{\pgfqpoint{0.000000in}{-0.052500in}}{\pgfqpoint{-0.013923in}{-0.052500in}}{\pgfqpoint{-0.027278in}{-0.046968in}}%
\pgfpathcurveto{\pgfqpoint{-0.037123in}{-0.037123in}}{\pgfqpoint{-0.046968in}{-0.027278in}}{\pgfqpoint{-0.052500in}{-0.013923in}}%
\pgfpathcurveto{\pgfqpoint{-0.052500in}{0.000000in}}{\pgfqpoint{-0.052500in}{0.013923in}}{\pgfqpoint{-0.046968in}{0.027278in}}%
\pgfpathcurveto{\pgfqpoint{-0.037123in}{0.037123in}}{\pgfqpoint{-0.027278in}{0.046968in}}{\pgfqpoint{-0.013923in}{0.052500in}}%
\pgfpathcurveto{\pgfqpoint{0.000000in}{0.052500in}}{\pgfqpoint{0.013923in}{0.052500in}}{\pgfqpoint{0.027278in}{0.046968in}}%
\pgfpathcurveto{\pgfqpoint{0.037123in}{0.037123in}}{\pgfqpoint{0.046968in}{0.027278in}}{\pgfqpoint{0.052500in}{0.013923in}}%
\pgfpathcurveto{\pgfqpoint{0.052500in}{0.000000in}}{\pgfqpoint{0.052500in}{-0.013923in}}{\pgfqpoint{0.046968in}{-0.027278in}}%
\pgfpathcurveto{\pgfqpoint{0.037123in}{-0.037123in}}{\pgfqpoint{0.027278in}{-0.046968in}}{\pgfqpoint{0.013923in}{-0.052500in}}%
\pgfpathclose%
\pgfpathmoveto{\pgfqpoint{0.166667in}{-0.058333in}}%
\pgfpathcurveto{\pgfqpoint{0.182137in}{-0.058333in}}{\pgfqpoint{0.196975in}{-0.052187in}}{\pgfqpoint{0.207915in}{-0.041248in}}%
\pgfpathcurveto{\pgfqpoint{0.218854in}{-0.030309in}}{\pgfqpoint{0.225000in}{-0.015470in}}{\pgfqpoint{0.225000in}{0.000000in}}%
\pgfpathcurveto{\pgfqpoint{0.225000in}{0.015470in}}{\pgfqpoint{0.218854in}{0.030309in}}{\pgfqpoint{0.207915in}{0.041248in}}%
\pgfpathcurveto{\pgfqpoint{0.196975in}{0.052187in}}{\pgfqpoint{0.182137in}{0.058333in}}{\pgfqpoint{0.166667in}{0.058333in}}%
\pgfpathcurveto{\pgfqpoint{0.151196in}{0.058333in}}{\pgfqpoint{0.136358in}{0.052187in}}{\pgfqpoint{0.125419in}{0.041248in}}%
\pgfpathcurveto{\pgfqpoint{0.114480in}{0.030309in}}{\pgfqpoint{0.108333in}{0.015470in}}{\pgfqpoint{0.108333in}{0.000000in}}%
\pgfpathcurveto{\pgfqpoint{0.108333in}{-0.015470in}}{\pgfqpoint{0.114480in}{-0.030309in}}{\pgfqpoint{0.125419in}{-0.041248in}}%
\pgfpathcurveto{\pgfqpoint{0.136358in}{-0.052187in}}{\pgfqpoint{0.151196in}{-0.058333in}}{\pgfqpoint{0.166667in}{-0.058333in}}%
\pgfpathclose%
\pgfpathmoveto{\pgfqpoint{0.166667in}{-0.052500in}}%
\pgfpathcurveto{\pgfqpoint{0.166667in}{-0.052500in}}{\pgfqpoint{0.152744in}{-0.052500in}}{\pgfqpoint{0.139389in}{-0.046968in}}%
\pgfpathcurveto{\pgfqpoint{0.129544in}{-0.037123in}}{\pgfqpoint{0.119698in}{-0.027278in}}{\pgfqpoint{0.114167in}{-0.013923in}}%
\pgfpathcurveto{\pgfqpoint{0.114167in}{0.000000in}}{\pgfqpoint{0.114167in}{0.013923in}}{\pgfqpoint{0.119698in}{0.027278in}}%
\pgfpathcurveto{\pgfqpoint{0.129544in}{0.037123in}}{\pgfqpoint{0.139389in}{0.046968in}}{\pgfqpoint{0.152744in}{0.052500in}}%
\pgfpathcurveto{\pgfqpoint{0.166667in}{0.052500in}}{\pgfqpoint{0.180590in}{0.052500in}}{\pgfqpoint{0.193945in}{0.046968in}}%
\pgfpathcurveto{\pgfqpoint{0.203790in}{0.037123in}}{\pgfqpoint{0.213635in}{0.027278in}}{\pgfqpoint{0.219167in}{0.013923in}}%
\pgfpathcurveto{\pgfqpoint{0.219167in}{0.000000in}}{\pgfqpoint{0.219167in}{-0.013923in}}{\pgfqpoint{0.213635in}{-0.027278in}}%
\pgfpathcurveto{\pgfqpoint{0.203790in}{-0.037123in}}{\pgfqpoint{0.193945in}{-0.046968in}}{\pgfqpoint{0.180590in}{-0.052500in}}%
\pgfpathclose%
\pgfpathmoveto{\pgfqpoint{0.333333in}{-0.058333in}}%
\pgfpathcurveto{\pgfqpoint{0.348804in}{-0.058333in}}{\pgfqpoint{0.363642in}{-0.052187in}}{\pgfqpoint{0.374581in}{-0.041248in}}%
\pgfpathcurveto{\pgfqpoint{0.385520in}{-0.030309in}}{\pgfqpoint{0.391667in}{-0.015470in}}{\pgfqpoint{0.391667in}{0.000000in}}%
\pgfpathcurveto{\pgfqpoint{0.391667in}{0.015470in}}{\pgfqpoint{0.385520in}{0.030309in}}{\pgfqpoint{0.374581in}{0.041248in}}%
\pgfpathcurveto{\pgfqpoint{0.363642in}{0.052187in}}{\pgfqpoint{0.348804in}{0.058333in}}{\pgfqpoint{0.333333in}{0.058333in}}%
\pgfpathcurveto{\pgfqpoint{0.317863in}{0.058333in}}{\pgfqpoint{0.303025in}{0.052187in}}{\pgfqpoint{0.292085in}{0.041248in}}%
\pgfpathcurveto{\pgfqpoint{0.281146in}{0.030309in}}{\pgfqpoint{0.275000in}{0.015470in}}{\pgfqpoint{0.275000in}{0.000000in}}%
\pgfpathcurveto{\pgfqpoint{0.275000in}{-0.015470in}}{\pgfqpoint{0.281146in}{-0.030309in}}{\pgfqpoint{0.292085in}{-0.041248in}}%
\pgfpathcurveto{\pgfqpoint{0.303025in}{-0.052187in}}{\pgfqpoint{0.317863in}{-0.058333in}}{\pgfqpoint{0.333333in}{-0.058333in}}%
\pgfpathclose%
\pgfpathmoveto{\pgfqpoint{0.333333in}{-0.052500in}}%
\pgfpathcurveto{\pgfqpoint{0.333333in}{-0.052500in}}{\pgfqpoint{0.319410in}{-0.052500in}}{\pgfqpoint{0.306055in}{-0.046968in}}%
\pgfpathcurveto{\pgfqpoint{0.296210in}{-0.037123in}}{\pgfqpoint{0.286365in}{-0.027278in}}{\pgfqpoint{0.280833in}{-0.013923in}}%
\pgfpathcurveto{\pgfqpoint{0.280833in}{0.000000in}}{\pgfqpoint{0.280833in}{0.013923in}}{\pgfqpoint{0.286365in}{0.027278in}}%
\pgfpathcurveto{\pgfqpoint{0.296210in}{0.037123in}}{\pgfqpoint{0.306055in}{0.046968in}}{\pgfqpoint{0.319410in}{0.052500in}}%
\pgfpathcurveto{\pgfqpoint{0.333333in}{0.052500in}}{\pgfqpoint{0.347256in}{0.052500in}}{\pgfqpoint{0.360611in}{0.046968in}}%
\pgfpathcurveto{\pgfqpoint{0.370456in}{0.037123in}}{\pgfqpoint{0.380302in}{0.027278in}}{\pgfqpoint{0.385833in}{0.013923in}}%
\pgfpathcurveto{\pgfqpoint{0.385833in}{0.000000in}}{\pgfqpoint{0.385833in}{-0.013923in}}{\pgfqpoint{0.380302in}{-0.027278in}}%
\pgfpathcurveto{\pgfqpoint{0.370456in}{-0.037123in}}{\pgfqpoint{0.360611in}{-0.046968in}}{\pgfqpoint{0.347256in}{-0.052500in}}%
\pgfpathclose%
\pgfpathmoveto{\pgfqpoint{0.500000in}{-0.058333in}}%
\pgfpathcurveto{\pgfqpoint{0.515470in}{-0.058333in}}{\pgfqpoint{0.530309in}{-0.052187in}}{\pgfqpoint{0.541248in}{-0.041248in}}%
\pgfpathcurveto{\pgfqpoint{0.552187in}{-0.030309in}}{\pgfqpoint{0.558333in}{-0.015470in}}{\pgfqpoint{0.558333in}{0.000000in}}%
\pgfpathcurveto{\pgfqpoint{0.558333in}{0.015470in}}{\pgfqpoint{0.552187in}{0.030309in}}{\pgfqpoint{0.541248in}{0.041248in}}%
\pgfpathcurveto{\pgfqpoint{0.530309in}{0.052187in}}{\pgfqpoint{0.515470in}{0.058333in}}{\pgfqpoint{0.500000in}{0.058333in}}%
\pgfpathcurveto{\pgfqpoint{0.484530in}{0.058333in}}{\pgfqpoint{0.469691in}{0.052187in}}{\pgfqpoint{0.458752in}{0.041248in}}%
\pgfpathcurveto{\pgfqpoint{0.447813in}{0.030309in}}{\pgfqpoint{0.441667in}{0.015470in}}{\pgfqpoint{0.441667in}{0.000000in}}%
\pgfpathcurveto{\pgfqpoint{0.441667in}{-0.015470in}}{\pgfqpoint{0.447813in}{-0.030309in}}{\pgfqpoint{0.458752in}{-0.041248in}}%
\pgfpathcurveto{\pgfqpoint{0.469691in}{-0.052187in}}{\pgfqpoint{0.484530in}{-0.058333in}}{\pgfqpoint{0.500000in}{-0.058333in}}%
\pgfpathclose%
\pgfpathmoveto{\pgfqpoint{0.500000in}{-0.052500in}}%
\pgfpathcurveto{\pgfqpoint{0.500000in}{-0.052500in}}{\pgfqpoint{0.486077in}{-0.052500in}}{\pgfqpoint{0.472722in}{-0.046968in}}%
\pgfpathcurveto{\pgfqpoint{0.462877in}{-0.037123in}}{\pgfqpoint{0.453032in}{-0.027278in}}{\pgfqpoint{0.447500in}{-0.013923in}}%
\pgfpathcurveto{\pgfqpoint{0.447500in}{0.000000in}}{\pgfqpoint{0.447500in}{0.013923in}}{\pgfqpoint{0.453032in}{0.027278in}}%
\pgfpathcurveto{\pgfqpoint{0.462877in}{0.037123in}}{\pgfqpoint{0.472722in}{0.046968in}}{\pgfqpoint{0.486077in}{0.052500in}}%
\pgfpathcurveto{\pgfqpoint{0.500000in}{0.052500in}}{\pgfqpoint{0.513923in}{0.052500in}}{\pgfqpoint{0.527278in}{0.046968in}}%
\pgfpathcurveto{\pgfqpoint{0.537123in}{0.037123in}}{\pgfqpoint{0.546968in}{0.027278in}}{\pgfqpoint{0.552500in}{0.013923in}}%
\pgfpathcurveto{\pgfqpoint{0.552500in}{0.000000in}}{\pgfqpoint{0.552500in}{-0.013923in}}{\pgfqpoint{0.546968in}{-0.027278in}}%
\pgfpathcurveto{\pgfqpoint{0.537123in}{-0.037123in}}{\pgfqpoint{0.527278in}{-0.046968in}}{\pgfqpoint{0.513923in}{-0.052500in}}%
\pgfpathclose%
\pgfpathmoveto{\pgfqpoint{0.666667in}{-0.058333in}}%
\pgfpathcurveto{\pgfqpoint{0.682137in}{-0.058333in}}{\pgfqpoint{0.696975in}{-0.052187in}}{\pgfqpoint{0.707915in}{-0.041248in}}%
\pgfpathcurveto{\pgfqpoint{0.718854in}{-0.030309in}}{\pgfqpoint{0.725000in}{-0.015470in}}{\pgfqpoint{0.725000in}{0.000000in}}%
\pgfpathcurveto{\pgfqpoint{0.725000in}{0.015470in}}{\pgfqpoint{0.718854in}{0.030309in}}{\pgfqpoint{0.707915in}{0.041248in}}%
\pgfpathcurveto{\pgfqpoint{0.696975in}{0.052187in}}{\pgfqpoint{0.682137in}{0.058333in}}{\pgfqpoint{0.666667in}{0.058333in}}%
\pgfpathcurveto{\pgfqpoint{0.651196in}{0.058333in}}{\pgfqpoint{0.636358in}{0.052187in}}{\pgfqpoint{0.625419in}{0.041248in}}%
\pgfpathcurveto{\pgfqpoint{0.614480in}{0.030309in}}{\pgfqpoint{0.608333in}{0.015470in}}{\pgfqpoint{0.608333in}{0.000000in}}%
\pgfpathcurveto{\pgfqpoint{0.608333in}{-0.015470in}}{\pgfqpoint{0.614480in}{-0.030309in}}{\pgfqpoint{0.625419in}{-0.041248in}}%
\pgfpathcurveto{\pgfqpoint{0.636358in}{-0.052187in}}{\pgfqpoint{0.651196in}{-0.058333in}}{\pgfqpoint{0.666667in}{-0.058333in}}%
\pgfpathclose%
\pgfpathmoveto{\pgfqpoint{0.666667in}{-0.052500in}}%
\pgfpathcurveto{\pgfqpoint{0.666667in}{-0.052500in}}{\pgfqpoint{0.652744in}{-0.052500in}}{\pgfqpoint{0.639389in}{-0.046968in}}%
\pgfpathcurveto{\pgfqpoint{0.629544in}{-0.037123in}}{\pgfqpoint{0.619698in}{-0.027278in}}{\pgfqpoint{0.614167in}{-0.013923in}}%
\pgfpathcurveto{\pgfqpoint{0.614167in}{0.000000in}}{\pgfqpoint{0.614167in}{0.013923in}}{\pgfqpoint{0.619698in}{0.027278in}}%
\pgfpathcurveto{\pgfqpoint{0.629544in}{0.037123in}}{\pgfqpoint{0.639389in}{0.046968in}}{\pgfqpoint{0.652744in}{0.052500in}}%
\pgfpathcurveto{\pgfqpoint{0.666667in}{0.052500in}}{\pgfqpoint{0.680590in}{0.052500in}}{\pgfqpoint{0.693945in}{0.046968in}}%
\pgfpathcurveto{\pgfqpoint{0.703790in}{0.037123in}}{\pgfqpoint{0.713635in}{0.027278in}}{\pgfqpoint{0.719167in}{0.013923in}}%
\pgfpathcurveto{\pgfqpoint{0.719167in}{0.000000in}}{\pgfqpoint{0.719167in}{-0.013923in}}{\pgfqpoint{0.713635in}{-0.027278in}}%
\pgfpathcurveto{\pgfqpoint{0.703790in}{-0.037123in}}{\pgfqpoint{0.693945in}{-0.046968in}}{\pgfqpoint{0.680590in}{-0.052500in}}%
\pgfpathclose%
\pgfpathmoveto{\pgfqpoint{0.833333in}{-0.058333in}}%
\pgfpathcurveto{\pgfqpoint{0.848804in}{-0.058333in}}{\pgfqpoint{0.863642in}{-0.052187in}}{\pgfqpoint{0.874581in}{-0.041248in}}%
\pgfpathcurveto{\pgfqpoint{0.885520in}{-0.030309in}}{\pgfqpoint{0.891667in}{-0.015470in}}{\pgfqpoint{0.891667in}{0.000000in}}%
\pgfpathcurveto{\pgfqpoint{0.891667in}{0.015470in}}{\pgfqpoint{0.885520in}{0.030309in}}{\pgfqpoint{0.874581in}{0.041248in}}%
\pgfpathcurveto{\pgfqpoint{0.863642in}{0.052187in}}{\pgfqpoint{0.848804in}{0.058333in}}{\pgfqpoint{0.833333in}{0.058333in}}%
\pgfpathcurveto{\pgfqpoint{0.817863in}{0.058333in}}{\pgfqpoint{0.803025in}{0.052187in}}{\pgfqpoint{0.792085in}{0.041248in}}%
\pgfpathcurveto{\pgfqpoint{0.781146in}{0.030309in}}{\pgfqpoint{0.775000in}{0.015470in}}{\pgfqpoint{0.775000in}{0.000000in}}%
\pgfpathcurveto{\pgfqpoint{0.775000in}{-0.015470in}}{\pgfqpoint{0.781146in}{-0.030309in}}{\pgfqpoint{0.792085in}{-0.041248in}}%
\pgfpathcurveto{\pgfqpoint{0.803025in}{-0.052187in}}{\pgfqpoint{0.817863in}{-0.058333in}}{\pgfqpoint{0.833333in}{-0.058333in}}%
\pgfpathclose%
\pgfpathmoveto{\pgfqpoint{0.833333in}{-0.052500in}}%
\pgfpathcurveto{\pgfqpoint{0.833333in}{-0.052500in}}{\pgfqpoint{0.819410in}{-0.052500in}}{\pgfqpoint{0.806055in}{-0.046968in}}%
\pgfpathcurveto{\pgfqpoint{0.796210in}{-0.037123in}}{\pgfqpoint{0.786365in}{-0.027278in}}{\pgfqpoint{0.780833in}{-0.013923in}}%
\pgfpathcurveto{\pgfqpoint{0.780833in}{0.000000in}}{\pgfqpoint{0.780833in}{0.013923in}}{\pgfqpoint{0.786365in}{0.027278in}}%
\pgfpathcurveto{\pgfqpoint{0.796210in}{0.037123in}}{\pgfqpoint{0.806055in}{0.046968in}}{\pgfqpoint{0.819410in}{0.052500in}}%
\pgfpathcurveto{\pgfqpoint{0.833333in}{0.052500in}}{\pgfqpoint{0.847256in}{0.052500in}}{\pgfqpoint{0.860611in}{0.046968in}}%
\pgfpathcurveto{\pgfqpoint{0.870456in}{0.037123in}}{\pgfqpoint{0.880302in}{0.027278in}}{\pgfqpoint{0.885833in}{0.013923in}}%
\pgfpathcurveto{\pgfqpoint{0.885833in}{0.000000in}}{\pgfqpoint{0.885833in}{-0.013923in}}{\pgfqpoint{0.880302in}{-0.027278in}}%
\pgfpathcurveto{\pgfqpoint{0.870456in}{-0.037123in}}{\pgfqpoint{0.860611in}{-0.046968in}}{\pgfqpoint{0.847256in}{-0.052500in}}%
\pgfpathclose%
\pgfpathmoveto{\pgfqpoint{1.000000in}{-0.058333in}}%
\pgfpathcurveto{\pgfqpoint{1.015470in}{-0.058333in}}{\pgfqpoint{1.030309in}{-0.052187in}}{\pgfqpoint{1.041248in}{-0.041248in}}%
\pgfpathcurveto{\pgfqpoint{1.052187in}{-0.030309in}}{\pgfqpoint{1.058333in}{-0.015470in}}{\pgfqpoint{1.058333in}{0.000000in}}%
\pgfpathcurveto{\pgfqpoint{1.058333in}{0.015470in}}{\pgfqpoint{1.052187in}{0.030309in}}{\pgfqpoint{1.041248in}{0.041248in}}%
\pgfpathcurveto{\pgfqpoint{1.030309in}{0.052187in}}{\pgfqpoint{1.015470in}{0.058333in}}{\pgfqpoint{1.000000in}{0.058333in}}%
\pgfpathcurveto{\pgfqpoint{0.984530in}{0.058333in}}{\pgfqpoint{0.969691in}{0.052187in}}{\pgfqpoint{0.958752in}{0.041248in}}%
\pgfpathcurveto{\pgfqpoint{0.947813in}{0.030309in}}{\pgfqpoint{0.941667in}{0.015470in}}{\pgfqpoint{0.941667in}{0.000000in}}%
\pgfpathcurveto{\pgfqpoint{0.941667in}{-0.015470in}}{\pgfqpoint{0.947813in}{-0.030309in}}{\pgfqpoint{0.958752in}{-0.041248in}}%
\pgfpathcurveto{\pgfqpoint{0.969691in}{-0.052187in}}{\pgfqpoint{0.984530in}{-0.058333in}}{\pgfqpoint{1.000000in}{-0.058333in}}%
\pgfpathclose%
\pgfpathmoveto{\pgfqpoint{1.000000in}{-0.052500in}}%
\pgfpathcurveto{\pgfqpoint{1.000000in}{-0.052500in}}{\pgfqpoint{0.986077in}{-0.052500in}}{\pgfqpoint{0.972722in}{-0.046968in}}%
\pgfpathcurveto{\pgfqpoint{0.962877in}{-0.037123in}}{\pgfqpoint{0.953032in}{-0.027278in}}{\pgfqpoint{0.947500in}{-0.013923in}}%
\pgfpathcurveto{\pgfqpoint{0.947500in}{0.000000in}}{\pgfqpoint{0.947500in}{0.013923in}}{\pgfqpoint{0.953032in}{0.027278in}}%
\pgfpathcurveto{\pgfqpoint{0.962877in}{0.037123in}}{\pgfqpoint{0.972722in}{0.046968in}}{\pgfqpoint{0.986077in}{0.052500in}}%
\pgfpathcurveto{\pgfqpoint{1.000000in}{0.052500in}}{\pgfqpoint{1.013923in}{0.052500in}}{\pgfqpoint{1.027278in}{0.046968in}}%
\pgfpathcurveto{\pgfqpoint{1.037123in}{0.037123in}}{\pgfqpoint{1.046968in}{0.027278in}}{\pgfqpoint{1.052500in}{0.013923in}}%
\pgfpathcurveto{\pgfqpoint{1.052500in}{0.000000in}}{\pgfqpoint{1.052500in}{-0.013923in}}{\pgfqpoint{1.046968in}{-0.027278in}}%
\pgfpathcurveto{\pgfqpoint{1.037123in}{-0.037123in}}{\pgfqpoint{1.027278in}{-0.046968in}}{\pgfqpoint{1.013923in}{-0.052500in}}%
\pgfpathclose%
\pgfpathmoveto{\pgfqpoint{0.083333in}{0.108333in}}%
\pgfpathcurveto{\pgfqpoint{0.098804in}{0.108333in}}{\pgfqpoint{0.113642in}{0.114480in}}{\pgfqpoint{0.124581in}{0.125419in}}%
\pgfpathcurveto{\pgfqpoint{0.135520in}{0.136358in}}{\pgfqpoint{0.141667in}{0.151196in}}{\pgfqpoint{0.141667in}{0.166667in}}%
\pgfpathcurveto{\pgfqpoint{0.141667in}{0.182137in}}{\pgfqpoint{0.135520in}{0.196975in}}{\pgfqpoint{0.124581in}{0.207915in}}%
\pgfpathcurveto{\pgfqpoint{0.113642in}{0.218854in}}{\pgfqpoint{0.098804in}{0.225000in}}{\pgfqpoint{0.083333in}{0.225000in}}%
\pgfpathcurveto{\pgfqpoint{0.067863in}{0.225000in}}{\pgfqpoint{0.053025in}{0.218854in}}{\pgfqpoint{0.042085in}{0.207915in}}%
\pgfpathcurveto{\pgfqpoint{0.031146in}{0.196975in}}{\pgfqpoint{0.025000in}{0.182137in}}{\pgfqpoint{0.025000in}{0.166667in}}%
\pgfpathcurveto{\pgfqpoint{0.025000in}{0.151196in}}{\pgfqpoint{0.031146in}{0.136358in}}{\pgfqpoint{0.042085in}{0.125419in}}%
\pgfpathcurveto{\pgfqpoint{0.053025in}{0.114480in}}{\pgfqpoint{0.067863in}{0.108333in}}{\pgfqpoint{0.083333in}{0.108333in}}%
\pgfpathclose%
\pgfpathmoveto{\pgfqpoint{0.083333in}{0.114167in}}%
\pgfpathcurveto{\pgfqpoint{0.083333in}{0.114167in}}{\pgfqpoint{0.069410in}{0.114167in}}{\pgfqpoint{0.056055in}{0.119698in}}%
\pgfpathcurveto{\pgfqpoint{0.046210in}{0.129544in}}{\pgfqpoint{0.036365in}{0.139389in}}{\pgfqpoint{0.030833in}{0.152744in}}%
\pgfpathcurveto{\pgfqpoint{0.030833in}{0.166667in}}{\pgfqpoint{0.030833in}{0.180590in}}{\pgfqpoint{0.036365in}{0.193945in}}%
\pgfpathcurveto{\pgfqpoint{0.046210in}{0.203790in}}{\pgfqpoint{0.056055in}{0.213635in}}{\pgfqpoint{0.069410in}{0.219167in}}%
\pgfpathcurveto{\pgfqpoint{0.083333in}{0.219167in}}{\pgfqpoint{0.097256in}{0.219167in}}{\pgfqpoint{0.110611in}{0.213635in}}%
\pgfpathcurveto{\pgfqpoint{0.120456in}{0.203790in}}{\pgfqpoint{0.130302in}{0.193945in}}{\pgfqpoint{0.135833in}{0.180590in}}%
\pgfpathcurveto{\pgfqpoint{0.135833in}{0.166667in}}{\pgfqpoint{0.135833in}{0.152744in}}{\pgfqpoint{0.130302in}{0.139389in}}%
\pgfpathcurveto{\pgfqpoint{0.120456in}{0.129544in}}{\pgfqpoint{0.110611in}{0.119698in}}{\pgfqpoint{0.097256in}{0.114167in}}%
\pgfpathclose%
\pgfpathmoveto{\pgfqpoint{0.250000in}{0.108333in}}%
\pgfpathcurveto{\pgfqpoint{0.265470in}{0.108333in}}{\pgfqpoint{0.280309in}{0.114480in}}{\pgfqpoint{0.291248in}{0.125419in}}%
\pgfpathcurveto{\pgfqpoint{0.302187in}{0.136358in}}{\pgfqpoint{0.308333in}{0.151196in}}{\pgfqpoint{0.308333in}{0.166667in}}%
\pgfpathcurveto{\pgfqpoint{0.308333in}{0.182137in}}{\pgfqpoint{0.302187in}{0.196975in}}{\pgfqpoint{0.291248in}{0.207915in}}%
\pgfpathcurveto{\pgfqpoint{0.280309in}{0.218854in}}{\pgfqpoint{0.265470in}{0.225000in}}{\pgfqpoint{0.250000in}{0.225000in}}%
\pgfpathcurveto{\pgfqpoint{0.234530in}{0.225000in}}{\pgfqpoint{0.219691in}{0.218854in}}{\pgfqpoint{0.208752in}{0.207915in}}%
\pgfpathcurveto{\pgfqpoint{0.197813in}{0.196975in}}{\pgfqpoint{0.191667in}{0.182137in}}{\pgfqpoint{0.191667in}{0.166667in}}%
\pgfpathcurveto{\pgfqpoint{0.191667in}{0.151196in}}{\pgfqpoint{0.197813in}{0.136358in}}{\pgfqpoint{0.208752in}{0.125419in}}%
\pgfpathcurveto{\pgfqpoint{0.219691in}{0.114480in}}{\pgfqpoint{0.234530in}{0.108333in}}{\pgfqpoint{0.250000in}{0.108333in}}%
\pgfpathclose%
\pgfpathmoveto{\pgfqpoint{0.250000in}{0.114167in}}%
\pgfpathcurveto{\pgfqpoint{0.250000in}{0.114167in}}{\pgfqpoint{0.236077in}{0.114167in}}{\pgfqpoint{0.222722in}{0.119698in}}%
\pgfpathcurveto{\pgfqpoint{0.212877in}{0.129544in}}{\pgfqpoint{0.203032in}{0.139389in}}{\pgfqpoint{0.197500in}{0.152744in}}%
\pgfpathcurveto{\pgfqpoint{0.197500in}{0.166667in}}{\pgfqpoint{0.197500in}{0.180590in}}{\pgfqpoint{0.203032in}{0.193945in}}%
\pgfpathcurveto{\pgfqpoint{0.212877in}{0.203790in}}{\pgfqpoint{0.222722in}{0.213635in}}{\pgfqpoint{0.236077in}{0.219167in}}%
\pgfpathcurveto{\pgfqpoint{0.250000in}{0.219167in}}{\pgfqpoint{0.263923in}{0.219167in}}{\pgfqpoint{0.277278in}{0.213635in}}%
\pgfpathcurveto{\pgfqpoint{0.287123in}{0.203790in}}{\pgfqpoint{0.296968in}{0.193945in}}{\pgfqpoint{0.302500in}{0.180590in}}%
\pgfpathcurveto{\pgfqpoint{0.302500in}{0.166667in}}{\pgfqpoint{0.302500in}{0.152744in}}{\pgfqpoint{0.296968in}{0.139389in}}%
\pgfpathcurveto{\pgfqpoint{0.287123in}{0.129544in}}{\pgfqpoint{0.277278in}{0.119698in}}{\pgfqpoint{0.263923in}{0.114167in}}%
\pgfpathclose%
\pgfpathmoveto{\pgfqpoint{0.416667in}{0.108333in}}%
\pgfpathcurveto{\pgfqpoint{0.432137in}{0.108333in}}{\pgfqpoint{0.446975in}{0.114480in}}{\pgfqpoint{0.457915in}{0.125419in}}%
\pgfpathcurveto{\pgfqpoint{0.468854in}{0.136358in}}{\pgfqpoint{0.475000in}{0.151196in}}{\pgfqpoint{0.475000in}{0.166667in}}%
\pgfpathcurveto{\pgfqpoint{0.475000in}{0.182137in}}{\pgfqpoint{0.468854in}{0.196975in}}{\pgfqpoint{0.457915in}{0.207915in}}%
\pgfpathcurveto{\pgfqpoint{0.446975in}{0.218854in}}{\pgfqpoint{0.432137in}{0.225000in}}{\pgfqpoint{0.416667in}{0.225000in}}%
\pgfpathcurveto{\pgfqpoint{0.401196in}{0.225000in}}{\pgfqpoint{0.386358in}{0.218854in}}{\pgfqpoint{0.375419in}{0.207915in}}%
\pgfpathcurveto{\pgfqpoint{0.364480in}{0.196975in}}{\pgfqpoint{0.358333in}{0.182137in}}{\pgfqpoint{0.358333in}{0.166667in}}%
\pgfpathcurveto{\pgfqpoint{0.358333in}{0.151196in}}{\pgfqpoint{0.364480in}{0.136358in}}{\pgfqpoint{0.375419in}{0.125419in}}%
\pgfpathcurveto{\pgfqpoint{0.386358in}{0.114480in}}{\pgfqpoint{0.401196in}{0.108333in}}{\pgfqpoint{0.416667in}{0.108333in}}%
\pgfpathclose%
\pgfpathmoveto{\pgfqpoint{0.416667in}{0.114167in}}%
\pgfpathcurveto{\pgfqpoint{0.416667in}{0.114167in}}{\pgfqpoint{0.402744in}{0.114167in}}{\pgfqpoint{0.389389in}{0.119698in}}%
\pgfpathcurveto{\pgfqpoint{0.379544in}{0.129544in}}{\pgfqpoint{0.369698in}{0.139389in}}{\pgfqpoint{0.364167in}{0.152744in}}%
\pgfpathcurveto{\pgfqpoint{0.364167in}{0.166667in}}{\pgfqpoint{0.364167in}{0.180590in}}{\pgfqpoint{0.369698in}{0.193945in}}%
\pgfpathcurveto{\pgfqpoint{0.379544in}{0.203790in}}{\pgfqpoint{0.389389in}{0.213635in}}{\pgfqpoint{0.402744in}{0.219167in}}%
\pgfpathcurveto{\pgfqpoint{0.416667in}{0.219167in}}{\pgfqpoint{0.430590in}{0.219167in}}{\pgfqpoint{0.443945in}{0.213635in}}%
\pgfpathcurveto{\pgfqpoint{0.453790in}{0.203790in}}{\pgfqpoint{0.463635in}{0.193945in}}{\pgfqpoint{0.469167in}{0.180590in}}%
\pgfpathcurveto{\pgfqpoint{0.469167in}{0.166667in}}{\pgfqpoint{0.469167in}{0.152744in}}{\pgfqpoint{0.463635in}{0.139389in}}%
\pgfpathcurveto{\pgfqpoint{0.453790in}{0.129544in}}{\pgfqpoint{0.443945in}{0.119698in}}{\pgfqpoint{0.430590in}{0.114167in}}%
\pgfpathclose%
\pgfpathmoveto{\pgfqpoint{0.583333in}{0.108333in}}%
\pgfpathcurveto{\pgfqpoint{0.598804in}{0.108333in}}{\pgfqpoint{0.613642in}{0.114480in}}{\pgfqpoint{0.624581in}{0.125419in}}%
\pgfpathcurveto{\pgfqpoint{0.635520in}{0.136358in}}{\pgfqpoint{0.641667in}{0.151196in}}{\pgfqpoint{0.641667in}{0.166667in}}%
\pgfpathcurveto{\pgfqpoint{0.641667in}{0.182137in}}{\pgfqpoint{0.635520in}{0.196975in}}{\pgfqpoint{0.624581in}{0.207915in}}%
\pgfpathcurveto{\pgfqpoint{0.613642in}{0.218854in}}{\pgfqpoint{0.598804in}{0.225000in}}{\pgfqpoint{0.583333in}{0.225000in}}%
\pgfpathcurveto{\pgfqpoint{0.567863in}{0.225000in}}{\pgfqpoint{0.553025in}{0.218854in}}{\pgfqpoint{0.542085in}{0.207915in}}%
\pgfpathcurveto{\pgfqpoint{0.531146in}{0.196975in}}{\pgfqpoint{0.525000in}{0.182137in}}{\pgfqpoint{0.525000in}{0.166667in}}%
\pgfpathcurveto{\pgfqpoint{0.525000in}{0.151196in}}{\pgfqpoint{0.531146in}{0.136358in}}{\pgfqpoint{0.542085in}{0.125419in}}%
\pgfpathcurveto{\pgfqpoint{0.553025in}{0.114480in}}{\pgfqpoint{0.567863in}{0.108333in}}{\pgfqpoint{0.583333in}{0.108333in}}%
\pgfpathclose%
\pgfpathmoveto{\pgfqpoint{0.583333in}{0.114167in}}%
\pgfpathcurveto{\pgfqpoint{0.583333in}{0.114167in}}{\pgfqpoint{0.569410in}{0.114167in}}{\pgfqpoint{0.556055in}{0.119698in}}%
\pgfpathcurveto{\pgfqpoint{0.546210in}{0.129544in}}{\pgfqpoint{0.536365in}{0.139389in}}{\pgfqpoint{0.530833in}{0.152744in}}%
\pgfpathcurveto{\pgfqpoint{0.530833in}{0.166667in}}{\pgfqpoint{0.530833in}{0.180590in}}{\pgfqpoint{0.536365in}{0.193945in}}%
\pgfpathcurveto{\pgfqpoint{0.546210in}{0.203790in}}{\pgfqpoint{0.556055in}{0.213635in}}{\pgfqpoint{0.569410in}{0.219167in}}%
\pgfpathcurveto{\pgfqpoint{0.583333in}{0.219167in}}{\pgfqpoint{0.597256in}{0.219167in}}{\pgfqpoint{0.610611in}{0.213635in}}%
\pgfpathcurveto{\pgfqpoint{0.620456in}{0.203790in}}{\pgfqpoint{0.630302in}{0.193945in}}{\pgfqpoint{0.635833in}{0.180590in}}%
\pgfpathcurveto{\pgfqpoint{0.635833in}{0.166667in}}{\pgfqpoint{0.635833in}{0.152744in}}{\pgfqpoint{0.630302in}{0.139389in}}%
\pgfpathcurveto{\pgfqpoint{0.620456in}{0.129544in}}{\pgfqpoint{0.610611in}{0.119698in}}{\pgfqpoint{0.597256in}{0.114167in}}%
\pgfpathclose%
\pgfpathmoveto{\pgfqpoint{0.750000in}{0.108333in}}%
\pgfpathcurveto{\pgfqpoint{0.765470in}{0.108333in}}{\pgfqpoint{0.780309in}{0.114480in}}{\pgfqpoint{0.791248in}{0.125419in}}%
\pgfpathcurveto{\pgfqpoint{0.802187in}{0.136358in}}{\pgfqpoint{0.808333in}{0.151196in}}{\pgfqpoint{0.808333in}{0.166667in}}%
\pgfpathcurveto{\pgfqpoint{0.808333in}{0.182137in}}{\pgfqpoint{0.802187in}{0.196975in}}{\pgfqpoint{0.791248in}{0.207915in}}%
\pgfpathcurveto{\pgfqpoint{0.780309in}{0.218854in}}{\pgfqpoint{0.765470in}{0.225000in}}{\pgfqpoint{0.750000in}{0.225000in}}%
\pgfpathcurveto{\pgfqpoint{0.734530in}{0.225000in}}{\pgfqpoint{0.719691in}{0.218854in}}{\pgfqpoint{0.708752in}{0.207915in}}%
\pgfpathcurveto{\pgfqpoint{0.697813in}{0.196975in}}{\pgfqpoint{0.691667in}{0.182137in}}{\pgfqpoint{0.691667in}{0.166667in}}%
\pgfpathcurveto{\pgfqpoint{0.691667in}{0.151196in}}{\pgfqpoint{0.697813in}{0.136358in}}{\pgfqpoint{0.708752in}{0.125419in}}%
\pgfpathcurveto{\pgfqpoint{0.719691in}{0.114480in}}{\pgfqpoint{0.734530in}{0.108333in}}{\pgfqpoint{0.750000in}{0.108333in}}%
\pgfpathclose%
\pgfpathmoveto{\pgfqpoint{0.750000in}{0.114167in}}%
\pgfpathcurveto{\pgfqpoint{0.750000in}{0.114167in}}{\pgfqpoint{0.736077in}{0.114167in}}{\pgfqpoint{0.722722in}{0.119698in}}%
\pgfpathcurveto{\pgfqpoint{0.712877in}{0.129544in}}{\pgfqpoint{0.703032in}{0.139389in}}{\pgfqpoint{0.697500in}{0.152744in}}%
\pgfpathcurveto{\pgfqpoint{0.697500in}{0.166667in}}{\pgfqpoint{0.697500in}{0.180590in}}{\pgfqpoint{0.703032in}{0.193945in}}%
\pgfpathcurveto{\pgfqpoint{0.712877in}{0.203790in}}{\pgfqpoint{0.722722in}{0.213635in}}{\pgfqpoint{0.736077in}{0.219167in}}%
\pgfpathcurveto{\pgfqpoint{0.750000in}{0.219167in}}{\pgfqpoint{0.763923in}{0.219167in}}{\pgfqpoint{0.777278in}{0.213635in}}%
\pgfpathcurveto{\pgfqpoint{0.787123in}{0.203790in}}{\pgfqpoint{0.796968in}{0.193945in}}{\pgfqpoint{0.802500in}{0.180590in}}%
\pgfpathcurveto{\pgfqpoint{0.802500in}{0.166667in}}{\pgfqpoint{0.802500in}{0.152744in}}{\pgfqpoint{0.796968in}{0.139389in}}%
\pgfpathcurveto{\pgfqpoint{0.787123in}{0.129544in}}{\pgfqpoint{0.777278in}{0.119698in}}{\pgfqpoint{0.763923in}{0.114167in}}%
\pgfpathclose%
\pgfpathmoveto{\pgfqpoint{0.916667in}{0.108333in}}%
\pgfpathcurveto{\pgfqpoint{0.932137in}{0.108333in}}{\pgfqpoint{0.946975in}{0.114480in}}{\pgfqpoint{0.957915in}{0.125419in}}%
\pgfpathcurveto{\pgfqpoint{0.968854in}{0.136358in}}{\pgfqpoint{0.975000in}{0.151196in}}{\pgfqpoint{0.975000in}{0.166667in}}%
\pgfpathcurveto{\pgfqpoint{0.975000in}{0.182137in}}{\pgfqpoint{0.968854in}{0.196975in}}{\pgfqpoint{0.957915in}{0.207915in}}%
\pgfpathcurveto{\pgfqpoint{0.946975in}{0.218854in}}{\pgfqpoint{0.932137in}{0.225000in}}{\pgfqpoint{0.916667in}{0.225000in}}%
\pgfpathcurveto{\pgfqpoint{0.901196in}{0.225000in}}{\pgfqpoint{0.886358in}{0.218854in}}{\pgfqpoint{0.875419in}{0.207915in}}%
\pgfpathcurveto{\pgfqpoint{0.864480in}{0.196975in}}{\pgfqpoint{0.858333in}{0.182137in}}{\pgfqpoint{0.858333in}{0.166667in}}%
\pgfpathcurveto{\pgfqpoint{0.858333in}{0.151196in}}{\pgfqpoint{0.864480in}{0.136358in}}{\pgfqpoint{0.875419in}{0.125419in}}%
\pgfpathcurveto{\pgfqpoint{0.886358in}{0.114480in}}{\pgfqpoint{0.901196in}{0.108333in}}{\pgfqpoint{0.916667in}{0.108333in}}%
\pgfpathclose%
\pgfpathmoveto{\pgfqpoint{0.916667in}{0.114167in}}%
\pgfpathcurveto{\pgfqpoint{0.916667in}{0.114167in}}{\pgfqpoint{0.902744in}{0.114167in}}{\pgfqpoint{0.889389in}{0.119698in}}%
\pgfpathcurveto{\pgfqpoint{0.879544in}{0.129544in}}{\pgfqpoint{0.869698in}{0.139389in}}{\pgfqpoint{0.864167in}{0.152744in}}%
\pgfpathcurveto{\pgfqpoint{0.864167in}{0.166667in}}{\pgfqpoint{0.864167in}{0.180590in}}{\pgfqpoint{0.869698in}{0.193945in}}%
\pgfpathcurveto{\pgfqpoint{0.879544in}{0.203790in}}{\pgfqpoint{0.889389in}{0.213635in}}{\pgfqpoint{0.902744in}{0.219167in}}%
\pgfpathcurveto{\pgfqpoint{0.916667in}{0.219167in}}{\pgfqpoint{0.930590in}{0.219167in}}{\pgfqpoint{0.943945in}{0.213635in}}%
\pgfpathcurveto{\pgfqpoint{0.953790in}{0.203790in}}{\pgfqpoint{0.963635in}{0.193945in}}{\pgfqpoint{0.969167in}{0.180590in}}%
\pgfpathcurveto{\pgfqpoint{0.969167in}{0.166667in}}{\pgfqpoint{0.969167in}{0.152744in}}{\pgfqpoint{0.963635in}{0.139389in}}%
\pgfpathcurveto{\pgfqpoint{0.953790in}{0.129544in}}{\pgfqpoint{0.943945in}{0.119698in}}{\pgfqpoint{0.930590in}{0.114167in}}%
\pgfpathclose%
\pgfpathmoveto{\pgfqpoint{0.000000in}{0.275000in}}%
\pgfpathcurveto{\pgfqpoint{0.015470in}{0.275000in}}{\pgfqpoint{0.030309in}{0.281146in}}{\pgfqpoint{0.041248in}{0.292085in}}%
\pgfpathcurveto{\pgfqpoint{0.052187in}{0.303025in}}{\pgfqpoint{0.058333in}{0.317863in}}{\pgfqpoint{0.058333in}{0.333333in}}%
\pgfpathcurveto{\pgfqpoint{0.058333in}{0.348804in}}{\pgfqpoint{0.052187in}{0.363642in}}{\pgfqpoint{0.041248in}{0.374581in}}%
\pgfpathcurveto{\pgfqpoint{0.030309in}{0.385520in}}{\pgfqpoint{0.015470in}{0.391667in}}{\pgfqpoint{0.000000in}{0.391667in}}%
\pgfpathcurveto{\pgfqpoint{-0.015470in}{0.391667in}}{\pgfqpoint{-0.030309in}{0.385520in}}{\pgfqpoint{-0.041248in}{0.374581in}}%
\pgfpathcurveto{\pgfqpoint{-0.052187in}{0.363642in}}{\pgfqpoint{-0.058333in}{0.348804in}}{\pgfqpoint{-0.058333in}{0.333333in}}%
\pgfpathcurveto{\pgfqpoint{-0.058333in}{0.317863in}}{\pgfqpoint{-0.052187in}{0.303025in}}{\pgfqpoint{-0.041248in}{0.292085in}}%
\pgfpathcurveto{\pgfqpoint{-0.030309in}{0.281146in}}{\pgfqpoint{-0.015470in}{0.275000in}}{\pgfqpoint{0.000000in}{0.275000in}}%
\pgfpathclose%
\pgfpathmoveto{\pgfqpoint{0.000000in}{0.280833in}}%
\pgfpathcurveto{\pgfqpoint{0.000000in}{0.280833in}}{\pgfqpoint{-0.013923in}{0.280833in}}{\pgfqpoint{-0.027278in}{0.286365in}}%
\pgfpathcurveto{\pgfqpoint{-0.037123in}{0.296210in}}{\pgfqpoint{-0.046968in}{0.306055in}}{\pgfqpoint{-0.052500in}{0.319410in}}%
\pgfpathcurveto{\pgfqpoint{-0.052500in}{0.333333in}}{\pgfqpoint{-0.052500in}{0.347256in}}{\pgfqpoint{-0.046968in}{0.360611in}}%
\pgfpathcurveto{\pgfqpoint{-0.037123in}{0.370456in}}{\pgfqpoint{-0.027278in}{0.380302in}}{\pgfqpoint{-0.013923in}{0.385833in}}%
\pgfpathcurveto{\pgfqpoint{0.000000in}{0.385833in}}{\pgfqpoint{0.013923in}{0.385833in}}{\pgfqpoint{0.027278in}{0.380302in}}%
\pgfpathcurveto{\pgfqpoint{0.037123in}{0.370456in}}{\pgfqpoint{0.046968in}{0.360611in}}{\pgfqpoint{0.052500in}{0.347256in}}%
\pgfpathcurveto{\pgfqpoint{0.052500in}{0.333333in}}{\pgfqpoint{0.052500in}{0.319410in}}{\pgfqpoint{0.046968in}{0.306055in}}%
\pgfpathcurveto{\pgfqpoint{0.037123in}{0.296210in}}{\pgfqpoint{0.027278in}{0.286365in}}{\pgfqpoint{0.013923in}{0.280833in}}%
\pgfpathclose%
\pgfpathmoveto{\pgfqpoint{0.166667in}{0.275000in}}%
\pgfpathcurveto{\pgfqpoint{0.182137in}{0.275000in}}{\pgfqpoint{0.196975in}{0.281146in}}{\pgfqpoint{0.207915in}{0.292085in}}%
\pgfpathcurveto{\pgfqpoint{0.218854in}{0.303025in}}{\pgfqpoint{0.225000in}{0.317863in}}{\pgfqpoint{0.225000in}{0.333333in}}%
\pgfpathcurveto{\pgfqpoint{0.225000in}{0.348804in}}{\pgfqpoint{0.218854in}{0.363642in}}{\pgfqpoint{0.207915in}{0.374581in}}%
\pgfpathcurveto{\pgfqpoint{0.196975in}{0.385520in}}{\pgfqpoint{0.182137in}{0.391667in}}{\pgfqpoint{0.166667in}{0.391667in}}%
\pgfpathcurveto{\pgfqpoint{0.151196in}{0.391667in}}{\pgfqpoint{0.136358in}{0.385520in}}{\pgfqpoint{0.125419in}{0.374581in}}%
\pgfpathcurveto{\pgfqpoint{0.114480in}{0.363642in}}{\pgfqpoint{0.108333in}{0.348804in}}{\pgfqpoint{0.108333in}{0.333333in}}%
\pgfpathcurveto{\pgfqpoint{0.108333in}{0.317863in}}{\pgfqpoint{0.114480in}{0.303025in}}{\pgfqpoint{0.125419in}{0.292085in}}%
\pgfpathcurveto{\pgfqpoint{0.136358in}{0.281146in}}{\pgfqpoint{0.151196in}{0.275000in}}{\pgfqpoint{0.166667in}{0.275000in}}%
\pgfpathclose%
\pgfpathmoveto{\pgfqpoint{0.166667in}{0.280833in}}%
\pgfpathcurveto{\pgfqpoint{0.166667in}{0.280833in}}{\pgfqpoint{0.152744in}{0.280833in}}{\pgfqpoint{0.139389in}{0.286365in}}%
\pgfpathcurveto{\pgfqpoint{0.129544in}{0.296210in}}{\pgfqpoint{0.119698in}{0.306055in}}{\pgfqpoint{0.114167in}{0.319410in}}%
\pgfpathcurveto{\pgfqpoint{0.114167in}{0.333333in}}{\pgfqpoint{0.114167in}{0.347256in}}{\pgfqpoint{0.119698in}{0.360611in}}%
\pgfpathcurveto{\pgfqpoint{0.129544in}{0.370456in}}{\pgfqpoint{0.139389in}{0.380302in}}{\pgfqpoint{0.152744in}{0.385833in}}%
\pgfpathcurveto{\pgfqpoint{0.166667in}{0.385833in}}{\pgfqpoint{0.180590in}{0.385833in}}{\pgfqpoint{0.193945in}{0.380302in}}%
\pgfpathcurveto{\pgfqpoint{0.203790in}{0.370456in}}{\pgfqpoint{0.213635in}{0.360611in}}{\pgfqpoint{0.219167in}{0.347256in}}%
\pgfpathcurveto{\pgfqpoint{0.219167in}{0.333333in}}{\pgfqpoint{0.219167in}{0.319410in}}{\pgfqpoint{0.213635in}{0.306055in}}%
\pgfpathcurveto{\pgfqpoint{0.203790in}{0.296210in}}{\pgfqpoint{0.193945in}{0.286365in}}{\pgfqpoint{0.180590in}{0.280833in}}%
\pgfpathclose%
\pgfpathmoveto{\pgfqpoint{0.333333in}{0.275000in}}%
\pgfpathcurveto{\pgfqpoint{0.348804in}{0.275000in}}{\pgfqpoint{0.363642in}{0.281146in}}{\pgfqpoint{0.374581in}{0.292085in}}%
\pgfpathcurveto{\pgfqpoint{0.385520in}{0.303025in}}{\pgfqpoint{0.391667in}{0.317863in}}{\pgfqpoint{0.391667in}{0.333333in}}%
\pgfpathcurveto{\pgfqpoint{0.391667in}{0.348804in}}{\pgfqpoint{0.385520in}{0.363642in}}{\pgfqpoint{0.374581in}{0.374581in}}%
\pgfpathcurveto{\pgfqpoint{0.363642in}{0.385520in}}{\pgfqpoint{0.348804in}{0.391667in}}{\pgfqpoint{0.333333in}{0.391667in}}%
\pgfpathcurveto{\pgfqpoint{0.317863in}{0.391667in}}{\pgfqpoint{0.303025in}{0.385520in}}{\pgfqpoint{0.292085in}{0.374581in}}%
\pgfpathcurveto{\pgfqpoint{0.281146in}{0.363642in}}{\pgfqpoint{0.275000in}{0.348804in}}{\pgfqpoint{0.275000in}{0.333333in}}%
\pgfpathcurveto{\pgfqpoint{0.275000in}{0.317863in}}{\pgfqpoint{0.281146in}{0.303025in}}{\pgfqpoint{0.292085in}{0.292085in}}%
\pgfpathcurveto{\pgfqpoint{0.303025in}{0.281146in}}{\pgfqpoint{0.317863in}{0.275000in}}{\pgfqpoint{0.333333in}{0.275000in}}%
\pgfpathclose%
\pgfpathmoveto{\pgfqpoint{0.333333in}{0.280833in}}%
\pgfpathcurveto{\pgfqpoint{0.333333in}{0.280833in}}{\pgfqpoint{0.319410in}{0.280833in}}{\pgfqpoint{0.306055in}{0.286365in}}%
\pgfpathcurveto{\pgfqpoint{0.296210in}{0.296210in}}{\pgfqpoint{0.286365in}{0.306055in}}{\pgfqpoint{0.280833in}{0.319410in}}%
\pgfpathcurveto{\pgfqpoint{0.280833in}{0.333333in}}{\pgfqpoint{0.280833in}{0.347256in}}{\pgfqpoint{0.286365in}{0.360611in}}%
\pgfpathcurveto{\pgfqpoint{0.296210in}{0.370456in}}{\pgfqpoint{0.306055in}{0.380302in}}{\pgfqpoint{0.319410in}{0.385833in}}%
\pgfpathcurveto{\pgfqpoint{0.333333in}{0.385833in}}{\pgfqpoint{0.347256in}{0.385833in}}{\pgfqpoint{0.360611in}{0.380302in}}%
\pgfpathcurveto{\pgfqpoint{0.370456in}{0.370456in}}{\pgfqpoint{0.380302in}{0.360611in}}{\pgfqpoint{0.385833in}{0.347256in}}%
\pgfpathcurveto{\pgfqpoint{0.385833in}{0.333333in}}{\pgfqpoint{0.385833in}{0.319410in}}{\pgfqpoint{0.380302in}{0.306055in}}%
\pgfpathcurveto{\pgfqpoint{0.370456in}{0.296210in}}{\pgfqpoint{0.360611in}{0.286365in}}{\pgfqpoint{0.347256in}{0.280833in}}%
\pgfpathclose%
\pgfpathmoveto{\pgfqpoint{0.500000in}{0.275000in}}%
\pgfpathcurveto{\pgfqpoint{0.515470in}{0.275000in}}{\pgfqpoint{0.530309in}{0.281146in}}{\pgfqpoint{0.541248in}{0.292085in}}%
\pgfpathcurveto{\pgfqpoint{0.552187in}{0.303025in}}{\pgfqpoint{0.558333in}{0.317863in}}{\pgfqpoint{0.558333in}{0.333333in}}%
\pgfpathcurveto{\pgfqpoint{0.558333in}{0.348804in}}{\pgfqpoint{0.552187in}{0.363642in}}{\pgfqpoint{0.541248in}{0.374581in}}%
\pgfpathcurveto{\pgfqpoint{0.530309in}{0.385520in}}{\pgfqpoint{0.515470in}{0.391667in}}{\pgfqpoint{0.500000in}{0.391667in}}%
\pgfpathcurveto{\pgfqpoint{0.484530in}{0.391667in}}{\pgfqpoint{0.469691in}{0.385520in}}{\pgfqpoint{0.458752in}{0.374581in}}%
\pgfpathcurveto{\pgfqpoint{0.447813in}{0.363642in}}{\pgfqpoint{0.441667in}{0.348804in}}{\pgfqpoint{0.441667in}{0.333333in}}%
\pgfpathcurveto{\pgfqpoint{0.441667in}{0.317863in}}{\pgfqpoint{0.447813in}{0.303025in}}{\pgfqpoint{0.458752in}{0.292085in}}%
\pgfpathcurveto{\pgfqpoint{0.469691in}{0.281146in}}{\pgfqpoint{0.484530in}{0.275000in}}{\pgfqpoint{0.500000in}{0.275000in}}%
\pgfpathclose%
\pgfpathmoveto{\pgfqpoint{0.500000in}{0.280833in}}%
\pgfpathcurveto{\pgfqpoint{0.500000in}{0.280833in}}{\pgfqpoint{0.486077in}{0.280833in}}{\pgfqpoint{0.472722in}{0.286365in}}%
\pgfpathcurveto{\pgfqpoint{0.462877in}{0.296210in}}{\pgfqpoint{0.453032in}{0.306055in}}{\pgfqpoint{0.447500in}{0.319410in}}%
\pgfpathcurveto{\pgfqpoint{0.447500in}{0.333333in}}{\pgfqpoint{0.447500in}{0.347256in}}{\pgfqpoint{0.453032in}{0.360611in}}%
\pgfpathcurveto{\pgfqpoint{0.462877in}{0.370456in}}{\pgfqpoint{0.472722in}{0.380302in}}{\pgfqpoint{0.486077in}{0.385833in}}%
\pgfpathcurveto{\pgfqpoint{0.500000in}{0.385833in}}{\pgfqpoint{0.513923in}{0.385833in}}{\pgfqpoint{0.527278in}{0.380302in}}%
\pgfpathcurveto{\pgfqpoint{0.537123in}{0.370456in}}{\pgfqpoint{0.546968in}{0.360611in}}{\pgfqpoint{0.552500in}{0.347256in}}%
\pgfpathcurveto{\pgfqpoint{0.552500in}{0.333333in}}{\pgfqpoint{0.552500in}{0.319410in}}{\pgfqpoint{0.546968in}{0.306055in}}%
\pgfpathcurveto{\pgfqpoint{0.537123in}{0.296210in}}{\pgfqpoint{0.527278in}{0.286365in}}{\pgfqpoint{0.513923in}{0.280833in}}%
\pgfpathclose%
\pgfpathmoveto{\pgfqpoint{0.666667in}{0.275000in}}%
\pgfpathcurveto{\pgfqpoint{0.682137in}{0.275000in}}{\pgfqpoint{0.696975in}{0.281146in}}{\pgfqpoint{0.707915in}{0.292085in}}%
\pgfpathcurveto{\pgfqpoint{0.718854in}{0.303025in}}{\pgfqpoint{0.725000in}{0.317863in}}{\pgfqpoint{0.725000in}{0.333333in}}%
\pgfpathcurveto{\pgfqpoint{0.725000in}{0.348804in}}{\pgfqpoint{0.718854in}{0.363642in}}{\pgfqpoint{0.707915in}{0.374581in}}%
\pgfpathcurveto{\pgfqpoint{0.696975in}{0.385520in}}{\pgfqpoint{0.682137in}{0.391667in}}{\pgfqpoint{0.666667in}{0.391667in}}%
\pgfpathcurveto{\pgfqpoint{0.651196in}{0.391667in}}{\pgfqpoint{0.636358in}{0.385520in}}{\pgfqpoint{0.625419in}{0.374581in}}%
\pgfpathcurveto{\pgfqpoint{0.614480in}{0.363642in}}{\pgfqpoint{0.608333in}{0.348804in}}{\pgfqpoint{0.608333in}{0.333333in}}%
\pgfpathcurveto{\pgfqpoint{0.608333in}{0.317863in}}{\pgfqpoint{0.614480in}{0.303025in}}{\pgfqpoint{0.625419in}{0.292085in}}%
\pgfpathcurveto{\pgfqpoint{0.636358in}{0.281146in}}{\pgfqpoint{0.651196in}{0.275000in}}{\pgfqpoint{0.666667in}{0.275000in}}%
\pgfpathclose%
\pgfpathmoveto{\pgfqpoint{0.666667in}{0.280833in}}%
\pgfpathcurveto{\pgfqpoint{0.666667in}{0.280833in}}{\pgfqpoint{0.652744in}{0.280833in}}{\pgfqpoint{0.639389in}{0.286365in}}%
\pgfpathcurveto{\pgfqpoint{0.629544in}{0.296210in}}{\pgfqpoint{0.619698in}{0.306055in}}{\pgfqpoint{0.614167in}{0.319410in}}%
\pgfpathcurveto{\pgfqpoint{0.614167in}{0.333333in}}{\pgfqpoint{0.614167in}{0.347256in}}{\pgfqpoint{0.619698in}{0.360611in}}%
\pgfpathcurveto{\pgfqpoint{0.629544in}{0.370456in}}{\pgfqpoint{0.639389in}{0.380302in}}{\pgfqpoint{0.652744in}{0.385833in}}%
\pgfpathcurveto{\pgfqpoint{0.666667in}{0.385833in}}{\pgfqpoint{0.680590in}{0.385833in}}{\pgfqpoint{0.693945in}{0.380302in}}%
\pgfpathcurveto{\pgfqpoint{0.703790in}{0.370456in}}{\pgfqpoint{0.713635in}{0.360611in}}{\pgfqpoint{0.719167in}{0.347256in}}%
\pgfpathcurveto{\pgfqpoint{0.719167in}{0.333333in}}{\pgfqpoint{0.719167in}{0.319410in}}{\pgfqpoint{0.713635in}{0.306055in}}%
\pgfpathcurveto{\pgfqpoint{0.703790in}{0.296210in}}{\pgfqpoint{0.693945in}{0.286365in}}{\pgfqpoint{0.680590in}{0.280833in}}%
\pgfpathclose%
\pgfpathmoveto{\pgfqpoint{0.833333in}{0.275000in}}%
\pgfpathcurveto{\pgfqpoint{0.848804in}{0.275000in}}{\pgfqpoint{0.863642in}{0.281146in}}{\pgfqpoint{0.874581in}{0.292085in}}%
\pgfpathcurveto{\pgfqpoint{0.885520in}{0.303025in}}{\pgfqpoint{0.891667in}{0.317863in}}{\pgfqpoint{0.891667in}{0.333333in}}%
\pgfpathcurveto{\pgfqpoint{0.891667in}{0.348804in}}{\pgfqpoint{0.885520in}{0.363642in}}{\pgfqpoint{0.874581in}{0.374581in}}%
\pgfpathcurveto{\pgfqpoint{0.863642in}{0.385520in}}{\pgfqpoint{0.848804in}{0.391667in}}{\pgfqpoint{0.833333in}{0.391667in}}%
\pgfpathcurveto{\pgfqpoint{0.817863in}{0.391667in}}{\pgfqpoint{0.803025in}{0.385520in}}{\pgfqpoint{0.792085in}{0.374581in}}%
\pgfpathcurveto{\pgfqpoint{0.781146in}{0.363642in}}{\pgfqpoint{0.775000in}{0.348804in}}{\pgfqpoint{0.775000in}{0.333333in}}%
\pgfpathcurveto{\pgfqpoint{0.775000in}{0.317863in}}{\pgfqpoint{0.781146in}{0.303025in}}{\pgfqpoint{0.792085in}{0.292085in}}%
\pgfpathcurveto{\pgfqpoint{0.803025in}{0.281146in}}{\pgfqpoint{0.817863in}{0.275000in}}{\pgfqpoint{0.833333in}{0.275000in}}%
\pgfpathclose%
\pgfpathmoveto{\pgfqpoint{0.833333in}{0.280833in}}%
\pgfpathcurveto{\pgfqpoint{0.833333in}{0.280833in}}{\pgfqpoint{0.819410in}{0.280833in}}{\pgfqpoint{0.806055in}{0.286365in}}%
\pgfpathcurveto{\pgfqpoint{0.796210in}{0.296210in}}{\pgfqpoint{0.786365in}{0.306055in}}{\pgfqpoint{0.780833in}{0.319410in}}%
\pgfpathcurveto{\pgfqpoint{0.780833in}{0.333333in}}{\pgfqpoint{0.780833in}{0.347256in}}{\pgfqpoint{0.786365in}{0.360611in}}%
\pgfpathcurveto{\pgfqpoint{0.796210in}{0.370456in}}{\pgfqpoint{0.806055in}{0.380302in}}{\pgfqpoint{0.819410in}{0.385833in}}%
\pgfpathcurveto{\pgfqpoint{0.833333in}{0.385833in}}{\pgfqpoint{0.847256in}{0.385833in}}{\pgfqpoint{0.860611in}{0.380302in}}%
\pgfpathcurveto{\pgfqpoint{0.870456in}{0.370456in}}{\pgfqpoint{0.880302in}{0.360611in}}{\pgfqpoint{0.885833in}{0.347256in}}%
\pgfpathcurveto{\pgfqpoint{0.885833in}{0.333333in}}{\pgfqpoint{0.885833in}{0.319410in}}{\pgfqpoint{0.880302in}{0.306055in}}%
\pgfpathcurveto{\pgfqpoint{0.870456in}{0.296210in}}{\pgfqpoint{0.860611in}{0.286365in}}{\pgfqpoint{0.847256in}{0.280833in}}%
\pgfpathclose%
\pgfpathmoveto{\pgfqpoint{1.000000in}{0.275000in}}%
\pgfpathcurveto{\pgfqpoint{1.015470in}{0.275000in}}{\pgfqpoint{1.030309in}{0.281146in}}{\pgfqpoint{1.041248in}{0.292085in}}%
\pgfpathcurveto{\pgfqpoint{1.052187in}{0.303025in}}{\pgfqpoint{1.058333in}{0.317863in}}{\pgfqpoint{1.058333in}{0.333333in}}%
\pgfpathcurveto{\pgfqpoint{1.058333in}{0.348804in}}{\pgfqpoint{1.052187in}{0.363642in}}{\pgfqpoint{1.041248in}{0.374581in}}%
\pgfpathcurveto{\pgfqpoint{1.030309in}{0.385520in}}{\pgfqpoint{1.015470in}{0.391667in}}{\pgfqpoint{1.000000in}{0.391667in}}%
\pgfpathcurveto{\pgfqpoint{0.984530in}{0.391667in}}{\pgfqpoint{0.969691in}{0.385520in}}{\pgfqpoint{0.958752in}{0.374581in}}%
\pgfpathcurveto{\pgfqpoint{0.947813in}{0.363642in}}{\pgfqpoint{0.941667in}{0.348804in}}{\pgfqpoint{0.941667in}{0.333333in}}%
\pgfpathcurveto{\pgfqpoint{0.941667in}{0.317863in}}{\pgfqpoint{0.947813in}{0.303025in}}{\pgfqpoint{0.958752in}{0.292085in}}%
\pgfpathcurveto{\pgfqpoint{0.969691in}{0.281146in}}{\pgfqpoint{0.984530in}{0.275000in}}{\pgfqpoint{1.000000in}{0.275000in}}%
\pgfpathclose%
\pgfpathmoveto{\pgfqpoint{1.000000in}{0.280833in}}%
\pgfpathcurveto{\pgfqpoint{1.000000in}{0.280833in}}{\pgfqpoint{0.986077in}{0.280833in}}{\pgfqpoint{0.972722in}{0.286365in}}%
\pgfpathcurveto{\pgfqpoint{0.962877in}{0.296210in}}{\pgfqpoint{0.953032in}{0.306055in}}{\pgfqpoint{0.947500in}{0.319410in}}%
\pgfpathcurveto{\pgfqpoint{0.947500in}{0.333333in}}{\pgfqpoint{0.947500in}{0.347256in}}{\pgfqpoint{0.953032in}{0.360611in}}%
\pgfpathcurveto{\pgfqpoint{0.962877in}{0.370456in}}{\pgfqpoint{0.972722in}{0.380302in}}{\pgfqpoint{0.986077in}{0.385833in}}%
\pgfpathcurveto{\pgfqpoint{1.000000in}{0.385833in}}{\pgfqpoint{1.013923in}{0.385833in}}{\pgfqpoint{1.027278in}{0.380302in}}%
\pgfpathcurveto{\pgfqpoint{1.037123in}{0.370456in}}{\pgfqpoint{1.046968in}{0.360611in}}{\pgfqpoint{1.052500in}{0.347256in}}%
\pgfpathcurveto{\pgfqpoint{1.052500in}{0.333333in}}{\pgfqpoint{1.052500in}{0.319410in}}{\pgfqpoint{1.046968in}{0.306055in}}%
\pgfpathcurveto{\pgfqpoint{1.037123in}{0.296210in}}{\pgfqpoint{1.027278in}{0.286365in}}{\pgfqpoint{1.013923in}{0.280833in}}%
\pgfpathclose%
\pgfpathmoveto{\pgfqpoint{0.083333in}{0.441667in}}%
\pgfpathcurveto{\pgfqpoint{0.098804in}{0.441667in}}{\pgfqpoint{0.113642in}{0.447813in}}{\pgfqpoint{0.124581in}{0.458752in}}%
\pgfpathcurveto{\pgfqpoint{0.135520in}{0.469691in}}{\pgfqpoint{0.141667in}{0.484530in}}{\pgfqpoint{0.141667in}{0.500000in}}%
\pgfpathcurveto{\pgfqpoint{0.141667in}{0.515470in}}{\pgfqpoint{0.135520in}{0.530309in}}{\pgfqpoint{0.124581in}{0.541248in}}%
\pgfpathcurveto{\pgfqpoint{0.113642in}{0.552187in}}{\pgfqpoint{0.098804in}{0.558333in}}{\pgfqpoint{0.083333in}{0.558333in}}%
\pgfpathcurveto{\pgfqpoint{0.067863in}{0.558333in}}{\pgfqpoint{0.053025in}{0.552187in}}{\pgfqpoint{0.042085in}{0.541248in}}%
\pgfpathcurveto{\pgfqpoint{0.031146in}{0.530309in}}{\pgfqpoint{0.025000in}{0.515470in}}{\pgfqpoint{0.025000in}{0.500000in}}%
\pgfpathcurveto{\pgfqpoint{0.025000in}{0.484530in}}{\pgfqpoint{0.031146in}{0.469691in}}{\pgfqpoint{0.042085in}{0.458752in}}%
\pgfpathcurveto{\pgfqpoint{0.053025in}{0.447813in}}{\pgfqpoint{0.067863in}{0.441667in}}{\pgfqpoint{0.083333in}{0.441667in}}%
\pgfpathclose%
\pgfpathmoveto{\pgfqpoint{0.083333in}{0.447500in}}%
\pgfpathcurveto{\pgfqpoint{0.083333in}{0.447500in}}{\pgfqpoint{0.069410in}{0.447500in}}{\pgfqpoint{0.056055in}{0.453032in}}%
\pgfpathcurveto{\pgfqpoint{0.046210in}{0.462877in}}{\pgfqpoint{0.036365in}{0.472722in}}{\pgfqpoint{0.030833in}{0.486077in}}%
\pgfpathcurveto{\pgfqpoint{0.030833in}{0.500000in}}{\pgfqpoint{0.030833in}{0.513923in}}{\pgfqpoint{0.036365in}{0.527278in}}%
\pgfpathcurveto{\pgfqpoint{0.046210in}{0.537123in}}{\pgfqpoint{0.056055in}{0.546968in}}{\pgfqpoint{0.069410in}{0.552500in}}%
\pgfpathcurveto{\pgfqpoint{0.083333in}{0.552500in}}{\pgfqpoint{0.097256in}{0.552500in}}{\pgfqpoint{0.110611in}{0.546968in}}%
\pgfpathcurveto{\pgfqpoint{0.120456in}{0.537123in}}{\pgfqpoint{0.130302in}{0.527278in}}{\pgfqpoint{0.135833in}{0.513923in}}%
\pgfpathcurveto{\pgfqpoint{0.135833in}{0.500000in}}{\pgfqpoint{0.135833in}{0.486077in}}{\pgfqpoint{0.130302in}{0.472722in}}%
\pgfpathcurveto{\pgfqpoint{0.120456in}{0.462877in}}{\pgfqpoint{0.110611in}{0.453032in}}{\pgfqpoint{0.097256in}{0.447500in}}%
\pgfpathclose%
\pgfpathmoveto{\pgfqpoint{0.250000in}{0.441667in}}%
\pgfpathcurveto{\pgfqpoint{0.265470in}{0.441667in}}{\pgfqpoint{0.280309in}{0.447813in}}{\pgfqpoint{0.291248in}{0.458752in}}%
\pgfpathcurveto{\pgfqpoint{0.302187in}{0.469691in}}{\pgfqpoint{0.308333in}{0.484530in}}{\pgfqpoint{0.308333in}{0.500000in}}%
\pgfpathcurveto{\pgfqpoint{0.308333in}{0.515470in}}{\pgfqpoint{0.302187in}{0.530309in}}{\pgfqpoint{0.291248in}{0.541248in}}%
\pgfpathcurveto{\pgfqpoint{0.280309in}{0.552187in}}{\pgfqpoint{0.265470in}{0.558333in}}{\pgfqpoint{0.250000in}{0.558333in}}%
\pgfpathcurveto{\pgfqpoint{0.234530in}{0.558333in}}{\pgfqpoint{0.219691in}{0.552187in}}{\pgfqpoint{0.208752in}{0.541248in}}%
\pgfpathcurveto{\pgfqpoint{0.197813in}{0.530309in}}{\pgfqpoint{0.191667in}{0.515470in}}{\pgfqpoint{0.191667in}{0.500000in}}%
\pgfpathcurveto{\pgfqpoint{0.191667in}{0.484530in}}{\pgfqpoint{0.197813in}{0.469691in}}{\pgfqpoint{0.208752in}{0.458752in}}%
\pgfpathcurveto{\pgfqpoint{0.219691in}{0.447813in}}{\pgfqpoint{0.234530in}{0.441667in}}{\pgfqpoint{0.250000in}{0.441667in}}%
\pgfpathclose%
\pgfpathmoveto{\pgfqpoint{0.250000in}{0.447500in}}%
\pgfpathcurveto{\pgfqpoint{0.250000in}{0.447500in}}{\pgfqpoint{0.236077in}{0.447500in}}{\pgfqpoint{0.222722in}{0.453032in}}%
\pgfpathcurveto{\pgfqpoint{0.212877in}{0.462877in}}{\pgfqpoint{0.203032in}{0.472722in}}{\pgfqpoint{0.197500in}{0.486077in}}%
\pgfpathcurveto{\pgfqpoint{0.197500in}{0.500000in}}{\pgfqpoint{0.197500in}{0.513923in}}{\pgfqpoint{0.203032in}{0.527278in}}%
\pgfpathcurveto{\pgfqpoint{0.212877in}{0.537123in}}{\pgfqpoint{0.222722in}{0.546968in}}{\pgfqpoint{0.236077in}{0.552500in}}%
\pgfpathcurveto{\pgfqpoint{0.250000in}{0.552500in}}{\pgfqpoint{0.263923in}{0.552500in}}{\pgfqpoint{0.277278in}{0.546968in}}%
\pgfpathcurveto{\pgfqpoint{0.287123in}{0.537123in}}{\pgfqpoint{0.296968in}{0.527278in}}{\pgfqpoint{0.302500in}{0.513923in}}%
\pgfpathcurveto{\pgfqpoint{0.302500in}{0.500000in}}{\pgfqpoint{0.302500in}{0.486077in}}{\pgfqpoint{0.296968in}{0.472722in}}%
\pgfpathcurveto{\pgfqpoint{0.287123in}{0.462877in}}{\pgfqpoint{0.277278in}{0.453032in}}{\pgfqpoint{0.263923in}{0.447500in}}%
\pgfpathclose%
\pgfpathmoveto{\pgfqpoint{0.416667in}{0.441667in}}%
\pgfpathcurveto{\pgfqpoint{0.432137in}{0.441667in}}{\pgfqpoint{0.446975in}{0.447813in}}{\pgfqpoint{0.457915in}{0.458752in}}%
\pgfpathcurveto{\pgfqpoint{0.468854in}{0.469691in}}{\pgfqpoint{0.475000in}{0.484530in}}{\pgfqpoint{0.475000in}{0.500000in}}%
\pgfpathcurveto{\pgfqpoint{0.475000in}{0.515470in}}{\pgfqpoint{0.468854in}{0.530309in}}{\pgfqpoint{0.457915in}{0.541248in}}%
\pgfpathcurveto{\pgfqpoint{0.446975in}{0.552187in}}{\pgfqpoint{0.432137in}{0.558333in}}{\pgfqpoint{0.416667in}{0.558333in}}%
\pgfpathcurveto{\pgfqpoint{0.401196in}{0.558333in}}{\pgfqpoint{0.386358in}{0.552187in}}{\pgfqpoint{0.375419in}{0.541248in}}%
\pgfpathcurveto{\pgfqpoint{0.364480in}{0.530309in}}{\pgfqpoint{0.358333in}{0.515470in}}{\pgfqpoint{0.358333in}{0.500000in}}%
\pgfpathcurveto{\pgfqpoint{0.358333in}{0.484530in}}{\pgfqpoint{0.364480in}{0.469691in}}{\pgfqpoint{0.375419in}{0.458752in}}%
\pgfpathcurveto{\pgfqpoint{0.386358in}{0.447813in}}{\pgfqpoint{0.401196in}{0.441667in}}{\pgfqpoint{0.416667in}{0.441667in}}%
\pgfpathclose%
\pgfpathmoveto{\pgfqpoint{0.416667in}{0.447500in}}%
\pgfpathcurveto{\pgfqpoint{0.416667in}{0.447500in}}{\pgfqpoint{0.402744in}{0.447500in}}{\pgfqpoint{0.389389in}{0.453032in}}%
\pgfpathcurveto{\pgfqpoint{0.379544in}{0.462877in}}{\pgfqpoint{0.369698in}{0.472722in}}{\pgfqpoint{0.364167in}{0.486077in}}%
\pgfpathcurveto{\pgfqpoint{0.364167in}{0.500000in}}{\pgfqpoint{0.364167in}{0.513923in}}{\pgfqpoint{0.369698in}{0.527278in}}%
\pgfpathcurveto{\pgfqpoint{0.379544in}{0.537123in}}{\pgfqpoint{0.389389in}{0.546968in}}{\pgfqpoint{0.402744in}{0.552500in}}%
\pgfpathcurveto{\pgfqpoint{0.416667in}{0.552500in}}{\pgfqpoint{0.430590in}{0.552500in}}{\pgfqpoint{0.443945in}{0.546968in}}%
\pgfpathcurveto{\pgfqpoint{0.453790in}{0.537123in}}{\pgfqpoint{0.463635in}{0.527278in}}{\pgfqpoint{0.469167in}{0.513923in}}%
\pgfpathcurveto{\pgfqpoint{0.469167in}{0.500000in}}{\pgfqpoint{0.469167in}{0.486077in}}{\pgfqpoint{0.463635in}{0.472722in}}%
\pgfpathcurveto{\pgfqpoint{0.453790in}{0.462877in}}{\pgfqpoint{0.443945in}{0.453032in}}{\pgfqpoint{0.430590in}{0.447500in}}%
\pgfpathclose%
\pgfpathmoveto{\pgfqpoint{0.583333in}{0.441667in}}%
\pgfpathcurveto{\pgfqpoint{0.598804in}{0.441667in}}{\pgfqpoint{0.613642in}{0.447813in}}{\pgfqpoint{0.624581in}{0.458752in}}%
\pgfpathcurveto{\pgfqpoint{0.635520in}{0.469691in}}{\pgfqpoint{0.641667in}{0.484530in}}{\pgfqpoint{0.641667in}{0.500000in}}%
\pgfpathcurveto{\pgfqpoint{0.641667in}{0.515470in}}{\pgfqpoint{0.635520in}{0.530309in}}{\pgfqpoint{0.624581in}{0.541248in}}%
\pgfpathcurveto{\pgfqpoint{0.613642in}{0.552187in}}{\pgfqpoint{0.598804in}{0.558333in}}{\pgfqpoint{0.583333in}{0.558333in}}%
\pgfpathcurveto{\pgfqpoint{0.567863in}{0.558333in}}{\pgfqpoint{0.553025in}{0.552187in}}{\pgfqpoint{0.542085in}{0.541248in}}%
\pgfpathcurveto{\pgfqpoint{0.531146in}{0.530309in}}{\pgfqpoint{0.525000in}{0.515470in}}{\pgfqpoint{0.525000in}{0.500000in}}%
\pgfpathcurveto{\pgfqpoint{0.525000in}{0.484530in}}{\pgfqpoint{0.531146in}{0.469691in}}{\pgfqpoint{0.542085in}{0.458752in}}%
\pgfpathcurveto{\pgfqpoint{0.553025in}{0.447813in}}{\pgfqpoint{0.567863in}{0.441667in}}{\pgfqpoint{0.583333in}{0.441667in}}%
\pgfpathclose%
\pgfpathmoveto{\pgfqpoint{0.583333in}{0.447500in}}%
\pgfpathcurveto{\pgfqpoint{0.583333in}{0.447500in}}{\pgfqpoint{0.569410in}{0.447500in}}{\pgfqpoint{0.556055in}{0.453032in}}%
\pgfpathcurveto{\pgfqpoint{0.546210in}{0.462877in}}{\pgfqpoint{0.536365in}{0.472722in}}{\pgfqpoint{0.530833in}{0.486077in}}%
\pgfpathcurveto{\pgfqpoint{0.530833in}{0.500000in}}{\pgfqpoint{0.530833in}{0.513923in}}{\pgfqpoint{0.536365in}{0.527278in}}%
\pgfpathcurveto{\pgfqpoint{0.546210in}{0.537123in}}{\pgfqpoint{0.556055in}{0.546968in}}{\pgfqpoint{0.569410in}{0.552500in}}%
\pgfpathcurveto{\pgfqpoint{0.583333in}{0.552500in}}{\pgfqpoint{0.597256in}{0.552500in}}{\pgfqpoint{0.610611in}{0.546968in}}%
\pgfpathcurveto{\pgfqpoint{0.620456in}{0.537123in}}{\pgfqpoint{0.630302in}{0.527278in}}{\pgfqpoint{0.635833in}{0.513923in}}%
\pgfpathcurveto{\pgfqpoint{0.635833in}{0.500000in}}{\pgfqpoint{0.635833in}{0.486077in}}{\pgfqpoint{0.630302in}{0.472722in}}%
\pgfpathcurveto{\pgfqpoint{0.620456in}{0.462877in}}{\pgfqpoint{0.610611in}{0.453032in}}{\pgfqpoint{0.597256in}{0.447500in}}%
\pgfpathclose%
\pgfpathmoveto{\pgfqpoint{0.750000in}{0.441667in}}%
\pgfpathcurveto{\pgfqpoint{0.765470in}{0.441667in}}{\pgfqpoint{0.780309in}{0.447813in}}{\pgfqpoint{0.791248in}{0.458752in}}%
\pgfpathcurveto{\pgfqpoint{0.802187in}{0.469691in}}{\pgfqpoint{0.808333in}{0.484530in}}{\pgfqpoint{0.808333in}{0.500000in}}%
\pgfpathcurveto{\pgfqpoint{0.808333in}{0.515470in}}{\pgfqpoint{0.802187in}{0.530309in}}{\pgfqpoint{0.791248in}{0.541248in}}%
\pgfpathcurveto{\pgfqpoint{0.780309in}{0.552187in}}{\pgfqpoint{0.765470in}{0.558333in}}{\pgfqpoint{0.750000in}{0.558333in}}%
\pgfpathcurveto{\pgfqpoint{0.734530in}{0.558333in}}{\pgfqpoint{0.719691in}{0.552187in}}{\pgfqpoint{0.708752in}{0.541248in}}%
\pgfpathcurveto{\pgfqpoint{0.697813in}{0.530309in}}{\pgfqpoint{0.691667in}{0.515470in}}{\pgfqpoint{0.691667in}{0.500000in}}%
\pgfpathcurveto{\pgfqpoint{0.691667in}{0.484530in}}{\pgfqpoint{0.697813in}{0.469691in}}{\pgfqpoint{0.708752in}{0.458752in}}%
\pgfpathcurveto{\pgfqpoint{0.719691in}{0.447813in}}{\pgfqpoint{0.734530in}{0.441667in}}{\pgfqpoint{0.750000in}{0.441667in}}%
\pgfpathclose%
\pgfpathmoveto{\pgfqpoint{0.750000in}{0.447500in}}%
\pgfpathcurveto{\pgfqpoint{0.750000in}{0.447500in}}{\pgfqpoint{0.736077in}{0.447500in}}{\pgfqpoint{0.722722in}{0.453032in}}%
\pgfpathcurveto{\pgfqpoint{0.712877in}{0.462877in}}{\pgfqpoint{0.703032in}{0.472722in}}{\pgfqpoint{0.697500in}{0.486077in}}%
\pgfpathcurveto{\pgfqpoint{0.697500in}{0.500000in}}{\pgfqpoint{0.697500in}{0.513923in}}{\pgfqpoint{0.703032in}{0.527278in}}%
\pgfpathcurveto{\pgfqpoint{0.712877in}{0.537123in}}{\pgfqpoint{0.722722in}{0.546968in}}{\pgfqpoint{0.736077in}{0.552500in}}%
\pgfpathcurveto{\pgfqpoint{0.750000in}{0.552500in}}{\pgfqpoint{0.763923in}{0.552500in}}{\pgfqpoint{0.777278in}{0.546968in}}%
\pgfpathcurveto{\pgfqpoint{0.787123in}{0.537123in}}{\pgfqpoint{0.796968in}{0.527278in}}{\pgfqpoint{0.802500in}{0.513923in}}%
\pgfpathcurveto{\pgfqpoint{0.802500in}{0.500000in}}{\pgfqpoint{0.802500in}{0.486077in}}{\pgfqpoint{0.796968in}{0.472722in}}%
\pgfpathcurveto{\pgfqpoint{0.787123in}{0.462877in}}{\pgfqpoint{0.777278in}{0.453032in}}{\pgfqpoint{0.763923in}{0.447500in}}%
\pgfpathclose%
\pgfpathmoveto{\pgfqpoint{0.916667in}{0.441667in}}%
\pgfpathcurveto{\pgfqpoint{0.932137in}{0.441667in}}{\pgfqpoint{0.946975in}{0.447813in}}{\pgfqpoint{0.957915in}{0.458752in}}%
\pgfpathcurveto{\pgfqpoint{0.968854in}{0.469691in}}{\pgfqpoint{0.975000in}{0.484530in}}{\pgfqpoint{0.975000in}{0.500000in}}%
\pgfpathcurveto{\pgfqpoint{0.975000in}{0.515470in}}{\pgfqpoint{0.968854in}{0.530309in}}{\pgfqpoint{0.957915in}{0.541248in}}%
\pgfpathcurveto{\pgfqpoint{0.946975in}{0.552187in}}{\pgfqpoint{0.932137in}{0.558333in}}{\pgfqpoint{0.916667in}{0.558333in}}%
\pgfpathcurveto{\pgfqpoint{0.901196in}{0.558333in}}{\pgfqpoint{0.886358in}{0.552187in}}{\pgfqpoint{0.875419in}{0.541248in}}%
\pgfpathcurveto{\pgfqpoint{0.864480in}{0.530309in}}{\pgfqpoint{0.858333in}{0.515470in}}{\pgfqpoint{0.858333in}{0.500000in}}%
\pgfpathcurveto{\pgfqpoint{0.858333in}{0.484530in}}{\pgfqpoint{0.864480in}{0.469691in}}{\pgfqpoint{0.875419in}{0.458752in}}%
\pgfpathcurveto{\pgfqpoint{0.886358in}{0.447813in}}{\pgfqpoint{0.901196in}{0.441667in}}{\pgfqpoint{0.916667in}{0.441667in}}%
\pgfpathclose%
\pgfpathmoveto{\pgfqpoint{0.916667in}{0.447500in}}%
\pgfpathcurveto{\pgfqpoint{0.916667in}{0.447500in}}{\pgfqpoint{0.902744in}{0.447500in}}{\pgfqpoint{0.889389in}{0.453032in}}%
\pgfpathcurveto{\pgfqpoint{0.879544in}{0.462877in}}{\pgfqpoint{0.869698in}{0.472722in}}{\pgfqpoint{0.864167in}{0.486077in}}%
\pgfpathcurveto{\pgfqpoint{0.864167in}{0.500000in}}{\pgfqpoint{0.864167in}{0.513923in}}{\pgfqpoint{0.869698in}{0.527278in}}%
\pgfpathcurveto{\pgfqpoint{0.879544in}{0.537123in}}{\pgfqpoint{0.889389in}{0.546968in}}{\pgfqpoint{0.902744in}{0.552500in}}%
\pgfpathcurveto{\pgfqpoint{0.916667in}{0.552500in}}{\pgfqpoint{0.930590in}{0.552500in}}{\pgfqpoint{0.943945in}{0.546968in}}%
\pgfpathcurveto{\pgfqpoint{0.953790in}{0.537123in}}{\pgfqpoint{0.963635in}{0.527278in}}{\pgfqpoint{0.969167in}{0.513923in}}%
\pgfpathcurveto{\pgfqpoint{0.969167in}{0.500000in}}{\pgfqpoint{0.969167in}{0.486077in}}{\pgfqpoint{0.963635in}{0.472722in}}%
\pgfpathcurveto{\pgfqpoint{0.953790in}{0.462877in}}{\pgfqpoint{0.943945in}{0.453032in}}{\pgfqpoint{0.930590in}{0.447500in}}%
\pgfpathclose%
\pgfpathmoveto{\pgfqpoint{0.000000in}{0.608333in}}%
\pgfpathcurveto{\pgfqpoint{0.015470in}{0.608333in}}{\pgfqpoint{0.030309in}{0.614480in}}{\pgfqpoint{0.041248in}{0.625419in}}%
\pgfpathcurveto{\pgfqpoint{0.052187in}{0.636358in}}{\pgfqpoint{0.058333in}{0.651196in}}{\pgfqpoint{0.058333in}{0.666667in}}%
\pgfpathcurveto{\pgfqpoint{0.058333in}{0.682137in}}{\pgfqpoint{0.052187in}{0.696975in}}{\pgfqpoint{0.041248in}{0.707915in}}%
\pgfpathcurveto{\pgfqpoint{0.030309in}{0.718854in}}{\pgfqpoint{0.015470in}{0.725000in}}{\pgfqpoint{0.000000in}{0.725000in}}%
\pgfpathcurveto{\pgfqpoint{-0.015470in}{0.725000in}}{\pgfqpoint{-0.030309in}{0.718854in}}{\pgfqpoint{-0.041248in}{0.707915in}}%
\pgfpathcurveto{\pgfqpoint{-0.052187in}{0.696975in}}{\pgfqpoint{-0.058333in}{0.682137in}}{\pgfqpoint{-0.058333in}{0.666667in}}%
\pgfpathcurveto{\pgfqpoint{-0.058333in}{0.651196in}}{\pgfqpoint{-0.052187in}{0.636358in}}{\pgfqpoint{-0.041248in}{0.625419in}}%
\pgfpathcurveto{\pgfqpoint{-0.030309in}{0.614480in}}{\pgfqpoint{-0.015470in}{0.608333in}}{\pgfqpoint{0.000000in}{0.608333in}}%
\pgfpathclose%
\pgfpathmoveto{\pgfqpoint{0.000000in}{0.614167in}}%
\pgfpathcurveto{\pgfqpoint{0.000000in}{0.614167in}}{\pgfqpoint{-0.013923in}{0.614167in}}{\pgfqpoint{-0.027278in}{0.619698in}}%
\pgfpathcurveto{\pgfqpoint{-0.037123in}{0.629544in}}{\pgfqpoint{-0.046968in}{0.639389in}}{\pgfqpoint{-0.052500in}{0.652744in}}%
\pgfpathcurveto{\pgfqpoint{-0.052500in}{0.666667in}}{\pgfqpoint{-0.052500in}{0.680590in}}{\pgfqpoint{-0.046968in}{0.693945in}}%
\pgfpathcurveto{\pgfqpoint{-0.037123in}{0.703790in}}{\pgfqpoint{-0.027278in}{0.713635in}}{\pgfqpoint{-0.013923in}{0.719167in}}%
\pgfpathcurveto{\pgfqpoint{0.000000in}{0.719167in}}{\pgfqpoint{0.013923in}{0.719167in}}{\pgfqpoint{0.027278in}{0.713635in}}%
\pgfpathcurveto{\pgfqpoint{0.037123in}{0.703790in}}{\pgfqpoint{0.046968in}{0.693945in}}{\pgfqpoint{0.052500in}{0.680590in}}%
\pgfpathcurveto{\pgfqpoint{0.052500in}{0.666667in}}{\pgfqpoint{0.052500in}{0.652744in}}{\pgfqpoint{0.046968in}{0.639389in}}%
\pgfpathcurveto{\pgfqpoint{0.037123in}{0.629544in}}{\pgfqpoint{0.027278in}{0.619698in}}{\pgfqpoint{0.013923in}{0.614167in}}%
\pgfpathclose%
\pgfpathmoveto{\pgfqpoint{0.166667in}{0.608333in}}%
\pgfpathcurveto{\pgfqpoint{0.182137in}{0.608333in}}{\pgfqpoint{0.196975in}{0.614480in}}{\pgfqpoint{0.207915in}{0.625419in}}%
\pgfpathcurveto{\pgfqpoint{0.218854in}{0.636358in}}{\pgfqpoint{0.225000in}{0.651196in}}{\pgfqpoint{0.225000in}{0.666667in}}%
\pgfpathcurveto{\pgfqpoint{0.225000in}{0.682137in}}{\pgfqpoint{0.218854in}{0.696975in}}{\pgfqpoint{0.207915in}{0.707915in}}%
\pgfpathcurveto{\pgfqpoint{0.196975in}{0.718854in}}{\pgfqpoint{0.182137in}{0.725000in}}{\pgfqpoint{0.166667in}{0.725000in}}%
\pgfpathcurveto{\pgfqpoint{0.151196in}{0.725000in}}{\pgfqpoint{0.136358in}{0.718854in}}{\pgfqpoint{0.125419in}{0.707915in}}%
\pgfpathcurveto{\pgfqpoint{0.114480in}{0.696975in}}{\pgfqpoint{0.108333in}{0.682137in}}{\pgfqpoint{0.108333in}{0.666667in}}%
\pgfpathcurveto{\pgfqpoint{0.108333in}{0.651196in}}{\pgfqpoint{0.114480in}{0.636358in}}{\pgfqpoint{0.125419in}{0.625419in}}%
\pgfpathcurveto{\pgfqpoint{0.136358in}{0.614480in}}{\pgfqpoint{0.151196in}{0.608333in}}{\pgfqpoint{0.166667in}{0.608333in}}%
\pgfpathclose%
\pgfpathmoveto{\pgfqpoint{0.166667in}{0.614167in}}%
\pgfpathcurveto{\pgfqpoint{0.166667in}{0.614167in}}{\pgfqpoint{0.152744in}{0.614167in}}{\pgfqpoint{0.139389in}{0.619698in}}%
\pgfpathcurveto{\pgfqpoint{0.129544in}{0.629544in}}{\pgfqpoint{0.119698in}{0.639389in}}{\pgfqpoint{0.114167in}{0.652744in}}%
\pgfpathcurveto{\pgfqpoint{0.114167in}{0.666667in}}{\pgfqpoint{0.114167in}{0.680590in}}{\pgfqpoint{0.119698in}{0.693945in}}%
\pgfpathcurveto{\pgfqpoint{0.129544in}{0.703790in}}{\pgfqpoint{0.139389in}{0.713635in}}{\pgfqpoint{0.152744in}{0.719167in}}%
\pgfpathcurveto{\pgfqpoint{0.166667in}{0.719167in}}{\pgfqpoint{0.180590in}{0.719167in}}{\pgfqpoint{0.193945in}{0.713635in}}%
\pgfpathcurveto{\pgfqpoint{0.203790in}{0.703790in}}{\pgfqpoint{0.213635in}{0.693945in}}{\pgfqpoint{0.219167in}{0.680590in}}%
\pgfpathcurveto{\pgfqpoint{0.219167in}{0.666667in}}{\pgfqpoint{0.219167in}{0.652744in}}{\pgfqpoint{0.213635in}{0.639389in}}%
\pgfpathcurveto{\pgfqpoint{0.203790in}{0.629544in}}{\pgfqpoint{0.193945in}{0.619698in}}{\pgfqpoint{0.180590in}{0.614167in}}%
\pgfpathclose%
\pgfpathmoveto{\pgfqpoint{0.333333in}{0.608333in}}%
\pgfpathcurveto{\pgfqpoint{0.348804in}{0.608333in}}{\pgfqpoint{0.363642in}{0.614480in}}{\pgfqpoint{0.374581in}{0.625419in}}%
\pgfpathcurveto{\pgfqpoint{0.385520in}{0.636358in}}{\pgfqpoint{0.391667in}{0.651196in}}{\pgfqpoint{0.391667in}{0.666667in}}%
\pgfpathcurveto{\pgfqpoint{0.391667in}{0.682137in}}{\pgfqpoint{0.385520in}{0.696975in}}{\pgfqpoint{0.374581in}{0.707915in}}%
\pgfpathcurveto{\pgfqpoint{0.363642in}{0.718854in}}{\pgfqpoint{0.348804in}{0.725000in}}{\pgfqpoint{0.333333in}{0.725000in}}%
\pgfpathcurveto{\pgfqpoint{0.317863in}{0.725000in}}{\pgfqpoint{0.303025in}{0.718854in}}{\pgfqpoint{0.292085in}{0.707915in}}%
\pgfpathcurveto{\pgfqpoint{0.281146in}{0.696975in}}{\pgfqpoint{0.275000in}{0.682137in}}{\pgfqpoint{0.275000in}{0.666667in}}%
\pgfpathcurveto{\pgfqpoint{0.275000in}{0.651196in}}{\pgfqpoint{0.281146in}{0.636358in}}{\pgfqpoint{0.292085in}{0.625419in}}%
\pgfpathcurveto{\pgfqpoint{0.303025in}{0.614480in}}{\pgfqpoint{0.317863in}{0.608333in}}{\pgfqpoint{0.333333in}{0.608333in}}%
\pgfpathclose%
\pgfpathmoveto{\pgfqpoint{0.333333in}{0.614167in}}%
\pgfpathcurveto{\pgfqpoint{0.333333in}{0.614167in}}{\pgfqpoint{0.319410in}{0.614167in}}{\pgfqpoint{0.306055in}{0.619698in}}%
\pgfpathcurveto{\pgfqpoint{0.296210in}{0.629544in}}{\pgfqpoint{0.286365in}{0.639389in}}{\pgfqpoint{0.280833in}{0.652744in}}%
\pgfpathcurveto{\pgfqpoint{0.280833in}{0.666667in}}{\pgfqpoint{0.280833in}{0.680590in}}{\pgfqpoint{0.286365in}{0.693945in}}%
\pgfpathcurveto{\pgfqpoint{0.296210in}{0.703790in}}{\pgfqpoint{0.306055in}{0.713635in}}{\pgfqpoint{0.319410in}{0.719167in}}%
\pgfpathcurveto{\pgfqpoint{0.333333in}{0.719167in}}{\pgfqpoint{0.347256in}{0.719167in}}{\pgfqpoint{0.360611in}{0.713635in}}%
\pgfpathcurveto{\pgfqpoint{0.370456in}{0.703790in}}{\pgfqpoint{0.380302in}{0.693945in}}{\pgfqpoint{0.385833in}{0.680590in}}%
\pgfpathcurveto{\pgfqpoint{0.385833in}{0.666667in}}{\pgfqpoint{0.385833in}{0.652744in}}{\pgfqpoint{0.380302in}{0.639389in}}%
\pgfpathcurveto{\pgfqpoint{0.370456in}{0.629544in}}{\pgfqpoint{0.360611in}{0.619698in}}{\pgfqpoint{0.347256in}{0.614167in}}%
\pgfpathclose%
\pgfpathmoveto{\pgfqpoint{0.500000in}{0.608333in}}%
\pgfpathcurveto{\pgfqpoint{0.515470in}{0.608333in}}{\pgfqpoint{0.530309in}{0.614480in}}{\pgfqpoint{0.541248in}{0.625419in}}%
\pgfpathcurveto{\pgfqpoint{0.552187in}{0.636358in}}{\pgfqpoint{0.558333in}{0.651196in}}{\pgfqpoint{0.558333in}{0.666667in}}%
\pgfpathcurveto{\pgfqpoint{0.558333in}{0.682137in}}{\pgfqpoint{0.552187in}{0.696975in}}{\pgfqpoint{0.541248in}{0.707915in}}%
\pgfpathcurveto{\pgfqpoint{0.530309in}{0.718854in}}{\pgfqpoint{0.515470in}{0.725000in}}{\pgfqpoint{0.500000in}{0.725000in}}%
\pgfpathcurveto{\pgfqpoint{0.484530in}{0.725000in}}{\pgfqpoint{0.469691in}{0.718854in}}{\pgfqpoint{0.458752in}{0.707915in}}%
\pgfpathcurveto{\pgfqpoint{0.447813in}{0.696975in}}{\pgfqpoint{0.441667in}{0.682137in}}{\pgfqpoint{0.441667in}{0.666667in}}%
\pgfpathcurveto{\pgfqpoint{0.441667in}{0.651196in}}{\pgfqpoint{0.447813in}{0.636358in}}{\pgfqpoint{0.458752in}{0.625419in}}%
\pgfpathcurveto{\pgfqpoint{0.469691in}{0.614480in}}{\pgfqpoint{0.484530in}{0.608333in}}{\pgfqpoint{0.500000in}{0.608333in}}%
\pgfpathclose%
\pgfpathmoveto{\pgfqpoint{0.500000in}{0.614167in}}%
\pgfpathcurveto{\pgfqpoint{0.500000in}{0.614167in}}{\pgfqpoint{0.486077in}{0.614167in}}{\pgfqpoint{0.472722in}{0.619698in}}%
\pgfpathcurveto{\pgfqpoint{0.462877in}{0.629544in}}{\pgfqpoint{0.453032in}{0.639389in}}{\pgfqpoint{0.447500in}{0.652744in}}%
\pgfpathcurveto{\pgfqpoint{0.447500in}{0.666667in}}{\pgfqpoint{0.447500in}{0.680590in}}{\pgfqpoint{0.453032in}{0.693945in}}%
\pgfpathcurveto{\pgfqpoint{0.462877in}{0.703790in}}{\pgfqpoint{0.472722in}{0.713635in}}{\pgfqpoint{0.486077in}{0.719167in}}%
\pgfpathcurveto{\pgfqpoint{0.500000in}{0.719167in}}{\pgfqpoint{0.513923in}{0.719167in}}{\pgfqpoint{0.527278in}{0.713635in}}%
\pgfpathcurveto{\pgfqpoint{0.537123in}{0.703790in}}{\pgfqpoint{0.546968in}{0.693945in}}{\pgfqpoint{0.552500in}{0.680590in}}%
\pgfpathcurveto{\pgfqpoint{0.552500in}{0.666667in}}{\pgfqpoint{0.552500in}{0.652744in}}{\pgfqpoint{0.546968in}{0.639389in}}%
\pgfpathcurveto{\pgfqpoint{0.537123in}{0.629544in}}{\pgfqpoint{0.527278in}{0.619698in}}{\pgfqpoint{0.513923in}{0.614167in}}%
\pgfpathclose%
\pgfpathmoveto{\pgfqpoint{0.666667in}{0.608333in}}%
\pgfpathcurveto{\pgfqpoint{0.682137in}{0.608333in}}{\pgfqpoint{0.696975in}{0.614480in}}{\pgfqpoint{0.707915in}{0.625419in}}%
\pgfpathcurveto{\pgfqpoint{0.718854in}{0.636358in}}{\pgfqpoint{0.725000in}{0.651196in}}{\pgfqpoint{0.725000in}{0.666667in}}%
\pgfpathcurveto{\pgfqpoint{0.725000in}{0.682137in}}{\pgfqpoint{0.718854in}{0.696975in}}{\pgfqpoint{0.707915in}{0.707915in}}%
\pgfpathcurveto{\pgfqpoint{0.696975in}{0.718854in}}{\pgfqpoint{0.682137in}{0.725000in}}{\pgfqpoint{0.666667in}{0.725000in}}%
\pgfpathcurveto{\pgfqpoint{0.651196in}{0.725000in}}{\pgfqpoint{0.636358in}{0.718854in}}{\pgfqpoint{0.625419in}{0.707915in}}%
\pgfpathcurveto{\pgfqpoint{0.614480in}{0.696975in}}{\pgfqpoint{0.608333in}{0.682137in}}{\pgfqpoint{0.608333in}{0.666667in}}%
\pgfpathcurveto{\pgfqpoint{0.608333in}{0.651196in}}{\pgfqpoint{0.614480in}{0.636358in}}{\pgfqpoint{0.625419in}{0.625419in}}%
\pgfpathcurveto{\pgfqpoint{0.636358in}{0.614480in}}{\pgfqpoint{0.651196in}{0.608333in}}{\pgfqpoint{0.666667in}{0.608333in}}%
\pgfpathclose%
\pgfpathmoveto{\pgfqpoint{0.666667in}{0.614167in}}%
\pgfpathcurveto{\pgfqpoint{0.666667in}{0.614167in}}{\pgfqpoint{0.652744in}{0.614167in}}{\pgfqpoint{0.639389in}{0.619698in}}%
\pgfpathcurveto{\pgfqpoint{0.629544in}{0.629544in}}{\pgfqpoint{0.619698in}{0.639389in}}{\pgfqpoint{0.614167in}{0.652744in}}%
\pgfpathcurveto{\pgfqpoint{0.614167in}{0.666667in}}{\pgfqpoint{0.614167in}{0.680590in}}{\pgfqpoint{0.619698in}{0.693945in}}%
\pgfpathcurveto{\pgfqpoint{0.629544in}{0.703790in}}{\pgfqpoint{0.639389in}{0.713635in}}{\pgfqpoint{0.652744in}{0.719167in}}%
\pgfpathcurveto{\pgfqpoint{0.666667in}{0.719167in}}{\pgfqpoint{0.680590in}{0.719167in}}{\pgfqpoint{0.693945in}{0.713635in}}%
\pgfpathcurveto{\pgfqpoint{0.703790in}{0.703790in}}{\pgfqpoint{0.713635in}{0.693945in}}{\pgfqpoint{0.719167in}{0.680590in}}%
\pgfpathcurveto{\pgfqpoint{0.719167in}{0.666667in}}{\pgfqpoint{0.719167in}{0.652744in}}{\pgfqpoint{0.713635in}{0.639389in}}%
\pgfpathcurveto{\pgfqpoint{0.703790in}{0.629544in}}{\pgfqpoint{0.693945in}{0.619698in}}{\pgfqpoint{0.680590in}{0.614167in}}%
\pgfpathclose%
\pgfpathmoveto{\pgfqpoint{0.833333in}{0.608333in}}%
\pgfpathcurveto{\pgfqpoint{0.848804in}{0.608333in}}{\pgfqpoint{0.863642in}{0.614480in}}{\pgfqpoint{0.874581in}{0.625419in}}%
\pgfpathcurveto{\pgfqpoint{0.885520in}{0.636358in}}{\pgfqpoint{0.891667in}{0.651196in}}{\pgfqpoint{0.891667in}{0.666667in}}%
\pgfpathcurveto{\pgfqpoint{0.891667in}{0.682137in}}{\pgfqpoint{0.885520in}{0.696975in}}{\pgfqpoint{0.874581in}{0.707915in}}%
\pgfpathcurveto{\pgfqpoint{0.863642in}{0.718854in}}{\pgfqpoint{0.848804in}{0.725000in}}{\pgfqpoint{0.833333in}{0.725000in}}%
\pgfpathcurveto{\pgfqpoint{0.817863in}{0.725000in}}{\pgfqpoint{0.803025in}{0.718854in}}{\pgfqpoint{0.792085in}{0.707915in}}%
\pgfpathcurveto{\pgfqpoint{0.781146in}{0.696975in}}{\pgfqpoint{0.775000in}{0.682137in}}{\pgfqpoint{0.775000in}{0.666667in}}%
\pgfpathcurveto{\pgfqpoint{0.775000in}{0.651196in}}{\pgfqpoint{0.781146in}{0.636358in}}{\pgfqpoint{0.792085in}{0.625419in}}%
\pgfpathcurveto{\pgfqpoint{0.803025in}{0.614480in}}{\pgfqpoint{0.817863in}{0.608333in}}{\pgfqpoint{0.833333in}{0.608333in}}%
\pgfpathclose%
\pgfpathmoveto{\pgfqpoint{0.833333in}{0.614167in}}%
\pgfpathcurveto{\pgfqpoint{0.833333in}{0.614167in}}{\pgfqpoint{0.819410in}{0.614167in}}{\pgfqpoint{0.806055in}{0.619698in}}%
\pgfpathcurveto{\pgfqpoint{0.796210in}{0.629544in}}{\pgfqpoint{0.786365in}{0.639389in}}{\pgfqpoint{0.780833in}{0.652744in}}%
\pgfpathcurveto{\pgfqpoint{0.780833in}{0.666667in}}{\pgfqpoint{0.780833in}{0.680590in}}{\pgfqpoint{0.786365in}{0.693945in}}%
\pgfpathcurveto{\pgfqpoint{0.796210in}{0.703790in}}{\pgfqpoint{0.806055in}{0.713635in}}{\pgfqpoint{0.819410in}{0.719167in}}%
\pgfpathcurveto{\pgfqpoint{0.833333in}{0.719167in}}{\pgfqpoint{0.847256in}{0.719167in}}{\pgfqpoint{0.860611in}{0.713635in}}%
\pgfpathcurveto{\pgfqpoint{0.870456in}{0.703790in}}{\pgfqpoint{0.880302in}{0.693945in}}{\pgfqpoint{0.885833in}{0.680590in}}%
\pgfpathcurveto{\pgfqpoint{0.885833in}{0.666667in}}{\pgfqpoint{0.885833in}{0.652744in}}{\pgfqpoint{0.880302in}{0.639389in}}%
\pgfpathcurveto{\pgfqpoint{0.870456in}{0.629544in}}{\pgfqpoint{0.860611in}{0.619698in}}{\pgfqpoint{0.847256in}{0.614167in}}%
\pgfpathclose%
\pgfpathmoveto{\pgfqpoint{1.000000in}{0.608333in}}%
\pgfpathcurveto{\pgfqpoint{1.015470in}{0.608333in}}{\pgfqpoint{1.030309in}{0.614480in}}{\pgfqpoint{1.041248in}{0.625419in}}%
\pgfpathcurveto{\pgfqpoint{1.052187in}{0.636358in}}{\pgfqpoint{1.058333in}{0.651196in}}{\pgfqpoint{1.058333in}{0.666667in}}%
\pgfpathcurveto{\pgfqpoint{1.058333in}{0.682137in}}{\pgfqpoint{1.052187in}{0.696975in}}{\pgfqpoint{1.041248in}{0.707915in}}%
\pgfpathcurveto{\pgfqpoint{1.030309in}{0.718854in}}{\pgfqpoint{1.015470in}{0.725000in}}{\pgfqpoint{1.000000in}{0.725000in}}%
\pgfpathcurveto{\pgfqpoint{0.984530in}{0.725000in}}{\pgfqpoint{0.969691in}{0.718854in}}{\pgfqpoint{0.958752in}{0.707915in}}%
\pgfpathcurveto{\pgfqpoint{0.947813in}{0.696975in}}{\pgfqpoint{0.941667in}{0.682137in}}{\pgfqpoint{0.941667in}{0.666667in}}%
\pgfpathcurveto{\pgfqpoint{0.941667in}{0.651196in}}{\pgfqpoint{0.947813in}{0.636358in}}{\pgfqpoint{0.958752in}{0.625419in}}%
\pgfpathcurveto{\pgfqpoint{0.969691in}{0.614480in}}{\pgfqpoint{0.984530in}{0.608333in}}{\pgfqpoint{1.000000in}{0.608333in}}%
\pgfpathclose%
\pgfpathmoveto{\pgfqpoint{1.000000in}{0.614167in}}%
\pgfpathcurveto{\pgfqpoint{1.000000in}{0.614167in}}{\pgfqpoint{0.986077in}{0.614167in}}{\pgfqpoint{0.972722in}{0.619698in}}%
\pgfpathcurveto{\pgfqpoint{0.962877in}{0.629544in}}{\pgfqpoint{0.953032in}{0.639389in}}{\pgfqpoint{0.947500in}{0.652744in}}%
\pgfpathcurveto{\pgfqpoint{0.947500in}{0.666667in}}{\pgfqpoint{0.947500in}{0.680590in}}{\pgfqpoint{0.953032in}{0.693945in}}%
\pgfpathcurveto{\pgfqpoint{0.962877in}{0.703790in}}{\pgfqpoint{0.972722in}{0.713635in}}{\pgfqpoint{0.986077in}{0.719167in}}%
\pgfpathcurveto{\pgfqpoint{1.000000in}{0.719167in}}{\pgfqpoint{1.013923in}{0.719167in}}{\pgfqpoint{1.027278in}{0.713635in}}%
\pgfpathcurveto{\pgfqpoint{1.037123in}{0.703790in}}{\pgfqpoint{1.046968in}{0.693945in}}{\pgfqpoint{1.052500in}{0.680590in}}%
\pgfpathcurveto{\pgfqpoint{1.052500in}{0.666667in}}{\pgfqpoint{1.052500in}{0.652744in}}{\pgfqpoint{1.046968in}{0.639389in}}%
\pgfpathcurveto{\pgfqpoint{1.037123in}{0.629544in}}{\pgfqpoint{1.027278in}{0.619698in}}{\pgfqpoint{1.013923in}{0.614167in}}%
\pgfpathclose%
\pgfpathmoveto{\pgfqpoint{0.083333in}{0.775000in}}%
\pgfpathcurveto{\pgfqpoint{0.098804in}{0.775000in}}{\pgfqpoint{0.113642in}{0.781146in}}{\pgfqpoint{0.124581in}{0.792085in}}%
\pgfpathcurveto{\pgfqpoint{0.135520in}{0.803025in}}{\pgfqpoint{0.141667in}{0.817863in}}{\pgfqpoint{0.141667in}{0.833333in}}%
\pgfpathcurveto{\pgfqpoint{0.141667in}{0.848804in}}{\pgfqpoint{0.135520in}{0.863642in}}{\pgfqpoint{0.124581in}{0.874581in}}%
\pgfpathcurveto{\pgfqpoint{0.113642in}{0.885520in}}{\pgfqpoint{0.098804in}{0.891667in}}{\pgfqpoint{0.083333in}{0.891667in}}%
\pgfpathcurveto{\pgfqpoint{0.067863in}{0.891667in}}{\pgfqpoint{0.053025in}{0.885520in}}{\pgfqpoint{0.042085in}{0.874581in}}%
\pgfpathcurveto{\pgfqpoint{0.031146in}{0.863642in}}{\pgfqpoint{0.025000in}{0.848804in}}{\pgfqpoint{0.025000in}{0.833333in}}%
\pgfpathcurveto{\pgfqpoint{0.025000in}{0.817863in}}{\pgfqpoint{0.031146in}{0.803025in}}{\pgfqpoint{0.042085in}{0.792085in}}%
\pgfpathcurveto{\pgfqpoint{0.053025in}{0.781146in}}{\pgfqpoint{0.067863in}{0.775000in}}{\pgfqpoint{0.083333in}{0.775000in}}%
\pgfpathclose%
\pgfpathmoveto{\pgfqpoint{0.083333in}{0.780833in}}%
\pgfpathcurveto{\pgfqpoint{0.083333in}{0.780833in}}{\pgfqpoint{0.069410in}{0.780833in}}{\pgfqpoint{0.056055in}{0.786365in}}%
\pgfpathcurveto{\pgfqpoint{0.046210in}{0.796210in}}{\pgfqpoint{0.036365in}{0.806055in}}{\pgfqpoint{0.030833in}{0.819410in}}%
\pgfpathcurveto{\pgfqpoint{0.030833in}{0.833333in}}{\pgfqpoint{0.030833in}{0.847256in}}{\pgfqpoint{0.036365in}{0.860611in}}%
\pgfpathcurveto{\pgfqpoint{0.046210in}{0.870456in}}{\pgfqpoint{0.056055in}{0.880302in}}{\pgfqpoint{0.069410in}{0.885833in}}%
\pgfpathcurveto{\pgfqpoint{0.083333in}{0.885833in}}{\pgfqpoint{0.097256in}{0.885833in}}{\pgfqpoint{0.110611in}{0.880302in}}%
\pgfpathcurveto{\pgfqpoint{0.120456in}{0.870456in}}{\pgfqpoint{0.130302in}{0.860611in}}{\pgfqpoint{0.135833in}{0.847256in}}%
\pgfpathcurveto{\pgfqpoint{0.135833in}{0.833333in}}{\pgfqpoint{0.135833in}{0.819410in}}{\pgfqpoint{0.130302in}{0.806055in}}%
\pgfpathcurveto{\pgfqpoint{0.120456in}{0.796210in}}{\pgfqpoint{0.110611in}{0.786365in}}{\pgfqpoint{0.097256in}{0.780833in}}%
\pgfpathclose%
\pgfpathmoveto{\pgfqpoint{0.250000in}{0.775000in}}%
\pgfpathcurveto{\pgfqpoint{0.265470in}{0.775000in}}{\pgfqpoint{0.280309in}{0.781146in}}{\pgfqpoint{0.291248in}{0.792085in}}%
\pgfpathcurveto{\pgfqpoint{0.302187in}{0.803025in}}{\pgfqpoint{0.308333in}{0.817863in}}{\pgfqpoint{0.308333in}{0.833333in}}%
\pgfpathcurveto{\pgfqpoint{0.308333in}{0.848804in}}{\pgfqpoint{0.302187in}{0.863642in}}{\pgfqpoint{0.291248in}{0.874581in}}%
\pgfpathcurveto{\pgfqpoint{0.280309in}{0.885520in}}{\pgfqpoint{0.265470in}{0.891667in}}{\pgfqpoint{0.250000in}{0.891667in}}%
\pgfpathcurveto{\pgfqpoint{0.234530in}{0.891667in}}{\pgfqpoint{0.219691in}{0.885520in}}{\pgfqpoint{0.208752in}{0.874581in}}%
\pgfpathcurveto{\pgfqpoint{0.197813in}{0.863642in}}{\pgfqpoint{0.191667in}{0.848804in}}{\pgfqpoint{0.191667in}{0.833333in}}%
\pgfpathcurveto{\pgfqpoint{0.191667in}{0.817863in}}{\pgfqpoint{0.197813in}{0.803025in}}{\pgfqpoint{0.208752in}{0.792085in}}%
\pgfpathcurveto{\pgfqpoint{0.219691in}{0.781146in}}{\pgfqpoint{0.234530in}{0.775000in}}{\pgfqpoint{0.250000in}{0.775000in}}%
\pgfpathclose%
\pgfpathmoveto{\pgfqpoint{0.250000in}{0.780833in}}%
\pgfpathcurveto{\pgfqpoint{0.250000in}{0.780833in}}{\pgfqpoint{0.236077in}{0.780833in}}{\pgfqpoint{0.222722in}{0.786365in}}%
\pgfpathcurveto{\pgfqpoint{0.212877in}{0.796210in}}{\pgfqpoint{0.203032in}{0.806055in}}{\pgfqpoint{0.197500in}{0.819410in}}%
\pgfpathcurveto{\pgfqpoint{0.197500in}{0.833333in}}{\pgfqpoint{0.197500in}{0.847256in}}{\pgfqpoint{0.203032in}{0.860611in}}%
\pgfpathcurveto{\pgfqpoint{0.212877in}{0.870456in}}{\pgfqpoint{0.222722in}{0.880302in}}{\pgfqpoint{0.236077in}{0.885833in}}%
\pgfpathcurveto{\pgfqpoint{0.250000in}{0.885833in}}{\pgfqpoint{0.263923in}{0.885833in}}{\pgfqpoint{0.277278in}{0.880302in}}%
\pgfpathcurveto{\pgfqpoint{0.287123in}{0.870456in}}{\pgfqpoint{0.296968in}{0.860611in}}{\pgfqpoint{0.302500in}{0.847256in}}%
\pgfpathcurveto{\pgfqpoint{0.302500in}{0.833333in}}{\pgfqpoint{0.302500in}{0.819410in}}{\pgfqpoint{0.296968in}{0.806055in}}%
\pgfpathcurveto{\pgfqpoint{0.287123in}{0.796210in}}{\pgfqpoint{0.277278in}{0.786365in}}{\pgfqpoint{0.263923in}{0.780833in}}%
\pgfpathclose%
\pgfpathmoveto{\pgfqpoint{0.416667in}{0.775000in}}%
\pgfpathcurveto{\pgfqpoint{0.432137in}{0.775000in}}{\pgfqpoint{0.446975in}{0.781146in}}{\pgfqpoint{0.457915in}{0.792085in}}%
\pgfpathcurveto{\pgfqpoint{0.468854in}{0.803025in}}{\pgfqpoint{0.475000in}{0.817863in}}{\pgfqpoint{0.475000in}{0.833333in}}%
\pgfpathcurveto{\pgfqpoint{0.475000in}{0.848804in}}{\pgfqpoint{0.468854in}{0.863642in}}{\pgfqpoint{0.457915in}{0.874581in}}%
\pgfpathcurveto{\pgfqpoint{0.446975in}{0.885520in}}{\pgfqpoint{0.432137in}{0.891667in}}{\pgfqpoint{0.416667in}{0.891667in}}%
\pgfpathcurveto{\pgfqpoint{0.401196in}{0.891667in}}{\pgfqpoint{0.386358in}{0.885520in}}{\pgfqpoint{0.375419in}{0.874581in}}%
\pgfpathcurveto{\pgfqpoint{0.364480in}{0.863642in}}{\pgfqpoint{0.358333in}{0.848804in}}{\pgfqpoint{0.358333in}{0.833333in}}%
\pgfpathcurveto{\pgfqpoint{0.358333in}{0.817863in}}{\pgfqpoint{0.364480in}{0.803025in}}{\pgfqpoint{0.375419in}{0.792085in}}%
\pgfpathcurveto{\pgfqpoint{0.386358in}{0.781146in}}{\pgfqpoint{0.401196in}{0.775000in}}{\pgfqpoint{0.416667in}{0.775000in}}%
\pgfpathclose%
\pgfpathmoveto{\pgfqpoint{0.416667in}{0.780833in}}%
\pgfpathcurveto{\pgfqpoint{0.416667in}{0.780833in}}{\pgfqpoint{0.402744in}{0.780833in}}{\pgfqpoint{0.389389in}{0.786365in}}%
\pgfpathcurveto{\pgfqpoint{0.379544in}{0.796210in}}{\pgfqpoint{0.369698in}{0.806055in}}{\pgfqpoint{0.364167in}{0.819410in}}%
\pgfpathcurveto{\pgfqpoint{0.364167in}{0.833333in}}{\pgfqpoint{0.364167in}{0.847256in}}{\pgfqpoint{0.369698in}{0.860611in}}%
\pgfpathcurveto{\pgfqpoint{0.379544in}{0.870456in}}{\pgfqpoint{0.389389in}{0.880302in}}{\pgfqpoint{0.402744in}{0.885833in}}%
\pgfpathcurveto{\pgfqpoint{0.416667in}{0.885833in}}{\pgfqpoint{0.430590in}{0.885833in}}{\pgfqpoint{0.443945in}{0.880302in}}%
\pgfpathcurveto{\pgfqpoint{0.453790in}{0.870456in}}{\pgfqpoint{0.463635in}{0.860611in}}{\pgfqpoint{0.469167in}{0.847256in}}%
\pgfpathcurveto{\pgfqpoint{0.469167in}{0.833333in}}{\pgfqpoint{0.469167in}{0.819410in}}{\pgfqpoint{0.463635in}{0.806055in}}%
\pgfpathcurveto{\pgfqpoint{0.453790in}{0.796210in}}{\pgfqpoint{0.443945in}{0.786365in}}{\pgfqpoint{0.430590in}{0.780833in}}%
\pgfpathclose%
\pgfpathmoveto{\pgfqpoint{0.583333in}{0.775000in}}%
\pgfpathcurveto{\pgfqpoint{0.598804in}{0.775000in}}{\pgfqpoint{0.613642in}{0.781146in}}{\pgfqpoint{0.624581in}{0.792085in}}%
\pgfpathcurveto{\pgfqpoint{0.635520in}{0.803025in}}{\pgfqpoint{0.641667in}{0.817863in}}{\pgfqpoint{0.641667in}{0.833333in}}%
\pgfpathcurveto{\pgfqpoint{0.641667in}{0.848804in}}{\pgfqpoint{0.635520in}{0.863642in}}{\pgfqpoint{0.624581in}{0.874581in}}%
\pgfpathcurveto{\pgfqpoint{0.613642in}{0.885520in}}{\pgfqpoint{0.598804in}{0.891667in}}{\pgfqpoint{0.583333in}{0.891667in}}%
\pgfpathcurveto{\pgfqpoint{0.567863in}{0.891667in}}{\pgfqpoint{0.553025in}{0.885520in}}{\pgfqpoint{0.542085in}{0.874581in}}%
\pgfpathcurveto{\pgfqpoint{0.531146in}{0.863642in}}{\pgfqpoint{0.525000in}{0.848804in}}{\pgfqpoint{0.525000in}{0.833333in}}%
\pgfpathcurveto{\pgfqpoint{0.525000in}{0.817863in}}{\pgfqpoint{0.531146in}{0.803025in}}{\pgfqpoint{0.542085in}{0.792085in}}%
\pgfpathcurveto{\pgfqpoint{0.553025in}{0.781146in}}{\pgfqpoint{0.567863in}{0.775000in}}{\pgfqpoint{0.583333in}{0.775000in}}%
\pgfpathclose%
\pgfpathmoveto{\pgfqpoint{0.583333in}{0.780833in}}%
\pgfpathcurveto{\pgfqpoint{0.583333in}{0.780833in}}{\pgfqpoint{0.569410in}{0.780833in}}{\pgfqpoint{0.556055in}{0.786365in}}%
\pgfpathcurveto{\pgfqpoint{0.546210in}{0.796210in}}{\pgfqpoint{0.536365in}{0.806055in}}{\pgfqpoint{0.530833in}{0.819410in}}%
\pgfpathcurveto{\pgfqpoint{0.530833in}{0.833333in}}{\pgfqpoint{0.530833in}{0.847256in}}{\pgfqpoint{0.536365in}{0.860611in}}%
\pgfpathcurveto{\pgfqpoint{0.546210in}{0.870456in}}{\pgfqpoint{0.556055in}{0.880302in}}{\pgfqpoint{0.569410in}{0.885833in}}%
\pgfpathcurveto{\pgfqpoint{0.583333in}{0.885833in}}{\pgfqpoint{0.597256in}{0.885833in}}{\pgfqpoint{0.610611in}{0.880302in}}%
\pgfpathcurveto{\pgfqpoint{0.620456in}{0.870456in}}{\pgfqpoint{0.630302in}{0.860611in}}{\pgfqpoint{0.635833in}{0.847256in}}%
\pgfpathcurveto{\pgfqpoint{0.635833in}{0.833333in}}{\pgfqpoint{0.635833in}{0.819410in}}{\pgfqpoint{0.630302in}{0.806055in}}%
\pgfpathcurveto{\pgfqpoint{0.620456in}{0.796210in}}{\pgfqpoint{0.610611in}{0.786365in}}{\pgfqpoint{0.597256in}{0.780833in}}%
\pgfpathclose%
\pgfpathmoveto{\pgfqpoint{0.750000in}{0.775000in}}%
\pgfpathcurveto{\pgfqpoint{0.765470in}{0.775000in}}{\pgfqpoint{0.780309in}{0.781146in}}{\pgfqpoint{0.791248in}{0.792085in}}%
\pgfpathcurveto{\pgfqpoint{0.802187in}{0.803025in}}{\pgfqpoint{0.808333in}{0.817863in}}{\pgfqpoint{0.808333in}{0.833333in}}%
\pgfpathcurveto{\pgfqpoint{0.808333in}{0.848804in}}{\pgfqpoint{0.802187in}{0.863642in}}{\pgfqpoint{0.791248in}{0.874581in}}%
\pgfpathcurveto{\pgfqpoint{0.780309in}{0.885520in}}{\pgfqpoint{0.765470in}{0.891667in}}{\pgfqpoint{0.750000in}{0.891667in}}%
\pgfpathcurveto{\pgfqpoint{0.734530in}{0.891667in}}{\pgfqpoint{0.719691in}{0.885520in}}{\pgfqpoint{0.708752in}{0.874581in}}%
\pgfpathcurveto{\pgfqpoint{0.697813in}{0.863642in}}{\pgfqpoint{0.691667in}{0.848804in}}{\pgfqpoint{0.691667in}{0.833333in}}%
\pgfpathcurveto{\pgfqpoint{0.691667in}{0.817863in}}{\pgfqpoint{0.697813in}{0.803025in}}{\pgfqpoint{0.708752in}{0.792085in}}%
\pgfpathcurveto{\pgfqpoint{0.719691in}{0.781146in}}{\pgfqpoint{0.734530in}{0.775000in}}{\pgfqpoint{0.750000in}{0.775000in}}%
\pgfpathclose%
\pgfpathmoveto{\pgfqpoint{0.750000in}{0.780833in}}%
\pgfpathcurveto{\pgfqpoint{0.750000in}{0.780833in}}{\pgfqpoint{0.736077in}{0.780833in}}{\pgfqpoint{0.722722in}{0.786365in}}%
\pgfpathcurveto{\pgfqpoint{0.712877in}{0.796210in}}{\pgfqpoint{0.703032in}{0.806055in}}{\pgfqpoint{0.697500in}{0.819410in}}%
\pgfpathcurveto{\pgfqpoint{0.697500in}{0.833333in}}{\pgfqpoint{0.697500in}{0.847256in}}{\pgfqpoint{0.703032in}{0.860611in}}%
\pgfpathcurveto{\pgfqpoint{0.712877in}{0.870456in}}{\pgfqpoint{0.722722in}{0.880302in}}{\pgfqpoint{0.736077in}{0.885833in}}%
\pgfpathcurveto{\pgfqpoint{0.750000in}{0.885833in}}{\pgfqpoint{0.763923in}{0.885833in}}{\pgfqpoint{0.777278in}{0.880302in}}%
\pgfpathcurveto{\pgfqpoint{0.787123in}{0.870456in}}{\pgfqpoint{0.796968in}{0.860611in}}{\pgfqpoint{0.802500in}{0.847256in}}%
\pgfpathcurveto{\pgfqpoint{0.802500in}{0.833333in}}{\pgfqpoint{0.802500in}{0.819410in}}{\pgfqpoint{0.796968in}{0.806055in}}%
\pgfpathcurveto{\pgfqpoint{0.787123in}{0.796210in}}{\pgfqpoint{0.777278in}{0.786365in}}{\pgfqpoint{0.763923in}{0.780833in}}%
\pgfpathclose%
\pgfpathmoveto{\pgfqpoint{0.916667in}{0.775000in}}%
\pgfpathcurveto{\pgfqpoint{0.932137in}{0.775000in}}{\pgfqpoint{0.946975in}{0.781146in}}{\pgfqpoint{0.957915in}{0.792085in}}%
\pgfpathcurveto{\pgfqpoint{0.968854in}{0.803025in}}{\pgfqpoint{0.975000in}{0.817863in}}{\pgfqpoint{0.975000in}{0.833333in}}%
\pgfpathcurveto{\pgfqpoint{0.975000in}{0.848804in}}{\pgfqpoint{0.968854in}{0.863642in}}{\pgfqpoint{0.957915in}{0.874581in}}%
\pgfpathcurveto{\pgfqpoint{0.946975in}{0.885520in}}{\pgfqpoint{0.932137in}{0.891667in}}{\pgfqpoint{0.916667in}{0.891667in}}%
\pgfpathcurveto{\pgfqpoint{0.901196in}{0.891667in}}{\pgfqpoint{0.886358in}{0.885520in}}{\pgfqpoint{0.875419in}{0.874581in}}%
\pgfpathcurveto{\pgfqpoint{0.864480in}{0.863642in}}{\pgfqpoint{0.858333in}{0.848804in}}{\pgfqpoint{0.858333in}{0.833333in}}%
\pgfpathcurveto{\pgfqpoint{0.858333in}{0.817863in}}{\pgfqpoint{0.864480in}{0.803025in}}{\pgfqpoint{0.875419in}{0.792085in}}%
\pgfpathcurveto{\pgfqpoint{0.886358in}{0.781146in}}{\pgfqpoint{0.901196in}{0.775000in}}{\pgfqpoint{0.916667in}{0.775000in}}%
\pgfpathclose%
\pgfpathmoveto{\pgfqpoint{0.916667in}{0.780833in}}%
\pgfpathcurveto{\pgfqpoint{0.916667in}{0.780833in}}{\pgfqpoint{0.902744in}{0.780833in}}{\pgfqpoint{0.889389in}{0.786365in}}%
\pgfpathcurveto{\pgfqpoint{0.879544in}{0.796210in}}{\pgfqpoint{0.869698in}{0.806055in}}{\pgfqpoint{0.864167in}{0.819410in}}%
\pgfpathcurveto{\pgfqpoint{0.864167in}{0.833333in}}{\pgfqpoint{0.864167in}{0.847256in}}{\pgfqpoint{0.869698in}{0.860611in}}%
\pgfpathcurveto{\pgfqpoint{0.879544in}{0.870456in}}{\pgfqpoint{0.889389in}{0.880302in}}{\pgfqpoint{0.902744in}{0.885833in}}%
\pgfpathcurveto{\pgfqpoint{0.916667in}{0.885833in}}{\pgfqpoint{0.930590in}{0.885833in}}{\pgfqpoint{0.943945in}{0.880302in}}%
\pgfpathcurveto{\pgfqpoint{0.953790in}{0.870456in}}{\pgfqpoint{0.963635in}{0.860611in}}{\pgfqpoint{0.969167in}{0.847256in}}%
\pgfpathcurveto{\pgfqpoint{0.969167in}{0.833333in}}{\pgfqpoint{0.969167in}{0.819410in}}{\pgfqpoint{0.963635in}{0.806055in}}%
\pgfpathcurveto{\pgfqpoint{0.953790in}{0.796210in}}{\pgfqpoint{0.943945in}{0.786365in}}{\pgfqpoint{0.930590in}{0.780833in}}%
\pgfpathclose%
\pgfpathmoveto{\pgfqpoint{0.000000in}{0.941667in}}%
\pgfpathcurveto{\pgfqpoint{0.015470in}{0.941667in}}{\pgfqpoint{0.030309in}{0.947813in}}{\pgfqpoint{0.041248in}{0.958752in}}%
\pgfpathcurveto{\pgfqpoint{0.052187in}{0.969691in}}{\pgfqpoint{0.058333in}{0.984530in}}{\pgfqpoint{0.058333in}{1.000000in}}%
\pgfpathcurveto{\pgfqpoint{0.058333in}{1.015470in}}{\pgfqpoint{0.052187in}{1.030309in}}{\pgfqpoint{0.041248in}{1.041248in}}%
\pgfpathcurveto{\pgfqpoint{0.030309in}{1.052187in}}{\pgfqpoint{0.015470in}{1.058333in}}{\pgfqpoint{0.000000in}{1.058333in}}%
\pgfpathcurveto{\pgfqpoint{-0.015470in}{1.058333in}}{\pgfqpoint{-0.030309in}{1.052187in}}{\pgfqpoint{-0.041248in}{1.041248in}}%
\pgfpathcurveto{\pgfqpoint{-0.052187in}{1.030309in}}{\pgfqpoint{-0.058333in}{1.015470in}}{\pgfqpoint{-0.058333in}{1.000000in}}%
\pgfpathcurveto{\pgfqpoint{-0.058333in}{0.984530in}}{\pgfqpoint{-0.052187in}{0.969691in}}{\pgfqpoint{-0.041248in}{0.958752in}}%
\pgfpathcurveto{\pgfqpoint{-0.030309in}{0.947813in}}{\pgfqpoint{-0.015470in}{0.941667in}}{\pgfqpoint{0.000000in}{0.941667in}}%
\pgfpathclose%
\pgfpathmoveto{\pgfqpoint{0.000000in}{0.947500in}}%
\pgfpathcurveto{\pgfqpoint{0.000000in}{0.947500in}}{\pgfqpoint{-0.013923in}{0.947500in}}{\pgfqpoint{-0.027278in}{0.953032in}}%
\pgfpathcurveto{\pgfqpoint{-0.037123in}{0.962877in}}{\pgfqpoint{-0.046968in}{0.972722in}}{\pgfqpoint{-0.052500in}{0.986077in}}%
\pgfpathcurveto{\pgfqpoint{-0.052500in}{1.000000in}}{\pgfqpoint{-0.052500in}{1.013923in}}{\pgfqpoint{-0.046968in}{1.027278in}}%
\pgfpathcurveto{\pgfqpoint{-0.037123in}{1.037123in}}{\pgfqpoint{-0.027278in}{1.046968in}}{\pgfqpoint{-0.013923in}{1.052500in}}%
\pgfpathcurveto{\pgfqpoint{0.000000in}{1.052500in}}{\pgfqpoint{0.013923in}{1.052500in}}{\pgfqpoint{0.027278in}{1.046968in}}%
\pgfpathcurveto{\pgfqpoint{0.037123in}{1.037123in}}{\pgfqpoint{0.046968in}{1.027278in}}{\pgfqpoint{0.052500in}{1.013923in}}%
\pgfpathcurveto{\pgfqpoint{0.052500in}{1.000000in}}{\pgfqpoint{0.052500in}{0.986077in}}{\pgfqpoint{0.046968in}{0.972722in}}%
\pgfpathcurveto{\pgfqpoint{0.037123in}{0.962877in}}{\pgfqpoint{0.027278in}{0.953032in}}{\pgfqpoint{0.013923in}{0.947500in}}%
\pgfpathclose%
\pgfpathmoveto{\pgfqpoint{0.166667in}{0.941667in}}%
\pgfpathcurveto{\pgfqpoint{0.182137in}{0.941667in}}{\pgfqpoint{0.196975in}{0.947813in}}{\pgfqpoint{0.207915in}{0.958752in}}%
\pgfpathcurveto{\pgfqpoint{0.218854in}{0.969691in}}{\pgfqpoint{0.225000in}{0.984530in}}{\pgfqpoint{0.225000in}{1.000000in}}%
\pgfpathcurveto{\pgfqpoint{0.225000in}{1.015470in}}{\pgfqpoint{0.218854in}{1.030309in}}{\pgfqpoint{0.207915in}{1.041248in}}%
\pgfpathcurveto{\pgfqpoint{0.196975in}{1.052187in}}{\pgfqpoint{0.182137in}{1.058333in}}{\pgfqpoint{0.166667in}{1.058333in}}%
\pgfpathcurveto{\pgfqpoint{0.151196in}{1.058333in}}{\pgfqpoint{0.136358in}{1.052187in}}{\pgfqpoint{0.125419in}{1.041248in}}%
\pgfpathcurveto{\pgfqpoint{0.114480in}{1.030309in}}{\pgfqpoint{0.108333in}{1.015470in}}{\pgfqpoint{0.108333in}{1.000000in}}%
\pgfpathcurveto{\pgfqpoint{0.108333in}{0.984530in}}{\pgfqpoint{0.114480in}{0.969691in}}{\pgfqpoint{0.125419in}{0.958752in}}%
\pgfpathcurveto{\pgfqpoint{0.136358in}{0.947813in}}{\pgfqpoint{0.151196in}{0.941667in}}{\pgfqpoint{0.166667in}{0.941667in}}%
\pgfpathclose%
\pgfpathmoveto{\pgfqpoint{0.166667in}{0.947500in}}%
\pgfpathcurveto{\pgfqpoint{0.166667in}{0.947500in}}{\pgfqpoint{0.152744in}{0.947500in}}{\pgfqpoint{0.139389in}{0.953032in}}%
\pgfpathcurveto{\pgfqpoint{0.129544in}{0.962877in}}{\pgfqpoint{0.119698in}{0.972722in}}{\pgfqpoint{0.114167in}{0.986077in}}%
\pgfpathcurveto{\pgfqpoint{0.114167in}{1.000000in}}{\pgfqpoint{0.114167in}{1.013923in}}{\pgfqpoint{0.119698in}{1.027278in}}%
\pgfpathcurveto{\pgfqpoint{0.129544in}{1.037123in}}{\pgfqpoint{0.139389in}{1.046968in}}{\pgfqpoint{0.152744in}{1.052500in}}%
\pgfpathcurveto{\pgfqpoint{0.166667in}{1.052500in}}{\pgfqpoint{0.180590in}{1.052500in}}{\pgfqpoint{0.193945in}{1.046968in}}%
\pgfpathcurveto{\pgfqpoint{0.203790in}{1.037123in}}{\pgfqpoint{0.213635in}{1.027278in}}{\pgfqpoint{0.219167in}{1.013923in}}%
\pgfpathcurveto{\pgfqpoint{0.219167in}{1.000000in}}{\pgfqpoint{0.219167in}{0.986077in}}{\pgfqpoint{0.213635in}{0.972722in}}%
\pgfpathcurveto{\pgfqpoint{0.203790in}{0.962877in}}{\pgfqpoint{0.193945in}{0.953032in}}{\pgfqpoint{0.180590in}{0.947500in}}%
\pgfpathclose%
\pgfpathmoveto{\pgfqpoint{0.333333in}{0.941667in}}%
\pgfpathcurveto{\pgfqpoint{0.348804in}{0.941667in}}{\pgfqpoint{0.363642in}{0.947813in}}{\pgfqpoint{0.374581in}{0.958752in}}%
\pgfpathcurveto{\pgfqpoint{0.385520in}{0.969691in}}{\pgfqpoint{0.391667in}{0.984530in}}{\pgfqpoint{0.391667in}{1.000000in}}%
\pgfpathcurveto{\pgfqpoint{0.391667in}{1.015470in}}{\pgfqpoint{0.385520in}{1.030309in}}{\pgfqpoint{0.374581in}{1.041248in}}%
\pgfpathcurveto{\pgfqpoint{0.363642in}{1.052187in}}{\pgfqpoint{0.348804in}{1.058333in}}{\pgfqpoint{0.333333in}{1.058333in}}%
\pgfpathcurveto{\pgfqpoint{0.317863in}{1.058333in}}{\pgfqpoint{0.303025in}{1.052187in}}{\pgfqpoint{0.292085in}{1.041248in}}%
\pgfpathcurveto{\pgfqpoint{0.281146in}{1.030309in}}{\pgfqpoint{0.275000in}{1.015470in}}{\pgfqpoint{0.275000in}{1.000000in}}%
\pgfpathcurveto{\pgfqpoint{0.275000in}{0.984530in}}{\pgfqpoint{0.281146in}{0.969691in}}{\pgfqpoint{0.292085in}{0.958752in}}%
\pgfpathcurveto{\pgfqpoint{0.303025in}{0.947813in}}{\pgfqpoint{0.317863in}{0.941667in}}{\pgfqpoint{0.333333in}{0.941667in}}%
\pgfpathclose%
\pgfpathmoveto{\pgfqpoint{0.333333in}{0.947500in}}%
\pgfpathcurveto{\pgfqpoint{0.333333in}{0.947500in}}{\pgfqpoint{0.319410in}{0.947500in}}{\pgfqpoint{0.306055in}{0.953032in}}%
\pgfpathcurveto{\pgfqpoint{0.296210in}{0.962877in}}{\pgfqpoint{0.286365in}{0.972722in}}{\pgfqpoint{0.280833in}{0.986077in}}%
\pgfpathcurveto{\pgfqpoint{0.280833in}{1.000000in}}{\pgfqpoint{0.280833in}{1.013923in}}{\pgfqpoint{0.286365in}{1.027278in}}%
\pgfpathcurveto{\pgfqpoint{0.296210in}{1.037123in}}{\pgfqpoint{0.306055in}{1.046968in}}{\pgfqpoint{0.319410in}{1.052500in}}%
\pgfpathcurveto{\pgfqpoint{0.333333in}{1.052500in}}{\pgfqpoint{0.347256in}{1.052500in}}{\pgfqpoint{0.360611in}{1.046968in}}%
\pgfpathcurveto{\pgfqpoint{0.370456in}{1.037123in}}{\pgfqpoint{0.380302in}{1.027278in}}{\pgfqpoint{0.385833in}{1.013923in}}%
\pgfpathcurveto{\pgfqpoint{0.385833in}{1.000000in}}{\pgfqpoint{0.385833in}{0.986077in}}{\pgfqpoint{0.380302in}{0.972722in}}%
\pgfpathcurveto{\pgfqpoint{0.370456in}{0.962877in}}{\pgfqpoint{0.360611in}{0.953032in}}{\pgfqpoint{0.347256in}{0.947500in}}%
\pgfpathclose%
\pgfpathmoveto{\pgfqpoint{0.500000in}{0.941667in}}%
\pgfpathcurveto{\pgfqpoint{0.515470in}{0.941667in}}{\pgfqpoint{0.530309in}{0.947813in}}{\pgfqpoint{0.541248in}{0.958752in}}%
\pgfpathcurveto{\pgfqpoint{0.552187in}{0.969691in}}{\pgfqpoint{0.558333in}{0.984530in}}{\pgfqpoint{0.558333in}{1.000000in}}%
\pgfpathcurveto{\pgfqpoint{0.558333in}{1.015470in}}{\pgfqpoint{0.552187in}{1.030309in}}{\pgfqpoint{0.541248in}{1.041248in}}%
\pgfpathcurveto{\pgfqpoint{0.530309in}{1.052187in}}{\pgfqpoint{0.515470in}{1.058333in}}{\pgfqpoint{0.500000in}{1.058333in}}%
\pgfpathcurveto{\pgfqpoint{0.484530in}{1.058333in}}{\pgfqpoint{0.469691in}{1.052187in}}{\pgfqpoint{0.458752in}{1.041248in}}%
\pgfpathcurveto{\pgfqpoint{0.447813in}{1.030309in}}{\pgfqpoint{0.441667in}{1.015470in}}{\pgfqpoint{0.441667in}{1.000000in}}%
\pgfpathcurveto{\pgfqpoint{0.441667in}{0.984530in}}{\pgfqpoint{0.447813in}{0.969691in}}{\pgfqpoint{0.458752in}{0.958752in}}%
\pgfpathcurveto{\pgfqpoint{0.469691in}{0.947813in}}{\pgfqpoint{0.484530in}{0.941667in}}{\pgfqpoint{0.500000in}{0.941667in}}%
\pgfpathclose%
\pgfpathmoveto{\pgfqpoint{0.500000in}{0.947500in}}%
\pgfpathcurveto{\pgfqpoint{0.500000in}{0.947500in}}{\pgfqpoint{0.486077in}{0.947500in}}{\pgfqpoint{0.472722in}{0.953032in}}%
\pgfpathcurveto{\pgfqpoint{0.462877in}{0.962877in}}{\pgfqpoint{0.453032in}{0.972722in}}{\pgfqpoint{0.447500in}{0.986077in}}%
\pgfpathcurveto{\pgfqpoint{0.447500in}{1.000000in}}{\pgfqpoint{0.447500in}{1.013923in}}{\pgfqpoint{0.453032in}{1.027278in}}%
\pgfpathcurveto{\pgfqpoint{0.462877in}{1.037123in}}{\pgfqpoint{0.472722in}{1.046968in}}{\pgfqpoint{0.486077in}{1.052500in}}%
\pgfpathcurveto{\pgfqpoint{0.500000in}{1.052500in}}{\pgfqpoint{0.513923in}{1.052500in}}{\pgfqpoint{0.527278in}{1.046968in}}%
\pgfpathcurveto{\pgfqpoint{0.537123in}{1.037123in}}{\pgfqpoint{0.546968in}{1.027278in}}{\pgfqpoint{0.552500in}{1.013923in}}%
\pgfpathcurveto{\pgfqpoint{0.552500in}{1.000000in}}{\pgfqpoint{0.552500in}{0.986077in}}{\pgfqpoint{0.546968in}{0.972722in}}%
\pgfpathcurveto{\pgfqpoint{0.537123in}{0.962877in}}{\pgfqpoint{0.527278in}{0.953032in}}{\pgfqpoint{0.513923in}{0.947500in}}%
\pgfpathclose%
\pgfpathmoveto{\pgfqpoint{0.666667in}{0.941667in}}%
\pgfpathcurveto{\pgfqpoint{0.682137in}{0.941667in}}{\pgfqpoint{0.696975in}{0.947813in}}{\pgfqpoint{0.707915in}{0.958752in}}%
\pgfpathcurveto{\pgfqpoint{0.718854in}{0.969691in}}{\pgfqpoint{0.725000in}{0.984530in}}{\pgfqpoint{0.725000in}{1.000000in}}%
\pgfpathcurveto{\pgfqpoint{0.725000in}{1.015470in}}{\pgfqpoint{0.718854in}{1.030309in}}{\pgfqpoint{0.707915in}{1.041248in}}%
\pgfpathcurveto{\pgfqpoint{0.696975in}{1.052187in}}{\pgfqpoint{0.682137in}{1.058333in}}{\pgfqpoint{0.666667in}{1.058333in}}%
\pgfpathcurveto{\pgfqpoint{0.651196in}{1.058333in}}{\pgfqpoint{0.636358in}{1.052187in}}{\pgfqpoint{0.625419in}{1.041248in}}%
\pgfpathcurveto{\pgfqpoint{0.614480in}{1.030309in}}{\pgfqpoint{0.608333in}{1.015470in}}{\pgfqpoint{0.608333in}{1.000000in}}%
\pgfpathcurveto{\pgfqpoint{0.608333in}{0.984530in}}{\pgfqpoint{0.614480in}{0.969691in}}{\pgfqpoint{0.625419in}{0.958752in}}%
\pgfpathcurveto{\pgfqpoint{0.636358in}{0.947813in}}{\pgfqpoint{0.651196in}{0.941667in}}{\pgfqpoint{0.666667in}{0.941667in}}%
\pgfpathclose%
\pgfpathmoveto{\pgfqpoint{0.666667in}{0.947500in}}%
\pgfpathcurveto{\pgfqpoint{0.666667in}{0.947500in}}{\pgfqpoint{0.652744in}{0.947500in}}{\pgfqpoint{0.639389in}{0.953032in}}%
\pgfpathcurveto{\pgfqpoint{0.629544in}{0.962877in}}{\pgfqpoint{0.619698in}{0.972722in}}{\pgfqpoint{0.614167in}{0.986077in}}%
\pgfpathcurveto{\pgfqpoint{0.614167in}{1.000000in}}{\pgfqpoint{0.614167in}{1.013923in}}{\pgfqpoint{0.619698in}{1.027278in}}%
\pgfpathcurveto{\pgfqpoint{0.629544in}{1.037123in}}{\pgfqpoint{0.639389in}{1.046968in}}{\pgfqpoint{0.652744in}{1.052500in}}%
\pgfpathcurveto{\pgfqpoint{0.666667in}{1.052500in}}{\pgfqpoint{0.680590in}{1.052500in}}{\pgfqpoint{0.693945in}{1.046968in}}%
\pgfpathcurveto{\pgfqpoint{0.703790in}{1.037123in}}{\pgfqpoint{0.713635in}{1.027278in}}{\pgfqpoint{0.719167in}{1.013923in}}%
\pgfpathcurveto{\pgfqpoint{0.719167in}{1.000000in}}{\pgfqpoint{0.719167in}{0.986077in}}{\pgfqpoint{0.713635in}{0.972722in}}%
\pgfpathcurveto{\pgfqpoint{0.703790in}{0.962877in}}{\pgfqpoint{0.693945in}{0.953032in}}{\pgfqpoint{0.680590in}{0.947500in}}%
\pgfpathclose%
\pgfpathmoveto{\pgfqpoint{0.833333in}{0.941667in}}%
\pgfpathcurveto{\pgfqpoint{0.848804in}{0.941667in}}{\pgfqpoint{0.863642in}{0.947813in}}{\pgfqpoint{0.874581in}{0.958752in}}%
\pgfpathcurveto{\pgfqpoint{0.885520in}{0.969691in}}{\pgfqpoint{0.891667in}{0.984530in}}{\pgfqpoint{0.891667in}{1.000000in}}%
\pgfpathcurveto{\pgfqpoint{0.891667in}{1.015470in}}{\pgfqpoint{0.885520in}{1.030309in}}{\pgfqpoint{0.874581in}{1.041248in}}%
\pgfpathcurveto{\pgfqpoint{0.863642in}{1.052187in}}{\pgfqpoint{0.848804in}{1.058333in}}{\pgfqpoint{0.833333in}{1.058333in}}%
\pgfpathcurveto{\pgfqpoint{0.817863in}{1.058333in}}{\pgfqpoint{0.803025in}{1.052187in}}{\pgfqpoint{0.792085in}{1.041248in}}%
\pgfpathcurveto{\pgfqpoint{0.781146in}{1.030309in}}{\pgfqpoint{0.775000in}{1.015470in}}{\pgfqpoint{0.775000in}{1.000000in}}%
\pgfpathcurveto{\pgfqpoint{0.775000in}{0.984530in}}{\pgfqpoint{0.781146in}{0.969691in}}{\pgfqpoint{0.792085in}{0.958752in}}%
\pgfpathcurveto{\pgfqpoint{0.803025in}{0.947813in}}{\pgfqpoint{0.817863in}{0.941667in}}{\pgfqpoint{0.833333in}{0.941667in}}%
\pgfpathclose%
\pgfpathmoveto{\pgfqpoint{0.833333in}{0.947500in}}%
\pgfpathcurveto{\pgfqpoint{0.833333in}{0.947500in}}{\pgfqpoint{0.819410in}{0.947500in}}{\pgfqpoint{0.806055in}{0.953032in}}%
\pgfpathcurveto{\pgfqpoint{0.796210in}{0.962877in}}{\pgfqpoint{0.786365in}{0.972722in}}{\pgfqpoint{0.780833in}{0.986077in}}%
\pgfpathcurveto{\pgfqpoint{0.780833in}{1.000000in}}{\pgfqpoint{0.780833in}{1.013923in}}{\pgfqpoint{0.786365in}{1.027278in}}%
\pgfpathcurveto{\pgfqpoint{0.796210in}{1.037123in}}{\pgfqpoint{0.806055in}{1.046968in}}{\pgfqpoint{0.819410in}{1.052500in}}%
\pgfpathcurveto{\pgfqpoint{0.833333in}{1.052500in}}{\pgfqpoint{0.847256in}{1.052500in}}{\pgfqpoint{0.860611in}{1.046968in}}%
\pgfpathcurveto{\pgfqpoint{0.870456in}{1.037123in}}{\pgfqpoint{0.880302in}{1.027278in}}{\pgfqpoint{0.885833in}{1.013923in}}%
\pgfpathcurveto{\pgfqpoint{0.885833in}{1.000000in}}{\pgfqpoint{0.885833in}{0.986077in}}{\pgfqpoint{0.880302in}{0.972722in}}%
\pgfpathcurveto{\pgfqpoint{0.870456in}{0.962877in}}{\pgfqpoint{0.860611in}{0.953032in}}{\pgfqpoint{0.847256in}{0.947500in}}%
\pgfpathclose%
\pgfpathmoveto{\pgfqpoint{1.000000in}{0.941667in}}%
\pgfpathcurveto{\pgfqpoint{1.015470in}{0.941667in}}{\pgfqpoint{1.030309in}{0.947813in}}{\pgfqpoint{1.041248in}{0.958752in}}%
\pgfpathcurveto{\pgfqpoint{1.052187in}{0.969691in}}{\pgfqpoint{1.058333in}{0.984530in}}{\pgfqpoint{1.058333in}{1.000000in}}%
\pgfpathcurveto{\pgfqpoint{1.058333in}{1.015470in}}{\pgfqpoint{1.052187in}{1.030309in}}{\pgfqpoint{1.041248in}{1.041248in}}%
\pgfpathcurveto{\pgfqpoint{1.030309in}{1.052187in}}{\pgfqpoint{1.015470in}{1.058333in}}{\pgfqpoint{1.000000in}{1.058333in}}%
\pgfpathcurveto{\pgfqpoint{0.984530in}{1.058333in}}{\pgfqpoint{0.969691in}{1.052187in}}{\pgfqpoint{0.958752in}{1.041248in}}%
\pgfpathcurveto{\pgfqpoint{0.947813in}{1.030309in}}{\pgfqpoint{0.941667in}{1.015470in}}{\pgfqpoint{0.941667in}{1.000000in}}%
\pgfpathcurveto{\pgfqpoint{0.941667in}{0.984530in}}{\pgfqpoint{0.947813in}{0.969691in}}{\pgfqpoint{0.958752in}{0.958752in}}%
\pgfpathcurveto{\pgfqpoint{0.969691in}{0.947813in}}{\pgfqpoint{0.984530in}{0.941667in}}{\pgfqpoint{1.000000in}{0.941667in}}%
\pgfpathclose%
\pgfpathmoveto{\pgfqpoint{1.000000in}{0.947500in}}%
\pgfpathcurveto{\pgfqpoint{1.000000in}{0.947500in}}{\pgfqpoint{0.986077in}{0.947500in}}{\pgfqpoint{0.972722in}{0.953032in}}%
\pgfpathcurveto{\pgfqpoint{0.962877in}{0.962877in}}{\pgfqpoint{0.953032in}{0.972722in}}{\pgfqpoint{0.947500in}{0.986077in}}%
\pgfpathcurveto{\pgfqpoint{0.947500in}{1.000000in}}{\pgfqpoint{0.947500in}{1.013923in}}{\pgfqpoint{0.953032in}{1.027278in}}%
\pgfpathcurveto{\pgfqpoint{0.962877in}{1.037123in}}{\pgfqpoint{0.972722in}{1.046968in}}{\pgfqpoint{0.986077in}{1.052500in}}%
\pgfpathcurveto{\pgfqpoint{1.000000in}{1.052500in}}{\pgfqpoint{1.013923in}{1.052500in}}{\pgfqpoint{1.027278in}{1.046968in}}%
\pgfpathcurveto{\pgfqpoint{1.037123in}{1.037123in}}{\pgfqpoint{1.046968in}{1.027278in}}{\pgfqpoint{1.052500in}{1.013923in}}%
\pgfpathcurveto{\pgfqpoint{1.052500in}{1.000000in}}{\pgfqpoint{1.052500in}{0.986077in}}{\pgfqpoint{1.046968in}{0.972722in}}%
\pgfpathcurveto{\pgfqpoint{1.037123in}{0.962877in}}{\pgfqpoint{1.027278in}{0.953032in}}{\pgfqpoint{1.013923in}{0.947500in}}%
\pgfpathclose%
\pgfusepath{stroke}%
\end{pgfscope}%
}%
\pgfsys@transformshift{6.128174in}{6.550538in}%
\pgfsys@useobject{currentpattern}{}%
\pgfsys@transformshift{1in}{0in}%
\pgfsys@transformshift{-1in}{0in}%
\pgfsys@transformshift{0in}{1in}%
\pgfsys@useobject{currentpattern}{}%
\pgfsys@transformshift{1in}{0in}%
\pgfsys@transformshift{-1in}{0in}%
\pgfsys@transformshift{0in}{1in}%
\pgfsys@useobject{currentpattern}{}%
\pgfsys@transformshift{1in}{0in}%
\pgfsys@transformshift{-1in}{0in}%
\pgfsys@transformshift{0in}{1in}%
\end{pgfscope}%
\begin{pgfscope}%
\pgfpathrectangle{\pgfqpoint{1.090674in}{0.637495in}}{\pgfqpoint{9.300000in}{9.060000in}}%
\pgfusepath{clip}%
\pgfsetbuttcap%
\pgfsetmiterjoin%
\definecolor{currentfill}{rgb}{0.890196,0.466667,0.760784}%
\pgfsetfillcolor{currentfill}%
\pgfsetfillopacity{0.990000}%
\pgfsetlinewidth{0.000000pt}%
\definecolor{currentstroke}{rgb}{0.000000,0.000000,0.000000}%
\pgfsetstrokecolor{currentstroke}%
\pgfsetstrokeopacity{0.990000}%
\pgfsetdash{}{0pt}%
\pgfpathmoveto{\pgfqpoint{7.678174in}{6.807627in}}%
\pgfpathlineto{\pgfqpoint{8.453174in}{6.807627in}}%
\pgfpathlineto{\pgfqpoint{8.453174in}{8.920924in}}%
\pgfpathlineto{\pgfqpoint{7.678174in}{8.920924in}}%
\pgfpathclose%
\pgfusepath{fill}%
\end{pgfscope}%
\begin{pgfscope}%
\pgfsetbuttcap%
\pgfsetmiterjoin%
\definecolor{currentfill}{rgb}{0.890196,0.466667,0.760784}%
\pgfsetfillcolor{currentfill}%
\pgfsetfillopacity{0.990000}%
\pgfsetlinewidth{0.000000pt}%
\definecolor{currentstroke}{rgb}{0.000000,0.000000,0.000000}%
\pgfsetstrokecolor{currentstroke}%
\pgfsetstrokeopacity{0.990000}%
\pgfsetdash{}{0pt}%
\pgfpathrectangle{\pgfqpoint{1.090674in}{0.637495in}}{\pgfqpoint{9.300000in}{9.060000in}}%
\pgfusepath{clip}%
\pgfpathmoveto{\pgfqpoint{7.678174in}{6.807627in}}%
\pgfpathlineto{\pgfqpoint{8.453174in}{6.807627in}}%
\pgfpathlineto{\pgfqpoint{8.453174in}{8.920924in}}%
\pgfpathlineto{\pgfqpoint{7.678174in}{8.920924in}}%
\pgfpathclose%
\pgfusepath{clip}%
\pgfsys@defobject{currentpattern}{\pgfqpoint{0in}{0in}}{\pgfqpoint{1in}{1in}}{%
\begin{pgfscope}%
\pgfpathrectangle{\pgfqpoint{0in}{0in}}{\pgfqpoint{1in}{1in}}%
\pgfusepath{clip}%
\pgfpathmoveto{\pgfqpoint{0.000000in}{-0.058333in}}%
\pgfpathcurveto{\pgfqpoint{0.015470in}{-0.058333in}}{\pgfqpoint{0.030309in}{-0.052187in}}{\pgfqpoint{0.041248in}{-0.041248in}}%
\pgfpathcurveto{\pgfqpoint{0.052187in}{-0.030309in}}{\pgfqpoint{0.058333in}{-0.015470in}}{\pgfqpoint{0.058333in}{0.000000in}}%
\pgfpathcurveto{\pgfqpoint{0.058333in}{0.015470in}}{\pgfqpoint{0.052187in}{0.030309in}}{\pgfqpoint{0.041248in}{0.041248in}}%
\pgfpathcurveto{\pgfqpoint{0.030309in}{0.052187in}}{\pgfqpoint{0.015470in}{0.058333in}}{\pgfqpoint{0.000000in}{0.058333in}}%
\pgfpathcurveto{\pgfqpoint{-0.015470in}{0.058333in}}{\pgfqpoint{-0.030309in}{0.052187in}}{\pgfqpoint{-0.041248in}{0.041248in}}%
\pgfpathcurveto{\pgfqpoint{-0.052187in}{0.030309in}}{\pgfqpoint{-0.058333in}{0.015470in}}{\pgfqpoint{-0.058333in}{0.000000in}}%
\pgfpathcurveto{\pgfqpoint{-0.058333in}{-0.015470in}}{\pgfqpoint{-0.052187in}{-0.030309in}}{\pgfqpoint{-0.041248in}{-0.041248in}}%
\pgfpathcurveto{\pgfqpoint{-0.030309in}{-0.052187in}}{\pgfqpoint{-0.015470in}{-0.058333in}}{\pgfqpoint{0.000000in}{-0.058333in}}%
\pgfpathclose%
\pgfpathmoveto{\pgfqpoint{0.000000in}{-0.052500in}}%
\pgfpathcurveto{\pgfqpoint{0.000000in}{-0.052500in}}{\pgfqpoint{-0.013923in}{-0.052500in}}{\pgfqpoint{-0.027278in}{-0.046968in}}%
\pgfpathcurveto{\pgfqpoint{-0.037123in}{-0.037123in}}{\pgfqpoint{-0.046968in}{-0.027278in}}{\pgfqpoint{-0.052500in}{-0.013923in}}%
\pgfpathcurveto{\pgfqpoint{-0.052500in}{0.000000in}}{\pgfqpoint{-0.052500in}{0.013923in}}{\pgfqpoint{-0.046968in}{0.027278in}}%
\pgfpathcurveto{\pgfqpoint{-0.037123in}{0.037123in}}{\pgfqpoint{-0.027278in}{0.046968in}}{\pgfqpoint{-0.013923in}{0.052500in}}%
\pgfpathcurveto{\pgfqpoint{0.000000in}{0.052500in}}{\pgfqpoint{0.013923in}{0.052500in}}{\pgfqpoint{0.027278in}{0.046968in}}%
\pgfpathcurveto{\pgfqpoint{0.037123in}{0.037123in}}{\pgfqpoint{0.046968in}{0.027278in}}{\pgfqpoint{0.052500in}{0.013923in}}%
\pgfpathcurveto{\pgfqpoint{0.052500in}{0.000000in}}{\pgfqpoint{0.052500in}{-0.013923in}}{\pgfqpoint{0.046968in}{-0.027278in}}%
\pgfpathcurveto{\pgfqpoint{0.037123in}{-0.037123in}}{\pgfqpoint{0.027278in}{-0.046968in}}{\pgfqpoint{0.013923in}{-0.052500in}}%
\pgfpathclose%
\pgfpathmoveto{\pgfqpoint{0.166667in}{-0.058333in}}%
\pgfpathcurveto{\pgfqpoint{0.182137in}{-0.058333in}}{\pgfqpoint{0.196975in}{-0.052187in}}{\pgfqpoint{0.207915in}{-0.041248in}}%
\pgfpathcurveto{\pgfqpoint{0.218854in}{-0.030309in}}{\pgfqpoint{0.225000in}{-0.015470in}}{\pgfqpoint{0.225000in}{0.000000in}}%
\pgfpathcurveto{\pgfqpoint{0.225000in}{0.015470in}}{\pgfqpoint{0.218854in}{0.030309in}}{\pgfqpoint{0.207915in}{0.041248in}}%
\pgfpathcurveto{\pgfqpoint{0.196975in}{0.052187in}}{\pgfqpoint{0.182137in}{0.058333in}}{\pgfqpoint{0.166667in}{0.058333in}}%
\pgfpathcurveto{\pgfqpoint{0.151196in}{0.058333in}}{\pgfqpoint{0.136358in}{0.052187in}}{\pgfqpoint{0.125419in}{0.041248in}}%
\pgfpathcurveto{\pgfqpoint{0.114480in}{0.030309in}}{\pgfqpoint{0.108333in}{0.015470in}}{\pgfqpoint{0.108333in}{0.000000in}}%
\pgfpathcurveto{\pgfqpoint{0.108333in}{-0.015470in}}{\pgfqpoint{0.114480in}{-0.030309in}}{\pgfqpoint{0.125419in}{-0.041248in}}%
\pgfpathcurveto{\pgfqpoint{0.136358in}{-0.052187in}}{\pgfqpoint{0.151196in}{-0.058333in}}{\pgfqpoint{0.166667in}{-0.058333in}}%
\pgfpathclose%
\pgfpathmoveto{\pgfqpoint{0.166667in}{-0.052500in}}%
\pgfpathcurveto{\pgfqpoint{0.166667in}{-0.052500in}}{\pgfqpoint{0.152744in}{-0.052500in}}{\pgfqpoint{0.139389in}{-0.046968in}}%
\pgfpathcurveto{\pgfqpoint{0.129544in}{-0.037123in}}{\pgfqpoint{0.119698in}{-0.027278in}}{\pgfqpoint{0.114167in}{-0.013923in}}%
\pgfpathcurveto{\pgfqpoint{0.114167in}{0.000000in}}{\pgfqpoint{0.114167in}{0.013923in}}{\pgfqpoint{0.119698in}{0.027278in}}%
\pgfpathcurveto{\pgfqpoint{0.129544in}{0.037123in}}{\pgfqpoint{0.139389in}{0.046968in}}{\pgfqpoint{0.152744in}{0.052500in}}%
\pgfpathcurveto{\pgfqpoint{0.166667in}{0.052500in}}{\pgfqpoint{0.180590in}{0.052500in}}{\pgfqpoint{0.193945in}{0.046968in}}%
\pgfpathcurveto{\pgfqpoint{0.203790in}{0.037123in}}{\pgfqpoint{0.213635in}{0.027278in}}{\pgfqpoint{0.219167in}{0.013923in}}%
\pgfpathcurveto{\pgfqpoint{0.219167in}{0.000000in}}{\pgfqpoint{0.219167in}{-0.013923in}}{\pgfqpoint{0.213635in}{-0.027278in}}%
\pgfpathcurveto{\pgfqpoint{0.203790in}{-0.037123in}}{\pgfqpoint{0.193945in}{-0.046968in}}{\pgfqpoint{0.180590in}{-0.052500in}}%
\pgfpathclose%
\pgfpathmoveto{\pgfqpoint{0.333333in}{-0.058333in}}%
\pgfpathcurveto{\pgfqpoint{0.348804in}{-0.058333in}}{\pgfqpoint{0.363642in}{-0.052187in}}{\pgfqpoint{0.374581in}{-0.041248in}}%
\pgfpathcurveto{\pgfqpoint{0.385520in}{-0.030309in}}{\pgfqpoint{0.391667in}{-0.015470in}}{\pgfqpoint{0.391667in}{0.000000in}}%
\pgfpathcurveto{\pgfqpoint{0.391667in}{0.015470in}}{\pgfqpoint{0.385520in}{0.030309in}}{\pgfqpoint{0.374581in}{0.041248in}}%
\pgfpathcurveto{\pgfqpoint{0.363642in}{0.052187in}}{\pgfqpoint{0.348804in}{0.058333in}}{\pgfqpoint{0.333333in}{0.058333in}}%
\pgfpathcurveto{\pgfqpoint{0.317863in}{0.058333in}}{\pgfqpoint{0.303025in}{0.052187in}}{\pgfqpoint{0.292085in}{0.041248in}}%
\pgfpathcurveto{\pgfqpoint{0.281146in}{0.030309in}}{\pgfqpoint{0.275000in}{0.015470in}}{\pgfqpoint{0.275000in}{0.000000in}}%
\pgfpathcurveto{\pgfqpoint{0.275000in}{-0.015470in}}{\pgfqpoint{0.281146in}{-0.030309in}}{\pgfqpoint{0.292085in}{-0.041248in}}%
\pgfpathcurveto{\pgfqpoint{0.303025in}{-0.052187in}}{\pgfqpoint{0.317863in}{-0.058333in}}{\pgfqpoint{0.333333in}{-0.058333in}}%
\pgfpathclose%
\pgfpathmoveto{\pgfqpoint{0.333333in}{-0.052500in}}%
\pgfpathcurveto{\pgfqpoint{0.333333in}{-0.052500in}}{\pgfqpoint{0.319410in}{-0.052500in}}{\pgfqpoint{0.306055in}{-0.046968in}}%
\pgfpathcurveto{\pgfqpoint{0.296210in}{-0.037123in}}{\pgfqpoint{0.286365in}{-0.027278in}}{\pgfqpoint{0.280833in}{-0.013923in}}%
\pgfpathcurveto{\pgfqpoint{0.280833in}{0.000000in}}{\pgfqpoint{0.280833in}{0.013923in}}{\pgfqpoint{0.286365in}{0.027278in}}%
\pgfpathcurveto{\pgfqpoint{0.296210in}{0.037123in}}{\pgfqpoint{0.306055in}{0.046968in}}{\pgfqpoint{0.319410in}{0.052500in}}%
\pgfpathcurveto{\pgfqpoint{0.333333in}{0.052500in}}{\pgfqpoint{0.347256in}{0.052500in}}{\pgfqpoint{0.360611in}{0.046968in}}%
\pgfpathcurveto{\pgfqpoint{0.370456in}{0.037123in}}{\pgfqpoint{0.380302in}{0.027278in}}{\pgfqpoint{0.385833in}{0.013923in}}%
\pgfpathcurveto{\pgfqpoint{0.385833in}{0.000000in}}{\pgfqpoint{0.385833in}{-0.013923in}}{\pgfqpoint{0.380302in}{-0.027278in}}%
\pgfpathcurveto{\pgfqpoint{0.370456in}{-0.037123in}}{\pgfqpoint{0.360611in}{-0.046968in}}{\pgfqpoint{0.347256in}{-0.052500in}}%
\pgfpathclose%
\pgfpathmoveto{\pgfqpoint{0.500000in}{-0.058333in}}%
\pgfpathcurveto{\pgfqpoint{0.515470in}{-0.058333in}}{\pgfqpoint{0.530309in}{-0.052187in}}{\pgfqpoint{0.541248in}{-0.041248in}}%
\pgfpathcurveto{\pgfqpoint{0.552187in}{-0.030309in}}{\pgfqpoint{0.558333in}{-0.015470in}}{\pgfqpoint{0.558333in}{0.000000in}}%
\pgfpathcurveto{\pgfqpoint{0.558333in}{0.015470in}}{\pgfqpoint{0.552187in}{0.030309in}}{\pgfqpoint{0.541248in}{0.041248in}}%
\pgfpathcurveto{\pgfqpoint{0.530309in}{0.052187in}}{\pgfqpoint{0.515470in}{0.058333in}}{\pgfqpoint{0.500000in}{0.058333in}}%
\pgfpathcurveto{\pgfqpoint{0.484530in}{0.058333in}}{\pgfqpoint{0.469691in}{0.052187in}}{\pgfqpoint{0.458752in}{0.041248in}}%
\pgfpathcurveto{\pgfqpoint{0.447813in}{0.030309in}}{\pgfqpoint{0.441667in}{0.015470in}}{\pgfqpoint{0.441667in}{0.000000in}}%
\pgfpathcurveto{\pgfqpoint{0.441667in}{-0.015470in}}{\pgfqpoint{0.447813in}{-0.030309in}}{\pgfqpoint{0.458752in}{-0.041248in}}%
\pgfpathcurveto{\pgfqpoint{0.469691in}{-0.052187in}}{\pgfqpoint{0.484530in}{-0.058333in}}{\pgfqpoint{0.500000in}{-0.058333in}}%
\pgfpathclose%
\pgfpathmoveto{\pgfqpoint{0.500000in}{-0.052500in}}%
\pgfpathcurveto{\pgfqpoint{0.500000in}{-0.052500in}}{\pgfqpoint{0.486077in}{-0.052500in}}{\pgfqpoint{0.472722in}{-0.046968in}}%
\pgfpathcurveto{\pgfqpoint{0.462877in}{-0.037123in}}{\pgfqpoint{0.453032in}{-0.027278in}}{\pgfqpoint{0.447500in}{-0.013923in}}%
\pgfpathcurveto{\pgfqpoint{0.447500in}{0.000000in}}{\pgfqpoint{0.447500in}{0.013923in}}{\pgfqpoint{0.453032in}{0.027278in}}%
\pgfpathcurveto{\pgfqpoint{0.462877in}{0.037123in}}{\pgfqpoint{0.472722in}{0.046968in}}{\pgfqpoint{0.486077in}{0.052500in}}%
\pgfpathcurveto{\pgfqpoint{0.500000in}{0.052500in}}{\pgfqpoint{0.513923in}{0.052500in}}{\pgfqpoint{0.527278in}{0.046968in}}%
\pgfpathcurveto{\pgfqpoint{0.537123in}{0.037123in}}{\pgfqpoint{0.546968in}{0.027278in}}{\pgfqpoint{0.552500in}{0.013923in}}%
\pgfpathcurveto{\pgfqpoint{0.552500in}{0.000000in}}{\pgfqpoint{0.552500in}{-0.013923in}}{\pgfqpoint{0.546968in}{-0.027278in}}%
\pgfpathcurveto{\pgfqpoint{0.537123in}{-0.037123in}}{\pgfqpoint{0.527278in}{-0.046968in}}{\pgfqpoint{0.513923in}{-0.052500in}}%
\pgfpathclose%
\pgfpathmoveto{\pgfqpoint{0.666667in}{-0.058333in}}%
\pgfpathcurveto{\pgfqpoint{0.682137in}{-0.058333in}}{\pgfqpoint{0.696975in}{-0.052187in}}{\pgfqpoint{0.707915in}{-0.041248in}}%
\pgfpathcurveto{\pgfqpoint{0.718854in}{-0.030309in}}{\pgfqpoint{0.725000in}{-0.015470in}}{\pgfqpoint{0.725000in}{0.000000in}}%
\pgfpathcurveto{\pgfqpoint{0.725000in}{0.015470in}}{\pgfqpoint{0.718854in}{0.030309in}}{\pgfqpoint{0.707915in}{0.041248in}}%
\pgfpathcurveto{\pgfqpoint{0.696975in}{0.052187in}}{\pgfqpoint{0.682137in}{0.058333in}}{\pgfqpoint{0.666667in}{0.058333in}}%
\pgfpathcurveto{\pgfqpoint{0.651196in}{0.058333in}}{\pgfqpoint{0.636358in}{0.052187in}}{\pgfqpoint{0.625419in}{0.041248in}}%
\pgfpathcurveto{\pgfqpoint{0.614480in}{0.030309in}}{\pgfqpoint{0.608333in}{0.015470in}}{\pgfqpoint{0.608333in}{0.000000in}}%
\pgfpathcurveto{\pgfqpoint{0.608333in}{-0.015470in}}{\pgfqpoint{0.614480in}{-0.030309in}}{\pgfqpoint{0.625419in}{-0.041248in}}%
\pgfpathcurveto{\pgfqpoint{0.636358in}{-0.052187in}}{\pgfqpoint{0.651196in}{-0.058333in}}{\pgfqpoint{0.666667in}{-0.058333in}}%
\pgfpathclose%
\pgfpathmoveto{\pgfqpoint{0.666667in}{-0.052500in}}%
\pgfpathcurveto{\pgfqpoint{0.666667in}{-0.052500in}}{\pgfqpoint{0.652744in}{-0.052500in}}{\pgfqpoint{0.639389in}{-0.046968in}}%
\pgfpathcurveto{\pgfqpoint{0.629544in}{-0.037123in}}{\pgfqpoint{0.619698in}{-0.027278in}}{\pgfqpoint{0.614167in}{-0.013923in}}%
\pgfpathcurveto{\pgfqpoint{0.614167in}{0.000000in}}{\pgfqpoint{0.614167in}{0.013923in}}{\pgfqpoint{0.619698in}{0.027278in}}%
\pgfpathcurveto{\pgfqpoint{0.629544in}{0.037123in}}{\pgfqpoint{0.639389in}{0.046968in}}{\pgfqpoint{0.652744in}{0.052500in}}%
\pgfpathcurveto{\pgfqpoint{0.666667in}{0.052500in}}{\pgfqpoint{0.680590in}{0.052500in}}{\pgfqpoint{0.693945in}{0.046968in}}%
\pgfpathcurveto{\pgfqpoint{0.703790in}{0.037123in}}{\pgfqpoint{0.713635in}{0.027278in}}{\pgfqpoint{0.719167in}{0.013923in}}%
\pgfpathcurveto{\pgfqpoint{0.719167in}{0.000000in}}{\pgfqpoint{0.719167in}{-0.013923in}}{\pgfqpoint{0.713635in}{-0.027278in}}%
\pgfpathcurveto{\pgfqpoint{0.703790in}{-0.037123in}}{\pgfqpoint{0.693945in}{-0.046968in}}{\pgfqpoint{0.680590in}{-0.052500in}}%
\pgfpathclose%
\pgfpathmoveto{\pgfqpoint{0.833333in}{-0.058333in}}%
\pgfpathcurveto{\pgfqpoint{0.848804in}{-0.058333in}}{\pgfqpoint{0.863642in}{-0.052187in}}{\pgfqpoint{0.874581in}{-0.041248in}}%
\pgfpathcurveto{\pgfqpoint{0.885520in}{-0.030309in}}{\pgfqpoint{0.891667in}{-0.015470in}}{\pgfqpoint{0.891667in}{0.000000in}}%
\pgfpathcurveto{\pgfqpoint{0.891667in}{0.015470in}}{\pgfqpoint{0.885520in}{0.030309in}}{\pgfqpoint{0.874581in}{0.041248in}}%
\pgfpathcurveto{\pgfqpoint{0.863642in}{0.052187in}}{\pgfqpoint{0.848804in}{0.058333in}}{\pgfqpoint{0.833333in}{0.058333in}}%
\pgfpathcurveto{\pgfqpoint{0.817863in}{0.058333in}}{\pgfqpoint{0.803025in}{0.052187in}}{\pgfqpoint{0.792085in}{0.041248in}}%
\pgfpathcurveto{\pgfqpoint{0.781146in}{0.030309in}}{\pgfqpoint{0.775000in}{0.015470in}}{\pgfqpoint{0.775000in}{0.000000in}}%
\pgfpathcurveto{\pgfqpoint{0.775000in}{-0.015470in}}{\pgfqpoint{0.781146in}{-0.030309in}}{\pgfqpoint{0.792085in}{-0.041248in}}%
\pgfpathcurveto{\pgfqpoint{0.803025in}{-0.052187in}}{\pgfqpoint{0.817863in}{-0.058333in}}{\pgfqpoint{0.833333in}{-0.058333in}}%
\pgfpathclose%
\pgfpathmoveto{\pgfqpoint{0.833333in}{-0.052500in}}%
\pgfpathcurveto{\pgfqpoint{0.833333in}{-0.052500in}}{\pgfqpoint{0.819410in}{-0.052500in}}{\pgfqpoint{0.806055in}{-0.046968in}}%
\pgfpathcurveto{\pgfqpoint{0.796210in}{-0.037123in}}{\pgfqpoint{0.786365in}{-0.027278in}}{\pgfqpoint{0.780833in}{-0.013923in}}%
\pgfpathcurveto{\pgfqpoint{0.780833in}{0.000000in}}{\pgfqpoint{0.780833in}{0.013923in}}{\pgfqpoint{0.786365in}{0.027278in}}%
\pgfpathcurveto{\pgfqpoint{0.796210in}{0.037123in}}{\pgfqpoint{0.806055in}{0.046968in}}{\pgfqpoint{0.819410in}{0.052500in}}%
\pgfpathcurveto{\pgfqpoint{0.833333in}{0.052500in}}{\pgfqpoint{0.847256in}{0.052500in}}{\pgfqpoint{0.860611in}{0.046968in}}%
\pgfpathcurveto{\pgfqpoint{0.870456in}{0.037123in}}{\pgfqpoint{0.880302in}{0.027278in}}{\pgfqpoint{0.885833in}{0.013923in}}%
\pgfpathcurveto{\pgfqpoint{0.885833in}{0.000000in}}{\pgfqpoint{0.885833in}{-0.013923in}}{\pgfqpoint{0.880302in}{-0.027278in}}%
\pgfpathcurveto{\pgfqpoint{0.870456in}{-0.037123in}}{\pgfqpoint{0.860611in}{-0.046968in}}{\pgfqpoint{0.847256in}{-0.052500in}}%
\pgfpathclose%
\pgfpathmoveto{\pgfqpoint{1.000000in}{-0.058333in}}%
\pgfpathcurveto{\pgfqpoint{1.015470in}{-0.058333in}}{\pgfqpoint{1.030309in}{-0.052187in}}{\pgfqpoint{1.041248in}{-0.041248in}}%
\pgfpathcurveto{\pgfqpoint{1.052187in}{-0.030309in}}{\pgfqpoint{1.058333in}{-0.015470in}}{\pgfqpoint{1.058333in}{0.000000in}}%
\pgfpathcurveto{\pgfqpoint{1.058333in}{0.015470in}}{\pgfqpoint{1.052187in}{0.030309in}}{\pgfqpoint{1.041248in}{0.041248in}}%
\pgfpathcurveto{\pgfqpoint{1.030309in}{0.052187in}}{\pgfqpoint{1.015470in}{0.058333in}}{\pgfqpoint{1.000000in}{0.058333in}}%
\pgfpathcurveto{\pgfqpoint{0.984530in}{0.058333in}}{\pgfqpoint{0.969691in}{0.052187in}}{\pgfqpoint{0.958752in}{0.041248in}}%
\pgfpathcurveto{\pgfqpoint{0.947813in}{0.030309in}}{\pgfqpoint{0.941667in}{0.015470in}}{\pgfqpoint{0.941667in}{0.000000in}}%
\pgfpathcurveto{\pgfqpoint{0.941667in}{-0.015470in}}{\pgfqpoint{0.947813in}{-0.030309in}}{\pgfqpoint{0.958752in}{-0.041248in}}%
\pgfpathcurveto{\pgfqpoint{0.969691in}{-0.052187in}}{\pgfqpoint{0.984530in}{-0.058333in}}{\pgfqpoint{1.000000in}{-0.058333in}}%
\pgfpathclose%
\pgfpathmoveto{\pgfqpoint{1.000000in}{-0.052500in}}%
\pgfpathcurveto{\pgfqpoint{1.000000in}{-0.052500in}}{\pgfqpoint{0.986077in}{-0.052500in}}{\pgfqpoint{0.972722in}{-0.046968in}}%
\pgfpathcurveto{\pgfqpoint{0.962877in}{-0.037123in}}{\pgfqpoint{0.953032in}{-0.027278in}}{\pgfqpoint{0.947500in}{-0.013923in}}%
\pgfpathcurveto{\pgfqpoint{0.947500in}{0.000000in}}{\pgfqpoint{0.947500in}{0.013923in}}{\pgfqpoint{0.953032in}{0.027278in}}%
\pgfpathcurveto{\pgfqpoint{0.962877in}{0.037123in}}{\pgfqpoint{0.972722in}{0.046968in}}{\pgfqpoint{0.986077in}{0.052500in}}%
\pgfpathcurveto{\pgfqpoint{1.000000in}{0.052500in}}{\pgfqpoint{1.013923in}{0.052500in}}{\pgfqpoint{1.027278in}{0.046968in}}%
\pgfpathcurveto{\pgfqpoint{1.037123in}{0.037123in}}{\pgfqpoint{1.046968in}{0.027278in}}{\pgfqpoint{1.052500in}{0.013923in}}%
\pgfpathcurveto{\pgfqpoint{1.052500in}{0.000000in}}{\pgfqpoint{1.052500in}{-0.013923in}}{\pgfqpoint{1.046968in}{-0.027278in}}%
\pgfpathcurveto{\pgfqpoint{1.037123in}{-0.037123in}}{\pgfqpoint{1.027278in}{-0.046968in}}{\pgfqpoint{1.013923in}{-0.052500in}}%
\pgfpathclose%
\pgfpathmoveto{\pgfqpoint{0.083333in}{0.108333in}}%
\pgfpathcurveto{\pgfqpoint{0.098804in}{0.108333in}}{\pgfqpoint{0.113642in}{0.114480in}}{\pgfqpoint{0.124581in}{0.125419in}}%
\pgfpathcurveto{\pgfqpoint{0.135520in}{0.136358in}}{\pgfqpoint{0.141667in}{0.151196in}}{\pgfqpoint{0.141667in}{0.166667in}}%
\pgfpathcurveto{\pgfqpoint{0.141667in}{0.182137in}}{\pgfqpoint{0.135520in}{0.196975in}}{\pgfqpoint{0.124581in}{0.207915in}}%
\pgfpathcurveto{\pgfqpoint{0.113642in}{0.218854in}}{\pgfqpoint{0.098804in}{0.225000in}}{\pgfqpoint{0.083333in}{0.225000in}}%
\pgfpathcurveto{\pgfqpoint{0.067863in}{0.225000in}}{\pgfqpoint{0.053025in}{0.218854in}}{\pgfqpoint{0.042085in}{0.207915in}}%
\pgfpathcurveto{\pgfqpoint{0.031146in}{0.196975in}}{\pgfqpoint{0.025000in}{0.182137in}}{\pgfqpoint{0.025000in}{0.166667in}}%
\pgfpathcurveto{\pgfqpoint{0.025000in}{0.151196in}}{\pgfqpoint{0.031146in}{0.136358in}}{\pgfqpoint{0.042085in}{0.125419in}}%
\pgfpathcurveto{\pgfqpoint{0.053025in}{0.114480in}}{\pgfqpoint{0.067863in}{0.108333in}}{\pgfqpoint{0.083333in}{0.108333in}}%
\pgfpathclose%
\pgfpathmoveto{\pgfqpoint{0.083333in}{0.114167in}}%
\pgfpathcurveto{\pgfqpoint{0.083333in}{0.114167in}}{\pgfqpoint{0.069410in}{0.114167in}}{\pgfqpoint{0.056055in}{0.119698in}}%
\pgfpathcurveto{\pgfqpoint{0.046210in}{0.129544in}}{\pgfqpoint{0.036365in}{0.139389in}}{\pgfqpoint{0.030833in}{0.152744in}}%
\pgfpathcurveto{\pgfqpoint{0.030833in}{0.166667in}}{\pgfqpoint{0.030833in}{0.180590in}}{\pgfqpoint{0.036365in}{0.193945in}}%
\pgfpathcurveto{\pgfqpoint{0.046210in}{0.203790in}}{\pgfqpoint{0.056055in}{0.213635in}}{\pgfqpoint{0.069410in}{0.219167in}}%
\pgfpathcurveto{\pgfqpoint{0.083333in}{0.219167in}}{\pgfqpoint{0.097256in}{0.219167in}}{\pgfqpoint{0.110611in}{0.213635in}}%
\pgfpathcurveto{\pgfqpoint{0.120456in}{0.203790in}}{\pgfqpoint{0.130302in}{0.193945in}}{\pgfqpoint{0.135833in}{0.180590in}}%
\pgfpathcurveto{\pgfqpoint{0.135833in}{0.166667in}}{\pgfqpoint{0.135833in}{0.152744in}}{\pgfqpoint{0.130302in}{0.139389in}}%
\pgfpathcurveto{\pgfqpoint{0.120456in}{0.129544in}}{\pgfqpoint{0.110611in}{0.119698in}}{\pgfqpoint{0.097256in}{0.114167in}}%
\pgfpathclose%
\pgfpathmoveto{\pgfqpoint{0.250000in}{0.108333in}}%
\pgfpathcurveto{\pgfqpoint{0.265470in}{0.108333in}}{\pgfqpoint{0.280309in}{0.114480in}}{\pgfqpoint{0.291248in}{0.125419in}}%
\pgfpathcurveto{\pgfqpoint{0.302187in}{0.136358in}}{\pgfqpoint{0.308333in}{0.151196in}}{\pgfqpoint{0.308333in}{0.166667in}}%
\pgfpathcurveto{\pgfqpoint{0.308333in}{0.182137in}}{\pgfqpoint{0.302187in}{0.196975in}}{\pgfqpoint{0.291248in}{0.207915in}}%
\pgfpathcurveto{\pgfqpoint{0.280309in}{0.218854in}}{\pgfqpoint{0.265470in}{0.225000in}}{\pgfqpoint{0.250000in}{0.225000in}}%
\pgfpathcurveto{\pgfqpoint{0.234530in}{0.225000in}}{\pgfqpoint{0.219691in}{0.218854in}}{\pgfqpoint{0.208752in}{0.207915in}}%
\pgfpathcurveto{\pgfqpoint{0.197813in}{0.196975in}}{\pgfqpoint{0.191667in}{0.182137in}}{\pgfqpoint{0.191667in}{0.166667in}}%
\pgfpathcurveto{\pgfqpoint{0.191667in}{0.151196in}}{\pgfqpoint{0.197813in}{0.136358in}}{\pgfqpoint{0.208752in}{0.125419in}}%
\pgfpathcurveto{\pgfqpoint{0.219691in}{0.114480in}}{\pgfqpoint{0.234530in}{0.108333in}}{\pgfqpoint{0.250000in}{0.108333in}}%
\pgfpathclose%
\pgfpathmoveto{\pgfqpoint{0.250000in}{0.114167in}}%
\pgfpathcurveto{\pgfqpoint{0.250000in}{0.114167in}}{\pgfqpoint{0.236077in}{0.114167in}}{\pgfqpoint{0.222722in}{0.119698in}}%
\pgfpathcurveto{\pgfqpoint{0.212877in}{0.129544in}}{\pgfqpoint{0.203032in}{0.139389in}}{\pgfqpoint{0.197500in}{0.152744in}}%
\pgfpathcurveto{\pgfqpoint{0.197500in}{0.166667in}}{\pgfqpoint{0.197500in}{0.180590in}}{\pgfqpoint{0.203032in}{0.193945in}}%
\pgfpathcurveto{\pgfqpoint{0.212877in}{0.203790in}}{\pgfqpoint{0.222722in}{0.213635in}}{\pgfqpoint{0.236077in}{0.219167in}}%
\pgfpathcurveto{\pgfqpoint{0.250000in}{0.219167in}}{\pgfqpoint{0.263923in}{0.219167in}}{\pgfqpoint{0.277278in}{0.213635in}}%
\pgfpathcurveto{\pgfqpoint{0.287123in}{0.203790in}}{\pgfqpoint{0.296968in}{0.193945in}}{\pgfqpoint{0.302500in}{0.180590in}}%
\pgfpathcurveto{\pgfqpoint{0.302500in}{0.166667in}}{\pgfqpoint{0.302500in}{0.152744in}}{\pgfqpoint{0.296968in}{0.139389in}}%
\pgfpathcurveto{\pgfqpoint{0.287123in}{0.129544in}}{\pgfqpoint{0.277278in}{0.119698in}}{\pgfqpoint{0.263923in}{0.114167in}}%
\pgfpathclose%
\pgfpathmoveto{\pgfqpoint{0.416667in}{0.108333in}}%
\pgfpathcurveto{\pgfqpoint{0.432137in}{0.108333in}}{\pgfqpoint{0.446975in}{0.114480in}}{\pgfqpoint{0.457915in}{0.125419in}}%
\pgfpathcurveto{\pgfqpoint{0.468854in}{0.136358in}}{\pgfqpoint{0.475000in}{0.151196in}}{\pgfqpoint{0.475000in}{0.166667in}}%
\pgfpathcurveto{\pgfqpoint{0.475000in}{0.182137in}}{\pgfqpoint{0.468854in}{0.196975in}}{\pgfqpoint{0.457915in}{0.207915in}}%
\pgfpathcurveto{\pgfqpoint{0.446975in}{0.218854in}}{\pgfqpoint{0.432137in}{0.225000in}}{\pgfqpoint{0.416667in}{0.225000in}}%
\pgfpathcurveto{\pgfqpoint{0.401196in}{0.225000in}}{\pgfqpoint{0.386358in}{0.218854in}}{\pgfqpoint{0.375419in}{0.207915in}}%
\pgfpathcurveto{\pgfqpoint{0.364480in}{0.196975in}}{\pgfqpoint{0.358333in}{0.182137in}}{\pgfqpoint{0.358333in}{0.166667in}}%
\pgfpathcurveto{\pgfqpoint{0.358333in}{0.151196in}}{\pgfqpoint{0.364480in}{0.136358in}}{\pgfqpoint{0.375419in}{0.125419in}}%
\pgfpathcurveto{\pgfqpoint{0.386358in}{0.114480in}}{\pgfqpoint{0.401196in}{0.108333in}}{\pgfqpoint{0.416667in}{0.108333in}}%
\pgfpathclose%
\pgfpathmoveto{\pgfqpoint{0.416667in}{0.114167in}}%
\pgfpathcurveto{\pgfqpoint{0.416667in}{0.114167in}}{\pgfqpoint{0.402744in}{0.114167in}}{\pgfqpoint{0.389389in}{0.119698in}}%
\pgfpathcurveto{\pgfqpoint{0.379544in}{0.129544in}}{\pgfqpoint{0.369698in}{0.139389in}}{\pgfqpoint{0.364167in}{0.152744in}}%
\pgfpathcurveto{\pgfqpoint{0.364167in}{0.166667in}}{\pgfqpoint{0.364167in}{0.180590in}}{\pgfqpoint{0.369698in}{0.193945in}}%
\pgfpathcurveto{\pgfqpoint{0.379544in}{0.203790in}}{\pgfqpoint{0.389389in}{0.213635in}}{\pgfqpoint{0.402744in}{0.219167in}}%
\pgfpathcurveto{\pgfqpoint{0.416667in}{0.219167in}}{\pgfqpoint{0.430590in}{0.219167in}}{\pgfqpoint{0.443945in}{0.213635in}}%
\pgfpathcurveto{\pgfqpoint{0.453790in}{0.203790in}}{\pgfqpoint{0.463635in}{0.193945in}}{\pgfqpoint{0.469167in}{0.180590in}}%
\pgfpathcurveto{\pgfqpoint{0.469167in}{0.166667in}}{\pgfqpoint{0.469167in}{0.152744in}}{\pgfqpoint{0.463635in}{0.139389in}}%
\pgfpathcurveto{\pgfqpoint{0.453790in}{0.129544in}}{\pgfqpoint{0.443945in}{0.119698in}}{\pgfqpoint{0.430590in}{0.114167in}}%
\pgfpathclose%
\pgfpathmoveto{\pgfqpoint{0.583333in}{0.108333in}}%
\pgfpathcurveto{\pgfqpoint{0.598804in}{0.108333in}}{\pgfqpoint{0.613642in}{0.114480in}}{\pgfqpoint{0.624581in}{0.125419in}}%
\pgfpathcurveto{\pgfqpoint{0.635520in}{0.136358in}}{\pgfqpoint{0.641667in}{0.151196in}}{\pgfqpoint{0.641667in}{0.166667in}}%
\pgfpathcurveto{\pgfqpoint{0.641667in}{0.182137in}}{\pgfqpoint{0.635520in}{0.196975in}}{\pgfqpoint{0.624581in}{0.207915in}}%
\pgfpathcurveto{\pgfqpoint{0.613642in}{0.218854in}}{\pgfqpoint{0.598804in}{0.225000in}}{\pgfqpoint{0.583333in}{0.225000in}}%
\pgfpathcurveto{\pgfqpoint{0.567863in}{0.225000in}}{\pgfqpoint{0.553025in}{0.218854in}}{\pgfqpoint{0.542085in}{0.207915in}}%
\pgfpathcurveto{\pgfqpoint{0.531146in}{0.196975in}}{\pgfqpoint{0.525000in}{0.182137in}}{\pgfqpoint{0.525000in}{0.166667in}}%
\pgfpathcurveto{\pgfqpoint{0.525000in}{0.151196in}}{\pgfqpoint{0.531146in}{0.136358in}}{\pgfqpoint{0.542085in}{0.125419in}}%
\pgfpathcurveto{\pgfqpoint{0.553025in}{0.114480in}}{\pgfqpoint{0.567863in}{0.108333in}}{\pgfqpoint{0.583333in}{0.108333in}}%
\pgfpathclose%
\pgfpathmoveto{\pgfqpoint{0.583333in}{0.114167in}}%
\pgfpathcurveto{\pgfqpoint{0.583333in}{0.114167in}}{\pgfqpoint{0.569410in}{0.114167in}}{\pgfqpoint{0.556055in}{0.119698in}}%
\pgfpathcurveto{\pgfqpoint{0.546210in}{0.129544in}}{\pgfqpoint{0.536365in}{0.139389in}}{\pgfqpoint{0.530833in}{0.152744in}}%
\pgfpathcurveto{\pgfqpoint{0.530833in}{0.166667in}}{\pgfqpoint{0.530833in}{0.180590in}}{\pgfqpoint{0.536365in}{0.193945in}}%
\pgfpathcurveto{\pgfqpoint{0.546210in}{0.203790in}}{\pgfqpoint{0.556055in}{0.213635in}}{\pgfqpoint{0.569410in}{0.219167in}}%
\pgfpathcurveto{\pgfqpoint{0.583333in}{0.219167in}}{\pgfqpoint{0.597256in}{0.219167in}}{\pgfqpoint{0.610611in}{0.213635in}}%
\pgfpathcurveto{\pgfqpoint{0.620456in}{0.203790in}}{\pgfqpoint{0.630302in}{0.193945in}}{\pgfqpoint{0.635833in}{0.180590in}}%
\pgfpathcurveto{\pgfqpoint{0.635833in}{0.166667in}}{\pgfqpoint{0.635833in}{0.152744in}}{\pgfqpoint{0.630302in}{0.139389in}}%
\pgfpathcurveto{\pgfqpoint{0.620456in}{0.129544in}}{\pgfqpoint{0.610611in}{0.119698in}}{\pgfqpoint{0.597256in}{0.114167in}}%
\pgfpathclose%
\pgfpathmoveto{\pgfqpoint{0.750000in}{0.108333in}}%
\pgfpathcurveto{\pgfqpoint{0.765470in}{0.108333in}}{\pgfqpoint{0.780309in}{0.114480in}}{\pgfqpoint{0.791248in}{0.125419in}}%
\pgfpathcurveto{\pgfqpoint{0.802187in}{0.136358in}}{\pgfqpoint{0.808333in}{0.151196in}}{\pgfqpoint{0.808333in}{0.166667in}}%
\pgfpathcurveto{\pgfqpoint{0.808333in}{0.182137in}}{\pgfqpoint{0.802187in}{0.196975in}}{\pgfqpoint{0.791248in}{0.207915in}}%
\pgfpathcurveto{\pgfqpoint{0.780309in}{0.218854in}}{\pgfqpoint{0.765470in}{0.225000in}}{\pgfqpoint{0.750000in}{0.225000in}}%
\pgfpathcurveto{\pgfqpoint{0.734530in}{0.225000in}}{\pgfqpoint{0.719691in}{0.218854in}}{\pgfqpoint{0.708752in}{0.207915in}}%
\pgfpathcurveto{\pgfqpoint{0.697813in}{0.196975in}}{\pgfqpoint{0.691667in}{0.182137in}}{\pgfqpoint{0.691667in}{0.166667in}}%
\pgfpathcurveto{\pgfqpoint{0.691667in}{0.151196in}}{\pgfqpoint{0.697813in}{0.136358in}}{\pgfqpoint{0.708752in}{0.125419in}}%
\pgfpathcurveto{\pgfqpoint{0.719691in}{0.114480in}}{\pgfqpoint{0.734530in}{0.108333in}}{\pgfqpoint{0.750000in}{0.108333in}}%
\pgfpathclose%
\pgfpathmoveto{\pgfqpoint{0.750000in}{0.114167in}}%
\pgfpathcurveto{\pgfqpoint{0.750000in}{0.114167in}}{\pgfqpoint{0.736077in}{0.114167in}}{\pgfqpoint{0.722722in}{0.119698in}}%
\pgfpathcurveto{\pgfqpoint{0.712877in}{0.129544in}}{\pgfqpoint{0.703032in}{0.139389in}}{\pgfqpoint{0.697500in}{0.152744in}}%
\pgfpathcurveto{\pgfqpoint{0.697500in}{0.166667in}}{\pgfqpoint{0.697500in}{0.180590in}}{\pgfqpoint{0.703032in}{0.193945in}}%
\pgfpathcurveto{\pgfqpoint{0.712877in}{0.203790in}}{\pgfqpoint{0.722722in}{0.213635in}}{\pgfqpoint{0.736077in}{0.219167in}}%
\pgfpathcurveto{\pgfqpoint{0.750000in}{0.219167in}}{\pgfqpoint{0.763923in}{0.219167in}}{\pgfqpoint{0.777278in}{0.213635in}}%
\pgfpathcurveto{\pgfqpoint{0.787123in}{0.203790in}}{\pgfqpoint{0.796968in}{0.193945in}}{\pgfqpoint{0.802500in}{0.180590in}}%
\pgfpathcurveto{\pgfqpoint{0.802500in}{0.166667in}}{\pgfqpoint{0.802500in}{0.152744in}}{\pgfqpoint{0.796968in}{0.139389in}}%
\pgfpathcurveto{\pgfqpoint{0.787123in}{0.129544in}}{\pgfqpoint{0.777278in}{0.119698in}}{\pgfqpoint{0.763923in}{0.114167in}}%
\pgfpathclose%
\pgfpathmoveto{\pgfqpoint{0.916667in}{0.108333in}}%
\pgfpathcurveto{\pgfqpoint{0.932137in}{0.108333in}}{\pgfqpoint{0.946975in}{0.114480in}}{\pgfqpoint{0.957915in}{0.125419in}}%
\pgfpathcurveto{\pgfqpoint{0.968854in}{0.136358in}}{\pgfqpoint{0.975000in}{0.151196in}}{\pgfqpoint{0.975000in}{0.166667in}}%
\pgfpathcurveto{\pgfqpoint{0.975000in}{0.182137in}}{\pgfqpoint{0.968854in}{0.196975in}}{\pgfqpoint{0.957915in}{0.207915in}}%
\pgfpathcurveto{\pgfqpoint{0.946975in}{0.218854in}}{\pgfqpoint{0.932137in}{0.225000in}}{\pgfqpoint{0.916667in}{0.225000in}}%
\pgfpathcurveto{\pgfqpoint{0.901196in}{0.225000in}}{\pgfqpoint{0.886358in}{0.218854in}}{\pgfqpoint{0.875419in}{0.207915in}}%
\pgfpathcurveto{\pgfqpoint{0.864480in}{0.196975in}}{\pgfqpoint{0.858333in}{0.182137in}}{\pgfqpoint{0.858333in}{0.166667in}}%
\pgfpathcurveto{\pgfqpoint{0.858333in}{0.151196in}}{\pgfqpoint{0.864480in}{0.136358in}}{\pgfqpoint{0.875419in}{0.125419in}}%
\pgfpathcurveto{\pgfqpoint{0.886358in}{0.114480in}}{\pgfqpoint{0.901196in}{0.108333in}}{\pgfqpoint{0.916667in}{0.108333in}}%
\pgfpathclose%
\pgfpathmoveto{\pgfqpoint{0.916667in}{0.114167in}}%
\pgfpathcurveto{\pgfqpoint{0.916667in}{0.114167in}}{\pgfqpoint{0.902744in}{0.114167in}}{\pgfqpoint{0.889389in}{0.119698in}}%
\pgfpathcurveto{\pgfqpoint{0.879544in}{0.129544in}}{\pgfqpoint{0.869698in}{0.139389in}}{\pgfqpoint{0.864167in}{0.152744in}}%
\pgfpathcurveto{\pgfqpoint{0.864167in}{0.166667in}}{\pgfqpoint{0.864167in}{0.180590in}}{\pgfqpoint{0.869698in}{0.193945in}}%
\pgfpathcurveto{\pgfqpoint{0.879544in}{0.203790in}}{\pgfqpoint{0.889389in}{0.213635in}}{\pgfqpoint{0.902744in}{0.219167in}}%
\pgfpathcurveto{\pgfqpoint{0.916667in}{0.219167in}}{\pgfqpoint{0.930590in}{0.219167in}}{\pgfqpoint{0.943945in}{0.213635in}}%
\pgfpathcurveto{\pgfqpoint{0.953790in}{0.203790in}}{\pgfqpoint{0.963635in}{0.193945in}}{\pgfqpoint{0.969167in}{0.180590in}}%
\pgfpathcurveto{\pgfqpoint{0.969167in}{0.166667in}}{\pgfqpoint{0.969167in}{0.152744in}}{\pgfqpoint{0.963635in}{0.139389in}}%
\pgfpathcurveto{\pgfqpoint{0.953790in}{0.129544in}}{\pgfqpoint{0.943945in}{0.119698in}}{\pgfqpoint{0.930590in}{0.114167in}}%
\pgfpathclose%
\pgfpathmoveto{\pgfqpoint{0.000000in}{0.275000in}}%
\pgfpathcurveto{\pgfqpoint{0.015470in}{0.275000in}}{\pgfqpoint{0.030309in}{0.281146in}}{\pgfqpoint{0.041248in}{0.292085in}}%
\pgfpathcurveto{\pgfqpoint{0.052187in}{0.303025in}}{\pgfqpoint{0.058333in}{0.317863in}}{\pgfqpoint{0.058333in}{0.333333in}}%
\pgfpathcurveto{\pgfqpoint{0.058333in}{0.348804in}}{\pgfqpoint{0.052187in}{0.363642in}}{\pgfqpoint{0.041248in}{0.374581in}}%
\pgfpathcurveto{\pgfqpoint{0.030309in}{0.385520in}}{\pgfqpoint{0.015470in}{0.391667in}}{\pgfqpoint{0.000000in}{0.391667in}}%
\pgfpathcurveto{\pgfqpoint{-0.015470in}{0.391667in}}{\pgfqpoint{-0.030309in}{0.385520in}}{\pgfqpoint{-0.041248in}{0.374581in}}%
\pgfpathcurveto{\pgfqpoint{-0.052187in}{0.363642in}}{\pgfqpoint{-0.058333in}{0.348804in}}{\pgfqpoint{-0.058333in}{0.333333in}}%
\pgfpathcurveto{\pgfqpoint{-0.058333in}{0.317863in}}{\pgfqpoint{-0.052187in}{0.303025in}}{\pgfqpoint{-0.041248in}{0.292085in}}%
\pgfpathcurveto{\pgfqpoint{-0.030309in}{0.281146in}}{\pgfqpoint{-0.015470in}{0.275000in}}{\pgfqpoint{0.000000in}{0.275000in}}%
\pgfpathclose%
\pgfpathmoveto{\pgfqpoint{0.000000in}{0.280833in}}%
\pgfpathcurveto{\pgfqpoint{0.000000in}{0.280833in}}{\pgfqpoint{-0.013923in}{0.280833in}}{\pgfqpoint{-0.027278in}{0.286365in}}%
\pgfpathcurveto{\pgfqpoint{-0.037123in}{0.296210in}}{\pgfqpoint{-0.046968in}{0.306055in}}{\pgfqpoint{-0.052500in}{0.319410in}}%
\pgfpathcurveto{\pgfqpoint{-0.052500in}{0.333333in}}{\pgfqpoint{-0.052500in}{0.347256in}}{\pgfqpoint{-0.046968in}{0.360611in}}%
\pgfpathcurveto{\pgfqpoint{-0.037123in}{0.370456in}}{\pgfqpoint{-0.027278in}{0.380302in}}{\pgfqpoint{-0.013923in}{0.385833in}}%
\pgfpathcurveto{\pgfqpoint{0.000000in}{0.385833in}}{\pgfqpoint{0.013923in}{0.385833in}}{\pgfqpoint{0.027278in}{0.380302in}}%
\pgfpathcurveto{\pgfqpoint{0.037123in}{0.370456in}}{\pgfqpoint{0.046968in}{0.360611in}}{\pgfqpoint{0.052500in}{0.347256in}}%
\pgfpathcurveto{\pgfqpoint{0.052500in}{0.333333in}}{\pgfqpoint{0.052500in}{0.319410in}}{\pgfqpoint{0.046968in}{0.306055in}}%
\pgfpathcurveto{\pgfqpoint{0.037123in}{0.296210in}}{\pgfqpoint{0.027278in}{0.286365in}}{\pgfqpoint{0.013923in}{0.280833in}}%
\pgfpathclose%
\pgfpathmoveto{\pgfqpoint{0.166667in}{0.275000in}}%
\pgfpathcurveto{\pgfqpoint{0.182137in}{0.275000in}}{\pgfqpoint{0.196975in}{0.281146in}}{\pgfqpoint{0.207915in}{0.292085in}}%
\pgfpathcurveto{\pgfqpoint{0.218854in}{0.303025in}}{\pgfqpoint{0.225000in}{0.317863in}}{\pgfqpoint{0.225000in}{0.333333in}}%
\pgfpathcurveto{\pgfqpoint{0.225000in}{0.348804in}}{\pgfqpoint{0.218854in}{0.363642in}}{\pgfqpoint{0.207915in}{0.374581in}}%
\pgfpathcurveto{\pgfqpoint{0.196975in}{0.385520in}}{\pgfqpoint{0.182137in}{0.391667in}}{\pgfqpoint{0.166667in}{0.391667in}}%
\pgfpathcurveto{\pgfqpoint{0.151196in}{0.391667in}}{\pgfqpoint{0.136358in}{0.385520in}}{\pgfqpoint{0.125419in}{0.374581in}}%
\pgfpathcurveto{\pgfqpoint{0.114480in}{0.363642in}}{\pgfqpoint{0.108333in}{0.348804in}}{\pgfqpoint{0.108333in}{0.333333in}}%
\pgfpathcurveto{\pgfqpoint{0.108333in}{0.317863in}}{\pgfqpoint{0.114480in}{0.303025in}}{\pgfqpoint{0.125419in}{0.292085in}}%
\pgfpathcurveto{\pgfqpoint{0.136358in}{0.281146in}}{\pgfqpoint{0.151196in}{0.275000in}}{\pgfqpoint{0.166667in}{0.275000in}}%
\pgfpathclose%
\pgfpathmoveto{\pgfqpoint{0.166667in}{0.280833in}}%
\pgfpathcurveto{\pgfqpoint{0.166667in}{0.280833in}}{\pgfqpoint{0.152744in}{0.280833in}}{\pgfqpoint{0.139389in}{0.286365in}}%
\pgfpathcurveto{\pgfqpoint{0.129544in}{0.296210in}}{\pgfqpoint{0.119698in}{0.306055in}}{\pgfqpoint{0.114167in}{0.319410in}}%
\pgfpathcurveto{\pgfqpoint{0.114167in}{0.333333in}}{\pgfqpoint{0.114167in}{0.347256in}}{\pgfqpoint{0.119698in}{0.360611in}}%
\pgfpathcurveto{\pgfqpoint{0.129544in}{0.370456in}}{\pgfqpoint{0.139389in}{0.380302in}}{\pgfqpoint{0.152744in}{0.385833in}}%
\pgfpathcurveto{\pgfqpoint{0.166667in}{0.385833in}}{\pgfqpoint{0.180590in}{0.385833in}}{\pgfqpoint{0.193945in}{0.380302in}}%
\pgfpathcurveto{\pgfqpoint{0.203790in}{0.370456in}}{\pgfqpoint{0.213635in}{0.360611in}}{\pgfqpoint{0.219167in}{0.347256in}}%
\pgfpathcurveto{\pgfqpoint{0.219167in}{0.333333in}}{\pgfqpoint{0.219167in}{0.319410in}}{\pgfqpoint{0.213635in}{0.306055in}}%
\pgfpathcurveto{\pgfqpoint{0.203790in}{0.296210in}}{\pgfqpoint{0.193945in}{0.286365in}}{\pgfqpoint{0.180590in}{0.280833in}}%
\pgfpathclose%
\pgfpathmoveto{\pgfqpoint{0.333333in}{0.275000in}}%
\pgfpathcurveto{\pgfqpoint{0.348804in}{0.275000in}}{\pgfqpoint{0.363642in}{0.281146in}}{\pgfqpoint{0.374581in}{0.292085in}}%
\pgfpathcurveto{\pgfqpoint{0.385520in}{0.303025in}}{\pgfqpoint{0.391667in}{0.317863in}}{\pgfqpoint{0.391667in}{0.333333in}}%
\pgfpathcurveto{\pgfqpoint{0.391667in}{0.348804in}}{\pgfqpoint{0.385520in}{0.363642in}}{\pgfqpoint{0.374581in}{0.374581in}}%
\pgfpathcurveto{\pgfqpoint{0.363642in}{0.385520in}}{\pgfqpoint{0.348804in}{0.391667in}}{\pgfqpoint{0.333333in}{0.391667in}}%
\pgfpathcurveto{\pgfqpoint{0.317863in}{0.391667in}}{\pgfqpoint{0.303025in}{0.385520in}}{\pgfqpoint{0.292085in}{0.374581in}}%
\pgfpathcurveto{\pgfqpoint{0.281146in}{0.363642in}}{\pgfqpoint{0.275000in}{0.348804in}}{\pgfqpoint{0.275000in}{0.333333in}}%
\pgfpathcurveto{\pgfqpoint{0.275000in}{0.317863in}}{\pgfqpoint{0.281146in}{0.303025in}}{\pgfqpoint{0.292085in}{0.292085in}}%
\pgfpathcurveto{\pgfqpoint{0.303025in}{0.281146in}}{\pgfqpoint{0.317863in}{0.275000in}}{\pgfqpoint{0.333333in}{0.275000in}}%
\pgfpathclose%
\pgfpathmoveto{\pgfqpoint{0.333333in}{0.280833in}}%
\pgfpathcurveto{\pgfqpoint{0.333333in}{0.280833in}}{\pgfqpoint{0.319410in}{0.280833in}}{\pgfqpoint{0.306055in}{0.286365in}}%
\pgfpathcurveto{\pgfqpoint{0.296210in}{0.296210in}}{\pgfqpoint{0.286365in}{0.306055in}}{\pgfqpoint{0.280833in}{0.319410in}}%
\pgfpathcurveto{\pgfqpoint{0.280833in}{0.333333in}}{\pgfqpoint{0.280833in}{0.347256in}}{\pgfqpoint{0.286365in}{0.360611in}}%
\pgfpathcurveto{\pgfqpoint{0.296210in}{0.370456in}}{\pgfqpoint{0.306055in}{0.380302in}}{\pgfqpoint{0.319410in}{0.385833in}}%
\pgfpathcurveto{\pgfqpoint{0.333333in}{0.385833in}}{\pgfqpoint{0.347256in}{0.385833in}}{\pgfqpoint{0.360611in}{0.380302in}}%
\pgfpathcurveto{\pgfqpoint{0.370456in}{0.370456in}}{\pgfqpoint{0.380302in}{0.360611in}}{\pgfqpoint{0.385833in}{0.347256in}}%
\pgfpathcurveto{\pgfqpoint{0.385833in}{0.333333in}}{\pgfqpoint{0.385833in}{0.319410in}}{\pgfqpoint{0.380302in}{0.306055in}}%
\pgfpathcurveto{\pgfqpoint{0.370456in}{0.296210in}}{\pgfqpoint{0.360611in}{0.286365in}}{\pgfqpoint{0.347256in}{0.280833in}}%
\pgfpathclose%
\pgfpathmoveto{\pgfqpoint{0.500000in}{0.275000in}}%
\pgfpathcurveto{\pgfqpoint{0.515470in}{0.275000in}}{\pgfqpoint{0.530309in}{0.281146in}}{\pgfqpoint{0.541248in}{0.292085in}}%
\pgfpathcurveto{\pgfqpoint{0.552187in}{0.303025in}}{\pgfqpoint{0.558333in}{0.317863in}}{\pgfqpoint{0.558333in}{0.333333in}}%
\pgfpathcurveto{\pgfqpoint{0.558333in}{0.348804in}}{\pgfqpoint{0.552187in}{0.363642in}}{\pgfqpoint{0.541248in}{0.374581in}}%
\pgfpathcurveto{\pgfqpoint{0.530309in}{0.385520in}}{\pgfqpoint{0.515470in}{0.391667in}}{\pgfqpoint{0.500000in}{0.391667in}}%
\pgfpathcurveto{\pgfqpoint{0.484530in}{0.391667in}}{\pgfqpoint{0.469691in}{0.385520in}}{\pgfqpoint{0.458752in}{0.374581in}}%
\pgfpathcurveto{\pgfqpoint{0.447813in}{0.363642in}}{\pgfqpoint{0.441667in}{0.348804in}}{\pgfqpoint{0.441667in}{0.333333in}}%
\pgfpathcurveto{\pgfqpoint{0.441667in}{0.317863in}}{\pgfqpoint{0.447813in}{0.303025in}}{\pgfqpoint{0.458752in}{0.292085in}}%
\pgfpathcurveto{\pgfqpoint{0.469691in}{0.281146in}}{\pgfqpoint{0.484530in}{0.275000in}}{\pgfqpoint{0.500000in}{0.275000in}}%
\pgfpathclose%
\pgfpathmoveto{\pgfqpoint{0.500000in}{0.280833in}}%
\pgfpathcurveto{\pgfqpoint{0.500000in}{0.280833in}}{\pgfqpoint{0.486077in}{0.280833in}}{\pgfqpoint{0.472722in}{0.286365in}}%
\pgfpathcurveto{\pgfqpoint{0.462877in}{0.296210in}}{\pgfqpoint{0.453032in}{0.306055in}}{\pgfqpoint{0.447500in}{0.319410in}}%
\pgfpathcurveto{\pgfqpoint{0.447500in}{0.333333in}}{\pgfqpoint{0.447500in}{0.347256in}}{\pgfqpoint{0.453032in}{0.360611in}}%
\pgfpathcurveto{\pgfqpoint{0.462877in}{0.370456in}}{\pgfqpoint{0.472722in}{0.380302in}}{\pgfqpoint{0.486077in}{0.385833in}}%
\pgfpathcurveto{\pgfqpoint{0.500000in}{0.385833in}}{\pgfqpoint{0.513923in}{0.385833in}}{\pgfqpoint{0.527278in}{0.380302in}}%
\pgfpathcurveto{\pgfqpoint{0.537123in}{0.370456in}}{\pgfqpoint{0.546968in}{0.360611in}}{\pgfqpoint{0.552500in}{0.347256in}}%
\pgfpathcurveto{\pgfqpoint{0.552500in}{0.333333in}}{\pgfqpoint{0.552500in}{0.319410in}}{\pgfqpoint{0.546968in}{0.306055in}}%
\pgfpathcurveto{\pgfqpoint{0.537123in}{0.296210in}}{\pgfqpoint{0.527278in}{0.286365in}}{\pgfqpoint{0.513923in}{0.280833in}}%
\pgfpathclose%
\pgfpathmoveto{\pgfqpoint{0.666667in}{0.275000in}}%
\pgfpathcurveto{\pgfqpoint{0.682137in}{0.275000in}}{\pgfqpoint{0.696975in}{0.281146in}}{\pgfqpoint{0.707915in}{0.292085in}}%
\pgfpathcurveto{\pgfqpoint{0.718854in}{0.303025in}}{\pgfqpoint{0.725000in}{0.317863in}}{\pgfqpoint{0.725000in}{0.333333in}}%
\pgfpathcurveto{\pgfqpoint{0.725000in}{0.348804in}}{\pgfqpoint{0.718854in}{0.363642in}}{\pgfqpoint{0.707915in}{0.374581in}}%
\pgfpathcurveto{\pgfqpoint{0.696975in}{0.385520in}}{\pgfqpoint{0.682137in}{0.391667in}}{\pgfqpoint{0.666667in}{0.391667in}}%
\pgfpathcurveto{\pgfqpoint{0.651196in}{0.391667in}}{\pgfqpoint{0.636358in}{0.385520in}}{\pgfqpoint{0.625419in}{0.374581in}}%
\pgfpathcurveto{\pgfqpoint{0.614480in}{0.363642in}}{\pgfqpoint{0.608333in}{0.348804in}}{\pgfqpoint{0.608333in}{0.333333in}}%
\pgfpathcurveto{\pgfqpoint{0.608333in}{0.317863in}}{\pgfqpoint{0.614480in}{0.303025in}}{\pgfqpoint{0.625419in}{0.292085in}}%
\pgfpathcurveto{\pgfqpoint{0.636358in}{0.281146in}}{\pgfqpoint{0.651196in}{0.275000in}}{\pgfqpoint{0.666667in}{0.275000in}}%
\pgfpathclose%
\pgfpathmoveto{\pgfqpoint{0.666667in}{0.280833in}}%
\pgfpathcurveto{\pgfqpoint{0.666667in}{0.280833in}}{\pgfqpoint{0.652744in}{0.280833in}}{\pgfqpoint{0.639389in}{0.286365in}}%
\pgfpathcurveto{\pgfqpoint{0.629544in}{0.296210in}}{\pgfqpoint{0.619698in}{0.306055in}}{\pgfqpoint{0.614167in}{0.319410in}}%
\pgfpathcurveto{\pgfqpoint{0.614167in}{0.333333in}}{\pgfqpoint{0.614167in}{0.347256in}}{\pgfqpoint{0.619698in}{0.360611in}}%
\pgfpathcurveto{\pgfqpoint{0.629544in}{0.370456in}}{\pgfqpoint{0.639389in}{0.380302in}}{\pgfqpoint{0.652744in}{0.385833in}}%
\pgfpathcurveto{\pgfqpoint{0.666667in}{0.385833in}}{\pgfqpoint{0.680590in}{0.385833in}}{\pgfqpoint{0.693945in}{0.380302in}}%
\pgfpathcurveto{\pgfqpoint{0.703790in}{0.370456in}}{\pgfqpoint{0.713635in}{0.360611in}}{\pgfqpoint{0.719167in}{0.347256in}}%
\pgfpathcurveto{\pgfqpoint{0.719167in}{0.333333in}}{\pgfqpoint{0.719167in}{0.319410in}}{\pgfqpoint{0.713635in}{0.306055in}}%
\pgfpathcurveto{\pgfqpoint{0.703790in}{0.296210in}}{\pgfqpoint{0.693945in}{0.286365in}}{\pgfqpoint{0.680590in}{0.280833in}}%
\pgfpathclose%
\pgfpathmoveto{\pgfqpoint{0.833333in}{0.275000in}}%
\pgfpathcurveto{\pgfqpoint{0.848804in}{0.275000in}}{\pgfqpoint{0.863642in}{0.281146in}}{\pgfqpoint{0.874581in}{0.292085in}}%
\pgfpathcurveto{\pgfqpoint{0.885520in}{0.303025in}}{\pgfqpoint{0.891667in}{0.317863in}}{\pgfqpoint{0.891667in}{0.333333in}}%
\pgfpathcurveto{\pgfqpoint{0.891667in}{0.348804in}}{\pgfqpoint{0.885520in}{0.363642in}}{\pgfqpoint{0.874581in}{0.374581in}}%
\pgfpathcurveto{\pgfqpoint{0.863642in}{0.385520in}}{\pgfqpoint{0.848804in}{0.391667in}}{\pgfqpoint{0.833333in}{0.391667in}}%
\pgfpathcurveto{\pgfqpoint{0.817863in}{0.391667in}}{\pgfqpoint{0.803025in}{0.385520in}}{\pgfqpoint{0.792085in}{0.374581in}}%
\pgfpathcurveto{\pgfqpoint{0.781146in}{0.363642in}}{\pgfqpoint{0.775000in}{0.348804in}}{\pgfqpoint{0.775000in}{0.333333in}}%
\pgfpathcurveto{\pgfqpoint{0.775000in}{0.317863in}}{\pgfqpoint{0.781146in}{0.303025in}}{\pgfqpoint{0.792085in}{0.292085in}}%
\pgfpathcurveto{\pgfqpoint{0.803025in}{0.281146in}}{\pgfqpoint{0.817863in}{0.275000in}}{\pgfqpoint{0.833333in}{0.275000in}}%
\pgfpathclose%
\pgfpathmoveto{\pgfqpoint{0.833333in}{0.280833in}}%
\pgfpathcurveto{\pgfqpoint{0.833333in}{0.280833in}}{\pgfqpoint{0.819410in}{0.280833in}}{\pgfqpoint{0.806055in}{0.286365in}}%
\pgfpathcurveto{\pgfqpoint{0.796210in}{0.296210in}}{\pgfqpoint{0.786365in}{0.306055in}}{\pgfqpoint{0.780833in}{0.319410in}}%
\pgfpathcurveto{\pgfqpoint{0.780833in}{0.333333in}}{\pgfqpoint{0.780833in}{0.347256in}}{\pgfqpoint{0.786365in}{0.360611in}}%
\pgfpathcurveto{\pgfqpoint{0.796210in}{0.370456in}}{\pgfqpoint{0.806055in}{0.380302in}}{\pgfqpoint{0.819410in}{0.385833in}}%
\pgfpathcurveto{\pgfqpoint{0.833333in}{0.385833in}}{\pgfqpoint{0.847256in}{0.385833in}}{\pgfqpoint{0.860611in}{0.380302in}}%
\pgfpathcurveto{\pgfqpoint{0.870456in}{0.370456in}}{\pgfqpoint{0.880302in}{0.360611in}}{\pgfqpoint{0.885833in}{0.347256in}}%
\pgfpathcurveto{\pgfqpoint{0.885833in}{0.333333in}}{\pgfqpoint{0.885833in}{0.319410in}}{\pgfqpoint{0.880302in}{0.306055in}}%
\pgfpathcurveto{\pgfqpoint{0.870456in}{0.296210in}}{\pgfqpoint{0.860611in}{0.286365in}}{\pgfqpoint{0.847256in}{0.280833in}}%
\pgfpathclose%
\pgfpathmoveto{\pgfqpoint{1.000000in}{0.275000in}}%
\pgfpathcurveto{\pgfqpoint{1.015470in}{0.275000in}}{\pgfqpoint{1.030309in}{0.281146in}}{\pgfqpoint{1.041248in}{0.292085in}}%
\pgfpathcurveto{\pgfqpoint{1.052187in}{0.303025in}}{\pgfqpoint{1.058333in}{0.317863in}}{\pgfqpoint{1.058333in}{0.333333in}}%
\pgfpathcurveto{\pgfqpoint{1.058333in}{0.348804in}}{\pgfqpoint{1.052187in}{0.363642in}}{\pgfqpoint{1.041248in}{0.374581in}}%
\pgfpathcurveto{\pgfqpoint{1.030309in}{0.385520in}}{\pgfqpoint{1.015470in}{0.391667in}}{\pgfqpoint{1.000000in}{0.391667in}}%
\pgfpathcurveto{\pgfqpoint{0.984530in}{0.391667in}}{\pgfqpoint{0.969691in}{0.385520in}}{\pgfqpoint{0.958752in}{0.374581in}}%
\pgfpathcurveto{\pgfqpoint{0.947813in}{0.363642in}}{\pgfqpoint{0.941667in}{0.348804in}}{\pgfqpoint{0.941667in}{0.333333in}}%
\pgfpathcurveto{\pgfqpoint{0.941667in}{0.317863in}}{\pgfqpoint{0.947813in}{0.303025in}}{\pgfqpoint{0.958752in}{0.292085in}}%
\pgfpathcurveto{\pgfqpoint{0.969691in}{0.281146in}}{\pgfqpoint{0.984530in}{0.275000in}}{\pgfqpoint{1.000000in}{0.275000in}}%
\pgfpathclose%
\pgfpathmoveto{\pgfqpoint{1.000000in}{0.280833in}}%
\pgfpathcurveto{\pgfqpoint{1.000000in}{0.280833in}}{\pgfqpoint{0.986077in}{0.280833in}}{\pgfqpoint{0.972722in}{0.286365in}}%
\pgfpathcurveto{\pgfqpoint{0.962877in}{0.296210in}}{\pgfqpoint{0.953032in}{0.306055in}}{\pgfqpoint{0.947500in}{0.319410in}}%
\pgfpathcurveto{\pgfqpoint{0.947500in}{0.333333in}}{\pgfqpoint{0.947500in}{0.347256in}}{\pgfqpoint{0.953032in}{0.360611in}}%
\pgfpathcurveto{\pgfqpoint{0.962877in}{0.370456in}}{\pgfqpoint{0.972722in}{0.380302in}}{\pgfqpoint{0.986077in}{0.385833in}}%
\pgfpathcurveto{\pgfqpoint{1.000000in}{0.385833in}}{\pgfqpoint{1.013923in}{0.385833in}}{\pgfqpoint{1.027278in}{0.380302in}}%
\pgfpathcurveto{\pgfqpoint{1.037123in}{0.370456in}}{\pgfqpoint{1.046968in}{0.360611in}}{\pgfqpoint{1.052500in}{0.347256in}}%
\pgfpathcurveto{\pgfqpoint{1.052500in}{0.333333in}}{\pgfqpoint{1.052500in}{0.319410in}}{\pgfqpoint{1.046968in}{0.306055in}}%
\pgfpathcurveto{\pgfqpoint{1.037123in}{0.296210in}}{\pgfqpoint{1.027278in}{0.286365in}}{\pgfqpoint{1.013923in}{0.280833in}}%
\pgfpathclose%
\pgfpathmoveto{\pgfqpoint{0.083333in}{0.441667in}}%
\pgfpathcurveto{\pgfqpoint{0.098804in}{0.441667in}}{\pgfqpoint{0.113642in}{0.447813in}}{\pgfqpoint{0.124581in}{0.458752in}}%
\pgfpathcurveto{\pgfqpoint{0.135520in}{0.469691in}}{\pgfqpoint{0.141667in}{0.484530in}}{\pgfqpoint{0.141667in}{0.500000in}}%
\pgfpathcurveto{\pgfqpoint{0.141667in}{0.515470in}}{\pgfqpoint{0.135520in}{0.530309in}}{\pgfqpoint{0.124581in}{0.541248in}}%
\pgfpathcurveto{\pgfqpoint{0.113642in}{0.552187in}}{\pgfqpoint{0.098804in}{0.558333in}}{\pgfqpoint{0.083333in}{0.558333in}}%
\pgfpathcurveto{\pgfqpoint{0.067863in}{0.558333in}}{\pgfqpoint{0.053025in}{0.552187in}}{\pgfqpoint{0.042085in}{0.541248in}}%
\pgfpathcurveto{\pgfqpoint{0.031146in}{0.530309in}}{\pgfqpoint{0.025000in}{0.515470in}}{\pgfqpoint{0.025000in}{0.500000in}}%
\pgfpathcurveto{\pgfqpoint{0.025000in}{0.484530in}}{\pgfqpoint{0.031146in}{0.469691in}}{\pgfqpoint{0.042085in}{0.458752in}}%
\pgfpathcurveto{\pgfqpoint{0.053025in}{0.447813in}}{\pgfqpoint{0.067863in}{0.441667in}}{\pgfqpoint{0.083333in}{0.441667in}}%
\pgfpathclose%
\pgfpathmoveto{\pgfqpoint{0.083333in}{0.447500in}}%
\pgfpathcurveto{\pgfqpoint{0.083333in}{0.447500in}}{\pgfqpoint{0.069410in}{0.447500in}}{\pgfqpoint{0.056055in}{0.453032in}}%
\pgfpathcurveto{\pgfqpoint{0.046210in}{0.462877in}}{\pgfqpoint{0.036365in}{0.472722in}}{\pgfqpoint{0.030833in}{0.486077in}}%
\pgfpathcurveto{\pgfqpoint{0.030833in}{0.500000in}}{\pgfqpoint{0.030833in}{0.513923in}}{\pgfqpoint{0.036365in}{0.527278in}}%
\pgfpathcurveto{\pgfqpoint{0.046210in}{0.537123in}}{\pgfqpoint{0.056055in}{0.546968in}}{\pgfqpoint{0.069410in}{0.552500in}}%
\pgfpathcurveto{\pgfqpoint{0.083333in}{0.552500in}}{\pgfqpoint{0.097256in}{0.552500in}}{\pgfqpoint{0.110611in}{0.546968in}}%
\pgfpathcurveto{\pgfqpoint{0.120456in}{0.537123in}}{\pgfqpoint{0.130302in}{0.527278in}}{\pgfqpoint{0.135833in}{0.513923in}}%
\pgfpathcurveto{\pgfqpoint{0.135833in}{0.500000in}}{\pgfqpoint{0.135833in}{0.486077in}}{\pgfqpoint{0.130302in}{0.472722in}}%
\pgfpathcurveto{\pgfqpoint{0.120456in}{0.462877in}}{\pgfqpoint{0.110611in}{0.453032in}}{\pgfqpoint{0.097256in}{0.447500in}}%
\pgfpathclose%
\pgfpathmoveto{\pgfqpoint{0.250000in}{0.441667in}}%
\pgfpathcurveto{\pgfqpoint{0.265470in}{0.441667in}}{\pgfqpoint{0.280309in}{0.447813in}}{\pgfqpoint{0.291248in}{0.458752in}}%
\pgfpathcurveto{\pgfqpoint{0.302187in}{0.469691in}}{\pgfqpoint{0.308333in}{0.484530in}}{\pgfqpoint{0.308333in}{0.500000in}}%
\pgfpathcurveto{\pgfqpoint{0.308333in}{0.515470in}}{\pgfqpoint{0.302187in}{0.530309in}}{\pgfqpoint{0.291248in}{0.541248in}}%
\pgfpathcurveto{\pgfqpoint{0.280309in}{0.552187in}}{\pgfqpoint{0.265470in}{0.558333in}}{\pgfqpoint{0.250000in}{0.558333in}}%
\pgfpathcurveto{\pgfqpoint{0.234530in}{0.558333in}}{\pgfqpoint{0.219691in}{0.552187in}}{\pgfqpoint{0.208752in}{0.541248in}}%
\pgfpathcurveto{\pgfqpoint{0.197813in}{0.530309in}}{\pgfqpoint{0.191667in}{0.515470in}}{\pgfqpoint{0.191667in}{0.500000in}}%
\pgfpathcurveto{\pgfqpoint{0.191667in}{0.484530in}}{\pgfqpoint{0.197813in}{0.469691in}}{\pgfqpoint{0.208752in}{0.458752in}}%
\pgfpathcurveto{\pgfqpoint{0.219691in}{0.447813in}}{\pgfqpoint{0.234530in}{0.441667in}}{\pgfqpoint{0.250000in}{0.441667in}}%
\pgfpathclose%
\pgfpathmoveto{\pgfqpoint{0.250000in}{0.447500in}}%
\pgfpathcurveto{\pgfqpoint{0.250000in}{0.447500in}}{\pgfqpoint{0.236077in}{0.447500in}}{\pgfqpoint{0.222722in}{0.453032in}}%
\pgfpathcurveto{\pgfqpoint{0.212877in}{0.462877in}}{\pgfqpoint{0.203032in}{0.472722in}}{\pgfqpoint{0.197500in}{0.486077in}}%
\pgfpathcurveto{\pgfqpoint{0.197500in}{0.500000in}}{\pgfqpoint{0.197500in}{0.513923in}}{\pgfqpoint{0.203032in}{0.527278in}}%
\pgfpathcurveto{\pgfqpoint{0.212877in}{0.537123in}}{\pgfqpoint{0.222722in}{0.546968in}}{\pgfqpoint{0.236077in}{0.552500in}}%
\pgfpathcurveto{\pgfqpoint{0.250000in}{0.552500in}}{\pgfqpoint{0.263923in}{0.552500in}}{\pgfqpoint{0.277278in}{0.546968in}}%
\pgfpathcurveto{\pgfqpoint{0.287123in}{0.537123in}}{\pgfqpoint{0.296968in}{0.527278in}}{\pgfqpoint{0.302500in}{0.513923in}}%
\pgfpathcurveto{\pgfqpoint{0.302500in}{0.500000in}}{\pgfqpoint{0.302500in}{0.486077in}}{\pgfqpoint{0.296968in}{0.472722in}}%
\pgfpathcurveto{\pgfqpoint{0.287123in}{0.462877in}}{\pgfqpoint{0.277278in}{0.453032in}}{\pgfqpoint{0.263923in}{0.447500in}}%
\pgfpathclose%
\pgfpathmoveto{\pgfqpoint{0.416667in}{0.441667in}}%
\pgfpathcurveto{\pgfqpoint{0.432137in}{0.441667in}}{\pgfqpoint{0.446975in}{0.447813in}}{\pgfqpoint{0.457915in}{0.458752in}}%
\pgfpathcurveto{\pgfqpoint{0.468854in}{0.469691in}}{\pgfqpoint{0.475000in}{0.484530in}}{\pgfqpoint{0.475000in}{0.500000in}}%
\pgfpathcurveto{\pgfqpoint{0.475000in}{0.515470in}}{\pgfqpoint{0.468854in}{0.530309in}}{\pgfqpoint{0.457915in}{0.541248in}}%
\pgfpathcurveto{\pgfqpoint{0.446975in}{0.552187in}}{\pgfqpoint{0.432137in}{0.558333in}}{\pgfqpoint{0.416667in}{0.558333in}}%
\pgfpathcurveto{\pgfqpoint{0.401196in}{0.558333in}}{\pgfqpoint{0.386358in}{0.552187in}}{\pgfqpoint{0.375419in}{0.541248in}}%
\pgfpathcurveto{\pgfqpoint{0.364480in}{0.530309in}}{\pgfqpoint{0.358333in}{0.515470in}}{\pgfqpoint{0.358333in}{0.500000in}}%
\pgfpathcurveto{\pgfqpoint{0.358333in}{0.484530in}}{\pgfqpoint{0.364480in}{0.469691in}}{\pgfqpoint{0.375419in}{0.458752in}}%
\pgfpathcurveto{\pgfqpoint{0.386358in}{0.447813in}}{\pgfqpoint{0.401196in}{0.441667in}}{\pgfqpoint{0.416667in}{0.441667in}}%
\pgfpathclose%
\pgfpathmoveto{\pgfqpoint{0.416667in}{0.447500in}}%
\pgfpathcurveto{\pgfqpoint{0.416667in}{0.447500in}}{\pgfqpoint{0.402744in}{0.447500in}}{\pgfqpoint{0.389389in}{0.453032in}}%
\pgfpathcurveto{\pgfqpoint{0.379544in}{0.462877in}}{\pgfqpoint{0.369698in}{0.472722in}}{\pgfqpoint{0.364167in}{0.486077in}}%
\pgfpathcurveto{\pgfqpoint{0.364167in}{0.500000in}}{\pgfqpoint{0.364167in}{0.513923in}}{\pgfqpoint{0.369698in}{0.527278in}}%
\pgfpathcurveto{\pgfqpoint{0.379544in}{0.537123in}}{\pgfqpoint{0.389389in}{0.546968in}}{\pgfqpoint{0.402744in}{0.552500in}}%
\pgfpathcurveto{\pgfqpoint{0.416667in}{0.552500in}}{\pgfqpoint{0.430590in}{0.552500in}}{\pgfqpoint{0.443945in}{0.546968in}}%
\pgfpathcurveto{\pgfqpoint{0.453790in}{0.537123in}}{\pgfqpoint{0.463635in}{0.527278in}}{\pgfqpoint{0.469167in}{0.513923in}}%
\pgfpathcurveto{\pgfqpoint{0.469167in}{0.500000in}}{\pgfqpoint{0.469167in}{0.486077in}}{\pgfqpoint{0.463635in}{0.472722in}}%
\pgfpathcurveto{\pgfqpoint{0.453790in}{0.462877in}}{\pgfqpoint{0.443945in}{0.453032in}}{\pgfqpoint{0.430590in}{0.447500in}}%
\pgfpathclose%
\pgfpathmoveto{\pgfqpoint{0.583333in}{0.441667in}}%
\pgfpathcurveto{\pgfqpoint{0.598804in}{0.441667in}}{\pgfqpoint{0.613642in}{0.447813in}}{\pgfqpoint{0.624581in}{0.458752in}}%
\pgfpathcurveto{\pgfqpoint{0.635520in}{0.469691in}}{\pgfqpoint{0.641667in}{0.484530in}}{\pgfqpoint{0.641667in}{0.500000in}}%
\pgfpathcurveto{\pgfqpoint{0.641667in}{0.515470in}}{\pgfqpoint{0.635520in}{0.530309in}}{\pgfqpoint{0.624581in}{0.541248in}}%
\pgfpathcurveto{\pgfqpoint{0.613642in}{0.552187in}}{\pgfqpoint{0.598804in}{0.558333in}}{\pgfqpoint{0.583333in}{0.558333in}}%
\pgfpathcurveto{\pgfqpoint{0.567863in}{0.558333in}}{\pgfqpoint{0.553025in}{0.552187in}}{\pgfqpoint{0.542085in}{0.541248in}}%
\pgfpathcurveto{\pgfqpoint{0.531146in}{0.530309in}}{\pgfqpoint{0.525000in}{0.515470in}}{\pgfqpoint{0.525000in}{0.500000in}}%
\pgfpathcurveto{\pgfqpoint{0.525000in}{0.484530in}}{\pgfqpoint{0.531146in}{0.469691in}}{\pgfqpoint{0.542085in}{0.458752in}}%
\pgfpathcurveto{\pgfqpoint{0.553025in}{0.447813in}}{\pgfqpoint{0.567863in}{0.441667in}}{\pgfqpoint{0.583333in}{0.441667in}}%
\pgfpathclose%
\pgfpathmoveto{\pgfqpoint{0.583333in}{0.447500in}}%
\pgfpathcurveto{\pgfqpoint{0.583333in}{0.447500in}}{\pgfqpoint{0.569410in}{0.447500in}}{\pgfqpoint{0.556055in}{0.453032in}}%
\pgfpathcurveto{\pgfqpoint{0.546210in}{0.462877in}}{\pgfqpoint{0.536365in}{0.472722in}}{\pgfqpoint{0.530833in}{0.486077in}}%
\pgfpathcurveto{\pgfqpoint{0.530833in}{0.500000in}}{\pgfqpoint{0.530833in}{0.513923in}}{\pgfqpoint{0.536365in}{0.527278in}}%
\pgfpathcurveto{\pgfqpoint{0.546210in}{0.537123in}}{\pgfqpoint{0.556055in}{0.546968in}}{\pgfqpoint{0.569410in}{0.552500in}}%
\pgfpathcurveto{\pgfqpoint{0.583333in}{0.552500in}}{\pgfqpoint{0.597256in}{0.552500in}}{\pgfqpoint{0.610611in}{0.546968in}}%
\pgfpathcurveto{\pgfqpoint{0.620456in}{0.537123in}}{\pgfqpoint{0.630302in}{0.527278in}}{\pgfqpoint{0.635833in}{0.513923in}}%
\pgfpathcurveto{\pgfqpoint{0.635833in}{0.500000in}}{\pgfqpoint{0.635833in}{0.486077in}}{\pgfqpoint{0.630302in}{0.472722in}}%
\pgfpathcurveto{\pgfqpoint{0.620456in}{0.462877in}}{\pgfqpoint{0.610611in}{0.453032in}}{\pgfqpoint{0.597256in}{0.447500in}}%
\pgfpathclose%
\pgfpathmoveto{\pgfqpoint{0.750000in}{0.441667in}}%
\pgfpathcurveto{\pgfqpoint{0.765470in}{0.441667in}}{\pgfqpoint{0.780309in}{0.447813in}}{\pgfqpoint{0.791248in}{0.458752in}}%
\pgfpathcurveto{\pgfqpoint{0.802187in}{0.469691in}}{\pgfqpoint{0.808333in}{0.484530in}}{\pgfqpoint{0.808333in}{0.500000in}}%
\pgfpathcurveto{\pgfqpoint{0.808333in}{0.515470in}}{\pgfqpoint{0.802187in}{0.530309in}}{\pgfqpoint{0.791248in}{0.541248in}}%
\pgfpathcurveto{\pgfqpoint{0.780309in}{0.552187in}}{\pgfqpoint{0.765470in}{0.558333in}}{\pgfqpoint{0.750000in}{0.558333in}}%
\pgfpathcurveto{\pgfqpoint{0.734530in}{0.558333in}}{\pgfqpoint{0.719691in}{0.552187in}}{\pgfqpoint{0.708752in}{0.541248in}}%
\pgfpathcurveto{\pgfqpoint{0.697813in}{0.530309in}}{\pgfqpoint{0.691667in}{0.515470in}}{\pgfqpoint{0.691667in}{0.500000in}}%
\pgfpathcurveto{\pgfqpoint{0.691667in}{0.484530in}}{\pgfqpoint{0.697813in}{0.469691in}}{\pgfqpoint{0.708752in}{0.458752in}}%
\pgfpathcurveto{\pgfqpoint{0.719691in}{0.447813in}}{\pgfqpoint{0.734530in}{0.441667in}}{\pgfqpoint{0.750000in}{0.441667in}}%
\pgfpathclose%
\pgfpathmoveto{\pgfqpoint{0.750000in}{0.447500in}}%
\pgfpathcurveto{\pgfqpoint{0.750000in}{0.447500in}}{\pgfqpoint{0.736077in}{0.447500in}}{\pgfqpoint{0.722722in}{0.453032in}}%
\pgfpathcurveto{\pgfqpoint{0.712877in}{0.462877in}}{\pgfqpoint{0.703032in}{0.472722in}}{\pgfqpoint{0.697500in}{0.486077in}}%
\pgfpathcurveto{\pgfqpoint{0.697500in}{0.500000in}}{\pgfqpoint{0.697500in}{0.513923in}}{\pgfqpoint{0.703032in}{0.527278in}}%
\pgfpathcurveto{\pgfqpoint{0.712877in}{0.537123in}}{\pgfqpoint{0.722722in}{0.546968in}}{\pgfqpoint{0.736077in}{0.552500in}}%
\pgfpathcurveto{\pgfqpoint{0.750000in}{0.552500in}}{\pgfqpoint{0.763923in}{0.552500in}}{\pgfqpoint{0.777278in}{0.546968in}}%
\pgfpathcurveto{\pgfqpoint{0.787123in}{0.537123in}}{\pgfqpoint{0.796968in}{0.527278in}}{\pgfqpoint{0.802500in}{0.513923in}}%
\pgfpathcurveto{\pgfqpoint{0.802500in}{0.500000in}}{\pgfqpoint{0.802500in}{0.486077in}}{\pgfqpoint{0.796968in}{0.472722in}}%
\pgfpathcurveto{\pgfqpoint{0.787123in}{0.462877in}}{\pgfqpoint{0.777278in}{0.453032in}}{\pgfqpoint{0.763923in}{0.447500in}}%
\pgfpathclose%
\pgfpathmoveto{\pgfqpoint{0.916667in}{0.441667in}}%
\pgfpathcurveto{\pgfqpoint{0.932137in}{0.441667in}}{\pgfqpoint{0.946975in}{0.447813in}}{\pgfqpoint{0.957915in}{0.458752in}}%
\pgfpathcurveto{\pgfqpoint{0.968854in}{0.469691in}}{\pgfqpoint{0.975000in}{0.484530in}}{\pgfqpoint{0.975000in}{0.500000in}}%
\pgfpathcurveto{\pgfqpoint{0.975000in}{0.515470in}}{\pgfqpoint{0.968854in}{0.530309in}}{\pgfqpoint{0.957915in}{0.541248in}}%
\pgfpathcurveto{\pgfqpoint{0.946975in}{0.552187in}}{\pgfqpoint{0.932137in}{0.558333in}}{\pgfqpoint{0.916667in}{0.558333in}}%
\pgfpathcurveto{\pgfqpoint{0.901196in}{0.558333in}}{\pgfqpoint{0.886358in}{0.552187in}}{\pgfqpoint{0.875419in}{0.541248in}}%
\pgfpathcurveto{\pgfqpoint{0.864480in}{0.530309in}}{\pgfqpoint{0.858333in}{0.515470in}}{\pgfqpoint{0.858333in}{0.500000in}}%
\pgfpathcurveto{\pgfqpoint{0.858333in}{0.484530in}}{\pgfqpoint{0.864480in}{0.469691in}}{\pgfqpoint{0.875419in}{0.458752in}}%
\pgfpathcurveto{\pgfqpoint{0.886358in}{0.447813in}}{\pgfqpoint{0.901196in}{0.441667in}}{\pgfqpoint{0.916667in}{0.441667in}}%
\pgfpathclose%
\pgfpathmoveto{\pgfqpoint{0.916667in}{0.447500in}}%
\pgfpathcurveto{\pgfqpoint{0.916667in}{0.447500in}}{\pgfqpoint{0.902744in}{0.447500in}}{\pgfqpoint{0.889389in}{0.453032in}}%
\pgfpathcurveto{\pgfqpoint{0.879544in}{0.462877in}}{\pgfqpoint{0.869698in}{0.472722in}}{\pgfqpoint{0.864167in}{0.486077in}}%
\pgfpathcurveto{\pgfqpoint{0.864167in}{0.500000in}}{\pgfqpoint{0.864167in}{0.513923in}}{\pgfqpoint{0.869698in}{0.527278in}}%
\pgfpathcurveto{\pgfqpoint{0.879544in}{0.537123in}}{\pgfqpoint{0.889389in}{0.546968in}}{\pgfqpoint{0.902744in}{0.552500in}}%
\pgfpathcurveto{\pgfqpoint{0.916667in}{0.552500in}}{\pgfqpoint{0.930590in}{0.552500in}}{\pgfqpoint{0.943945in}{0.546968in}}%
\pgfpathcurveto{\pgfqpoint{0.953790in}{0.537123in}}{\pgfqpoint{0.963635in}{0.527278in}}{\pgfqpoint{0.969167in}{0.513923in}}%
\pgfpathcurveto{\pgfqpoint{0.969167in}{0.500000in}}{\pgfqpoint{0.969167in}{0.486077in}}{\pgfqpoint{0.963635in}{0.472722in}}%
\pgfpathcurveto{\pgfqpoint{0.953790in}{0.462877in}}{\pgfqpoint{0.943945in}{0.453032in}}{\pgfqpoint{0.930590in}{0.447500in}}%
\pgfpathclose%
\pgfpathmoveto{\pgfqpoint{0.000000in}{0.608333in}}%
\pgfpathcurveto{\pgfqpoint{0.015470in}{0.608333in}}{\pgfqpoint{0.030309in}{0.614480in}}{\pgfqpoint{0.041248in}{0.625419in}}%
\pgfpathcurveto{\pgfqpoint{0.052187in}{0.636358in}}{\pgfqpoint{0.058333in}{0.651196in}}{\pgfqpoint{0.058333in}{0.666667in}}%
\pgfpathcurveto{\pgfqpoint{0.058333in}{0.682137in}}{\pgfqpoint{0.052187in}{0.696975in}}{\pgfqpoint{0.041248in}{0.707915in}}%
\pgfpathcurveto{\pgfqpoint{0.030309in}{0.718854in}}{\pgfqpoint{0.015470in}{0.725000in}}{\pgfqpoint{0.000000in}{0.725000in}}%
\pgfpathcurveto{\pgfqpoint{-0.015470in}{0.725000in}}{\pgfqpoint{-0.030309in}{0.718854in}}{\pgfqpoint{-0.041248in}{0.707915in}}%
\pgfpathcurveto{\pgfqpoint{-0.052187in}{0.696975in}}{\pgfqpoint{-0.058333in}{0.682137in}}{\pgfqpoint{-0.058333in}{0.666667in}}%
\pgfpathcurveto{\pgfqpoint{-0.058333in}{0.651196in}}{\pgfqpoint{-0.052187in}{0.636358in}}{\pgfqpoint{-0.041248in}{0.625419in}}%
\pgfpathcurveto{\pgfqpoint{-0.030309in}{0.614480in}}{\pgfqpoint{-0.015470in}{0.608333in}}{\pgfqpoint{0.000000in}{0.608333in}}%
\pgfpathclose%
\pgfpathmoveto{\pgfqpoint{0.000000in}{0.614167in}}%
\pgfpathcurveto{\pgfqpoint{0.000000in}{0.614167in}}{\pgfqpoint{-0.013923in}{0.614167in}}{\pgfqpoint{-0.027278in}{0.619698in}}%
\pgfpathcurveto{\pgfqpoint{-0.037123in}{0.629544in}}{\pgfqpoint{-0.046968in}{0.639389in}}{\pgfqpoint{-0.052500in}{0.652744in}}%
\pgfpathcurveto{\pgfqpoint{-0.052500in}{0.666667in}}{\pgfqpoint{-0.052500in}{0.680590in}}{\pgfqpoint{-0.046968in}{0.693945in}}%
\pgfpathcurveto{\pgfqpoint{-0.037123in}{0.703790in}}{\pgfqpoint{-0.027278in}{0.713635in}}{\pgfqpoint{-0.013923in}{0.719167in}}%
\pgfpathcurveto{\pgfqpoint{0.000000in}{0.719167in}}{\pgfqpoint{0.013923in}{0.719167in}}{\pgfqpoint{0.027278in}{0.713635in}}%
\pgfpathcurveto{\pgfqpoint{0.037123in}{0.703790in}}{\pgfqpoint{0.046968in}{0.693945in}}{\pgfqpoint{0.052500in}{0.680590in}}%
\pgfpathcurveto{\pgfqpoint{0.052500in}{0.666667in}}{\pgfqpoint{0.052500in}{0.652744in}}{\pgfqpoint{0.046968in}{0.639389in}}%
\pgfpathcurveto{\pgfqpoint{0.037123in}{0.629544in}}{\pgfqpoint{0.027278in}{0.619698in}}{\pgfqpoint{0.013923in}{0.614167in}}%
\pgfpathclose%
\pgfpathmoveto{\pgfqpoint{0.166667in}{0.608333in}}%
\pgfpathcurveto{\pgfqpoint{0.182137in}{0.608333in}}{\pgfqpoint{0.196975in}{0.614480in}}{\pgfqpoint{0.207915in}{0.625419in}}%
\pgfpathcurveto{\pgfqpoint{0.218854in}{0.636358in}}{\pgfqpoint{0.225000in}{0.651196in}}{\pgfqpoint{0.225000in}{0.666667in}}%
\pgfpathcurveto{\pgfqpoint{0.225000in}{0.682137in}}{\pgfqpoint{0.218854in}{0.696975in}}{\pgfqpoint{0.207915in}{0.707915in}}%
\pgfpathcurveto{\pgfqpoint{0.196975in}{0.718854in}}{\pgfqpoint{0.182137in}{0.725000in}}{\pgfqpoint{0.166667in}{0.725000in}}%
\pgfpathcurveto{\pgfqpoint{0.151196in}{0.725000in}}{\pgfqpoint{0.136358in}{0.718854in}}{\pgfqpoint{0.125419in}{0.707915in}}%
\pgfpathcurveto{\pgfqpoint{0.114480in}{0.696975in}}{\pgfqpoint{0.108333in}{0.682137in}}{\pgfqpoint{0.108333in}{0.666667in}}%
\pgfpathcurveto{\pgfqpoint{0.108333in}{0.651196in}}{\pgfqpoint{0.114480in}{0.636358in}}{\pgfqpoint{0.125419in}{0.625419in}}%
\pgfpathcurveto{\pgfqpoint{0.136358in}{0.614480in}}{\pgfqpoint{0.151196in}{0.608333in}}{\pgfqpoint{0.166667in}{0.608333in}}%
\pgfpathclose%
\pgfpathmoveto{\pgfqpoint{0.166667in}{0.614167in}}%
\pgfpathcurveto{\pgfqpoint{0.166667in}{0.614167in}}{\pgfqpoint{0.152744in}{0.614167in}}{\pgfqpoint{0.139389in}{0.619698in}}%
\pgfpathcurveto{\pgfqpoint{0.129544in}{0.629544in}}{\pgfqpoint{0.119698in}{0.639389in}}{\pgfqpoint{0.114167in}{0.652744in}}%
\pgfpathcurveto{\pgfqpoint{0.114167in}{0.666667in}}{\pgfqpoint{0.114167in}{0.680590in}}{\pgfqpoint{0.119698in}{0.693945in}}%
\pgfpathcurveto{\pgfqpoint{0.129544in}{0.703790in}}{\pgfqpoint{0.139389in}{0.713635in}}{\pgfqpoint{0.152744in}{0.719167in}}%
\pgfpathcurveto{\pgfqpoint{0.166667in}{0.719167in}}{\pgfqpoint{0.180590in}{0.719167in}}{\pgfqpoint{0.193945in}{0.713635in}}%
\pgfpathcurveto{\pgfqpoint{0.203790in}{0.703790in}}{\pgfqpoint{0.213635in}{0.693945in}}{\pgfqpoint{0.219167in}{0.680590in}}%
\pgfpathcurveto{\pgfqpoint{0.219167in}{0.666667in}}{\pgfqpoint{0.219167in}{0.652744in}}{\pgfqpoint{0.213635in}{0.639389in}}%
\pgfpathcurveto{\pgfqpoint{0.203790in}{0.629544in}}{\pgfqpoint{0.193945in}{0.619698in}}{\pgfqpoint{0.180590in}{0.614167in}}%
\pgfpathclose%
\pgfpathmoveto{\pgfqpoint{0.333333in}{0.608333in}}%
\pgfpathcurveto{\pgfqpoint{0.348804in}{0.608333in}}{\pgfqpoint{0.363642in}{0.614480in}}{\pgfqpoint{0.374581in}{0.625419in}}%
\pgfpathcurveto{\pgfqpoint{0.385520in}{0.636358in}}{\pgfqpoint{0.391667in}{0.651196in}}{\pgfqpoint{0.391667in}{0.666667in}}%
\pgfpathcurveto{\pgfqpoint{0.391667in}{0.682137in}}{\pgfqpoint{0.385520in}{0.696975in}}{\pgfqpoint{0.374581in}{0.707915in}}%
\pgfpathcurveto{\pgfqpoint{0.363642in}{0.718854in}}{\pgfqpoint{0.348804in}{0.725000in}}{\pgfqpoint{0.333333in}{0.725000in}}%
\pgfpathcurveto{\pgfqpoint{0.317863in}{0.725000in}}{\pgfqpoint{0.303025in}{0.718854in}}{\pgfqpoint{0.292085in}{0.707915in}}%
\pgfpathcurveto{\pgfqpoint{0.281146in}{0.696975in}}{\pgfqpoint{0.275000in}{0.682137in}}{\pgfqpoint{0.275000in}{0.666667in}}%
\pgfpathcurveto{\pgfqpoint{0.275000in}{0.651196in}}{\pgfqpoint{0.281146in}{0.636358in}}{\pgfqpoint{0.292085in}{0.625419in}}%
\pgfpathcurveto{\pgfqpoint{0.303025in}{0.614480in}}{\pgfqpoint{0.317863in}{0.608333in}}{\pgfqpoint{0.333333in}{0.608333in}}%
\pgfpathclose%
\pgfpathmoveto{\pgfqpoint{0.333333in}{0.614167in}}%
\pgfpathcurveto{\pgfqpoint{0.333333in}{0.614167in}}{\pgfqpoint{0.319410in}{0.614167in}}{\pgfqpoint{0.306055in}{0.619698in}}%
\pgfpathcurveto{\pgfqpoint{0.296210in}{0.629544in}}{\pgfqpoint{0.286365in}{0.639389in}}{\pgfqpoint{0.280833in}{0.652744in}}%
\pgfpathcurveto{\pgfqpoint{0.280833in}{0.666667in}}{\pgfqpoint{0.280833in}{0.680590in}}{\pgfqpoint{0.286365in}{0.693945in}}%
\pgfpathcurveto{\pgfqpoint{0.296210in}{0.703790in}}{\pgfqpoint{0.306055in}{0.713635in}}{\pgfqpoint{0.319410in}{0.719167in}}%
\pgfpathcurveto{\pgfqpoint{0.333333in}{0.719167in}}{\pgfqpoint{0.347256in}{0.719167in}}{\pgfqpoint{0.360611in}{0.713635in}}%
\pgfpathcurveto{\pgfqpoint{0.370456in}{0.703790in}}{\pgfqpoint{0.380302in}{0.693945in}}{\pgfqpoint{0.385833in}{0.680590in}}%
\pgfpathcurveto{\pgfqpoint{0.385833in}{0.666667in}}{\pgfqpoint{0.385833in}{0.652744in}}{\pgfqpoint{0.380302in}{0.639389in}}%
\pgfpathcurveto{\pgfqpoint{0.370456in}{0.629544in}}{\pgfqpoint{0.360611in}{0.619698in}}{\pgfqpoint{0.347256in}{0.614167in}}%
\pgfpathclose%
\pgfpathmoveto{\pgfqpoint{0.500000in}{0.608333in}}%
\pgfpathcurveto{\pgfqpoint{0.515470in}{0.608333in}}{\pgfqpoint{0.530309in}{0.614480in}}{\pgfqpoint{0.541248in}{0.625419in}}%
\pgfpathcurveto{\pgfqpoint{0.552187in}{0.636358in}}{\pgfqpoint{0.558333in}{0.651196in}}{\pgfqpoint{0.558333in}{0.666667in}}%
\pgfpathcurveto{\pgfqpoint{0.558333in}{0.682137in}}{\pgfqpoint{0.552187in}{0.696975in}}{\pgfqpoint{0.541248in}{0.707915in}}%
\pgfpathcurveto{\pgfqpoint{0.530309in}{0.718854in}}{\pgfqpoint{0.515470in}{0.725000in}}{\pgfqpoint{0.500000in}{0.725000in}}%
\pgfpathcurveto{\pgfqpoint{0.484530in}{0.725000in}}{\pgfqpoint{0.469691in}{0.718854in}}{\pgfqpoint{0.458752in}{0.707915in}}%
\pgfpathcurveto{\pgfqpoint{0.447813in}{0.696975in}}{\pgfqpoint{0.441667in}{0.682137in}}{\pgfqpoint{0.441667in}{0.666667in}}%
\pgfpathcurveto{\pgfqpoint{0.441667in}{0.651196in}}{\pgfqpoint{0.447813in}{0.636358in}}{\pgfqpoint{0.458752in}{0.625419in}}%
\pgfpathcurveto{\pgfqpoint{0.469691in}{0.614480in}}{\pgfqpoint{0.484530in}{0.608333in}}{\pgfqpoint{0.500000in}{0.608333in}}%
\pgfpathclose%
\pgfpathmoveto{\pgfqpoint{0.500000in}{0.614167in}}%
\pgfpathcurveto{\pgfqpoint{0.500000in}{0.614167in}}{\pgfqpoint{0.486077in}{0.614167in}}{\pgfqpoint{0.472722in}{0.619698in}}%
\pgfpathcurveto{\pgfqpoint{0.462877in}{0.629544in}}{\pgfqpoint{0.453032in}{0.639389in}}{\pgfqpoint{0.447500in}{0.652744in}}%
\pgfpathcurveto{\pgfqpoint{0.447500in}{0.666667in}}{\pgfqpoint{0.447500in}{0.680590in}}{\pgfqpoint{0.453032in}{0.693945in}}%
\pgfpathcurveto{\pgfqpoint{0.462877in}{0.703790in}}{\pgfqpoint{0.472722in}{0.713635in}}{\pgfqpoint{0.486077in}{0.719167in}}%
\pgfpathcurveto{\pgfqpoint{0.500000in}{0.719167in}}{\pgfqpoint{0.513923in}{0.719167in}}{\pgfqpoint{0.527278in}{0.713635in}}%
\pgfpathcurveto{\pgfqpoint{0.537123in}{0.703790in}}{\pgfqpoint{0.546968in}{0.693945in}}{\pgfqpoint{0.552500in}{0.680590in}}%
\pgfpathcurveto{\pgfqpoint{0.552500in}{0.666667in}}{\pgfqpoint{0.552500in}{0.652744in}}{\pgfqpoint{0.546968in}{0.639389in}}%
\pgfpathcurveto{\pgfqpoint{0.537123in}{0.629544in}}{\pgfqpoint{0.527278in}{0.619698in}}{\pgfqpoint{0.513923in}{0.614167in}}%
\pgfpathclose%
\pgfpathmoveto{\pgfqpoint{0.666667in}{0.608333in}}%
\pgfpathcurveto{\pgfqpoint{0.682137in}{0.608333in}}{\pgfqpoint{0.696975in}{0.614480in}}{\pgfqpoint{0.707915in}{0.625419in}}%
\pgfpathcurveto{\pgfqpoint{0.718854in}{0.636358in}}{\pgfqpoint{0.725000in}{0.651196in}}{\pgfqpoint{0.725000in}{0.666667in}}%
\pgfpathcurveto{\pgfqpoint{0.725000in}{0.682137in}}{\pgfqpoint{0.718854in}{0.696975in}}{\pgfqpoint{0.707915in}{0.707915in}}%
\pgfpathcurveto{\pgfqpoint{0.696975in}{0.718854in}}{\pgfqpoint{0.682137in}{0.725000in}}{\pgfqpoint{0.666667in}{0.725000in}}%
\pgfpathcurveto{\pgfqpoint{0.651196in}{0.725000in}}{\pgfqpoint{0.636358in}{0.718854in}}{\pgfqpoint{0.625419in}{0.707915in}}%
\pgfpathcurveto{\pgfqpoint{0.614480in}{0.696975in}}{\pgfqpoint{0.608333in}{0.682137in}}{\pgfqpoint{0.608333in}{0.666667in}}%
\pgfpathcurveto{\pgfqpoint{0.608333in}{0.651196in}}{\pgfqpoint{0.614480in}{0.636358in}}{\pgfqpoint{0.625419in}{0.625419in}}%
\pgfpathcurveto{\pgfqpoint{0.636358in}{0.614480in}}{\pgfqpoint{0.651196in}{0.608333in}}{\pgfqpoint{0.666667in}{0.608333in}}%
\pgfpathclose%
\pgfpathmoveto{\pgfqpoint{0.666667in}{0.614167in}}%
\pgfpathcurveto{\pgfqpoint{0.666667in}{0.614167in}}{\pgfqpoint{0.652744in}{0.614167in}}{\pgfqpoint{0.639389in}{0.619698in}}%
\pgfpathcurveto{\pgfqpoint{0.629544in}{0.629544in}}{\pgfqpoint{0.619698in}{0.639389in}}{\pgfqpoint{0.614167in}{0.652744in}}%
\pgfpathcurveto{\pgfqpoint{0.614167in}{0.666667in}}{\pgfqpoint{0.614167in}{0.680590in}}{\pgfqpoint{0.619698in}{0.693945in}}%
\pgfpathcurveto{\pgfqpoint{0.629544in}{0.703790in}}{\pgfqpoint{0.639389in}{0.713635in}}{\pgfqpoint{0.652744in}{0.719167in}}%
\pgfpathcurveto{\pgfqpoint{0.666667in}{0.719167in}}{\pgfqpoint{0.680590in}{0.719167in}}{\pgfqpoint{0.693945in}{0.713635in}}%
\pgfpathcurveto{\pgfqpoint{0.703790in}{0.703790in}}{\pgfqpoint{0.713635in}{0.693945in}}{\pgfqpoint{0.719167in}{0.680590in}}%
\pgfpathcurveto{\pgfqpoint{0.719167in}{0.666667in}}{\pgfqpoint{0.719167in}{0.652744in}}{\pgfqpoint{0.713635in}{0.639389in}}%
\pgfpathcurveto{\pgfqpoint{0.703790in}{0.629544in}}{\pgfqpoint{0.693945in}{0.619698in}}{\pgfqpoint{0.680590in}{0.614167in}}%
\pgfpathclose%
\pgfpathmoveto{\pgfqpoint{0.833333in}{0.608333in}}%
\pgfpathcurveto{\pgfqpoint{0.848804in}{0.608333in}}{\pgfqpoint{0.863642in}{0.614480in}}{\pgfqpoint{0.874581in}{0.625419in}}%
\pgfpathcurveto{\pgfqpoint{0.885520in}{0.636358in}}{\pgfqpoint{0.891667in}{0.651196in}}{\pgfqpoint{0.891667in}{0.666667in}}%
\pgfpathcurveto{\pgfqpoint{0.891667in}{0.682137in}}{\pgfqpoint{0.885520in}{0.696975in}}{\pgfqpoint{0.874581in}{0.707915in}}%
\pgfpathcurveto{\pgfqpoint{0.863642in}{0.718854in}}{\pgfqpoint{0.848804in}{0.725000in}}{\pgfqpoint{0.833333in}{0.725000in}}%
\pgfpathcurveto{\pgfqpoint{0.817863in}{0.725000in}}{\pgfqpoint{0.803025in}{0.718854in}}{\pgfqpoint{0.792085in}{0.707915in}}%
\pgfpathcurveto{\pgfqpoint{0.781146in}{0.696975in}}{\pgfqpoint{0.775000in}{0.682137in}}{\pgfqpoint{0.775000in}{0.666667in}}%
\pgfpathcurveto{\pgfqpoint{0.775000in}{0.651196in}}{\pgfqpoint{0.781146in}{0.636358in}}{\pgfqpoint{0.792085in}{0.625419in}}%
\pgfpathcurveto{\pgfqpoint{0.803025in}{0.614480in}}{\pgfqpoint{0.817863in}{0.608333in}}{\pgfqpoint{0.833333in}{0.608333in}}%
\pgfpathclose%
\pgfpathmoveto{\pgfqpoint{0.833333in}{0.614167in}}%
\pgfpathcurveto{\pgfqpoint{0.833333in}{0.614167in}}{\pgfqpoint{0.819410in}{0.614167in}}{\pgfqpoint{0.806055in}{0.619698in}}%
\pgfpathcurveto{\pgfqpoint{0.796210in}{0.629544in}}{\pgfqpoint{0.786365in}{0.639389in}}{\pgfqpoint{0.780833in}{0.652744in}}%
\pgfpathcurveto{\pgfqpoint{0.780833in}{0.666667in}}{\pgfqpoint{0.780833in}{0.680590in}}{\pgfqpoint{0.786365in}{0.693945in}}%
\pgfpathcurveto{\pgfqpoint{0.796210in}{0.703790in}}{\pgfqpoint{0.806055in}{0.713635in}}{\pgfqpoint{0.819410in}{0.719167in}}%
\pgfpathcurveto{\pgfqpoint{0.833333in}{0.719167in}}{\pgfqpoint{0.847256in}{0.719167in}}{\pgfqpoint{0.860611in}{0.713635in}}%
\pgfpathcurveto{\pgfqpoint{0.870456in}{0.703790in}}{\pgfqpoint{0.880302in}{0.693945in}}{\pgfqpoint{0.885833in}{0.680590in}}%
\pgfpathcurveto{\pgfqpoint{0.885833in}{0.666667in}}{\pgfqpoint{0.885833in}{0.652744in}}{\pgfqpoint{0.880302in}{0.639389in}}%
\pgfpathcurveto{\pgfqpoint{0.870456in}{0.629544in}}{\pgfqpoint{0.860611in}{0.619698in}}{\pgfqpoint{0.847256in}{0.614167in}}%
\pgfpathclose%
\pgfpathmoveto{\pgfqpoint{1.000000in}{0.608333in}}%
\pgfpathcurveto{\pgfqpoint{1.015470in}{0.608333in}}{\pgfqpoint{1.030309in}{0.614480in}}{\pgfqpoint{1.041248in}{0.625419in}}%
\pgfpathcurveto{\pgfqpoint{1.052187in}{0.636358in}}{\pgfqpoint{1.058333in}{0.651196in}}{\pgfqpoint{1.058333in}{0.666667in}}%
\pgfpathcurveto{\pgfqpoint{1.058333in}{0.682137in}}{\pgfqpoint{1.052187in}{0.696975in}}{\pgfqpoint{1.041248in}{0.707915in}}%
\pgfpathcurveto{\pgfqpoint{1.030309in}{0.718854in}}{\pgfqpoint{1.015470in}{0.725000in}}{\pgfqpoint{1.000000in}{0.725000in}}%
\pgfpathcurveto{\pgfqpoint{0.984530in}{0.725000in}}{\pgfqpoint{0.969691in}{0.718854in}}{\pgfqpoint{0.958752in}{0.707915in}}%
\pgfpathcurveto{\pgfqpoint{0.947813in}{0.696975in}}{\pgfqpoint{0.941667in}{0.682137in}}{\pgfqpoint{0.941667in}{0.666667in}}%
\pgfpathcurveto{\pgfqpoint{0.941667in}{0.651196in}}{\pgfqpoint{0.947813in}{0.636358in}}{\pgfqpoint{0.958752in}{0.625419in}}%
\pgfpathcurveto{\pgfqpoint{0.969691in}{0.614480in}}{\pgfqpoint{0.984530in}{0.608333in}}{\pgfqpoint{1.000000in}{0.608333in}}%
\pgfpathclose%
\pgfpathmoveto{\pgfqpoint{1.000000in}{0.614167in}}%
\pgfpathcurveto{\pgfqpoint{1.000000in}{0.614167in}}{\pgfqpoint{0.986077in}{0.614167in}}{\pgfqpoint{0.972722in}{0.619698in}}%
\pgfpathcurveto{\pgfqpoint{0.962877in}{0.629544in}}{\pgfqpoint{0.953032in}{0.639389in}}{\pgfqpoint{0.947500in}{0.652744in}}%
\pgfpathcurveto{\pgfqpoint{0.947500in}{0.666667in}}{\pgfqpoint{0.947500in}{0.680590in}}{\pgfqpoint{0.953032in}{0.693945in}}%
\pgfpathcurveto{\pgfqpoint{0.962877in}{0.703790in}}{\pgfqpoint{0.972722in}{0.713635in}}{\pgfqpoint{0.986077in}{0.719167in}}%
\pgfpathcurveto{\pgfqpoint{1.000000in}{0.719167in}}{\pgfqpoint{1.013923in}{0.719167in}}{\pgfqpoint{1.027278in}{0.713635in}}%
\pgfpathcurveto{\pgfqpoint{1.037123in}{0.703790in}}{\pgfqpoint{1.046968in}{0.693945in}}{\pgfqpoint{1.052500in}{0.680590in}}%
\pgfpathcurveto{\pgfqpoint{1.052500in}{0.666667in}}{\pgfqpoint{1.052500in}{0.652744in}}{\pgfqpoint{1.046968in}{0.639389in}}%
\pgfpathcurveto{\pgfqpoint{1.037123in}{0.629544in}}{\pgfqpoint{1.027278in}{0.619698in}}{\pgfqpoint{1.013923in}{0.614167in}}%
\pgfpathclose%
\pgfpathmoveto{\pgfqpoint{0.083333in}{0.775000in}}%
\pgfpathcurveto{\pgfqpoint{0.098804in}{0.775000in}}{\pgfqpoint{0.113642in}{0.781146in}}{\pgfqpoint{0.124581in}{0.792085in}}%
\pgfpathcurveto{\pgfqpoint{0.135520in}{0.803025in}}{\pgfqpoint{0.141667in}{0.817863in}}{\pgfqpoint{0.141667in}{0.833333in}}%
\pgfpathcurveto{\pgfqpoint{0.141667in}{0.848804in}}{\pgfqpoint{0.135520in}{0.863642in}}{\pgfqpoint{0.124581in}{0.874581in}}%
\pgfpathcurveto{\pgfqpoint{0.113642in}{0.885520in}}{\pgfqpoint{0.098804in}{0.891667in}}{\pgfqpoint{0.083333in}{0.891667in}}%
\pgfpathcurveto{\pgfqpoint{0.067863in}{0.891667in}}{\pgfqpoint{0.053025in}{0.885520in}}{\pgfqpoint{0.042085in}{0.874581in}}%
\pgfpathcurveto{\pgfqpoint{0.031146in}{0.863642in}}{\pgfqpoint{0.025000in}{0.848804in}}{\pgfqpoint{0.025000in}{0.833333in}}%
\pgfpathcurveto{\pgfqpoint{0.025000in}{0.817863in}}{\pgfqpoint{0.031146in}{0.803025in}}{\pgfqpoint{0.042085in}{0.792085in}}%
\pgfpathcurveto{\pgfqpoint{0.053025in}{0.781146in}}{\pgfqpoint{0.067863in}{0.775000in}}{\pgfqpoint{0.083333in}{0.775000in}}%
\pgfpathclose%
\pgfpathmoveto{\pgfqpoint{0.083333in}{0.780833in}}%
\pgfpathcurveto{\pgfqpoint{0.083333in}{0.780833in}}{\pgfqpoint{0.069410in}{0.780833in}}{\pgfqpoint{0.056055in}{0.786365in}}%
\pgfpathcurveto{\pgfqpoint{0.046210in}{0.796210in}}{\pgfqpoint{0.036365in}{0.806055in}}{\pgfqpoint{0.030833in}{0.819410in}}%
\pgfpathcurveto{\pgfqpoint{0.030833in}{0.833333in}}{\pgfqpoint{0.030833in}{0.847256in}}{\pgfqpoint{0.036365in}{0.860611in}}%
\pgfpathcurveto{\pgfqpoint{0.046210in}{0.870456in}}{\pgfqpoint{0.056055in}{0.880302in}}{\pgfqpoint{0.069410in}{0.885833in}}%
\pgfpathcurveto{\pgfqpoint{0.083333in}{0.885833in}}{\pgfqpoint{0.097256in}{0.885833in}}{\pgfqpoint{0.110611in}{0.880302in}}%
\pgfpathcurveto{\pgfqpoint{0.120456in}{0.870456in}}{\pgfqpoint{0.130302in}{0.860611in}}{\pgfqpoint{0.135833in}{0.847256in}}%
\pgfpathcurveto{\pgfqpoint{0.135833in}{0.833333in}}{\pgfqpoint{0.135833in}{0.819410in}}{\pgfqpoint{0.130302in}{0.806055in}}%
\pgfpathcurveto{\pgfqpoint{0.120456in}{0.796210in}}{\pgfqpoint{0.110611in}{0.786365in}}{\pgfqpoint{0.097256in}{0.780833in}}%
\pgfpathclose%
\pgfpathmoveto{\pgfqpoint{0.250000in}{0.775000in}}%
\pgfpathcurveto{\pgfqpoint{0.265470in}{0.775000in}}{\pgfqpoint{0.280309in}{0.781146in}}{\pgfqpoint{0.291248in}{0.792085in}}%
\pgfpathcurveto{\pgfqpoint{0.302187in}{0.803025in}}{\pgfqpoint{0.308333in}{0.817863in}}{\pgfqpoint{0.308333in}{0.833333in}}%
\pgfpathcurveto{\pgfqpoint{0.308333in}{0.848804in}}{\pgfqpoint{0.302187in}{0.863642in}}{\pgfqpoint{0.291248in}{0.874581in}}%
\pgfpathcurveto{\pgfqpoint{0.280309in}{0.885520in}}{\pgfqpoint{0.265470in}{0.891667in}}{\pgfqpoint{0.250000in}{0.891667in}}%
\pgfpathcurveto{\pgfqpoint{0.234530in}{0.891667in}}{\pgfqpoint{0.219691in}{0.885520in}}{\pgfqpoint{0.208752in}{0.874581in}}%
\pgfpathcurveto{\pgfqpoint{0.197813in}{0.863642in}}{\pgfqpoint{0.191667in}{0.848804in}}{\pgfqpoint{0.191667in}{0.833333in}}%
\pgfpathcurveto{\pgfqpoint{0.191667in}{0.817863in}}{\pgfqpoint{0.197813in}{0.803025in}}{\pgfqpoint{0.208752in}{0.792085in}}%
\pgfpathcurveto{\pgfqpoint{0.219691in}{0.781146in}}{\pgfqpoint{0.234530in}{0.775000in}}{\pgfqpoint{0.250000in}{0.775000in}}%
\pgfpathclose%
\pgfpathmoveto{\pgfqpoint{0.250000in}{0.780833in}}%
\pgfpathcurveto{\pgfqpoint{0.250000in}{0.780833in}}{\pgfqpoint{0.236077in}{0.780833in}}{\pgfqpoint{0.222722in}{0.786365in}}%
\pgfpathcurveto{\pgfqpoint{0.212877in}{0.796210in}}{\pgfqpoint{0.203032in}{0.806055in}}{\pgfqpoint{0.197500in}{0.819410in}}%
\pgfpathcurveto{\pgfqpoint{0.197500in}{0.833333in}}{\pgfqpoint{0.197500in}{0.847256in}}{\pgfqpoint{0.203032in}{0.860611in}}%
\pgfpathcurveto{\pgfqpoint{0.212877in}{0.870456in}}{\pgfqpoint{0.222722in}{0.880302in}}{\pgfqpoint{0.236077in}{0.885833in}}%
\pgfpathcurveto{\pgfqpoint{0.250000in}{0.885833in}}{\pgfqpoint{0.263923in}{0.885833in}}{\pgfqpoint{0.277278in}{0.880302in}}%
\pgfpathcurveto{\pgfqpoint{0.287123in}{0.870456in}}{\pgfqpoint{0.296968in}{0.860611in}}{\pgfqpoint{0.302500in}{0.847256in}}%
\pgfpathcurveto{\pgfqpoint{0.302500in}{0.833333in}}{\pgfqpoint{0.302500in}{0.819410in}}{\pgfqpoint{0.296968in}{0.806055in}}%
\pgfpathcurveto{\pgfqpoint{0.287123in}{0.796210in}}{\pgfqpoint{0.277278in}{0.786365in}}{\pgfqpoint{0.263923in}{0.780833in}}%
\pgfpathclose%
\pgfpathmoveto{\pgfqpoint{0.416667in}{0.775000in}}%
\pgfpathcurveto{\pgfqpoint{0.432137in}{0.775000in}}{\pgfqpoint{0.446975in}{0.781146in}}{\pgfqpoint{0.457915in}{0.792085in}}%
\pgfpathcurveto{\pgfqpoint{0.468854in}{0.803025in}}{\pgfqpoint{0.475000in}{0.817863in}}{\pgfqpoint{0.475000in}{0.833333in}}%
\pgfpathcurveto{\pgfqpoint{0.475000in}{0.848804in}}{\pgfqpoint{0.468854in}{0.863642in}}{\pgfqpoint{0.457915in}{0.874581in}}%
\pgfpathcurveto{\pgfqpoint{0.446975in}{0.885520in}}{\pgfqpoint{0.432137in}{0.891667in}}{\pgfqpoint{0.416667in}{0.891667in}}%
\pgfpathcurveto{\pgfqpoint{0.401196in}{0.891667in}}{\pgfqpoint{0.386358in}{0.885520in}}{\pgfqpoint{0.375419in}{0.874581in}}%
\pgfpathcurveto{\pgfqpoint{0.364480in}{0.863642in}}{\pgfqpoint{0.358333in}{0.848804in}}{\pgfqpoint{0.358333in}{0.833333in}}%
\pgfpathcurveto{\pgfqpoint{0.358333in}{0.817863in}}{\pgfqpoint{0.364480in}{0.803025in}}{\pgfqpoint{0.375419in}{0.792085in}}%
\pgfpathcurveto{\pgfqpoint{0.386358in}{0.781146in}}{\pgfqpoint{0.401196in}{0.775000in}}{\pgfqpoint{0.416667in}{0.775000in}}%
\pgfpathclose%
\pgfpathmoveto{\pgfqpoint{0.416667in}{0.780833in}}%
\pgfpathcurveto{\pgfqpoint{0.416667in}{0.780833in}}{\pgfqpoint{0.402744in}{0.780833in}}{\pgfqpoint{0.389389in}{0.786365in}}%
\pgfpathcurveto{\pgfqpoint{0.379544in}{0.796210in}}{\pgfqpoint{0.369698in}{0.806055in}}{\pgfqpoint{0.364167in}{0.819410in}}%
\pgfpathcurveto{\pgfqpoint{0.364167in}{0.833333in}}{\pgfqpoint{0.364167in}{0.847256in}}{\pgfqpoint{0.369698in}{0.860611in}}%
\pgfpathcurveto{\pgfqpoint{0.379544in}{0.870456in}}{\pgfqpoint{0.389389in}{0.880302in}}{\pgfqpoint{0.402744in}{0.885833in}}%
\pgfpathcurveto{\pgfqpoint{0.416667in}{0.885833in}}{\pgfqpoint{0.430590in}{0.885833in}}{\pgfqpoint{0.443945in}{0.880302in}}%
\pgfpathcurveto{\pgfqpoint{0.453790in}{0.870456in}}{\pgfqpoint{0.463635in}{0.860611in}}{\pgfqpoint{0.469167in}{0.847256in}}%
\pgfpathcurveto{\pgfqpoint{0.469167in}{0.833333in}}{\pgfqpoint{0.469167in}{0.819410in}}{\pgfqpoint{0.463635in}{0.806055in}}%
\pgfpathcurveto{\pgfqpoint{0.453790in}{0.796210in}}{\pgfqpoint{0.443945in}{0.786365in}}{\pgfqpoint{0.430590in}{0.780833in}}%
\pgfpathclose%
\pgfpathmoveto{\pgfqpoint{0.583333in}{0.775000in}}%
\pgfpathcurveto{\pgfqpoint{0.598804in}{0.775000in}}{\pgfqpoint{0.613642in}{0.781146in}}{\pgfqpoint{0.624581in}{0.792085in}}%
\pgfpathcurveto{\pgfqpoint{0.635520in}{0.803025in}}{\pgfqpoint{0.641667in}{0.817863in}}{\pgfqpoint{0.641667in}{0.833333in}}%
\pgfpathcurveto{\pgfqpoint{0.641667in}{0.848804in}}{\pgfqpoint{0.635520in}{0.863642in}}{\pgfqpoint{0.624581in}{0.874581in}}%
\pgfpathcurveto{\pgfqpoint{0.613642in}{0.885520in}}{\pgfqpoint{0.598804in}{0.891667in}}{\pgfqpoint{0.583333in}{0.891667in}}%
\pgfpathcurveto{\pgfqpoint{0.567863in}{0.891667in}}{\pgfqpoint{0.553025in}{0.885520in}}{\pgfqpoint{0.542085in}{0.874581in}}%
\pgfpathcurveto{\pgfqpoint{0.531146in}{0.863642in}}{\pgfqpoint{0.525000in}{0.848804in}}{\pgfqpoint{0.525000in}{0.833333in}}%
\pgfpathcurveto{\pgfqpoint{0.525000in}{0.817863in}}{\pgfqpoint{0.531146in}{0.803025in}}{\pgfqpoint{0.542085in}{0.792085in}}%
\pgfpathcurveto{\pgfqpoint{0.553025in}{0.781146in}}{\pgfqpoint{0.567863in}{0.775000in}}{\pgfqpoint{0.583333in}{0.775000in}}%
\pgfpathclose%
\pgfpathmoveto{\pgfqpoint{0.583333in}{0.780833in}}%
\pgfpathcurveto{\pgfqpoint{0.583333in}{0.780833in}}{\pgfqpoint{0.569410in}{0.780833in}}{\pgfqpoint{0.556055in}{0.786365in}}%
\pgfpathcurveto{\pgfqpoint{0.546210in}{0.796210in}}{\pgfqpoint{0.536365in}{0.806055in}}{\pgfqpoint{0.530833in}{0.819410in}}%
\pgfpathcurveto{\pgfqpoint{0.530833in}{0.833333in}}{\pgfqpoint{0.530833in}{0.847256in}}{\pgfqpoint{0.536365in}{0.860611in}}%
\pgfpathcurveto{\pgfqpoint{0.546210in}{0.870456in}}{\pgfqpoint{0.556055in}{0.880302in}}{\pgfqpoint{0.569410in}{0.885833in}}%
\pgfpathcurveto{\pgfqpoint{0.583333in}{0.885833in}}{\pgfqpoint{0.597256in}{0.885833in}}{\pgfqpoint{0.610611in}{0.880302in}}%
\pgfpathcurveto{\pgfqpoint{0.620456in}{0.870456in}}{\pgfqpoint{0.630302in}{0.860611in}}{\pgfqpoint{0.635833in}{0.847256in}}%
\pgfpathcurveto{\pgfqpoint{0.635833in}{0.833333in}}{\pgfqpoint{0.635833in}{0.819410in}}{\pgfqpoint{0.630302in}{0.806055in}}%
\pgfpathcurveto{\pgfqpoint{0.620456in}{0.796210in}}{\pgfqpoint{0.610611in}{0.786365in}}{\pgfqpoint{0.597256in}{0.780833in}}%
\pgfpathclose%
\pgfpathmoveto{\pgfqpoint{0.750000in}{0.775000in}}%
\pgfpathcurveto{\pgfqpoint{0.765470in}{0.775000in}}{\pgfqpoint{0.780309in}{0.781146in}}{\pgfqpoint{0.791248in}{0.792085in}}%
\pgfpathcurveto{\pgfqpoint{0.802187in}{0.803025in}}{\pgfqpoint{0.808333in}{0.817863in}}{\pgfqpoint{0.808333in}{0.833333in}}%
\pgfpathcurveto{\pgfqpoint{0.808333in}{0.848804in}}{\pgfqpoint{0.802187in}{0.863642in}}{\pgfqpoint{0.791248in}{0.874581in}}%
\pgfpathcurveto{\pgfqpoint{0.780309in}{0.885520in}}{\pgfqpoint{0.765470in}{0.891667in}}{\pgfqpoint{0.750000in}{0.891667in}}%
\pgfpathcurveto{\pgfqpoint{0.734530in}{0.891667in}}{\pgfqpoint{0.719691in}{0.885520in}}{\pgfqpoint{0.708752in}{0.874581in}}%
\pgfpathcurveto{\pgfqpoint{0.697813in}{0.863642in}}{\pgfqpoint{0.691667in}{0.848804in}}{\pgfqpoint{0.691667in}{0.833333in}}%
\pgfpathcurveto{\pgfqpoint{0.691667in}{0.817863in}}{\pgfqpoint{0.697813in}{0.803025in}}{\pgfqpoint{0.708752in}{0.792085in}}%
\pgfpathcurveto{\pgfqpoint{0.719691in}{0.781146in}}{\pgfqpoint{0.734530in}{0.775000in}}{\pgfqpoint{0.750000in}{0.775000in}}%
\pgfpathclose%
\pgfpathmoveto{\pgfqpoint{0.750000in}{0.780833in}}%
\pgfpathcurveto{\pgfqpoint{0.750000in}{0.780833in}}{\pgfqpoint{0.736077in}{0.780833in}}{\pgfqpoint{0.722722in}{0.786365in}}%
\pgfpathcurveto{\pgfqpoint{0.712877in}{0.796210in}}{\pgfqpoint{0.703032in}{0.806055in}}{\pgfqpoint{0.697500in}{0.819410in}}%
\pgfpathcurveto{\pgfqpoint{0.697500in}{0.833333in}}{\pgfqpoint{0.697500in}{0.847256in}}{\pgfqpoint{0.703032in}{0.860611in}}%
\pgfpathcurveto{\pgfqpoint{0.712877in}{0.870456in}}{\pgfqpoint{0.722722in}{0.880302in}}{\pgfqpoint{0.736077in}{0.885833in}}%
\pgfpathcurveto{\pgfqpoint{0.750000in}{0.885833in}}{\pgfqpoint{0.763923in}{0.885833in}}{\pgfqpoint{0.777278in}{0.880302in}}%
\pgfpathcurveto{\pgfqpoint{0.787123in}{0.870456in}}{\pgfqpoint{0.796968in}{0.860611in}}{\pgfqpoint{0.802500in}{0.847256in}}%
\pgfpathcurveto{\pgfqpoint{0.802500in}{0.833333in}}{\pgfqpoint{0.802500in}{0.819410in}}{\pgfqpoint{0.796968in}{0.806055in}}%
\pgfpathcurveto{\pgfqpoint{0.787123in}{0.796210in}}{\pgfqpoint{0.777278in}{0.786365in}}{\pgfqpoint{0.763923in}{0.780833in}}%
\pgfpathclose%
\pgfpathmoveto{\pgfqpoint{0.916667in}{0.775000in}}%
\pgfpathcurveto{\pgfqpoint{0.932137in}{0.775000in}}{\pgfqpoint{0.946975in}{0.781146in}}{\pgfqpoint{0.957915in}{0.792085in}}%
\pgfpathcurveto{\pgfqpoint{0.968854in}{0.803025in}}{\pgfqpoint{0.975000in}{0.817863in}}{\pgfqpoint{0.975000in}{0.833333in}}%
\pgfpathcurveto{\pgfqpoint{0.975000in}{0.848804in}}{\pgfqpoint{0.968854in}{0.863642in}}{\pgfqpoint{0.957915in}{0.874581in}}%
\pgfpathcurveto{\pgfqpoint{0.946975in}{0.885520in}}{\pgfqpoint{0.932137in}{0.891667in}}{\pgfqpoint{0.916667in}{0.891667in}}%
\pgfpathcurveto{\pgfqpoint{0.901196in}{0.891667in}}{\pgfqpoint{0.886358in}{0.885520in}}{\pgfqpoint{0.875419in}{0.874581in}}%
\pgfpathcurveto{\pgfqpoint{0.864480in}{0.863642in}}{\pgfqpoint{0.858333in}{0.848804in}}{\pgfqpoint{0.858333in}{0.833333in}}%
\pgfpathcurveto{\pgfqpoint{0.858333in}{0.817863in}}{\pgfqpoint{0.864480in}{0.803025in}}{\pgfqpoint{0.875419in}{0.792085in}}%
\pgfpathcurveto{\pgfqpoint{0.886358in}{0.781146in}}{\pgfqpoint{0.901196in}{0.775000in}}{\pgfqpoint{0.916667in}{0.775000in}}%
\pgfpathclose%
\pgfpathmoveto{\pgfqpoint{0.916667in}{0.780833in}}%
\pgfpathcurveto{\pgfqpoint{0.916667in}{0.780833in}}{\pgfqpoint{0.902744in}{0.780833in}}{\pgfqpoint{0.889389in}{0.786365in}}%
\pgfpathcurveto{\pgfqpoint{0.879544in}{0.796210in}}{\pgfqpoint{0.869698in}{0.806055in}}{\pgfqpoint{0.864167in}{0.819410in}}%
\pgfpathcurveto{\pgfqpoint{0.864167in}{0.833333in}}{\pgfqpoint{0.864167in}{0.847256in}}{\pgfqpoint{0.869698in}{0.860611in}}%
\pgfpathcurveto{\pgfqpoint{0.879544in}{0.870456in}}{\pgfqpoint{0.889389in}{0.880302in}}{\pgfqpoint{0.902744in}{0.885833in}}%
\pgfpathcurveto{\pgfqpoint{0.916667in}{0.885833in}}{\pgfqpoint{0.930590in}{0.885833in}}{\pgfqpoint{0.943945in}{0.880302in}}%
\pgfpathcurveto{\pgfqpoint{0.953790in}{0.870456in}}{\pgfqpoint{0.963635in}{0.860611in}}{\pgfqpoint{0.969167in}{0.847256in}}%
\pgfpathcurveto{\pgfqpoint{0.969167in}{0.833333in}}{\pgfqpoint{0.969167in}{0.819410in}}{\pgfqpoint{0.963635in}{0.806055in}}%
\pgfpathcurveto{\pgfqpoint{0.953790in}{0.796210in}}{\pgfqpoint{0.943945in}{0.786365in}}{\pgfqpoint{0.930590in}{0.780833in}}%
\pgfpathclose%
\pgfpathmoveto{\pgfqpoint{0.000000in}{0.941667in}}%
\pgfpathcurveto{\pgfqpoint{0.015470in}{0.941667in}}{\pgfqpoint{0.030309in}{0.947813in}}{\pgfqpoint{0.041248in}{0.958752in}}%
\pgfpathcurveto{\pgfqpoint{0.052187in}{0.969691in}}{\pgfqpoint{0.058333in}{0.984530in}}{\pgfqpoint{0.058333in}{1.000000in}}%
\pgfpathcurveto{\pgfqpoint{0.058333in}{1.015470in}}{\pgfqpoint{0.052187in}{1.030309in}}{\pgfqpoint{0.041248in}{1.041248in}}%
\pgfpathcurveto{\pgfqpoint{0.030309in}{1.052187in}}{\pgfqpoint{0.015470in}{1.058333in}}{\pgfqpoint{0.000000in}{1.058333in}}%
\pgfpathcurveto{\pgfqpoint{-0.015470in}{1.058333in}}{\pgfqpoint{-0.030309in}{1.052187in}}{\pgfqpoint{-0.041248in}{1.041248in}}%
\pgfpathcurveto{\pgfqpoint{-0.052187in}{1.030309in}}{\pgfqpoint{-0.058333in}{1.015470in}}{\pgfqpoint{-0.058333in}{1.000000in}}%
\pgfpathcurveto{\pgfqpoint{-0.058333in}{0.984530in}}{\pgfqpoint{-0.052187in}{0.969691in}}{\pgfqpoint{-0.041248in}{0.958752in}}%
\pgfpathcurveto{\pgfqpoint{-0.030309in}{0.947813in}}{\pgfqpoint{-0.015470in}{0.941667in}}{\pgfqpoint{0.000000in}{0.941667in}}%
\pgfpathclose%
\pgfpathmoveto{\pgfqpoint{0.000000in}{0.947500in}}%
\pgfpathcurveto{\pgfqpoint{0.000000in}{0.947500in}}{\pgfqpoint{-0.013923in}{0.947500in}}{\pgfqpoint{-0.027278in}{0.953032in}}%
\pgfpathcurveto{\pgfqpoint{-0.037123in}{0.962877in}}{\pgfqpoint{-0.046968in}{0.972722in}}{\pgfqpoint{-0.052500in}{0.986077in}}%
\pgfpathcurveto{\pgfqpoint{-0.052500in}{1.000000in}}{\pgfqpoint{-0.052500in}{1.013923in}}{\pgfqpoint{-0.046968in}{1.027278in}}%
\pgfpathcurveto{\pgfqpoint{-0.037123in}{1.037123in}}{\pgfqpoint{-0.027278in}{1.046968in}}{\pgfqpoint{-0.013923in}{1.052500in}}%
\pgfpathcurveto{\pgfqpoint{0.000000in}{1.052500in}}{\pgfqpoint{0.013923in}{1.052500in}}{\pgfqpoint{0.027278in}{1.046968in}}%
\pgfpathcurveto{\pgfqpoint{0.037123in}{1.037123in}}{\pgfqpoint{0.046968in}{1.027278in}}{\pgfqpoint{0.052500in}{1.013923in}}%
\pgfpathcurveto{\pgfqpoint{0.052500in}{1.000000in}}{\pgfqpoint{0.052500in}{0.986077in}}{\pgfqpoint{0.046968in}{0.972722in}}%
\pgfpathcurveto{\pgfqpoint{0.037123in}{0.962877in}}{\pgfqpoint{0.027278in}{0.953032in}}{\pgfqpoint{0.013923in}{0.947500in}}%
\pgfpathclose%
\pgfpathmoveto{\pgfqpoint{0.166667in}{0.941667in}}%
\pgfpathcurveto{\pgfqpoint{0.182137in}{0.941667in}}{\pgfqpoint{0.196975in}{0.947813in}}{\pgfqpoint{0.207915in}{0.958752in}}%
\pgfpathcurveto{\pgfqpoint{0.218854in}{0.969691in}}{\pgfqpoint{0.225000in}{0.984530in}}{\pgfqpoint{0.225000in}{1.000000in}}%
\pgfpathcurveto{\pgfqpoint{0.225000in}{1.015470in}}{\pgfqpoint{0.218854in}{1.030309in}}{\pgfqpoint{0.207915in}{1.041248in}}%
\pgfpathcurveto{\pgfqpoint{0.196975in}{1.052187in}}{\pgfqpoint{0.182137in}{1.058333in}}{\pgfqpoint{0.166667in}{1.058333in}}%
\pgfpathcurveto{\pgfqpoint{0.151196in}{1.058333in}}{\pgfqpoint{0.136358in}{1.052187in}}{\pgfqpoint{0.125419in}{1.041248in}}%
\pgfpathcurveto{\pgfqpoint{0.114480in}{1.030309in}}{\pgfqpoint{0.108333in}{1.015470in}}{\pgfqpoint{0.108333in}{1.000000in}}%
\pgfpathcurveto{\pgfqpoint{0.108333in}{0.984530in}}{\pgfqpoint{0.114480in}{0.969691in}}{\pgfqpoint{0.125419in}{0.958752in}}%
\pgfpathcurveto{\pgfqpoint{0.136358in}{0.947813in}}{\pgfqpoint{0.151196in}{0.941667in}}{\pgfqpoint{0.166667in}{0.941667in}}%
\pgfpathclose%
\pgfpathmoveto{\pgfqpoint{0.166667in}{0.947500in}}%
\pgfpathcurveto{\pgfqpoint{0.166667in}{0.947500in}}{\pgfqpoint{0.152744in}{0.947500in}}{\pgfqpoint{0.139389in}{0.953032in}}%
\pgfpathcurveto{\pgfqpoint{0.129544in}{0.962877in}}{\pgfqpoint{0.119698in}{0.972722in}}{\pgfqpoint{0.114167in}{0.986077in}}%
\pgfpathcurveto{\pgfqpoint{0.114167in}{1.000000in}}{\pgfqpoint{0.114167in}{1.013923in}}{\pgfqpoint{0.119698in}{1.027278in}}%
\pgfpathcurveto{\pgfqpoint{0.129544in}{1.037123in}}{\pgfqpoint{0.139389in}{1.046968in}}{\pgfqpoint{0.152744in}{1.052500in}}%
\pgfpathcurveto{\pgfqpoint{0.166667in}{1.052500in}}{\pgfqpoint{0.180590in}{1.052500in}}{\pgfqpoint{0.193945in}{1.046968in}}%
\pgfpathcurveto{\pgfqpoint{0.203790in}{1.037123in}}{\pgfqpoint{0.213635in}{1.027278in}}{\pgfqpoint{0.219167in}{1.013923in}}%
\pgfpathcurveto{\pgfqpoint{0.219167in}{1.000000in}}{\pgfqpoint{0.219167in}{0.986077in}}{\pgfqpoint{0.213635in}{0.972722in}}%
\pgfpathcurveto{\pgfqpoint{0.203790in}{0.962877in}}{\pgfqpoint{0.193945in}{0.953032in}}{\pgfqpoint{0.180590in}{0.947500in}}%
\pgfpathclose%
\pgfpathmoveto{\pgfqpoint{0.333333in}{0.941667in}}%
\pgfpathcurveto{\pgfqpoint{0.348804in}{0.941667in}}{\pgfqpoint{0.363642in}{0.947813in}}{\pgfqpoint{0.374581in}{0.958752in}}%
\pgfpathcurveto{\pgfqpoint{0.385520in}{0.969691in}}{\pgfqpoint{0.391667in}{0.984530in}}{\pgfqpoint{0.391667in}{1.000000in}}%
\pgfpathcurveto{\pgfqpoint{0.391667in}{1.015470in}}{\pgfqpoint{0.385520in}{1.030309in}}{\pgfqpoint{0.374581in}{1.041248in}}%
\pgfpathcurveto{\pgfqpoint{0.363642in}{1.052187in}}{\pgfqpoint{0.348804in}{1.058333in}}{\pgfqpoint{0.333333in}{1.058333in}}%
\pgfpathcurveto{\pgfqpoint{0.317863in}{1.058333in}}{\pgfqpoint{0.303025in}{1.052187in}}{\pgfqpoint{0.292085in}{1.041248in}}%
\pgfpathcurveto{\pgfqpoint{0.281146in}{1.030309in}}{\pgfqpoint{0.275000in}{1.015470in}}{\pgfqpoint{0.275000in}{1.000000in}}%
\pgfpathcurveto{\pgfqpoint{0.275000in}{0.984530in}}{\pgfqpoint{0.281146in}{0.969691in}}{\pgfqpoint{0.292085in}{0.958752in}}%
\pgfpathcurveto{\pgfqpoint{0.303025in}{0.947813in}}{\pgfqpoint{0.317863in}{0.941667in}}{\pgfqpoint{0.333333in}{0.941667in}}%
\pgfpathclose%
\pgfpathmoveto{\pgfqpoint{0.333333in}{0.947500in}}%
\pgfpathcurveto{\pgfqpoint{0.333333in}{0.947500in}}{\pgfqpoint{0.319410in}{0.947500in}}{\pgfqpoint{0.306055in}{0.953032in}}%
\pgfpathcurveto{\pgfqpoint{0.296210in}{0.962877in}}{\pgfqpoint{0.286365in}{0.972722in}}{\pgfqpoint{0.280833in}{0.986077in}}%
\pgfpathcurveto{\pgfqpoint{0.280833in}{1.000000in}}{\pgfqpoint{0.280833in}{1.013923in}}{\pgfqpoint{0.286365in}{1.027278in}}%
\pgfpathcurveto{\pgfqpoint{0.296210in}{1.037123in}}{\pgfqpoint{0.306055in}{1.046968in}}{\pgfqpoint{0.319410in}{1.052500in}}%
\pgfpathcurveto{\pgfqpoint{0.333333in}{1.052500in}}{\pgfqpoint{0.347256in}{1.052500in}}{\pgfqpoint{0.360611in}{1.046968in}}%
\pgfpathcurveto{\pgfqpoint{0.370456in}{1.037123in}}{\pgfqpoint{0.380302in}{1.027278in}}{\pgfqpoint{0.385833in}{1.013923in}}%
\pgfpathcurveto{\pgfqpoint{0.385833in}{1.000000in}}{\pgfqpoint{0.385833in}{0.986077in}}{\pgfqpoint{0.380302in}{0.972722in}}%
\pgfpathcurveto{\pgfqpoint{0.370456in}{0.962877in}}{\pgfqpoint{0.360611in}{0.953032in}}{\pgfqpoint{0.347256in}{0.947500in}}%
\pgfpathclose%
\pgfpathmoveto{\pgfqpoint{0.500000in}{0.941667in}}%
\pgfpathcurveto{\pgfqpoint{0.515470in}{0.941667in}}{\pgfqpoint{0.530309in}{0.947813in}}{\pgfqpoint{0.541248in}{0.958752in}}%
\pgfpathcurveto{\pgfqpoint{0.552187in}{0.969691in}}{\pgfqpoint{0.558333in}{0.984530in}}{\pgfqpoint{0.558333in}{1.000000in}}%
\pgfpathcurveto{\pgfqpoint{0.558333in}{1.015470in}}{\pgfqpoint{0.552187in}{1.030309in}}{\pgfqpoint{0.541248in}{1.041248in}}%
\pgfpathcurveto{\pgfqpoint{0.530309in}{1.052187in}}{\pgfqpoint{0.515470in}{1.058333in}}{\pgfqpoint{0.500000in}{1.058333in}}%
\pgfpathcurveto{\pgfqpoint{0.484530in}{1.058333in}}{\pgfqpoint{0.469691in}{1.052187in}}{\pgfqpoint{0.458752in}{1.041248in}}%
\pgfpathcurveto{\pgfqpoint{0.447813in}{1.030309in}}{\pgfqpoint{0.441667in}{1.015470in}}{\pgfqpoint{0.441667in}{1.000000in}}%
\pgfpathcurveto{\pgfqpoint{0.441667in}{0.984530in}}{\pgfqpoint{0.447813in}{0.969691in}}{\pgfqpoint{0.458752in}{0.958752in}}%
\pgfpathcurveto{\pgfqpoint{0.469691in}{0.947813in}}{\pgfqpoint{0.484530in}{0.941667in}}{\pgfqpoint{0.500000in}{0.941667in}}%
\pgfpathclose%
\pgfpathmoveto{\pgfqpoint{0.500000in}{0.947500in}}%
\pgfpathcurveto{\pgfqpoint{0.500000in}{0.947500in}}{\pgfqpoint{0.486077in}{0.947500in}}{\pgfqpoint{0.472722in}{0.953032in}}%
\pgfpathcurveto{\pgfqpoint{0.462877in}{0.962877in}}{\pgfqpoint{0.453032in}{0.972722in}}{\pgfqpoint{0.447500in}{0.986077in}}%
\pgfpathcurveto{\pgfqpoint{0.447500in}{1.000000in}}{\pgfqpoint{0.447500in}{1.013923in}}{\pgfqpoint{0.453032in}{1.027278in}}%
\pgfpathcurveto{\pgfqpoint{0.462877in}{1.037123in}}{\pgfqpoint{0.472722in}{1.046968in}}{\pgfqpoint{0.486077in}{1.052500in}}%
\pgfpathcurveto{\pgfqpoint{0.500000in}{1.052500in}}{\pgfqpoint{0.513923in}{1.052500in}}{\pgfqpoint{0.527278in}{1.046968in}}%
\pgfpathcurveto{\pgfqpoint{0.537123in}{1.037123in}}{\pgfqpoint{0.546968in}{1.027278in}}{\pgfqpoint{0.552500in}{1.013923in}}%
\pgfpathcurveto{\pgfqpoint{0.552500in}{1.000000in}}{\pgfqpoint{0.552500in}{0.986077in}}{\pgfqpoint{0.546968in}{0.972722in}}%
\pgfpathcurveto{\pgfqpoint{0.537123in}{0.962877in}}{\pgfqpoint{0.527278in}{0.953032in}}{\pgfqpoint{0.513923in}{0.947500in}}%
\pgfpathclose%
\pgfpathmoveto{\pgfqpoint{0.666667in}{0.941667in}}%
\pgfpathcurveto{\pgfqpoint{0.682137in}{0.941667in}}{\pgfqpoint{0.696975in}{0.947813in}}{\pgfqpoint{0.707915in}{0.958752in}}%
\pgfpathcurveto{\pgfqpoint{0.718854in}{0.969691in}}{\pgfqpoint{0.725000in}{0.984530in}}{\pgfqpoint{0.725000in}{1.000000in}}%
\pgfpathcurveto{\pgfqpoint{0.725000in}{1.015470in}}{\pgfqpoint{0.718854in}{1.030309in}}{\pgfqpoint{0.707915in}{1.041248in}}%
\pgfpathcurveto{\pgfqpoint{0.696975in}{1.052187in}}{\pgfqpoint{0.682137in}{1.058333in}}{\pgfqpoint{0.666667in}{1.058333in}}%
\pgfpathcurveto{\pgfqpoint{0.651196in}{1.058333in}}{\pgfqpoint{0.636358in}{1.052187in}}{\pgfqpoint{0.625419in}{1.041248in}}%
\pgfpathcurveto{\pgfqpoint{0.614480in}{1.030309in}}{\pgfqpoint{0.608333in}{1.015470in}}{\pgfqpoint{0.608333in}{1.000000in}}%
\pgfpathcurveto{\pgfqpoint{0.608333in}{0.984530in}}{\pgfqpoint{0.614480in}{0.969691in}}{\pgfqpoint{0.625419in}{0.958752in}}%
\pgfpathcurveto{\pgfqpoint{0.636358in}{0.947813in}}{\pgfqpoint{0.651196in}{0.941667in}}{\pgfqpoint{0.666667in}{0.941667in}}%
\pgfpathclose%
\pgfpathmoveto{\pgfqpoint{0.666667in}{0.947500in}}%
\pgfpathcurveto{\pgfqpoint{0.666667in}{0.947500in}}{\pgfqpoint{0.652744in}{0.947500in}}{\pgfqpoint{0.639389in}{0.953032in}}%
\pgfpathcurveto{\pgfqpoint{0.629544in}{0.962877in}}{\pgfqpoint{0.619698in}{0.972722in}}{\pgfqpoint{0.614167in}{0.986077in}}%
\pgfpathcurveto{\pgfqpoint{0.614167in}{1.000000in}}{\pgfqpoint{0.614167in}{1.013923in}}{\pgfqpoint{0.619698in}{1.027278in}}%
\pgfpathcurveto{\pgfqpoint{0.629544in}{1.037123in}}{\pgfqpoint{0.639389in}{1.046968in}}{\pgfqpoint{0.652744in}{1.052500in}}%
\pgfpathcurveto{\pgfqpoint{0.666667in}{1.052500in}}{\pgfqpoint{0.680590in}{1.052500in}}{\pgfqpoint{0.693945in}{1.046968in}}%
\pgfpathcurveto{\pgfqpoint{0.703790in}{1.037123in}}{\pgfqpoint{0.713635in}{1.027278in}}{\pgfqpoint{0.719167in}{1.013923in}}%
\pgfpathcurveto{\pgfqpoint{0.719167in}{1.000000in}}{\pgfqpoint{0.719167in}{0.986077in}}{\pgfqpoint{0.713635in}{0.972722in}}%
\pgfpathcurveto{\pgfqpoint{0.703790in}{0.962877in}}{\pgfqpoint{0.693945in}{0.953032in}}{\pgfqpoint{0.680590in}{0.947500in}}%
\pgfpathclose%
\pgfpathmoveto{\pgfqpoint{0.833333in}{0.941667in}}%
\pgfpathcurveto{\pgfqpoint{0.848804in}{0.941667in}}{\pgfqpoint{0.863642in}{0.947813in}}{\pgfqpoint{0.874581in}{0.958752in}}%
\pgfpathcurveto{\pgfqpoint{0.885520in}{0.969691in}}{\pgfqpoint{0.891667in}{0.984530in}}{\pgfqpoint{0.891667in}{1.000000in}}%
\pgfpathcurveto{\pgfqpoint{0.891667in}{1.015470in}}{\pgfqpoint{0.885520in}{1.030309in}}{\pgfqpoint{0.874581in}{1.041248in}}%
\pgfpathcurveto{\pgfqpoint{0.863642in}{1.052187in}}{\pgfqpoint{0.848804in}{1.058333in}}{\pgfqpoint{0.833333in}{1.058333in}}%
\pgfpathcurveto{\pgfqpoint{0.817863in}{1.058333in}}{\pgfqpoint{0.803025in}{1.052187in}}{\pgfqpoint{0.792085in}{1.041248in}}%
\pgfpathcurveto{\pgfqpoint{0.781146in}{1.030309in}}{\pgfqpoint{0.775000in}{1.015470in}}{\pgfqpoint{0.775000in}{1.000000in}}%
\pgfpathcurveto{\pgfqpoint{0.775000in}{0.984530in}}{\pgfqpoint{0.781146in}{0.969691in}}{\pgfqpoint{0.792085in}{0.958752in}}%
\pgfpathcurveto{\pgfqpoint{0.803025in}{0.947813in}}{\pgfqpoint{0.817863in}{0.941667in}}{\pgfqpoint{0.833333in}{0.941667in}}%
\pgfpathclose%
\pgfpathmoveto{\pgfqpoint{0.833333in}{0.947500in}}%
\pgfpathcurveto{\pgfqpoint{0.833333in}{0.947500in}}{\pgfqpoint{0.819410in}{0.947500in}}{\pgfqpoint{0.806055in}{0.953032in}}%
\pgfpathcurveto{\pgfqpoint{0.796210in}{0.962877in}}{\pgfqpoint{0.786365in}{0.972722in}}{\pgfqpoint{0.780833in}{0.986077in}}%
\pgfpathcurveto{\pgfqpoint{0.780833in}{1.000000in}}{\pgfqpoint{0.780833in}{1.013923in}}{\pgfqpoint{0.786365in}{1.027278in}}%
\pgfpathcurveto{\pgfqpoint{0.796210in}{1.037123in}}{\pgfqpoint{0.806055in}{1.046968in}}{\pgfqpoint{0.819410in}{1.052500in}}%
\pgfpathcurveto{\pgfqpoint{0.833333in}{1.052500in}}{\pgfqpoint{0.847256in}{1.052500in}}{\pgfqpoint{0.860611in}{1.046968in}}%
\pgfpathcurveto{\pgfqpoint{0.870456in}{1.037123in}}{\pgfqpoint{0.880302in}{1.027278in}}{\pgfqpoint{0.885833in}{1.013923in}}%
\pgfpathcurveto{\pgfqpoint{0.885833in}{1.000000in}}{\pgfqpoint{0.885833in}{0.986077in}}{\pgfqpoint{0.880302in}{0.972722in}}%
\pgfpathcurveto{\pgfqpoint{0.870456in}{0.962877in}}{\pgfqpoint{0.860611in}{0.953032in}}{\pgfqpoint{0.847256in}{0.947500in}}%
\pgfpathclose%
\pgfpathmoveto{\pgfqpoint{1.000000in}{0.941667in}}%
\pgfpathcurveto{\pgfqpoint{1.015470in}{0.941667in}}{\pgfqpoint{1.030309in}{0.947813in}}{\pgfqpoint{1.041248in}{0.958752in}}%
\pgfpathcurveto{\pgfqpoint{1.052187in}{0.969691in}}{\pgfqpoint{1.058333in}{0.984530in}}{\pgfqpoint{1.058333in}{1.000000in}}%
\pgfpathcurveto{\pgfqpoint{1.058333in}{1.015470in}}{\pgfqpoint{1.052187in}{1.030309in}}{\pgfqpoint{1.041248in}{1.041248in}}%
\pgfpathcurveto{\pgfqpoint{1.030309in}{1.052187in}}{\pgfqpoint{1.015470in}{1.058333in}}{\pgfqpoint{1.000000in}{1.058333in}}%
\pgfpathcurveto{\pgfqpoint{0.984530in}{1.058333in}}{\pgfqpoint{0.969691in}{1.052187in}}{\pgfqpoint{0.958752in}{1.041248in}}%
\pgfpathcurveto{\pgfqpoint{0.947813in}{1.030309in}}{\pgfqpoint{0.941667in}{1.015470in}}{\pgfqpoint{0.941667in}{1.000000in}}%
\pgfpathcurveto{\pgfqpoint{0.941667in}{0.984530in}}{\pgfqpoint{0.947813in}{0.969691in}}{\pgfqpoint{0.958752in}{0.958752in}}%
\pgfpathcurveto{\pgfqpoint{0.969691in}{0.947813in}}{\pgfqpoint{0.984530in}{0.941667in}}{\pgfqpoint{1.000000in}{0.941667in}}%
\pgfpathclose%
\pgfpathmoveto{\pgfqpoint{1.000000in}{0.947500in}}%
\pgfpathcurveto{\pgfqpoint{1.000000in}{0.947500in}}{\pgfqpoint{0.986077in}{0.947500in}}{\pgfqpoint{0.972722in}{0.953032in}}%
\pgfpathcurveto{\pgfqpoint{0.962877in}{0.962877in}}{\pgfqpoint{0.953032in}{0.972722in}}{\pgfqpoint{0.947500in}{0.986077in}}%
\pgfpathcurveto{\pgfqpoint{0.947500in}{1.000000in}}{\pgfqpoint{0.947500in}{1.013923in}}{\pgfqpoint{0.953032in}{1.027278in}}%
\pgfpathcurveto{\pgfqpoint{0.962877in}{1.037123in}}{\pgfqpoint{0.972722in}{1.046968in}}{\pgfqpoint{0.986077in}{1.052500in}}%
\pgfpathcurveto{\pgfqpoint{1.000000in}{1.052500in}}{\pgfqpoint{1.013923in}{1.052500in}}{\pgfqpoint{1.027278in}{1.046968in}}%
\pgfpathcurveto{\pgfqpoint{1.037123in}{1.037123in}}{\pgfqpoint{1.046968in}{1.027278in}}{\pgfqpoint{1.052500in}{1.013923in}}%
\pgfpathcurveto{\pgfqpoint{1.052500in}{1.000000in}}{\pgfqpoint{1.052500in}{0.986077in}}{\pgfqpoint{1.046968in}{0.972722in}}%
\pgfpathcurveto{\pgfqpoint{1.037123in}{0.962877in}}{\pgfqpoint{1.027278in}{0.953032in}}{\pgfqpoint{1.013923in}{0.947500in}}%
\pgfpathclose%
\pgfusepath{stroke}%
\end{pgfscope}%
}%
\pgfsys@transformshift{7.678174in}{6.807627in}%
\pgfsys@useobject{currentpattern}{}%
\pgfsys@transformshift{1in}{0in}%
\pgfsys@transformshift{-1in}{0in}%
\pgfsys@transformshift{0in}{1in}%
\pgfsys@useobject{currentpattern}{}%
\pgfsys@transformshift{1in}{0in}%
\pgfsys@transformshift{-1in}{0in}%
\pgfsys@transformshift{0in}{1in}%
\pgfsys@useobject{currentpattern}{}%
\pgfsys@transformshift{1in}{0in}%
\pgfsys@transformshift{-1in}{0in}%
\pgfsys@transformshift{0in}{1in}%
\end{pgfscope}%
\begin{pgfscope}%
\pgfpathrectangle{\pgfqpoint{1.090674in}{0.637495in}}{\pgfqpoint{9.300000in}{9.060000in}}%
\pgfusepath{clip}%
\pgfsetbuttcap%
\pgfsetmiterjoin%
\definecolor{currentfill}{rgb}{0.890196,0.466667,0.760784}%
\pgfsetfillcolor{currentfill}%
\pgfsetfillopacity{0.990000}%
\pgfsetlinewidth{0.000000pt}%
\definecolor{currentstroke}{rgb}{0.000000,0.000000,0.000000}%
\pgfsetstrokecolor{currentstroke}%
\pgfsetstrokeopacity{0.990000}%
\pgfsetdash{}{0pt}%
\pgfpathmoveto{\pgfqpoint{9.228174in}{7.064715in}}%
\pgfpathlineto{\pgfqpoint{10.003174in}{7.064715in}}%
\pgfpathlineto{\pgfqpoint{10.003174in}{9.266067in}}%
\pgfpathlineto{\pgfqpoint{9.228174in}{9.266067in}}%
\pgfpathclose%
\pgfusepath{fill}%
\end{pgfscope}%
\begin{pgfscope}%
\pgfsetbuttcap%
\pgfsetmiterjoin%
\definecolor{currentfill}{rgb}{0.890196,0.466667,0.760784}%
\pgfsetfillcolor{currentfill}%
\pgfsetfillopacity{0.990000}%
\pgfsetlinewidth{0.000000pt}%
\definecolor{currentstroke}{rgb}{0.000000,0.000000,0.000000}%
\pgfsetstrokecolor{currentstroke}%
\pgfsetstrokeopacity{0.990000}%
\pgfsetdash{}{0pt}%
\pgfpathrectangle{\pgfqpoint{1.090674in}{0.637495in}}{\pgfqpoint{9.300000in}{9.060000in}}%
\pgfusepath{clip}%
\pgfpathmoveto{\pgfqpoint{9.228174in}{7.064715in}}%
\pgfpathlineto{\pgfqpoint{10.003174in}{7.064715in}}%
\pgfpathlineto{\pgfqpoint{10.003174in}{9.266067in}}%
\pgfpathlineto{\pgfqpoint{9.228174in}{9.266067in}}%
\pgfpathclose%
\pgfusepath{clip}%
\pgfsys@defobject{currentpattern}{\pgfqpoint{0in}{0in}}{\pgfqpoint{1in}{1in}}{%
\begin{pgfscope}%
\pgfpathrectangle{\pgfqpoint{0in}{0in}}{\pgfqpoint{1in}{1in}}%
\pgfusepath{clip}%
\pgfpathmoveto{\pgfqpoint{0.000000in}{-0.058333in}}%
\pgfpathcurveto{\pgfqpoint{0.015470in}{-0.058333in}}{\pgfqpoint{0.030309in}{-0.052187in}}{\pgfqpoint{0.041248in}{-0.041248in}}%
\pgfpathcurveto{\pgfqpoint{0.052187in}{-0.030309in}}{\pgfqpoint{0.058333in}{-0.015470in}}{\pgfqpoint{0.058333in}{0.000000in}}%
\pgfpathcurveto{\pgfqpoint{0.058333in}{0.015470in}}{\pgfqpoint{0.052187in}{0.030309in}}{\pgfqpoint{0.041248in}{0.041248in}}%
\pgfpathcurveto{\pgfqpoint{0.030309in}{0.052187in}}{\pgfqpoint{0.015470in}{0.058333in}}{\pgfqpoint{0.000000in}{0.058333in}}%
\pgfpathcurveto{\pgfqpoint{-0.015470in}{0.058333in}}{\pgfqpoint{-0.030309in}{0.052187in}}{\pgfqpoint{-0.041248in}{0.041248in}}%
\pgfpathcurveto{\pgfqpoint{-0.052187in}{0.030309in}}{\pgfqpoint{-0.058333in}{0.015470in}}{\pgfqpoint{-0.058333in}{0.000000in}}%
\pgfpathcurveto{\pgfqpoint{-0.058333in}{-0.015470in}}{\pgfqpoint{-0.052187in}{-0.030309in}}{\pgfqpoint{-0.041248in}{-0.041248in}}%
\pgfpathcurveto{\pgfqpoint{-0.030309in}{-0.052187in}}{\pgfqpoint{-0.015470in}{-0.058333in}}{\pgfqpoint{0.000000in}{-0.058333in}}%
\pgfpathclose%
\pgfpathmoveto{\pgfqpoint{0.000000in}{-0.052500in}}%
\pgfpathcurveto{\pgfqpoint{0.000000in}{-0.052500in}}{\pgfqpoint{-0.013923in}{-0.052500in}}{\pgfqpoint{-0.027278in}{-0.046968in}}%
\pgfpathcurveto{\pgfqpoint{-0.037123in}{-0.037123in}}{\pgfqpoint{-0.046968in}{-0.027278in}}{\pgfqpoint{-0.052500in}{-0.013923in}}%
\pgfpathcurveto{\pgfqpoint{-0.052500in}{0.000000in}}{\pgfqpoint{-0.052500in}{0.013923in}}{\pgfqpoint{-0.046968in}{0.027278in}}%
\pgfpathcurveto{\pgfqpoint{-0.037123in}{0.037123in}}{\pgfqpoint{-0.027278in}{0.046968in}}{\pgfqpoint{-0.013923in}{0.052500in}}%
\pgfpathcurveto{\pgfqpoint{0.000000in}{0.052500in}}{\pgfqpoint{0.013923in}{0.052500in}}{\pgfqpoint{0.027278in}{0.046968in}}%
\pgfpathcurveto{\pgfqpoint{0.037123in}{0.037123in}}{\pgfqpoint{0.046968in}{0.027278in}}{\pgfqpoint{0.052500in}{0.013923in}}%
\pgfpathcurveto{\pgfqpoint{0.052500in}{0.000000in}}{\pgfqpoint{0.052500in}{-0.013923in}}{\pgfqpoint{0.046968in}{-0.027278in}}%
\pgfpathcurveto{\pgfqpoint{0.037123in}{-0.037123in}}{\pgfqpoint{0.027278in}{-0.046968in}}{\pgfqpoint{0.013923in}{-0.052500in}}%
\pgfpathclose%
\pgfpathmoveto{\pgfqpoint{0.166667in}{-0.058333in}}%
\pgfpathcurveto{\pgfqpoint{0.182137in}{-0.058333in}}{\pgfqpoint{0.196975in}{-0.052187in}}{\pgfqpoint{0.207915in}{-0.041248in}}%
\pgfpathcurveto{\pgfqpoint{0.218854in}{-0.030309in}}{\pgfqpoint{0.225000in}{-0.015470in}}{\pgfqpoint{0.225000in}{0.000000in}}%
\pgfpathcurveto{\pgfqpoint{0.225000in}{0.015470in}}{\pgfqpoint{0.218854in}{0.030309in}}{\pgfqpoint{0.207915in}{0.041248in}}%
\pgfpathcurveto{\pgfqpoint{0.196975in}{0.052187in}}{\pgfqpoint{0.182137in}{0.058333in}}{\pgfqpoint{0.166667in}{0.058333in}}%
\pgfpathcurveto{\pgfqpoint{0.151196in}{0.058333in}}{\pgfqpoint{0.136358in}{0.052187in}}{\pgfqpoint{0.125419in}{0.041248in}}%
\pgfpathcurveto{\pgfqpoint{0.114480in}{0.030309in}}{\pgfqpoint{0.108333in}{0.015470in}}{\pgfqpoint{0.108333in}{0.000000in}}%
\pgfpathcurveto{\pgfqpoint{0.108333in}{-0.015470in}}{\pgfqpoint{0.114480in}{-0.030309in}}{\pgfqpoint{0.125419in}{-0.041248in}}%
\pgfpathcurveto{\pgfqpoint{0.136358in}{-0.052187in}}{\pgfqpoint{0.151196in}{-0.058333in}}{\pgfqpoint{0.166667in}{-0.058333in}}%
\pgfpathclose%
\pgfpathmoveto{\pgfqpoint{0.166667in}{-0.052500in}}%
\pgfpathcurveto{\pgfqpoint{0.166667in}{-0.052500in}}{\pgfqpoint{0.152744in}{-0.052500in}}{\pgfqpoint{0.139389in}{-0.046968in}}%
\pgfpathcurveto{\pgfqpoint{0.129544in}{-0.037123in}}{\pgfqpoint{0.119698in}{-0.027278in}}{\pgfqpoint{0.114167in}{-0.013923in}}%
\pgfpathcurveto{\pgfqpoint{0.114167in}{0.000000in}}{\pgfqpoint{0.114167in}{0.013923in}}{\pgfqpoint{0.119698in}{0.027278in}}%
\pgfpathcurveto{\pgfqpoint{0.129544in}{0.037123in}}{\pgfqpoint{0.139389in}{0.046968in}}{\pgfqpoint{0.152744in}{0.052500in}}%
\pgfpathcurveto{\pgfqpoint{0.166667in}{0.052500in}}{\pgfqpoint{0.180590in}{0.052500in}}{\pgfqpoint{0.193945in}{0.046968in}}%
\pgfpathcurveto{\pgfqpoint{0.203790in}{0.037123in}}{\pgfqpoint{0.213635in}{0.027278in}}{\pgfqpoint{0.219167in}{0.013923in}}%
\pgfpathcurveto{\pgfqpoint{0.219167in}{0.000000in}}{\pgfqpoint{0.219167in}{-0.013923in}}{\pgfqpoint{0.213635in}{-0.027278in}}%
\pgfpathcurveto{\pgfqpoint{0.203790in}{-0.037123in}}{\pgfqpoint{0.193945in}{-0.046968in}}{\pgfqpoint{0.180590in}{-0.052500in}}%
\pgfpathclose%
\pgfpathmoveto{\pgfqpoint{0.333333in}{-0.058333in}}%
\pgfpathcurveto{\pgfqpoint{0.348804in}{-0.058333in}}{\pgfqpoint{0.363642in}{-0.052187in}}{\pgfqpoint{0.374581in}{-0.041248in}}%
\pgfpathcurveto{\pgfqpoint{0.385520in}{-0.030309in}}{\pgfqpoint{0.391667in}{-0.015470in}}{\pgfqpoint{0.391667in}{0.000000in}}%
\pgfpathcurveto{\pgfqpoint{0.391667in}{0.015470in}}{\pgfqpoint{0.385520in}{0.030309in}}{\pgfqpoint{0.374581in}{0.041248in}}%
\pgfpathcurveto{\pgfqpoint{0.363642in}{0.052187in}}{\pgfqpoint{0.348804in}{0.058333in}}{\pgfqpoint{0.333333in}{0.058333in}}%
\pgfpathcurveto{\pgfqpoint{0.317863in}{0.058333in}}{\pgfqpoint{0.303025in}{0.052187in}}{\pgfqpoint{0.292085in}{0.041248in}}%
\pgfpathcurveto{\pgfqpoint{0.281146in}{0.030309in}}{\pgfqpoint{0.275000in}{0.015470in}}{\pgfqpoint{0.275000in}{0.000000in}}%
\pgfpathcurveto{\pgfqpoint{0.275000in}{-0.015470in}}{\pgfqpoint{0.281146in}{-0.030309in}}{\pgfqpoint{0.292085in}{-0.041248in}}%
\pgfpathcurveto{\pgfqpoint{0.303025in}{-0.052187in}}{\pgfqpoint{0.317863in}{-0.058333in}}{\pgfqpoint{0.333333in}{-0.058333in}}%
\pgfpathclose%
\pgfpathmoveto{\pgfqpoint{0.333333in}{-0.052500in}}%
\pgfpathcurveto{\pgfqpoint{0.333333in}{-0.052500in}}{\pgfqpoint{0.319410in}{-0.052500in}}{\pgfqpoint{0.306055in}{-0.046968in}}%
\pgfpathcurveto{\pgfqpoint{0.296210in}{-0.037123in}}{\pgfqpoint{0.286365in}{-0.027278in}}{\pgfqpoint{0.280833in}{-0.013923in}}%
\pgfpathcurveto{\pgfqpoint{0.280833in}{0.000000in}}{\pgfqpoint{0.280833in}{0.013923in}}{\pgfqpoint{0.286365in}{0.027278in}}%
\pgfpathcurveto{\pgfqpoint{0.296210in}{0.037123in}}{\pgfqpoint{0.306055in}{0.046968in}}{\pgfqpoint{0.319410in}{0.052500in}}%
\pgfpathcurveto{\pgfqpoint{0.333333in}{0.052500in}}{\pgfqpoint{0.347256in}{0.052500in}}{\pgfqpoint{0.360611in}{0.046968in}}%
\pgfpathcurveto{\pgfqpoint{0.370456in}{0.037123in}}{\pgfqpoint{0.380302in}{0.027278in}}{\pgfqpoint{0.385833in}{0.013923in}}%
\pgfpathcurveto{\pgfqpoint{0.385833in}{0.000000in}}{\pgfqpoint{0.385833in}{-0.013923in}}{\pgfqpoint{0.380302in}{-0.027278in}}%
\pgfpathcurveto{\pgfqpoint{0.370456in}{-0.037123in}}{\pgfqpoint{0.360611in}{-0.046968in}}{\pgfqpoint{0.347256in}{-0.052500in}}%
\pgfpathclose%
\pgfpathmoveto{\pgfqpoint{0.500000in}{-0.058333in}}%
\pgfpathcurveto{\pgfqpoint{0.515470in}{-0.058333in}}{\pgfqpoint{0.530309in}{-0.052187in}}{\pgfqpoint{0.541248in}{-0.041248in}}%
\pgfpathcurveto{\pgfqpoint{0.552187in}{-0.030309in}}{\pgfqpoint{0.558333in}{-0.015470in}}{\pgfqpoint{0.558333in}{0.000000in}}%
\pgfpathcurveto{\pgfqpoint{0.558333in}{0.015470in}}{\pgfqpoint{0.552187in}{0.030309in}}{\pgfqpoint{0.541248in}{0.041248in}}%
\pgfpathcurveto{\pgfqpoint{0.530309in}{0.052187in}}{\pgfqpoint{0.515470in}{0.058333in}}{\pgfqpoint{0.500000in}{0.058333in}}%
\pgfpathcurveto{\pgfqpoint{0.484530in}{0.058333in}}{\pgfqpoint{0.469691in}{0.052187in}}{\pgfqpoint{0.458752in}{0.041248in}}%
\pgfpathcurveto{\pgfqpoint{0.447813in}{0.030309in}}{\pgfqpoint{0.441667in}{0.015470in}}{\pgfqpoint{0.441667in}{0.000000in}}%
\pgfpathcurveto{\pgfqpoint{0.441667in}{-0.015470in}}{\pgfqpoint{0.447813in}{-0.030309in}}{\pgfqpoint{0.458752in}{-0.041248in}}%
\pgfpathcurveto{\pgfqpoint{0.469691in}{-0.052187in}}{\pgfqpoint{0.484530in}{-0.058333in}}{\pgfqpoint{0.500000in}{-0.058333in}}%
\pgfpathclose%
\pgfpathmoveto{\pgfqpoint{0.500000in}{-0.052500in}}%
\pgfpathcurveto{\pgfqpoint{0.500000in}{-0.052500in}}{\pgfqpoint{0.486077in}{-0.052500in}}{\pgfqpoint{0.472722in}{-0.046968in}}%
\pgfpathcurveto{\pgfqpoint{0.462877in}{-0.037123in}}{\pgfqpoint{0.453032in}{-0.027278in}}{\pgfqpoint{0.447500in}{-0.013923in}}%
\pgfpathcurveto{\pgfqpoint{0.447500in}{0.000000in}}{\pgfqpoint{0.447500in}{0.013923in}}{\pgfqpoint{0.453032in}{0.027278in}}%
\pgfpathcurveto{\pgfqpoint{0.462877in}{0.037123in}}{\pgfqpoint{0.472722in}{0.046968in}}{\pgfqpoint{0.486077in}{0.052500in}}%
\pgfpathcurveto{\pgfqpoint{0.500000in}{0.052500in}}{\pgfqpoint{0.513923in}{0.052500in}}{\pgfqpoint{0.527278in}{0.046968in}}%
\pgfpathcurveto{\pgfqpoint{0.537123in}{0.037123in}}{\pgfqpoint{0.546968in}{0.027278in}}{\pgfqpoint{0.552500in}{0.013923in}}%
\pgfpathcurveto{\pgfqpoint{0.552500in}{0.000000in}}{\pgfqpoint{0.552500in}{-0.013923in}}{\pgfqpoint{0.546968in}{-0.027278in}}%
\pgfpathcurveto{\pgfqpoint{0.537123in}{-0.037123in}}{\pgfqpoint{0.527278in}{-0.046968in}}{\pgfqpoint{0.513923in}{-0.052500in}}%
\pgfpathclose%
\pgfpathmoveto{\pgfqpoint{0.666667in}{-0.058333in}}%
\pgfpathcurveto{\pgfqpoint{0.682137in}{-0.058333in}}{\pgfqpoint{0.696975in}{-0.052187in}}{\pgfqpoint{0.707915in}{-0.041248in}}%
\pgfpathcurveto{\pgfqpoint{0.718854in}{-0.030309in}}{\pgfqpoint{0.725000in}{-0.015470in}}{\pgfqpoint{0.725000in}{0.000000in}}%
\pgfpathcurveto{\pgfqpoint{0.725000in}{0.015470in}}{\pgfqpoint{0.718854in}{0.030309in}}{\pgfqpoint{0.707915in}{0.041248in}}%
\pgfpathcurveto{\pgfqpoint{0.696975in}{0.052187in}}{\pgfqpoint{0.682137in}{0.058333in}}{\pgfqpoint{0.666667in}{0.058333in}}%
\pgfpathcurveto{\pgfqpoint{0.651196in}{0.058333in}}{\pgfqpoint{0.636358in}{0.052187in}}{\pgfqpoint{0.625419in}{0.041248in}}%
\pgfpathcurveto{\pgfqpoint{0.614480in}{0.030309in}}{\pgfqpoint{0.608333in}{0.015470in}}{\pgfqpoint{0.608333in}{0.000000in}}%
\pgfpathcurveto{\pgfqpoint{0.608333in}{-0.015470in}}{\pgfqpoint{0.614480in}{-0.030309in}}{\pgfqpoint{0.625419in}{-0.041248in}}%
\pgfpathcurveto{\pgfqpoint{0.636358in}{-0.052187in}}{\pgfqpoint{0.651196in}{-0.058333in}}{\pgfqpoint{0.666667in}{-0.058333in}}%
\pgfpathclose%
\pgfpathmoveto{\pgfqpoint{0.666667in}{-0.052500in}}%
\pgfpathcurveto{\pgfqpoint{0.666667in}{-0.052500in}}{\pgfqpoint{0.652744in}{-0.052500in}}{\pgfqpoint{0.639389in}{-0.046968in}}%
\pgfpathcurveto{\pgfqpoint{0.629544in}{-0.037123in}}{\pgfqpoint{0.619698in}{-0.027278in}}{\pgfqpoint{0.614167in}{-0.013923in}}%
\pgfpathcurveto{\pgfqpoint{0.614167in}{0.000000in}}{\pgfqpoint{0.614167in}{0.013923in}}{\pgfqpoint{0.619698in}{0.027278in}}%
\pgfpathcurveto{\pgfqpoint{0.629544in}{0.037123in}}{\pgfqpoint{0.639389in}{0.046968in}}{\pgfqpoint{0.652744in}{0.052500in}}%
\pgfpathcurveto{\pgfqpoint{0.666667in}{0.052500in}}{\pgfqpoint{0.680590in}{0.052500in}}{\pgfqpoint{0.693945in}{0.046968in}}%
\pgfpathcurveto{\pgfqpoint{0.703790in}{0.037123in}}{\pgfqpoint{0.713635in}{0.027278in}}{\pgfqpoint{0.719167in}{0.013923in}}%
\pgfpathcurveto{\pgfqpoint{0.719167in}{0.000000in}}{\pgfqpoint{0.719167in}{-0.013923in}}{\pgfqpoint{0.713635in}{-0.027278in}}%
\pgfpathcurveto{\pgfqpoint{0.703790in}{-0.037123in}}{\pgfqpoint{0.693945in}{-0.046968in}}{\pgfqpoint{0.680590in}{-0.052500in}}%
\pgfpathclose%
\pgfpathmoveto{\pgfqpoint{0.833333in}{-0.058333in}}%
\pgfpathcurveto{\pgfqpoint{0.848804in}{-0.058333in}}{\pgfqpoint{0.863642in}{-0.052187in}}{\pgfqpoint{0.874581in}{-0.041248in}}%
\pgfpathcurveto{\pgfqpoint{0.885520in}{-0.030309in}}{\pgfqpoint{0.891667in}{-0.015470in}}{\pgfqpoint{0.891667in}{0.000000in}}%
\pgfpathcurveto{\pgfqpoint{0.891667in}{0.015470in}}{\pgfqpoint{0.885520in}{0.030309in}}{\pgfqpoint{0.874581in}{0.041248in}}%
\pgfpathcurveto{\pgfqpoint{0.863642in}{0.052187in}}{\pgfqpoint{0.848804in}{0.058333in}}{\pgfqpoint{0.833333in}{0.058333in}}%
\pgfpathcurveto{\pgfqpoint{0.817863in}{0.058333in}}{\pgfqpoint{0.803025in}{0.052187in}}{\pgfqpoint{0.792085in}{0.041248in}}%
\pgfpathcurveto{\pgfqpoint{0.781146in}{0.030309in}}{\pgfqpoint{0.775000in}{0.015470in}}{\pgfqpoint{0.775000in}{0.000000in}}%
\pgfpathcurveto{\pgfqpoint{0.775000in}{-0.015470in}}{\pgfqpoint{0.781146in}{-0.030309in}}{\pgfqpoint{0.792085in}{-0.041248in}}%
\pgfpathcurveto{\pgfqpoint{0.803025in}{-0.052187in}}{\pgfqpoint{0.817863in}{-0.058333in}}{\pgfqpoint{0.833333in}{-0.058333in}}%
\pgfpathclose%
\pgfpathmoveto{\pgfqpoint{0.833333in}{-0.052500in}}%
\pgfpathcurveto{\pgfqpoint{0.833333in}{-0.052500in}}{\pgfqpoint{0.819410in}{-0.052500in}}{\pgfqpoint{0.806055in}{-0.046968in}}%
\pgfpathcurveto{\pgfqpoint{0.796210in}{-0.037123in}}{\pgfqpoint{0.786365in}{-0.027278in}}{\pgfqpoint{0.780833in}{-0.013923in}}%
\pgfpathcurveto{\pgfqpoint{0.780833in}{0.000000in}}{\pgfqpoint{0.780833in}{0.013923in}}{\pgfqpoint{0.786365in}{0.027278in}}%
\pgfpathcurveto{\pgfqpoint{0.796210in}{0.037123in}}{\pgfqpoint{0.806055in}{0.046968in}}{\pgfqpoint{0.819410in}{0.052500in}}%
\pgfpathcurveto{\pgfqpoint{0.833333in}{0.052500in}}{\pgfqpoint{0.847256in}{0.052500in}}{\pgfqpoint{0.860611in}{0.046968in}}%
\pgfpathcurveto{\pgfqpoint{0.870456in}{0.037123in}}{\pgfqpoint{0.880302in}{0.027278in}}{\pgfqpoint{0.885833in}{0.013923in}}%
\pgfpathcurveto{\pgfqpoint{0.885833in}{0.000000in}}{\pgfqpoint{0.885833in}{-0.013923in}}{\pgfqpoint{0.880302in}{-0.027278in}}%
\pgfpathcurveto{\pgfqpoint{0.870456in}{-0.037123in}}{\pgfqpoint{0.860611in}{-0.046968in}}{\pgfqpoint{0.847256in}{-0.052500in}}%
\pgfpathclose%
\pgfpathmoveto{\pgfqpoint{1.000000in}{-0.058333in}}%
\pgfpathcurveto{\pgfqpoint{1.015470in}{-0.058333in}}{\pgfqpoint{1.030309in}{-0.052187in}}{\pgfqpoint{1.041248in}{-0.041248in}}%
\pgfpathcurveto{\pgfqpoint{1.052187in}{-0.030309in}}{\pgfqpoint{1.058333in}{-0.015470in}}{\pgfqpoint{1.058333in}{0.000000in}}%
\pgfpathcurveto{\pgfqpoint{1.058333in}{0.015470in}}{\pgfqpoint{1.052187in}{0.030309in}}{\pgfqpoint{1.041248in}{0.041248in}}%
\pgfpathcurveto{\pgfqpoint{1.030309in}{0.052187in}}{\pgfqpoint{1.015470in}{0.058333in}}{\pgfqpoint{1.000000in}{0.058333in}}%
\pgfpathcurveto{\pgfqpoint{0.984530in}{0.058333in}}{\pgfqpoint{0.969691in}{0.052187in}}{\pgfqpoint{0.958752in}{0.041248in}}%
\pgfpathcurveto{\pgfqpoint{0.947813in}{0.030309in}}{\pgfqpoint{0.941667in}{0.015470in}}{\pgfqpoint{0.941667in}{0.000000in}}%
\pgfpathcurveto{\pgfqpoint{0.941667in}{-0.015470in}}{\pgfqpoint{0.947813in}{-0.030309in}}{\pgfqpoint{0.958752in}{-0.041248in}}%
\pgfpathcurveto{\pgfqpoint{0.969691in}{-0.052187in}}{\pgfqpoint{0.984530in}{-0.058333in}}{\pgfqpoint{1.000000in}{-0.058333in}}%
\pgfpathclose%
\pgfpathmoveto{\pgfqpoint{1.000000in}{-0.052500in}}%
\pgfpathcurveto{\pgfqpoint{1.000000in}{-0.052500in}}{\pgfqpoint{0.986077in}{-0.052500in}}{\pgfqpoint{0.972722in}{-0.046968in}}%
\pgfpathcurveto{\pgfqpoint{0.962877in}{-0.037123in}}{\pgfqpoint{0.953032in}{-0.027278in}}{\pgfqpoint{0.947500in}{-0.013923in}}%
\pgfpathcurveto{\pgfqpoint{0.947500in}{0.000000in}}{\pgfqpoint{0.947500in}{0.013923in}}{\pgfqpoint{0.953032in}{0.027278in}}%
\pgfpathcurveto{\pgfqpoint{0.962877in}{0.037123in}}{\pgfqpoint{0.972722in}{0.046968in}}{\pgfqpoint{0.986077in}{0.052500in}}%
\pgfpathcurveto{\pgfqpoint{1.000000in}{0.052500in}}{\pgfqpoint{1.013923in}{0.052500in}}{\pgfqpoint{1.027278in}{0.046968in}}%
\pgfpathcurveto{\pgfqpoint{1.037123in}{0.037123in}}{\pgfqpoint{1.046968in}{0.027278in}}{\pgfqpoint{1.052500in}{0.013923in}}%
\pgfpathcurveto{\pgfqpoint{1.052500in}{0.000000in}}{\pgfqpoint{1.052500in}{-0.013923in}}{\pgfqpoint{1.046968in}{-0.027278in}}%
\pgfpathcurveto{\pgfqpoint{1.037123in}{-0.037123in}}{\pgfqpoint{1.027278in}{-0.046968in}}{\pgfqpoint{1.013923in}{-0.052500in}}%
\pgfpathclose%
\pgfpathmoveto{\pgfqpoint{0.083333in}{0.108333in}}%
\pgfpathcurveto{\pgfqpoint{0.098804in}{0.108333in}}{\pgfqpoint{0.113642in}{0.114480in}}{\pgfqpoint{0.124581in}{0.125419in}}%
\pgfpathcurveto{\pgfqpoint{0.135520in}{0.136358in}}{\pgfqpoint{0.141667in}{0.151196in}}{\pgfqpoint{0.141667in}{0.166667in}}%
\pgfpathcurveto{\pgfqpoint{0.141667in}{0.182137in}}{\pgfqpoint{0.135520in}{0.196975in}}{\pgfqpoint{0.124581in}{0.207915in}}%
\pgfpathcurveto{\pgfqpoint{0.113642in}{0.218854in}}{\pgfqpoint{0.098804in}{0.225000in}}{\pgfqpoint{0.083333in}{0.225000in}}%
\pgfpathcurveto{\pgfqpoint{0.067863in}{0.225000in}}{\pgfqpoint{0.053025in}{0.218854in}}{\pgfqpoint{0.042085in}{0.207915in}}%
\pgfpathcurveto{\pgfqpoint{0.031146in}{0.196975in}}{\pgfqpoint{0.025000in}{0.182137in}}{\pgfqpoint{0.025000in}{0.166667in}}%
\pgfpathcurveto{\pgfqpoint{0.025000in}{0.151196in}}{\pgfqpoint{0.031146in}{0.136358in}}{\pgfqpoint{0.042085in}{0.125419in}}%
\pgfpathcurveto{\pgfqpoint{0.053025in}{0.114480in}}{\pgfqpoint{0.067863in}{0.108333in}}{\pgfqpoint{0.083333in}{0.108333in}}%
\pgfpathclose%
\pgfpathmoveto{\pgfqpoint{0.083333in}{0.114167in}}%
\pgfpathcurveto{\pgfqpoint{0.083333in}{0.114167in}}{\pgfqpoint{0.069410in}{0.114167in}}{\pgfqpoint{0.056055in}{0.119698in}}%
\pgfpathcurveto{\pgfqpoint{0.046210in}{0.129544in}}{\pgfqpoint{0.036365in}{0.139389in}}{\pgfqpoint{0.030833in}{0.152744in}}%
\pgfpathcurveto{\pgfqpoint{0.030833in}{0.166667in}}{\pgfqpoint{0.030833in}{0.180590in}}{\pgfqpoint{0.036365in}{0.193945in}}%
\pgfpathcurveto{\pgfqpoint{0.046210in}{0.203790in}}{\pgfqpoint{0.056055in}{0.213635in}}{\pgfqpoint{0.069410in}{0.219167in}}%
\pgfpathcurveto{\pgfqpoint{0.083333in}{0.219167in}}{\pgfqpoint{0.097256in}{0.219167in}}{\pgfqpoint{0.110611in}{0.213635in}}%
\pgfpathcurveto{\pgfqpoint{0.120456in}{0.203790in}}{\pgfqpoint{0.130302in}{0.193945in}}{\pgfqpoint{0.135833in}{0.180590in}}%
\pgfpathcurveto{\pgfqpoint{0.135833in}{0.166667in}}{\pgfqpoint{0.135833in}{0.152744in}}{\pgfqpoint{0.130302in}{0.139389in}}%
\pgfpathcurveto{\pgfqpoint{0.120456in}{0.129544in}}{\pgfqpoint{0.110611in}{0.119698in}}{\pgfqpoint{0.097256in}{0.114167in}}%
\pgfpathclose%
\pgfpathmoveto{\pgfqpoint{0.250000in}{0.108333in}}%
\pgfpathcurveto{\pgfqpoint{0.265470in}{0.108333in}}{\pgfqpoint{0.280309in}{0.114480in}}{\pgfqpoint{0.291248in}{0.125419in}}%
\pgfpathcurveto{\pgfqpoint{0.302187in}{0.136358in}}{\pgfqpoint{0.308333in}{0.151196in}}{\pgfqpoint{0.308333in}{0.166667in}}%
\pgfpathcurveto{\pgfqpoint{0.308333in}{0.182137in}}{\pgfqpoint{0.302187in}{0.196975in}}{\pgfqpoint{0.291248in}{0.207915in}}%
\pgfpathcurveto{\pgfqpoint{0.280309in}{0.218854in}}{\pgfqpoint{0.265470in}{0.225000in}}{\pgfqpoint{0.250000in}{0.225000in}}%
\pgfpathcurveto{\pgfqpoint{0.234530in}{0.225000in}}{\pgfqpoint{0.219691in}{0.218854in}}{\pgfqpoint{0.208752in}{0.207915in}}%
\pgfpathcurveto{\pgfqpoint{0.197813in}{0.196975in}}{\pgfqpoint{0.191667in}{0.182137in}}{\pgfqpoint{0.191667in}{0.166667in}}%
\pgfpathcurveto{\pgfqpoint{0.191667in}{0.151196in}}{\pgfqpoint{0.197813in}{0.136358in}}{\pgfqpoint{0.208752in}{0.125419in}}%
\pgfpathcurveto{\pgfqpoint{0.219691in}{0.114480in}}{\pgfqpoint{0.234530in}{0.108333in}}{\pgfqpoint{0.250000in}{0.108333in}}%
\pgfpathclose%
\pgfpathmoveto{\pgfqpoint{0.250000in}{0.114167in}}%
\pgfpathcurveto{\pgfqpoint{0.250000in}{0.114167in}}{\pgfqpoint{0.236077in}{0.114167in}}{\pgfqpoint{0.222722in}{0.119698in}}%
\pgfpathcurveto{\pgfqpoint{0.212877in}{0.129544in}}{\pgfqpoint{0.203032in}{0.139389in}}{\pgfqpoint{0.197500in}{0.152744in}}%
\pgfpathcurveto{\pgfqpoint{0.197500in}{0.166667in}}{\pgfqpoint{0.197500in}{0.180590in}}{\pgfqpoint{0.203032in}{0.193945in}}%
\pgfpathcurveto{\pgfqpoint{0.212877in}{0.203790in}}{\pgfqpoint{0.222722in}{0.213635in}}{\pgfqpoint{0.236077in}{0.219167in}}%
\pgfpathcurveto{\pgfqpoint{0.250000in}{0.219167in}}{\pgfqpoint{0.263923in}{0.219167in}}{\pgfqpoint{0.277278in}{0.213635in}}%
\pgfpathcurveto{\pgfqpoint{0.287123in}{0.203790in}}{\pgfqpoint{0.296968in}{0.193945in}}{\pgfqpoint{0.302500in}{0.180590in}}%
\pgfpathcurveto{\pgfqpoint{0.302500in}{0.166667in}}{\pgfqpoint{0.302500in}{0.152744in}}{\pgfqpoint{0.296968in}{0.139389in}}%
\pgfpathcurveto{\pgfqpoint{0.287123in}{0.129544in}}{\pgfqpoint{0.277278in}{0.119698in}}{\pgfqpoint{0.263923in}{0.114167in}}%
\pgfpathclose%
\pgfpathmoveto{\pgfqpoint{0.416667in}{0.108333in}}%
\pgfpathcurveto{\pgfqpoint{0.432137in}{0.108333in}}{\pgfqpoint{0.446975in}{0.114480in}}{\pgfqpoint{0.457915in}{0.125419in}}%
\pgfpathcurveto{\pgfqpoint{0.468854in}{0.136358in}}{\pgfqpoint{0.475000in}{0.151196in}}{\pgfqpoint{0.475000in}{0.166667in}}%
\pgfpathcurveto{\pgfqpoint{0.475000in}{0.182137in}}{\pgfqpoint{0.468854in}{0.196975in}}{\pgfqpoint{0.457915in}{0.207915in}}%
\pgfpathcurveto{\pgfqpoint{0.446975in}{0.218854in}}{\pgfqpoint{0.432137in}{0.225000in}}{\pgfqpoint{0.416667in}{0.225000in}}%
\pgfpathcurveto{\pgfqpoint{0.401196in}{0.225000in}}{\pgfqpoint{0.386358in}{0.218854in}}{\pgfqpoint{0.375419in}{0.207915in}}%
\pgfpathcurveto{\pgfqpoint{0.364480in}{0.196975in}}{\pgfqpoint{0.358333in}{0.182137in}}{\pgfqpoint{0.358333in}{0.166667in}}%
\pgfpathcurveto{\pgfqpoint{0.358333in}{0.151196in}}{\pgfqpoint{0.364480in}{0.136358in}}{\pgfqpoint{0.375419in}{0.125419in}}%
\pgfpathcurveto{\pgfqpoint{0.386358in}{0.114480in}}{\pgfqpoint{0.401196in}{0.108333in}}{\pgfqpoint{0.416667in}{0.108333in}}%
\pgfpathclose%
\pgfpathmoveto{\pgfqpoint{0.416667in}{0.114167in}}%
\pgfpathcurveto{\pgfqpoint{0.416667in}{0.114167in}}{\pgfqpoint{0.402744in}{0.114167in}}{\pgfqpoint{0.389389in}{0.119698in}}%
\pgfpathcurveto{\pgfqpoint{0.379544in}{0.129544in}}{\pgfqpoint{0.369698in}{0.139389in}}{\pgfqpoint{0.364167in}{0.152744in}}%
\pgfpathcurveto{\pgfqpoint{0.364167in}{0.166667in}}{\pgfqpoint{0.364167in}{0.180590in}}{\pgfqpoint{0.369698in}{0.193945in}}%
\pgfpathcurveto{\pgfqpoint{0.379544in}{0.203790in}}{\pgfqpoint{0.389389in}{0.213635in}}{\pgfqpoint{0.402744in}{0.219167in}}%
\pgfpathcurveto{\pgfqpoint{0.416667in}{0.219167in}}{\pgfqpoint{0.430590in}{0.219167in}}{\pgfqpoint{0.443945in}{0.213635in}}%
\pgfpathcurveto{\pgfqpoint{0.453790in}{0.203790in}}{\pgfqpoint{0.463635in}{0.193945in}}{\pgfqpoint{0.469167in}{0.180590in}}%
\pgfpathcurveto{\pgfqpoint{0.469167in}{0.166667in}}{\pgfqpoint{0.469167in}{0.152744in}}{\pgfqpoint{0.463635in}{0.139389in}}%
\pgfpathcurveto{\pgfqpoint{0.453790in}{0.129544in}}{\pgfqpoint{0.443945in}{0.119698in}}{\pgfqpoint{0.430590in}{0.114167in}}%
\pgfpathclose%
\pgfpathmoveto{\pgfqpoint{0.583333in}{0.108333in}}%
\pgfpathcurveto{\pgfqpoint{0.598804in}{0.108333in}}{\pgfqpoint{0.613642in}{0.114480in}}{\pgfqpoint{0.624581in}{0.125419in}}%
\pgfpathcurveto{\pgfqpoint{0.635520in}{0.136358in}}{\pgfqpoint{0.641667in}{0.151196in}}{\pgfqpoint{0.641667in}{0.166667in}}%
\pgfpathcurveto{\pgfqpoint{0.641667in}{0.182137in}}{\pgfqpoint{0.635520in}{0.196975in}}{\pgfqpoint{0.624581in}{0.207915in}}%
\pgfpathcurveto{\pgfqpoint{0.613642in}{0.218854in}}{\pgfqpoint{0.598804in}{0.225000in}}{\pgfqpoint{0.583333in}{0.225000in}}%
\pgfpathcurveto{\pgfqpoint{0.567863in}{0.225000in}}{\pgfqpoint{0.553025in}{0.218854in}}{\pgfqpoint{0.542085in}{0.207915in}}%
\pgfpathcurveto{\pgfqpoint{0.531146in}{0.196975in}}{\pgfqpoint{0.525000in}{0.182137in}}{\pgfqpoint{0.525000in}{0.166667in}}%
\pgfpathcurveto{\pgfqpoint{0.525000in}{0.151196in}}{\pgfqpoint{0.531146in}{0.136358in}}{\pgfqpoint{0.542085in}{0.125419in}}%
\pgfpathcurveto{\pgfqpoint{0.553025in}{0.114480in}}{\pgfqpoint{0.567863in}{0.108333in}}{\pgfqpoint{0.583333in}{0.108333in}}%
\pgfpathclose%
\pgfpathmoveto{\pgfqpoint{0.583333in}{0.114167in}}%
\pgfpathcurveto{\pgfqpoint{0.583333in}{0.114167in}}{\pgfqpoint{0.569410in}{0.114167in}}{\pgfqpoint{0.556055in}{0.119698in}}%
\pgfpathcurveto{\pgfqpoint{0.546210in}{0.129544in}}{\pgfqpoint{0.536365in}{0.139389in}}{\pgfqpoint{0.530833in}{0.152744in}}%
\pgfpathcurveto{\pgfqpoint{0.530833in}{0.166667in}}{\pgfqpoint{0.530833in}{0.180590in}}{\pgfqpoint{0.536365in}{0.193945in}}%
\pgfpathcurveto{\pgfqpoint{0.546210in}{0.203790in}}{\pgfqpoint{0.556055in}{0.213635in}}{\pgfqpoint{0.569410in}{0.219167in}}%
\pgfpathcurveto{\pgfqpoint{0.583333in}{0.219167in}}{\pgfqpoint{0.597256in}{0.219167in}}{\pgfqpoint{0.610611in}{0.213635in}}%
\pgfpathcurveto{\pgfqpoint{0.620456in}{0.203790in}}{\pgfqpoint{0.630302in}{0.193945in}}{\pgfqpoint{0.635833in}{0.180590in}}%
\pgfpathcurveto{\pgfqpoint{0.635833in}{0.166667in}}{\pgfqpoint{0.635833in}{0.152744in}}{\pgfqpoint{0.630302in}{0.139389in}}%
\pgfpathcurveto{\pgfqpoint{0.620456in}{0.129544in}}{\pgfqpoint{0.610611in}{0.119698in}}{\pgfqpoint{0.597256in}{0.114167in}}%
\pgfpathclose%
\pgfpathmoveto{\pgfqpoint{0.750000in}{0.108333in}}%
\pgfpathcurveto{\pgfqpoint{0.765470in}{0.108333in}}{\pgfqpoint{0.780309in}{0.114480in}}{\pgfqpoint{0.791248in}{0.125419in}}%
\pgfpathcurveto{\pgfqpoint{0.802187in}{0.136358in}}{\pgfqpoint{0.808333in}{0.151196in}}{\pgfqpoint{0.808333in}{0.166667in}}%
\pgfpathcurveto{\pgfqpoint{0.808333in}{0.182137in}}{\pgfqpoint{0.802187in}{0.196975in}}{\pgfqpoint{0.791248in}{0.207915in}}%
\pgfpathcurveto{\pgfqpoint{0.780309in}{0.218854in}}{\pgfqpoint{0.765470in}{0.225000in}}{\pgfqpoint{0.750000in}{0.225000in}}%
\pgfpathcurveto{\pgfqpoint{0.734530in}{0.225000in}}{\pgfqpoint{0.719691in}{0.218854in}}{\pgfqpoint{0.708752in}{0.207915in}}%
\pgfpathcurveto{\pgfqpoint{0.697813in}{0.196975in}}{\pgfqpoint{0.691667in}{0.182137in}}{\pgfqpoint{0.691667in}{0.166667in}}%
\pgfpathcurveto{\pgfqpoint{0.691667in}{0.151196in}}{\pgfqpoint{0.697813in}{0.136358in}}{\pgfqpoint{0.708752in}{0.125419in}}%
\pgfpathcurveto{\pgfqpoint{0.719691in}{0.114480in}}{\pgfqpoint{0.734530in}{0.108333in}}{\pgfqpoint{0.750000in}{0.108333in}}%
\pgfpathclose%
\pgfpathmoveto{\pgfqpoint{0.750000in}{0.114167in}}%
\pgfpathcurveto{\pgfqpoint{0.750000in}{0.114167in}}{\pgfqpoint{0.736077in}{0.114167in}}{\pgfqpoint{0.722722in}{0.119698in}}%
\pgfpathcurveto{\pgfqpoint{0.712877in}{0.129544in}}{\pgfqpoint{0.703032in}{0.139389in}}{\pgfqpoint{0.697500in}{0.152744in}}%
\pgfpathcurveto{\pgfqpoint{0.697500in}{0.166667in}}{\pgfqpoint{0.697500in}{0.180590in}}{\pgfqpoint{0.703032in}{0.193945in}}%
\pgfpathcurveto{\pgfqpoint{0.712877in}{0.203790in}}{\pgfqpoint{0.722722in}{0.213635in}}{\pgfqpoint{0.736077in}{0.219167in}}%
\pgfpathcurveto{\pgfqpoint{0.750000in}{0.219167in}}{\pgfqpoint{0.763923in}{0.219167in}}{\pgfqpoint{0.777278in}{0.213635in}}%
\pgfpathcurveto{\pgfqpoint{0.787123in}{0.203790in}}{\pgfqpoint{0.796968in}{0.193945in}}{\pgfqpoint{0.802500in}{0.180590in}}%
\pgfpathcurveto{\pgfqpoint{0.802500in}{0.166667in}}{\pgfqpoint{0.802500in}{0.152744in}}{\pgfqpoint{0.796968in}{0.139389in}}%
\pgfpathcurveto{\pgfqpoint{0.787123in}{0.129544in}}{\pgfqpoint{0.777278in}{0.119698in}}{\pgfqpoint{0.763923in}{0.114167in}}%
\pgfpathclose%
\pgfpathmoveto{\pgfqpoint{0.916667in}{0.108333in}}%
\pgfpathcurveto{\pgfqpoint{0.932137in}{0.108333in}}{\pgfqpoint{0.946975in}{0.114480in}}{\pgfqpoint{0.957915in}{0.125419in}}%
\pgfpathcurveto{\pgfqpoint{0.968854in}{0.136358in}}{\pgfqpoint{0.975000in}{0.151196in}}{\pgfqpoint{0.975000in}{0.166667in}}%
\pgfpathcurveto{\pgfqpoint{0.975000in}{0.182137in}}{\pgfqpoint{0.968854in}{0.196975in}}{\pgfqpoint{0.957915in}{0.207915in}}%
\pgfpathcurveto{\pgfqpoint{0.946975in}{0.218854in}}{\pgfqpoint{0.932137in}{0.225000in}}{\pgfqpoint{0.916667in}{0.225000in}}%
\pgfpathcurveto{\pgfqpoint{0.901196in}{0.225000in}}{\pgfqpoint{0.886358in}{0.218854in}}{\pgfqpoint{0.875419in}{0.207915in}}%
\pgfpathcurveto{\pgfqpoint{0.864480in}{0.196975in}}{\pgfqpoint{0.858333in}{0.182137in}}{\pgfqpoint{0.858333in}{0.166667in}}%
\pgfpathcurveto{\pgfqpoint{0.858333in}{0.151196in}}{\pgfqpoint{0.864480in}{0.136358in}}{\pgfqpoint{0.875419in}{0.125419in}}%
\pgfpathcurveto{\pgfqpoint{0.886358in}{0.114480in}}{\pgfqpoint{0.901196in}{0.108333in}}{\pgfqpoint{0.916667in}{0.108333in}}%
\pgfpathclose%
\pgfpathmoveto{\pgfqpoint{0.916667in}{0.114167in}}%
\pgfpathcurveto{\pgfqpoint{0.916667in}{0.114167in}}{\pgfqpoint{0.902744in}{0.114167in}}{\pgfqpoint{0.889389in}{0.119698in}}%
\pgfpathcurveto{\pgfqpoint{0.879544in}{0.129544in}}{\pgfqpoint{0.869698in}{0.139389in}}{\pgfqpoint{0.864167in}{0.152744in}}%
\pgfpathcurveto{\pgfqpoint{0.864167in}{0.166667in}}{\pgfqpoint{0.864167in}{0.180590in}}{\pgfqpoint{0.869698in}{0.193945in}}%
\pgfpathcurveto{\pgfqpoint{0.879544in}{0.203790in}}{\pgfqpoint{0.889389in}{0.213635in}}{\pgfqpoint{0.902744in}{0.219167in}}%
\pgfpathcurveto{\pgfqpoint{0.916667in}{0.219167in}}{\pgfqpoint{0.930590in}{0.219167in}}{\pgfqpoint{0.943945in}{0.213635in}}%
\pgfpathcurveto{\pgfqpoint{0.953790in}{0.203790in}}{\pgfqpoint{0.963635in}{0.193945in}}{\pgfqpoint{0.969167in}{0.180590in}}%
\pgfpathcurveto{\pgfqpoint{0.969167in}{0.166667in}}{\pgfqpoint{0.969167in}{0.152744in}}{\pgfqpoint{0.963635in}{0.139389in}}%
\pgfpathcurveto{\pgfqpoint{0.953790in}{0.129544in}}{\pgfqpoint{0.943945in}{0.119698in}}{\pgfqpoint{0.930590in}{0.114167in}}%
\pgfpathclose%
\pgfpathmoveto{\pgfqpoint{0.000000in}{0.275000in}}%
\pgfpathcurveto{\pgfqpoint{0.015470in}{0.275000in}}{\pgfqpoint{0.030309in}{0.281146in}}{\pgfqpoint{0.041248in}{0.292085in}}%
\pgfpathcurveto{\pgfqpoint{0.052187in}{0.303025in}}{\pgfqpoint{0.058333in}{0.317863in}}{\pgfqpoint{0.058333in}{0.333333in}}%
\pgfpathcurveto{\pgfqpoint{0.058333in}{0.348804in}}{\pgfqpoint{0.052187in}{0.363642in}}{\pgfqpoint{0.041248in}{0.374581in}}%
\pgfpathcurveto{\pgfqpoint{0.030309in}{0.385520in}}{\pgfqpoint{0.015470in}{0.391667in}}{\pgfqpoint{0.000000in}{0.391667in}}%
\pgfpathcurveto{\pgfqpoint{-0.015470in}{0.391667in}}{\pgfqpoint{-0.030309in}{0.385520in}}{\pgfqpoint{-0.041248in}{0.374581in}}%
\pgfpathcurveto{\pgfqpoint{-0.052187in}{0.363642in}}{\pgfqpoint{-0.058333in}{0.348804in}}{\pgfqpoint{-0.058333in}{0.333333in}}%
\pgfpathcurveto{\pgfqpoint{-0.058333in}{0.317863in}}{\pgfqpoint{-0.052187in}{0.303025in}}{\pgfqpoint{-0.041248in}{0.292085in}}%
\pgfpathcurveto{\pgfqpoint{-0.030309in}{0.281146in}}{\pgfqpoint{-0.015470in}{0.275000in}}{\pgfqpoint{0.000000in}{0.275000in}}%
\pgfpathclose%
\pgfpathmoveto{\pgfqpoint{0.000000in}{0.280833in}}%
\pgfpathcurveto{\pgfqpoint{0.000000in}{0.280833in}}{\pgfqpoint{-0.013923in}{0.280833in}}{\pgfqpoint{-0.027278in}{0.286365in}}%
\pgfpathcurveto{\pgfqpoint{-0.037123in}{0.296210in}}{\pgfqpoint{-0.046968in}{0.306055in}}{\pgfqpoint{-0.052500in}{0.319410in}}%
\pgfpathcurveto{\pgfqpoint{-0.052500in}{0.333333in}}{\pgfqpoint{-0.052500in}{0.347256in}}{\pgfqpoint{-0.046968in}{0.360611in}}%
\pgfpathcurveto{\pgfqpoint{-0.037123in}{0.370456in}}{\pgfqpoint{-0.027278in}{0.380302in}}{\pgfqpoint{-0.013923in}{0.385833in}}%
\pgfpathcurveto{\pgfqpoint{0.000000in}{0.385833in}}{\pgfqpoint{0.013923in}{0.385833in}}{\pgfqpoint{0.027278in}{0.380302in}}%
\pgfpathcurveto{\pgfqpoint{0.037123in}{0.370456in}}{\pgfqpoint{0.046968in}{0.360611in}}{\pgfqpoint{0.052500in}{0.347256in}}%
\pgfpathcurveto{\pgfqpoint{0.052500in}{0.333333in}}{\pgfqpoint{0.052500in}{0.319410in}}{\pgfqpoint{0.046968in}{0.306055in}}%
\pgfpathcurveto{\pgfqpoint{0.037123in}{0.296210in}}{\pgfqpoint{0.027278in}{0.286365in}}{\pgfqpoint{0.013923in}{0.280833in}}%
\pgfpathclose%
\pgfpathmoveto{\pgfqpoint{0.166667in}{0.275000in}}%
\pgfpathcurveto{\pgfqpoint{0.182137in}{0.275000in}}{\pgfqpoint{0.196975in}{0.281146in}}{\pgfqpoint{0.207915in}{0.292085in}}%
\pgfpathcurveto{\pgfqpoint{0.218854in}{0.303025in}}{\pgfqpoint{0.225000in}{0.317863in}}{\pgfqpoint{0.225000in}{0.333333in}}%
\pgfpathcurveto{\pgfqpoint{0.225000in}{0.348804in}}{\pgfqpoint{0.218854in}{0.363642in}}{\pgfqpoint{0.207915in}{0.374581in}}%
\pgfpathcurveto{\pgfqpoint{0.196975in}{0.385520in}}{\pgfqpoint{0.182137in}{0.391667in}}{\pgfqpoint{0.166667in}{0.391667in}}%
\pgfpathcurveto{\pgfqpoint{0.151196in}{0.391667in}}{\pgfqpoint{0.136358in}{0.385520in}}{\pgfqpoint{0.125419in}{0.374581in}}%
\pgfpathcurveto{\pgfqpoint{0.114480in}{0.363642in}}{\pgfqpoint{0.108333in}{0.348804in}}{\pgfqpoint{0.108333in}{0.333333in}}%
\pgfpathcurveto{\pgfqpoint{0.108333in}{0.317863in}}{\pgfqpoint{0.114480in}{0.303025in}}{\pgfqpoint{0.125419in}{0.292085in}}%
\pgfpathcurveto{\pgfqpoint{0.136358in}{0.281146in}}{\pgfqpoint{0.151196in}{0.275000in}}{\pgfqpoint{0.166667in}{0.275000in}}%
\pgfpathclose%
\pgfpathmoveto{\pgfqpoint{0.166667in}{0.280833in}}%
\pgfpathcurveto{\pgfqpoint{0.166667in}{0.280833in}}{\pgfqpoint{0.152744in}{0.280833in}}{\pgfqpoint{0.139389in}{0.286365in}}%
\pgfpathcurveto{\pgfqpoint{0.129544in}{0.296210in}}{\pgfqpoint{0.119698in}{0.306055in}}{\pgfqpoint{0.114167in}{0.319410in}}%
\pgfpathcurveto{\pgfqpoint{0.114167in}{0.333333in}}{\pgfqpoint{0.114167in}{0.347256in}}{\pgfqpoint{0.119698in}{0.360611in}}%
\pgfpathcurveto{\pgfqpoint{0.129544in}{0.370456in}}{\pgfqpoint{0.139389in}{0.380302in}}{\pgfqpoint{0.152744in}{0.385833in}}%
\pgfpathcurveto{\pgfqpoint{0.166667in}{0.385833in}}{\pgfqpoint{0.180590in}{0.385833in}}{\pgfqpoint{0.193945in}{0.380302in}}%
\pgfpathcurveto{\pgfqpoint{0.203790in}{0.370456in}}{\pgfqpoint{0.213635in}{0.360611in}}{\pgfqpoint{0.219167in}{0.347256in}}%
\pgfpathcurveto{\pgfqpoint{0.219167in}{0.333333in}}{\pgfqpoint{0.219167in}{0.319410in}}{\pgfqpoint{0.213635in}{0.306055in}}%
\pgfpathcurveto{\pgfqpoint{0.203790in}{0.296210in}}{\pgfqpoint{0.193945in}{0.286365in}}{\pgfqpoint{0.180590in}{0.280833in}}%
\pgfpathclose%
\pgfpathmoveto{\pgfqpoint{0.333333in}{0.275000in}}%
\pgfpathcurveto{\pgfqpoint{0.348804in}{0.275000in}}{\pgfqpoint{0.363642in}{0.281146in}}{\pgfqpoint{0.374581in}{0.292085in}}%
\pgfpathcurveto{\pgfqpoint{0.385520in}{0.303025in}}{\pgfqpoint{0.391667in}{0.317863in}}{\pgfqpoint{0.391667in}{0.333333in}}%
\pgfpathcurveto{\pgfqpoint{0.391667in}{0.348804in}}{\pgfqpoint{0.385520in}{0.363642in}}{\pgfqpoint{0.374581in}{0.374581in}}%
\pgfpathcurveto{\pgfqpoint{0.363642in}{0.385520in}}{\pgfqpoint{0.348804in}{0.391667in}}{\pgfqpoint{0.333333in}{0.391667in}}%
\pgfpathcurveto{\pgfqpoint{0.317863in}{0.391667in}}{\pgfqpoint{0.303025in}{0.385520in}}{\pgfqpoint{0.292085in}{0.374581in}}%
\pgfpathcurveto{\pgfqpoint{0.281146in}{0.363642in}}{\pgfqpoint{0.275000in}{0.348804in}}{\pgfqpoint{0.275000in}{0.333333in}}%
\pgfpathcurveto{\pgfqpoint{0.275000in}{0.317863in}}{\pgfqpoint{0.281146in}{0.303025in}}{\pgfqpoint{0.292085in}{0.292085in}}%
\pgfpathcurveto{\pgfqpoint{0.303025in}{0.281146in}}{\pgfqpoint{0.317863in}{0.275000in}}{\pgfqpoint{0.333333in}{0.275000in}}%
\pgfpathclose%
\pgfpathmoveto{\pgfqpoint{0.333333in}{0.280833in}}%
\pgfpathcurveto{\pgfqpoint{0.333333in}{0.280833in}}{\pgfqpoint{0.319410in}{0.280833in}}{\pgfqpoint{0.306055in}{0.286365in}}%
\pgfpathcurveto{\pgfqpoint{0.296210in}{0.296210in}}{\pgfqpoint{0.286365in}{0.306055in}}{\pgfqpoint{0.280833in}{0.319410in}}%
\pgfpathcurveto{\pgfqpoint{0.280833in}{0.333333in}}{\pgfqpoint{0.280833in}{0.347256in}}{\pgfqpoint{0.286365in}{0.360611in}}%
\pgfpathcurveto{\pgfqpoint{0.296210in}{0.370456in}}{\pgfqpoint{0.306055in}{0.380302in}}{\pgfqpoint{0.319410in}{0.385833in}}%
\pgfpathcurveto{\pgfqpoint{0.333333in}{0.385833in}}{\pgfqpoint{0.347256in}{0.385833in}}{\pgfqpoint{0.360611in}{0.380302in}}%
\pgfpathcurveto{\pgfqpoint{0.370456in}{0.370456in}}{\pgfqpoint{0.380302in}{0.360611in}}{\pgfqpoint{0.385833in}{0.347256in}}%
\pgfpathcurveto{\pgfqpoint{0.385833in}{0.333333in}}{\pgfqpoint{0.385833in}{0.319410in}}{\pgfqpoint{0.380302in}{0.306055in}}%
\pgfpathcurveto{\pgfqpoint{0.370456in}{0.296210in}}{\pgfqpoint{0.360611in}{0.286365in}}{\pgfqpoint{0.347256in}{0.280833in}}%
\pgfpathclose%
\pgfpathmoveto{\pgfqpoint{0.500000in}{0.275000in}}%
\pgfpathcurveto{\pgfqpoint{0.515470in}{0.275000in}}{\pgfqpoint{0.530309in}{0.281146in}}{\pgfqpoint{0.541248in}{0.292085in}}%
\pgfpathcurveto{\pgfqpoint{0.552187in}{0.303025in}}{\pgfqpoint{0.558333in}{0.317863in}}{\pgfqpoint{0.558333in}{0.333333in}}%
\pgfpathcurveto{\pgfqpoint{0.558333in}{0.348804in}}{\pgfqpoint{0.552187in}{0.363642in}}{\pgfqpoint{0.541248in}{0.374581in}}%
\pgfpathcurveto{\pgfqpoint{0.530309in}{0.385520in}}{\pgfqpoint{0.515470in}{0.391667in}}{\pgfqpoint{0.500000in}{0.391667in}}%
\pgfpathcurveto{\pgfqpoint{0.484530in}{0.391667in}}{\pgfqpoint{0.469691in}{0.385520in}}{\pgfqpoint{0.458752in}{0.374581in}}%
\pgfpathcurveto{\pgfqpoint{0.447813in}{0.363642in}}{\pgfqpoint{0.441667in}{0.348804in}}{\pgfqpoint{0.441667in}{0.333333in}}%
\pgfpathcurveto{\pgfqpoint{0.441667in}{0.317863in}}{\pgfqpoint{0.447813in}{0.303025in}}{\pgfqpoint{0.458752in}{0.292085in}}%
\pgfpathcurveto{\pgfqpoint{0.469691in}{0.281146in}}{\pgfqpoint{0.484530in}{0.275000in}}{\pgfqpoint{0.500000in}{0.275000in}}%
\pgfpathclose%
\pgfpathmoveto{\pgfqpoint{0.500000in}{0.280833in}}%
\pgfpathcurveto{\pgfqpoint{0.500000in}{0.280833in}}{\pgfqpoint{0.486077in}{0.280833in}}{\pgfqpoint{0.472722in}{0.286365in}}%
\pgfpathcurveto{\pgfqpoint{0.462877in}{0.296210in}}{\pgfqpoint{0.453032in}{0.306055in}}{\pgfqpoint{0.447500in}{0.319410in}}%
\pgfpathcurveto{\pgfqpoint{0.447500in}{0.333333in}}{\pgfqpoint{0.447500in}{0.347256in}}{\pgfqpoint{0.453032in}{0.360611in}}%
\pgfpathcurveto{\pgfqpoint{0.462877in}{0.370456in}}{\pgfqpoint{0.472722in}{0.380302in}}{\pgfqpoint{0.486077in}{0.385833in}}%
\pgfpathcurveto{\pgfqpoint{0.500000in}{0.385833in}}{\pgfqpoint{0.513923in}{0.385833in}}{\pgfqpoint{0.527278in}{0.380302in}}%
\pgfpathcurveto{\pgfqpoint{0.537123in}{0.370456in}}{\pgfqpoint{0.546968in}{0.360611in}}{\pgfqpoint{0.552500in}{0.347256in}}%
\pgfpathcurveto{\pgfqpoint{0.552500in}{0.333333in}}{\pgfqpoint{0.552500in}{0.319410in}}{\pgfqpoint{0.546968in}{0.306055in}}%
\pgfpathcurveto{\pgfqpoint{0.537123in}{0.296210in}}{\pgfqpoint{0.527278in}{0.286365in}}{\pgfqpoint{0.513923in}{0.280833in}}%
\pgfpathclose%
\pgfpathmoveto{\pgfqpoint{0.666667in}{0.275000in}}%
\pgfpathcurveto{\pgfqpoint{0.682137in}{0.275000in}}{\pgfqpoint{0.696975in}{0.281146in}}{\pgfqpoint{0.707915in}{0.292085in}}%
\pgfpathcurveto{\pgfqpoint{0.718854in}{0.303025in}}{\pgfqpoint{0.725000in}{0.317863in}}{\pgfqpoint{0.725000in}{0.333333in}}%
\pgfpathcurveto{\pgfqpoint{0.725000in}{0.348804in}}{\pgfqpoint{0.718854in}{0.363642in}}{\pgfqpoint{0.707915in}{0.374581in}}%
\pgfpathcurveto{\pgfqpoint{0.696975in}{0.385520in}}{\pgfqpoint{0.682137in}{0.391667in}}{\pgfqpoint{0.666667in}{0.391667in}}%
\pgfpathcurveto{\pgfqpoint{0.651196in}{0.391667in}}{\pgfqpoint{0.636358in}{0.385520in}}{\pgfqpoint{0.625419in}{0.374581in}}%
\pgfpathcurveto{\pgfqpoint{0.614480in}{0.363642in}}{\pgfqpoint{0.608333in}{0.348804in}}{\pgfqpoint{0.608333in}{0.333333in}}%
\pgfpathcurveto{\pgfqpoint{0.608333in}{0.317863in}}{\pgfqpoint{0.614480in}{0.303025in}}{\pgfqpoint{0.625419in}{0.292085in}}%
\pgfpathcurveto{\pgfqpoint{0.636358in}{0.281146in}}{\pgfqpoint{0.651196in}{0.275000in}}{\pgfqpoint{0.666667in}{0.275000in}}%
\pgfpathclose%
\pgfpathmoveto{\pgfqpoint{0.666667in}{0.280833in}}%
\pgfpathcurveto{\pgfqpoint{0.666667in}{0.280833in}}{\pgfqpoint{0.652744in}{0.280833in}}{\pgfqpoint{0.639389in}{0.286365in}}%
\pgfpathcurveto{\pgfqpoint{0.629544in}{0.296210in}}{\pgfqpoint{0.619698in}{0.306055in}}{\pgfqpoint{0.614167in}{0.319410in}}%
\pgfpathcurveto{\pgfqpoint{0.614167in}{0.333333in}}{\pgfqpoint{0.614167in}{0.347256in}}{\pgfqpoint{0.619698in}{0.360611in}}%
\pgfpathcurveto{\pgfqpoint{0.629544in}{0.370456in}}{\pgfqpoint{0.639389in}{0.380302in}}{\pgfqpoint{0.652744in}{0.385833in}}%
\pgfpathcurveto{\pgfqpoint{0.666667in}{0.385833in}}{\pgfqpoint{0.680590in}{0.385833in}}{\pgfqpoint{0.693945in}{0.380302in}}%
\pgfpathcurveto{\pgfqpoint{0.703790in}{0.370456in}}{\pgfqpoint{0.713635in}{0.360611in}}{\pgfqpoint{0.719167in}{0.347256in}}%
\pgfpathcurveto{\pgfqpoint{0.719167in}{0.333333in}}{\pgfqpoint{0.719167in}{0.319410in}}{\pgfqpoint{0.713635in}{0.306055in}}%
\pgfpathcurveto{\pgfqpoint{0.703790in}{0.296210in}}{\pgfqpoint{0.693945in}{0.286365in}}{\pgfqpoint{0.680590in}{0.280833in}}%
\pgfpathclose%
\pgfpathmoveto{\pgfqpoint{0.833333in}{0.275000in}}%
\pgfpathcurveto{\pgfqpoint{0.848804in}{0.275000in}}{\pgfqpoint{0.863642in}{0.281146in}}{\pgfqpoint{0.874581in}{0.292085in}}%
\pgfpathcurveto{\pgfqpoint{0.885520in}{0.303025in}}{\pgfqpoint{0.891667in}{0.317863in}}{\pgfqpoint{0.891667in}{0.333333in}}%
\pgfpathcurveto{\pgfqpoint{0.891667in}{0.348804in}}{\pgfqpoint{0.885520in}{0.363642in}}{\pgfqpoint{0.874581in}{0.374581in}}%
\pgfpathcurveto{\pgfqpoint{0.863642in}{0.385520in}}{\pgfqpoint{0.848804in}{0.391667in}}{\pgfqpoint{0.833333in}{0.391667in}}%
\pgfpathcurveto{\pgfqpoint{0.817863in}{0.391667in}}{\pgfqpoint{0.803025in}{0.385520in}}{\pgfqpoint{0.792085in}{0.374581in}}%
\pgfpathcurveto{\pgfqpoint{0.781146in}{0.363642in}}{\pgfqpoint{0.775000in}{0.348804in}}{\pgfqpoint{0.775000in}{0.333333in}}%
\pgfpathcurveto{\pgfqpoint{0.775000in}{0.317863in}}{\pgfqpoint{0.781146in}{0.303025in}}{\pgfqpoint{0.792085in}{0.292085in}}%
\pgfpathcurveto{\pgfqpoint{0.803025in}{0.281146in}}{\pgfqpoint{0.817863in}{0.275000in}}{\pgfqpoint{0.833333in}{0.275000in}}%
\pgfpathclose%
\pgfpathmoveto{\pgfqpoint{0.833333in}{0.280833in}}%
\pgfpathcurveto{\pgfqpoint{0.833333in}{0.280833in}}{\pgfqpoint{0.819410in}{0.280833in}}{\pgfqpoint{0.806055in}{0.286365in}}%
\pgfpathcurveto{\pgfqpoint{0.796210in}{0.296210in}}{\pgfqpoint{0.786365in}{0.306055in}}{\pgfqpoint{0.780833in}{0.319410in}}%
\pgfpathcurveto{\pgfqpoint{0.780833in}{0.333333in}}{\pgfqpoint{0.780833in}{0.347256in}}{\pgfqpoint{0.786365in}{0.360611in}}%
\pgfpathcurveto{\pgfqpoint{0.796210in}{0.370456in}}{\pgfqpoint{0.806055in}{0.380302in}}{\pgfqpoint{0.819410in}{0.385833in}}%
\pgfpathcurveto{\pgfqpoint{0.833333in}{0.385833in}}{\pgfqpoint{0.847256in}{0.385833in}}{\pgfqpoint{0.860611in}{0.380302in}}%
\pgfpathcurveto{\pgfqpoint{0.870456in}{0.370456in}}{\pgfqpoint{0.880302in}{0.360611in}}{\pgfqpoint{0.885833in}{0.347256in}}%
\pgfpathcurveto{\pgfqpoint{0.885833in}{0.333333in}}{\pgfqpoint{0.885833in}{0.319410in}}{\pgfqpoint{0.880302in}{0.306055in}}%
\pgfpathcurveto{\pgfqpoint{0.870456in}{0.296210in}}{\pgfqpoint{0.860611in}{0.286365in}}{\pgfqpoint{0.847256in}{0.280833in}}%
\pgfpathclose%
\pgfpathmoveto{\pgfqpoint{1.000000in}{0.275000in}}%
\pgfpathcurveto{\pgfqpoint{1.015470in}{0.275000in}}{\pgfqpoint{1.030309in}{0.281146in}}{\pgfqpoint{1.041248in}{0.292085in}}%
\pgfpathcurveto{\pgfqpoint{1.052187in}{0.303025in}}{\pgfqpoint{1.058333in}{0.317863in}}{\pgfqpoint{1.058333in}{0.333333in}}%
\pgfpathcurveto{\pgfqpoint{1.058333in}{0.348804in}}{\pgfqpoint{1.052187in}{0.363642in}}{\pgfqpoint{1.041248in}{0.374581in}}%
\pgfpathcurveto{\pgfqpoint{1.030309in}{0.385520in}}{\pgfqpoint{1.015470in}{0.391667in}}{\pgfqpoint{1.000000in}{0.391667in}}%
\pgfpathcurveto{\pgfqpoint{0.984530in}{0.391667in}}{\pgfqpoint{0.969691in}{0.385520in}}{\pgfqpoint{0.958752in}{0.374581in}}%
\pgfpathcurveto{\pgfqpoint{0.947813in}{0.363642in}}{\pgfqpoint{0.941667in}{0.348804in}}{\pgfqpoint{0.941667in}{0.333333in}}%
\pgfpathcurveto{\pgfqpoint{0.941667in}{0.317863in}}{\pgfqpoint{0.947813in}{0.303025in}}{\pgfqpoint{0.958752in}{0.292085in}}%
\pgfpathcurveto{\pgfqpoint{0.969691in}{0.281146in}}{\pgfqpoint{0.984530in}{0.275000in}}{\pgfqpoint{1.000000in}{0.275000in}}%
\pgfpathclose%
\pgfpathmoveto{\pgfqpoint{1.000000in}{0.280833in}}%
\pgfpathcurveto{\pgfqpoint{1.000000in}{0.280833in}}{\pgfqpoint{0.986077in}{0.280833in}}{\pgfqpoint{0.972722in}{0.286365in}}%
\pgfpathcurveto{\pgfqpoint{0.962877in}{0.296210in}}{\pgfqpoint{0.953032in}{0.306055in}}{\pgfqpoint{0.947500in}{0.319410in}}%
\pgfpathcurveto{\pgfqpoint{0.947500in}{0.333333in}}{\pgfqpoint{0.947500in}{0.347256in}}{\pgfqpoint{0.953032in}{0.360611in}}%
\pgfpathcurveto{\pgfqpoint{0.962877in}{0.370456in}}{\pgfqpoint{0.972722in}{0.380302in}}{\pgfqpoint{0.986077in}{0.385833in}}%
\pgfpathcurveto{\pgfqpoint{1.000000in}{0.385833in}}{\pgfqpoint{1.013923in}{0.385833in}}{\pgfqpoint{1.027278in}{0.380302in}}%
\pgfpathcurveto{\pgfqpoint{1.037123in}{0.370456in}}{\pgfqpoint{1.046968in}{0.360611in}}{\pgfqpoint{1.052500in}{0.347256in}}%
\pgfpathcurveto{\pgfqpoint{1.052500in}{0.333333in}}{\pgfqpoint{1.052500in}{0.319410in}}{\pgfqpoint{1.046968in}{0.306055in}}%
\pgfpathcurveto{\pgfqpoint{1.037123in}{0.296210in}}{\pgfqpoint{1.027278in}{0.286365in}}{\pgfqpoint{1.013923in}{0.280833in}}%
\pgfpathclose%
\pgfpathmoveto{\pgfqpoint{0.083333in}{0.441667in}}%
\pgfpathcurveto{\pgfqpoint{0.098804in}{0.441667in}}{\pgfqpoint{0.113642in}{0.447813in}}{\pgfqpoint{0.124581in}{0.458752in}}%
\pgfpathcurveto{\pgfqpoint{0.135520in}{0.469691in}}{\pgfqpoint{0.141667in}{0.484530in}}{\pgfqpoint{0.141667in}{0.500000in}}%
\pgfpathcurveto{\pgfqpoint{0.141667in}{0.515470in}}{\pgfqpoint{0.135520in}{0.530309in}}{\pgfqpoint{0.124581in}{0.541248in}}%
\pgfpathcurveto{\pgfqpoint{0.113642in}{0.552187in}}{\pgfqpoint{0.098804in}{0.558333in}}{\pgfqpoint{0.083333in}{0.558333in}}%
\pgfpathcurveto{\pgfqpoint{0.067863in}{0.558333in}}{\pgfqpoint{0.053025in}{0.552187in}}{\pgfqpoint{0.042085in}{0.541248in}}%
\pgfpathcurveto{\pgfqpoint{0.031146in}{0.530309in}}{\pgfqpoint{0.025000in}{0.515470in}}{\pgfqpoint{0.025000in}{0.500000in}}%
\pgfpathcurveto{\pgfqpoint{0.025000in}{0.484530in}}{\pgfqpoint{0.031146in}{0.469691in}}{\pgfqpoint{0.042085in}{0.458752in}}%
\pgfpathcurveto{\pgfqpoint{0.053025in}{0.447813in}}{\pgfqpoint{0.067863in}{0.441667in}}{\pgfqpoint{0.083333in}{0.441667in}}%
\pgfpathclose%
\pgfpathmoveto{\pgfqpoint{0.083333in}{0.447500in}}%
\pgfpathcurveto{\pgfqpoint{0.083333in}{0.447500in}}{\pgfqpoint{0.069410in}{0.447500in}}{\pgfqpoint{0.056055in}{0.453032in}}%
\pgfpathcurveto{\pgfqpoint{0.046210in}{0.462877in}}{\pgfqpoint{0.036365in}{0.472722in}}{\pgfqpoint{0.030833in}{0.486077in}}%
\pgfpathcurveto{\pgfqpoint{0.030833in}{0.500000in}}{\pgfqpoint{0.030833in}{0.513923in}}{\pgfqpoint{0.036365in}{0.527278in}}%
\pgfpathcurveto{\pgfqpoint{0.046210in}{0.537123in}}{\pgfqpoint{0.056055in}{0.546968in}}{\pgfqpoint{0.069410in}{0.552500in}}%
\pgfpathcurveto{\pgfqpoint{0.083333in}{0.552500in}}{\pgfqpoint{0.097256in}{0.552500in}}{\pgfqpoint{0.110611in}{0.546968in}}%
\pgfpathcurveto{\pgfqpoint{0.120456in}{0.537123in}}{\pgfqpoint{0.130302in}{0.527278in}}{\pgfqpoint{0.135833in}{0.513923in}}%
\pgfpathcurveto{\pgfqpoint{0.135833in}{0.500000in}}{\pgfqpoint{0.135833in}{0.486077in}}{\pgfqpoint{0.130302in}{0.472722in}}%
\pgfpathcurveto{\pgfqpoint{0.120456in}{0.462877in}}{\pgfqpoint{0.110611in}{0.453032in}}{\pgfqpoint{0.097256in}{0.447500in}}%
\pgfpathclose%
\pgfpathmoveto{\pgfqpoint{0.250000in}{0.441667in}}%
\pgfpathcurveto{\pgfqpoint{0.265470in}{0.441667in}}{\pgfqpoint{0.280309in}{0.447813in}}{\pgfqpoint{0.291248in}{0.458752in}}%
\pgfpathcurveto{\pgfqpoint{0.302187in}{0.469691in}}{\pgfqpoint{0.308333in}{0.484530in}}{\pgfqpoint{0.308333in}{0.500000in}}%
\pgfpathcurveto{\pgfqpoint{0.308333in}{0.515470in}}{\pgfqpoint{0.302187in}{0.530309in}}{\pgfqpoint{0.291248in}{0.541248in}}%
\pgfpathcurveto{\pgfqpoint{0.280309in}{0.552187in}}{\pgfqpoint{0.265470in}{0.558333in}}{\pgfqpoint{0.250000in}{0.558333in}}%
\pgfpathcurveto{\pgfqpoint{0.234530in}{0.558333in}}{\pgfqpoint{0.219691in}{0.552187in}}{\pgfqpoint{0.208752in}{0.541248in}}%
\pgfpathcurveto{\pgfqpoint{0.197813in}{0.530309in}}{\pgfqpoint{0.191667in}{0.515470in}}{\pgfqpoint{0.191667in}{0.500000in}}%
\pgfpathcurveto{\pgfqpoint{0.191667in}{0.484530in}}{\pgfqpoint{0.197813in}{0.469691in}}{\pgfqpoint{0.208752in}{0.458752in}}%
\pgfpathcurveto{\pgfqpoint{0.219691in}{0.447813in}}{\pgfqpoint{0.234530in}{0.441667in}}{\pgfqpoint{0.250000in}{0.441667in}}%
\pgfpathclose%
\pgfpathmoveto{\pgfqpoint{0.250000in}{0.447500in}}%
\pgfpathcurveto{\pgfqpoint{0.250000in}{0.447500in}}{\pgfqpoint{0.236077in}{0.447500in}}{\pgfqpoint{0.222722in}{0.453032in}}%
\pgfpathcurveto{\pgfqpoint{0.212877in}{0.462877in}}{\pgfqpoint{0.203032in}{0.472722in}}{\pgfqpoint{0.197500in}{0.486077in}}%
\pgfpathcurveto{\pgfqpoint{0.197500in}{0.500000in}}{\pgfqpoint{0.197500in}{0.513923in}}{\pgfqpoint{0.203032in}{0.527278in}}%
\pgfpathcurveto{\pgfqpoint{0.212877in}{0.537123in}}{\pgfqpoint{0.222722in}{0.546968in}}{\pgfqpoint{0.236077in}{0.552500in}}%
\pgfpathcurveto{\pgfqpoint{0.250000in}{0.552500in}}{\pgfqpoint{0.263923in}{0.552500in}}{\pgfqpoint{0.277278in}{0.546968in}}%
\pgfpathcurveto{\pgfqpoint{0.287123in}{0.537123in}}{\pgfqpoint{0.296968in}{0.527278in}}{\pgfqpoint{0.302500in}{0.513923in}}%
\pgfpathcurveto{\pgfqpoint{0.302500in}{0.500000in}}{\pgfqpoint{0.302500in}{0.486077in}}{\pgfqpoint{0.296968in}{0.472722in}}%
\pgfpathcurveto{\pgfqpoint{0.287123in}{0.462877in}}{\pgfqpoint{0.277278in}{0.453032in}}{\pgfqpoint{0.263923in}{0.447500in}}%
\pgfpathclose%
\pgfpathmoveto{\pgfqpoint{0.416667in}{0.441667in}}%
\pgfpathcurveto{\pgfqpoint{0.432137in}{0.441667in}}{\pgfqpoint{0.446975in}{0.447813in}}{\pgfqpoint{0.457915in}{0.458752in}}%
\pgfpathcurveto{\pgfqpoint{0.468854in}{0.469691in}}{\pgfqpoint{0.475000in}{0.484530in}}{\pgfqpoint{0.475000in}{0.500000in}}%
\pgfpathcurveto{\pgfqpoint{0.475000in}{0.515470in}}{\pgfqpoint{0.468854in}{0.530309in}}{\pgfqpoint{0.457915in}{0.541248in}}%
\pgfpathcurveto{\pgfqpoint{0.446975in}{0.552187in}}{\pgfqpoint{0.432137in}{0.558333in}}{\pgfqpoint{0.416667in}{0.558333in}}%
\pgfpathcurveto{\pgfqpoint{0.401196in}{0.558333in}}{\pgfqpoint{0.386358in}{0.552187in}}{\pgfqpoint{0.375419in}{0.541248in}}%
\pgfpathcurveto{\pgfqpoint{0.364480in}{0.530309in}}{\pgfqpoint{0.358333in}{0.515470in}}{\pgfqpoint{0.358333in}{0.500000in}}%
\pgfpathcurveto{\pgfqpoint{0.358333in}{0.484530in}}{\pgfqpoint{0.364480in}{0.469691in}}{\pgfqpoint{0.375419in}{0.458752in}}%
\pgfpathcurveto{\pgfqpoint{0.386358in}{0.447813in}}{\pgfqpoint{0.401196in}{0.441667in}}{\pgfqpoint{0.416667in}{0.441667in}}%
\pgfpathclose%
\pgfpathmoveto{\pgfqpoint{0.416667in}{0.447500in}}%
\pgfpathcurveto{\pgfqpoint{0.416667in}{0.447500in}}{\pgfqpoint{0.402744in}{0.447500in}}{\pgfqpoint{0.389389in}{0.453032in}}%
\pgfpathcurveto{\pgfqpoint{0.379544in}{0.462877in}}{\pgfqpoint{0.369698in}{0.472722in}}{\pgfqpoint{0.364167in}{0.486077in}}%
\pgfpathcurveto{\pgfqpoint{0.364167in}{0.500000in}}{\pgfqpoint{0.364167in}{0.513923in}}{\pgfqpoint{0.369698in}{0.527278in}}%
\pgfpathcurveto{\pgfqpoint{0.379544in}{0.537123in}}{\pgfqpoint{0.389389in}{0.546968in}}{\pgfqpoint{0.402744in}{0.552500in}}%
\pgfpathcurveto{\pgfqpoint{0.416667in}{0.552500in}}{\pgfqpoint{0.430590in}{0.552500in}}{\pgfqpoint{0.443945in}{0.546968in}}%
\pgfpathcurveto{\pgfqpoint{0.453790in}{0.537123in}}{\pgfqpoint{0.463635in}{0.527278in}}{\pgfqpoint{0.469167in}{0.513923in}}%
\pgfpathcurveto{\pgfqpoint{0.469167in}{0.500000in}}{\pgfqpoint{0.469167in}{0.486077in}}{\pgfqpoint{0.463635in}{0.472722in}}%
\pgfpathcurveto{\pgfqpoint{0.453790in}{0.462877in}}{\pgfqpoint{0.443945in}{0.453032in}}{\pgfqpoint{0.430590in}{0.447500in}}%
\pgfpathclose%
\pgfpathmoveto{\pgfqpoint{0.583333in}{0.441667in}}%
\pgfpathcurveto{\pgfqpoint{0.598804in}{0.441667in}}{\pgfqpoint{0.613642in}{0.447813in}}{\pgfqpoint{0.624581in}{0.458752in}}%
\pgfpathcurveto{\pgfqpoint{0.635520in}{0.469691in}}{\pgfqpoint{0.641667in}{0.484530in}}{\pgfqpoint{0.641667in}{0.500000in}}%
\pgfpathcurveto{\pgfqpoint{0.641667in}{0.515470in}}{\pgfqpoint{0.635520in}{0.530309in}}{\pgfqpoint{0.624581in}{0.541248in}}%
\pgfpathcurveto{\pgfqpoint{0.613642in}{0.552187in}}{\pgfqpoint{0.598804in}{0.558333in}}{\pgfqpoint{0.583333in}{0.558333in}}%
\pgfpathcurveto{\pgfqpoint{0.567863in}{0.558333in}}{\pgfqpoint{0.553025in}{0.552187in}}{\pgfqpoint{0.542085in}{0.541248in}}%
\pgfpathcurveto{\pgfqpoint{0.531146in}{0.530309in}}{\pgfqpoint{0.525000in}{0.515470in}}{\pgfqpoint{0.525000in}{0.500000in}}%
\pgfpathcurveto{\pgfqpoint{0.525000in}{0.484530in}}{\pgfqpoint{0.531146in}{0.469691in}}{\pgfqpoint{0.542085in}{0.458752in}}%
\pgfpathcurveto{\pgfqpoint{0.553025in}{0.447813in}}{\pgfqpoint{0.567863in}{0.441667in}}{\pgfqpoint{0.583333in}{0.441667in}}%
\pgfpathclose%
\pgfpathmoveto{\pgfqpoint{0.583333in}{0.447500in}}%
\pgfpathcurveto{\pgfqpoint{0.583333in}{0.447500in}}{\pgfqpoint{0.569410in}{0.447500in}}{\pgfqpoint{0.556055in}{0.453032in}}%
\pgfpathcurveto{\pgfqpoint{0.546210in}{0.462877in}}{\pgfqpoint{0.536365in}{0.472722in}}{\pgfqpoint{0.530833in}{0.486077in}}%
\pgfpathcurveto{\pgfqpoint{0.530833in}{0.500000in}}{\pgfqpoint{0.530833in}{0.513923in}}{\pgfqpoint{0.536365in}{0.527278in}}%
\pgfpathcurveto{\pgfqpoint{0.546210in}{0.537123in}}{\pgfqpoint{0.556055in}{0.546968in}}{\pgfqpoint{0.569410in}{0.552500in}}%
\pgfpathcurveto{\pgfqpoint{0.583333in}{0.552500in}}{\pgfqpoint{0.597256in}{0.552500in}}{\pgfqpoint{0.610611in}{0.546968in}}%
\pgfpathcurveto{\pgfqpoint{0.620456in}{0.537123in}}{\pgfqpoint{0.630302in}{0.527278in}}{\pgfqpoint{0.635833in}{0.513923in}}%
\pgfpathcurveto{\pgfqpoint{0.635833in}{0.500000in}}{\pgfqpoint{0.635833in}{0.486077in}}{\pgfqpoint{0.630302in}{0.472722in}}%
\pgfpathcurveto{\pgfqpoint{0.620456in}{0.462877in}}{\pgfqpoint{0.610611in}{0.453032in}}{\pgfqpoint{0.597256in}{0.447500in}}%
\pgfpathclose%
\pgfpathmoveto{\pgfqpoint{0.750000in}{0.441667in}}%
\pgfpathcurveto{\pgfqpoint{0.765470in}{0.441667in}}{\pgfqpoint{0.780309in}{0.447813in}}{\pgfqpoint{0.791248in}{0.458752in}}%
\pgfpathcurveto{\pgfqpoint{0.802187in}{0.469691in}}{\pgfqpoint{0.808333in}{0.484530in}}{\pgfqpoint{0.808333in}{0.500000in}}%
\pgfpathcurveto{\pgfqpoint{0.808333in}{0.515470in}}{\pgfqpoint{0.802187in}{0.530309in}}{\pgfqpoint{0.791248in}{0.541248in}}%
\pgfpathcurveto{\pgfqpoint{0.780309in}{0.552187in}}{\pgfqpoint{0.765470in}{0.558333in}}{\pgfqpoint{0.750000in}{0.558333in}}%
\pgfpathcurveto{\pgfqpoint{0.734530in}{0.558333in}}{\pgfqpoint{0.719691in}{0.552187in}}{\pgfqpoint{0.708752in}{0.541248in}}%
\pgfpathcurveto{\pgfqpoint{0.697813in}{0.530309in}}{\pgfqpoint{0.691667in}{0.515470in}}{\pgfqpoint{0.691667in}{0.500000in}}%
\pgfpathcurveto{\pgfqpoint{0.691667in}{0.484530in}}{\pgfqpoint{0.697813in}{0.469691in}}{\pgfqpoint{0.708752in}{0.458752in}}%
\pgfpathcurveto{\pgfqpoint{0.719691in}{0.447813in}}{\pgfqpoint{0.734530in}{0.441667in}}{\pgfqpoint{0.750000in}{0.441667in}}%
\pgfpathclose%
\pgfpathmoveto{\pgfqpoint{0.750000in}{0.447500in}}%
\pgfpathcurveto{\pgfqpoint{0.750000in}{0.447500in}}{\pgfqpoint{0.736077in}{0.447500in}}{\pgfqpoint{0.722722in}{0.453032in}}%
\pgfpathcurveto{\pgfqpoint{0.712877in}{0.462877in}}{\pgfqpoint{0.703032in}{0.472722in}}{\pgfqpoint{0.697500in}{0.486077in}}%
\pgfpathcurveto{\pgfqpoint{0.697500in}{0.500000in}}{\pgfqpoint{0.697500in}{0.513923in}}{\pgfqpoint{0.703032in}{0.527278in}}%
\pgfpathcurveto{\pgfqpoint{0.712877in}{0.537123in}}{\pgfqpoint{0.722722in}{0.546968in}}{\pgfqpoint{0.736077in}{0.552500in}}%
\pgfpathcurveto{\pgfqpoint{0.750000in}{0.552500in}}{\pgfqpoint{0.763923in}{0.552500in}}{\pgfqpoint{0.777278in}{0.546968in}}%
\pgfpathcurveto{\pgfqpoint{0.787123in}{0.537123in}}{\pgfqpoint{0.796968in}{0.527278in}}{\pgfqpoint{0.802500in}{0.513923in}}%
\pgfpathcurveto{\pgfqpoint{0.802500in}{0.500000in}}{\pgfqpoint{0.802500in}{0.486077in}}{\pgfqpoint{0.796968in}{0.472722in}}%
\pgfpathcurveto{\pgfqpoint{0.787123in}{0.462877in}}{\pgfqpoint{0.777278in}{0.453032in}}{\pgfqpoint{0.763923in}{0.447500in}}%
\pgfpathclose%
\pgfpathmoveto{\pgfqpoint{0.916667in}{0.441667in}}%
\pgfpathcurveto{\pgfqpoint{0.932137in}{0.441667in}}{\pgfqpoint{0.946975in}{0.447813in}}{\pgfqpoint{0.957915in}{0.458752in}}%
\pgfpathcurveto{\pgfqpoint{0.968854in}{0.469691in}}{\pgfqpoint{0.975000in}{0.484530in}}{\pgfqpoint{0.975000in}{0.500000in}}%
\pgfpathcurveto{\pgfqpoint{0.975000in}{0.515470in}}{\pgfqpoint{0.968854in}{0.530309in}}{\pgfqpoint{0.957915in}{0.541248in}}%
\pgfpathcurveto{\pgfqpoint{0.946975in}{0.552187in}}{\pgfqpoint{0.932137in}{0.558333in}}{\pgfqpoint{0.916667in}{0.558333in}}%
\pgfpathcurveto{\pgfqpoint{0.901196in}{0.558333in}}{\pgfqpoint{0.886358in}{0.552187in}}{\pgfqpoint{0.875419in}{0.541248in}}%
\pgfpathcurveto{\pgfqpoint{0.864480in}{0.530309in}}{\pgfqpoint{0.858333in}{0.515470in}}{\pgfqpoint{0.858333in}{0.500000in}}%
\pgfpathcurveto{\pgfqpoint{0.858333in}{0.484530in}}{\pgfqpoint{0.864480in}{0.469691in}}{\pgfqpoint{0.875419in}{0.458752in}}%
\pgfpathcurveto{\pgfqpoint{0.886358in}{0.447813in}}{\pgfqpoint{0.901196in}{0.441667in}}{\pgfqpoint{0.916667in}{0.441667in}}%
\pgfpathclose%
\pgfpathmoveto{\pgfqpoint{0.916667in}{0.447500in}}%
\pgfpathcurveto{\pgfqpoint{0.916667in}{0.447500in}}{\pgfqpoint{0.902744in}{0.447500in}}{\pgfqpoint{0.889389in}{0.453032in}}%
\pgfpathcurveto{\pgfqpoint{0.879544in}{0.462877in}}{\pgfqpoint{0.869698in}{0.472722in}}{\pgfqpoint{0.864167in}{0.486077in}}%
\pgfpathcurveto{\pgfqpoint{0.864167in}{0.500000in}}{\pgfqpoint{0.864167in}{0.513923in}}{\pgfqpoint{0.869698in}{0.527278in}}%
\pgfpathcurveto{\pgfqpoint{0.879544in}{0.537123in}}{\pgfqpoint{0.889389in}{0.546968in}}{\pgfqpoint{0.902744in}{0.552500in}}%
\pgfpathcurveto{\pgfqpoint{0.916667in}{0.552500in}}{\pgfqpoint{0.930590in}{0.552500in}}{\pgfqpoint{0.943945in}{0.546968in}}%
\pgfpathcurveto{\pgfqpoint{0.953790in}{0.537123in}}{\pgfqpoint{0.963635in}{0.527278in}}{\pgfqpoint{0.969167in}{0.513923in}}%
\pgfpathcurveto{\pgfqpoint{0.969167in}{0.500000in}}{\pgfqpoint{0.969167in}{0.486077in}}{\pgfqpoint{0.963635in}{0.472722in}}%
\pgfpathcurveto{\pgfqpoint{0.953790in}{0.462877in}}{\pgfqpoint{0.943945in}{0.453032in}}{\pgfqpoint{0.930590in}{0.447500in}}%
\pgfpathclose%
\pgfpathmoveto{\pgfqpoint{0.000000in}{0.608333in}}%
\pgfpathcurveto{\pgfqpoint{0.015470in}{0.608333in}}{\pgfqpoint{0.030309in}{0.614480in}}{\pgfqpoint{0.041248in}{0.625419in}}%
\pgfpathcurveto{\pgfqpoint{0.052187in}{0.636358in}}{\pgfqpoint{0.058333in}{0.651196in}}{\pgfqpoint{0.058333in}{0.666667in}}%
\pgfpathcurveto{\pgfqpoint{0.058333in}{0.682137in}}{\pgfqpoint{0.052187in}{0.696975in}}{\pgfqpoint{0.041248in}{0.707915in}}%
\pgfpathcurveto{\pgfqpoint{0.030309in}{0.718854in}}{\pgfqpoint{0.015470in}{0.725000in}}{\pgfqpoint{0.000000in}{0.725000in}}%
\pgfpathcurveto{\pgfqpoint{-0.015470in}{0.725000in}}{\pgfqpoint{-0.030309in}{0.718854in}}{\pgfqpoint{-0.041248in}{0.707915in}}%
\pgfpathcurveto{\pgfqpoint{-0.052187in}{0.696975in}}{\pgfqpoint{-0.058333in}{0.682137in}}{\pgfqpoint{-0.058333in}{0.666667in}}%
\pgfpathcurveto{\pgfqpoint{-0.058333in}{0.651196in}}{\pgfqpoint{-0.052187in}{0.636358in}}{\pgfqpoint{-0.041248in}{0.625419in}}%
\pgfpathcurveto{\pgfqpoint{-0.030309in}{0.614480in}}{\pgfqpoint{-0.015470in}{0.608333in}}{\pgfqpoint{0.000000in}{0.608333in}}%
\pgfpathclose%
\pgfpathmoveto{\pgfqpoint{0.000000in}{0.614167in}}%
\pgfpathcurveto{\pgfqpoint{0.000000in}{0.614167in}}{\pgfqpoint{-0.013923in}{0.614167in}}{\pgfqpoint{-0.027278in}{0.619698in}}%
\pgfpathcurveto{\pgfqpoint{-0.037123in}{0.629544in}}{\pgfqpoint{-0.046968in}{0.639389in}}{\pgfqpoint{-0.052500in}{0.652744in}}%
\pgfpathcurveto{\pgfqpoint{-0.052500in}{0.666667in}}{\pgfqpoint{-0.052500in}{0.680590in}}{\pgfqpoint{-0.046968in}{0.693945in}}%
\pgfpathcurveto{\pgfqpoint{-0.037123in}{0.703790in}}{\pgfqpoint{-0.027278in}{0.713635in}}{\pgfqpoint{-0.013923in}{0.719167in}}%
\pgfpathcurveto{\pgfqpoint{0.000000in}{0.719167in}}{\pgfqpoint{0.013923in}{0.719167in}}{\pgfqpoint{0.027278in}{0.713635in}}%
\pgfpathcurveto{\pgfqpoint{0.037123in}{0.703790in}}{\pgfqpoint{0.046968in}{0.693945in}}{\pgfqpoint{0.052500in}{0.680590in}}%
\pgfpathcurveto{\pgfqpoint{0.052500in}{0.666667in}}{\pgfqpoint{0.052500in}{0.652744in}}{\pgfqpoint{0.046968in}{0.639389in}}%
\pgfpathcurveto{\pgfqpoint{0.037123in}{0.629544in}}{\pgfqpoint{0.027278in}{0.619698in}}{\pgfqpoint{0.013923in}{0.614167in}}%
\pgfpathclose%
\pgfpathmoveto{\pgfqpoint{0.166667in}{0.608333in}}%
\pgfpathcurveto{\pgfqpoint{0.182137in}{0.608333in}}{\pgfqpoint{0.196975in}{0.614480in}}{\pgfqpoint{0.207915in}{0.625419in}}%
\pgfpathcurveto{\pgfqpoint{0.218854in}{0.636358in}}{\pgfqpoint{0.225000in}{0.651196in}}{\pgfqpoint{0.225000in}{0.666667in}}%
\pgfpathcurveto{\pgfqpoint{0.225000in}{0.682137in}}{\pgfqpoint{0.218854in}{0.696975in}}{\pgfqpoint{0.207915in}{0.707915in}}%
\pgfpathcurveto{\pgfqpoint{0.196975in}{0.718854in}}{\pgfqpoint{0.182137in}{0.725000in}}{\pgfqpoint{0.166667in}{0.725000in}}%
\pgfpathcurveto{\pgfqpoint{0.151196in}{0.725000in}}{\pgfqpoint{0.136358in}{0.718854in}}{\pgfqpoint{0.125419in}{0.707915in}}%
\pgfpathcurveto{\pgfqpoint{0.114480in}{0.696975in}}{\pgfqpoint{0.108333in}{0.682137in}}{\pgfqpoint{0.108333in}{0.666667in}}%
\pgfpathcurveto{\pgfqpoint{0.108333in}{0.651196in}}{\pgfqpoint{0.114480in}{0.636358in}}{\pgfqpoint{0.125419in}{0.625419in}}%
\pgfpathcurveto{\pgfqpoint{0.136358in}{0.614480in}}{\pgfqpoint{0.151196in}{0.608333in}}{\pgfqpoint{0.166667in}{0.608333in}}%
\pgfpathclose%
\pgfpathmoveto{\pgfqpoint{0.166667in}{0.614167in}}%
\pgfpathcurveto{\pgfqpoint{0.166667in}{0.614167in}}{\pgfqpoint{0.152744in}{0.614167in}}{\pgfqpoint{0.139389in}{0.619698in}}%
\pgfpathcurveto{\pgfqpoint{0.129544in}{0.629544in}}{\pgfqpoint{0.119698in}{0.639389in}}{\pgfqpoint{0.114167in}{0.652744in}}%
\pgfpathcurveto{\pgfqpoint{0.114167in}{0.666667in}}{\pgfqpoint{0.114167in}{0.680590in}}{\pgfqpoint{0.119698in}{0.693945in}}%
\pgfpathcurveto{\pgfqpoint{0.129544in}{0.703790in}}{\pgfqpoint{0.139389in}{0.713635in}}{\pgfqpoint{0.152744in}{0.719167in}}%
\pgfpathcurveto{\pgfqpoint{0.166667in}{0.719167in}}{\pgfqpoint{0.180590in}{0.719167in}}{\pgfqpoint{0.193945in}{0.713635in}}%
\pgfpathcurveto{\pgfqpoint{0.203790in}{0.703790in}}{\pgfqpoint{0.213635in}{0.693945in}}{\pgfqpoint{0.219167in}{0.680590in}}%
\pgfpathcurveto{\pgfqpoint{0.219167in}{0.666667in}}{\pgfqpoint{0.219167in}{0.652744in}}{\pgfqpoint{0.213635in}{0.639389in}}%
\pgfpathcurveto{\pgfqpoint{0.203790in}{0.629544in}}{\pgfqpoint{0.193945in}{0.619698in}}{\pgfqpoint{0.180590in}{0.614167in}}%
\pgfpathclose%
\pgfpathmoveto{\pgfqpoint{0.333333in}{0.608333in}}%
\pgfpathcurveto{\pgfqpoint{0.348804in}{0.608333in}}{\pgfqpoint{0.363642in}{0.614480in}}{\pgfqpoint{0.374581in}{0.625419in}}%
\pgfpathcurveto{\pgfqpoint{0.385520in}{0.636358in}}{\pgfqpoint{0.391667in}{0.651196in}}{\pgfqpoint{0.391667in}{0.666667in}}%
\pgfpathcurveto{\pgfqpoint{0.391667in}{0.682137in}}{\pgfqpoint{0.385520in}{0.696975in}}{\pgfqpoint{0.374581in}{0.707915in}}%
\pgfpathcurveto{\pgfqpoint{0.363642in}{0.718854in}}{\pgfqpoint{0.348804in}{0.725000in}}{\pgfqpoint{0.333333in}{0.725000in}}%
\pgfpathcurveto{\pgfqpoint{0.317863in}{0.725000in}}{\pgfqpoint{0.303025in}{0.718854in}}{\pgfqpoint{0.292085in}{0.707915in}}%
\pgfpathcurveto{\pgfqpoint{0.281146in}{0.696975in}}{\pgfqpoint{0.275000in}{0.682137in}}{\pgfqpoint{0.275000in}{0.666667in}}%
\pgfpathcurveto{\pgfqpoint{0.275000in}{0.651196in}}{\pgfqpoint{0.281146in}{0.636358in}}{\pgfqpoint{0.292085in}{0.625419in}}%
\pgfpathcurveto{\pgfqpoint{0.303025in}{0.614480in}}{\pgfqpoint{0.317863in}{0.608333in}}{\pgfqpoint{0.333333in}{0.608333in}}%
\pgfpathclose%
\pgfpathmoveto{\pgfqpoint{0.333333in}{0.614167in}}%
\pgfpathcurveto{\pgfqpoint{0.333333in}{0.614167in}}{\pgfqpoint{0.319410in}{0.614167in}}{\pgfqpoint{0.306055in}{0.619698in}}%
\pgfpathcurveto{\pgfqpoint{0.296210in}{0.629544in}}{\pgfqpoint{0.286365in}{0.639389in}}{\pgfqpoint{0.280833in}{0.652744in}}%
\pgfpathcurveto{\pgfqpoint{0.280833in}{0.666667in}}{\pgfqpoint{0.280833in}{0.680590in}}{\pgfqpoint{0.286365in}{0.693945in}}%
\pgfpathcurveto{\pgfqpoint{0.296210in}{0.703790in}}{\pgfqpoint{0.306055in}{0.713635in}}{\pgfqpoint{0.319410in}{0.719167in}}%
\pgfpathcurveto{\pgfqpoint{0.333333in}{0.719167in}}{\pgfqpoint{0.347256in}{0.719167in}}{\pgfqpoint{0.360611in}{0.713635in}}%
\pgfpathcurveto{\pgfqpoint{0.370456in}{0.703790in}}{\pgfqpoint{0.380302in}{0.693945in}}{\pgfqpoint{0.385833in}{0.680590in}}%
\pgfpathcurveto{\pgfqpoint{0.385833in}{0.666667in}}{\pgfqpoint{0.385833in}{0.652744in}}{\pgfqpoint{0.380302in}{0.639389in}}%
\pgfpathcurveto{\pgfqpoint{0.370456in}{0.629544in}}{\pgfqpoint{0.360611in}{0.619698in}}{\pgfqpoint{0.347256in}{0.614167in}}%
\pgfpathclose%
\pgfpathmoveto{\pgfqpoint{0.500000in}{0.608333in}}%
\pgfpathcurveto{\pgfqpoint{0.515470in}{0.608333in}}{\pgfqpoint{0.530309in}{0.614480in}}{\pgfqpoint{0.541248in}{0.625419in}}%
\pgfpathcurveto{\pgfqpoint{0.552187in}{0.636358in}}{\pgfqpoint{0.558333in}{0.651196in}}{\pgfqpoint{0.558333in}{0.666667in}}%
\pgfpathcurveto{\pgfqpoint{0.558333in}{0.682137in}}{\pgfqpoint{0.552187in}{0.696975in}}{\pgfqpoint{0.541248in}{0.707915in}}%
\pgfpathcurveto{\pgfqpoint{0.530309in}{0.718854in}}{\pgfqpoint{0.515470in}{0.725000in}}{\pgfqpoint{0.500000in}{0.725000in}}%
\pgfpathcurveto{\pgfqpoint{0.484530in}{0.725000in}}{\pgfqpoint{0.469691in}{0.718854in}}{\pgfqpoint{0.458752in}{0.707915in}}%
\pgfpathcurveto{\pgfqpoint{0.447813in}{0.696975in}}{\pgfqpoint{0.441667in}{0.682137in}}{\pgfqpoint{0.441667in}{0.666667in}}%
\pgfpathcurveto{\pgfqpoint{0.441667in}{0.651196in}}{\pgfqpoint{0.447813in}{0.636358in}}{\pgfqpoint{0.458752in}{0.625419in}}%
\pgfpathcurveto{\pgfqpoint{0.469691in}{0.614480in}}{\pgfqpoint{0.484530in}{0.608333in}}{\pgfqpoint{0.500000in}{0.608333in}}%
\pgfpathclose%
\pgfpathmoveto{\pgfqpoint{0.500000in}{0.614167in}}%
\pgfpathcurveto{\pgfqpoint{0.500000in}{0.614167in}}{\pgfqpoint{0.486077in}{0.614167in}}{\pgfqpoint{0.472722in}{0.619698in}}%
\pgfpathcurveto{\pgfqpoint{0.462877in}{0.629544in}}{\pgfqpoint{0.453032in}{0.639389in}}{\pgfqpoint{0.447500in}{0.652744in}}%
\pgfpathcurveto{\pgfqpoint{0.447500in}{0.666667in}}{\pgfqpoint{0.447500in}{0.680590in}}{\pgfqpoint{0.453032in}{0.693945in}}%
\pgfpathcurveto{\pgfqpoint{0.462877in}{0.703790in}}{\pgfqpoint{0.472722in}{0.713635in}}{\pgfqpoint{0.486077in}{0.719167in}}%
\pgfpathcurveto{\pgfqpoint{0.500000in}{0.719167in}}{\pgfqpoint{0.513923in}{0.719167in}}{\pgfqpoint{0.527278in}{0.713635in}}%
\pgfpathcurveto{\pgfqpoint{0.537123in}{0.703790in}}{\pgfqpoint{0.546968in}{0.693945in}}{\pgfqpoint{0.552500in}{0.680590in}}%
\pgfpathcurveto{\pgfqpoint{0.552500in}{0.666667in}}{\pgfqpoint{0.552500in}{0.652744in}}{\pgfqpoint{0.546968in}{0.639389in}}%
\pgfpathcurveto{\pgfqpoint{0.537123in}{0.629544in}}{\pgfqpoint{0.527278in}{0.619698in}}{\pgfqpoint{0.513923in}{0.614167in}}%
\pgfpathclose%
\pgfpathmoveto{\pgfqpoint{0.666667in}{0.608333in}}%
\pgfpathcurveto{\pgfqpoint{0.682137in}{0.608333in}}{\pgfqpoint{0.696975in}{0.614480in}}{\pgfqpoint{0.707915in}{0.625419in}}%
\pgfpathcurveto{\pgfqpoint{0.718854in}{0.636358in}}{\pgfqpoint{0.725000in}{0.651196in}}{\pgfqpoint{0.725000in}{0.666667in}}%
\pgfpathcurveto{\pgfqpoint{0.725000in}{0.682137in}}{\pgfqpoint{0.718854in}{0.696975in}}{\pgfqpoint{0.707915in}{0.707915in}}%
\pgfpathcurveto{\pgfqpoint{0.696975in}{0.718854in}}{\pgfqpoint{0.682137in}{0.725000in}}{\pgfqpoint{0.666667in}{0.725000in}}%
\pgfpathcurveto{\pgfqpoint{0.651196in}{0.725000in}}{\pgfqpoint{0.636358in}{0.718854in}}{\pgfqpoint{0.625419in}{0.707915in}}%
\pgfpathcurveto{\pgfqpoint{0.614480in}{0.696975in}}{\pgfqpoint{0.608333in}{0.682137in}}{\pgfqpoint{0.608333in}{0.666667in}}%
\pgfpathcurveto{\pgfqpoint{0.608333in}{0.651196in}}{\pgfqpoint{0.614480in}{0.636358in}}{\pgfqpoint{0.625419in}{0.625419in}}%
\pgfpathcurveto{\pgfqpoint{0.636358in}{0.614480in}}{\pgfqpoint{0.651196in}{0.608333in}}{\pgfqpoint{0.666667in}{0.608333in}}%
\pgfpathclose%
\pgfpathmoveto{\pgfqpoint{0.666667in}{0.614167in}}%
\pgfpathcurveto{\pgfqpoint{0.666667in}{0.614167in}}{\pgfqpoint{0.652744in}{0.614167in}}{\pgfqpoint{0.639389in}{0.619698in}}%
\pgfpathcurveto{\pgfqpoint{0.629544in}{0.629544in}}{\pgfqpoint{0.619698in}{0.639389in}}{\pgfqpoint{0.614167in}{0.652744in}}%
\pgfpathcurveto{\pgfqpoint{0.614167in}{0.666667in}}{\pgfqpoint{0.614167in}{0.680590in}}{\pgfqpoint{0.619698in}{0.693945in}}%
\pgfpathcurveto{\pgfqpoint{0.629544in}{0.703790in}}{\pgfqpoint{0.639389in}{0.713635in}}{\pgfqpoint{0.652744in}{0.719167in}}%
\pgfpathcurveto{\pgfqpoint{0.666667in}{0.719167in}}{\pgfqpoint{0.680590in}{0.719167in}}{\pgfqpoint{0.693945in}{0.713635in}}%
\pgfpathcurveto{\pgfqpoint{0.703790in}{0.703790in}}{\pgfqpoint{0.713635in}{0.693945in}}{\pgfqpoint{0.719167in}{0.680590in}}%
\pgfpathcurveto{\pgfqpoint{0.719167in}{0.666667in}}{\pgfqpoint{0.719167in}{0.652744in}}{\pgfqpoint{0.713635in}{0.639389in}}%
\pgfpathcurveto{\pgfqpoint{0.703790in}{0.629544in}}{\pgfqpoint{0.693945in}{0.619698in}}{\pgfqpoint{0.680590in}{0.614167in}}%
\pgfpathclose%
\pgfpathmoveto{\pgfqpoint{0.833333in}{0.608333in}}%
\pgfpathcurveto{\pgfqpoint{0.848804in}{0.608333in}}{\pgfqpoint{0.863642in}{0.614480in}}{\pgfqpoint{0.874581in}{0.625419in}}%
\pgfpathcurveto{\pgfqpoint{0.885520in}{0.636358in}}{\pgfqpoint{0.891667in}{0.651196in}}{\pgfqpoint{0.891667in}{0.666667in}}%
\pgfpathcurveto{\pgfqpoint{0.891667in}{0.682137in}}{\pgfqpoint{0.885520in}{0.696975in}}{\pgfqpoint{0.874581in}{0.707915in}}%
\pgfpathcurveto{\pgfqpoint{0.863642in}{0.718854in}}{\pgfqpoint{0.848804in}{0.725000in}}{\pgfqpoint{0.833333in}{0.725000in}}%
\pgfpathcurveto{\pgfqpoint{0.817863in}{0.725000in}}{\pgfqpoint{0.803025in}{0.718854in}}{\pgfqpoint{0.792085in}{0.707915in}}%
\pgfpathcurveto{\pgfqpoint{0.781146in}{0.696975in}}{\pgfqpoint{0.775000in}{0.682137in}}{\pgfqpoint{0.775000in}{0.666667in}}%
\pgfpathcurveto{\pgfqpoint{0.775000in}{0.651196in}}{\pgfqpoint{0.781146in}{0.636358in}}{\pgfqpoint{0.792085in}{0.625419in}}%
\pgfpathcurveto{\pgfqpoint{0.803025in}{0.614480in}}{\pgfqpoint{0.817863in}{0.608333in}}{\pgfqpoint{0.833333in}{0.608333in}}%
\pgfpathclose%
\pgfpathmoveto{\pgfqpoint{0.833333in}{0.614167in}}%
\pgfpathcurveto{\pgfqpoint{0.833333in}{0.614167in}}{\pgfqpoint{0.819410in}{0.614167in}}{\pgfqpoint{0.806055in}{0.619698in}}%
\pgfpathcurveto{\pgfqpoint{0.796210in}{0.629544in}}{\pgfqpoint{0.786365in}{0.639389in}}{\pgfqpoint{0.780833in}{0.652744in}}%
\pgfpathcurveto{\pgfqpoint{0.780833in}{0.666667in}}{\pgfqpoint{0.780833in}{0.680590in}}{\pgfqpoint{0.786365in}{0.693945in}}%
\pgfpathcurveto{\pgfqpoint{0.796210in}{0.703790in}}{\pgfqpoint{0.806055in}{0.713635in}}{\pgfqpoint{0.819410in}{0.719167in}}%
\pgfpathcurveto{\pgfqpoint{0.833333in}{0.719167in}}{\pgfqpoint{0.847256in}{0.719167in}}{\pgfqpoint{0.860611in}{0.713635in}}%
\pgfpathcurveto{\pgfqpoint{0.870456in}{0.703790in}}{\pgfqpoint{0.880302in}{0.693945in}}{\pgfqpoint{0.885833in}{0.680590in}}%
\pgfpathcurveto{\pgfqpoint{0.885833in}{0.666667in}}{\pgfqpoint{0.885833in}{0.652744in}}{\pgfqpoint{0.880302in}{0.639389in}}%
\pgfpathcurveto{\pgfqpoint{0.870456in}{0.629544in}}{\pgfqpoint{0.860611in}{0.619698in}}{\pgfqpoint{0.847256in}{0.614167in}}%
\pgfpathclose%
\pgfpathmoveto{\pgfqpoint{1.000000in}{0.608333in}}%
\pgfpathcurveto{\pgfqpoint{1.015470in}{0.608333in}}{\pgfqpoint{1.030309in}{0.614480in}}{\pgfqpoint{1.041248in}{0.625419in}}%
\pgfpathcurveto{\pgfqpoint{1.052187in}{0.636358in}}{\pgfqpoint{1.058333in}{0.651196in}}{\pgfqpoint{1.058333in}{0.666667in}}%
\pgfpathcurveto{\pgfqpoint{1.058333in}{0.682137in}}{\pgfqpoint{1.052187in}{0.696975in}}{\pgfqpoint{1.041248in}{0.707915in}}%
\pgfpathcurveto{\pgfqpoint{1.030309in}{0.718854in}}{\pgfqpoint{1.015470in}{0.725000in}}{\pgfqpoint{1.000000in}{0.725000in}}%
\pgfpathcurveto{\pgfqpoint{0.984530in}{0.725000in}}{\pgfqpoint{0.969691in}{0.718854in}}{\pgfqpoint{0.958752in}{0.707915in}}%
\pgfpathcurveto{\pgfqpoint{0.947813in}{0.696975in}}{\pgfqpoint{0.941667in}{0.682137in}}{\pgfqpoint{0.941667in}{0.666667in}}%
\pgfpathcurveto{\pgfqpoint{0.941667in}{0.651196in}}{\pgfqpoint{0.947813in}{0.636358in}}{\pgfqpoint{0.958752in}{0.625419in}}%
\pgfpathcurveto{\pgfqpoint{0.969691in}{0.614480in}}{\pgfqpoint{0.984530in}{0.608333in}}{\pgfqpoint{1.000000in}{0.608333in}}%
\pgfpathclose%
\pgfpathmoveto{\pgfqpoint{1.000000in}{0.614167in}}%
\pgfpathcurveto{\pgfqpoint{1.000000in}{0.614167in}}{\pgfqpoint{0.986077in}{0.614167in}}{\pgfqpoint{0.972722in}{0.619698in}}%
\pgfpathcurveto{\pgfqpoint{0.962877in}{0.629544in}}{\pgfqpoint{0.953032in}{0.639389in}}{\pgfqpoint{0.947500in}{0.652744in}}%
\pgfpathcurveto{\pgfqpoint{0.947500in}{0.666667in}}{\pgfqpoint{0.947500in}{0.680590in}}{\pgfqpoint{0.953032in}{0.693945in}}%
\pgfpathcurveto{\pgfqpoint{0.962877in}{0.703790in}}{\pgfqpoint{0.972722in}{0.713635in}}{\pgfqpoint{0.986077in}{0.719167in}}%
\pgfpathcurveto{\pgfqpoint{1.000000in}{0.719167in}}{\pgfqpoint{1.013923in}{0.719167in}}{\pgfqpoint{1.027278in}{0.713635in}}%
\pgfpathcurveto{\pgfqpoint{1.037123in}{0.703790in}}{\pgfqpoint{1.046968in}{0.693945in}}{\pgfqpoint{1.052500in}{0.680590in}}%
\pgfpathcurveto{\pgfqpoint{1.052500in}{0.666667in}}{\pgfqpoint{1.052500in}{0.652744in}}{\pgfqpoint{1.046968in}{0.639389in}}%
\pgfpathcurveto{\pgfqpoint{1.037123in}{0.629544in}}{\pgfqpoint{1.027278in}{0.619698in}}{\pgfqpoint{1.013923in}{0.614167in}}%
\pgfpathclose%
\pgfpathmoveto{\pgfqpoint{0.083333in}{0.775000in}}%
\pgfpathcurveto{\pgfqpoint{0.098804in}{0.775000in}}{\pgfqpoint{0.113642in}{0.781146in}}{\pgfqpoint{0.124581in}{0.792085in}}%
\pgfpathcurveto{\pgfqpoint{0.135520in}{0.803025in}}{\pgfqpoint{0.141667in}{0.817863in}}{\pgfqpoint{0.141667in}{0.833333in}}%
\pgfpathcurveto{\pgfqpoint{0.141667in}{0.848804in}}{\pgfqpoint{0.135520in}{0.863642in}}{\pgfqpoint{0.124581in}{0.874581in}}%
\pgfpathcurveto{\pgfqpoint{0.113642in}{0.885520in}}{\pgfqpoint{0.098804in}{0.891667in}}{\pgfqpoint{0.083333in}{0.891667in}}%
\pgfpathcurveto{\pgfqpoint{0.067863in}{0.891667in}}{\pgfqpoint{0.053025in}{0.885520in}}{\pgfqpoint{0.042085in}{0.874581in}}%
\pgfpathcurveto{\pgfqpoint{0.031146in}{0.863642in}}{\pgfqpoint{0.025000in}{0.848804in}}{\pgfqpoint{0.025000in}{0.833333in}}%
\pgfpathcurveto{\pgfqpoint{0.025000in}{0.817863in}}{\pgfqpoint{0.031146in}{0.803025in}}{\pgfqpoint{0.042085in}{0.792085in}}%
\pgfpathcurveto{\pgfqpoint{0.053025in}{0.781146in}}{\pgfqpoint{0.067863in}{0.775000in}}{\pgfqpoint{0.083333in}{0.775000in}}%
\pgfpathclose%
\pgfpathmoveto{\pgfqpoint{0.083333in}{0.780833in}}%
\pgfpathcurveto{\pgfqpoint{0.083333in}{0.780833in}}{\pgfqpoint{0.069410in}{0.780833in}}{\pgfqpoint{0.056055in}{0.786365in}}%
\pgfpathcurveto{\pgfqpoint{0.046210in}{0.796210in}}{\pgfqpoint{0.036365in}{0.806055in}}{\pgfqpoint{0.030833in}{0.819410in}}%
\pgfpathcurveto{\pgfqpoint{0.030833in}{0.833333in}}{\pgfqpoint{0.030833in}{0.847256in}}{\pgfqpoint{0.036365in}{0.860611in}}%
\pgfpathcurveto{\pgfqpoint{0.046210in}{0.870456in}}{\pgfqpoint{0.056055in}{0.880302in}}{\pgfqpoint{0.069410in}{0.885833in}}%
\pgfpathcurveto{\pgfqpoint{0.083333in}{0.885833in}}{\pgfqpoint{0.097256in}{0.885833in}}{\pgfqpoint{0.110611in}{0.880302in}}%
\pgfpathcurveto{\pgfqpoint{0.120456in}{0.870456in}}{\pgfqpoint{0.130302in}{0.860611in}}{\pgfqpoint{0.135833in}{0.847256in}}%
\pgfpathcurveto{\pgfqpoint{0.135833in}{0.833333in}}{\pgfqpoint{0.135833in}{0.819410in}}{\pgfqpoint{0.130302in}{0.806055in}}%
\pgfpathcurveto{\pgfqpoint{0.120456in}{0.796210in}}{\pgfqpoint{0.110611in}{0.786365in}}{\pgfqpoint{0.097256in}{0.780833in}}%
\pgfpathclose%
\pgfpathmoveto{\pgfqpoint{0.250000in}{0.775000in}}%
\pgfpathcurveto{\pgfqpoint{0.265470in}{0.775000in}}{\pgfqpoint{0.280309in}{0.781146in}}{\pgfqpoint{0.291248in}{0.792085in}}%
\pgfpathcurveto{\pgfqpoint{0.302187in}{0.803025in}}{\pgfqpoint{0.308333in}{0.817863in}}{\pgfqpoint{0.308333in}{0.833333in}}%
\pgfpathcurveto{\pgfqpoint{0.308333in}{0.848804in}}{\pgfqpoint{0.302187in}{0.863642in}}{\pgfqpoint{0.291248in}{0.874581in}}%
\pgfpathcurveto{\pgfqpoint{0.280309in}{0.885520in}}{\pgfqpoint{0.265470in}{0.891667in}}{\pgfqpoint{0.250000in}{0.891667in}}%
\pgfpathcurveto{\pgfqpoint{0.234530in}{0.891667in}}{\pgfqpoint{0.219691in}{0.885520in}}{\pgfqpoint{0.208752in}{0.874581in}}%
\pgfpathcurveto{\pgfqpoint{0.197813in}{0.863642in}}{\pgfqpoint{0.191667in}{0.848804in}}{\pgfqpoint{0.191667in}{0.833333in}}%
\pgfpathcurveto{\pgfqpoint{0.191667in}{0.817863in}}{\pgfqpoint{0.197813in}{0.803025in}}{\pgfqpoint{0.208752in}{0.792085in}}%
\pgfpathcurveto{\pgfqpoint{0.219691in}{0.781146in}}{\pgfqpoint{0.234530in}{0.775000in}}{\pgfqpoint{0.250000in}{0.775000in}}%
\pgfpathclose%
\pgfpathmoveto{\pgfqpoint{0.250000in}{0.780833in}}%
\pgfpathcurveto{\pgfqpoint{0.250000in}{0.780833in}}{\pgfqpoint{0.236077in}{0.780833in}}{\pgfqpoint{0.222722in}{0.786365in}}%
\pgfpathcurveto{\pgfqpoint{0.212877in}{0.796210in}}{\pgfqpoint{0.203032in}{0.806055in}}{\pgfqpoint{0.197500in}{0.819410in}}%
\pgfpathcurveto{\pgfqpoint{0.197500in}{0.833333in}}{\pgfqpoint{0.197500in}{0.847256in}}{\pgfqpoint{0.203032in}{0.860611in}}%
\pgfpathcurveto{\pgfqpoint{0.212877in}{0.870456in}}{\pgfqpoint{0.222722in}{0.880302in}}{\pgfqpoint{0.236077in}{0.885833in}}%
\pgfpathcurveto{\pgfqpoint{0.250000in}{0.885833in}}{\pgfqpoint{0.263923in}{0.885833in}}{\pgfqpoint{0.277278in}{0.880302in}}%
\pgfpathcurveto{\pgfqpoint{0.287123in}{0.870456in}}{\pgfqpoint{0.296968in}{0.860611in}}{\pgfqpoint{0.302500in}{0.847256in}}%
\pgfpathcurveto{\pgfqpoint{0.302500in}{0.833333in}}{\pgfqpoint{0.302500in}{0.819410in}}{\pgfqpoint{0.296968in}{0.806055in}}%
\pgfpathcurveto{\pgfqpoint{0.287123in}{0.796210in}}{\pgfqpoint{0.277278in}{0.786365in}}{\pgfqpoint{0.263923in}{0.780833in}}%
\pgfpathclose%
\pgfpathmoveto{\pgfqpoint{0.416667in}{0.775000in}}%
\pgfpathcurveto{\pgfqpoint{0.432137in}{0.775000in}}{\pgfqpoint{0.446975in}{0.781146in}}{\pgfqpoint{0.457915in}{0.792085in}}%
\pgfpathcurveto{\pgfqpoint{0.468854in}{0.803025in}}{\pgfqpoint{0.475000in}{0.817863in}}{\pgfqpoint{0.475000in}{0.833333in}}%
\pgfpathcurveto{\pgfqpoint{0.475000in}{0.848804in}}{\pgfqpoint{0.468854in}{0.863642in}}{\pgfqpoint{0.457915in}{0.874581in}}%
\pgfpathcurveto{\pgfqpoint{0.446975in}{0.885520in}}{\pgfqpoint{0.432137in}{0.891667in}}{\pgfqpoint{0.416667in}{0.891667in}}%
\pgfpathcurveto{\pgfqpoint{0.401196in}{0.891667in}}{\pgfqpoint{0.386358in}{0.885520in}}{\pgfqpoint{0.375419in}{0.874581in}}%
\pgfpathcurveto{\pgfqpoint{0.364480in}{0.863642in}}{\pgfqpoint{0.358333in}{0.848804in}}{\pgfqpoint{0.358333in}{0.833333in}}%
\pgfpathcurveto{\pgfqpoint{0.358333in}{0.817863in}}{\pgfqpoint{0.364480in}{0.803025in}}{\pgfqpoint{0.375419in}{0.792085in}}%
\pgfpathcurveto{\pgfqpoint{0.386358in}{0.781146in}}{\pgfqpoint{0.401196in}{0.775000in}}{\pgfqpoint{0.416667in}{0.775000in}}%
\pgfpathclose%
\pgfpathmoveto{\pgfqpoint{0.416667in}{0.780833in}}%
\pgfpathcurveto{\pgfqpoint{0.416667in}{0.780833in}}{\pgfqpoint{0.402744in}{0.780833in}}{\pgfqpoint{0.389389in}{0.786365in}}%
\pgfpathcurveto{\pgfqpoint{0.379544in}{0.796210in}}{\pgfqpoint{0.369698in}{0.806055in}}{\pgfqpoint{0.364167in}{0.819410in}}%
\pgfpathcurveto{\pgfqpoint{0.364167in}{0.833333in}}{\pgfqpoint{0.364167in}{0.847256in}}{\pgfqpoint{0.369698in}{0.860611in}}%
\pgfpathcurveto{\pgfqpoint{0.379544in}{0.870456in}}{\pgfqpoint{0.389389in}{0.880302in}}{\pgfqpoint{0.402744in}{0.885833in}}%
\pgfpathcurveto{\pgfqpoint{0.416667in}{0.885833in}}{\pgfqpoint{0.430590in}{0.885833in}}{\pgfqpoint{0.443945in}{0.880302in}}%
\pgfpathcurveto{\pgfqpoint{0.453790in}{0.870456in}}{\pgfqpoint{0.463635in}{0.860611in}}{\pgfqpoint{0.469167in}{0.847256in}}%
\pgfpathcurveto{\pgfqpoint{0.469167in}{0.833333in}}{\pgfqpoint{0.469167in}{0.819410in}}{\pgfqpoint{0.463635in}{0.806055in}}%
\pgfpathcurveto{\pgfqpoint{0.453790in}{0.796210in}}{\pgfqpoint{0.443945in}{0.786365in}}{\pgfqpoint{0.430590in}{0.780833in}}%
\pgfpathclose%
\pgfpathmoveto{\pgfqpoint{0.583333in}{0.775000in}}%
\pgfpathcurveto{\pgfqpoint{0.598804in}{0.775000in}}{\pgfqpoint{0.613642in}{0.781146in}}{\pgfqpoint{0.624581in}{0.792085in}}%
\pgfpathcurveto{\pgfqpoint{0.635520in}{0.803025in}}{\pgfqpoint{0.641667in}{0.817863in}}{\pgfqpoint{0.641667in}{0.833333in}}%
\pgfpathcurveto{\pgfqpoint{0.641667in}{0.848804in}}{\pgfqpoint{0.635520in}{0.863642in}}{\pgfqpoint{0.624581in}{0.874581in}}%
\pgfpathcurveto{\pgfqpoint{0.613642in}{0.885520in}}{\pgfqpoint{0.598804in}{0.891667in}}{\pgfqpoint{0.583333in}{0.891667in}}%
\pgfpathcurveto{\pgfqpoint{0.567863in}{0.891667in}}{\pgfqpoint{0.553025in}{0.885520in}}{\pgfqpoint{0.542085in}{0.874581in}}%
\pgfpathcurveto{\pgfqpoint{0.531146in}{0.863642in}}{\pgfqpoint{0.525000in}{0.848804in}}{\pgfqpoint{0.525000in}{0.833333in}}%
\pgfpathcurveto{\pgfqpoint{0.525000in}{0.817863in}}{\pgfqpoint{0.531146in}{0.803025in}}{\pgfqpoint{0.542085in}{0.792085in}}%
\pgfpathcurveto{\pgfqpoint{0.553025in}{0.781146in}}{\pgfqpoint{0.567863in}{0.775000in}}{\pgfqpoint{0.583333in}{0.775000in}}%
\pgfpathclose%
\pgfpathmoveto{\pgfqpoint{0.583333in}{0.780833in}}%
\pgfpathcurveto{\pgfqpoint{0.583333in}{0.780833in}}{\pgfqpoint{0.569410in}{0.780833in}}{\pgfqpoint{0.556055in}{0.786365in}}%
\pgfpathcurveto{\pgfqpoint{0.546210in}{0.796210in}}{\pgfqpoint{0.536365in}{0.806055in}}{\pgfqpoint{0.530833in}{0.819410in}}%
\pgfpathcurveto{\pgfqpoint{0.530833in}{0.833333in}}{\pgfqpoint{0.530833in}{0.847256in}}{\pgfqpoint{0.536365in}{0.860611in}}%
\pgfpathcurveto{\pgfqpoint{0.546210in}{0.870456in}}{\pgfqpoint{0.556055in}{0.880302in}}{\pgfqpoint{0.569410in}{0.885833in}}%
\pgfpathcurveto{\pgfqpoint{0.583333in}{0.885833in}}{\pgfqpoint{0.597256in}{0.885833in}}{\pgfqpoint{0.610611in}{0.880302in}}%
\pgfpathcurveto{\pgfqpoint{0.620456in}{0.870456in}}{\pgfqpoint{0.630302in}{0.860611in}}{\pgfqpoint{0.635833in}{0.847256in}}%
\pgfpathcurveto{\pgfqpoint{0.635833in}{0.833333in}}{\pgfqpoint{0.635833in}{0.819410in}}{\pgfqpoint{0.630302in}{0.806055in}}%
\pgfpathcurveto{\pgfqpoint{0.620456in}{0.796210in}}{\pgfqpoint{0.610611in}{0.786365in}}{\pgfqpoint{0.597256in}{0.780833in}}%
\pgfpathclose%
\pgfpathmoveto{\pgfqpoint{0.750000in}{0.775000in}}%
\pgfpathcurveto{\pgfqpoint{0.765470in}{0.775000in}}{\pgfqpoint{0.780309in}{0.781146in}}{\pgfqpoint{0.791248in}{0.792085in}}%
\pgfpathcurveto{\pgfqpoint{0.802187in}{0.803025in}}{\pgfqpoint{0.808333in}{0.817863in}}{\pgfqpoint{0.808333in}{0.833333in}}%
\pgfpathcurveto{\pgfqpoint{0.808333in}{0.848804in}}{\pgfqpoint{0.802187in}{0.863642in}}{\pgfqpoint{0.791248in}{0.874581in}}%
\pgfpathcurveto{\pgfqpoint{0.780309in}{0.885520in}}{\pgfqpoint{0.765470in}{0.891667in}}{\pgfqpoint{0.750000in}{0.891667in}}%
\pgfpathcurveto{\pgfqpoint{0.734530in}{0.891667in}}{\pgfqpoint{0.719691in}{0.885520in}}{\pgfqpoint{0.708752in}{0.874581in}}%
\pgfpathcurveto{\pgfqpoint{0.697813in}{0.863642in}}{\pgfqpoint{0.691667in}{0.848804in}}{\pgfqpoint{0.691667in}{0.833333in}}%
\pgfpathcurveto{\pgfqpoint{0.691667in}{0.817863in}}{\pgfqpoint{0.697813in}{0.803025in}}{\pgfqpoint{0.708752in}{0.792085in}}%
\pgfpathcurveto{\pgfqpoint{0.719691in}{0.781146in}}{\pgfqpoint{0.734530in}{0.775000in}}{\pgfqpoint{0.750000in}{0.775000in}}%
\pgfpathclose%
\pgfpathmoveto{\pgfqpoint{0.750000in}{0.780833in}}%
\pgfpathcurveto{\pgfqpoint{0.750000in}{0.780833in}}{\pgfqpoint{0.736077in}{0.780833in}}{\pgfqpoint{0.722722in}{0.786365in}}%
\pgfpathcurveto{\pgfqpoint{0.712877in}{0.796210in}}{\pgfqpoint{0.703032in}{0.806055in}}{\pgfqpoint{0.697500in}{0.819410in}}%
\pgfpathcurveto{\pgfqpoint{0.697500in}{0.833333in}}{\pgfqpoint{0.697500in}{0.847256in}}{\pgfqpoint{0.703032in}{0.860611in}}%
\pgfpathcurveto{\pgfqpoint{0.712877in}{0.870456in}}{\pgfqpoint{0.722722in}{0.880302in}}{\pgfqpoint{0.736077in}{0.885833in}}%
\pgfpathcurveto{\pgfqpoint{0.750000in}{0.885833in}}{\pgfqpoint{0.763923in}{0.885833in}}{\pgfqpoint{0.777278in}{0.880302in}}%
\pgfpathcurveto{\pgfqpoint{0.787123in}{0.870456in}}{\pgfqpoint{0.796968in}{0.860611in}}{\pgfqpoint{0.802500in}{0.847256in}}%
\pgfpathcurveto{\pgfqpoint{0.802500in}{0.833333in}}{\pgfqpoint{0.802500in}{0.819410in}}{\pgfqpoint{0.796968in}{0.806055in}}%
\pgfpathcurveto{\pgfqpoint{0.787123in}{0.796210in}}{\pgfqpoint{0.777278in}{0.786365in}}{\pgfqpoint{0.763923in}{0.780833in}}%
\pgfpathclose%
\pgfpathmoveto{\pgfqpoint{0.916667in}{0.775000in}}%
\pgfpathcurveto{\pgfqpoint{0.932137in}{0.775000in}}{\pgfqpoint{0.946975in}{0.781146in}}{\pgfqpoint{0.957915in}{0.792085in}}%
\pgfpathcurveto{\pgfqpoint{0.968854in}{0.803025in}}{\pgfqpoint{0.975000in}{0.817863in}}{\pgfqpoint{0.975000in}{0.833333in}}%
\pgfpathcurveto{\pgfqpoint{0.975000in}{0.848804in}}{\pgfqpoint{0.968854in}{0.863642in}}{\pgfqpoint{0.957915in}{0.874581in}}%
\pgfpathcurveto{\pgfqpoint{0.946975in}{0.885520in}}{\pgfqpoint{0.932137in}{0.891667in}}{\pgfqpoint{0.916667in}{0.891667in}}%
\pgfpathcurveto{\pgfqpoint{0.901196in}{0.891667in}}{\pgfqpoint{0.886358in}{0.885520in}}{\pgfqpoint{0.875419in}{0.874581in}}%
\pgfpathcurveto{\pgfqpoint{0.864480in}{0.863642in}}{\pgfqpoint{0.858333in}{0.848804in}}{\pgfqpoint{0.858333in}{0.833333in}}%
\pgfpathcurveto{\pgfqpoint{0.858333in}{0.817863in}}{\pgfqpoint{0.864480in}{0.803025in}}{\pgfqpoint{0.875419in}{0.792085in}}%
\pgfpathcurveto{\pgfqpoint{0.886358in}{0.781146in}}{\pgfqpoint{0.901196in}{0.775000in}}{\pgfqpoint{0.916667in}{0.775000in}}%
\pgfpathclose%
\pgfpathmoveto{\pgfqpoint{0.916667in}{0.780833in}}%
\pgfpathcurveto{\pgfqpoint{0.916667in}{0.780833in}}{\pgfqpoint{0.902744in}{0.780833in}}{\pgfqpoint{0.889389in}{0.786365in}}%
\pgfpathcurveto{\pgfqpoint{0.879544in}{0.796210in}}{\pgfqpoint{0.869698in}{0.806055in}}{\pgfqpoint{0.864167in}{0.819410in}}%
\pgfpathcurveto{\pgfqpoint{0.864167in}{0.833333in}}{\pgfqpoint{0.864167in}{0.847256in}}{\pgfqpoint{0.869698in}{0.860611in}}%
\pgfpathcurveto{\pgfqpoint{0.879544in}{0.870456in}}{\pgfqpoint{0.889389in}{0.880302in}}{\pgfqpoint{0.902744in}{0.885833in}}%
\pgfpathcurveto{\pgfqpoint{0.916667in}{0.885833in}}{\pgfqpoint{0.930590in}{0.885833in}}{\pgfqpoint{0.943945in}{0.880302in}}%
\pgfpathcurveto{\pgfqpoint{0.953790in}{0.870456in}}{\pgfqpoint{0.963635in}{0.860611in}}{\pgfqpoint{0.969167in}{0.847256in}}%
\pgfpathcurveto{\pgfqpoint{0.969167in}{0.833333in}}{\pgfqpoint{0.969167in}{0.819410in}}{\pgfqpoint{0.963635in}{0.806055in}}%
\pgfpathcurveto{\pgfqpoint{0.953790in}{0.796210in}}{\pgfqpoint{0.943945in}{0.786365in}}{\pgfqpoint{0.930590in}{0.780833in}}%
\pgfpathclose%
\pgfpathmoveto{\pgfqpoint{0.000000in}{0.941667in}}%
\pgfpathcurveto{\pgfqpoint{0.015470in}{0.941667in}}{\pgfqpoint{0.030309in}{0.947813in}}{\pgfqpoint{0.041248in}{0.958752in}}%
\pgfpathcurveto{\pgfqpoint{0.052187in}{0.969691in}}{\pgfqpoint{0.058333in}{0.984530in}}{\pgfqpoint{0.058333in}{1.000000in}}%
\pgfpathcurveto{\pgfqpoint{0.058333in}{1.015470in}}{\pgfqpoint{0.052187in}{1.030309in}}{\pgfqpoint{0.041248in}{1.041248in}}%
\pgfpathcurveto{\pgfqpoint{0.030309in}{1.052187in}}{\pgfqpoint{0.015470in}{1.058333in}}{\pgfqpoint{0.000000in}{1.058333in}}%
\pgfpathcurveto{\pgfqpoint{-0.015470in}{1.058333in}}{\pgfqpoint{-0.030309in}{1.052187in}}{\pgfqpoint{-0.041248in}{1.041248in}}%
\pgfpathcurveto{\pgfqpoint{-0.052187in}{1.030309in}}{\pgfqpoint{-0.058333in}{1.015470in}}{\pgfqpoint{-0.058333in}{1.000000in}}%
\pgfpathcurveto{\pgfqpoint{-0.058333in}{0.984530in}}{\pgfqpoint{-0.052187in}{0.969691in}}{\pgfqpoint{-0.041248in}{0.958752in}}%
\pgfpathcurveto{\pgfqpoint{-0.030309in}{0.947813in}}{\pgfqpoint{-0.015470in}{0.941667in}}{\pgfqpoint{0.000000in}{0.941667in}}%
\pgfpathclose%
\pgfpathmoveto{\pgfqpoint{0.000000in}{0.947500in}}%
\pgfpathcurveto{\pgfqpoint{0.000000in}{0.947500in}}{\pgfqpoint{-0.013923in}{0.947500in}}{\pgfqpoint{-0.027278in}{0.953032in}}%
\pgfpathcurveto{\pgfqpoint{-0.037123in}{0.962877in}}{\pgfqpoint{-0.046968in}{0.972722in}}{\pgfqpoint{-0.052500in}{0.986077in}}%
\pgfpathcurveto{\pgfqpoint{-0.052500in}{1.000000in}}{\pgfqpoint{-0.052500in}{1.013923in}}{\pgfqpoint{-0.046968in}{1.027278in}}%
\pgfpathcurveto{\pgfqpoint{-0.037123in}{1.037123in}}{\pgfqpoint{-0.027278in}{1.046968in}}{\pgfqpoint{-0.013923in}{1.052500in}}%
\pgfpathcurveto{\pgfqpoint{0.000000in}{1.052500in}}{\pgfqpoint{0.013923in}{1.052500in}}{\pgfqpoint{0.027278in}{1.046968in}}%
\pgfpathcurveto{\pgfqpoint{0.037123in}{1.037123in}}{\pgfqpoint{0.046968in}{1.027278in}}{\pgfqpoint{0.052500in}{1.013923in}}%
\pgfpathcurveto{\pgfqpoint{0.052500in}{1.000000in}}{\pgfqpoint{0.052500in}{0.986077in}}{\pgfqpoint{0.046968in}{0.972722in}}%
\pgfpathcurveto{\pgfqpoint{0.037123in}{0.962877in}}{\pgfqpoint{0.027278in}{0.953032in}}{\pgfqpoint{0.013923in}{0.947500in}}%
\pgfpathclose%
\pgfpathmoveto{\pgfqpoint{0.166667in}{0.941667in}}%
\pgfpathcurveto{\pgfqpoint{0.182137in}{0.941667in}}{\pgfqpoint{0.196975in}{0.947813in}}{\pgfqpoint{0.207915in}{0.958752in}}%
\pgfpathcurveto{\pgfqpoint{0.218854in}{0.969691in}}{\pgfqpoint{0.225000in}{0.984530in}}{\pgfqpoint{0.225000in}{1.000000in}}%
\pgfpathcurveto{\pgfqpoint{0.225000in}{1.015470in}}{\pgfqpoint{0.218854in}{1.030309in}}{\pgfqpoint{0.207915in}{1.041248in}}%
\pgfpathcurveto{\pgfqpoint{0.196975in}{1.052187in}}{\pgfqpoint{0.182137in}{1.058333in}}{\pgfqpoint{0.166667in}{1.058333in}}%
\pgfpathcurveto{\pgfqpoint{0.151196in}{1.058333in}}{\pgfqpoint{0.136358in}{1.052187in}}{\pgfqpoint{0.125419in}{1.041248in}}%
\pgfpathcurveto{\pgfqpoint{0.114480in}{1.030309in}}{\pgfqpoint{0.108333in}{1.015470in}}{\pgfqpoint{0.108333in}{1.000000in}}%
\pgfpathcurveto{\pgfqpoint{0.108333in}{0.984530in}}{\pgfqpoint{0.114480in}{0.969691in}}{\pgfqpoint{0.125419in}{0.958752in}}%
\pgfpathcurveto{\pgfqpoint{0.136358in}{0.947813in}}{\pgfqpoint{0.151196in}{0.941667in}}{\pgfqpoint{0.166667in}{0.941667in}}%
\pgfpathclose%
\pgfpathmoveto{\pgfqpoint{0.166667in}{0.947500in}}%
\pgfpathcurveto{\pgfqpoint{0.166667in}{0.947500in}}{\pgfqpoint{0.152744in}{0.947500in}}{\pgfqpoint{0.139389in}{0.953032in}}%
\pgfpathcurveto{\pgfqpoint{0.129544in}{0.962877in}}{\pgfqpoint{0.119698in}{0.972722in}}{\pgfqpoint{0.114167in}{0.986077in}}%
\pgfpathcurveto{\pgfqpoint{0.114167in}{1.000000in}}{\pgfqpoint{0.114167in}{1.013923in}}{\pgfqpoint{0.119698in}{1.027278in}}%
\pgfpathcurveto{\pgfqpoint{0.129544in}{1.037123in}}{\pgfqpoint{0.139389in}{1.046968in}}{\pgfqpoint{0.152744in}{1.052500in}}%
\pgfpathcurveto{\pgfqpoint{0.166667in}{1.052500in}}{\pgfqpoint{0.180590in}{1.052500in}}{\pgfqpoint{0.193945in}{1.046968in}}%
\pgfpathcurveto{\pgfqpoint{0.203790in}{1.037123in}}{\pgfqpoint{0.213635in}{1.027278in}}{\pgfqpoint{0.219167in}{1.013923in}}%
\pgfpathcurveto{\pgfqpoint{0.219167in}{1.000000in}}{\pgfqpoint{0.219167in}{0.986077in}}{\pgfqpoint{0.213635in}{0.972722in}}%
\pgfpathcurveto{\pgfqpoint{0.203790in}{0.962877in}}{\pgfqpoint{0.193945in}{0.953032in}}{\pgfqpoint{0.180590in}{0.947500in}}%
\pgfpathclose%
\pgfpathmoveto{\pgfqpoint{0.333333in}{0.941667in}}%
\pgfpathcurveto{\pgfqpoint{0.348804in}{0.941667in}}{\pgfqpoint{0.363642in}{0.947813in}}{\pgfqpoint{0.374581in}{0.958752in}}%
\pgfpathcurveto{\pgfqpoint{0.385520in}{0.969691in}}{\pgfqpoint{0.391667in}{0.984530in}}{\pgfqpoint{0.391667in}{1.000000in}}%
\pgfpathcurveto{\pgfqpoint{0.391667in}{1.015470in}}{\pgfqpoint{0.385520in}{1.030309in}}{\pgfqpoint{0.374581in}{1.041248in}}%
\pgfpathcurveto{\pgfqpoint{0.363642in}{1.052187in}}{\pgfqpoint{0.348804in}{1.058333in}}{\pgfqpoint{0.333333in}{1.058333in}}%
\pgfpathcurveto{\pgfqpoint{0.317863in}{1.058333in}}{\pgfqpoint{0.303025in}{1.052187in}}{\pgfqpoint{0.292085in}{1.041248in}}%
\pgfpathcurveto{\pgfqpoint{0.281146in}{1.030309in}}{\pgfqpoint{0.275000in}{1.015470in}}{\pgfqpoint{0.275000in}{1.000000in}}%
\pgfpathcurveto{\pgfqpoint{0.275000in}{0.984530in}}{\pgfqpoint{0.281146in}{0.969691in}}{\pgfqpoint{0.292085in}{0.958752in}}%
\pgfpathcurveto{\pgfqpoint{0.303025in}{0.947813in}}{\pgfqpoint{0.317863in}{0.941667in}}{\pgfqpoint{0.333333in}{0.941667in}}%
\pgfpathclose%
\pgfpathmoveto{\pgfqpoint{0.333333in}{0.947500in}}%
\pgfpathcurveto{\pgfqpoint{0.333333in}{0.947500in}}{\pgfqpoint{0.319410in}{0.947500in}}{\pgfqpoint{0.306055in}{0.953032in}}%
\pgfpathcurveto{\pgfqpoint{0.296210in}{0.962877in}}{\pgfqpoint{0.286365in}{0.972722in}}{\pgfqpoint{0.280833in}{0.986077in}}%
\pgfpathcurveto{\pgfqpoint{0.280833in}{1.000000in}}{\pgfqpoint{0.280833in}{1.013923in}}{\pgfqpoint{0.286365in}{1.027278in}}%
\pgfpathcurveto{\pgfqpoint{0.296210in}{1.037123in}}{\pgfqpoint{0.306055in}{1.046968in}}{\pgfqpoint{0.319410in}{1.052500in}}%
\pgfpathcurveto{\pgfqpoint{0.333333in}{1.052500in}}{\pgfqpoint{0.347256in}{1.052500in}}{\pgfqpoint{0.360611in}{1.046968in}}%
\pgfpathcurveto{\pgfqpoint{0.370456in}{1.037123in}}{\pgfqpoint{0.380302in}{1.027278in}}{\pgfqpoint{0.385833in}{1.013923in}}%
\pgfpathcurveto{\pgfqpoint{0.385833in}{1.000000in}}{\pgfqpoint{0.385833in}{0.986077in}}{\pgfqpoint{0.380302in}{0.972722in}}%
\pgfpathcurveto{\pgfqpoint{0.370456in}{0.962877in}}{\pgfqpoint{0.360611in}{0.953032in}}{\pgfqpoint{0.347256in}{0.947500in}}%
\pgfpathclose%
\pgfpathmoveto{\pgfqpoint{0.500000in}{0.941667in}}%
\pgfpathcurveto{\pgfqpoint{0.515470in}{0.941667in}}{\pgfqpoint{0.530309in}{0.947813in}}{\pgfqpoint{0.541248in}{0.958752in}}%
\pgfpathcurveto{\pgfqpoint{0.552187in}{0.969691in}}{\pgfqpoint{0.558333in}{0.984530in}}{\pgfqpoint{0.558333in}{1.000000in}}%
\pgfpathcurveto{\pgfqpoint{0.558333in}{1.015470in}}{\pgfqpoint{0.552187in}{1.030309in}}{\pgfqpoint{0.541248in}{1.041248in}}%
\pgfpathcurveto{\pgfqpoint{0.530309in}{1.052187in}}{\pgfqpoint{0.515470in}{1.058333in}}{\pgfqpoint{0.500000in}{1.058333in}}%
\pgfpathcurveto{\pgfqpoint{0.484530in}{1.058333in}}{\pgfqpoint{0.469691in}{1.052187in}}{\pgfqpoint{0.458752in}{1.041248in}}%
\pgfpathcurveto{\pgfqpoint{0.447813in}{1.030309in}}{\pgfqpoint{0.441667in}{1.015470in}}{\pgfqpoint{0.441667in}{1.000000in}}%
\pgfpathcurveto{\pgfqpoint{0.441667in}{0.984530in}}{\pgfqpoint{0.447813in}{0.969691in}}{\pgfqpoint{0.458752in}{0.958752in}}%
\pgfpathcurveto{\pgfqpoint{0.469691in}{0.947813in}}{\pgfqpoint{0.484530in}{0.941667in}}{\pgfqpoint{0.500000in}{0.941667in}}%
\pgfpathclose%
\pgfpathmoveto{\pgfqpoint{0.500000in}{0.947500in}}%
\pgfpathcurveto{\pgfqpoint{0.500000in}{0.947500in}}{\pgfqpoint{0.486077in}{0.947500in}}{\pgfqpoint{0.472722in}{0.953032in}}%
\pgfpathcurveto{\pgfqpoint{0.462877in}{0.962877in}}{\pgfqpoint{0.453032in}{0.972722in}}{\pgfqpoint{0.447500in}{0.986077in}}%
\pgfpathcurveto{\pgfqpoint{0.447500in}{1.000000in}}{\pgfqpoint{0.447500in}{1.013923in}}{\pgfqpoint{0.453032in}{1.027278in}}%
\pgfpathcurveto{\pgfqpoint{0.462877in}{1.037123in}}{\pgfqpoint{0.472722in}{1.046968in}}{\pgfqpoint{0.486077in}{1.052500in}}%
\pgfpathcurveto{\pgfqpoint{0.500000in}{1.052500in}}{\pgfqpoint{0.513923in}{1.052500in}}{\pgfqpoint{0.527278in}{1.046968in}}%
\pgfpathcurveto{\pgfqpoint{0.537123in}{1.037123in}}{\pgfqpoint{0.546968in}{1.027278in}}{\pgfqpoint{0.552500in}{1.013923in}}%
\pgfpathcurveto{\pgfqpoint{0.552500in}{1.000000in}}{\pgfqpoint{0.552500in}{0.986077in}}{\pgfqpoint{0.546968in}{0.972722in}}%
\pgfpathcurveto{\pgfqpoint{0.537123in}{0.962877in}}{\pgfqpoint{0.527278in}{0.953032in}}{\pgfqpoint{0.513923in}{0.947500in}}%
\pgfpathclose%
\pgfpathmoveto{\pgfqpoint{0.666667in}{0.941667in}}%
\pgfpathcurveto{\pgfqpoint{0.682137in}{0.941667in}}{\pgfqpoint{0.696975in}{0.947813in}}{\pgfqpoint{0.707915in}{0.958752in}}%
\pgfpathcurveto{\pgfqpoint{0.718854in}{0.969691in}}{\pgfqpoint{0.725000in}{0.984530in}}{\pgfqpoint{0.725000in}{1.000000in}}%
\pgfpathcurveto{\pgfqpoint{0.725000in}{1.015470in}}{\pgfqpoint{0.718854in}{1.030309in}}{\pgfqpoint{0.707915in}{1.041248in}}%
\pgfpathcurveto{\pgfqpoint{0.696975in}{1.052187in}}{\pgfqpoint{0.682137in}{1.058333in}}{\pgfqpoint{0.666667in}{1.058333in}}%
\pgfpathcurveto{\pgfqpoint{0.651196in}{1.058333in}}{\pgfqpoint{0.636358in}{1.052187in}}{\pgfqpoint{0.625419in}{1.041248in}}%
\pgfpathcurveto{\pgfqpoint{0.614480in}{1.030309in}}{\pgfqpoint{0.608333in}{1.015470in}}{\pgfqpoint{0.608333in}{1.000000in}}%
\pgfpathcurveto{\pgfqpoint{0.608333in}{0.984530in}}{\pgfqpoint{0.614480in}{0.969691in}}{\pgfqpoint{0.625419in}{0.958752in}}%
\pgfpathcurveto{\pgfqpoint{0.636358in}{0.947813in}}{\pgfqpoint{0.651196in}{0.941667in}}{\pgfqpoint{0.666667in}{0.941667in}}%
\pgfpathclose%
\pgfpathmoveto{\pgfqpoint{0.666667in}{0.947500in}}%
\pgfpathcurveto{\pgfqpoint{0.666667in}{0.947500in}}{\pgfqpoint{0.652744in}{0.947500in}}{\pgfqpoint{0.639389in}{0.953032in}}%
\pgfpathcurveto{\pgfqpoint{0.629544in}{0.962877in}}{\pgfqpoint{0.619698in}{0.972722in}}{\pgfqpoint{0.614167in}{0.986077in}}%
\pgfpathcurveto{\pgfqpoint{0.614167in}{1.000000in}}{\pgfqpoint{0.614167in}{1.013923in}}{\pgfqpoint{0.619698in}{1.027278in}}%
\pgfpathcurveto{\pgfqpoint{0.629544in}{1.037123in}}{\pgfqpoint{0.639389in}{1.046968in}}{\pgfqpoint{0.652744in}{1.052500in}}%
\pgfpathcurveto{\pgfqpoint{0.666667in}{1.052500in}}{\pgfqpoint{0.680590in}{1.052500in}}{\pgfqpoint{0.693945in}{1.046968in}}%
\pgfpathcurveto{\pgfqpoint{0.703790in}{1.037123in}}{\pgfqpoint{0.713635in}{1.027278in}}{\pgfqpoint{0.719167in}{1.013923in}}%
\pgfpathcurveto{\pgfqpoint{0.719167in}{1.000000in}}{\pgfqpoint{0.719167in}{0.986077in}}{\pgfqpoint{0.713635in}{0.972722in}}%
\pgfpathcurveto{\pgfqpoint{0.703790in}{0.962877in}}{\pgfqpoint{0.693945in}{0.953032in}}{\pgfqpoint{0.680590in}{0.947500in}}%
\pgfpathclose%
\pgfpathmoveto{\pgfqpoint{0.833333in}{0.941667in}}%
\pgfpathcurveto{\pgfqpoint{0.848804in}{0.941667in}}{\pgfqpoint{0.863642in}{0.947813in}}{\pgfqpoint{0.874581in}{0.958752in}}%
\pgfpathcurveto{\pgfqpoint{0.885520in}{0.969691in}}{\pgfqpoint{0.891667in}{0.984530in}}{\pgfqpoint{0.891667in}{1.000000in}}%
\pgfpathcurveto{\pgfqpoint{0.891667in}{1.015470in}}{\pgfqpoint{0.885520in}{1.030309in}}{\pgfqpoint{0.874581in}{1.041248in}}%
\pgfpathcurveto{\pgfqpoint{0.863642in}{1.052187in}}{\pgfqpoint{0.848804in}{1.058333in}}{\pgfqpoint{0.833333in}{1.058333in}}%
\pgfpathcurveto{\pgfqpoint{0.817863in}{1.058333in}}{\pgfqpoint{0.803025in}{1.052187in}}{\pgfqpoint{0.792085in}{1.041248in}}%
\pgfpathcurveto{\pgfqpoint{0.781146in}{1.030309in}}{\pgfqpoint{0.775000in}{1.015470in}}{\pgfqpoint{0.775000in}{1.000000in}}%
\pgfpathcurveto{\pgfqpoint{0.775000in}{0.984530in}}{\pgfqpoint{0.781146in}{0.969691in}}{\pgfqpoint{0.792085in}{0.958752in}}%
\pgfpathcurveto{\pgfqpoint{0.803025in}{0.947813in}}{\pgfqpoint{0.817863in}{0.941667in}}{\pgfqpoint{0.833333in}{0.941667in}}%
\pgfpathclose%
\pgfpathmoveto{\pgfqpoint{0.833333in}{0.947500in}}%
\pgfpathcurveto{\pgfqpoint{0.833333in}{0.947500in}}{\pgfqpoint{0.819410in}{0.947500in}}{\pgfqpoint{0.806055in}{0.953032in}}%
\pgfpathcurveto{\pgfqpoint{0.796210in}{0.962877in}}{\pgfqpoint{0.786365in}{0.972722in}}{\pgfqpoint{0.780833in}{0.986077in}}%
\pgfpathcurveto{\pgfqpoint{0.780833in}{1.000000in}}{\pgfqpoint{0.780833in}{1.013923in}}{\pgfqpoint{0.786365in}{1.027278in}}%
\pgfpathcurveto{\pgfqpoint{0.796210in}{1.037123in}}{\pgfqpoint{0.806055in}{1.046968in}}{\pgfqpoint{0.819410in}{1.052500in}}%
\pgfpathcurveto{\pgfqpoint{0.833333in}{1.052500in}}{\pgfqpoint{0.847256in}{1.052500in}}{\pgfqpoint{0.860611in}{1.046968in}}%
\pgfpathcurveto{\pgfqpoint{0.870456in}{1.037123in}}{\pgfqpoint{0.880302in}{1.027278in}}{\pgfqpoint{0.885833in}{1.013923in}}%
\pgfpathcurveto{\pgfqpoint{0.885833in}{1.000000in}}{\pgfqpoint{0.885833in}{0.986077in}}{\pgfqpoint{0.880302in}{0.972722in}}%
\pgfpathcurveto{\pgfqpoint{0.870456in}{0.962877in}}{\pgfqpoint{0.860611in}{0.953032in}}{\pgfqpoint{0.847256in}{0.947500in}}%
\pgfpathclose%
\pgfpathmoveto{\pgfqpoint{1.000000in}{0.941667in}}%
\pgfpathcurveto{\pgfqpoint{1.015470in}{0.941667in}}{\pgfqpoint{1.030309in}{0.947813in}}{\pgfqpoint{1.041248in}{0.958752in}}%
\pgfpathcurveto{\pgfqpoint{1.052187in}{0.969691in}}{\pgfqpoint{1.058333in}{0.984530in}}{\pgfqpoint{1.058333in}{1.000000in}}%
\pgfpathcurveto{\pgfqpoint{1.058333in}{1.015470in}}{\pgfqpoint{1.052187in}{1.030309in}}{\pgfqpoint{1.041248in}{1.041248in}}%
\pgfpathcurveto{\pgfqpoint{1.030309in}{1.052187in}}{\pgfqpoint{1.015470in}{1.058333in}}{\pgfqpoint{1.000000in}{1.058333in}}%
\pgfpathcurveto{\pgfqpoint{0.984530in}{1.058333in}}{\pgfqpoint{0.969691in}{1.052187in}}{\pgfqpoint{0.958752in}{1.041248in}}%
\pgfpathcurveto{\pgfqpoint{0.947813in}{1.030309in}}{\pgfqpoint{0.941667in}{1.015470in}}{\pgfqpoint{0.941667in}{1.000000in}}%
\pgfpathcurveto{\pgfqpoint{0.941667in}{0.984530in}}{\pgfqpoint{0.947813in}{0.969691in}}{\pgfqpoint{0.958752in}{0.958752in}}%
\pgfpathcurveto{\pgfqpoint{0.969691in}{0.947813in}}{\pgfqpoint{0.984530in}{0.941667in}}{\pgfqpoint{1.000000in}{0.941667in}}%
\pgfpathclose%
\pgfpathmoveto{\pgfqpoint{1.000000in}{0.947500in}}%
\pgfpathcurveto{\pgfqpoint{1.000000in}{0.947500in}}{\pgfqpoint{0.986077in}{0.947500in}}{\pgfqpoint{0.972722in}{0.953032in}}%
\pgfpathcurveto{\pgfqpoint{0.962877in}{0.962877in}}{\pgfqpoint{0.953032in}{0.972722in}}{\pgfqpoint{0.947500in}{0.986077in}}%
\pgfpathcurveto{\pgfqpoint{0.947500in}{1.000000in}}{\pgfqpoint{0.947500in}{1.013923in}}{\pgfqpoint{0.953032in}{1.027278in}}%
\pgfpathcurveto{\pgfqpoint{0.962877in}{1.037123in}}{\pgfqpoint{0.972722in}{1.046968in}}{\pgfqpoint{0.986077in}{1.052500in}}%
\pgfpathcurveto{\pgfqpoint{1.000000in}{1.052500in}}{\pgfqpoint{1.013923in}{1.052500in}}{\pgfqpoint{1.027278in}{1.046968in}}%
\pgfpathcurveto{\pgfqpoint{1.037123in}{1.037123in}}{\pgfqpoint{1.046968in}{1.027278in}}{\pgfqpoint{1.052500in}{1.013923in}}%
\pgfpathcurveto{\pgfqpoint{1.052500in}{1.000000in}}{\pgfqpoint{1.052500in}{0.986077in}}{\pgfqpoint{1.046968in}{0.972722in}}%
\pgfpathcurveto{\pgfqpoint{1.037123in}{0.962877in}}{\pgfqpoint{1.027278in}{0.953032in}}{\pgfqpoint{1.013923in}{0.947500in}}%
\pgfpathclose%
\pgfusepath{stroke}%
\end{pgfscope}%
}%
\pgfsys@transformshift{9.228174in}{7.064715in}%
\pgfsys@useobject{currentpattern}{}%
\pgfsys@transformshift{1in}{0in}%
\pgfsys@transformshift{-1in}{0in}%
\pgfsys@transformshift{0in}{1in}%
\pgfsys@useobject{currentpattern}{}%
\pgfsys@transformshift{1in}{0in}%
\pgfsys@transformshift{-1in}{0in}%
\pgfsys@transformshift{0in}{1in}%
\pgfsys@useobject{currentpattern}{}%
\pgfsys@transformshift{1in}{0in}%
\pgfsys@transformshift{-1in}{0in}%
\pgfsys@transformshift{0in}{1in}%
\end{pgfscope}%
\begin{pgfscope}%
\pgfsetrectcap%
\pgfsetmiterjoin%
\pgfsetlinewidth{1.003750pt}%
\definecolor{currentstroke}{rgb}{1.000000,1.000000,1.000000}%
\pgfsetstrokecolor{currentstroke}%
\pgfsetdash{}{0pt}%
\pgfpathmoveto{\pgfqpoint{1.090674in}{0.637495in}}%
\pgfpathlineto{\pgfqpoint{1.090674in}{9.697495in}}%
\pgfusepath{stroke}%
\end{pgfscope}%
\begin{pgfscope}%
\pgfsetrectcap%
\pgfsetmiterjoin%
\pgfsetlinewidth{1.003750pt}%
\definecolor{currentstroke}{rgb}{1.000000,1.000000,1.000000}%
\pgfsetstrokecolor{currentstroke}%
\pgfsetdash{}{0pt}%
\pgfpathmoveto{\pgfqpoint{10.390674in}{0.637495in}}%
\pgfpathlineto{\pgfqpoint{10.390674in}{9.697495in}}%
\pgfusepath{stroke}%
\end{pgfscope}%
\begin{pgfscope}%
\pgfsetrectcap%
\pgfsetmiterjoin%
\pgfsetlinewidth{1.003750pt}%
\definecolor{currentstroke}{rgb}{1.000000,1.000000,1.000000}%
\pgfsetstrokecolor{currentstroke}%
\pgfsetdash{}{0pt}%
\pgfpathmoveto{\pgfqpoint{1.090674in}{0.637495in}}%
\pgfpathlineto{\pgfqpoint{10.390674in}{0.637495in}}%
\pgfusepath{stroke}%
\end{pgfscope}%
\begin{pgfscope}%
\pgfsetrectcap%
\pgfsetmiterjoin%
\pgfsetlinewidth{1.003750pt}%
\definecolor{currentstroke}{rgb}{1.000000,1.000000,1.000000}%
\pgfsetstrokecolor{currentstroke}%
\pgfsetdash{}{0pt}%
\pgfpathmoveto{\pgfqpoint{1.090674in}{9.697495in}}%
\pgfpathlineto{\pgfqpoint{10.390674in}{9.697495in}}%
\pgfusepath{stroke}%
\end{pgfscope}%
\begin{pgfscope}%
\definecolor{textcolor}{rgb}{0.000000,0.000000,0.000000}%
\pgfsetstrokecolor{textcolor}%
\pgfsetfillcolor{textcolor}%
\pgftext[x=5.740674in,y=9.780828in,,base]{\color{textcolor}\rmfamily\fontsize{24.000000}{28.800000}\selectfont UIUC Cooling Capacity}%
\end{pgfscope}%
\begin{pgfscope}%
\pgfsetbuttcap%
\pgfsetmiterjoin%
\definecolor{currentfill}{rgb}{0.269412,0.269412,0.269412}%
\pgfsetfillcolor{currentfill}%
\pgfsetfillopacity{0.500000}%
\pgfsetlinewidth{0.501875pt}%
\definecolor{currentstroke}{rgb}{0.269412,0.269412,0.269412}%
\pgfsetstrokecolor{currentstroke}%
\pgfsetstrokeopacity{0.500000}%
\pgfsetdash{}{0pt}%
\pgfpathmoveto{\pgfqpoint{1.274007in}{8.843020in}}%
\pgfpathlineto{\pgfqpoint{3.557776in}{8.843020in}}%
\pgfpathquadraticcurveto{\pgfqpoint{3.602221in}{8.843020in}}{\pgfqpoint{3.602221in}{8.887465in}}%
\pgfpathlineto{\pgfqpoint{3.602221in}{9.514162in}}%
\pgfpathquadraticcurveto{\pgfqpoint{3.602221in}{9.558606in}}{\pgfqpoint{3.557776in}{9.558606in}}%
\pgfpathlineto{\pgfqpoint{1.274007in}{9.558606in}}%
\pgfpathquadraticcurveto{\pgfqpoint{1.229563in}{9.558606in}}{\pgfqpoint{1.229563in}{9.514162in}}%
\pgfpathlineto{\pgfqpoint{1.229563in}{8.887465in}}%
\pgfpathquadraticcurveto{\pgfqpoint{1.229563in}{8.843020in}}{\pgfqpoint{1.274007in}{8.843020in}}%
\pgfpathclose%
\pgfusepath{stroke,fill}%
\end{pgfscope}%
\begin{pgfscope}%
\pgfsetbuttcap%
\pgfsetmiterjoin%
\definecolor{currentfill}{rgb}{0.898039,0.898039,0.898039}%
\pgfsetfillcolor{currentfill}%
\pgfsetlinewidth{0.501875pt}%
\definecolor{currentstroke}{rgb}{0.800000,0.800000,0.800000}%
\pgfsetstrokecolor{currentstroke}%
\pgfsetdash{}{0pt}%
\pgfpathmoveto{\pgfqpoint{1.246230in}{8.870798in}}%
\pgfpathlineto{\pgfqpoint{3.529998in}{8.870798in}}%
\pgfpathquadraticcurveto{\pgfqpoint{3.574443in}{8.870798in}}{\pgfqpoint{3.574443in}{8.915242in}}%
\pgfpathlineto{\pgfqpoint{3.574443in}{9.541940in}}%
\pgfpathquadraticcurveto{\pgfqpoint{3.574443in}{9.586384in}}{\pgfqpoint{3.529998in}{9.586384in}}%
\pgfpathlineto{\pgfqpoint{1.246230in}{9.586384in}}%
\pgfpathquadraticcurveto{\pgfqpoint{1.201785in}{9.586384in}}{\pgfqpoint{1.201785in}{9.541940in}}%
\pgfpathlineto{\pgfqpoint{1.201785in}{8.915242in}}%
\pgfpathquadraticcurveto{\pgfqpoint{1.201785in}{8.870798in}}{\pgfqpoint{1.246230in}{8.870798in}}%
\pgfpathclose%
\pgfusepath{stroke,fill}%
\end{pgfscope}%
\begin{pgfscope}%
\pgfsetbuttcap%
\pgfsetmiterjoin%
\definecolor{currentfill}{rgb}{0.580392,0.403922,0.741176}%
\pgfsetfillcolor{currentfill}%
\pgfsetfillopacity{0.990000}%
\pgfsetlinewidth{0.000000pt}%
\definecolor{currentstroke}{rgb}{0.000000,0.000000,0.000000}%
\pgfsetstrokecolor{currentstroke}%
\pgfsetstrokeopacity{0.990000}%
\pgfsetdash{}{0pt}%
\pgfpathmoveto{\pgfqpoint{1.290674in}{9.330828in}}%
\pgfpathlineto{\pgfqpoint{1.735119in}{9.330828in}}%
\pgfpathlineto{\pgfqpoint{1.735119in}{9.486384in}}%
\pgfpathlineto{\pgfqpoint{1.290674in}{9.486384in}}%
\pgfpathclose%
\pgfusepath{fill}%
\end{pgfscope}%
\begin{pgfscope}%
\pgfsetbuttcap%
\pgfsetmiterjoin%
\definecolor{currentfill}{rgb}{0.580392,0.403922,0.741176}%
\pgfsetfillcolor{currentfill}%
\pgfsetfillopacity{0.990000}%
\pgfsetlinewidth{0.000000pt}%
\definecolor{currentstroke}{rgb}{0.000000,0.000000,0.000000}%
\pgfsetstrokecolor{currentstroke}%
\pgfsetstrokeopacity{0.990000}%
\pgfsetdash{}{0pt}%
\pgfpathmoveto{\pgfqpoint{1.290674in}{9.330828in}}%
\pgfpathlineto{\pgfqpoint{1.735119in}{9.330828in}}%
\pgfpathlineto{\pgfqpoint{1.735119in}{9.486384in}}%
\pgfpathlineto{\pgfqpoint{1.290674in}{9.486384in}}%
\pgfpathclose%
\pgfusepath{clip}%
\pgfsys@defobject{currentpattern}{\pgfqpoint{0in}{0in}}{\pgfqpoint{1in}{1in}}{%
\begin{pgfscope}%
\pgfpathrectangle{\pgfqpoint{0in}{0in}}{\pgfqpoint{1in}{1in}}%
\pgfusepath{clip}%
\pgfpathmoveto{\pgfqpoint{-0.500000in}{0.500000in}}%
\pgfpathlineto{\pgfqpoint{0.500000in}{1.500000in}}%
\pgfpathmoveto{\pgfqpoint{-0.333333in}{0.333333in}}%
\pgfpathlineto{\pgfqpoint{0.666667in}{1.333333in}}%
\pgfpathmoveto{\pgfqpoint{-0.166667in}{0.166667in}}%
\pgfpathlineto{\pgfqpoint{0.833333in}{1.166667in}}%
\pgfpathmoveto{\pgfqpoint{0.000000in}{0.000000in}}%
\pgfpathlineto{\pgfqpoint{1.000000in}{1.000000in}}%
\pgfpathmoveto{\pgfqpoint{0.166667in}{-0.166667in}}%
\pgfpathlineto{\pgfqpoint{1.166667in}{0.833333in}}%
\pgfpathmoveto{\pgfqpoint{0.333333in}{-0.333333in}}%
\pgfpathlineto{\pgfqpoint{1.333333in}{0.666667in}}%
\pgfpathmoveto{\pgfqpoint{0.500000in}{-0.500000in}}%
\pgfpathlineto{\pgfqpoint{1.500000in}{0.500000in}}%
\pgfpathmoveto{\pgfqpoint{-0.500000in}{0.500000in}}%
\pgfpathlineto{\pgfqpoint{0.500000in}{-0.500000in}}%
\pgfpathmoveto{\pgfqpoint{-0.333333in}{0.666667in}}%
\pgfpathlineto{\pgfqpoint{0.666667in}{-0.333333in}}%
\pgfpathmoveto{\pgfqpoint{-0.166667in}{0.833333in}}%
\pgfpathlineto{\pgfqpoint{0.833333in}{-0.166667in}}%
\pgfpathmoveto{\pgfqpoint{0.000000in}{1.000000in}}%
\pgfpathlineto{\pgfqpoint{1.000000in}{0.000000in}}%
\pgfpathmoveto{\pgfqpoint{0.166667in}{1.166667in}}%
\pgfpathlineto{\pgfqpoint{1.166667in}{0.166667in}}%
\pgfpathmoveto{\pgfqpoint{0.333333in}{1.333333in}}%
\pgfpathlineto{\pgfqpoint{1.333333in}{0.333333in}}%
\pgfpathmoveto{\pgfqpoint{0.500000in}{1.500000in}}%
\pgfpathlineto{\pgfqpoint{1.500000in}{0.500000in}}%
\pgfusepath{stroke}%
\end{pgfscope}%
}%
\pgfsys@transformshift{1.290674in}{9.330828in}%
\pgfsys@useobject{currentpattern}{}%
\pgfsys@transformshift{1in}{0in}%
\pgfsys@transformshift{-1in}{0in}%
\pgfsys@transformshift{0in}{1in}%
\end{pgfscope}%
\begin{pgfscope}%
\definecolor{textcolor}{rgb}{0.000000,0.000000,0.000000}%
\pgfsetstrokecolor{textcolor}%
\pgfsetfillcolor{textcolor}%
\pgftext[x=1.912896in,y=9.330828in,left,base]{\color{textcolor}\rmfamily\fontsize{16.000000}{19.200000}\selectfont CWS}%
\end{pgfscope}%
\begin{pgfscope}%
\pgfsetbuttcap%
\pgfsetmiterjoin%
\definecolor{currentfill}{rgb}{0.890196,0.466667,0.760784}%
\pgfsetfillcolor{currentfill}%
\pgfsetfillopacity{0.990000}%
\pgfsetlinewidth{0.000000pt}%
\definecolor{currentstroke}{rgb}{0.000000,0.000000,0.000000}%
\pgfsetstrokecolor{currentstroke}%
\pgfsetstrokeopacity{0.990000}%
\pgfsetdash{}{0pt}%
\pgfpathmoveto{\pgfqpoint{1.290674in}{9.006369in}}%
\pgfpathlineto{\pgfqpoint{1.735119in}{9.006369in}}%
\pgfpathlineto{\pgfqpoint{1.735119in}{9.161924in}}%
\pgfpathlineto{\pgfqpoint{1.290674in}{9.161924in}}%
\pgfpathclose%
\pgfusepath{fill}%
\end{pgfscope}%
\begin{pgfscope}%
\pgfsetbuttcap%
\pgfsetmiterjoin%
\definecolor{currentfill}{rgb}{0.890196,0.466667,0.760784}%
\pgfsetfillcolor{currentfill}%
\pgfsetfillopacity{0.990000}%
\pgfsetlinewidth{0.000000pt}%
\definecolor{currentstroke}{rgb}{0.000000,0.000000,0.000000}%
\pgfsetstrokecolor{currentstroke}%
\pgfsetstrokeopacity{0.990000}%
\pgfsetdash{}{0pt}%
\pgfpathmoveto{\pgfqpoint{1.290674in}{9.006369in}}%
\pgfpathlineto{\pgfqpoint{1.735119in}{9.006369in}}%
\pgfpathlineto{\pgfqpoint{1.735119in}{9.161924in}}%
\pgfpathlineto{\pgfqpoint{1.290674in}{9.161924in}}%
\pgfpathclose%
\pgfusepath{clip}%
\pgfsys@defobject{currentpattern}{\pgfqpoint{0in}{0in}}{\pgfqpoint{1in}{1in}}{%
\begin{pgfscope}%
\pgfpathrectangle{\pgfqpoint{0in}{0in}}{\pgfqpoint{1in}{1in}}%
\pgfusepath{clip}%
\pgfpathmoveto{\pgfqpoint{0.000000in}{-0.058333in}}%
\pgfpathcurveto{\pgfqpoint{0.015470in}{-0.058333in}}{\pgfqpoint{0.030309in}{-0.052187in}}{\pgfqpoint{0.041248in}{-0.041248in}}%
\pgfpathcurveto{\pgfqpoint{0.052187in}{-0.030309in}}{\pgfqpoint{0.058333in}{-0.015470in}}{\pgfqpoint{0.058333in}{0.000000in}}%
\pgfpathcurveto{\pgfqpoint{0.058333in}{0.015470in}}{\pgfqpoint{0.052187in}{0.030309in}}{\pgfqpoint{0.041248in}{0.041248in}}%
\pgfpathcurveto{\pgfqpoint{0.030309in}{0.052187in}}{\pgfqpoint{0.015470in}{0.058333in}}{\pgfqpoint{0.000000in}{0.058333in}}%
\pgfpathcurveto{\pgfqpoint{-0.015470in}{0.058333in}}{\pgfqpoint{-0.030309in}{0.052187in}}{\pgfqpoint{-0.041248in}{0.041248in}}%
\pgfpathcurveto{\pgfqpoint{-0.052187in}{0.030309in}}{\pgfqpoint{-0.058333in}{0.015470in}}{\pgfqpoint{-0.058333in}{0.000000in}}%
\pgfpathcurveto{\pgfqpoint{-0.058333in}{-0.015470in}}{\pgfqpoint{-0.052187in}{-0.030309in}}{\pgfqpoint{-0.041248in}{-0.041248in}}%
\pgfpathcurveto{\pgfqpoint{-0.030309in}{-0.052187in}}{\pgfqpoint{-0.015470in}{-0.058333in}}{\pgfqpoint{0.000000in}{-0.058333in}}%
\pgfpathclose%
\pgfpathmoveto{\pgfqpoint{0.000000in}{-0.052500in}}%
\pgfpathcurveto{\pgfqpoint{0.000000in}{-0.052500in}}{\pgfqpoint{-0.013923in}{-0.052500in}}{\pgfqpoint{-0.027278in}{-0.046968in}}%
\pgfpathcurveto{\pgfqpoint{-0.037123in}{-0.037123in}}{\pgfqpoint{-0.046968in}{-0.027278in}}{\pgfqpoint{-0.052500in}{-0.013923in}}%
\pgfpathcurveto{\pgfqpoint{-0.052500in}{0.000000in}}{\pgfqpoint{-0.052500in}{0.013923in}}{\pgfqpoint{-0.046968in}{0.027278in}}%
\pgfpathcurveto{\pgfqpoint{-0.037123in}{0.037123in}}{\pgfqpoint{-0.027278in}{0.046968in}}{\pgfqpoint{-0.013923in}{0.052500in}}%
\pgfpathcurveto{\pgfqpoint{0.000000in}{0.052500in}}{\pgfqpoint{0.013923in}{0.052500in}}{\pgfqpoint{0.027278in}{0.046968in}}%
\pgfpathcurveto{\pgfqpoint{0.037123in}{0.037123in}}{\pgfqpoint{0.046968in}{0.027278in}}{\pgfqpoint{0.052500in}{0.013923in}}%
\pgfpathcurveto{\pgfqpoint{0.052500in}{0.000000in}}{\pgfqpoint{0.052500in}{-0.013923in}}{\pgfqpoint{0.046968in}{-0.027278in}}%
\pgfpathcurveto{\pgfqpoint{0.037123in}{-0.037123in}}{\pgfqpoint{0.027278in}{-0.046968in}}{\pgfqpoint{0.013923in}{-0.052500in}}%
\pgfpathclose%
\pgfpathmoveto{\pgfqpoint{0.166667in}{-0.058333in}}%
\pgfpathcurveto{\pgfqpoint{0.182137in}{-0.058333in}}{\pgfqpoint{0.196975in}{-0.052187in}}{\pgfqpoint{0.207915in}{-0.041248in}}%
\pgfpathcurveto{\pgfqpoint{0.218854in}{-0.030309in}}{\pgfqpoint{0.225000in}{-0.015470in}}{\pgfqpoint{0.225000in}{0.000000in}}%
\pgfpathcurveto{\pgfqpoint{0.225000in}{0.015470in}}{\pgfqpoint{0.218854in}{0.030309in}}{\pgfqpoint{0.207915in}{0.041248in}}%
\pgfpathcurveto{\pgfqpoint{0.196975in}{0.052187in}}{\pgfqpoint{0.182137in}{0.058333in}}{\pgfqpoint{0.166667in}{0.058333in}}%
\pgfpathcurveto{\pgfqpoint{0.151196in}{0.058333in}}{\pgfqpoint{0.136358in}{0.052187in}}{\pgfqpoint{0.125419in}{0.041248in}}%
\pgfpathcurveto{\pgfqpoint{0.114480in}{0.030309in}}{\pgfqpoint{0.108333in}{0.015470in}}{\pgfqpoint{0.108333in}{0.000000in}}%
\pgfpathcurveto{\pgfqpoint{0.108333in}{-0.015470in}}{\pgfqpoint{0.114480in}{-0.030309in}}{\pgfqpoint{0.125419in}{-0.041248in}}%
\pgfpathcurveto{\pgfqpoint{0.136358in}{-0.052187in}}{\pgfqpoint{0.151196in}{-0.058333in}}{\pgfqpoint{0.166667in}{-0.058333in}}%
\pgfpathclose%
\pgfpathmoveto{\pgfqpoint{0.166667in}{-0.052500in}}%
\pgfpathcurveto{\pgfqpoint{0.166667in}{-0.052500in}}{\pgfqpoint{0.152744in}{-0.052500in}}{\pgfqpoint{0.139389in}{-0.046968in}}%
\pgfpathcurveto{\pgfqpoint{0.129544in}{-0.037123in}}{\pgfqpoint{0.119698in}{-0.027278in}}{\pgfqpoint{0.114167in}{-0.013923in}}%
\pgfpathcurveto{\pgfqpoint{0.114167in}{0.000000in}}{\pgfqpoint{0.114167in}{0.013923in}}{\pgfqpoint{0.119698in}{0.027278in}}%
\pgfpathcurveto{\pgfqpoint{0.129544in}{0.037123in}}{\pgfqpoint{0.139389in}{0.046968in}}{\pgfqpoint{0.152744in}{0.052500in}}%
\pgfpathcurveto{\pgfqpoint{0.166667in}{0.052500in}}{\pgfqpoint{0.180590in}{0.052500in}}{\pgfqpoint{0.193945in}{0.046968in}}%
\pgfpathcurveto{\pgfqpoint{0.203790in}{0.037123in}}{\pgfqpoint{0.213635in}{0.027278in}}{\pgfqpoint{0.219167in}{0.013923in}}%
\pgfpathcurveto{\pgfqpoint{0.219167in}{0.000000in}}{\pgfqpoint{0.219167in}{-0.013923in}}{\pgfqpoint{0.213635in}{-0.027278in}}%
\pgfpathcurveto{\pgfqpoint{0.203790in}{-0.037123in}}{\pgfqpoint{0.193945in}{-0.046968in}}{\pgfqpoint{0.180590in}{-0.052500in}}%
\pgfpathclose%
\pgfpathmoveto{\pgfqpoint{0.333333in}{-0.058333in}}%
\pgfpathcurveto{\pgfqpoint{0.348804in}{-0.058333in}}{\pgfqpoint{0.363642in}{-0.052187in}}{\pgfqpoint{0.374581in}{-0.041248in}}%
\pgfpathcurveto{\pgfqpoint{0.385520in}{-0.030309in}}{\pgfqpoint{0.391667in}{-0.015470in}}{\pgfqpoint{0.391667in}{0.000000in}}%
\pgfpathcurveto{\pgfqpoint{0.391667in}{0.015470in}}{\pgfqpoint{0.385520in}{0.030309in}}{\pgfqpoint{0.374581in}{0.041248in}}%
\pgfpathcurveto{\pgfqpoint{0.363642in}{0.052187in}}{\pgfqpoint{0.348804in}{0.058333in}}{\pgfqpoint{0.333333in}{0.058333in}}%
\pgfpathcurveto{\pgfqpoint{0.317863in}{0.058333in}}{\pgfqpoint{0.303025in}{0.052187in}}{\pgfqpoint{0.292085in}{0.041248in}}%
\pgfpathcurveto{\pgfqpoint{0.281146in}{0.030309in}}{\pgfqpoint{0.275000in}{0.015470in}}{\pgfqpoint{0.275000in}{0.000000in}}%
\pgfpathcurveto{\pgfqpoint{0.275000in}{-0.015470in}}{\pgfqpoint{0.281146in}{-0.030309in}}{\pgfqpoint{0.292085in}{-0.041248in}}%
\pgfpathcurveto{\pgfqpoint{0.303025in}{-0.052187in}}{\pgfqpoint{0.317863in}{-0.058333in}}{\pgfqpoint{0.333333in}{-0.058333in}}%
\pgfpathclose%
\pgfpathmoveto{\pgfqpoint{0.333333in}{-0.052500in}}%
\pgfpathcurveto{\pgfqpoint{0.333333in}{-0.052500in}}{\pgfqpoint{0.319410in}{-0.052500in}}{\pgfqpoint{0.306055in}{-0.046968in}}%
\pgfpathcurveto{\pgfqpoint{0.296210in}{-0.037123in}}{\pgfqpoint{0.286365in}{-0.027278in}}{\pgfqpoint{0.280833in}{-0.013923in}}%
\pgfpathcurveto{\pgfqpoint{0.280833in}{0.000000in}}{\pgfqpoint{0.280833in}{0.013923in}}{\pgfqpoint{0.286365in}{0.027278in}}%
\pgfpathcurveto{\pgfqpoint{0.296210in}{0.037123in}}{\pgfqpoint{0.306055in}{0.046968in}}{\pgfqpoint{0.319410in}{0.052500in}}%
\pgfpathcurveto{\pgfqpoint{0.333333in}{0.052500in}}{\pgfqpoint{0.347256in}{0.052500in}}{\pgfqpoint{0.360611in}{0.046968in}}%
\pgfpathcurveto{\pgfqpoint{0.370456in}{0.037123in}}{\pgfqpoint{0.380302in}{0.027278in}}{\pgfqpoint{0.385833in}{0.013923in}}%
\pgfpathcurveto{\pgfqpoint{0.385833in}{0.000000in}}{\pgfqpoint{0.385833in}{-0.013923in}}{\pgfqpoint{0.380302in}{-0.027278in}}%
\pgfpathcurveto{\pgfqpoint{0.370456in}{-0.037123in}}{\pgfqpoint{0.360611in}{-0.046968in}}{\pgfqpoint{0.347256in}{-0.052500in}}%
\pgfpathclose%
\pgfpathmoveto{\pgfqpoint{0.500000in}{-0.058333in}}%
\pgfpathcurveto{\pgfqpoint{0.515470in}{-0.058333in}}{\pgfqpoint{0.530309in}{-0.052187in}}{\pgfqpoint{0.541248in}{-0.041248in}}%
\pgfpathcurveto{\pgfqpoint{0.552187in}{-0.030309in}}{\pgfqpoint{0.558333in}{-0.015470in}}{\pgfqpoint{0.558333in}{0.000000in}}%
\pgfpathcurveto{\pgfqpoint{0.558333in}{0.015470in}}{\pgfqpoint{0.552187in}{0.030309in}}{\pgfqpoint{0.541248in}{0.041248in}}%
\pgfpathcurveto{\pgfqpoint{0.530309in}{0.052187in}}{\pgfqpoint{0.515470in}{0.058333in}}{\pgfqpoint{0.500000in}{0.058333in}}%
\pgfpathcurveto{\pgfqpoint{0.484530in}{0.058333in}}{\pgfqpoint{0.469691in}{0.052187in}}{\pgfqpoint{0.458752in}{0.041248in}}%
\pgfpathcurveto{\pgfqpoint{0.447813in}{0.030309in}}{\pgfqpoint{0.441667in}{0.015470in}}{\pgfqpoint{0.441667in}{0.000000in}}%
\pgfpathcurveto{\pgfqpoint{0.441667in}{-0.015470in}}{\pgfqpoint{0.447813in}{-0.030309in}}{\pgfqpoint{0.458752in}{-0.041248in}}%
\pgfpathcurveto{\pgfqpoint{0.469691in}{-0.052187in}}{\pgfqpoint{0.484530in}{-0.058333in}}{\pgfqpoint{0.500000in}{-0.058333in}}%
\pgfpathclose%
\pgfpathmoveto{\pgfqpoint{0.500000in}{-0.052500in}}%
\pgfpathcurveto{\pgfqpoint{0.500000in}{-0.052500in}}{\pgfqpoint{0.486077in}{-0.052500in}}{\pgfqpoint{0.472722in}{-0.046968in}}%
\pgfpathcurveto{\pgfqpoint{0.462877in}{-0.037123in}}{\pgfqpoint{0.453032in}{-0.027278in}}{\pgfqpoint{0.447500in}{-0.013923in}}%
\pgfpathcurveto{\pgfqpoint{0.447500in}{0.000000in}}{\pgfqpoint{0.447500in}{0.013923in}}{\pgfqpoint{0.453032in}{0.027278in}}%
\pgfpathcurveto{\pgfqpoint{0.462877in}{0.037123in}}{\pgfqpoint{0.472722in}{0.046968in}}{\pgfqpoint{0.486077in}{0.052500in}}%
\pgfpathcurveto{\pgfqpoint{0.500000in}{0.052500in}}{\pgfqpoint{0.513923in}{0.052500in}}{\pgfqpoint{0.527278in}{0.046968in}}%
\pgfpathcurveto{\pgfqpoint{0.537123in}{0.037123in}}{\pgfqpoint{0.546968in}{0.027278in}}{\pgfqpoint{0.552500in}{0.013923in}}%
\pgfpathcurveto{\pgfqpoint{0.552500in}{0.000000in}}{\pgfqpoint{0.552500in}{-0.013923in}}{\pgfqpoint{0.546968in}{-0.027278in}}%
\pgfpathcurveto{\pgfqpoint{0.537123in}{-0.037123in}}{\pgfqpoint{0.527278in}{-0.046968in}}{\pgfqpoint{0.513923in}{-0.052500in}}%
\pgfpathclose%
\pgfpathmoveto{\pgfqpoint{0.666667in}{-0.058333in}}%
\pgfpathcurveto{\pgfqpoint{0.682137in}{-0.058333in}}{\pgfqpoint{0.696975in}{-0.052187in}}{\pgfqpoint{0.707915in}{-0.041248in}}%
\pgfpathcurveto{\pgfqpoint{0.718854in}{-0.030309in}}{\pgfqpoint{0.725000in}{-0.015470in}}{\pgfqpoint{0.725000in}{0.000000in}}%
\pgfpathcurveto{\pgfqpoint{0.725000in}{0.015470in}}{\pgfqpoint{0.718854in}{0.030309in}}{\pgfqpoint{0.707915in}{0.041248in}}%
\pgfpathcurveto{\pgfqpoint{0.696975in}{0.052187in}}{\pgfqpoint{0.682137in}{0.058333in}}{\pgfqpoint{0.666667in}{0.058333in}}%
\pgfpathcurveto{\pgfqpoint{0.651196in}{0.058333in}}{\pgfqpoint{0.636358in}{0.052187in}}{\pgfqpoint{0.625419in}{0.041248in}}%
\pgfpathcurveto{\pgfqpoint{0.614480in}{0.030309in}}{\pgfqpoint{0.608333in}{0.015470in}}{\pgfqpoint{0.608333in}{0.000000in}}%
\pgfpathcurveto{\pgfqpoint{0.608333in}{-0.015470in}}{\pgfqpoint{0.614480in}{-0.030309in}}{\pgfqpoint{0.625419in}{-0.041248in}}%
\pgfpathcurveto{\pgfqpoint{0.636358in}{-0.052187in}}{\pgfqpoint{0.651196in}{-0.058333in}}{\pgfqpoint{0.666667in}{-0.058333in}}%
\pgfpathclose%
\pgfpathmoveto{\pgfqpoint{0.666667in}{-0.052500in}}%
\pgfpathcurveto{\pgfqpoint{0.666667in}{-0.052500in}}{\pgfqpoint{0.652744in}{-0.052500in}}{\pgfqpoint{0.639389in}{-0.046968in}}%
\pgfpathcurveto{\pgfqpoint{0.629544in}{-0.037123in}}{\pgfqpoint{0.619698in}{-0.027278in}}{\pgfqpoint{0.614167in}{-0.013923in}}%
\pgfpathcurveto{\pgfqpoint{0.614167in}{0.000000in}}{\pgfqpoint{0.614167in}{0.013923in}}{\pgfqpoint{0.619698in}{0.027278in}}%
\pgfpathcurveto{\pgfqpoint{0.629544in}{0.037123in}}{\pgfqpoint{0.639389in}{0.046968in}}{\pgfqpoint{0.652744in}{0.052500in}}%
\pgfpathcurveto{\pgfqpoint{0.666667in}{0.052500in}}{\pgfqpoint{0.680590in}{0.052500in}}{\pgfqpoint{0.693945in}{0.046968in}}%
\pgfpathcurveto{\pgfqpoint{0.703790in}{0.037123in}}{\pgfqpoint{0.713635in}{0.027278in}}{\pgfqpoint{0.719167in}{0.013923in}}%
\pgfpathcurveto{\pgfqpoint{0.719167in}{0.000000in}}{\pgfqpoint{0.719167in}{-0.013923in}}{\pgfqpoint{0.713635in}{-0.027278in}}%
\pgfpathcurveto{\pgfqpoint{0.703790in}{-0.037123in}}{\pgfqpoint{0.693945in}{-0.046968in}}{\pgfqpoint{0.680590in}{-0.052500in}}%
\pgfpathclose%
\pgfpathmoveto{\pgfqpoint{0.833333in}{-0.058333in}}%
\pgfpathcurveto{\pgfqpoint{0.848804in}{-0.058333in}}{\pgfqpoint{0.863642in}{-0.052187in}}{\pgfqpoint{0.874581in}{-0.041248in}}%
\pgfpathcurveto{\pgfqpoint{0.885520in}{-0.030309in}}{\pgfqpoint{0.891667in}{-0.015470in}}{\pgfqpoint{0.891667in}{0.000000in}}%
\pgfpathcurveto{\pgfqpoint{0.891667in}{0.015470in}}{\pgfqpoint{0.885520in}{0.030309in}}{\pgfqpoint{0.874581in}{0.041248in}}%
\pgfpathcurveto{\pgfqpoint{0.863642in}{0.052187in}}{\pgfqpoint{0.848804in}{0.058333in}}{\pgfqpoint{0.833333in}{0.058333in}}%
\pgfpathcurveto{\pgfqpoint{0.817863in}{0.058333in}}{\pgfqpoint{0.803025in}{0.052187in}}{\pgfqpoint{0.792085in}{0.041248in}}%
\pgfpathcurveto{\pgfqpoint{0.781146in}{0.030309in}}{\pgfqpoint{0.775000in}{0.015470in}}{\pgfqpoint{0.775000in}{0.000000in}}%
\pgfpathcurveto{\pgfqpoint{0.775000in}{-0.015470in}}{\pgfqpoint{0.781146in}{-0.030309in}}{\pgfqpoint{0.792085in}{-0.041248in}}%
\pgfpathcurveto{\pgfqpoint{0.803025in}{-0.052187in}}{\pgfqpoint{0.817863in}{-0.058333in}}{\pgfqpoint{0.833333in}{-0.058333in}}%
\pgfpathclose%
\pgfpathmoveto{\pgfqpoint{0.833333in}{-0.052500in}}%
\pgfpathcurveto{\pgfqpoint{0.833333in}{-0.052500in}}{\pgfqpoint{0.819410in}{-0.052500in}}{\pgfqpoint{0.806055in}{-0.046968in}}%
\pgfpathcurveto{\pgfqpoint{0.796210in}{-0.037123in}}{\pgfqpoint{0.786365in}{-0.027278in}}{\pgfqpoint{0.780833in}{-0.013923in}}%
\pgfpathcurveto{\pgfqpoint{0.780833in}{0.000000in}}{\pgfqpoint{0.780833in}{0.013923in}}{\pgfqpoint{0.786365in}{0.027278in}}%
\pgfpathcurveto{\pgfqpoint{0.796210in}{0.037123in}}{\pgfqpoint{0.806055in}{0.046968in}}{\pgfqpoint{0.819410in}{0.052500in}}%
\pgfpathcurveto{\pgfqpoint{0.833333in}{0.052500in}}{\pgfqpoint{0.847256in}{0.052500in}}{\pgfqpoint{0.860611in}{0.046968in}}%
\pgfpathcurveto{\pgfqpoint{0.870456in}{0.037123in}}{\pgfqpoint{0.880302in}{0.027278in}}{\pgfqpoint{0.885833in}{0.013923in}}%
\pgfpathcurveto{\pgfqpoint{0.885833in}{0.000000in}}{\pgfqpoint{0.885833in}{-0.013923in}}{\pgfqpoint{0.880302in}{-0.027278in}}%
\pgfpathcurveto{\pgfqpoint{0.870456in}{-0.037123in}}{\pgfqpoint{0.860611in}{-0.046968in}}{\pgfqpoint{0.847256in}{-0.052500in}}%
\pgfpathclose%
\pgfpathmoveto{\pgfqpoint{1.000000in}{-0.058333in}}%
\pgfpathcurveto{\pgfqpoint{1.015470in}{-0.058333in}}{\pgfqpoint{1.030309in}{-0.052187in}}{\pgfqpoint{1.041248in}{-0.041248in}}%
\pgfpathcurveto{\pgfqpoint{1.052187in}{-0.030309in}}{\pgfqpoint{1.058333in}{-0.015470in}}{\pgfqpoint{1.058333in}{0.000000in}}%
\pgfpathcurveto{\pgfqpoint{1.058333in}{0.015470in}}{\pgfqpoint{1.052187in}{0.030309in}}{\pgfqpoint{1.041248in}{0.041248in}}%
\pgfpathcurveto{\pgfqpoint{1.030309in}{0.052187in}}{\pgfqpoint{1.015470in}{0.058333in}}{\pgfqpoint{1.000000in}{0.058333in}}%
\pgfpathcurveto{\pgfqpoint{0.984530in}{0.058333in}}{\pgfqpoint{0.969691in}{0.052187in}}{\pgfqpoint{0.958752in}{0.041248in}}%
\pgfpathcurveto{\pgfqpoint{0.947813in}{0.030309in}}{\pgfqpoint{0.941667in}{0.015470in}}{\pgfqpoint{0.941667in}{0.000000in}}%
\pgfpathcurveto{\pgfqpoint{0.941667in}{-0.015470in}}{\pgfqpoint{0.947813in}{-0.030309in}}{\pgfqpoint{0.958752in}{-0.041248in}}%
\pgfpathcurveto{\pgfqpoint{0.969691in}{-0.052187in}}{\pgfqpoint{0.984530in}{-0.058333in}}{\pgfqpoint{1.000000in}{-0.058333in}}%
\pgfpathclose%
\pgfpathmoveto{\pgfqpoint{1.000000in}{-0.052500in}}%
\pgfpathcurveto{\pgfqpoint{1.000000in}{-0.052500in}}{\pgfqpoint{0.986077in}{-0.052500in}}{\pgfqpoint{0.972722in}{-0.046968in}}%
\pgfpathcurveto{\pgfqpoint{0.962877in}{-0.037123in}}{\pgfqpoint{0.953032in}{-0.027278in}}{\pgfqpoint{0.947500in}{-0.013923in}}%
\pgfpathcurveto{\pgfqpoint{0.947500in}{0.000000in}}{\pgfqpoint{0.947500in}{0.013923in}}{\pgfqpoint{0.953032in}{0.027278in}}%
\pgfpathcurveto{\pgfqpoint{0.962877in}{0.037123in}}{\pgfqpoint{0.972722in}{0.046968in}}{\pgfqpoint{0.986077in}{0.052500in}}%
\pgfpathcurveto{\pgfqpoint{1.000000in}{0.052500in}}{\pgfqpoint{1.013923in}{0.052500in}}{\pgfqpoint{1.027278in}{0.046968in}}%
\pgfpathcurveto{\pgfqpoint{1.037123in}{0.037123in}}{\pgfqpoint{1.046968in}{0.027278in}}{\pgfqpoint{1.052500in}{0.013923in}}%
\pgfpathcurveto{\pgfqpoint{1.052500in}{0.000000in}}{\pgfqpoint{1.052500in}{-0.013923in}}{\pgfqpoint{1.046968in}{-0.027278in}}%
\pgfpathcurveto{\pgfqpoint{1.037123in}{-0.037123in}}{\pgfqpoint{1.027278in}{-0.046968in}}{\pgfqpoint{1.013923in}{-0.052500in}}%
\pgfpathclose%
\pgfpathmoveto{\pgfqpoint{0.083333in}{0.108333in}}%
\pgfpathcurveto{\pgfqpoint{0.098804in}{0.108333in}}{\pgfqpoint{0.113642in}{0.114480in}}{\pgfqpoint{0.124581in}{0.125419in}}%
\pgfpathcurveto{\pgfqpoint{0.135520in}{0.136358in}}{\pgfqpoint{0.141667in}{0.151196in}}{\pgfqpoint{0.141667in}{0.166667in}}%
\pgfpathcurveto{\pgfqpoint{0.141667in}{0.182137in}}{\pgfqpoint{0.135520in}{0.196975in}}{\pgfqpoint{0.124581in}{0.207915in}}%
\pgfpathcurveto{\pgfqpoint{0.113642in}{0.218854in}}{\pgfqpoint{0.098804in}{0.225000in}}{\pgfqpoint{0.083333in}{0.225000in}}%
\pgfpathcurveto{\pgfqpoint{0.067863in}{0.225000in}}{\pgfqpoint{0.053025in}{0.218854in}}{\pgfqpoint{0.042085in}{0.207915in}}%
\pgfpathcurveto{\pgfqpoint{0.031146in}{0.196975in}}{\pgfqpoint{0.025000in}{0.182137in}}{\pgfqpoint{0.025000in}{0.166667in}}%
\pgfpathcurveto{\pgfqpoint{0.025000in}{0.151196in}}{\pgfqpoint{0.031146in}{0.136358in}}{\pgfqpoint{0.042085in}{0.125419in}}%
\pgfpathcurveto{\pgfqpoint{0.053025in}{0.114480in}}{\pgfqpoint{0.067863in}{0.108333in}}{\pgfqpoint{0.083333in}{0.108333in}}%
\pgfpathclose%
\pgfpathmoveto{\pgfqpoint{0.083333in}{0.114167in}}%
\pgfpathcurveto{\pgfqpoint{0.083333in}{0.114167in}}{\pgfqpoint{0.069410in}{0.114167in}}{\pgfqpoint{0.056055in}{0.119698in}}%
\pgfpathcurveto{\pgfqpoint{0.046210in}{0.129544in}}{\pgfqpoint{0.036365in}{0.139389in}}{\pgfqpoint{0.030833in}{0.152744in}}%
\pgfpathcurveto{\pgfqpoint{0.030833in}{0.166667in}}{\pgfqpoint{0.030833in}{0.180590in}}{\pgfqpoint{0.036365in}{0.193945in}}%
\pgfpathcurveto{\pgfqpoint{0.046210in}{0.203790in}}{\pgfqpoint{0.056055in}{0.213635in}}{\pgfqpoint{0.069410in}{0.219167in}}%
\pgfpathcurveto{\pgfqpoint{0.083333in}{0.219167in}}{\pgfqpoint{0.097256in}{0.219167in}}{\pgfqpoint{0.110611in}{0.213635in}}%
\pgfpathcurveto{\pgfqpoint{0.120456in}{0.203790in}}{\pgfqpoint{0.130302in}{0.193945in}}{\pgfqpoint{0.135833in}{0.180590in}}%
\pgfpathcurveto{\pgfqpoint{0.135833in}{0.166667in}}{\pgfqpoint{0.135833in}{0.152744in}}{\pgfqpoint{0.130302in}{0.139389in}}%
\pgfpathcurveto{\pgfqpoint{0.120456in}{0.129544in}}{\pgfqpoint{0.110611in}{0.119698in}}{\pgfqpoint{0.097256in}{0.114167in}}%
\pgfpathclose%
\pgfpathmoveto{\pgfqpoint{0.250000in}{0.108333in}}%
\pgfpathcurveto{\pgfqpoint{0.265470in}{0.108333in}}{\pgfqpoint{0.280309in}{0.114480in}}{\pgfqpoint{0.291248in}{0.125419in}}%
\pgfpathcurveto{\pgfqpoint{0.302187in}{0.136358in}}{\pgfqpoint{0.308333in}{0.151196in}}{\pgfqpoint{0.308333in}{0.166667in}}%
\pgfpathcurveto{\pgfqpoint{0.308333in}{0.182137in}}{\pgfqpoint{0.302187in}{0.196975in}}{\pgfqpoint{0.291248in}{0.207915in}}%
\pgfpathcurveto{\pgfqpoint{0.280309in}{0.218854in}}{\pgfqpoint{0.265470in}{0.225000in}}{\pgfqpoint{0.250000in}{0.225000in}}%
\pgfpathcurveto{\pgfqpoint{0.234530in}{0.225000in}}{\pgfqpoint{0.219691in}{0.218854in}}{\pgfqpoint{0.208752in}{0.207915in}}%
\pgfpathcurveto{\pgfqpoint{0.197813in}{0.196975in}}{\pgfqpoint{0.191667in}{0.182137in}}{\pgfqpoint{0.191667in}{0.166667in}}%
\pgfpathcurveto{\pgfqpoint{0.191667in}{0.151196in}}{\pgfqpoint{0.197813in}{0.136358in}}{\pgfqpoint{0.208752in}{0.125419in}}%
\pgfpathcurveto{\pgfqpoint{0.219691in}{0.114480in}}{\pgfqpoint{0.234530in}{0.108333in}}{\pgfqpoint{0.250000in}{0.108333in}}%
\pgfpathclose%
\pgfpathmoveto{\pgfqpoint{0.250000in}{0.114167in}}%
\pgfpathcurveto{\pgfqpoint{0.250000in}{0.114167in}}{\pgfqpoint{0.236077in}{0.114167in}}{\pgfqpoint{0.222722in}{0.119698in}}%
\pgfpathcurveto{\pgfqpoint{0.212877in}{0.129544in}}{\pgfqpoint{0.203032in}{0.139389in}}{\pgfqpoint{0.197500in}{0.152744in}}%
\pgfpathcurveto{\pgfqpoint{0.197500in}{0.166667in}}{\pgfqpoint{0.197500in}{0.180590in}}{\pgfqpoint{0.203032in}{0.193945in}}%
\pgfpathcurveto{\pgfqpoint{0.212877in}{0.203790in}}{\pgfqpoint{0.222722in}{0.213635in}}{\pgfqpoint{0.236077in}{0.219167in}}%
\pgfpathcurveto{\pgfqpoint{0.250000in}{0.219167in}}{\pgfqpoint{0.263923in}{0.219167in}}{\pgfqpoint{0.277278in}{0.213635in}}%
\pgfpathcurveto{\pgfqpoint{0.287123in}{0.203790in}}{\pgfqpoint{0.296968in}{0.193945in}}{\pgfqpoint{0.302500in}{0.180590in}}%
\pgfpathcurveto{\pgfqpoint{0.302500in}{0.166667in}}{\pgfqpoint{0.302500in}{0.152744in}}{\pgfqpoint{0.296968in}{0.139389in}}%
\pgfpathcurveto{\pgfqpoint{0.287123in}{0.129544in}}{\pgfqpoint{0.277278in}{0.119698in}}{\pgfqpoint{0.263923in}{0.114167in}}%
\pgfpathclose%
\pgfpathmoveto{\pgfqpoint{0.416667in}{0.108333in}}%
\pgfpathcurveto{\pgfqpoint{0.432137in}{0.108333in}}{\pgfqpoint{0.446975in}{0.114480in}}{\pgfqpoint{0.457915in}{0.125419in}}%
\pgfpathcurveto{\pgfqpoint{0.468854in}{0.136358in}}{\pgfqpoint{0.475000in}{0.151196in}}{\pgfqpoint{0.475000in}{0.166667in}}%
\pgfpathcurveto{\pgfqpoint{0.475000in}{0.182137in}}{\pgfqpoint{0.468854in}{0.196975in}}{\pgfqpoint{0.457915in}{0.207915in}}%
\pgfpathcurveto{\pgfqpoint{0.446975in}{0.218854in}}{\pgfqpoint{0.432137in}{0.225000in}}{\pgfqpoint{0.416667in}{0.225000in}}%
\pgfpathcurveto{\pgfqpoint{0.401196in}{0.225000in}}{\pgfqpoint{0.386358in}{0.218854in}}{\pgfqpoint{0.375419in}{0.207915in}}%
\pgfpathcurveto{\pgfqpoint{0.364480in}{0.196975in}}{\pgfqpoint{0.358333in}{0.182137in}}{\pgfqpoint{0.358333in}{0.166667in}}%
\pgfpathcurveto{\pgfqpoint{0.358333in}{0.151196in}}{\pgfqpoint{0.364480in}{0.136358in}}{\pgfqpoint{0.375419in}{0.125419in}}%
\pgfpathcurveto{\pgfqpoint{0.386358in}{0.114480in}}{\pgfqpoint{0.401196in}{0.108333in}}{\pgfqpoint{0.416667in}{0.108333in}}%
\pgfpathclose%
\pgfpathmoveto{\pgfqpoint{0.416667in}{0.114167in}}%
\pgfpathcurveto{\pgfqpoint{0.416667in}{0.114167in}}{\pgfqpoint{0.402744in}{0.114167in}}{\pgfqpoint{0.389389in}{0.119698in}}%
\pgfpathcurveto{\pgfqpoint{0.379544in}{0.129544in}}{\pgfqpoint{0.369698in}{0.139389in}}{\pgfqpoint{0.364167in}{0.152744in}}%
\pgfpathcurveto{\pgfqpoint{0.364167in}{0.166667in}}{\pgfqpoint{0.364167in}{0.180590in}}{\pgfqpoint{0.369698in}{0.193945in}}%
\pgfpathcurveto{\pgfqpoint{0.379544in}{0.203790in}}{\pgfqpoint{0.389389in}{0.213635in}}{\pgfqpoint{0.402744in}{0.219167in}}%
\pgfpathcurveto{\pgfqpoint{0.416667in}{0.219167in}}{\pgfqpoint{0.430590in}{0.219167in}}{\pgfqpoint{0.443945in}{0.213635in}}%
\pgfpathcurveto{\pgfqpoint{0.453790in}{0.203790in}}{\pgfqpoint{0.463635in}{0.193945in}}{\pgfqpoint{0.469167in}{0.180590in}}%
\pgfpathcurveto{\pgfqpoint{0.469167in}{0.166667in}}{\pgfqpoint{0.469167in}{0.152744in}}{\pgfqpoint{0.463635in}{0.139389in}}%
\pgfpathcurveto{\pgfqpoint{0.453790in}{0.129544in}}{\pgfqpoint{0.443945in}{0.119698in}}{\pgfqpoint{0.430590in}{0.114167in}}%
\pgfpathclose%
\pgfpathmoveto{\pgfqpoint{0.583333in}{0.108333in}}%
\pgfpathcurveto{\pgfqpoint{0.598804in}{0.108333in}}{\pgfqpoint{0.613642in}{0.114480in}}{\pgfqpoint{0.624581in}{0.125419in}}%
\pgfpathcurveto{\pgfqpoint{0.635520in}{0.136358in}}{\pgfqpoint{0.641667in}{0.151196in}}{\pgfqpoint{0.641667in}{0.166667in}}%
\pgfpathcurveto{\pgfqpoint{0.641667in}{0.182137in}}{\pgfqpoint{0.635520in}{0.196975in}}{\pgfqpoint{0.624581in}{0.207915in}}%
\pgfpathcurveto{\pgfqpoint{0.613642in}{0.218854in}}{\pgfqpoint{0.598804in}{0.225000in}}{\pgfqpoint{0.583333in}{0.225000in}}%
\pgfpathcurveto{\pgfqpoint{0.567863in}{0.225000in}}{\pgfqpoint{0.553025in}{0.218854in}}{\pgfqpoint{0.542085in}{0.207915in}}%
\pgfpathcurveto{\pgfqpoint{0.531146in}{0.196975in}}{\pgfqpoint{0.525000in}{0.182137in}}{\pgfqpoint{0.525000in}{0.166667in}}%
\pgfpathcurveto{\pgfqpoint{0.525000in}{0.151196in}}{\pgfqpoint{0.531146in}{0.136358in}}{\pgfqpoint{0.542085in}{0.125419in}}%
\pgfpathcurveto{\pgfqpoint{0.553025in}{0.114480in}}{\pgfqpoint{0.567863in}{0.108333in}}{\pgfqpoint{0.583333in}{0.108333in}}%
\pgfpathclose%
\pgfpathmoveto{\pgfqpoint{0.583333in}{0.114167in}}%
\pgfpathcurveto{\pgfqpoint{0.583333in}{0.114167in}}{\pgfqpoint{0.569410in}{0.114167in}}{\pgfqpoint{0.556055in}{0.119698in}}%
\pgfpathcurveto{\pgfqpoint{0.546210in}{0.129544in}}{\pgfqpoint{0.536365in}{0.139389in}}{\pgfqpoint{0.530833in}{0.152744in}}%
\pgfpathcurveto{\pgfqpoint{0.530833in}{0.166667in}}{\pgfqpoint{0.530833in}{0.180590in}}{\pgfqpoint{0.536365in}{0.193945in}}%
\pgfpathcurveto{\pgfqpoint{0.546210in}{0.203790in}}{\pgfqpoint{0.556055in}{0.213635in}}{\pgfqpoint{0.569410in}{0.219167in}}%
\pgfpathcurveto{\pgfqpoint{0.583333in}{0.219167in}}{\pgfqpoint{0.597256in}{0.219167in}}{\pgfqpoint{0.610611in}{0.213635in}}%
\pgfpathcurveto{\pgfqpoint{0.620456in}{0.203790in}}{\pgfqpoint{0.630302in}{0.193945in}}{\pgfqpoint{0.635833in}{0.180590in}}%
\pgfpathcurveto{\pgfqpoint{0.635833in}{0.166667in}}{\pgfqpoint{0.635833in}{0.152744in}}{\pgfqpoint{0.630302in}{0.139389in}}%
\pgfpathcurveto{\pgfqpoint{0.620456in}{0.129544in}}{\pgfqpoint{0.610611in}{0.119698in}}{\pgfqpoint{0.597256in}{0.114167in}}%
\pgfpathclose%
\pgfpathmoveto{\pgfqpoint{0.750000in}{0.108333in}}%
\pgfpathcurveto{\pgfqpoint{0.765470in}{0.108333in}}{\pgfqpoint{0.780309in}{0.114480in}}{\pgfqpoint{0.791248in}{0.125419in}}%
\pgfpathcurveto{\pgfqpoint{0.802187in}{0.136358in}}{\pgfqpoint{0.808333in}{0.151196in}}{\pgfqpoint{0.808333in}{0.166667in}}%
\pgfpathcurveto{\pgfqpoint{0.808333in}{0.182137in}}{\pgfqpoint{0.802187in}{0.196975in}}{\pgfqpoint{0.791248in}{0.207915in}}%
\pgfpathcurveto{\pgfqpoint{0.780309in}{0.218854in}}{\pgfqpoint{0.765470in}{0.225000in}}{\pgfqpoint{0.750000in}{0.225000in}}%
\pgfpathcurveto{\pgfqpoint{0.734530in}{0.225000in}}{\pgfqpoint{0.719691in}{0.218854in}}{\pgfqpoint{0.708752in}{0.207915in}}%
\pgfpathcurveto{\pgfqpoint{0.697813in}{0.196975in}}{\pgfqpoint{0.691667in}{0.182137in}}{\pgfqpoint{0.691667in}{0.166667in}}%
\pgfpathcurveto{\pgfqpoint{0.691667in}{0.151196in}}{\pgfqpoint{0.697813in}{0.136358in}}{\pgfqpoint{0.708752in}{0.125419in}}%
\pgfpathcurveto{\pgfqpoint{0.719691in}{0.114480in}}{\pgfqpoint{0.734530in}{0.108333in}}{\pgfqpoint{0.750000in}{0.108333in}}%
\pgfpathclose%
\pgfpathmoveto{\pgfqpoint{0.750000in}{0.114167in}}%
\pgfpathcurveto{\pgfqpoint{0.750000in}{0.114167in}}{\pgfqpoint{0.736077in}{0.114167in}}{\pgfqpoint{0.722722in}{0.119698in}}%
\pgfpathcurveto{\pgfqpoint{0.712877in}{0.129544in}}{\pgfqpoint{0.703032in}{0.139389in}}{\pgfqpoint{0.697500in}{0.152744in}}%
\pgfpathcurveto{\pgfqpoint{0.697500in}{0.166667in}}{\pgfqpoint{0.697500in}{0.180590in}}{\pgfqpoint{0.703032in}{0.193945in}}%
\pgfpathcurveto{\pgfqpoint{0.712877in}{0.203790in}}{\pgfqpoint{0.722722in}{0.213635in}}{\pgfqpoint{0.736077in}{0.219167in}}%
\pgfpathcurveto{\pgfqpoint{0.750000in}{0.219167in}}{\pgfqpoint{0.763923in}{0.219167in}}{\pgfqpoint{0.777278in}{0.213635in}}%
\pgfpathcurveto{\pgfqpoint{0.787123in}{0.203790in}}{\pgfqpoint{0.796968in}{0.193945in}}{\pgfqpoint{0.802500in}{0.180590in}}%
\pgfpathcurveto{\pgfqpoint{0.802500in}{0.166667in}}{\pgfqpoint{0.802500in}{0.152744in}}{\pgfqpoint{0.796968in}{0.139389in}}%
\pgfpathcurveto{\pgfqpoint{0.787123in}{0.129544in}}{\pgfqpoint{0.777278in}{0.119698in}}{\pgfqpoint{0.763923in}{0.114167in}}%
\pgfpathclose%
\pgfpathmoveto{\pgfqpoint{0.916667in}{0.108333in}}%
\pgfpathcurveto{\pgfqpoint{0.932137in}{0.108333in}}{\pgfqpoint{0.946975in}{0.114480in}}{\pgfqpoint{0.957915in}{0.125419in}}%
\pgfpathcurveto{\pgfqpoint{0.968854in}{0.136358in}}{\pgfqpoint{0.975000in}{0.151196in}}{\pgfqpoint{0.975000in}{0.166667in}}%
\pgfpathcurveto{\pgfqpoint{0.975000in}{0.182137in}}{\pgfqpoint{0.968854in}{0.196975in}}{\pgfqpoint{0.957915in}{0.207915in}}%
\pgfpathcurveto{\pgfqpoint{0.946975in}{0.218854in}}{\pgfqpoint{0.932137in}{0.225000in}}{\pgfqpoint{0.916667in}{0.225000in}}%
\pgfpathcurveto{\pgfqpoint{0.901196in}{0.225000in}}{\pgfqpoint{0.886358in}{0.218854in}}{\pgfqpoint{0.875419in}{0.207915in}}%
\pgfpathcurveto{\pgfqpoint{0.864480in}{0.196975in}}{\pgfqpoint{0.858333in}{0.182137in}}{\pgfqpoint{0.858333in}{0.166667in}}%
\pgfpathcurveto{\pgfqpoint{0.858333in}{0.151196in}}{\pgfqpoint{0.864480in}{0.136358in}}{\pgfqpoint{0.875419in}{0.125419in}}%
\pgfpathcurveto{\pgfqpoint{0.886358in}{0.114480in}}{\pgfqpoint{0.901196in}{0.108333in}}{\pgfqpoint{0.916667in}{0.108333in}}%
\pgfpathclose%
\pgfpathmoveto{\pgfqpoint{0.916667in}{0.114167in}}%
\pgfpathcurveto{\pgfqpoint{0.916667in}{0.114167in}}{\pgfqpoint{0.902744in}{0.114167in}}{\pgfqpoint{0.889389in}{0.119698in}}%
\pgfpathcurveto{\pgfqpoint{0.879544in}{0.129544in}}{\pgfqpoint{0.869698in}{0.139389in}}{\pgfqpoint{0.864167in}{0.152744in}}%
\pgfpathcurveto{\pgfqpoint{0.864167in}{0.166667in}}{\pgfqpoint{0.864167in}{0.180590in}}{\pgfqpoint{0.869698in}{0.193945in}}%
\pgfpathcurveto{\pgfqpoint{0.879544in}{0.203790in}}{\pgfqpoint{0.889389in}{0.213635in}}{\pgfqpoint{0.902744in}{0.219167in}}%
\pgfpathcurveto{\pgfqpoint{0.916667in}{0.219167in}}{\pgfqpoint{0.930590in}{0.219167in}}{\pgfqpoint{0.943945in}{0.213635in}}%
\pgfpathcurveto{\pgfqpoint{0.953790in}{0.203790in}}{\pgfqpoint{0.963635in}{0.193945in}}{\pgfqpoint{0.969167in}{0.180590in}}%
\pgfpathcurveto{\pgfqpoint{0.969167in}{0.166667in}}{\pgfqpoint{0.969167in}{0.152744in}}{\pgfqpoint{0.963635in}{0.139389in}}%
\pgfpathcurveto{\pgfqpoint{0.953790in}{0.129544in}}{\pgfqpoint{0.943945in}{0.119698in}}{\pgfqpoint{0.930590in}{0.114167in}}%
\pgfpathclose%
\pgfpathmoveto{\pgfqpoint{0.000000in}{0.275000in}}%
\pgfpathcurveto{\pgfqpoint{0.015470in}{0.275000in}}{\pgfqpoint{0.030309in}{0.281146in}}{\pgfqpoint{0.041248in}{0.292085in}}%
\pgfpathcurveto{\pgfqpoint{0.052187in}{0.303025in}}{\pgfqpoint{0.058333in}{0.317863in}}{\pgfqpoint{0.058333in}{0.333333in}}%
\pgfpathcurveto{\pgfqpoint{0.058333in}{0.348804in}}{\pgfqpoint{0.052187in}{0.363642in}}{\pgfqpoint{0.041248in}{0.374581in}}%
\pgfpathcurveto{\pgfqpoint{0.030309in}{0.385520in}}{\pgfqpoint{0.015470in}{0.391667in}}{\pgfqpoint{0.000000in}{0.391667in}}%
\pgfpathcurveto{\pgfqpoint{-0.015470in}{0.391667in}}{\pgfqpoint{-0.030309in}{0.385520in}}{\pgfqpoint{-0.041248in}{0.374581in}}%
\pgfpathcurveto{\pgfqpoint{-0.052187in}{0.363642in}}{\pgfqpoint{-0.058333in}{0.348804in}}{\pgfqpoint{-0.058333in}{0.333333in}}%
\pgfpathcurveto{\pgfqpoint{-0.058333in}{0.317863in}}{\pgfqpoint{-0.052187in}{0.303025in}}{\pgfqpoint{-0.041248in}{0.292085in}}%
\pgfpathcurveto{\pgfqpoint{-0.030309in}{0.281146in}}{\pgfqpoint{-0.015470in}{0.275000in}}{\pgfqpoint{0.000000in}{0.275000in}}%
\pgfpathclose%
\pgfpathmoveto{\pgfqpoint{0.000000in}{0.280833in}}%
\pgfpathcurveto{\pgfqpoint{0.000000in}{0.280833in}}{\pgfqpoint{-0.013923in}{0.280833in}}{\pgfqpoint{-0.027278in}{0.286365in}}%
\pgfpathcurveto{\pgfqpoint{-0.037123in}{0.296210in}}{\pgfqpoint{-0.046968in}{0.306055in}}{\pgfqpoint{-0.052500in}{0.319410in}}%
\pgfpathcurveto{\pgfqpoint{-0.052500in}{0.333333in}}{\pgfqpoint{-0.052500in}{0.347256in}}{\pgfqpoint{-0.046968in}{0.360611in}}%
\pgfpathcurveto{\pgfqpoint{-0.037123in}{0.370456in}}{\pgfqpoint{-0.027278in}{0.380302in}}{\pgfqpoint{-0.013923in}{0.385833in}}%
\pgfpathcurveto{\pgfqpoint{0.000000in}{0.385833in}}{\pgfqpoint{0.013923in}{0.385833in}}{\pgfqpoint{0.027278in}{0.380302in}}%
\pgfpathcurveto{\pgfqpoint{0.037123in}{0.370456in}}{\pgfqpoint{0.046968in}{0.360611in}}{\pgfqpoint{0.052500in}{0.347256in}}%
\pgfpathcurveto{\pgfqpoint{0.052500in}{0.333333in}}{\pgfqpoint{0.052500in}{0.319410in}}{\pgfqpoint{0.046968in}{0.306055in}}%
\pgfpathcurveto{\pgfqpoint{0.037123in}{0.296210in}}{\pgfqpoint{0.027278in}{0.286365in}}{\pgfqpoint{0.013923in}{0.280833in}}%
\pgfpathclose%
\pgfpathmoveto{\pgfqpoint{0.166667in}{0.275000in}}%
\pgfpathcurveto{\pgfqpoint{0.182137in}{0.275000in}}{\pgfqpoint{0.196975in}{0.281146in}}{\pgfqpoint{0.207915in}{0.292085in}}%
\pgfpathcurveto{\pgfqpoint{0.218854in}{0.303025in}}{\pgfqpoint{0.225000in}{0.317863in}}{\pgfqpoint{0.225000in}{0.333333in}}%
\pgfpathcurveto{\pgfqpoint{0.225000in}{0.348804in}}{\pgfqpoint{0.218854in}{0.363642in}}{\pgfqpoint{0.207915in}{0.374581in}}%
\pgfpathcurveto{\pgfqpoint{0.196975in}{0.385520in}}{\pgfqpoint{0.182137in}{0.391667in}}{\pgfqpoint{0.166667in}{0.391667in}}%
\pgfpathcurveto{\pgfqpoint{0.151196in}{0.391667in}}{\pgfqpoint{0.136358in}{0.385520in}}{\pgfqpoint{0.125419in}{0.374581in}}%
\pgfpathcurveto{\pgfqpoint{0.114480in}{0.363642in}}{\pgfqpoint{0.108333in}{0.348804in}}{\pgfqpoint{0.108333in}{0.333333in}}%
\pgfpathcurveto{\pgfqpoint{0.108333in}{0.317863in}}{\pgfqpoint{0.114480in}{0.303025in}}{\pgfqpoint{0.125419in}{0.292085in}}%
\pgfpathcurveto{\pgfqpoint{0.136358in}{0.281146in}}{\pgfqpoint{0.151196in}{0.275000in}}{\pgfqpoint{0.166667in}{0.275000in}}%
\pgfpathclose%
\pgfpathmoveto{\pgfqpoint{0.166667in}{0.280833in}}%
\pgfpathcurveto{\pgfqpoint{0.166667in}{0.280833in}}{\pgfqpoint{0.152744in}{0.280833in}}{\pgfqpoint{0.139389in}{0.286365in}}%
\pgfpathcurveto{\pgfqpoint{0.129544in}{0.296210in}}{\pgfqpoint{0.119698in}{0.306055in}}{\pgfqpoint{0.114167in}{0.319410in}}%
\pgfpathcurveto{\pgfqpoint{0.114167in}{0.333333in}}{\pgfqpoint{0.114167in}{0.347256in}}{\pgfqpoint{0.119698in}{0.360611in}}%
\pgfpathcurveto{\pgfqpoint{0.129544in}{0.370456in}}{\pgfqpoint{0.139389in}{0.380302in}}{\pgfqpoint{0.152744in}{0.385833in}}%
\pgfpathcurveto{\pgfqpoint{0.166667in}{0.385833in}}{\pgfqpoint{0.180590in}{0.385833in}}{\pgfqpoint{0.193945in}{0.380302in}}%
\pgfpathcurveto{\pgfqpoint{0.203790in}{0.370456in}}{\pgfqpoint{0.213635in}{0.360611in}}{\pgfqpoint{0.219167in}{0.347256in}}%
\pgfpathcurveto{\pgfqpoint{0.219167in}{0.333333in}}{\pgfqpoint{0.219167in}{0.319410in}}{\pgfqpoint{0.213635in}{0.306055in}}%
\pgfpathcurveto{\pgfqpoint{0.203790in}{0.296210in}}{\pgfqpoint{0.193945in}{0.286365in}}{\pgfqpoint{0.180590in}{0.280833in}}%
\pgfpathclose%
\pgfpathmoveto{\pgfqpoint{0.333333in}{0.275000in}}%
\pgfpathcurveto{\pgfqpoint{0.348804in}{0.275000in}}{\pgfqpoint{0.363642in}{0.281146in}}{\pgfqpoint{0.374581in}{0.292085in}}%
\pgfpathcurveto{\pgfqpoint{0.385520in}{0.303025in}}{\pgfqpoint{0.391667in}{0.317863in}}{\pgfqpoint{0.391667in}{0.333333in}}%
\pgfpathcurveto{\pgfqpoint{0.391667in}{0.348804in}}{\pgfqpoint{0.385520in}{0.363642in}}{\pgfqpoint{0.374581in}{0.374581in}}%
\pgfpathcurveto{\pgfqpoint{0.363642in}{0.385520in}}{\pgfqpoint{0.348804in}{0.391667in}}{\pgfqpoint{0.333333in}{0.391667in}}%
\pgfpathcurveto{\pgfqpoint{0.317863in}{0.391667in}}{\pgfqpoint{0.303025in}{0.385520in}}{\pgfqpoint{0.292085in}{0.374581in}}%
\pgfpathcurveto{\pgfqpoint{0.281146in}{0.363642in}}{\pgfqpoint{0.275000in}{0.348804in}}{\pgfqpoint{0.275000in}{0.333333in}}%
\pgfpathcurveto{\pgfqpoint{0.275000in}{0.317863in}}{\pgfqpoint{0.281146in}{0.303025in}}{\pgfqpoint{0.292085in}{0.292085in}}%
\pgfpathcurveto{\pgfqpoint{0.303025in}{0.281146in}}{\pgfqpoint{0.317863in}{0.275000in}}{\pgfqpoint{0.333333in}{0.275000in}}%
\pgfpathclose%
\pgfpathmoveto{\pgfqpoint{0.333333in}{0.280833in}}%
\pgfpathcurveto{\pgfqpoint{0.333333in}{0.280833in}}{\pgfqpoint{0.319410in}{0.280833in}}{\pgfqpoint{0.306055in}{0.286365in}}%
\pgfpathcurveto{\pgfqpoint{0.296210in}{0.296210in}}{\pgfqpoint{0.286365in}{0.306055in}}{\pgfqpoint{0.280833in}{0.319410in}}%
\pgfpathcurveto{\pgfqpoint{0.280833in}{0.333333in}}{\pgfqpoint{0.280833in}{0.347256in}}{\pgfqpoint{0.286365in}{0.360611in}}%
\pgfpathcurveto{\pgfqpoint{0.296210in}{0.370456in}}{\pgfqpoint{0.306055in}{0.380302in}}{\pgfqpoint{0.319410in}{0.385833in}}%
\pgfpathcurveto{\pgfqpoint{0.333333in}{0.385833in}}{\pgfqpoint{0.347256in}{0.385833in}}{\pgfqpoint{0.360611in}{0.380302in}}%
\pgfpathcurveto{\pgfqpoint{0.370456in}{0.370456in}}{\pgfqpoint{0.380302in}{0.360611in}}{\pgfqpoint{0.385833in}{0.347256in}}%
\pgfpathcurveto{\pgfqpoint{0.385833in}{0.333333in}}{\pgfqpoint{0.385833in}{0.319410in}}{\pgfqpoint{0.380302in}{0.306055in}}%
\pgfpathcurveto{\pgfqpoint{0.370456in}{0.296210in}}{\pgfqpoint{0.360611in}{0.286365in}}{\pgfqpoint{0.347256in}{0.280833in}}%
\pgfpathclose%
\pgfpathmoveto{\pgfqpoint{0.500000in}{0.275000in}}%
\pgfpathcurveto{\pgfqpoint{0.515470in}{0.275000in}}{\pgfqpoint{0.530309in}{0.281146in}}{\pgfqpoint{0.541248in}{0.292085in}}%
\pgfpathcurveto{\pgfqpoint{0.552187in}{0.303025in}}{\pgfqpoint{0.558333in}{0.317863in}}{\pgfqpoint{0.558333in}{0.333333in}}%
\pgfpathcurveto{\pgfqpoint{0.558333in}{0.348804in}}{\pgfqpoint{0.552187in}{0.363642in}}{\pgfqpoint{0.541248in}{0.374581in}}%
\pgfpathcurveto{\pgfqpoint{0.530309in}{0.385520in}}{\pgfqpoint{0.515470in}{0.391667in}}{\pgfqpoint{0.500000in}{0.391667in}}%
\pgfpathcurveto{\pgfqpoint{0.484530in}{0.391667in}}{\pgfqpoint{0.469691in}{0.385520in}}{\pgfqpoint{0.458752in}{0.374581in}}%
\pgfpathcurveto{\pgfqpoint{0.447813in}{0.363642in}}{\pgfqpoint{0.441667in}{0.348804in}}{\pgfqpoint{0.441667in}{0.333333in}}%
\pgfpathcurveto{\pgfqpoint{0.441667in}{0.317863in}}{\pgfqpoint{0.447813in}{0.303025in}}{\pgfqpoint{0.458752in}{0.292085in}}%
\pgfpathcurveto{\pgfqpoint{0.469691in}{0.281146in}}{\pgfqpoint{0.484530in}{0.275000in}}{\pgfqpoint{0.500000in}{0.275000in}}%
\pgfpathclose%
\pgfpathmoveto{\pgfqpoint{0.500000in}{0.280833in}}%
\pgfpathcurveto{\pgfqpoint{0.500000in}{0.280833in}}{\pgfqpoint{0.486077in}{0.280833in}}{\pgfqpoint{0.472722in}{0.286365in}}%
\pgfpathcurveto{\pgfqpoint{0.462877in}{0.296210in}}{\pgfqpoint{0.453032in}{0.306055in}}{\pgfqpoint{0.447500in}{0.319410in}}%
\pgfpathcurveto{\pgfqpoint{0.447500in}{0.333333in}}{\pgfqpoint{0.447500in}{0.347256in}}{\pgfqpoint{0.453032in}{0.360611in}}%
\pgfpathcurveto{\pgfqpoint{0.462877in}{0.370456in}}{\pgfqpoint{0.472722in}{0.380302in}}{\pgfqpoint{0.486077in}{0.385833in}}%
\pgfpathcurveto{\pgfqpoint{0.500000in}{0.385833in}}{\pgfqpoint{0.513923in}{0.385833in}}{\pgfqpoint{0.527278in}{0.380302in}}%
\pgfpathcurveto{\pgfqpoint{0.537123in}{0.370456in}}{\pgfqpoint{0.546968in}{0.360611in}}{\pgfqpoint{0.552500in}{0.347256in}}%
\pgfpathcurveto{\pgfqpoint{0.552500in}{0.333333in}}{\pgfqpoint{0.552500in}{0.319410in}}{\pgfqpoint{0.546968in}{0.306055in}}%
\pgfpathcurveto{\pgfqpoint{0.537123in}{0.296210in}}{\pgfqpoint{0.527278in}{0.286365in}}{\pgfqpoint{0.513923in}{0.280833in}}%
\pgfpathclose%
\pgfpathmoveto{\pgfqpoint{0.666667in}{0.275000in}}%
\pgfpathcurveto{\pgfqpoint{0.682137in}{0.275000in}}{\pgfqpoint{0.696975in}{0.281146in}}{\pgfqpoint{0.707915in}{0.292085in}}%
\pgfpathcurveto{\pgfqpoint{0.718854in}{0.303025in}}{\pgfqpoint{0.725000in}{0.317863in}}{\pgfqpoint{0.725000in}{0.333333in}}%
\pgfpathcurveto{\pgfqpoint{0.725000in}{0.348804in}}{\pgfqpoint{0.718854in}{0.363642in}}{\pgfqpoint{0.707915in}{0.374581in}}%
\pgfpathcurveto{\pgfqpoint{0.696975in}{0.385520in}}{\pgfqpoint{0.682137in}{0.391667in}}{\pgfqpoint{0.666667in}{0.391667in}}%
\pgfpathcurveto{\pgfqpoint{0.651196in}{0.391667in}}{\pgfqpoint{0.636358in}{0.385520in}}{\pgfqpoint{0.625419in}{0.374581in}}%
\pgfpathcurveto{\pgfqpoint{0.614480in}{0.363642in}}{\pgfqpoint{0.608333in}{0.348804in}}{\pgfqpoint{0.608333in}{0.333333in}}%
\pgfpathcurveto{\pgfqpoint{0.608333in}{0.317863in}}{\pgfqpoint{0.614480in}{0.303025in}}{\pgfqpoint{0.625419in}{0.292085in}}%
\pgfpathcurveto{\pgfqpoint{0.636358in}{0.281146in}}{\pgfqpoint{0.651196in}{0.275000in}}{\pgfqpoint{0.666667in}{0.275000in}}%
\pgfpathclose%
\pgfpathmoveto{\pgfqpoint{0.666667in}{0.280833in}}%
\pgfpathcurveto{\pgfqpoint{0.666667in}{0.280833in}}{\pgfqpoint{0.652744in}{0.280833in}}{\pgfqpoint{0.639389in}{0.286365in}}%
\pgfpathcurveto{\pgfqpoint{0.629544in}{0.296210in}}{\pgfqpoint{0.619698in}{0.306055in}}{\pgfqpoint{0.614167in}{0.319410in}}%
\pgfpathcurveto{\pgfqpoint{0.614167in}{0.333333in}}{\pgfqpoint{0.614167in}{0.347256in}}{\pgfqpoint{0.619698in}{0.360611in}}%
\pgfpathcurveto{\pgfqpoint{0.629544in}{0.370456in}}{\pgfqpoint{0.639389in}{0.380302in}}{\pgfqpoint{0.652744in}{0.385833in}}%
\pgfpathcurveto{\pgfqpoint{0.666667in}{0.385833in}}{\pgfqpoint{0.680590in}{0.385833in}}{\pgfqpoint{0.693945in}{0.380302in}}%
\pgfpathcurveto{\pgfqpoint{0.703790in}{0.370456in}}{\pgfqpoint{0.713635in}{0.360611in}}{\pgfqpoint{0.719167in}{0.347256in}}%
\pgfpathcurveto{\pgfqpoint{0.719167in}{0.333333in}}{\pgfqpoint{0.719167in}{0.319410in}}{\pgfqpoint{0.713635in}{0.306055in}}%
\pgfpathcurveto{\pgfqpoint{0.703790in}{0.296210in}}{\pgfqpoint{0.693945in}{0.286365in}}{\pgfqpoint{0.680590in}{0.280833in}}%
\pgfpathclose%
\pgfpathmoveto{\pgfqpoint{0.833333in}{0.275000in}}%
\pgfpathcurveto{\pgfqpoint{0.848804in}{0.275000in}}{\pgfqpoint{0.863642in}{0.281146in}}{\pgfqpoint{0.874581in}{0.292085in}}%
\pgfpathcurveto{\pgfqpoint{0.885520in}{0.303025in}}{\pgfqpoint{0.891667in}{0.317863in}}{\pgfqpoint{0.891667in}{0.333333in}}%
\pgfpathcurveto{\pgfqpoint{0.891667in}{0.348804in}}{\pgfqpoint{0.885520in}{0.363642in}}{\pgfqpoint{0.874581in}{0.374581in}}%
\pgfpathcurveto{\pgfqpoint{0.863642in}{0.385520in}}{\pgfqpoint{0.848804in}{0.391667in}}{\pgfqpoint{0.833333in}{0.391667in}}%
\pgfpathcurveto{\pgfqpoint{0.817863in}{0.391667in}}{\pgfqpoint{0.803025in}{0.385520in}}{\pgfqpoint{0.792085in}{0.374581in}}%
\pgfpathcurveto{\pgfqpoint{0.781146in}{0.363642in}}{\pgfqpoint{0.775000in}{0.348804in}}{\pgfqpoint{0.775000in}{0.333333in}}%
\pgfpathcurveto{\pgfqpoint{0.775000in}{0.317863in}}{\pgfqpoint{0.781146in}{0.303025in}}{\pgfqpoint{0.792085in}{0.292085in}}%
\pgfpathcurveto{\pgfqpoint{0.803025in}{0.281146in}}{\pgfqpoint{0.817863in}{0.275000in}}{\pgfqpoint{0.833333in}{0.275000in}}%
\pgfpathclose%
\pgfpathmoveto{\pgfqpoint{0.833333in}{0.280833in}}%
\pgfpathcurveto{\pgfqpoint{0.833333in}{0.280833in}}{\pgfqpoint{0.819410in}{0.280833in}}{\pgfqpoint{0.806055in}{0.286365in}}%
\pgfpathcurveto{\pgfqpoint{0.796210in}{0.296210in}}{\pgfqpoint{0.786365in}{0.306055in}}{\pgfqpoint{0.780833in}{0.319410in}}%
\pgfpathcurveto{\pgfqpoint{0.780833in}{0.333333in}}{\pgfqpoint{0.780833in}{0.347256in}}{\pgfqpoint{0.786365in}{0.360611in}}%
\pgfpathcurveto{\pgfqpoint{0.796210in}{0.370456in}}{\pgfqpoint{0.806055in}{0.380302in}}{\pgfqpoint{0.819410in}{0.385833in}}%
\pgfpathcurveto{\pgfqpoint{0.833333in}{0.385833in}}{\pgfqpoint{0.847256in}{0.385833in}}{\pgfqpoint{0.860611in}{0.380302in}}%
\pgfpathcurveto{\pgfqpoint{0.870456in}{0.370456in}}{\pgfqpoint{0.880302in}{0.360611in}}{\pgfqpoint{0.885833in}{0.347256in}}%
\pgfpathcurveto{\pgfqpoint{0.885833in}{0.333333in}}{\pgfqpoint{0.885833in}{0.319410in}}{\pgfqpoint{0.880302in}{0.306055in}}%
\pgfpathcurveto{\pgfqpoint{0.870456in}{0.296210in}}{\pgfqpoint{0.860611in}{0.286365in}}{\pgfqpoint{0.847256in}{0.280833in}}%
\pgfpathclose%
\pgfpathmoveto{\pgfqpoint{1.000000in}{0.275000in}}%
\pgfpathcurveto{\pgfqpoint{1.015470in}{0.275000in}}{\pgfqpoint{1.030309in}{0.281146in}}{\pgfqpoint{1.041248in}{0.292085in}}%
\pgfpathcurveto{\pgfqpoint{1.052187in}{0.303025in}}{\pgfqpoint{1.058333in}{0.317863in}}{\pgfqpoint{1.058333in}{0.333333in}}%
\pgfpathcurveto{\pgfqpoint{1.058333in}{0.348804in}}{\pgfqpoint{1.052187in}{0.363642in}}{\pgfqpoint{1.041248in}{0.374581in}}%
\pgfpathcurveto{\pgfqpoint{1.030309in}{0.385520in}}{\pgfqpoint{1.015470in}{0.391667in}}{\pgfqpoint{1.000000in}{0.391667in}}%
\pgfpathcurveto{\pgfqpoint{0.984530in}{0.391667in}}{\pgfqpoint{0.969691in}{0.385520in}}{\pgfqpoint{0.958752in}{0.374581in}}%
\pgfpathcurveto{\pgfqpoint{0.947813in}{0.363642in}}{\pgfqpoint{0.941667in}{0.348804in}}{\pgfqpoint{0.941667in}{0.333333in}}%
\pgfpathcurveto{\pgfqpoint{0.941667in}{0.317863in}}{\pgfqpoint{0.947813in}{0.303025in}}{\pgfqpoint{0.958752in}{0.292085in}}%
\pgfpathcurveto{\pgfqpoint{0.969691in}{0.281146in}}{\pgfqpoint{0.984530in}{0.275000in}}{\pgfqpoint{1.000000in}{0.275000in}}%
\pgfpathclose%
\pgfpathmoveto{\pgfqpoint{1.000000in}{0.280833in}}%
\pgfpathcurveto{\pgfqpoint{1.000000in}{0.280833in}}{\pgfqpoint{0.986077in}{0.280833in}}{\pgfqpoint{0.972722in}{0.286365in}}%
\pgfpathcurveto{\pgfqpoint{0.962877in}{0.296210in}}{\pgfqpoint{0.953032in}{0.306055in}}{\pgfqpoint{0.947500in}{0.319410in}}%
\pgfpathcurveto{\pgfqpoint{0.947500in}{0.333333in}}{\pgfqpoint{0.947500in}{0.347256in}}{\pgfqpoint{0.953032in}{0.360611in}}%
\pgfpathcurveto{\pgfqpoint{0.962877in}{0.370456in}}{\pgfqpoint{0.972722in}{0.380302in}}{\pgfqpoint{0.986077in}{0.385833in}}%
\pgfpathcurveto{\pgfqpoint{1.000000in}{0.385833in}}{\pgfqpoint{1.013923in}{0.385833in}}{\pgfqpoint{1.027278in}{0.380302in}}%
\pgfpathcurveto{\pgfqpoint{1.037123in}{0.370456in}}{\pgfqpoint{1.046968in}{0.360611in}}{\pgfqpoint{1.052500in}{0.347256in}}%
\pgfpathcurveto{\pgfqpoint{1.052500in}{0.333333in}}{\pgfqpoint{1.052500in}{0.319410in}}{\pgfqpoint{1.046968in}{0.306055in}}%
\pgfpathcurveto{\pgfqpoint{1.037123in}{0.296210in}}{\pgfqpoint{1.027278in}{0.286365in}}{\pgfqpoint{1.013923in}{0.280833in}}%
\pgfpathclose%
\pgfpathmoveto{\pgfqpoint{0.083333in}{0.441667in}}%
\pgfpathcurveto{\pgfqpoint{0.098804in}{0.441667in}}{\pgfqpoint{0.113642in}{0.447813in}}{\pgfqpoint{0.124581in}{0.458752in}}%
\pgfpathcurveto{\pgfqpoint{0.135520in}{0.469691in}}{\pgfqpoint{0.141667in}{0.484530in}}{\pgfqpoint{0.141667in}{0.500000in}}%
\pgfpathcurveto{\pgfqpoint{0.141667in}{0.515470in}}{\pgfqpoint{0.135520in}{0.530309in}}{\pgfqpoint{0.124581in}{0.541248in}}%
\pgfpathcurveto{\pgfqpoint{0.113642in}{0.552187in}}{\pgfqpoint{0.098804in}{0.558333in}}{\pgfqpoint{0.083333in}{0.558333in}}%
\pgfpathcurveto{\pgfqpoint{0.067863in}{0.558333in}}{\pgfqpoint{0.053025in}{0.552187in}}{\pgfqpoint{0.042085in}{0.541248in}}%
\pgfpathcurveto{\pgfqpoint{0.031146in}{0.530309in}}{\pgfqpoint{0.025000in}{0.515470in}}{\pgfqpoint{0.025000in}{0.500000in}}%
\pgfpathcurveto{\pgfqpoint{0.025000in}{0.484530in}}{\pgfqpoint{0.031146in}{0.469691in}}{\pgfqpoint{0.042085in}{0.458752in}}%
\pgfpathcurveto{\pgfqpoint{0.053025in}{0.447813in}}{\pgfqpoint{0.067863in}{0.441667in}}{\pgfqpoint{0.083333in}{0.441667in}}%
\pgfpathclose%
\pgfpathmoveto{\pgfqpoint{0.083333in}{0.447500in}}%
\pgfpathcurveto{\pgfqpoint{0.083333in}{0.447500in}}{\pgfqpoint{0.069410in}{0.447500in}}{\pgfqpoint{0.056055in}{0.453032in}}%
\pgfpathcurveto{\pgfqpoint{0.046210in}{0.462877in}}{\pgfqpoint{0.036365in}{0.472722in}}{\pgfqpoint{0.030833in}{0.486077in}}%
\pgfpathcurveto{\pgfqpoint{0.030833in}{0.500000in}}{\pgfqpoint{0.030833in}{0.513923in}}{\pgfqpoint{0.036365in}{0.527278in}}%
\pgfpathcurveto{\pgfqpoint{0.046210in}{0.537123in}}{\pgfqpoint{0.056055in}{0.546968in}}{\pgfqpoint{0.069410in}{0.552500in}}%
\pgfpathcurveto{\pgfqpoint{0.083333in}{0.552500in}}{\pgfqpoint{0.097256in}{0.552500in}}{\pgfqpoint{0.110611in}{0.546968in}}%
\pgfpathcurveto{\pgfqpoint{0.120456in}{0.537123in}}{\pgfqpoint{0.130302in}{0.527278in}}{\pgfqpoint{0.135833in}{0.513923in}}%
\pgfpathcurveto{\pgfqpoint{0.135833in}{0.500000in}}{\pgfqpoint{0.135833in}{0.486077in}}{\pgfqpoint{0.130302in}{0.472722in}}%
\pgfpathcurveto{\pgfqpoint{0.120456in}{0.462877in}}{\pgfqpoint{0.110611in}{0.453032in}}{\pgfqpoint{0.097256in}{0.447500in}}%
\pgfpathclose%
\pgfpathmoveto{\pgfqpoint{0.250000in}{0.441667in}}%
\pgfpathcurveto{\pgfqpoint{0.265470in}{0.441667in}}{\pgfqpoint{0.280309in}{0.447813in}}{\pgfqpoint{0.291248in}{0.458752in}}%
\pgfpathcurveto{\pgfqpoint{0.302187in}{0.469691in}}{\pgfqpoint{0.308333in}{0.484530in}}{\pgfqpoint{0.308333in}{0.500000in}}%
\pgfpathcurveto{\pgfqpoint{0.308333in}{0.515470in}}{\pgfqpoint{0.302187in}{0.530309in}}{\pgfqpoint{0.291248in}{0.541248in}}%
\pgfpathcurveto{\pgfqpoint{0.280309in}{0.552187in}}{\pgfqpoint{0.265470in}{0.558333in}}{\pgfqpoint{0.250000in}{0.558333in}}%
\pgfpathcurveto{\pgfqpoint{0.234530in}{0.558333in}}{\pgfqpoint{0.219691in}{0.552187in}}{\pgfqpoint{0.208752in}{0.541248in}}%
\pgfpathcurveto{\pgfqpoint{0.197813in}{0.530309in}}{\pgfqpoint{0.191667in}{0.515470in}}{\pgfqpoint{0.191667in}{0.500000in}}%
\pgfpathcurveto{\pgfqpoint{0.191667in}{0.484530in}}{\pgfqpoint{0.197813in}{0.469691in}}{\pgfqpoint{0.208752in}{0.458752in}}%
\pgfpathcurveto{\pgfqpoint{0.219691in}{0.447813in}}{\pgfqpoint{0.234530in}{0.441667in}}{\pgfqpoint{0.250000in}{0.441667in}}%
\pgfpathclose%
\pgfpathmoveto{\pgfqpoint{0.250000in}{0.447500in}}%
\pgfpathcurveto{\pgfqpoint{0.250000in}{0.447500in}}{\pgfqpoint{0.236077in}{0.447500in}}{\pgfqpoint{0.222722in}{0.453032in}}%
\pgfpathcurveto{\pgfqpoint{0.212877in}{0.462877in}}{\pgfqpoint{0.203032in}{0.472722in}}{\pgfqpoint{0.197500in}{0.486077in}}%
\pgfpathcurveto{\pgfqpoint{0.197500in}{0.500000in}}{\pgfqpoint{0.197500in}{0.513923in}}{\pgfqpoint{0.203032in}{0.527278in}}%
\pgfpathcurveto{\pgfqpoint{0.212877in}{0.537123in}}{\pgfqpoint{0.222722in}{0.546968in}}{\pgfqpoint{0.236077in}{0.552500in}}%
\pgfpathcurveto{\pgfqpoint{0.250000in}{0.552500in}}{\pgfqpoint{0.263923in}{0.552500in}}{\pgfqpoint{0.277278in}{0.546968in}}%
\pgfpathcurveto{\pgfqpoint{0.287123in}{0.537123in}}{\pgfqpoint{0.296968in}{0.527278in}}{\pgfqpoint{0.302500in}{0.513923in}}%
\pgfpathcurveto{\pgfqpoint{0.302500in}{0.500000in}}{\pgfqpoint{0.302500in}{0.486077in}}{\pgfqpoint{0.296968in}{0.472722in}}%
\pgfpathcurveto{\pgfqpoint{0.287123in}{0.462877in}}{\pgfqpoint{0.277278in}{0.453032in}}{\pgfqpoint{0.263923in}{0.447500in}}%
\pgfpathclose%
\pgfpathmoveto{\pgfqpoint{0.416667in}{0.441667in}}%
\pgfpathcurveto{\pgfqpoint{0.432137in}{0.441667in}}{\pgfqpoint{0.446975in}{0.447813in}}{\pgfqpoint{0.457915in}{0.458752in}}%
\pgfpathcurveto{\pgfqpoint{0.468854in}{0.469691in}}{\pgfqpoint{0.475000in}{0.484530in}}{\pgfqpoint{0.475000in}{0.500000in}}%
\pgfpathcurveto{\pgfqpoint{0.475000in}{0.515470in}}{\pgfqpoint{0.468854in}{0.530309in}}{\pgfqpoint{0.457915in}{0.541248in}}%
\pgfpathcurveto{\pgfqpoint{0.446975in}{0.552187in}}{\pgfqpoint{0.432137in}{0.558333in}}{\pgfqpoint{0.416667in}{0.558333in}}%
\pgfpathcurveto{\pgfqpoint{0.401196in}{0.558333in}}{\pgfqpoint{0.386358in}{0.552187in}}{\pgfqpoint{0.375419in}{0.541248in}}%
\pgfpathcurveto{\pgfqpoint{0.364480in}{0.530309in}}{\pgfqpoint{0.358333in}{0.515470in}}{\pgfqpoint{0.358333in}{0.500000in}}%
\pgfpathcurveto{\pgfqpoint{0.358333in}{0.484530in}}{\pgfqpoint{0.364480in}{0.469691in}}{\pgfqpoint{0.375419in}{0.458752in}}%
\pgfpathcurveto{\pgfqpoint{0.386358in}{0.447813in}}{\pgfqpoint{0.401196in}{0.441667in}}{\pgfqpoint{0.416667in}{0.441667in}}%
\pgfpathclose%
\pgfpathmoveto{\pgfqpoint{0.416667in}{0.447500in}}%
\pgfpathcurveto{\pgfqpoint{0.416667in}{0.447500in}}{\pgfqpoint{0.402744in}{0.447500in}}{\pgfqpoint{0.389389in}{0.453032in}}%
\pgfpathcurveto{\pgfqpoint{0.379544in}{0.462877in}}{\pgfqpoint{0.369698in}{0.472722in}}{\pgfqpoint{0.364167in}{0.486077in}}%
\pgfpathcurveto{\pgfqpoint{0.364167in}{0.500000in}}{\pgfqpoint{0.364167in}{0.513923in}}{\pgfqpoint{0.369698in}{0.527278in}}%
\pgfpathcurveto{\pgfqpoint{0.379544in}{0.537123in}}{\pgfqpoint{0.389389in}{0.546968in}}{\pgfqpoint{0.402744in}{0.552500in}}%
\pgfpathcurveto{\pgfqpoint{0.416667in}{0.552500in}}{\pgfqpoint{0.430590in}{0.552500in}}{\pgfqpoint{0.443945in}{0.546968in}}%
\pgfpathcurveto{\pgfqpoint{0.453790in}{0.537123in}}{\pgfqpoint{0.463635in}{0.527278in}}{\pgfqpoint{0.469167in}{0.513923in}}%
\pgfpathcurveto{\pgfqpoint{0.469167in}{0.500000in}}{\pgfqpoint{0.469167in}{0.486077in}}{\pgfqpoint{0.463635in}{0.472722in}}%
\pgfpathcurveto{\pgfqpoint{0.453790in}{0.462877in}}{\pgfqpoint{0.443945in}{0.453032in}}{\pgfqpoint{0.430590in}{0.447500in}}%
\pgfpathclose%
\pgfpathmoveto{\pgfqpoint{0.583333in}{0.441667in}}%
\pgfpathcurveto{\pgfqpoint{0.598804in}{0.441667in}}{\pgfqpoint{0.613642in}{0.447813in}}{\pgfqpoint{0.624581in}{0.458752in}}%
\pgfpathcurveto{\pgfqpoint{0.635520in}{0.469691in}}{\pgfqpoint{0.641667in}{0.484530in}}{\pgfqpoint{0.641667in}{0.500000in}}%
\pgfpathcurveto{\pgfqpoint{0.641667in}{0.515470in}}{\pgfqpoint{0.635520in}{0.530309in}}{\pgfqpoint{0.624581in}{0.541248in}}%
\pgfpathcurveto{\pgfqpoint{0.613642in}{0.552187in}}{\pgfqpoint{0.598804in}{0.558333in}}{\pgfqpoint{0.583333in}{0.558333in}}%
\pgfpathcurveto{\pgfqpoint{0.567863in}{0.558333in}}{\pgfqpoint{0.553025in}{0.552187in}}{\pgfqpoint{0.542085in}{0.541248in}}%
\pgfpathcurveto{\pgfqpoint{0.531146in}{0.530309in}}{\pgfqpoint{0.525000in}{0.515470in}}{\pgfqpoint{0.525000in}{0.500000in}}%
\pgfpathcurveto{\pgfqpoint{0.525000in}{0.484530in}}{\pgfqpoint{0.531146in}{0.469691in}}{\pgfqpoint{0.542085in}{0.458752in}}%
\pgfpathcurveto{\pgfqpoint{0.553025in}{0.447813in}}{\pgfqpoint{0.567863in}{0.441667in}}{\pgfqpoint{0.583333in}{0.441667in}}%
\pgfpathclose%
\pgfpathmoveto{\pgfqpoint{0.583333in}{0.447500in}}%
\pgfpathcurveto{\pgfqpoint{0.583333in}{0.447500in}}{\pgfqpoint{0.569410in}{0.447500in}}{\pgfqpoint{0.556055in}{0.453032in}}%
\pgfpathcurveto{\pgfqpoint{0.546210in}{0.462877in}}{\pgfqpoint{0.536365in}{0.472722in}}{\pgfqpoint{0.530833in}{0.486077in}}%
\pgfpathcurveto{\pgfqpoint{0.530833in}{0.500000in}}{\pgfqpoint{0.530833in}{0.513923in}}{\pgfqpoint{0.536365in}{0.527278in}}%
\pgfpathcurveto{\pgfqpoint{0.546210in}{0.537123in}}{\pgfqpoint{0.556055in}{0.546968in}}{\pgfqpoint{0.569410in}{0.552500in}}%
\pgfpathcurveto{\pgfqpoint{0.583333in}{0.552500in}}{\pgfqpoint{0.597256in}{0.552500in}}{\pgfqpoint{0.610611in}{0.546968in}}%
\pgfpathcurveto{\pgfqpoint{0.620456in}{0.537123in}}{\pgfqpoint{0.630302in}{0.527278in}}{\pgfqpoint{0.635833in}{0.513923in}}%
\pgfpathcurveto{\pgfqpoint{0.635833in}{0.500000in}}{\pgfqpoint{0.635833in}{0.486077in}}{\pgfqpoint{0.630302in}{0.472722in}}%
\pgfpathcurveto{\pgfqpoint{0.620456in}{0.462877in}}{\pgfqpoint{0.610611in}{0.453032in}}{\pgfqpoint{0.597256in}{0.447500in}}%
\pgfpathclose%
\pgfpathmoveto{\pgfqpoint{0.750000in}{0.441667in}}%
\pgfpathcurveto{\pgfqpoint{0.765470in}{0.441667in}}{\pgfqpoint{0.780309in}{0.447813in}}{\pgfqpoint{0.791248in}{0.458752in}}%
\pgfpathcurveto{\pgfqpoint{0.802187in}{0.469691in}}{\pgfqpoint{0.808333in}{0.484530in}}{\pgfqpoint{0.808333in}{0.500000in}}%
\pgfpathcurveto{\pgfqpoint{0.808333in}{0.515470in}}{\pgfqpoint{0.802187in}{0.530309in}}{\pgfqpoint{0.791248in}{0.541248in}}%
\pgfpathcurveto{\pgfqpoint{0.780309in}{0.552187in}}{\pgfqpoint{0.765470in}{0.558333in}}{\pgfqpoint{0.750000in}{0.558333in}}%
\pgfpathcurveto{\pgfqpoint{0.734530in}{0.558333in}}{\pgfqpoint{0.719691in}{0.552187in}}{\pgfqpoint{0.708752in}{0.541248in}}%
\pgfpathcurveto{\pgfqpoint{0.697813in}{0.530309in}}{\pgfqpoint{0.691667in}{0.515470in}}{\pgfqpoint{0.691667in}{0.500000in}}%
\pgfpathcurveto{\pgfqpoint{0.691667in}{0.484530in}}{\pgfqpoint{0.697813in}{0.469691in}}{\pgfqpoint{0.708752in}{0.458752in}}%
\pgfpathcurveto{\pgfqpoint{0.719691in}{0.447813in}}{\pgfqpoint{0.734530in}{0.441667in}}{\pgfqpoint{0.750000in}{0.441667in}}%
\pgfpathclose%
\pgfpathmoveto{\pgfqpoint{0.750000in}{0.447500in}}%
\pgfpathcurveto{\pgfqpoint{0.750000in}{0.447500in}}{\pgfqpoint{0.736077in}{0.447500in}}{\pgfqpoint{0.722722in}{0.453032in}}%
\pgfpathcurveto{\pgfqpoint{0.712877in}{0.462877in}}{\pgfqpoint{0.703032in}{0.472722in}}{\pgfqpoint{0.697500in}{0.486077in}}%
\pgfpathcurveto{\pgfqpoint{0.697500in}{0.500000in}}{\pgfqpoint{0.697500in}{0.513923in}}{\pgfqpoint{0.703032in}{0.527278in}}%
\pgfpathcurveto{\pgfqpoint{0.712877in}{0.537123in}}{\pgfqpoint{0.722722in}{0.546968in}}{\pgfqpoint{0.736077in}{0.552500in}}%
\pgfpathcurveto{\pgfqpoint{0.750000in}{0.552500in}}{\pgfqpoint{0.763923in}{0.552500in}}{\pgfqpoint{0.777278in}{0.546968in}}%
\pgfpathcurveto{\pgfqpoint{0.787123in}{0.537123in}}{\pgfqpoint{0.796968in}{0.527278in}}{\pgfqpoint{0.802500in}{0.513923in}}%
\pgfpathcurveto{\pgfqpoint{0.802500in}{0.500000in}}{\pgfqpoint{0.802500in}{0.486077in}}{\pgfqpoint{0.796968in}{0.472722in}}%
\pgfpathcurveto{\pgfqpoint{0.787123in}{0.462877in}}{\pgfqpoint{0.777278in}{0.453032in}}{\pgfqpoint{0.763923in}{0.447500in}}%
\pgfpathclose%
\pgfpathmoveto{\pgfqpoint{0.916667in}{0.441667in}}%
\pgfpathcurveto{\pgfqpoint{0.932137in}{0.441667in}}{\pgfqpoint{0.946975in}{0.447813in}}{\pgfqpoint{0.957915in}{0.458752in}}%
\pgfpathcurveto{\pgfqpoint{0.968854in}{0.469691in}}{\pgfqpoint{0.975000in}{0.484530in}}{\pgfqpoint{0.975000in}{0.500000in}}%
\pgfpathcurveto{\pgfqpoint{0.975000in}{0.515470in}}{\pgfqpoint{0.968854in}{0.530309in}}{\pgfqpoint{0.957915in}{0.541248in}}%
\pgfpathcurveto{\pgfqpoint{0.946975in}{0.552187in}}{\pgfqpoint{0.932137in}{0.558333in}}{\pgfqpoint{0.916667in}{0.558333in}}%
\pgfpathcurveto{\pgfqpoint{0.901196in}{0.558333in}}{\pgfqpoint{0.886358in}{0.552187in}}{\pgfqpoint{0.875419in}{0.541248in}}%
\pgfpathcurveto{\pgfqpoint{0.864480in}{0.530309in}}{\pgfqpoint{0.858333in}{0.515470in}}{\pgfqpoint{0.858333in}{0.500000in}}%
\pgfpathcurveto{\pgfqpoint{0.858333in}{0.484530in}}{\pgfqpoint{0.864480in}{0.469691in}}{\pgfqpoint{0.875419in}{0.458752in}}%
\pgfpathcurveto{\pgfqpoint{0.886358in}{0.447813in}}{\pgfqpoint{0.901196in}{0.441667in}}{\pgfqpoint{0.916667in}{0.441667in}}%
\pgfpathclose%
\pgfpathmoveto{\pgfqpoint{0.916667in}{0.447500in}}%
\pgfpathcurveto{\pgfqpoint{0.916667in}{0.447500in}}{\pgfqpoint{0.902744in}{0.447500in}}{\pgfqpoint{0.889389in}{0.453032in}}%
\pgfpathcurveto{\pgfqpoint{0.879544in}{0.462877in}}{\pgfqpoint{0.869698in}{0.472722in}}{\pgfqpoint{0.864167in}{0.486077in}}%
\pgfpathcurveto{\pgfqpoint{0.864167in}{0.500000in}}{\pgfqpoint{0.864167in}{0.513923in}}{\pgfqpoint{0.869698in}{0.527278in}}%
\pgfpathcurveto{\pgfqpoint{0.879544in}{0.537123in}}{\pgfqpoint{0.889389in}{0.546968in}}{\pgfqpoint{0.902744in}{0.552500in}}%
\pgfpathcurveto{\pgfqpoint{0.916667in}{0.552500in}}{\pgfqpoint{0.930590in}{0.552500in}}{\pgfqpoint{0.943945in}{0.546968in}}%
\pgfpathcurveto{\pgfqpoint{0.953790in}{0.537123in}}{\pgfqpoint{0.963635in}{0.527278in}}{\pgfqpoint{0.969167in}{0.513923in}}%
\pgfpathcurveto{\pgfqpoint{0.969167in}{0.500000in}}{\pgfqpoint{0.969167in}{0.486077in}}{\pgfqpoint{0.963635in}{0.472722in}}%
\pgfpathcurveto{\pgfqpoint{0.953790in}{0.462877in}}{\pgfqpoint{0.943945in}{0.453032in}}{\pgfqpoint{0.930590in}{0.447500in}}%
\pgfpathclose%
\pgfpathmoveto{\pgfqpoint{0.000000in}{0.608333in}}%
\pgfpathcurveto{\pgfqpoint{0.015470in}{0.608333in}}{\pgfqpoint{0.030309in}{0.614480in}}{\pgfqpoint{0.041248in}{0.625419in}}%
\pgfpathcurveto{\pgfqpoint{0.052187in}{0.636358in}}{\pgfqpoint{0.058333in}{0.651196in}}{\pgfqpoint{0.058333in}{0.666667in}}%
\pgfpathcurveto{\pgfqpoint{0.058333in}{0.682137in}}{\pgfqpoint{0.052187in}{0.696975in}}{\pgfqpoint{0.041248in}{0.707915in}}%
\pgfpathcurveto{\pgfqpoint{0.030309in}{0.718854in}}{\pgfqpoint{0.015470in}{0.725000in}}{\pgfqpoint{0.000000in}{0.725000in}}%
\pgfpathcurveto{\pgfqpoint{-0.015470in}{0.725000in}}{\pgfqpoint{-0.030309in}{0.718854in}}{\pgfqpoint{-0.041248in}{0.707915in}}%
\pgfpathcurveto{\pgfqpoint{-0.052187in}{0.696975in}}{\pgfqpoint{-0.058333in}{0.682137in}}{\pgfqpoint{-0.058333in}{0.666667in}}%
\pgfpathcurveto{\pgfqpoint{-0.058333in}{0.651196in}}{\pgfqpoint{-0.052187in}{0.636358in}}{\pgfqpoint{-0.041248in}{0.625419in}}%
\pgfpathcurveto{\pgfqpoint{-0.030309in}{0.614480in}}{\pgfqpoint{-0.015470in}{0.608333in}}{\pgfqpoint{0.000000in}{0.608333in}}%
\pgfpathclose%
\pgfpathmoveto{\pgfqpoint{0.000000in}{0.614167in}}%
\pgfpathcurveto{\pgfqpoint{0.000000in}{0.614167in}}{\pgfqpoint{-0.013923in}{0.614167in}}{\pgfqpoint{-0.027278in}{0.619698in}}%
\pgfpathcurveto{\pgfqpoint{-0.037123in}{0.629544in}}{\pgfqpoint{-0.046968in}{0.639389in}}{\pgfqpoint{-0.052500in}{0.652744in}}%
\pgfpathcurveto{\pgfqpoint{-0.052500in}{0.666667in}}{\pgfqpoint{-0.052500in}{0.680590in}}{\pgfqpoint{-0.046968in}{0.693945in}}%
\pgfpathcurveto{\pgfqpoint{-0.037123in}{0.703790in}}{\pgfqpoint{-0.027278in}{0.713635in}}{\pgfqpoint{-0.013923in}{0.719167in}}%
\pgfpathcurveto{\pgfqpoint{0.000000in}{0.719167in}}{\pgfqpoint{0.013923in}{0.719167in}}{\pgfqpoint{0.027278in}{0.713635in}}%
\pgfpathcurveto{\pgfqpoint{0.037123in}{0.703790in}}{\pgfqpoint{0.046968in}{0.693945in}}{\pgfqpoint{0.052500in}{0.680590in}}%
\pgfpathcurveto{\pgfqpoint{0.052500in}{0.666667in}}{\pgfqpoint{0.052500in}{0.652744in}}{\pgfqpoint{0.046968in}{0.639389in}}%
\pgfpathcurveto{\pgfqpoint{0.037123in}{0.629544in}}{\pgfqpoint{0.027278in}{0.619698in}}{\pgfqpoint{0.013923in}{0.614167in}}%
\pgfpathclose%
\pgfpathmoveto{\pgfqpoint{0.166667in}{0.608333in}}%
\pgfpathcurveto{\pgfqpoint{0.182137in}{0.608333in}}{\pgfqpoint{0.196975in}{0.614480in}}{\pgfqpoint{0.207915in}{0.625419in}}%
\pgfpathcurveto{\pgfqpoint{0.218854in}{0.636358in}}{\pgfqpoint{0.225000in}{0.651196in}}{\pgfqpoint{0.225000in}{0.666667in}}%
\pgfpathcurveto{\pgfqpoint{0.225000in}{0.682137in}}{\pgfqpoint{0.218854in}{0.696975in}}{\pgfqpoint{0.207915in}{0.707915in}}%
\pgfpathcurveto{\pgfqpoint{0.196975in}{0.718854in}}{\pgfqpoint{0.182137in}{0.725000in}}{\pgfqpoint{0.166667in}{0.725000in}}%
\pgfpathcurveto{\pgfqpoint{0.151196in}{0.725000in}}{\pgfqpoint{0.136358in}{0.718854in}}{\pgfqpoint{0.125419in}{0.707915in}}%
\pgfpathcurveto{\pgfqpoint{0.114480in}{0.696975in}}{\pgfqpoint{0.108333in}{0.682137in}}{\pgfqpoint{0.108333in}{0.666667in}}%
\pgfpathcurveto{\pgfqpoint{0.108333in}{0.651196in}}{\pgfqpoint{0.114480in}{0.636358in}}{\pgfqpoint{0.125419in}{0.625419in}}%
\pgfpathcurveto{\pgfqpoint{0.136358in}{0.614480in}}{\pgfqpoint{0.151196in}{0.608333in}}{\pgfqpoint{0.166667in}{0.608333in}}%
\pgfpathclose%
\pgfpathmoveto{\pgfqpoint{0.166667in}{0.614167in}}%
\pgfpathcurveto{\pgfqpoint{0.166667in}{0.614167in}}{\pgfqpoint{0.152744in}{0.614167in}}{\pgfqpoint{0.139389in}{0.619698in}}%
\pgfpathcurveto{\pgfqpoint{0.129544in}{0.629544in}}{\pgfqpoint{0.119698in}{0.639389in}}{\pgfqpoint{0.114167in}{0.652744in}}%
\pgfpathcurveto{\pgfqpoint{0.114167in}{0.666667in}}{\pgfqpoint{0.114167in}{0.680590in}}{\pgfqpoint{0.119698in}{0.693945in}}%
\pgfpathcurveto{\pgfqpoint{0.129544in}{0.703790in}}{\pgfqpoint{0.139389in}{0.713635in}}{\pgfqpoint{0.152744in}{0.719167in}}%
\pgfpathcurveto{\pgfqpoint{0.166667in}{0.719167in}}{\pgfqpoint{0.180590in}{0.719167in}}{\pgfqpoint{0.193945in}{0.713635in}}%
\pgfpathcurveto{\pgfqpoint{0.203790in}{0.703790in}}{\pgfqpoint{0.213635in}{0.693945in}}{\pgfqpoint{0.219167in}{0.680590in}}%
\pgfpathcurveto{\pgfqpoint{0.219167in}{0.666667in}}{\pgfqpoint{0.219167in}{0.652744in}}{\pgfqpoint{0.213635in}{0.639389in}}%
\pgfpathcurveto{\pgfqpoint{0.203790in}{0.629544in}}{\pgfqpoint{0.193945in}{0.619698in}}{\pgfqpoint{0.180590in}{0.614167in}}%
\pgfpathclose%
\pgfpathmoveto{\pgfqpoint{0.333333in}{0.608333in}}%
\pgfpathcurveto{\pgfqpoint{0.348804in}{0.608333in}}{\pgfqpoint{0.363642in}{0.614480in}}{\pgfqpoint{0.374581in}{0.625419in}}%
\pgfpathcurveto{\pgfqpoint{0.385520in}{0.636358in}}{\pgfqpoint{0.391667in}{0.651196in}}{\pgfqpoint{0.391667in}{0.666667in}}%
\pgfpathcurveto{\pgfqpoint{0.391667in}{0.682137in}}{\pgfqpoint{0.385520in}{0.696975in}}{\pgfqpoint{0.374581in}{0.707915in}}%
\pgfpathcurveto{\pgfqpoint{0.363642in}{0.718854in}}{\pgfqpoint{0.348804in}{0.725000in}}{\pgfqpoint{0.333333in}{0.725000in}}%
\pgfpathcurveto{\pgfqpoint{0.317863in}{0.725000in}}{\pgfqpoint{0.303025in}{0.718854in}}{\pgfqpoint{0.292085in}{0.707915in}}%
\pgfpathcurveto{\pgfqpoint{0.281146in}{0.696975in}}{\pgfqpoint{0.275000in}{0.682137in}}{\pgfqpoint{0.275000in}{0.666667in}}%
\pgfpathcurveto{\pgfqpoint{0.275000in}{0.651196in}}{\pgfqpoint{0.281146in}{0.636358in}}{\pgfqpoint{0.292085in}{0.625419in}}%
\pgfpathcurveto{\pgfqpoint{0.303025in}{0.614480in}}{\pgfqpoint{0.317863in}{0.608333in}}{\pgfqpoint{0.333333in}{0.608333in}}%
\pgfpathclose%
\pgfpathmoveto{\pgfqpoint{0.333333in}{0.614167in}}%
\pgfpathcurveto{\pgfqpoint{0.333333in}{0.614167in}}{\pgfqpoint{0.319410in}{0.614167in}}{\pgfqpoint{0.306055in}{0.619698in}}%
\pgfpathcurveto{\pgfqpoint{0.296210in}{0.629544in}}{\pgfqpoint{0.286365in}{0.639389in}}{\pgfqpoint{0.280833in}{0.652744in}}%
\pgfpathcurveto{\pgfqpoint{0.280833in}{0.666667in}}{\pgfqpoint{0.280833in}{0.680590in}}{\pgfqpoint{0.286365in}{0.693945in}}%
\pgfpathcurveto{\pgfqpoint{0.296210in}{0.703790in}}{\pgfqpoint{0.306055in}{0.713635in}}{\pgfqpoint{0.319410in}{0.719167in}}%
\pgfpathcurveto{\pgfqpoint{0.333333in}{0.719167in}}{\pgfqpoint{0.347256in}{0.719167in}}{\pgfqpoint{0.360611in}{0.713635in}}%
\pgfpathcurveto{\pgfqpoint{0.370456in}{0.703790in}}{\pgfqpoint{0.380302in}{0.693945in}}{\pgfqpoint{0.385833in}{0.680590in}}%
\pgfpathcurveto{\pgfqpoint{0.385833in}{0.666667in}}{\pgfqpoint{0.385833in}{0.652744in}}{\pgfqpoint{0.380302in}{0.639389in}}%
\pgfpathcurveto{\pgfqpoint{0.370456in}{0.629544in}}{\pgfqpoint{0.360611in}{0.619698in}}{\pgfqpoint{0.347256in}{0.614167in}}%
\pgfpathclose%
\pgfpathmoveto{\pgfqpoint{0.500000in}{0.608333in}}%
\pgfpathcurveto{\pgfqpoint{0.515470in}{0.608333in}}{\pgfqpoint{0.530309in}{0.614480in}}{\pgfqpoint{0.541248in}{0.625419in}}%
\pgfpathcurveto{\pgfqpoint{0.552187in}{0.636358in}}{\pgfqpoint{0.558333in}{0.651196in}}{\pgfqpoint{0.558333in}{0.666667in}}%
\pgfpathcurveto{\pgfqpoint{0.558333in}{0.682137in}}{\pgfqpoint{0.552187in}{0.696975in}}{\pgfqpoint{0.541248in}{0.707915in}}%
\pgfpathcurveto{\pgfqpoint{0.530309in}{0.718854in}}{\pgfqpoint{0.515470in}{0.725000in}}{\pgfqpoint{0.500000in}{0.725000in}}%
\pgfpathcurveto{\pgfqpoint{0.484530in}{0.725000in}}{\pgfqpoint{0.469691in}{0.718854in}}{\pgfqpoint{0.458752in}{0.707915in}}%
\pgfpathcurveto{\pgfqpoint{0.447813in}{0.696975in}}{\pgfqpoint{0.441667in}{0.682137in}}{\pgfqpoint{0.441667in}{0.666667in}}%
\pgfpathcurveto{\pgfqpoint{0.441667in}{0.651196in}}{\pgfqpoint{0.447813in}{0.636358in}}{\pgfqpoint{0.458752in}{0.625419in}}%
\pgfpathcurveto{\pgfqpoint{0.469691in}{0.614480in}}{\pgfqpoint{0.484530in}{0.608333in}}{\pgfqpoint{0.500000in}{0.608333in}}%
\pgfpathclose%
\pgfpathmoveto{\pgfqpoint{0.500000in}{0.614167in}}%
\pgfpathcurveto{\pgfqpoint{0.500000in}{0.614167in}}{\pgfqpoint{0.486077in}{0.614167in}}{\pgfqpoint{0.472722in}{0.619698in}}%
\pgfpathcurveto{\pgfqpoint{0.462877in}{0.629544in}}{\pgfqpoint{0.453032in}{0.639389in}}{\pgfqpoint{0.447500in}{0.652744in}}%
\pgfpathcurveto{\pgfqpoint{0.447500in}{0.666667in}}{\pgfqpoint{0.447500in}{0.680590in}}{\pgfqpoint{0.453032in}{0.693945in}}%
\pgfpathcurveto{\pgfqpoint{0.462877in}{0.703790in}}{\pgfqpoint{0.472722in}{0.713635in}}{\pgfqpoint{0.486077in}{0.719167in}}%
\pgfpathcurveto{\pgfqpoint{0.500000in}{0.719167in}}{\pgfqpoint{0.513923in}{0.719167in}}{\pgfqpoint{0.527278in}{0.713635in}}%
\pgfpathcurveto{\pgfqpoint{0.537123in}{0.703790in}}{\pgfqpoint{0.546968in}{0.693945in}}{\pgfqpoint{0.552500in}{0.680590in}}%
\pgfpathcurveto{\pgfqpoint{0.552500in}{0.666667in}}{\pgfqpoint{0.552500in}{0.652744in}}{\pgfqpoint{0.546968in}{0.639389in}}%
\pgfpathcurveto{\pgfqpoint{0.537123in}{0.629544in}}{\pgfqpoint{0.527278in}{0.619698in}}{\pgfqpoint{0.513923in}{0.614167in}}%
\pgfpathclose%
\pgfpathmoveto{\pgfqpoint{0.666667in}{0.608333in}}%
\pgfpathcurveto{\pgfqpoint{0.682137in}{0.608333in}}{\pgfqpoint{0.696975in}{0.614480in}}{\pgfqpoint{0.707915in}{0.625419in}}%
\pgfpathcurveto{\pgfqpoint{0.718854in}{0.636358in}}{\pgfqpoint{0.725000in}{0.651196in}}{\pgfqpoint{0.725000in}{0.666667in}}%
\pgfpathcurveto{\pgfqpoint{0.725000in}{0.682137in}}{\pgfqpoint{0.718854in}{0.696975in}}{\pgfqpoint{0.707915in}{0.707915in}}%
\pgfpathcurveto{\pgfqpoint{0.696975in}{0.718854in}}{\pgfqpoint{0.682137in}{0.725000in}}{\pgfqpoint{0.666667in}{0.725000in}}%
\pgfpathcurveto{\pgfqpoint{0.651196in}{0.725000in}}{\pgfqpoint{0.636358in}{0.718854in}}{\pgfqpoint{0.625419in}{0.707915in}}%
\pgfpathcurveto{\pgfqpoint{0.614480in}{0.696975in}}{\pgfqpoint{0.608333in}{0.682137in}}{\pgfqpoint{0.608333in}{0.666667in}}%
\pgfpathcurveto{\pgfqpoint{0.608333in}{0.651196in}}{\pgfqpoint{0.614480in}{0.636358in}}{\pgfqpoint{0.625419in}{0.625419in}}%
\pgfpathcurveto{\pgfqpoint{0.636358in}{0.614480in}}{\pgfqpoint{0.651196in}{0.608333in}}{\pgfqpoint{0.666667in}{0.608333in}}%
\pgfpathclose%
\pgfpathmoveto{\pgfqpoint{0.666667in}{0.614167in}}%
\pgfpathcurveto{\pgfqpoint{0.666667in}{0.614167in}}{\pgfqpoint{0.652744in}{0.614167in}}{\pgfqpoint{0.639389in}{0.619698in}}%
\pgfpathcurveto{\pgfqpoint{0.629544in}{0.629544in}}{\pgfqpoint{0.619698in}{0.639389in}}{\pgfqpoint{0.614167in}{0.652744in}}%
\pgfpathcurveto{\pgfqpoint{0.614167in}{0.666667in}}{\pgfqpoint{0.614167in}{0.680590in}}{\pgfqpoint{0.619698in}{0.693945in}}%
\pgfpathcurveto{\pgfqpoint{0.629544in}{0.703790in}}{\pgfqpoint{0.639389in}{0.713635in}}{\pgfqpoint{0.652744in}{0.719167in}}%
\pgfpathcurveto{\pgfqpoint{0.666667in}{0.719167in}}{\pgfqpoint{0.680590in}{0.719167in}}{\pgfqpoint{0.693945in}{0.713635in}}%
\pgfpathcurveto{\pgfqpoint{0.703790in}{0.703790in}}{\pgfqpoint{0.713635in}{0.693945in}}{\pgfqpoint{0.719167in}{0.680590in}}%
\pgfpathcurveto{\pgfqpoint{0.719167in}{0.666667in}}{\pgfqpoint{0.719167in}{0.652744in}}{\pgfqpoint{0.713635in}{0.639389in}}%
\pgfpathcurveto{\pgfqpoint{0.703790in}{0.629544in}}{\pgfqpoint{0.693945in}{0.619698in}}{\pgfqpoint{0.680590in}{0.614167in}}%
\pgfpathclose%
\pgfpathmoveto{\pgfqpoint{0.833333in}{0.608333in}}%
\pgfpathcurveto{\pgfqpoint{0.848804in}{0.608333in}}{\pgfqpoint{0.863642in}{0.614480in}}{\pgfqpoint{0.874581in}{0.625419in}}%
\pgfpathcurveto{\pgfqpoint{0.885520in}{0.636358in}}{\pgfqpoint{0.891667in}{0.651196in}}{\pgfqpoint{0.891667in}{0.666667in}}%
\pgfpathcurveto{\pgfqpoint{0.891667in}{0.682137in}}{\pgfqpoint{0.885520in}{0.696975in}}{\pgfqpoint{0.874581in}{0.707915in}}%
\pgfpathcurveto{\pgfqpoint{0.863642in}{0.718854in}}{\pgfqpoint{0.848804in}{0.725000in}}{\pgfqpoint{0.833333in}{0.725000in}}%
\pgfpathcurveto{\pgfqpoint{0.817863in}{0.725000in}}{\pgfqpoint{0.803025in}{0.718854in}}{\pgfqpoint{0.792085in}{0.707915in}}%
\pgfpathcurveto{\pgfqpoint{0.781146in}{0.696975in}}{\pgfqpoint{0.775000in}{0.682137in}}{\pgfqpoint{0.775000in}{0.666667in}}%
\pgfpathcurveto{\pgfqpoint{0.775000in}{0.651196in}}{\pgfqpoint{0.781146in}{0.636358in}}{\pgfqpoint{0.792085in}{0.625419in}}%
\pgfpathcurveto{\pgfqpoint{0.803025in}{0.614480in}}{\pgfqpoint{0.817863in}{0.608333in}}{\pgfqpoint{0.833333in}{0.608333in}}%
\pgfpathclose%
\pgfpathmoveto{\pgfqpoint{0.833333in}{0.614167in}}%
\pgfpathcurveto{\pgfqpoint{0.833333in}{0.614167in}}{\pgfqpoint{0.819410in}{0.614167in}}{\pgfqpoint{0.806055in}{0.619698in}}%
\pgfpathcurveto{\pgfqpoint{0.796210in}{0.629544in}}{\pgfqpoint{0.786365in}{0.639389in}}{\pgfqpoint{0.780833in}{0.652744in}}%
\pgfpathcurveto{\pgfqpoint{0.780833in}{0.666667in}}{\pgfqpoint{0.780833in}{0.680590in}}{\pgfqpoint{0.786365in}{0.693945in}}%
\pgfpathcurveto{\pgfqpoint{0.796210in}{0.703790in}}{\pgfqpoint{0.806055in}{0.713635in}}{\pgfqpoint{0.819410in}{0.719167in}}%
\pgfpathcurveto{\pgfqpoint{0.833333in}{0.719167in}}{\pgfqpoint{0.847256in}{0.719167in}}{\pgfqpoint{0.860611in}{0.713635in}}%
\pgfpathcurveto{\pgfqpoint{0.870456in}{0.703790in}}{\pgfqpoint{0.880302in}{0.693945in}}{\pgfqpoint{0.885833in}{0.680590in}}%
\pgfpathcurveto{\pgfqpoint{0.885833in}{0.666667in}}{\pgfqpoint{0.885833in}{0.652744in}}{\pgfqpoint{0.880302in}{0.639389in}}%
\pgfpathcurveto{\pgfqpoint{0.870456in}{0.629544in}}{\pgfqpoint{0.860611in}{0.619698in}}{\pgfqpoint{0.847256in}{0.614167in}}%
\pgfpathclose%
\pgfpathmoveto{\pgfqpoint{1.000000in}{0.608333in}}%
\pgfpathcurveto{\pgfqpoint{1.015470in}{0.608333in}}{\pgfqpoint{1.030309in}{0.614480in}}{\pgfqpoint{1.041248in}{0.625419in}}%
\pgfpathcurveto{\pgfqpoint{1.052187in}{0.636358in}}{\pgfqpoint{1.058333in}{0.651196in}}{\pgfqpoint{1.058333in}{0.666667in}}%
\pgfpathcurveto{\pgfqpoint{1.058333in}{0.682137in}}{\pgfqpoint{1.052187in}{0.696975in}}{\pgfqpoint{1.041248in}{0.707915in}}%
\pgfpathcurveto{\pgfqpoint{1.030309in}{0.718854in}}{\pgfqpoint{1.015470in}{0.725000in}}{\pgfqpoint{1.000000in}{0.725000in}}%
\pgfpathcurveto{\pgfqpoint{0.984530in}{0.725000in}}{\pgfqpoint{0.969691in}{0.718854in}}{\pgfqpoint{0.958752in}{0.707915in}}%
\pgfpathcurveto{\pgfqpoint{0.947813in}{0.696975in}}{\pgfqpoint{0.941667in}{0.682137in}}{\pgfqpoint{0.941667in}{0.666667in}}%
\pgfpathcurveto{\pgfqpoint{0.941667in}{0.651196in}}{\pgfqpoint{0.947813in}{0.636358in}}{\pgfqpoint{0.958752in}{0.625419in}}%
\pgfpathcurveto{\pgfqpoint{0.969691in}{0.614480in}}{\pgfqpoint{0.984530in}{0.608333in}}{\pgfqpoint{1.000000in}{0.608333in}}%
\pgfpathclose%
\pgfpathmoveto{\pgfqpoint{1.000000in}{0.614167in}}%
\pgfpathcurveto{\pgfqpoint{1.000000in}{0.614167in}}{\pgfqpoint{0.986077in}{0.614167in}}{\pgfqpoint{0.972722in}{0.619698in}}%
\pgfpathcurveto{\pgfqpoint{0.962877in}{0.629544in}}{\pgfqpoint{0.953032in}{0.639389in}}{\pgfqpoint{0.947500in}{0.652744in}}%
\pgfpathcurveto{\pgfqpoint{0.947500in}{0.666667in}}{\pgfqpoint{0.947500in}{0.680590in}}{\pgfqpoint{0.953032in}{0.693945in}}%
\pgfpathcurveto{\pgfqpoint{0.962877in}{0.703790in}}{\pgfqpoint{0.972722in}{0.713635in}}{\pgfqpoint{0.986077in}{0.719167in}}%
\pgfpathcurveto{\pgfqpoint{1.000000in}{0.719167in}}{\pgfqpoint{1.013923in}{0.719167in}}{\pgfqpoint{1.027278in}{0.713635in}}%
\pgfpathcurveto{\pgfqpoint{1.037123in}{0.703790in}}{\pgfqpoint{1.046968in}{0.693945in}}{\pgfqpoint{1.052500in}{0.680590in}}%
\pgfpathcurveto{\pgfqpoint{1.052500in}{0.666667in}}{\pgfqpoint{1.052500in}{0.652744in}}{\pgfqpoint{1.046968in}{0.639389in}}%
\pgfpathcurveto{\pgfqpoint{1.037123in}{0.629544in}}{\pgfqpoint{1.027278in}{0.619698in}}{\pgfqpoint{1.013923in}{0.614167in}}%
\pgfpathclose%
\pgfpathmoveto{\pgfqpoint{0.083333in}{0.775000in}}%
\pgfpathcurveto{\pgfqpoint{0.098804in}{0.775000in}}{\pgfqpoint{0.113642in}{0.781146in}}{\pgfqpoint{0.124581in}{0.792085in}}%
\pgfpathcurveto{\pgfqpoint{0.135520in}{0.803025in}}{\pgfqpoint{0.141667in}{0.817863in}}{\pgfqpoint{0.141667in}{0.833333in}}%
\pgfpathcurveto{\pgfqpoint{0.141667in}{0.848804in}}{\pgfqpoint{0.135520in}{0.863642in}}{\pgfqpoint{0.124581in}{0.874581in}}%
\pgfpathcurveto{\pgfqpoint{0.113642in}{0.885520in}}{\pgfqpoint{0.098804in}{0.891667in}}{\pgfqpoint{0.083333in}{0.891667in}}%
\pgfpathcurveto{\pgfqpoint{0.067863in}{0.891667in}}{\pgfqpoint{0.053025in}{0.885520in}}{\pgfqpoint{0.042085in}{0.874581in}}%
\pgfpathcurveto{\pgfqpoint{0.031146in}{0.863642in}}{\pgfqpoint{0.025000in}{0.848804in}}{\pgfqpoint{0.025000in}{0.833333in}}%
\pgfpathcurveto{\pgfqpoint{0.025000in}{0.817863in}}{\pgfqpoint{0.031146in}{0.803025in}}{\pgfqpoint{0.042085in}{0.792085in}}%
\pgfpathcurveto{\pgfqpoint{0.053025in}{0.781146in}}{\pgfqpoint{0.067863in}{0.775000in}}{\pgfqpoint{0.083333in}{0.775000in}}%
\pgfpathclose%
\pgfpathmoveto{\pgfqpoint{0.083333in}{0.780833in}}%
\pgfpathcurveto{\pgfqpoint{0.083333in}{0.780833in}}{\pgfqpoint{0.069410in}{0.780833in}}{\pgfqpoint{0.056055in}{0.786365in}}%
\pgfpathcurveto{\pgfqpoint{0.046210in}{0.796210in}}{\pgfqpoint{0.036365in}{0.806055in}}{\pgfqpoint{0.030833in}{0.819410in}}%
\pgfpathcurveto{\pgfqpoint{0.030833in}{0.833333in}}{\pgfqpoint{0.030833in}{0.847256in}}{\pgfqpoint{0.036365in}{0.860611in}}%
\pgfpathcurveto{\pgfqpoint{0.046210in}{0.870456in}}{\pgfqpoint{0.056055in}{0.880302in}}{\pgfqpoint{0.069410in}{0.885833in}}%
\pgfpathcurveto{\pgfqpoint{0.083333in}{0.885833in}}{\pgfqpoint{0.097256in}{0.885833in}}{\pgfqpoint{0.110611in}{0.880302in}}%
\pgfpathcurveto{\pgfqpoint{0.120456in}{0.870456in}}{\pgfqpoint{0.130302in}{0.860611in}}{\pgfqpoint{0.135833in}{0.847256in}}%
\pgfpathcurveto{\pgfqpoint{0.135833in}{0.833333in}}{\pgfqpoint{0.135833in}{0.819410in}}{\pgfqpoint{0.130302in}{0.806055in}}%
\pgfpathcurveto{\pgfqpoint{0.120456in}{0.796210in}}{\pgfqpoint{0.110611in}{0.786365in}}{\pgfqpoint{0.097256in}{0.780833in}}%
\pgfpathclose%
\pgfpathmoveto{\pgfqpoint{0.250000in}{0.775000in}}%
\pgfpathcurveto{\pgfqpoint{0.265470in}{0.775000in}}{\pgfqpoint{0.280309in}{0.781146in}}{\pgfqpoint{0.291248in}{0.792085in}}%
\pgfpathcurveto{\pgfqpoint{0.302187in}{0.803025in}}{\pgfqpoint{0.308333in}{0.817863in}}{\pgfqpoint{0.308333in}{0.833333in}}%
\pgfpathcurveto{\pgfqpoint{0.308333in}{0.848804in}}{\pgfqpoint{0.302187in}{0.863642in}}{\pgfqpoint{0.291248in}{0.874581in}}%
\pgfpathcurveto{\pgfqpoint{0.280309in}{0.885520in}}{\pgfqpoint{0.265470in}{0.891667in}}{\pgfqpoint{0.250000in}{0.891667in}}%
\pgfpathcurveto{\pgfqpoint{0.234530in}{0.891667in}}{\pgfqpoint{0.219691in}{0.885520in}}{\pgfqpoint{0.208752in}{0.874581in}}%
\pgfpathcurveto{\pgfqpoint{0.197813in}{0.863642in}}{\pgfqpoint{0.191667in}{0.848804in}}{\pgfqpoint{0.191667in}{0.833333in}}%
\pgfpathcurveto{\pgfqpoint{0.191667in}{0.817863in}}{\pgfqpoint{0.197813in}{0.803025in}}{\pgfqpoint{0.208752in}{0.792085in}}%
\pgfpathcurveto{\pgfqpoint{0.219691in}{0.781146in}}{\pgfqpoint{0.234530in}{0.775000in}}{\pgfqpoint{0.250000in}{0.775000in}}%
\pgfpathclose%
\pgfpathmoveto{\pgfqpoint{0.250000in}{0.780833in}}%
\pgfpathcurveto{\pgfqpoint{0.250000in}{0.780833in}}{\pgfqpoint{0.236077in}{0.780833in}}{\pgfqpoint{0.222722in}{0.786365in}}%
\pgfpathcurveto{\pgfqpoint{0.212877in}{0.796210in}}{\pgfqpoint{0.203032in}{0.806055in}}{\pgfqpoint{0.197500in}{0.819410in}}%
\pgfpathcurveto{\pgfqpoint{0.197500in}{0.833333in}}{\pgfqpoint{0.197500in}{0.847256in}}{\pgfqpoint{0.203032in}{0.860611in}}%
\pgfpathcurveto{\pgfqpoint{0.212877in}{0.870456in}}{\pgfqpoint{0.222722in}{0.880302in}}{\pgfqpoint{0.236077in}{0.885833in}}%
\pgfpathcurveto{\pgfqpoint{0.250000in}{0.885833in}}{\pgfqpoint{0.263923in}{0.885833in}}{\pgfqpoint{0.277278in}{0.880302in}}%
\pgfpathcurveto{\pgfqpoint{0.287123in}{0.870456in}}{\pgfqpoint{0.296968in}{0.860611in}}{\pgfqpoint{0.302500in}{0.847256in}}%
\pgfpathcurveto{\pgfqpoint{0.302500in}{0.833333in}}{\pgfqpoint{0.302500in}{0.819410in}}{\pgfqpoint{0.296968in}{0.806055in}}%
\pgfpathcurveto{\pgfqpoint{0.287123in}{0.796210in}}{\pgfqpoint{0.277278in}{0.786365in}}{\pgfqpoint{0.263923in}{0.780833in}}%
\pgfpathclose%
\pgfpathmoveto{\pgfqpoint{0.416667in}{0.775000in}}%
\pgfpathcurveto{\pgfqpoint{0.432137in}{0.775000in}}{\pgfqpoint{0.446975in}{0.781146in}}{\pgfqpoint{0.457915in}{0.792085in}}%
\pgfpathcurveto{\pgfqpoint{0.468854in}{0.803025in}}{\pgfqpoint{0.475000in}{0.817863in}}{\pgfqpoint{0.475000in}{0.833333in}}%
\pgfpathcurveto{\pgfqpoint{0.475000in}{0.848804in}}{\pgfqpoint{0.468854in}{0.863642in}}{\pgfqpoint{0.457915in}{0.874581in}}%
\pgfpathcurveto{\pgfqpoint{0.446975in}{0.885520in}}{\pgfqpoint{0.432137in}{0.891667in}}{\pgfqpoint{0.416667in}{0.891667in}}%
\pgfpathcurveto{\pgfqpoint{0.401196in}{0.891667in}}{\pgfqpoint{0.386358in}{0.885520in}}{\pgfqpoint{0.375419in}{0.874581in}}%
\pgfpathcurveto{\pgfqpoint{0.364480in}{0.863642in}}{\pgfqpoint{0.358333in}{0.848804in}}{\pgfqpoint{0.358333in}{0.833333in}}%
\pgfpathcurveto{\pgfqpoint{0.358333in}{0.817863in}}{\pgfqpoint{0.364480in}{0.803025in}}{\pgfqpoint{0.375419in}{0.792085in}}%
\pgfpathcurveto{\pgfqpoint{0.386358in}{0.781146in}}{\pgfqpoint{0.401196in}{0.775000in}}{\pgfqpoint{0.416667in}{0.775000in}}%
\pgfpathclose%
\pgfpathmoveto{\pgfqpoint{0.416667in}{0.780833in}}%
\pgfpathcurveto{\pgfqpoint{0.416667in}{0.780833in}}{\pgfqpoint{0.402744in}{0.780833in}}{\pgfqpoint{0.389389in}{0.786365in}}%
\pgfpathcurveto{\pgfqpoint{0.379544in}{0.796210in}}{\pgfqpoint{0.369698in}{0.806055in}}{\pgfqpoint{0.364167in}{0.819410in}}%
\pgfpathcurveto{\pgfqpoint{0.364167in}{0.833333in}}{\pgfqpoint{0.364167in}{0.847256in}}{\pgfqpoint{0.369698in}{0.860611in}}%
\pgfpathcurveto{\pgfqpoint{0.379544in}{0.870456in}}{\pgfqpoint{0.389389in}{0.880302in}}{\pgfqpoint{0.402744in}{0.885833in}}%
\pgfpathcurveto{\pgfqpoint{0.416667in}{0.885833in}}{\pgfqpoint{0.430590in}{0.885833in}}{\pgfqpoint{0.443945in}{0.880302in}}%
\pgfpathcurveto{\pgfqpoint{0.453790in}{0.870456in}}{\pgfqpoint{0.463635in}{0.860611in}}{\pgfqpoint{0.469167in}{0.847256in}}%
\pgfpathcurveto{\pgfqpoint{0.469167in}{0.833333in}}{\pgfqpoint{0.469167in}{0.819410in}}{\pgfqpoint{0.463635in}{0.806055in}}%
\pgfpathcurveto{\pgfqpoint{0.453790in}{0.796210in}}{\pgfqpoint{0.443945in}{0.786365in}}{\pgfqpoint{0.430590in}{0.780833in}}%
\pgfpathclose%
\pgfpathmoveto{\pgfqpoint{0.583333in}{0.775000in}}%
\pgfpathcurveto{\pgfqpoint{0.598804in}{0.775000in}}{\pgfqpoint{0.613642in}{0.781146in}}{\pgfqpoint{0.624581in}{0.792085in}}%
\pgfpathcurveto{\pgfqpoint{0.635520in}{0.803025in}}{\pgfqpoint{0.641667in}{0.817863in}}{\pgfqpoint{0.641667in}{0.833333in}}%
\pgfpathcurveto{\pgfqpoint{0.641667in}{0.848804in}}{\pgfqpoint{0.635520in}{0.863642in}}{\pgfqpoint{0.624581in}{0.874581in}}%
\pgfpathcurveto{\pgfqpoint{0.613642in}{0.885520in}}{\pgfqpoint{0.598804in}{0.891667in}}{\pgfqpoint{0.583333in}{0.891667in}}%
\pgfpathcurveto{\pgfqpoint{0.567863in}{0.891667in}}{\pgfqpoint{0.553025in}{0.885520in}}{\pgfqpoint{0.542085in}{0.874581in}}%
\pgfpathcurveto{\pgfqpoint{0.531146in}{0.863642in}}{\pgfqpoint{0.525000in}{0.848804in}}{\pgfqpoint{0.525000in}{0.833333in}}%
\pgfpathcurveto{\pgfqpoint{0.525000in}{0.817863in}}{\pgfqpoint{0.531146in}{0.803025in}}{\pgfqpoint{0.542085in}{0.792085in}}%
\pgfpathcurveto{\pgfqpoint{0.553025in}{0.781146in}}{\pgfqpoint{0.567863in}{0.775000in}}{\pgfqpoint{0.583333in}{0.775000in}}%
\pgfpathclose%
\pgfpathmoveto{\pgfqpoint{0.583333in}{0.780833in}}%
\pgfpathcurveto{\pgfqpoint{0.583333in}{0.780833in}}{\pgfqpoint{0.569410in}{0.780833in}}{\pgfqpoint{0.556055in}{0.786365in}}%
\pgfpathcurveto{\pgfqpoint{0.546210in}{0.796210in}}{\pgfqpoint{0.536365in}{0.806055in}}{\pgfqpoint{0.530833in}{0.819410in}}%
\pgfpathcurveto{\pgfqpoint{0.530833in}{0.833333in}}{\pgfqpoint{0.530833in}{0.847256in}}{\pgfqpoint{0.536365in}{0.860611in}}%
\pgfpathcurveto{\pgfqpoint{0.546210in}{0.870456in}}{\pgfqpoint{0.556055in}{0.880302in}}{\pgfqpoint{0.569410in}{0.885833in}}%
\pgfpathcurveto{\pgfqpoint{0.583333in}{0.885833in}}{\pgfqpoint{0.597256in}{0.885833in}}{\pgfqpoint{0.610611in}{0.880302in}}%
\pgfpathcurveto{\pgfqpoint{0.620456in}{0.870456in}}{\pgfqpoint{0.630302in}{0.860611in}}{\pgfqpoint{0.635833in}{0.847256in}}%
\pgfpathcurveto{\pgfqpoint{0.635833in}{0.833333in}}{\pgfqpoint{0.635833in}{0.819410in}}{\pgfqpoint{0.630302in}{0.806055in}}%
\pgfpathcurveto{\pgfqpoint{0.620456in}{0.796210in}}{\pgfqpoint{0.610611in}{0.786365in}}{\pgfqpoint{0.597256in}{0.780833in}}%
\pgfpathclose%
\pgfpathmoveto{\pgfqpoint{0.750000in}{0.775000in}}%
\pgfpathcurveto{\pgfqpoint{0.765470in}{0.775000in}}{\pgfqpoint{0.780309in}{0.781146in}}{\pgfqpoint{0.791248in}{0.792085in}}%
\pgfpathcurveto{\pgfqpoint{0.802187in}{0.803025in}}{\pgfqpoint{0.808333in}{0.817863in}}{\pgfqpoint{0.808333in}{0.833333in}}%
\pgfpathcurveto{\pgfqpoint{0.808333in}{0.848804in}}{\pgfqpoint{0.802187in}{0.863642in}}{\pgfqpoint{0.791248in}{0.874581in}}%
\pgfpathcurveto{\pgfqpoint{0.780309in}{0.885520in}}{\pgfqpoint{0.765470in}{0.891667in}}{\pgfqpoint{0.750000in}{0.891667in}}%
\pgfpathcurveto{\pgfqpoint{0.734530in}{0.891667in}}{\pgfqpoint{0.719691in}{0.885520in}}{\pgfqpoint{0.708752in}{0.874581in}}%
\pgfpathcurveto{\pgfqpoint{0.697813in}{0.863642in}}{\pgfqpoint{0.691667in}{0.848804in}}{\pgfqpoint{0.691667in}{0.833333in}}%
\pgfpathcurveto{\pgfqpoint{0.691667in}{0.817863in}}{\pgfqpoint{0.697813in}{0.803025in}}{\pgfqpoint{0.708752in}{0.792085in}}%
\pgfpathcurveto{\pgfqpoint{0.719691in}{0.781146in}}{\pgfqpoint{0.734530in}{0.775000in}}{\pgfqpoint{0.750000in}{0.775000in}}%
\pgfpathclose%
\pgfpathmoveto{\pgfqpoint{0.750000in}{0.780833in}}%
\pgfpathcurveto{\pgfqpoint{0.750000in}{0.780833in}}{\pgfqpoint{0.736077in}{0.780833in}}{\pgfqpoint{0.722722in}{0.786365in}}%
\pgfpathcurveto{\pgfqpoint{0.712877in}{0.796210in}}{\pgfqpoint{0.703032in}{0.806055in}}{\pgfqpoint{0.697500in}{0.819410in}}%
\pgfpathcurveto{\pgfqpoint{0.697500in}{0.833333in}}{\pgfqpoint{0.697500in}{0.847256in}}{\pgfqpoint{0.703032in}{0.860611in}}%
\pgfpathcurveto{\pgfqpoint{0.712877in}{0.870456in}}{\pgfqpoint{0.722722in}{0.880302in}}{\pgfqpoint{0.736077in}{0.885833in}}%
\pgfpathcurveto{\pgfqpoint{0.750000in}{0.885833in}}{\pgfqpoint{0.763923in}{0.885833in}}{\pgfqpoint{0.777278in}{0.880302in}}%
\pgfpathcurveto{\pgfqpoint{0.787123in}{0.870456in}}{\pgfqpoint{0.796968in}{0.860611in}}{\pgfqpoint{0.802500in}{0.847256in}}%
\pgfpathcurveto{\pgfqpoint{0.802500in}{0.833333in}}{\pgfqpoint{0.802500in}{0.819410in}}{\pgfqpoint{0.796968in}{0.806055in}}%
\pgfpathcurveto{\pgfqpoint{0.787123in}{0.796210in}}{\pgfqpoint{0.777278in}{0.786365in}}{\pgfqpoint{0.763923in}{0.780833in}}%
\pgfpathclose%
\pgfpathmoveto{\pgfqpoint{0.916667in}{0.775000in}}%
\pgfpathcurveto{\pgfqpoint{0.932137in}{0.775000in}}{\pgfqpoint{0.946975in}{0.781146in}}{\pgfqpoint{0.957915in}{0.792085in}}%
\pgfpathcurveto{\pgfqpoint{0.968854in}{0.803025in}}{\pgfqpoint{0.975000in}{0.817863in}}{\pgfqpoint{0.975000in}{0.833333in}}%
\pgfpathcurveto{\pgfqpoint{0.975000in}{0.848804in}}{\pgfqpoint{0.968854in}{0.863642in}}{\pgfqpoint{0.957915in}{0.874581in}}%
\pgfpathcurveto{\pgfqpoint{0.946975in}{0.885520in}}{\pgfqpoint{0.932137in}{0.891667in}}{\pgfqpoint{0.916667in}{0.891667in}}%
\pgfpathcurveto{\pgfqpoint{0.901196in}{0.891667in}}{\pgfqpoint{0.886358in}{0.885520in}}{\pgfqpoint{0.875419in}{0.874581in}}%
\pgfpathcurveto{\pgfqpoint{0.864480in}{0.863642in}}{\pgfqpoint{0.858333in}{0.848804in}}{\pgfqpoint{0.858333in}{0.833333in}}%
\pgfpathcurveto{\pgfqpoint{0.858333in}{0.817863in}}{\pgfqpoint{0.864480in}{0.803025in}}{\pgfqpoint{0.875419in}{0.792085in}}%
\pgfpathcurveto{\pgfqpoint{0.886358in}{0.781146in}}{\pgfqpoint{0.901196in}{0.775000in}}{\pgfqpoint{0.916667in}{0.775000in}}%
\pgfpathclose%
\pgfpathmoveto{\pgfqpoint{0.916667in}{0.780833in}}%
\pgfpathcurveto{\pgfqpoint{0.916667in}{0.780833in}}{\pgfqpoint{0.902744in}{0.780833in}}{\pgfqpoint{0.889389in}{0.786365in}}%
\pgfpathcurveto{\pgfqpoint{0.879544in}{0.796210in}}{\pgfqpoint{0.869698in}{0.806055in}}{\pgfqpoint{0.864167in}{0.819410in}}%
\pgfpathcurveto{\pgfqpoint{0.864167in}{0.833333in}}{\pgfqpoint{0.864167in}{0.847256in}}{\pgfqpoint{0.869698in}{0.860611in}}%
\pgfpathcurveto{\pgfqpoint{0.879544in}{0.870456in}}{\pgfqpoint{0.889389in}{0.880302in}}{\pgfqpoint{0.902744in}{0.885833in}}%
\pgfpathcurveto{\pgfqpoint{0.916667in}{0.885833in}}{\pgfqpoint{0.930590in}{0.885833in}}{\pgfqpoint{0.943945in}{0.880302in}}%
\pgfpathcurveto{\pgfqpoint{0.953790in}{0.870456in}}{\pgfqpoint{0.963635in}{0.860611in}}{\pgfqpoint{0.969167in}{0.847256in}}%
\pgfpathcurveto{\pgfqpoint{0.969167in}{0.833333in}}{\pgfqpoint{0.969167in}{0.819410in}}{\pgfqpoint{0.963635in}{0.806055in}}%
\pgfpathcurveto{\pgfqpoint{0.953790in}{0.796210in}}{\pgfqpoint{0.943945in}{0.786365in}}{\pgfqpoint{0.930590in}{0.780833in}}%
\pgfpathclose%
\pgfpathmoveto{\pgfqpoint{0.000000in}{0.941667in}}%
\pgfpathcurveto{\pgfqpoint{0.015470in}{0.941667in}}{\pgfqpoint{0.030309in}{0.947813in}}{\pgfqpoint{0.041248in}{0.958752in}}%
\pgfpathcurveto{\pgfqpoint{0.052187in}{0.969691in}}{\pgfqpoint{0.058333in}{0.984530in}}{\pgfqpoint{0.058333in}{1.000000in}}%
\pgfpathcurveto{\pgfqpoint{0.058333in}{1.015470in}}{\pgfqpoint{0.052187in}{1.030309in}}{\pgfqpoint{0.041248in}{1.041248in}}%
\pgfpathcurveto{\pgfqpoint{0.030309in}{1.052187in}}{\pgfqpoint{0.015470in}{1.058333in}}{\pgfqpoint{0.000000in}{1.058333in}}%
\pgfpathcurveto{\pgfqpoint{-0.015470in}{1.058333in}}{\pgfqpoint{-0.030309in}{1.052187in}}{\pgfqpoint{-0.041248in}{1.041248in}}%
\pgfpathcurveto{\pgfqpoint{-0.052187in}{1.030309in}}{\pgfqpoint{-0.058333in}{1.015470in}}{\pgfqpoint{-0.058333in}{1.000000in}}%
\pgfpathcurveto{\pgfqpoint{-0.058333in}{0.984530in}}{\pgfqpoint{-0.052187in}{0.969691in}}{\pgfqpoint{-0.041248in}{0.958752in}}%
\pgfpathcurveto{\pgfqpoint{-0.030309in}{0.947813in}}{\pgfqpoint{-0.015470in}{0.941667in}}{\pgfqpoint{0.000000in}{0.941667in}}%
\pgfpathclose%
\pgfpathmoveto{\pgfqpoint{0.000000in}{0.947500in}}%
\pgfpathcurveto{\pgfqpoint{0.000000in}{0.947500in}}{\pgfqpoint{-0.013923in}{0.947500in}}{\pgfqpoint{-0.027278in}{0.953032in}}%
\pgfpathcurveto{\pgfqpoint{-0.037123in}{0.962877in}}{\pgfqpoint{-0.046968in}{0.972722in}}{\pgfqpoint{-0.052500in}{0.986077in}}%
\pgfpathcurveto{\pgfqpoint{-0.052500in}{1.000000in}}{\pgfqpoint{-0.052500in}{1.013923in}}{\pgfqpoint{-0.046968in}{1.027278in}}%
\pgfpathcurveto{\pgfqpoint{-0.037123in}{1.037123in}}{\pgfqpoint{-0.027278in}{1.046968in}}{\pgfqpoint{-0.013923in}{1.052500in}}%
\pgfpathcurveto{\pgfqpoint{0.000000in}{1.052500in}}{\pgfqpoint{0.013923in}{1.052500in}}{\pgfqpoint{0.027278in}{1.046968in}}%
\pgfpathcurveto{\pgfqpoint{0.037123in}{1.037123in}}{\pgfqpoint{0.046968in}{1.027278in}}{\pgfqpoint{0.052500in}{1.013923in}}%
\pgfpathcurveto{\pgfqpoint{0.052500in}{1.000000in}}{\pgfqpoint{0.052500in}{0.986077in}}{\pgfqpoint{0.046968in}{0.972722in}}%
\pgfpathcurveto{\pgfqpoint{0.037123in}{0.962877in}}{\pgfqpoint{0.027278in}{0.953032in}}{\pgfqpoint{0.013923in}{0.947500in}}%
\pgfpathclose%
\pgfpathmoveto{\pgfqpoint{0.166667in}{0.941667in}}%
\pgfpathcurveto{\pgfqpoint{0.182137in}{0.941667in}}{\pgfqpoint{0.196975in}{0.947813in}}{\pgfqpoint{0.207915in}{0.958752in}}%
\pgfpathcurveto{\pgfqpoint{0.218854in}{0.969691in}}{\pgfqpoint{0.225000in}{0.984530in}}{\pgfqpoint{0.225000in}{1.000000in}}%
\pgfpathcurveto{\pgfqpoint{0.225000in}{1.015470in}}{\pgfqpoint{0.218854in}{1.030309in}}{\pgfqpoint{0.207915in}{1.041248in}}%
\pgfpathcurveto{\pgfqpoint{0.196975in}{1.052187in}}{\pgfqpoint{0.182137in}{1.058333in}}{\pgfqpoint{0.166667in}{1.058333in}}%
\pgfpathcurveto{\pgfqpoint{0.151196in}{1.058333in}}{\pgfqpoint{0.136358in}{1.052187in}}{\pgfqpoint{0.125419in}{1.041248in}}%
\pgfpathcurveto{\pgfqpoint{0.114480in}{1.030309in}}{\pgfqpoint{0.108333in}{1.015470in}}{\pgfqpoint{0.108333in}{1.000000in}}%
\pgfpathcurveto{\pgfqpoint{0.108333in}{0.984530in}}{\pgfqpoint{0.114480in}{0.969691in}}{\pgfqpoint{0.125419in}{0.958752in}}%
\pgfpathcurveto{\pgfqpoint{0.136358in}{0.947813in}}{\pgfqpoint{0.151196in}{0.941667in}}{\pgfqpoint{0.166667in}{0.941667in}}%
\pgfpathclose%
\pgfpathmoveto{\pgfqpoint{0.166667in}{0.947500in}}%
\pgfpathcurveto{\pgfqpoint{0.166667in}{0.947500in}}{\pgfqpoint{0.152744in}{0.947500in}}{\pgfqpoint{0.139389in}{0.953032in}}%
\pgfpathcurveto{\pgfqpoint{0.129544in}{0.962877in}}{\pgfqpoint{0.119698in}{0.972722in}}{\pgfqpoint{0.114167in}{0.986077in}}%
\pgfpathcurveto{\pgfqpoint{0.114167in}{1.000000in}}{\pgfqpoint{0.114167in}{1.013923in}}{\pgfqpoint{0.119698in}{1.027278in}}%
\pgfpathcurveto{\pgfqpoint{0.129544in}{1.037123in}}{\pgfqpoint{0.139389in}{1.046968in}}{\pgfqpoint{0.152744in}{1.052500in}}%
\pgfpathcurveto{\pgfqpoint{0.166667in}{1.052500in}}{\pgfqpoint{0.180590in}{1.052500in}}{\pgfqpoint{0.193945in}{1.046968in}}%
\pgfpathcurveto{\pgfqpoint{0.203790in}{1.037123in}}{\pgfqpoint{0.213635in}{1.027278in}}{\pgfqpoint{0.219167in}{1.013923in}}%
\pgfpathcurveto{\pgfqpoint{0.219167in}{1.000000in}}{\pgfqpoint{0.219167in}{0.986077in}}{\pgfqpoint{0.213635in}{0.972722in}}%
\pgfpathcurveto{\pgfqpoint{0.203790in}{0.962877in}}{\pgfqpoint{0.193945in}{0.953032in}}{\pgfqpoint{0.180590in}{0.947500in}}%
\pgfpathclose%
\pgfpathmoveto{\pgfqpoint{0.333333in}{0.941667in}}%
\pgfpathcurveto{\pgfqpoint{0.348804in}{0.941667in}}{\pgfqpoint{0.363642in}{0.947813in}}{\pgfqpoint{0.374581in}{0.958752in}}%
\pgfpathcurveto{\pgfqpoint{0.385520in}{0.969691in}}{\pgfqpoint{0.391667in}{0.984530in}}{\pgfqpoint{0.391667in}{1.000000in}}%
\pgfpathcurveto{\pgfqpoint{0.391667in}{1.015470in}}{\pgfqpoint{0.385520in}{1.030309in}}{\pgfqpoint{0.374581in}{1.041248in}}%
\pgfpathcurveto{\pgfqpoint{0.363642in}{1.052187in}}{\pgfqpoint{0.348804in}{1.058333in}}{\pgfqpoint{0.333333in}{1.058333in}}%
\pgfpathcurveto{\pgfqpoint{0.317863in}{1.058333in}}{\pgfqpoint{0.303025in}{1.052187in}}{\pgfqpoint{0.292085in}{1.041248in}}%
\pgfpathcurveto{\pgfqpoint{0.281146in}{1.030309in}}{\pgfqpoint{0.275000in}{1.015470in}}{\pgfqpoint{0.275000in}{1.000000in}}%
\pgfpathcurveto{\pgfqpoint{0.275000in}{0.984530in}}{\pgfqpoint{0.281146in}{0.969691in}}{\pgfqpoint{0.292085in}{0.958752in}}%
\pgfpathcurveto{\pgfqpoint{0.303025in}{0.947813in}}{\pgfqpoint{0.317863in}{0.941667in}}{\pgfqpoint{0.333333in}{0.941667in}}%
\pgfpathclose%
\pgfpathmoveto{\pgfqpoint{0.333333in}{0.947500in}}%
\pgfpathcurveto{\pgfqpoint{0.333333in}{0.947500in}}{\pgfqpoint{0.319410in}{0.947500in}}{\pgfqpoint{0.306055in}{0.953032in}}%
\pgfpathcurveto{\pgfqpoint{0.296210in}{0.962877in}}{\pgfqpoint{0.286365in}{0.972722in}}{\pgfqpoint{0.280833in}{0.986077in}}%
\pgfpathcurveto{\pgfqpoint{0.280833in}{1.000000in}}{\pgfqpoint{0.280833in}{1.013923in}}{\pgfqpoint{0.286365in}{1.027278in}}%
\pgfpathcurveto{\pgfqpoint{0.296210in}{1.037123in}}{\pgfqpoint{0.306055in}{1.046968in}}{\pgfqpoint{0.319410in}{1.052500in}}%
\pgfpathcurveto{\pgfqpoint{0.333333in}{1.052500in}}{\pgfqpoint{0.347256in}{1.052500in}}{\pgfqpoint{0.360611in}{1.046968in}}%
\pgfpathcurveto{\pgfqpoint{0.370456in}{1.037123in}}{\pgfqpoint{0.380302in}{1.027278in}}{\pgfqpoint{0.385833in}{1.013923in}}%
\pgfpathcurveto{\pgfqpoint{0.385833in}{1.000000in}}{\pgfqpoint{0.385833in}{0.986077in}}{\pgfqpoint{0.380302in}{0.972722in}}%
\pgfpathcurveto{\pgfqpoint{0.370456in}{0.962877in}}{\pgfqpoint{0.360611in}{0.953032in}}{\pgfqpoint{0.347256in}{0.947500in}}%
\pgfpathclose%
\pgfpathmoveto{\pgfqpoint{0.500000in}{0.941667in}}%
\pgfpathcurveto{\pgfqpoint{0.515470in}{0.941667in}}{\pgfqpoint{0.530309in}{0.947813in}}{\pgfqpoint{0.541248in}{0.958752in}}%
\pgfpathcurveto{\pgfqpoint{0.552187in}{0.969691in}}{\pgfqpoint{0.558333in}{0.984530in}}{\pgfqpoint{0.558333in}{1.000000in}}%
\pgfpathcurveto{\pgfqpoint{0.558333in}{1.015470in}}{\pgfqpoint{0.552187in}{1.030309in}}{\pgfqpoint{0.541248in}{1.041248in}}%
\pgfpathcurveto{\pgfqpoint{0.530309in}{1.052187in}}{\pgfqpoint{0.515470in}{1.058333in}}{\pgfqpoint{0.500000in}{1.058333in}}%
\pgfpathcurveto{\pgfqpoint{0.484530in}{1.058333in}}{\pgfqpoint{0.469691in}{1.052187in}}{\pgfqpoint{0.458752in}{1.041248in}}%
\pgfpathcurveto{\pgfqpoint{0.447813in}{1.030309in}}{\pgfqpoint{0.441667in}{1.015470in}}{\pgfqpoint{0.441667in}{1.000000in}}%
\pgfpathcurveto{\pgfqpoint{0.441667in}{0.984530in}}{\pgfqpoint{0.447813in}{0.969691in}}{\pgfqpoint{0.458752in}{0.958752in}}%
\pgfpathcurveto{\pgfqpoint{0.469691in}{0.947813in}}{\pgfqpoint{0.484530in}{0.941667in}}{\pgfqpoint{0.500000in}{0.941667in}}%
\pgfpathclose%
\pgfpathmoveto{\pgfqpoint{0.500000in}{0.947500in}}%
\pgfpathcurveto{\pgfqpoint{0.500000in}{0.947500in}}{\pgfqpoint{0.486077in}{0.947500in}}{\pgfqpoint{0.472722in}{0.953032in}}%
\pgfpathcurveto{\pgfqpoint{0.462877in}{0.962877in}}{\pgfqpoint{0.453032in}{0.972722in}}{\pgfqpoint{0.447500in}{0.986077in}}%
\pgfpathcurveto{\pgfqpoint{0.447500in}{1.000000in}}{\pgfqpoint{0.447500in}{1.013923in}}{\pgfqpoint{0.453032in}{1.027278in}}%
\pgfpathcurveto{\pgfqpoint{0.462877in}{1.037123in}}{\pgfqpoint{0.472722in}{1.046968in}}{\pgfqpoint{0.486077in}{1.052500in}}%
\pgfpathcurveto{\pgfqpoint{0.500000in}{1.052500in}}{\pgfqpoint{0.513923in}{1.052500in}}{\pgfqpoint{0.527278in}{1.046968in}}%
\pgfpathcurveto{\pgfqpoint{0.537123in}{1.037123in}}{\pgfqpoint{0.546968in}{1.027278in}}{\pgfqpoint{0.552500in}{1.013923in}}%
\pgfpathcurveto{\pgfqpoint{0.552500in}{1.000000in}}{\pgfqpoint{0.552500in}{0.986077in}}{\pgfqpoint{0.546968in}{0.972722in}}%
\pgfpathcurveto{\pgfqpoint{0.537123in}{0.962877in}}{\pgfqpoint{0.527278in}{0.953032in}}{\pgfqpoint{0.513923in}{0.947500in}}%
\pgfpathclose%
\pgfpathmoveto{\pgfqpoint{0.666667in}{0.941667in}}%
\pgfpathcurveto{\pgfqpoint{0.682137in}{0.941667in}}{\pgfqpoint{0.696975in}{0.947813in}}{\pgfqpoint{0.707915in}{0.958752in}}%
\pgfpathcurveto{\pgfqpoint{0.718854in}{0.969691in}}{\pgfqpoint{0.725000in}{0.984530in}}{\pgfqpoint{0.725000in}{1.000000in}}%
\pgfpathcurveto{\pgfqpoint{0.725000in}{1.015470in}}{\pgfqpoint{0.718854in}{1.030309in}}{\pgfqpoint{0.707915in}{1.041248in}}%
\pgfpathcurveto{\pgfqpoint{0.696975in}{1.052187in}}{\pgfqpoint{0.682137in}{1.058333in}}{\pgfqpoint{0.666667in}{1.058333in}}%
\pgfpathcurveto{\pgfqpoint{0.651196in}{1.058333in}}{\pgfqpoint{0.636358in}{1.052187in}}{\pgfqpoint{0.625419in}{1.041248in}}%
\pgfpathcurveto{\pgfqpoint{0.614480in}{1.030309in}}{\pgfqpoint{0.608333in}{1.015470in}}{\pgfqpoint{0.608333in}{1.000000in}}%
\pgfpathcurveto{\pgfqpoint{0.608333in}{0.984530in}}{\pgfqpoint{0.614480in}{0.969691in}}{\pgfqpoint{0.625419in}{0.958752in}}%
\pgfpathcurveto{\pgfqpoint{0.636358in}{0.947813in}}{\pgfqpoint{0.651196in}{0.941667in}}{\pgfqpoint{0.666667in}{0.941667in}}%
\pgfpathclose%
\pgfpathmoveto{\pgfqpoint{0.666667in}{0.947500in}}%
\pgfpathcurveto{\pgfqpoint{0.666667in}{0.947500in}}{\pgfqpoint{0.652744in}{0.947500in}}{\pgfqpoint{0.639389in}{0.953032in}}%
\pgfpathcurveto{\pgfqpoint{0.629544in}{0.962877in}}{\pgfqpoint{0.619698in}{0.972722in}}{\pgfqpoint{0.614167in}{0.986077in}}%
\pgfpathcurveto{\pgfqpoint{0.614167in}{1.000000in}}{\pgfqpoint{0.614167in}{1.013923in}}{\pgfqpoint{0.619698in}{1.027278in}}%
\pgfpathcurveto{\pgfqpoint{0.629544in}{1.037123in}}{\pgfqpoint{0.639389in}{1.046968in}}{\pgfqpoint{0.652744in}{1.052500in}}%
\pgfpathcurveto{\pgfqpoint{0.666667in}{1.052500in}}{\pgfqpoint{0.680590in}{1.052500in}}{\pgfqpoint{0.693945in}{1.046968in}}%
\pgfpathcurveto{\pgfqpoint{0.703790in}{1.037123in}}{\pgfqpoint{0.713635in}{1.027278in}}{\pgfqpoint{0.719167in}{1.013923in}}%
\pgfpathcurveto{\pgfqpoint{0.719167in}{1.000000in}}{\pgfqpoint{0.719167in}{0.986077in}}{\pgfqpoint{0.713635in}{0.972722in}}%
\pgfpathcurveto{\pgfqpoint{0.703790in}{0.962877in}}{\pgfqpoint{0.693945in}{0.953032in}}{\pgfqpoint{0.680590in}{0.947500in}}%
\pgfpathclose%
\pgfpathmoveto{\pgfqpoint{0.833333in}{0.941667in}}%
\pgfpathcurveto{\pgfqpoint{0.848804in}{0.941667in}}{\pgfqpoint{0.863642in}{0.947813in}}{\pgfqpoint{0.874581in}{0.958752in}}%
\pgfpathcurveto{\pgfqpoint{0.885520in}{0.969691in}}{\pgfqpoint{0.891667in}{0.984530in}}{\pgfqpoint{0.891667in}{1.000000in}}%
\pgfpathcurveto{\pgfqpoint{0.891667in}{1.015470in}}{\pgfqpoint{0.885520in}{1.030309in}}{\pgfqpoint{0.874581in}{1.041248in}}%
\pgfpathcurveto{\pgfqpoint{0.863642in}{1.052187in}}{\pgfqpoint{0.848804in}{1.058333in}}{\pgfqpoint{0.833333in}{1.058333in}}%
\pgfpathcurveto{\pgfqpoint{0.817863in}{1.058333in}}{\pgfqpoint{0.803025in}{1.052187in}}{\pgfqpoint{0.792085in}{1.041248in}}%
\pgfpathcurveto{\pgfqpoint{0.781146in}{1.030309in}}{\pgfqpoint{0.775000in}{1.015470in}}{\pgfqpoint{0.775000in}{1.000000in}}%
\pgfpathcurveto{\pgfqpoint{0.775000in}{0.984530in}}{\pgfqpoint{0.781146in}{0.969691in}}{\pgfqpoint{0.792085in}{0.958752in}}%
\pgfpathcurveto{\pgfqpoint{0.803025in}{0.947813in}}{\pgfqpoint{0.817863in}{0.941667in}}{\pgfqpoint{0.833333in}{0.941667in}}%
\pgfpathclose%
\pgfpathmoveto{\pgfqpoint{0.833333in}{0.947500in}}%
\pgfpathcurveto{\pgfqpoint{0.833333in}{0.947500in}}{\pgfqpoint{0.819410in}{0.947500in}}{\pgfqpoint{0.806055in}{0.953032in}}%
\pgfpathcurveto{\pgfqpoint{0.796210in}{0.962877in}}{\pgfqpoint{0.786365in}{0.972722in}}{\pgfqpoint{0.780833in}{0.986077in}}%
\pgfpathcurveto{\pgfqpoint{0.780833in}{1.000000in}}{\pgfqpoint{0.780833in}{1.013923in}}{\pgfqpoint{0.786365in}{1.027278in}}%
\pgfpathcurveto{\pgfqpoint{0.796210in}{1.037123in}}{\pgfqpoint{0.806055in}{1.046968in}}{\pgfqpoint{0.819410in}{1.052500in}}%
\pgfpathcurveto{\pgfqpoint{0.833333in}{1.052500in}}{\pgfqpoint{0.847256in}{1.052500in}}{\pgfqpoint{0.860611in}{1.046968in}}%
\pgfpathcurveto{\pgfqpoint{0.870456in}{1.037123in}}{\pgfqpoint{0.880302in}{1.027278in}}{\pgfqpoint{0.885833in}{1.013923in}}%
\pgfpathcurveto{\pgfqpoint{0.885833in}{1.000000in}}{\pgfqpoint{0.885833in}{0.986077in}}{\pgfqpoint{0.880302in}{0.972722in}}%
\pgfpathcurveto{\pgfqpoint{0.870456in}{0.962877in}}{\pgfqpoint{0.860611in}{0.953032in}}{\pgfqpoint{0.847256in}{0.947500in}}%
\pgfpathclose%
\pgfpathmoveto{\pgfqpoint{1.000000in}{0.941667in}}%
\pgfpathcurveto{\pgfqpoint{1.015470in}{0.941667in}}{\pgfqpoint{1.030309in}{0.947813in}}{\pgfqpoint{1.041248in}{0.958752in}}%
\pgfpathcurveto{\pgfqpoint{1.052187in}{0.969691in}}{\pgfqpoint{1.058333in}{0.984530in}}{\pgfqpoint{1.058333in}{1.000000in}}%
\pgfpathcurveto{\pgfqpoint{1.058333in}{1.015470in}}{\pgfqpoint{1.052187in}{1.030309in}}{\pgfqpoint{1.041248in}{1.041248in}}%
\pgfpathcurveto{\pgfqpoint{1.030309in}{1.052187in}}{\pgfqpoint{1.015470in}{1.058333in}}{\pgfqpoint{1.000000in}{1.058333in}}%
\pgfpathcurveto{\pgfqpoint{0.984530in}{1.058333in}}{\pgfqpoint{0.969691in}{1.052187in}}{\pgfqpoint{0.958752in}{1.041248in}}%
\pgfpathcurveto{\pgfqpoint{0.947813in}{1.030309in}}{\pgfqpoint{0.941667in}{1.015470in}}{\pgfqpoint{0.941667in}{1.000000in}}%
\pgfpathcurveto{\pgfqpoint{0.941667in}{0.984530in}}{\pgfqpoint{0.947813in}{0.969691in}}{\pgfqpoint{0.958752in}{0.958752in}}%
\pgfpathcurveto{\pgfqpoint{0.969691in}{0.947813in}}{\pgfqpoint{0.984530in}{0.941667in}}{\pgfqpoint{1.000000in}{0.941667in}}%
\pgfpathclose%
\pgfpathmoveto{\pgfqpoint{1.000000in}{0.947500in}}%
\pgfpathcurveto{\pgfqpoint{1.000000in}{0.947500in}}{\pgfqpoint{0.986077in}{0.947500in}}{\pgfqpoint{0.972722in}{0.953032in}}%
\pgfpathcurveto{\pgfqpoint{0.962877in}{0.962877in}}{\pgfqpoint{0.953032in}{0.972722in}}{\pgfqpoint{0.947500in}{0.986077in}}%
\pgfpathcurveto{\pgfqpoint{0.947500in}{1.000000in}}{\pgfqpoint{0.947500in}{1.013923in}}{\pgfqpoint{0.953032in}{1.027278in}}%
\pgfpathcurveto{\pgfqpoint{0.962877in}{1.037123in}}{\pgfqpoint{0.972722in}{1.046968in}}{\pgfqpoint{0.986077in}{1.052500in}}%
\pgfpathcurveto{\pgfqpoint{1.000000in}{1.052500in}}{\pgfqpoint{1.013923in}{1.052500in}}{\pgfqpoint{1.027278in}{1.046968in}}%
\pgfpathcurveto{\pgfqpoint{1.037123in}{1.037123in}}{\pgfqpoint{1.046968in}{1.027278in}}{\pgfqpoint{1.052500in}{1.013923in}}%
\pgfpathcurveto{\pgfqpoint{1.052500in}{1.000000in}}{\pgfqpoint{1.052500in}{0.986077in}}{\pgfqpoint{1.046968in}{0.972722in}}%
\pgfpathcurveto{\pgfqpoint{1.037123in}{0.962877in}}{\pgfqpoint{1.027278in}{0.953032in}}{\pgfqpoint{1.013923in}{0.947500in}}%
\pgfpathclose%
\pgfusepath{stroke}%
\end{pgfscope}%
}%
\pgfsys@transformshift{1.290674in}{9.006369in}%
\pgfsys@useobject{currentpattern}{}%
\pgfsys@transformshift{1in}{0in}%
\pgfsys@transformshift{-1in}{0in}%
\pgfsys@transformshift{0in}{1in}%
\end{pgfscope}%
\begin{pgfscope}%
\definecolor{textcolor}{rgb}{0.000000,0.000000,0.000000}%
\pgfsetstrokecolor{textcolor}%
\pgfsetfillcolor{textcolor}%
\pgftext[x=1.912896in,y=9.006369in,left,base]{\color{textcolor}\rmfamily\fontsize{16.000000}{19.200000}\selectfont CW\_STORAGE}%
\end{pgfscope}%
\end{pgfpicture}%
\makeatother%
\endgroup%
}
    \caption[]{Least cost refrigeration capacity.}
    \label{fig:uiuc_chw_cap}
  \end{minipage}
\end{figure}

\section{Alternative Pathways}

Typical formulations of capacity expansion problems use linear programming to optimize
an objective function, usually system cost. However, these models often miss
other appealing options due to unmodeled, structural uncertainty. \gls{mga}
works around this by relaxing the objective function according to a user-defined
slack variable and searching for near-optimal solutions in decision-space. This
section presents the results for \gls{uiuc} using \gls{mga} to explore alternative
pathways to net-zero carbon emissions.

Figure \ref{fig:uiuc_elc_mga} shows the \gls{mga} results for \gls{uiuc}'s electric
sector. The first panel of Figure \ref{fig:uiuc_elc_mga} shows that a one percent
increase in system cost allows \gls{uiuc} to retire Abbott Power Plant's
natural gas turbines fully and does not require nuclear-powered steam turbines
in exchange for slightly higher solar capacity and electricity imports. All other
\gls{mga} and slack values have identical results for the electric sector.
This lack of unique alternatives is possibly the result of the capacity constraints on
wind and solar and the limited set of technology options.

\begin{figure}[H]
  \centering
  \resizebox{0.95\columnwidth}{!}{%% Creator: Matplotlib, PGF backend
%%
%% To include the figure in your LaTeX document, write
%%   \input{<filename>.pgf}
%%
%% Make sure the required packages are loaded in your preamble
%%   \usepackage{pgf}
%%
%% Figures using additional raster images can only be included by \input if
%% they are in the same directory as the main LaTeX file. For loading figures
%% from other directories you can use the `import` package
%%   \usepackage{import}
%%
%% and then include the figures with
%%   \import{<path to file>}{<filename>.pgf}
%%
%% Matplotlib used the following preamble
%%
\begingroup%
\makeatletter%
\begin{pgfpicture}%
\pgfpathrectangle{\pgfpointorigin}{\pgfqpoint{14.040327in}{4.950000in}}%
\pgfusepath{use as bounding box, clip}%
\begin{pgfscope}%
\pgfsetbuttcap%
\pgfsetmiterjoin%
\definecolor{currentfill}{rgb}{1.000000,1.000000,1.000000}%
\pgfsetfillcolor{currentfill}%
\pgfsetlinewidth{0.000000pt}%
\definecolor{currentstroke}{rgb}{0.000000,0.000000,0.000000}%
\pgfsetstrokecolor{currentstroke}%
\pgfsetdash{}{0pt}%
\pgfpathmoveto{\pgfqpoint{0.000000in}{0.000000in}}%
\pgfpathlineto{\pgfqpoint{14.040327in}{0.000000in}}%
\pgfpathlineto{\pgfqpoint{14.040327in}{4.950000in}}%
\pgfpathlineto{\pgfqpoint{0.000000in}{4.950000in}}%
\pgfpathclose%
\pgfusepath{fill}%
\end{pgfscope}%
\begin{pgfscope}%
\pgfsetbuttcap%
\pgfsetmiterjoin%
\definecolor{currentfill}{rgb}{0.898039,0.898039,0.898039}%
\pgfsetfillcolor{currentfill}%
\pgfsetlinewidth{0.000000pt}%
\definecolor{currentstroke}{rgb}{0.000000,0.000000,0.000000}%
\pgfsetstrokecolor{currentstroke}%
\pgfsetstrokeopacity{0.000000}%
\pgfsetdash{}{0pt}%
\pgfpathmoveto{\pgfqpoint{0.759873in}{1.263068in}}%
\pgfpathlineto{\pgfqpoint{3.349611in}{1.263068in}}%
\pgfpathlineto{\pgfqpoint{3.349611in}{4.342817in}}%
\pgfpathlineto{\pgfqpoint{0.759873in}{4.342817in}}%
\pgfpathclose%
\pgfusepath{fill}%
\end{pgfscope}%
\begin{pgfscope}%
\pgfpathrectangle{\pgfqpoint{0.759873in}{1.263068in}}{\pgfqpoint{2.589738in}{3.079750in}}%
\pgfusepath{clip}%
\pgfsetrectcap%
\pgfsetroundjoin%
\pgfsetlinewidth{0.803000pt}%
\definecolor{currentstroke}{rgb}{1.000000,1.000000,1.000000}%
\pgfsetstrokecolor{currentstroke}%
\pgfsetstrokeopacity{0.000000}%
\pgfsetdash{}{0pt}%
\pgfpathmoveto{\pgfqpoint{0.877588in}{1.263068in}}%
\pgfpathlineto{\pgfqpoint{0.877588in}{4.342817in}}%
\pgfusepath{stroke}%
\end{pgfscope}%
\begin{pgfscope}%
\pgfsetbuttcap%
\pgfsetroundjoin%
\definecolor{currentfill}{rgb}{0.333333,0.333333,0.333333}%
\pgfsetfillcolor{currentfill}%
\pgfsetlinewidth{0.803000pt}%
\definecolor{currentstroke}{rgb}{0.333333,0.333333,0.333333}%
\pgfsetstrokecolor{currentstroke}%
\pgfsetdash{}{0pt}%
\pgfsys@defobject{currentmarker}{\pgfqpoint{0.000000in}{-0.048611in}}{\pgfqpoint{0.000000in}{0.000000in}}{%
\pgfpathmoveto{\pgfqpoint{0.000000in}{0.000000in}}%
\pgfpathlineto{\pgfqpoint{0.000000in}{-0.048611in}}%
\pgfusepath{stroke,fill}%
}%
\begin{pgfscope}%
\pgfsys@transformshift{0.877588in}{1.263068in}%
\pgfsys@useobject{currentmarker}{}%
\end{pgfscope}%
\end{pgfscope}%
\begin{pgfscope}%
\definecolor{textcolor}{rgb}{0.333333,0.333333,0.333333}%
\pgfsetstrokecolor{textcolor}%
\pgfsetfillcolor{textcolor}%
\pgftext[x=0.944139in, y=0.210068in, left, base,rotate=90.000000]{\color{textcolor}\rmfamily\fontsize{16.000000}{19.200000}\selectfont mga-0-1\%}%
\end{pgfscope}%
\begin{pgfscope}%
\pgfpathrectangle{\pgfqpoint{0.759873in}{1.263068in}}{\pgfqpoint{2.589738in}{3.079750in}}%
\pgfusepath{clip}%
\pgfsetrectcap%
\pgfsetroundjoin%
\pgfsetlinewidth{0.803000pt}%
\definecolor{currentstroke}{rgb}{1.000000,1.000000,1.000000}%
\pgfsetstrokecolor{currentstroke}%
\pgfsetstrokeopacity{0.000000}%
\pgfsetdash{}{0pt}%
\pgfpathmoveto{\pgfqpoint{1.466165in}{1.263068in}}%
\pgfpathlineto{\pgfqpoint{1.466165in}{4.342817in}}%
\pgfusepath{stroke}%
\end{pgfscope}%
\begin{pgfscope}%
\pgfsetbuttcap%
\pgfsetroundjoin%
\definecolor{currentfill}{rgb}{0.333333,0.333333,0.333333}%
\pgfsetfillcolor{currentfill}%
\pgfsetlinewidth{0.803000pt}%
\definecolor{currentstroke}{rgb}{0.333333,0.333333,0.333333}%
\pgfsetstrokecolor{currentstroke}%
\pgfsetdash{}{0pt}%
\pgfsys@defobject{currentmarker}{\pgfqpoint{0.000000in}{-0.048611in}}{\pgfqpoint{0.000000in}{0.000000in}}{%
\pgfpathmoveto{\pgfqpoint{0.000000in}{0.000000in}}%
\pgfpathlineto{\pgfqpoint{0.000000in}{-0.048611in}}%
\pgfusepath{stroke,fill}%
}%
\begin{pgfscope}%
\pgfsys@transformshift{1.466165in}{1.263068in}%
\pgfsys@useobject{currentmarker}{}%
\end{pgfscope}%
\end{pgfscope}%
\begin{pgfscope}%
\definecolor{textcolor}{rgb}{0.333333,0.333333,0.333333}%
\pgfsetstrokecolor{textcolor}%
\pgfsetfillcolor{textcolor}%
\pgftext[x=1.532716in, y=0.210068in, left, base,rotate=90.000000]{\color{textcolor}\rmfamily\fontsize{16.000000}{19.200000}\selectfont mga-1-1\%}%
\end{pgfscope}%
\begin{pgfscope}%
\pgfpathrectangle{\pgfqpoint{0.759873in}{1.263068in}}{\pgfqpoint{2.589738in}{3.079750in}}%
\pgfusepath{clip}%
\pgfsetrectcap%
\pgfsetroundjoin%
\pgfsetlinewidth{0.803000pt}%
\definecolor{currentstroke}{rgb}{1.000000,1.000000,1.000000}%
\pgfsetstrokecolor{currentstroke}%
\pgfsetstrokeopacity{0.000000}%
\pgfsetdash{}{0pt}%
\pgfpathmoveto{\pgfqpoint{2.054742in}{1.263068in}}%
\pgfpathlineto{\pgfqpoint{2.054742in}{4.342817in}}%
\pgfusepath{stroke}%
\end{pgfscope}%
\begin{pgfscope}%
\pgfsetbuttcap%
\pgfsetroundjoin%
\definecolor{currentfill}{rgb}{0.333333,0.333333,0.333333}%
\pgfsetfillcolor{currentfill}%
\pgfsetlinewidth{0.803000pt}%
\definecolor{currentstroke}{rgb}{0.333333,0.333333,0.333333}%
\pgfsetstrokecolor{currentstroke}%
\pgfsetdash{}{0pt}%
\pgfsys@defobject{currentmarker}{\pgfqpoint{0.000000in}{-0.048611in}}{\pgfqpoint{0.000000in}{0.000000in}}{%
\pgfpathmoveto{\pgfqpoint{0.000000in}{0.000000in}}%
\pgfpathlineto{\pgfqpoint{0.000000in}{-0.048611in}}%
\pgfusepath{stroke,fill}%
}%
\begin{pgfscope}%
\pgfsys@transformshift{2.054742in}{1.263068in}%
\pgfsys@useobject{currentmarker}{}%
\end{pgfscope}%
\end{pgfscope}%
\begin{pgfscope}%
\definecolor{textcolor}{rgb}{0.333333,0.333333,0.333333}%
\pgfsetstrokecolor{textcolor}%
\pgfsetfillcolor{textcolor}%
\pgftext[x=2.121292in, y=0.210068in, left, base,rotate=90.000000]{\color{textcolor}\rmfamily\fontsize{16.000000}{19.200000}\selectfont mga-2-1\%}%
\end{pgfscope}%
\begin{pgfscope}%
\pgfpathrectangle{\pgfqpoint{0.759873in}{1.263068in}}{\pgfqpoint{2.589738in}{3.079750in}}%
\pgfusepath{clip}%
\pgfsetrectcap%
\pgfsetroundjoin%
\pgfsetlinewidth{0.803000pt}%
\definecolor{currentstroke}{rgb}{1.000000,1.000000,1.000000}%
\pgfsetstrokecolor{currentstroke}%
\pgfsetstrokeopacity{0.000000}%
\pgfsetdash{}{0pt}%
\pgfpathmoveto{\pgfqpoint{2.643318in}{1.263068in}}%
\pgfpathlineto{\pgfqpoint{2.643318in}{4.342817in}}%
\pgfusepath{stroke}%
\end{pgfscope}%
\begin{pgfscope}%
\pgfsetbuttcap%
\pgfsetroundjoin%
\definecolor{currentfill}{rgb}{0.333333,0.333333,0.333333}%
\pgfsetfillcolor{currentfill}%
\pgfsetlinewidth{0.803000pt}%
\definecolor{currentstroke}{rgb}{0.333333,0.333333,0.333333}%
\pgfsetstrokecolor{currentstroke}%
\pgfsetdash{}{0pt}%
\pgfsys@defobject{currentmarker}{\pgfqpoint{0.000000in}{-0.048611in}}{\pgfqpoint{0.000000in}{0.000000in}}{%
\pgfpathmoveto{\pgfqpoint{0.000000in}{0.000000in}}%
\pgfpathlineto{\pgfqpoint{0.000000in}{-0.048611in}}%
\pgfusepath{stroke,fill}%
}%
\begin{pgfscope}%
\pgfsys@transformshift{2.643318in}{1.263068in}%
\pgfsys@useobject{currentmarker}{}%
\end{pgfscope}%
\end{pgfscope}%
\begin{pgfscope}%
\definecolor{textcolor}{rgb}{0.333333,0.333333,0.333333}%
\pgfsetstrokecolor{textcolor}%
\pgfsetfillcolor{textcolor}%
\pgftext[x=2.709869in, y=0.210068in, left, base,rotate=90.000000]{\color{textcolor}\rmfamily\fontsize{16.000000}{19.200000}\selectfont mga-3-1\%}%
\end{pgfscope}%
\begin{pgfscope}%
\pgfpathrectangle{\pgfqpoint{0.759873in}{1.263068in}}{\pgfqpoint{2.589738in}{3.079750in}}%
\pgfusepath{clip}%
\pgfsetrectcap%
\pgfsetroundjoin%
\pgfsetlinewidth{0.803000pt}%
\definecolor{currentstroke}{rgb}{1.000000,1.000000,1.000000}%
\pgfsetstrokecolor{currentstroke}%
\pgfsetstrokeopacity{0.000000}%
\pgfsetdash{}{0pt}%
\pgfpathmoveto{\pgfqpoint{3.231895in}{1.263068in}}%
\pgfpathlineto{\pgfqpoint{3.231895in}{4.342817in}}%
\pgfusepath{stroke}%
\end{pgfscope}%
\begin{pgfscope}%
\pgfsetbuttcap%
\pgfsetroundjoin%
\definecolor{currentfill}{rgb}{0.333333,0.333333,0.333333}%
\pgfsetfillcolor{currentfill}%
\pgfsetlinewidth{0.803000pt}%
\definecolor{currentstroke}{rgb}{0.333333,0.333333,0.333333}%
\pgfsetstrokecolor{currentstroke}%
\pgfsetdash{}{0pt}%
\pgfsys@defobject{currentmarker}{\pgfqpoint{0.000000in}{-0.048611in}}{\pgfqpoint{0.000000in}{0.000000in}}{%
\pgfpathmoveto{\pgfqpoint{0.000000in}{0.000000in}}%
\pgfpathlineto{\pgfqpoint{0.000000in}{-0.048611in}}%
\pgfusepath{stroke,fill}%
}%
\begin{pgfscope}%
\pgfsys@transformshift{3.231895in}{1.263068in}%
\pgfsys@useobject{currentmarker}{}%
\end{pgfscope}%
\end{pgfscope}%
\begin{pgfscope}%
\definecolor{textcolor}{rgb}{0.333333,0.333333,0.333333}%
\pgfsetstrokecolor{textcolor}%
\pgfsetfillcolor{textcolor}%
\pgftext[x=3.298446in, y=0.210068in, left, base,rotate=90.000000]{\color{textcolor}\rmfamily\fontsize{16.000000}{19.200000}\selectfont mga-4-1\%}%
\end{pgfscope}%
\begin{pgfscope}%
\pgfpathrectangle{\pgfqpoint{0.759873in}{1.263068in}}{\pgfqpoint{2.589738in}{3.079750in}}%
\pgfusepath{clip}%
\pgfsetrectcap%
\pgfsetroundjoin%
\pgfsetlinewidth{0.803000pt}%
\definecolor{currentstroke}{rgb}{1.000000,1.000000,1.000000}%
\pgfsetstrokecolor{currentstroke}%
\pgfsetdash{}{0pt}%
\pgfpathmoveto{\pgfqpoint{0.759873in}{1.403056in}}%
\pgfpathlineto{\pgfqpoint{3.349611in}{1.403056in}}%
\pgfusepath{stroke}%
\end{pgfscope}%
\begin{pgfscope}%
\pgfsetbuttcap%
\pgfsetroundjoin%
\definecolor{currentfill}{rgb}{0.333333,0.333333,0.333333}%
\pgfsetfillcolor{currentfill}%
\pgfsetlinewidth{0.803000pt}%
\definecolor{currentstroke}{rgb}{0.333333,0.333333,0.333333}%
\pgfsetstrokecolor{currentstroke}%
\pgfsetdash{}{0pt}%
\pgfsys@defobject{currentmarker}{\pgfqpoint{-0.048611in}{0.000000in}}{\pgfqpoint{-0.000000in}{0.000000in}}{%
\pgfpathmoveto{\pgfqpoint{-0.000000in}{0.000000in}}%
\pgfpathlineto{\pgfqpoint{-0.048611in}{0.000000in}}%
\pgfusepath{stroke,fill}%
}%
\begin{pgfscope}%
\pgfsys@transformshift{0.759873in}{1.403056in}%
\pgfsys@useobject{currentmarker}{}%
\end{pgfscope}%
\end{pgfscope}%
\begin{pgfscope}%
\definecolor{textcolor}{rgb}{0.333333,0.333333,0.333333}%
\pgfsetstrokecolor{textcolor}%
\pgfsetfillcolor{textcolor}%
\pgftext[x=0.564735in, y=1.333612in, left, base]{\color{textcolor}\rmfamily\fontsize{14.000000}{16.800000}\selectfont \(\displaystyle {0}\)}%
\end{pgfscope}%
\begin{pgfscope}%
\pgfpathrectangle{\pgfqpoint{0.759873in}{1.263068in}}{\pgfqpoint{2.589738in}{3.079750in}}%
\pgfusepath{clip}%
\pgfsetrectcap%
\pgfsetroundjoin%
\pgfsetlinewidth{0.803000pt}%
\definecolor{currentstroke}{rgb}{1.000000,1.000000,1.000000}%
\pgfsetstrokecolor{currentstroke}%
\pgfsetdash{}{0pt}%
\pgfpathmoveto{\pgfqpoint{0.759873in}{1.858235in}}%
\pgfpathlineto{\pgfqpoint{3.349611in}{1.858235in}}%
\pgfusepath{stroke}%
\end{pgfscope}%
\begin{pgfscope}%
\pgfsetbuttcap%
\pgfsetroundjoin%
\definecolor{currentfill}{rgb}{0.333333,0.333333,0.333333}%
\pgfsetfillcolor{currentfill}%
\pgfsetlinewidth{0.803000pt}%
\definecolor{currentstroke}{rgb}{0.333333,0.333333,0.333333}%
\pgfsetstrokecolor{currentstroke}%
\pgfsetdash{}{0pt}%
\pgfsys@defobject{currentmarker}{\pgfqpoint{-0.048611in}{0.000000in}}{\pgfqpoint{-0.000000in}{0.000000in}}{%
\pgfpathmoveto{\pgfqpoint{-0.000000in}{0.000000in}}%
\pgfpathlineto{\pgfqpoint{-0.048611in}{0.000000in}}%
\pgfusepath{stroke,fill}%
}%
\begin{pgfscope}%
\pgfsys@transformshift{0.759873in}{1.858235in}%
\pgfsys@useobject{currentmarker}{}%
\end{pgfscope}%
\end{pgfscope}%
\begin{pgfscope}%
\definecolor{textcolor}{rgb}{0.333333,0.333333,0.333333}%
\pgfsetstrokecolor{textcolor}%
\pgfsetfillcolor{textcolor}%
\pgftext[x=0.466820in, y=1.788791in, left, base]{\color{textcolor}\rmfamily\fontsize{14.000000}{16.800000}\selectfont \(\displaystyle {20}\)}%
\end{pgfscope}%
\begin{pgfscope}%
\pgfpathrectangle{\pgfqpoint{0.759873in}{1.263068in}}{\pgfqpoint{2.589738in}{3.079750in}}%
\pgfusepath{clip}%
\pgfsetrectcap%
\pgfsetroundjoin%
\pgfsetlinewidth{0.803000pt}%
\definecolor{currentstroke}{rgb}{1.000000,1.000000,1.000000}%
\pgfsetstrokecolor{currentstroke}%
\pgfsetdash{}{0pt}%
\pgfpathmoveto{\pgfqpoint{0.759873in}{2.313414in}}%
\pgfpathlineto{\pgfqpoint{3.349611in}{2.313414in}}%
\pgfusepath{stroke}%
\end{pgfscope}%
\begin{pgfscope}%
\pgfsetbuttcap%
\pgfsetroundjoin%
\definecolor{currentfill}{rgb}{0.333333,0.333333,0.333333}%
\pgfsetfillcolor{currentfill}%
\pgfsetlinewidth{0.803000pt}%
\definecolor{currentstroke}{rgb}{0.333333,0.333333,0.333333}%
\pgfsetstrokecolor{currentstroke}%
\pgfsetdash{}{0pt}%
\pgfsys@defobject{currentmarker}{\pgfqpoint{-0.048611in}{0.000000in}}{\pgfqpoint{-0.000000in}{0.000000in}}{%
\pgfpathmoveto{\pgfqpoint{-0.000000in}{0.000000in}}%
\pgfpathlineto{\pgfqpoint{-0.048611in}{0.000000in}}%
\pgfusepath{stroke,fill}%
}%
\begin{pgfscope}%
\pgfsys@transformshift{0.759873in}{2.313414in}%
\pgfsys@useobject{currentmarker}{}%
\end{pgfscope}%
\end{pgfscope}%
\begin{pgfscope}%
\definecolor{textcolor}{rgb}{0.333333,0.333333,0.333333}%
\pgfsetstrokecolor{textcolor}%
\pgfsetfillcolor{textcolor}%
\pgftext[x=0.466820in, y=2.243969in, left, base]{\color{textcolor}\rmfamily\fontsize{14.000000}{16.800000}\selectfont \(\displaystyle {40}\)}%
\end{pgfscope}%
\begin{pgfscope}%
\pgfpathrectangle{\pgfqpoint{0.759873in}{1.263068in}}{\pgfqpoint{2.589738in}{3.079750in}}%
\pgfusepath{clip}%
\pgfsetrectcap%
\pgfsetroundjoin%
\pgfsetlinewidth{0.803000pt}%
\definecolor{currentstroke}{rgb}{1.000000,1.000000,1.000000}%
\pgfsetstrokecolor{currentstroke}%
\pgfsetdash{}{0pt}%
\pgfpathmoveto{\pgfqpoint{0.759873in}{2.768592in}}%
\pgfpathlineto{\pgfqpoint{3.349611in}{2.768592in}}%
\pgfusepath{stroke}%
\end{pgfscope}%
\begin{pgfscope}%
\pgfsetbuttcap%
\pgfsetroundjoin%
\definecolor{currentfill}{rgb}{0.333333,0.333333,0.333333}%
\pgfsetfillcolor{currentfill}%
\pgfsetlinewidth{0.803000pt}%
\definecolor{currentstroke}{rgb}{0.333333,0.333333,0.333333}%
\pgfsetstrokecolor{currentstroke}%
\pgfsetdash{}{0pt}%
\pgfsys@defobject{currentmarker}{\pgfqpoint{-0.048611in}{0.000000in}}{\pgfqpoint{-0.000000in}{0.000000in}}{%
\pgfpathmoveto{\pgfqpoint{-0.000000in}{0.000000in}}%
\pgfpathlineto{\pgfqpoint{-0.048611in}{0.000000in}}%
\pgfusepath{stroke,fill}%
}%
\begin{pgfscope}%
\pgfsys@transformshift{0.759873in}{2.768592in}%
\pgfsys@useobject{currentmarker}{}%
\end{pgfscope}%
\end{pgfscope}%
\begin{pgfscope}%
\definecolor{textcolor}{rgb}{0.333333,0.333333,0.333333}%
\pgfsetstrokecolor{textcolor}%
\pgfsetfillcolor{textcolor}%
\pgftext[x=0.466820in, y=2.699148in, left, base]{\color{textcolor}\rmfamily\fontsize{14.000000}{16.800000}\selectfont \(\displaystyle {60}\)}%
\end{pgfscope}%
\begin{pgfscope}%
\pgfpathrectangle{\pgfqpoint{0.759873in}{1.263068in}}{\pgfqpoint{2.589738in}{3.079750in}}%
\pgfusepath{clip}%
\pgfsetrectcap%
\pgfsetroundjoin%
\pgfsetlinewidth{0.803000pt}%
\definecolor{currentstroke}{rgb}{1.000000,1.000000,1.000000}%
\pgfsetstrokecolor{currentstroke}%
\pgfsetdash{}{0pt}%
\pgfpathmoveto{\pgfqpoint{0.759873in}{3.223771in}}%
\pgfpathlineto{\pgfqpoint{3.349611in}{3.223771in}}%
\pgfusepath{stroke}%
\end{pgfscope}%
\begin{pgfscope}%
\pgfsetbuttcap%
\pgfsetroundjoin%
\definecolor{currentfill}{rgb}{0.333333,0.333333,0.333333}%
\pgfsetfillcolor{currentfill}%
\pgfsetlinewidth{0.803000pt}%
\definecolor{currentstroke}{rgb}{0.333333,0.333333,0.333333}%
\pgfsetstrokecolor{currentstroke}%
\pgfsetdash{}{0pt}%
\pgfsys@defobject{currentmarker}{\pgfqpoint{-0.048611in}{0.000000in}}{\pgfqpoint{-0.000000in}{0.000000in}}{%
\pgfpathmoveto{\pgfqpoint{-0.000000in}{0.000000in}}%
\pgfpathlineto{\pgfqpoint{-0.048611in}{0.000000in}}%
\pgfusepath{stroke,fill}%
}%
\begin{pgfscope}%
\pgfsys@transformshift{0.759873in}{3.223771in}%
\pgfsys@useobject{currentmarker}{}%
\end{pgfscope}%
\end{pgfscope}%
\begin{pgfscope}%
\definecolor{textcolor}{rgb}{0.333333,0.333333,0.333333}%
\pgfsetstrokecolor{textcolor}%
\pgfsetfillcolor{textcolor}%
\pgftext[x=0.466820in, y=3.154327in, left, base]{\color{textcolor}\rmfamily\fontsize{14.000000}{16.800000}\selectfont \(\displaystyle {80}\)}%
\end{pgfscope}%
\begin{pgfscope}%
\pgfpathrectangle{\pgfqpoint{0.759873in}{1.263068in}}{\pgfqpoint{2.589738in}{3.079750in}}%
\pgfusepath{clip}%
\pgfsetrectcap%
\pgfsetroundjoin%
\pgfsetlinewidth{0.803000pt}%
\definecolor{currentstroke}{rgb}{1.000000,1.000000,1.000000}%
\pgfsetstrokecolor{currentstroke}%
\pgfsetdash{}{0pt}%
\pgfpathmoveto{\pgfqpoint{0.759873in}{3.678950in}}%
\pgfpathlineto{\pgfqpoint{3.349611in}{3.678950in}}%
\pgfusepath{stroke}%
\end{pgfscope}%
\begin{pgfscope}%
\pgfsetbuttcap%
\pgfsetroundjoin%
\definecolor{currentfill}{rgb}{0.333333,0.333333,0.333333}%
\pgfsetfillcolor{currentfill}%
\pgfsetlinewidth{0.803000pt}%
\definecolor{currentstroke}{rgb}{0.333333,0.333333,0.333333}%
\pgfsetstrokecolor{currentstroke}%
\pgfsetdash{}{0pt}%
\pgfsys@defobject{currentmarker}{\pgfqpoint{-0.048611in}{0.000000in}}{\pgfqpoint{-0.000000in}{0.000000in}}{%
\pgfpathmoveto{\pgfqpoint{-0.000000in}{0.000000in}}%
\pgfpathlineto{\pgfqpoint{-0.048611in}{0.000000in}}%
\pgfusepath{stroke,fill}%
}%
\begin{pgfscope}%
\pgfsys@transformshift{0.759873in}{3.678950in}%
\pgfsys@useobject{currentmarker}{}%
\end{pgfscope}%
\end{pgfscope}%
\begin{pgfscope}%
\definecolor{textcolor}{rgb}{0.333333,0.333333,0.333333}%
\pgfsetstrokecolor{textcolor}%
\pgfsetfillcolor{textcolor}%
\pgftext[x=0.368904in, y=3.609505in, left, base]{\color{textcolor}\rmfamily\fontsize{14.000000}{16.800000}\selectfont \(\displaystyle {100}\)}%
\end{pgfscope}%
\begin{pgfscope}%
\pgfpathrectangle{\pgfqpoint{0.759873in}{1.263068in}}{\pgfqpoint{2.589738in}{3.079750in}}%
\pgfusepath{clip}%
\pgfsetrectcap%
\pgfsetroundjoin%
\pgfsetlinewidth{0.803000pt}%
\definecolor{currentstroke}{rgb}{1.000000,1.000000,1.000000}%
\pgfsetstrokecolor{currentstroke}%
\pgfsetdash{}{0pt}%
\pgfpathmoveto{\pgfqpoint{0.759873in}{4.134128in}}%
\pgfpathlineto{\pgfqpoint{3.349611in}{4.134128in}}%
\pgfusepath{stroke}%
\end{pgfscope}%
\begin{pgfscope}%
\pgfsetbuttcap%
\pgfsetroundjoin%
\definecolor{currentfill}{rgb}{0.333333,0.333333,0.333333}%
\pgfsetfillcolor{currentfill}%
\pgfsetlinewidth{0.803000pt}%
\definecolor{currentstroke}{rgb}{0.333333,0.333333,0.333333}%
\pgfsetstrokecolor{currentstroke}%
\pgfsetdash{}{0pt}%
\pgfsys@defobject{currentmarker}{\pgfqpoint{-0.048611in}{0.000000in}}{\pgfqpoint{-0.000000in}{0.000000in}}{%
\pgfpathmoveto{\pgfqpoint{-0.000000in}{0.000000in}}%
\pgfpathlineto{\pgfqpoint{-0.048611in}{0.000000in}}%
\pgfusepath{stroke,fill}%
}%
\begin{pgfscope}%
\pgfsys@transformshift{0.759873in}{4.134128in}%
\pgfsys@useobject{currentmarker}{}%
\end{pgfscope}%
\end{pgfscope}%
\begin{pgfscope}%
\definecolor{textcolor}{rgb}{0.333333,0.333333,0.333333}%
\pgfsetstrokecolor{textcolor}%
\pgfsetfillcolor{textcolor}%
\pgftext[x=0.368904in, y=4.064684in, left, base]{\color{textcolor}\rmfamily\fontsize{14.000000}{16.800000}\selectfont \(\displaystyle {120}\)}%
\end{pgfscope}%
\begin{pgfscope}%
\definecolor{textcolor}{rgb}{0.333333,0.333333,0.333333}%
\pgfsetstrokecolor{textcolor}%
\pgfsetfillcolor{textcolor}%
\pgftext[x=0.313349in,y=2.802942in,,bottom,rotate=90.000000]{\color{textcolor}\rmfamily\fontsize{16.000000}{19.200000}\selectfont MW\(\displaystyle _e\)}%
\end{pgfscope}%
\begin{pgfscope}%
\pgfpathrectangle{\pgfqpoint{0.759873in}{1.263068in}}{\pgfqpoint{2.589738in}{3.079750in}}%
\pgfusepath{clip}%
\pgfsetbuttcap%
\pgfsetroundjoin%
\pgfsetlinewidth{1.505625pt}%
\definecolor{currentstroke}{rgb}{0.839216,0.152941,0.156863}%
\pgfsetstrokecolor{currentstroke}%
\pgfsetdash{{5.550000pt}{2.400000pt}}{0.000000pt}%
\pgfpathmoveto{\pgfqpoint{0.877588in}{4.005751in}}%
\pgfpathlineto{\pgfqpoint{1.466165in}{1.404210in}}%
\pgfpathlineto{\pgfqpoint{2.054742in}{1.404210in}}%
\pgfpathlineto{\pgfqpoint{2.643318in}{1.403056in}}%
\pgfpathlineto{\pgfqpoint{3.231895in}{1.404386in}}%
\pgfusepath{stroke}%
\end{pgfscope}%
\begin{pgfscope}%
\pgfpathrectangle{\pgfqpoint{0.759873in}{1.263068in}}{\pgfqpoint{2.589738in}{3.079750in}}%
\pgfusepath{clip}%
\pgfsetbuttcap%
\pgfsetroundjoin%
\definecolor{currentfill}{rgb}{0.839216,0.152941,0.156863}%
\pgfsetfillcolor{currentfill}%
\pgfsetlinewidth{1.003750pt}%
\definecolor{currentstroke}{rgb}{0.839216,0.152941,0.156863}%
\pgfsetstrokecolor{currentstroke}%
\pgfsetdash{}{0pt}%
\pgfsys@defobject{currentmarker}{\pgfqpoint{-0.041667in}{-0.041667in}}{\pgfqpoint{0.041667in}{0.041667in}}{%
\pgfpathmoveto{\pgfqpoint{0.000000in}{-0.041667in}}%
\pgfpathcurveto{\pgfqpoint{0.011050in}{-0.041667in}}{\pgfqpoint{0.021649in}{-0.037276in}}{\pgfqpoint{0.029463in}{-0.029463in}}%
\pgfpathcurveto{\pgfqpoint{0.037276in}{-0.021649in}}{\pgfqpoint{0.041667in}{-0.011050in}}{\pgfqpoint{0.041667in}{0.000000in}}%
\pgfpathcurveto{\pgfqpoint{0.041667in}{0.011050in}}{\pgfqpoint{0.037276in}{0.021649in}}{\pgfqpoint{0.029463in}{0.029463in}}%
\pgfpathcurveto{\pgfqpoint{0.021649in}{0.037276in}}{\pgfqpoint{0.011050in}{0.041667in}}{\pgfqpoint{0.000000in}{0.041667in}}%
\pgfpathcurveto{\pgfqpoint{-0.011050in}{0.041667in}}{\pgfqpoint{-0.021649in}{0.037276in}}{\pgfqpoint{-0.029463in}{0.029463in}}%
\pgfpathcurveto{\pgfqpoint{-0.037276in}{0.021649in}}{\pgfqpoint{-0.041667in}{0.011050in}}{\pgfqpoint{-0.041667in}{0.000000in}}%
\pgfpathcurveto{\pgfqpoint{-0.041667in}{-0.011050in}}{\pgfqpoint{-0.037276in}{-0.021649in}}{\pgfqpoint{-0.029463in}{-0.029463in}}%
\pgfpathcurveto{\pgfqpoint{-0.021649in}{-0.037276in}}{\pgfqpoint{-0.011050in}{-0.041667in}}{\pgfqpoint{0.000000in}{-0.041667in}}%
\pgfpathclose%
\pgfusepath{stroke,fill}%
}%
\begin{pgfscope}%
\pgfsys@transformshift{0.877588in}{4.005751in}%
\pgfsys@useobject{currentmarker}{}%
\end{pgfscope}%
\begin{pgfscope}%
\pgfsys@transformshift{1.466165in}{1.404210in}%
\pgfsys@useobject{currentmarker}{}%
\end{pgfscope}%
\begin{pgfscope}%
\pgfsys@transformshift{2.054742in}{1.404210in}%
\pgfsys@useobject{currentmarker}{}%
\end{pgfscope}%
\begin{pgfscope}%
\pgfsys@transformshift{2.643318in}{1.403056in}%
\pgfsys@useobject{currentmarker}{}%
\end{pgfscope}%
\begin{pgfscope}%
\pgfsys@transformshift{3.231895in}{1.404386in}%
\pgfsys@useobject{currentmarker}{}%
\end{pgfscope}%
\end{pgfscope}%
\begin{pgfscope}%
\pgfpathrectangle{\pgfqpoint{0.759873in}{1.263068in}}{\pgfqpoint{2.589738in}{3.079750in}}%
\pgfusepath{clip}%
\pgfsetbuttcap%
\pgfsetroundjoin%
\pgfsetlinewidth{1.505625pt}%
\definecolor{currentstroke}{rgb}{0.549020,0.337255,0.294118}%
\pgfsetstrokecolor{currentstroke}%
\pgfsetdash{{5.550000pt}{2.400000pt}}{0.000000pt}%
\pgfpathmoveto{\pgfqpoint{0.877588in}{4.031153in}}%
\pgfpathlineto{\pgfqpoint{1.466165in}{4.197695in}}%
\pgfpathlineto{\pgfqpoint{2.054742in}{4.197695in}}%
\pgfpathlineto{\pgfqpoint{2.643318in}{4.197695in}}%
\pgfpathlineto{\pgfqpoint{3.231895in}{4.197695in}}%
\pgfusepath{stroke}%
\end{pgfscope}%
\begin{pgfscope}%
\pgfpathrectangle{\pgfqpoint{0.759873in}{1.263068in}}{\pgfqpoint{2.589738in}{3.079750in}}%
\pgfusepath{clip}%
\pgfsetbuttcap%
\pgfsetroundjoin%
\definecolor{currentfill}{rgb}{0.549020,0.337255,0.294118}%
\pgfsetfillcolor{currentfill}%
\pgfsetlinewidth{1.003750pt}%
\definecolor{currentstroke}{rgb}{0.549020,0.337255,0.294118}%
\pgfsetstrokecolor{currentstroke}%
\pgfsetdash{}{0pt}%
\pgfsys@defobject{currentmarker}{\pgfqpoint{-0.041667in}{-0.041667in}}{\pgfqpoint{0.041667in}{0.041667in}}{%
\pgfpathmoveto{\pgfqpoint{0.000000in}{-0.041667in}}%
\pgfpathcurveto{\pgfqpoint{0.011050in}{-0.041667in}}{\pgfqpoint{0.021649in}{-0.037276in}}{\pgfqpoint{0.029463in}{-0.029463in}}%
\pgfpathcurveto{\pgfqpoint{0.037276in}{-0.021649in}}{\pgfqpoint{0.041667in}{-0.011050in}}{\pgfqpoint{0.041667in}{0.000000in}}%
\pgfpathcurveto{\pgfqpoint{0.041667in}{0.011050in}}{\pgfqpoint{0.037276in}{0.021649in}}{\pgfqpoint{0.029463in}{0.029463in}}%
\pgfpathcurveto{\pgfqpoint{0.021649in}{0.037276in}}{\pgfqpoint{0.011050in}{0.041667in}}{\pgfqpoint{0.000000in}{0.041667in}}%
\pgfpathcurveto{\pgfqpoint{-0.011050in}{0.041667in}}{\pgfqpoint{-0.021649in}{0.037276in}}{\pgfqpoint{-0.029463in}{0.029463in}}%
\pgfpathcurveto{\pgfqpoint{-0.037276in}{0.021649in}}{\pgfqpoint{-0.041667in}{0.011050in}}{\pgfqpoint{-0.041667in}{0.000000in}}%
\pgfpathcurveto{\pgfqpoint{-0.041667in}{-0.011050in}}{\pgfqpoint{-0.037276in}{-0.021649in}}{\pgfqpoint{-0.029463in}{-0.029463in}}%
\pgfpathcurveto{\pgfqpoint{-0.021649in}{-0.037276in}}{\pgfqpoint{-0.011050in}{-0.041667in}}{\pgfqpoint{0.000000in}{-0.041667in}}%
\pgfpathclose%
\pgfusepath{stroke,fill}%
}%
\begin{pgfscope}%
\pgfsys@transformshift{0.877588in}{4.031153in}%
\pgfsys@useobject{currentmarker}{}%
\end{pgfscope}%
\begin{pgfscope}%
\pgfsys@transformshift{1.466165in}{4.197695in}%
\pgfsys@useobject{currentmarker}{}%
\end{pgfscope}%
\begin{pgfscope}%
\pgfsys@transformshift{2.054742in}{4.197695in}%
\pgfsys@useobject{currentmarker}{}%
\end{pgfscope}%
\begin{pgfscope}%
\pgfsys@transformshift{2.643318in}{4.197695in}%
\pgfsys@useobject{currentmarker}{}%
\end{pgfscope}%
\begin{pgfscope}%
\pgfsys@transformshift{3.231895in}{4.197695in}%
\pgfsys@useobject{currentmarker}{}%
\end{pgfscope}%
\end{pgfscope}%
\begin{pgfscope}%
\pgfpathrectangle{\pgfqpoint{0.759873in}{1.263068in}}{\pgfqpoint{2.589738in}{3.079750in}}%
\pgfusepath{clip}%
\pgfsetbuttcap%
\pgfsetroundjoin%
\pgfsetlinewidth{1.505625pt}%
\definecolor{currentstroke}{rgb}{0.411765,0.411765,0.411765}%
\pgfsetstrokecolor{currentstroke}%
\pgfsetdash{{5.550000pt}{2.400000pt}}{0.000000pt}%
\pgfpathmoveto{\pgfqpoint{0.877588in}{1.403056in}}%
\pgfpathlineto{\pgfqpoint{1.466165in}{1.403056in}}%
\pgfpathlineto{\pgfqpoint{2.054742in}{1.403056in}}%
\pgfpathlineto{\pgfqpoint{2.643318in}{1.403056in}}%
\pgfpathlineto{\pgfqpoint{3.231895in}{1.403056in}}%
\pgfusepath{stroke}%
\end{pgfscope}%
\begin{pgfscope}%
\pgfpathrectangle{\pgfqpoint{0.759873in}{1.263068in}}{\pgfqpoint{2.589738in}{3.079750in}}%
\pgfusepath{clip}%
\pgfsetbuttcap%
\pgfsetroundjoin%
\definecolor{currentfill}{rgb}{0.411765,0.411765,0.411765}%
\pgfsetfillcolor{currentfill}%
\pgfsetlinewidth{1.003750pt}%
\definecolor{currentstroke}{rgb}{0.411765,0.411765,0.411765}%
\pgfsetstrokecolor{currentstroke}%
\pgfsetdash{}{0pt}%
\pgfsys@defobject{currentmarker}{\pgfqpoint{-0.041667in}{-0.041667in}}{\pgfqpoint{0.041667in}{0.041667in}}{%
\pgfpathmoveto{\pgfqpoint{0.000000in}{-0.041667in}}%
\pgfpathcurveto{\pgfqpoint{0.011050in}{-0.041667in}}{\pgfqpoint{0.021649in}{-0.037276in}}{\pgfqpoint{0.029463in}{-0.029463in}}%
\pgfpathcurveto{\pgfqpoint{0.037276in}{-0.021649in}}{\pgfqpoint{0.041667in}{-0.011050in}}{\pgfqpoint{0.041667in}{0.000000in}}%
\pgfpathcurveto{\pgfqpoint{0.041667in}{0.011050in}}{\pgfqpoint{0.037276in}{0.021649in}}{\pgfqpoint{0.029463in}{0.029463in}}%
\pgfpathcurveto{\pgfqpoint{0.021649in}{0.037276in}}{\pgfqpoint{0.011050in}{0.041667in}}{\pgfqpoint{0.000000in}{0.041667in}}%
\pgfpathcurveto{\pgfqpoint{-0.011050in}{0.041667in}}{\pgfqpoint{-0.021649in}{0.037276in}}{\pgfqpoint{-0.029463in}{0.029463in}}%
\pgfpathcurveto{\pgfqpoint{-0.037276in}{0.021649in}}{\pgfqpoint{-0.041667in}{0.011050in}}{\pgfqpoint{-0.041667in}{0.000000in}}%
\pgfpathcurveto{\pgfqpoint{-0.041667in}{-0.011050in}}{\pgfqpoint{-0.037276in}{-0.021649in}}{\pgfqpoint{-0.029463in}{-0.029463in}}%
\pgfpathcurveto{\pgfqpoint{-0.021649in}{-0.037276in}}{\pgfqpoint{-0.011050in}{-0.041667in}}{\pgfqpoint{0.000000in}{-0.041667in}}%
\pgfpathclose%
\pgfusepath{stroke,fill}%
}%
\begin{pgfscope}%
\pgfsys@transformshift{0.877588in}{1.403056in}%
\pgfsys@useobject{currentmarker}{}%
\end{pgfscope}%
\begin{pgfscope}%
\pgfsys@transformshift{1.466165in}{1.403056in}%
\pgfsys@useobject{currentmarker}{}%
\end{pgfscope}%
\begin{pgfscope}%
\pgfsys@transformshift{2.054742in}{1.403056in}%
\pgfsys@useobject{currentmarker}{}%
\end{pgfscope}%
\begin{pgfscope}%
\pgfsys@transformshift{2.643318in}{1.403056in}%
\pgfsys@useobject{currentmarker}{}%
\end{pgfscope}%
\begin{pgfscope}%
\pgfsys@transformshift{3.231895in}{1.403056in}%
\pgfsys@useobject{currentmarker}{}%
\end{pgfscope}%
\end{pgfscope}%
\begin{pgfscope}%
\pgfpathrectangle{\pgfqpoint{0.759873in}{1.263068in}}{\pgfqpoint{2.589738in}{3.079750in}}%
\pgfusepath{clip}%
\pgfsetbuttcap%
\pgfsetroundjoin%
\pgfsetlinewidth{1.505625pt}%
\definecolor{currentstroke}{rgb}{0.172549,0.627451,0.172549}%
\pgfsetstrokecolor{currentstroke}%
\pgfsetdash{{5.550000pt}{2.400000pt}}{0.000000pt}%
\pgfpathmoveto{\pgfqpoint{0.877588in}{2.075567in}}%
\pgfpathlineto{\pgfqpoint{1.466165in}{1.403056in}}%
\pgfpathlineto{\pgfqpoint{2.054742in}{1.403056in}}%
\pgfpathlineto{\pgfqpoint{2.643318in}{1.405909in}}%
\pgfpathlineto{\pgfqpoint{3.231895in}{1.403056in}}%
\pgfusepath{stroke}%
\end{pgfscope}%
\begin{pgfscope}%
\pgfpathrectangle{\pgfqpoint{0.759873in}{1.263068in}}{\pgfqpoint{2.589738in}{3.079750in}}%
\pgfusepath{clip}%
\pgfsetbuttcap%
\pgfsetroundjoin%
\definecolor{currentfill}{rgb}{0.172549,0.627451,0.172549}%
\pgfsetfillcolor{currentfill}%
\pgfsetlinewidth{1.003750pt}%
\definecolor{currentstroke}{rgb}{0.172549,0.627451,0.172549}%
\pgfsetstrokecolor{currentstroke}%
\pgfsetdash{}{0pt}%
\pgfsys@defobject{currentmarker}{\pgfqpoint{-0.041667in}{-0.041667in}}{\pgfqpoint{0.041667in}{0.041667in}}{%
\pgfpathmoveto{\pgfqpoint{0.000000in}{-0.041667in}}%
\pgfpathcurveto{\pgfqpoint{0.011050in}{-0.041667in}}{\pgfqpoint{0.021649in}{-0.037276in}}{\pgfqpoint{0.029463in}{-0.029463in}}%
\pgfpathcurveto{\pgfqpoint{0.037276in}{-0.021649in}}{\pgfqpoint{0.041667in}{-0.011050in}}{\pgfqpoint{0.041667in}{0.000000in}}%
\pgfpathcurveto{\pgfqpoint{0.041667in}{0.011050in}}{\pgfqpoint{0.037276in}{0.021649in}}{\pgfqpoint{0.029463in}{0.029463in}}%
\pgfpathcurveto{\pgfqpoint{0.021649in}{0.037276in}}{\pgfqpoint{0.011050in}{0.041667in}}{\pgfqpoint{0.000000in}{0.041667in}}%
\pgfpathcurveto{\pgfqpoint{-0.011050in}{0.041667in}}{\pgfqpoint{-0.021649in}{0.037276in}}{\pgfqpoint{-0.029463in}{0.029463in}}%
\pgfpathcurveto{\pgfqpoint{-0.037276in}{0.021649in}}{\pgfqpoint{-0.041667in}{0.011050in}}{\pgfqpoint{-0.041667in}{0.000000in}}%
\pgfpathcurveto{\pgfqpoint{-0.041667in}{-0.011050in}}{\pgfqpoint{-0.037276in}{-0.021649in}}{\pgfqpoint{-0.029463in}{-0.029463in}}%
\pgfpathcurveto{\pgfqpoint{-0.021649in}{-0.037276in}}{\pgfqpoint{-0.011050in}{-0.041667in}}{\pgfqpoint{0.000000in}{-0.041667in}}%
\pgfpathclose%
\pgfusepath{stroke,fill}%
}%
\begin{pgfscope}%
\pgfsys@transformshift{0.877588in}{2.075567in}%
\pgfsys@useobject{currentmarker}{}%
\end{pgfscope}%
\begin{pgfscope}%
\pgfsys@transformshift{1.466165in}{1.403056in}%
\pgfsys@useobject{currentmarker}{}%
\end{pgfscope}%
\begin{pgfscope}%
\pgfsys@transformshift{2.054742in}{1.403056in}%
\pgfsys@useobject{currentmarker}{}%
\end{pgfscope}%
\begin{pgfscope}%
\pgfsys@transformshift{2.643318in}{1.405909in}%
\pgfsys@useobject{currentmarker}{}%
\end{pgfscope}%
\begin{pgfscope}%
\pgfsys@transformshift{3.231895in}{1.403056in}%
\pgfsys@useobject{currentmarker}{}%
\end{pgfscope}%
\end{pgfscope}%
\begin{pgfscope}%
\pgfpathrectangle{\pgfqpoint{0.759873in}{1.263068in}}{\pgfqpoint{2.589738in}{3.079750in}}%
\pgfusepath{clip}%
\pgfsetbuttcap%
\pgfsetroundjoin%
\pgfsetlinewidth{1.505625pt}%
\definecolor{currentstroke}{rgb}{1.000000,1.000000,0.000000}%
\pgfsetstrokecolor{currentstroke}%
\pgfsetdash{{5.550000pt}{2.400000pt}}{0.000000pt}%
\pgfpathmoveto{\pgfqpoint{0.877588in}{2.704346in}}%
\pgfpathlineto{\pgfqpoint{1.466165in}{2.846531in}}%
\pgfpathlineto{\pgfqpoint{2.054742in}{2.845535in}}%
\pgfpathlineto{\pgfqpoint{2.643318in}{2.846531in}}%
\pgfpathlineto{\pgfqpoint{3.231895in}{2.846531in}}%
\pgfusepath{stroke}%
\end{pgfscope}%
\begin{pgfscope}%
\pgfpathrectangle{\pgfqpoint{0.759873in}{1.263068in}}{\pgfqpoint{2.589738in}{3.079750in}}%
\pgfusepath{clip}%
\pgfsetbuttcap%
\pgfsetroundjoin%
\definecolor{currentfill}{rgb}{1.000000,1.000000,0.000000}%
\pgfsetfillcolor{currentfill}%
\pgfsetlinewidth{1.003750pt}%
\definecolor{currentstroke}{rgb}{1.000000,1.000000,0.000000}%
\pgfsetstrokecolor{currentstroke}%
\pgfsetdash{}{0pt}%
\pgfsys@defobject{currentmarker}{\pgfqpoint{-0.041667in}{-0.041667in}}{\pgfqpoint{0.041667in}{0.041667in}}{%
\pgfpathmoveto{\pgfqpoint{0.000000in}{-0.041667in}}%
\pgfpathcurveto{\pgfqpoint{0.011050in}{-0.041667in}}{\pgfqpoint{0.021649in}{-0.037276in}}{\pgfqpoint{0.029463in}{-0.029463in}}%
\pgfpathcurveto{\pgfqpoint{0.037276in}{-0.021649in}}{\pgfqpoint{0.041667in}{-0.011050in}}{\pgfqpoint{0.041667in}{0.000000in}}%
\pgfpathcurveto{\pgfqpoint{0.041667in}{0.011050in}}{\pgfqpoint{0.037276in}{0.021649in}}{\pgfqpoint{0.029463in}{0.029463in}}%
\pgfpathcurveto{\pgfqpoint{0.021649in}{0.037276in}}{\pgfqpoint{0.011050in}{0.041667in}}{\pgfqpoint{0.000000in}{0.041667in}}%
\pgfpathcurveto{\pgfqpoint{-0.011050in}{0.041667in}}{\pgfqpoint{-0.021649in}{0.037276in}}{\pgfqpoint{-0.029463in}{0.029463in}}%
\pgfpathcurveto{\pgfqpoint{-0.037276in}{0.021649in}}{\pgfqpoint{-0.041667in}{0.011050in}}{\pgfqpoint{-0.041667in}{0.000000in}}%
\pgfpathcurveto{\pgfqpoint{-0.041667in}{-0.011050in}}{\pgfqpoint{-0.037276in}{-0.021649in}}{\pgfqpoint{-0.029463in}{-0.029463in}}%
\pgfpathcurveto{\pgfqpoint{-0.021649in}{-0.037276in}}{\pgfqpoint{-0.011050in}{-0.041667in}}{\pgfqpoint{0.000000in}{-0.041667in}}%
\pgfpathclose%
\pgfusepath{stroke,fill}%
}%
\begin{pgfscope}%
\pgfsys@transformshift{0.877588in}{2.704346in}%
\pgfsys@useobject{currentmarker}{}%
\end{pgfscope}%
\begin{pgfscope}%
\pgfsys@transformshift{1.466165in}{2.846531in}%
\pgfsys@useobject{currentmarker}{}%
\end{pgfscope}%
\begin{pgfscope}%
\pgfsys@transformshift{2.054742in}{2.845535in}%
\pgfsys@useobject{currentmarker}{}%
\end{pgfscope}%
\begin{pgfscope}%
\pgfsys@transformshift{2.643318in}{2.846531in}%
\pgfsys@useobject{currentmarker}{}%
\end{pgfscope}%
\begin{pgfscope}%
\pgfsys@transformshift{3.231895in}{2.846531in}%
\pgfsys@useobject{currentmarker}{}%
\end{pgfscope}%
\end{pgfscope}%
\begin{pgfscope}%
\pgfpathrectangle{\pgfqpoint{0.759873in}{1.263068in}}{\pgfqpoint{2.589738in}{3.079750in}}%
\pgfusepath{clip}%
\pgfsetbuttcap%
\pgfsetroundjoin%
\pgfsetlinewidth{1.505625pt}%
\definecolor{currentstroke}{rgb}{0.121569,0.466667,0.705882}%
\pgfsetstrokecolor{currentstroke}%
\pgfsetdash{{5.550000pt}{2.400000pt}}{0.000000pt}%
\pgfpathmoveto{\pgfqpoint{0.877588in}{3.690329in}}%
\pgfpathlineto{\pgfqpoint{1.466165in}{3.690329in}}%
\pgfpathlineto{\pgfqpoint{2.054742in}{3.690329in}}%
\pgfpathlineto{\pgfqpoint{2.643318in}{3.690329in}}%
\pgfpathlineto{\pgfqpoint{3.231895in}{3.690329in}}%
\pgfusepath{stroke}%
\end{pgfscope}%
\begin{pgfscope}%
\pgfpathrectangle{\pgfqpoint{0.759873in}{1.263068in}}{\pgfqpoint{2.589738in}{3.079750in}}%
\pgfusepath{clip}%
\pgfsetbuttcap%
\pgfsetroundjoin%
\definecolor{currentfill}{rgb}{0.121569,0.466667,0.705882}%
\pgfsetfillcolor{currentfill}%
\pgfsetlinewidth{1.003750pt}%
\definecolor{currentstroke}{rgb}{0.121569,0.466667,0.705882}%
\pgfsetstrokecolor{currentstroke}%
\pgfsetdash{}{0pt}%
\pgfsys@defobject{currentmarker}{\pgfqpoint{-0.041667in}{-0.041667in}}{\pgfqpoint{0.041667in}{0.041667in}}{%
\pgfpathmoveto{\pgfqpoint{0.000000in}{-0.041667in}}%
\pgfpathcurveto{\pgfqpoint{0.011050in}{-0.041667in}}{\pgfqpoint{0.021649in}{-0.037276in}}{\pgfqpoint{0.029463in}{-0.029463in}}%
\pgfpathcurveto{\pgfqpoint{0.037276in}{-0.021649in}}{\pgfqpoint{0.041667in}{-0.011050in}}{\pgfqpoint{0.041667in}{0.000000in}}%
\pgfpathcurveto{\pgfqpoint{0.041667in}{0.011050in}}{\pgfqpoint{0.037276in}{0.021649in}}{\pgfqpoint{0.029463in}{0.029463in}}%
\pgfpathcurveto{\pgfqpoint{0.021649in}{0.037276in}}{\pgfqpoint{0.011050in}{0.041667in}}{\pgfqpoint{0.000000in}{0.041667in}}%
\pgfpathcurveto{\pgfqpoint{-0.011050in}{0.041667in}}{\pgfqpoint{-0.021649in}{0.037276in}}{\pgfqpoint{-0.029463in}{0.029463in}}%
\pgfpathcurveto{\pgfqpoint{-0.037276in}{0.021649in}}{\pgfqpoint{-0.041667in}{0.011050in}}{\pgfqpoint{-0.041667in}{0.000000in}}%
\pgfpathcurveto{\pgfqpoint{-0.041667in}{-0.011050in}}{\pgfqpoint{-0.037276in}{-0.021649in}}{\pgfqpoint{-0.029463in}{-0.029463in}}%
\pgfpathcurveto{\pgfqpoint{-0.021649in}{-0.037276in}}{\pgfqpoint{-0.011050in}{-0.041667in}}{\pgfqpoint{0.000000in}{-0.041667in}}%
\pgfpathclose%
\pgfusepath{stroke,fill}%
}%
\begin{pgfscope}%
\pgfsys@transformshift{0.877588in}{3.690329in}%
\pgfsys@useobject{currentmarker}{}%
\end{pgfscope}%
\begin{pgfscope}%
\pgfsys@transformshift{1.466165in}{3.690329in}%
\pgfsys@useobject{currentmarker}{}%
\end{pgfscope}%
\begin{pgfscope}%
\pgfsys@transformshift{2.054742in}{3.690329in}%
\pgfsys@useobject{currentmarker}{}%
\end{pgfscope}%
\begin{pgfscope}%
\pgfsys@transformshift{2.643318in}{3.690329in}%
\pgfsys@useobject{currentmarker}{}%
\end{pgfscope}%
\begin{pgfscope}%
\pgfsys@transformshift{3.231895in}{3.690329in}%
\pgfsys@useobject{currentmarker}{}%
\end{pgfscope}%
\end{pgfscope}%
\begin{pgfscope}%
\pgfsetrectcap%
\pgfsetmiterjoin%
\pgfsetlinewidth{1.003750pt}%
\definecolor{currentstroke}{rgb}{1.000000,1.000000,1.000000}%
\pgfsetstrokecolor{currentstroke}%
\pgfsetdash{}{0pt}%
\pgfpathmoveto{\pgfqpoint{0.759873in}{1.263067in}}%
\pgfpathlineto{\pgfqpoint{0.759873in}{4.342817in}}%
\pgfusepath{stroke}%
\end{pgfscope}%
\begin{pgfscope}%
\pgfsetrectcap%
\pgfsetmiterjoin%
\pgfsetlinewidth{1.003750pt}%
\definecolor{currentstroke}{rgb}{1.000000,1.000000,1.000000}%
\pgfsetstrokecolor{currentstroke}%
\pgfsetdash{}{0pt}%
\pgfpathmoveto{\pgfqpoint{3.349611in}{1.263067in}}%
\pgfpathlineto{\pgfqpoint{3.349611in}{4.342817in}}%
\pgfusepath{stroke}%
\end{pgfscope}%
\begin{pgfscope}%
\pgfsetrectcap%
\pgfsetmiterjoin%
\pgfsetlinewidth{1.003750pt}%
\definecolor{currentstroke}{rgb}{1.000000,1.000000,1.000000}%
\pgfsetstrokecolor{currentstroke}%
\pgfsetdash{}{0pt}%
\pgfpathmoveto{\pgfqpoint{0.759873in}{1.263068in}}%
\pgfpathlineto{\pgfqpoint{3.349611in}{1.263068in}}%
\pgfusepath{stroke}%
\end{pgfscope}%
\begin{pgfscope}%
\pgfsetrectcap%
\pgfsetmiterjoin%
\pgfsetlinewidth{1.003750pt}%
\definecolor{currentstroke}{rgb}{1.000000,1.000000,1.000000}%
\pgfsetstrokecolor{currentstroke}%
\pgfsetdash{}{0pt}%
\pgfpathmoveto{\pgfqpoint{0.759873in}{4.342817in}}%
\pgfpathlineto{\pgfqpoint{3.349611in}{4.342817in}}%
\pgfusepath{stroke}%
\end{pgfscope}%
\begin{pgfscope}%
\pgfsetbuttcap%
\pgfsetmiterjoin%
\definecolor{currentfill}{rgb}{0.898039,0.898039,0.898039}%
\pgfsetfillcolor{currentfill}%
\pgfsetlinewidth{0.000000pt}%
\definecolor{currentstroke}{rgb}{0.000000,0.000000,0.000000}%
\pgfsetstrokecolor{currentstroke}%
\pgfsetstrokeopacity{0.000000}%
\pgfsetdash{}{0pt}%
\pgfpathmoveto{\pgfqpoint{3.569557in}{1.263068in}}%
\pgfpathlineto{\pgfqpoint{6.159295in}{1.263068in}}%
\pgfpathlineto{\pgfqpoint{6.159295in}{4.342817in}}%
\pgfpathlineto{\pgfqpoint{3.569557in}{4.342817in}}%
\pgfpathclose%
\pgfusepath{fill}%
\end{pgfscope}%
\begin{pgfscope}%
\pgfpathrectangle{\pgfqpoint{3.569557in}{1.263068in}}{\pgfqpoint{2.589738in}{3.079750in}}%
\pgfusepath{clip}%
\pgfsetrectcap%
\pgfsetroundjoin%
\pgfsetlinewidth{0.803000pt}%
\definecolor{currentstroke}{rgb}{1.000000,1.000000,1.000000}%
\pgfsetstrokecolor{currentstroke}%
\pgfsetstrokeopacity{0.000000}%
\pgfsetdash{}{0pt}%
\pgfpathmoveto{\pgfqpoint{3.687273in}{1.263068in}}%
\pgfpathlineto{\pgfqpoint{3.687273in}{4.342817in}}%
\pgfusepath{stroke}%
\end{pgfscope}%
\begin{pgfscope}%
\pgfsetbuttcap%
\pgfsetroundjoin%
\definecolor{currentfill}{rgb}{0.333333,0.333333,0.333333}%
\pgfsetfillcolor{currentfill}%
\pgfsetlinewidth{0.803000pt}%
\definecolor{currentstroke}{rgb}{0.333333,0.333333,0.333333}%
\pgfsetstrokecolor{currentstroke}%
\pgfsetdash{}{0pt}%
\pgfsys@defobject{currentmarker}{\pgfqpoint{0.000000in}{-0.048611in}}{\pgfqpoint{0.000000in}{0.000000in}}{%
\pgfpathmoveto{\pgfqpoint{0.000000in}{0.000000in}}%
\pgfpathlineto{\pgfqpoint{0.000000in}{-0.048611in}}%
\pgfusepath{stroke,fill}%
}%
\begin{pgfscope}%
\pgfsys@transformshift{3.687273in}{1.263068in}%
\pgfsys@useobject{currentmarker}{}%
\end{pgfscope}%
\end{pgfscope}%
\begin{pgfscope}%
\definecolor{textcolor}{rgb}{0.333333,0.333333,0.333333}%
\pgfsetstrokecolor{textcolor}%
\pgfsetfillcolor{textcolor}%
\pgftext[x=3.753824in, y=0.210068in, left, base,rotate=90.000000]{\color{textcolor}\rmfamily\fontsize{16.000000}{19.200000}\selectfont mga-0-5\%}%
\end{pgfscope}%
\begin{pgfscope}%
\pgfpathrectangle{\pgfqpoint{3.569557in}{1.263068in}}{\pgfqpoint{2.589738in}{3.079750in}}%
\pgfusepath{clip}%
\pgfsetrectcap%
\pgfsetroundjoin%
\pgfsetlinewidth{0.803000pt}%
\definecolor{currentstroke}{rgb}{1.000000,1.000000,1.000000}%
\pgfsetstrokecolor{currentstroke}%
\pgfsetstrokeopacity{0.000000}%
\pgfsetdash{}{0pt}%
\pgfpathmoveto{\pgfqpoint{4.275849in}{1.263068in}}%
\pgfpathlineto{\pgfqpoint{4.275849in}{4.342817in}}%
\pgfusepath{stroke}%
\end{pgfscope}%
\begin{pgfscope}%
\pgfsetbuttcap%
\pgfsetroundjoin%
\definecolor{currentfill}{rgb}{0.333333,0.333333,0.333333}%
\pgfsetfillcolor{currentfill}%
\pgfsetlinewidth{0.803000pt}%
\definecolor{currentstroke}{rgb}{0.333333,0.333333,0.333333}%
\pgfsetstrokecolor{currentstroke}%
\pgfsetdash{}{0pt}%
\pgfsys@defobject{currentmarker}{\pgfqpoint{0.000000in}{-0.048611in}}{\pgfqpoint{0.000000in}{0.000000in}}{%
\pgfpathmoveto{\pgfqpoint{0.000000in}{0.000000in}}%
\pgfpathlineto{\pgfqpoint{0.000000in}{-0.048611in}}%
\pgfusepath{stroke,fill}%
}%
\begin{pgfscope}%
\pgfsys@transformshift{4.275849in}{1.263068in}%
\pgfsys@useobject{currentmarker}{}%
\end{pgfscope}%
\end{pgfscope}%
\begin{pgfscope}%
\definecolor{textcolor}{rgb}{0.333333,0.333333,0.333333}%
\pgfsetstrokecolor{textcolor}%
\pgfsetfillcolor{textcolor}%
\pgftext[x=4.342400in, y=0.210068in, left, base,rotate=90.000000]{\color{textcolor}\rmfamily\fontsize{16.000000}{19.200000}\selectfont mga-1-5\%}%
\end{pgfscope}%
\begin{pgfscope}%
\pgfpathrectangle{\pgfqpoint{3.569557in}{1.263068in}}{\pgfqpoint{2.589738in}{3.079750in}}%
\pgfusepath{clip}%
\pgfsetrectcap%
\pgfsetroundjoin%
\pgfsetlinewidth{0.803000pt}%
\definecolor{currentstroke}{rgb}{1.000000,1.000000,1.000000}%
\pgfsetstrokecolor{currentstroke}%
\pgfsetstrokeopacity{0.000000}%
\pgfsetdash{}{0pt}%
\pgfpathmoveto{\pgfqpoint{4.864426in}{1.263068in}}%
\pgfpathlineto{\pgfqpoint{4.864426in}{4.342817in}}%
\pgfusepath{stroke}%
\end{pgfscope}%
\begin{pgfscope}%
\pgfsetbuttcap%
\pgfsetroundjoin%
\definecolor{currentfill}{rgb}{0.333333,0.333333,0.333333}%
\pgfsetfillcolor{currentfill}%
\pgfsetlinewidth{0.803000pt}%
\definecolor{currentstroke}{rgb}{0.333333,0.333333,0.333333}%
\pgfsetstrokecolor{currentstroke}%
\pgfsetdash{}{0pt}%
\pgfsys@defobject{currentmarker}{\pgfqpoint{0.000000in}{-0.048611in}}{\pgfqpoint{0.000000in}{0.000000in}}{%
\pgfpathmoveto{\pgfqpoint{0.000000in}{0.000000in}}%
\pgfpathlineto{\pgfqpoint{0.000000in}{-0.048611in}}%
\pgfusepath{stroke,fill}%
}%
\begin{pgfscope}%
\pgfsys@transformshift{4.864426in}{1.263068in}%
\pgfsys@useobject{currentmarker}{}%
\end{pgfscope}%
\end{pgfscope}%
\begin{pgfscope}%
\definecolor{textcolor}{rgb}{0.333333,0.333333,0.333333}%
\pgfsetstrokecolor{textcolor}%
\pgfsetfillcolor{textcolor}%
\pgftext[x=4.930977in, y=0.210068in, left, base,rotate=90.000000]{\color{textcolor}\rmfamily\fontsize{16.000000}{19.200000}\selectfont mga-2-5\%}%
\end{pgfscope}%
\begin{pgfscope}%
\pgfpathrectangle{\pgfqpoint{3.569557in}{1.263068in}}{\pgfqpoint{2.589738in}{3.079750in}}%
\pgfusepath{clip}%
\pgfsetrectcap%
\pgfsetroundjoin%
\pgfsetlinewidth{0.803000pt}%
\definecolor{currentstroke}{rgb}{1.000000,1.000000,1.000000}%
\pgfsetstrokecolor{currentstroke}%
\pgfsetstrokeopacity{0.000000}%
\pgfsetdash{}{0pt}%
\pgfpathmoveto{\pgfqpoint{5.453003in}{1.263068in}}%
\pgfpathlineto{\pgfqpoint{5.453003in}{4.342817in}}%
\pgfusepath{stroke}%
\end{pgfscope}%
\begin{pgfscope}%
\pgfsetbuttcap%
\pgfsetroundjoin%
\definecolor{currentfill}{rgb}{0.333333,0.333333,0.333333}%
\pgfsetfillcolor{currentfill}%
\pgfsetlinewidth{0.803000pt}%
\definecolor{currentstroke}{rgb}{0.333333,0.333333,0.333333}%
\pgfsetstrokecolor{currentstroke}%
\pgfsetdash{}{0pt}%
\pgfsys@defobject{currentmarker}{\pgfqpoint{0.000000in}{-0.048611in}}{\pgfqpoint{0.000000in}{0.000000in}}{%
\pgfpathmoveto{\pgfqpoint{0.000000in}{0.000000in}}%
\pgfpathlineto{\pgfqpoint{0.000000in}{-0.048611in}}%
\pgfusepath{stroke,fill}%
}%
\begin{pgfscope}%
\pgfsys@transformshift{5.453003in}{1.263068in}%
\pgfsys@useobject{currentmarker}{}%
\end{pgfscope}%
\end{pgfscope}%
\begin{pgfscope}%
\definecolor{textcolor}{rgb}{0.333333,0.333333,0.333333}%
\pgfsetstrokecolor{textcolor}%
\pgfsetfillcolor{textcolor}%
\pgftext[x=5.519554in, y=0.210068in, left, base,rotate=90.000000]{\color{textcolor}\rmfamily\fontsize{16.000000}{19.200000}\selectfont mga-3-5\%}%
\end{pgfscope}%
\begin{pgfscope}%
\pgfpathrectangle{\pgfqpoint{3.569557in}{1.263068in}}{\pgfqpoint{2.589738in}{3.079750in}}%
\pgfusepath{clip}%
\pgfsetrectcap%
\pgfsetroundjoin%
\pgfsetlinewidth{0.803000pt}%
\definecolor{currentstroke}{rgb}{1.000000,1.000000,1.000000}%
\pgfsetstrokecolor{currentstroke}%
\pgfsetstrokeopacity{0.000000}%
\pgfsetdash{}{0pt}%
\pgfpathmoveto{\pgfqpoint{6.041580in}{1.263068in}}%
\pgfpathlineto{\pgfqpoint{6.041580in}{4.342817in}}%
\pgfusepath{stroke}%
\end{pgfscope}%
\begin{pgfscope}%
\pgfsetbuttcap%
\pgfsetroundjoin%
\definecolor{currentfill}{rgb}{0.333333,0.333333,0.333333}%
\pgfsetfillcolor{currentfill}%
\pgfsetlinewidth{0.803000pt}%
\definecolor{currentstroke}{rgb}{0.333333,0.333333,0.333333}%
\pgfsetstrokecolor{currentstroke}%
\pgfsetdash{}{0pt}%
\pgfsys@defobject{currentmarker}{\pgfqpoint{0.000000in}{-0.048611in}}{\pgfqpoint{0.000000in}{0.000000in}}{%
\pgfpathmoveto{\pgfqpoint{0.000000in}{0.000000in}}%
\pgfpathlineto{\pgfqpoint{0.000000in}{-0.048611in}}%
\pgfusepath{stroke,fill}%
}%
\begin{pgfscope}%
\pgfsys@transformshift{6.041580in}{1.263068in}%
\pgfsys@useobject{currentmarker}{}%
\end{pgfscope}%
\end{pgfscope}%
\begin{pgfscope}%
\definecolor{textcolor}{rgb}{0.333333,0.333333,0.333333}%
\pgfsetstrokecolor{textcolor}%
\pgfsetfillcolor{textcolor}%
\pgftext[x=6.108131in, y=0.210068in, left, base,rotate=90.000000]{\color{textcolor}\rmfamily\fontsize{16.000000}{19.200000}\selectfont mga-4-5\%}%
\end{pgfscope}%
\begin{pgfscope}%
\pgfpathrectangle{\pgfqpoint{3.569557in}{1.263068in}}{\pgfqpoint{2.589738in}{3.079750in}}%
\pgfusepath{clip}%
\pgfsetrectcap%
\pgfsetroundjoin%
\pgfsetlinewidth{0.803000pt}%
\definecolor{currentstroke}{rgb}{1.000000,1.000000,1.000000}%
\pgfsetstrokecolor{currentstroke}%
\pgfsetdash{}{0pt}%
\pgfpathmoveto{\pgfqpoint{3.569557in}{1.403056in}}%
\pgfpathlineto{\pgfqpoint{6.159295in}{1.403056in}}%
\pgfusepath{stroke}%
\end{pgfscope}%
\begin{pgfscope}%
\pgfsetbuttcap%
\pgfsetroundjoin%
\definecolor{currentfill}{rgb}{0.333333,0.333333,0.333333}%
\pgfsetfillcolor{currentfill}%
\pgfsetlinewidth{0.803000pt}%
\definecolor{currentstroke}{rgb}{0.333333,0.333333,0.333333}%
\pgfsetstrokecolor{currentstroke}%
\pgfsetdash{}{0pt}%
\pgfsys@defobject{currentmarker}{\pgfqpoint{-0.048611in}{0.000000in}}{\pgfqpoint{-0.000000in}{0.000000in}}{%
\pgfpathmoveto{\pgfqpoint{-0.000000in}{0.000000in}}%
\pgfpathlineto{\pgfqpoint{-0.048611in}{0.000000in}}%
\pgfusepath{stroke,fill}%
}%
\begin{pgfscope}%
\pgfsys@transformshift{3.569557in}{1.403056in}%
\pgfsys@useobject{currentmarker}{}%
\end{pgfscope}%
\end{pgfscope}%
\begin{pgfscope}%
\pgfpathrectangle{\pgfqpoint{3.569557in}{1.263068in}}{\pgfqpoint{2.589738in}{3.079750in}}%
\pgfusepath{clip}%
\pgfsetrectcap%
\pgfsetroundjoin%
\pgfsetlinewidth{0.803000pt}%
\definecolor{currentstroke}{rgb}{1.000000,1.000000,1.000000}%
\pgfsetstrokecolor{currentstroke}%
\pgfsetdash{}{0pt}%
\pgfpathmoveto{\pgfqpoint{3.569557in}{1.858235in}}%
\pgfpathlineto{\pgfqpoint{6.159295in}{1.858235in}}%
\pgfusepath{stroke}%
\end{pgfscope}%
\begin{pgfscope}%
\pgfsetbuttcap%
\pgfsetroundjoin%
\definecolor{currentfill}{rgb}{0.333333,0.333333,0.333333}%
\pgfsetfillcolor{currentfill}%
\pgfsetlinewidth{0.803000pt}%
\definecolor{currentstroke}{rgb}{0.333333,0.333333,0.333333}%
\pgfsetstrokecolor{currentstroke}%
\pgfsetdash{}{0pt}%
\pgfsys@defobject{currentmarker}{\pgfqpoint{-0.048611in}{0.000000in}}{\pgfqpoint{-0.000000in}{0.000000in}}{%
\pgfpathmoveto{\pgfqpoint{-0.000000in}{0.000000in}}%
\pgfpathlineto{\pgfqpoint{-0.048611in}{0.000000in}}%
\pgfusepath{stroke,fill}%
}%
\begin{pgfscope}%
\pgfsys@transformshift{3.569557in}{1.858235in}%
\pgfsys@useobject{currentmarker}{}%
\end{pgfscope}%
\end{pgfscope}%
\begin{pgfscope}%
\pgfpathrectangle{\pgfqpoint{3.569557in}{1.263068in}}{\pgfqpoint{2.589738in}{3.079750in}}%
\pgfusepath{clip}%
\pgfsetrectcap%
\pgfsetroundjoin%
\pgfsetlinewidth{0.803000pt}%
\definecolor{currentstroke}{rgb}{1.000000,1.000000,1.000000}%
\pgfsetstrokecolor{currentstroke}%
\pgfsetdash{}{0pt}%
\pgfpathmoveto{\pgfqpoint{3.569557in}{2.313414in}}%
\pgfpathlineto{\pgfqpoint{6.159295in}{2.313414in}}%
\pgfusepath{stroke}%
\end{pgfscope}%
\begin{pgfscope}%
\pgfsetbuttcap%
\pgfsetroundjoin%
\definecolor{currentfill}{rgb}{0.333333,0.333333,0.333333}%
\pgfsetfillcolor{currentfill}%
\pgfsetlinewidth{0.803000pt}%
\definecolor{currentstroke}{rgb}{0.333333,0.333333,0.333333}%
\pgfsetstrokecolor{currentstroke}%
\pgfsetdash{}{0pt}%
\pgfsys@defobject{currentmarker}{\pgfqpoint{-0.048611in}{0.000000in}}{\pgfqpoint{-0.000000in}{0.000000in}}{%
\pgfpathmoveto{\pgfqpoint{-0.000000in}{0.000000in}}%
\pgfpathlineto{\pgfqpoint{-0.048611in}{0.000000in}}%
\pgfusepath{stroke,fill}%
}%
\begin{pgfscope}%
\pgfsys@transformshift{3.569557in}{2.313414in}%
\pgfsys@useobject{currentmarker}{}%
\end{pgfscope}%
\end{pgfscope}%
\begin{pgfscope}%
\pgfpathrectangle{\pgfqpoint{3.569557in}{1.263068in}}{\pgfqpoint{2.589738in}{3.079750in}}%
\pgfusepath{clip}%
\pgfsetrectcap%
\pgfsetroundjoin%
\pgfsetlinewidth{0.803000pt}%
\definecolor{currentstroke}{rgb}{1.000000,1.000000,1.000000}%
\pgfsetstrokecolor{currentstroke}%
\pgfsetdash{}{0pt}%
\pgfpathmoveto{\pgfqpoint{3.569557in}{2.768592in}}%
\pgfpathlineto{\pgfqpoint{6.159295in}{2.768592in}}%
\pgfusepath{stroke}%
\end{pgfscope}%
\begin{pgfscope}%
\pgfsetbuttcap%
\pgfsetroundjoin%
\definecolor{currentfill}{rgb}{0.333333,0.333333,0.333333}%
\pgfsetfillcolor{currentfill}%
\pgfsetlinewidth{0.803000pt}%
\definecolor{currentstroke}{rgb}{0.333333,0.333333,0.333333}%
\pgfsetstrokecolor{currentstroke}%
\pgfsetdash{}{0pt}%
\pgfsys@defobject{currentmarker}{\pgfqpoint{-0.048611in}{0.000000in}}{\pgfqpoint{-0.000000in}{0.000000in}}{%
\pgfpathmoveto{\pgfqpoint{-0.000000in}{0.000000in}}%
\pgfpathlineto{\pgfqpoint{-0.048611in}{0.000000in}}%
\pgfusepath{stroke,fill}%
}%
\begin{pgfscope}%
\pgfsys@transformshift{3.569557in}{2.768592in}%
\pgfsys@useobject{currentmarker}{}%
\end{pgfscope}%
\end{pgfscope}%
\begin{pgfscope}%
\pgfpathrectangle{\pgfqpoint{3.569557in}{1.263068in}}{\pgfqpoint{2.589738in}{3.079750in}}%
\pgfusepath{clip}%
\pgfsetrectcap%
\pgfsetroundjoin%
\pgfsetlinewidth{0.803000pt}%
\definecolor{currentstroke}{rgb}{1.000000,1.000000,1.000000}%
\pgfsetstrokecolor{currentstroke}%
\pgfsetdash{}{0pt}%
\pgfpathmoveto{\pgfqpoint{3.569557in}{3.223771in}}%
\pgfpathlineto{\pgfqpoint{6.159295in}{3.223771in}}%
\pgfusepath{stroke}%
\end{pgfscope}%
\begin{pgfscope}%
\pgfsetbuttcap%
\pgfsetroundjoin%
\definecolor{currentfill}{rgb}{0.333333,0.333333,0.333333}%
\pgfsetfillcolor{currentfill}%
\pgfsetlinewidth{0.803000pt}%
\definecolor{currentstroke}{rgb}{0.333333,0.333333,0.333333}%
\pgfsetstrokecolor{currentstroke}%
\pgfsetdash{}{0pt}%
\pgfsys@defobject{currentmarker}{\pgfqpoint{-0.048611in}{0.000000in}}{\pgfqpoint{-0.000000in}{0.000000in}}{%
\pgfpathmoveto{\pgfqpoint{-0.000000in}{0.000000in}}%
\pgfpathlineto{\pgfqpoint{-0.048611in}{0.000000in}}%
\pgfusepath{stroke,fill}%
}%
\begin{pgfscope}%
\pgfsys@transformshift{3.569557in}{3.223771in}%
\pgfsys@useobject{currentmarker}{}%
\end{pgfscope}%
\end{pgfscope}%
\begin{pgfscope}%
\pgfpathrectangle{\pgfqpoint{3.569557in}{1.263068in}}{\pgfqpoint{2.589738in}{3.079750in}}%
\pgfusepath{clip}%
\pgfsetrectcap%
\pgfsetroundjoin%
\pgfsetlinewidth{0.803000pt}%
\definecolor{currentstroke}{rgb}{1.000000,1.000000,1.000000}%
\pgfsetstrokecolor{currentstroke}%
\pgfsetdash{}{0pt}%
\pgfpathmoveto{\pgfqpoint{3.569557in}{3.678950in}}%
\pgfpathlineto{\pgfqpoint{6.159295in}{3.678950in}}%
\pgfusepath{stroke}%
\end{pgfscope}%
\begin{pgfscope}%
\pgfsetbuttcap%
\pgfsetroundjoin%
\definecolor{currentfill}{rgb}{0.333333,0.333333,0.333333}%
\pgfsetfillcolor{currentfill}%
\pgfsetlinewidth{0.803000pt}%
\definecolor{currentstroke}{rgb}{0.333333,0.333333,0.333333}%
\pgfsetstrokecolor{currentstroke}%
\pgfsetdash{}{0pt}%
\pgfsys@defobject{currentmarker}{\pgfqpoint{-0.048611in}{0.000000in}}{\pgfqpoint{-0.000000in}{0.000000in}}{%
\pgfpathmoveto{\pgfqpoint{-0.000000in}{0.000000in}}%
\pgfpathlineto{\pgfqpoint{-0.048611in}{0.000000in}}%
\pgfusepath{stroke,fill}%
}%
\begin{pgfscope}%
\pgfsys@transformshift{3.569557in}{3.678950in}%
\pgfsys@useobject{currentmarker}{}%
\end{pgfscope}%
\end{pgfscope}%
\begin{pgfscope}%
\pgfpathrectangle{\pgfqpoint{3.569557in}{1.263068in}}{\pgfqpoint{2.589738in}{3.079750in}}%
\pgfusepath{clip}%
\pgfsetrectcap%
\pgfsetroundjoin%
\pgfsetlinewidth{0.803000pt}%
\definecolor{currentstroke}{rgb}{1.000000,1.000000,1.000000}%
\pgfsetstrokecolor{currentstroke}%
\pgfsetdash{}{0pt}%
\pgfpathmoveto{\pgfqpoint{3.569557in}{4.134128in}}%
\pgfpathlineto{\pgfqpoint{6.159295in}{4.134128in}}%
\pgfusepath{stroke}%
\end{pgfscope}%
\begin{pgfscope}%
\pgfsetbuttcap%
\pgfsetroundjoin%
\definecolor{currentfill}{rgb}{0.333333,0.333333,0.333333}%
\pgfsetfillcolor{currentfill}%
\pgfsetlinewidth{0.803000pt}%
\definecolor{currentstroke}{rgb}{0.333333,0.333333,0.333333}%
\pgfsetstrokecolor{currentstroke}%
\pgfsetdash{}{0pt}%
\pgfsys@defobject{currentmarker}{\pgfqpoint{-0.048611in}{0.000000in}}{\pgfqpoint{-0.000000in}{0.000000in}}{%
\pgfpathmoveto{\pgfqpoint{-0.000000in}{0.000000in}}%
\pgfpathlineto{\pgfqpoint{-0.048611in}{0.000000in}}%
\pgfusepath{stroke,fill}%
}%
\begin{pgfscope}%
\pgfsys@transformshift{3.569557in}{4.134128in}%
\pgfsys@useobject{currentmarker}{}%
\end{pgfscope}%
\end{pgfscope}%
\begin{pgfscope}%
\pgfpathrectangle{\pgfqpoint{3.569557in}{1.263068in}}{\pgfqpoint{2.589738in}{3.079750in}}%
\pgfusepath{clip}%
\pgfsetbuttcap%
\pgfsetroundjoin%
\pgfsetlinewidth{1.505625pt}%
\definecolor{currentstroke}{rgb}{0.839216,0.152941,0.156863}%
\pgfsetstrokecolor{currentstroke}%
\pgfsetdash{{5.550000pt}{2.400000pt}}{0.000000pt}%
\pgfpathmoveto{\pgfqpoint{3.687273in}{4.005751in}}%
\pgfpathlineto{\pgfqpoint{4.275849in}{1.403056in}}%
\pgfpathlineto{\pgfqpoint{4.864426in}{1.403056in}}%
\pgfpathlineto{\pgfqpoint{5.453003in}{1.403056in}}%
\pgfpathlineto{\pgfqpoint{6.041580in}{1.403056in}}%
\pgfusepath{stroke}%
\end{pgfscope}%
\begin{pgfscope}%
\pgfpathrectangle{\pgfqpoint{3.569557in}{1.263068in}}{\pgfqpoint{2.589738in}{3.079750in}}%
\pgfusepath{clip}%
\pgfsetbuttcap%
\pgfsetroundjoin%
\definecolor{currentfill}{rgb}{0.839216,0.152941,0.156863}%
\pgfsetfillcolor{currentfill}%
\pgfsetlinewidth{1.003750pt}%
\definecolor{currentstroke}{rgb}{0.839216,0.152941,0.156863}%
\pgfsetstrokecolor{currentstroke}%
\pgfsetdash{}{0pt}%
\pgfsys@defobject{currentmarker}{\pgfqpoint{-0.041667in}{-0.041667in}}{\pgfqpoint{0.041667in}{0.041667in}}{%
\pgfpathmoveto{\pgfqpoint{0.000000in}{-0.041667in}}%
\pgfpathcurveto{\pgfqpoint{0.011050in}{-0.041667in}}{\pgfqpoint{0.021649in}{-0.037276in}}{\pgfqpoint{0.029463in}{-0.029463in}}%
\pgfpathcurveto{\pgfqpoint{0.037276in}{-0.021649in}}{\pgfqpoint{0.041667in}{-0.011050in}}{\pgfqpoint{0.041667in}{0.000000in}}%
\pgfpathcurveto{\pgfqpoint{0.041667in}{0.011050in}}{\pgfqpoint{0.037276in}{0.021649in}}{\pgfqpoint{0.029463in}{0.029463in}}%
\pgfpathcurveto{\pgfqpoint{0.021649in}{0.037276in}}{\pgfqpoint{0.011050in}{0.041667in}}{\pgfqpoint{0.000000in}{0.041667in}}%
\pgfpathcurveto{\pgfqpoint{-0.011050in}{0.041667in}}{\pgfqpoint{-0.021649in}{0.037276in}}{\pgfqpoint{-0.029463in}{0.029463in}}%
\pgfpathcurveto{\pgfqpoint{-0.037276in}{0.021649in}}{\pgfqpoint{-0.041667in}{0.011050in}}{\pgfqpoint{-0.041667in}{0.000000in}}%
\pgfpathcurveto{\pgfqpoint{-0.041667in}{-0.011050in}}{\pgfqpoint{-0.037276in}{-0.021649in}}{\pgfqpoint{-0.029463in}{-0.029463in}}%
\pgfpathcurveto{\pgfqpoint{-0.021649in}{-0.037276in}}{\pgfqpoint{-0.011050in}{-0.041667in}}{\pgfqpoint{0.000000in}{-0.041667in}}%
\pgfpathclose%
\pgfusepath{stroke,fill}%
}%
\begin{pgfscope}%
\pgfsys@transformshift{3.687273in}{4.005751in}%
\pgfsys@useobject{currentmarker}{}%
\end{pgfscope}%
\begin{pgfscope}%
\pgfsys@transformshift{4.275849in}{1.403056in}%
\pgfsys@useobject{currentmarker}{}%
\end{pgfscope}%
\begin{pgfscope}%
\pgfsys@transformshift{4.864426in}{1.403056in}%
\pgfsys@useobject{currentmarker}{}%
\end{pgfscope}%
\begin{pgfscope}%
\pgfsys@transformshift{5.453003in}{1.403056in}%
\pgfsys@useobject{currentmarker}{}%
\end{pgfscope}%
\begin{pgfscope}%
\pgfsys@transformshift{6.041580in}{1.403056in}%
\pgfsys@useobject{currentmarker}{}%
\end{pgfscope}%
\end{pgfscope}%
\begin{pgfscope}%
\pgfpathrectangle{\pgfqpoint{3.569557in}{1.263068in}}{\pgfqpoint{2.589738in}{3.079750in}}%
\pgfusepath{clip}%
\pgfsetbuttcap%
\pgfsetroundjoin%
\pgfsetlinewidth{1.505625pt}%
\definecolor{currentstroke}{rgb}{0.549020,0.337255,0.294118}%
\pgfsetstrokecolor{currentstroke}%
\pgfsetdash{{5.550000pt}{2.400000pt}}{0.000000pt}%
\pgfpathmoveto{\pgfqpoint{3.687273in}{4.031153in}}%
\pgfpathlineto{\pgfqpoint{4.275849in}{4.197693in}}%
\pgfpathlineto{\pgfqpoint{4.864426in}{4.197693in}}%
\pgfpathlineto{\pgfqpoint{5.453003in}{4.197694in}}%
\pgfpathlineto{\pgfqpoint{6.041580in}{4.197693in}}%
\pgfusepath{stroke}%
\end{pgfscope}%
\begin{pgfscope}%
\pgfpathrectangle{\pgfqpoint{3.569557in}{1.263068in}}{\pgfqpoint{2.589738in}{3.079750in}}%
\pgfusepath{clip}%
\pgfsetbuttcap%
\pgfsetroundjoin%
\definecolor{currentfill}{rgb}{0.549020,0.337255,0.294118}%
\pgfsetfillcolor{currentfill}%
\pgfsetlinewidth{1.003750pt}%
\definecolor{currentstroke}{rgb}{0.549020,0.337255,0.294118}%
\pgfsetstrokecolor{currentstroke}%
\pgfsetdash{}{0pt}%
\pgfsys@defobject{currentmarker}{\pgfqpoint{-0.041667in}{-0.041667in}}{\pgfqpoint{0.041667in}{0.041667in}}{%
\pgfpathmoveto{\pgfqpoint{0.000000in}{-0.041667in}}%
\pgfpathcurveto{\pgfqpoint{0.011050in}{-0.041667in}}{\pgfqpoint{0.021649in}{-0.037276in}}{\pgfqpoint{0.029463in}{-0.029463in}}%
\pgfpathcurveto{\pgfqpoint{0.037276in}{-0.021649in}}{\pgfqpoint{0.041667in}{-0.011050in}}{\pgfqpoint{0.041667in}{0.000000in}}%
\pgfpathcurveto{\pgfqpoint{0.041667in}{0.011050in}}{\pgfqpoint{0.037276in}{0.021649in}}{\pgfqpoint{0.029463in}{0.029463in}}%
\pgfpathcurveto{\pgfqpoint{0.021649in}{0.037276in}}{\pgfqpoint{0.011050in}{0.041667in}}{\pgfqpoint{0.000000in}{0.041667in}}%
\pgfpathcurveto{\pgfqpoint{-0.011050in}{0.041667in}}{\pgfqpoint{-0.021649in}{0.037276in}}{\pgfqpoint{-0.029463in}{0.029463in}}%
\pgfpathcurveto{\pgfqpoint{-0.037276in}{0.021649in}}{\pgfqpoint{-0.041667in}{0.011050in}}{\pgfqpoint{-0.041667in}{0.000000in}}%
\pgfpathcurveto{\pgfqpoint{-0.041667in}{-0.011050in}}{\pgfqpoint{-0.037276in}{-0.021649in}}{\pgfqpoint{-0.029463in}{-0.029463in}}%
\pgfpathcurveto{\pgfqpoint{-0.021649in}{-0.037276in}}{\pgfqpoint{-0.011050in}{-0.041667in}}{\pgfqpoint{0.000000in}{-0.041667in}}%
\pgfpathclose%
\pgfusepath{stroke,fill}%
}%
\begin{pgfscope}%
\pgfsys@transformshift{3.687273in}{4.031153in}%
\pgfsys@useobject{currentmarker}{}%
\end{pgfscope}%
\begin{pgfscope}%
\pgfsys@transformshift{4.275849in}{4.197693in}%
\pgfsys@useobject{currentmarker}{}%
\end{pgfscope}%
\begin{pgfscope}%
\pgfsys@transformshift{4.864426in}{4.197693in}%
\pgfsys@useobject{currentmarker}{}%
\end{pgfscope}%
\begin{pgfscope}%
\pgfsys@transformshift{5.453003in}{4.197694in}%
\pgfsys@useobject{currentmarker}{}%
\end{pgfscope}%
\begin{pgfscope}%
\pgfsys@transformshift{6.041580in}{4.197693in}%
\pgfsys@useobject{currentmarker}{}%
\end{pgfscope}%
\end{pgfscope}%
\begin{pgfscope}%
\pgfpathrectangle{\pgfqpoint{3.569557in}{1.263068in}}{\pgfqpoint{2.589738in}{3.079750in}}%
\pgfusepath{clip}%
\pgfsetbuttcap%
\pgfsetroundjoin%
\pgfsetlinewidth{1.505625pt}%
\definecolor{currentstroke}{rgb}{0.411765,0.411765,0.411765}%
\pgfsetstrokecolor{currentstroke}%
\pgfsetdash{{5.550000pt}{2.400000pt}}{0.000000pt}%
\pgfpathmoveto{\pgfqpoint{3.687273in}{1.403056in}}%
\pgfpathlineto{\pgfqpoint{4.275849in}{1.403056in}}%
\pgfpathlineto{\pgfqpoint{4.864426in}{1.403056in}}%
\pgfpathlineto{\pgfqpoint{5.453003in}{1.403056in}}%
\pgfpathlineto{\pgfqpoint{6.041580in}{1.403056in}}%
\pgfusepath{stroke}%
\end{pgfscope}%
\begin{pgfscope}%
\pgfpathrectangle{\pgfqpoint{3.569557in}{1.263068in}}{\pgfqpoint{2.589738in}{3.079750in}}%
\pgfusepath{clip}%
\pgfsetbuttcap%
\pgfsetroundjoin%
\definecolor{currentfill}{rgb}{0.411765,0.411765,0.411765}%
\pgfsetfillcolor{currentfill}%
\pgfsetlinewidth{1.003750pt}%
\definecolor{currentstroke}{rgb}{0.411765,0.411765,0.411765}%
\pgfsetstrokecolor{currentstroke}%
\pgfsetdash{}{0pt}%
\pgfsys@defobject{currentmarker}{\pgfqpoint{-0.041667in}{-0.041667in}}{\pgfqpoint{0.041667in}{0.041667in}}{%
\pgfpathmoveto{\pgfqpoint{0.000000in}{-0.041667in}}%
\pgfpathcurveto{\pgfqpoint{0.011050in}{-0.041667in}}{\pgfqpoint{0.021649in}{-0.037276in}}{\pgfqpoint{0.029463in}{-0.029463in}}%
\pgfpathcurveto{\pgfqpoint{0.037276in}{-0.021649in}}{\pgfqpoint{0.041667in}{-0.011050in}}{\pgfqpoint{0.041667in}{0.000000in}}%
\pgfpathcurveto{\pgfqpoint{0.041667in}{0.011050in}}{\pgfqpoint{0.037276in}{0.021649in}}{\pgfqpoint{0.029463in}{0.029463in}}%
\pgfpathcurveto{\pgfqpoint{0.021649in}{0.037276in}}{\pgfqpoint{0.011050in}{0.041667in}}{\pgfqpoint{0.000000in}{0.041667in}}%
\pgfpathcurveto{\pgfqpoint{-0.011050in}{0.041667in}}{\pgfqpoint{-0.021649in}{0.037276in}}{\pgfqpoint{-0.029463in}{0.029463in}}%
\pgfpathcurveto{\pgfqpoint{-0.037276in}{0.021649in}}{\pgfqpoint{-0.041667in}{0.011050in}}{\pgfqpoint{-0.041667in}{0.000000in}}%
\pgfpathcurveto{\pgfqpoint{-0.041667in}{-0.011050in}}{\pgfqpoint{-0.037276in}{-0.021649in}}{\pgfqpoint{-0.029463in}{-0.029463in}}%
\pgfpathcurveto{\pgfqpoint{-0.021649in}{-0.037276in}}{\pgfqpoint{-0.011050in}{-0.041667in}}{\pgfqpoint{0.000000in}{-0.041667in}}%
\pgfpathclose%
\pgfusepath{stroke,fill}%
}%
\begin{pgfscope}%
\pgfsys@transformshift{3.687273in}{1.403056in}%
\pgfsys@useobject{currentmarker}{}%
\end{pgfscope}%
\begin{pgfscope}%
\pgfsys@transformshift{4.275849in}{1.403056in}%
\pgfsys@useobject{currentmarker}{}%
\end{pgfscope}%
\begin{pgfscope}%
\pgfsys@transformshift{4.864426in}{1.403056in}%
\pgfsys@useobject{currentmarker}{}%
\end{pgfscope}%
\begin{pgfscope}%
\pgfsys@transformshift{5.453003in}{1.403056in}%
\pgfsys@useobject{currentmarker}{}%
\end{pgfscope}%
\begin{pgfscope}%
\pgfsys@transformshift{6.041580in}{1.403056in}%
\pgfsys@useobject{currentmarker}{}%
\end{pgfscope}%
\end{pgfscope}%
\begin{pgfscope}%
\pgfpathrectangle{\pgfqpoint{3.569557in}{1.263068in}}{\pgfqpoint{2.589738in}{3.079750in}}%
\pgfusepath{clip}%
\pgfsetbuttcap%
\pgfsetroundjoin%
\pgfsetlinewidth{1.505625pt}%
\definecolor{currentstroke}{rgb}{0.172549,0.627451,0.172549}%
\pgfsetstrokecolor{currentstroke}%
\pgfsetdash{{5.550000pt}{2.400000pt}}{0.000000pt}%
\pgfpathmoveto{\pgfqpoint{3.687273in}{2.075567in}}%
\pgfpathlineto{\pgfqpoint{4.275849in}{1.403215in}}%
\pgfpathlineto{\pgfqpoint{4.864426in}{1.403962in}}%
\pgfpathlineto{\pgfqpoint{5.453003in}{1.403056in}}%
\pgfpathlineto{\pgfqpoint{6.041580in}{1.403056in}}%
\pgfusepath{stroke}%
\end{pgfscope}%
\begin{pgfscope}%
\pgfpathrectangle{\pgfqpoint{3.569557in}{1.263068in}}{\pgfqpoint{2.589738in}{3.079750in}}%
\pgfusepath{clip}%
\pgfsetbuttcap%
\pgfsetroundjoin%
\definecolor{currentfill}{rgb}{0.172549,0.627451,0.172549}%
\pgfsetfillcolor{currentfill}%
\pgfsetlinewidth{1.003750pt}%
\definecolor{currentstroke}{rgb}{0.172549,0.627451,0.172549}%
\pgfsetstrokecolor{currentstroke}%
\pgfsetdash{}{0pt}%
\pgfsys@defobject{currentmarker}{\pgfqpoint{-0.041667in}{-0.041667in}}{\pgfqpoint{0.041667in}{0.041667in}}{%
\pgfpathmoveto{\pgfqpoint{0.000000in}{-0.041667in}}%
\pgfpathcurveto{\pgfqpoint{0.011050in}{-0.041667in}}{\pgfqpoint{0.021649in}{-0.037276in}}{\pgfqpoint{0.029463in}{-0.029463in}}%
\pgfpathcurveto{\pgfqpoint{0.037276in}{-0.021649in}}{\pgfqpoint{0.041667in}{-0.011050in}}{\pgfqpoint{0.041667in}{0.000000in}}%
\pgfpathcurveto{\pgfqpoint{0.041667in}{0.011050in}}{\pgfqpoint{0.037276in}{0.021649in}}{\pgfqpoint{0.029463in}{0.029463in}}%
\pgfpathcurveto{\pgfqpoint{0.021649in}{0.037276in}}{\pgfqpoint{0.011050in}{0.041667in}}{\pgfqpoint{0.000000in}{0.041667in}}%
\pgfpathcurveto{\pgfqpoint{-0.011050in}{0.041667in}}{\pgfqpoint{-0.021649in}{0.037276in}}{\pgfqpoint{-0.029463in}{0.029463in}}%
\pgfpathcurveto{\pgfqpoint{-0.037276in}{0.021649in}}{\pgfqpoint{-0.041667in}{0.011050in}}{\pgfqpoint{-0.041667in}{0.000000in}}%
\pgfpathcurveto{\pgfqpoint{-0.041667in}{-0.011050in}}{\pgfqpoint{-0.037276in}{-0.021649in}}{\pgfqpoint{-0.029463in}{-0.029463in}}%
\pgfpathcurveto{\pgfqpoint{-0.021649in}{-0.037276in}}{\pgfqpoint{-0.011050in}{-0.041667in}}{\pgfqpoint{0.000000in}{-0.041667in}}%
\pgfpathclose%
\pgfusepath{stroke,fill}%
}%
\begin{pgfscope}%
\pgfsys@transformshift{3.687273in}{2.075567in}%
\pgfsys@useobject{currentmarker}{}%
\end{pgfscope}%
\begin{pgfscope}%
\pgfsys@transformshift{4.275849in}{1.403215in}%
\pgfsys@useobject{currentmarker}{}%
\end{pgfscope}%
\begin{pgfscope}%
\pgfsys@transformshift{4.864426in}{1.403962in}%
\pgfsys@useobject{currentmarker}{}%
\end{pgfscope}%
\begin{pgfscope}%
\pgfsys@transformshift{5.453003in}{1.403056in}%
\pgfsys@useobject{currentmarker}{}%
\end{pgfscope}%
\begin{pgfscope}%
\pgfsys@transformshift{6.041580in}{1.403056in}%
\pgfsys@useobject{currentmarker}{}%
\end{pgfscope}%
\end{pgfscope}%
\begin{pgfscope}%
\pgfpathrectangle{\pgfqpoint{3.569557in}{1.263068in}}{\pgfqpoint{2.589738in}{3.079750in}}%
\pgfusepath{clip}%
\pgfsetbuttcap%
\pgfsetroundjoin%
\pgfsetlinewidth{1.505625pt}%
\definecolor{currentstroke}{rgb}{1.000000,1.000000,0.000000}%
\pgfsetstrokecolor{currentstroke}%
\pgfsetdash{{5.550000pt}{2.400000pt}}{0.000000pt}%
\pgfpathmoveto{\pgfqpoint{3.687273in}{2.704346in}}%
\pgfpathlineto{\pgfqpoint{4.275849in}{2.850598in}}%
\pgfpathlineto{\pgfqpoint{4.864426in}{2.850599in}}%
\pgfpathlineto{\pgfqpoint{5.453003in}{2.850598in}}%
\pgfpathlineto{\pgfqpoint{6.041580in}{2.850598in}}%
\pgfusepath{stroke}%
\end{pgfscope}%
\begin{pgfscope}%
\pgfpathrectangle{\pgfqpoint{3.569557in}{1.263068in}}{\pgfqpoint{2.589738in}{3.079750in}}%
\pgfusepath{clip}%
\pgfsetbuttcap%
\pgfsetroundjoin%
\definecolor{currentfill}{rgb}{1.000000,1.000000,0.000000}%
\pgfsetfillcolor{currentfill}%
\pgfsetlinewidth{1.003750pt}%
\definecolor{currentstroke}{rgb}{1.000000,1.000000,0.000000}%
\pgfsetstrokecolor{currentstroke}%
\pgfsetdash{}{0pt}%
\pgfsys@defobject{currentmarker}{\pgfqpoint{-0.041667in}{-0.041667in}}{\pgfqpoint{0.041667in}{0.041667in}}{%
\pgfpathmoveto{\pgfqpoint{0.000000in}{-0.041667in}}%
\pgfpathcurveto{\pgfqpoint{0.011050in}{-0.041667in}}{\pgfqpoint{0.021649in}{-0.037276in}}{\pgfqpoint{0.029463in}{-0.029463in}}%
\pgfpathcurveto{\pgfqpoint{0.037276in}{-0.021649in}}{\pgfqpoint{0.041667in}{-0.011050in}}{\pgfqpoint{0.041667in}{0.000000in}}%
\pgfpathcurveto{\pgfqpoint{0.041667in}{0.011050in}}{\pgfqpoint{0.037276in}{0.021649in}}{\pgfqpoint{0.029463in}{0.029463in}}%
\pgfpathcurveto{\pgfqpoint{0.021649in}{0.037276in}}{\pgfqpoint{0.011050in}{0.041667in}}{\pgfqpoint{0.000000in}{0.041667in}}%
\pgfpathcurveto{\pgfqpoint{-0.011050in}{0.041667in}}{\pgfqpoint{-0.021649in}{0.037276in}}{\pgfqpoint{-0.029463in}{0.029463in}}%
\pgfpathcurveto{\pgfqpoint{-0.037276in}{0.021649in}}{\pgfqpoint{-0.041667in}{0.011050in}}{\pgfqpoint{-0.041667in}{0.000000in}}%
\pgfpathcurveto{\pgfqpoint{-0.041667in}{-0.011050in}}{\pgfqpoint{-0.037276in}{-0.021649in}}{\pgfqpoint{-0.029463in}{-0.029463in}}%
\pgfpathcurveto{\pgfqpoint{-0.021649in}{-0.037276in}}{\pgfqpoint{-0.011050in}{-0.041667in}}{\pgfqpoint{0.000000in}{-0.041667in}}%
\pgfpathclose%
\pgfusepath{stroke,fill}%
}%
\begin{pgfscope}%
\pgfsys@transformshift{3.687273in}{2.704346in}%
\pgfsys@useobject{currentmarker}{}%
\end{pgfscope}%
\begin{pgfscope}%
\pgfsys@transformshift{4.275849in}{2.850598in}%
\pgfsys@useobject{currentmarker}{}%
\end{pgfscope}%
\begin{pgfscope}%
\pgfsys@transformshift{4.864426in}{2.850599in}%
\pgfsys@useobject{currentmarker}{}%
\end{pgfscope}%
\begin{pgfscope}%
\pgfsys@transformshift{5.453003in}{2.850598in}%
\pgfsys@useobject{currentmarker}{}%
\end{pgfscope}%
\begin{pgfscope}%
\pgfsys@transformshift{6.041580in}{2.850598in}%
\pgfsys@useobject{currentmarker}{}%
\end{pgfscope}%
\end{pgfscope}%
\begin{pgfscope}%
\pgfpathrectangle{\pgfqpoint{3.569557in}{1.263068in}}{\pgfqpoint{2.589738in}{3.079750in}}%
\pgfusepath{clip}%
\pgfsetbuttcap%
\pgfsetroundjoin%
\pgfsetlinewidth{1.505625pt}%
\definecolor{currentstroke}{rgb}{0.121569,0.466667,0.705882}%
\pgfsetstrokecolor{currentstroke}%
\pgfsetdash{{5.550000pt}{2.400000pt}}{0.000000pt}%
\pgfpathmoveto{\pgfqpoint{3.687273in}{3.690329in}}%
\pgfpathlineto{\pgfqpoint{4.275849in}{3.690329in}}%
\pgfpathlineto{\pgfqpoint{4.864426in}{3.690329in}}%
\pgfpathlineto{\pgfqpoint{5.453003in}{3.690329in}}%
\pgfpathlineto{\pgfqpoint{6.041580in}{3.690329in}}%
\pgfusepath{stroke}%
\end{pgfscope}%
\begin{pgfscope}%
\pgfpathrectangle{\pgfqpoint{3.569557in}{1.263068in}}{\pgfqpoint{2.589738in}{3.079750in}}%
\pgfusepath{clip}%
\pgfsetbuttcap%
\pgfsetroundjoin%
\definecolor{currentfill}{rgb}{0.121569,0.466667,0.705882}%
\pgfsetfillcolor{currentfill}%
\pgfsetlinewidth{1.003750pt}%
\definecolor{currentstroke}{rgb}{0.121569,0.466667,0.705882}%
\pgfsetstrokecolor{currentstroke}%
\pgfsetdash{}{0pt}%
\pgfsys@defobject{currentmarker}{\pgfqpoint{-0.041667in}{-0.041667in}}{\pgfqpoint{0.041667in}{0.041667in}}{%
\pgfpathmoveto{\pgfqpoint{0.000000in}{-0.041667in}}%
\pgfpathcurveto{\pgfqpoint{0.011050in}{-0.041667in}}{\pgfqpoint{0.021649in}{-0.037276in}}{\pgfqpoint{0.029463in}{-0.029463in}}%
\pgfpathcurveto{\pgfqpoint{0.037276in}{-0.021649in}}{\pgfqpoint{0.041667in}{-0.011050in}}{\pgfqpoint{0.041667in}{0.000000in}}%
\pgfpathcurveto{\pgfqpoint{0.041667in}{0.011050in}}{\pgfqpoint{0.037276in}{0.021649in}}{\pgfqpoint{0.029463in}{0.029463in}}%
\pgfpathcurveto{\pgfqpoint{0.021649in}{0.037276in}}{\pgfqpoint{0.011050in}{0.041667in}}{\pgfqpoint{0.000000in}{0.041667in}}%
\pgfpathcurveto{\pgfqpoint{-0.011050in}{0.041667in}}{\pgfqpoint{-0.021649in}{0.037276in}}{\pgfqpoint{-0.029463in}{0.029463in}}%
\pgfpathcurveto{\pgfqpoint{-0.037276in}{0.021649in}}{\pgfqpoint{-0.041667in}{0.011050in}}{\pgfqpoint{-0.041667in}{0.000000in}}%
\pgfpathcurveto{\pgfqpoint{-0.041667in}{-0.011050in}}{\pgfqpoint{-0.037276in}{-0.021649in}}{\pgfqpoint{-0.029463in}{-0.029463in}}%
\pgfpathcurveto{\pgfqpoint{-0.021649in}{-0.037276in}}{\pgfqpoint{-0.011050in}{-0.041667in}}{\pgfqpoint{0.000000in}{-0.041667in}}%
\pgfpathclose%
\pgfusepath{stroke,fill}%
}%
\begin{pgfscope}%
\pgfsys@transformshift{3.687273in}{3.690329in}%
\pgfsys@useobject{currentmarker}{}%
\end{pgfscope}%
\begin{pgfscope}%
\pgfsys@transformshift{4.275849in}{3.690329in}%
\pgfsys@useobject{currentmarker}{}%
\end{pgfscope}%
\begin{pgfscope}%
\pgfsys@transformshift{4.864426in}{3.690329in}%
\pgfsys@useobject{currentmarker}{}%
\end{pgfscope}%
\begin{pgfscope}%
\pgfsys@transformshift{5.453003in}{3.690329in}%
\pgfsys@useobject{currentmarker}{}%
\end{pgfscope}%
\begin{pgfscope}%
\pgfsys@transformshift{6.041580in}{3.690329in}%
\pgfsys@useobject{currentmarker}{}%
\end{pgfscope}%
\end{pgfscope}%
\begin{pgfscope}%
\pgfsetrectcap%
\pgfsetmiterjoin%
\pgfsetlinewidth{1.003750pt}%
\definecolor{currentstroke}{rgb}{1.000000,1.000000,1.000000}%
\pgfsetstrokecolor{currentstroke}%
\pgfsetdash{}{0pt}%
\pgfpathmoveto{\pgfqpoint{3.569557in}{1.263067in}}%
\pgfpathlineto{\pgfqpoint{3.569557in}{4.342817in}}%
\pgfusepath{stroke}%
\end{pgfscope}%
\begin{pgfscope}%
\pgfsetrectcap%
\pgfsetmiterjoin%
\pgfsetlinewidth{1.003750pt}%
\definecolor{currentstroke}{rgb}{1.000000,1.000000,1.000000}%
\pgfsetstrokecolor{currentstroke}%
\pgfsetdash{}{0pt}%
\pgfpathmoveto{\pgfqpoint{6.159295in}{1.263067in}}%
\pgfpathlineto{\pgfqpoint{6.159295in}{4.342817in}}%
\pgfusepath{stroke}%
\end{pgfscope}%
\begin{pgfscope}%
\pgfsetrectcap%
\pgfsetmiterjoin%
\pgfsetlinewidth{1.003750pt}%
\definecolor{currentstroke}{rgb}{1.000000,1.000000,1.000000}%
\pgfsetstrokecolor{currentstroke}%
\pgfsetdash{}{0pt}%
\pgfpathmoveto{\pgfqpoint{3.569557in}{1.263068in}}%
\pgfpathlineto{\pgfqpoint{6.159295in}{1.263068in}}%
\pgfusepath{stroke}%
\end{pgfscope}%
\begin{pgfscope}%
\pgfsetrectcap%
\pgfsetmiterjoin%
\pgfsetlinewidth{1.003750pt}%
\definecolor{currentstroke}{rgb}{1.000000,1.000000,1.000000}%
\pgfsetstrokecolor{currentstroke}%
\pgfsetdash{}{0pt}%
\pgfpathmoveto{\pgfqpoint{3.569557in}{4.342817in}}%
\pgfpathlineto{\pgfqpoint{6.159295in}{4.342817in}}%
\pgfusepath{stroke}%
\end{pgfscope}%
\begin{pgfscope}%
\pgfsetbuttcap%
\pgfsetmiterjoin%
\definecolor{currentfill}{rgb}{0.898039,0.898039,0.898039}%
\pgfsetfillcolor{currentfill}%
\pgfsetlinewidth{0.000000pt}%
\definecolor{currentstroke}{rgb}{0.000000,0.000000,0.000000}%
\pgfsetstrokecolor{currentstroke}%
\pgfsetstrokeopacity{0.000000}%
\pgfsetdash{}{0pt}%
\pgfpathmoveto{\pgfqpoint{6.379242in}{1.263068in}}%
\pgfpathlineto{\pgfqpoint{8.968980in}{1.263068in}}%
\pgfpathlineto{\pgfqpoint{8.968980in}{4.342817in}}%
\pgfpathlineto{\pgfqpoint{6.379242in}{4.342817in}}%
\pgfpathclose%
\pgfusepath{fill}%
\end{pgfscope}%
\begin{pgfscope}%
\pgfpathrectangle{\pgfqpoint{6.379242in}{1.263068in}}{\pgfqpoint{2.589738in}{3.079750in}}%
\pgfusepath{clip}%
\pgfsetrectcap%
\pgfsetroundjoin%
\pgfsetlinewidth{0.803000pt}%
\definecolor{currentstroke}{rgb}{1.000000,1.000000,1.000000}%
\pgfsetstrokecolor{currentstroke}%
\pgfsetstrokeopacity{0.000000}%
\pgfsetdash{}{0pt}%
\pgfpathmoveto{\pgfqpoint{6.496957in}{1.263068in}}%
\pgfpathlineto{\pgfqpoint{6.496957in}{4.342817in}}%
\pgfusepath{stroke}%
\end{pgfscope}%
\begin{pgfscope}%
\pgfsetbuttcap%
\pgfsetroundjoin%
\definecolor{currentfill}{rgb}{0.333333,0.333333,0.333333}%
\pgfsetfillcolor{currentfill}%
\pgfsetlinewidth{0.803000pt}%
\definecolor{currentstroke}{rgb}{0.333333,0.333333,0.333333}%
\pgfsetstrokecolor{currentstroke}%
\pgfsetdash{}{0pt}%
\pgfsys@defobject{currentmarker}{\pgfqpoint{0.000000in}{-0.048611in}}{\pgfqpoint{0.000000in}{0.000000in}}{%
\pgfpathmoveto{\pgfqpoint{0.000000in}{0.000000in}}%
\pgfpathlineto{\pgfqpoint{0.000000in}{-0.048611in}}%
\pgfusepath{stroke,fill}%
}%
\begin{pgfscope}%
\pgfsys@transformshift{6.496957in}{1.263068in}%
\pgfsys@useobject{currentmarker}{}%
\end{pgfscope}%
\end{pgfscope}%
\begin{pgfscope}%
\definecolor{textcolor}{rgb}{0.333333,0.333333,0.333333}%
\pgfsetstrokecolor{textcolor}%
\pgfsetfillcolor{textcolor}%
\pgftext[x=6.563508in, y=0.100000in, left, base,rotate=90.000000]{\color{textcolor}\rmfamily\fontsize{16.000000}{19.200000}\selectfont mga-0-10\%}%
\end{pgfscope}%
\begin{pgfscope}%
\pgfpathrectangle{\pgfqpoint{6.379242in}{1.263068in}}{\pgfqpoint{2.589738in}{3.079750in}}%
\pgfusepath{clip}%
\pgfsetrectcap%
\pgfsetroundjoin%
\pgfsetlinewidth{0.803000pt}%
\definecolor{currentstroke}{rgb}{1.000000,1.000000,1.000000}%
\pgfsetstrokecolor{currentstroke}%
\pgfsetstrokeopacity{0.000000}%
\pgfsetdash{}{0pt}%
\pgfpathmoveto{\pgfqpoint{7.085534in}{1.263068in}}%
\pgfpathlineto{\pgfqpoint{7.085534in}{4.342817in}}%
\pgfusepath{stroke}%
\end{pgfscope}%
\begin{pgfscope}%
\pgfsetbuttcap%
\pgfsetroundjoin%
\definecolor{currentfill}{rgb}{0.333333,0.333333,0.333333}%
\pgfsetfillcolor{currentfill}%
\pgfsetlinewidth{0.803000pt}%
\definecolor{currentstroke}{rgb}{0.333333,0.333333,0.333333}%
\pgfsetstrokecolor{currentstroke}%
\pgfsetdash{}{0pt}%
\pgfsys@defobject{currentmarker}{\pgfqpoint{0.000000in}{-0.048611in}}{\pgfqpoint{0.000000in}{0.000000in}}{%
\pgfpathmoveto{\pgfqpoint{0.000000in}{0.000000in}}%
\pgfpathlineto{\pgfqpoint{0.000000in}{-0.048611in}}%
\pgfusepath{stroke,fill}%
}%
\begin{pgfscope}%
\pgfsys@transformshift{7.085534in}{1.263068in}%
\pgfsys@useobject{currentmarker}{}%
\end{pgfscope}%
\end{pgfscope}%
\begin{pgfscope}%
\definecolor{textcolor}{rgb}{0.333333,0.333333,0.333333}%
\pgfsetstrokecolor{textcolor}%
\pgfsetfillcolor{textcolor}%
\pgftext[x=7.152085in, y=0.100000in, left, base,rotate=90.000000]{\color{textcolor}\rmfamily\fontsize{16.000000}{19.200000}\selectfont mga-1-10\%}%
\end{pgfscope}%
\begin{pgfscope}%
\pgfpathrectangle{\pgfqpoint{6.379242in}{1.263068in}}{\pgfqpoint{2.589738in}{3.079750in}}%
\pgfusepath{clip}%
\pgfsetrectcap%
\pgfsetroundjoin%
\pgfsetlinewidth{0.803000pt}%
\definecolor{currentstroke}{rgb}{1.000000,1.000000,1.000000}%
\pgfsetstrokecolor{currentstroke}%
\pgfsetstrokeopacity{0.000000}%
\pgfsetdash{}{0pt}%
\pgfpathmoveto{\pgfqpoint{7.674111in}{1.263068in}}%
\pgfpathlineto{\pgfqpoint{7.674111in}{4.342817in}}%
\pgfusepath{stroke}%
\end{pgfscope}%
\begin{pgfscope}%
\pgfsetbuttcap%
\pgfsetroundjoin%
\definecolor{currentfill}{rgb}{0.333333,0.333333,0.333333}%
\pgfsetfillcolor{currentfill}%
\pgfsetlinewidth{0.803000pt}%
\definecolor{currentstroke}{rgb}{0.333333,0.333333,0.333333}%
\pgfsetstrokecolor{currentstroke}%
\pgfsetdash{}{0pt}%
\pgfsys@defobject{currentmarker}{\pgfqpoint{0.000000in}{-0.048611in}}{\pgfqpoint{0.000000in}{0.000000in}}{%
\pgfpathmoveto{\pgfqpoint{0.000000in}{0.000000in}}%
\pgfpathlineto{\pgfqpoint{0.000000in}{-0.048611in}}%
\pgfusepath{stroke,fill}%
}%
\begin{pgfscope}%
\pgfsys@transformshift{7.674111in}{1.263068in}%
\pgfsys@useobject{currentmarker}{}%
\end{pgfscope}%
\end{pgfscope}%
\begin{pgfscope}%
\definecolor{textcolor}{rgb}{0.333333,0.333333,0.333333}%
\pgfsetstrokecolor{textcolor}%
\pgfsetfillcolor{textcolor}%
\pgftext[x=7.740662in, y=0.100000in, left, base,rotate=90.000000]{\color{textcolor}\rmfamily\fontsize{16.000000}{19.200000}\selectfont mga-2-10\%}%
\end{pgfscope}%
\begin{pgfscope}%
\pgfpathrectangle{\pgfqpoint{6.379242in}{1.263068in}}{\pgfqpoint{2.589738in}{3.079750in}}%
\pgfusepath{clip}%
\pgfsetrectcap%
\pgfsetroundjoin%
\pgfsetlinewidth{0.803000pt}%
\definecolor{currentstroke}{rgb}{1.000000,1.000000,1.000000}%
\pgfsetstrokecolor{currentstroke}%
\pgfsetstrokeopacity{0.000000}%
\pgfsetdash{}{0pt}%
\pgfpathmoveto{\pgfqpoint{8.262688in}{1.263068in}}%
\pgfpathlineto{\pgfqpoint{8.262688in}{4.342817in}}%
\pgfusepath{stroke}%
\end{pgfscope}%
\begin{pgfscope}%
\pgfsetbuttcap%
\pgfsetroundjoin%
\definecolor{currentfill}{rgb}{0.333333,0.333333,0.333333}%
\pgfsetfillcolor{currentfill}%
\pgfsetlinewidth{0.803000pt}%
\definecolor{currentstroke}{rgb}{0.333333,0.333333,0.333333}%
\pgfsetstrokecolor{currentstroke}%
\pgfsetdash{}{0pt}%
\pgfsys@defobject{currentmarker}{\pgfqpoint{0.000000in}{-0.048611in}}{\pgfqpoint{0.000000in}{0.000000in}}{%
\pgfpathmoveto{\pgfqpoint{0.000000in}{0.000000in}}%
\pgfpathlineto{\pgfqpoint{0.000000in}{-0.048611in}}%
\pgfusepath{stroke,fill}%
}%
\begin{pgfscope}%
\pgfsys@transformshift{8.262688in}{1.263068in}%
\pgfsys@useobject{currentmarker}{}%
\end{pgfscope}%
\end{pgfscope}%
\begin{pgfscope}%
\definecolor{textcolor}{rgb}{0.333333,0.333333,0.333333}%
\pgfsetstrokecolor{textcolor}%
\pgfsetfillcolor{textcolor}%
\pgftext[x=8.329239in, y=0.100000in, left, base,rotate=90.000000]{\color{textcolor}\rmfamily\fontsize{16.000000}{19.200000}\selectfont mga-3-10\%}%
\end{pgfscope}%
\begin{pgfscope}%
\pgfpathrectangle{\pgfqpoint{6.379242in}{1.263068in}}{\pgfqpoint{2.589738in}{3.079750in}}%
\pgfusepath{clip}%
\pgfsetrectcap%
\pgfsetroundjoin%
\pgfsetlinewidth{0.803000pt}%
\definecolor{currentstroke}{rgb}{1.000000,1.000000,1.000000}%
\pgfsetstrokecolor{currentstroke}%
\pgfsetstrokeopacity{0.000000}%
\pgfsetdash{}{0pt}%
\pgfpathmoveto{\pgfqpoint{8.851264in}{1.263068in}}%
\pgfpathlineto{\pgfqpoint{8.851264in}{4.342817in}}%
\pgfusepath{stroke}%
\end{pgfscope}%
\begin{pgfscope}%
\pgfsetbuttcap%
\pgfsetroundjoin%
\definecolor{currentfill}{rgb}{0.333333,0.333333,0.333333}%
\pgfsetfillcolor{currentfill}%
\pgfsetlinewidth{0.803000pt}%
\definecolor{currentstroke}{rgb}{0.333333,0.333333,0.333333}%
\pgfsetstrokecolor{currentstroke}%
\pgfsetdash{}{0pt}%
\pgfsys@defobject{currentmarker}{\pgfqpoint{0.000000in}{-0.048611in}}{\pgfqpoint{0.000000in}{0.000000in}}{%
\pgfpathmoveto{\pgfqpoint{0.000000in}{0.000000in}}%
\pgfpathlineto{\pgfqpoint{0.000000in}{-0.048611in}}%
\pgfusepath{stroke,fill}%
}%
\begin{pgfscope}%
\pgfsys@transformshift{8.851264in}{1.263068in}%
\pgfsys@useobject{currentmarker}{}%
\end{pgfscope}%
\end{pgfscope}%
\begin{pgfscope}%
\definecolor{textcolor}{rgb}{0.333333,0.333333,0.333333}%
\pgfsetstrokecolor{textcolor}%
\pgfsetfillcolor{textcolor}%
\pgftext[x=8.917815in, y=0.100000in, left, base,rotate=90.000000]{\color{textcolor}\rmfamily\fontsize{16.000000}{19.200000}\selectfont mga-4-10\%}%
\end{pgfscope}%
\begin{pgfscope}%
\pgfpathrectangle{\pgfqpoint{6.379242in}{1.263068in}}{\pgfqpoint{2.589738in}{3.079750in}}%
\pgfusepath{clip}%
\pgfsetrectcap%
\pgfsetroundjoin%
\pgfsetlinewidth{0.803000pt}%
\definecolor{currentstroke}{rgb}{1.000000,1.000000,1.000000}%
\pgfsetstrokecolor{currentstroke}%
\pgfsetdash{}{0pt}%
\pgfpathmoveto{\pgfqpoint{6.379242in}{1.403056in}}%
\pgfpathlineto{\pgfqpoint{8.968980in}{1.403056in}}%
\pgfusepath{stroke}%
\end{pgfscope}%
\begin{pgfscope}%
\pgfsetbuttcap%
\pgfsetroundjoin%
\definecolor{currentfill}{rgb}{0.333333,0.333333,0.333333}%
\pgfsetfillcolor{currentfill}%
\pgfsetlinewidth{0.803000pt}%
\definecolor{currentstroke}{rgb}{0.333333,0.333333,0.333333}%
\pgfsetstrokecolor{currentstroke}%
\pgfsetdash{}{0pt}%
\pgfsys@defobject{currentmarker}{\pgfqpoint{-0.048611in}{0.000000in}}{\pgfqpoint{-0.000000in}{0.000000in}}{%
\pgfpathmoveto{\pgfqpoint{-0.000000in}{0.000000in}}%
\pgfpathlineto{\pgfqpoint{-0.048611in}{0.000000in}}%
\pgfusepath{stroke,fill}%
}%
\begin{pgfscope}%
\pgfsys@transformshift{6.379242in}{1.403056in}%
\pgfsys@useobject{currentmarker}{}%
\end{pgfscope}%
\end{pgfscope}%
\begin{pgfscope}%
\pgfpathrectangle{\pgfqpoint{6.379242in}{1.263068in}}{\pgfqpoint{2.589738in}{3.079750in}}%
\pgfusepath{clip}%
\pgfsetrectcap%
\pgfsetroundjoin%
\pgfsetlinewidth{0.803000pt}%
\definecolor{currentstroke}{rgb}{1.000000,1.000000,1.000000}%
\pgfsetstrokecolor{currentstroke}%
\pgfsetdash{}{0pt}%
\pgfpathmoveto{\pgfqpoint{6.379242in}{1.858235in}}%
\pgfpathlineto{\pgfqpoint{8.968980in}{1.858235in}}%
\pgfusepath{stroke}%
\end{pgfscope}%
\begin{pgfscope}%
\pgfsetbuttcap%
\pgfsetroundjoin%
\definecolor{currentfill}{rgb}{0.333333,0.333333,0.333333}%
\pgfsetfillcolor{currentfill}%
\pgfsetlinewidth{0.803000pt}%
\definecolor{currentstroke}{rgb}{0.333333,0.333333,0.333333}%
\pgfsetstrokecolor{currentstroke}%
\pgfsetdash{}{0pt}%
\pgfsys@defobject{currentmarker}{\pgfqpoint{-0.048611in}{0.000000in}}{\pgfqpoint{-0.000000in}{0.000000in}}{%
\pgfpathmoveto{\pgfqpoint{-0.000000in}{0.000000in}}%
\pgfpathlineto{\pgfqpoint{-0.048611in}{0.000000in}}%
\pgfusepath{stroke,fill}%
}%
\begin{pgfscope}%
\pgfsys@transformshift{6.379242in}{1.858235in}%
\pgfsys@useobject{currentmarker}{}%
\end{pgfscope}%
\end{pgfscope}%
\begin{pgfscope}%
\pgfpathrectangle{\pgfqpoint{6.379242in}{1.263068in}}{\pgfqpoint{2.589738in}{3.079750in}}%
\pgfusepath{clip}%
\pgfsetrectcap%
\pgfsetroundjoin%
\pgfsetlinewidth{0.803000pt}%
\definecolor{currentstroke}{rgb}{1.000000,1.000000,1.000000}%
\pgfsetstrokecolor{currentstroke}%
\pgfsetdash{}{0pt}%
\pgfpathmoveto{\pgfqpoint{6.379242in}{2.313414in}}%
\pgfpathlineto{\pgfqpoint{8.968980in}{2.313414in}}%
\pgfusepath{stroke}%
\end{pgfscope}%
\begin{pgfscope}%
\pgfsetbuttcap%
\pgfsetroundjoin%
\definecolor{currentfill}{rgb}{0.333333,0.333333,0.333333}%
\pgfsetfillcolor{currentfill}%
\pgfsetlinewidth{0.803000pt}%
\definecolor{currentstroke}{rgb}{0.333333,0.333333,0.333333}%
\pgfsetstrokecolor{currentstroke}%
\pgfsetdash{}{0pt}%
\pgfsys@defobject{currentmarker}{\pgfqpoint{-0.048611in}{0.000000in}}{\pgfqpoint{-0.000000in}{0.000000in}}{%
\pgfpathmoveto{\pgfqpoint{-0.000000in}{0.000000in}}%
\pgfpathlineto{\pgfqpoint{-0.048611in}{0.000000in}}%
\pgfusepath{stroke,fill}%
}%
\begin{pgfscope}%
\pgfsys@transformshift{6.379242in}{2.313414in}%
\pgfsys@useobject{currentmarker}{}%
\end{pgfscope}%
\end{pgfscope}%
\begin{pgfscope}%
\pgfpathrectangle{\pgfqpoint{6.379242in}{1.263068in}}{\pgfqpoint{2.589738in}{3.079750in}}%
\pgfusepath{clip}%
\pgfsetrectcap%
\pgfsetroundjoin%
\pgfsetlinewidth{0.803000pt}%
\definecolor{currentstroke}{rgb}{1.000000,1.000000,1.000000}%
\pgfsetstrokecolor{currentstroke}%
\pgfsetdash{}{0pt}%
\pgfpathmoveto{\pgfqpoint{6.379242in}{2.768592in}}%
\pgfpathlineto{\pgfqpoint{8.968980in}{2.768592in}}%
\pgfusepath{stroke}%
\end{pgfscope}%
\begin{pgfscope}%
\pgfsetbuttcap%
\pgfsetroundjoin%
\definecolor{currentfill}{rgb}{0.333333,0.333333,0.333333}%
\pgfsetfillcolor{currentfill}%
\pgfsetlinewidth{0.803000pt}%
\definecolor{currentstroke}{rgb}{0.333333,0.333333,0.333333}%
\pgfsetstrokecolor{currentstroke}%
\pgfsetdash{}{0pt}%
\pgfsys@defobject{currentmarker}{\pgfqpoint{-0.048611in}{0.000000in}}{\pgfqpoint{-0.000000in}{0.000000in}}{%
\pgfpathmoveto{\pgfqpoint{-0.000000in}{0.000000in}}%
\pgfpathlineto{\pgfqpoint{-0.048611in}{0.000000in}}%
\pgfusepath{stroke,fill}%
}%
\begin{pgfscope}%
\pgfsys@transformshift{6.379242in}{2.768592in}%
\pgfsys@useobject{currentmarker}{}%
\end{pgfscope}%
\end{pgfscope}%
\begin{pgfscope}%
\pgfpathrectangle{\pgfqpoint{6.379242in}{1.263068in}}{\pgfqpoint{2.589738in}{3.079750in}}%
\pgfusepath{clip}%
\pgfsetrectcap%
\pgfsetroundjoin%
\pgfsetlinewidth{0.803000pt}%
\definecolor{currentstroke}{rgb}{1.000000,1.000000,1.000000}%
\pgfsetstrokecolor{currentstroke}%
\pgfsetdash{}{0pt}%
\pgfpathmoveto{\pgfqpoint{6.379242in}{3.223771in}}%
\pgfpathlineto{\pgfqpoint{8.968980in}{3.223771in}}%
\pgfusepath{stroke}%
\end{pgfscope}%
\begin{pgfscope}%
\pgfsetbuttcap%
\pgfsetroundjoin%
\definecolor{currentfill}{rgb}{0.333333,0.333333,0.333333}%
\pgfsetfillcolor{currentfill}%
\pgfsetlinewidth{0.803000pt}%
\definecolor{currentstroke}{rgb}{0.333333,0.333333,0.333333}%
\pgfsetstrokecolor{currentstroke}%
\pgfsetdash{}{0pt}%
\pgfsys@defobject{currentmarker}{\pgfqpoint{-0.048611in}{0.000000in}}{\pgfqpoint{-0.000000in}{0.000000in}}{%
\pgfpathmoveto{\pgfqpoint{-0.000000in}{0.000000in}}%
\pgfpathlineto{\pgfqpoint{-0.048611in}{0.000000in}}%
\pgfusepath{stroke,fill}%
}%
\begin{pgfscope}%
\pgfsys@transformshift{6.379242in}{3.223771in}%
\pgfsys@useobject{currentmarker}{}%
\end{pgfscope}%
\end{pgfscope}%
\begin{pgfscope}%
\pgfpathrectangle{\pgfqpoint{6.379242in}{1.263068in}}{\pgfqpoint{2.589738in}{3.079750in}}%
\pgfusepath{clip}%
\pgfsetrectcap%
\pgfsetroundjoin%
\pgfsetlinewidth{0.803000pt}%
\definecolor{currentstroke}{rgb}{1.000000,1.000000,1.000000}%
\pgfsetstrokecolor{currentstroke}%
\pgfsetdash{}{0pt}%
\pgfpathmoveto{\pgfqpoint{6.379242in}{3.678950in}}%
\pgfpathlineto{\pgfqpoint{8.968980in}{3.678950in}}%
\pgfusepath{stroke}%
\end{pgfscope}%
\begin{pgfscope}%
\pgfsetbuttcap%
\pgfsetroundjoin%
\definecolor{currentfill}{rgb}{0.333333,0.333333,0.333333}%
\pgfsetfillcolor{currentfill}%
\pgfsetlinewidth{0.803000pt}%
\definecolor{currentstroke}{rgb}{0.333333,0.333333,0.333333}%
\pgfsetstrokecolor{currentstroke}%
\pgfsetdash{}{0pt}%
\pgfsys@defobject{currentmarker}{\pgfqpoint{-0.048611in}{0.000000in}}{\pgfqpoint{-0.000000in}{0.000000in}}{%
\pgfpathmoveto{\pgfqpoint{-0.000000in}{0.000000in}}%
\pgfpathlineto{\pgfqpoint{-0.048611in}{0.000000in}}%
\pgfusepath{stroke,fill}%
}%
\begin{pgfscope}%
\pgfsys@transformshift{6.379242in}{3.678950in}%
\pgfsys@useobject{currentmarker}{}%
\end{pgfscope}%
\end{pgfscope}%
\begin{pgfscope}%
\pgfpathrectangle{\pgfqpoint{6.379242in}{1.263068in}}{\pgfqpoint{2.589738in}{3.079750in}}%
\pgfusepath{clip}%
\pgfsetrectcap%
\pgfsetroundjoin%
\pgfsetlinewidth{0.803000pt}%
\definecolor{currentstroke}{rgb}{1.000000,1.000000,1.000000}%
\pgfsetstrokecolor{currentstroke}%
\pgfsetdash{}{0pt}%
\pgfpathmoveto{\pgfqpoint{6.379242in}{4.134128in}}%
\pgfpathlineto{\pgfqpoint{8.968980in}{4.134128in}}%
\pgfusepath{stroke}%
\end{pgfscope}%
\begin{pgfscope}%
\pgfsetbuttcap%
\pgfsetroundjoin%
\definecolor{currentfill}{rgb}{0.333333,0.333333,0.333333}%
\pgfsetfillcolor{currentfill}%
\pgfsetlinewidth{0.803000pt}%
\definecolor{currentstroke}{rgb}{0.333333,0.333333,0.333333}%
\pgfsetstrokecolor{currentstroke}%
\pgfsetdash{}{0pt}%
\pgfsys@defobject{currentmarker}{\pgfqpoint{-0.048611in}{0.000000in}}{\pgfqpoint{-0.000000in}{0.000000in}}{%
\pgfpathmoveto{\pgfqpoint{-0.000000in}{0.000000in}}%
\pgfpathlineto{\pgfqpoint{-0.048611in}{0.000000in}}%
\pgfusepath{stroke,fill}%
}%
\begin{pgfscope}%
\pgfsys@transformshift{6.379242in}{4.134128in}%
\pgfsys@useobject{currentmarker}{}%
\end{pgfscope}%
\end{pgfscope}%
\begin{pgfscope}%
\pgfpathrectangle{\pgfqpoint{6.379242in}{1.263068in}}{\pgfqpoint{2.589738in}{3.079750in}}%
\pgfusepath{clip}%
\pgfsetbuttcap%
\pgfsetroundjoin%
\pgfsetlinewidth{1.505625pt}%
\definecolor{currentstroke}{rgb}{0.839216,0.152941,0.156863}%
\pgfsetstrokecolor{currentstroke}%
\pgfsetdash{{5.550000pt}{2.400000pt}}{0.000000pt}%
\pgfpathmoveto{\pgfqpoint{6.496957in}{4.005751in}}%
\pgfpathlineto{\pgfqpoint{7.085534in}{1.403056in}}%
\pgfpathlineto{\pgfqpoint{7.674111in}{1.403056in}}%
\pgfpathlineto{\pgfqpoint{8.262688in}{1.403056in}}%
\pgfpathlineto{\pgfqpoint{8.851264in}{1.403056in}}%
\pgfusepath{stroke}%
\end{pgfscope}%
\begin{pgfscope}%
\pgfpathrectangle{\pgfqpoint{6.379242in}{1.263068in}}{\pgfqpoint{2.589738in}{3.079750in}}%
\pgfusepath{clip}%
\pgfsetbuttcap%
\pgfsetroundjoin%
\definecolor{currentfill}{rgb}{0.839216,0.152941,0.156863}%
\pgfsetfillcolor{currentfill}%
\pgfsetlinewidth{1.003750pt}%
\definecolor{currentstroke}{rgb}{0.839216,0.152941,0.156863}%
\pgfsetstrokecolor{currentstroke}%
\pgfsetdash{}{0pt}%
\pgfsys@defobject{currentmarker}{\pgfqpoint{-0.041667in}{-0.041667in}}{\pgfqpoint{0.041667in}{0.041667in}}{%
\pgfpathmoveto{\pgfqpoint{0.000000in}{-0.041667in}}%
\pgfpathcurveto{\pgfqpoint{0.011050in}{-0.041667in}}{\pgfqpoint{0.021649in}{-0.037276in}}{\pgfqpoint{0.029463in}{-0.029463in}}%
\pgfpathcurveto{\pgfqpoint{0.037276in}{-0.021649in}}{\pgfqpoint{0.041667in}{-0.011050in}}{\pgfqpoint{0.041667in}{0.000000in}}%
\pgfpathcurveto{\pgfqpoint{0.041667in}{0.011050in}}{\pgfqpoint{0.037276in}{0.021649in}}{\pgfqpoint{0.029463in}{0.029463in}}%
\pgfpathcurveto{\pgfqpoint{0.021649in}{0.037276in}}{\pgfqpoint{0.011050in}{0.041667in}}{\pgfqpoint{0.000000in}{0.041667in}}%
\pgfpathcurveto{\pgfqpoint{-0.011050in}{0.041667in}}{\pgfqpoint{-0.021649in}{0.037276in}}{\pgfqpoint{-0.029463in}{0.029463in}}%
\pgfpathcurveto{\pgfqpoint{-0.037276in}{0.021649in}}{\pgfqpoint{-0.041667in}{0.011050in}}{\pgfqpoint{-0.041667in}{0.000000in}}%
\pgfpathcurveto{\pgfqpoint{-0.041667in}{-0.011050in}}{\pgfqpoint{-0.037276in}{-0.021649in}}{\pgfqpoint{-0.029463in}{-0.029463in}}%
\pgfpathcurveto{\pgfqpoint{-0.021649in}{-0.037276in}}{\pgfqpoint{-0.011050in}{-0.041667in}}{\pgfqpoint{0.000000in}{-0.041667in}}%
\pgfpathclose%
\pgfusepath{stroke,fill}%
}%
\begin{pgfscope}%
\pgfsys@transformshift{6.496957in}{4.005751in}%
\pgfsys@useobject{currentmarker}{}%
\end{pgfscope}%
\begin{pgfscope}%
\pgfsys@transformshift{7.085534in}{1.403056in}%
\pgfsys@useobject{currentmarker}{}%
\end{pgfscope}%
\begin{pgfscope}%
\pgfsys@transformshift{7.674111in}{1.403056in}%
\pgfsys@useobject{currentmarker}{}%
\end{pgfscope}%
\begin{pgfscope}%
\pgfsys@transformshift{8.262688in}{1.403056in}%
\pgfsys@useobject{currentmarker}{}%
\end{pgfscope}%
\begin{pgfscope}%
\pgfsys@transformshift{8.851264in}{1.403056in}%
\pgfsys@useobject{currentmarker}{}%
\end{pgfscope}%
\end{pgfscope}%
\begin{pgfscope}%
\pgfpathrectangle{\pgfqpoint{6.379242in}{1.263068in}}{\pgfqpoint{2.589738in}{3.079750in}}%
\pgfusepath{clip}%
\pgfsetbuttcap%
\pgfsetroundjoin%
\pgfsetlinewidth{1.505625pt}%
\definecolor{currentstroke}{rgb}{0.549020,0.337255,0.294118}%
\pgfsetstrokecolor{currentstroke}%
\pgfsetdash{{5.550000pt}{2.400000pt}}{0.000000pt}%
\pgfpathmoveto{\pgfqpoint{6.496957in}{4.031153in}}%
\pgfpathlineto{\pgfqpoint{7.085534in}{4.202822in}}%
\pgfpathlineto{\pgfqpoint{7.674111in}{4.202827in}}%
\pgfpathlineto{\pgfqpoint{8.262688in}{4.202823in}}%
\pgfpathlineto{\pgfqpoint{8.851264in}{4.202821in}}%
\pgfusepath{stroke}%
\end{pgfscope}%
\begin{pgfscope}%
\pgfpathrectangle{\pgfqpoint{6.379242in}{1.263068in}}{\pgfqpoint{2.589738in}{3.079750in}}%
\pgfusepath{clip}%
\pgfsetbuttcap%
\pgfsetroundjoin%
\definecolor{currentfill}{rgb}{0.549020,0.337255,0.294118}%
\pgfsetfillcolor{currentfill}%
\pgfsetlinewidth{1.003750pt}%
\definecolor{currentstroke}{rgb}{0.549020,0.337255,0.294118}%
\pgfsetstrokecolor{currentstroke}%
\pgfsetdash{}{0pt}%
\pgfsys@defobject{currentmarker}{\pgfqpoint{-0.041667in}{-0.041667in}}{\pgfqpoint{0.041667in}{0.041667in}}{%
\pgfpathmoveto{\pgfqpoint{0.000000in}{-0.041667in}}%
\pgfpathcurveto{\pgfqpoint{0.011050in}{-0.041667in}}{\pgfqpoint{0.021649in}{-0.037276in}}{\pgfqpoint{0.029463in}{-0.029463in}}%
\pgfpathcurveto{\pgfqpoint{0.037276in}{-0.021649in}}{\pgfqpoint{0.041667in}{-0.011050in}}{\pgfqpoint{0.041667in}{0.000000in}}%
\pgfpathcurveto{\pgfqpoint{0.041667in}{0.011050in}}{\pgfqpoint{0.037276in}{0.021649in}}{\pgfqpoint{0.029463in}{0.029463in}}%
\pgfpathcurveto{\pgfqpoint{0.021649in}{0.037276in}}{\pgfqpoint{0.011050in}{0.041667in}}{\pgfqpoint{0.000000in}{0.041667in}}%
\pgfpathcurveto{\pgfqpoint{-0.011050in}{0.041667in}}{\pgfqpoint{-0.021649in}{0.037276in}}{\pgfqpoint{-0.029463in}{0.029463in}}%
\pgfpathcurveto{\pgfqpoint{-0.037276in}{0.021649in}}{\pgfqpoint{-0.041667in}{0.011050in}}{\pgfqpoint{-0.041667in}{0.000000in}}%
\pgfpathcurveto{\pgfqpoint{-0.041667in}{-0.011050in}}{\pgfqpoint{-0.037276in}{-0.021649in}}{\pgfqpoint{-0.029463in}{-0.029463in}}%
\pgfpathcurveto{\pgfqpoint{-0.021649in}{-0.037276in}}{\pgfqpoint{-0.011050in}{-0.041667in}}{\pgfqpoint{0.000000in}{-0.041667in}}%
\pgfpathclose%
\pgfusepath{stroke,fill}%
}%
\begin{pgfscope}%
\pgfsys@transformshift{6.496957in}{4.031153in}%
\pgfsys@useobject{currentmarker}{}%
\end{pgfscope}%
\begin{pgfscope}%
\pgfsys@transformshift{7.085534in}{4.202822in}%
\pgfsys@useobject{currentmarker}{}%
\end{pgfscope}%
\begin{pgfscope}%
\pgfsys@transformshift{7.674111in}{4.202827in}%
\pgfsys@useobject{currentmarker}{}%
\end{pgfscope}%
\begin{pgfscope}%
\pgfsys@transformshift{8.262688in}{4.202823in}%
\pgfsys@useobject{currentmarker}{}%
\end{pgfscope}%
\begin{pgfscope}%
\pgfsys@transformshift{8.851264in}{4.202821in}%
\pgfsys@useobject{currentmarker}{}%
\end{pgfscope}%
\end{pgfscope}%
\begin{pgfscope}%
\pgfpathrectangle{\pgfqpoint{6.379242in}{1.263068in}}{\pgfqpoint{2.589738in}{3.079750in}}%
\pgfusepath{clip}%
\pgfsetbuttcap%
\pgfsetroundjoin%
\pgfsetlinewidth{1.505625pt}%
\definecolor{currentstroke}{rgb}{0.411765,0.411765,0.411765}%
\pgfsetstrokecolor{currentstroke}%
\pgfsetdash{{5.550000pt}{2.400000pt}}{0.000000pt}%
\pgfpathmoveto{\pgfqpoint{6.496957in}{1.403056in}}%
\pgfpathlineto{\pgfqpoint{7.085534in}{1.403056in}}%
\pgfpathlineto{\pgfqpoint{7.674111in}{1.403056in}}%
\pgfpathlineto{\pgfqpoint{8.262688in}{1.403056in}}%
\pgfpathlineto{\pgfqpoint{8.851264in}{1.403061in}}%
\pgfusepath{stroke}%
\end{pgfscope}%
\begin{pgfscope}%
\pgfpathrectangle{\pgfqpoint{6.379242in}{1.263068in}}{\pgfqpoint{2.589738in}{3.079750in}}%
\pgfusepath{clip}%
\pgfsetbuttcap%
\pgfsetroundjoin%
\definecolor{currentfill}{rgb}{0.411765,0.411765,0.411765}%
\pgfsetfillcolor{currentfill}%
\pgfsetlinewidth{1.003750pt}%
\definecolor{currentstroke}{rgb}{0.411765,0.411765,0.411765}%
\pgfsetstrokecolor{currentstroke}%
\pgfsetdash{}{0pt}%
\pgfsys@defobject{currentmarker}{\pgfqpoint{-0.041667in}{-0.041667in}}{\pgfqpoint{0.041667in}{0.041667in}}{%
\pgfpathmoveto{\pgfqpoint{0.000000in}{-0.041667in}}%
\pgfpathcurveto{\pgfqpoint{0.011050in}{-0.041667in}}{\pgfqpoint{0.021649in}{-0.037276in}}{\pgfqpoint{0.029463in}{-0.029463in}}%
\pgfpathcurveto{\pgfqpoint{0.037276in}{-0.021649in}}{\pgfqpoint{0.041667in}{-0.011050in}}{\pgfqpoint{0.041667in}{0.000000in}}%
\pgfpathcurveto{\pgfqpoint{0.041667in}{0.011050in}}{\pgfqpoint{0.037276in}{0.021649in}}{\pgfqpoint{0.029463in}{0.029463in}}%
\pgfpathcurveto{\pgfqpoint{0.021649in}{0.037276in}}{\pgfqpoint{0.011050in}{0.041667in}}{\pgfqpoint{0.000000in}{0.041667in}}%
\pgfpathcurveto{\pgfqpoint{-0.011050in}{0.041667in}}{\pgfqpoint{-0.021649in}{0.037276in}}{\pgfqpoint{-0.029463in}{0.029463in}}%
\pgfpathcurveto{\pgfqpoint{-0.037276in}{0.021649in}}{\pgfqpoint{-0.041667in}{0.011050in}}{\pgfqpoint{-0.041667in}{0.000000in}}%
\pgfpathcurveto{\pgfqpoint{-0.041667in}{-0.011050in}}{\pgfqpoint{-0.037276in}{-0.021649in}}{\pgfqpoint{-0.029463in}{-0.029463in}}%
\pgfpathcurveto{\pgfqpoint{-0.021649in}{-0.037276in}}{\pgfqpoint{-0.011050in}{-0.041667in}}{\pgfqpoint{0.000000in}{-0.041667in}}%
\pgfpathclose%
\pgfusepath{stroke,fill}%
}%
\begin{pgfscope}%
\pgfsys@transformshift{6.496957in}{1.403056in}%
\pgfsys@useobject{currentmarker}{}%
\end{pgfscope}%
\begin{pgfscope}%
\pgfsys@transformshift{7.085534in}{1.403056in}%
\pgfsys@useobject{currentmarker}{}%
\end{pgfscope}%
\begin{pgfscope}%
\pgfsys@transformshift{7.674111in}{1.403056in}%
\pgfsys@useobject{currentmarker}{}%
\end{pgfscope}%
\begin{pgfscope}%
\pgfsys@transformshift{8.262688in}{1.403056in}%
\pgfsys@useobject{currentmarker}{}%
\end{pgfscope}%
\begin{pgfscope}%
\pgfsys@transformshift{8.851264in}{1.403061in}%
\pgfsys@useobject{currentmarker}{}%
\end{pgfscope}%
\end{pgfscope}%
\begin{pgfscope}%
\pgfpathrectangle{\pgfqpoint{6.379242in}{1.263068in}}{\pgfqpoint{2.589738in}{3.079750in}}%
\pgfusepath{clip}%
\pgfsetbuttcap%
\pgfsetroundjoin%
\pgfsetlinewidth{1.505625pt}%
\definecolor{currentstroke}{rgb}{0.172549,0.627451,0.172549}%
\pgfsetstrokecolor{currentstroke}%
\pgfsetdash{{5.550000pt}{2.400000pt}}{0.000000pt}%
\pgfpathmoveto{\pgfqpoint{6.496957in}{2.075567in}}%
\pgfpathlineto{\pgfqpoint{7.085534in}{1.403056in}}%
\pgfpathlineto{\pgfqpoint{7.674111in}{1.403068in}}%
\pgfpathlineto{\pgfqpoint{8.262688in}{1.403056in}}%
\pgfpathlineto{\pgfqpoint{8.851264in}{1.403056in}}%
\pgfusepath{stroke}%
\end{pgfscope}%
\begin{pgfscope}%
\pgfpathrectangle{\pgfqpoint{6.379242in}{1.263068in}}{\pgfqpoint{2.589738in}{3.079750in}}%
\pgfusepath{clip}%
\pgfsetbuttcap%
\pgfsetroundjoin%
\definecolor{currentfill}{rgb}{0.172549,0.627451,0.172549}%
\pgfsetfillcolor{currentfill}%
\pgfsetlinewidth{1.003750pt}%
\definecolor{currentstroke}{rgb}{0.172549,0.627451,0.172549}%
\pgfsetstrokecolor{currentstroke}%
\pgfsetdash{}{0pt}%
\pgfsys@defobject{currentmarker}{\pgfqpoint{-0.041667in}{-0.041667in}}{\pgfqpoint{0.041667in}{0.041667in}}{%
\pgfpathmoveto{\pgfqpoint{0.000000in}{-0.041667in}}%
\pgfpathcurveto{\pgfqpoint{0.011050in}{-0.041667in}}{\pgfqpoint{0.021649in}{-0.037276in}}{\pgfqpoint{0.029463in}{-0.029463in}}%
\pgfpathcurveto{\pgfqpoint{0.037276in}{-0.021649in}}{\pgfqpoint{0.041667in}{-0.011050in}}{\pgfqpoint{0.041667in}{0.000000in}}%
\pgfpathcurveto{\pgfqpoint{0.041667in}{0.011050in}}{\pgfqpoint{0.037276in}{0.021649in}}{\pgfqpoint{0.029463in}{0.029463in}}%
\pgfpathcurveto{\pgfqpoint{0.021649in}{0.037276in}}{\pgfqpoint{0.011050in}{0.041667in}}{\pgfqpoint{0.000000in}{0.041667in}}%
\pgfpathcurveto{\pgfqpoint{-0.011050in}{0.041667in}}{\pgfqpoint{-0.021649in}{0.037276in}}{\pgfqpoint{-0.029463in}{0.029463in}}%
\pgfpathcurveto{\pgfqpoint{-0.037276in}{0.021649in}}{\pgfqpoint{-0.041667in}{0.011050in}}{\pgfqpoint{-0.041667in}{0.000000in}}%
\pgfpathcurveto{\pgfqpoint{-0.041667in}{-0.011050in}}{\pgfqpoint{-0.037276in}{-0.021649in}}{\pgfqpoint{-0.029463in}{-0.029463in}}%
\pgfpathcurveto{\pgfqpoint{-0.021649in}{-0.037276in}}{\pgfqpoint{-0.011050in}{-0.041667in}}{\pgfqpoint{0.000000in}{-0.041667in}}%
\pgfpathclose%
\pgfusepath{stroke,fill}%
}%
\begin{pgfscope}%
\pgfsys@transformshift{6.496957in}{2.075567in}%
\pgfsys@useobject{currentmarker}{}%
\end{pgfscope}%
\begin{pgfscope}%
\pgfsys@transformshift{7.085534in}{1.403056in}%
\pgfsys@useobject{currentmarker}{}%
\end{pgfscope}%
\begin{pgfscope}%
\pgfsys@transformshift{7.674111in}{1.403068in}%
\pgfsys@useobject{currentmarker}{}%
\end{pgfscope}%
\begin{pgfscope}%
\pgfsys@transformshift{8.262688in}{1.403056in}%
\pgfsys@useobject{currentmarker}{}%
\end{pgfscope}%
\begin{pgfscope}%
\pgfsys@transformshift{8.851264in}{1.403056in}%
\pgfsys@useobject{currentmarker}{}%
\end{pgfscope}%
\end{pgfscope}%
\begin{pgfscope}%
\pgfpathrectangle{\pgfqpoint{6.379242in}{1.263068in}}{\pgfqpoint{2.589738in}{3.079750in}}%
\pgfusepath{clip}%
\pgfsetbuttcap%
\pgfsetroundjoin%
\pgfsetlinewidth{1.505625pt}%
\definecolor{currentstroke}{rgb}{1.000000,1.000000,0.000000}%
\pgfsetstrokecolor{currentstroke}%
\pgfsetdash{{5.550000pt}{2.400000pt}}{0.000000pt}%
\pgfpathmoveto{\pgfqpoint{6.496957in}{2.704346in}}%
\pgfpathlineto{\pgfqpoint{7.085534in}{2.852132in}}%
\pgfpathlineto{\pgfqpoint{7.674111in}{2.854042in}}%
\pgfpathlineto{\pgfqpoint{8.262688in}{2.852131in}}%
\pgfpathlineto{\pgfqpoint{8.851264in}{2.854578in}}%
\pgfusepath{stroke}%
\end{pgfscope}%
\begin{pgfscope}%
\pgfpathrectangle{\pgfqpoint{6.379242in}{1.263068in}}{\pgfqpoint{2.589738in}{3.079750in}}%
\pgfusepath{clip}%
\pgfsetbuttcap%
\pgfsetroundjoin%
\definecolor{currentfill}{rgb}{1.000000,1.000000,0.000000}%
\pgfsetfillcolor{currentfill}%
\pgfsetlinewidth{1.003750pt}%
\definecolor{currentstroke}{rgb}{1.000000,1.000000,0.000000}%
\pgfsetstrokecolor{currentstroke}%
\pgfsetdash{}{0pt}%
\pgfsys@defobject{currentmarker}{\pgfqpoint{-0.041667in}{-0.041667in}}{\pgfqpoint{0.041667in}{0.041667in}}{%
\pgfpathmoveto{\pgfqpoint{0.000000in}{-0.041667in}}%
\pgfpathcurveto{\pgfqpoint{0.011050in}{-0.041667in}}{\pgfqpoint{0.021649in}{-0.037276in}}{\pgfqpoint{0.029463in}{-0.029463in}}%
\pgfpathcurveto{\pgfqpoint{0.037276in}{-0.021649in}}{\pgfqpoint{0.041667in}{-0.011050in}}{\pgfqpoint{0.041667in}{0.000000in}}%
\pgfpathcurveto{\pgfqpoint{0.041667in}{0.011050in}}{\pgfqpoint{0.037276in}{0.021649in}}{\pgfqpoint{0.029463in}{0.029463in}}%
\pgfpathcurveto{\pgfqpoint{0.021649in}{0.037276in}}{\pgfqpoint{0.011050in}{0.041667in}}{\pgfqpoint{0.000000in}{0.041667in}}%
\pgfpathcurveto{\pgfqpoint{-0.011050in}{0.041667in}}{\pgfqpoint{-0.021649in}{0.037276in}}{\pgfqpoint{-0.029463in}{0.029463in}}%
\pgfpathcurveto{\pgfqpoint{-0.037276in}{0.021649in}}{\pgfqpoint{-0.041667in}{0.011050in}}{\pgfqpoint{-0.041667in}{0.000000in}}%
\pgfpathcurveto{\pgfqpoint{-0.041667in}{-0.011050in}}{\pgfqpoint{-0.037276in}{-0.021649in}}{\pgfqpoint{-0.029463in}{-0.029463in}}%
\pgfpathcurveto{\pgfqpoint{-0.021649in}{-0.037276in}}{\pgfqpoint{-0.011050in}{-0.041667in}}{\pgfqpoint{0.000000in}{-0.041667in}}%
\pgfpathclose%
\pgfusepath{stroke,fill}%
}%
\begin{pgfscope}%
\pgfsys@transformshift{6.496957in}{2.704346in}%
\pgfsys@useobject{currentmarker}{}%
\end{pgfscope}%
\begin{pgfscope}%
\pgfsys@transformshift{7.085534in}{2.852132in}%
\pgfsys@useobject{currentmarker}{}%
\end{pgfscope}%
\begin{pgfscope}%
\pgfsys@transformshift{7.674111in}{2.854042in}%
\pgfsys@useobject{currentmarker}{}%
\end{pgfscope}%
\begin{pgfscope}%
\pgfsys@transformshift{8.262688in}{2.852131in}%
\pgfsys@useobject{currentmarker}{}%
\end{pgfscope}%
\begin{pgfscope}%
\pgfsys@transformshift{8.851264in}{2.854578in}%
\pgfsys@useobject{currentmarker}{}%
\end{pgfscope}%
\end{pgfscope}%
\begin{pgfscope}%
\pgfpathrectangle{\pgfqpoint{6.379242in}{1.263068in}}{\pgfqpoint{2.589738in}{3.079750in}}%
\pgfusepath{clip}%
\pgfsetbuttcap%
\pgfsetroundjoin%
\pgfsetlinewidth{1.505625pt}%
\definecolor{currentstroke}{rgb}{0.121569,0.466667,0.705882}%
\pgfsetstrokecolor{currentstroke}%
\pgfsetdash{{5.550000pt}{2.400000pt}}{0.000000pt}%
\pgfpathmoveto{\pgfqpoint{6.496957in}{3.690329in}}%
\pgfpathlineto{\pgfqpoint{7.085534in}{3.690329in}}%
\pgfpathlineto{\pgfqpoint{7.674111in}{3.690329in}}%
\pgfpathlineto{\pgfqpoint{8.262688in}{3.690329in}}%
\pgfpathlineto{\pgfqpoint{8.851264in}{3.690329in}}%
\pgfusepath{stroke}%
\end{pgfscope}%
\begin{pgfscope}%
\pgfpathrectangle{\pgfqpoint{6.379242in}{1.263068in}}{\pgfqpoint{2.589738in}{3.079750in}}%
\pgfusepath{clip}%
\pgfsetbuttcap%
\pgfsetroundjoin%
\definecolor{currentfill}{rgb}{0.121569,0.466667,0.705882}%
\pgfsetfillcolor{currentfill}%
\pgfsetlinewidth{1.003750pt}%
\definecolor{currentstroke}{rgb}{0.121569,0.466667,0.705882}%
\pgfsetstrokecolor{currentstroke}%
\pgfsetdash{}{0pt}%
\pgfsys@defobject{currentmarker}{\pgfqpoint{-0.041667in}{-0.041667in}}{\pgfqpoint{0.041667in}{0.041667in}}{%
\pgfpathmoveto{\pgfqpoint{0.000000in}{-0.041667in}}%
\pgfpathcurveto{\pgfqpoint{0.011050in}{-0.041667in}}{\pgfqpoint{0.021649in}{-0.037276in}}{\pgfqpoint{0.029463in}{-0.029463in}}%
\pgfpathcurveto{\pgfqpoint{0.037276in}{-0.021649in}}{\pgfqpoint{0.041667in}{-0.011050in}}{\pgfqpoint{0.041667in}{0.000000in}}%
\pgfpathcurveto{\pgfqpoint{0.041667in}{0.011050in}}{\pgfqpoint{0.037276in}{0.021649in}}{\pgfqpoint{0.029463in}{0.029463in}}%
\pgfpathcurveto{\pgfqpoint{0.021649in}{0.037276in}}{\pgfqpoint{0.011050in}{0.041667in}}{\pgfqpoint{0.000000in}{0.041667in}}%
\pgfpathcurveto{\pgfqpoint{-0.011050in}{0.041667in}}{\pgfqpoint{-0.021649in}{0.037276in}}{\pgfqpoint{-0.029463in}{0.029463in}}%
\pgfpathcurveto{\pgfqpoint{-0.037276in}{0.021649in}}{\pgfqpoint{-0.041667in}{0.011050in}}{\pgfqpoint{-0.041667in}{0.000000in}}%
\pgfpathcurveto{\pgfqpoint{-0.041667in}{-0.011050in}}{\pgfqpoint{-0.037276in}{-0.021649in}}{\pgfqpoint{-0.029463in}{-0.029463in}}%
\pgfpathcurveto{\pgfqpoint{-0.021649in}{-0.037276in}}{\pgfqpoint{-0.011050in}{-0.041667in}}{\pgfqpoint{0.000000in}{-0.041667in}}%
\pgfpathclose%
\pgfusepath{stroke,fill}%
}%
\begin{pgfscope}%
\pgfsys@transformshift{6.496957in}{3.690329in}%
\pgfsys@useobject{currentmarker}{}%
\end{pgfscope}%
\begin{pgfscope}%
\pgfsys@transformshift{7.085534in}{3.690329in}%
\pgfsys@useobject{currentmarker}{}%
\end{pgfscope}%
\begin{pgfscope}%
\pgfsys@transformshift{7.674111in}{3.690329in}%
\pgfsys@useobject{currentmarker}{}%
\end{pgfscope}%
\begin{pgfscope}%
\pgfsys@transformshift{8.262688in}{3.690329in}%
\pgfsys@useobject{currentmarker}{}%
\end{pgfscope}%
\begin{pgfscope}%
\pgfsys@transformshift{8.851264in}{3.690329in}%
\pgfsys@useobject{currentmarker}{}%
\end{pgfscope}%
\end{pgfscope}%
\begin{pgfscope}%
\pgfsetrectcap%
\pgfsetmiterjoin%
\pgfsetlinewidth{1.003750pt}%
\definecolor{currentstroke}{rgb}{1.000000,1.000000,1.000000}%
\pgfsetstrokecolor{currentstroke}%
\pgfsetdash{}{0pt}%
\pgfpathmoveto{\pgfqpoint{6.379242in}{1.263067in}}%
\pgfpathlineto{\pgfqpoint{6.379242in}{4.342817in}}%
\pgfusepath{stroke}%
\end{pgfscope}%
\begin{pgfscope}%
\pgfsetrectcap%
\pgfsetmiterjoin%
\pgfsetlinewidth{1.003750pt}%
\definecolor{currentstroke}{rgb}{1.000000,1.000000,1.000000}%
\pgfsetstrokecolor{currentstroke}%
\pgfsetdash{}{0pt}%
\pgfpathmoveto{\pgfqpoint{8.968980in}{1.263067in}}%
\pgfpathlineto{\pgfqpoint{8.968980in}{4.342817in}}%
\pgfusepath{stroke}%
\end{pgfscope}%
\begin{pgfscope}%
\pgfsetrectcap%
\pgfsetmiterjoin%
\pgfsetlinewidth{1.003750pt}%
\definecolor{currentstroke}{rgb}{1.000000,1.000000,1.000000}%
\pgfsetstrokecolor{currentstroke}%
\pgfsetdash{}{0pt}%
\pgfpathmoveto{\pgfqpoint{6.379242in}{1.263068in}}%
\pgfpathlineto{\pgfqpoint{8.968980in}{1.263068in}}%
\pgfusepath{stroke}%
\end{pgfscope}%
\begin{pgfscope}%
\pgfsetrectcap%
\pgfsetmiterjoin%
\pgfsetlinewidth{1.003750pt}%
\definecolor{currentstroke}{rgb}{1.000000,1.000000,1.000000}%
\pgfsetstrokecolor{currentstroke}%
\pgfsetdash{}{0pt}%
\pgfpathmoveto{\pgfqpoint{6.379242in}{4.342817in}}%
\pgfpathlineto{\pgfqpoint{8.968980in}{4.342817in}}%
\pgfusepath{stroke}%
\end{pgfscope}%
\begin{pgfscope}%
\pgfsetbuttcap%
\pgfsetmiterjoin%
\definecolor{currentfill}{rgb}{0.898039,0.898039,0.898039}%
\pgfsetfillcolor{currentfill}%
\pgfsetlinewidth{0.000000pt}%
\definecolor{currentstroke}{rgb}{0.000000,0.000000,0.000000}%
\pgfsetstrokecolor{currentstroke}%
\pgfsetstrokeopacity{0.000000}%
\pgfsetdash{}{0pt}%
\pgfpathmoveto{\pgfqpoint{9.188926in}{1.263068in}}%
\pgfpathlineto{\pgfqpoint{11.778664in}{1.263068in}}%
\pgfpathlineto{\pgfqpoint{11.778664in}{4.342817in}}%
\pgfpathlineto{\pgfqpoint{9.188926in}{4.342817in}}%
\pgfpathclose%
\pgfusepath{fill}%
\end{pgfscope}%
\begin{pgfscope}%
\pgfpathrectangle{\pgfqpoint{9.188926in}{1.263068in}}{\pgfqpoint{2.589738in}{3.079750in}}%
\pgfusepath{clip}%
\pgfsetrectcap%
\pgfsetroundjoin%
\pgfsetlinewidth{0.803000pt}%
\definecolor{currentstroke}{rgb}{1.000000,1.000000,1.000000}%
\pgfsetstrokecolor{currentstroke}%
\pgfsetstrokeopacity{0.000000}%
\pgfsetdash{}{0pt}%
\pgfpathmoveto{\pgfqpoint{9.306642in}{1.263068in}}%
\pgfpathlineto{\pgfqpoint{9.306642in}{4.342817in}}%
\pgfusepath{stroke}%
\end{pgfscope}%
\begin{pgfscope}%
\pgfsetbuttcap%
\pgfsetroundjoin%
\definecolor{currentfill}{rgb}{0.333333,0.333333,0.333333}%
\pgfsetfillcolor{currentfill}%
\pgfsetlinewidth{0.803000pt}%
\definecolor{currentstroke}{rgb}{0.333333,0.333333,0.333333}%
\pgfsetstrokecolor{currentstroke}%
\pgfsetdash{}{0pt}%
\pgfsys@defobject{currentmarker}{\pgfqpoint{0.000000in}{-0.048611in}}{\pgfqpoint{0.000000in}{0.000000in}}{%
\pgfpathmoveto{\pgfqpoint{0.000000in}{0.000000in}}%
\pgfpathlineto{\pgfqpoint{0.000000in}{-0.048611in}}%
\pgfusepath{stroke,fill}%
}%
\begin{pgfscope}%
\pgfsys@transformshift{9.306642in}{1.263068in}%
\pgfsys@useobject{currentmarker}{}%
\end{pgfscope}%
\end{pgfscope}%
\begin{pgfscope}%
\definecolor{textcolor}{rgb}{0.333333,0.333333,0.333333}%
\pgfsetstrokecolor{textcolor}%
\pgfsetfillcolor{textcolor}%
\pgftext[x=9.373193in, y=0.100000in, left, base,rotate=90.000000]{\color{textcolor}\rmfamily\fontsize{16.000000}{19.200000}\selectfont mga-0-20\%}%
\end{pgfscope}%
\begin{pgfscope}%
\pgfpathrectangle{\pgfqpoint{9.188926in}{1.263068in}}{\pgfqpoint{2.589738in}{3.079750in}}%
\pgfusepath{clip}%
\pgfsetrectcap%
\pgfsetroundjoin%
\pgfsetlinewidth{0.803000pt}%
\definecolor{currentstroke}{rgb}{1.000000,1.000000,1.000000}%
\pgfsetstrokecolor{currentstroke}%
\pgfsetstrokeopacity{0.000000}%
\pgfsetdash{}{0pt}%
\pgfpathmoveto{\pgfqpoint{9.895219in}{1.263068in}}%
\pgfpathlineto{\pgfqpoint{9.895219in}{4.342817in}}%
\pgfusepath{stroke}%
\end{pgfscope}%
\begin{pgfscope}%
\pgfsetbuttcap%
\pgfsetroundjoin%
\definecolor{currentfill}{rgb}{0.333333,0.333333,0.333333}%
\pgfsetfillcolor{currentfill}%
\pgfsetlinewidth{0.803000pt}%
\definecolor{currentstroke}{rgb}{0.333333,0.333333,0.333333}%
\pgfsetstrokecolor{currentstroke}%
\pgfsetdash{}{0pt}%
\pgfsys@defobject{currentmarker}{\pgfqpoint{0.000000in}{-0.048611in}}{\pgfqpoint{0.000000in}{0.000000in}}{%
\pgfpathmoveto{\pgfqpoint{0.000000in}{0.000000in}}%
\pgfpathlineto{\pgfqpoint{0.000000in}{-0.048611in}}%
\pgfusepath{stroke,fill}%
}%
\begin{pgfscope}%
\pgfsys@transformshift{9.895219in}{1.263068in}%
\pgfsys@useobject{currentmarker}{}%
\end{pgfscope}%
\end{pgfscope}%
\begin{pgfscope}%
\definecolor{textcolor}{rgb}{0.333333,0.333333,0.333333}%
\pgfsetstrokecolor{textcolor}%
\pgfsetfillcolor{textcolor}%
\pgftext[x=9.961770in, y=0.100000in, left, base,rotate=90.000000]{\color{textcolor}\rmfamily\fontsize{16.000000}{19.200000}\selectfont mga-1-20\%}%
\end{pgfscope}%
\begin{pgfscope}%
\pgfpathrectangle{\pgfqpoint{9.188926in}{1.263068in}}{\pgfqpoint{2.589738in}{3.079750in}}%
\pgfusepath{clip}%
\pgfsetrectcap%
\pgfsetroundjoin%
\pgfsetlinewidth{0.803000pt}%
\definecolor{currentstroke}{rgb}{1.000000,1.000000,1.000000}%
\pgfsetstrokecolor{currentstroke}%
\pgfsetstrokeopacity{0.000000}%
\pgfsetdash{}{0pt}%
\pgfpathmoveto{\pgfqpoint{10.483795in}{1.263068in}}%
\pgfpathlineto{\pgfqpoint{10.483795in}{4.342817in}}%
\pgfusepath{stroke}%
\end{pgfscope}%
\begin{pgfscope}%
\pgfsetbuttcap%
\pgfsetroundjoin%
\definecolor{currentfill}{rgb}{0.333333,0.333333,0.333333}%
\pgfsetfillcolor{currentfill}%
\pgfsetlinewidth{0.803000pt}%
\definecolor{currentstroke}{rgb}{0.333333,0.333333,0.333333}%
\pgfsetstrokecolor{currentstroke}%
\pgfsetdash{}{0pt}%
\pgfsys@defobject{currentmarker}{\pgfqpoint{0.000000in}{-0.048611in}}{\pgfqpoint{0.000000in}{0.000000in}}{%
\pgfpathmoveto{\pgfqpoint{0.000000in}{0.000000in}}%
\pgfpathlineto{\pgfqpoint{0.000000in}{-0.048611in}}%
\pgfusepath{stroke,fill}%
}%
\begin{pgfscope}%
\pgfsys@transformshift{10.483795in}{1.263068in}%
\pgfsys@useobject{currentmarker}{}%
\end{pgfscope}%
\end{pgfscope}%
\begin{pgfscope}%
\definecolor{textcolor}{rgb}{0.333333,0.333333,0.333333}%
\pgfsetstrokecolor{textcolor}%
\pgfsetfillcolor{textcolor}%
\pgftext[x=10.550346in, y=0.100000in, left, base,rotate=90.000000]{\color{textcolor}\rmfamily\fontsize{16.000000}{19.200000}\selectfont mga-2-20\%}%
\end{pgfscope}%
\begin{pgfscope}%
\pgfpathrectangle{\pgfqpoint{9.188926in}{1.263068in}}{\pgfqpoint{2.589738in}{3.079750in}}%
\pgfusepath{clip}%
\pgfsetrectcap%
\pgfsetroundjoin%
\pgfsetlinewidth{0.803000pt}%
\definecolor{currentstroke}{rgb}{1.000000,1.000000,1.000000}%
\pgfsetstrokecolor{currentstroke}%
\pgfsetstrokeopacity{0.000000}%
\pgfsetdash{}{0pt}%
\pgfpathmoveto{\pgfqpoint{11.072372in}{1.263068in}}%
\pgfpathlineto{\pgfqpoint{11.072372in}{4.342817in}}%
\pgfusepath{stroke}%
\end{pgfscope}%
\begin{pgfscope}%
\pgfsetbuttcap%
\pgfsetroundjoin%
\definecolor{currentfill}{rgb}{0.333333,0.333333,0.333333}%
\pgfsetfillcolor{currentfill}%
\pgfsetlinewidth{0.803000pt}%
\definecolor{currentstroke}{rgb}{0.333333,0.333333,0.333333}%
\pgfsetstrokecolor{currentstroke}%
\pgfsetdash{}{0pt}%
\pgfsys@defobject{currentmarker}{\pgfqpoint{0.000000in}{-0.048611in}}{\pgfqpoint{0.000000in}{0.000000in}}{%
\pgfpathmoveto{\pgfqpoint{0.000000in}{0.000000in}}%
\pgfpathlineto{\pgfqpoint{0.000000in}{-0.048611in}}%
\pgfusepath{stroke,fill}%
}%
\begin{pgfscope}%
\pgfsys@transformshift{11.072372in}{1.263068in}%
\pgfsys@useobject{currentmarker}{}%
\end{pgfscope}%
\end{pgfscope}%
\begin{pgfscope}%
\definecolor{textcolor}{rgb}{0.333333,0.333333,0.333333}%
\pgfsetstrokecolor{textcolor}%
\pgfsetfillcolor{textcolor}%
\pgftext[x=11.138923in, y=0.100000in, left, base,rotate=90.000000]{\color{textcolor}\rmfamily\fontsize{16.000000}{19.200000}\selectfont mga-3-20\%}%
\end{pgfscope}%
\begin{pgfscope}%
\pgfpathrectangle{\pgfqpoint{9.188926in}{1.263068in}}{\pgfqpoint{2.589738in}{3.079750in}}%
\pgfusepath{clip}%
\pgfsetrectcap%
\pgfsetroundjoin%
\pgfsetlinewidth{0.803000pt}%
\definecolor{currentstroke}{rgb}{1.000000,1.000000,1.000000}%
\pgfsetstrokecolor{currentstroke}%
\pgfsetstrokeopacity{0.000000}%
\pgfsetdash{}{0pt}%
\pgfpathmoveto{\pgfqpoint{11.660949in}{1.263068in}}%
\pgfpathlineto{\pgfqpoint{11.660949in}{4.342817in}}%
\pgfusepath{stroke}%
\end{pgfscope}%
\begin{pgfscope}%
\pgfsetbuttcap%
\pgfsetroundjoin%
\definecolor{currentfill}{rgb}{0.333333,0.333333,0.333333}%
\pgfsetfillcolor{currentfill}%
\pgfsetlinewidth{0.803000pt}%
\definecolor{currentstroke}{rgb}{0.333333,0.333333,0.333333}%
\pgfsetstrokecolor{currentstroke}%
\pgfsetdash{}{0pt}%
\pgfsys@defobject{currentmarker}{\pgfqpoint{0.000000in}{-0.048611in}}{\pgfqpoint{0.000000in}{0.000000in}}{%
\pgfpathmoveto{\pgfqpoint{0.000000in}{0.000000in}}%
\pgfpathlineto{\pgfqpoint{0.000000in}{-0.048611in}}%
\pgfusepath{stroke,fill}%
}%
\begin{pgfscope}%
\pgfsys@transformshift{11.660949in}{1.263068in}%
\pgfsys@useobject{currentmarker}{}%
\end{pgfscope}%
\end{pgfscope}%
\begin{pgfscope}%
\definecolor{textcolor}{rgb}{0.333333,0.333333,0.333333}%
\pgfsetstrokecolor{textcolor}%
\pgfsetfillcolor{textcolor}%
\pgftext[x=11.727500in, y=0.100000in, left, base,rotate=90.000000]{\color{textcolor}\rmfamily\fontsize{16.000000}{19.200000}\selectfont mga-4-20\%}%
\end{pgfscope}%
\begin{pgfscope}%
\pgfpathrectangle{\pgfqpoint{9.188926in}{1.263068in}}{\pgfqpoint{2.589738in}{3.079750in}}%
\pgfusepath{clip}%
\pgfsetrectcap%
\pgfsetroundjoin%
\pgfsetlinewidth{0.803000pt}%
\definecolor{currentstroke}{rgb}{1.000000,1.000000,1.000000}%
\pgfsetstrokecolor{currentstroke}%
\pgfsetdash{}{0pt}%
\pgfpathmoveto{\pgfqpoint{9.188926in}{1.403056in}}%
\pgfpathlineto{\pgfqpoint{11.778664in}{1.403056in}}%
\pgfusepath{stroke}%
\end{pgfscope}%
\begin{pgfscope}%
\pgfsetbuttcap%
\pgfsetroundjoin%
\definecolor{currentfill}{rgb}{0.333333,0.333333,0.333333}%
\pgfsetfillcolor{currentfill}%
\pgfsetlinewidth{0.803000pt}%
\definecolor{currentstroke}{rgb}{0.333333,0.333333,0.333333}%
\pgfsetstrokecolor{currentstroke}%
\pgfsetdash{}{0pt}%
\pgfsys@defobject{currentmarker}{\pgfqpoint{-0.048611in}{0.000000in}}{\pgfqpoint{-0.000000in}{0.000000in}}{%
\pgfpathmoveto{\pgfqpoint{-0.000000in}{0.000000in}}%
\pgfpathlineto{\pgfqpoint{-0.048611in}{0.000000in}}%
\pgfusepath{stroke,fill}%
}%
\begin{pgfscope}%
\pgfsys@transformshift{9.188926in}{1.403056in}%
\pgfsys@useobject{currentmarker}{}%
\end{pgfscope}%
\end{pgfscope}%
\begin{pgfscope}%
\pgfpathrectangle{\pgfqpoint{9.188926in}{1.263068in}}{\pgfqpoint{2.589738in}{3.079750in}}%
\pgfusepath{clip}%
\pgfsetrectcap%
\pgfsetroundjoin%
\pgfsetlinewidth{0.803000pt}%
\definecolor{currentstroke}{rgb}{1.000000,1.000000,1.000000}%
\pgfsetstrokecolor{currentstroke}%
\pgfsetdash{}{0pt}%
\pgfpathmoveto{\pgfqpoint{9.188926in}{1.858235in}}%
\pgfpathlineto{\pgfqpoint{11.778664in}{1.858235in}}%
\pgfusepath{stroke}%
\end{pgfscope}%
\begin{pgfscope}%
\pgfsetbuttcap%
\pgfsetroundjoin%
\definecolor{currentfill}{rgb}{0.333333,0.333333,0.333333}%
\pgfsetfillcolor{currentfill}%
\pgfsetlinewidth{0.803000pt}%
\definecolor{currentstroke}{rgb}{0.333333,0.333333,0.333333}%
\pgfsetstrokecolor{currentstroke}%
\pgfsetdash{}{0pt}%
\pgfsys@defobject{currentmarker}{\pgfqpoint{-0.048611in}{0.000000in}}{\pgfqpoint{-0.000000in}{0.000000in}}{%
\pgfpathmoveto{\pgfqpoint{-0.000000in}{0.000000in}}%
\pgfpathlineto{\pgfqpoint{-0.048611in}{0.000000in}}%
\pgfusepath{stroke,fill}%
}%
\begin{pgfscope}%
\pgfsys@transformshift{9.188926in}{1.858235in}%
\pgfsys@useobject{currentmarker}{}%
\end{pgfscope}%
\end{pgfscope}%
\begin{pgfscope}%
\pgfpathrectangle{\pgfqpoint{9.188926in}{1.263068in}}{\pgfqpoint{2.589738in}{3.079750in}}%
\pgfusepath{clip}%
\pgfsetrectcap%
\pgfsetroundjoin%
\pgfsetlinewidth{0.803000pt}%
\definecolor{currentstroke}{rgb}{1.000000,1.000000,1.000000}%
\pgfsetstrokecolor{currentstroke}%
\pgfsetdash{}{0pt}%
\pgfpathmoveto{\pgfqpoint{9.188926in}{2.313414in}}%
\pgfpathlineto{\pgfqpoint{11.778664in}{2.313414in}}%
\pgfusepath{stroke}%
\end{pgfscope}%
\begin{pgfscope}%
\pgfsetbuttcap%
\pgfsetroundjoin%
\definecolor{currentfill}{rgb}{0.333333,0.333333,0.333333}%
\pgfsetfillcolor{currentfill}%
\pgfsetlinewidth{0.803000pt}%
\definecolor{currentstroke}{rgb}{0.333333,0.333333,0.333333}%
\pgfsetstrokecolor{currentstroke}%
\pgfsetdash{}{0pt}%
\pgfsys@defobject{currentmarker}{\pgfqpoint{-0.048611in}{0.000000in}}{\pgfqpoint{-0.000000in}{0.000000in}}{%
\pgfpathmoveto{\pgfqpoint{-0.000000in}{0.000000in}}%
\pgfpathlineto{\pgfqpoint{-0.048611in}{0.000000in}}%
\pgfusepath{stroke,fill}%
}%
\begin{pgfscope}%
\pgfsys@transformshift{9.188926in}{2.313414in}%
\pgfsys@useobject{currentmarker}{}%
\end{pgfscope}%
\end{pgfscope}%
\begin{pgfscope}%
\pgfpathrectangle{\pgfqpoint{9.188926in}{1.263068in}}{\pgfqpoint{2.589738in}{3.079750in}}%
\pgfusepath{clip}%
\pgfsetrectcap%
\pgfsetroundjoin%
\pgfsetlinewidth{0.803000pt}%
\definecolor{currentstroke}{rgb}{1.000000,1.000000,1.000000}%
\pgfsetstrokecolor{currentstroke}%
\pgfsetdash{}{0pt}%
\pgfpathmoveto{\pgfqpoint{9.188926in}{2.768592in}}%
\pgfpathlineto{\pgfqpoint{11.778664in}{2.768592in}}%
\pgfusepath{stroke}%
\end{pgfscope}%
\begin{pgfscope}%
\pgfsetbuttcap%
\pgfsetroundjoin%
\definecolor{currentfill}{rgb}{0.333333,0.333333,0.333333}%
\pgfsetfillcolor{currentfill}%
\pgfsetlinewidth{0.803000pt}%
\definecolor{currentstroke}{rgb}{0.333333,0.333333,0.333333}%
\pgfsetstrokecolor{currentstroke}%
\pgfsetdash{}{0pt}%
\pgfsys@defobject{currentmarker}{\pgfqpoint{-0.048611in}{0.000000in}}{\pgfqpoint{-0.000000in}{0.000000in}}{%
\pgfpathmoveto{\pgfqpoint{-0.000000in}{0.000000in}}%
\pgfpathlineto{\pgfqpoint{-0.048611in}{0.000000in}}%
\pgfusepath{stroke,fill}%
}%
\begin{pgfscope}%
\pgfsys@transformshift{9.188926in}{2.768592in}%
\pgfsys@useobject{currentmarker}{}%
\end{pgfscope}%
\end{pgfscope}%
\begin{pgfscope}%
\pgfpathrectangle{\pgfqpoint{9.188926in}{1.263068in}}{\pgfqpoint{2.589738in}{3.079750in}}%
\pgfusepath{clip}%
\pgfsetrectcap%
\pgfsetroundjoin%
\pgfsetlinewidth{0.803000pt}%
\definecolor{currentstroke}{rgb}{1.000000,1.000000,1.000000}%
\pgfsetstrokecolor{currentstroke}%
\pgfsetdash{}{0pt}%
\pgfpathmoveto{\pgfqpoint{9.188926in}{3.223771in}}%
\pgfpathlineto{\pgfqpoint{11.778664in}{3.223771in}}%
\pgfusepath{stroke}%
\end{pgfscope}%
\begin{pgfscope}%
\pgfsetbuttcap%
\pgfsetroundjoin%
\definecolor{currentfill}{rgb}{0.333333,0.333333,0.333333}%
\pgfsetfillcolor{currentfill}%
\pgfsetlinewidth{0.803000pt}%
\definecolor{currentstroke}{rgb}{0.333333,0.333333,0.333333}%
\pgfsetstrokecolor{currentstroke}%
\pgfsetdash{}{0pt}%
\pgfsys@defobject{currentmarker}{\pgfqpoint{-0.048611in}{0.000000in}}{\pgfqpoint{-0.000000in}{0.000000in}}{%
\pgfpathmoveto{\pgfqpoint{-0.000000in}{0.000000in}}%
\pgfpathlineto{\pgfqpoint{-0.048611in}{0.000000in}}%
\pgfusepath{stroke,fill}%
}%
\begin{pgfscope}%
\pgfsys@transformshift{9.188926in}{3.223771in}%
\pgfsys@useobject{currentmarker}{}%
\end{pgfscope}%
\end{pgfscope}%
\begin{pgfscope}%
\pgfpathrectangle{\pgfqpoint{9.188926in}{1.263068in}}{\pgfqpoint{2.589738in}{3.079750in}}%
\pgfusepath{clip}%
\pgfsetrectcap%
\pgfsetroundjoin%
\pgfsetlinewidth{0.803000pt}%
\definecolor{currentstroke}{rgb}{1.000000,1.000000,1.000000}%
\pgfsetstrokecolor{currentstroke}%
\pgfsetdash{}{0pt}%
\pgfpathmoveto{\pgfqpoint{9.188926in}{3.678950in}}%
\pgfpathlineto{\pgfqpoint{11.778664in}{3.678950in}}%
\pgfusepath{stroke}%
\end{pgfscope}%
\begin{pgfscope}%
\pgfsetbuttcap%
\pgfsetroundjoin%
\definecolor{currentfill}{rgb}{0.333333,0.333333,0.333333}%
\pgfsetfillcolor{currentfill}%
\pgfsetlinewidth{0.803000pt}%
\definecolor{currentstroke}{rgb}{0.333333,0.333333,0.333333}%
\pgfsetstrokecolor{currentstroke}%
\pgfsetdash{}{0pt}%
\pgfsys@defobject{currentmarker}{\pgfqpoint{-0.048611in}{0.000000in}}{\pgfqpoint{-0.000000in}{0.000000in}}{%
\pgfpathmoveto{\pgfqpoint{-0.000000in}{0.000000in}}%
\pgfpathlineto{\pgfqpoint{-0.048611in}{0.000000in}}%
\pgfusepath{stroke,fill}%
}%
\begin{pgfscope}%
\pgfsys@transformshift{9.188926in}{3.678950in}%
\pgfsys@useobject{currentmarker}{}%
\end{pgfscope}%
\end{pgfscope}%
\begin{pgfscope}%
\pgfpathrectangle{\pgfqpoint{9.188926in}{1.263068in}}{\pgfqpoint{2.589738in}{3.079750in}}%
\pgfusepath{clip}%
\pgfsetrectcap%
\pgfsetroundjoin%
\pgfsetlinewidth{0.803000pt}%
\definecolor{currentstroke}{rgb}{1.000000,1.000000,1.000000}%
\pgfsetstrokecolor{currentstroke}%
\pgfsetdash{}{0pt}%
\pgfpathmoveto{\pgfqpoint{9.188926in}{4.134128in}}%
\pgfpathlineto{\pgfqpoint{11.778664in}{4.134128in}}%
\pgfusepath{stroke}%
\end{pgfscope}%
\begin{pgfscope}%
\pgfsetbuttcap%
\pgfsetroundjoin%
\definecolor{currentfill}{rgb}{0.333333,0.333333,0.333333}%
\pgfsetfillcolor{currentfill}%
\pgfsetlinewidth{0.803000pt}%
\definecolor{currentstroke}{rgb}{0.333333,0.333333,0.333333}%
\pgfsetstrokecolor{currentstroke}%
\pgfsetdash{}{0pt}%
\pgfsys@defobject{currentmarker}{\pgfqpoint{-0.048611in}{0.000000in}}{\pgfqpoint{-0.000000in}{0.000000in}}{%
\pgfpathmoveto{\pgfqpoint{-0.000000in}{0.000000in}}%
\pgfpathlineto{\pgfqpoint{-0.048611in}{0.000000in}}%
\pgfusepath{stroke,fill}%
}%
\begin{pgfscope}%
\pgfsys@transformshift{9.188926in}{4.134128in}%
\pgfsys@useobject{currentmarker}{}%
\end{pgfscope}%
\end{pgfscope}%
\begin{pgfscope}%
\pgfpathrectangle{\pgfqpoint{9.188926in}{1.263068in}}{\pgfqpoint{2.589738in}{3.079750in}}%
\pgfusepath{clip}%
\pgfsetbuttcap%
\pgfsetroundjoin%
\pgfsetlinewidth{1.505625pt}%
\definecolor{currentstroke}{rgb}{0.839216,0.152941,0.156863}%
\pgfsetstrokecolor{currentstroke}%
\pgfsetdash{{5.550000pt}{2.400000pt}}{0.000000pt}%
\pgfpathmoveto{\pgfqpoint{9.306642in}{4.005751in}}%
\pgfpathlineto{\pgfqpoint{9.895219in}{1.403056in}}%
\pgfpathlineto{\pgfqpoint{10.483795in}{1.403056in}}%
\pgfpathlineto{\pgfqpoint{11.072372in}{1.403056in}}%
\pgfpathlineto{\pgfqpoint{11.660949in}{1.403056in}}%
\pgfusepath{stroke}%
\end{pgfscope}%
\begin{pgfscope}%
\pgfpathrectangle{\pgfqpoint{9.188926in}{1.263068in}}{\pgfqpoint{2.589738in}{3.079750in}}%
\pgfusepath{clip}%
\pgfsetbuttcap%
\pgfsetroundjoin%
\definecolor{currentfill}{rgb}{0.839216,0.152941,0.156863}%
\pgfsetfillcolor{currentfill}%
\pgfsetlinewidth{1.003750pt}%
\definecolor{currentstroke}{rgb}{0.839216,0.152941,0.156863}%
\pgfsetstrokecolor{currentstroke}%
\pgfsetdash{}{0pt}%
\pgfsys@defobject{currentmarker}{\pgfqpoint{-0.041667in}{-0.041667in}}{\pgfqpoint{0.041667in}{0.041667in}}{%
\pgfpathmoveto{\pgfqpoint{0.000000in}{-0.041667in}}%
\pgfpathcurveto{\pgfqpoint{0.011050in}{-0.041667in}}{\pgfqpoint{0.021649in}{-0.037276in}}{\pgfqpoint{0.029463in}{-0.029463in}}%
\pgfpathcurveto{\pgfqpoint{0.037276in}{-0.021649in}}{\pgfqpoint{0.041667in}{-0.011050in}}{\pgfqpoint{0.041667in}{0.000000in}}%
\pgfpathcurveto{\pgfqpoint{0.041667in}{0.011050in}}{\pgfqpoint{0.037276in}{0.021649in}}{\pgfqpoint{0.029463in}{0.029463in}}%
\pgfpathcurveto{\pgfqpoint{0.021649in}{0.037276in}}{\pgfqpoint{0.011050in}{0.041667in}}{\pgfqpoint{0.000000in}{0.041667in}}%
\pgfpathcurveto{\pgfqpoint{-0.011050in}{0.041667in}}{\pgfqpoint{-0.021649in}{0.037276in}}{\pgfqpoint{-0.029463in}{0.029463in}}%
\pgfpathcurveto{\pgfqpoint{-0.037276in}{0.021649in}}{\pgfqpoint{-0.041667in}{0.011050in}}{\pgfqpoint{-0.041667in}{0.000000in}}%
\pgfpathcurveto{\pgfqpoint{-0.041667in}{-0.011050in}}{\pgfqpoint{-0.037276in}{-0.021649in}}{\pgfqpoint{-0.029463in}{-0.029463in}}%
\pgfpathcurveto{\pgfqpoint{-0.021649in}{-0.037276in}}{\pgfqpoint{-0.011050in}{-0.041667in}}{\pgfqpoint{0.000000in}{-0.041667in}}%
\pgfpathclose%
\pgfusepath{stroke,fill}%
}%
\begin{pgfscope}%
\pgfsys@transformshift{9.306642in}{4.005751in}%
\pgfsys@useobject{currentmarker}{}%
\end{pgfscope}%
\begin{pgfscope}%
\pgfsys@transformshift{9.895219in}{1.403056in}%
\pgfsys@useobject{currentmarker}{}%
\end{pgfscope}%
\begin{pgfscope}%
\pgfsys@transformshift{10.483795in}{1.403056in}%
\pgfsys@useobject{currentmarker}{}%
\end{pgfscope}%
\begin{pgfscope}%
\pgfsys@transformshift{11.072372in}{1.403056in}%
\pgfsys@useobject{currentmarker}{}%
\end{pgfscope}%
\begin{pgfscope}%
\pgfsys@transformshift{11.660949in}{1.403056in}%
\pgfsys@useobject{currentmarker}{}%
\end{pgfscope}%
\end{pgfscope}%
\begin{pgfscope}%
\pgfpathrectangle{\pgfqpoint{9.188926in}{1.263068in}}{\pgfqpoint{2.589738in}{3.079750in}}%
\pgfusepath{clip}%
\pgfsetbuttcap%
\pgfsetroundjoin%
\pgfsetlinewidth{1.505625pt}%
\definecolor{currentstroke}{rgb}{0.549020,0.337255,0.294118}%
\pgfsetstrokecolor{currentstroke}%
\pgfsetdash{{5.550000pt}{2.400000pt}}{0.000000pt}%
\pgfpathmoveto{\pgfqpoint{9.306642in}{4.031153in}}%
\pgfpathlineto{\pgfqpoint{9.895219in}{4.202829in}}%
\pgfpathlineto{\pgfqpoint{10.483795in}{4.202819in}}%
\pgfpathlineto{\pgfqpoint{11.072372in}{4.202823in}}%
\pgfpathlineto{\pgfqpoint{11.660949in}{4.202823in}}%
\pgfusepath{stroke}%
\end{pgfscope}%
\begin{pgfscope}%
\pgfpathrectangle{\pgfqpoint{9.188926in}{1.263068in}}{\pgfqpoint{2.589738in}{3.079750in}}%
\pgfusepath{clip}%
\pgfsetbuttcap%
\pgfsetroundjoin%
\definecolor{currentfill}{rgb}{0.549020,0.337255,0.294118}%
\pgfsetfillcolor{currentfill}%
\pgfsetlinewidth{1.003750pt}%
\definecolor{currentstroke}{rgb}{0.549020,0.337255,0.294118}%
\pgfsetstrokecolor{currentstroke}%
\pgfsetdash{}{0pt}%
\pgfsys@defobject{currentmarker}{\pgfqpoint{-0.041667in}{-0.041667in}}{\pgfqpoint{0.041667in}{0.041667in}}{%
\pgfpathmoveto{\pgfqpoint{0.000000in}{-0.041667in}}%
\pgfpathcurveto{\pgfqpoint{0.011050in}{-0.041667in}}{\pgfqpoint{0.021649in}{-0.037276in}}{\pgfqpoint{0.029463in}{-0.029463in}}%
\pgfpathcurveto{\pgfqpoint{0.037276in}{-0.021649in}}{\pgfqpoint{0.041667in}{-0.011050in}}{\pgfqpoint{0.041667in}{0.000000in}}%
\pgfpathcurveto{\pgfqpoint{0.041667in}{0.011050in}}{\pgfqpoint{0.037276in}{0.021649in}}{\pgfqpoint{0.029463in}{0.029463in}}%
\pgfpathcurveto{\pgfqpoint{0.021649in}{0.037276in}}{\pgfqpoint{0.011050in}{0.041667in}}{\pgfqpoint{0.000000in}{0.041667in}}%
\pgfpathcurveto{\pgfqpoint{-0.011050in}{0.041667in}}{\pgfqpoint{-0.021649in}{0.037276in}}{\pgfqpoint{-0.029463in}{0.029463in}}%
\pgfpathcurveto{\pgfqpoint{-0.037276in}{0.021649in}}{\pgfqpoint{-0.041667in}{0.011050in}}{\pgfqpoint{-0.041667in}{0.000000in}}%
\pgfpathcurveto{\pgfqpoint{-0.041667in}{-0.011050in}}{\pgfqpoint{-0.037276in}{-0.021649in}}{\pgfqpoint{-0.029463in}{-0.029463in}}%
\pgfpathcurveto{\pgfqpoint{-0.021649in}{-0.037276in}}{\pgfqpoint{-0.011050in}{-0.041667in}}{\pgfqpoint{0.000000in}{-0.041667in}}%
\pgfpathclose%
\pgfusepath{stroke,fill}%
}%
\begin{pgfscope}%
\pgfsys@transformshift{9.306642in}{4.031153in}%
\pgfsys@useobject{currentmarker}{}%
\end{pgfscope}%
\begin{pgfscope}%
\pgfsys@transformshift{9.895219in}{4.202829in}%
\pgfsys@useobject{currentmarker}{}%
\end{pgfscope}%
\begin{pgfscope}%
\pgfsys@transformshift{10.483795in}{4.202819in}%
\pgfsys@useobject{currentmarker}{}%
\end{pgfscope}%
\begin{pgfscope}%
\pgfsys@transformshift{11.072372in}{4.202823in}%
\pgfsys@useobject{currentmarker}{}%
\end{pgfscope}%
\begin{pgfscope}%
\pgfsys@transformshift{11.660949in}{4.202823in}%
\pgfsys@useobject{currentmarker}{}%
\end{pgfscope}%
\end{pgfscope}%
\begin{pgfscope}%
\pgfpathrectangle{\pgfqpoint{9.188926in}{1.263068in}}{\pgfqpoint{2.589738in}{3.079750in}}%
\pgfusepath{clip}%
\pgfsetbuttcap%
\pgfsetroundjoin%
\pgfsetlinewidth{1.505625pt}%
\definecolor{currentstroke}{rgb}{0.411765,0.411765,0.411765}%
\pgfsetstrokecolor{currentstroke}%
\pgfsetdash{{5.550000pt}{2.400000pt}}{0.000000pt}%
\pgfpathmoveto{\pgfqpoint{9.306642in}{1.403056in}}%
\pgfpathlineto{\pgfqpoint{9.895219in}{1.403056in}}%
\pgfpathlineto{\pgfqpoint{10.483795in}{1.403056in}}%
\pgfpathlineto{\pgfqpoint{11.072372in}{1.403056in}}%
\pgfpathlineto{\pgfqpoint{11.660949in}{1.403056in}}%
\pgfusepath{stroke}%
\end{pgfscope}%
\begin{pgfscope}%
\pgfpathrectangle{\pgfqpoint{9.188926in}{1.263068in}}{\pgfqpoint{2.589738in}{3.079750in}}%
\pgfusepath{clip}%
\pgfsetbuttcap%
\pgfsetroundjoin%
\definecolor{currentfill}{rgb}{0.411765,0.411765,0.411765}%
\pgfsetfillcolor{currentfill}%
\pgfsetlinewidth{1.003750pt}%
\definecolor{currentstroke}{rgb}{0.411765,0.411765,0.411765}%
\pgfsetstrokecolor{currentstroke}%
\pgfsetdash{}{0pt}%
\pgfsys@defobject{currentmarker}{\pgfqpoint{-0.041667in}{-0.041667in}}{\pgfqpoint{0.041667in}{0.041667in}}{%
\pgfpathmoveto{\pgfqpoint{0.000000in}{-0.041667in}}%
\pgfpathcurveto{\pgfqpoint{0.011050in}{-0.041667in}}{\pgfqpoint{0.021649in}{-0.037276in}}{\pgfqpoint{0.029463in}{-0.029463in}}%
\pgfpathcurveto{\pgfqpoint{0.037276in}{-0.021649in}}{\pgfqpoint{0.041667in}{-0.011050in}}{\pgfqpoint{0.041667in}{0.000000in}}%
\pgfpathcurveto{\pgfqpoint{0.041667in}{0.011050in}}{\pgfqpoint{0.037276in}{0.021649in}}{\pgfqpoint{0.029463in}{0.029463in}}%
\pgfpathcurveto{\pgfqpoint{0.021649in}{0.037276in}}{\pgfqpoint{0.011050in}{0.041667in}}{\pgfqpoint{0.000000in}{0.041667in}}%
\pgfpathcurveto{\pgfqpoint{-0.011050in}{0.041667in}}{\pgfqpoint{-0.021649in}{0.037276in}}{\pgfqpoint{-0.029463in}{0.029463in}}%
\pgfpathcurveto{\pgfqpoint{-0.037276in}{0.021649in}}{\pgfqpoint{-0.041667in}{0.011050in}}{\pgfqpoint{-0.041667in}{0.000000in}}%
\pgfpathcurveto{\pgfqpoint{-0.041667in}{-0.011050in}}{\pgfqpoint{-0.037276in}{-0.021649in}}{\pgfqpoint{-0.029463in}{-0.029463in}}%
\pgfpathcurveto{\pgfqpoint{-0.021649in}{-0.037276in}}{\pgfqpoint{-0.011050in}{-0.041667in}}{\pgfqpoint{0.000000in}{-0.041667in}}%
\pgfpathclose%
\pgfusepath{stroke,fill}%
}%
\begin{pgfscope}%
\pgfsys@transformshift{9.306642in}{1.403056in}%
\pgfsys@useobject{currentmarker}{}%
\end{pgfscope}%
\begin{pgfscope}%
\pgfsys@transformshift{9.895219in}{1.403056in}%
\pgfsys@useobject{currentmarker}{}%
\end{pgfscope}%
\begin{pgfscope}%
\pgfsys@transformshift{10.483795in}{1.403056in}%
\pgfsys@useobject{currentmarker}{}%
\end{pgfscope}%
\begin{pgfscope}%
\pgfsys@transformshift{11.072372in}{1.403056in}%
\pgfsys@useobject{currentmarker}{}%
\end{pgfscope}%
\begin{pgfscope}%
\pgfsys@transformshift{11.660949in}{1.403056in}%
\pgfsys@useobject{currentmarker}{}%
\end{pgfscope}%
\end{pgfscope}%
\begin{pgfscope}%
\pgfpathrectangle{\pgfqpoint{9.188926in}{1.263068in}}{\pgfqpoint{2.589738in}{3.079750in}}%
\pgfusepath{clip}%
\pgfsetbuttcap%
\pgfsetroundjoin%
\pgfsetlinewidth{1.505625pt}%
\definecolor{currentstroke}{rgb}{0.172549,0.627451,0.172549}%
\pgfsetstrokecolor{currentstroke}%
\pgfsetdash{{5.550000pt}{2.400000pt}}{0.000000pt}%
\pgfpathmoveto{\pgfqpoint{9.306642in}{2.075567in}}%
\pgfpathlineto{\pgfqpoint{9.895219in}{1.403056in}}%
\pgfpathlineto{\pgfqpoint{10.483795in}{1.403056in}}%
\pgfpathlineto{\pgfqpoint{11.072372in}{1.403056in}}%
\pgfpathlineto{\pgfqpoint{11.660949in}{1.403056in}}%
\pgfusepath{stroke}%
\end{pgfscope}%
\begin{pgfscope}%
\pgfpathrectangle{\pgfqpoint{9.188926in}{1.263068in}}{\pgfqpoint{2.589738in}{3.079750in}}%
\pgfusepath{clip}%
\pgfsetbuttcap%
\pgfsetroundjoin%
\definecolor{currentfill}{rgb}{0.172549,0.627451,0.172549}%
\pgfsetfillcolor{currentfill}%
\pgfsetlinewidth{1.003750pt}%
\definecolor{currentstroke}{rgb}{0.172549,0.627451,0.172549}%
\pgfsetstrokecolor{currentstroke}%
\pgfsetdash{}{0pt}%
\pgfsys@defobject{currentmarker}{\pgfqpoint{-0.041667in}{-0.041667in}}{\pgfqpoint{0.041667in}{0.041667in}}{%
\pgfpathmoveto{\pgfqpoint{0.000000in}{-0.041667in}}%
\pgfpathcurveto{\pgfqpoint{0.011050in}{-0.041667in}}{\pgfqpoint{0.021649in}{-0.037276in}}{\pgfqpoint{0.029463in}{-0.029463in}}%
\pgfpathcurveto{\pgfqpoint{0.037276in}{-0.021649in}}{\pgfqpoint{0.041667in}{-0.011050in}}{\pgfqpoint{0.041667in}{0.000000in}}%
\pgfpathcurveto{\pgfqpoint{0.041667in}{0.011050in}}{\pgfqpoint{0.037276in}{0.021649in}}{\pgfqpoint{0.029463in}{0.029463in}}%
\pgfpathcurveto{\pgfqpoint{0.021649in}{0.037276in}}{\pgfqpoint{0.011050in}{0.041667in}}{\pgfqpoint{0.000000in}{0.041667in}}%
\pgfpathcurveto{\pgfqpoint{-0.011050in}{0.041667in}}{\pgfqpoint{-0.021649in}{0.037276in}}{\pgfqpoint{-0.029463in}{0.029463in}}%
\pgfpathcurveto{\pgfqpoint{-0.037276in}{0.021649in}}{\pgfqpoint{-0.041667in}{0.011050in}}{\pgfqpoint{-0.041667in}{0.000000in}}%
\pgfpathcurveto{\pgfqpoint{-0.041667in}{-0.011050in}}{\pgfqpoint{-0.037276in}{-0.021649in}}{\pgfqpoint{-0.029463in}{-0.029463in}}%
\pgfpathcurveto{\pgfqpoint{-0.021649in}{-0.037276in}}{\pgfqpoint{-0.011050in}{-0.041667in}}{\pgfqpoint{0.000000in}{-0.041667in}}%
\pgfpathclose%
\pgfusepath{stroke,fill}%
}%
\begin{pgfscope}%
\pgfsys@transformshift{9.306642in}{2.075567in}%
\pgfsys@useobject{currentmarker}{}%
\end{pgfscope}%
\begin{pgfscope}%
\pgfsys@transformshift{9.895219in}{1.403056in}%
\pgfsys@useobject{currentmarker}{}%
\end{pgfscope}%
\begin{pgfscope}%
\pgfsys@transformshift{10.483795in}{1.403056in}%
\pgfsys@useobject{currentmarker}{}%
\end{pgfscope}%
\begin{pgfscope}%
\pgfsys@transformshift{11.072372in}{1.403056in}%
\pgfsys@useobject{currentmarker}{}%
\end{pgfscope}%
\begin{pgfscope}%
\pgfsys@transformshift{11.660949in}{1.403056in}%
\pgfsys@useobject{currentmarker}{}%
\end{pgfscope}%
\end{pgfscope}%
\begin{pgfscope}%
\pgfpathrectangle{\pgfqpoint{9.188926in}{1.263068in}}{\pgfqpoint{2.589738in}{3.079750in}}%
\pgfusepath{clip}%
\pgfsetbuttcap%
\pgfsetroundjoin%
\pgfsetlinewidth{1.505625pt}%
\definecolor{currentstroke}{rgb}{1.000000,1.000000,0.000000}%
\pgfsetstrokecolor{currentstroke}%
\pgfsetdash{{5.550000pt}{2.400000pt}}{0.000000pt}%
\pgfpathmoveto{\pgfqpoint{9.306642in}{2.704346in}}%
\pgfpathlineto{\pgfqpoint{9.895219in}{2.837169in}}%
\pgfpathlineto{\pgfqpoint{10.483795in}{2.852132in}}%
\pgfpathlineto{\pgfqpoint{11.072372in}{2.852132in}}%
\pgfpathlineto{\pgfqpoint{11.660949in}{2.852133in}}%
\pgfusepath{stroke}%
\end{pgfscope}%
\begin{pgfscope}%
\pgfpathrectangle{\pgfqpoint{9.188926in}{1.263068in}}{\pgfqpoint{2.589738in}{3.079750in}}%
\pgfusepath{clip}%
\pgfsetbuttcap%
\pgfsetroundjoin%
\definecolor{currentfill}{rgb}{1.000000,1.000000,0.000000}%
\pgfsetfillcolor{currentfill}%
\pgfsetlinewidth{1.003750pt}%
\definecolor{currentstroke}{rgb}{1.000000,1.000000,0.000000}%
\pgfsetstrokecolor{currentstroke}%
\pgfsetdash{}{0pt}%
\pgfsys@defobject{currentmarker}{\pgfqpoint{-0.041667in}{-0.041667in}}{\pgfqpoint{0.041667in}{0.041667in}}{%
\pgfpathmoveto{\pgfqpoint{0.000000in}{-0.041667in}}%
\pgfpathcurveto{\pgfqpoint{0.011050in}{-0.041667in}}{\pgfqpoint{0.021649in}{-0.037276in}}{\pgfqpoint{0.029463in}{-0.029463in}}%
\pgfpathcurveto{\pgfqpoint{0.037276in}{-0.021649in}}{\pgfqpoint{0.041667in}{-0.011050in}}{\pgfqpoint{0.041667in}{0.000000in}}%
\pgfpathcurveto{\pgfqpoint{0.041667in}{0.011050in}}{\pgfqpoint{0.037276in}{0.021649in}}{\pgfqpoint{0.029463in}{0.029463in}}%
\pgfpathcurveto{\pgfqpoint{0.021649in}{0.037276in}}{\pgfqpoint{0.011050in}{0.041667in}}{\pgfqpoint{0.000000in}{0.041667in}}%
\pgfpathcurveto{\pgfqpoint{-0.011050in}{0.041667in}}{\pgfqpoint{-0.021649in}{0.037276in}}{\pgfqpoint{-0.029463in}{0.029463in}}%
\pgfpathcurveto{\pgfqpoint{-0.037276in}{0.021649in}}{\pgfqpoint{-0.041667in}{0.011050in}}{\pgfqpoint{-0.041667in}{0.000000in}}%
\pgfpathcurveto{\pgfqpoint{-0.041667in}{-0.011050in}}{\pgfqpoint{-0.037276in}{-0.021649in}}{\pgfqpoint{-0.029463in}{-0.029463in}}%
\pgfpathcurveto{\pgfqpoint{-0.021649in}{-0.037276in}}{\pgfqpoint{-0.011050in}{-0.041667in}}{\pgfqpoint{0.000000in}{-0.041667in}}%
\pgfpathclose%
\pgfusepath{stroke,fill}%
}%
\begin{pgfscope}%
\pgfsys@transformshift{9.306642in}{2.704346in}%
\pgfsys@useobject{currentmarker}{}%
\end{pgfscope}%
\begin{pgfscope}%
\pgfsys@transformshift{9.895219in}{2.837169in}%
\pgfsys@useobject{currentmarker}{}%
\end{pgfscope}%
\begin{pgfscope}%
\pgfsys@transformshift{10.483795in}{2.852132in}%
\pgfsys@useobject{currentmarker}{}%
\end{pgfscope}%
\begin{pgfscope}%
\pgfsys@transformshift{11.072372in}{2.852132in}%
\pgfsys@useobject{currentmarker}{}%
\end{pgfscope}%
\begin{pgfscope}%
\pgfsys@transformshift{11.660949in}{2.852133in}%
\pgfsys@useobject{currentmarker}{}%
\end{pgfscope}%
\end{pgfscope}%
\begin{pgfscope}%
\pgfpathrectangle{\pgfqpoint{9.188926in}{1.263068in}}{\pgfqpoint{2.589738in}{3.079750in}}%
\pgfusepath{clip}%
\pgfsetbuttcap%
\pgfsetroundjoin%
\pgfsetlinewidth{1.505625pt}%
\definecolor{currentstroke}{rgb}{0.121569,0.466667,0.705882}%
\pgfsetstrokecolor{currentstroke}%
\pgfsetdash{{5.550000pt}{2.400000pt}}{0.000000pt}%
\pgfpathmoveto{\pgfqpoint{9.306642in}{3.690329in}}%
\pgfpathlineto{\pgfqpoint{9.895219in}{3.690329in}}%
\pgfpathlineto{\pgfqpoint{10.483795in}{3.690329in}}%
\pgfpathlineto{\pgfqpoint{11.072372in}{3.690329in}}%
\pgfpathlineto{\pgfqpoint{11.660949in}{3.690329in}}%
\pgfusepath{stroke}%
\end{pgfscope}%
\begin{pgfscope}%
\pgfpathrectangle{\pgfqpoint{9.188926in}{1.263068in}}{\pgfqpoint{2.589738in}{3.079750in}}%
\pgfusepath{clip}%
\pgfsetbuttcap%
\pgfsetroundjoin%
\definecolor{currentfill}{rgb}{0.121569,0.466667,0.705882}%
\pgfsetfillcolor{currentfill}%
\pgfsetlinewidth{1.003750pt}%
\definecolor{currentstroke}{rgb}{0.121569,0.466667,0.705882}%
\pgfsetstrokecolor{currentstroke}%
\pgfsetdash{}{0pt}%
\pgfsys@defobject{currentmarker}{\pgfqpoint{-0.041667in}{-0.041667in}}{\pgfqpoint{0.041667in}{0.041667in}}{%
\pgfpathmoveto{\pgfqpoint{0.000000in}{-0.041667in}}%
\pgfpathcurveto{\pgfqpoint{0.011050in}{-0.041667in}}{\pgfqpoint{0.021649in}{-0.037276in}}{\pgfqpoint{0.029463in}{-0.029463in}}%
\pgfpathcurveto{\pgfqpoint{0.037276in}{-0.021649in}}{\pgfqpoint{0.041667in}{-0.011050in}}{\pgfqpoint{0.041667in}{0.000000in}}%
\pgfpathcurveto{\pgfqpoint{0.041667in}{0.011050in}}{\pgfqpoint{0.037276in}{0.021649in}}{\pgfqpoint{0.029463in}{0.029463in}}%
\pgfpathcurveto{\pgfqpoint{0.021649in}{0.037276in}}{\pgfqpoint{0.011050in}{0.041667in}}{\pgfqpoint{0.000000in}{0.041667in}}%
\pgfpathcurveto{\pgfqpoint{-0.011050in}{0.041667in}}{\pgfqpoint{-0.021649in}{0.037276in}}{\pgfqpoint{-0.029463in}{0.029463in}}%
\pgfpathcurveto{\pgfqpoint{-0.037276in}{0.021649in}}{\pgfqpoint{-0.041667in}{0.011050in}}{\pgfqpoint{-0.041667in}{0.000000in}}%
\pgfpathcurveto{\pgfqpoint{-0.041667in}{-0.011050in}}{\pgfqpoint{-0.037276in}{-0.021649in}}{\pgfqpoint{-0.029463in}{-0.029463in}}%
\pgfpathcurveto{\pgfqpoint{-0.021649in}{-0.037276in}}{\pgfqpoint{-0.011050in}{-0.041667in}}{\pgfqpoint{0.000000in}{-0.041667in}}%
\pgfpathclose%
\pgfusepath{stroke,fill}%
}%
\begin{pgfscope}%
\pgfsys@transformshift{9.306642in}{3.690329in}%
\pgfsys@useobject{currentmarker}{}%
\end{pgfscope}%
\begin{pgfscope}%
\pgfsys@transformshift{9.895219in}{3.690329in}%
\pgfsys@useobject{currentmarker}{}%
\end{pgfscope}%
\begin{pgfscope}%
\pgfsys@transformshift{10.483795in}{3.690329in}%
\pgfsys@useobject{currentmarker}{}%
\end{pgfscope}%
\begin{pgfscope}%
\pgfsys@transformshift{11.072372in}{3.690329in}%
\pgfsys@useobject{currentmarker}{}%
\end{pgfscope}%
\begin{pgfscope}%
\pgfsys@transformshift{11.660949in}{3.690329in}%
\pgfsys@useobject{currentmarker}{}%
\end{pgfscope}%
\end{pgfscope}%
\begin{pgfscope}%
\pgfsetrectcap%
\pgfsetmiterjoin%
\pgfsetlinewidth{1.003750pt}%
\definecolor{currentstroke}{rgb}{1.000000,1.000000,1.000000}%
\pgfsetstrokecolor{currentstroke}%
\pgfsetdash{}{0pt}%
\pgfpathmoveto{\pgfqpoint{9.188926in}{1.263067in}}%
\pgfpathlineto{\pgfqpoint{9.188926in}{4.342817in}}%
\pgfusepath{stroke}%
\end{pgfscope}%
\begin{pgfscope}%
\pgfsetrectcap%
\pgfsetmiterjoin%
\pgfsetlinewidth{1.003750pt}%
\definecolor{currentstroke}{rgb}{1.000000,1.000000,1.000000}%
\pgfsetstrokecolor{currentstroke}%
\pgfsetdash{}{0pt}%
\pgfpathmoveto{\pgfqpoint{11.778664in}{1.263067in}}%
\pgfpathlineto{\pgfqpoint{11.778664in}{4.342817in}}%
\pgfusepath{stroke}%
\end{pgfscope}%
\begin{pgfscope}%
\pgfsetrectcap%
\pgfsetmiterjoin%
\pgfsetlinewidth{1.003750pt}%
\definecolor{currentstroke}{rgb}{1.000000,1.000000,1.000000}%
\pgfsetstrokecolor{currentstroke}%
\pgfsetdash{}{0pt}%
\pgfpathmoveto{\pgfqpoint{9.188926in}{1.263068in}}%
\pgfpathlineto{\pgfqpoint{11.778664in}{1.263068in}}%
\pgfusepath{stroke}%
\end{pgfscope}%
\begin{pgfscope}%
\pgfsetrectcap%
\pgfsetmiterjoin%
\pgfsetlinewidth{1.003750pt}%
\definecolor{currentstroke}{rgb}{1.000000,1.000000,1.000000}%
\pgfsetstrokecolor{currentstroke}%
\pgfsetdash{}{0pt}%
\pgfpathmoveto{\pgfqpoint{9.188926in}{4.342817in}}%
\pgfpathlineto{\pgfqpoint{11.778664in}{4.342817in}}%
\pgfusepath{stroke}%
\end{pgfscope}%
\begin{pgfscope}%
\pgfsetbuttcap%
\pgfsetmiterjoin%
\definecolor{currentfill}{rgb}{0.269412,0.269412,0.269412}%
\pgfsetfillcolor{currentfill}%
\pgfsetfillopacity{0.500000}%
\pgfsetlinewidth{0.501875pt}%
\definecolor{currentstroke}{rgb}{0.269412,0.269412,0.269412}%
\pgfsetstrokecolor{currentstroke}%
\pgfsetstrokeopacity{0.500000}%
\pgfsetdash{}{0pt}%
\pgfpathmoveto{\pgfqpoint{11.974818in}{1.851240in}}%
\pgfpathlineto{\pgfqpoint{13.929216in}{1.851240in}}%
\pgfpathquadraticcurveto{\pgfqpoint{13.968105in}{1.851240in}}{\pgfqpoint{13.968105in}{1.890129in}}%
\pgfpathlineto{\pgfqpoint{13.968105in}{3.520682in}}%
\pgfpathquadraticcurveto{\pgfqpoint{13.968105in}{3.559570in}}{\pgfqpoint{13.929216in}{3.559570in}}%
\pgfpathlineto{\pgfqpoint{11.974818in}{3.559570in}}%
\pgfpathquadraticcurveto{\pgfqpoint{11.935929in}{3.559570in}}{\pgfqpoint{11.935929in}{3.520682in}}%
\pgfpathlineto{\pgfqpoint{11.935929in}{1.890129in}}%
\pgfpathquadraticcurveto{\pgfqpoint{11.935929in}{1.851240in}}{\pgfqpoint{11.974818in}{1.851240in}}%
\pgfpathclose%
\pgfusepath{stroke,fill}%
\end{pgfscope}%
\begin{pgfscope}%
\pgfsetbuttcap%
\pgfsetmiterjoin%
\definecolor{currentfill}{rgb}{0.898039,0.898039,0.898039}%
\pgfsetfillcolor{currentfill}%
\pgfsetlinewidth{0.501875pt}%
\definecolor{currentstroke}{rgb}{0.800000,0.800000,0.800000}%
\pgfsetstrokecolor{currentstroke}%
\pgfsetdash{}{0pt}%
\pgfpathmoveto{\pgfqpoint{11.947040in}{1.879017in}}%
\pgfpathlineto{\pgfqpoint{13.901438in}{1.879017in}}%
\pgfpathquadraticcurveto{\pgfqpoint{13.940327in}{1.879017in}}{\pgfqpoint{13.940327in}{1.917906in}}%
\pgfpathlineto{\pgfqpoint{13.940327in}{3.548459in}}%
\pgfpathquadraticcurveto{\pgfqpoint{13.940327in}{3.587348in}}{\pgfqpoint{13.901438in}{3.587348in}}%
\pgfpathlineto{\pgfqpoint{11.947040in}{3.587348in}}%
\pgfpathquadraticcurveto{\pgfqpoint{11.908151in}{3.587348in}}{\pgfqpoint{11.908151in}{3.548459in}}%
\pgfpathlineto{\pgfqpoint{11.908151in}{1.917906in}}%
\pgfpathquadraticcurveto{\pgfqpoint{11.908151in}{1.879017in}}{\pgfqpoint{11.947040in}{1.879017in}}%
\pgfpathclose%
\pgfusepath{stroke,fill}%
\end{pgfscope}%
\begin{pgfscope}%
\pgfsetbuttcap%
\pgfsetroundjoin%
\pgfsetlinewidth{1.505625pt}%
\definecolor{currentstroke}{rgb}{0.839216,0.152941,0.156863}%
\pgfsetstrokecolor{currentstroke}%
\pgfsetdash{{5.550000pt}{2.400000pt}}{0.000000pt}%
\pgfpathmoveto{\pgfqpoint{11.985929in}{3.438737in}}%
\pgfpathlineto{\pgfqpoint{12.374818in}{3.438737in}}%
\pgfusepath{stroke}%
\end{pgfscope}%
\begin{pgfscope}%
\pgfsetbuttcap%
\pgfsetroundjoin%
\definecolor{currentfill}{rgb}{0.839216,0.152941,0.156863}%
\pgfsetfillcolor{currentfill}%
\pgfsetlinewidth{1.003750pt}%
\definecolor{currentstroke}{rgb}{0.839216,0.152941,0.156863}%
\pgfsetstrokecolor{currentstroke}%
\pgfsetdash{}{0pt}%
\pgfsys@defobject{currentmarker}{\pgfqpoint{-0.041667in}{-0.041667in}}{\pgfqpoint{0.041667in}{0.041667in}}{%
\pgfpathmoveto{\pgfqpoint{0.000000in}{-0.041667in}}%
\pgfpathcurveto{\pgfqpoint{0.011050in}{-0.041667in}}{\pgfqpoint{0.021649in}{-0.037276in}}{\pgfqpoint{0.029463in}{-0.029463in}}%
\pgfpathcurveto{\pgfqpoint{0.037276in}{-0.021649in}}{\pgfqpoint{0.041667in}{-0.011050in}}{\pgfqpoint{0.041667in}{0.000000in}}%
\pgfpathcurveto{\pgfqpoint{0.041667in}{0.011050in}}{\pgfqpoint{0.037276in}{0.021649in}}{\pgfqpoint{0.029463in}{0.029463in}}%
\pgfpathcurveto{\pgfqpoint{0.021649in}{0.037276in}}{\pgfqpoint{0.011050in}{0.041667in}}{\pgfqpoint{0.000000in}{0.041667in}}%
\pgfpathcurveto{\pgfqpoint{-0.011050in}{0.041667in}}{\pgfqpoint{-0.021649in}{0.037276in}}{\pgfqpoint{-0.029463in}{0.029463in}}%
\pgfpathcurveto{\pgfqpoint{-0.037276in}{0.021649in}}{\pgfqpoint{-0.041667in}{0.011050in}}{\pgfqpoint{-0.041667in}{0.000000in}}%
\pgfpathcurveto{\pgfqpoint{-0.041667in}{-0.011050in}}{\pgfqpoint{-0.037276in}{-0.021649in}}{\pgfqpoint{-0.029463in}{-0.029463in}}%
\pgfpathcurveto{\pgfqpoint{-0.021649in}{-0.037276in}}{\pgfqpoint{-0.011050in}{-0.041667in}}{\pgfqpoint{0.000000in}{-0.041667in}}%
\pgfpathclose%
\pgfusepath{stroke,fill}%
}%
\begin{pgfscope}%
\pgfsys@transformshift{12.180374in}{3.438737in}%
\pgfsys@useobject{currentmarker}{}%
\end{pgfscope}%
\end{pgfscope}%
\begin{pgfscope}%
\definecolor{textcolor}{rgb}{0.000000,0.000000,0.000000}%
\pgfsetstrokecolor{textcolor}%
\pgfsetfillcolor{textcolor}%
\pgftext[x=12.530374in,y=3.370682in,left,base]{\color{textcolor}\rmfamily\fontsize{14.000000}{16.800000}\selectfont ABBOTT\_TB}%
\end{pgfscope}%
\begin{pgfscope}%
\pgfsetbuttcap%
\pgfsetroundjoin%
\pgfsetlinewidth{1.505625pt}%
\definecolor{currentstroke}{rgb}{0.549020,0.337255,0.294118}%
\pgfsetstrokecolor{currentstroke}%
\pgfsetdash{{5.550000pt}{2.400000pt}}{0.000000pt}%
\pgfpathmoveto{\pgfqpoint{11.985929in}{3.163738in}}%
\pgfpathlineto{\pgfqpoint{12.374818in}{3.163738in}}%
\pgfusepath{stroke}%
\end{pgfscope}%
\begin{pgfscope}%
\pgfsetbuttcap%
\pgfsetroundjoin%
\definecolor{currentfill}{rgb}{0.549020,0.337255,0.294118}%
\pgfsetfillcolor{currentfill}%
\pgfsetlinewidth{1.003750pt}%
\definecolor{currentstroke}{rgb}{0.549020,0.337255,0.294118}%
\pgfsetstrokecolor{currentstroke}%
\pgfsetdash{}{0pt}%
\pgfsys@defobject{currentmarker}{\pgfqpoint{-0.041667in}{-0.041667in}}{\pgfqpoint{0.041667in}{0.041667in}}{%
\pgfpathmoveto{\pgfqpoint{0.000000in}{-0.041667in}}%
\pgfpathcurveto{\pgfqpoint{0.011050in}{-0.041667in}}{\pgfqpoint{0.021649in}{-0.037276in}}{\pgfqpoint{0.029463in}{-0.029463in}}%
\pgfpathcurveto{\pgfqpoint{0.037276in}{-0.021649in}}{\pgfqpoint{0.041667in}{-0.011050in}}{\pgfqpoint{0.041667in}{0.000000in}}%
\pgfpathcurveto{\pgfqpoint{0.041667in}{0.011050in}}{\pgfqpoint{0.037276in}{0.021649in}}{\pgfqpoint{0.029463in}{0.029463in}}%
\pgfpathcurveto{\pgfqpoint{0.021649in}{0.037276in}}{\pgfqpoint{0.011050in}{0.041667in}}{\pgfqpoint{0.000000in}{0.041667in}}%
\pgfpathcurveto{\pgfqpoint{-0.011050in}{0.041667in}}{\pgfqpoint{-0.021649in}{0.037276in}}{\pgfqpoint{-0.029463in}{0.029463in}}%
\pgfpathcurveto{\pgfqpoint{-0.037276in}{0.021649in}}{\pgfqpoint{-0.041667in}{0.011050in}}{\pgfqpoint{-0.041667in}{0.000000in}}%
\pgfpathcurveto{\pgfqpoint{-0.041667in}{-0.011050in}}{\pgfqpoint{-0.037276in}{-0.021649in}}{\pgfqpoint{-0.029463in}{-0.029463in}}%
\pgfpathcurveto{\pgfqpoint{-0.021649in}{-0.037276in}}{\pgfqpoint{-0.011050in}{-0.041667in}}{\pgfqpoint{0.000000in}{-0.041667in}}%
\pgfpathclose%
\pgfusepath{stroke,fill}%
}%
\begin{pgfscope}%
\pgfsys@transformshift{12.180374in}{3.163738in}%
\pgfsys@useobject{currentmarker}{}%
\end{pgfscope}%
\end{pgfscope}%
\begin{pgfscope}%
\definecolor{textcolor}{rgb}{0.000000,0.000000,0.000000}%
\pgfsetstrokecolor{textcolor}%
\pgfsetfillcolor{textcolor}%
\pgftext[x=12.530374in,y=3.095682in,left,base]{\color{textcolor}\rmfamily\fontsize{14.000000}{16.800000}\selectfont IMP\_ELC}%
\end{pgfscope}%
\begin{pgfscope}%
\pgfsetbuttcap%
\pgfsetroundjoin%
\pgfsetlinewidth{1.505625pt}%
\definecolor{currentstroke}{rgb}{0.411765,0.411765,0.411765}%
\pgfsetstrokecolor{currentstroke}%
\pgfsetdash{{5.550000pt}{2.400000pt}}{0.000000pt}%
\pgfpathmoveto{\pgfqpoint{11.985929in}{2.888738in}}%
\pgfpathlineto{\pgfqpoint{12.374818in}{2.888738in}}%
\pgfusepath{stroke}%
\end{pgfscope}%
\begin{pgfscope}%
\pgfsetbuttcap%
\pgfsetroundjoin%
\definecolor{currentfill}{rgb}{0.411765,0.411765,0.411765}%
\pgfsetfillcolor{currentfill}%
\pgfsetlinewidth{1.003750pt}%
\definecolor{currentstroke}{rgb}{0.411765,0.411765,0.411765}%
\pgfsetstrokecolor{currentstroke}%
\pgfsetdash{}{0pt}%
\pgfsys@defobject{currentmarker}{\pgfqpoint{-0.041667in}{-0.041667in}}{\pgfqpoint{0.041667in}{0.041667in}}{%
\pgfpathmoveto{\pgfqpoint{0.000000in}{-0.041667in}}%
\pgfpathcurveto{\pgfqpoint{0.011050in}{-0.041667in}}{\pgfqpoint{0.021649in}{-0.037276in}}{\pgfqpoint{0.029463in}{-0.029463in}}%
\pgfpathcurveto{\pgfqpoint{0.037276in}{-0.021649in}}{\pgfqpoint{0.041667in}{-0.011050in}}{\pgfqpoint{0.041667in}{0.000000in}}%
\pgfpathcurveto{\pgfqpoint{0.041667in}{0.011050in}}{\pgfqpoint{0.037276in}{0.021649in}}{\pgfqpoint{0.029463in}{0.029463in}}%
\pgfpathcurveto{\pgfqpoint{0.021649in}{0.037276in}}{\pgfqpoint{0.011050in}{0.041667in}}{\pgfqpoint{0.000000in}{0.041667in}}%
\pgfpathcurveto{\pgfqpoint{-0.011050in}{0.041667in}}{\pgfqpoint{-0.021649in}{0.037276in}}{\pgfqpoint{-0.029463in}{0.029463in}}%
\pgfpathcurveto{\pgfqpoint{-0.037276in}{0.021649in}}{\pgfqpoint{-0.041667in}{0.011050in}}{\pgfqpoint{-0.041667in}{0.000000in}}%
\pgfpathcurveto{\pgfqpoint{-0.041667in}{-0.011050in}}{\pgfqpoint{-0.037276in}{-0.021649in}}{\pgfqpoint{-0.029463in}{-0.029463in}}%
\pgfpathcurveto{\pgfqpoint{-0.021649in}{-0.037276in}}{\pgfqpoint{-0.011050in}{-0.041667in}}{\pgfqpoint{0.000000in}{-0.041667in}}%
\pgfpathclose%
\pgfusepath{stroke,fill}%
}%
\begin{pgfscope}%
\pgfsys@transformshift{12.180374in}{2.888738in}%
\pgfsys@useobject{currentmarker}{}%
\end{pgfscope}%
\end{pgfscope}%
\begin{pgfscope}%
\definecolor{textcolor}{rgb}{0.000000,0.000000,0.000000}%
\pgfsetstrokecolor{textcolor}%
\pgfsetfillcolor{textcolor}%
\pgftext[x=12.530374in,y=2.820683in,left,base]{\color{textcolor}\rmfamily\fontsize{14.000000}{16.800000}\selectfont LI\_BATTERY}%
\end{pgfscope}%
\begin{pgfscope}%
\pgfsetbuttcap%
\pgfsetroundjoin%
\pgfsetlinewidth{1.505625pt}%
\definecolor{currentstroke}{rgb}{0.172549,0.627451,0.172549}%
\pgfsetstrokecolor{currentstroke}%
\pgfsetdash{{5.550000pt}{2.400000pt}}{0.000000pt}%
\pgfpathmoveto{\pgfqpoint{11.985929in}{2.613739in}}%
\pgfpathlineto{\pgfqpoint{12.374818in}{2.613739in}}%
\pgfusepath{stroke}%
\end{pgfscope}%
\begin{pgfscope}%
\pgfsetbuttcap%
\pgfsetroundjoin%
\definecolor{currentfill}{rgb}{0.172549,0.627451,0.172549}%
\pgfsetfillcolor{currentfill}%
\pgfsetlinewidth{1.003750pt}%
\definecolor{currentstroke}{rgb}{0.172549,0.627451,0.172549}%
\pgfsetstrokecolor{currentstroke}%
\pgfsetdash{}{0pt}%
\pgfsys@defobject{currentmarker}{\pgfqpoint{-0.041667in}{-0.041667in}}{\pgfqpoint{0.041667in}{0.041667in}}{%
\pgfpathmoveto{\pgfqpoint{0.000000in}{-0.041667in}}%
\pgfpathcurveto{\pgfqpoint{0.011050in}{-0.041667in}}{\pgfqpoint{0.021649in}{-0.037276in}}{\pgfqpoint{0.029463in}{-0.029463in}}%
\pgfpathcurveto{\pgfqpoint{0.037276in}{-0.021649in}}{\pgfqpoint{0.041667in}{-0.011050in}}{\pgfqpoint{0.041667in}{0.000000in}}%
\pgfpathcurveto{\pgfqpoint{0.041667in}{0.011050in}}{\pgfqpoint{0.037276in}{0.021649in}}{\pgfqpoint{0.029463in}{0.029463in}}%
\pgfpathcurveto{\pgfqpoint{0.021649in}{0.037276in}}{\pgfqpoint{0.011050in}{0.041667in}}{\pgfqpoint{0.000000in}{0.041667in}}%
\pgfpathcurveto{\pgfqpoint{-0.011050in}{0.041667in}}{\pgfqpoint{-0.021649in}{0.037276in}}{\pgfqpoint{-0.029463in}{0.029463in}}%
\pgfpathcurveto{\pgfqpoint{-0.037276in}{0.021649in}}{\pgfqpoint{-0.041667in}{0.011050in}}{\pgfqpoint{-0.041667in}{0.000000in}}%
\pgfpathcurveto{\pgfqpoint{-0.041667in}{-0.011050in}}{\pgfqpoint{-0.037276in}{-0.021649in}}{\pgfqpoint{-0.029463in}{-0.029463in}}%
\pgfpathcurveto{\pgfqpoint{-0.021649in}{-0.037276in}}{\pgfqpoint{-0.011050in}{-0.041667in}}{\pgfqpoint{0.000000in}{-0.041667in}}%
\pgfpathclose%
\pgfusepath{stroke,fill}%
}%
\begin{pgfscope}%
\pgfsys@transformshift{12.180374in}{2.613739in}%
\pgfsys@useobject{currentmarker}{}%
\end{pgfscope}%
\end{pgfscope}%
\begin{pgfscope}%
\definecolor{textcolor}{rgb}{0.000000,0.000000,0.000000}%
\pgfsetstrokecolor{textcolor}%
\pgfsetfillcolor{textcolor}%
\pgftext[x=12.530374in,y=2.545683in,left,base]{\color{textcolor}\rmfamily\fontsize{14.000000}{16.800000}\selectfont NUCLEAR\_TB}%
\end{pgfscope}%
\begin{pgfscope}%
\pgfsetbuttcap%
\pgfsetroundjoin%
\pgfsetlinewidth{1.505625pt}%
\definecolor{currentstroke}{rgb}{1.000000,1.000000,0.000000}%
\pgfsetstrokecolor{currentstroke}%
\pgfsetdash{{5.550000pt}{2.400000pt}}{0.000000pt}%
\pgfpathmoveto{\pgfqpoint{11.985929in}{2.338739in}}%
\pgfpathlineto{\pgfqpoint{12.374818in}{2.338739in}}%
\pgfusepath{stroke}%
\end{pgfscope}%
\begin{pgfscope}%
\pgfsetbuttcap%
\pgfsetroundjoin%
\definecolor{currentfill}{rgb}{1.000000,1.000000,0.000000}%
\pgfsetfillcolor{currentfill}%
\pgfsetlinewidth{1.003750pt}%
\definecolor{currentstroke}{rgb}{1.000000,1.000000,0.000000}%
\pgfsetstrokecolor{currentstroke}%
\pgfsetdash{}{0pt}%
\pgfsys@defobject{currentmarker}{\pgfqpoint{-0.041667in}{-0.041667in}}{\pgfqpoint{0.041667in}{0.041667in}}{%
\pgfpathmoveto{\pgfqpoint{0.000000in}{-0.041667in}}%
\pgfpathcurveto{\pgfqpoint{0.011050in}{-0.041667in}}{\pgfqpoint{0.021649in}{-0.037276in}}{\pgfqpoint{0.029463in}{-0.029463in}}%
\pgfpathcurveto{\pgfqpoint{0.037276in}{-0.021649in}}{\pgfqpoint{0.041667in}{-0.011050in}}{\pgfqpoint{0.041667in}{0.000000in}}%
\pgfpathcurveto{\pgfqpoint{0.041667in}{0.011050in}}{\pgfqpoint{0.037276in}{0.021649in}}{\pgfqpoint{0.029463in}{0.029463in}}%
\pgfpathcurveto{\pgfqpoint{0.021649in}{0.037276in}}{\pgfqpoint{0.011050in}{0.041667in}}{\pgfqpoint{0.000000in}{0.041667in}}%
\pgfpathcurveto{\pgfqpoint{-0.011050in}{0.041667in}}{\pgfqpoint{-0.021649in}{0.037276in}}{\pgfqpoint{-0.029463in}{0.029463in}}%
\pgfpathcurveto{\pgfqpoint{-0.037276in}{0.021649in}}{\pgfqpoint{-0.041667in}{0.011050in}}{\pgfqpoint{-0.041667in}{0.000000in}}%
\pgfpathcurveto{\pgfqpoint{-0.041667in}{-0.011050in}}{\pgfqpoint{-0.037276in}{-0.021649in}}{\pgfqpoint{-0.029463in}{-0.029463in}}%
\pgfpathcurveto{\pgfqpoint{-0.021649in}{-0.037276in}}{\pgfqpoint{-0.011050in}{-0.041667in}}{\pgfqpoint{0.000000in}{-0.041667in}}%
\pgfpathclose%
\pgfusepath{stroke,fill}%
}%
\begin{pgfscope}%
\pgfsys@transformshift{12.180374in}{2.338739in}%
\pgfsys@useobject{currentmarker}{}%
\end{pgfscope}%
\end{pgfscope}%
\begin{pgfscope}%
\definecolor{textcolor}{rgb}{0.000000,0.000000,0.000000}%
\pgfsetstrokecolor{textcolor}%
\pgfsetfillcolor{textcolor}%
\pgftext[x=12.530374in,y=2.270684in,left,base]{\color{textcolor}\rmfamily\fontsize{14.000000}{16.800000}\selectfont SOLAR\_FARM}%
\end{pgfscope}%
\begin{pgfscope}%
\pgfsetbuttcap%
\pgfsetroundjoin%
\pgfsetlinewidth{1.505625pt}%
\definecolor{currentstroke}{rgb}{0.121569,0.466667,0.705882}%
\pgfsetstrokecolor{currentstroke}%
\pgfsetdash{{5.550000pt}{2.400000pt}}{0.000000pt}%
\pgfpathmoveto{\pgfqpoint{11.985929in}{2.063740in}}%
\pgfpathlineto{\pgfqpoint{12.374818in}{2.063740in}}%
\pgfusepath{stroke}%
\end{pgfscope}%
\begin{pgfscope}%
\pgfsetbuttcap%
\pgfsetroundjoin%
\definecolor{currentfill}{rgb}{0.121569,0.466667,0.705882}%
\pgfsetfillcolor{currentfill}%
\pgfsetlinewidth{1.003750pt}%
\definecolor{currentstroke}{rgb}{0.121569,0.466667,0.705882}%
\pgfsetstrokecolor{currentstroke}%
\pgfsetdash{}{0pt}%
\pgfsys@defobject{currentmarker}{\pgfqpoint{-0.041667in}{-0.041667in}}{\pgfqpoint{0.041667in}{0.041667in}}{%
\pgfpathmoveto{\pgfqpoint{0.000000in}{-0.041667in}}%
\pgfpathcurveto{\pgfqpoint{0.011050in}{-0.041667in}}{\pgfqpoint{0.021649in}{-0.037276in}}{\pgfqpoint{0.029463in}{-0.029463in}}%
\pgfpathcurveto{\pgfqpoint{0.037276in}{-0.021649in}}{\pgfqpoint{0.041667in}{-0.011050in}}{\pgfqpoint{0.041667in}{0.000000in}}%
\pgfpathcurveto{\pgfqpoint{0.041667in}{0.011050in}}{\pgfqpoint{0.037276in}{0.021649in}}{\pgfqpoint{0.029463in}{0.029463in}}%
\pgfpathcurveto{\pgfqpoint{0.021649in}{0.037276in}}{\pgfqpoint{0.011050in}{0.041667in}}{\pgfqpoint{0.000000in}{0.041667in}}%
\pgfpathcurveto{\pgfqpoint{-0.011050in}{0.041667in}}{\pgfqpoint{-0.021649in}{0.037276in}}{\pgfqpoint{-0.029463in}{0.029463in}}%
\pgfpathcurveto{\pgfqpoint{-0.037276in}{0.021649in}}{\pgfqpoint{-0.041667in}{0.011050in}}{\pgfqpoint{-0.041667in}{0.000000in}}%
\pgfpathcurveto{\pgfqpoint{-0.041667in}{-0.011050in}}{\pgfqpoint{-0.037276in}{-0.021649in}}{\pgfqpoint{-0.029463in}{-0.029463in}}%
\pgfpathcurveto{\pgfqpoint{-0.021649in}{-0.037276in}}{\pgfqpoint{-0.011050in}{-0.041667in}}{\pgfqpoint{0.000000in}{-0.041667in}}%
\pgfpathclose%
\pgfusepath{stroke,fill}%
}%
\begin{pgfscope}%
\pgfsys@transformshift{12.180374in}{2.063740in}%
\pgfsys@useobject{currentmarker}{}%
\end{pgfscope}%
\end{pgfscope}%
\begin{pgfscope}%
\definecolor{textcolor}{rgb}{0.000000,0.000000,0.000000}%
\pgfsetstrokecolor{textcolor}%
\pgfsetfillcolor{textcolor}%
\pgftext[x=12.530374in,y=1.995684in,left,base]{\color{textcolor}\rmfamily\fontsize{14.000000}{16.800000}\selectfont WIND\_FARM}%
\end{pgfscope}%
\begin{pgfscope}%
\definecolor{textcolor}{rgb}{0.000000,0.000000,0.000000}%
\pgfsetstrokecolor{textcolor}%
\pgfsetfillcolor{textcolor}%
\pgftext[x=5.950000in,y=4.850000in,,top]{\color{textcolor}\rmfamily\fontsize{24.000000}{28.800000}\selectfont UIUC Electric Generating Capacity in 2050}%
\end{pgfscope}%
\end{pgfpicture}%
\makeatother%
\endgroup%
}
  \caption{Results for capacity expansion under various slack values using modeling-
  to-generate alternatives. The x-axis ticks represent a different \gls{mga} run.
  The label indicates the run number and the value of the slack variable. For example,
  ``mga-1-1\%'' indicates the first \gls{mga} run with a one percent
  slack value. For reference, the cost minimized solution ``mga-0-x\%,'' is also
  plotted.}
  \label{fig:uiuc_elc_mga}
\end{figure}

I set the capacity limit for solar to 191 MW of photovoltaic cells, which corresponds
roughly to one-eighth of University-owned land. The fact that installed solar capacity
is so far below this limit suggests that further increases in solar capacity would
raise system costs beyond 20\% of the optimal solution.

Figure \ref{fig:uiuc_thm_mga} shows the \gls{mga} results for the steam sector
of \gls{uiuc}. Similar to the electric sector, these results change very little.
Allowing the system cost to increase by 1\% allows \gls{uiuc} to build additional
natural gas capacity, which is antithetical to the goal of net-zero carbon emissions.
The nuclear capacity does not change at all. These two features are also caused
by a lack of technology options in the steam sector. As discussed in section
\ref{section:alt_firm}, there are no actual steam production alternatives to coal
and natural gas besides nuclear power in Illinois.


\begin{figure}[H]
  \centering
  \resizebox{0.95\columnwidth}{!}{%% Creator: Matplotlib, PGF backend
%%
%% To include the figure in your LaTeX document, write
%%   \input{<filename>.pgf}
%%
%% Make sure the required packages are loaded in your preamble
%%   \usepackage{pgf}
%%
%% Figures using additional raster images can only be included by \input if
%% they are in the same directory as the main LaTeX file. For loading figures
%% from other directories you can use the `import` package
%%   \usepackage{import}
%%
%% and then include the figures with
%%   \import{<path to file>}{<filename>.pgf}
%%
%% Matplotlib used the following preamble
%%
\begingroup%
\makeatletter%
\begin{pgfpicture}%
\pgfpathrectangle{\pgfpointorigin}{\pgfqpoint{11.900880in}{4.950000in}}%
\pgfusepath{use as bounding box, clip}%
\begin{pgfscope}%
\pgfsetbuttcap%
\pgfsetmiterjoin%
\definecolor{currentfill}{rgb}{1.000000,1.000000,1.000000}%
\pgfsetfillcolor{currentfill}%
\pgfsetlinewidth{0.000000pt}%
\definecolor{currentstroke}{rgb}{0.000000,0.000000,0.000000}%
\pgfsetstrokecolor{currentstroke}%
\pgfsetdash{}{0pt}%
\pgfpathmoveto{\pgfqpoint{0.000000in}{0.000000in}}%
\pgfpathlineto{\pgfqpoint{11.900880in}{0.000000in}}%
\pgfpathlineto{\pgfqpoint{11.900880in}{4.950000in}}%
\pgfpathlineto{\pgfqpoint{0.000000in}{4.950000in}}%
\pgfpathclose%
\pgfusepath{fill}%
\end{pgfscope}%
\begin{pgfscope}%
\pgfsetbuttcap%
\pgfsetmiterjoin%
\definecolor{currentfill}{rgb}{0.898039,0.898039,0.898039}%
\pgfsetfillcolor{currentfill}%
\pgfsetlinewidth{0.000000pt}%
\definecolor{currentstroke}{rgb}{0.000000,0.000000,0.000000}%
\pgfsetstrokecolor{currentstroke}%
\pgfsetstrokeopacity{0.000000}%
\pgfsetdash{}{0pt}%
\pgfpathmoveto{\pgfqpoint{0.661957in}{1.263068in}}%
\pgfpathlineto{\pgfqpoint{2.701300in}{1.263068in}}%
\pgfpathlineto{\pgfqpoint{2.701300in}{4.342817in}}%
\pgfpathlineto{\pgfqpoint{0.661957in}{4.342817in}}%
\pgfpathclose%
\pgfusepath{fill}%
\end{pgfscope}%
\begin{pgfscope}%
\pgfpathrectangle{\pgfqpoint{0.661957in}{1.263068in}}{\pgfqpoint{2.039343in}{3.079750in}}%
\pgfusepath{clip}%
\pgfsetrectcap%
\pgfsetroundjoin%
\pgfsetlinewidth{0.803000pt}%
\definecolor{currentstroke}{rgb}{1.000000,1.000000,1.000000}%
\pgfsetstrokecolor{currentstroke}%
\pgfsetstrokeopacity{0.000000}%
\pgfsetdash{}{0pt}%
\pgfpathmoveto{\pgfqpoint{0.754655in}{1.263068in}}%
\pgfpathlineto{\pgfqpoint{0.754655in}{4.342817in}}%
\pgfusepath{stroke}%
\end{pgfscope}%
\begin{pgfscope}%
\pgfsetbuttcap%
\pgfsetroundjoin%
\definecolor{currentfill}{rgb}{0.333333,0.333333,0.333333}%
\pgfsetfillcolor{currentfill}%
\pgfsetlinewidth{0.803000pt}%
\definecolor{currentstroke}{rgb}{0.333333,0.333333,0.333333}%
\pgfsetstrokecolor{currentstroke}%
\pgfsetdash{}{0pt}%
\pgfsys@defobject{currentmarker}{\pgfqpoint{0.000000in}{-0.048611in}}{\pgfqpoint{0.000000in}{0.000000in}}{%
\pgfpathmoveto{\pgfqpoint{0.000000in}{0.000000in}}%
\pgfpathlineto{\pgfqpoint{0.000000in}{-0.048611in}}%
\pgfusepath{stroke,fill}%
}%
\begin{pgfscope}%
\pgfsys@transformshift{0.754655in}{1.263068in}%
\pgfsys@useobject{currentmarker}{}%
\end{pgfscope}%
\end{pgfscope}%
\begin{pgfscope}%
\definecolor{textcolor}{rgb}{0.333333,0.333333,0.333333}%
\pgfsetstrokecolor{textcolor}%
\pgfsetfillcolor{textcolor}%
\pgftext[x=0.821206in, y=0.210068in, left, base,rotate=90.000000]{\color{textcolor}\rmfamily\fontsize{16.000000}{19.200000}\selectfont mga-0-1\%}%
\end{pgfscope}%
\begin{pgfscope}%
\pgfpathrectangle{\pgfqpoint{0.661957in}{1.263068in}}{\pgfqpoint{2.039343in}{3.079750in}}%
\pgfusepath{clip}%
\pgfsetrectcap%
\pgfsetroundjoin%
\pgfsetlinewidth{0.803000pt}%
\definecolor{currentstroke}{rgb}{1.000000,1.000000,1.000000}%
\pgfsetstrokecolor{currentstroke}%
\pgfsetstrokeopacity{0.000000}%
\pgfsetdash{}{0pt}%
\pgfpathmoveto{\pgfqpoint{1.218142in}{1.263068in}}%
\pgfpathlineto{\pgfqpoint{1.218142in}{4.342817in}}%
\pgfusepath{stroke}%
\end{pgfscope}%
\begin{pgfscope}%
\pgfsetbuttcap%
\pgfsetroundjoin%
\definecolor{currentfill}{rgb}{0.333333,0.333333,0.333333}%
\pgfsetfillcolor{currentfill}%
\pgfsetlinewidth{0.803000pt}%
\definecolor{currentstroke}{rgb}{0.333333,0.333333,0.333333}%
\pgfsetstrokecolor{currentstroke}%
\pgfsetdash{}{0pt}%
\pgfsys@defobject{currentmarker}{\pgfqpoint{0.000000in}{-0.048611in}}{\pgfqpoint{0.000000in}{0.000000in}}{%
\pgfpathmoveto{\pgfqpoint{0.000000in}{0.000000in}}%
\pgfpathlineto{\pgfqpoint{0.000000in}{-0.048611in}}%
\pgfusepath{stroke,fill}%
}%
\begin{pgfscope}%
\pgfsys@transformshift{1.218142in}{1.263068in}%
\pgfsys@useobject{currentmarker}{}%
\end{pgfscope}%
\end{pgfscope}%
\begin{pgfscope}%
\definecolor{textcolor}{rgb}{0.333333,0.333333,0.333333}%
\pgfsetstrokecolor{textcolor}%
\pgfsetfillcolor{textcolor}%
\pgftext[x=1.284693in, y=0.210068in, left, base,rotate=90.000000]{\color{textcolor}\rmfamily\fontsize{16.000000}{19.200000}\selectfont mga-1-1\%}%
\end{pgfscope}%
\begin{pgfscope}%
\pgfpathrectangle{\pgfqpoint{0.661957in}{1.263068in}}{\pgfqpoint{2.039343in}{3.079750in}}%
\pgfusepath{clip}%
\pgfsetrectcap%
\pgfsetroundjoin%
\pgfsetlinewidth{0.803000pt}%
\definecolor{currentstroke}{rgb}{1.000000,1.000000,1.000000}%
\pgfsetstrokecolor{currentstroke}%
\pgfsetstrokeopacity{0.000000}%
\pgfsetdash{}{0pt}%
\pgfpathmoveto{\pgfqpoint{1.681629in}{1.263068in}}%
\pgfpathlineto{\pgfqpoint{1.681629in}{4.342817in}}%
\pgfusepath{stroke}%
\end{pgfscope}%
\begin{pgfscope}%
\pgfsetbuttcap%
\pgfsetroundjoin%
\definecolor{currentfill}{rgb}{0.333333,0.333333,0.333333}%
\pgfsetfillcolor{currentfill}%
\pgfsetlinewidth{0.803000pt}%
\definecolor{currentstroke}{rgb}{0.333333,0.333333,0.333333}%
\pgfsetstrokecolor{currentstroke}%
\pgfsetdash{}{0pt}%
\pgfsys@defobject{currentmarker}{\pgfqpoint{0.000000in}{-0.048611in}}{\pgfqpoint{0.000000in}{0.000000in}}{%
\pgfpathmoveto{\pgfqpoint{0.000000in}{0.000000in}}%
\pgfpathlineto{\pgfqpoint{0.000000in}{-0.048611in}}%
\pgfusepath{stroke,fill}%
}%
\begin{pgfscope}%
\pgfsys@transformshift{1.681629in}{1.263068in}%
\pgfsys@useobject{currentmarker}{}%
\end{pgfscope}%
\end{pgfscope}%
\begin{pgfscope}%
\definecolor{textcolor}{rgb}{0.333333,0.333333,0.333333}%
\pgfsetstrokecolor{textcolor}%
\pgfsetfillcolor{textcolor}%
\pgftext[x=1.748180in, y=0.210068in, left, base,rotate=90.000000]{\color{textcolor}\rmfamily\fontsize{16.000000}{19.200000}\selectfont mga-2-1\%}%
\end{pgfscope}%
\begin{pgfscope}%
\pgfpathrectangle{\pgfqpoint{0.661957in}{1.263068in}}{\pgfqpoint{2.039343in}{3.079750in}}%
\pgfusepath{clip}%
\pgfsetrectcap%
\pgfsetroundjoin%
\pgfsetlinewidth{0.803000pt}%
\definecolor{currentstroke}{rgb}{1.000000,1.000000,1.000000}%
\pgfsetstrokecolor{currentstroke}%
\pgfsetstrokeopacity{0.000000}%
\pgfsetdash{}{0pt}%
\pgfpathmoveto{\pgfqpoint{2.145116in}{1.263068in}}%
\pgfpathlineto{\pgfqpoint{2.145116in}{4.342817in}}%
\pgfusepath{stroke}%
\end{pgfscope}%
\begin{pgfscope}%
\pgfsetbuttcap%
\pgfsetroundjoin%
\definecolor{currentfill}{rgb}{0.333333,0.333333,0.333333}%
\pgfsetfillcolor{currentfill}%
\pgfsetlinewidth{0.803000pt}%
\definecolor{currentstroke}{rgb}{0.333333,0.333333,0.333333}%
\pgfsetstrokecolor{currentstroke}%
\pgfsetdash{}{0pt}%
\pgfsys@defobject{currentmarker}{\pgfqpoint{0.000000in}{-0.048611in}}{\pgfqpoint{0.000000in}{0.000000in}}{%
\pgfpathmoveto{\pgfqpoint{0.000000in}{0.000000in}}%
\pgfpathlineto{\pgfqpoint{0.000000in}{-0.048611in}}%
\pgfusepath{stroke,fill}%
}%
\begin{pgfscope}%
\pgfsys@transformshift{2.145116in}{1.263068in}%
\pgfsys@useobject{currentmarker}{}%
\end{pgfscope}%
\end{pgfscope}%
\begin{pgfscope}%
\definecolor{textcolor}{rgb}{0.333333,0.333333,0.333333}%
\pgfsetstrokecolor{textcolor}%
\pgfsetfillcolor{textcolor}%
\pgftext[x=2.211667in, y=0.210068in, left, base,rotate=90.000000]{\color{textcolor}\rmfamily\fontsize{16.000000}{19.200000}\selectfont mga-3-1\%}%
\end{pgfscope}%
\begin{pgfscope}%
\pgfpathrectangle{\pgfqpoint{0.661957in}{1.263068in}}{\pgfqpoint{2.039343in}{3.079750in}}%
\pgfusepath{clip}%
\pgfsetrectcap%
\pgfsetroundjoin%
\pgfsetlinewidth{0.803000pt}%
\definecolor{currentstroke}{rgb}{1.000000,1.000000,1.000000}%
\pgfsetstrokecolor{currentstroke}%
\pgfsetstrokeopacity{0.000000}%
\pgfsetdash{}{0pt}%
\pgfpathmoveto{\pgfqpoint{2.608603in}{1.263068in}}%
\pgfpathlineto{\pgfqpoint{2.608603in}{4.342817in}}%
\pgfusepath{stroke}%
\end{pgfscope}%
\begin{pgfscope}%
\pgfsetbuttcap%
\pgfsetroundjoin%
\definecolor{currentfill}{rgb}{0.333333,0.333333,0.333333}%
\pgfsetfillcolor{currentfill}%
\pgfsetlinewidth{0.803000pt}%
\definecolor{currentstroke}{rgb}{0.333333,0.333333,0.333333}%
\pgfsetstrokecolor{currentstroke}%
\pgfsetdash{}{0pt}%
\pgfsys@defobject{currentmarker}{\pgfqpoint{0.000000in}{-0.048611in}}{\pgfqpoint{0.000000in}{0.000000in}}{%
\pgfpathmoveto{\pgfqpoint{0.000000in}{0.000000in}}%
\pgfpathlineto{\pgfqpoint{0.000000in}{-0.048611in}}%
\pgfusepath{stroke,fill}%
}%
\begin{pgfscope}%
\pgfsys@transformshift{2.608603in}{1.263068in}%
\pgfsys@useobject{currentmarker}{}%
\end{pgfscope}%
\end{pgfscope}%
\begin{pgfscope}%
\definecolor{textcolor}{rgb}{0.333333,0.333333,0.333333}%
\pgfsetstrokecolor{textcolor}%
\pgfsetfillcolor{textcolor}%
\pgftext[x=2.675154in, y=0.210068in, left, base,rotate=90.000000]{\color{textcolor}\rmfamily\fontsize{16.000000}{19.200000}\selectfont mga-4-1\%}%
\end{pgfscope}%
\begin{pgfscope}%
\pgfpathrectangle{\pgfqpoint{0.661957in}{1.263068in}}{\pgfqpoint{2.039343in}{3.079750in}}%
\pgfusepath{clip}%
\pgfsetrectcap%
\pgfsetroundjoin%
\pgfsetlinewidth{0.803000pt}%
\definecolor{currentstroke}{rgb}{1.000000,1.000000,1.000000}%
\pgfsetstrokecolor{currentstroke}%
\pgfsetdash{}{0pt}%
\pgfpathmoveto{\pgfqpoint{0.661957in}{1.486456in}}%
\pgfpathlineto{\pgfqpoint{2.701300in}{1.486456in}}%
\pgfusepath{stroke}%
\end{pgfscope}%
\begin{pgfscope}%
\pgfsetbuttcap%
\pgfsetroundjoin%
\definecolor{currentfill}{rgb}{0.333333,0.333333,0.333333}%
\pgfsetfillcolor{currentfill}%
\pgfsetlinewidth{0.803000pt}%
\definecolor{currentstroke}{rgb}{0.333333,0.333333,0.333333}%
\pgfsetstrokecolor{currentstroke}%
\pgfsetdash{}{0pt}%
\pgfsys@defobject{currentmarker}{\pgfqpoint{-0.048611in}{0.000000in}}{\pgfqpoint{-0.000000in}{0.000000in}}{%
\pgfpathmoveto{\pgfqpoint{-0.000000in}{0.000000in}}%
\pgfpathlineto{\pgfqpoint{-0.048611in}{0.000000in}}%
\pgfusepath{stroke,fill}%
}%
\begin{pgfscope}%
\pgfsys@transformshift{0.661957in}{1.486456in}%
\pgfsys@useobject{currentmarker}{}%
\end{pgfscope}%
\end{pgfscope}%
\begin{pgfscope}%
\definecolor{textcolor}{rgb}{0.333333,0.333333,0.333333}%
\pgfsetstrokecolor{textcolor}%
\pgfsetfillcolor{textcolor}%
\pgftext[x=0.368904in, y=1.417012in, left, base]{\color{textcolor}\rmfamily\fontsize{14.000000}{16.800000}\selectfont \(\displaystyle {60}\)}%
\end{pgfscope}%
\begin{pgfscope}%
\pgfpathrectangle{\pgfqpoint{0.661957in}{1.263068in}}{\pgfqpoint{2.039343in}{3.079750in}}%
\pgfusepath{clip}%
\pgfsetrectcap%
\pgfsetroundjoin%
\pgfsetlinewidth{0.803000pt}%
\definecolor{currentstroke}{rgb}{1.000000,1.000000,1.000000}%
\pgfsetstrokecolor{currentstroke}%
\pgfsetdash{}{0pt}%
\pgfpathmoveto{\pgfqpoint{0.661957in}{1.975300in}}%
\pgfpathlineto{\pgfqpoint{2.701300in}{1.975300in}}%
\pgfusepath{stroke}%
\end{pgfscope}%
\begin{pgfscope}%
\pgfsetbuttcap%
\pgfsetroundjoin%
\definecolor{currentfill}{rgb}{0.333333,0.333333,0.333333}%
\pgfsetfillcolor{currentfill}%
\pgfsetlinewidth{0.803000pt}%
\definecolor{currentstroke}{rgb}{0.333333,0.333333,0.333333}%
\pgfsetstrokecolor{currentstroke}%
\pgfsetdash{}{0pt}%
\pgfsys@defobject{currentmarker}{\pgfqpoint{-0.048611in}{0.000000in}}{\pgfqpoint{-0.000000in}{0.000000in}}{%
\pgfpathmoveto{\pgfqpoint{-0.000000in}{0.000000in}}%
\pgfpathlineto{\pgfqpoint{-0.048611in}{0.000000in}}%
\pgfusepath{stroke,fill}%
}%
\begin{pgfscope}%
\pgfsys@transformshift{0.661957in}{1.975300in}%
\pgfsys@useobject{currentmarker}{}%
\end{pgfscope}%
\end{pgfscope}%
\begin{pgfscope}%
\definecolor{textcolor}{rgb}{0.333333,0.333333,0.333333}%
\pgfsetstrokecolor{textcolor}%
\pgfsetfillcolor{textcolor}%
\pgftext[x=0.368904in, y=1.905855in, left, base]{\color{textcolor}\rmfamily\fontsize{14.000000}{16.800000}\selectfont \(\displaystyle {65}\)}%
\end{pgfscope}%
\begin{pgfscope}%
\pgfpathrectangle{\pgfqpoint{0.661957in}{1.263068in}}{\pgfqpoint{2.039343in}{3.079750in}}%
\pgfusepath{clip}%
\pgfsetrectcap%
\pgfsetroundjoin%
\pgfsetlinewidth{0.803000pt}%
\definecolor{currentstroke}{rgb}{1.000000,1.000000,1.000000}%
\pgfsetstrokecolor{currentstroke}%
\pgfsetdash{}{0pt}%
\pgfpathmoveto{\pgfqpoint{0.661957in}{2.464144in}}%
\pgfpathlineto{\pgfqpoint{2.701300in}{2.464144in}}%
\pgfusepath{stroke}%
\end{pgfscope}%
\begin{pgfscope}%
\pgfsetbuttcap%
\pgfsetroundjoin%
\definecolor{currentfill}{rgb}{0.333333,0.333333,0.333333}%
\pgfsetfillcolor{currentfill}%
\pgfsetlinewidth{0.803000pt}%
\definecolor{currentstroke}{rgb}{0.333333,0.333333,0.333333}%
\pgfsetstrokecolor{currentstroke}%
\pgfsetdash{}{0pt}%
\pgfsys@defobject{currentmarker}{\pgfqpoint{-0.048611in}{0.000000in}}{\pgfqpoint{-0.000000in}{0.000000in}}{%
\pgfpathmoveto{\pgfqpoint{-0.000000in}{0.000000in}}%
\pgfpathlineto{\pgfqpoint{-0.048611in}{0.000000in}}%
\pgfusepath{stroke,fill}%
}%
\begin{pgfscope}%
\pgfsys@transformshift{0.661957in}{2.464144in}%
\pgfsys@useobject{currentmarker}{}%
\end{pgfscope}%
\end{pgfscope}%
\begin{pgfscope}%
\definecolor{textcolor}{rgb}{0.333333,0.333333,0.333333}%
\pgfsetstrokecolor{textcolor}%
\pgfsetfillcolor{textcolor}%
\pgftext[x=0.368904in, y=2.394699in, left, base]{\color{textcolor}\rmfamily\fontsize{14.000000}{16.800000}\selectfont \(\displaystyle {70}\)}%
\end{pgfscope}%
\begin{pgfscope}%
\pgfpathrectangle{\pgfqpoint{0.661957in}{1.263068in}}{\pgfqpoint{2.039343in}{3.079750in}}%
\pgfusepath{clip}%
\pgfsetrectcap%
\pgfsetroundjoin%
\pgfsetlinewidth{0.803000pt}%
\definecolor{currentstroke}{rgb}{1.000000,1.000000,1.000000}%
\pgfsetstrokecolor{currentstroke}%
\pgfsetdash{}{0pt}%
\pgfpathmoveto{\pgfqpoint{0.661957in}{2.952987in}}%
\pgfpathlineto{\pgfqpoint{2.701300in}{2.952987in}}%
\pgfusepath{stroke}%
\end{pgfscope}%
\begin{pgfscope}%
\pgfsetbuttcap%
\pgfsetroundjoin%
\definecolor{currentfill}{rgb}{0.333333,0.333333,0.333333}%
\pgfsetfillcolor{currentfill}%
\pgfsetlinewidth{0.803000pt}%
\definecolor{currentstroke}{rgb}{0.333333,0.333333,0.333333}%
\pgfsetstrokecolor{currentstroke}%
\pgfsetdash{}{0pt}%
\pgfsys@defobject{currentmarker}{\pgfqpoint{-0.048611in}{0.000000in}}{\pgfqpoint{-0.000000in}{0.000000in}}{%
\pgfpathmoveto{\pgfqpoint{-0.000000in}{0.000000in}}%
\pgfpathlineto{\pgfqpoint{-0.048611in}{0.000000in}}%
\pgfusepath{stroke,fill}%
}%
\begin{pgfscope}%
\pgfsys@transformshift{0.661957in}{2.952987in}%
\pgfsys@useobject{currentmarker}{}%
\end{pgfscope}%
\end{pgfscope}%
\begin{pgfscope}%
\definecolor{textcolor}{rgb}{0.333333,0.333333,0.333333}%
\pgfsetstrokecolor{textcolor}%
\pgfsetfillcolor{textcolor}%
\pgftext[x=0.368904in, y=2.883543in, left, base]{\color{textcolor}\rmfamily\fontsize{14.000000}{16.800000}\selectfont \(\displaystyle {75}\)}%
\end{pgfscope}%
\begin{pgfscope}%
\pgfpathrectangle{\pgfqpoint{0.661957in}{1.263068in}}{\pgfqpoint{2.039343in}{3.079750in}}%
\pgfusepath{clip}%
\pgfsetrectcap%
\pgfsetroundjoin%
\pgfsetlinewidth{0.803000pt}%
\definecolor{currentstroke}{rgb}{1.000000,1.000000,1.000000}%
\pgfsetstrokecolor{currentstroke}%
\pgfsetdash{}{0pt}%
\pgfpathmoveto{\pgfqpoint{0.661957in}{3.441831in}}%
\pgfpathlineto{\pgfqpoint{2.701300in}{3.441831in}}%
\pgfusepath{stroke}%
\end{pgfscope}%
\begin{pgfscope}%
\pgfsetbuttcap%
\pgfsetroundjoin%
\definecolor{currentfill}{rgb}{0.333333,0.333333,0.333333}%
\pgfsetfillcolor{currentfill}%
\pgfsetlinewidth{0.803000pt}%
\definecolor{currentstroke}{rgb}{0.333333,0.333333,0.333333}%
\pgfsetstrokecolor{currentstroke}%
\pgfsetdash{}{0pt}%
\pgfsys@defobject{currentmarker}{\pgfqpoint{-0.048611in}{0.000000in}}{\pgfqpoint{-0.000000in}{0.000000in}}{%
\pgfpathmoveto{\pgfqpoint{-0.000000in}{0.000000in}}%
\pgfpathlineto{\pgfqpoint{-0.048611in}{0.000000in}}%
\pgfusepath{stroke,fill}%
}%
\begin{pgfscope}%
\pgfsys@transformshift{0.661957in}{3.441831in}%
\pgfsys@useobject{currentmarker}{}%
\end{pgfscope}%
\end{pgfscope}%
\begin{pgfscope}%
\definecolor{textcolor}{rgb}{0.333333,0.333333,0.333333}%
\pgfsetstrokecolor{textcolor}%
\pgfsetfillcolor{textcolor}%
\pgftext[x=0.368904in, y=3.372387in, left, base]{\color{textcolor}\rmfamily\fontsize{14.000000}{16.800000}\selectfont \(\displaystyle {80}\)}%
\end{pgfscope}%
\begin{pgfscope}%
\pgfpathrectangle{\pgfqpoint{0.661957in}{1.263068in}}{\pgfqpoint{2.039343in}{3.079750in}}%
\pgfusepath{clip}%
\pgfsetrectcap%
\pgfsetroundjoin%
\pgfsetlinewidth{0.803000pt}%
\definecolor{currentstroke}{rgb}{1.000000,1.000000,1.000000}%
\pgfsetstrokecolor{currentstroke}%
\pgfsetdash{}{0pt}%
\pgfpathmoveto{\pgfqpoint{0.661957in}{3.930675in}}%
\pgfpathlineto{\pgfqpoint{2.701300in}{3.930675in}}%
\pgfusepath{stroke}%
\end{pgfscope}%
\begin{pgfscope}%
\pgfsetbuttcap%
\pgfsetroundjoin%
\definecolor{currentfill}{rgb}{0.333333,0.333333,0.333333}%
\pgfsetfillcolor{currentfill}%
\pgfsetlinewidth{0.803000pt}%
\definecolor{currentstroke}{rgb}{0.333333,0.333333,0.333333}%
\pgfsetstrokecolor{currentstroke}%
\pgfsetdash{}{0pt}%
\pgfsys@defobject{currentmarker}{\pgfqpoint{-0.048611in}{0.000000in}}{\pgfqpoint{-0.000000in}{0.000000in}}{%
\pgfpathmoveto{\pgfqpoint{-0.000000in}{0.000000in}}%
\pgfpathlineto{\pgfqpoint{-0.048611in}{0.000000in}}%
\pgfusepath{stroke,fill}%
}%
\begin{pgfscope}%
\pgfsys@transformshift{0.661957in}{3.930675in}%
\pgfsys@useobject{currentmarker}{}%
\end{pgfscope}%
\end{pgfscope}%
\begin{pgfscope}%
\definecolor{textcolor}{rgb}{0.333333,0.333333,0.333333}%
\pgfsetstrokecolor{textcolor}%
\pgfsetfillcolor{textcolor}%
\pgftext[x=0.368904in, y=3.861231in, left, base]{\color{textcolor}\rmfamily\fontsize{14.000000}{16.800000}\selectfont \(\displaystyle {85}\)}%
\end{pgfscope}%
\begin{pgfscope}%
\definecolor{textcolor}{rgb}{0.333333,0.333333,0.333333}%
\pgfsetstrokecolor{textcolor}%
\pgfsetfillcolor{textcolor}%
\pgftext[x=0.313349in,y=2.802942in,,bottom,rotate=90.000000]{\color{textcolor}\rmfamily\fontsize{16.000000}{19.200000}\selectfont MW\(\displaystyle _{th}\)}%
\end{pgfscope}%
\begin{pgfscope}%
\pgfpathrectangle{\pgfqpoint{0.661957in}{1.263068in}}{\pgfqpoint{2.039343in}{3.079750in}}%
\pgfusepath{clip}%
\pgfsetbuttcap%
\pgfsetroundjoin%
\pgfsetlinewidth{1.505625pt}%
\definecolor{currentstroke}{rgb}{0.839216,0.152941,0.156863}%
\pgfsetstrokecolor{currentstroke}%
\pgfsetdash{{5.550000pt}{2.400000pt}}{0.000000pt}%
\pgfpathmoveto{\pgfqpoint{0.754655in}{1.403056in}}%
\pgfpathlineto{\pgfqpoint{1.218142in}{1.898950in}}%
\pgfpathlineto{\pgfqpoint{1.681629in}{1.893526in}}%
\pgfpathlineto{\pgfqpoint{2.145116in}{1.899779in}}%
\pgfpathlineto{\pgfqpoint{2.608603in}{1.903201in}}%
\pgfusepath{stroke}%
\end{pgfscope}%
\begin{pgfscope}%
\pgfpathrectangle{\pgfqpoint{0.661957in}{1.263068in}}{\pgfqpoint{2.039343in}{3.079750in}}%
\pgfusepath{clip}%
\pgfsetbuttcap%
\pgfsetroundjoin%
\definecolor{currentfill}{rgb}{0.839216,0.152941,0.156863}%
\pgfsetfillcolor{currentfill}%
\pgfsetlinewidth{1.003750pt}%
\definecolor{currentstroke}{rgb}{0.839216,0.152941,0.156863}%
\pgfsetstrokecolor{currentstroke}%
\pgfsetdash{}{0pt}%
\pgfsys@defobject{currentmarker}{\pgfqpoint{-0.041667in}{-0.041667in}}{\pgfqpoint{0.041667in}{0.041667in}}{%
\pgfpathmoveto{\pgfqpoint{0.000000in}{-0.041667in}}%
\pgfpathcurveto{\pgfqpoint{0.011050in}{-0.041667in}}{\pgfqpoint{0.021649in}{-0.037276in}}{\pgfqpoint{0.029463in}{-0.029463in}}%
\pgfpathcurveto{\pgfqpoint{0.037276in}{-0.021649in}}{\pgfqpoint{0.041667in}{-0.011050in}}{\pgfqpoint{0.041667in}{0.000000in}}%
\pgfpathcurveto{\pgfqpoint{0.041667in}{0.011050in}}{\pgfqpoint{0.037276in}{0.021649in}}{\pgfqpoint{0.029463in}{0.029463in}}%
\pgfpathcurveto{\pgfqpoint{0.021649in}{0.037276in}}{\pgfqpoint{0.011050in}{0.041667in}}{\pgfqpoint{0.000000in}{0.041667in}}%
\pgfpathcurveto{\pgfqpoint{-0.011050in}{0.041667in}}{\pgfqpoint{-0.021649in}{0.037276in}}{\pgfqpoint{-0.029463in}{0.029463in}}%
\pgfpathcurveto{\pgfqpoint{-0.037276in}{0.021649in}}{\pgfqpoint{-0.041667in}{0.011050in}}{\pgfqpoint{-0.041667in}{0.000000in}}%
\pgfpathcurveto{\pgfqpoint{-0.041667in}{-0.011050in}}{\pgfqpoint{-0.037276in}{-0.021649in}}{\pgfqpoint{-0.029463in}{-0.029463in}}%
\pgfpathcurveto{\pgfqpoint{-0.021649in}{-0.037276in}}{\pgfqpoint{-0.011050in}{-0.041667in}}{\pgfqpoint{0.000000in}{-0.041667in}}%
\pgfpathclose%
\pgfusepath{stroke,fill}%
}%
\begin{pgfscope}%
\pgfsys@transformshift{0.754655in}{1.403056in}%
\pgfsys@useobject{currentmarker}{}%
\end{pgfscope}%
\begin{pgfscope}%
\pgfsys@transformshift{1.218142in}{1.898950in}%
\pgfsys@useobject{currentmarker}{}%
\end{pgfscope}%
\begin{pgfscope}%
\pgfsys@transformshift{1.681629in}{1.893526in}%
\pgfsys@useobject{currentmarker}{}%
\end{pgfscope}%
\begin{pgfscope}%
\pgfsys@transformshift{2.145116in}{1.899779in}%
\pgfsys@useobject{currentmarker}{}%
\end{pgfscope}%
\begin{pgfscope}%
\pgfsys@transformshift{2.608603in}{1.903201in}%
\pgfsys@useobject{currentmarker}{}%
\end{pgfscope}%
\end{pgfscope}%
\begin{pgfscope}%
\pgfpathrectangle{\pgfqpoint{0.661957in}{1.263068in}}{\pgfqpoint{2.039343in}{3.079750in}}%
\pgfusepath{clip}%
\pgfsetbuttcap%
\pgfsetroundjoin%
\pgfsetlinewidth{1.505625pt}%
\definecolor{currentstroke}{rgb}{0.172549,0.627451,0.172549}%
\pgfsetstrokecolor{currentstroke}%
\pgfsetdash{{9.600000pt}{2.400000pt}{1.500000pt}{2.400000pt}}{0.000000pt}%
\pgfpathmoveto{\pgfqpoint{0.754655in}{4.202829in}}%
\pgfpathlineto{\pgfqpoint{1.218142in}{4.202331in}}%
\pgfpathlineto{\pgfqpoint{1.681629in}{4.202829in}}%
\pgfpathlineto{\pgfqpoint{2.145116in}{4.202829in}}%
\pgfpathlineto{\pgfqpoint{2.608603in}{4.202829in}}%
\pgfusepath{stroke}%
\end{pgfscope}%
\begin{pgfscope}%
\pgfpathrectangle{\pgfqpoint{0.661957in}{1.263068in}}{\pgfqpoint{2.039343in}{3.079750in}}%
\pgfusepath{clip}%
\pgfsetbuttcap%
\pgfsetmiterjoin%
\definecolor{currentfill}{rgb}{0.172549,0.627451,0.172549}%
\pgfsetfillcolor{currentfill}%
\pgfsetlinewidth{1.003750pt}%
\definecolor{currentstroke}{rgb}{0.172549,0.627451,0.172549}%
\pgfsetstrokecolor{currentstroke}%
\pgfsetdash{}{0pt}%
\pgfsys@defobject{currentmarker}{\pgfqpoint{-0.041667in}{-0.041667in}}{\pgfqpoint{0.041667in}{0.041667in}}{%
\pgfpathmoveto{\pgfqpoint{0.000000in}{0.041667in}}%
\pgfpathlineto{\pgfqpoint{-0.041667in}{-0.041667in}}%
\pgfpathlineto{\pgfqpoint{0.041667in}{-0.041667in}}%
\pgfpathclose%
\pgfusepath{stroke,fill}%
}%
\begin{pgfscope}%
\pgfsys@transformshift{0.754655in}{4.202829in}%
\pgfsys@useobject{currentmarker}{}%
\end{pgfscope}%
\begin{pgfscope}%
\pgfsys@transformshift{1.218142in}{4.202331in}%
\pgfsys@useobject{currentmarker}{}%
\end{pgfscope}%
\begin{pgfscope}%
\pgfsys@transformshift{1.681629in}{4.202829in}%
\pgfsys@useobject{currentmarker}{}%
\end{pgfscope}%
\begin{pgfscope}%
\pgfsys@transformshift{2.145116in}{4.202829in}%
\pgfsys@useobject{currentmarker}{}%
\end{pgfscope}%
\begin{pgfscope}%
\pgfsys@transformshift{2.608603in}{4.202829in}%
\pgfsys@useobject{currentmarker}{}%
\end{pgfscope}%
\end{pgfscope}%
\begin{pgfscope}%
\pgfsetrectcap%
\pgfsetmiterjoin%
\pgfsetlinewidth{1.003750pt}%
\definecolor{currentstroke}{rgb}{1.000000,1.000000,1.000000}%
\pgfsetstrokecolor{currentstroke}%
\pgfsetdash{}{0pt}%
\pgfpathmoveto{\pgfqpoint{0.661957in}{1.263067in}}%
\pgfpathlineto{\pgfqpoint{0.661957in}{4.342817in}}%
\pgfusepath{stroke}%
\end{pgfscope}%
\begin{pgfscope}%
\pgfsetrectcap%
\pgfsetmiterjoin%
\pgfsetlinewidth{1.003750pt}%
\definecolor{currentstroke}{rgb}{1.000000,1.000000,1.000000}%
\pgfsetstrokecolor{currentstroke}%
\pgfsetdash{}{0pt}%
\pgfpathmoveto{\pgfqpoint{2.701300in}{1.263067in}}%
\pgfpathlineto{\pgfqpoint{2.701300in}{4.342817in}}%
\pgfusepath{stroke}%
\end{pgfscope}%
\begin{pgfscope}%
\pgfsetrectcap%
\pgfsetmiterjoin%
\pgfsetlinewidth{1.003750pt}%
\definecolor{currentstroke}{rgb}{1.000000,1.000000,1.000000}%
\pgfsetstrokecolor{currentstroke}%
\pgfsetdash{}{0pt}%
\pgfpathmoveto{\pgfqpoint{0.661957in}{1.263068in}}%
\pgfpathlineto{\pgfqpoint{2.701300in}{1.263068in}}%
\pgfusepath{stroke}%
\end{pgfscope}%
\begin{pgfscope}%
\pgfsetrectcap%
\pgfsetmiterjoin%
\pgfsetlinewidth{1.003750pt}%
\definecolor{currentstroke}{rgb}{1.000000,1.000000,1.000000}%
\pgfsetstrokecolor{currentstroke}%
\pgfsetdash{}{0pt}%
\pgfpathmoveto{\pgfqpoint{0.661957in}{4.342817in}}%
\pgfpathlineto{\pgfqpoint{2.701300in}{4.342817in}}%
\pgfusepath{stroke}%
\end{pgfscope}%
\begin{pgfscope}%
\pgfsetbuttcap%
\pgfsetmiterjoin%
\definecolor{currentfill}{rgb}{0.898039,0.898039,0.898039}%
\pgfsetfillcolor{currentfill}%
\pgfsetlinewidth{0.000000pt}%
\definecolor{currentstroke}{rgb}{0.000000,0.000000,0.000000}%
\pgfsetstrokecolor{currentstroke}%
\pgfsetstrokeopacity{0.000000}%
\pgfsetdash{}{0pt}%
\pgfpathmoveto{\pgfqpoint{2.921247in}{1.263068in}}%
\pgfpathlineto{\pgfqpoint{4.960590in}{1.263068in}}%
\pgfpathlineto{\pgfqpoint{4.960590in}{4.342817in}}%
\pgfpathlineto{\pgfqpoint{2.921247in}{4.342817in}}%
\pgfpathclose%
\pgfusepath{fill}%
\end{pgfscope}%
\begin{pgfscope}%
\pgfpathrectangle{\pgfqpoint{2.921247in}{1.263068in}}{\pgfqpoint{2.039343in}{3.079750in}}%
\pgfusepath{clip}%
\pgfsetrectcap%
\pgfsetroundjoin%
\pgfsetlinewidth{0.803000pt}%
\definecolor{currentstroke}{rgb}{1.000000,1.000000,1.000000}%
\pgfsetstrokecolor{currentstroke}%
\pgfsetstrokeopacity{0.000000}%
\pgfsetdash{}{0pt}%
\pgfpathmoveto{\pgfqpoint{3.013944in}{1.263068in}}%
\pgfpathlineto{\pgfqpoint{3.013944in}{4.342817in}}%
\pgfusepath{stroke}%
\end{pgfscope}%
\begin{pgfscope}%
\pgfsetbuttcap%
\pgfsetroundjoin%
\definecolor{currentfill}{rgb}{0.333333,0.333333,0.333333}%
\pgfsetfillcolor{currentfill}%
\pgfsetlinewidth{0.803000pt}%
\definecolor{currentstroke}{rgb}{0.333333,0.333333,0.333333}%
\pgfsetstrokecolor{currentstroke}%
\pgfsetdash{}{0pt}%
\pgfsys@defobject{currentmarker}{\pgfqpoint{0.000000in}{-0.048611in}}{\pgfqpoint{0.000000in}{0.000000in}}{%
\pgfpathmoveto{\pgfqpoint{0.000000in}{0.000000in}}%
\pgfpathlineto{\pgfqpoint{0.000000in}{-0.048611in}}%
\pgfusepath{stroke,fill}%
}%
\begin{pgfscope}%
\pgfsys@transformshift{3.013944in}{1.263068in}%
\pgfsys@useobject{currentmarker}{}%
\end{pgfscope}%
\end{pgfscope}%
\begin{pgfscope}%
\definecolor{textcolor}{rgb}{0.333333,0.333333,0.333333}%
\pgfsetstrokecolor{textcolor}%
\pgfsetfillcolor{textcolor}%
\pgftext[x=3.080495in, y=0.210068in, left, base,rotate=90.000000]{\color{textcolor}\rmfamily\fontsize{16.000000}{19.200000}\selectfont mga-0-5\%}%
\end{pgfscope}%
\begin{pgfscope}%
\pgfpathrectangle{\pgfqpoint{2.921247in}{1.263068in}}{\pgfqpoint{2.039343in}{3.079750in}}%
\pgfusepath{clip}%
\pgfsetrectcap%
\pgfsetroundjoin%
\pgfsetlinewidth{0.803000pt}%
\definecolor{currentstroke}{rgb}{1.000000,1.000000,1.000000}%
\pgfsetstrokecolor{currentstroke}%
\pgfsetstrokeopacity{0.000000}%
\pgfsetdash{}{0pt}%
\pgfpathmoveto{\pgfqpoint{3.477432in}{1.263068in}}%
\pgfpathlineto{\pgfqpoint{3.477432in}{4.342817in}}%
\pgfusepath{stroke}%
\end{pgfscope}%
\begin{pgfscope}%
\pgfsetbuttcap%
\pgfsetroundjoin%
\definecolor{currentfill}{rgb}{0.333333,0.333333,0.333333}%
\pgfsetfillcolor{currentfill}%
\pgfsetlinewidth{0.803000pt}%
\definecolor{currentstroke}{rgb}{0.333333,0.333333,0.333333}%
\pgfsetstrokecolor{currentstroke}%
\pgfsetdash{}{0pt}%
\pgfsys@defobject{currentmarker}{\pgfqpoint{0.000000in}{-0.048611in}}{\pgfqpoint{0.000000in}{0.000000in}}{%
\pgfpathmoveto{\pgfqpoint{0.000000in}{0.000000in}}%
\pgfpathlineto{\pgfqpoint{0.000000in}{-0.048611in}}%
\pgfusepath{stroke,fill}%
}%
\begin{pgfscope}%
\pgfsys@transformshift{3.477432in}{1.263068in}%
\pgfsys@useobject{currentmarker}{}%
\end{pgfscope}%
\end{pgfscope}%
\begin{pgfscope}%
\definecolor{textcolor}{rgb}{0.333333,0.333333,0.333333}%
\pgfsetstrokecolor{textcolor}%
\pgfsetfillcolor{textcolor}%
\pgftext[x=3.543982in, y=0.210068in, left, base,rotate=90.000000]{\color{textcolor}\rmfamily\fontsize{16.000000}{19.200000}\selectfont mga-1-5\%}%
\end{pgfscope}%
\begin{pgfscope}%
\pgfpathrectangle{\pgfqpoint{2.921247in}{1.263068in}}{\pgfqpoint{2.039343in}{3.079750in}}%
\pgfusepath{clip}%
\pgfsetrectcap%
\pgfsetroundjoin%
\pgfsetlinewidth{0.803000pt}%
\definecolor{currentstroke}{rgb}{1.000000,1.000000,1.000000}%
\pgfsetstrokecolor{currentstroke}%
\pgfsetstrokeopacity{0.000000}%
\pgfsetdash{}{0pt}%
\pgfpathmoveto{\pgfqpoint{3.940919in}{1.263068in}}%
\pgfpathlineto{\pgfqpoint{3.940919in}{4.342817in}}%
\pgfusepath{stroke}%
\end{pgfscope}%
\begin{pgfscope}%
\pgfsetbuttcap%
\pgfsetroundjoin%
\definecolor{currentfill}{rgb}{0.333333,0.333333,0.333333}%
\pgfsetfillcolor{currentfill}%
\pgfsetlinewidth{0.803000pt}%
\definecolor{currentstroke}{rgb}{0.333333,0.333333,0.333333}%
\pgfsetstrokecolor{currentstroke}%
\pgfsetdash{}{0pt}%
\pgfsys@defobject{currentmarker}{\pgfqpoint{0.000000in}{-0.048611in}}{\pgfqpoint{0.000000in}{0.000000in}}{%
\pgfpathmoveto{\pgfqpoint{0.000000in}{0.000000in}}%
\pgfpathlineto{\pgfqpoint{0.000000in}{-0.048611in}}%
\pgfusepath{stroke,fill}%
}%
\begin{pgfscope}%
\pgfsys@transformshift{3.940919in}{1.263068in}%
\pgfsys@useobject{currentmarker}{}%
\end{pgfscope}%
\end{pgfscope}%
\begin{pgfscope}%
\definecolor{textcolor}{rgb}{0.333333,0.333333,0.333333}%
\pgfsetstrokecolor{textcolor}%
\pgfsetfillcolor{textcolor}%
\pgftext[x=4.007470in, y=0.210068in, left, base,rotate=90.000000]{\color{textcolor}\rmfamily\fontsize{16.000000}{19.200000}\selectfont mga-2-5\%}%
\end{pgfscope}%
\begin{pgfscope}%
\pgfpathrectangle{\pgfqpoint{2.921247in}{1.263068in}}{\pgfqpoint{2.039343in}{3.079750in}}%
\pgfusepath{clip}%
\pgfsetrectcap%
\pgfsetroundjoin%
\pgfsetlinewidth{0.803000pt}%
\definecolor{currentstroke}{rgb}{1.000000,1.000000,1.000000}%
\pgfsetstrokecolor{currentstroke}%
\pgfsetstrokeopacity{0.000000}%
\pgfsetdash{}{0pt}%
\pgfpathmoveto{\pgfqpoint{4.404406in}{1.263068in}}%
\pgfpathlineto{\pgfqpoint{4.404406in}{4.342817in}}%
\pgfusepath{stroke}%
\end{pgfscope}%
\begin{pgfscope}%
\pgfsetbuttcap%
\pgfsetroundjoin%
\definecolor{currentfill}{rgb}{0.333333,0.333333,0.333333}%
\pgfsetfillcolor{currentfill}%
\pgfsetlinewidth{0.803000pt}%
\definecolor{currentstroke}{rgb}{0.333333,0.333333,0.333333}%
\pgfsetstrokecolor{currentstroke}%
\pgfsetdash{}{0pt}%
\pgfsys@defobject{currentmarker}{\pgfqpoint{0.000000in}{-0.048611in}}{\pgfqpoint{0.000000in}{0.000000in}}{%
\pgfpathmoveto{\pgfqpoint{0.000000in}{0.000000in}}%
\pgfpathlineto{\pgfqpoint{0.000000in}{-0.048611in}}%
\pgfusepath{stroke,fill}%
}%
\begin{pgfscope}%
\pgfsys@transformshift{4.404406in}{1.263068in}%
\pgfsys@useobject{currentmarker}{}%
\end{pgfscope}%
\end{pgfscope}%
\begin{pgfscope}%
\definecolor{textcolor}{rgb}{0.333333,0.333333,0.333333}%
\pgfsetstrokecolor{textcolor}%
\pgfsetfillcolor{textcolor}%
\pgftext[x=4.470957in, y=0.210068in, left, base,rotate=90.000000]{\color{textcolor}\rmfamily\fontsize{16.000000}{19.200000}\selectfont mga-3-5\%}%
\end{pgfscope}%
\begin{pgfscope}%
\pgfpathrectangle{\pgfqpoint{2.921247in}{1.263068in}}{\pgfqpoint{2.039343in}{3.079750in}}%
\pgfusepath{clip}%
\pgfsetrectcap%
\pgfsetroundjoin%
\pgfsetlinewidth{0.803000pt}%
\definecolor{currentstroke}{rgb}{1.000000,1.000000,1.000000}%
\pgfsetstrokecolor{currentstroke}%
\pgfsetstrokeopacity{0.000000}%
\pgfsetdash{}{0pt}%
\pgfpathmoveto{\pgfqpoint{4.867893in}{1.263068in}}%
\pgfpathlineto{\pgfqpoint{4.867893in}{4.342817in}}%
\pgfusepath{stroke}%
\end{pgfscope}%
\begin{pgfscope}%
\pgfsetbuttcap%
\pgfsetroundjoin%
\definecolor{currentfill}{rgb}{0.333333,0.333333,0.333333}%
\pgfsetfillcolor{currentfill}%
\pgfsetlinewidth{0.803000pt}%
\definecolor{currentstroke}{rgb}{0.333333,0.333333,0.333333}%
\pgfsetstrokecolor{currentstroke}%
\pgfsetdash{}{0pt}%
\pgfsys@defobject{currentmarker}{\pgfqpoint{0.000000in}{-0.048611in}}{\pgfqpoint{0.000000in}{0.000000in}}{%
\pgfpathmoveto{\pgfqpoint{0.000000in}{0.000000in}}%
\pgfpathlineto{\pgfqpoint{0.000000in}{-0.048611in}}%
\pgfusepath{stroke,fill}%
}%
\begin{pgfscope}%
\pgfsys@transformshift{4.867893in}{1.263068in}%
\pgfsys@useobject{currentmarker}{}%
\end{pgfscope}%
\end{pgfscope}%
\begin{pgfscope}%
\definecolor{textcolor}{rgb}{0.333333,0.333333,0.333333}%
\pgfsetstrokecolor{textcolor}%
\pgfsetfillcolor{textcolor}%
\pgftext[x=4.934444in, y=0.210068in, left, base,rotate=90.000000]{\color{textcolor}\rmfamily\fontsize{16.000000}{19.200000}\selectfont mga-4-5\%}%
\end{pgfscope}%
\begin{pgfscope}%
\pgfpathrectangle{\pgfqpoint{2.921247in}{1.263068in}}{\pgfqpoint{2.039343in}{3.079750in}}%
\pgfusepath{clip}%
\pgfsetrectcap%
\pgfsetroundjoin%
\pgfsetlinewidth{0.803000pt}%
\definecolor{currentstroke}{rgb}{1.000000,1.000000,1.000000}%
\pgfsetstrokecolor{currentstroke}%
\pgfsetdash{}{0pt}%
\pgfpathmoveto{\pgfqpoint{2.921247in}{1.486456in}}%
\pgfpathlineto{\pgfqpoint{4.960590in}{1.486456in}}%
\pgfusepath{stroke}%
\end{pgfscope}%
\begin{pgfscope}%
\pgfsetbuttcap%
\pgfsetroundjoin%
\definecolor{currentfill}{rgb}{0.333333,0.333333,0.333333}%
\pgfsetfillcolor{currentfill}%
\pgfsetlinewidth{0.803000pt}%
\definecolor{currentstroke}{rgb}{0.333333,0.333333,0.333333}%
\pgfsetstrokecolor{currentstroke}%
\pgfsetdash{}{0pt}%
\pgfsys@defobject{currentmarker}{\pgfqpoint{-0.048611in}{0.000000in}}{\pgfqpoint{-0.000000in}{0.000000in}}{%
\pgfpathmoveto{\pgfqpoint{-0.000000in}{0.000000in}}%
\pgfpathlineto{\pgfqpoint{-0.048611in}{0.000000in}}%
\pgfusepath{stroke,fill}%
}%
\begin{pgfscope}%
\pgfsys@transformshift{2.921247in}{1.486456in}%
\pgfsys@useobject{currentmarker}{}%
\end{pgfscope}%
\end{pgfscope}%
\begin{pgfscope}%
\pgfpathrectangle{\pgfqpoint{2.921247in}{1.263068in}}{\pgfqpoint{2.039343in}{3.079750in}}%
\pgfusepath{clip}%
\pgfsetrectcap%
\pgfsetroundjoin%
\pgfsetlinewidth{0.803000pt}%
\definecolor{currentstroke}{rgb}{1.000000,1.000000,1.000000}%
\pgfsetstrokecolor{currentstroke}%
\pgfsetdash{}{0pt}%
\pgfpathmoveto{\pgfqpoint{2.921247in}{1.975300in}}%
\pgfpathlineto{\pgfqpoint{4.960590in}{1.975300in}}%
\pgfusepath{stroke}%
\end{pgfscope}%
\begin{pgfscope}%
\pgfsetbuttcap%
\pgfsetroundjoin%
\definecolor{currentfill}{rgb}{0.333333,0.333333,0.333333}%
\pgfsetfillcolor{currentfill}%
\pgfsetlinewidth{0.803000pt}%
\definecolor{currentstroke}{rgb}{0.333333,0.333333,0.333333}%
\pgfsetstrokecolor{currentstroke}%
\pgfsetdash{}{0pt}%
\pgfsys@defobject{currentmarker}{\pgfqpoint{-0.048611in}{0.000000in}}{\pgfqpoint{-0.000000in}{0.000000in}}{%
\pgfpathmoveto{\pgfqpoint{-0.000000in}{0.000000in}}%
\pgfpathlineto{\pgfqpoint{-0.048611in}{0.000000in}}%
\pgfusepath{stroke,fill}%
}%
\begin{pgfscope}%
\pgfsys@transformshift{2.921247in}{1.975300in}%
\pgfsys@useobject{currentmarker}{}%
\end{pgfscope}%
\end{pgfscope}%
\begin{pgfscope}%
\pgfpathrectangle{\pgfqpoint{2.921247in}{1.263068in}}{\pgfqpoint{2.039343in}{3.079750in}}%
\pgfusepath{clip}%
\pgfsetrectcap%
\pgfsetroundjoin%
\pgfsetlinewidth{0.803000pt}%
\definecolor{currentstroke}{rgb}{1.000000,1.000000,1.000000}%
\pgfsetstrokecolor{currentstroke}%
\pgfsetdash{}{0pt}%
\pgfpathmoveto{\pgfqpoint{2.921247in}{2.464144in}}%
\pgfpathlineto{\pgfqpoint{4.960590in}{2.464144in}}%
\pgfusepath{stroke}%
\end{pgfscope}%
\begin{pgfscope}%
\pgfsetbuttcap%
\pgfsetroundjoin%
\definecolor{currentfill}{rgb}{0.333333,0.333333,0.333333}%
\pgfsetfillcolor{currentfill}%
\pgfsetlinewidth{0.803000pt}%
\definecolor{currentstroke}{rgb}{0.333333,0.333333,0.333333}%
\pgfsetstrokecolor{currentstroke}%
\pgfsetdash{}{0pt}%
\pgfsys@defobject{currentmarker}{\pgfqpoint{-0.048611in}{0.000000in}}{\pgfqpoint{-0.000000in}{0.000000in}}{%
\pgfpathmoveto{\pgfqpoint{-0.000000in}{0.000000in}}%
\pgfpathlineto{\pgfqpoint{-0.048611in}{0.000000in}}%
\pgfusepath{stroke,fill}%
}%
\begin{pgfscope}%
\pgfsys@transformshift{2.921247in}{2.464144in}%
\pgfsys@useobject{currentmarker}{}%
\end{pgfscope}%
\end{pgfscope}%
\begin{pgfscope}%
\pgfpathrectangle{\pgfqpoint{2.921247in}{1.263068in}}{\pgfqpoint{2.039343in}{3.079750in}}%
\pgfusepath{clip}%
\pgfsetrectcap%
\pgfsetroundjoin%
\pgfsetlinewidth{0.803000pt}%
\definecolor{currentstroke}{rgb}{1.000000,1.000000,1.000000}%
\pgfsetstrokecolor{currentstroke}%
\pgfsetdash{}{0pt}%
\pgfpathmoveto{\pgfqpoint{2.921247in}{2.952987in}}%
\pgfpathlineto{\pgfqpoint{4.960590in}{2.952987in}}%
\pgfusepath{stroke}%
\end{pgfscope}%
\begin{pgfscope}%
\pgfsetbuttcap%
\pgfsetroundjoin%
\definecolor{currentfill}{rgb}{0.333333,0.333333,0.333333}%
\pgfsetfillcolor{currentfill}%
\pgfsetlinewidth{0.803000pt}%
\definecolor{currentstroke}{rgb}{0.333333,0.333333,0.333333}%
\pgfsetstrokecolor{currentstroke}%
\pgfsetdash{}{0pt}%
\pgfsys@defobject{currentmarker}{\pgfqpoint{-0.048611in}{0.000000in}}{\pgfqpoint{-0.000000in}{0.000000in}}{%
\pgfpathmoveto{\pgfqpoint{-0.000000in}{0.000000in}}%
\pgfpathlineto{\pgfqpoint{-0.048611in}{0.000000in}}%
\pgfusepath{stroke,fill}%
}%
\begin{pgfscope}%
\pgfsys@transformshift{2.921247in}{2.952987in}%
\pgfsys@useobject{currentmarker}{}%
\end{pgfscope}%
\end{pgfscope}%
\begin{pgfscope}%
\pgfpathrectangle{\pgfqpoint{2.921247in}{1.263068in}}{\pgfqpoint{2.039343in}{3.079750in}}%
\pgfusepath{clip}%
\pgfsetrectcap%
\pgfsetroundjoin%
\pgfsetlinewidth{0.803000pt}%
\definecolor{currentstroke}{rgb}{1.000000,1.000000,1.000000}%
\pgfsetstrokecolor{currentstroke}%
\pgfsetdash{}{0pt}%
\pgfpathmoveto{\pgfqpoint{2.921247in}{3.441831in}}%
\pgfpathlineto{\pgfqpoint{4.960590in}{3.441831in}}%
\pgfusepath{stroke}%
\end{pgfscope}%
\begin{pgfscope}%
\pgfsetbuttcap%
\pgfsetroundjoin%
\definecolor{currentfill}{rgb}{0.333333,0.333333,0.333333}%
\pgfsetfillcolor{currentfill}%
\pgfsetlinewidth{0.803000pt}%
\definecolor{currentstroke}{rgb}{0.333333,0.333333,0.333333}%
\pgfsetstrokecolor{currentstroke}%
\pgfsetdash{}{0pt}%
\pgfsys@defobject{currentmarker}{\pgfqpoint{-0.048611in}{0.000000in}}{\pgfqpoint{-0.000000in}{0.000000in}}{%
\pgfpathmoveto{\pgfqpoint{-0.000000in}{0.000000in}}%
\pgfpathlineto{\pgfqpoint{-0.048611in}{0.000000in}}%
\pgfusepath{stroke,fill}%
}%
\begin{pgfscope}%
\pgfsys@transformshift{2.921247in}{3.441831in}%
\pgfsys@useobject{currentmarker}{}%
\end{pgfscope}%
\end{pgfscope}%
\begin{pgfscope}%
\pgfpathrectangle{\pgfqpoint{2.921247in}{1.263068in}}{\pgfqpoint{2.039343in}{3.079750in}}%
\pgfusepath{clip}%
\pgfsetrectcap%
\pgfsetroundjoin%
\pgfsetlinewidth{0.803000pt}%
\definecolor{currentstroke}{rgb}{1.000000,1.000000,1.000000}%
\pgfsetstrokecolor{currentstroke}%
\pgfsetdash{}{0pt}%
\pgfpathmoveto{\pgfqpoint{2.921247in}{3.930675in}}%
\pgfpathlineto{\pgfqpoint{4.960590in}{3.930675in}}%
\pgfusepath{stroke}%
\end{pgfscope}%
\begin{pgfscope}%
\pgfsetbuttcap%
\pgfsetroundjoin%
\definecolor{currentfill}{rgb}{0.333333,0.333333,0.333333}%
\pgfsetfillcolor{currentfill}%
\pgfsetlinewidth{0.803000pt}%
\definecolor{currentstroke}{rgb}{0.333333,0.333333,0.333333}%
\pgfsetstrokecolor{currentstroke}%
\pgfsetdash{}{0pt}%
\pgfsys@defobject{currentmarker}{\pgfqpoint{-0.048611in}{0.000000in}}{\pgfqpoint{-0.000000in}{0.000000in}}{%
\pgfpathmoveto{\pgfqpoint{-0.000000in}{0.000000in}}%
\pgfpathlineto{\pgfqpoint{-0.048611in}{0.000000in}}%
\pgfusepath{stroke,fill}%
}%
\begin{pgfscope}%
\pgfsys@transformshift{2.921247in}{3.930675in}%
\pgfsys@useobject{currentmarker}{}%
\end{pgfscope}%
\end{pgfscope}%
\begin{pgfscope}%
\pgfpathrectangle{\pgfqpoint{2.921247in}{1.263068in}}{\pgfqpoint{2.039343in}{3.079750in}}%
\pgfusepath{clip}%
\pgfsetbuttcap%
\pgfsetroundjoin%
\pgfsetlinewidth{1.505625pt}%
\definecolor{currentstroke}{rgb}{0.839216,0.152941,0.156863}%
\pgfsetstrokecolor{currentstroke}%
\pgfsetdash{{5.550000pt}{2.400000pt}}{0.000000pt}%
\pgfpathmoveto{\pgfqpoint{3.013944in}{1.403056in}}%
\pgfpathlineto{\pgfqpoint{3.477432in}{1.901308in}}%
\pgfpathlineto{\pgfqpoint{3.940919in}{1.898716in}}%
\pgfpathlineto{\pgfqpoint{4.404406in}{1.902009in}}%
\pgfpathlineto{\pgfqpoint{4.867893in}{1.903523in}}%
\pgfusepath{stroke}%
\end{pgfscope}%
\begin{pgfscope}%
\pgfpathrectangle{\pgfqpoint{2.921247in}{1.263068in}}{\pgfqpoint{2.039343in}{3.079750in}}%
\pgfusepath{clip}%
\pgfsetbuttcap%
\pgfsetroundjoin%
\definecolor{currentfill}{rgb}{0.839216,0.152941,0.156863}%
\pgfsetfillcolor{currentfill}%
\pgfsetlinewidth{1.003750pt}%
\definecolor{currentstroke}{rgb}{0.839216,0.152941,0.156863}%
\pgfsetstrokecolor{currentstroke}%
\pgfsetdash{}{0pt}%
\pgfsys@defobject{currentmarker}{\pgfqpoint{-0.041667in}{-0.041667in}}{\pgfqpoint{0.041667in}{0.041667in}}{%
\pgfpathmoveto{\pgfqpoint{0.000000in}{-0.041667in}}%
\pgfpathcurveto{\pgfqpoint{0.011050in}{-0.041667in}}{\pgfqpoint{0.021649in}{-0.037276in}}{\pgfqpoint{0.029463in}{-0.029463in}}%
\pgfpathcurveto{\pgfqpoint{0.037276in}{-0.021649in}}{\pgfqpoint{0.041667in}{-0.011050in}}{\pgfqpoint{0.041667in}{0.000000in}}%
\pgfpathcurveto{\pgfqpoint{0.041667in}{0.011050in}}{\pgfqpoint{0.037276in}{0.021649in}}{\pgfqpoint{0.029463in}{0.029463in}}%
\pgfpathcurveto{\pgfqpoint{0.021649in}{0.037276in}}{\pgfqpoint{0.011050in}{0.041667in}}{\pgfqpoint{0.000000in}{0.041667in}}%
\pgfpathcurveto{\pgfqpoint{-0.011050in}{0.041667in}}{\pgfqpoint{-0.021649in}{0.037276in}}{\pgfqpoint{-0.029463in}{0.029463in}}%
\pgfpathcurveto{\pgfqpoint{-0.037276in}{0.021649in}}{\pgfqpoint{-0.041667in}{0.011050in}}{\pgfqpoint{-0.041667in}{0.000000in}}%
\pgfpathcurveto{\pgfqpoint{-0.041667in}{-0.011050in}}{\pgfqpoint{-0.037276in}{-0.021649in}}{\pgfqpoint{-0.029463in}{-0.029463in}}%
\pgfpathcurveto{\pgfqpoint{-0.021649in}{-0.037276in}}{\pgfqpoint{-0.011050in}{-0.041667in}}{\pgfqpoint{0.000000in}{-0.041667in}}%
\pgfpathclose%
\pgfusepath{stroke,fill}%
}%
\begin{pgfscope}%
\pgfsys@transformshift{3.013944in}{1.403056in}%
\pgfsys@useobject{currentmarker}{}%
\end{pgfscope}%
\begin{pgfscope}%
\pgfsys@transformshift{3.477432in}{1.901308in}%
\pgfsys@useobject{currentmarker}{}%
\end{pgfscope}%
\begin{pgfscope}%
\pgfsys@transformshift{3.940919in}{1.898716in}%
\pgfsys@useobject{currentmarker}{}%
\end{pgfscope}%
\begin{pgfscope}%
\pgfsys@transformshift{4.404406in}{1.902009in}%
\pgfsys@useobject{currentmarker}{}%
\end{pgfscope}%
\begin{pgfscope}%
\pgfsys@transformshift{4.867893in}{1.903523in}%
\pgfsys@useobject{currentmarker}{}%
\end{pgfscope}%
\end{pgfscope}%
\begin{pgfscope}%
\pgfpathrectangle{\pgfqpoint{2.921247in}{1.263068in}}{\pgfqpoint{2.039343in}{3.079750in}}%
\pgfusepath{clip}%
\pgfsetbuttcap%
\pgfsetroundjoin%
\pgfsetlinewidth{1.505625pt}%
\definecolor{currentstroke}{rgb}{0.172549,0.627451,0.172549}%
\pgfsetstrokecolor{currentstroke}%
\pgfsetdash{{9.600000pt}{2.400000pt}{1.500000pt}{2.400000pt}}{0.000000pt}%
\pgfpathmoveto{\pgfqpoint{3.013944in}{4.202829in}}%
\pgfpathlineto{\pgfqpoint{3.477432in}{4.202829in}}%
\pgfpathlineto{\pgfqpoint{3.940919in}{4.202829in}}%
\pgfpathlineto{\pgfqpoint{4.404406in}{4.202829in}}%
\pgfpathlineto{\pgfqpoint{4.867893in}{4.202829in}}%
\pgfusepath{stroke}%
\end{pgfscope}%
\begin{pgfscope}%
\pgfpathrectangle{\pgfqpoint{2.921247in}{1.263068in}}{\pgfqpoint{2.039343in}{3.079750in}}%
\pgfusepath{clip}%
\pgfsetbuttcap%
\pgfsetmiterjoin%
\definecolor{currentfill}{rgb}{0.172549,0.627451,0.172549}%
\pgfsetfillcolor{currentfill}%
\pgfsetlinewidth{1.003750pt}%
\definecolor{currentstroke}{rgb}{0.172549,0.627451,0.172549}%
\pgfsetstrokecolor{currentstroke}%
\pgfsetdash{}{0pt}%
\pgfsys@defobject{currentmarker}{\pgfqpoint{-0.041667in}{-0.041667in}}{\pgfqpoint{0.041667in}{0.041667in}}{%
\pgfpathmoveto{\pgfqpoint{0.000000in}{0.041667in}}%
\pgfpathlineto{\pgfqpoint{-0.041667in}{-0.041667in}}%
\pgfpathlineto{\pgfqpoint{0.041667in}{-0.041667in}}%
\pgfpathclose%
\pgfusepath{stroke,fill}%
}%
\begin{pgfscope}%
\pgfsys@transformshift{3.013944in}{4.202829in}%
\pgfsys@useobject{currentmarker}{}%
\end{pgfscope}%
\begin{pgfscope}%
\pgfsys@transformshift{3.477432in}{4.202829in}%
\pgfsys@useobject{currentmarker}{}%
\end{pgfscope}%
\begin{pgfscope}%
\pgfsys@transformshift{3.940919in}{4.202829in}%
\pgfsys@useobject{currentmarker}{}%
\end{pgfscope}%
\begin{pgfscope}%
\pgfsys@transformshift{4.404406in}{4.202829in}%
\pgfsys@useobject{currentmarker}{}%
\end{pgfscope}%
\begin{pgfscope}%
\pgfsys@transformshift{4.867893in}{4.202829in}%
\pgfsys@useobject{currentmarker}{}%
\end{pgfscope}%
\end{pgfscope}%
\begin{pgfscope}%
\pgfsetrectcap%
\pgfsetmiterjoin%
\pgfsetlinewidth{1.003750pt}%
\definecolor{currentstroke}{rgb}{1.000000,1.000000,1.000000}%
\pgfsetstrokecolor{currentstroke}%
\pgfsetdash{}{0pt}%
\pgfpathmoveto{\pgfqpoint{2.921247in}{1.263067in}}%
\pgfpathlineto{\pgfqpoint{2.921247in}{4.342817in}}%
\pgfusepath{stroke}%
\end{pgfscope}%
\begin{pgfscope}%
\pgfsetrectcap%
\pgfsetmiterjoin%
\pgfsetlinewidth{1.003750pt}%
\definecolor{currentstroke}{rgb}{1.000000,1.000000,1.000000}%
\pgfsetstrokecolor{currentstroke}%
\pgfsetdash{}{0pt}%
\pgfpathmoveto{\pgfqpoint{4.960590in}{1.263067in}}%
\pgfpathlineto{\pgfqpoint{4.960590in}{4.342817in}}%
\pgfusepath{stroke}%
\end{pgfscope}%
\begin{pgfscope}%
\pgfsetrectcap%
\pgfsetmiterjoin%
\pgfsetlinewidth{1.003750pt}%
\definecolor{currentstroke}{rgb}{1.000000,1.000000,1.000000}%
\pgfsetstrokecolor{currentstroke}%
\pgfsetdash{}{0pt}%
\pgfpathmoveto{\pgfqpoint{2.921247in}{1.263068in}}%
\pgfpathlineto{\pgfqpoint{4.960590in}{1.263068in}}%
\pgfusepath{stroke}%
\end{pgfscope}%
\begin{pgfscope}%
\pgfsetrectcap%
\pgfsetmiterjoin%
\pgfsetlinewidth{1.003750pt}%
\definecolor{currentstroke}{rgb}{1.000000,1.000000,1.000000}%
\pgfsetstrokecolor{currentstroke}%
\pgfsetdash{}{0pt}%
\pgfpathmoveto{\pgfqpoint{2.921247in}{4.342817in}}%
\pgfpathlineto{\pgfqpoint{4.960590in}{4.342817in}}%
\pgfusepath{stroke}%
\end{pgfscope}%
\begin{pgfscope}%
\pgfsetbuttcap%
\pgfsetmiterjoin%
\definecolor{currentfill}{rgb}{0.898039,0.898039,0.898039}%
\pgfsetfillcolor{currentfill}%
\pgfsetlinewidth{0.000000pt}%
\definecolor{currentstroke}{rgb}{0.000000,0.000000,0.000000}%
\pgfsetstrokecolor{currentstroke}%
\pgfsetstrokeopacity{0.000000}%
\pgfsetdash{}{0pt}%
\pgfpathmoveto{\pgfqpoint{5.180537in}{1.263068in}}%
\pgfpathlineto{\pgfqpoint{7.219880in}{1.263068in}}%
\pgfpathlineto{\pgfqpoint{7.219880in}{4.342817in}}%
\pgfpathlineto{\pgfqpoint{5.180537in}{4.342817in}}%
\pgfpathclose%
\pgfusepath{fill}%
\end{pgfscope}%
\begin{pgfscope}%
\pgfpathrectangle{\pgfqpoint{5.180537in}{1.263068in}}{\pgfqpoint{2.039343in}{3.079750in}}%
\pgfusepath{clip}%
\pgfsetrectcap%
\pgfsetroundjoin%
\pgfsetlinewidth{0.803000pt}%
\definecolor{currentstroke}{rgb}{1.000000,1.000000,1.000000}%
\pgfsetstrokecolor{currentstroke}%
\pgfsetstrokeopacity{0.000000}%
\pgfsetdash{}{0pt}%
\pgfpathmoveto{\pgfqpoint{5.273234in}{1.263068in}}%
\pgfpathlineto{\pgfqpoint{5.273234in}{4.342817in}}%
\pgfusepath{stroke}%
\end{pgfscope}%
\begin{pgfscope}%
\pgfsetbuttcap%
\pgfsetroundjoin%
\definecolor{currentfill}{rgb}{0.333333,0.333333,0.333333}%
\pgfsetfillcolor{currentfill}%
\pgfsetlinewidth{0.803000pt}%
\definecolor{currentstroke}{rgb}{0.333333,0.333333,0.333333}%
\pgfsetstrokecolor{currentstroke}%
\pgfsetdash{}{0pt}%
\pgfsys@defobject{currentmarker}{\pgfqpoint{0.000000in}{-0.048611in}}{\pgfqpoint{0.000000in}{0.000000in}}{%
\pgfpathmoveto{\pgfqpoint{0.000000in}{0.000000in}}%
\pgfpathlineto{\pgfqpoint{0.000000in}{-0.048611in}}%
\pgfusepath{stroke,fill}%
}%
\begin{pgfscope}%
\pgfsys@transformshift{5.273234in}{1.263068in}%
\pgfsys@useobject{currentmarker}{}%
\end{pgfscope}%
\end{pgfscope}%
\begin{pgfscope}%
\definecolor{textcolor}{rgb}{0.333333,0.333333,0.333333}%
\pgfsetstrokecolor{textcolor}%
\pgfsetfillcolor{textcolor}%
\pgftext[x=5.339785in, y=0.100000in, left, base,rotate=90.000000]{\color{textcolor}\rmfamily\fontsize{16.000000}{19.200000}\selectfont mga-0-10\%}%
\end{pgfscope}%
\begin{pgfscope}%
\pgfpathrectangle{\pgfqpoint{5.180537in}{1.263068in}}{\pgfqpoint{2.039343in}{3.079750in}}%
\pgfusepath{clip}%
\pgfsetrectcap%
\pgfsetroundjoin%
\pgfsetlinewidth{0.803000pt}%
\definecolor{currentstroke}{rgb}{1.000000,1.000000,1.000000}%
\pgfsetstrokecolor{currentstroke}%
\pgfsetstrokeopacity{0.000000}%
\pgfsetdash{}{0pt}%
\pgfpathmoveto{\pgfqpoint{5.736721in}{1.263068in}}%
\pgfpathlineto{\pgfqpoint{5.736721in}{4.342817in}}%
\pgfusepath{stroke}%
\end{pgfscope}%
\begin{pgfscope}%
\pgfsetbuttcap%
\pgfsetroundjoin%
\definecolor{currentfill}{rgb}{0.333333,0.333333,0.333333}%
\pgfsetfillcolor{currentfill}%
\pgfsetlinewidth{0.803000pt}%
\definecolor{currentstroke}{rgb}{0.333333,0.333333,0.333333}%
\pgfsetstrokecolor{currentstroke}%
\pgfsetdash{}{0pt}%
\pgfsys@defobject{currentmarker}{\pgfqpoint{0.000000in}{-0.048611in}}{\pgfqpoint{0.000000in}{0.000000in}}{%
\pgfpathmoveto{\pgfqpoint{0.000000in}{0.000000in}}%
\pgfpathlineto{\pgfqpoint{0.000000in}{-0.048611in}}%
\pgfusepath{stroke,fill}%
}%
\begin{pgfscope}%
\pgfsys@transformshift{5.736721in}{1.263068in}%
\pgfsys@useobject{currentmarker}{}%
\end{pgfscope}%
\end{pgfscope}%
\begin{pgfscope}%
\definecolor{textcolor}{rgb}{0.333333,0.333333,0.333333}%
\pgfsetstrokecolor{textcolor}%
\pgfsetfillcolor{textcolor}%
\pgftext[x=5.803272in, y=0.100000in, left, base,rotate=90.000000]{\color{textcolor}\rmfamily\fontsize{16.000000}{19.200000}\selectfont mga-1-10\%}%
\end{pgfscope}%
\begin{pgfscope}%
\pgfpathrectangle{\pgfqpoint{5.180537in}{1.263068in}}{\pgfqpoint{2.039343in}{3.079750in}}%
\pgfusepath{clip}%
\pgfsetrectcap%
\pgfsetroundjoin%
\pgfsetlinewidth{0.803000pt}%
\definecolor{currentstroke}{rgb}{1.000000,1.000000,1.000000}%
\pgfsetstrokecolor{currentstroke}%
\pgfsetstrokeopacity{0.000000}%
\pgfsetdash{}{0pt}%
\pgfpathmoveto{\pgfqpoint{6.200208in}{1.263068in}}%
\pgfpathlineto{\pgfqpoint{6.200208in}{4.342817in}}%
\pgfusepath{stroke}%
\end{pgfscope}%
\begin{pgfscope}%
\pgfsetbuttcap%
\pgfsetroundjoin%
\definecolor{currentfill}{rgb}{0.333333,0.333333,0.333333}%
\pgfsetfillcolor{currentfill}%
\pgfsetlinewidth{0.803000pt}%
\definecolor{currentstroke}{rgb}{0.333333,0.333333,0.333333}%
\pgfsetstrokecolor{currentstroke}%
\pgfsetdash{}{0pt}%
\pgfsys@defobject{currentmarker}{\pgfqpoint{0.000000in}{-0.048611in}}{\pgfqpoint{0.000000in}{0.000000in}}{%
\pgfpathmoveto{\pgfqpoint{0.000000in}{0.000000in}}%
\pgfpathlineto{\pgfqpoint{0.000000in}{-0.048611in}}%
\pgfusepath{stroke,fill}%
}%
\begin{pgfscope}%
\pgfsys@transformshift{6.200208in}{1.263068in}%
\pgfsys@useobject{currentmarker}{}%
\end{pgfscope}%
\end{pgfscope}%
\begin{pgfscope}%
\definecolor{textcolor}{rgb}{0.333333,0.333333,0.333333}%
\pgfsetstrokecolor{textcolor}%
\pgfsetfillcolor{textcolor}%
\pgftext[x=6.266759in, y=0.100000in, left, base,rotate=90.000000]{\color{textcolor}\rmfamily\fontsize{16.000000}{19.200000}\selectfont mga-2-10\%}%
\end{pgfscope}%
\begin{pgfscope}%
\pgfpathrectangle{\pgfqpoint{5.180537in}{1.263068in}}{\pgfqpoint{2.039343in}{3.079750in}}%
\pgfusepath{clip}%
\pgfsetrectcap%
\pgfsetroundjoin%
\pgfsetlinewidth{0.803000pt}%
\definecolor{currentstroke}{rgb}{1.000000,1.000000,1.000000}%
\pgfsetstrokecolor{currentstroke}%
\pgfsetstrokeopacity{0.000000}%
\pgfsetdash{}{0pt}%
\pgfpathmoveto{\pgfqpoint{6.663696in}{1.263068in}}%
\pgfpathlineto{\pgfqpoint{6.663696in}{4.342817in}}%
\pgfusepath{stroke}%
\end{pgfscope}%
\begin{pgfscope}%
\pgfsetbuttcap%
\pgfsetroundjoin%
\definecolor{currentfill}{rgb}{0.333333,0.333333,0.333333}%
\pgfsetfillcolor{currentfill}%
\pgfsetlinewidth{0.803000pt}%
\definecolor{currentstroke}{rgb}{0.333333,0.333333,0.333333}%
\pgfsetstrokecolor{currentstroke}%
\pgfsetdash{}{0pt}%
\pgfsys@defobject{currentmarker}{\pgfqpoint{0.000000in}{-0.048611in}}{\pgfqpoint{0.000000in}{0.000000in}}{%
\pgfpathmoveto{\pgfqpoint{0.000000in}{0.000000in}}%
\pgfpathlineto{\pgfqpoint{0.000000in}{-0.048611in}}%
\pgfusepath{stroke,fill}%
}%
\begin{pgfscope}%
\pgfsys@transformshift{6.663696in}{1.263068in}%
\pgfsys@useobject{currentmarker}{}%
\end{pgfscope}%
\end{pgfscope}%
\begin{pgfscope}%
\definecolor{textcolor}{rgb}{0.333333,0.333333,0.333333}%
\pgfsetstrokecolor{textcolor}%
\pgfsetfillcolor{textcolor}%
\pgftext[x=6.730246in, y=0.100000in, left, base,rotate=90.000000]{\color{textcolor}\rmfamily\fontsize{16.000000}{19.200000}\selectfont mga-3-10\%}%
\end{pgfscope}%
\begin{pgfscope}%
\pgfpathrectangle{\pgfqpoint{5.180537in}{1.263068in}}{\pgfqpoint{2.039343in}{3.079750in}}%
\pgfusepath{clip}%
\pgfsetrectcap%
\pgfsetroundjoin%
\pgfsetlinewidth{0.803000pt}%
\definecolor{currentstroke}{rgb}{1.000000,1.000000,1.000000}%
\pgfsetstrokecolor{currentstroke}%
\pgfsetstrokeopacity{0.000000}%
\pgfsetdash{}{0pt}%
\pgfpathmoveto{\pgfqpoint{7.127183in}{1.263068in}}%
\pgfpathlineto{\pgfqpoint{7.127183in}{4.342817in}}%
\pgfusepath{stroke}%
\end{pgfscope}%
\begin{pgfscope}%
\pgfsetbuttcap%
\pgfsetroundjoin%
\definecolor{currentfill}{rgb}{0.333333,0.333333,0.333333}%
\pgfsetfillcolor{currentfill}%
\pgfsetlinewidth{0.803000pt}%
\definecolor{currentstroke}{rgb}{0.333333,0.333333,0.333333}%
\pgfsetstrokecolor{currentstroke}%
\pgfsetdash{}{0pt}%
\pgfsys@defobject{currentmarker}{\pgfqpoint{0.000000in}{-0.048611in}}{\pgfqpoint{0.000000in}{0.000000in}}{%
\pgfpathmoveto{\pgfqpoint{0.000000in}{0.000000in}}%
\pgfpathlineto{\pgfqpoint{0.000000in}{-0.048611in}}%
\pgfusepath{stroke,fill}%
}%
\begin{pgfscope}%
\pgfsys@transformshift{7.127183in}{1.263068in}%
\pgfsys@useobject{currentmarker}{}%
\end{pgfscope}%
\end{pgfscope}%
\begin{pgfscope}%
\definecolor{textcolor}{rgb}{0.333333,0.333333,0.333333}%
\pgfsetstrokecolor{textcolor}%
\pgfsetfillcolor{textcolor}%
\pgftext[x=7.193733in, y=0.100000in, left, base,rotate=90.000000]{\color{textcolor}\rmfamily\fontsize{16.000000}{19.200000}\selectfont mga-4-10\%}%
\end{pgfscope}%
\begin{pgfscope}%
\pgfpathrectangle{\pgfqpoint{5.180537in}{1.263068in}}{\pgfqpoint{2.039343in}{3.079750in}}%
\pgfusepath{clip}%
\pgfsetrectcap%
\pgfsetroundjoin%
\pgfsetlinewidth{0.803000pt}%
\definecolor{currentstroke}{rgb}{1.000000,1.000000,1.000000}%
\pgfsetstrokecolor{currentstroke}%
\pgfsetdash{}{0pt}%
\pgfpathmoveto{\pgfqpoint{5.180537in}{1.486456in}}%
\pgfpathlineto{\pgfqpoint{7.219880in}{1.486456in}}%
\pgfusepath{stroke}%
\end{pgfscope}%
\begin{pgfscope}%
\pgfsetbuttcap%
\pgfsetroundjoin%
\definecolor{currentfill}{rgb}{0.333333,0.333333,0.333333}%
\pgfsetfillcolor{currentfill}%
\pgfsetlinewidth{0.803000pt}%
\definecolor{currentstroke}{rgb}{0.333333,0.333333,0.333333}%
\pgfsetstrokecolor{currentstroke}%
\pgfsetdash{}{0pt}%
\pgfsys@defobject{currentmarker}{\pgfqpoint{-0.048611in}{0.000000in}}{\pgfqpoint{-0.000000in}{0.000000in}}{%
\pgfpathmoveto{\pgfqpoint{-0.000000in}{0.000000in}}%
\pgfpathlineto{\pgfqpoint{-0.048611in}{0.000000in}}%
\pgfusepath{stroke,fill}%
}%
\begin{pgfscope}%
\pgfsys@transformshift{5.180537in}{1.486456in}%
\pgfsys@useobject{currentmarker}{}%
\end{pgfscope}%
\end{pgfscope}%
\begin{pgfscope}%
\pgfpathrectangle{\pgfqpoint{5.180537in}{1.263068in}}{\pgfqpoint{2.039343in}{3.079750in}}%
\pgfusepath{clip}%
\pgfsetrectcap%
\pgfsetroundjoin%
\pgfsetlinewidth{0.803000pt}%
\definecolor{currentstroke}{rgb}{1.000000,1.000000,1.000000}%
\pgfsetstrokecolor{currentstroke}%
\pgfsetdash{}{0pt}%
\pgfpathmoveto{\pgfqpoint{5.180537in}{1.975300in}}%
\pgfpathlineto{\pgfqpoint{7.219880in}{1.975300in}}%
\pgfusepath{stroke}%
\end{pgfscope}%
\begin{pgfscope}%
\pgfsetbuttcap%
\pgfsetroundjoin%
\definecolor{currentfill}{rgb}{0.333333,0.333333,0.333333}%
\pgfsetfillcolor{currentfill}%
\pgfsetlinewidth{0.803000pt}%
\definecolor{currentstroke}{rgb}{0.333333,0.333333,0.333333}%
\pgfsetstrokecolor{currentstroke}%
\pgfsetdash{}{0pt}%
\pgfsys@defobject{currentmarker}{\pgfqpoint{-0.048611in}{0.000000in}}{\pgfqpoint{-0.000000in}{0.000000in}}{%
\pgfpathmoveto{\pgfqpoint{-0.000000in}{0.000000in}}%
\pgfpathlineto{\pgfqpoint{-0.048611in}{0.000000in}}%
\pgfusepath{stroke,fill}%
}%
\begin{pgfscope}%
\pgfsys@transformshift{5.180537in}{1.975300in}%
\pgfsys@useobject{currentmarker}{}%
\end{pgfscope}%
\end{pgfscope}%
\begin{pgfscope}%
\pgfpathrectangle{\pgfqpoint{5.180537in}{1.263068in}}{\pgfqpoint{2.039343in}{3.079750in}}%
\pgfusepath{clip}%
\pgfsetrectcap%
\pgfsetroundjoin%
\pgfsetlinewidth{0.803000pt}%
\definecolor{currentstroke}{rgb}{1.000000,1.000000,1.000000}%
\pgfsetstrokecolor{currentstroke}%
\pgfsetdash{}{0pt}%
\pgfpathmoveto{\pgfqpoint{5.180537in}{2.464144in}}%
\pgfpathlineto{\pgfqpoint{7.219880in}{2.464144in}}%
\pgfusepath{stroke}%
\end{pgfscope}%
\begin{pgfscope}%
\pgfsetbuttcap%
\pgfsetroundjoin%
\definecolor{currentfill}{rgb}{0.333333,0.333333,0.333333}%
\pgfsetfillcolor{currentfill}%
\pgfsetlinewidth{0.803000pt}%
\definecolor{currentstroke}{rgb}{0.333333,0.333333,0.333333}%
\pgfsetstrokecolor{currentstroke}%
\pgfsetdash{}{0pt}%
\pgfsys@defobject{currentmarker}{\pgfqpoint{-0.048611in}{0.000000in}}{\pgfqpoint{-0.000000in}{0.000000in}}{%
\pgfpathmoveto{\pgfqpoint{-0.000000in}{0.000000in}}%
\pgfpathlineto{\pgfqpoint{-0.048611in}{0.000000in}}%
\pgfusepath{stroke,fill}%
}%
\begin{pgfscope}%
\pgfsys@transformshift{5.180537in}{2.464144in}%
\pgfsys@useobject{currentmarker}{}%
\end{pgfscope}%
\end{pgfscope}%
\begin{pgfscope}%
\pgfpathrectangle{\pgfqpoint{5.180537in}{1.263068in}}{\pgfqpoint{2.039343in}{3.079750in}}%
\pgfusepath{clip}%
\pgfsetrectcap%
\pgfsetroundjoin%
\pgfsetlinewidth{0.803000pt}%
\definecolor{currentstroke}{rgb}{1.000000,1.000000,1.000000}%
\pgfsetstrokecolor{currentstroke}%
\pgfsetdash{}{0pt}%
\pgfpathmoveto{\pgfqpoint{5.180537in}{2.952987in}}%
\pgfpathlineto{\pgfqpoint{7.219880in}{2.952987in}}%
\pgfusepath{stroke}%
\end{pgfscope}%
\begin{pgfscope}%
\pgfsetbuttcap%
\pgfsetroundjoin%
\definecolor{currentfill}{rgb}{0.333333,0.333333,0.333333}%
\pgfsetfillcolor{currentfill}%
\pgfsetlinewidth{0.803000pt}%
\definecolor{currentstroke}{rgb}{0.333333,0.333333,0.333333}%
\pgfsetstrokecolor{currentstroke}%
\pgfsetdash{}{0pt}%
\pgfsys@defobject{currentmarker}{\pgfqpoint{-0.048611in}{0.000000in}}{\pgfqpoint{-0.000000in}{0.000000in}}{%
\pgfpathmoveto{\pgfqpoint{-0.000000in}{0.000000in}}%
\pgfpathlineto{\pgfqpoint{-0.048611in}{0.000000in}}%
\pgfusepath{stroke,fill}%
}%
\begin{pgfscope}%
\pgfsys@transformshift{5.180537in}{2.952987in}%
\pgfsys@useobject{currentmarker}{}%
\end{pgfscope}%
\end{pgfscope}%
\begin{pgfscope}%
\pgfpathrectangle{\pgfqpoint{5.180537in}{1.263068in}}{\pgfqpoint{2.039343in}{3.079750in}}%
\pgfusepath{clip}%
\pgfsetrectcap%
\pgfsetroundjoin%
\pgfsetlinewidth{0.803000pt}%
\definecolor{currentstroke}{rgb}{1.000000,1.000000,1.000000}%
\pgfsetstrokecolor{currentstroke}%
\pgfsetdash{}{0pt}%
\pgfpathmoveto{\pgfqpoint{5.180537in}{3.441831in}}%
\pgfpathlineto{\pgfqpoint{7.219880in}{3.441831in}}%
\pgfusepath{stroke}%
\end{pgfscope}%
\begin{pgfscope}%
\pgfsetbuttcap%
\pgfsetroundjoin%
\definecolor{currentfill}{rgb}{0.333333,0.333333,0.333333}%
\pgfsetfillcolor{currentfill}%
\pgfsetlinewidth{0.803000pt}%
\definecolor{currentstroke}{rgb}{0.333333,0.333333,0.333333}%
\pgfsetstrokecolor{currentstroke}%
\pgfsetdash{}{0pt}%
\pgfsys@defobject{currentmarker}{\pgfqpoint{-0.048611in}{0.000000in}}{\pgfqpoint{-0.000000in}{0.000000in}}{%
\pgfpathmoveto{\pgfqpoint{-0.000000in}{0.000000in}}%
\pgfpathlineto{\pgfqpoint{-0.048611in}{0.000000in}}%
\pgfusepath{stroke,fill}%
}%
\begin{pgfscope}%
\pgfsys@transformshift{5.180537in}{3.441831in}%
\pgfsys@useobject{currentmarker}{}%
\end{pgfscope}%
\end{pgfscope}%
\begin{pgfscope}%
\pgfpathrectangle{\pgfqpoint{5.180537in}{1.263068in}}{\pgfqpoint{2.039343in}{3.079750in}}%
\pgfusepath{clip}%
\pgfsetrectcap%
\pgfsetroundjoin%
\pgfsetlinewidth{0.803000pt}%
\definecolor{currentstroke}{rgb}{1.000000,1.000000,1.000000}%
\pgfsetstrokecolor{currentstroke}%
\pgfsetdash{}{0pt}%
\pgfpathmoveto{\pgfqpoint{5.180537in}{3.930675in}}%
\pgfpathlineto{\pgfqpoint{7.219880in}{3.930675in}}%
\pgfusepath{stroke}%
\end{pgfscope}%
\begin{pgfscope}%
\pgfsetbuttcap%
\pgfsetroundjoin%
\definecolor{currentfill}{rgb}{0.333333,0.333333,0.333333}%
\pgfsetfillcolor{currentfill}%
\pgfsetlinewidth{0.803000pt}%
\definecolor{currentstroke}{rgb}{0.333333,0.333333,0.333333}%
\pgfsetstrokecolor{currentstroke}%
\pgfsetdash{}{0pt}%
\pgfsys@defobject{currentmarker}{\pgfqpoint{-0.048611in}{0.000000in}}{\pgfqpoint{-0.000000in}{0.000000in}}{%
\pgfpathmoveto{\pgfqpoint{-0.000000in}{0.000000in}}%
\pgfpathlineto{\pgfqpoint{-0.048611in}{0.000000in}}%
\pgfusepath{stroke,fill}%
}%
\begin{pgfscope}%
\pgfsys@transformshift{5.180537in}{3.930675in}%
\pgfsys@useobject{currentmarker}{}%
\end{pgfscope}%
\end{pgfscope}%
\begin{pgfscope}%
\pgfpathrectangle{\pgfqpoint{5.180537in}{1.263068in}}{\pgfqpoint{2.039343in}{3.079750in}}%
\pgfusepath{clip}%
\pgfsetbuttcap%
\pgfsetroundjoin%
\pgfsetlinewidth{1.505625pt}%
\definecolor{currentstroke}{rgb}{0.839216,0.152941,0.156863}%
\pgfsetstrokecolor{currentstroke}%
\pgfsetdash{{5.550000pt}{2.400000pt}}{0.000000pt}%
\pgfpathmoveto{\pgfqpoint{5.273234in}{1.403056in}}%
\pgfpathlineto{\pgfqpoint{5.736721in}{1.903522in}}%
\pgfpathlineto{\pgfqpoint{6.200208in}{1.903809in}}%
\pgfpathlineto{\pgfqpoint{6.663696in}{1.904295in}}%
\pgfpathlineto{\pgfqpoint{7.127183in}{1.905208in}}%
\pgfusepath{stroke}%
\end{pgfscope}%
\begin{pgfscope}%
\pgfpathrectangle{\pgfqpoint{5.180537in}{1.263068in}}{\pgfqpoint{2.039343in}{3.079750in}}%
\pgfusepath{clip}%
\pgfsetbuttcap%
\pgfsetroundjoin%
\definecolor{currentfill}{rgb}{0.839216,0.152941,0.156863}%
\pgfsetfillcolor{currentfill}%
\pgfsetlinewidth{1.003750pt}%
\definecolor{currentstroke}{rgb}{0.839216,0.152941,0.156863}%
\pgfsetstrokecolor{currentstroke}%
\pgfsetdash{}{0pt}%
\pgfsys@defobject{currentmarker}{\pgfqpoint{-0.041667in}{-0.041667in}}{\pgfqpoint{0.041667in}{0.041667in}}{%
\pgfpathmoveto{\pgfqpoint{0.000000in}{-0.041667in}}%
\pgfpathcurveto{\pgfqpoint{0.011050in}{-0.041667in}}{\pgfqpoint{0.021649in}{-0.037276in}}{\pgfqpoint{0.029463in}{-0.029463in}}%
\pgfpathcurveto{\pgfqpoint{0.037276in}{-0.021649in}}{\pgfqpoint{0.041667in}{-0.011050in}}{\pgfqpoint{0.041667in}{0.000000in}}%
\pgfpathcurveto{\pgfqpoint{0.041667in}{0.011050in}}{\pgfqpoint{0.037276in}{0.021649in}}{\pgfqpoint{0.029463in}{0.029463in}}%
\pgfpathcurveto{\pgfqpoint{0.021649in}{0.037276in}}{\pgfqpoint{0.011050in}{0.041667in}}{\pgfqpoint{0.000000in}{0.041667in}}%
\pgfpathcurveto{\pgfqpoint{-0.011050in}{0.041667in}}{\pgfqpoint{-0.021649in}{0.037276in}}{\pgfqpoint{-0.029463in}{0.029463in}}%
\pgfpathcurveto{\pgfqpoint{-0.037276in}{0.021649in}}{\pgfqpoint{-0.041667in}{0.011050in}}{\pgfqpoint{-0.041667in}{0.000000in}}%
\pgfpathcurveto{\pgfqpoint{-0.041667in}{-0.011050in}}{\pgfqpoint{-0.037276in}{-0.021649in}}{\pgfqpoint{-0.029463in}{-0.029463in}}%
\pgfpathcurveto{\pgfqpoint{-0.021649in}{-0.037276in}}{\pgfqpoint{-0.011050in}{-0.041667in}}{\pgfqpoint{0.000000in}{-0.041667in}}%
\pgfpathclose%
\pgfusepath{stroke,fill}%
}%
\begin{pgfscope}%
\pgfsys@transformshift{5.273234in}{1.403056in}%
\pgfsys@useobject{currentmarker}{}%
\end{pgfscope}%
\begin{pgfscope}%
\pgfsys@transformshift{5.736721in}{1.903522in}%
\pgfsys@useobject{currentmarker}{}%
\end{pgfscope}%
\begin{pgfscope}%
\pgfsys@transformshift{6.200208in}{1.903809in}%
\pgfsys@useobject{currentmarker}{}%
\end{pgfscope}%
\begin{pgfscope}%
\pgfsys@transformshift{6.663696in}{1.904295in}%
\pgfsys@useobject{currentmarker}{}%
\end{pgfscope}%
\begin{pgfscope}%
\pgfsys@transformshift{7.127183in}{1.905208in}%
\pgfsys@useobject{currentmarker}{}%
\end{pgfscope}%
\end{pgfscope}%
\begin{pgfscope}%
\pgfpathrectangle{\pgfqpoint{5.180537in}{1.263068in}}{\pgfqpoint{2.039343in}{3.079750in}}%
\pgfusepath{clip}%
\pgfsetbuttcap%
\pgfsetroundjoin%
\pgfsetlinewidth{1.505625pt}%
\definecolor{currentstroke}{rgb}{0.172549,0.627451,0.172549}%
\pgfsetstrokecolor{currentstroke}%
\pgfsetdash{{9.600000pt}{2.400000pt}{1.500000pt}{2.400000pt}}{0.000000pt}%
\pgfpathmoveto{\pgfqpoint{5.273234in}{4.202829in}}%
\pgfpathlineto{\pgfqpoint{5.736721in}{4.202829in}}%
\pgfpathlineto{\pgfqpoint{6.200208in}{4.202829in}}%
\pgfpathlineto{\pgfqpoint{6.663696in}{4.202829in}}%
\pgfpathlineto{\pgfqpoint{7.127183in}{4.202829in}}%
\pgfusepath{stroke}%
\end{pgfscope}%
\begin{pgfscope}%
\pgfpathrectangle{\pgfqpoint{5.180537in}{1.263068in}}{\pgfqpoint{2.039343in}{3.079750in}}%
\pgfusepath{clip}%
\pgfsetbuttcap%
\pgfsetmiterjoin%
\definecolor{currentfill}{rgb}{0.172549,0.627451,0.172549}%
\pgfsetfillcolor{currentfill}%
\pgfsetlinewidth{1.003750pt}%
\definecolor{currentstroke}{rgb}{0.172549,0.627451,0.172549}%
\pgfsetstrokecolor{currentstroke}%
\pgfsetdash{}{0pt}%
\pgfsys@defobject{currentmarker}{\pgfqpoint{-0.041667in}{-0.041667in}}{\pgfqpoint{0.041667in}{0.041667in}}{%
\pgfpathmoveto{\pgfqpoint{0.000000in}{0.041667in}}%
\pgfpathlineto{\pgfqpoint{-0.041667in}{-0.041667in}}%
\pgfpathlineto{\pgfqpoint{0.041667in}{-0.041667in}}%
\pgfpathclose%
\pgfusepath{stroke,fill}%
}%
\begin{pgfscope}%
\pgfsys@transformshift{5.273234in}{4.202829in}%
\pgfsys@useobject{currentmarker}{}%
\end{pgfscope}%
\begin{pgfscope}%
\pgfsys@transformshift{5.736721in}{4.202829in}%
\pgfsys@useobject{currentmarker}{}%
\end{pgfscope}%
\begin{pgfscope}%
\pgfsys@transformshift{6.200208in}{4.202829in}%
\pgfsys@useobject{currentmarker}{}%
\end{pgfscope}%
\begin{pgfscope}%
\pgfsys@transformshift{6.663696in}{4.202829in}%
\pgfsys@useobject{currentmarker}{}%
\end{pgfscope}%
\begin{pgfscope}%
\pgfsys@transformshift{7.127183in}{4.202829in}%
\pgfsys@useobject{currentmarker}{}%
\end{pgfscope}%
\end{pgfscope}%
\begin{pgfscope}%
\pgfsetrectcap%
\pgfsetmiterjoin%
\pgfsetlinewidth{1.003750pt}%
\definecolor{currentstroke}{rgb}{1.000000,1.000000,1.000000}%
\pgfsetstrokecolor{currentstroke}%
\pgfsetdash{}{0pt}%
\pgfpathmoveto{\pgfqpoint{5.180537in}{1.263067in}}%
\pgfpathlineto{\pgfqpoint{5.180537in}{4.342817in}}%
\pgfusepath{stroke}%
\end{pgfscope}%
\begin{pgfscope}%
\pgfsetrectcap%
\pgfsetmiterjoin%
\pgfsetlinewidth{1.003750pt}%
\definecolor{currentstroke}{rgb}{1.000000,1.000000,1.000000}%
\pgfsetstrokecolor{currentstroke}%
\pgfsetdash{}{0pt}%
\pgfpathmoveto{\pgfqpoint{7.219880in}{1.263067in}}%
\pgfpathlineto{\pgfqpoint{7.219880in}{4.342817in}}%
\pgfusepath{stroke}%
\end{pgfscope}%
\begin{pgfscope}%
\pgfsetrectcap%
\pgfsetmiterjoin%
\pgfsetlinewidth{1.003750pt}%
\definecolor{currentstroke}{rgb}{1.000000,1.000000,1.000000}%
\pgfsetstrokecolor{currentstroke}%
\pgfsetdash{}{0pt}%
\pgfpathmoveto{\pgfqpoint{5.180537in}{1.263068in}}%
\pgfpathlineto{\pgfqpoint{7.219880in}{1.263068in}}%
\pgfusepath{stroke}%
\end{pgfscope}%
\begin{pgfscope}%
\pgfsetrectcap%
\pgfsetmiterjoin%
\pgfsetlinewidth{1.003750pt}%
\definecolor{currentstroke}{rgb}{1.000000,1.000000,1.000000}%
\pgfsetstrokecolor{currentstroke}%
\pgfsetdash{}{0pt}%
\pgfpathmoveto{\pgfqpoint{5.180537in}{4.342817in}}%
\pgfpathlineto{\pgfqpoint{7.219880in}{4.342817in}}%
\pgfusepath{stroke}%
\end{pgfscope}%
\begin{pgfscope}%
\pgfsetbuttcap%
\pgfsetmiterjoin%
\definecolor{currentfill}{rgb}{0.898039,0.898039,0.898039}%
\pgfsetfillcolor{currentfill}%
\pgfsetlinewidth{0.000000pt}%
\definecolor{currentstroke}{rgb}{0.000000,0.000000,0.000000}%
\pgfsetstrokecolor{currentstroke}%
\pgfsetstrokeopacity{0.000000}%
\pgfsetdash{}{0pt}%
\pgfpathmoveto{\pgfqpoint{7.439827in}{1.263068in}}%
\pgfpathlineto{\pgfqpoint{9.479170in}{1.263068in}}%
\pgfpathlineto{\pgfqpoint{9.479170in}{4.342817in}}%
\pgfpathlineto{\pgfqpoint{7.439827in}{4.342817in}}%
\pgfpathclose%
\pgfusepath{fill}%
\end{pgfscope}%
\begin{pgfscope}%
\pgfpathrectangle{\pgfqpoint{7.439827in}{1.263068in}}{\pgfqpoint{2.039343in}{3.079750in}}%
\pgfusepath{clip}%
\pgfsetrectcap%
\pgfsetroundjoin%
\pgfsetlinewidth{0.803000pt}%
\definecolor{currentstroke}{rgb}{1.000000,1.000000,1.000000}%
\pgfsetstrokecolor{currentstroke}%
\pgfsetstrokeopacity{0.000000}%
\pgfsetdash{}{0pt}%
\pgfpathmoveto{\pgfqpoint{7.532524in}{1.263068in}}%
\pgfpathlineto{\pgfqpoint{7.532524in}{4.342817in}}%
\pgfusepath{stroke}%
\end{pgfscope}%
\begin{pgfscope}%
\pgfsetbuttcap%
\pgfsetroundjoin%
\definecolor{currentfill}{rgb}{0.333333,0.333333,0.333333}%
\pgfsetfillcolor{currentfill}%
\pgfsetlinewidth{0.803000pt}%
\definecolor{currentstroke}{rgb}{0.333333,0.333333,0.333333}%
\pgfsetstrokecolor{currentstroke}%
\pgfsetdash{}{0pt}%
\pgfsys@defobject{currentmarker}{\pgfqpoint{0.000000in}{-0.048611in}}{\pgfqpoint{0.000000in}{0.000000in}}{%
\pgfpathmoveto{\pgfqpoint{0.000000in}{0.000000in}}%
\pgfpathlineto{\pgfqpoint{0.000000in}{-0.048611in}}%
\pgfusepath{stroke,fill}%
}%
\begin{pgfscope}%
\pgfsys@transformshift{7.532524in}{1.263068in}%
\pgfsys@useobject{currentmarker}{}%
\end{pgfscope}%
\end{pgfscope}%
\begin{pgfscope}%
\definecolor{textcolor}{rgb}{0.333333,0.333333,0.333333}%
\pgfsetstrokecolor{textcolor}%
\pgfsetfillcolor{textcolor}%
\pgftext[x=7.599075in, y=0.100000in, left, base,rotate=90.000000]{\color{textcolor}\rmfamily\fontsize{16.000000}{19.200000}\selectfont mga-0-20\%}%
\end{pgfscope}%
\begin{pgfscope}%
\pgfpathrectangle{\pgfqpoint{7.439827in}{1.263068in}}{\pgfqpoint{2.039343in}{3.079750in}}%
\pgfusepath{clip}%
\pgfsetrectcap%
\pgfsetroundjoin%
\pgfsetlinewidth{0.803000pt}%
\definecolor{currentstroke}{rgb}{1.000000,1.000000,1.000000}%
\pgfsetstrokecolor{currentstroke}%
\pgfsetstrokeopacity{0.000000}%
\pgfsetdash{}{0pt}%
\pgfpathmoveto{\pgfqpoint{7.996011in}{1.263068in}}%
\pgfpathlineto{\pgfqpoint{7.996011in}{4.342817in}}%
\pgfusepath{stroke}%
\end{pgfscope}%
\begin{pgfscope}%
\pgfsetbuttcap%
\pgfsetroundjoin%
\definecolor{currentfill}{rgb}{0.333333,0.333333,0.333333}%
\pgfsetfillcolor{currentfill}%
\pgfsetlinewidth{0.803000pt}%
\definecolor{currentstroke}{rgb}{0.333333,0.333333,0.333333}%
\pgfsetstrokecolor{currentstroke}%
\pgfsetdash{}{0pt}%
\pgfsys@defobject{currentmarker}{\pgfqpoint{0.000000in}{-0.048611in}}{\pgfqpoint{0.000000in}{0.000000in}}{%
\pgfpathmoveto{\pgfqpoint{0.000000in}{0.000000in}}%
\pgfpathlineto{\pgfqpoint{0.000000in}{-0.048611in}}%
\pgfusepath{stroke,fill}%
}%
\begin{pgfscope}%
\pgfsys@transformshift{7.996011in}{1.263068in}%
\pgfsys@useobject{currentmarker}{}%
\end{pgfscope}%
\end{pgfscope}%
\begin{pgfscope}%
\definecolor{textcolor}{rgb}{0.333333,0.333333,0.333333}%
\pgfsetstrokecolor{textcolor}%
\pgfsetfillcolor{textcolor}%
\pgftext[x=8.062562in, y=0.100000in, left, base,rotate=90.000000]{\color{textcolor}\rmfamily\fontsize{16.000000}{19.200000}\selectfont mga-1-20\%}%
\end{pgfscope}%
\begin{pgfscope}%
\pgfpathrectangle{\pgfqpoint{7.439827in}{1.263068in}}{\pgfqpoint{2.039343in}{3.079750in}}%
\pgfusepath{clip}%
\pgfsetrectcap%
\pgfsetroundjoin%
\pgfsetlinewidth{0.803000pt}%
\definecolor{currentstroke}{rgb}{1.000000,1.000000,1.000000}%
\pgfsetstrokecolor{currentstroke}%
\pgfsetstrokeopacity{0.000000}%
\pgfsetdash{}{0pt}%
\pgfpathmoveto{\pgfqpoint{8.459498in}{1.263068in}}%
\pgfpathlineto{\pgfqpoint{8.459498in}{4.342817in}}%
\pgfusepath{stroke}%
\end{pgfscope}%
\begin{pgfscope}%
\pgfsetbuttcap%
\pgfsetroundjoin%
\definecolor{currentfill}{rgb}{0.333333,0.333333,0.333333}%
\pgfsetfillcolor{currentfill}%
\pgfsetlinewidth{0.803000pt}%
\definecolor{currentstroke}{rgb}{0.333333,0.333333,0.333333}%
\pgfsetstrokecolor{currentstroke}%
\pgfsetdash{}{0pt}%
\pgfsys@defobject{currentmarker}{\pgfqpoint{0.000000in}{-0.048611in}}{\pgfqpoint{0.000000in}{0.000000in}}{%
\pgfpathmoveto{\pgfqpoint{0.000000in}{0.000000in}}%
\pgfpathlineto{\pgfqpoint{0.000000in}{-0.048611in}}%
\pgfusepath{stroke,fill}%
}%
\begin{pgfscope}%
\pgfsys@transformshift{8.459498in}{1.263068in}%
\pgfsys@useobject{currentmarker}{}%
\end{pgfscope}%
\end{pgfscope}%
\begin{pgfscope}%
\definecolor{textcolor}{rgb}{0.333333,0.333333,0.333333}%
\pgfsetstrokecolor{textcolor}%
\pgfsetfillcolor{textcolor}%
\pgftext[x=8.526049in, y=0.100000in, left, base,rotate=90.000000]{\color{textcolor}\rmfamily\fontsize{16.000000}{19.200000}\selectfont mga-2-20\%}%
\end{pgfscope}%
\begin{pgfscope}%
\pgfpathrectangle{\pgfqpoint{7.439827in}{1.263068in}}{\pgfqpoint{2.039343in}{3.079750in}}%
\pgfusepath{clip}%
\pgfsetrectcap%
\pgfsetroundjoin%
\pgfsetlinewidth{0.803000pt}%
\definecolor{currentstroke}{rgb}{1.000000,1.000000,1.000000}%
\pgfsetstrokecolor{currentstroke}%
\pgfsetstrokeopacity{0.000000}%
\pgfsetdash{}{0pt}%
\pgfpathmoveto{\pgfqpoint{8.922985in}{1.263068in}}%
\pgfpathlineto{\pgfqpoint{8.922985in}{4.342817in}}%
\pgfusepath{stroke}%
\end{pgfscope}%
\begin{pgfscope}%
\pgfsetbuttcap%
\pgfsetroundjoin%
\definecolor{currentfill}{rgb}{0.333333,0.333333,0.333333}%
\pgfsetfillcolor{currentfill}%
\pgfsetlinewidth{0.803000pt}%
\definecolor{currentstroke}{rgb}{0.333333,0.333333,0.333333}%
\pgfsetstrokecolor{currentstroke}%
\pgfsetdash{}{0pt}%
\pgfsys@defobject{currentmarker}{\pgfqpoint{0.000000in}{-0.048611in}}{\pgfqpoint{0.000000in}{0.000000in}}{%
\pgfpathmoveto{\pgfqpoint{0.000000in}{0.000000in}}%
\pgfpathlineto{\pgfqpoint{0.000000in}{-0.048611in}}%
\pgfusepath{stroke,fill}%
}%
\begin{pgfscope}%
\pgfsys@transformshift{8.922985in}{1.263068in}%
\pgfsys@useobject{currentmarker}{}%
\end{pgfscope}%
\end{pgfscope}%
\begin{pgfscope}%
\definecolor{textcolor}{rgb}{0.333333,0.333333,0.333333}%
\pgfsetstrokecolor{textcolor}%
\pgfsetfillcolor{textcolor}%
\pgftext[x=8.989536in, y=0.100000in, left, base,rotate=90.000000]{\color{textcolor}\rmfamily\fontsize{16.000000}{19.200000}\selectfont mga-3-20\%}%
\end{pgfscope}%
\begin{pgfscope}%
\pgfpathrectangle{\pgfqpoint{7.439827in}{1.263068in}}{\pgfqpoint{2.039343in}{3.079750in}}%
\pgfusepath{clip}%
\pgfsetrectcap%
\pgfsetroundjoin%
\pgfsetlinewidth{0.803000pt}%
\definecolor{currentstroke}{rgb}{1.000000,1.000000,1.000000}%
\pgfsetstrokecolor{currentstroke}%
\pgfsetstrokeopacity{0.000000}%
\pgfsetdash{}{0pt}%
\pgfpathmoveto{\pgfqpoint{9.386472in}{1.263068in}}%
\pgfpathlineto{\pgfqpoint{9.386472in}{4.342817in}}%
\pgfusepath{stroke}%
\end{pgfscope}%
\begin{pgfscope}%
\pgfsetbuttcap%
\pgfsetroundjoin%
\definecolor{currentfill}{rgb}{0.333333,0.333333,0.333333}%
\pgfsetfillcolor{currentfill}%
\pgfsetlinewidth{0.803000pt}%
\definecolor{currentstroke}{rgb}{0.333333,0.333333,0.333333}%
\pgfsetstrokecolor{currentstroke}%
\pgfsetdash{}{0pt}%
\pgfsys@defobject{currentmarker}{\pgfqpoint{0.000000in}{-0.048611in}}{\pgfqpoint{0.000000in}{0.000000in}}{%
\pgfpathmoveto{\pgfqpoint{0.000000in}{0.000000in}}%
\pgfpathlineto{\pgfqpoint{0.000000in}{-0.048611in}}%
\pgfusepath{stroke,fill}%
}%
\begin{pgfscope}%
\pgfsys@transformshift{9.386472in}{1.263068in}%
\pgfsys@useobject{currentmarker}{}%
\end{pgfscope}%
\end{pgfscope}%
\begin{pgfscope}%
\definecolor{textcolor}{rgb}{0.333333,0.333333,0.333333}%
\pgfsetstrokecolor{textcolor}%
\pgfsetfillcolor{textcolor}%
\pgftext[x=9.453023in, y=0.100000in, left, base,rotate=90.000000]{\color{textcolor}\rmfamily\fontsize{16.000000}{19.200000}\selectfont mga-4-20\%}%
\end{pgfscope}%
\begin{pgfscope}%
\pgfpathrectangle{\pgfqpoint{7.439827in}{1.263068in}}{\pgfqpoint{2.039343in}{3.079750in}}%
\pgfusepath{clip}%
\pgfsetrectcap%
\pgfsetroundjoin%
\pgfsetlinewidth{0.803000pt}%
\definecolor{currentstroke}{rgb}{1.000000,1.000000,1.000000}%
\pgfsetstrokecolor{currentstroke}%
\pgfsetdash{}{0pt}%
\pgfpathmoveto{\pgfqpoint{7.439827in}{1.486456in}}%
\pgfpathlineto{\pgfqpoint{9.479170in}{1.486456in}}%
\pgfusepath{stroke}%
\end{pgfscope}%
\begin{pgfscope}%
\pgfsetbuttcap%
\pgfsetroundjoin%
\definecolor{currentfill}{rgb}{0.333333,0.333333,0.333333}%
\pgfsetfillcolor{currentfill}%
\pgfsetlinewidth{0.803000pt}%
\definecolor{currentstroke}{rgb}{0.333333,0.333333,0.333333}%
\pgfsetstrokecolor{currentstroke}%
\pgfsetdash{}{0pt}%
\pgfsys@defobject{currentmarker}{\pgfqpoint{-0.048611in}{0.000000in}}{\pgfqpoint{-0.000000in}{0.000000in}}{%
\pgfpathmoveto{\pgfqpoint{-0.000000in}{0.000000in}}%
\pgfpathlineto{\pgfqpoint{-0.048611in}{0.000000in}}%
\pgfusepath{stroke,fill}%
}%
\begin{pgfscope}%
\pgfsys@transformshift{7.439827in}{1.486456in}%
\pgfsys@useobject{currentmarker}{}%
\end{pgfscope}%
\end{pgfscope}%
\begin{pgfscope}%
\pgfpathrectangle{\pgfqpoint{7.439827in}{1.263068in}}{\pgfqpoint{2.039343in}{3.079750in}}%
\pgfusepath{clip}%
\pgfsetrectcap%
\pgfsetroundjoin%
\pgfsetlinewidth{0.803000pt}%
\definecolor{currentstroke}{rgb}{1.000000,1.000000,1.000000}%
\pgfsetstrokecolor{currentstroke}%
\pgfsetdash{}{0pt}%
\pgfpathmoveto{\pgfqpoint{7.439827in}{1.975300in}}%
\pgfpathlineto{\pgfqpoint{9.479170in}{1.975300in}}%
\pgfusepath{stroke}%
\end{pgfscope}%
\begin{pgfscope}%
\pgfsetbuttcap%
\pgfsetroundjoin%
\definecolor{currentfill}{rgb}{0.333333,0.333333,0.333333}%
\pgfsetfillcolor{currentfill}%
\pgfsetlinewidth{0.803000pt}%
\definecolor{currentstroke}{rgb}{0.333333,0.333333,0.333333}%
\pgfsetstrokecolor{currentstroke}%
\pgfsetdash{}{0pt}%
\pgfsys@defobject{currentmarker}{\pgfqpoint{-0.048611in}{0.000000in}}{\pgfqpoint{-0.000000in}{0.000000in}}{%
\pgfpathmoveto{\pgfqpoint{-0.000000in}{0.000000in}}%
\pgfpathlineto{\pgfqpoint{-0.048611in}{0.000000in}}%
\pgfusepath{stroke,fill}%
}%
\begin{pgfscope}%
\pgfsys@transformshift{7.439827in}{1.975300in}%
\pgfsys@useobject{currentmarker}{}%
\end{pgfscope}%
\end{pgfscope}%
\begin{pgfscope}%
\pgfpathrectangle{\pgfqpoint{7.439827in}{1.263068in}}{\pgfqpoint{2.039343in}{3.079750in}}%
\pgfusepath{clip}%
\pgfsetrectcap%
\pgfsetroundjoin%
\pgfsetlinewidth{0.803000pt}%
\definecolor{currentstroke}{rgb}{1.000000,1.000000,1.000000}%
\pgfsetstrokecolor{currentstroke}%
\pgfsetdash{}{0pt}%
\pgfpathmoveto{\pgfqpoint{7.439827in}{2.464144in}}%
\pgfpathlineto{\pgfqpoint{9.479170in}{2.464144in}}%
\pgfusepath{stroke}%
\end{pgfscope}%
\begin{pgfscope}%
\pgfsetbuttcap%
\pgfsetroundjoin%
\definecolor{currentfill}{rgb}{0.333333,0.333333,0.333333}%
\pgfsetfillcolor{currentfill}%
\pgfsetlinewidth{0.803000pt}%
\definecolor{currentstroke}{rgb}{0.333333,0.333333,0.333333}%
\pgfsetstrokecolor{currentstroke}%
\pgfsetdash{}{0pt}%
\pgfsys@defobject{currentmarker}{\pgfqpoint{-0.048611in}{0.000000in}}{\pgfqpoint{-0.000000in}{0.000000in}}{%
\pgfpathmoveto{\pgfqpoint{-0.000000in}{0.000000in}}%
\pgfpathlineto{\pgfqpoint{-0.048611in}{0.000000in}}%
\pgfusepath{stroke,fill}%
}%
\begin{pgfscope}%
\pgfsys@transformshift{7.439827in}{2.464144in}%
\pgfsys@useobject{currentmarker}{}%
\end{pgfscope}%
\end{pgfscope}%
\begin{pgfscope}%
\pgfpathrectangle{\pgfqpoint{7.439827in}{1.263068in}}{\pgfqpoint{2.039343in}{3.079750in}}%
\pgfusepath{clip}%
\pgfsetrectcap%
\pgfsetroundjoin%
\pgfsetlinewidth{0.803000pt}%
\definecolor{currentstroke}{rgb}{1.000000,1.000000,1.000000}%
\pgfsetstrokecolor{currentstroke}%
\pgfsetdash{}{0pt}%
\pgfpathmoveto{\pgfqpoint{7.439827in}{2.952987in}}%
\pgfpathlineto{\pgfqpoint{9.479170in}{2.952987in}}%
\pgfusepath{stroke}%
\end{pgfscope}%
\begin{pgfscope}%
\pgfsetbuttcap%
\pgfsetroundjoin%
\definecolor{currentfill}{rgb}{0.333333,0.333333,0.333333}%
\pgfsetfillcolor{currentfill}%
\pgfsetlinewidth{0.803000pt}%
\definecolor{currentstroke}{rgb}{0.333333,0.333333,0.333333}%
\pgfsetstrokecolor{currentstroke}%
\pgfsetdash{}{0pt}%
\pgfsys@defobject{currentmarker}{\pgfqpoint{-0.048611in}{0.000000in}}{\pgfqpoint{-0.000000in}{0.000000in}}{%
\pgfpathmoveto{\pgfqpoint{-0.000000in}{0.000000in}}%
\pgfpathlineto{\pgfqpoint{-0.048611in}{0.000000in}}%
\pgfusepath{stroke,fill}%
}%
\begin{pgfscope}%
\pgfsys@transformshift{7.439827in}{2.952987in}%
\pgfsys@useobject{currentmarker}{}%
\end{pgfscope}%
\end{pgfscope}%
\begin{pgfscope}%
\pgfpathrectangle{\pgfqpoint{7.439827in}{1.263068in}}{\pgfqpoint{2.039343in}{3.079750in}}%
\pgfusepath{clip}%
\pgfsetrectcap%
\pgfsetroundjoin%
\pgfsetlinewidth{0.803000pt}%
\definecolor{currentstroke}{rgb}{1.000000,1.000000,1.000000}%
\pgfsetstrokecolor{currentstroke}%
\pgfsetdash{}{0pt}%
\pgfpathmoveto{\pgfqpoint{7.439827in}{3.441831in}}%
\pgfpathlineto{\pgfqpoint{9.479170in}{3.441831in}}%
\pgfusepath{stroke}%
\end{pgfscope}%
\begin{pgfscope}%
\pgfsetbuttcap%
\pgfsetroundjoin%
\definecolor{currentfill}{rgb}{0.333333,0.333333,0.333333}%
\pgfsetfillcolor{currentfill}%
\pgfsetlinewidth{0.803000pt}%
\definecolor{currentstroke}{rgb}{0.333333,0.333333,0.333333}%
\pgfsetstrokecolor{currentstroke}%
\pgfsetdash{}{0pt}%
\pgfsys@defobject{currentmarker}{\pgfqpoint{-0.048611in}{0.000000in}}{\pgfqpoint{-0.000000in}{0.000000in}}{%
\pgfpathmoveto{\pgfqpoint{-0.000000in}{0.000000in}}%
\pgfpathlineto{\pgfqpoint{-0.048611in}{0.000000in}}%
\pgfusepath{stroke,fill}%
}%
\begin{pgfscope}%
\pgfsys@transformshift{7.439827in}{3.441831in}%
\pgfsys@useobject{currentmarker}{}%
\end{pgfscope}%
\end{pgfscope}%
\begin{pgfscope}%
\pgfpathrectangle{\pgfqpoint{7.439827in}{1.263068in}}{\pgfqpoint{2.039343in}{3.079750in}}%
\pgfusepath{clip}%
\pgfsetrectcap%
\pgfsetroundjoin%
\pgfsetlinewidth{0.803000pt}%
\definecolor{currentstroke}{rgb}{1.000000,1.000000,1.000000}%
\pgfsetstrokecolor{currentstroke}%
\pgfsetdash{}{0pt}%
\pgfpathmoveto{\pgfqpoint{7.439827in}{3.930675in}}%
\pgfpathlineto{\pgfqpoint{9.479170in}{3.930675in}}%
\pgfusepath{stroke}%
\end{pgfscope}%
\begin{pgfscope}%
\pgfsetbuttcap%
\pgfsetroundjoin%
\definecolor{currentfill}{rgb}{0.333333,0.333333,0.333333}%
\pgfsetfillcolor{currentfill}%
\pgfsetlinewidth{0.803000pt}%
\definecolor{currentstroke}{rgb}{0.333333,0.333333,0.333333}%
\pgfsetstrokecolor{currentstroke}%
\pgfsetdash{}{0pt}%
\pgfsys@defobject{currentmarker}{\pgfqpoint{-0.048611in}{0.000000in}}{\pgfqpoint{-0.000000in}{0.000000in}}{%
\pgfpathmoveto{\pgfqpoint{-0.000000in}{0.000000in}}%
\pgfpathlineto{\pgfqpoint{-0.048611in}{0.000000in}}%
\pgfusepath{stroke,fill}%
}%
\begin{pgfscope}%
\pgfsys@transformshift{7.439827in}{3.930675in}%
\pgfsys@useobject{currentmarker}{}%
\end{pgfscope}%
\end{pgfscope}%
\begin{pgfscope}%
\pgfpathrectangle{\pgfqpoint{7.439827in}{1.263068in}}{\pgfqpoint{2.039343in}{3.079750in}}%
\pgfusepath{clip}%
\pgfsetbuttcap%
\pgfsetroundjoin%
\pgfsetlinewidth{1.505625pt}%
\definecolor{currentstroke}{rgb}{0.839216,0.152941,0.156863}%
\pgfsetstrokecolor{currentstroke}%
\pgfsetdash{{5.550000pt}{2.400000pt}}{0.000000pt}%
\pgfpathmoveto{\pgfqpoint{7.532524in}{1.403056in}}%
\pgfpathlineto{\pgfqpoint{7.996011in}{1.903846in}}%
\pgfpathlineto{\pgfqpoint{8.459498in}{1.903773in}}%
\pgfpathlineto{\pgfqpoint{8.922985in}{1.903843in}}%
\pgfpathlineto{\pgfqpoint{9.386472in}{1.903732in}}%
\pgfusepath{stroke}%
\end{pgfscope}%
\begin{pgfscope}%
\pgfpathrectangle{\pgfqpoint{7.439827in}{1.263068in}}{\pgfqpoint{2.039343in}{3.079750in}}%
\pgfusepath{clip}%
\pgfsetbuttcap%
\pgfsetroundjoin%
\definecolor{currentfill}{rgb}{0.839216,0.152941,0.156863}%
\pgfsetfillcolor{currentfill}%
\pgfsetlinewidth{1.003750pt}%
\definecolor{currentstroke}{rgb}{0.839216,0.152941,0.156863}%
\pgfsetstrokecolor{currentstroke}%
\pgfsetdash{}{0pt}%
\pgfsys@defobject{currentmarker}{\pgfqpoint{-0.041667in}{-0.041667in}}{\pgfqpoint{0.041667in}{0.041667in}}{%
\pgfpathmoveto{\pgfqpoint{0.000000in}{-0.041667in}}%
\pgfpathcurveto{\pgfqpoint{0.011050in}{-0.041667in}}{\pgfqpoint{0.021649in}{-0.037276in}}{\pgfqpoint{0.029463in}{-0.029463in}}%
\pgfpathcurveto{\pgfqpoint{0.037276in}{-0.021649in}}{\pgfqpoint{0.041667in}{-0.011050in}}{\pgfqpoint{0.041667in}{0.000000in}}%
\pgfpathcurveto{\pgfqpoint{0.041667in}{0.011050in}}{\pgfqpoint{0.037276in}{0.021649in}}{\pgfqpoint{0.029463in}{0.029463in}}%
\pgfpathcurveto{\pgfqpoint{0.021649in}{0.037276in}}{\pgfqpoint{0.011050in}{0.041667in}}{\pgfqpoint{0.000000in}{0.041667in}}%
\pgfpathcurveto{\pgfqpoint{-0.011050in}{0.041667in}}{\pgfqpoint{-0.021649in}{0.037276in}}{\pgfqpoint{-0.029463in}{0.029463in}}%
\pgfpathcurveto{\pgfqpoint{-0.037276in}{0.021649in}}{\pgfqpoint{-0.041667in}{0.011050in}}{\pgfqpoint{-0.041667in}{0.000000in}}%
\pgfpathcurveto{\pgfqpoint{-0.041667in}{-0.011050in}}{\pgfqpoint{-0.037276in}{-0.021649in}}{\pgfqpoint{-0.029463in}{-0.029463in}}%
\pgfpathcurveto{\pgfqpoint{-0.021649in}{-0.037276in}}{\pgfqpoint{-0.011050in}{-0.041667in}}{\pgfqpoint{0.000000in}{-0.041667in}}%
\pgfpathclose%
\pgfusepath{stroke,fill}%
}%
\begin{pgfscope}%
\pgfsys@transformshift{7.532524in}{1.403056in}%
\pgfsys@useobject{currentmarker}{}%
\end{pgfscope}%
\begin{pgfscope}%
\pgfsys@transformshift{7.996011in}{1.903846in}%
\pgfsys@useobject{currentmarker}{}%
\end{pgfscope}%
\begin{pgfscope}%
\pgfsys@transformshift{8.459498in}{1.903773in}%
\pgfsys@useobject{currentmarker}{}%
\end{pgfscope}%
\begin{pgfscope}%
\pgfsys@transformshift{8.922985in}{1.903843in}%
\pgfsys@useobject{currentmarker}{}%
\end{pgfscope}%
\begin{pgfscope}%
\pgfsys@transformshift{9.386472in}{1.903732in}%
\pgfsys@useobject{currentmarker}{}%
\end{pgfscope}%
\end{pgfscope}%
\begin{pgfscope}%
\pgfpathrectangle{\pgfqpoint{7.439827in}{1.263068in}}{\pgfqpoint{2.039343in}{3.079750in}}%
\pgfusepath{clip}%
\pgfsetbuttcap%
\pgfsetroundjoin%
\pgfsetlinewidth{1.505625pt}%
\definecolor{currentstroke}{rgb}{0.172549,0.627451,0.172549}%
\pgfsetstrokecolor{currentstroke}%
\pgfsetdash{{9.600000pt}{2.400000pt}{1.500000pt}{2.400000pt}}{0.000000pt}%
\pgfpathmoveto{\pgfqpoint{7.532524in}{4.202829in}}%
\pgfpathlineto{\pgfqpoint{7.996011in}{4.202829in}}%
\pgfpathlineto{\pgfqpoint{8.459498in}{4.202829in}}%
\pgfpathlineto{\pgfqpoint{8.922985in}{4.202829in}}%
\pgfpathlineto{\pgfqpoint{9.386472in}{4.202829in}}%
\pgfusepath{stroke}%
\end{pgfscope}%
\begin{pgfscope}%
\pgfpathrectangle{\pgfqpoint{7.439827in}{1.263068in}}{\pgfqpoint{2.039343in}{3.079750in}}%
\pgfusepath{clip}%
\pgfsetbuttcap%
\pgfsetmiterjoin%
\definecolor{currentfill}{rgb}{0.172549,0.627451,0.172549}%
\pgfsetfillcolor{currentfill}%
\pgfsetlinewidth{1.003750pt}%
\definecolor{currentstroke}{rgb}{0.172549,0.627451,0.172549}%
\pgfsetstrokecolor{currentstroke}%
\pgfsetdash{}{0pt}%
\pgfsys@defobject{currentmarker}{\pgfqpoint{-0.041667in}{-0.041667in}}{\pgfqpoint{0.041667in}{0.041667in}}{%
\pgfpathmoveto{\pgfqpoint{0.000000in}{0.041667in}}%
\pgfpathlineto{\pgfqpoint{-0.041667in}{-0.041667in}}%
\pgfpathlineto{\pgfqpoint{0.041667in}{-0.041667in}}%
\pgfpathclose%
\pgfusepath{stroke,fill}%
}%
\begin{pgfscope}%
\pgfsys@transformshift{7.532524in}{4.202829in}%
\pgfsys@useobject{currentmarker}{}%
\end{pgfscope}%
\begin{pgfscope}%
\pgfsys@transformshift{7.996011in}{4.202829in}%
\pgfsys@useobject{currentmarker}{}%
\end{pgfscope}%
\begin{pgfscope}%
\pgfsys@transformshift{8.459498in}{4.202829in}%
\pgfsys@useobject{currentmarker}{}%
\end{pgfscope}%
\begin{pgfscope}%
\pgfsys@transformshift{8.922985in}{4.202829in}%
\pgfsys@useobject{currentmarker}{}%
\end{pgfscope}%
\begin{pgfscope}%
\pgfsys@transformshift{9.386472in}{4.202829in}%
\pgfsys@useobject{currentmarker}{}%
\end{pgfscope}%
\end{pgfscope}%
\begin{pgfscope}%
\pgfsetrectcap%
\pgfsetmiterjoin%
\pgfsetlinewidth{1.003750pt}%
\definecolor{currentstroke}{rgb}{1.000000,1.000000,1.000000}%
\pgfsetstrokecolor{currentstroke}%
\pgfsetdash{}{0pt}%
\pgfpathmoveto{\pgfqpoint{7.439827in}{1.263067in}}%
\pgfpathlineto{\pgfqpoint{7.439827in}{4.342817in}}%
\pgfusepath{stroke}%
\end{pgfscope}%
\begin{pgfscope}%
\pgfsetrectcap%
\pgfsetmiterjoin%
\pgfsetlinewidth{1.003750pt}%
\definecolor{currentstroke}{rgb}{1.000000,1.000000,1.000000}%
\pgfsetstrokecolor{currentstroke}%
\pgfsetdash{}{0pt}%
\pgfpathmoveto{\pgfqpoint{9.479170in}{1.263067in}}%
\pgfpathlineto{\pgfqpoint{9.479170in}{4.342817in}}%
\pgfusepath{stroke}%
\end{pgfscope}%
\begin{pgfscope}%
\pgfsetrectcap%
\pgfsetmiterjoin%
\pgfsetlinewidth{1.003750pt}%
\definecolor{currentstroke}{rgb}{1.000000,1.000000,1.000000}%
\pgfsetstrokecolor{currentstroke}%
\pgfsetdash{}{0pt}%
\pgfpathmoveto{\pgfqpoint{7.439827in}{1.263068in}}%
\pgfpathlineto{\pgfqpoint{9.479170in}{1.263068in}}%
\pgfusepath{stroke}%
\end{pgfscope}%
\begin{pgfscope}%
\pgfsetrectcap%
\pgfsetmiterjoin%
\pgfsetlinewidth{1.003750pt}%
\definecolor{currentstroke}{rgb}{1.000000,1.000000,1.000000}%
\pgfsetstrokecolor{currentstroke}%
\pgfsetdash{}{0pt}%
\pgfpathmoveto{\pgfqpoint{7.439827in}{4.342817in}}%
\pgfpathlineto{\pgfqpoint{9.479170in}{4.342817in}}%
\pgfusepath{stroke}%
\end{pgfscope}%
\begin{pgfscope}%
\pgfsetbuttcap%
\pgfsetmiterjoin%
\definecolor{currentfill}{rgb}{0.269412,0.269412,0.269412}%
\pgfsetfillcolor{currentfill}%
\pgfsetfillopacity{0.500000}%
\pgfsetlinewidth{0.501875pt}%
\definecolor{currentstroke}{rgb}{0.269412,0.269412,0.269412}%
\pgfsetstrokecolor{currentstroke}%
\pgfsetstrokeopacity{0.500000}%
\pgfsetdash{}{0pt}%
\pgfpathmoveto{\pgfqpoint{9.647804in}{2.467190in}}%
\pgfpathlineto{\pgfqpoint{11.789769in}{2.467190in}}%
\pgfpathquadraticcurveto{\pgfqpoint{11.828658in}{2.467190in}}{\pgfqpoint{11.828658in}{2.506079in}}%
\pgfpathlineto{\pgfqpoint{11.828658in}{3.036633in}}%
\pgfpathquadraticcurveto{\pgfqpoint{11.828658in}{3.075522in}}{\pgfqpoint{11.789769in}{3.075522in}}%
\pgfpathlineto{\pgfqpoint{9.647804in}{3.075522in}}%
\pgfpathquadraticcurveto{\pgfqpoint{9.608915in}{3.075522in}}{\pgfqpoint{9.608915in}{3.036633in}}%
\pgfpathlineto{\pgfqpoint{9.608915in}{2.506079in}}%
\pgfpathquadraticcurveto{\pgfqpoint{9.608915in}{2.467190in}}{\pgfqpoint{9.647804in}{2.467190in}}%
\pgfpathclose%
\pgfusepath{stroke,fill}%
\end{pgfscope}%
\begin{pgfscope}%
\pgfsetbuttcap%
\pgfsetmiterjoin%
\definecolor{currentfill}{rgb}{0.898039,0.898039,0.898039}%
\pgfsetfillcolor{currentfill}%
\pgfsetlinewidth{0.501875pt}%
\definecolor{currentstroke}{rgb}{0.800000,0.800000,0.800000}%
\pgfsetstrokecolor{currentstroke}%
\pgfsetdash{}{0pt}%
\pgfpathmoveto{\pgfqpoint{9.620026in}{2.494967in}}%
\pgfpathlineto{\pgfqpoint{11.761991in}{2.494967in}}%
\pgfpathquadraticcurveto{\pgfqpoint{11.800880in}{2.494967in}}{\pgfqpoint{11.800880in}{2.533856in}}%
\pgfpathlineto{\pgfqpoint{11.800880in}{3.064411in}}%
\pgfpathquadraticcurveto{\pgfqpoint{11.800880in}{3.103300in}}{\pgfqpoint{11.761991in}{3.103300in}}%
\pgfpathlineto{\pgfqpoint{9.620026in}{3.103300in}}%
\pgfpathquadraticcurveto{\pgfqpoint{9.581137in}{3.103300in}}{\pgfqpoint{9.581137in}{3.064411in}}%
\pgfpathlineto{\pgfqpoint{9.581137in}{2.533856in}}%
\pgfpathquadraticcurveto{\pgfqpoint{9.581137in}{2.494967in}}{\pgfqpoint{9.620026in}{2.494967in}}%
\pgfpathclose%
\pgfusepath{stroke,fill}%
\end{pgfscope}%
\begin{pgfscope}%
\pgfsetbuttcap%
\pgfsetroundjoin%
\pgfsetlinewidth{1.505625pt}%
\definecolor{currentstroke}{rgb}{0.839216,0.152941,0.156863}%
\pgfsetstrokecolor{currentstroke}%
\pgfsetdash{{5.550000pt}{2.400000pt}}{0.000000pt}%
\pgfpathmoveto{\pgfqpoint{9.658915in}{2.954689in}}%
\pgfpathlineto{\pgfqpoint{10.047804in}{2.954689in}}%
\pgfusepath{stroke}%
\end{pgfscope}%
\begin{pgfscope}%
\pgfsetbuttcap%
\pgfsetroundjoin%
\definecolor{currentfill}{rgb}{0.839216,0.152941,0.156863}%
\pgfsetfillcolor{currentfill}%
\pgfsetlinewidth{1.003750pt}%
\definecolor{currentstroke}{rgb}{0.839216,0.152941,0.156863}%
\pgfsetstrokecolor{currentstroke}%
\pgfsetdash{}{0pt}%
\pgfsys@defobject{currentmarker}{\pgfqpoint{-0.041667in}{-0.041667in}}{\pgfqpoint{0.041667in}{0.041667in}}{%
\pgfpathmoveto{\pgfqpoint{0.000000in}{-0.041667in}}%
\pgfpathcurveto{\pgfqpoint{0.011050in}{-0.041667in}}{\pgfqpoint{0.021649in}{-0.037276in}}{\pgfqpoint{0.029463in}{-0.029463in}}%
\pgfpathcurveto{\pgfqpoint{0.037276in}{-0.021649in}}{\pgfqpoint{0.041667in}{-0.011050in}}{\pgfqpoint{0.041667in}{0.000000in}}%
\pgfpathcurveto{\pgfqpoint{0.041667in}{0.011050in}}{\pgfqpoint{0.037276in}{0.021649in}}{\pgfqpoint{0.029463in}{0.029463in}}%
\pgfpathcurveto{\pgfqpoint{0.021649in}{0.037276in}}{\pgfqpoint{0.011050in}{0.041667in}}{\pgfqpoint{0.000000in}{0.041667in}}%
\pgfpathcurveto{\pgfqpoint{-0.011050in}{0.041667in}}{\pgfqpoint{-0.021649in}{0.037276in}}{\pgfqpoint{-0.029463in}{0.029463in}}%
\pgfpathcurveto{\pgfqpoint{-0.037276in}{0.021649in}}{\pgfqpoint{-0.041667in}{0.011050in}}{\pgfqpoint{-0.041667in}{0.000000in}}%
\pgfpathcurveto{\pgfqpoint{-0.041667in}{-0.011050in}}{\pgfqpoint{-0.037276in}{-0.021649in}}{\pgfqpoint{-0.029463in}{-0.029463in}}%
\pgfpathcurveto{\pgfqpoint{-0.021649in}{-0.037276in}}{\pgfqpoint{-0.011050in}{-0.041667in}}{\pgfqpoint{0.000000in}{-0.041667in}}%
\pgfpathclose%
\pgfusepath{stroke,fill}%
}%
\begin{pgfscope}%
\pgfsys@transformshift{9.853359in}{2.954689in}%
\pgfsys@useobject{currentmarker}{}%
\end{pgfscope}%
\end{pgfscope}%
\begin{pgfscope}%
\definecolor{textcolor}{rgb}{0.000000,0.000000,0.000000}%
\pgfsetstrokecolor{textcolor}%
\pgfsetfillcolor{textcolor}%
\pgftext[x=10.203359in,y=2.886633in,left,base]{\color{textcolor}\rmfamily\fontsize{14.000000}{16.800000}\selectfont ABBOTT}%
\end{pgfscope}%
\begin{pgfscope}%
\pgfsetbuttcap%
\pgfsetroundjoin%
\pgfsetlinewidth{1.505625pt}%
\definecolor{currentstroke}{rgb}{0.172549,0.627451,0.172549}%
\pgfsetstrokecolor{currentstroke}%
\pgfsetdash{{9.600000pt}{2.400000pt}{1.500000pt}{2.400000pt}}{0.000000pt}%
\pgfpathmoveto{\pgfqpoint{9.658915in}{2.679689in}}%
\pgfpathlineto{\pgfqpoint{10.047804in}{2.679689in}}%
\pgfusepath{stroke}%
\end{pgfscope}%
\begin{pgfscope}%
\pgfsetbuttcap%
\pgfsetmiterjoin%
\definecolor{currentfill}{rgb}{0.172549,0.627451,0.172549}%
\pgfsetfillcolor{currentfill}%
\pgfsetlinewidth{1.003750pt}%
\definecolor{currentstroke}{rgb}{0.172549,0.627451,0.172549}%
\pgfsetstrokecolor{currentstroke}%
\pgfsetdash{}{0pt}%
\pgfsys@defobject{currentmarker}{\pgfqpoint{-0.041667in}{-0.041667in}}{\pgfqpoint{0.041667in}{0.041667in}}{%
\pgfpathmoveto{\pgfqpoint{0.000000in}{0.041667in}}%
\pgfpathlineto{\pgfqpoint{-0.041667in}{-0.041667in}}%
\pgfpathlineto{\pgfqpoint{0.041667in}{-0.041667in}}%
\pgfpathclose%
\pgfusepath{stroke,fill}%
}%
\begin{pgfscope}%
\pgfsys@transformshift{9.853359in}{2.679689in}%
\pgfsys@useobject{currentmarker}{}%
\end{pgfscope}%
\end{pgfscope}%
\begin{pgfscope}%
\definecolor{textcolor}{rgb}{0.000000,0.000000,0.000000}%
\pgfsetstrokecolor{textcolor}%
\pgfsetfillcolor{textcolor}%
\pgftext[x=10.203359in,y=2.611634in,left,base]{\color{textcolor}\rmfamily\fontsize{14.000000}{16.800000}\selectfont NUCLEAR\_THM}%
\end{pgfscope}%
\begin{pgfscope}%
\definecolor{textcolor}{rgb}{0.000000,0.000000,0.000000}%
\pgfsetstrokecolor{textcolor}%
\pgfsetfillcolor{textcolor}%
\pgftext[x=5.950000in,y=4.850000in,,top]{\color{textcolor}\rmfamily\fontsize{24.000000}{28.800000}\selectfont UIUC Thermal Generating Capacity in 2050}%
\end{pgfscope}%
\end{pgfpicture}%
\makeatother%
\endgroup%
}
  \caption{Results for capacity expansion under various slack values using modeling-
  to-generate alternatives. The x-axis ticks represent a different \gls{mga} run.
  The label indicates the run number and the value of the slack variable. For example,
  ``mga-1-1\%'' indicates the first \gls{mga} run with a one percent
  slack value. For reference, the cost minimized solution ``mga-0-x\%,'' is also
  plotted.}
  \label{fig:uiuc_thm_mga}
\end{figure}

Finally, Figure \ref{fig:uiuc_chw_mga} shows the \gls{mga} runs for the chilled
water sector. As with the other two sectors, the chilled water sector lacks a
diverse set of technology options, leading to persistent results. However,
unique to this sector, the slack value changes the results. Increasing the slack
value shows a tradeoff between chilled water storage and electric chilling capacity.
The chilled water storage only decreases with slack value, indicating an
essential role for storage capacity.

\begin{figure}[H]
  \centering
  \resizebox{0.95\columnwidth}{!}{%% Creator: Matplotlib, PGF backend
%%
%% To include the figure in your LaTeX document, write
%%   \input{<filename>.pgf}
%%
%% Make sure the required packages are loaded in your preamble
%%   \usepackage{pgf}
%%
%% Figures using additional raster images can only be included by \input if
%% they are in the same directory as the main LaTeX file. For loading figures
%% from other directories you can use the `import` package
%%   \usepackage{import}
%%
%% and then include the figures with
%%   \import{<path to file>}{<filename>.pgf}
%%
%% Matplotlib used the following preamble
%%
\begingroup%
\makeatletter%
\begin{pgfpicture}%
\pgfpathrectangle{\pgfpointorigin}{\pgfqpoint{11.900000in}{5.930000in}}%
\pgfusepath{use as bounding box, clip}%
\begin{pgfscope}%
\pgfsetbuttcap%
\pgfsetmiterjoin%
\definecolor{currentfill}{rgb}{1.000000,1.000000,1.000000}%
\pgfsetfillcolor{currentfill}%
\pgfsetlinewidth{0.000000pt}%
\definecolor{currentstroke}{rgb}{0.000000,0.000000,0.000000}%
\pgfsetstrokecolor{currentstroke}%
\pgfsetdash{}{0pt}%
\pgfpathmoveto{\pgfqpoint{0.000000in}{-0.000000in}}%
\pgfpathlineto{\pgfqpoint{11.900000in}{-0.000000in}}%
\pgfpathlineto{\pgfqpoint{11.900000in}{5.930000in}}%
\pgfpathlineto{\pgfqpoint{0.000000in}{5.930000in}}%
\pgfpathclose%
\pgfusepath{fill}%
\end{pgfscope}%
\begin{pgfscope}%
\pgfsetbuttcap%
\pgfsetmiterjoin%
\definecolor{currentfill}{rgb}{0.898039,0.898039,0.898039}%
\pgfsetfillcolor{currentfill}%
\pgfsetlinewidth{0.000000pt}%
\definecolor{currentstroke}{rgb}{0.000000,0.000000,0.000000}%
\pgfsetstrokecolor{currentstroke}%
\pgfsetstrokeopacity{0.000000}%
\pgfsetdash{}{0pt}%
\pgfpathmoveto{\pgfqpoint{0.955704in}{3.402248in}}%
\pgfpathlineto{\pgfqpoint{3.432373in}{3.402248in}}%
\pgfpathlineto{\pgfqpoint{3.432373in}{5.342817in}}%
\pgfpathlineto{\pgfqpoint{0.955704in}{5.342817in}}%
\pgfpathclose%
\pgfusepath{fill}%
\end{pgfscope}%
\begin{pgfscope}%
\pgfpathrectangle{\pgfqpoint{0.955704in}{3.402248in}}{\pgfqpoint{2.476670in}{1.940569in}}%
\pgfusepath{clip}%
\pgfsetrectcap%
\pgfsetroundjoin%
\pgfsetlinewidth{0.803000pt}%
\definecolor{currentstroke}{rgb}{1.000000,1.000000,1.000000}%
\pgfsetstrokecolor{currentstroke}%
\pgfsetdash{}{0pt}%
\pgfpathmoveto{\pgfqpoint{0.955704in}{4.254602in}}%
\pgfpathlineto{\pgfqpoint{3.432373in}{4.254602in}}%
\pgfusepath{stroke}%
\end{pgfscope}%
\begin{pgfscope}%
\pgfsetbuttcap%
\pgfsetroundjoin%
\definecolor{currentfill}{rgb}{0.333333,0.333333,0.333333}%
\pgfsetfillcolor{currentfill}%
\pgfsetlinewidth{0.803000pt}%
\definecolor{currentstroke}{rgb}{0.333333,0.333333,0.333333}%
\pgfsetstrokecolor{currentstroke}%
\pgfsetdash{}{0pt}%
\pgfsys@defobject{currentmarker}{\pgfqpoint{-0.048611in}{0.000000in}}{\pgfqpoint{-0.000000in}{0.000000in}}{%
\pgfpathmoveto{\pgfqpoint{-0.000000in}{0.000000in}}%
\pgfpathlineto{\pgfqpoint{-0.048611in}{0.000000in}}%
\pgfusepath{stroke,fill}%
}%
\begin{pgfscope}%
\pgfsys@transformshift{0.955704in}{4.254602in}%
\pgfsys@useobject{currentmarker}{}%
\end{pgfscope}%
\end{pgfscope}%
\begin{pgfscope}%
\definecolor{textcolor}{rgb}{0.333333,0.333333,0.333333}%
\pgfsetstrokecolor{textcolor}%
\pgfsetfillcolor{textcolor}%
\pgftext[x=0.368904in, y=4.185157in, left, base]{\color{textcolor}\rmfamily\fontsize{14.000000}{16.800000}\selectfont \(\displaystyle {20000}\)}%
\end{pgfscope}%
\begin{pgfscope}%
\pgfpathrectangle{\pgfqpoint{0.955704in}{3.402248in}}{\pgfqpoint{2.476670in}{1.940569in}}%
\pgfusepath{clip}%
\pgfsetrectcap%
\pgfsetroundjoin%
\pgfsetlinewidth{0.803000pt}%
\definecolor{currentstroke}{rgb}{1.000000,1.000000,1.000000}%
\pgfsetstrokecolor{currentstroke}%
\pgfsetdash{}{0pt}%
\pgfpathmoveto{\pgfqpoint{0.955704in}{5.129423in}}%
\pgfpathlineto{\pgfqpoint{3.432373in}{5.129423in}}%
\pgfusepath{stroke}%
\end{pgfscope}%
\begin{pgfscope}%
\pgfsetbuttcap%
\pgfsetroundjoin%
\definecolor{currentfill}{rgb}{0.333333,0.333333,0.333333}%
\pgfsetfillcolor{currentfill}%
\pgfsetlinewidth{0.803000pt}%
\definecolor{currentstroke}{rgb}{0.333333,0.333333,0.333333}%
\pgfsetstrokecolor{currentstroke}%
\pgfsetdash{}{0pt}%
\pgfsys@defobject{currentmarker}{\pgfqpoint{-0.048611in}{0.000000in}}{\pgfqpoint{-0.000000in}{0.000000in}}{%
\pgfpathmoveto{\pgfqpoint{-0.000000in}{0.000000in}}%
\pgfpathlineto{\pgfqpoint{-0.048611in}{0.000000in}}%
\pgfusepath{stroke,fill}%
}%
\begin{pgfscope}%
\pgfsys@transformshift{0.955704in}{5.129423in}%
\pgfsys@useobject{currentmarker}{}%
\end{pgfscope}%
\end{pgfscope}%
\begin{pgfscope}%
\definecolor{textcolor}{rgb}{0.333333,0.333333,0.333333}%
\pgfsetstrokecolor{textcolor}%
\pgfsetfillcolor{textcolor}%
\pgftext[x=0.368904in, y=5.059978in, left, base]{\color{textcolor}\rmfamily\fontsize{14.000000}{16.800000}\selectfont \(\displaystyle {40000}\)}%
\end{pgfscope}%
\begin{pgfscope}%
\definecolor{textcolor}{rgb}{0.333333,0.333333,0.333333}%
\pgfsetstrokecolor{textcolor}%
\pgfsetfillcolor{textcolor}%
\pgftext[x=0.313349in,y=4.372533in,,bottom,rotate=90.000000]{\color{textcolor}\rmfamily\fontsize{16.000000}{19.200000}\selectfont Tons of Refrigeration\(\displaystyle \)}%
\end{pgfscope}%
\begin{pgfscope}%
\pgfpathrectangle{\pgfqpoint{0.955704in}{3.402248in}}{\pgfqpoint{2.476670in}{1.940569in}}%
\pgfusepath{clip}%
\pgfsetbuttcap%
\pgfsetroundjoin%
\pgfsetlinewidth{1.505625pt}%
\definecolor{currentstroke}{rgb}{0.580392,0.403922,0.741176}%
\pgfsetstrokecolor{currentstroke}%
\pgfsetdash{{5.550000pt}{2.400000pt}}{0.000000pt}%
\pgfpathmoveto{\pgfqpoint{1.068280in}{4.779522in}}%
\pgfpathlineto{\pgfqpoint{1.631159in}{5.090121in}}%
\pgfpathlineto{\pgfqpoint{2.194038in}{5.090121in}}%
\pgfpathlineto{\pgfqpoint{2.756918in}{5.090121in}}%
\pgfpathlineto{\pgfqpoint{3.319797in}{5.090121in}}%
\pgfusepath{stroke}%
\end{pgfscope}%
\begin{pgfscope}%
\pgfpathrectangle{\pgfqpoint{0.955704in}{3.402248in}}{\pgfqpoint{2.476670in}{1.940569in}}%
\pgfusepath{clip}%
\pgfsetbuttcap%
\pgfsetroundjoin%
\definecolor{currentfill}{rgb}{0.580392,0.403922,0.741176}%
\pgfsetfillcolor{currentfill}%
\pgfsetlinewidth{1.003750pt}%
\definecolor{currentstroke}{rgb}{0.580392,0.403922,0.741176}%
\pgfsetstrokecolor{currentstroke}%
\pgfsetdash{}{0pt}%
\pgfsys@defobject{currentmarker}{\pgfqpoint{-0.041667in}{-0.041667in}}{\pgfqpoint{0.041667in}{0.041667in}}{%
\pgfpathmoveto{\pgfqpoint{0.000000in}{-0.041667in}}%
\pgfpathcurveto{\pgfqpoint{0.011050in}{-0.041667in}}{\pgfqpoint{0.021649in}{-0.037276in}}{\pgfqpoint{0.029463in}{-0.029463in}}%
\pgfpathcurveto{\pgfqpoint{0.037276in}{-0.021649in}}{\pgfqpoint{0.041667in}{-0.011050in}}{\pgfqpoint{0.041667in}{0.000000in}}%
\pgfpathcurveto{\pgfqpoint{0.041667in}{0.011050in}}{\pgfqpoint{0.037276in}{0.021649in}}{\pgfqpoint{0.029463in}{0.029463in}}%
\pgfpathcurveto{\pgfqpoint{0.021649in}{0.037276in}}{\pgfqpoint{0.011050in}{0.041667in}}{\pgfqpoint{0.000000in}{0.041667in}}%
\pgfpathcurveto{\pgfqpoint{-0.011050in}{0.041667in}}{\pgfqpoint{-0.021649in}{0.037276in}}{\pgfqpoint{-0.029463in}{0.029463in}}%
\pgfpathcurveto{\pgfqpoint{-0.037276in}{0.021649in}}{\pgfqpoint{-0.041667in}{0.011050in}}{\pgfqpoint{-0.041667in}{0.000000in}}%
\pgfpathcurveto{\pgfqpoint{-0.041667in}{-0.011050in}}{\pgfqpoint{-0.037276in}{-0.021649in}}{\pgfqpoint{-0.029463in}{-0.029463in}}%
\pgfpathcurveto{\pgfqpoint{-0.021649in}{-0.037276in}}{\pgfqpoint{-0.011050in}{-0.041667in}}{\pgfqpoint{0.000000in}{-0.041667in}}%
\pgfpathclose%
\pgfusepath{stroke,fill}%
}%
\begin{pgfscope}%
\pgfsys@transformshift{1.068280in}{4.779522in}%
\pgfsys@useobject{currentmarker}{}%
\end{pgfscope}%
\begin{pgfscope}%
\pgfsys@transformshift{1.631159in}{5.090121in}%
\pgfsys@useobject{currentmarker}{}%
\end{pgfscope}%
\begin{pgfscope}%
\pgfsys@transformshift{2.194038in}{5.090121in}%
\pgfsys@useobject{currentmarker}{}%
\end{pgfscope}%
\begin{pgfscope}%
\pgfsys@transformshift{2.756918in}{5.090121in}%
\pgfsys@useobject{currentmarker}{}%
\end{pgfscope}%
\begin{pgfscope}%
\pgfsys@transformshift{3.319797in}{5.090121in}%
\pgfsys@useobject{currentmarker}{}%
\end{pgfscope}%
\end{pgfscope}%
\begin{pgfscope}%
\pgfsetrectcap%
\pgfsetmiterjoin%
\pgfsetlinewidth{1.003750pt}%
\definecolor{currentstroke}{rgb}{1.000000,1.000000,1.000000}%
\pgfsetstrokecolor{currentstroke}%
\pgfsetdash{}{0pt}%
\pgfpathmoveto{\pgfqpoint{0.955704in}{3.402248in}}%
\pgfpathlineto{\pgfqpoint{0.955704in}{5.342817in}}%
\pgfusepath{stroke}%
\end{pgfscope}%
\begin{pgfscope}%
\pgfsetrectcap%
\pgfsetmiterjoin%
\pgfsetlinewidth{1.003750pt}%
\definecolor{currentstroke}{rgb}{1.000000,1.000000,1.000000}%
\pgfsetstrokecolor{currentstroke}%
\pgfsetdash{}{0pt}%
\pgfpathmoveto{\pgfqpoint{3.432373in}{3.402248in}}%
\pgfpathlineto{\pgfqpoint{3.432373in}{5.342817in}}%
\pgfusepath{stroke}%
\end{pgfscope}%
\begin{pgfscope}%
\pgfsetrectcap%
\pgfsetmiterjoin%
\pgfsetlinewidth{1.003750pt}%
\definecolor{currentstroke}{rgb}{1.000000,1.000000,1.000000}%
\pgfsetstrokecolor{currentstroke}%
\pgfsetdash{}{0pt}%
\pgfpathmoveto{\pgfqpoint{0.955704in}{3.402248in}}%
\pgfpathlineto{\pgfqpoint{3.432373in}{3.402248in}}%
\pgfusepath{stroke}%
\end{pgfscope}%
\begin{pgfscope}%
\pgfsetrectcap%
\pgfsetmiterjoin%
\pgfsetlinewidth{1.003750pt}%
\definecolor{currentstroke}{rgb}{1.000000,1.000000,1.000000}%
\pgfsetstrokecolor{currentstroke}%
\pgfsetdash{}{0pt}%
\pgfpathmoveto{\pgfqpoint{0.955704in}{5.342817in}}%
\pgfpathlineto{\pgfqpoint{3.432373in}{5.342817in}}%
\pgfusepath{stroke}%
\end{pgfscope}%
\begin{pgfscope}%
\pgfsetbuttcap%
\pgfsetmiterjoin%
\definecolor{currentfill}{rgb}{0.898039,0.898039,0.898039}%
\pgfsetfillcolor{currentfill}%
\pgfsetlinewidth{0.000000pt}%
\definecolor{currentstroke}{rgb}{0.000000,0.000000,0.000000}%
\pgfsetstrokecolor{currentstroke}%
\pgfsetstrokeopacity{0.000000}%
\pgfsetdash{}{0pt}%
\pgfpathmoveto{\pgfqpoint{3.652320in}{3.402248in}}%
\pgfpathlineto{\pgfqpoint{6.128990in}{3.402248in}}%
\pgfpathlineto{\pgfqpoint{6.128990in}{5.342817in}}%
\pgfpathlineto{\pgfqpoint{3.652320in}{5.342817in}}%
\pgfpathclose%
\pgfusepath{fill}%
\end{pgfscope}%
\begin{pgfscope}%
\pgfpathrectangle{\pgfqpoint{3.652320in}{3.402248in}}{\pgfqpoint{2.476670in}{1.940569in}}%
\pgfusepath{clip}%
\pgfsetrectcap%
\pgfsetroundjoin%
\pgfsetlinewidth{0.803000pt}%
\definecolor{currentstroke}{rgb}{1.000000,1.000000,1.000000}%
\pgfsetstrokecolor{currentstroke}%
\pgfsetdash{}{0pt}%
\pgfpathmoveto{\pgfqpoint{3.652320in}{4.254602in}}%
\pgfpathlineto{\pgfqpoint{6.128990in}{4.254602in}}%
\pgfusepath{stroke}%
\end{pgfscope}%
\begin{pgfscope}%
\pgfsetbuttcap%
\pgfsetroundjoin%
\definecolor{currentfill}{rgb}{0.333333,0.333333,0.333333}%
\pgfsetfillcolor{currentfill}%
\pgfsetlinewidth{0.803000pt}%
\definecolor{currentstroke}{rgb}{0.333333,0.333333,0.333333}%
\pgfsetstrokecolor{currentstroke}%
\pgfsetdash{}{0pt}%
\pgfsys@defobject{currentmarker}{\pgfqpoint{-0.048611in}{0.000000in}}{\pgfqpoint{-0.000000in}{0.000000in}}{%
\pgfpathmoveto{\pgfqpoint{-0.000000in}{0.000000in}}%
\pgfpathlineto{\pgfqpoint{-0.048611in}{0.000000in}}%
\pgfusepath{stroke,fill}%
}%
\begin{pgfscope}%
\pgfsys@transformshift{3.652320in}{4.254602in}%
\pgfsys@useobject{currentmarker}{}%
\end{pgfscope}%
\end{pgfscope}%
\begin{pgfscope}%
\pgfpathrectangle{\pgfqpoint{3.652320in}{3.402248in}}{\pgfqpoint{2.476670in}{1.940569in}}%
\pgfusepath{clip}%
\pgfsetrectcap%
\pgfsetroundjoin%
\pgfsetlinewidth{0.803000pt}%
\definecolor{currentstroke}{rgb}{1.000000,1.000000,1.000000}%
\pgfsetstrokecolor{currentstroke}%
\pgfsetdash{}{0pt}%
\pgfpathmoveto{\pgfqpoint{3.652320in}{5.129423in}}%
\pgfpathlineto{\pgfqpoint{6.128990in}{5.129423in}}%
\pgfusepath{stroke}%
\end{pgfscope}%
\begin{pgfscope}%
\pgfsetbuttcap%
\pgfsetroundjoin%
\definecolor{currentfill}{rgb}{0.333333,0.333333,0.333333}%
\pgfsetfillcolor{currentfill}%
\pgfsetlinewidth{0.803000pt}%
\definecolor{currentstroke}{rgb}{0.333333,0.333333,0.333333}%
\pgfsetstrokecolor{currentstroke}%
\pgfsetdash{}{0pt}%
\pgfsys@defobject{currentmarker}{\pgfqpoint{-0.048611in}{0.000000in}}{\pgfqpoint{-0.000000in}{0.000000in}}{%
\pgfpathmoveto{\pgfqpoint{-0.000000in}{0.000000in}}%
\pgfpathlineto{\pgfqpoint{-0.048611in}{0.000000in}}%
\pgfusepath{stroke,fill}%
}%
\begin{pgfscope}%
\pgfsys@transformshift{3.652320in}{5.129423in}%
\pgfsys@useobject{currentmarker}{}%
\end{pgfscope}%
\end{pgfscope}%
\begin{pgfscope}%
\pgfpathrectangle{\pgfqpoint{3.652320in}{3.402248in}}{\pgfqpoint{2.476670in}{1.940569in}}%
\pgfusepath{clip}%
\pgfsetbuttcap%
\pgfsetroundjoin%
\pgfsetlinewidth{1.505625pt}%
\definecolor{currentstroke}{rgb}{0.580392,0.403922,0.741176}%
\pgfsetstrokecolor{currentstroke}%
\pgfsetdash{{5.550000pt}{2.400000pt}}{0.000000pt}%
\pgfpathmoveto{\pgfqpoint{3.764896in}{4.779522in}}%
\pgfpathlineto{\pgfqpoint{4.327775in}{5.161650in}}%
\pgfpathlineto{\pgfqpoint{4.890655in}{5.161651in}}%
\pgfpathlineto{\pgfqpoint{5.453534in}{5.161651in}}%
\pgfpathlineto{\pgfqpoint{6.016414in}{5.161651in}}%
\pgfusepath{stroke}%
\end{pgfscope}%
\begin{pgfscope}%
\pgfpathrectangle{\pgfqpoint{3.652320in}{3.402248in}}{\pgfqpoint{2.476670in}{1.940569in}}%
\pgfusepath{clip}%
\pgfsetbuttcap%
\pgfsetroundjoin%
\definecolor{currentfill}{rgb}{0.580392,0.403922,0.741176}%
\pgfsetfillcolor{currentfill}%
\pgfsetlinewidth{1.003750pt}%
\definecolor{currentstroke}{rgb}{0.580392,0.403922,0.741176}%
\pgfsetstrokecolor{currentstroke}%
\pgfsetdash{}{0pt}%
\pgfsys@defobject{currentmarker}{\pgfqpoint{-0.041667in}{-0.041667in}}{\pgfqpoint{0.041667in}{0.041667in}}{%
\pgfpathmoveto{\pgfqpoint{0.000000in}{-0.041667in}}%
\pgfpathcurveto{\pgfqpoint{0.011050in}{-0.041667in}}{\pgfqpoint{0.021649in}{-0.037276in}}{\pgfqpoint{0.029463in}{-0.029463in}}%
\pgfpathcurveto{\pgfqpoint{0.037276in}{-0.021649in}}{\pgfqpoint{0.041667in}{-0.011050in}}{\pgfqpoint{0.041667in}{0.000000in}}%
\pgfpathcurveto{\pgfqpoint{0.041667in}{0.011050in}}{\pgfqpoint{0.037276in}{0.021649in}}{\pgfqpoint{0.029463in}{0.029463in}}%
\pgfpathcurveto{\pgfqpoint{0.021649in}{0.037276in}}{\pgfqpoint{0.011050in}{0.041667in}}{\pgfqpoint{0.000000in}{0.041667in}}%
\pgfpathcurveto{\pgfqpoint{-0.011050in}{0.041667in}}{\pgfqpoint{-0.021649in}{0.037276in}}{\pgfqpoint{-0.029463in}{0.029463in}}%
\pgfpathcurveto{\pgfqpoint{-0.037276in}{0.021649in}}{\pgfqpoint{-0.041667in}{0.011050in}}{\pgfqpoint{-0.041667in}{0.000000in}}%
\pgfpathcurveto{\pgfqpoint{-0.041667in}{-0.011050in}}{\pgfqpoint{-0.037276in}{-0.021649in}}{\pgfqpoint{-0.029463in}{-0.029463in}}%
\pgfpathcurveto{\pgfqpoint{-0.021649in}{-0.037276in}}{\pgfqpoint{-0.011050in}{-0.041667in}}{\pgfqpoint{0.000000in}{-0.041667in}}%
\pgfpathclose%
\pgfusepath{stroke,fill}%
}%
\begin{pgfscope}%
\pgfsys@transformshift{3.764896in}{4.779522in}%
\pgfsys@useobject{currentmarker}{}%
\end{pgfscope}%
\begin{pgfscope}%
\pgfsys@transformshift{4.327775in}{5.161650in}%
\pgfsys@useobject{currentmarker}{}%
\end{pgfscope}%
\begin{pgfscope}%
\pgfsys@transformshift{4.890655in}{5.161651in}%
\pgfsys@useobject{currentmarker}{}%
\end{pgfscope}%
\begin{pgfscope}%
\pgfsys@transformshift{5.453534in}{5.161651in}%
\pgfsys@useobject{currentmarker}{}%
\end{pgfscope}%
\begin{pgfscope}%
\pgfsys@transformshift{6.016414in}{5.161651in}%
\pgfsys@useobject{currentmarker}{}%
\end{pgfscope}%
\end{pgfscope}%
\begin{pgfscope}%
\pgfsetrectcap%
\pgfsetmiterjoin%
\pgfsetlinewidth{1.003750pt}%
\definecolor{currentstroke}{rgb}{1.000000,1.000000,1.000000}%
\pgfsetstrokecolor{currentstroke}%
\pgfsetdash{}{0pt}%
\pgfpathmoveto{\pgfqpoint{3.652320in}{3.402248in}}%
\pgfpathlineto{\pgfqpoint{3.652320in}{5.342817in}}%
\pgfusepath{stroke}%
\end{pgfscope}%
\begin{pgfscope}%
\pgfsetrectcap%
\pgfsetmiterjoin%
\pgfsetlinewidth{1.003750pt}%
\definecolor{currentstroke}{rgb}{1.000000,1.000000,1.000000}%
\pgfsetstrokecolor{currentstroke}%
\pgfsetdash{}{0pt}%
\pgfpathmoveto{\pgfqpoint{6.128990in}{3.402248in}}%
\pgfpathlineto{\pgfqpoint{6.128990in}{5.342817in}}%
\pgfusepath{stroke}%
\end{pgfscope}%
\begin{pgfscope}%
\pgfsetrectcap%
\pgfsetmiterjoin%
\pgfsetlinewidth{1.003750pt}%
\definecolor{currentstroke}{rgb}{1.000000,1.000000,1.000000}%
\pgfsetstrokecolor{currentstroke}%
\pgfsetdash{}{0pt}%
\pgfpathmoveto{\pgfqpoint{3.652320in}{3.402248in}}%
\pgfpathlineto{\pgfqpoint{6.128990in}{3.402248in}}%
\pgfusepath{stroke}%
\end{pgfscope}%
\begin{pgfscope}%
\pgfsetrectcap%
\pgfsetmiterjoin%
\pgfsetlinewidth{1.003750pt}%
\definecolor{currentstroke}{rgb}{1.000000,1.000000,1.000000}%
\pgfsetstrokecolor{currentstroke}%
\pgfsetdash{}{0pt}%
\pgfpathmoveto{\pgfqpoint{3.652320in}{5.342817in}}%
\pgfpathlineto{\pgfqpoint{6.128990in}{5.342817in}}%
\pgfusepath{stroke}%
\end{pgfscope}%
\begin{pgfscope}%
\pgfsetbuttcap%
\pgfsetmiterjoin%
\definecolor{currentfill}{rgb}{0.898039,0.898039,0.898039}%
\pgfsetfillcolor{currentfill}%
\pgfsetlinewidth{0.000000pt}%
\definecolor{currentstroke}{rgb}{0.000000,0.000000,0.000000}%
\pgfsetstrokecolor{currentstroke}%
\pgfsetstrokeopacity{0.000000}%
\pgfsetdash{}{0pt}%
\pgfpathmoveto{\pgfqpoint{6.348936in}{3.402248in}}%
\pgfpathlineto{\pgfqpoint{8.825606in}{3.402248in}}%
\pgfpathlineto{\pgfqpoint{8.825606in}{5.342817in}}%
\pgfpathlineto{\pgfqpoint{6.348936in}{5.342817in}}%
\pgfpathclose%
\pgfusepath{fill}%
\end{pgfscope}%
\begin{pgfscope}%
\pgfpathrectangle{\pgfqpoint{6.348936in}{3.402248in}}{\pgfqpoint{2.476670in}{1.940569in}}%
\pgfusepath{clip}%
\pgfsetrectcap%
\pgfsetroundjoin%
\pgfsetlinewidth{0.803000pt}%
\definecolor{currentstroke}{rgb}{1.000000,1.000000,1.000000}%
\pgfsetstrokecolor{currentstroke}%
\pgfsetdash{}{0pt}%
\pgfpathmoveto{\pgfqpoint{6.348936in}{4.254602in}}%
\pgfpathlineto{\pgfqpoint{8.825606in}{4.254602in}}%
\pgfusepath{stroke}%
\end{pgfscope}%
\begin{pgfscope}%
\pgfsetbuttcap%
\pgfsetroundjoin%
\definecolor{currentfill}{rgb}{0.333333,0.333333,0.333333}%
\pgfsetfillcolor{currentfill}%
\pgfsetlinewidth{0.803000pt}%
\definecolor{currentstroke}{rgb}{0.333333,0.333333,0.333333}%
\pgfsetstrokecolor{currentstroke}%
\pgfsetdash{}{0pt}%
\pgfsys@defobject{currentmarker}{\pgfqpoint{-0.048611in}{0.000000in}}{\pgfqpoint{-0.000000in}{0.000000in}}{%
\pgfpathmoveto{\pgfqpoint{-0.000000in}{0.000000in}}%
\pgfpathlineto{\pgfqpoint{-0.048611in}{0.000000in}}%
\pgfusepath{stroke,fill}%
}%
\begin{pgfscope}%
\pgfsys@transformshift{6.348936in}{4.254602in}%
\pgfsys@useobject{currentmarker}{}%
\end{pgfscope}%
\end{pgfscope}%
\begin{pgfscope}%
\pgfpathrectangle{\pgfqpoint{6.348936in}{3.402248in}}{\pgfqpoint{2.476670in}{1.940569in}}%
\pgfusepath{clip}%
\pgfsetrectcap%
\pgfsetroundjoin%
\pgfsetlinewidth{0.803000pt}%
\definecolor{currentstroke}{rgb}{1.000000,1.000000,1.000000}%
\pgfsetstrokecolor{currentstroke}%
\pgfsetdash{}{0pt}%
\pgfpathmoveto{\pgfqpoint{6.348936in}{5.129423in}}%
\pgfpathlineto{\pgfqpoint{8.825606in}{5.129423in}}%
\pgfusepath{stroke}%
\end{pgfscope}%
\begin{pgfscope}%
\pgfsetbuttcap%
\pgfsetroundjoin%
\definecolor{currentfill}{rgb}{0.333333,0.333333,0.333333}%
\pgfsetfillcolor{currentfill}%
\pgfsetlinewidth{0.803000pt}%
\definecolor{currentstroke}{rgb}{0.333333,0.333333,0.333333}%
\pgfsetstrokecolor{currentstroke}%
\pgfsetdash{}{0pt}%
\pgfsys@defobject{currentmarker}{\pgfqpoint{-0.048611in}{0.000000in}}{\pgfqpoint{-0.000000in}{0.000000in}}{%
\pgfpathmoveto{\pgfqpoint{-0.000000in}{0.000000in}}%
\pgfpathlineto{\pgfqpoint{-0.048611in}{0.000000in}}%
\pgfusepath{stroke,fill}%
}%
\begin{pgfscope}%
\pgfsys@transformshift{6.348936in}{5.129423in}%
\pgfsys@useobject{currentmarker}{}%
\end{pgfscope}%
\end{pgfscope}%
\begin{pgfscope}%
\pgfpathrectangle{\pgfqpoint{6.348936in}{3.402248in}}{\pgfqpoint{2.476670in}{1.940569in}}%
\pgfusepath{clip}%
\pgfsetbuttcap%
\pgfsetroundjoin%
\pgfsetlinewidth{1.505625pt}%
\definecolor{currentstroke}{rgb}{0.580392,0.403922,0.741176}%
\pgfsetstrokecolor{currentstroke}%
\pgfsetdash{{5.550000pt}{2.400000pt}}{0.000000pt}%
\pgfpathmoveto{\pgfqpoint{6.461512in}{4.779522in}}%
\pgfpathlineto{\pgfqpoint{7.024392in}{5.191406in}}%
\pgfpathlineto{\pgfqpoint{7.587271in}{5.191406in}}%
\pgfpathlineto{\pgfqpoint{8.150151in}{5.191406in}}%
\pgfpathlineto{\pgfqpoint{8.713030in}{5.191406in}}%
\pgfusepath{stroke}%
\end{pgfscope}%
\begin{pgfscope}%
\pgfpathrectangle{\pgfqpoint{6.348936in}{3.402248in}}{\pgfqpoint{2.476670in}{1.940569in}}%
\pgfusepath{clip}%
\pgfsetbuttcap%
\pgfsetroundjoin%
\definecolor{currentfill}{rgb}{0.580392,0.403922,0.741176}%
\pgfsetfillcolor{currentfill}%
\pgfsetlinewidth{1.003750pt}%
\definecolor{currentstroke}{rgb}{0.580392,0.403922,0.741176}%
\pgfsetstrokecolor{currentstroke}%
\pgfsetdash{}{0pt}%
\pgfsys@defobject{currentmarker}{\pgfqpoint{-0.041667in}{-0.041667in}}{\pgfqpoint{0.041667in}{0.041667in}}{%
\pgfpathmoveto{\pgfqpoint{0.000000in}{-0.041667in}}%
\pgfpathcurveto{\pgfqpoint{0.011050in}{-0.041667in}}{\pgfqpoint{0.021649in}{-0.037276in}}{\pgfqpoint{0.029463in}{-0.029463in}}%
\pgfpathcurveto{\pgfqpoint{0.037276in}{-0.021649in}}{\pgfqpoint{0.041667in}{-0.011050in}}{\pgfqpoint{0.041667in}{0.000000in}}%
\pgfpathcurveto{\pgfqpoint{0.041667in}{0.011050in}}{\pgfqpoint{0.037276in}{0.021649in}}{\pgfqpoint{0.029463in}{0.029463in}}%
\pgfpathcurveto{\pgfqpoint{0.021649in}{0.037276in}}{\pgfqpoint{0.011050in}{0.041667in}}{\pgfqpoint{0.000000in}{0.041667in}}%
\pgfpathcurveto{\pgfqpoint{-0.011050in}{0.041667in}}{\pgfqpoint{-0.021649in}{0.037276in}}{\pgfqpoint{-0.029463in}{0.029463in}}%
\pgfpathcurveto{\pgfqpoint{-0.037276in}{0.021649in}}{\pgfqpoint{-0.041667in}{0.011050in}}{\pgfqpoint{-0.041667in}{0.000000in}}%
\pgfpathcurveto{\pgfqpoint{-0.041667in}{-0.011050in}}{\pgfqpoint{-0.037276in}{-0.021649in}}{\pgfqpoint{-0.029463in}{-0.029463in}}%
\pgfpathcurveto{\pgfqpoint{-0.021649in}{-0.037276in}}{\pgfqpoint{-0.011050in}{-0.041667in}}{\pgfqpoint{0.000000in}{-0.041667in}}%
\pgfpathclose%
\pgfusepath{stroke,fill}%
}%
\begin{pgfscope}%
\pgfsys@transformshift{6.461512in}{4.779522in}%
\pgfsys@useobject{currentmarker}{}%
\end{pgfscope}%
\begin{pgfscope}%
\pgfsys@transformshift{7.024392in}{5.191406in}%
\pgfsys@useobject{currentmarker}{}%
\end{pgfscope}%
\begin{pgfscope}%
\pgfsys@transformshift{7.587271in}{5.191406in}%
\pgfsys@useobject{currentmarker}{}%
\end{pgfscope}%
\begin{pgfscope}%
\pgfsys@transformshift{8.150151in}{5.191406in}%
\pgfsys@useobject{currentmarker}{}%
\end{pgfscope}%
\begin{pgfscope}%
\pgfsys@transformshift{8.713030in}{5.191406in}%
\pgfsys@useobject{currentmarker}{}%
\end{pgfscope}%
\end{pgfscope}%
\begin{pgfscope}%
\pgfsetrectcap%
\pgfsetmiterjoin%
\pgfsetlinewidth{1.003750pt}%
\definecolor{currentstroke}{rgb}{1.000000,1.000000,1.000000}%
\pgfsetstrokecolor{currentstroke}%
\pgfsetdash{}{0pt}%
\pgfpathmoveto{\pgfqpoint{6.348936in}{3.402248in}}%
\pgfpathlineto{\pgfqpoint{6.348936in}{5.342817in}}%
\pgfusepath{stroke}%
\end{pgfscope}%
\begin{pgfscope}%
\pgfsetrectcap%
\pgfsetmiterjoin%
\pgfsetlinewidth{1.003750pt}%
\definecolor{currentstroke}{rgb}{1.000000,1.000000,1.000000}%
\pgfsetstrokecolor{currentstroke}%
\pgfsetdash{}{0pt}%
\pgfpathmoveto{\pgfqpoint{8.825606in}{3.402248in}}%
\pgfpathlineto{\pgfqpoint{8.825606in}{5.342817in}}%
\pgfusepath{stroke}%
\end{pgfscope}%
\begin{pgfscope}%
\pgfsetrectcap%
\pgfsetmiterjoin%
\pgfsetlinewidth{1.003750pt}%
\definecolor{currentstroke}{rgb}{1.000000,1.000000,1.000000}%
\pgfsetstrokecolor{currentstroke}%
\pgfsetdash{}{0pt}%
\pgfpathmoveto{\pgfqpoint{6.348936in}{3.402248in}}%
\pgfpathlineto{\pgfqpoint{8.825606in}{3.402248in}}%
\pgfusepath{stroke}%
\end{pgfscope}%
\begin{pgfscope}%
\pgfsetrectcap%
\pgfsetmiterjoin%
\pgfsetlinewidth{1.003750pt}%
\definecolor{currentstroke}{rgb}{1.000000,1.000000,1.000000}%
\pgfsetstrokecolor{currentstroke}%
\pgfsetdash{}{0pt}%
\pgfpathmoveto{\pgfqpoint{6.348936in}{5.342817in}}%
\pgfpathlineto{\pgfqpoint{8.825606in}{5.342817in}}%
\pgfusepath{stroke}%
\end{pgfscope}%
\begin{pgfscope}%
\pgfsetbuttcap%
\pgfsetmiterjoin%
\definecolor{currentfill}{rgb}{0.898039,0.898039,0.898039}%
\pgfsetfillcolor{currentfill}%
\pgfsetlinewidth{0.000000pt}%
\definecolor{currentstroke}{rgb}{0.000000,0.000000,0.000000}%
\pgfsetstrokecolor{currentstroke}%
\pgfsetstrokeopacity{0.000000}%
\pgfsetdash{}{0pt}%
\pgfpathmoveto{\pgfqpoint{9.045553in}{3.402248in}}%
\pgfpathlineto{\pgfqpoint{11.522222in}{3.402248in}}%
\pgfpathlineto{\pgfqpoint{11.522222in}{5.342817in}}%
\pgfpathlineto{\pgfqpoint{9.045553in}{5.342817in}}%
\pgfpathclose%
\pgfusepath{fill}%
\end{pgfscope}%
\begin{pgfscope}%
\pgfpathrectangle{\pgfqpoint{9.045553in}{3.402248in}}{\pgfqpoint{2.476670in}{1.940569in}}%
\pgfusepath{clip}%
\pgfsetrectcap%
\pgfsetroundjoin%
\pgfsetlinewidth{0.803000pt}%
\definecolor{currentstroke}{rgb}{1.000000,1.000000,1.000000}%
\pgfsetstrokecolor{currentstroke}%
\pgfsetdash{}{0pt}%
\pgfpathmoveto{\pgfqpoint{9.045553in}{4.254602in}}%
\pgfpathlineto{\pgfqpoint{11.522222in}{4.254602in}}%
\pgfusepath{stroke}%
\end{pgfscope}%
\begin{pgfscope}%
\pgfsetbuttcap%
\pgfsetroundjoin%
\definecolor{currentfill}{rgb}{0.333333,0.333333,0.333333}%
\pgfsetfillcolor{currentfill}%
\pgfsetlinewidth{0.803000pt}%
\definecolor{currentstroke}{rgb}{0.333333,0.333333,0.333333}%
\pgfsetstrokecolor{currentstroke}%
\pgfsetdash{}{0pt}%
\pgfsys@defobject{currentmarker}{\pgfqpoint{-0.048611in}{0.000000in}}{\pgfqpoint{-0.000000in}{0.000000in}}{%
\pgfpathmoveto{\pgfqpoint{-0.000000in}{0.000000in}}%
\pgfpathlineto{\pgfqpoint{-0.048611in}{0.000000in}}%
\pgfusepath{stroke,fill}%
}%
\begin{pgfscope}%
\pgfsys@transformshift{9.045553in}{4.254602in}%
\pgfsys@useobject{currentmarker}{}%
\end{pgfscope}%
\end{pgfscope}%
\begin{pgfscope}%
\pgfpathrectangle{\pgfqpoint{9.045553in}{3.402248in}}{\pgfqpoint{2.476670in}{1.940569in}}%
\pgfusepath{clip}%
\pgfsetrectcap%
\pgfsetroundjoin%
\pgfsetlinewidth{0.803000pt}%
\definecolor{currentstroke}{rgb}{1.000000,1.000000,1.000000}%
\pgfsetstrokecolor{currentstroke}%
\pgfsetdash{}{0pt}%
\pgfpathmoveto{\pgfqpoint{9.045553in}{5.129423in}}%
\pgfpathlineto{\pgfqpoint{11.522222in}{5.129423in}}%
\pgfusepath{stroke}%
\end{pgfscope}%
\begin{pgfscope}%
\pgfsetbuttcap%
\pgfsetroundjoin%
\definecolor{currentfill}{rgb}{0.333333,0.333333,0.333333}%
\pgfsetfillcolor{currentfill}%
\pgfsetlinewidth{0.803000pt}%
\definecolor{currentstroke}{rgb}{0.333333,0.333333,0.333333}%
\pgfsetstrokecolor{currentstroke}%
\pgfsetdash{}{0pt}%
\pgfsys@defobject{currentmarker}{\pgfqpoint{-0.048611in}{0.000000in}}{\pgfqpoint{-0.000000in}{0.000000in}}{%
\pgfpathmoveto{\pgfqpoint{-0.000000in}{0.000000in}}%
\pgfpathlineto{\pgfqpoint{-0.048611in}{0.000000in}}%
\pgfusepath{stroke,fill}%
}%
\begin{pgfscope}%
\pgfsys@transformshift{9.045553in}{5.129423in}%
\pgfsys@useobject{currentmarker}{}%
\end{pgfscope}%
\end{pgfscope}%
\begin{pgfscope}%
\pgfpathrectangle{\pgfqpoint{9.045553in}{3.402248in}}{\pgfqpoint{2.476670in}{1.940569in}}%
\pgfusepath{clip}%
\pgfsetbuttcap%
\pgfsetroundjoin%
\pgfsetlinewidth{1.505625pt}%
\definecolor{currentstroke}{rgb}{0.580392,0.403922,0.741176}%
\pgfsetstrokecolor{currentstroke}%
\pgfsetdash{{5.550000pt}{2.400000pt}}{0.000000pt}%
\pgfpathmoveto{\pgfqpoint{9.158129in}{4.779522in}}%
\pgfpathlineto{\pgfqpoint{9.721008in}{5.254610in}}%
\pgfpathlineto{\pgfqpoint{10.283887in}{5.254610in}}%
\pgfpathlineto{\pgfqpoint{10.846767in}{5.254610in}}%
\pgfpathlineto{\pgfqpoint{11.409646in}{5.254610in}}%
\pgfusepath{stroke}%
\end{pgfscope}%
\begin{pgfscope}%
\pgfpathrectangle{\pgfqpoint{9.045553in}{3.402248in}}{\pgfqpoint{2.476670in}{1.940569in}}%
\pgfusepath{clip}%
\pgfsetbuttcap%
\pgfsetroundjoin%
\definecolor{currentfill}{rgb}{0.580392,0.403922,0.741176}%
\pgfsetfillcolor{currentfill}%
\pgfsetlinewidth{1.003750pt}%
\definecolor{currentstroke}{rgb}{0.580392,0.403922,0.741176}%
\pgfsetstrokecolor{currentstroke}%
\pgfsetdash{}{0pt}%
\pgfsys@defobject{currentmarker}{\pgfqpoint{-0.041667in}{-0.041667in}}{\pgfqpoint{0.041667in}{0.041667in}}{%
\pgfpathmoveto{\pgfqpoint{0.000000in}{-0.041667in}}%
\pgfpathcurveto{\pgfqpoint{0.011050in}{-0.041667in}}{\pgfqpoint{0.021649in}{-0.037276in}}{\pgfqpoint{0.029463in}{-0.029463in}}%
\pgfpathcurveto{\pgfqpoint{0.037276in}{-0.021649in}}{\pgfqpoint{0.041667in}{-0.011050in}}{\pgfqpoint{0.041667in}{0.000000in}}%
\pgfpathcurveto{\pgfqpoint{0.041667in}{0.011050in}}{\pgfqpoint{0.037276in}{0.021649in}}{\pgfqpoint{0.029463in}{0.029463in}}%
\pgfpathcurveto{\pgfqpoint{0.021649in}{0.037276in}}{\pgfqpoint{0.011050in}{0.041667in}}{\pgfqpoint{0.000000in}{0.041667in}}%
\pgfpathcurveto{\pgfqpoint{-0.011050in}{0.041667in}}{\pgfqpoint{-0.021649in}{0.037276in}}{\pgfqpoint{-0.029463in}{0.029463in}}%
\pgfpathcurveto{\pgfqpoint{-0.037276in}{0.021649in}}{\pgfqpoint{-0.041667in}{0.011050in}}{\pgfqpoint{-0.041667in}{0.000000in}}%
\pgfpathcurveto{\pgfqpoint{-0.041667in}{-0.011050in}}{\pgfqpoint{-0.037276in}{-0.021649in}}{\pgfqpoint{-0.029463in}{-0.029463in}}%
\pgfpathcurveto{\pgfqpoint{-0.021649in}{-0.037276in}}{\pgfqpoint{-0.011050in}{-0.041667in}}{\pgfqpoint{0.000000in}{-0.041667in}}%
\pgfpathclose%
\pgfusepath{stroke,fill}%
}%
\begin{pgfscope}%
\pgfsys@transformshift{9.158129in}{4.779522in}%
\pgfsys@useobject{currentmarker}{}%
\end{pgfscope}%
\begin{pgfscope}%
\pgfsys@transformshift{9.721008in}{5.254610in}%
\pgfsys@useobject{currentmarker}{}%
\end{pgfscope}%
\begin{pgfscope}%
\pgfsys@transformshift{10.283887in}{5.254610in}%
\pgfsys@useobject{currentmarker}{}%
\end{pgfscope}%
\begin{pgfscope}%
\pgfsys@transformshift{10.846767in}{5.254610in}%
\pgfsys@useobject{currentmarker}{}%
\end{pgfscope}%
\begin{pgfscope}%
\pgfsys@transformshift{11.409646in}{5.254610in}%
\pgfsys@useobject{currentmarker}{}%
\end{pgfscope}%
\end{pgfscope}%
\begin{pgfscope}%
\pgfsetrectcap%
\pgfsetmiterjoin%
\pgfsetlinewidth{1.003750pt}%
\definecolor{currentstroke}{rgb}{1.000000,1.000000,1.000000}%
\pgfsetstrokecolor{currentstroke}%
\pgfsetdash{}{0pt}%
\pgfpathmoveto{\pgfqpoint{9.045553in}{3.402248in}}%
\pgfpathlineto{\pgfqpoint{9.045553in}{5.342817in}}%
\pgfusepath{stroke}%
\end{pgfscope}%
\begin{pgfscope}%
\pgfsetrectcap%
\pgfsetmiterjoin%
\pgfsetlinewidth{1.003750pt}%
\definecolor{currentstroke}{rgb}{1.000000,1.000000,1.000000}%
\pgfsetstrokecolor{currentstroke}%
\pgfsetdash{}{0pt}%
\pgfpathmoveto{\pgfqpoint{11.522222in}{3.402248in}}%
\pgfpathlineto{\pgfqpoint{11.522222in}{5.342817in}}%
\pgfusepath{stroke}%
\end{pgfscope}%
\begin{pgfscope}%
\pgfsetrectcap%
\pgfsetmiterjoin%
\pgfsetlinewidth{1.003750pt}%
\definecolor{currentstroke}{rgb}{1.000000,1.000000,1.000000}%
\pgfsetstrokecolor{currentstroke}%
\pgfsetdash{}{0pt}%
\pgfpathmoveto{\pgfqpoint{9.045553in}{3.402248in}}%
\pgfpathlineto{\pgfqpoint{11.522222in}{3.402248in}}%
\pgfusepath{stroke}%
\end{pgfscope}%
\begin{pgfscope}%
\pgfsetrectcap%
\pgfsetmiterjoin%
\pgfsetlinewidth{1.003750pt}%
\definecolor{currentstroke}{rgb}{1.000000,1.000000,1.000000}%
\pgfsetstrokecolor{currentstroke}%
\pgfsetdash{}{0pt}%
\pgfpathmoveto{\pgfqpoint{9.045553in}{5.342817in}}%
\pgfpathlineto{\pgfqpoint{11.522222in}{5.342817in}}%
\pgfusepath{stroke}%
\end{pgfscope}%
\begin{pgfscope}%
\pgfsetbuttcap%
\pgfsetmiterjoin%
\definecolor{currentfill}{rgb}{0.898039,0.898039,0.898039}%
\pgfsetfillcolor{currentfill}%
\pgfsetlinewidth{0.000000pt}%
\definecolor{currentstroke}{rgb}{0.000000,0.000000,0.000000}%
\pgfsetstrokecolor{currentstroke}%
\pgfsetstrokeopacity{0.000000}%
\pgfsetdash{}{0pt}%
\pgfpathmoveto{\pgfqpoint{0.955704in}{1.263068in}}%
\pgfpathlineto{\pgfqpoint{3.432373in}{1.263068in}}%
\pgfpathlineto{\pgfqpoint{3.432373in}{3.203637in}}%
\pgfpathlineto{\pgfqpoint{0.955704in}{3.203637in}}%
\pgfpathclose%
\pgfusepath{fill}%
\end{pgfscope}%
\begin{pgfscope}%
\pgfpathrectangle{\pgfqpoint{0.955704in}{1.263068in}}{\pgfqpoint{2.476670in}{1.940569in}}%
\pgfusepath{clip}%
\pgfsetrectcap%
\pgfsetroundjoin%
\pgfsetlinewidth{0.803000pt}%
\definecolor{currentstroke}{rgb}{1.000000,1.000000,1.000000}%
\pgfsetstrokecolor{currentstroke}%
\pgfsetstrokeopacity{0.000000}%
\pgfsetdash{}{0pt}%
\pgfpathmoveto{\pgfqpoint{1.068280in}{1.263068in}}%
\pgfpathlineto{\pgfqpoint{1.068280in}{3.203637in}}%
\pgfusepath{stroke}%
\end{pgfscope}%
\begin{pgfscope}%
\pgfsetbuttcap%
\pgfsetroundjoin%
\definecolor{currentfill}{rgb}{0.333333,0.333333,0.333333}%
\pgfsetfillcolor{currentfill}%
\pgfsetlinewidth{0.803000pt}%
\definecolor{currentstroke}{rgb}{0.333333,0.333333,0.333333}%
\pgfsetstrokecolor{currentstroke}%
\pgfsetdash{}{0pt}%
\pgfsys@defobject{currentmarker}{\pgfqpoint{0.000000in}{-0.048611in}}{\pgfqpoint{0.000000in}{0.000000in}}{%
\pgfpathmoveto{\pgfqpoint{0.000000in}{0.000000in}}%
\pgfpathlineto{\pgfqpoint{0.000000in}{-0.048611in}}%
\pgfusepath{stroke,fill}%
}%
\begin{pgfscope}%
\pgfsys@transformshift{1.068280in}{1.263068in}%
\pgfsys@useobject{currentmarker}{}%
\end{pgfscope}%
\end{pgfscope}%
\begin{pgfscope}%
\definecolor{textcolor}{rgb}{0.333333,0.333333,0.333333}%
\pgfsetstrokecolor{textcolor}%
\pgfsetfillcolor{textcolor}%
\pgftext[x=1.134830in, y=0.210068in, left, base,rotate=90.000000]{\color{textcolor}\rmfamily\fontsize{16.000000}{19.200000}\selectfont mga-0-1\%}%
\end{pgfscope}%
\begin{pgfscope}%
\pgfpathrectangle{\pgfqpoint{0.955704in}{1.263068in}}{\pgfqpoint{2.476670in}{1.940569in}}%
\pgfusepath{clip}%
\pgfsetrectcap%
\pgfsetroundjoin%
\pgfsetlinewidth{0.803000pt}%
\definecolor{currentstroke}{rgb}{1.000000,1.000000,1.000000}%
\pgfsetstrokecolor{currentstroke}%
\pgfsetstrokeopacity{0.000000}%
\pgfsetdash{}{0pt}%
\pgfpathmoveto{\pgfqpoint{1.631159in}{1.263068in}}%
\pgfpathlineto{\pgfqpoint{1.631159in}{3.203637in}}%
\pgfusepath{stroke}%
\end{pgfscope}%
\begin{pgfscope}%
\pgfsetbuttcap%
\pgfsetroundjoin%
\definecolor{currentfill}{rgb}{0.333333,0.333333,0.333333}%
\pgfsetfillcolor{currentfill}%
\pgfsetlinewidth{0.803000pt}%
\definecolor{currentstroke}{rgb}{0.333333,0.333333,0.333333}%
\pgfsetstrokecolor{currentstroke}%
\pgfsetdash{}{0pt}%
\pgfsys@defobject{currentmarker}{\pgfqpoint{0.000000in}{-0.048611in}}{\pgfqpoint{0.000000in}{0.000000in}}{%
\pgfpathmoveto{\pgfqpoint{0.000000in}{0.000000in}}%
\pgfpathlineto{\pgfqpoint{0.000000in}{-0.048611in}}%
\pgfusepath{stroke,fill}%
}%
\begin{pgfscope}%
\pgfsys@transformshift{1.631159in}{1.263068in}%
\pgfsys@useobject{currentmarker}{}%
\end{pgfscope}%
\end{pgfscope}%
\begin{pgfscope}%
\definecolor{textcolor}{rgb}{0.333333,0.333333,0.333333}%
\pgfsetstrokecolor{textcolor}%
\pgfsetfillcolor{textcolor}%
\pgftext[x=1.697710in, y=0.210068in, left, base,rotate=90.000000]{\color{textcolor}\rmfamily\fontsize{16.000000}{19.200000}\selectfont mga-1-1\%}%
\end{pgfscope}%
\begin{pgfscope}%
\pgfpathrectangle{\pgfqpoint{0.955704in}{1.263068in}}{\pgfqpoint{2.476670in}{1.940569in}}%
\pgfusepath{clip}%
\pgfsetrectcap%
\pgfsetroundjoin%
\pgfsetlinewidth{0.803000pt}%
\definecolor{currentstroke}{rgb}{1.000000,1.000000,1.000000}%
\pgfsetstrokecolor{currentstroke}%
\pgfsetstrokeopacity{0.000000}%
\pgfsetdash{}{0pt}%
\pgfpathmoveto{\pgfqpoint{2.194038in}{1.263068in}}%
\pgfpathlineto{\pgfqpoint{2.194038in}{3.203637in}}%
\pgfusepath{stroke}%
\end{pgfscope}%
\begin{pgfscope}%
\pgfsetbuttcap%
\pgfsetroundjoin%
\definecolor{currentfill}{rgb}{0.333333,0.333333,0.333333}%
\pgfsetfillcolor{currentfill}%
\pgfsetlinewidth{0.803000pt}%
\definecolor{currentstroke}{rgb}{0.333333,0.333333,0.333333}%
\pgfsetstrokecolor{currentstroke}%
\pgfsetdash{}{0pt}%
\pgfsys@defobject{currentmarker}{\pgfqpoint{0.000000in}{-0.048611in}}{\pgfqpoint{0.000000in}{0.000000in}}{%
\pgfpathmoveto{\pgfqpoint{0.000000in}{0.000000in}}%
\pgfpathlineto{\pgfqpoint{0.000000in}{-0.048611in}}%
\pgfusepath{stroke,fill}%
}%
\begin{pgfscope}%
\pgfsys@transformshift{2.194038in}{1.263068in}%
\pgfsys@useobject{currentmarker}{}%
\end{pgfscope}%
\end{pgfscope}%
\begin{pgfscope}%
\definecolor{textcolor}{rgb}{0.333333,0.333333,0.333333}%
\pgfsetstrokecolor{textcolor}%
\pgfsetfillcolor{textcolor}%
\pgftext[x=2.260589in, y=0.210068in, left, base,rotate=90.000000]{\color{textcolor}\rmfamily\fontsize{16.000000}{19.200000}\selectfont mga-2-1\%}%
\end{pgfscope}%
\begin{pgfscope}%
\pgfpathrectangle{\pgfqpoint{0.955704in}{1.263068in}}{\pgfqpoint{2.476670in}{1.940569in}}%
\pgfusepath{clip}%
\pgfsetrectcap%
\pgfsetroundjoin%
\pgfsetlinewidth{0.803000pt}%
\definecolor{currentstroke}{rgb}{1.000000,1.000000,1.000000}%
\pgfsetstrokecolor{currentstroke}%
\pgfsetstrokeopacity{0.000000}%
\pgfsetdash{}{0pt}%
\pgfpathmoveto{\pgfqpoint{2.756918in}{1.263068in}}%
\pgfpathlineto{\pgfqpoint{2.756918in}{3.203637in}}%
\pgfusepath{stroke}%
\end{pgfscope}%
\begin{pgfscope}%
\pgfsetbuttcap%
\pgfsetroundjoin%
\definecolor{currentfill}{rgb}{0.333333,0.333333,0.333333}%
\pgfsetfillcolor{currentfill}%
\pgfsetlinewidth{0.803000pt}%
\definecolor{currentstroke}{rgb}{0.333333,0.333333,0.333333}%
\pgfsetstrokecolor{currentstroke}%
\pgfsetdash{}{0pt}%
\pgfsys@defobject{currentmarker}{\pgfqpoint{0.000000in}{-0.048611in}}{\pgfqpoint{0.000000in}{0.000000in}}{%
\pgfpathmoveto{\pgfqpoint{0.000000in}{0.000000in}}%
\pgfpathlineto{\pgfqpoint{0.000000in}{-0.048611in}}%
\pgfusepath{stroke,fill}%
}%
\begin{pgfscope}%
\pgfsys@transformshift{2.756918in}{1.263068in}%
\pgfsys@useobject{currentmarker}{}%
\end{pgfscope}%
\end{pgfscope}%
\begin{pgfscope}%
\definecolor{textcolor}{rgb}{0.333333,0.333333,0.333333}%
\pgfsetstrokecolor{textcolor}%
\pgfsetfillcolor{textcolor}%
\pgftext[x=2.823469in, y=0.210068in, left, base,rotate=90.000000]{\color{textcolor}\rmfamily\fontsize{16.000000}{19.200000}\selectfont mga-3-1\%}%
\end{pgfscope}%
\begin{pgfscope}%
\pgfpathrectangle{\pgfqpoint{0.955704in}{1.263068in}}{\pgfqpoint{2.476670in}{1.940569in}}%
\pgfusepath{clip}%
\pgfsetrectcap%
\pgfsetroundjoin%
\pgfsetlinewidth{0.803000pt}%
\definecolor{currentstroke}{rgb}{1.000000,1.000000,1.000000}%
\pgfsetstrokecolor{currentstroke}%
\pgfsetstrokeopacity{0.000000}%
\pgfsetdash{}{0pt}%
\pgfpathmoveto{\pgfqpoint{3.319797in}{1.263068in}}%
\pgfpathlineto{\pgfqpoint{3.319797in}{3.203637in}}%
\pgfusepath{stroke}%
\end{pgfscope}%
\begin{pgfscope}%
\pgfsetbuttcap%
\pgfsetroundjoin%
\definecolor{currentfill}{rgb}{0.333333,0.333333,0.333333}%
\pgfsetfillcolor{currentfill}%
\pgfsetlinewidth{0.803000pt}%
\definecolor{currentstroke}{rgb}{0.333333,0.333333,0.333333}%
\pgfsetstrokecolor{currentstroke}%
\pgfsetdash{}{0pt}%
\pgfsys@defobject{currentmarker}{\pgfqpoint{0.000000in}{-0.048611in}}{\pgfqpoint{0.000000in}{0.000000in}}{%
\pgfpathmoveto{\pgfqpoint{0.000000in}{0.000000in}}%
\pgfpathlineto{\pgfqpoint{0.000000in}{-0.048611in}}%
\pgfusepath{stroke,fill}%
}%
\begin{pgfscope}%
\pgfsys@transformshift{3.319797in}{1.263068in}%
\pgfsys@useobject{currentmarker}{}%
\end{pgfscope}%
\end{pgfscope}%
\begin{pgfscope}%
\definecolor{textcolor}{rgb}{0.333333,0.333333,0.333333}%
\pgfsetstrokecolor{textcolor}%
\pgfsetfillcolor{textcolor}%
\pgftext[x=3.386348in, y=0.210068in, left, base,rotate=90.000000]{\color{textcolor}\rmfamily\fontsize{16.000000}{19.200000}\selectfont mga-4-1\%}%
\end{pgfscope}%
\begin{pgfscope}%
\pgfpathrectangle{\pgfqpoint{0.955704in}{1.263068in}}{\pgfqpoint{2.476670in}{1.940569in}}%
\pgfusepath{clip}%
\pgfsetrectcap%
\pgfsetroundjoin%
\pgfsetlinewidth{0.803000pt}%
\definecolor{currentstroke}{rgb}{1.000000,1.000000,1.000000}%
\pgfsetstrokecolor{currentstroke}%
\pgfsetdash{}{0pt}%
\pgfpathmoveto{\pgfqpoint{0.955704in}{2.115421in}}%
\pgfpathlineto{\pgfqpoint{3.432373in}{2.115421in}}%
\pgfusepath{stroke}%
\end{pgfscope}%
\begin{pgfscope}%
\pgfsetbuttcap%
\pgfsetroundjoin%
\definecolor{currentfill}{rgb}{0.333333,0.333333,0.333333}%
\pgfsetfillcolor{currentfill}%
\pgfsetlinewidth{0.803000pt}%
\definecolor{currentstroke}{rgb}{0.333333,0.333333,0.333333}%
\pgfsetstrokecolor{currentstroke}%
\pgfsetdash{}{0pt}%
\pgfsys@defobject{currentmarker}{\pgfqpoint{-0.048611in}{0.000000in}}{\pgfqpoint{-0.000000in}{0.000000in}}{%
\pgfpathmoveto{\pgfqpoint{-0.000000in}{0.000000in}}%
\pgfpathlineto{\pgfqpoint{-0.048611in}{0.000000in}}%
\pgfusepath{stroke,fill}%
}%
\begin{pgfscope}%
\pgfsys@transformshift{0.955704in}{2.115421in}%
\pgfsys@useobject{currentmarker}{}%
\end{pgfscope}%
\end{pgfscope}%
\begin{pgfscope}%
\definecolor{textcolor}{rgb}{0.333333,0.333333,0.333333}%
\pgfsetstrokecolor{textcolor}%
\pgfsetfillcolor{textcolor}%
\pgftext[x=0.368904in, y=2.045977in, left, base]{\color{textcolor}\rmfamily\fontsize{14.000000}{16.800000}\selectfont \(\displaystyle {20000}\)}%
\end{pgfscope}%
\begin{pgfscope}%
\pgfpathrectangle{\pgfqpoint{0.955704in}{1.263068in}}{\pgfqpoint{2.476670in}{1.940569in}}%
\pgfusepath{clip}%
\pgfsetrectcap%
\pgfsetroundjoin%
\pgfsetlinewidth{0.803000pt}%
\definecolor{currentstroke}{rgb}{1.000000,1.000000,1.000000}%
\pgfsetstrokecolor{currentstroke}%
\pgfsetdash{}{0pt}%
\pgfpathmoveto{\pgfqpoint{0.955704in}{2.990242in}}%
\pgfpathlineto{\pgfqpoint{3.432373in}{2.990242in}}%
\pgfusepath{stroke}%
\end{pgfscope}%
\begin{pgfscope}%
\pgfsetbuttcap%
\pgfsetroundjoin%
\definecolor{currentfill}{rgb}{0.333333,0.333333,0.333333}%
\pgfsetfillcolor{currentfill}%
\pgfsetlinewidth{0.803000pt}%
\definecolor{currentstroke}{rgb}{0.333333,0.333333,0.333333}%
\pgfsetstrokecolor{currentstroke}%
\pgfsetdash{}{0pt}%
\pgfsys@defobject{currentmarker}{\pgfqpoint{-0.048611in}{0.000000in}}{\pgfqpoint{-0.000000in}{0.000000in}}{%
\pgfpathmoveto{\pgfqpoint{-0.000000in}{0.000000in}}%
\pgfpathlineto{\pgfqpoint{-0.048611in}{0.000000in}}%
\pgfusepath{stroke,fill}%
}%
\begin{pgfscope}%
\pgfsys@transformshift{0.955704in}{2.990242in}%
\pgfsys@useobject{currentmarker}{}%
\end{pgfscope}%
\end{pgfscope}%
\begin{pgfscope}%
\definecolor{textcolor}{rgb}{0.333333,0.333333,0.333333}%
\pgfsetstrokecolor{textcolor}%
\pgfsetfillcolor{textcolor}%
\pgftext[x=0.368904in, y=2.920798in, left, base]{\color{textcolor}\rmfamily\fontsize{14.000000}{16.800000}\selectfont \(\displaystyle {40000}\)}%
\end{pgfscope}%
\begin{pgfscope}%
\definecolor{textcolor}{rgb}{0.333333,0.333333,0.333333}%
\pgfsetstrokecolor{textcolor}%
\pgfsetfillcolor{textcolor}%
\pgftext[x=0.313349in,y=2.233352in,,bottom,rotate=90.000000]{\color{textcolor}\rmfamily\fontsize{16.000000}{19.200000}\selectfont Tons of Refrigeration\(\displaystyle \)}%
\end{pgfscope}%
\begin{pgfscope}%
\pgfpathrectangle{\pgfqpoint{0.955704in}{1.263068in}}{\pgfqpoint{2.476670in}{1.940569in}}%
\pgfusepath{clip}%
\pgfsetbuttcap%
\pgfsetroundjoin%
\pgfsetlinewidth{1.505625pt}%
\definecolor{currentstroke}{rgb}{0.580392,0.403922,0.741176}%
\pgfsetstrokecolor{currentstroke}%
\pgfsetdash{{5.550000pt}{2.400000pt}}{0.000000pt}%
\pgfpathmoveto{\pgfqpoint{1.068280in}{1.720018in}}%
\pgfpathlineto{\pgfqpoint{1.631159in}{1.662492in}}%
\pgfpathlineto{\pgfqpoint{2.194038in}{1.662507in}}%
\pgfpathlineto{\pgfqpoint{2.756918in}{1.662488in}}%
\pgfpathlineto{\pgfqpoint{3.319797in}{1.662499in}}%
\pgfusepath{stroke}%
\end{pgfscope}%
\begin{pgfscope}%
\pgfpathrectangle{\pgfqpoint{0.955704in}{1.263068in}}{\pgfqpoint{2.476670in}{1.940569in}}%
\pgfusepath{clip}%
\pgfsetbuttcap%
\pgfsetroundjoin%
\definecolor{currentfill}{rgb}{0.580392,0.403922,0.741176}%
\pgfsetfillcolor{currentfill}%
\pgfsetlinewidth{1.003750pt}%
\definecolor{currentstroke}{rgb}{0.580392,0.403922,0.741176}%
\pgfsetstrokecolor{currentstroke}%
\pgfsetdash{}{0pt}%
\pgfsys@defobject{currentmarker}{\pgfqpoint{-0.041667in}{-0.041667in}}{\pgfqpoint{0.041667in}{0.041667in}}{%
\pgfpathmoveto{\pgfqpoint{0.000000in}{-0.041667in}}%
\pgfpathcurveto{\pgfqpoint{0.011050in}{-0.041667in}}{\pgfqpoint{0.021649in}{-0.037276in}}{\pgfqpoint{0.029463in}{-0.029463in}}%
\pgfpathcurveto{\pgfqpoint{0.037276in}{-0.021649in}}{\pgfqpoint{0.041667in}{-0.011050in}}{\pgfqpoint{0.041667in}{0.000000in}}%
\pgfpathcurveto{\pgfqpoint{0.041667in}{0.011050in}}{\pgfqpoint{0.037276in}{0.021649in}}{\pgfqpoint{0.029463in}{0.029463in}}%
\pgfpathcurveto{\pgfqpoint{0.021649in}{0.037276in}}{\pgfqpoint{0.011050in}{0.041667in}}{\pgfqpoint{0.000000in}{0.041667in}}%
\pgfpathcurveto{\pgfqpoint{-0.011050in}{0.041667in}}{\pgfqpoint{-0.021649in}{0.037276in}}{\pgfqpoint{-0.029463in}{0.029463in}}%
\pgfpathcurveto{\pgfqpoint{-0.037276in}{0.021649in}}{\pgfqpoint{-0.041667in}{0.011050in}}{\pgfqpoint{-0.041667in}{0.000000in}}%
\pgfpathcurveto{\pgfqpoint{-0.041667in}{-0.011050in}}{\pgfqpoint{-0.037276in}{-0.021649in}}{\pgfqpoint{-0.029463in}{-0.029463in}}%
\pgfpathcurveto{\pgfqpoint{-0.021649in}{-0.037276in}}{\pgfqpoint{-0.011050in}{-0.041667in}}{\pgfqpoint{0.000000in}{-0.041667in}}%
\pgfpathclose%
\pgfusepath{stroke,fill}%
}%
\begin{pgfscope}%
\pgfsys@transformshift{1.068280in}{1.720018in}%
\pgfsys@useobject{currentmarker}{}%
\end{pgfscope}%
\begin{pgfscope}%
\pgfsys@transformshift{1.631159in}{1.662492in}%
\pgfsys@useobject{currentmarker}{}%
\end{pgfscope}%
\begin{pgfscope}%
\pgfsys@transformshift{2.194038in}{1.662507in}%
\pgfsys@useobject{currentmarker}{}%
\end{pgfscope}%
\begin{pgfscope}%
\pgfsys@transformshift{2.756918in}{1.662488in}%
\pgfsys@useobject{currentmarker}{}%
\end{pgfscope}%
\begin{pgfscope}%
\pgfsys@transformshift{3.319797in}{1.662499in}%
\pgfsys@useobject{currentmarker}{}%
\end{pgfscope}%
\end{pgfscope}%
\begin{pgfscope}%
\pgfsetrectcap%
\pgfsetmiterjoin%
\pgfsetlinewidth{1.003750pt}%
\definecolor{currentstroke}{rgb}{1.000000,1.000000,1.000000}%
\pgfsetstrokecolor{currentstroke}%
\pgfsetdash{}{0pt}%
\pgfpathmoveto{\pgfqpoint{0.955704in}{1.263068in}}%
\pgfpathlineto{\pgfqpoint{0.955704in}{3.203637in}}%
\pgfusepath{stroke}%
\end{pgfscope}%
\begin{pgfscope}%
\pgfsetrectcap%
\pgfsetmiterjoin%
\pgfsetlinewidth{1.003750pt}%
\definecolor{currentstroke}{rgb}{1.000000,1.000000,1.000000}%
\pgfsetstrokecolor{currentstroke}%
\pgfsetdash{}{0pt}%
\pgfpathmoveto{\pgfqpoint{3.432373in}{1.263068in}}%
\pgfpathlineto{\pgfqpoint{3.432373in}{3.203637in}}%
\pgfusepath{stroke}%
\end{pgfscope}%
\begin{pgfscope}%
\pgfsetrectcap%
\pgfsetmiterjoin%
\pgfsetlinewidth{1.003750pt}%
\definecolor{currentstroke}{rgb}{1.000000,1.000000,1.000000}%
\pgfsetstrokecolor{currentstroke}%
\pgfsetdash{}{0pt}%
\pgfpathmoveto{\pgfqpoint{0.955704in}{1.263068in}}%
\pgfpathlineto{\pgfqpoint{3.432373in}{1.263068in}}%
\pgfusepath{stroke}%
\end{pgfscope}%
\begin{pgfscope}%
\pgfsetrectcap%
\pgfsetmiterjoin%
\pgfsetlinewidth{1.003750pt}%
\definecolor{currentstroke}{rgb}{1.000000,1.000000,1.000000}%
\pgfsetstrokecolor{currentstroke}%
\pgfsetdash{}{0pt}%
\pgfpathmoveto{\pgfqpoint{0.955704in}{3.203637in}}%
\pgfpathlineto{\pgfqpoint{3.432373in}{3.203637in}}%
\pgfusepath{stroke}%
\end{pgfscope}%
\begin{pgfscope}%
\pgfsetbuttcap%
\pgfsetmiterjoin%
\definecolor{currentfill}{rgb}{0.898039,0.898039,0.898039}%
\pgfsetfillcolor{currentfill}%
\pgfsetlinewidth{0.000000pt}%
\definecolor{currentstroke}{rgb}{0.000000,0.000000,0.000000}%
\pgfsetstrokecolor{currentstroke}%
\pgfsetstrokeopacity{0.000000}%
\pgfsetdash{}{0pt}%
\pgfpathmoveto{\pgfqpoint{3.652320in}{1.263068in}}%
\pgfpathlineto{\pgfqpoint{6.128990in}{1.263068in}}%
\pgfpathlineto{\pgfqpoint{6.128990in}{3.203637in}}%
\pgfpathlineto{\pgfqpoint{3.652320in}{3.203637in}}%
\pgfpathclose%
\pgfusepath{fill}%
\end{pgfscope}%
\begin{pgfscope}%
\pgfpathrectangle{\pgfqpoint{3.652320in}{1.263068in}}{\pgfqpoint{2.476670in}{1.940569in}}%
\pgfusepath{clip}%
\pgfsetrectcap%
\pgfsetroundjoin%
\pgfsetlinewidth{0.803000pt}%
\definecolor{currentstroke}{rgb}{1.000000,1.000000,1.000000}%
\pgfsetstrokecolor{currentstroke}%
\pgfsetstrokeopacity{0.000000}%
\pgfsetdash{}{0pt}%
\pgfpathmoveto{\pgfqpoint{3.764896in}{1.263068in}}%
\pgfpathlineto{\pgfqpoint{3.764896in}{3.203637in}}%
\pgfusepath{stroke}%
\end{pgfscope}%
\begin{pgfscope}%
\pgfsetbuttcap%
\pgfsetroundjoin%
\definecolor{currentfill}{rgb}{0.333333,0.333333,0.333333}%
\pgfsetfillcolor{currentfill}%
\pgfsetlinewidth{0.803000pt}%
\definecolor{currentstroke}{rgb}{0.333333,0.333333,0.333333}%
\pgfsetstrokecolor{currentstroke}%
\pgfsetdash{}{0pt}%
\pgfsys@defobject{currentmarker}{\pgfqpoint{0.000000in}{-0.048611in}}{\pgfqpoint{0.000000in}{0.000000in}}{%
\pgfpathmoveto{\pgfqpoint{0.000000in}{0.000000in}}%
\pgfpathlineto{\pgfqpoint{0.000000in}{-0.048611in}}%
\pgfusepath{stroke,fill}%
}%
\begin{pgfscope}%
\pgfsys@transformshift{3.764896in}{1.263068in}%
\pgfsys@useobject{currentmarker}{}%
\end{pgfscope}%
\end{pgfscope}%
\begin{pgfscope}%
\definecolor{textcolor}{rgb}{0.333333,0.333333,0.333333}%
\pgfsetstrokecolor{textcolor}%
\pgfsetfillcolor{textcolor}%
\pgftext[x=3.831447in, y=0.210068in, left, base,rotate=90.000000]{\color{textcolor}\rmfamily\fontsize{16.000000}{19.200000}\selectfont mga-0-5\%}%
\end{pgfscope}%
\begin{pgfscope}%
\pgfpathrectangle{\pgfqpoint{3.652320in}{1.263068in}}{\pgfqpoint{2.476670in}{1.940569in}}%
\pgfusepath{clip}%
\pgfsetrectcap%
\pgfsetroundjoin%
\pgfsetlinewidth{0.803000pt}%
\definecolor{currentstroke}{rgb}{1.000000,1.000000,1.000000}%
\pgfsetstrokecolor{currentstroke}%
\pgfsetstrokeopacity{0.000000}%
\pgfsetdash{}{0pt}%
\pgfpathmoveto{\pgfqpoint{4.327775in}{1.263068in}}%
\pgfpathlineto{\pgfqpoint{4.327775in}{3.203637in}}%
\pgfusepath{stroke}%
\end{pgfscope}%
\begin{pgfscope}%
\pgfsetbuttcap%
\pgfsetroundjoin%
\definecolor{currentfill}{rgb}{0.333333,0.333333,0.333333}%
\pgfsetfillcolor{currentfill}%
\pgfsetlinewidth{0.803000pt}%
\definecolor{currentstroke}{rgb}{0.333333,0.333333,0.333333}%
\pgfsetstrokecolor{currentstroke}%
\pgfsetdash{}{0pt}%
\pgfsys@defobject{currentmarker}{\pgfqpoint{0.000000in}{-0.048611in}}{\pgfqpoint{0.000000in}{0.000000in}}{%
\pgfpathmoveto{\pgfqpoint{0.000000in}{0.000000in}}%
\pgfpathlineto{\pgfqpoint{0.000000in}{-0.048611in}}%
\pgfusepath{stroke,fill}%
}%
\begin{pgfscope}%
\pgfsys@transformshift{4.327775in}{1.263068in}%
\pgfsys@useobject{currentmarker}{}%
\end{pgfscope}%
\end{pgfscope}%
\begin{pgfscope}%
\definecolor{textcolor}{rgb}{0.333333,0.333333,0.333333}%
\pgfsetstrokecolor{textcolor}%
\pgfsetfillcolor{textcolor}%
\pgftext[x=4.394326in, y=0.210068in, left, base,rotate=90.000000]{\color{textcolor}\rmfamily\fontsize{16.000000}{19.200000}\selectfont mga-1-5\%}%
\end{pgfscope}%
\begin{pgfscope}%
\pgfpathrectangle{\pgfqpoint{3.652320in}{1.263068in}}{\pgfqpoint{2.476670in}{1.940569in}}%
\pgfusepath{clip}%
\pgfsetrectcap%
\pgfsetroundjoin%
\pgfsetlinewidth{0.803000pt}%
\definecolor{currentstroke}{rgb}{1.000000,1.000000,1.000000}%
\pgfsetstrokecolor{currentstroke}%
\pgfsetstrokeopacity{0.000000}%
\pgfsetdash{}{0pt}%
\pgfpathmoveto{\pgfqpoint{4.890655in}{1.263068in}}%
\pgfpathlineto{\pgfqpoint{4.890655in}{3.203637in}}%
\pgfusepath{stroke}%
\end{pgfscope}%
\begin{pgfscope}%
\pgfsetbuttcap%
\pgfsetroundjoin%
\definecolor{currentfill}{rgb}{0.333333,0.333333,0.333333}%
\pgfsetfillcolor{currentfill}%
\pgfsetlinewidth{0.803000pt}%
\definecolor{currentstroke}{rgb}{0.333333,0.333333,0.333333}%
\pgfsetstrokecolor{currentstroke}%
\pgfsetdash{}{0pt}%
\pgfsys@defobject{currentmarker}{\pgfqpoint{0.000000in}{-0.048611in}}{\pgfqpoint{0.000000in}{0.000000in}}{%
\pgfpathmoveto{\pgfqpoint{0.000000in}{0.000000in}}%
\pgfpathlineto{\pgfqpoint{0.000000in}{-0.048611in}}%
\pgfusepath{stroke,fill}%
}%
\begin{pgfscope}%
\pgfsys@transformshift{4.890655in}{1.263068in}%
\pgfsys@useobject{currentmarker}{}%
\end{pgfscope}%
\end{pgfscope}%
\begin{pgfscope}%
\definecolor{textcolor}{rgb}{0.333333,0.333333,0.333333}%
\pgfsetstrokecolor{textcolor}%
\pgfsetfillcolor{textcolor}%
\pgftext[x=4.957206in, y=0.210068in, left, base,rotate=90.000000]{\color{textcolor}\rmfamily\fontsize{16.000000}{19.200000}\selectfont mga-2-5\%}%
\end{pgfscope}%
\begin{pgfscope}%
\pgfpathrectangle{\pgfqpoint{3.652320in}{1.263068in}}{\pgfqpoint{2.476670in}{1.940569in}}%
\pgfusepath{clip}%
\pgfsetrectcap%
\pgfsetroundjoin%
\pgfsetlinewidth{0.803000pt}%
\definecolor{currentstroke}{rgb}{1.000000,1.000000,1.000000}%
\pgfsetstrokecolor{currentstroke}%
\pgfsetstrokeopacity{0.000000}%
\pgfsetdash{}{0pt}%
\pgfpathmoveto{\pgfqpoint{5.453534in}{1.263068in}}%
\pgfpathlineto{\pgfqpoint{5.453534in}{3.203637in}}%
\pgfusepath{stroke}%
\end{pgfscope}%
\begin{pgfscope}%
\pgfsetbuttcap%
\pgfsetroundjoin%
\definecolor{currentfill}{rgb}{0.333333,0.333333,0.333333}%
\pgfsetfillcolor{currentfill}%
\pgfsetlinewidth{0.803000pt}%
\definecolor{currentstroke}{rgb}{0.333333,0.333333,0.333333}%
\pgfsetstrokecolor{currentstroke}%
\pgfsetdash{}{0pt}%
\pgfsys@defobject{currentmarker}{\pgfqpoint{0.000000in}{-0.048611in}}{\pgfqpoint{0.000000in}{0.000000in}}{%
\pgfpathmoveto{\pgfqpoint{0.000000in}{0.000000in}}%
\pgfpathlineto{\pgfqpoint{0.000000in}{-0.048611in}}%
\pgfusepath{stroke,fill}%
}%
\begin{pgfscope}%
\pgfsys@transformshift{5.453534in}{1.263068in}%
\pgfsys@useobject{currentmarker}{}%
\end{pgfscope}%
\end{pgfscope}%
\begin{pgfscope}%
\definecolor{textcolor}{rgb}{0.333333,0.333333,0.333333}%
\pgfsetstrokecolor{textcolor}%
\pgfsetfillcolor{textcolor}%
\pgftext[x=5.520085in, y=0.210068in, left, base,rotate=90.000000]{\color{textcolor}\rmfamily\fontsize{16.000000}{19.200000}\selectfont mga-3-5\%}%
\end{pgfscope}%
\begin{pgfscope}%
\pgfpathrectangle{\pgfqpoint{3.652320in}{1.263068in}}{\pgfqpoint{2.476670in}{1.940569in}}%
\pgfusepath{clip}%
\pgfsetrectcap%
\pgfsetroundjoin%
\pgfsetlinewidth{0.803000pt}%
\definecolor{currentstroke}{rgb}{1.000000,1.000000,1.000000}%
\pgfsetstrokecolor{currentstroke}%
\pgfsetstrokeopacity{0.000000}%
\pgfsetdash{}{0pt}%
\pgfpathmoveto{\pgfqpoint{6.016414in}{1.263068in}}%
\pgfpathlineto{\pgfqpoint{6.016414in}{3.203637in}}%
\pgfusepath{stroke}%
\end{pgfscope}%
\begin{pgfscope}%
\pgfsetbuttcap%
\pgfsetroundjoin%
\definecolor{currentfill}{rgb}{0.333333,0.333333,0.333333}%
\pgfsetfillcolor{currentfill}%
\pgfsetlinewidth{0.803000pt}%
\definecolor{currentstroke}{rgb}{0.333333,0.333333,0.333333}%
\pgfsetstrokecolor{currentstroke}%
\pgfsetdash{}{0pt}%
\pgfsys@defobject{currentmarker}{\pgfqpoint{0.000000in}{-0.048611in}}{\pgfqpoint{0.000000in}{0.000000in}}{%
\pgfpathmoveto{\pgfqpoint{0.000000in}{0.000000in}}%
\pgfpathlineto{\pgfqpoint{0.000000in}{-0.048611in}}%
\pgfusepath{stroke,fill}%
}%
\begin{pgfscope}%
\pgfsys@transformshift{6.016414in}{1.263068in}%
\pgfsys@useobject{currentmarker}{}%
\end{pgfscope}%
\end{pgfscope}%
\begin{pgfscope}%
\definecolor{textcolor}{rgb}{0.333333,0.333333,0.333333}%
\pgfsetstrokecolor{textcolor}%
\pgfsetfillcolor{textcolor}%
\pgftext[x=6.082965in, y=0.210068in, left, base,rotate=90.000000]{\color{textcolor}\rmfamily\fontsize{16.000000}{19.200000}\selectfont mga-4-5\%}%
\end{pgfscope}%
\begin{pgfscope}%
\pgfpathrectangle{\pgfqpoint{3.652320in}{1.263068in}}{\pgfqpoint{2.476670in}{1.940569in}}%
\pgfusepath{clip}%
\pgfsetrectcap%
\pgfsetroundjoin%
\pgfsetlinewidth{0.803000pt}%
\definecolor{currentstroke}{rgb}{1.000000,1.000000,1.000000}%
\pgfsetstrokecolor{currentstroke}%
\pgfsetdash{}{0pt}%
\pgfpathmoveto{\pgfqpoint{3.652320in}{2.115421in}}%
\pgfpathlineto{\pgfqpoint{6.128990in}{2.115421in}}%
\pgfusepath{stroke}%
\end{pgfscope}%
\begin{pgfscope}%
\pgfsetbuttcap%
\pgfsetroundjoin%
\definecolor{currentfill}{rgb}{0.333333,0.333333,0.333333}%
\pgfsetfillcolor{currentfill}%
\pgfsetlinewidth{0.803000pt}%
\definecolor{currentstroke}{rgb}{0.333333,0.333333,0.333333}%
\pgfsetstrokecolor{currentstroke}%
\pgfsetdash{}{0pt}%
\pgfsys@defobject{currentmarker}{\pgfqpoint{-0.048611in}{0.000000in}}{\pgfqpoint{-0.000000in}{0.000000in}}{%
\pgfpathmoveto{\pgfqpoint{-0.000000in}{0.000000in}}%
\pgfpathlineto{\pgfqpoint{-0.048611in}{0.000000in}}%
\pgfusepath{stroke,fill}%
}%
\begin{pgfscope}%
\pgfsys@transformshift{3.652320in}{2.115421in}%
\pgfsys@useobject{currentmarker}{}%
\end{pgfscope}%
\end{pgfscope}%
\begin{pgfscope}%
\pgfpathrectangle{\pgfqpoint{3.652320in}{1.263068in}}{\pgfqpoint{2.476670in}{1.940569in}}%
\pgfusepath{clip}%
\pgfsetrectcap%
\pgfsetroundjoin%
\pgfsetlinewidth{0.803000pt}%
\definecolor{currentstroke}{rgb}{1.000000,1.000000,1.000000}%
\pgfsetstrokecolor{currentstroke}%
\pgfsetdash{}{0pt}%
\pgfpathmoveto{\pgfqpoint{3.652320in}{2.990242in}}%
\pgfpathlineto{\pgfqpoint{6.128990in}{2.990242in}}%
\pgfusepath{stroke}%
\end{pgfscope}%
\begin{pgfscope}%
\pgfsetbuttcap%
\pgfsetroundjoin%
\definecolor{currentfill}{rgb}{0.333333,0.333333,0.333333}%
\pgfsetfillcolor{currentfill}%
\pgfsetlinewidth{0.803000pt}%
\definecolor{currentstroke}{rgb}{0.333333,0.333333,0.333333}%
\pgfsetstrokecolor{currentstroke}%
\pgfsetdash{}{0pt}%
\pgfsys@defobject{currentmarker}{\pgfqpoint{-0.048611in}{0.000000in}}{\pgfqpoint{-0.000000in}{0.000000in}}{%
\pgfpathmoveto{\pgfqpoint{-0.000000in}{0.000000in}}%
\pgfpathlineto{\pgfqpoint{-0.048611in}{0.000000in}}%
\pgfusepath{stroke,fill}%
}%
\begin{pgfscope}%
\pgfsys@transformshift{3.652320in}{2.990242in}%
\pgfsys@useobject{currentmarker}{}%
\end{pgfscope}%
\end{pgfscope}%
\begin{pgfscope}%
\pgfpathrectangle{\pgfqpoint{3.652320in}{1.263068in}}{\pgfqpoint{2.476670in}{1.940569in}}%
\pgfusepath{clip}%
\pgfsetbuttcap%
\pgfsetroundjoin%
\pgfsetlinewidth{1.505625pt}%
\definecolor{currentstroke}{rgb}{0.580392,0.403922,0.741176}%
\pgfsetstrokecolor{currentstroke}%
\pgfsetdash{{5.550000pt}{2.400000pt}}{0.000000pt}%
\pgfpathmoveto{\pgfqpoint{3.764896in}{1.720018in}}%
\pgfpathlineto{\pgfqpoint{4.327775in}{1.624145in}}%
\pgfpathlineto{\pgfqpoint{4.890655in}{1.624145in}}%
\pgfpathlineto{\pgfqpoint{5.453534in}{1.624166in}}%
\pgfpathlineto{\pgfqpoint{6.016414in}{1.624153in}}%
\pgfusepath{stroke}%
\end{pgfscope}%
\begin{pgfscope}%
\pgfpathrectangle{\pgfqpoint{3.652320in}{1.263068in}}{\pgfqpoint{2.476670in}{1.940569in}}%
\pgfusepath{clip}%
\pgfsetbuttcap%
\pgfsetroundjoin%
\definecolor{currentfill}{rgb}{0.580392,0.403922,0.741176}%
\pgfsetfillcolor{currentfill}%
\pgfsetlinewidth{1.003750pt}%
\definecolor{currentstroke}{rgb}{0.580392,0.403922,0.741176}%
\pgfsetstrokecolor{currentstroke}%
\pgfsetdash{}{0pt}%
\pgfsys@defobject{currentmarker}{\pgfqpoint{-0.041667in}{-0.041667in}}{\pgfqpoint{0.041667in}{0.041667in}}{%
\pgfpathmoveto{\pgfqpoint{0.000000in}{-0.041667in}}%
\pgfpathcurveto{\pgfqpoint{0.011050in}{-0.041667in}}{\pgfqpoint{0.021649in}{-0.037276in}}{\pgfqpoint{0.029463in}{-0.029463in}}%
\pgfpathcurveto{\pgfqpoint{0.037276in}{-0.021649in}}{\pgfqpoint{0.041667in}{-0.011050in}}{\pgfqpoint{0.041667in}{0.000000in}}%
\pgfpathcurveto{\pgfqpoint{0.041667in}{0.011050in}}{\pgfqpoint{0.037276in}{0.021649in}}{\pgfqpoint{0.029463in}{0.029463in}}%
\pgfpathcurveto{\pgfqpoint{0.021649in}{0.037276in}}{\pgfqpoint{0.011050in}{0.041667in}}{\pgfqpoint{0.000000in}{0.041667in}}%
\pgfpathcurveto{\pgfqpoint{-0.011050in}{0.041667in}}{\pgfqpoint{-0.021649in}{0.037276in}}{\pgfqpoint{-0.029463in}{0.029463in}}%
\pgfpathcurveto{\pgfqpoint{-0.037276in}{0.021649in}}{\pgfqpoint{-0.041667in}{0.011050in}}{\pgfqpoint{-0.041667in}{0.000000in}}%
\pgfpathcurveto{\pgfqpoint{-0.041667in}{-0.011050in}}{\pgfqpoint{-0.037276in}{-0.021649in}}{\pgfqpoint{-0.029463in}{-0.029463in}}%
\pgfpathcurveto{\pgfqpoint{-0.021649in}{-0.037276in}}{\pgfqpoint{-0.011050in}{-0.041667in}}{\pgfqpoint{0.000000in}{-0.041667in}}%
\pgfpathclose%
\pgfusepath{stroke,fill}%
}%
\begin{pgfscope}%
\pgfsys@transformshift{3.764896in}{1.720018in}%
\pgfsys@useobject{currentmarker}{}%
\end{pgfscope}%
\begin{pgfscope}%
\pgfsys@transformshift{4.327775in}{1.624145in}%
\pgfsys@useobject{currentmarker}{}%
\end{pgfscope}%
\begin{pgfscope}%
\pgfsys@transformshift{4.890655in}{1.624145in}%
\pgfsys@useobject{currentmarker}{}%
\end{pgfscope}%
\begin{pgfscope}%
\pgfsys@transformshift{5.453534in}{1.624166in}%
\pgfsys@useobject{currentmarker}{}%
\end{pgfscope}%
\begin{pgfscope}%
\pgfsys@transformshift{6.016414in}{1.624153in}%
\pgfsys@useobject{currentmarker}{}%
\end{pgfscope}%
\end{pgfscope}%
\begin{pgfscope}%
\pgfsetrectcap%
\pgfsetmiterjoin%
\pgfsetlinewidth{1.003750pt}%
\definecolor{currentstroke}{rgb}{1.000000,1.000000,1.000000}%
\pgfsetstrokecolor{currentstroke}%
\pgfsetdash{}{0pt}%
\pgfpathmoveto{\pgfqpoint{3.652320in}{1.263068in}}%
\pgfpathlineto{\pgfqpoint{3.652320in}{3.203637in}}%
\pgfusepath{stroke}%
\end{pgfscope}%
\begin{pgfscope}%
\pgfsetrectcap%
\pgfsetmiterjoin%
\pgfsetlinewidth{1.003750pt}%
\definecolor{currentstroke}{rgb}{1.000000,1.000000,1.000000}%
\pgfsetstrokecolor{currentstroke}%
\pgfsetdash{}{0pt}%
\pgfpathmoveto{\pgfqpoint{6.128990in}{1.263068in}}%
\pgfpathlineto{\pgfqpoint{6.128990in}{3.203637in}}%
\pgfusepath{stroke}%
\end{pgfscope}%
\begin{pgfscope}%
\pgfsetrectcap%
\pgfsetmiterjoin%
\pgfsetlinewidth{1.003750pt}%
\definecolor{currentstroke}{rgb}{1.000000,1.000000,1.000000}%
\pgfsetstrokecolor{currentstroke}%
\pgfsetdash{}{0pt}%
\pgfpathmoveto{\pgfqpoint{3.652320in}{1.263068in}}%
\pgfpathlineto{\pgfqpoint{6.128990in}{1.263068in}}%
\pgfusepath{stroke}%
\end{pgfscope}%
\begin{pgfscope}%
\pgfsetrectcap%
\pgfsetmiterjoin%
\pgfsetlinewidth{1.003750pt}%
\definecolor{currentstroke}{rgb}{1.000000,1.000000,1.000000}%
\pgfsetstrokecolor{currentstroke}%
\pgfsetdash{}{0pt}%
\pgfpathmoveto{\pgfqpoint{3.652320in}{3.203637in}}%
\pgfpathlineto{\pgfqpoint{6.128990in}{3.203637in}}%
\pgfusepath{stroke}%
\end{pgfscope}%
\begin{pgfscope}%
\pgfsetbuttcap%
\pgfsetmiterjoin%
\definecolor{currentfill}{rgb}{0.898039,0.898039,0.898039}%
\pgfsetfillcolor{currentfill}%
\pgfsetlinewidth{0.000000pt}%
\definecolor{currentstroke}{rgb}{0.000000,0.000000,0.000000}%
\pgfsetstrokecolor{currentstroke}%
\pgfsetstrokeopacity{0.000000}%
\pgfsetdash{}{0pt}%
\pgfpathmoveto{\pgfqpoint{6.348936in}{1.263068in}}%
\pgfpathlineto{\pgfqpoint{8.825606in}{1.263068in}}%
\pgfpathlineto{\pgfqpoint{8.825606in}{3.203637in}}%
\pgfpathlineto{\pgfqpoint{6.348936in}{3.203637in}}%
\pgfpathclose%
\pgfusepath{fill}%
\end{pgfscope}%
\begin{pgfscope}%
\pgfpathrectangle{\pgfqpoint{6.348936in}{1.263068in}}{\pgfqpoint{2.476670in}{1.940569in}}%
\pgfusepath{clip}%
\pgfsetrectcap%
\pgfsetroundjoin%
\pgfsetlinewidth{0.803000pt}%
\definecolor{currentstroke}{rgb}{1.000000,1.000000,1.000000}%
\pgfsetstrokecolor{currentstroke}%
\pgfsetstrokeopacity{0.000000}%
\pgfsetdash{}{0pt}%
\pgfpathmoveto{\pgfqpoint{6.461512in}{1.263068in}}%
\pgfpathlineto{\pgfqpoint{6.461512in}{3.203637in}}%
\pgfusepath{stroke}%
\end{pgfscope}%
\begin{pgfscope}%
\pgfsetbuttcap%
\pgfsetroundjoin%
\definecolor{currentfill}{rgb}{0.333333,0.333333,0.333333}%
\pgfsetfillcolor{currentfill}%
\pgfsetlinewidth{0.803000pt}%
\definecolor{currentstroke}{rgb}{0.333333,0.333333,0.333333}%
\pgfsetstrokecolor{currentstroke}%
\pgfsetdash{}{0pt}%
\pgfsys@defobject{currentmarker}{\pgfqpoint{0.000000in}{-0.048611in}}{\pgfqpoint{0.000000in}{0.000000in}}{%
\pgfpathmoveto{\pgfqpoint{0.000000in}{0.000000in}}%
\pgfpathlineto{\pgfqpoint{0.000000in}{-0.048611in}}%
\pgfusepath{stroke,fill}%
}%
\begin{pgfscope}%
\pgfsys@transformshift{6.461512in}{1.263068in}%
\pgfsys@useobject{currentmarker}{}%
\end{pgfscope}%
\end{pgfscope}%
\begin{pgfscope}%
\definecolor{textcolor}{rgb}{0.333333,0.333333,0.333333}%
\pgfsetstrokecolor{textcolor}%
\pgfsetfillcolor{textcolor}%
\pgftext[x=6.528063in, y=0.100000in, left, base,rotate=90.000000]{\color{textcolor}\rmfamily\fontsize{16.000000}{19.200000}\selectfont mga-0-10\%}%
\end{pgfscope}%
\begin{pgfscope}%
\pgfpathrectangle{\pgfqpoint{6.348936in}{1.263068in}}{\pgfqpoint{2.476670in}{1.940569in}}%
\pgfusepath{clip}%
\pgfsetrectcap%
\pgfsetroundjoin%
\pgfsetlinewidth{0.803000pt}%
\definecolor{currentstroke}{rgb}{1.000000,1.000000,1.000000}%
\pgfsetstrokecolor{currentstroke}%
\pgfsetstrokeopacity{0.000000}%
\pgfsetdash{}{0pt}%
\pgfpathmoveto{\pgfqpoint{7.024392in}{1.263068in}}%
\pgfpathlineto{\pgfqpoint{7.024392in}{3.203637in}}%
\pgfusepath{stroke}%
\end{pgfscope}%
\begin{pgfscope}%
\pgfsetbuttcap%
\pgfsetroundjoin%
\definecolor{currentfill}{rgb}{0.333333,0.333333,0.333333}%
\pgfsetfillcolor{currentfill}%
\pgfsetlinewidth{0.803000pt}%
\definecolor{currentstroke}{rgb}{0.333333,0.333333,0.333333}%
\pgfsetstrokecolor{currentstroke}%
\pgfsetdash{}{0pt}%
\pgfsys@defobject{currentmarker}{\pgfqpoint{0.000000in}{-0.048611in}}{\pgfqpoint{0.000000in}{0.000000in}}{%
\pgfpathmoveto{\pgfqpoint{0.000000in}{0.000000in}}%
\pgfpathlineto{\pgfqpoint{0.000000in}{-0.048611in}}%
\pgfusepath{stroke,fill}%
}%
\begin{pgfscope}%
\pgfsys@transformshift{7.024392in}{1.263068in}%
\pgfsys@useobject{currentmarker}{}%
\end{pgfscope}%
\end{pgfscope}%
\begin{pgfscope}%
\definecolor{textcolor}{rgb}{0.333333,0.333333,0.333333}%
\pgfsetstrokecolor{textcolor}%
\pgfsetfillcolor{textcolor}%
\pgftext[x=7.090943in, y=0.100000in, left, base,rotate=90.000000]{\color{textcolor}\rmfamily\fontsize{16.000000}{19.200000}\selectfont mga-1-10\%}%
\end{pgfscope}%
\begin{pgfscope}%
\pgfpathrectangle{\pgfqpoint{6.348936in}{1.263068in}}{\pgfqpoint{2.476670in}{1.940569in}}%
\pgfusepath{clip}%
\pgfsetrectcap%
\pgfsetroundjoin%
\pgfsetlinewidth{0.803000pt}%
\definecolor{currentstroke}{rgb}{1.000000,1.000000,1.000000}%
\pgfsetstrokecolor{currentstroke}%
\pgfsetstrokeopacity{0.000000}%
\pgfsetdash{}{0pt}%
\pgfpathmoveto{\pgfqpoint{7.587271in}{1.263068in}}%
\pgfpathlineto{\pgfqpoint{7.587271in}{3.203637in}}%
\pgfusepath{stroke}%
\end{pgfscope}%
\begin{pgfscope}%
\pgfsetbuttcap%
\pgfsetroundjoin%
\definecolor{currentfill}{rgb}{0.333333,0.333333,0.333333}%
\pgfsetfillcolor{currentfill}%
\pgfsetlinewidth{0.803000pt}%
\definecolor{currentstroke}{rgb}{0.333333,0.333333,0.333333}%
\pgfsetstrokecolor{currentstroke}%
\pgfsetdash{}{0pt}%
\pgfsys@defobject{currentmarker}{\pgfqpoint{0.000000in}{-0.048611in}}{\pgfqpoint{0.000000in}{0.000000in}}{%
\pgfpathmoveto{\pgfqpoint{0.000000in}{0.000000in}}%
\pgfpathlineto{\pgfqpoint{0.000000in}{-0.048611in}}%
\pgfusepath{stroke,fill}%
}%
\begin{pgfscope}%
\pgfsys@transformshift{7.587271in}{1.263068in}%
\pgfsys@useobject{currentmarker}{}%
\end{pgfscope}%
\end{pgfscope}%
\begin{pgfscope}%
\definecolor{textcolor}{rgb}{0.333333,0.333333,0.333333}%
\pgfsetstrokecolor{textcolor}%
\pgfsetfillcolor{textcolor}%
\pgftext[x=7.653822in, y=0.100000in, left, base,rotate=90.000000]{\color{textcolor}\rmfamily\fontsize{16.000000}{19.200000}\selectfont mga-2-10\%}%
\end{pgfscope}%
\begin{pgfscope}%
\pgfpathrectangle{\pgfqpoint{6.348936in}{1.263068in}}{\pgfqpoint{2.476670in}{1.940569in}}%
\pgfusepath{clip}%
\pgfsetrectcap%
\pgfsetroundjoin%
\pgfsetlinewidth{0.803000pt}%
\definecolor{currentstroke}{rgb}{1.000000,1.000000,1.000000}%
\pgfsetstrokecolor{currentstroke}%
\pgfsetstrokeopacity{0.000000}%
\pgfsetdash{}{0pt}%
\pgfpathmoveto{\pgfqpoint{8.150151in}{1.263068in}}%
\pgfpathlineto{\pgfqpoint{8.150151in}{3.203637in}}%
\pgfusepath{stroke}%
\end{pgfscope}%
\begin{pgfscope}%
\pgfsetbuttcap%
\pgfsetroundjoin%
\definecolor{currentfill}{rgb}{0.333333,0.333333,0.333333}%
\pgfsetfillcolor{currentfill}%
\pgfsetlinewidth{0.803000pt}%
\definecolor{currentstroke}{rgb}{0.333333,0.333333,0.333333}%
\pgfsetstrokecolor{currentstroke}%
\pgfsetdash{}{0pt}%
\pgfsys@defobject{currentmarker}{\pgfqpoint{0.000000in}{-0.048611in}}{\pgfqpoint{0.000000in}{0.000000in}}{%
\pgfpathmoveto{\pgfqpoint{0.000000in}{0.000000in}}%
\pgfpathlineto{\pgfqpoint{0.000000in}{-0.048611in}}%
\pgfusepath{stroke,fill}%
}%
\begin{pgfscope}%
\pgfsys@transformshift{8.150151in}{1.263068in}%
\pgfsys@useobject{currentmarker}{}%
\end{pgfscope}%
\end{pgfscope}%
\begin{pgfscope}%
\definecolor{textcolor}{rgb}{0.333333,0.333333,0.333333}%
\pgfsetstrokecolor{textcolor}%
\pgfsetfillcolor{textcolor}%
\pgftext[x=8.216701in, y=0.100000in, left, base,rotate=90.000000]{\color{textcolor}\rmfamily\fontsize{16.000000}{19.200000}\selectfont mga-3-10\%}%
\end{pgfscope}%
\begin{pgfscope}%
\pgfpathrectangle{\pgfqpoint{6.348936in}{1.263068in}}{\pgfqpoint{2.476670in}{1.940569in}}%
\pgfusepath{clip}%
\pgfsetrectcap%
\pgfsetroundjoin%
\pgfsetlinewidth{0.803000pt}%
\definecolor{currentstroke}{rgb}{1.000000,1.000000,1.000000}%
\pgfsetstrokecolor{currentstroke}%
\pgfsetstrokeopacity{0.000000}%
\pgfsetdash{}{0pt}%
\pgfpathmoveto{\pgfqpoint{8.713030in}{1.263068in}}%
\pgfpathlineto{\pgfqpoint{8.713030in}{3.203637in}}%
\pgfusepath{stroke}%
\end{pgfscope}%
\begin{pgfscope}%
\pgfsetbuttcap%
\pgfsetroundjoin%
\definecolor{currentfill}{rgb}{0.333333,0.333333,0.333333}%
\pgfsetfillcolor{currentfill}%
\pgfsetlinewidth{0.803000pt}%
\definecolor{currentstroke}{rgb}{0.333333,0.333333,0.333333}%
\pgfsetstrokecolor{currentstroke}%
\pgfsetdash{}{0pt}%
\pgfsys@defobject{currentmarker}{\pgfqpoint{0.000000in}{-0.048611in}}{\pgfqpoint{0.000000in}{0.000000in}}{%
\pgfpathmoveto{\pgfqpoint{0.000000in}{0.000000in}}%
\pgfpathlineto{\pgfqpoint{0.000000in}{-0.048611in}}%
\pgfusepath{stroke,fill}%
}%
\begin{pgfscope}%
\pgfsys@transformshift{8.713030in}{1.263068in}%
\pgfsys@useobject{currentmarker}{}%
\end{pgfscope}%
\end{pgfscope}%
\begin{pgfscope}%
\definecolor{textcolor}{rgb}{0.333333,0.333333,0.333333}%
\pgfsetstrokecolor{textcolor}%
\pgfsetfillcolor{textcolor}%
\pgftext[x=8.779581in, y=0.100000in, left, base,rotate=90.000000]{\color{textcolor}\rmfamily\fontsize{16.000000}{19.200000}\selectfont mga-4-10\%}%
\end{pgfscope}%
\begin{pgfscope}%
\pgfpathrectangle{\pgfqpoint{6.348936in}{1.263068in}}{\pgfqpoint{2.476670in}{1.940569in}}%
\pgfusepath{clip}%
\pgfsetrectcap%
\pgfsetroundjoin%
\pgfsetlinewidth{0.803000pt}%
\definecolor{currentstroke}{rgb}{1.000000,1.000000,1.000000}%
\pgfsetstrokecolor{currentstroke}%
\pgfsetdash{}{0pt}%
\pgfpathmoveto{\pgfqpoint{6.348936in}{2.115421in}}%
\pgfpathlineto{\pgfqpoint{8.825606in}{2.115421in}}%
\pgfusepath{stroke}%
\end{pgfscope}%
\begin{pgfscope}%
\pgfsetbuttcap%
\pgfsetroundjoin%
\definecolor{currentfill}{rgb}{0.333333,0.333333,0.333333}%
\pgfsetfillcolor{currentfill}%
\pgfsetlinewidth{0.803000pt}%
\definecolor{currentstroke}{rgb}{0.333333,0.333333,0.333333}%
\pgfsetstrokecolor{currentstroke}%
\pgfsetdash{}{0pt}%
\pgfsys@defobject{currentmarker}{\pgfqpoint{-0.048611in}{0.000000in}}{\pgfqpoint{-0.000000in}{0.000000in}}{%
\pgfpathmoveto{\pgfqpoint{-0.000000in}{0.000000in}}%
\pgfpathlineto{\pgfqpoint{-0.048611in}{0.000000in}}%
\pgfusepath{stroke,fill}%
}%
\begin{pgfscope}%
\pgfsys@transformshift{6.348936in}{2.115421in}%
\pgfsys@useobject{currentmarker}{}%
\end{pgfscope}%
\end{pgfscope}%
\begin{pgfscope}%
\pgfpathrectangle{\pgfqpoint{6.348936in}{1.263068in}}{\pgfqpoint{2.476670in}{1.940569in}}%
\pgfusepath{clip}%
\pgfsetrectcap%
\pgfsetroundjoin%
\pgfsetlinewidth{0.803000pt}%
\definecolor{currentstroke}{rgb}{1.000000,1.000000,1.000000}%
\pgfsetstrokecolor{currentstroke}%
\pgfsetdash{}{0pt}%
\pgfpathmoveto{\pgfqpoint{6.348936in}{2.990242in}}%
\pgfpathlineto{\pgfqpoint{8.825606in}{2.990242in}}%
\pgfusepath{stroke}%
\end{pgfscope}%
\begin{pgfscope}%
\pgfsetbuttcap%
\pgfsetroundjoin%
\definecolor{currentfill}{rgb}{0.333333,0.333333,0.333333}%
\pgfsetfillcolor{currentfill}%
\pgfsetlinewidth{0.803000pt}%
\definecolor{currentstroke}{rgb}{0.333333,0.333333,0.333333}%
\pgfsetstrokecolor{currentstroke}%
\pgfsetdash{}{0pt}%
\pgfsys@defobject{currentmarker}{\pgfqpoint{-0.048611in}{0.000000in}}{\pgfqpoint{-0.000000in}{0.000000in}}{%
\pgfpathmoveto{\pgfqpoint{-0.000000in}{0.000000in}}%
\pgfpathlineto{\pgfqpoint{-0.048611in}{0.000000in}}%
\pgfusepath{stroke,fill}%
}%
\begin{pgfscope}%
\pgfsys@transformshift{6.348936in}{2.990242in}%
\pgfsys@useobject{currentmarker}{}%
\end{pgfscope}%
\end{pgfscope}%
\begin{pgfscope}%
\pgfpathrectangle{\pgfqpoint{6.348936in}{1.263068in}}{\pgfqpoint{2.476670in}{1.940569in}}%
\pgfusepath{clip}%
\pgfsetbuttcap%
\pgfsetroundjoin%
\pgfsetlinewidth{1.505625pt}%
\definecolor{currentstroke}{rgb}{0.580392,0.403922,0.741176}%
\pgfsetstrokecolor{currentstroke}%
\pgfsetdash{{5.550000pt}{2.400000pt}}{0.000000pt}%
\pgfpathmoveto{\pgfqpoint{6.461512in}{1.720018in}}%
\pgfpathlineto{\pgfqpoint{7.024392in}{1.470213in}}%
\pgfpathlineto{\pgfqpoint{7.587271in}{1.470170in}}%
\pgfpathlineto{\pgfqpoint{8.150151in}{1.470166in}}%
\pgfpathlineto{\pgfqpoint{8.713030in}{1.470165in}}%
\pgfusepath{stroke}%
\end{pgfscope}%
\begin{pgfscope}%
\pgfpathrectangle{\pgfqpoint{6.348936in}{1.263068in}}{\pgfqpoint{2.476670in}{1.940569in}}%
\pgfusepath{clip}%
\pgfsetbuttcap%
\pgfsetroundjoin%
\definecolor{currentfill}{rgb}{0.580392,0.403922,0.741176}%
\pgfsetfillcolor{currentfill}%
\pgfsetlinewidth{1.003750pt}%
\definecolor{currentstroke}{rgb}{0.580392,0.403922,0.741176}%
\pgfsetstrokecolor{currentstroke}%
\pgfsetdash{}{0pt}%
\pgfsys@defobject{currentmarker}{\pgfqpoint{-0.041667in}{-0.041667in}}{\pgfqpoint{0.041667in}{0.041667in}}{%
\pgfpathmoveto{\pgfqpoint{0.000000in}{-0.041667in}}%
\pgfpathcurveto{\pgfqpoint{0.011050in}{-0.041667in}}{\pgfqpoint{0.021649in}{-0.037276in}}{\pgfqpoint{0.029463in}{-0.029463in}}%
\pgfpathcurveto{\pgfqpoint{0.037276in}{-0.021649in}}{\pgfqpoint{0.041667in}{-0.011050in}}{\pgfqpoint{0.041667in}{0.000000in}}%
\pgfpathcurveto{\pgfqpoint{0.041667in}{0.011050in}}{\pgfqpoint{0.037276in}{0.021649in}}{\pgfqpoint{0.029463in}{0.029463in}}%
\pgfpathcurveto{\pgfqpoint{0.021649in}{0.037276in}}{\pgfqpoint{0.011050in}{0.041667in}}{\pgfqpoint{0.000000in}{0.041667in}}%
\pgfpathcurveto{\pgfqpoint{-0.011050in}{0.041667in}}{\pgfqpoint{-0.021649in}{0.037276in}}{\pgfqpoint{-0.029463in}{0.029463in}}%
\pgfpathcurveto{\pgfqpoint{-0.037276in}{0.021649in}}{\pgfqpoint{-0.041667in}{0.011050in}}{\pgfqpoint{-0.041667in}{0.000000in}}%
\pgfpathcurveto{\pgfqpoint{-0.041667in}{-0.011050in}}{\pgfqpoint{-0.037276in}{-0.021649in}}{\pgfqpoint{-0.029463in}{-0.029463in}}%
\pgfpathcurveto{\pgfqpoint{-0.021649in}{-0.037276in}}{\pgfqpoint{-0.011050in}{-0.041667in}}{\pgfqpoint{0.000000in}{-0.041667in}}%
\pgfpathclose%
\pgfusepath{stroke,fill}%
}%
\begin{pgfscope}%
\pgfsys@transformshift{6.461512in}{1.720018in}%
\pgfsys@useobject{currentmarker}{}%
\end{pgfscope}%
\begin{pgfscope}%
\pgfsys@transformshift{7.024392in}{1.470213in}%
\pgfsys@useobject{currentmarker}{}%
\end{pgfscope}%
\begin{pgfscope}%
\pgfsys@transformshift{7.587271in}{1.470170in}%
\pgfsys@useobject{currentmarker}{}%
\end{pgfscope}%
\begin{pgfscope}%
\pgfsys@transformshift{8.150151in}{1.470166in}%
\pgfsys@useobject{currentmarker}{}%
\end{pgfscope}%
\begin{pgfscope}%
\pgfsys@transformshift{8.713030in}{1.470165in}%
\pgfsys@useobject{currentmarker}{}%
\end{pgfscope}%
\end{pgfscope}%
\begin{pgfscope}%
\pgfsetrectcap%
\pgfsetmiterjoin%
\pgfsetlinewidth{1.003750pt}%
\definecolor{currentstroke}{rgb}{1.000000,1.000000,1.000000}%
\pgfsetstrokecolor{currentstroke}%
\pgfsetdash{}{0pt}%
\pgfpathmoveto{\pgfqpoint{6.348936in}{1.263068in}}%
\pgfpathlineto{\pgfqpoint{6.348936in}{3.203637in}}%
\pgfusepath{stroke}%
\end{pgfscope}%
\begin{pgfscope}%
\pgfsetrectcap%
\pgfsetmiterjoin%
\pgfsetlinewidth{1.003750pt}%
\definecolor{currentstroke}{rgb}{1.000000,1.000000,1.000000}%
\pgfsetstrokecolor{currentstroke}%
\pgfsetdash{}{0pt}%
\pgfpathmoveto{\pgfqpoint{8.825606in}{1.263068in}}%
\pgfpathlineto{\pgfqpoint{8.825606in}{3.203637in}}%
\pgfusepath{stroke}%
\end{pgfscope}%
\begin{pgfscope}%
\pgfsetrectcap%
\pgfsetmiterjoin%
\pgfsetlinewidth{1.003750pt}%
\definecolor{currentstroke}{rgb}{1.000000,1.000000,1.000000}%
\pgfsetstrokecolor{currentstroke}%
\pgfsetdash{}{0pt}%
\pgfpathmoveto{\pgfqpoint{6.348936in}{1.263068in}}%
\pgfpathlineto{\pgfqpoint{8.825606in}{1.263068in}}%
\pgfusepath{stroke}%
\end{pgfscope}%
\begin{pgfscope}%
\pgfsetrectcap%
\pgfsetmiterjoin%
\pgfsetlinewidth{1.003750pt}%
\definecolor{currentstroke}{rgb}{1.000000,1.000000,1.000000}%
\pgfsetstrokecolor{currentstroke}%
\pgfsetdash{}{0pt}%
\pgfpathmoveto{\pgfqpoint{6.348936in}{3.203637in}}%
\pgfpathlineto{\pgfqpoint{8.825606in}{3.203637in}}%
\pgfusepath{stroke}%
\end{pgfscope}%
\begin{pgfscope}%
\pgfsetbuttcap%
\pgfsetmiterjoin%
\definecolor{currentfill}{rgb}{0.898039,0.898039,0.898039}%
\pgfsetfillcolor{currentfill}%
\pgfsetlinewidth{0.000000pt}%
\definecolor{currentstroke}{rgb}{0.000000,0.000000,0.000000}%
\pgfsetstrokecolor{currentstroke}%
\pgfsetstrokeopacity{0.000000}%
\pgfsetdash{}{0pt}%
\pgfpathmoveto{\pgfqpoint{9.045553in}{1.263068in}}%
\pgfpathlineto{\pgfqpoint{11.522222in}{1.263068in}}%
\pgfpathlineto{\pgfqpoint{11.522222in}{3.203637in}}%
\pgfpathlineto{\pgfqpoint{9.045553in}{3.203637in}}%
\pgfpathclose%
\pgfusepath{fill}%
\end{pgfscope}%
\begin{pgfscope}%
\pgfpathrectangle{\pgfqpoint{9.045553in}{1.263068in}}{\pgfqpoint{2.476670in}{1.940569in}}%
\pgfusepath{clip}%
\pgfsetrectcap%
\pgfsetroundjoin%
\pgfsetlinewidth{0.803000pt}%
\definecolor{currentstroke}{rgb}{1.000000,1.000000,1.000000}%
\pgfsetstrokecolor{currentstroke}%
\pgfsetstrokeopacity{0.000000}%
\pgfsetdash{}{0pt}%
\pgfpathmoveto{\pgfqpoint{9.158129in}{1.263068in}}%
\pgfpathlineto{\pgfqpoint{9.158129in}{3.203637in}}%
\pgfusepath{stroke}%
\end{pgfscope}%
\begin{pgfscope}%
\pgfsetbuttcap%
\pgfsetroundjoin%
\definecolor{currentfill}{rgb}{0.333333,0.333333,0.333333}%
\pgfsetfillcolor{currentfill}%
\pgfsetlinewidth{0.803000pt}%
\definecolor{currentstroke}{rgb}{0.333333,0.333333,0.333333}%
\pgfsetstrokecolor{currentstroke}%
\pgfsetdash{}{0pt}%
\pgfsys@defobject{currentmarker}{\pgfqpoint{0.000000in}{-0.048611in}}{\pgfqpoint{0.000000in}{0.000000in}}{%
\pgfpathmoveto{\pgfqpoint{0.000000in}{0.000000in}}%
\pgfpathlineto{\pgfqpoint{0.000000in}{-0.048611in}}%
\pgfusepath{stroke,fill}%
}%
\begin{pgfscope}%
\pgfsys@transformshift{9.158129in}{1.263068in}%
\pgfsys@useobject{currentmarker}{}%
\end{pgfscope}%
\end{pgfscope}%
\begin{pgfscope}%
\definecolor{textcolor}{rgb}{0.333333,0.333333,0.333333}%
\pgfsetstrokecolor{textcolor}%
\pgfsetfillcolor{textcolor}%
\pgftext[x=9.224679in, y=0.100000in, left, base,rotate=90.000000]{\color{textcolor}\rmfamily\fontsize{16.000000}{19.200000}\selectfont mga-0-20\%}%
\end{pgfscope}%
\begin{pgfscope}%
\pgfpathrectangle{\pgfqpoint{9.045553in}{1.263068in}}{\pgfqpoint{2.476670in}{1.940569in}}%
\pgfusepath{clip}%
\pgfsetrectcap%
\pgfsetroundjoin%
\pgfsetlinewidth{0.803000pt}%
\definecolor{currentstroke}{rgb}{1.000000,1.000000,1.000000}%
\pgfsetstrokecolor{currentstroke}%
\pgfsetstrokeopacity{0.000000}%
\pgfsetdash{}{0pt}%
\pgfpathmoveto{\pgfqpoint{9.721008in}{1.263068in}}%
\pgfpathlineto{\pgfqpoint{9.721008in}{3.203637in}}%
\pgfusepath{stroke}%
\end{pgfscope}%
\begin{pgfscope}%
\pgfsetbuttcap%
\pgfsetroundjoin%
\definecolor{currentfill}{rgb}{0.333333,0.333333,0.333333}%
\pgfsetfillcolor{currentfill}%
\pgfsetlinewidth{0.803000pt}%
\definecolor{currentstroke}{rgb}{0.333333,0.333333,0.333333}%
\pgfsetstrokecolor{currentstroke}%
\pgfsetdash{}{0pt}%
\pgfsys@defobject{currentmarker}{\pgfqpoint{0.000000in}{-0.048611in}}{\pgfqpoint{0.000000in}{0.000000in}}{%
\pgfpathmoveto{\pgfqpoint{0.000000in}{0.000000in}}%
\pgfpathlineto{\pgfqpoint{0.000000in}{-0.048611in}}%
\pgfusepath{stroke,fill}%
}%
\begin{pgfscope}%
\pgfsys@transformshift{9.721008in}{1.263068in}%
\pgfsys@useobject{currentmarker}{}%
\end{pgfscope}%
\end{pgfscope}%
\begin{pgfscope}%
\definecolor{textcolor}{rgb}{0.333333,0.333333,0.333333}%
\pgfsetstrokecolor{textcolor}%
\pgfsetfillcolor{textcolor}%
\pgftext[x=9.787559in, y=0.100000in, left, base,rotate=90.000000]{\color{textcolor}\rmfamily\fontsize{16.000000}{19.200000}\selectfont mga-1-20\%}%
\end{pgfscope}%
\begin{pgfscope}%
\pgfpathrectangle{\pgfqpoint{9.045553in}{1.263068in}}{\pgfqpoint{2.476670in}{1.940569in}}%
\pgfusepath{clip}%
\pgfsetrectcap%
\pgfsetroundjoin%
\pgfsetlinewidth{0.803000pt}%
\definecolor{currentstroke}{rgb}{1.000000,1.000000,1.000000}%
\pgfsetstrokecolor{currentstroke}%
\pgfsetstrokeopacity{0.000000}%
\pgfsetdash{}{0pt}%
\pgfpathmoveto{\pgfqpoint{10.283887in}{1.263068in}}%
\pgfpathlineto{\pgfqpoint{10.283887in}{3.203637in}}%
\pgfusepath{stroke}%
\end{pgfscope}%
\begin{pgfscope}%
\pgfsetbuttcap%
\pgfsetroundjoin%
\definecolor{currentfill}{rgb}{0.333333,0.333333,0.333333}%
\pgfsetfillcolor{currentfill}%
\pgfsetlinewidth{0.803000pt}%
\definecolor{currentstroke}{rgb}{0.333333,0.333333,0.333333}%
\pgfsetstrokecolor{currentstroke}%
\pgfsetdash{}{0pt}%
\pgfsys@defobject{currentmarker}{\pgfqpoint{0.000000in}{-0.048611in}}{\pgfqpoint{0.000000in}{0.000000in}}{%
\pgfpathmoveto{\pgfqpoint{0.000000in}{0.000000in}}%
\pgfpathlineto{\pgfqpoint{0.000000in}{-0.048611in}}%
\pgfusepath{stroke,fill}%
}%
\begin{pgfscope}%
\pgfsys@transformshift{10.283887in}{1.263068in}%
\pgfsys@useobject{currentmarker}{}%
\end{pgfscope}%
\end{pgfscope}%
\begin{pgfscope}%
\definecolor{textcolor}{rgb}{0.333333,0.333333,0.333333}%
\pgfsetstrokecolor{textcolor}%
\pgfsetfillcolor{textcolor}%
\pgftext[x=10.350438in, y=0.100000in, left, base,rotate=90.000000]{\color{textcolor}\rmfamily\fontsize{16.000000}{19.200000}\selectfont mga-2-20\%}%
\end{pgfscope}%
\begin{pgfscope}%
\pgfpathrectangle{\pgfqpoint{9.045553in}{1.263068in}}{\pgfqpoint{2.476670in}{1.940569in}}%
\pgfusepath{clip}%
\pgfsetrectcap%
\pgfsetroundjoin%
\pgfsetlinewidth{0.803000pt}%
\definecolor{currentstroke}{rgb}{1.000000,1.000000,1.000000}%
\pgfsetstrokecolor{currentstroke}%
\pgfsetstrokeopacity{0.000000}%
\pgfsetdash{}{0pt}%
\pgfpathmoveto{\pgfqpoint{10.846767in}{1.263068in}}%
\pgfpathlineto{\pgfqpoint{10.846767in}{3.203637in}}%
\pgfusepath{stroke}%
\end{pgfscope}%
\begin{pgfscope}%
\pgfsetbuttcap%
\pgfsetroundjoin%
\definecolor{currentfill}{rgb}{0.333333,0.333333,0.333333}%
\pgfsetfillcolor{currentfill}%
\pgfsetlinewidth{0.803000pt}%
\definecolor{currentstroke}{rgb}{0.333333,0.333333,0.333333}%
\pgfsetstrokecolor{currentstroke}%
\pgfsetdash{}{0pt}%
\pgfsys@defobject{currentmarker}{\pgfqpoint{0.000000in}{-0.048611in}}{\pgfqpoint{0.000000in}{0.000000in}}{%
\pgfpathmoveto{\pgfqpoint{0.000000in}{0.000000in}}%
\pgfpathlineto{\pgfqpoint{0.000000in}{-0.048611in}}%
\pgfusepath{stroke,fill}%
}%
\begin{pgfscope}%
\pgfsys@transformshift{10.846767in}{1.263068in}%
\pgfsys@useobject{currentmarker}{}%
\end{pgfscope}%
\end{pgfscope}%
\begin{pgfscope}%
\definecolor{textcolor}{rgb}{0.333333,0.333333,0.333333}%
\pgfsetstrokecolor{textcolor}%
\pgfsetfillcolor{textcolor}%
\pgftext[x=10.913318in, y=0.100000in, left, base,rotate=90.000000]{\color{textcolor}\rmfamily\fontsize{16.000000}{19.200000}\selectfont mga-3-20\%}%
\end{pgfscope}%
\begin{pgfscope}%
\pgfpathrectangle{\pgfqpoint{9.045553in}{1.263068in}}{\pgfqpoint{2.476670in}{1.940569in}}%
\pgfusepath{clip}%
\pgfsetrectcap%
\pgfsetroundjoin%
\pgfsetlinewidth{0.803000pt}%
\definecolor{currentstroke}{rgb}{1.000000,1.000000,1.000000}%
\pgfsetstrokecolor{currentstroke}%
\pgfsetstrokeopacity{0.000000}%
\pgfsetdash{}{0pt}%
\pgfpathmoveto{\pgfqpoint{11.409646in}{1.263068in}}%
\pgfpathlineto{\pgfqpoint{11.409646in}{3.203637in}}%
\pgfusepath{stroke}%
\end{pgfscope}%
\begin{pgfscope}%
\pgfsetbuttcap%
\pgfsetroundjoin%
\definecolor{currentfill}{rgb}{0.333333,0.333333,0.333333}%
\pgfsetfillcolor{currentfill}%
\pgfsetlinewidth{0.803000pt}%
\definecolor{currentstroke}{rgb}{0.333333,0.333333,0.333333}%
\pgfsetstrokecolor{currentstroke}%
\pgfsetdash{}{0pt}%
\pgfsys@defobject{currentmarker}{\pgfqpoint{0.000000in}{-0.048611in}}{\pgfqpoint{0.000000in}{0.000000in}}{%
\pgfpathmoveto{\pgfqpoint{0.000000in}{0.000000in}}%
\pgfpathlineto{\pgfqpoint{0.000000in}{-0.048611in}}%
\pgfusepath{stroke,fill}%
}%
\begin{pgfscope}%
\pgfsys@transformshift{11.409646in}{1.263068in}%
\pgfsys@useobject{currentmarker}{}%
\end{pgfscope}%
\end{pgfscope}%
\begin{pgfscope}%
\definecolor{textcolor}{rgb}{0.333333,0.333333,0.333333}%
\pgfsetstrokecolor{textcolor}%
\pgfsetfillcolor{textcolor}%
\pgftext[x=11.476197in, y=0.100000in, left, base,rotate=90.000000]{\color{textcolor}\rmfamily\fontsize{16.000000}{19.200000}\selectfont mga-4-20\%}%
\end{pgfscope}%
\begin{pgfscope}%
\pgfpathrectangle{\pgfqpoint{9.045553in}{1.263068in}}{\pgfqpoint{2.476670in}{1.940569in}}%
\pgfusepath{clip}%
\pgfsetrectcap%
\pgfsetroundjoin%
\pgfsetlinewidth{0.803000pt}%
\definecolor{currentstroke}{rgb}{1.000000,1.000000,1.000000}%
\pgfsetstrokecolor{currentstroke}%
\pgfsetdash{}{0pt}%
\pgfpathmoveto{\pgfqpoint{9.045553in}{2.115421in}}%
\pgfpathlineto{\pgfqpoint{11.522222in}{2.115421in}}%
\pgfusepath{stroke}%
\end{pgfscope}%
\begin{pgfscope}%
\pgfsetbuttcap%
\pgfsetroundjoin%
\definecolor{currentfill}{rgb}{0.333333,0.333333,0.333333}%
\pgfsetfillcolor{currentfill}%
\pgfsetlinewidth{0.803000pt}%
\definecolor{currentstroke}{rgb}{0.333333,0.333333,0.333333}%
\pgfsetstrokecolor{currentstroke}%
\pgfsetdash{}{0pt}%
\pgfsys@defobject{currentmarker}{\pgfqpoint{-0.048611in}{0.000000in}}{\pgfqpoint{-0.000000in}{0.000000in}}{%
\pgfpathmoveto{\pgfqpoint{-0.000000in}{0.000000in}}%
\pgfpathlineto{\pgfqpoint{-0.048611in}{0.000000in}}%
\pgfusepath{stroke,fill}%
}%
\begin{pgfscope}%
\pgfsys@transformshift{9.045553in}{2.115421in}%
\pgfsys@useobject{currentmarker}{}%
\end{pgfscope}%
\end{pgfscope}%
\begin{pgfscope}%
\pgfpathrectangle{\pgfqpoint{9.045553in}{1.263068in}}{\pgfqpoint{2.476670in}{1.940569in}}%
\pgfusepath{clip}%
\pgfsetrectcap%
\pgfsetroundjoin%
\pgfsetlinewidth{0.803000pt}%
\definecolor{currentstroke}{rgb}{1.000000,1.000000,1.000000}%
\pgfsetstrokecolor{currentstroke}%
\pgfsetdash{}{0pt}%
\pgfpathmoveto{\pgfqpoint{9.045553in}{2.990242in}}%
\pgfpathlineto{\pgfqpoint{11.522222in}{2.990242in}}%
\pgfusepath{stroke}%
\end{pgfscope}%
\begin{pgfscope}%
\pgfsetbuttcap%
\pgfsetroundjoin%
\definecolor{currentfill}{rgb}{0.333333,0.333333,0.333333}%
\pgfsetfillcolor{currentfill}%
\pgfsetlinewidth{0.803000pt}%
\definecolor{currentstroke}{rgb}{0.333333,0.333333,0.333333}%
\pgfsetstrokecolor{currentstroke}%
\pgfsetdash{}{0pt}%
\pgfsys@defobject{currentmarker}{\pgfqpoint{-0.048611in}{0.000000in}}{\pgfqpoint{-0.000000in}{0.000000in}}{%
\pgfpathmoveto{\pgfqpoint{-0.000000in}{0.000000in}}%
\pgfpathlineto{\pgfqpoint{-0.048611in}{0.000000in}}%
\pgfusepath{stroke,fill}%
}%
\begin{pgfscope}%
\pgfsys@transformshift{9.045553in}{2.990242in}%
\pgfsys@useobject{currentmarker}{}%
\end{pgfscope}%
\end{pgfscope}%
\begin{pgfscope}%
\pgfpathrectangle{\pgfqpoint{9.045553in}{1.263068in}}{\pgfqpoint{2.476670in}{1.940569in}}%
\pgfusepath{clip}%
\pgfsetbuttcap%
\pgfsetroundjoin%
\pgfsetlinewidth{1.505625pt}%
\definecolor{currentstroke}{rgb}{0.580392,0.403922,0.741176}%
\pgfsetstrokecolor{currentstroke}%
\pgfsetdash{{5.550000pt}{2.400000pt}}{0.000000pt}%
\pgfpathmoveto{\pgfqpoint{9.158129in}{1.720018in}}%
\pgfpathlineto{\pgfqpoint{9.721008in}{1.351275in}}%
\pgfpathlineto{\pgfqpoint{10.283887in}{1.351278in}}%
\pgfpathlineto{\pgfqpoint{10.846767in}{1.351278in}}%
\pgfpathlineto{\pgfqpoint{11.409646in}{1.351281in}}%
\pgfusepath{stroke}%
\end{pgfscope}%
\begin{pgfscope}%
\pgfpathrectangle{\pgfqpoint{9.045553in}{1.263068in}}{\pgfqpoint{2.476670in}{1.940569in}}%
\pgfusepath{clip}%
\pgfsetbuttcap%
\pgfsetroundjoin%
\definecolor{currentfill}{rgb}{0.580392,0.403922,0.741176}%
\pgfsetfillcolor{currentfill}%
\pgfsetlinewidth{1.003750pt}%
\definecolor{currentstroke}{rgb}{0.580392,0.403922,0.741176}%
\pgfsetstrokecolor{currentstroke}%
\pgfsetdash{}{0pt}%
\pgfsys@defobject{currentmarker}{\pgfqpoint{-0.041667in}{-0.041667in}}{\pgfqpoint{0.041667in}{0.041667in}}{%
\pgfpathmoveto{\pgfqpoint{0.000000in}{-0.041667in}}%
\pgfpathcurveto{\pgfqpoint{0.011050in}{-0.041667in}}{\pgfqpoint{0.021649in}{-0.037276in}}{\pgfqpoint{0.029463in}{-0.029463in}}%
\pgfpathcurveto{\pgfqpoint{0.037276in}{-0.021649in}}{\pgfqpoint{0.041667in}{-0.011050in}}{\pgfqpoint{0.041667in}{0.000000in}}%
\pgfpathcurveto{\pgfqpoint{0.041667in}{0.011050in}}{\pgfqpoint{0.037276in}{0.021649in}}{\pgfqpoint{0.029463in}{0.029463in}}%
\pgfpathcurveto{\pgfqpoint{0.021649in}{0.037276in}}{\pgfqpoint{0.011050in}{0.041667in}}{\pgfqpoint{0.000000in}{0.041667in}}%
\pgfpathcurveto{\pgfqpoint{-0.011050in}{0.041667in}}{\pgfqpoint{-0.021649in}{0.037276in}}{\pgfqpoint{-0.029463in}{0.029463in}}%
\pgfpathcurveto{\pgfqpoint{-0.037276in}{0.021649in}}{\pgfqpoint{-0.041667in}{0.011050in}}{\pgfqpoint{-0.041667in}{0.000000in}}%
\pgfpathcurveto{\pgfqpoint{-0.041667in}{-0.011050in}}{\pgfqpoint{-0.037276in}{-0.021649in}}{\pgfqpoint{-0.029463in}{-0.029463in}}%
\pgfpathcurveto{\pgfqpoint{-0.021649in}{-0.037276in}}{\pgfqpoint{-0.011050in}{-0.041667in}}{\pgfqpoint{0.000000in}{-0.041667in}}%
\pgfpathclose%
\pgfusepath{stroke,fill}%
}%
\begin{pgfscope}%
\pgfsys@transformshift{9.158129in}{1.720018in}%
\pgfsys@useobject{currentmarker}{}%
\end{pgfscope}%
\begin{pgfscope}%
\pgfsys@transformshift{9.721008in}{1.351275in}%
\pgfsys@useobject{currentmarker}{}%
\end{pgfscope}%
\begin{pgfscope}%
\pgfsys@transformshift{10.283887in}{1.351278in}%
\pgfsys@useobject{currentmarker}{}%
\end{pgfscope}%
\begin{pgfscope}%
\pgfsys@transformshift{10.846767in}{1.351278in}%
\pgfsys@useobject{currentmarker}{}%
\end{pgfscope}%
\begin{pgfscope}%
\pgfsys@transformshift{11.409646in}{1.351281in}%
\pgfsys@useobject{currentmarker}{}%
\end{pgfscope}%
\end{pgfscope}%
\begin{pgfscope}%
\pgfsetrectcap%
\pgfsetmiterjoin%
\pgfsetlinewidth{1.003750pt}%
\definecolor{currentstroke}{rgb}{1.000000,1.000000,1.000000}%
\pgfsetstrokecolor{currentstroke}%
\pgfsetdash{}{0pt}%
\pgfpathmoveto{\pgfqpoint{9.045553in}{1.263068in}}%
\pgfpathlineto{\pgfqpoint{9.045553in}{3.203637in}}%
\pgfusepath{stroke}%
\end{pgfscope}%
\begin{pgfscope}%
\pgfsetrectcap%
\pgfsetmiterjoin%
\pgfsetlinewidth{1.003750pt}%
\definecolor{currentstroke}{rgb}{1.000000,1.000000,1.000000}%
\pgfsetstrokecolor{currentstroke}%
\pgfsetdash{}{0pt}%
\pgfpathmoveto{\pgfqpoint{11.522222in}{1.263068in}}%
\pgfpathlineto{\pgfqpoint{11.522222in}{3.203637in}}%
\pgfusepath{stroke}%
\end{pgfscope}%
\begin{pgfscope}%
\pgfsetrectcap%
\pgfsetmiterjoin%
\pgfsetlinewidth{1.003750pt}%
\definecolor{currentstroke}{rgb}{1.000000,1.000000,1.000000}%
\pgfsetstrokecolor{currentstroke}%
\pgfsetdash{}{0pt}%
\pgfpathmoveto{\pgfqpoint{9.045553in}{1.263068in}}%
\pgfpathlineto{\pgfqpoint{11.522222in}{1.263068in}}%
\pgfusepath{stroke}%
\end{pgfscope}%
\begin{pgfscope}%
\pgfsetrectcap%
\pgfsetmiterjoin%
\pgfsetlinewidth{1.003750pt}%
\definecolor{currentstroke}{rgb}{1.000000,1.000000,1.000000}%
\pgfsetstrokecolor{currentstroke}%
\pgfsetdash{}{0pt}%
\pgfpathmoveto{\pgfqpoint{9.045553in}{3.203637in}}%
\pgfpathlineto{\pgfqpoint{11.522222in}{3.203637in}}%
\pgfusepath{stroke}%
\end{pgfscope}%
\begin{pgfscope}%
\definecolor{textcolor}{rgb}{0.333333,0.333333,0.333333}%
\pgfsetstrokecolor{textcolor}%
\pgfsetfillcolor{textcolor}%
\pgftext[x=11.800000in,y=4.372533in,,top,rotate=270.000000]{\color{textcolor}\rmfamily\fontsize{12.000000}{14.400000}\selectfont CWS}%
\end{pgfscope}%
\begin{pgfscope}%
\pgfsetrectcap%
\pgfsetmiterjoin%
\pgfsetlinewidth{1.003750pt}%
\definecolor{currentstroke}{rgb}{1.000000,1.000000,1.000000}%
\pgfsetstrokecolor{currentstroke}%
\pgfsetdash{}{0pt}%
\pgfpathmoveto{\pgfqpoint{9.045553in}{3.402248in}}%
\pgfpathlineto{\pgfqpoint{9.045553in}{5.342817in}}%
\pgfusepath{stroke}%
\end{pgfscope}%
\begin{pgfscope}%
\pgfsetrectcap%
\pgfsetmiterjoin%
\pgfsetlinewidth{1.003750pt}%
\definecolor{currentstroke}{rgb}{1.000000,1.000000,1.000000}%
\pgfsetstrokecolor{currentstroke}%
\pgfsetdash{}{0pt}%
\pgfpathmoveto{\pgfqpoint{11.522222in}{3.402248in}}%
\pgfpathlineto{\pgfqpoint{11.522222in}{5.342817in}}%
\pgfusepath{stroke}%
\end{pgfscope}%
\begin{pgfscope}%
\pgfsetrectcap%
\pgfsetmiterjoin%
\pgfsetlinewidth{1.003750pt}%
\definecolor{currentstroke}{rgb}{1.000000,1.000000,1.000000}%
\pgfsetstrokecolor{currentstroke}%
\pgfsetdash{}{0pt}%
\pgfpathmoveto{\pgfqpoint{9.045553in}{3.402248in}}%
\pgfpathlineto{\pgfqpoint{11.522222in}{3.402248in}}%
\pgfusepath{stroke}%
\end{pgfscope}%
\begin{pgfscope}%
\pgfsetrectcap%
\pgfsetmiterjoin%
\pgfsetlinewidth{1.003750pt}%
\definecolor{currentstroke}{rgb}{1.000000,1.000000,1.000000}%
\pgfsetstrokecolor{currentstroke}%
\pgfsetdash{}{0pt}%
\pgfpathmoveto{\pgfqpoint{9.045553in}{5.342817in}}%
\pgfpathlineto{\pgfqpoint{11.522222in}{5.342817in}}%
\pgfusepath{stroke}%
\end{pgfscope}%
\begin{pgfscope}%
\definecolor{textcolor}{rgb}{0.333333,0.333333,0.333333}%
\pgfsetstrokecolor{textcolor}%
\pgfsetfillcolor{textcolor}%
\pgftext[x=11.800000in,y=2.233352in,,top,rotate=270.000000]{\color{textcolor}\rmfamily\fontsize{12.000000}{14.400000}\selectfont CW\_STORAGE}%
\end{pgfscope}%
\begin{pgfscope}%
\pgfsetrectcap%
\pgfsetmiterjoin%
\pgfsetlinewidth{1.003750pt}%
\definecolor{currentstroke}{rgb}{1.000000,1.000000,1.000000}%
\pgfsetstrokecolor{currentstroke}%
\pgfsetdash{}{0pt}%
\pgfpathmoveto{\pgfqpoint{9.045553in}{1.263068in}}%
\pgfpathlineto{\pgfqpoint{9.045553in}{3.203637in}}%
\pgfusepath{stroke}%
\end{pgfscope}%
\begin{pgfscope}%
\pgfsetrectcap%
\pgfsetmiterjoin%
\pgfsetlinewidth{1.003750pt}%
\definecolor{currentstroke}{rgb}{1.000000,1.000000,1.000000}%
\pgfsetstrokecolor{currentstroke}%
\pgfsetdash{}{0pt}%
\pgfpathmoveto{\pgfqpoint{11.522222in}{1.263068in}}%
\pgfpathlineto{\pgfqpoint{11.522222in}{3.203637in}}%
\pgfusepath{stroke}%
\end{pgfscope}%
\begin{pgfscope}%
\pgfsetrectcap%
\pgfsetmiterjoin%
\pgfsetlinewidth{1.003750pt}%
\definecolor{currentstroke}{rgb}{1.000000,1.000000,1.000000}%
\pgfsetstrokecolor{currentstroke}%
\pgfsetdash{}{0pt}%
\pgfpathmoveto{\pgfqpoint{9.045553in}{1.263068in}}%
\pgfpathlineto{\pgfqpoint{11.522222in}{1.263068in}}%
\pgfusepath{stroke}%
\end{pgfscope}%
\begin{pgfscope}%
\pgfsetrectcap%
\pgfsetmiterjoin%
\pgfsetlinewidth{1.003750pt}%
\definecolor{currentstroke}{rgb}{1.000000,1.000000,1.000000}%
\pgfsetstrokecolor{currentstroke}%
\pgfsetdash{}{0pt}%
\pgfpathmoveto{\pgfqpoint{9.045553in}{3.203637in}}%
\pgfpathlineto{\pgfqpoint{11.522222in}{3.203637in}}%
\pgfusepath{stroke}%
\end{pgfscope}%
\begin{pgfscope}%
\definecolor{textcolor}{rgb}{0.333333,0.333333,0.333333}%
\pgfsetstrokecolor{textcolor}%
\pgfsetfillcolor{textcolor}%
\pgftext[x=11.800000in,y=4.372533in,,top,rotate=270.000000]{\color{textcolor}\rmfamily\fontsize{12.000000}{14.400000}\selectfont CWS}%
\end{pgfscope}%
\begin{pgfscope}%
\pgfsetrectcap%
\pgfsetmiterjoin%
\pgfsetlinewidth{1.003750pt}%
\definecolor{currentstroke}{rgb}{1.000000,1.000000,1.000000}%
\pgfsetstrokecolor{currentstroke}%
\pgfsetdash{}{0pt}%
\pgfpathmoveto{\pgfqpoint{9.045553in}{3.402248in}}%
\pgfpathlineto{\pgfqpoint{9.045553in}{5.342817in}}%
\pgfusepath{stroke}%
\end{pgfscope}%
\begin{pgfscope}%
\pgfsetrectcap%
\pgfsetmiterjoin%
\pgfsetlinewidth{1.003750pt}%
\definecolor{currentstroke}{rgb}{1.000000,1.000000,1.000000}%
\pgfsetstrokecolor{currentstroke}%
\pgfsetdash{}{0pt}%
\pgfpathmoveto{\pgfqpoint{11.522222in}{3.402248in}}%
\pgfpathlineto{\pgfqpoint{11.522222in}{5.342817in}}%
\pgfusepath{stroke}%
\end{pgfscope}%
\begin{pgfscope}%
\pgfsetrectcap%
\pgfsetmiterjoin%
\pgfsetlinewidth{1.003750pt}%
\definecolor{currentstroke}{rgb}{1.000000,1.000000,1.000000}%
\pgfsetstrokecolor{currentstroke}%
\pgfsetdash{}{0pt}%
\pgfpathmoveto{\pgfqpoint{9.045553in}{3.402248in}}%
\pgfpathlineto{\pgfqpoint{11.522222in}{3.402248in}}%
\pgfusepath{stroke}%
\end{pgfscope}%
\begin{pgfscope}%
\pgfsetrectcap%
\pgfsetmiterjoin%
\pgfsetlinewidth{1.003750pt}%
\definecolor{currentstroke}{rgb}{1.000000,1.000000,1.000000}%
\pgfsetstrokecolor{currentstroke}%
\pgfsetdash{}{0pt}%
\pgfpathmoveto{\pgfqpoint{9.045553in}{5.342817in}}%
\pgfpathlineto{\pgfqpoint{11.522222in}{5.342817in}}%
\pgfusepath{stroke}%
\end{pgfscope}%
\begin{pgfscope}%
\definecolor{textcolor}{rgb}{0.333333,0.333333,0.333333}%
\pgfsetstrokecolor{textcolor}%
\pgfsetfillcolor{textcolor}%
\pgftext[x=11.800000in,y=2.233352in,,top,rotate=270.000000]{\color{textcolor}\rmfamily\fontsize{12.000000}{14.400000}\selectfont CW\_STORAGE}%
\end{pgfscope}%
\begin{pgfscope}%
\pgfsetrectcap%
\pgfsetmiterjoin%
\pgfsetlinewidth{1.003750pt}%
\definecolor{currentstroke}{rgb}{1.000000,1.000000,1.000000}%
\pgfsetstrokecolor{currentstroke}%
\pgfsetdash{}{0pt}%
\pgfpathmoveto{\pgfqpoint{9.045553in}{1.263068in}}%
\pgfpathlineto{\pgfqpoint{9.045553in}{3.203637in}}%
\pgfusepath{stroke}%
\end{pgfscope}%
\begin{pgfscope}%
\pgfsetrectcap%
\pgfsetmiterjoin%
\pgfsetlinewidth{1.003750pt}%
\definecolor{currentstroke}{rgb}{1.000000,1.000000,1.000000}%
\pgfsetstrokecolor{currentstroke}%
\pgfsetdash{}{0pt}%
\pgfpathmoveto{\pgfqpoint{11.522222in}{1.263068in}}%
\pgfpathlineto{\pgfqpoint{11.522222in}{3.203637in}}%
\pgfusepath{stroke}%
\end{pgfscope}%
\begin{pgfscope}%
\pgfsetrectcap%
\pgfsetmiterjoin%
\pgfsetlinewidth{1.003750pt}%
\definecolor{currentstroke}{rgb}{1.000000,1.000000,1.000000}%
\pgfsetstrokecolor{currentstroke}%
\pgfsetdash{}{0pt}%
\pgfpathmoveto{\pgfqpoint{9.045553in}{1.263068in}}%
\pgfpathlineto{\pgfqpoint{11.522222in}{1.263068in}}%
\pgfusepath{stroke}%
\end{pgfscope}%
\begin{pgfscope}%
\pgfsetrectcap%
\pgfsetmiterjoin%
\pgfsetlinewidth{1.003750pt}%
\definecolor{currentstroke}{rgb}{1.000000,1.000000,1.000000}%
\pgfsetstrokecolor{currentstroke}%
\pgfsetdash{}{0pt}%
\pgfpathmoveto{\pgfqpoint{9.045553in}{3.203637in}}%
\pgfpathlineto{\pgfqpoint{11.522222in}{3.203637in}}%
\pgfusepath{stroke}%
\end{pgfscope}%
\begin{pgfscope}%
\definecolor{textcolor}{rgb}{0.333333,0.333333,0.333333}%
\pgfsetstrokecolor{textcolor}%
\pgfsetfillcolor{textcolor}%
\pgftext[x=11.800000in,y=4.372533in,,top,rotate=270.000000]{\color{textcolor}\rmfamily\fontsize{12.000000}{14.400000}\selectfont CWS}%
\end{pgfscope}%
\begin{pgfscope}%
\pgfsetrectcap%
\pgfsetmiterjoin%
\pgfsetlinewidth{1.003750pt}%
\definecolor{currentstroke}{rgb}{1.000000,1.000000,1.000000}%
\pgfsetstrokecolor{currentstroke}%
\pgfsetdash{}{0pt}%
\pgfpathmoveto{\pgfqpoint{9.045553in}{3.402248in}}%
\pgfpathlineto{\pgfqpoint{9.045553in}{5.342817in}}%
\pgfusepath{stroke}%
\end{pgfscope}%
\begin{pgfscope}%
\pgfsetrectcap%
\pgfsetmiterjoin%
\pgfsetlinewidth{1.003750pt}%
\definecolor{currentstroke}{rgb}{1.000000,1.000000,1.000000}%
\pgfsetstrokecolor{currentstroke}%
\pgfsetdash{}{0pt}%
\pgfpathmoveto{\pgfqpoint{11.522222in}{3.402248in}}%
\pgfpathlineto{\pgfqpoint{11.522222in}{5.342817in}}%
\pgfusepath{stroke}%
\end{pgfscope}%
\begin{pgfscope}%
\pgfsetrectcap%
\pgfsetmiterjoin%
\pgfsetlinewidth{1.003750pt}%
\definecolor{currentstroke}{rgb}{1.000000,1.000000,1.000000}%
\pgfsetstrokecolor{currentstroke}%
\pgfsetdash{}{0pt}%
\pgfpathmoveto{\pgfqpoint{9.045553in}{3.402248in}}%
\pgfpathlineto{\pgfqpoint{11.522222in}{3.402248in}}%
\pgfusepath{stroke}%
\end{pgfscope}%
\begin{pgfscope}%
\pgfsetrectcap%
\pgfsetmiterjoin%
\pgfsetlinewidth{1.003750pt}%
\definecolor{currentstroke}{rgb}{1.000000,1.000000,1.000000}%
\pgfsetstrokecolor{currentstroke}%
\pgfsetdash{}{0pt}%
\pgfpathmoveto{\pgfqpoint{9.045553in}{5.342817in}}%
\pgfpathlineto{\pgfqpoint{11.522222in}{5.342817in}}%
\pgfusepath{stroke}%
\end{pgfscope}%
\begin{pgfscope}%
\definecolor{textcolor}{rgb}{0.333333,0.333333,0.333333}%
\pgfsetstrokecolor{textcolor}%
\pgfsetfillcolor{textcolor}%
\pgftext[x=11.800000in,y=2.233352in,,top,rotate=270.000000]{\color{textcolor}\rmfamily\fontsize{12.000000}{14.400000}\selectfont CW\_STORAGE}%
\end{pgfscope}%
\begin{pgfscope}%
\pgfsetrectcap%
\pgfsetmiterjoin%
\pgfsetlinewidth{1.003750pt}%
\definecolor{currentstroke}{rgb}{1.000000,1.000000,1.000000}%
\pgfsetstrokecolor{currentstroke}%
\pgfsetdash{}{0pt}%
\pgfpathmoveto{\pgfqpoint{9.045553in}{1.263068in}}%
\pgfpathlineto{\pgfqpoint{9.045553in}{3.203637in}}%
\pgfusepath{stroke}%
\end{pgfscope}%
\begin{pgfscope}%
\pgfsetrectcap%
\pgfsetmiterjoin%
\pgfsetlinewidth{1.003750pt}%
\definecolor{currentstroke}{rgb}{1.000000,1.000000,1.000000}%
\pgfsetstrokecolor{currentstroke}%
\pgfsetdash{}{0pt}%
\pgfpathmoveto{\pgfqpoint{11.522222in}{1.263068in}}%
\pgfpathlineto{\pgfqpoint{11.522222in}{3.203637in}}%
\pgfusepath{stroke}%
\end{pgfscope}%
\begin{pgfscope}%
\pgfsetrectcap%
\pgfsetmiterjoin%
\pgfsetlinewidth{1.003750pt}%
\definecolor{currentstroke}{rgb}{1.000000,1.000000,1.000000}%
\pgfsetstrokecolor{currentstroke}%
\pgfsetdash{}{0pt}%
\pgfpathmoveto{\pgfqpoint{9.045553in}{1.263068in}}%
\pgfpathlineto{\pgfqpoint{11.522222in}{1.263068in}}%
\pgfusepath{stroke}%
\end{pgfscope}%
\begin{pgfscope}%
\pgfsetrectcap%
\pgfsetmiterjoin%
\pgfsetlinewidth{1.003750pt}%
\definecolor{currentstroke}{rgb}{1.000000,1.000000,1.000000}%
\pgfsetstrokecolor{currentstroke}%
\pgfsetdash{}{0pt}%
\pgfpathmoveto{\pgfqpoint{9.045553in}{3.203637in}}%
\pgfpathlineto{\pgfqpoint{11.522222in}{3.203637in}}%
\pgfusepath{stroke}%
\end{pgfscope}%
\begin{pgfscope}%
\definecolor{textcolor}{rgb}{0.333333,0.333333,0.333333}%
\pgfsetstrokecolor{textcolor}%
\pgfsetfillcolor{textcolor}%
\pgftext[x=11.800000in,y=4.372533in,,top,rotate=270.000000]{\color{textcolor}\rmfamily\fontsize{12.000000}{14.400000}\selectfont CWS}%
\end{pgfscope}%
\begin{pgfscope}%
\pgfsetrectcap%
\pgfsetmiterjoin%
\pgfsetlinewidth{1.003750pt}%
\definecolor{currentstroke}{rgb}{1.000000,1.000000,1.000000}%
\pgfsetstrokecolor{currentstroke}%
\pgfsetdash{}{0pt}%
\pgfpathmoveto{\pgfqpoint{9.045553in}{3.402248in}}%
\pgfpathlineto{\pgfqpoint{9.045553in}{5.342817in}}%
\pgfusepath{stroke}%
\end{pgfscope}%
\begin{pgfscope}%
\pgfsetrectcap%
\pgfsetmiterjoin%
\pgfsetlinewidth{1.003750pt}%
\definecolor{currentstroke}{rgb}{1.000000,1.000000,1.000000}%
\pgfsetstrokecolor{currentstroke}%
\pgfsetdash{}{0pt}%
\pgfpathmoveto{\pgfqpoint{11.522222in}{3.402248in}}%
\pgfpathlineto{\pgfqpoint{11.522222in}{5.342817in}}%
\pgfusepath{stroke}%
\end{pgfscope}%
\begin{pgfscope}%
\pgfsetrectcap%
\pgfsetmiterjoin%
\pgfsetlinewidth{1.003750pt}%
\definecolor{currentstroke}{rgb}{1.000000,1.000000,1.000000}%
\pgfsetstrokecolor{currentstroke}%
\pgfsetdash{}{0pt}%
\pgfpathmoveto{\pgfqpoint{9.045553in}{3.402248in}}%
\pgfpathlineto{\pgfqpoint{11.522222in}{3.402248in}}%
\pgfusepath{stroke}%
\end{pgfscope}%
\begin{pgfscope}%
\pgfsetrectcap%
\pgfsetmiterjoin%
\pgfsetlinewidth{1.003750pt}%
\definecolor{currentstroke}{rgb}{1.000000,1.000000,1.000000}%
\pgfsetstrokecolor{currentstroke}%
\pgfsetdash{}{0pt}%
\pgfpathmoveto{\pgfqpoint{9.045553in}{5.342817in}}%
\pgfpathlineto{\pgfqpoint{11.522222in}{5.342817in}}%
\pgfusepath{stroke}%
\end{pgfscope}%
\begin{pgfscope}%
\definecolor{textcolor}{rgb}{0.333333,0.333333,0.333333}%
\pgfsetstrokecolor{textcolor}%
\pgfsetfillcolor{textcolor}%
\pgftext[x=11.800000in,y=2.233352in,,top,rotate=270.000000]{\color{textcolor}\rmfamily\fontsize{12.000000}{14.400000}\selectfont CW\_STORAGE}%
\end{pgfscope}%
\begin{pgfscope}%
\pgfsetrectcap%
\pgfsetmiterjoin%
\pgfsetlinewidth{1.003750pt}%
\definecolor{currentstroke}{rgb}{1.000000,1.000000,1.000000}%
\pgfsetstrokecolor{currentstroke}%
\pgfsetdash{}{0pt}%
\pgfpathmoveto{\pgfqpoint{9.045553in}{1.263068in}}%
\pgfpathlineto{\pgfqpoint{9.045553in}{3.203637in}}%
\pgfusepath{stroke}%
\end{pgfscope}%
\begin{pgfscope}%
\pgfsetrectcap%
\pgfsetmiterjoin%
\pgfsetlinewidth{1.003750pt}%
\definecolor{currentstroke}{rgb}{1.000000,1.000000,1.000000}%
\pgfsetstrokecolor{currentstroke}%
\pgfsetdash{}{0pt}%
\pgfpathmoveto{\pgfqpoint{11.522222in}{1.263068in}}%
\pgfpathlineto{\pgfqpoint{11.522222in}{3.203637in}}%
\pgfusepath{stroke}%
\end{pgfscope}%
\begin{pgfscope}%
\pgfsetrectcap%
\pgfsetmiterjoin%
\pgfsetlinewidth{1.003750pt}%
\definecolor{currentstroke}{rgb}{1.000000,1.000000,1.000000}%
\pgfsetstrokecolor{currentstroke}%
\pgfsetdash{}{0pt}%
\pgfpathmoveto{\pgfqpoint{9.045553in}{1.263068in}}%
\pgfpathlineto{\pgfqpoint{11.522222in}{1.263068in}}%
\pgfusepath{stroke}%
\end{pgfscope}%
\begin{pgfscope}%
\pgfsetrectcap%
\pgfsetmiterjoin%
\pgfsetlinewidth{1.003750pt}%
\definecolor{currentstroke}{rgb}{1.000000,1.000000,1.000000}%
\pgfsetstrokecolor{currentstroke}%
\pgfsetdash{}{0pt}%
\pgfpathmoveto{\pgfqpoint{9.045553in}{3.203637in}}%
\pgfpathlineto{\pgfqpoint{11.522222in}{3.203637in}}%
\pgfusepath{stroke}%
\end{pgfscope}%
\begin{pgfscope}%
\definecolor{textcolor}{rgb}{0.000000,0.000000,0.000000}%
\pgfsetstrokecolor{textcolor}%
\pgfsetfillcolor{textcolor}%
\pgftext[x=5.950000in,y=5.830000in,,top]{\color{textcolor}\rmfamily\fontsize{24.000000}{28.800000}\selectfont UIUC Chiled Water Capacity in 2050}%
\end{pgfscope}%
\end{pgfpicture}%
\makeatother%
\endgroup%
}
  \caption{Results for capacity expansion under various slack values using modeling-
  to-generate alternatives. The x-axis ticks represent a different \gls{mga} run.
  The label indicates the run number and the value of the slack variable. For example,
  ``mga-1-1\%'' indicates the first \gls{mga} run with a one percent
  slack value. For reference, the cost minimized solution ``mga-0-x\%,'' is also
  plotted.}
  \label{fig:uiuc_chw_mga}
\end{figure}

\section{Discussion and Limitations}
The cheapest pathway to net-zero for \gls{uiuc} identified in this analysis has two key features:
\begin{enumerate}
\item Expanding electricity imports.
\item Replacing natural gas and coal with nuclear power.
\end{enumerate}
A significant limitation to the modeling of UIUC’s electricity system is that energy
from wind power \glspl{ppa} must flow through the same infrastructure as generalized
electricity imports. In this case, the results show an increase in electricity
imports up to 220 MW. If the wind capacity does not represent a greater investment
in \glspl{ppa}, then the wind capacity must be interpreted as owned and operated
by \gls{uiuc} on its land, which would require the use of 50\% of the University's
farmland \cite{lopez_us_2012}. The former interpretation challenges \gls{uiuc}'s
energy security and independence, the primary motivation for establishing a microgrid.
The latter interpretation preserves \gls{uiuc}'s' energy independence but may restrict
its agricultural research capabilities. As a land grant university, this may
pose challenges politically.

Another limitation of this model is that it assumes the larger Illinois grid will
achieve carbon neutrality by 2030, the current state policy. If Illinois fails to
achieve this goal by 2050, UIUC would have to eliminate electricity imports entirely.
This case is not considered in the present analysis but would force the university
to produce more of its own clean electricity through nuclear, solar, or wind power.

Separately, the nuclear capacity is nearly 100 MW$_{th}$ by 2050 in all simulations.
The reactor type is not specified, although it is assumed this would be a series
of microreactors or small modular reactors. \gls{uiuc} is already pursuing a pilot
program to build a small
research reactor developed by \gls{usnc} \cite{noauthor_university_2021}. This reactor
alone is not nearly enough to meet all campus energy demands. Therefore, subesquent
additions of nuclear capacity are necessary to meet demands. The dearth of
technology options for steam production is genuine. However, the University may
pursue other heating methods \textit{besides} steam, such as retrofitting all
campus buildings with electric heaters or heat pumps. The cost of such a retrofit
is beyond the scope of this work.

Lastly, this model excludes the transportation sector of campus since that sector
is responsible for less than 5\% of all emissions
\cite{institute_for_sustainability_energy_and_environment_illinois_2020}. However,
as the energy sector reduces emissions, the transportation sector will make up a
more significant portion of total carbon emissions. The last few tons of carbon
will be the most difficult to curtail. In any case, achieving true carbon
neutrality is an ambitious and costly endeavor.
