%\renewcommand{\thesection}{\Alph{section}}



\counterwithin{figure}{section}
\renewcommand{\thesection}{\Alph{section}}
\section{}

\label{app:pygen}
Code Listing \ref{code:sql} shows
the code \gls{temoa} requires to specify the model time horizon and time slices.
The \gls{pygen} equivalent is shown in code listing \ref{code:python}. Changing
the number of modeled ``seasons'' or years in the model horizon is non-trivial with
\gls{temoa} alone, especially since other tables, such as \texttt{CapacityFactorTech},
which specifies the availability of each technology at each time slice, depend
on the number of time slices and technologies (the product of ``number of seasons,''
``times of day,'' and number of technologies).
The example in code listing \ref{code:sql} produces 96 time slices, thus \texttt{CapacityFactorTech}
would have at least 96 rows for a single technology. With \gls{pygen}, users can
update the number of time slices and all affected tables with no additional lines
of code.


\begin{lstlisting}[style=sqlstyle,caption={Temoa tables that define the model horizon and time slices.}, label={code:sql}, floatplacement=H]
CREATE TABLE "time_season" (
  "t_season"	text,
  PRIMARY KEY("t_season")
  );
  INSERT INTO "time_season" VALUES('S1');
  INSERT INTO "time_season" VALUES('S2');
  INSERT INTO "time_season" VALUES('S3');
  INSERT INTO "time_season" VALUES('S4');

CREATE TABLE "time_periods" (
  "t_periods"	integer,
  "flag"	text,
  FOREIGN KEY("flag") REFERENCES "time_period_labels"("t_period_labels"),
  PRIMARY KEY("t_periods")
  );
  INSERT INTO "time_periods" VALUES (2025,'f');
  INSERT INTO "time_periods" VALUES (2030,'f');
  INSERT INTO "time_periods" VALUES (2035,'f');
  INSERT INTO "time_periods" VALUES (2040,'f');
  INSERT INTO "time_periods" VALUES (2045,'f');
  INSERT INTO "time_periods" VALUES (2050,'f');
  INSERT INTO "time_periods" VALUES (2051,'f');

CREATE TABLE "time_period_labels" (
  "t_period_labels"	text,
  "t_period_labels_desc"	text,
  PRIMARY KEY("t_period_labels")
  );
  INSERT INTO "time_period_labels" VALUES ('e','existing vintages');
  INSERT INTO "time_period_labels" VALUES ('f','future');

CREATE TABLE "time_of_day" (
  "t_day"	text,
  PRIMARY KEY("t_day")
  );
  INSERT INTO "time_of_day" VALUES('H1');
  INSERT INTO "time_of_day" VALUES('H2');
  INSERT INTO "time_of_day" VALUES('H3');
  INSERT INTO "time_of_day" VALUES('H4');
  INSERT INTO "time_of_day" VALUES('H5');
  INSERT INTO "time_of_day" VALUES('H6');
  INSERT INTO "time_of_day" VALUES('H7');
  INSERT INTO "time_of_day" VALUES('H8');
  INSERT INTO "time_of_day" VALUES('H9');
  INSERT INTO "time_of_day" VALUES('H10');
  INSERT INTO "time_of_day" VALUES('H11');
  INSERT INTO "time_of_day" VALUES('H12');
  INSERT INTO "time_of_day" VALUES('H13');
  INSERT INTO "time_of_day" VALUES('H14');
  INSERT INTO "time_of_day" VALUES('H15');
  INSERT INTO "time_of_day" VALUES('H16');
  INSERT INTO "time_of_day" VALUES('H17');
  INSERT INTO "time_of_day" VALUES('H18');
  INSERT INTO "time_of_day" VALUES('H19');
  INSERT INTO "time_of_day" VALUES('H20');
  INSERT INTO "time_of_day" VALUES('H21');
  INSERT INTO "time_of_day" VALUES('H22');
  INSERT INTO "time_of_day" VALUES('H23');
  INSERT INTO "time_of_day" VALUES('H24');
\end{lstlisting}


\begin{lstlisting}[style=pythonstyle, caption={Equivalent \gls{pygen} code to specify the model horizon and time slices.}, label={code:python}, floatplacement=H]
start_year = 2025  # the first year optimized by the model
end_year = 2050  # the last year optimized by the model
N_years = 6  # the number of years optimized by the model
N_seasons = 4 # the number of "seasons" in the model
N_hours = 24  # the number of hours in a day
\end{lstlisting}
