Climate change is a monumental problem with devastating consequences for the future.
A lack of strong federal policy in the United States means it is up to individual
states and entities to develop energy and environmental policies to combat climate
change. Fortunately, the State of Illinois and its flagship university, \gls{uiuc},
developed their own commitments with the Climate and Equitable Jobs Act and
the Illinois Climate Action Plan, respectively. These plans call for changes
to energy production in Illinois and at \gls{uiuc} but lack effective measures
to meet the affordability, reliability, and sustainability requirements for sound
energy policy. This thesis used \gls{temoa} and \gls{pygen} to evaluate these
plans and make recommendations for the future.

\section{Contributions}

\gls{esom} studies attempt to answer: how much capacity is enough? While
ensuring energy is available on-demand (reliability), that consumers are not
burdened by costs (affordability), and that pursuing a particular energy policy
does not restrict future generations' ability to meet their needs (sustainability).
Of course, this question can never be answered exactly due to ambiguities and deep
uncertainties about the future. Instead, \glspl{esom} offer a range of possible answers
using various tools to clarify these uncertainties. Most frequently, the varied
parameters in a given uncertainty analysis are technology
costs or fuel prices. The influence of time resolution, part of an \gls{esom}'s
structure, only recently became a feature of interest. How model results change
due to annual weather variability is essentially absent from the literature.
This thesis introduced the \gls{pygen} tool to facilitate temporal sensitivity
analysis and fill this gap in the literature.

The Illinois case study in this thesis
demonstrated \gls{pygen}' capabilities by varying wind and solar energy capacity factors
as well as varying the model's time resolution. The sensitivity to time resolution
in section \ref{section:time_res}
showed that using too few time-slices does not capture enough of the true variability
of solar and wind energy. At these coarse levels, wind energy's role is overstated,
and the role of clean firm capacity is significantly understated. In the fully
time-resolved scenarios, advanced nuclear reactors generate a significant fraction
of Illinois' electricity, even when the capital cost for these new reactors doubled
to simulate severe cost overruns as demonstrated by the \gls{XAN} scenario in section
\ref{section:zan_xan}. The sensitivity analysis over renewable resource
availability in section \ref{section:resource_sa} demonstrated that energy systems
that depend primarily on firm
capacity, such as advanced nuclear reactors, are the most robust to annual renewable
resource variability. Conversely, energy systems with a high penetration of intermittent
and variable energy sources are susceptible to significant uncertainties in both cost
and reliability. Ultimately, the results of this case study showed that Illinois'
energy policies could be improved by encouraging the continued operation of its
existing nuclear fleet through "zero emissions credits," increasing energy storage
targets, and expanding clean firm capacity through advanced nuclear projects.

\gls{uiuc}'s \gls{icap} report committed the University to net-zero carbon emissions
by 2050. That report identified a need for clean thermal energy but lacked any
discussion on achieving that goal which presented an opportunity to analyze scenarios
that led to net-zero carbon emissions. The \gls{uiuc} case study presented
in this thesis identifies nuclear energy as a solution to the campus' need for
carbon-free thermal power.

\section{Recommendations}

The results from the Illinois case study showed the strong influence of temporal
resolution and resource availability on model results. To address the former issue,
modelers should use a low time resolution during the development stage and then use the
highest computationally-tractable time resolution for sensitivity analyses and final
results. One way to do this is by using tools that allow modelers to tune space
and time resolution, such as \gls{pygen}. The latter issue is due to the common
assumption among \glspl{esom} that all modeled years are identical. The results
from section \ref{section:resource_sa} show that this assumption is not a valid
one for energy systems with highly intermittent and variable energy sources, such
as solar and wind. A further reason this assumption should be reevaluated, that
is not considered in the present work, is the coupling between energy demand and
temperature which will surely be impacted by a changing climate on the multi-decadal
timescales that \glspl{esom} operate. For these reasons, I recommend that modelers
find ways to enable different distributions for energy demand and resource availability
for each modeled year.


\section{Suggested Future Work}

This thesis focused on the effects of intra- and inter-annual variability on model
results and performance. Only one temporal aggregation technique was applied (averaging),
and there was no spatial resolution. Future work should investigate more efficient
temporal aggregation methods, such as those proposed by Kotzur et a. 2018
\cite{kotzur_impact_2018}. Using such methods could achieve results close to a reference
case with less computational cost. Additionally, future studies should examine the influence
of spatial resolution and different spatial aggregation techniques, as well as
the tradeoffs between spatial and temporal complexity. Increasing the spatial detail
may reduce the impacts of intermittency from renewable resources such as solar and
wind. Lastly, future work should be done on \gls{temoa}'s structure, and similar
\glspl{esom}, to allow users to specify different distributions for each modeled year.
